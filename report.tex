\documentclass[12pt]{article}
\usepackage{graphicx}

\usepackage[section]{placeins}
\graphicspath{ {./img/} }
\usepackage[a4paper, total={6in, 9in}]{geometry}
\usepackage{hyperref}
\usepackage[utf8]{inputenc}
\usepackage[english]{babel}
\usepackage{fancyhdr}
\usepackage{chemformula}
\usepackage{tabularx}
\usepackage{xcolor}
\usepackage{float}
\usepackage{tabto}
\usepackage{subcaption}
\usepackage{tikz}
\usepackage[section]{placeins}
\usepackage{booktabs}
\usepackage{adjustbox}
\usepackage{array}
\usepackage{gensymb}
\usepackage{pgf}
\usepackage[siunitx, RPvoltages]{circuitikz}
\usetikzlibrary{shapes, arrows}

% Figure Setup
\tikzstyle{boxes} = [rectangle, minimum width=2cm, minimum height=1cm, text centered, text
width=3cm, draw=black]

\tikzstyle{line} = [thick, -, >=stealth]
\tikzstyle{arrow} = [thick, ->, >=stealth]

\title{{\huge \textbf{ENEL420 Assignment 1}}\vspace{20pt}\\Digital Filtering of Additive Noise on an ECG Signal\vspace{24pt}}
\author{\large Joshua Hulbert (21385664)\\\vspace{12pt}\large Josiah Craw (35046080)}

\begin{document}
\maketitle
\thispagestyle{empty}

\newpage
\section*{Abstract}
\pagenumbering{roman}
An electrocardiogram (ECG) is a voltage versus time graph of the heart’s electrical activity. Important information is contained in
the frequency spectrum of an ECG that may be used in the diagnosis of heart conditions. Proper diagnosis is dependent on the observed 
ECG being free of noise. Digital filtering techniques are commonly employed to remove additive noise from an ECG.\\

\noindent In this assignment, an ECG signal corrupted by additive noise at two frequencies was provided. The noise frequencies were
identified by observing the ECG spectrum. Finite impulse response (FIR) and infinite impulse response (IIR) notch filtering techniques
for removing the noise were investigated. Three FIR filtering design methods were compared – windowing, optimal, and frequency-sampled.
The IIR filter was designed with pole-zero placement.\\ 

\noindent Filter performance was evaluated based on the notch attenuation, transition bandwidths, phase response, and computational
complexity. It was determined the IIR filter gave the best performance of the four filters. The window and optimal FIR filters were
implemented using the Python SciPy library. The frequency-sampled filter was designed with custom code and the IIR filter was designed
analytically. 

\newpage
\pagenumbering{arabic}

\section{Introduction}
This report describes the detection of additive noise on a digital electrocardiogram (ECG) and the design and implementation of digital
notch filters to remove this noise. An ECG is a voltage versus time signal of the heart’s electrical activity. Real-world ECGs may be
subject to additive noise. Notch filters are used to remove narrowband interference from ECGs, such as 50 Hz noise from the mains power
system.\\

\noindent An ECG signal provided for this assignment contains 50,000 samples taken at a sampling rate of 1024 Hz. It is corrupted by additive noise
at 32.6 Hz and 61.7 Hz; the noise frequencies were identified by plotting the signals magnitude spectrum. To remove the noise, three FIR
filters and an IIR filter were designed and implemented in Python. The filters can each be considered cascaded notch filters with stopbands
centred at the noise frequencies. The FIR filters were implemented with the window, optimal, and frequency sampling methods. The IIR filter
was implemented by cascading two dual-pole, dual-zero placed notch filters.\\

\noindent Each filter’s operation was verified by plotting the spectrum of the filtered ECG signals. The mean noise power was estimated by computing
the variance of the noise signal. This report compares the filter’s magnitude response, phase response, and computational efficiency.\\

\noindent The next section of this report describes the methods used to design each filter. Results are presented in Section 3. In Section 4 the
performance of the filters is discussed. The report is concluded in Section 5. References are found in Section 6.

\section{Methods}

\subsection{Noise Identification}
The noise frequencies were identified by calculating the fast Fourier transform (FFT) of the noisy ECG signal. The SciPy function \texttt{fft}
was used for this. The noise frequencies are characterised by impulses in the signal spectrum at 32.6 Hz and 61.7 Hz. Plots of the ECG signal
and its spectrum are found in Section 3.1.

\subsection{Window FIR Filter}
The window method designs an FIR filter by truncating or tapering the ideal filter response; this is to multiply it by a window function [1].
This filter was implemented using SciPy’s \texttt{firwin} function, which is described in Fig. 1. The specifications were to limit the number of FIR
coefficients to 400. Note that \texttt{firwin} designs a linear phase filter. It is not possible to design a linear phase FIR band-stop filter with
an even number of coefficients, as type II and type IV FIR filters have zeros at the Nyquist frequency and zero frequency, respectively.
Therefore, \texttt{numtaps} passed to \texttt{firwin} was 399. Using as many coefficients as possible was desirable as computational power was abundant for
this simulation. The sampling frequency was passed as 1024 Hz by the parameter \texttt{fs}.\\

\noindent To design a band-stop filter, the parameter \texttt{cutoff} should be an array which defines the filter cut-off frequencies. The default
\texttt{pass\_zero} value of \textit{True} means a band-stop filter is designed. A narrower stopband results in less stop-band attenuation. By experimentation,
it was found the noise frequencies were not present in the filtered output (i.e. sufficiently attenuated) for stop-bandwidths of 8 Hz. Therefore,
\texttt{cutoff} passed to \texttt{firwin} was $[28.6, 36.6, 57.7, 65.7]$. The window used was a Hamming window; a discussion on the effect of window types is found in Section 4.

\begin{figure}[H]
    \centering
    \includegraphics[width=\textwidth]{firwin.png}
    \caption{Function prototype and description for \texttt{firwin}.}
    \label{fig:firwin}
\end{figure}

\subsection{Optimal FIR Filter}
Digital filters designed via the window method have the largest ripples at the passband and stopband edges. The optimal method of FIR filter design aims to keep the ripple
constant in the passbands and stopbands. FIR filter coefficients are calculated by this method with the SciPy function remez, which is described in Fig. 2. For the same
reason described in Section 2.1, \texttt{numtaps} passed to \texttt{remez} was 399. The sampling frequency fs was 1024 Hz.\\

\noindent Ideally, the notch filters would attenuate only the additive noise frequencies, while not affecting frequencies. Zooming in on the ECG spectrum showed that
the noise existed almost entirely in the bands 32.5 – 32.7 Hz and 61.6 – 61.8 Hz; these frequencies define the stopband edges. To define the \texttt{bands} parameter, it is
also necessary to choose a transition bandwidth. Reducing the transition bandwidth results in less stop-band attenuation. By experimentation, it was found that a transition
bandwidth of 3.5 Hz eliminated the noise frequencies. The bands parameter was passed as $[0, 29, 32.5, 32.7, 36.2, 58.1, 61.6, 61.8, 65.3, 512]$. The desired gain in each
band of \texttt{bands} is defined by \texttt{desired}, which must be half the length of bands. The value for desired $[1, 0, 1, 0, 1]$.

\begin{figure}[H]
    \centering
    \includegraphics[width=\textwidth]{remez.png}
    \caption{Function prototype and description for \texttt{remez}.}
    \label{fig:remez}
\end{figure}

\subsection{Frequency Sampled FIR Filter}
ciPy does not define a function to design a frequency-sampled FIR filter, so a Python script was written to do this. The script was adapted from an example found in [2].
The script takes 400 equally spaced samples of the ideal frequency response with one sample taken at 0 Hz (i.e. a type 1 sampling scheme). Two transition samples were used,
though this can easily be changed with the \texttt{num\_transition\_samples} variable. The transition samples are equally spaced in magnitude, though ideal magnitude spacing may be
derived by extending the work in [3]. The inverse discrete Fourier transform of the ideal frequency response is calculated, giving the impulse response, which is then
shifted to make it symmetrical. The symmetrical impulse response is then tapered by a Hamming window.

\subsection{Pole-Zero Placed IIR Filters}
A simple method for IIR filter design is pole-zero placement on the z-plane. Zeros are placed in locations where the desired frequency response is zero. Poles are placed at
the same angle as the poles; their radii determine the transition bandwidth. To keep filter coefficients real, complex poles and zeros must come in complex-conjugate pairs.\\

\noindent For a notch filter, the angles the angle to place zeros (and poles) are:
\begin{equation}
    \textrm{arg}(z) = \pm \ang{360} \frac{f_{\textrm{notch}}}{f_s}
\end{equation}

\noindent With noise frequencies at 32.6 Hz and 61.7 Hz, zeros and poles were placed at ±11.46° for one notch filter and ±21.69° for the other notch filter. The 3 dB bandwidth of the
filters was specified as f3dB = 5 Hz, and the pole radius was calculated with:
\begin{equation}
    r = 1 - \frac{BW}{f_s}\pi
\end{equation}

\noindent Giving $r = 0.9847$ for both filters. To remove both noise frequencies, the IIR filters were cascaded together. Scaling factors of $0.99057$ and $0.98633$ were calculated and applied to
maintain unity passband gain. Figure 3 shows the final filter realisation.

\begin{figure}[H]
    \centering
    \includegraphics[width=\textwidth]{iir-block.png}
    \caption{Realisation of the cascaded IIR notch filters.}
    \label{fig:iir-filt}
\end{figure}
\section{Results}

\subsection{Noisy ECG Signal and Spectrum}
\begin{figure}[H]
    \centering
    \adjustbox{max width=0.75\textwidth}{
    %% Creator: Matplotlib, PGF backend
%%
%% To include the figure in your LaTeX document, write
%%   \input{<filename>.pgf}
%%
%% Make sure the required packages are loaded in your preamble
%%   \usepackage{pgf}
%%
%% Figures using additional raster images can only be included by \input if
%% they are in the same directory as the main LaTeX file. For loading figures
%% from other directories you can use the `import` package
%%   \usepackage{import}
%% and then include the figures with
%%   \import{<path to file>}{<filename>.pgf}
%%
%% Matplotlib used the following preamble
%%
\begingroup%
\makeatletter%
\begin{pgfpicture}%
\pgfpathrectangle{\pgfpointorigin}{\pgfqpoint{6.400000in}{4.800000in}}%
\pgfusepath{use as bounding box, clip}%
\begin{pgfscope}%
\pgfsetbuttcap%
\pgfsetmiterjoin%
\definecolor{currentfill}{rgb}{1.000000,1.000000,1.000000}%
\pgfsetfillcolor{currentfill}%
\pgfsetlinewidth{0.000000pt}%
\definecolor{currentstroke}{rgb}{1.000000,1.000000,1.000000}%
\pgfsetstrokecolor{currentstroke}%
\pgfsetdash{}{0pt}%
\pgfpathmoveto{\pgfqpoint{0.000000in}{0.000000in}}%
\pgfpathlineto{\pgfqpoint{6.400000in}{0.000000in}}%
\pgfpathlineto{\pgfqpoint{6.400000in}{4.800000in}}%
\pgfpathlineto{\pgfqpoint{0.000000in}{4.800000in}}%
\pgfpathclose%
\pgfusepath{fill}%
\end{pgfscope}%
\begin{pgfscope}%
\pgfsetbuttcap%
\pgfsetmiterjoin%
\definecolor{currentfill}{rgb}{1.000000,1.000000,1.000000}%
\pgfsetfillcolor{currentfill}%
\pgfsetlinewidth{0.000000pt}%
\definecolor{currentstroke}{rgb}{0.000000,0.000000,0.000000}%
\pgfsetstrokecolor{currentstroke}%
\pgfsetstrokeopacity{0.000000}%
\pgfsetdash{}{0pt}%
\pgfpathmoveto{\pgfqpoint{0.934300in}{2.889143in}}%
\pgfpathlineto{\pgfqpoint{6.146222in}{2.889143in}}%
\pgfpathlineto{\pgfqpoint{6.146222in}{4.451359in}}%
\pgfpathlineto{\pgfqpoint{0.934300in}{4.451359in}}%
\pgfpathclose%
\pgfusepath{fill}%
\end{pgfscope}%
\begin{pgfscope}%
\pgfsetbuttcap%
\pgfsetroundjoin%
\definecolor{currentfill}{rgb}{0.000000,0.000000,0.000000}%
\pgfsetfillcolor{currentfill}%
\pgfsetlinewidth{0.803000pt}%
\definecolor{currentstroke}{rgb}{0.000000,0.000000,0.000000}%
\pgfsetstrokecolor{currentstroke}%
\pgfsetdash{}{0pt}%
\pgfsys@defobject{currentmarker}{\pgfqpoint{0.000000in}{-0.048611in}}{\pgfqpoint{0.000000in}{0.000000in}}{%
\pgfpathmoveto{\pgfqpoint{0.000000in}{0.000000in}}%
\pgfpathlineto{\pgfqpoint{0.000000in}{-0.048611in}}%
\pgfusepath{stroke,fill}%
}%
\begin{pgfscope}%
\pgfsys@transformshift{1.171206in}{2.889143in}%
\pgfsys@useobject{currentmarker}{}%
\end{pgfscope}%
\end{pgfscope}%
\begin{pgfscope}%
\definecolor{textcolor}{rgb}{0.000000,0.000000,0.000000}%
\pgfsetstrokecolor{textcolor}%
\pgfsetfillcolor{textcolor}%
\pgftext[x=1.171206in,y=2.791921in,,top]{\color{textcolor}\rmfamily\fontsize{10.000000}{12.000000}\selectfont \(\displaystyle 0\)}%
\end{pgfscope}%
\begin{pgfscope}%
\pgfsetbuttcap%
\pgfsetroundjoin%
\definecolor{currentfill}{rgb}{0.000000,0.000000,0.000000}%
\pgfsetfillcolor{currentfill}%
\pgfsetlinewidth{0.803000pt}%
\definecolor{currentstroke}{rgb}{0.000000,0.000000,0.000000}%
\pgfsetstrokecolor{currentstroke}%
\pgfsetdash{}{0pt}%
\pgfsys@defobject{currentmarker}{\pgfqpoint{0.000000in}{-0.048611in}}{\pgfqpoint{0.000000in}{0.000000in}}{%
\pgfpathmoveto{\pgfqpoint{0.000000in}{0.000000in}}%
\pgfpathlineto{\pgfqpoint{0.000000in}{-0.048611in}}%
\pgfusepath{stroke,fill}%
}%
\begin{pgfscope}%
\pgfsys@transformshift{2.141590in}{2.889143in}%
\pgfsys@useobject{currentmarker}{}%
\end{pgfscope}%
\end{pgfscope}%
\begin{pgfscope}%
\definecolor{textcolor}{rgb}{0.000000,0.000000,0.000000}%
\pgfsetstrokecolor{textcolor}%
\pgfsetfillcolor{textcolor}%
\pgftext[x=2.141590in,y=2.791921in,,top]{\color{textcolor}\rmfamily\fontsize{10.000000}{12.000000}\selectfont \(\displaystyle 10\)}%
\end{pgfscope}%
\begin{pgfscope}%
\pgfsetbuttcap%
\pgfsetroundjoin%
\definecolor{currentfill}{rgb}{0.000000,0.000000,0.000000}%
\pgfsetfillcolor{currentfill}%
\pgfsetlinewidth{0.803000pt}%
\definecolor{currentstroke}{rgb}{0.000000,0.000000,0.000000}%
\pgfsetstrokecolor{currentstroke}%
\pgfsetdash{}{0pt}%
\pgfsys@defobject{currentmarker}{\pgfqpoint{0.000000in}{-0.048611in}}{\pgfqpoint{0.000000in}{0.000000in}}{%
\pgfpathmoveto{\pgfqpoint{0.000000in}{0.000000in}}%
\pgfpathlineto{\pgfqpoint{0.000000in}{-0.048611in}}%
\pgfusepath{stroke,fill}%
}%
\begin{pgfscope}%
\pgfsys@transformshift{3.111975in}{2.889143in}%
\pgfsys@useobject{currentmarker}{}%
\end{pgfscope}%
\end{pgfscope}%
\begin{pgfscope}%
\definecolor{textcolor}{rgb}{0.000000,0.000000,0.000000}%
\pgfsetstrokecolor{textcolor}%
\pgfsetfillcolor{textcolor}%
\pgftext[x=3.111975in,y=2.791921in,,top]{\color{textcolor}\rmfamily\fontsize{10.000000}{12.000000}\selectfont \(\displaystyle 20\)}%
\end{pgfscope}%
\begin{pgfscope}%
\pgfsetbuttcap%
\pgfsetroundjoin%
\definecolor{currentfill}{rgb}{0.000000,0.000000,0.000000}%
\pgfsetfillcolor{currentfill}%
\pgfsetlinewidth{0.803000pt}%
\definecolor{currentstroke}{rgb}{0.000000,0.000000,0.000000}%
\pgfsetstrokecolor{currentstroke}%
\pgfsetdash{}{0pt}%
\pgfsys@defobject{currentmarker}{\pgfqpoint{0.000000in}{-0.048611in}}{\pgfqpoint{0.000000in}{0.000000in}}{%
\pgfpathmoveto{\pgfqpoint{0.000000in}{0.000000in}}%
\pgfpathlineto{\pgfqpoint{0.000000in}{-0.048611in}}%
\pgfusepath{stroke,fill}%
}%
\begin{pgfscope}%
\pgfsys@transformshift{4.082359in}{2.889143in}%
\pgfsys@useobject{currentmarker}{}%
\end{pgfscope}%
\end{pgfscope}%
\begin{pgfscope}%
\definecolor{textcolor}{rgb}{0.000000,0.000000,0.000000}%
\pgfsetstrokecolor{textcolor}%
\pgfsetfillcolor{textcolor}%
\pgftext[x=4.082359in,y=2.791921in,,top]{\color{textcolor}\rmfamily\fontsize{10.000000}{12.000000}\selectfont \(\displaystyle 30\)}%
\end{pgfscope}%
\begin{pgfscope}%
\pgfsetbuttcap%
\pgfsetroundjoin%
\definecolor{currentfill}{rgb}{0.000000,0.000000,0.000000}%
\pgfsetfillcolor{currentfill}%
\pgfsetlinewidth{0.803000pt}%
\definecolor{currentstroke}{rgb}{0.000000,0.000000,0.000000}%
\pgfsetstrokecolor{currentstroke}%
\pgfsetdash{}{0pt}%
\pgfsys@defobject{currentmarker}{\pgfqpoint{0.000000in}{-0.048611in}}{\pgfqpoint{0.000000in}{0.000000in}}{%
\pgfpathmoveto{\pgfqpoint{0.000000in}{0.000000in}}%
\pgfpathlineto{\pgfqpoint{0.000000in}{-0.048611in}}%
\pgfusepath{stroke,fill}%
}%
\begin{pgfscope}%
\pgfsys@transformshift{5.052744in}{2.889143in}%
\pgfsys@useobject{currentmarker}{}%
\end{pgfscope}%
\end{pgfscope}%
\begin{pgfscope}%
\definecolor{textcolor}{rgb}{0.000000,0.000000,0.000000}%
\pgfsetstrokecolor{textcolor}%
\pgfsetfillcolor{textcolor}%
\pgftext[x=5.052744in,y=2.791921in,,top]{\color{textcolor}\rmfamily\fontsize{10.000000}{12.000000}\selectfont \(\displaystyle 40\)}%
\end{pgfscope}%
\begin{pgfscope}%
\pgfsetbuttcap%
\pgfsetroundjoin%
\definecolor{currentfill}{rgb}{0.000000,0.000000,0.000000}%
\pgfsetfillcolor{currentfill}%
\pgfsetlinewidth{0.803000pt}%
\definecolor{currentstroke}{rgb}{0.000000,0.000000,0.000000}%
\pgfsetstrokecolor{currentstroke}%
\pgfsetdash{}{0pt}%
\pgfsys@defobject{currentmarker}{\pgfqpoint{0.000000in}{-0.048611in}}{\pgfqpoint{0.000000in}{0.000000in}}{%
\pgfpathmoveto{\pgfqpoint{0.000000in}{0.000000in}}%
\pgfpathlineto{\pgfqpoint{0.000000in}{-0.048611in}}%
\pgfusepath{stroke,fill}%
}%
\begin{pgfscope}%
\pgfsys@transformshift{6.023128in}{2.889143in}%
\pgfsys@useobject{currentmarker}{}%
\end{pgfscope}%
\end{pgfscope}%
\begin{pgfscope}%
\definecolor{textcolor}{rgb}{0.000000,0.000000,0.000000}%
\pgfsetstrokecolor{textcolor}%
\pgfsetfillcolor{textcolor}%
\pgftext[x=6.023128in,y=2.791921in,,top]{\color{textcolor}\rmfamily\fontsize{10.000000}{12.000000}\selectfont \(\displaystyle 50\)}%
\end{pgfscope}%
\begin{pgfscope}%
\definecolor{textcolor}{rgb}{0.000000,0.000000,0.000000}%
\pgfsetstrokecolor{textcolor}%
\pgfsetfillcolor{textcolor}%
\pgftext[x=3.540261in,y=2.612909in,,top]{\color{textcolor}\rmfamily\fontsize{10.000000}{12.000000}\selectfont Time (s)}%
\end{pgfscope}%
\begin{pgfscope}%
\pgfsetbuttcap%
\pgfsetroundjoin%
\definecolor{currentfill}{rgb}{0.000000,0.000000,0.000000}%
\pgfsetfillcolor{currentfill}%
\pgfsetlinewidth{0.803000pt}%
\definecolor{currentstroke}{rgb}{0.000000,0.000000,0.000000}%
\pgfsetstrokecolor{currentstroke}%
\pgfsetdash{}{0pt}%
\pgfsys@defobject{currentmarker}{\pgfqpoint{-0.048611in}{0.000000in}}{\pgfqpoint{0.000000in}{0.000000in}}{%
\pgfpathmoveto{\pgfqpoint{0.000000in}{0.000000in}}%
\pgfpathlineto{\pgfqpoint{-0.048611in}{0.000000in}}%
\pgfusepath{stroke,fill}%
}%
\begin{pgfscope}%
\pgfsys@transformshift{0.934300in}{3.013981in}%
\pgfsys@useobject{currentmarker}{}%
\end{pgfscope}%
\end{pgfscope}%
\begin{pgfscope}%
\definecolor{textcolor}{rgb}{0.000000,0.000000,0.000000}%
\pgfsetstrokecolor{textcolor}%
\pgfsetfillcolor{textcolor}%
\pgftext[x=0.451275in,y=2.965756in,left,base]{\color{textcolor}\rmfamily\fontsize{10.000000}{12.000000}\selectfont \(\displaystyle -1000\)}%
\end{pgfscope}%
\begin{pgfscope}%
\pgfsetbuttcap%
\pgfsetroundjoin%
\definecolor{currentfill}{rgb}{0.000000,0.000000,0.000000}%
\pgfsetfillcolor{currentfill}%
\pgfsetlinewidth{0.803000pt}%
\definecolor{currentstroke}{rgb}{0.000000,0.000000,0.000000}%
\pgfsetstrokecolor{currentstroke}%
\pgfsetdash{}{0pt}%
\pgfsys@defobject{currentmarker}{\pgfqpoint{-0.048611in}{0.000000in}}{\pgfqpoint{0.000000in}{0.000000in}}{%
\pgfpathmoveto{\pgfqpoint{0.000000in}{0.000000in}}%
\pgfpathlineto{\pgfqpoint{-0.048611in}{0.000000in}}%
\pgfusepath{stroke,fill}%
}%
\begin{pgfscope}%
\pgfsys@transformshift{0.934300in}{3.558604in}%
\pgfsys@useobject{currentmarker}{}%
\end{pgfscope}%
\end{pgfscope}%
\begin{pgfscope}%
\definecolor{textcolor}{rgb}{0.000000,0.000000,0.000000}%
\pgfsetstrokecolor{textcolor}%
\pgfsetfillcolor{textcolor}%
\pgftext[x=0.767633in,y=3.510378in,left,base]{\color{textcolor}\rmfamily\fontsize{10.000000}{12.000000}\selectfont \(\displaystyle 0\)}%
\end{pgfscope}%
\begin{pgfscope}%
\pgfsetbuttcap%
\pgfsetroundjoin%
\definecolor{currentfill}{rgb}{0.000000,0.000000,0.000000}%
\pgfsetfillcolor{currentfill}%
\pgfsetlinewidth{0.803000pt}%
\definecolor{currentstroke}{rgb}{0.000000,0.000000,0.000000}%
\pgfsetstrokecolor{currentstroke}%
\pgfsetdash{}{0pt}%
\pgfsys@defobject{currentmarker}{\pgfqpoint{-0.048611in}{0.000000in}}{\pgfqpoint{0.000000in}{0.000000in}}{%
\pgfpathmoveto{\pgfqpoint{0.000000in}{0.000000in}}%
\pgfpathlineto{\pgfqpoint{-0.048611in}{0.000000in}}%
\pgfusepath{stroke,fill}%
}%
\begin{pgfscope}%
\pgfsys@transformshift{0.934300in}{4.103226in}%
\pgfsys@useobject{currentmarker}{}%
\end{pgfscope}%
\end{pgfscope}%
\begin{pgfscope}%
\definecolor{textcolor}{rgb}{0.000000,0.000000,0.000000}%
\pgfsetstrokecolor{textcolor}%
\pgfsetfillcolor{textcolor}%
\pgftext[x=0.559300in,y=4.055001in,left,base]{\color{textcolor}\rmfamily\fontsize{10.000000}{12.000000}\selectfont \(\displaystyle 1000\)}%
\end{pgfscope}%
\begin{pgfscope}%
\definecolor{textcolor}{rgb}{0.000000,0.000000,0.000000}%
\pgfsetstrokecolor{textcolor}%
\pgfsetfillcolor{textcolor}%
\pgftext[x=0.395719in,y=3.670251in,,bottom,rotate=90.000000]{\color{textcolor}\rmfamily\fontsize{10.000000}{12.000000}\selectfont ECG Voltage (\(\displaystyle \mu V\))}%
\end{pgfscope}%
\begin{pgfscope}%
\pgfpathrectangle{\pgfqpoint{0.934300in}{2.889143in}}{\pgfqpoint{5.211922in}{1.562215in}}%
\pgfusepath{clip}%
\pgfsetrectcap%
\pgfsetroundjoin%
\pgfsetlinewidth{1.505625pt}%
\definecolor{currentstroke}{rgb}{0.121569,0.466667,0.705882}%
\pgfsetstrokecolor{currentstroke}%
\pgfsetdash{}{0pt}%
\pgfpathmoveto{\pgfqpoint{1.171206in}{3.806574in}}%
\pgfpathlineto{\pgfqpoint{1.171395in}{3.858633in}}%
\pgfpathlineto{\pgfqpoint{1.171869in}{3.637116in}}%
\pgfpathlineto{\pgfqpoint{1.172248in}{3.487366in}}%
\pgfpathlineto{\pgfqpoint{1.172722in}{3.746014in}}%
\pgfpathlineto{\pgfqpoint{1.173006in}{3.810703in}}%
\pgfpathlineto{\pgfqpoint{1.173480in}{3.590769in}}%
\pgfpathlineto{\pgfqpoint{1.173765in}{3.483372in}}%
\pgfpathlineto{\pgfqpoint{1.174238in}{3.696475in}}%
\pgfpathlineto{\pgfqpoint{1.174523in}{3.808175in}}%
\pgfpathlineto{\pgfqpoint{1.174996in}{3.609359in}}%
\pgfpathlineto{\pgfqpoint{1.175376in}{3.456683in}}%
\pgfpathlineto{\pgfqpoint{1.175849in}{3.665882in}}%
\pgfpathlineto{\pgfqpoint{1.176134in}{3.765746in}}%
\pgfpathlineto{\pgfqpoint{1.176607in}{3.563229in}}%
\pgfpathlineto{\pgfqpoint{1.176892in}{3.464959in}}%
\pgfpathlineto{\pgfqpoint{1.177366in}{3.644841in}}%
\pgfpathlineto{\pgfqpoint{1.177650in}{3.766650in}}%
\pgfpathlineto{\pgfqpoint{1.178218in}{3.524647in}}%
\pgfpathlineto{\pgfqpoint{1.178503in}{3.418420in}}%
\pgfpathlineto{\pgfqpoint{1.178977in}{3.628015in}}%
\pgfpathlineto{\pgfqpoint{1.179261in}{3.712466in}}%
\pgfpathlineto{\pgfqpoint{1.179735in}{3.514412in}}%
\pgfpathlineto{\pgfqpoint{1.180114in}{3.388224in}}%
\pgfpathlineto{\pgfqpoint{1.180588in}{3.607152in}}%
\pgfpathlineto{\pgfqpoint{1.180872in}{3.673891in}}%
\pgfpathlineto{\pgfqpoint{1.181251in}{3.499424in}}%
\pgfpathlineto{\pgfqpoint{1.181630in}{3.329392in}}%
\pgfpathlineto{\pgfqpoint{1.182199in}{3.592316in}}%
\pgfpathlineto{\pgfqpoint{1.182388in}{3.639266in}}%
\pgfpathlineto{\pgfqpoint{1.182862in}{3.481436in}}%
\pgfpathlineto{\pgfqpoint{1.183241in}{3.359776in}}%
\pgfpathlineto{\pgfqpoint{1.183715in}{3.552700in}}%
\pgfpathlineto{\pgfqpoint{1.183999in}{3.642304in}}%
\pgfpathlineto{\pgfqpoint{1.184473in}{3.414723in}}%
\pgfpathlineto{\pgfqpoint{1.184757in}{3.318450in}}%
\pgfpathlineto{\pgfqpoint{1.185231in}{3.520414in}}%
\pgfpathlineto{\pgfqpoint{1.185610in}{3.644260in}}%
\pgfpathlineto{\pgfqpoint{1.186084in}{3.435280in}}%
\pgfpathlineto{\pgfqpoint{1.186368in}{3.355348in}}%
\pgfpathlineto{\pgfqpoint{1.186842in}{3.580212in}}%
\pgfpathlineto{\pgfqpoint{1.187126in}{3.674911in}}%
\pgfpathlineto{\pgfqpoint{1.187600in}{3.452198in}}%
\pgfpathlineto{\pgfqpoint{1.187884in}{3.353673in}}%
\pgfpathlineto{\pgfqpoint{1.188358in}{3.536031in}}%
\pgfpathlineto{\pgfqpoint{1.188737in}{3.672629in}}%
\pgfpathlineto{\pgfqpoint{1.189211in}{3.454205in}}%
\pgfpathlineto{\pgfqpoint{1.189495in}{3.368296in}}%
\pgfpathlineto{\pgfqpoint{1.189969in}{3.562048in}}%
\pgfpathlineto{\pgfqpoint{1.190253in}{3.684463in}}%
\pgfpathlineto{\pgfqpoint{1.190822in}{3.436989in}}%
\pgfpathlineto{\pgfqpoint{1.191012in}{3.381223in}}%
\pgfpathlineto{\pgfqpoint{1.191485in}{3.570192in}}%
\pgfpathlineto{\pgfqpoint{1.191864in}{3.722713in}}%
\pgfpathlineto{\pgfqpoint{1.192433in}{3.486497in}}%
\pgfpathlineto{\pgfqpoint{1.192623in}{3.427517in}}%
\pgfpathlineto{\pgfqpoint{1.193096in}{3.618480in}}%
\pgfpathlineto{\pgfqpoint{1.193381in}{3.726427in}}%
\pgfpathlineto{\pgfqpoint{1.193855in}{3.536312in}}%
\pgfpathlineto{\pgfqpoint{1.194139in}{3.413551in}}%
\pgfpathlineto{\pgfqpoint{1.194707in}{3.621406in}}%
\pgfpathlineto{\pgfqpoint{1.194992in}{3.710747in}}%
\pgfpathlineto{\pgfqpoint{1.195466in}{3.486605in}}%
\pgfpathlineto{\pgfqpoint{1.195845in}{3.337359in}}%
\pgfpathlineto{\pgfqpoint{1.196318in}{3.561283in}}%
\pgfpathlineto{\pgfqpoint{1.196603in}{3.634213in}}%
\pgfpathlineto{\pgfqpoint{1.196982in}{3.457850in}}%
\pgfpathlineto{\pgfqpoint{1.197361in}{3.303476in}}%
\pgfpathlineto{\pgfqpoint{1.197835in}{3.533798in}}%
\pgfpathlineto{\pgfqpoint{1.198214in}{3.653376in}}%
\pgfpathlineto{\pgfqpoint{1.198687in}{3.394245in}}%
\pgfpathlineto{\pgfqpoint{1.198972in}{3.319894in}}%
\pgfpathlineto{\pgfqpoint{1.199351in}{3.499165in}}%
\pgfpathlineto{\pgfqpoint{1.199730in}{3.656270in}}%
\pgfpathlineto{\pgfqpoint{1.200204in}{3.422118in}}%
\pgfpathlineto{\pgfqpoint{1.200488in}{3.334173in}}%
\pgfpathlineto{\pgfqpoint{1.200962in}{3.547950in}}%
\pgfpathlineto{\pgfqpoint{1.201341in}{3.672543in}}%
\pgfpathlineto{\pgfqpoint{1.201815in}{3.428084in}}%
\pgfpathlineto{\pgfqpoint{1.202099in}{3.318884in}}%
\pgfpathlineto{\pgfqpoint{1.202573in}{3.558827in}}%
\pgfpathlineto{\pgfqpoint{1.204184in}{4.004826in}}%
\pgfpathlineto{\pgfqpoint{1.204373in}{4.051678in}}%
\pgfpathlineto{\pgfqpoint{1.204752in}{3.864960in}}%
\pgfpathlineto{\pgfqpoint{1.205416in}{3.318088in}}%
\pgfpathlineto{\pgfqpoint{1.206079in}{3.644798in}}%
\pgfpathlineto{\pgfqpoint{1.206174in}{3.649848in}}%
\pgfpathlineto{\pgfqpoint{1.206269in}{3.628418in}}%
\pgfpathlineto{\pgfqpoint{1.206742in}{3.421826in}}%
\pgfpathlineto{\pgfqpoint{1.207311in}{3.657804in}}%
\pgfpathlineto{\pgfqpoint{1.207595in}{3.723852in}}%
\pgfpathlineto{\pgfqpoint{1.207974in}{3.560873in}}%
\pgfpathlineto{\pgfqpoint{1.208353in}{3.380229in}}%
\pgfpathlineto{\pgfqpoint{1.208922in}{3.643891in}}%
\pgfpathlineto{\pgfqpoint{1.209206in}{3.711172in}}%
\pgfpathlineto{\pgfqpoint{1.209680in}{3.509365in}}%
\pgfpathlineto{\pgfqpoint{1.209964in}{3.422715in}}%
\pgfpathlineto{\pgfqpoint{1.210438in}{3.638371in}}%
\pgfpathlineto{\pgfqpoint{1.210723in}{3.718343in}}%
\pgfpathlineto{\pgfqpoint{1.211196in}{3.504428in}}%
\pgfpathlineto{\pgfqpoint{1.211481in}{3.399368in}}%
\pgfpathlineto{\pgfqpoint{1.212049in}{3.633167in}}%
\pgfpathlineto{\pgfqpoint{1.212334in}{3.713750in}}%
\pgfpathlineto{\pgfqpoint{1.212807in}{3.502858in}}%
\pgfpathlineto{\pgfqpoint{1.213092in}{3.408127in}}%
\pgfpathlineto{\pgfqpoint{1.213565in}{3.621084in}}%
\pgfpathlineto{\pgfqpoint{1.213850in}{3.703867in}}%
\pgfpathlineto{\pgfqpoint{1.214324in}{3.499606in}}%
\pgfpathlineto{\pgfqpoint{1.214608in}{3.378378in}}%
\pgfpathlineto{\pgfqpoint{1.215176in}{3.633468in}}%
\pgfpathlineto{\pgfqpoint{1.215461in}{3.717361in}}%
\pgfpathlineto{\pgfqpoint{1.215935in}{3.517117in}}%
\pgfpathlineto{\pgfqpoint{1.216219in}{3.434530in}}%
\pgfpathlineto{\pgfqpoint{1.216693in}{3.626505in}}%
\pgfpathlineto{\pgfqpoint{1.216977in}{3.732162in}}%
\pgfpathlineto{\pgfqpoint{1.217451in}{3.521977in}}%
\pgfpathlineto{\pgfqpoint{1.217735in}{3.408875in}}%
\pgfpathlineto{\pgfqpoint{1.218304in}{3.631851in}}%
\pgfpathlineto{\pgfqpoint{1.218588in}{3.742680in}}%
\pgfpathlineto{\pgfqpoint{1.219062in}{3.547435in}}%
\pgfpathlineto{\pgfqpoint{1.219346in}{3.434952in}}%
\pgfpathlineto{\pgfqpoint{1.219915in}{3.664942in}}%
\pgfpathlineto{\pgfqpoint{1.220104in}{3.734632in}}%
\pgfpathlineto{\pgfqpoint{1.220578in}{3.545075in}}%
\pgfpathlineto{\pgfqpoint{1.220957in}{3.418442in}}%
\pgfpathlineto{\pgfqpoint{1.221431in}{3.625632in}}%
\pgfpathlineto{\pgfqpoint{1.221715in}{3.716182in}}%
\pgfpathlineto{\pgfqpoint{1.222189in}{3.517920in}}%
\pgfpathlineto{\pgfqpoint{1.222568in}{3.364381in}}%
\pgfpathlineto{\pgfqpoint{1.223137in}{3.588940in}}%
\pgfpathlineto{\pgfqpoint{1.223326in}{3.609875in}}%
\pgfpathlineto{\pgfqpoint{1.223610in}{3.483415in}}%
\pgfpathlineto{\pgfqpoint{1.224084in}{3.281252in}}%
\pgfpathlineto{\pgfqpoint{1.224558in}{3.516789in}}%
\pgfpathlineto{\pgfqpoint{1.224937in}{3.645223in}}%
\pgfpathlineto{\pgfqpoint{1.225411in}{3.443277in}}%
\pgfpathlineto{\pgfqpoint{1.225600in}{3.386050in}}%
\pgfpathlineto{\pgfqpoint{1.226074in}{3.552186in}}%
\pgfpathlineto{\pgfqpoint{1.226453in}{3.705722in}}%
\pgfpathlineto{\pgfqpoint{1.226927in}{3.474175in}}%
\pgfpathlineto{\pgfqpoint{1.227211in}{3.366088in}}%
\pgfpathlineto{\pgfqpoint{1.227685in}{3.560792in}}%
\pgfpathlineto{\pgfqpoint{1.227970in}{3.669423in}}%
\pgfpathlineto{\pgfqpoint{1.228538in}{3.417571in}}%
\pgfpathlineto{\pgfqpoint{1.228822in}{3.323111in}}%
\pgfpathlineto{\pgfqpoint{1.229296in}{3.531376in}}%
\pgfpathlineto{\pgfqpoint{1.229581in}{3.611522in}}%
\pgfpathlineto{\pgfqpoint{1.230054in}{3.423421in}}%
\pgfpathlineto{\pgfqpoint{1.230433in}{3.316976in}}%
\pgfpathlineto{\pgfqpoint{1.230813in}{3.502346in}}%
\pgfpathlineto{\pgfqpoint{1.231192in}{3.630227in}}%
\pgfpathlineto{\pgfqpoint{1.231665in}{3.388208in}}%
\pgfpathlineto{\pgfqpoint{1.231950in}{3.282645in}}%
\pgfpathlineto{\pgfqpoint{1.232518in}{3.523166in}}%
\pgfpathlineto{\pgfqpoint{1.232803in}{3.586404in}}%
\pgfpathlineto{\pgfqpoint{1.233182in}{3.425222in}}%
\pgfpathlineto{\pgfqpoint{1.233466in}{3.307875in}}%
\pgfpathlineto{\pgfqpoint{1.234034in}{3.547808in}}%
\pgfpathlineto{\pgfqpoint{1.234319in}{3.611608in}}%
\pgfpathlineto{\pgfqpoint{1.234698in}{3.444253in}}%
\pgfpathlineto{\pgfqpoint{1.235077in}{3.288586in}}%
\pgfpathlineto{\pgfqpoint{1.235645in}{3.526692in}}%
\pgfpathlineto{\pgfqpoint{1.235930in}{3.600369in}}%
\pgfpathlineto{\pgfqpoint{1.236404in}{3.370741in}}%
\pgfpathlineto{\pgfqpoint{1.236593in}{3.307372in}}%
\pgfpathlineto{\pgfqpoint{1.237162in}{3.517848in}}%
\pgfpathlineto{\pgfqpoint{1.237446in}{3.609954in}}%
\pgfpathlineto{\pgfqpoint{1.237920in}{3.400825in}}%
\pgfpathlineto{\pgfqpoint{1.238204in}{3.296787in}}%
\pgfpathlineto{\pgfqpoint{1.238678in}{3.493810in}}%
\pgfpathlineto{\pgfqpoint{1.239057in}{3.652096in}}%
\pgfpathlineto{\pgfqpoint{1.239626in}{3.426517in}}%
\pgfpathlineto{\pgfqpoint{1.239815in}{3.389846in}}%
\pgfpathlineto{\pgfqpoint{1.240194in}{3.545017in}}%
\pgfpathlineto{\pgfqpoint{1.240573in}{3.688423in}}%
\pgfpathlineto{\pgfqpoint{1.241047in}{3.485350in}}%
\pgfpathlineto{\pgfqpoint{1.241331in}{3.380053in}}%
\pgfpathlineto{\pgfqpoint{1.241900in}{3.603885in}}%
\pgfpathlineto{\pgfqpoint{1.242184in}{3.682145in}}%
\pgfpathlineto{\pgfqpoint{1.242658in}{3.473128in}}%
\pgfpathlineto{\pgfqpoint{1.242942in}{3.358546in}}%
\pgfpathlineto{\pgfqpoint{1.243511in}{3.567789in}}%
\pgfpathlineto{\pgfqpoint{1.243795in}{3.619484in}}%
\pgfpathlineto{\pgfqpoint{1.244174in}{3.449659in}}%
\pgfpathlineto{\pgfqpoint{1.244553in}{3.321660in}}%
\pgfpathlineto{\pgfqpoint{1.245027in}{3.544397in}}%
\pgfpathlineto{\pgfqpoint{1.245311in}{3.653896in}}%
\pgfpathlineto{\pgfqpoint{1.245880in}{3.408161in}}%
\pgfpathlineto{\pgfqpoint{1.246164in}{3.336160in}}%
\pgfpathlineto{\pgfqpoint{1.246638in}{3.570304in}}%
\pgfpathlineto{\pgfqpoint{1.246828in}{3.616134in}}%
\pgfpathlineto{\pgfqpoint{1.247301in}{3.459628in}}%
\pgfpathlineto{\pgfqpoint{1.247681in}{3.306271in}}%
\pgfpathlineto{\pgfqpoint{1.248154in}{3.542025in}}%
\pgfpathlineto{\pgfqpoint{1.248439in}{3.645120in}}%
\pgfpathlineto{\pgfqpoint{1.248912in}{3.453009in}}%
\pgfpathlineto{\pgfqpoint{1.249292in}{3.275198in}}%
\pgfpathlineto{\pgfqpoint{1.249765in}{3.521033in}}%
\pgfpathlineto{\pgfqpoint{1.251566in}{4.014702in}}%
\pgfpathlineto{\pgfqpoint{1.251850in}{3.915533in}}%
\pgfpathlineto{\pgfqpoint{1.252608in}{3.294991in}}%
\pgfpathlineto{\pgfqpoint{1.253366in}{3.587970in}}%
\pgfpathlineto{\pgfqpoint{1.253935in}{3.368328in}}%
\pgfpathlineto{\pgfqpoint{1.254409in}{3.572089in}}%
\pgfpathlineto{\pgfqpoint{1.254693in}{3.694751in}}%
\pgfpathlineto{\pgfqpoint{1.255262in}{3.456734in}}%
\pgfpathlineto{\pgfqpoint{1.255546in}{3.326685in}}%
\pgfpathlineto{\pgfqpoint{1.256115in}{3.596606in}}%
\pgfpathlineto{\pgfqpoint{1.256304in}{3.657906in}}%
\pgfpathlineto{\pgfqpoint{1.256778in}{3.457853in}}%
\pgfpathlineto{\pgfqpoint{1.257062in}{3.377540in}}%
\pgfpathlineto{\pgfqpoint{1.257536in}{3.548921in}}%
\pgfpathlineto{\pgfqpoint{1.257915in}{3.690142in}}%
\pgfpathlineto{\pgfqpoint{1.258389in}{3.436797in}}%
\pgfpathlineto{\pgfqpoint{1.258673in}{3.340855in}}%
\pgfpathlineto{\pgfqpoint{1.259147in}{3.521320in}}%
\pgfpathlineto{\pgfqpoint{1.259431in}{3.645136in}}%
\pgfpathlineto{\pgfqpoint{1.260000in}{3.439362in}}%
\pgfpathlineto{\pgfqpoint{1.260284in}{3.378717in}}%
\pgfpathlineto{\pgfqpoint{1.260663in}{3.549136in}}%
\pgfpathlineto{\pgfqpoint{1.261042in}{3.693290in}}%
\pgfpathlineto{\pgfqpoint{1.261516in}{3.469650in}}%
\pgfpathlineto{\pgfqpoint{1.261800in}{3.344274in}}%
\pgfpathlineto{\pgfqpoint{1.262369in}{3.606835in}}%
\pgfpathlineto{\pgfqpoint{1.262653in}{3.667891in}}%
\pgfpathlineto{\pgfqpoint{1.263032in}{3.516484in}}%
\pgfpathlineto{\pgfqpoint{1.263317in}{3.394983in}}%
\pgfpathlineto{\pgfqpoint{1.263885in}{3.620234in}}%
\pgfpathlineto{\pgfqpoint{1.264169in}{3.695632in}}%
\pgfpathlineto{\pgfqpoint{1.264643in}{3.496631in}}%
\pgfpathlineto{\pgfqpoint{1.265022in}{3.353703in}}%
\pgfpathlineto{\pgfqpoint{1.265496in}{3.629863in}}%
\pgfpathlineto{\pgfqpoint{1.265780in}{3.691338in}}%
\pgfpathlineto{\pgfqpoint{1.266254in}{3.517759in}}%
\pgfpathlineto{\pgfqpoint{1.266539in}{3.420198in}}%
\pgfpathlineto{\pgfqpoint{1.267012in}{3.608658in}}%
\pgfpathlineto{\pgfqpoint{1.267297in}{3.722263in}}%
\pgfpathlineto{\pgfqpoint{1.267865in}{3.473510in}}%
\pgfpathlineto{\pgfqpoint{1.268150in}{3.408859in}}%
\pgfpathlineto{\pgfqpoint{1.268529in}{3.590281in}}%
\pgfpathlineto{\pgfqpoint{1.268908in}{3.764677in}}%
\pgfpathlineto{\pgfqpoint{1.269476in}{3.539731in}}%
\pgfpathlineto{\pgfqpoint{1.269666in}{3.494228in}}%
\pgfpathlineto{\pgfqpoint{1.270140in}{3.694882in}}%
\pgfpathlineto{\pgfqpoint{1.270424in}{3.784260in}}%
\pgfpathlineto{\pgfqpoint{1.270993in}{3.571879in}}%
\pgfpathlineto{\pgfqpoint{1.271277in}{3.497127in}}%
\pgfpathlineto{\pgfqpoint{1.271751in}{3.743823in}}%
\pgfpathlineto{\pgfqpoint{1.272130in}{3.848389in}}%
\pgfpathlineto{\pgfqpoint{1.272603in}{3.633388in}}%
\pgfpathlineto{\pgfqpoint{1.272793in}{3.579105in}}%
\pgfpathlineto{\pgfqpoint{1.273362in}{3.763187in}}%
\pgfpathlineto{\pgfqpoint{1.273646in}{3.847415in}}%
\pgfpathlineto{\pgfqpoint{1.274025in}{3.658033in}}%
\pgfpathlineto{\pgfqpoint{1.274404in}{3.530679in}}%
\pgfpathlineto{\pgfqpoint{1.274878in}{3.730589in}}%
\pgfpathlineto{\pgfqpoint{1.275162in}{3.847697in}}%
\pgfpathlineto{\pgfqpoint{1.275731in}{3.598655in}}%
\pgfpathlineto{\pgfqpoint{1.276015in}{3.508167in}}%
\pgfpathlineto{\pgfqpoint{1.276584in}{3.706369in}}%
\pgfpathlineto{\pgfqpoint{1.276773in}{3.723470in}}%
\pgfpathlineto{\pgfqpoint{1.277057in}{3.631357in}}%
\pgfpathlineto{\pgfqpoint{1.277531in}{3.419802in}}%
\pgfpathlineto{\pgfqpoint{1.278005in}{3.629621in}}%
\pgfpathlineto{\pgfqpoint{1.278289in}{3.733919in}}%
\pgfpathlineto{\pgfqpoint{1.278858in}{3.499891in}}%
\pgfpathlineto{\pgfqpoint{1.279142in}{3.396241in}}%
\pgfpathlineto{\pgfqpoint{1.279711in}{3.633775in}}%
\pgfpathlineto{\pgfqpoint{1.279900in}{3.679361in}}%
\pgfpathlineto{\pgfqpoint{1.280279in}{3.551443in}}%
\pgfpathlineto{\pgfqpoint{1.280658in}{3.399166in}}%
\pgfpathlineto{\pgfqpoint{1.281132in}{3.601934in}}%
\pgfpathlineto{\pgfqpoint{1.281511in}{3.746026in}}%
\pgfpathlineto{\pgfqpoint{1.281985in}{3.519896in}}%
\pgfpathlineto{\pgfqpoint{1.282269in}{3.410204in}}%
\pgfpathlineto{\pgfqpoint{1.282838in}{3.654792in}}%
\pgfpathlineto{\pgfqpoint{1.283028in}{3.698014in}}%
\pgfpathlineto{\pgfqpoint{1.283501in}{3.544622in}}%
\pgfpathlineto{\pgfqpoint{1.283880in}{3.425747in}}%
\pgfpathlineto{\pgfqpoint{1.284354in}{3.640191in}}%
\pgfpathlineto{\pgfqpoint{1.284639in}{3.737305in}}%
\pgfpathlineto{\pgfqpoint{1.285112in}{3.517634in}}%
\pgfpathlineto{\pgfqpoint{1.285397in}{3.414184in}}%
\pgfpathlineto{\pgfqpoint{1.285965in}{3.644608in}}%
\pgfpathlineto{\pgfqpoint{1.286250in}{3.711019in}}%
\pgfpathlineto{\pgfqpoint{1.286723in}{3.510364in}}%
\pgfpathlineto{\pgfqpoint{1.287008in}{3.442920in}}%
\pgfpathlineto{\pgfqpoint{1.287387in}{3.618061in}}%
\pgfpathlineto{\pgfqpoint{1.287766in}{3.747164in}}%
\pgfpathlineto{\pgfqpoint{1.288240in}{3.528142in}}%
\pgfpathlineto{\pgfqpoint{1.288619in}{3.388194in}}%
\pgfpathlineto{\pgfqpoint{1.289092in}{3.597431in}}%
\pgfpathlineto{\pgfqpoint{1.289377in}{3.662200in}}%
\pgfpathlineto{\pgfqpoint{1.289851in}{3.453107in}}%
\pgfpathlineto{\pgfqpoint{1.290135in}{3.382965in}}%
\pgfpathlineto{\pgfqpoint{1.290609in}{3.587084in}}%
\pgfpathlineto{\pgfqpoint{1.290893in}{3.669796in}}%
\pgfpathlineto{\pgfqpoint{1.291367in}{3.462910in}}%
\pgfpathlineto{\pgfqpoint{1.291746in}{3.336115in}}%
\pgfpathlineto{\pgfqpoint{1.292220in}{3.566563in}}%
\pgfpathlineto{\pgfqpoint{1.292504in}{3.652345in}}%
\pgfpathlineto{\pgfqpoint{1.292978in}{3.453440in}}%
\pgfpathlineto{\pgfqpoint{1.293262in}{3.363990in}}%
\pgfpathlineto{\pgfqpoint{1.293831in}{3.577267in}}%
\pgfpathlineto{\pgfqpoint{1.294020in}{3.622384in}}%
\pgfpathlineto{\pgfqpoint{1.294494in}{3.429485in}}%
\pgfpathlineto{\pgfqpoint{1.294873in}{3.274814in}}%
\pgfpathlineto{\pgfqpoint{1.295442in}{3.522502in}}%
\pgfpathlineto{\pgfqpoint{1.295631in}{3.565764in}}%
\pgfpathlineto{\pgfqpoint{1.296010in}{3.389418in}}%
\pgfpathlineto{\pgfqpoint{1.296389in}{3.206405in}}%
\pgfpathlineto{\pgfqpoint{1.296863in}{3.516301in}}%
\pgfpathlineto{\pgfqpoint{1.297337in}{3.826821in}}%
\pgfpathlineto{\pgfqpoint{1.298095in}{3.660844in}}%
\pgfpathlineto{\pgfqpoint{1.298758in}{3.934774in}}%
\pgfpathlineto{\pgfqpoint{1.299043in}{3.741858in}}%
\pgfpathlineto{\pgfqpoint{1.299611in}{3.224656in}}%
\pgfpathlineto{\pgfqpoint{1.300180in}{3.609631in}}%
\pgfpathlineto{\pgfqpoint{1.300369in}{3.643384in}}%
\pgfpathlineto{\pgfqpoint{1.300748in}{3.483010in}}%
\pgfpathlineto{\pgfqpoint{1.301127in}{3.365199in}}%
\pgfpathlineto{\pgfqpoint{1.301601in}{3.573734in}}%
\pgfpathlineto{\pgfqpoint{1.301886in}{3.711160in}}%
\pgfpathlineto{\pgfqpoint{1.302454in}{3.491026in}}%
\pgfpathlineto{\pgfqpoint{1.302738in}{3.399839in}}%
\pgfpathlineto{\pgfqpoint{1.303212in}{3.615675in}}%
\pgfpathlineto{\pgfqpoint{1.303497in}{3.681346in}}%
\pgfpathlineto{\pgfqpoint{1.303970in}{3.486514in}}%
\pgfpathlineto{\pgfqpoint{1.304255in}{3.391126in}}%
\pgfpathlineto{\pgfqpoint{1.304729in}{3.614002in}}%
\pgfpathlineto{\pgfqpoint{1.305108in}{3.764831in}}%
\pgfpathlineto{\pgfqpoint{1.305581in}{3.537744in}}%
\pgfpathlineto{\pgfqpoint{1.305866in}{3.466157in}}%
\pgfpathlineto{\pgfqpoint{1.306340in}{3.671357in}}%
\pgfpathlineto{\pgfqpoint{1.306624in}{3.738319in}}%
\pgfpathlineto{\pgfqpoint{1.307098in}{3.566783in}}%
\pgfpathlineto{\pgfqpoint{1.307382in}{3.470602in}}%
\pgfpathlineto{\pgfqpoint{1.307856in}{3.672066in}}%
\pgfpathlineto{\pgfqpoint{1.308235in}{3.821524in}}%
\pgfpathlineto{\pgfqpoint{1.308709in}{3.608050in}}%
\pgfpathlineto{\pgfqpoint{1.308993in}{3.491400in}}%
\pgfpathlineto{\pgfqpoint{1.309561in}{3.747530in}}%
\pgfpathlineto{\pgfqpoint{1.309751in}{3.798969in}}%
\pgfpathlineto{\pgfqpoint{1.310225in}{3.639112in}}%
\pgfpathlineto{\pgfqpoint{1.310604in}{3.545297in}}%
\pgfpathlineto{\pgfqpoint{1.310983in}{3.726465in}}%
\pgfpathlineto{\pgfqpoint{1.311362in}{3.886543in}}%
\pgfpathlineto{\pgfqpoint{1.311931in}{3.630161in}}%
\pgfpathlineto{\pgfqpoint{1.312215in}{3.563158in}}%
\pgfpathlineto{\pgfqpoint{1.312689in}{3.772921in}}%
\pgfpathlineto{\pgfqpoint{1.312973in}{3.840872in}}%
\pgfpathlineto{\pgfqpoint{1.313352in}{3.674983in}}%
\pgfpathlineto{\pgfqpoint{1.313636in}{3.573222in}}%
\pgfpathlineto{\pgfqpoint{1.314205in}{3.793867in}}%
\pgfpathlineto{\pgfqpoint{1.314489in}{3.872822in}}%
\pgfpathlineto{\pgfqpoint{1.314963in}{3.658085in}}%
\pgfpathlineto{\pgfqpoint{1.315342in}{3.510537in}}%
\pgfpathlineto{\pgfqpoint{1.315816in}{3.754047in}}%
\pgfpathlineto{\pgfqpoint{1.316005in}{3.798827in}}%
\pgfpathlineto{\pgfqpoint{1.316479in}{3.682182in}}%
\pgfpathlineto{\pgfqpoint{1.316858in}{3.525182in}}%
\pgfpathlineto{\pgfqpoint{1.317332in}{3.739103in}}%
\pgfpathlineto{\pgfqpoint{1.317616in}{3.826494in}}%
\pgfpathlineto{\pgfqpoint{1.318090in}{3.604659in}}%
\pgfpathlineto{\pgfqpoint{1.318469in}{3.473401in}}%
\pgfpathlineto{\pgfqpoint{1.318943in}{3.698515in}}%
\pgfpathlineto{\pgfqpoint{1.319227in}{3.789530in}}%
\pgfpathlineto{\pgfqpoint{1.319701in}{3.584615in}}%
\pgfpathlineto{\pgfqpoint{1.320080in}{3.492743in}}%
\pgfpathlineto{\pgfqpoint{1.320554in}{3.693749in}}%
\pgfpathlineto{\pgfqpoint{1.320744in}{3.726978in}}%
\pgfpathlineto{\pgfqpoint{1.321123in}{3.570461in}}%
\pgfpathlineto{\pgfqpoint{1.321597in}{3.339579in}}%
\pgfpathlineto{\pgfqpoint{1.322165in}{3.595865in}}%
\pgfpathlineto{\pgfqpoint{1.322355in}{3.633887in}}%
\pgfpathlineto{\pgfqpoint{1.322734in}{3.512811in}}%
\pgfpathlineto{\pgfqpoint{1.323208in}{3.345156in}}%
\pgfpathlineto{\pgfqpoint{1.323681in}{3.541127in}}%
\pgfpathlineto{\pgfqpoint{1.323871in}{3.607826in}}%
\pgfpathlineto{\pgfqpoint{1.324345in}{3.400418in}}%
\pgfpathlineto{\pgfqpoint{1.324724in}{3.267296in}}%
\pgfpathlineto{\pgfqpoint{1.325198in}{3.474781in}}%
\pgfpathlineto{\pgfqpoint{1.325482in}{3.590962in}}%
\pgfpathlineto{\pgfqpoint{1.326050in}{3.356579in}}%
\pgfpathlineto{\pgfqpoint{1.326335in}{3.280726in}}%
\pgfpathlineto{\pgfqpoint{1.326809in}{3.477362in}}%
\pgfpathlineto{\pgfqpoint{1.327093in}{3.539345in}}%
\pgfpathlineto{\pgfqpoint{1.327472in}{3.365988in}}%
\pgfpathlineto{\pgfqpoint{1.327851in}{3.218385in}}%
\pgfpathlineto{\pgfqpoint{1.328325in}{3.434770in}}%
\pgfpathlineto{\pgfqpoint{1.328704in}{3.567213in}}%
\pgfpathlineto{\pgfqpoint{1.329178in}{3.315888in}}%
\pgfpathlineto{\pgfqpoint{1.329462in}{3.245845in}}%
\pgfpathlineto{\pgfqpoint{1.329936in}{3.441391in}}%
\pgfpathlineto{\pgfqpoint{1.330220in}{3.518573in}}%
\pgfpathlineto{\pgfqpoint{1.330694in}{3.300763in}}%
\pgfpathlineto{\pgfqpoint{1.330978in}{3.230055in}}%
\pgfpathlineto{\pgfqpoint{1.331452in}{3.454507in}}%
\pgfpathlineto{\pgfqpoint{1.331831in}{3.600461in}}%
\pgfpathlineto{\pgfqpoint{1.332305in}{3.369389in}}%
\pgfpathlineto{\pgfqpoint{1.332494in}{3.294463in}}%
\pgfpathlineto{\pgfqpoint{1.333063in}{3.479493in}}%
\pgfpathlineto{\pgfqpoint{1.333347in}{3.578475in}}%
\pgfpathlineto{\pgfqpoint{1.333821in}{3.382973in}}%
\pgfpathlineto{\pgfqpoint{1.334105in}{3.291290in}}%
\pgfpathlineto{\pgfqpoint{1.334579in}{3.493260in}}%
\pgfpathlineto{\pgfqpoint{1.334958in}{3.638260in}}%
\pgfpathlineto{\pgfqpoint{1.335432in}{3.420451in}}%
\pgfpathlineto{\pgfqpoint{1.335716in}{3.304986in}}%
\pgfpathlineto{\pgfqpoint{1.336285in}{3.526370in}}%
\pgfpathlineto{\pgfqpoint{1.336475in}{3.563319in}}%
\pgfpathlineto{\pgfqpoint{1.336948in}{3.408612in}}%
\pgfpathlineto{\pgfqpoint{1.337233in}{3.311873in}}%
\pgfpathlineto{\pgfqpoint{1.337706in}{3.503298in}}%
\pgfpathlineto{\pgfqpoint{1.338085in}{3.643787in}}%
\pgfpathlineto{\pgfqpoint{1.338559in}{3.446268in}}%
\pgfpathlineto{\pgfqpoint{1.338844in}{3.334415in}}%
\pgfpathlineto{\pgfqpoint{1.339412in}{3.554518in}}%
\pgfpathlineto{\pgfqpoint{1.339696in}{3.631044in}}%
\pgfpathlineto{\pgfqpoint{1.340170in}{3.434033in}}%
\pgfpathlineto{\pgfqpoint{1.340360in}{3.383890in}}%
\pgfpathlineto{\pgfqpoint{1.340834in}{3.541103in}}%
\pgfpathlineto{\pgfqpoint{1.341213in}{3.683493in}}%
\pgfpathlineto{\pgfqpoint{1.341687in}{3.483090in}}%
\pgfpathlineto{\pgfqpoint{1.342066in}{3.348413in}}%
\pgfpathlineto{\pgfqpoint{1.342634in}{3.555325in}}%
\pgfpathlineto{\pgfqpoint{1.344340in}{4.119366in}}%
\pgfpathlineto{\pgfqpoint{1.344814in}{3.866857in}}%
\pgfpathlineto{\pgfqpoint{1.346425in}{3.469530in}}%
\pgfpathlineto{\pgfqpoint{1.346614in}{3.456866in}}%
\pgfpathlineto{\pgfqpoint{1.346899in}{3.493750in}}%
\pgfpathlineto{\pgfqpoint{1.347467in}{3.736457in}}%
\pgfpathlineto{\pgfqpoint{1.347941in}{3.544740in}}%
\pgfpathlineto{\pgfqpoint{1.348320in}{3.396277in}}%
\pgfpathlineto{\pgfqpoint{1.348889in}{3.643092in}}%
\pgfpathlineto{\pgfqpoint{1.349078in}{3.694615in}}%
\pgfpathlineto{\pgfqpoint{1.349552in}{3.544323in}}%
\pgfpathlineto{\pgfqpoint{1.349931in}{3.439978in}}%
\pgfpathlineto{\pgfqpoint{1.350405in}{3.647004in}}%
\pgfpathlineto{\pgfqpoint{1.350594in}{3.687039in}}%
\pgfpathlineto{\pgfqpoint{1.351068in}{3.508922in}}%
\pgfpathlineto{\pgfqpoint{1.351447in}{3.359425in}}%
\pgfpathlineto{\pgfqpoint{1.351921in}{3.565813in}}%
\pgfpathlineto{\pgfqpoint{1.352205in}{3.679206in}}%
\pgfpathlineto{\pgfqpoint{1.352774in}{3.462634in}}%
\pgfpathlineto{\pgfqpoint{1.353058in}{3.395383in}}%
\pgfpathlineto{\pgfqpoint{1.353532in}{3.556886in}}%
\pgfpathlineto{\pgfqpoint{1.353816in}{3.607778in}}%
\pgfpathlineto{\pgfqpoint{1.354195in}{3.464984in}}%
\pgfpathlineto{\pgfqpoint{1.354574in}{3.298370in}}%
\pgfpathlineto{\pgfqpoint{1.355048in}{3.522893in}}%
\pgfpathlineto{\pgfqpoint{1.355427in}{3.658612in}}%
\pgfpathlineto{\pgfqpoint{1.355901in}{3.446178in}}%
\pgfpathlineto{\pgfqpoint{1.356185in}{3.352695in}}%
\pgfpathlineto{\pgfqpoint{1.356754in}{3.564452in}}%
\pgfpathlineto{\pgfqpoint{1.356944in}{3.607211in}}%
\pgfpathlineto{\pgfqpoint{1.357323in}{3.449369in}}%
\pgfpathlineto{\pgfqpoint{1.357702in}{3.309184in}}%
\pgfpathlineto{\pgfqpoint{1.358175in}{3.502990in}}%
\pgfpathlineto{\pgfqpoint{1.358555in}{3.648751in}}%
\pgfpathlineto{\pgfqpoint{1.359123in}{3.417201in}}%
\pgfpathlineto{\pgfqpoint{1.359313in}{3.369696in}}%
\pgfpathlineto{\pgfqpoint{1.359786in}{3.524107in}}%
\pgfpathlineto{\pgfqpoint{1.360071in}{3.608405in}}%
\pgfpathlineto{\pgfqpoint{1.360545in}{3.415831in}}%
\pgfpathlineto{\pgfqpoint{1.360829in}{3.347087in}}%
\pgfpathlineto{\pgfqpoint{1.361303in}{3.537153in}}%
\pgfpathlineto{\pgfqpoint{1.361682in}{3.699941in}}%
\pgfpathlineto{\pgfqpoint{1.362250in}{3.425344in}}%
\pgfpathlineto{\pgfqpoint{1.362440in}{3.375372in}}%
\pgfpathlineto{\pgfqpoint{1.362914in}{3.536374in}}%
\pgfpathlineto{\pgfqpoint{1.363198in}{3.627316in}}%
\pgfpathlineto{\pgfqpoint{1.363767in}{3.442476in}}%
\pgfpathlineto{\pgfqpoint{1.363956in}{3.391534in}}%
\pgfpathlineto{\pgfqpoint{1.364430in}{3.576981in}}%
\pgfpathlineto{\pgfqpoint{1.364809in}{3.736578in}}%
\pgfpathlineto{\pgfqpoint{1.365283in}{3.527534in}}%
\pgfpathlineto{\pgfqpoint{1.365662in}{3.384034in}}%
\pgfpathlineto{\pgfqpoint{1.366136in}{3.594605in}}%
\pgfpathlineto{\pgfqpoint{1.366325in}{3.658887in}}%
\pgfpathlineto{\pgfqpoint{1.366894in}{3.448487in}}%
\pgfpathlineto{\pgfqpoint{1.367178in}{3.377577in}}%
\pgfpathlineto{\pgfqpoint{1.367652in}{3.566530in}}%
\pgfpathlineto{\pgfqpoint{1.367936in}{3.642575in}}%
\pgfpathlineto{\pgfqpoint{1.368410in}{3.421366in}}%
\pgfpathlineto{\pgfqpoint{1.368789in}{3.304869in}}%
\pgfpathlineto{\pgfqpoint{1.369263in}{3.526177in}}%
\pgfpathlineto{\pgfqpoint{1.369642in}{3.659407in}}%
\pgfpathlineto{\pgfqpoint{1.370305in}{3.493309in}}%
\pgfpathlineto{\pgfqpoint{1.371063in}{3.814178in}}%
\pgfpathlineto{\pgfqpoint{1.371632in}{3.536589in}}%
\pgfpathlineto{\pgfqpoint{1.371916in}{3.460159in}}%
\pgfpathlineto{\pgfqpoint{1.372390in}{3.661494in}}%
\pgfpathlineto{\pgfqpoint{1.372769in}{3.780215in}}%
\pgfpathlineto{\pgfqpoint{1.373243in}{3.598659in}}%
\pgfpathlineto{\pgfqpoint{1.373433in}{3.574596in}}%
\pgfpathlineto{\pgfqpoint{1.373812in}{3.691951in}}%
\pgfpathlineto{\pgfqpoint{1.374191in}{3.851740in}}%
\pgfpathlineto{\pgfqpoint{1.374664in}{3.651463in}}%
\pgfpathlineto{\pgfqpoint{1.375043in}{3.520136in}}%
\pgfpathlineto{\pgfqpoint{1.375517in}{3.735627in}}%
\pgfpathlineto{\pgfqpoint{1.375802in}{3.835166in}}%
\pgfpathlineto{\pgfqpoint{1.376370in}{3.638218in}}%
\pgfpathlineto{\pgfqpoint{1.376560in}{3.586353in}}%
\pgfpathlineto{\pgfqpoint{1.377034in}{3.751834in}}%
\pgfpathlineto{\pgfqpoint{1.377413in}{3.850990in}}%
\pgfpathlineto{\pgfqpoint{1.377886in}{3.642695in}}%
\pgfpathlineto{\pgfqpoint{1.378171in}{3.572990in}}%
\pgfpathlineto{\pgfqpoint{1.378645in}{3.770856in}}%
\pgfpathlineto{\pgfqpoint{1.379024in}{3.913386in}}%
\pgfpathlineto{\pgfqpoint{1.379497in}{3.717528in}}%
\pgfpathlineto{\pgfqpoint{1.379782in}{3.636035in}}%
\pgfpathlineto{\pgfqpoint{1.380256in}{3.824784in}}%
\pgfpathlineto{\pgfqpoint{1.380540in}{3.876484in}}%
\pgfpathlineto{\pgfqpoint{1.380919in}{3.739217in}}%
\pgfpathlineto{\pgfqpoint{1.381298in}{3.574737in}}%
\pgfpathlineto{\pgfqpoint{1.381867in}{3.826896in}}%
\pgfpathlineto{\pgfqpoint{1.382151in}{3.907555in}}%
\pgfpathlineto{\pgfqpoint{1.382625in}{3.660104in}}%
\pgfpathlineto{\pgfqpoint{1.382909in}{3.588465in}}%
\pgfpathlineto{\pgfqpoint{1.383383in}{3.752575in}}%
\pgfpathlineto{\pgfqpoint{1.383667in}{3.810071in}}%
\pgfpathlineto{\pgfqpoint{1.384141in}{3.621429in}}%
\pgfpathlineto{\pgfqpoint{1.384425in}{3.516395in}}%
\pgfpathlineto{\pgfqpoint{1.384899in}{3.734440in}}%
\pgfpathlineto{\pgfqpoint{1.385278in}{3.873717in}}%
\pgfpathlineto{\pgfqpoint{1.385752in}{3.652014in}}%
\pgfpathlineto{\pgfqpoint{1.386036in}{3.552221in}}%
\pgfpathlineto{\pgfqpoint{1.386605in}{3.712314in}}%
\pgfpathlineto{\pgfqpoint{1.386794in}{3.761209in}}%
\pgfpathlineto{\pgfqpoint{1.387268in}{3.575034in}}%
\pgfpathlineto{\pgfqpoint{1.387552in}{3.473571in}}%
\pgfpathlineto{\pgfqpoint{1.388121in}{3.709159in}}%
\pgfpathlineto{\pgfqpoint{1.388405in}{3.799002in}}%
\pgfpathlineto{\pgfqpoint{1.388879in}{3.549134in}}%
\pgfpathlineto{\pgfqpoint{1.389258in}{3.410736in}}%
\pgfpathlineto{\pgfqpoint{1.389732in}{3.636120in}}%
\pgfpathlineto{\pgfqpoint{1.391438in}{4.064432in}}%
\pgfpathlineto{\pgfqpoint{1.391722in}{3.993744in}}%
\pgfpathlineto{\pgfqpoint{1.392670in}{3.318383in}}%
\pgfpathlineto{\pgfqpoint{1.393712in}{3.423970in}}%
\pgfpathlineto{\pgfqpoint{1.393902in}{3.400047in}}%
\pgfpathlineto{\pgfqpoint{1.394281in}{3.519283in}}%
\pgfpathlineto{\pgfqpoint{1.394660in}{3.673192in}}%
\pgfpathlineto{\pgfqpoint{1.395133in}{3.467931in}}%
\pgfpathlineto{\pgfqpoint{1.395513in}{3.321630in}}%
\pgfpathlineto{\pgfqpoint{1.395986in}{3.515564in}}%
\pgfpathlineto{\pgfqpoint{1.396271in}{3.615261in}}%
\pgfpathlineto{\pgfqpoint{1.396839in}{3.403531in}}%
\pgfpathlineto{\pgfqpoint{1.397029in}{3.382821in}}%
\pgfpathlineto{\pgfqpoint{1.397313in}{3.472241in}}%
\pgfpathlineto{\pgfqpoint{1.397787in}{3.677693in}}%
\pgfpathlineto{\pgfqpoint{1.398261in}{3.476380in}}%
\pgfpathlineto{\pgfqpoint{1.398640in}{3.323871in}}%
\pgfpathlineto{\pgfqpoint{1.399208in}{3.568201in}}%
\pgfpathlineto{\pgfqpoint{1.399493in}{3.621959in}}%
\pgfpathlineto{\pgfqpoint{1.399872in}{3.483504in}}%
\pgfpathlineto{\pgfqpoint{1.400156in}{3.405801in}}%
\pgfpathlineto{\pgfqpoint{1.400630in}{3.582640in}}%
\pgfpathlineto{\pgfqpoint{1.401009in}{3.655315in}}%
\pgfpathlineto{\pgfqpoint{1.401388in}{3.506061in}}%
\pgfpathlineto{\pgfqpoint{1.401767in}{3.332692in}}%
\pgfpathlineto{\pgfqpoint{1.402241in}{3.563909in}}%
\pgfpathlineto{\pgfqpoint{1.402525in}{3.673850in}}%
\pgfpathlineto{\pgfqpoint{1.403094in}{3.479025in}}%
\pgfpathlineto{\pgfqpoint{1.403283in}{3.435398in}}%
\pgfpathlineto{\pgfqpoint{1.403757in}{3.609909in}}%
\pgfpathlineto{\pgfqpoint{1.404041in}{3.703657in}}%
\pgfpathlineto{\pgfqpoint{1.404515in}{3.534754in}}%
\pgfpathlineto{\pgfqpoint{1.404894in}{3.393934in}}%
\pgfpathlineto{\pgfqpoint{1.405368in}{3.606387in}}%
\pgfpathlineto{\pgfqpoint{1.405747in}{3.746483in}}%
\pgfpathlineto{\pgfqpoint{1.406316in}{3.529825in}}%
\pgfpathlineto{\pgfqpoint{1.406410in}{3.516783in}}%
\pgfpathlineto{\pgfqpoint{1.406789in}{3.615490in}}%
\pgfpathlineto{\pgfqpoint{1.407263in}{3.780146in}}%
\pgfpathlineto{\pgfqpoint{1.407737in}{3.566841in}}%
\pgfpathlineto{\pgfqpoint{1.408021in}{3.482239in}}%
\pgfpathlineto{\pgfqpoint{1.408495in}{3.693388in}}%
\pgfpathlineto{\pgfqpoint{1.408874in}{3.844035in}}%
\pgfpathlineto{\pgfqpoint{1.409443in}{3.619525in}}%
\pgfpathlineto{\pgfqpoint{1.409632in}{3.583222in}}%
\pgfpathlineto{\pgfqpoint{1.410106in}{3.741312in}}%
\pgfpathlineto{\pgfqpoint{1.410391in}{3.816758in}}%
\pgfpathlineto{\pgfqpoint{1.410864in}{3.621500in}}%
\pgfpathlineto{\pgfqpoint{1.411149in}{3.517907in}}%
\pgfpathlineto{\pgfqpoint{1.411622in}{3.715385in}}%
\pgfpathlineto{\pgfqpoint{1.412001in}{3.862505in}}%
\pgfpathlineto{\pgfqpoint{1.412570in}{3.618007in}}%
\pgfpathlineto{\pgfqpoint{1.412760in}{3.564734in}}%
\pgfpathlineto{\pgfqpoint{1.413328in}{3.741940in}}%
\pgfpathlineto{\pgfqpoint{1.413518in}{3.789157in}}%
\pgfpathlineto{\pgfqpoint{1.413897in}{3.663591in}}%
\pgfpathlineto{\pgfqpoint{1.414371in}{3.482147in}}%
\pgfpathlineto{\pgfqpoint{1.414844in}{3.717243in}}%
\pgfpathlineto{\pgfqpoint{1.415129in}{3.817753in}}%
\pgfpathlineto{\pgfqpoint{1.415603in}{3.569976in}}%
\pgfpathlineto{\pgfqpoint{1.415982in}{3.446960in}}%
\pgfpathlineto{\pgfqpoint{1.416550in}{3.641051in}}%
\pgfpathlineto{\pgfqpoint{1.416645in}{3.658022in}}%
\pgfpathlineto{\pgfqpoint{1.417024in}{3.544684in}}%
\pgfpathlineto{\pgfqpoint{1.417403in}{3.386640in}}%
\pgfpathlineto{\pgfqpoint{1.417972in}{3.612157in}}%
\pgfpathlineto{\pgfqpoint{1.418256in}{3.682934in}}%
\pgfpathlineto{\pgfqpoint{1.418730in}{3.504611in}}%
\pgfpathlineto{\pgfqpoint{1.419109in}{3.378521in}}%
\pgfpathlineto{\pgfqpoint{1.419677in}{3.566219in}}%
\pgfpathlineto{\pgfqpoint{1.419867in}{3.596407in}}%
\pgfpathlineto{\pgfqpoint{1.420246in}{3.483090in}}%
\pgfpathlineto{\pgfqpoint{1.420530in}{3.383640in}}%
\pgfpathlineto{\pgfqpoint{1.421099in}{3.584662in}}%
\pgfpathlineto{\pgfqpoint{1.421383in}{3.684857in}}%
\pgfpathlineto{\pgfqpoint{1.421857in}{3.456035in}}%
\pgfpathlineto{\pgfqpoint{1.422236in}{3.318073in}}%
\pgfpathlineto{\pgfqpoint{1.422805in}{3.534762in}}%
\pgfpathlineto{\pgfqpoint{1.422994in}{3.575764in}}%
\pgfpathlineto{\pgfqpoint{1.423468in}{3.428973in}}%
\pgfpathlineto{\pgfqpoint{1.423752in}{3.347584in}}%
\pgfpathlineto{\pgfqpoint{1.424226in}{3.547007in}}%
\pgfpathlineto{\pgfqpoint{1.424510in}{3.634129in}}%
\pgfpathlineto{\pgfqpoint{1.424984in}{3.431401in}}%
\pgfpathlineto{\pgfqpoint{1.425363in}{3.239677in}}%
\pgfpathlineto{\pgfqpoint{1.425932in}{3.481961in}}%
\pgfpathlineto{\pgfqpoint{1.426121in}{3.523211in}}%
\pgfpathlineto{\pgfqpoint{1.426595in}{3.354935in}}%
\pgfpathlineto{\pgfqpoint{1.426974in}{3.254453in}}%
\pgfpathlineto{\pgfqpoint{1.427448in}{3.422439in}}%
\pgfpathlineto{\pgfqpoint{1.427638in}{3.481756in}}%
\pgfpathlineto{\pgfqpoint{1.428017in}{3.305862in}}%
\pgfpathlineto{\pgfqpoint{1.428490in}{3.050205in}}%
\pgfpathlineto{\pgfqpoint{1.429154in}{3.282349in}}%
\pgfpathlineto{\pgfqpoint{1.429343in}{3.304130in}}%
\pgfpathlineto{\pgfqpoint{1.429628in}{3.213194in}}%
\pgfpathlineto{\pgfqpoint{1.430007in}{3.064594in}}%
\pgfpathlineto{\pgfqpoint{1.430575in}{3.245020in}}%
\pgfpathlineto{\pgfqpoint{1.430860in}{3.298542in}}%
\pgfpathlineto{\pgfqpoint{1.431239in}{3.122060in}}%
\pgfpathlineto{\pgfqpoint{1.431618in}{2.986252in}}%
\pgfpathlineto{\pgfqpoint{1.432091in}{3.183083in}}%
\pgfpathlineto{\pgfqpoint{1.432471in}{3.327662in}}%
\pgfpathlineto{\pgfqpoint{1.433039in}{3.118360in}}%
\pgfpathlineto{\pgfqpoint{1.433229in}{3.086944in}}%
\pgfpathlineto{\pgfqpoint{1.433608in}{3.220792in}}%
\pgfpathlineto{\pgfqpoint{1.433987in}{3.342718in}}%
\pgfpathlineto{\pgfqpoint{1.434461in}{3.138929in}}%
\pgfpathlineto{\pgfqpoint{1.434745in}{3.032389in}}%
\pgfpathlineto{\pgfqpoint{1.435313in}{3.275605in}}%
\pgfpathlineto{\pgfqpoint{1.437114in}{3.791775in}}%
\pgfpathlineto{\pgfqpoint{1.437304in}{3.738868in}}%
\pgfpathlineto{\pgfqpoint{1.439483in}{3.132069in}}%
\pgfpathlineto{\pgfqpoint{1.439862in}{3.250203in}}%
\pgfpathlineto{\pgfqpoint{1.440336in}{3.427322in}}%
\pgfpathlineto{\pgfqpoint{1.440905in}{3.241135in}}%
\pgfpathlineto{\pgfqpoint{1.440999in}{3.236295in}}%
\pgfpathlineto{\pgfqpoint{1.441094in}{3.253575in}}%
\pgfpathlineto{\pgfqpoint{1.441852in}{3.625996in}}%
\pgfpathlineto{\pgfqpoint{1.442516in}{3.402873in}}%
\pgfpathlineto{\pgfqpoint{1.442705in}{3.375476in}}%
\pgfpathlineto{\pgfqpoint{1.443084in}{3.501359in}}%
\pgfpathlineto{\pgfqpoint{1.443463in}{3.622566in}}%
\pgfpathlineto{\pgfqpoint{1.443937in}{3.440094in}}%
\pgfpathlineto{\pgfqpoint{1.444127in}{3.402397in}}%
\pgfpathlineto{\pgfqpoint{1.444506in}{3.536248in}}%
\pgfpathlineto{\pgfqpoint{1.444979in}{3.772746in}}%
\pgfpathlineto{\pgfqpoint{1.445548in}{3.577383in}}%
\pgfpathlineto{\pgfqpoint{1.445832in}{3.493287in}}%
\pgfpathlineto{\pgfqpoint{1.446306in}{3.692836in}}%
\pgfpathlineto{\pgfqpoint{1.446590in}{3.774661in}}%
\pgfpathlineto{\pgfqpoint{1.447159in}{3.600017in}}%
\pgfpathlineto{\pgfqpoint{1.447254in}{3.582115in}}%
\pgfpathlineto{\pgfqpoint{1.447633in}{3.708274in}}%
\pgfpathlineto{\pgfqpoint{1.448107in}{3.941103in}}%
\pgfpathlineto{\pgfqpoint{1.448675in}{3.715565in}}%
\pgfpathlineto{\pgfqpoint{1.448960in}{3.627472in}}%
\pgfpathlineto{\pgfqpoint{1.449433in}{3.817555in}}%
\pgfpathlineto{\pgfqpoint{1.449718in}{3.897837in}}%
\pgfpathlineto{\pgfqpoint{1.450286in}{3.746411in}}%
\pgfpathlineto{\pgfqpoint{1.450476in}{3.700198in}}%
\pgfpathlineto{\pgfqpoint{1.450855in}{3.843639in}}%
\pgfpathlineto{\pgfqpoint{1.451234in}{3.977511in}}%
\pgfpathlineto{\pgfqpoint{1.451708in}{3.805106in}}%
\pgfpathlineto{\pgfqpoint{1.452087in}{3.637286in}}%
\pgfpathlineto{\pgfqpoint{1.452655in}{3.833208in}}%
\pgfpathlineto{\pgfqpoint{1.452845in}{3.883321in}}%
\pgfpathlineto{\pgfqpoint{1.453319in}{3.731116in}}%
\pgfpathlineto{\pgfqpoint{1.453603in}{3.661033in}}%
\pgfpathlineto{\pgfqpoint{1.454077in}{3.833419in}}%
\pgfpathlineto{\pgfqpoint{1.454361in}{3.918954in}}%
\pgfpathlineto{\pgfqpoint{1.454835in}{3.721200in}}%
\pgfpathlineto{\pgfqpoint{1.455214in}{3.558266in}}%
\pgfpathlineto{\pgfqpoint{1.455783in}{3.788157in}}%
\pgfpathlineto{\pgfqpoint{1.456067in}{3.855681in}}%
\pgfpathlineto{\pgfqpoint{1.456541in}{3.700611in}}%
\pgfpathlineto{\pgfqpoint{1.456825in}{3.629619in}}%
\pgfpathlineto{\pgfqpoint{1.457299in}{3.821104in}}%
\pgfpathlineto{\pgfqpoint{1.457488in}{3.862204in}}%
\pgfpathlineto{\pgfqpoint{1.457867in}{3.727356in}}%
\pgfpathlineto{\pgfqpoint{1.458341in}{3.513322in}}%
\pgfpathlineto{\pgfqpoint{1.458910in}{3.712964in}}%
\pgfpathlineto{\pgfqpoint{1.459194in}{3.801813in}}%
\pgfpathlineto{\pgfqpoint{1.459668in}{3.596079in}}%
\pgfpathlineto{\pgfqpoint{1.459952in}{3.530011in}}%
\pgfpathlineto{\pgfqpoint{1.460521in}{3.676801in}}%
\pgfpathlineto{\pgfqpoint{1.460615in}{3.693591in}}%
\pgfpathlineto{\pgfqpoint{1.460900in}{3.612247in}}%
\pgfpathlineto{\pgfqpoint{1.461563in}{3.272103in}}%
\pgfpathlineto{\pgfqpoint{1.462132in}{3.517178in}}%
\pgfpathlineto{\pgfqpoint{1.462226in}{3.533765in}}%
\pgfpathlineto{\pgfqpoint{1.462606in}{3.442878in}}%
\pgfpathlineto{\pgfqpoint{1.463174in}{3.218609in}}%
\pgfpathlineto{\pgfqpoint{1.463743in}{3.373720in}}%
\pgfpathlineto{\pgfqpoint{1.463837in}{3.380033in}}%
\pgfpathlineto{\pgfqpoint{1.464027in}{3.336901in}}%
\pgfpathlineto{\pgfqpoint{1.464690in}{3.034505in}}%
\pgfpathlineto{\pgfqpoint{1.465164in}{3.234519in}}%
\pgfpathlineto{\pgfqpoint{1.465448in}{3.349958in}}%
\pgfpathlineto{\pgfqpoint{1.466017in}{3.105015in}}%
\pgfpathlineto{\pgfqpoint{1.466301in}{3.050229in}}%
\pgfpathlineto{\pgfqpoint{1.466775in}{3.209964in}}%
\pgfpathlineto{\pgfqpoint{1.466965in}{3.250126in}}%
\pgfpathlineto{\pgfqpoint{1.467439in}{3.072559in}}%
\pgfpathlineto{\pgfqpoint{1.467723in}{2.993588in}}%
\pgfpathlineto{\pgfqpoint{1.468197in}{3.179350in}}%
\pgfpathlineto{\pgfqpoint{1.468576in}{3.346805in}}%
\pgfpathlineto{\pgfqpoint{1.469144in}{3.125887in}}%
\pgfpathlineto{\pgfqpoint{1.469429in}{3.064921in}}%
\pgfpathlineto{\pgfqpoint{1.469902in}{3.251747in}}%
\pgfpathlineto{\pgfqpoint{1.470187in}{3.310775in}}%
\pgfpathlineto{\pgfqpoint{1.470660in}{3.130876in}}%
\pgfpathlineto{\pgfqpoint{1.470850in}{3.085801in}}%
\pgfpathlineto{\pgfqpoint{1.471324in}{3.260034in}}%
\pgfpathlineto{\pgfqpoint{1.471798in}{3.439431in}}%
\pgfpathlineto{\pgfqpoint{1.472271in}{3.193008in}}%
\pgfpathlineto{\pgfqpoint{1.472556in}{3.107717in}}%
\pgfpathlineto{\pgfqpoint{1.473124in}{3.317784in}}%
\pgfpathlineto{\pgfqpoint{1.473314in}{3.359500in}}%
\pgfpathlineto{\pgfqpoint{1.473693in}{3.229312in}}%
\pgfpathlineto{\pgfqpoint{1.474072in}{3.131271in}}%
\pgfpathlineto{\pgfqpoint{1.474546in}{3.309803in}}%
\pgfpathlineto{\pgfqpoint{1.474830in}{3.411426in}}%
\pgfpathlineto{\pgfqpoint{1.475304in}{3.223774in}}%
\pgfpathlineto{\pgfqpoint{1.475683in}{3.094609in}}%
\pgfpathlineto{\pgfqpoint{1.476157in}{3.278542in}}%
\pgfpathlineto{\pgfqpoint{1.476441in}{3.346926in}}%
\pgfpathlineto{\pgfqpoint{1.477010in}{3.191933in}}%
\pgfpathlineto{\pgfqpoint{1.477199in}{3.169494in}}%
\pgfpathlineto{\pgfqpoint{1.477483in}{3.269392in}}%
\pgfpathlineto{\pgfqpoint{1.477957in}{3.495962in}}%
\pgfpathlineto{\pgfqpoint{1.478526in}{3.268450in}}%
\pgfpathlineto{\pgfqpoint{1.478810in}{3.194307in}}%
\pgfpathlineto{\pgfqpoint{1.479284in}{3.386712in}}%
\pgfpathlineto{\pgfqpoint{1.479663in}{3.517969in}}%
\pgfpathlineto{\pgfqpoint{1.480232in}{3.359607in}}%
\pgfpathlineto{\pgfqpoint{1.480326in}{3.349725in}}%
\pgfpathlineto{\pgfqpoint{1.480516in}{3.395008in}}%
\pgfpathlineto{\pgfqpoint{1.481085in}{3.682856in}}%
\pgfpathlineto{\pgfqpoint{1.481653in}{3.479469in}}%
\pgfpathlineto{\pgfqpoint{1.481843in}{3.448414in}}%
\pgfpathlineto{\pgfqpoint{1.482127in}{3.606223in}}%
\pgfpathlineto{\pgfqpoint{1.482790in}{4.203574in}}%
\pgfpathlineto{\pgfqpoint{1.483548in}{3.941200in}}%
\pgfpathlineto{\pgfqpoint{1.483643in}{3.937439in}}%
\pgfpathlineto{\pgfqpoint{1.483738in}{3.952316in}}%
\pgfpathlineto{\pgfqpoint{1.484117in}{4.044967in}}%
\pgfpathlineto{\pgfqpoint{1.484401in}{3.866310in}}%
\pgfpathlineto{\pgfqpoint{1.484970in}{3.335982in}}%
\pgfpathlineto{\pgfqpoint{1.485538in}{3.677423in}}%
\pgfpathlineto{\pgfqpoint{1.485918in}{3.836315in}}%
\pgfpathlineto{\pgfqpoint{1.486581in}{3.644876in}}%
\pgfpathlineto{\pgfqpoint{1.486676in}{3.642748in}}%
\pgfpathlineto{\pgfqpoint{1.486770in}{3.662010in}}%
\pgfpathlineto{\pgfqpoint{1.487434in}{3.888253in}}%
\pgfpathlineto{\pgfqpoint{1.487908in}{3.714995in}}%
\pgfpathlineto{\pgfqpoint{1.488192in}{3.593962in}}%
\pgfpathlineto{\pgfqpoint{1.488760in}{3.868048in}}%
\pgfpathlineto{\pgfqpoint{1.489045in}{3.948891in}}%
\pgfpathlineto{\pgfqpoint{1.489519in}{3.802114in}}%
\pgfpathlineto{\pgfqpoint{1.489803in}{3.698477in}}%
\pgfpathlineto{\pgfqpoint{1.490466in}{3.884999in}}%
\pgfpathlineto{\pgfqpoint{1.490561in}{3.884826in}}%
\pgfpathlineto{\pgfqpoint{1.491224in}{3.596755in}}%
\pgfpathlineto{\pgfqpoint{1.491414in}{3.569178in}}%
\pgfpathlineto{\pgfqpoint{1.491698in}{3.709808in}}%
\pgfpathlineto{\pgfqpoint{1.492267in}{3.895065in}}%
\pgfpathlineto{\pgfqpoint{1.492741in}{3.689544in}}%
\pgfpathlineto{\pgfqpoint{1.492930in}{3.616274in}}%
\pgfpathlineto{\pgfqpoint{1.493593in}{3.811178in}}%
\pgfpathlineto{\pgfqpoint{1.493688in}{3.823126in}}%
\pgfpathlineto{\pgfqpoint{1.493972in}{3.753679in}}%
\pgfpathlineto{\pgfqpoint{1.494446in}{3.566361in}}%
\pgfpathlineto{\pgfqpoint{1.495015in}{3.766612in}}%
\pgfpathlineto{\pgfqpoint{1.495394in}{3.893656in}}%
\pgfpathlineto{\pgfqpoint{1.495868in}{3.656203in}}%
\pgfpathlineto{\pgfqpoint{1.496152in}{3.587561in}}%
\pgfpathlineto{\pgfqpoint{1.496626in}{3.745505in}}%
\pgfpathlineto{\pgfqpoint{1.496815in}{3.787327in}}%
\pgfpathlineto{\pgfqpoint{1.497289in}{3.639945in}}%
\pgfpathlineto{\pgfqpoint{1.497668in}{3.550884in}}%
\pgfpathlineto{\pgfqpoint{1.498047in}{3.698644in}}%
\pgfpathlineto{\pgfqpoint{1.498426in}{3.859748in}}%
\pgfpathlineto{\pgfqpoint{1.498995in}{3.619645in}}%
\pgfpathlineto{\pgfqpoint{1.499279in}{3.505953in}}%
\pgfpathlineto{\pgfqpoint{1.499943in}{3.685378in}}%
\pgfpathlineto{\pgfqpoint{1.500416in}{3.533827in}}%
\pgfpathlineto{\pgfqpoint{1.500795in}{3.421331in}}%
\pgfpathlineto{\pgfqpoint{1.501269in}{3.597413in}}%
\pgfpathlineto{\pgfqpoint{1.501554in}{3.680655in}}%
\pgfpathlineto{\pgfqpoint{1.502027in}{3.493361in}}%
\pgfpathlineto{\pgfqpoint{1.502501in}{3.323888in}}%
\pgfpathlineto{\pgfqpoint{1.503070in}{3.508680in}}%
\pgfpathlineto{\pgfqpoint{1.503165in}{3.520003in}}%
\pgfpathlineto{\pgfqpoint{1.503449in}{3.448906in}}%
\pgfpathlineto{\pgfqpoint{1.503923in}{3.291345in}}%
\pgfpathlineto{\pgfqpoint{1.504397in}{3.439130in}}%
\pgfpathlineto{\pgfqpoint{1.504681in}{3.533568in}}%
\pgfpathlineto{\pgfqpoint{1.505155in}{3.339951in}}%
\pgfpathlineto{\pgfqpoint{1.505534in}{3.167706in}}%
\pgfpathlineto{\pgfqpoint{1.506197in}{3.379941in}}%
\pgfpathlineto{\pgfqpoint{1.506292in}{3.395805in}}%
\pgfpathlineto{\pgfqpoint{1.506671in}{3.300307in}}%
\pgfpathlineto{\pgfqpoint{1.507050in}{3.184928in}}%
\pgfpathlineto{\pgfqpoint{1.507618in}{3.365825in}}%
\pgfpathlineto{\pgfqpoint{1.507808in}{3.421501in}}%
\pgfpathlineto{\pgfqpoint{1.508282in}{3.235552in}}%
\pgfpathlineto{\pgfqpoint{1.508756in}{3.058855in}}%
\pgfpathlineto{\pgfqpoint{1.509229in}{3.275453in}}%
\pgfpathlineto{\pgfqpoint{1.509514in}{3.356150in}}%
\pgfpathlineto{\pgfqpoint{1.510082in}{3.168744in}}%
\pgfpathlineto{\pgfqpoint{1.510272in}{3.151069in}}%
\pgfpathlineto{\pgfqpoint{1.510556in}{3.237059in}}%
\pgfpathlineto{\pgfqpoint{1.510935in}{3.384454in}}%
\pgfpathlineto{\pgfqpoint{1.511504in}{3.187866in}}%
\pgfpathlineto{\pgfqpoint{1.511883in}{3.080551in}}%
\pgfpathlineto{\pgfqpoint{1.512357in}{3.274185in}}%
\pgfpathlineto{\pgfqpoint{1.512641in}{3.333234in}}%
\pgfpathlineto{\pgfqpoint{1.513210in}{3.179101in}}%
\pgfpathlineto{\pgfqpoint{1.513399in}{3.151114in}}%
\pgfpathlineto{\pgfqpoint{1.513778in}{3.272649in}}%
\pgfpathlineto{\pgfqpoint{1.514157in}{3.390828in}}%
\pgfpathlineto{\pgfqpoint{1.514726in}{3.218886in}}%
\pgfpathlineto{\pgfqpoint{1.514821in}{3.202281in}}%
\pgfpathlineto{\pgfqpoint{1.515200in}{3.318114in}}%
\pgfpathlineto{\pgfqpoint{1.516905in}{3.749685in}}%
\pgfpathlineto{\pgfqpoint{1.517284in}{3.892816in}}%
\pgfpathlineto{\pgfqpoint{1.517853in}{3.716333in}}%
\pgfpathlineto{\pgfqpoint{1.518137in}{3.660362in}}%
\pgfpathlineto{\pgfqpoint{1.518516in}{3.855581in}}%
\pgfpathlineto{\pgfqpoint{1.518895in}{4.000090in}}%
\pgfpathlineto{\pgfqpoint{1.519464in}{3.771496in}}%
\pgfpathlineto{\pgfqpoint{1.519748in}{3.709004in}}%
\pgfpathlineto{\pgfqpoint{1.520317in}{3.855035in}}%
\pgfpathlineto{\pgfqpoint{1.520412in}{3.866742in}}%
\pgfpathlineto{\pgfqpoint{1.520696in}{3.792127in}}%
\pgfpathlineto{\pgfqpoint{1.521265in}{3.581265in}}%
\pgfpathlineto{\pgfqpoint{1.521738in}{3.803967in}}%
\pgfpathlineto{\pgfqpoint{1.522117in}{3.916756in}}%
\pgfpathlineto{\pgfqpoint{1.522591in}{3.750556in}}%
\pgfpathlineto{\pgfqpoint{1.522876in}{3.675378in}}%
\pgfpathlineto{\pgfqpoint{1.523349in}{3.833657in}}%
\pgfpathlineto{\pgfqpoint{1.523539in}{3.880535in}}%
\pgfpathlineto{\pgfqpoint{1.524107in}{3.729378in}}%
\pgfpathlineto{\pgfqpoint{1.524392in}{3.675620in}}%
\pgfpathlineto{\pgfqpoint{1.524771in}{3.827448in}}%
\pgfpathlineto{\pgfqpoint{1.525245in}{4.007333in}}%
\pgfpathlineto{\pgfqpoint{1.525718in}{3.766823in}}%
\pgfpathlineto{\pgfqpoint{1.526003in}{3.695573in}}%
\pgfpathlineto{\pgfqpoint{1.526571in}{3.854920in}}%
\pgfpathlineto{\pgfqpoint{1.526761in}{3.877128in}}%
\pgfpathlineto{\pgfqpoint{1.527045in}{3.769845in}}%
\pgfpathlineto{\pgfqpoint{1.527424in}{3.615922in}}%
\pgfpathlineto{\pgfqpoint{1.527898in}{3.912178in}}%
\pgfpathlineto{\pgfqpoint{1.528372in}{4.300282in}}%
\pgfpathlineto{\pgfqpoint{1.529035in}{4.014551in}}%
\pgfpathlineto{\pgfqpoint{1.530646in}{3.373553in}}%
\pgfpathlineto{\pgfqpoint{1.529793in}{4.049099in}}%
\pgfpathlineto{\pgfqpoint{1.531120in}{3.672315in}}%
\pgfpathlineto{\pgfqpoint{1.531404in}{3.803252in}}%
\pgfpathlineto{\pgfqpoint{1.531973in}{3.562784in}}%
\pgfpathlineto{\pgfqpoint{1.532352in}{3.458133in}}%
\pgfpathlineto{\pgfqpoint{1.532826in}{3.612914in}}%
\pgfpathlineto{\pgfqpoint{1.533110in}{3.660788in}}%
\pgfpathlineto{\pgfqpoint{1.533489in}{3.527717in}}%
\pgfpathlineto{\pgfqpoint{1.533773in}{3.435142in}}%
\pgfpathlineto{\pgfqpoint{1.534342in}{3.636049in}}%
\pgfpathlineto{\pgfqpoint{1.534531in}{3.676730in}}%
\pgfpathlineto{\pgfqpoint{1.534911in}{3.568199in}}%
\pgfpathlineto{\pgfqpoint{1.535479in}{3.315025in}}%
\pgfpathlineto{\pgfqpoint{1.536048in}{3.522752in}}%
\pgfpathlineto{\pgfqpoint{1.536237in}{3.555776in}}%
\pgfpathlineto{\pgfqpoint{1.536616in}{3.417511in}}%
\pgfpathlineto{\pgfqpoint{1.536995in}{3.327687in}}%
\pgfpathlineto{\pgfqpoint{1.537469in}{3.492988in}}%
\pgfpathlineto{\pgfqpoint{1.537659in}{3.535504in}}%
\pgfpathlineto{\pgfqpoint{1.538133in}{3.359763in}}%
\pgfpathlineto{\pgfqpoint{1.538606in}{3.168801in}}%
\pgfpathlineto{\pgfqpoint{1.539175in}{3.390444in}}%
\pgfpathlineto{\pgfqpoint{1.539364in}{3.421701in}}%
\pgfpathlineto{\pgfqpoint{1.539744in}{3.315140in}}%
\pgfpathlineto{\pgfqpoint{1.540123in}{3.220542in}}%
\pgfpathlineto{\pgfqpoint{1.540691in}{3.356211in}}%
\pgfpathlineto{\pgfqpoint{1.540881in}{3.391432in}}%
\pgfpathlineto{\pgfqpoint{1.541260in}{3.242147in}}%
\pgfpathlineto{\pgfqpoint{1.541639in}{3.079138in}}%
\pgfpathlineto{\pgfqpoint{1.542207in}{3.280605in}}%
\pgfpathlineto{\pgfqpoint{1.542492in}{3.388786in}}%
\pgfpathlineto{\pgfqpoint{1.543155in}{3.199645in}}%
\pgfpathlineto{\pgfqpoint{1.543345in}{3.188635in}}%
\pgfpathlineto{\pgfqpoint{1.543629in}{3.255791in}}%
\pgfpathlineto{\pgfqpoint{1.544008in}{3.378918in}}%
\pgfpathlineto{\pgfqpoint{1.544482in}{3.178452in}}%
\pgfpathlineto{\pgfqpoint{1.544766in}{3.085631in}}%
\pgfpathlineto{\pgfqpoint{1.545240in}{3.248064in}}%
\pgfpathlineto{\pgfqpoint{1.545619in}{3.431374in}}%
\pgfpathlineto{\pgfqpoint{1.546282in}{3.225812in}}%
\pgfpathlineto{\pgfqpoint{1.546472in}{3.197289in}}%
\pgfpathlineto{\pgfqpoint{1.546851in}{3.300323in}}%
\pgfpathlineto{\pgfqpoint{1.547135in}{3.384237in}}%
\pgfpathlineto{\pgfqpoint{1.547704in}{3.195882in}}%
\pgfpathlineto{\pgfqpoint{1.547988in}{3.139698in}}%
\pgfpathlineto{\pgfqpoint{1.548367in}{3.306666in}}%
\pgfpathlineto{\pgfqpoint{1.548841in}{3.487526in}}%
\pgfpathlineto{\pgfqpoint{1.549409in}{3.283344in}}%
\pgfpathlineto{\pgfqpoint{1.549599in}{3.237841in}}%
\pgfpathlineto{\pgfqpoint{1.550073in}{3.382744in}}%
\pgfpathlineto{\pgfqpoint{1.550357in}{3.416847in}}%
\pgfpathlineto{\pgfqpoint{1.550736in}{3.318269in}}%
\pgfpathlineto{\pgfqpoint{1.551115in}{3.220331in}}%
\pgfpathlineto{\pgfqpoint{1.551494in}{3.388891in}}%
\pgfpathlineto{\pgfqpoint{1.551968in}{3.577757in}}%
\pgfpathlineto{\pgfqpoint{1.552442in}{3.385532in}}%
\pgfpathlineto{\pgfqpoint{1.552726in}{3.299037in}}%
\pgfpathlineto{\pgfqpoint{1.553295in}{3.491361in}}%
\pgfpathlineto{\pgfqpoint{1.553484in}{3.523107in}}%
\pgfpathlineto{\pgfqpoint{1.553958in}{3.416490in}}%
\pgfpathlineto{\pgfqpoint{1.554148in}{3.377188in}}%
\pgfpathlineto{\pgfqpoint{1.554527in}{3.513677in}}%
\pgfpathlineto{\pgfqpoint{1.555095in}{3.762451in}}%
\pgfpathlineto{\pgfqpoint{1.555664in}{3.563808in}}%
\pgfpathlineto{\pgfqpoint{1.555853in}{3.532226in}}%
\pgfpathlineto{\pgfqpoint{1.556232in}{3.656691in}}%
\pgfpathlineto{\pgfqpoint{1.556612in}{3.783768in}}%
\pgfpathlineto{\pgfqpoint{1.557275in}{3.640049in}}%
\pgfpathlineto{\pgfqpoint{1.557370in}{3.626927in}}%
\pgfpathlineto{\pgfqpoint{1.557654in}{3.693047in}}%
\pgfpathlineto{\pgfqpoint{1.558128in}{3.890506in}}%
\pgfpathlineto{\pgfqpoint{1.558696in}{3.670562in}}%
\pgfpathlineto{\pgfqpoint{1.559075in}{3.554014in}}%
\pgfpathlineto{\pgfqpoint{1.559644in}{3.755458in}}%
\pgfpathlineto{\pgfqpoint{1.559834in}{3.774413in}}%
\pgfpathlineto{\pgfqpoint{1.560213in}{3.657172in}}%
\pgfpathlineto{\pgfqpoint{1.560497in}{3.600412in}}%
\pgfpathlineto{\pgfqpoint{1.560876in}{3.709846in}}%
\pgfpathlineto{\pgfqpoint{1.561350in}{3.918245in}}%
\pgfpathlineto{\pgfqpoint{1.561824in}{3.711637in}}%
\pgfpathlineto{\pgfqpoint{1.562203in}{3.570923in}}%
\pgfpathlineto{\pgfqpoint{1.562676in}{3.794971in}}%
\pgfpathlineto{\pgfqpoint{1.562961in}{3.866894in}}%
\pgfpathlineto{\pgfqpoint{1.563529in}{3.693575in}}%
\pgfpathlineto{\pgfqpoint{1.563719in}{3.658974in}}%
\pgfpathlineto{\pgfqpoint{1.564193in}{3.813680in}}%
\pgfpathlineto{\pgfqpoint{1.564477in}{3.860625in}}%
\pgfpathlineto{\pgfqpoint{1.564856in}{3.699919in}}%
\pgfpathlineto{\pgfqpoint{1.565330in}{3.464239in}}%
\pgfpathlineto{\pgfqpoint{1.565898in}{3.665148in}}%
\pgfpathlineto{\pgfqpoint{1.566088in}{3.697711in}}%
\pgfpathlineto{\pgfqpoint{1.566562in}{3.553175in}}%
\pgfpathlineto{\pgfqpoint{1.566846in}{3.473542in}}%
\pgfpathlineto{\pgfqpoint{1.567415in}{3.643310in}}%
\pgfpathlineto{\pgfqpoint{1.567604in}{3.668888in}}%
\pgfpathlineto{\pgfqpoint{1.567888in}{3.570792in}}%
\pgfpathlineto{\pgfqpoint{1.568457in}{3.329006in}}%
\pgfpathlineto{\pgfqpoint{1.568931in}{3.516258in}}%
\pgfpathlineto{\pgfqpoint{1.569215in}{3.608955in}}%
\pgfpathlineto{\pgfqpoint{1.569784in}{3.445029in}}%
\pgfpathlineto{\pgfqpoint{1.570068in}{3.392816in}}%
\pgfpathlineto{\pgfqpoint{1.570447in}{3.519690in}}%
\pgfpathlineto{\pgfqpoint{1.570637in}{3.574916in}}%
\pgfpathlineto{\pgfqpoint{1.571110in}{3.428707in}}%
\pgfpathlineto{\pgfqpoint{1.571584in}{3.262454in}}%
\pgfpathlineto{\pgfqpoint{1.572058in}{3.461877in}}%
\pgfpathlineto{\pgfqpoint{1.572342in}{3.562397in}}%
\pgfpathlineto{\pgfqpoint{1.572911in}{3.352104in}}%
\pgfpathlineto{\pgfqpoint{1.573195in}{3.423996in}}%
\pgfpathlineto{\pgfqpoint{1.573953in}{3.887159in}}%
\pgfpathlineto{\pgfqpoint{1.574522in}{3.630613in}}%
\pgfpathlineto{\pgfqpoint{1.574711in}{3.587809in}}%
\pgfpathlineto{\pgfqpoint{1.575280in}{3.727892in}}%
\pgfpathlineto{\pgfqpoint{1.575564in}{3.608116in}}%
\pgfpathlineto{\pgfqpoint{1.576228in}{3.116713in}}%
\pgfpathlineto{\pgfqpoint{1.576891in}{3.377781in}}%
\pgfpathlineto{\pgfqpoint{1.576986in}{3.392865in}}%
\pgfpathlineto{\pgfqpoint{1.577365in}{3.294347in}}%
\pgfpathlineto{\pgfqpoint{1.577839in}{3.133830in}}%
\pgfpathlineto{\pgfqpoint{1.578313in}{3.321998in}}%
\pgfpathlineto{\pgfqpoint{1.578597in}{3.442122in}}%
\pgfpathlineto{\pgfqpoint{1.579165in}{3.242507in}}%
\pgfpathlineto{\pgfqpoint{1.579544in}{3.158532in}}%
\pgfpathlineto{\pgfqpoint{1.580018in}{3.336001in}}%
\pgfpathlineto{\pgfqpoint{1.580208in}{3.360792in}}%
\pgfpathlineto{\pgfqpoint{1.580587in}{3.250477in}}%
\pgfpathlineto{\pgfqpoint{1.580966in}{3.142615in}}%
\pgfpathlineto{\pgfqpoint{1.581440in}{3.338210in}}%
\pgfpathlineto{\pgfqpoint{1.581819in}{3.469574in}}%
\pgfpathlineto{\pgfqpoint{1.582293in}{3.278568in}}%
\pgfpathlineto{\pgfqpoint{1.582577in}{3.185927in}}%
\pgfpathlineto{\pgfqpoint{1.583145in}{3.352547in}}%
\pgfpathlineto{\pgfqpoint{1.583430in}{3.398860in}}%
\pgfpathlineto{\pgfqpoint{1.583904in}{3.279567in}}%
\pgfpathlineto{\pgfqpoint{1.584093in}{3.256898in}}%
\pgfpathlineto{\pgfqpoint{1.584472in}{3.369761in}}%
\pgfpathlineto{\pgfqpoint{1.584946in}{3.542709in}}%
\pgfpathlineto{\pgfqpoint{1.585420in}{3.325862in}}%
\pgfpathlineto{\pgfqpoint{1.585799in}{3.207299in}}%
\pgfpathlineto{\pgfqpoint{1.586367in}{3.401714in}}%
\pgfpathlineto{\pgfqpoint{1.586462in}{3.407644in}}%
\pgfpathlineto{\pgfqpoint{1.586841in}{3.363209in}}%
\pgfpathlineto{\pgfqpoint{1.587220in}{3.266297in}}%
\pgfpathlineto{\pgfqpoint{1.587599in}{3.406016in}}%
\pgfpathlineto{\pgfqpoint{1.588168in}{3.608435in}}%
\pgfpathlineto{\pgfqpoint{1.588642in}{3.432681in}}%
\pgfpathlineto{\pgfqpoint{1.588831in}{3.396243in}}%
\pgfpathlineto{\pgfqpoint{1.589210in}{3.532617in}}%
\pgfpathlineto{\pgfqpoint{1.589779in}{3.729464in}}%
\pgfpathlineto{\pgfqpoint{1.590442in}{3.605730in}}%
\pgfpathlineto{\pgfqpoint{1.591200in}{3.887699in}}%
\pgfpathlineto{\pgfqpoint{1.591674in}{3.722026in}}%
\pgfpathlineto{\pgfqpoint{1.592053in}{3.597597in}}%
\pgfpathlineto{\pgfqpoint{1.592622in}{3.796759in}}%
\pgfpathlineto{\pgfqpoint{1.592906in}{3.842629in}}%
\pgfpathlineto{\pgfqpoint{1.593285in}{3.725388in}}%
\pgfpathlineto{\pgfqpoint{1.593570in}{3.664982in}}%
\pgfpathlineto{\pgfqpoint{1.594043in}{3.815204in}}%
\pgfpathlineto{\pgfqpoint{1.594328in}{3.871300in}}%
\pgfpathlineto{\pgfqpoint{1.594707in}{3.717966in}}%
\pgfpathlineto{\pgfqpoint{1.595086in}{3.548974in}}%
\pgfpathlineto{\pgfqpoint{1.595654in}{3.752049in}}%
\pgfpathlineto{\pgfqpoint{1.596033in}{3.853249in}}%
\pgfpathlineto{\pgfqpoint{1.596602in}{3.702004in}}%
\pgfpathlineto{\pgfqpoint{1.596792in}{3.678333in}}%
\pgfpathlineto{\pgfqpoint{1.597171in}{3.790802in}}%
\pgfpathlineto{\pgfqpoint{1.597360in}{3.832711in}}%
\pgfpathlineto{\pgfqpoint{1.597834in}{3.706305in}}%
\pgfpathlineto{\pgfqpoint{1.598308in}{3.537574in}}%
\pgfpathlineto{\pgfqpoint{1.598782in}{3.721596in}}%
\pgfpathlineto{\pgfqpoint{1.599161in}{3.821047in}}%
\pgfpathlineto{\pgfqpoint{1.599634in}{3.665394in}}%
\pgfpathlineto{\pgfqpoint{1.599919in}{3.581879in}}%
\pgfpathlineto{\pgfqpoint{1.600487in}{3.736438in}}%
\pgfpathlineto{\pgfqpoint{1.600582in}{3.740384in}}%
\pgfpathlineto{\pgfqpoint{1.600772in}{3.710160in}}%
\pgfpathlineto{\pgfqpoint{1.601435in}{3.470402in}}%
\pgfpathlineto{\pgfqpoint{1.601909in}{3.668256in}}%
\pgfpathlineto{\pgfqpoint{1.602288in}{3.778052in}}%
\pgfpathlineto{\pgfqpoint{1.602762in}{3.597003in}}%
\pgfpathlineto{\pgfqpoint{1.603046in}{3.525740in}}%
\pgfpathlineto{\pgfqpoint{1.603615in}{3.654697in}}%
\pgfpathlineto{\pgfqpoint{1.603804in}{3.668209in}}%
\pgfpathlineto{\pgfqpoint{1.604088in}{3.582927in}}%
\pgfpathlineto{\pgfqpoint{1.604562in}{3.436878in}}%
\pgfpathlineto{\pgfqpoint{1.605036in}{3.630577in}}%
\pgfpathlineto{\pgfqpoint{1.605415in}{3.753832in}}%
\pgfpathlineto{\pgfqpoint{1.605889in}{3.580014in}}%
\pgfpathlineto{\pgfqpoint{1.606173in}{3.495246in}}%
\pgfpathlineto{\pgfqpoint{1.606742in}{3.642093in}}%
\pgfpathlineto{\pgfqpoint{1.606931in}{3.672340in}}%
\pgfpathlineto{\pgfqpoint{1.607310in}{3.542867in}}%
\pgfpathlineto{\pgfqpoint{1.607689in}{3.451578in}}%
\pgfpathlineto{\pgfqpoint{1.608163in}{3.644919in}}%
\pgfpathlineto{\pgfqpoint{1.608542in}{3.773472in}}%
\pgfpathlineto{\pgfqpoint{1.609016in}{3.564973in}}%
\pgfpathlineto{\pgfqpoint{1.609395in}{3.448070in}}%
\pgfpathlineto{\pgfqpoint{1.609964in}{3.616106in}}%
\pgfpathlineto{\pgfqpoint{1.610058in}{3.627253in}}%
\pgfpathlineto{\pgfqpoint{1.610343in}{3.550627in}}%
\pgfpathlineto{\pgfqpoint{1.610817in}{3.391451in}}%
\pgfpathlineto{\pgfqpoint{1.611385in}{3.579098in}}%
\pgfpathlineto{\pgfqpoint{1.611575in}{3.618357in}}%
\pgfpathlineto{\pgfqpoint{1.612049in}{3.491531in}}%
\pgfpathlineto{\pgfqpoint{1.612522in}{3.308928in}}%
\pgfpathlineto{\pgfqpoint{1.613091in}{3.486408in}}%
\pgfpathlineto{\pgfqpoint{1.613280in}{3.520759in}}%
\pgfpathlineto{\pgfqpoint{1.613754in}{3.384636in}}%
\pgfpathlineto{\pgfqpoint{1.613944in}{3.349448in}}%
\pgfpathlineto{\pgfqpoint{1.614418in}{3.491991in}}%
\pgfpathlineto{\pgfqpoint{1.614797in}{3.594257in}}%
\pgfpathlineto{\pgfqpoint{1.615271in}{3.411769in}}%
\pgfpathlineto{\pgfqpoint{1.615650in}{3.289311in}}%
\pgfpathlineto{\pgfqpoint{1.616218in}{3.483411in}}%
\pgfpathlineto{\pgfqpoint{1.616408in}{3.528930in}}%
\pgfpathlineto{\pgfqpoint{1.616976in}{3.388104in}}%
\pgfpathlineto{\pgfqpoint{1.617071in}{3.375467in}}%
\pgfpathlineto{\pgfqpoint{1.617450in}{3.451908in}}%
\pgfpathlineto{\pgfqpoint{1.617829in}{3.550935in}}%
\pgfpathlineto{\pgfqpoint{1.618492in}{3.443798in}}%
\pgfpathlineto{\pgfqpoint{1.618777in}{3.513846in}}%
\pgfpathlineto{\pgfqpoint{1.619630in}{3.986661in}}%
\pgfpathlineto{\pgfqpoint{1.620198in}{3.772273in}}%
\pgfpathlineto{\pgfqpoint{1.621335in}{3.399779in}}%
\pgfpathlineto{\pgfqpoint{1.621714in}{3.300809in}}%
\pgfpathlineto{\pgfqpoint{1.622188in}{3.447256in}}%
\pgfpathlineto{\pgfqpoint{1.622757in}{3.682619in}}%
\pgfpathlineto{\pgfqpoint{1.623325in}{3.521696in}}%
\pgfpathlineto{\pgfqpoint{1.623420in}{3.511857in}}%
\pgfpathlineto{\pgfqpoint{1.623705in}{3.574152in}}%
\pgfpathlineto{\pgfqpoint{1.624178in}{3.701326in}}%
\pgfpathlineto{\pgfqpoint{1.624652in}{3.547602in}}%
\pgfpathlineto{\pgfqpoint{1.625031in}{3.436923in}}%
\pgfpathlineto{\pgfqpoint{1.625505in}{3.629521in}}%
\pgfpathlineto{\pgfqpoint{1.625884in}{3.737091in}}%
\pgfpathlineto{\pgfqpoint{1.626453in}{3.572705in}}%
\pgfpathlineto{\pgfqpoint{1.626547in}{3.556466in}}%
\pgfpathlineto{\pgfqpoint{1.627021in}{3.662332in}}%
\pgfpathlineto{\pgfqpoint{1.627400in}{3.729420in}}%
\pgfpathlineto{\pgfqpoint{1.627779in}{3.595554in}}%
\pgfpathlineto{\pgfqpoint{1.628158in}{3.499901in}}%
\pgfpathlineto{\pgfqpoint{1.628632in}{3.682140in}}%
\pgfpathlineto{\pgfqpoint{1.629011in}{3.825831in}}%
\pgfpathlineto{\pgfqpoint{1.629580in}{3.617283in}}%
\pgfpathlineto{\pgfqpoint{1.629769in}{3.581604in}}%
\pgfpathlineto{\pgfqpoint{1.630243in}{3.700150in}}%
\pgfpathlineto{\pgfqpoint{1.630528in}{3.737559in}}%
\pgfpathlineto{\pgfqpoint{1.630907in}{3.622276in}}%
\pgfpathlineto{\pgfqpoint{1.631286in}{3.510833in}}%
\pgfpathlineto{\pgfqpoint{1.631759in}{3.719897in}}%
\pgfpathlineto{\pgfqpoint{1.632139in}{3.848661in}}%
\pgfpathlineto{\pgfqpoint{1.632707in}{3.656276in}}%
\pgfpathlineto{\pgfqpoint{1.632991in}{3.616585in}}%
\pgfpathlineto{\pgfqpoint{1.633465in}{3.730466in}}%
\pgfpathlineto{\pgfqpoint{1.633655in}{3.751935in}}%
\pgfpathlineto{\pgfqpoint{1.634034in}{3.663434in}}%
\pgfpathlineto{\pgfqpoint{1.634413in}{3.558461in}}%
\pgfpathlineto{\pgfqpoint{1.634887in}{3.749679in}}%
\pgfpathlineto{\pgfqpoint{1.635266in}{3.880073in}}%
\pgfpathlineto{\pgfqpoint{1.635740in}{3.700777in}}%
\pgfpathlineto{\pgfqpoint{1.636119in}{3.600930in}}%
\pgfpathlineto{\pgfqpoint{1.636687in}{3.768429in}}%
\pgfpathlineto{\pgfqpoint{1.636782in}{3.776397in}}%
\pgfpathlineto{\pgfqpoint{1.637066in}{3.722901in}}%
\pgfpathlineto{\pgfqpoint{1.637540in}{3.596571in}}%
\pgfpathlineto{\pgfqpoint{1.638014in}{3.743646in}}%
\pgfpathlineto{\pgfqpoint{1.638393in}{3.862192in}}%
\pgfpathlineto{\pgfqpoint{1.638867in}{3.663031in}}%
\pgfpathlineto{\pgfqpoint{1.639246in}{3.534222in}}%
\pgfpathlineto{\pgfqpoint{1.639814in}{3.718218in}}%
\pgfpathlineto{\pgfqpoint{1.640004in}{3.734788in}}%
\pgfpathlineto{\pgfqpoint{1.640383in}{3.627526in}}%
\pgfpathlineto{\pgfqpoint{1.640762in}{3.573580in}}%
\pgfpathlineto{\pgfqpoint{1.641141in}{3.695519in}}%
\pgfpathlineto{\pgfqpoint{1.641520in}{3.817796in}}%
\pgfpathlineto{\pgfqpoint{1.641994in}{3.623950in}}%
\pgfpathlineto{\pgfqpoint{1.642373in}{3.491302in}}%
\pgfpathlineto{\pgfqpoint{1.642942in}{3.678872in}}%
\pgfpathlineto{\pgfqpoint{1.643131in}{3.720426in}}%
\pgfpathlineto{\pgfqpoint{1.643605in}{3.581560in}}%
\pgfpathlineto{\pgfqpoint{1.643889in}{3.533479in}}%
\pgfpathlineto{\pgfqpoint{1.644268in}{3.657756in}}%
\pgfpathlineto{\pgfqpoint{1.644647in}{3.749033in}}%
\pgfpathlineto{\pgfqpoint{1.645121in}{3.570053in}}%
\pgfpathlineto{\pgfqpoint{1.645500in}{3.425358in}}%
\pgfpathlineto{\pgfqpoint{1.646069in}{3.643086in}}%
\pgfpathlineto{\pgfqpoint{1.646353in}{3.686622in}}%
\pgfpathlineto{\pgfqpoint{1.646827in}{3.556076in}}%
\pgfpathlineto{\pgfqpoint{1.647016in}{3.521039in}}%
\pgfpathlineto{\pgfqpoint{1.647490in}{3.666960in}}%
\pgfpathlineto{\pgfqpoint{1.647775in}{3.713616in}}%
\pgfpathlineto{\pgfqpoint{1.648154in}{3.595087in}}%
\pgfpathlineto{\pgfqpoint{1.648627in}{3.426448in}}%
\pgfpathlineto{\pgfqpoint{1.649196in}{3.628027in}}%
\pgfpathlineto{\pgfqpoint{1.649480in}{3.694060in}}%
\pgfpathlineto{\pgfqpoint{1.649954in}{3.564999in}}%
\pgfpathlineto{\pgfqpoint{1.650144in}{3.529613in}}%
\pgfpathlineto{\pgfqpoint{1.650618in}{3.645742in}}%
\pgfpathlineto{\pgfqpoint{1.650902in}{3.706528in}}%
\pgfpathlineto{\pgfqpoint{1.651376in}{3.547112in}}%
\pgfpathlineto{\pgfqpoint{1.651755in}{3.444696in}}%
\pgfpathlineto{\pgfqpoint{1.652134in}{3.584157in}}%
\pgfpathlineto{\pgfqpoint{1.652608in}{3.758259in}}%
\pgfpathlineto{\pgfqpoint{1.653176in}{3.593281in}}%
\pgfpathlineto{\pgfqpoint{1.653271in}{3.578129in}}%
\pgfpathlineto{\pgfqpoint{1.653650in}{3.668297in}}%
\pgfpathlineto{\pgfqpoint{1.654029in}{3.746471in}}%
\pgfpathlineto{\pgfqpoint{1.654503in}{3.593243in}}%
\pgfpathlineto{\pgfqpoint{1.654882in}{3.488991in}}%
\pgfpathlineto{\pgfqpoint{1.655450in}{3.667563in}}%
\pgfpathlineto{\pgfqpoint{1.655735in}{3.763499in}}%
\pgfpathlineto{\pgfqpoint{1.656303in}{3.547901in}}%
\pgfpathlineto{\pgfqpoint{1.656588in}{3.518190in}}%
\pgfpathlineto{\pgfqpoint{1.656967in}{3.585837in}}%
\pgfpathlineto{\pgfqpoint{1.657156in}{3.628406in}}%
\pgfpathlineto{\pgfqpoint{1.657630in}{3.514048in}}%
\pgfpathlineto{\pgfqpoint{1.658009in}{3.393874in}}%
\pgfpathlineto{\pgfqpoint{1.658483in}{3.591125in}}%
\pgfpathlineto{\pgfqpoint{1.658862in}{3.717868in}}%
\pgfpathlineto{\pgfqpoint{1.659431in}{3.530953in}}%
\pgfpathlineto{\pgfqpoint{1.659715in}{3.466273in}}%
\pgfpathlineto{\pgfqpoint{1.660283in}{3.606876in}}%
\pgfpathlineto{\pgfqpoint{1.660378in}{3.611025in}}%
\pgfpathlineto{\pgfqpoint{1.660473in}{3.596855in}}%
\pgfpathlineto{\pgfqpoint{1.661136in}{3.364202in}}%
\pgfpathlineto{\pgfqpoint{1.661610in}{3.546887in}}%
\pgfpathlineto{\pgfqpoint{1.661989in}{3.665758in}}%
\pgfpathlineto{\pgfqpoint{1.662558in}{3.493071in}}%
\pgfpathlineto{\pgfqpoint{1.662842in}{3.424619in}}%
\pgfpathlineto{\pgfqpoint{1.663221in}{3.574599in}}%
\pgfpathlineto{\pgfqpoint{1.665022in}{4.160816in}}%
\pgfpathlineto{\pgfqpoint{1.665211in}{4.134302in}}%
\pgfpathlineto{\pgfqpoint{1.666254in}{3.436388in}}%
\pgfpathlineto{\pgfqpoint{1.667675in}{3.617191in}}%
\pgfpathlineto{\pgfqpoint{1.668244in}{3.834465in}}%
\pgfpathlineto{\pgfqpoint{1.668717in}{3.637907in}}%
\pgfpathlineto{\pgfqpoint{1.669097in}{3.532143in}}%
\pgfpathlineto{\pgfqpoint{1.669665in}{3.702794in}}%
\pgfpathlineto{\pgfqpoint{1.669949in}{3.757073in}}%
\pgfpathlineto{\pgfqpoint{1.670518in}{3.615092in}}%
\pgfpathlineto{\pgfqpoint{1.670613in}{3.613078in}}%
\pgfpathlineto{\pgfqpoint{1.670708in}{3.628735in}}%
\pgfpathlineto{\pgfqpoint{1.671371in}{3.829359in}}%
\pgfpathlineto{\pgfqpoint{1.671845in}{3.673864in}}%
\pgfpathlineto{\pgfqpoint{1.672224in}{3.519442in}}%
\pgfpathlineto{\pgfqpoint{1.672792in}{3.726694in}}%
\pgfpathlineto{\pgfqpoint{1.672982in}{3.782543in}}%
\pgfpathlineto{\pgfqpoint{1.673645in}{3.650900in}}%
\pgfpathlineto{\pgfqpoint{1.673740in}{3.649115in}}%
\pgfpathlineto{\pgfqpoint{1.673835in}{3.657276in}}%
\pgfpathlineto{\pgfqpoint{1.674498in}{3.857727in}}%
\pgfpathlineto{\pgfqpoint{1.674972in}{3.690902in}}%
\pgfpathlineto{\pgfqpoint{1.675351in}{3.550682in}}%
\pgfpathlineto{\pgfqpoint{1.675920in}{3.773905in}}%
\pgfpathlineto{\pgfqpoint{1.676204in}{3.845190in}}%
\pgfpathlineto{\pgfqpoint{1.676772in}{3.700948in}}%
\pgfpathlineto{\pgfqpoint{1.676867in}{3.685625in}}%
\pgfpathlineto{\pgfqpoint{1.677246in}{3.772462in}}%
\pgfpathlineto{\pgfqpoint{1.677625in}{3.868730in}}%
\pgfpathlineto{\pgfqpoint{1.678099in}{3.711547in}}%
\pgfpathlineto{\pgfqpoint{1.678478in}{3.582684in}}%
\pgfpathlineto{\pgfqpoint{1.678952in}{3.766092in}}%
\pgfpathlineto{\pgfqpoint{1.679331in}{3.879194in}}%
\pgfpathlineto{\pgfqpoint{1.679900in}{3.732460in}}%
\pgfpathlineto{\pgfqpoint{1.680089in}{3.703248in}}%
\pgfpathlineto{\pgfqpoint{1.680563in}{3.814020in}}%
\pgfpathlineto{\pgfqpoint{1.680753in}{3.842372in}}%
\pgfpathlineto{\pgfqpoint{1.681132in}{3.752643in}}%
\pgfpathlineto{\pgfqpoint{1.681605in}{3.577129in}}%
\pgfpathlineto{\pgfqpoint{1.682079in}{3.779511in}}%
\pgfpathlineto{\pgfqpoint{1.682364in}{3.879498in}}%
\pgfpathlineto{\pgfqpoint{1.683027in}{3.709396in}}%
\pgfpathlineto{\pgfqpoint{1.683216in}{3.674073in}}%
\pgfpathlineto{\pgfqpoint{1.683785in}{3.798057in}}%
\pgfpathlineto{\pgfqpoint{1.683974in}{3.810366in}}%
\pgfpathlineto{\pgfqpoint{1.684259in}{3.735859in}}%
\pgfpathlineto{\pgfqpoint{1.684733in}{3.575670in}}%
\pgfpathlineto{\pgfqpoint{1.685206in}{3.753020in}}%
\pgfpathlineto{\pgfqpoint{1.685585in}{3.878358in}}%
\pgfpathlineto{\pgfqpoint{1.686154in}{3.704716in}}%
\pgfpathlineto{\pgfqpoint{1.686438in}{3.672194in}}%
\pgfpathlineto{\pgfqpoint{1.686912in}{3.747917in}}%
\pgfpathlineto{\pgfqpoint{1.687102in}{3.774089in}}%
\pgfpathlineto{\pgfqpoint{1.687481in}{3.673601in}}%
\pgfpathlineto{\pgfqpoint{1.687860in}{3.552370in}}%
\pgfpathlineto{\pgfqpoint{1.688334in}{3.721751in}}%
\pgfpathlineto{\pgfqpoint{1.688713in}{3.841841in}}%
\pgfpathlineto{\pgfqpoint{1.689187in}{3.674935in}}%
\pgfpathlineto{\pgfqpoint{1.689660in}{3.564486in}}%
\pgfpathlineto{\pgfqpoint{1.690229in}{3.695844in}}%
\pgfpathlineto{\pgfqpoint{1.690418in}{3.672830in}}%
\pgfpathlineto{\pgfqpoint{1.690987in}{3.514543in}}%
\pgfpathlineto{\pgfqpoint{1.691461in}{3.665316in}}%
\pgfpathlineto{\pgfqpoint{1.691840in}{3.795353in}}%
\pgfpathlineto{\pgfqpoint{1.692409in}{3.590614in}}%
\pgfpathlineto{\pgfqpoint{1.692693in}{3.511271in}}%
\pgfpathlineto{\pgfqpoint{1.693261in}{3.677869in}}%
\pgfpathlineto{\pgfqpoint{1.693451in}{3.701754in}}%
\pgfpathlineto{\pgfqpoint{1.693830in}{3.610419in}}%
\pgfpathlineto{\pgfqpoint{1.694114in}{3.560086in}}%
\pgfpathlineto{\pgfqpoint{1.694493in}{3.668302in}}%
\pgfpathlineto{\pgfqpoint{1.694967in}{3.830504in}}%
\pgfpathlineto{\pgfqpoint{1.695441in}{3.671955in}}%
\pgfpathlineto{\pgfqpoint{1.695820in}{3.544650in}}%
\pgfpathlineto{\pgfqpoint{1.696389in}{3.737680in}}%
\pgfpathlineto{\pgfqpoint{1.696578in}{3.779688in}}%
\pgfpathlineto{\pgfqpoint{1.697241in}{3.664165in}}%
\pgfpathlineto{\pgfqpoint{1.697431in}{3.679787in}}%
\pgfpathlineto{\pgfqpoint{1.698094in}{3.938233in}}%
\pgfpathlineto{\pgfqpoint{1.698568in}{3.747848in}}%
\pgfpathlineto{\pgfqpoint{1.698947in}{3.605424in}}%
\pgfpathlineto{\pgfqpoint{1.699516in}{3.769405in}}%
\pgfpathlineto{\pgfqpoint{1.699800in}{3.812900in}}%
\pgfpathlineto{\pgfqpoint{1.700274in}{3.686707in}}%
\pgfpathlineto{\pgfqpoint{1.700558in}{3.649660in}}%
\pgfpathlineto{\pgfqpoint{1.700937in}{3.771396in}}%
\pgfpathlineto{\pgfqpoint{1.701127in}{3.823829in}}%
\pgfpathlineto{\pgfqpoint{1.701601in}{3.666027in}}%
\pgfpathlineto{\pgfqpoint{1.702169in}{3.473455in}}%
\pgfpathlineto{\pgfqpoint{1.702643in}{3.665774in}}%
\pgfpathlineto{\pgfqpoint{1.702927in}{3.709662in}}%
\pgfpathlineto{\pgfqpoint{1.703401in}{3.602909in}}%
\pgfpathlineto{\pgfqpoint{1.703591in}{3.560511in}}%
\pgfpathlineto{\pgfqpoint{1.704159in}{3.694844in}}%
\pgfpathlineto{\pgfqpoint{1.704349in}{3.720034in}}%
\pgfpathlineto{\pgfqpoint{1.704728in}{3.618428in}}%
\pgfpathlineto{\pgfqpoint{1.705202in}{3.442412in}}%
\pgfpathlineto{\pgfqpoint{1.705770in}{3.632106in}}%
\pgfpathlineto{\pgfqpoint{1.706055in}{3.698594in}}%
\pgfpathlineto{\pgfqpoint{1.706623in}{3.554516in}}%
\pgfpathlineto{\pgfqpoint{1.706813in}{3.530555in}}%
\pgfpathlineto{\pgfqpoint{1.707286in}{3.628376in}}%
\pgfpathlineto{\pgfqpoint{1.707476in}{3.650998in}}%
\pgfpathlineto{\pgfqpoint{1.707760in}{3.552955in}}%
\pgfpathlineto{\pgfqpoint{1.708139in}{3.408144in}}%
\pgfpathlineto{\pgfqpoint{1.708613in}{3.637385in}}%
\pgfpathlineto{\pgfqpoint{1.709277in}{4.087466in}}%
\pgfpathlineto{\pgfqpoint{1.709845in}{3.888385in}}%
\pgfpathlineto{\pgfqpoint{1.711361in}{3.261977in}}%
\pgfpathlineto{\pgfqpoint{1.711930in}{3.591860in}}%
\pgfpathlineto{\pgfqpoint{1.712309in}{3.728370in}}%
\pgfpathlineto{\pgfqpoint{1.712878in}{3.546819in}}%
\pgfpathlineto{\pgfqpoint{1.713162in}{3.496823in}}%
\pgfpathlineto{\pgfqpoint{1.713730in}{3.622276in}}%
\pgfpathlineto{\pgfqpoint{1.713920in}{3.636595in}}%
\pgfpathlineto{\pgfqpoint{1.714204in}{3.545306in}}%
\pgfpathlineto{\pgfqpoint{1.714583in}{3.407915in}}%
\pgfpathlineto{\pgfqpoint{1.715057in}{3.598990in}}%
\pgfpathlineto{\pgfqpoint{1.715436in}{3.730288in}}%
\pgfpathlineto{\pgfqpoint{1.716005in}{3.526320in}}%
\pgfpathlineto{\pgfqpoint{1.716289in}{3.454925in}}%
\pgfpathlineto{\pgfqpoint{1.716858in}{3.593922in}}%
\pgfpathlineto{\pgfqpoint{1.716952in}{3.601252in}}%
\pgfpathlineto{\pgfqpoint{1.717237in}{3.558180in}}%
\pgfpathlineto{\pgfqpoint{1.717711in}{3.425213in}}%
\pgfpathlineto{\pgfqpoint{1.718184in}{3.603769in}}%
\pgfpathlineto{\pgfqpoint{1.718563in}{3.726741in}}%
\pgfpathlineto{\pgfqpoint{1.719132in}{3.542377in}}%
\pgfpathlineto{\pgfqpoint{1.719511in}{3.464309in}}%
\pgfpathlineto{\pgfqpoint{1.719985in}{3.625239in}}%
\pgfpathlineto{\pgfqpoint{1.720080in}{3.641359in}}%
\pgfpathlineto{\pgfqpoint{1.720553in}{3.552023in}}%
\pgfpathlineto{\pgfqpoint{1.720932in}{3.482182in}}%
\pgfpathlineto{\pgfqpoint{1.721312in}{3.624055in}}%
\pgfpathlineto{\pgfqpoint{1.721785in}{3.768526in}}%
\pgfpathlineto{\pgfqpoint{1.722259in}{3.554021in}}%
\pgfpathlineto{\pgfqpoint{1.722543in}{3.471560in}}%
\pgfpathlineto{\pgfqpoint{1.723112in}{3.644687in}}%
\pgfpathlineto{\pgfqpoint{1.723302in}{3.678106in}}%
\pgfpathlineto{\pgfqpoint{1.723870in}{3.559158in}}%
\pgfpathlineto{\pgfqpoint{1.723965in}{3.553218in}}%
\pgfpathlineto{\pgfqpoint{1.724154in}{3.581112in}}%
\pgfpathlineto{\pgfqpoint{1.724818in}{3.774984in}}%
\pgfpathlineto{\pgfqpoint{1.725292in}{3.620370in}}%
\pgfpathlineto{\pgfqpoint{1.725765in}{3.495337in}}%
\pgfpathlineto{\pgfqpoint{1.726239in}{3.650913in}}%
\pgfpathlineto{\pgfqpoint{1.726429in}{3.694638in}}%
\pgfpathlineto{\pgfqpoint{1.727092in}{3.577176in}}%
\pgfpathlineto{\pgfqpoint{1.727187in}{3.569050in}}%
\pgfpathlineto{\pgfqpoint{1.727471in}{3.622079in}}%
\pgfpathlineto{\pgfqpoint{1.727945in}{3.777909in}}%
\pgfpathlineto{\pgfqpoint{1.728419in}{3.611808in}}%
\pgfpathlineto{\pgfqpoint{1.728893in}{3.463850in}}%
\pgfpathlineto{\pgfqpoint{1.729367in}{3.641920in}}%
\pgfpathlineto{\pgfqpoint{1.729651in}{3.700400in}}%
\pgfpathlineto{\pgfqpoint{1.730219in}{3.571930in}}%
\pgfpathlineto{\pgfqpoint{1.730314in}{3.563826in}}%
\pgfpathlineto{\pgfqpoint{1.730598in}{3.600740in}}%
\pgfpathlineto{\pgfqpoint{1.731072in}{3.720597in}}%
\pgfpathlineto{\pgfqpoint{1.731546in}{3.556323in}}%
\pgfpathlineto{\pgfqpoint{1.732020in}{3.427321in}}%
\pgfpathlineto{\pgfqpoint{1.732494in}{3.590602in}}%
\pgfpathlineto{\pgfqpoint{1.732778in}{3.675672in}}%
\pgfpathlineto{\pgfqpoint{1.733347in}{3.504948in}}%
\pgfpathlineto{\pgfqpoint{1.733536in}{3.481493in}}%
\pgfpathlineto{\pgfqpoint{1.734105in}{3.567184in}}%
\pgfpathlineto{\pgfqpoint{1.734294in}{3.549502in}}%
\pgfpathlineto{\pgfqpoint{1.735147in}{3.249911in}}%
\pgfpathlineto{\pgfqpoint{1.735716in}{3.451130in}}%
\pgfpathlineto{\pgfqpoint{1.735905in}{3.488431in}}%
\pgfpathlineto{\pgfqpoint{1.736379in}{3.350054in}}%
\pgfpathlineto{\pgfqpoint{1.736569in}{3.320696in}}%
\pgfpathlineto{\pgfqpoint{1.737042in}{3.428572in}}%
\pgfpathlineto{\pgfqpoint{1.737421in}{3.515904in}}%
\pgfpathlineto{\pgfqpoint{1.737895in}{3.386619in}}%
\pgfpathlineto{\pgfqpoint{1.738180in}{3.311794in}}%
\pgfpathlineto{\pgfqpoint{1.738653in}{3.465540in}}%
\pgfpathlineto{\pgfqpoint{1.739032in}{3.605091in}}%
\pgfpathlineto{\pgfqpoint{1.739601in}{3.431371in}}%
\pgfpathlineto{\pgfqpoint{1.739885in}{3.389394in}}%
\pgfpathlineto{\pgfqpoint{1.740359in}{3.496412in}}%
\pgfpathlineto{\pgfqpoint{1.740549in}{3.522516in}}%
\pgfpathlineto{\pgfqpoint{1.740928in}{3.440867in}}%
\pgfpathlineto{\pgfqpoint{1.741307in}{3.359510in}}%
\pgfpathlineto{\pgfqpoint{1.741781in}{3.515249in}}%
\pgfpathlineto{\pgfqpoint{1.742254in}{3.660102in}}%
\pgfpathlineto{\pgfqpoint{1.742728in}{3.491672in}}%
\pgfpathlineto{\pgfqpoint{1.743013in}{3.434243in}}%
\pgfpathlineto{\pgfqpoint{1.743581in}{3.548537in}}%
\pgfpathlineto{\pgfqpoint{1.743676in}{3.557411in}}%
\pgfpathlineto{\pgfqpoint{1.743960in}{3.516805in}}%
\pgfpathlineto{\pgfqpoint{1.744434in}{3.384897in}}%
\pgfpathlineto{\pgfqpoint{1.744908in}{3.516834in}}%
\pgfpathlineto{\pgfqpoint{1.745287in}{3.630873in}}%
\pgfpathlineto{\pgfqpoint{1.745761in}{3.462176in}}%
\pgfpathlineto{\pgfqpoint{1.746140in}{3.343597in}}%
\pgfpathlineto{\pgfqpoint{1.746803in}{3.474248in}}%
\pgfpathlineto{\pgfqpoint{1.746898in}{3.481278in}}%
\pgfpathlineto{\pgfqpoint{1.747182in}{3.444280in}}%
\pgfpathlineto{\pgfqpoint{1.747561in}{3.330762in}}%
\pgfpathlineto{\pgfqpoint{1.748035in}{3.504392in}}%
\pgfpathlineto{\pgfqpoint{1.748414in}{3.627202in}}%
\pgfpathlineto{\pgfqpoint{1.748983in}{3.461725in}}%
\pgfpathlineto{\pgfqpoint{1.749267in}{3.397210in}}%
\pgfpathlineto{\pgfqpoint{1.749836in}{3.544749in}}%
\pgfpathlineto{\pgfqpoint{1.750120in}{3.570621in}}%
\pgfpathlineto{\pgfqpoint{1.750594in}{3.485105in}}%
\pgfpathlineto{\pgfqpoint{1.750783in}{3.463872in}}%
\pgfpathlineto{\pgfqpoint{1.751067in}{3.566433in}}%
\pgfpathlineto{\pgfqpoint{1.751541in}{3.725184in}}%
\pgfpathlineto{\pgfqpoint{1.752015in}{3.562043in}}%
\pgfpathlineto{\pgfqpoint{1.752394in}{3.373235in}}%
\pgfpathlineto{\pgfqpoint{1.752963in}{3.663184in}}%
\pgfpathlineto{\pgfqpoint{1.754574in}{4.092389in}}%
\pgfpathlineto{\pgfqpoint{1.754763in}{4.073749in}}%
\pgfpathlineto{\pgfqpoint{1.755806in}{3.336267in}}%
\pgfpathlineto{\pgfqpoint{1.757227in}{3.584157in}}%
\pgfpathlineto{\pgfqpoint{1.757796in}{3.730396in}}%
\pgfpathlineto{\pgfqpoint{1.758175in}{3.614024in}}%
\pgfpathlineto{\pgfqpoint{1.758743in}{3.405916in}}%
\pgfpathlineto{\pgfqpoint{1.759217in}{3.573362in}}%
\pgfpathlineto{\pgfqpoint{1.759501in}{3.653662in}}%
\pgfpathlineto{\pgfqpoint{1.760165in}{3.525299in}}%
\pgfpathlineto{\pgfqpoint{1.760260in}{3.521715in}}%
\pgfpathlineto{\pgfqpoint{1.760544in}{3.550064in}}%
\pgfpathlineto{\pgfqpoint{1.760923in}{3.623448in}}%
\pgfpathlineto{\pgfqpoint{1.761302in}{3.536240in}}%
\pgfpathlineto{\pgfqpoint{1.761871in}{3.362095in}}%
\pgfpathlineto{\pgfqpoint{1.762344in}{3.539926in}}%
\pgfpathlineto{\pgfqpoint{1.762629in}{3.614363in}}%
\pgfpathlineto{\pgfqpoint{1.763197in}{3.486827in}}%
\pgfpathlineto{\pgfqpoint{1.763387in}{3.464490in}}%
\pgfpathlineto{\pgfqpoint{1.763861in}{3.551746in}}%
\pgfpathlineto{\pgfqpoint{1.764050in}{3.592246in}}%
\pgfpathlineto{\pgfqpoint{1.764524in}{3.446347in}}%
\pgfpathlineto{\pgfqpoint{1.764903in}{3.349265in}}%
\pgfpathlineto{\pgfqpoint{1.765377in}{3.492857in}}%
\pgfpathlineto{\pgfqpoint{1.765851in}{3.632487in}}%
\pgfpathlineto{\pgfqpoint{1.766325in}{3.474339in}}%
\pgfpathlineto{\pgfqpoint{1.766609in}{3.434791in}}%
\pgfpathlineto{\pgfqpoint{1.767083in}{3.520065in}}%
\pgfpathlineto{\pgfqpoint{1.767272in}{3.553883in}}%
\pgfpathlineto{\pgfqpoint{1.767746in}{3.413381in}}%
\pgfpathlineto{\pgfqpoint{1.768030in}{3.339659in}}%
\pgfpathlineto{\pgfqpoint{1.768504in}{3.511983in}}%
\pgfpathlineto{\pgfqpoint{1.768978in}{3.632295in}}%
\pgfpathlineto{\pgfqpoint{1.769452in}{3.489754in}}%
\pgfpathlineto{\pgfqpoint{1.769736in}{3.409708in}}%
\pgfpathlineto{\pgfqpoint{1.770399in}{3.531090in}}%
\pgfpathlineto{\pgfqpoint{1.770778in}{3.461738in}}%
\pgfpathlineto{\pgfqpoint{1.771157in}{3.374584in}}%
\pgfpathlineto{\pgfqpoint{1.771631in}{3.512397in}}%
\pgfpathlineto{\pgfqpoint{1.772105in}{3.652064in}}%
\pgfpathlineto{\pgfqpoint{1.772579in}{3.456478in}}%
\pgfpathlineto{\pgfqpoint{1.772863in}{3.391807in}}%
\pgfpathlineto{\pgfqpoint{1.773432in}{3.498662in}}%
\pgfpathlineto{\pgfqpoint{1.773716in}{3.541933in}}%
\pgfpathlineto{\pgfqpoint{1.774095in}{3.437601in}}%
\pgfpathlineto{\pgfqpoint{1.774285in}{3.394215in}}%
\pgfpathlineto{\pgfqpoint{1.774759in}{3.542819in}}%
\pgfpathlineto{\pgfqpoint{1.775138in}{3.669127in}}%
\pgfpathlineto{\pgfqpoint{1.775706in}{3.470036in}}%
\pgfpathlineto{\pgfqpoint{1.776085in}{3.394522in}}%
\pgfpathlineto{\pgfqpoint{1.776654in}{3.509010in}}%
\pgfpathlineto{\pgfqpoint{1.776843in}{3.527875in}}%
\pgfpathlineto{\pgfqpoint{1.777222in}{3.431912in}}%
\pgfpathlineto{\pgfqpoint{1.777412in}{3.396649in}}%
\pgfpathlineto{\pgfqpoint{1.777886in}{3.493220in}}%
\pgfpathlineto{\pgfqpoint{1.778265in}{3.606906in}}%
\pgfpathlineto{\pgfqpoint{1.778739in}{3.457339in}}%
\pgfpathlineto{\pgfqpoint{1.779212in}{3.297725in}}%
\pgfpathlineto{\pgfqpoint{1.779781in}{3.471598in}}%
\pgfpathlineto{\pgfqpoint{1.779876in}{3.479537in}}%
\pgfpathlineto{\pgfqpoint{1.780255in}{3.425114in}}%
\pgfpathlineto{\pgfqpoint{1.780634in}{3.366355in}}%
\pgfpathlineto{\pgfqpoint{1.781013in}{3.453359in}}%
\pgfpathlineto{\pgfqpoint{1.781392in}{3.561081in}}%
\pgfpathlineto{\pgfqpoint{1.781866in}{3.411909in}}%
\pgfpathlineto{\pgfqpoint{1.782340in}{3.277639in}}%
\pgfpathlineto{\pgfqpoint{1.782813in}{3.421380in}}%
\pgfpathlineto{\pgfqpoint{1.783098in}{3.479898in}}%
\pgfpathlineto{\pgfqpoint{1.783761in}{3.373022in}}%
\pgfpathlineto{\pgfqpoint{1.783856in}{3.371731in}}%
\pgfpathlineto{\pgfqpoint{1.783951in}{3.382161in}}%
\pgfpathlineto{\pgfqpoint{1.784519in}{3.535886in}}%
\pgfpathlineto{\pgfqpoint{1.784993in}{3.413776in}}%
\pgfpathlineto{\pgfqpoint{1.785372in}{3.278230in}}%
\pgfpathlineto{\pgfqpoint{1.785941in}{3.471523in}}%
\pgfpathlineto{\pgfqpoint{1.786320in}{3.566617in}}%
\pgfpathlineto{\pgfqpoint{1.786888in}{3.444156in}}%
\pgfpathlineto{\pgfqpoint{1.786983in}{3.446420in}}%
\pgfpathlineto{\pgfqpoint{1.787646in}{3.588370in}}%
\pgfpathlineto{\pgfqpoint{1.788025in}{3.491527in}}%
\pgfpathlineto{\pgfqpoint{1.788499in}{3.300132in}}%
\pgfpathlineto{\pgfqpoint{1.789068in}{3.486511in}}%
\pgfpathlineto{\pgfqpoint{1.789352in}{3.569567in}}%
\pgfpathlineto{\pgfqpoint{1.789921in}{3.407085in}}%
\pgfpathlineto{\pgfqpoint{1.790205in}{3.386972in}}%
\pgfpathlineto{\pgfqpoint{1.790584in}{3.440894in}}%
\pgfpathlineto{\pgfqpoint{1.790774in}{3.472058in}}%
\pgfpathlineto{\pgfqpoint{1.791247in}{3.351872in}}%
\pgfpathlineto{\pgfqpoint{1.791627in}{3.248807in}}%
\pgfpathlineto{\pgfqpoint{1.792100in}{3.384675in}}%
\pgfpathlineto{\pgfqpoint{1.792574in}{3.535796in}}%
\pgfpathlineto{\pgfqpoint{1.793143in}{3.370367in}}%
\pgfpathlineto{\pgfqpoint{1.793332in}{3.354095in}}%
\pgfpathlineto{\pgfqpoint{1.793711in}{3.435966in}}%
\pgfpathlineto{\pgfqpoint{1.793901in}{3.467866in}}%
\pgfpathlineto{\pgfqpoint{1.794375in}{3.355893in}}%
\pgfpathlineto{\pgfqpoint{1.794754in}{3.267247in}}%
\pgfpathlineto{\pgfqpoint{1.795228in}{3.417694in}}%
\pgfpathlineto{\pgfqpoint{1.795701in}{3.563617in}}%
\pgfpathlineto{\pgfqpoint{1.796175in}{3.406692in}}%
\pgfpathlineto{\pgfqpoint{1.796554in}{3.328038in}}%
\pgfpathlineto{\pgfqpoint{1.797123in}{3.439488in}}%
\pgfpathlineto{\pgfqpoint{1.798734in}{4.001222in}}%
\pgfpathlineto{\pgfqpoint{1.799302in}{3.790128in}}%
\pgfpathlineto{\pgfqpoint{1.800250in}{3.388726in}}%
\pgfpathlineto{\pgfqpoint{1.801008in}{3.431975in}}%
\pgfpathlineto{\pgfqpoint{1.801672in}{3.633008in}}%
\pgfpathlineto{\pgfqpoint{1.801956in}{3.690461in}}%
\pgfpathlineto{\pgfqpoint{1.802335in}{3.549022in}}%
\pgfpathlineto{\pgfqpoint{1.802809in}{3.435158in}}%
\pgfpathlineto{\pgfqpoint{1.803377in}{3.540718in}}%
\pgfpathlineto{\pgfqpoint{1.803472in}{3.547392in}}%
\pgfpathlineto{\pgfqpoint{1.803756in}{3.513598in}}%
\pgfpathlineto{\pgfqpoint{1.804230in}{3.444229in}}%
\pgfpathlineto{\pgfqpoint{1.804609in}{3.555890in}}%
\pgfpathlineto{\pgfqpoint{1.804988in}{3.687004in}}%
\pgfpathlineto{\pgfqpoint{1.805462in}{3.514247in}}%
\pgfpathlineto{\pgfqpoint{1.805936in}{3.362517in}}%
\pgfpathlineto{\pgfqpoint{1.806504in}{3.502107in}}%
\pgfpathlineto{\pgfqpoint{1.806599in}{3.512885in}}%
\pgfpathlineto{\pgfqpoint{1.806978in}{3.440423in}}%
\pgfpathlineto{\pgfqpoint{1.807452in}{3.338549in}}%
\pgfpathlineto{\pgfqpoint{1.807831in}{3.461089in}}%
\pgfpathlineto{\pgfqpoint{1.808115in}{3.517885in}}%
\pgfpathlineto{\pgfqpoint{1.808589in}{3.352627in}}%
\pgfpathlineto{\pgfqpoint{1.808968in}{3.226877in}}%
\pgfpathlineto{\pgfqpoint{1.809537in}{3.416495in}}%
\pgfpathlineto{\pgfqpoint{1.809821in}{3.507363in}}%
\pgfpathlineto{\pgfqpoint{1.810485in}{3.407682in}}%
\pgfpathlineto{\pgfqpoint{1.810579in}{3.408833in}}%
\pgfpathlineto{\pgfqpoint{1.811243in}{3.551485in}}%
\pgfpathlineto{\pgfqpoint{1.811622in}{3.452443in}}%
\pgfpathlineto{\pgfqpoint{1.812190in}{3.249896in}}%
\pgfpathlineto{\pgfqpoint{1.812759in}{3.424917in}}%
\pgfpathlineto{\pgfqpoint{1.812948in}{3.455850in}}%
\pgfpathlineto{\pgfqpoint{1.813517in}{3.360810in}}%
\pgfpathlineto{\pgfqpoint{1.813707in}{3.342190in}}%
\pgfpathlineto{\pgfqpoint{1.814086in}{3.396268in}}%
\pgfpathlineto{\pgfqpoint{1.814370in}{3.439532in}}%
\pgfpathlineto{\pgfqpoint{1.814749in}{3.355617in}}%
\pgfpathlineto{\pgfqpoint{1.815318in}{3.158806in}}%
\pgfpathlineto{\pgfqpoint{1.815791in}{3.329863in}}%
\pgfpathlineto{\pgfqpoint{1.816170in}{3.390661in}}%
\pgfpathlineto{\pgfqpoint{1.816644in}{3.274908in}}%
\pgfpathlineto{\pgfqpoint{1.816834in}{3.245723in}}%
\pgfpathlineto{\pgfqpoint{1.817402in}{3.346851in}}%
\pgfpathlineto{\pgfqpoint{1.817497in}{3.353761in}}%
\pgfpathlineto{\pgfqpoint{1.817781in}{3.298500in}}%
\pgfpathlineto{\pgfqpoint{1.818350in}{3.112822in}}%
\pgfpathlineto{\pgfqpoint{1.818824in}{3.249433in}}%
\pgfpathlineto{\pgfqpoint{1.819298in}{3.389089in}}%
\pgfpathlineto{\pgfqpoint{1.819866in}{3.224100in}}%
\pgfpathlineto{\pgfqpoint{1.820056in}{3.205522in}}%
\pgfpathlineto{\pgfqpoint{1.820624in}{3.274987in}}%
\pgfpathlineto{\pgfqpoint{1.820719in}{3.277197in}}%
\pgfpathlineto{\pgfqpoint{1.820909in}{3.257894in}}%
\pgfpathlineto{\pgfqpoint{1.821477in}{3.073137in}}%
\pgfpathlineto{\pgfqpoint{1.821951in}{3.216642in}}%
\pgfpathlineto{\pgfqpoint{1.822330in}{3.337607in}}%
\pgfpathlineto{\pgfqpoint{1.822899in}{3.194523in}}%
\pgfpathlineto{\pgfqpoint{1.823183in}{3.136909in}}%
\pgfpathlineto{\pgfqpoint{1.823846in}{3.222126in}}%
\pgfpathlineto{\pgfqpoint{1.824225in}{3.144926in}}%
\pgfpathlineto{\pgfqpoint{1.824604in}{3.040580in}}%
\pgfpathlineto{\pgfqpoint{1.825078in}{3.169554in}}%
\pgfpathlineto{\pgfqpoint{1.825552in}{3.325536in}}%
\pgfpathlineto{\pgfqpoint{1.826026in}{3.147360in}}%
\pgfpathlineto{\pgfqpoint{1.826310in}{3.091280in}}%
\pgfpathlineto{\pgfqpoint{1.826974in}{3.187645in}}%
\pgfpathlineto{\pgfqpoint{1.827163in}{3.197447in}}%
\pgfpathlineto{\pgfqpoint{1.827447in}{3.142307in}}%
\pgfpathlineto{\pgfqpoint{1.827732in}{3.078103in}}%
\pgfpathlineto{\pgfqpoint{1.828205in}{3.204705in}}%
\pgfpathlineto{\pgfqpoint{1.828679in}{3.364610in}}%
\pgfpathlineto{\pgfqpoint{1.829248in}{3.181037in}}%
\pgfpathlineto{\pgfqpoint{1.829532in}{3.138967in}}%
\pgfpathlineto{\pgfqpoint{1.829911in}{3.258241in}}%
\pgfpathlineto{\pgfqpoint{1.830196in}{3.324315in}}%
\pgfpathlineto{\pgfqpoint{1.830859in}{3.229140in}}%
\pgfpathlineto{\pgfqpoint{1.830954in}{3.235959in}}%
\pgfpathlineto{\pgfqpoint{1.831807in}{3.523651in}}%
\pgfpathlineto{\pgfqpoint{1.832375in}{3.327135in}}%
\pgfpathlineto{\pgfqpoint{1.832659in}{3.251820in}}%
\pgfpathlineto{\pgfqpoint{1.833228in}{3.424290in}}%
\pgfpathlineto{\pgfqpoint{1.833323in}{3.434538in}}%
\pgfpathlineto{\pgfqpoint{1.833702in}{3.377230in}}%
\pgfpathlineto{\pgfqpoint{1.834081in}{3.307622in}}%
\pgfpathlineto{\pgfqpoint{1.834555in}{3.425087in}}%
\pgfpathlineto{\pgfqpoint{1.834839in}{3.483092in}}%
\pgfpathlineto{\pgfqpoint{1.835313in}{3.329278in}}%
\pgfpathlineto{\pgfqpoint{1.835787in}{3.190562in}}%
\pgfpathlineto{\pgfqpoint{1.836355in}{3.352820in}}%
\pgfpathlineto{\pgfqpoint{1.836639in}{3.370426in}}%
\pgfpathlineto{\pgfqpoint{1.837019in}{3.306702in}}%
\pgfpathlineto{\pgfqpoint{1.837208in}{3.275106in}}%
\pgfpathlineto{\pgfqpoint{1.837682in}{3.371564in}}%
\pgfpathlineto{\pgfqpoint{1.838061in}{3.426048in}}%
\pgfpathlineto{\pgfqpoint{1.838345in}{3.336141in}}%
\pgfpathlineto{\pgfqpoint{1.838914in}{3.140835in}}%
\pgfpathlineto{\pgfqpoint{1.839482in}{3.299120in}}%
\pgfpathlineto{\pgfqpoint{1.839767in}{3.353589in}}%
\pgfpathlineto{\pgfqpoint{1.840335in}{3.252128in}}%
\pgfpathlineto{\pgfqpoint{1.840430in}{3.254227in}}%
\pgfpathlineto{\pgfqpoint{1.842325in}{3.609032in}}%
\pgfpathlineto{\pgfqpoint{1.842894in}{3.824519in}}%
\pgfpathlineto{\pgfqpoint{1.843462in}{3.657864in}}%
\pgfpathlineto{\pgfqpoint{1.844789in}{3.177260in}}%
\pgfpathlineto{\pgfqpoint{1.845547in}{3.399255in}}%
\pgfpathlineto{\pgfqpoint{1.845926in}{3.527960in}}%
\pgfpathlineto{\pgfqpoint{1.846590in}{3.393650in}}%
\pgfpathlineto{\pgfqpoint{1.846779in}{3.361547in}}%
\pgfpathlineto{\pgfqpoint{1.847253in}{3.463063in}}%
\pgfpathlineto{\pgfqpoint{1.847443in}{3.480472in}}%
\pgfpathlineto{\pgfqpoint{1.847822in}{3.378962in}}%
\pgfpathlineto{\pgfqpoint{1.848201in}{3.285099in}}%
\pgfpathlineto{\pgfqpoint{1.848675in}{3.413210in}}%
\pgfpathlineto{\pgfqpoint{1.849148in}{3.565236in}}%
\pgfpathlineto{\pgfqpoint{1.849717in}{3.408758in}}%
\pgfpathlineto{\pgfqpoint{1.849906in}{3.381405in}}%
\pgfpathlineto{\pgfqpoint{1.850380in}{3.465676in}}%
\pgfpathlineto{\pgfqpoint{1.850570in}{3.489504in}}%
\pgfpathlineto{\pgfqpoint{1.851044in}{3.394644in}}%
\pgfpathlineto{\pgfqpoint{1.851328in}{3.322603in}}%
\pgfpathlineto{\pgfqpoint{1.851802in}{3.461929in}}%
\pgfpathlineto{\pgfqpoint{1.852276in}{3.633130in}}%
\pgfpathlineto{\pgfqpoint{1.852844in}{3.440778in}}%
\pgfpathlineto{\pgfqpoint{1.853128in}{3.406542in}}%
\pgfpathlineto{\pgfqpoint{1.853602in}{3.497635in}}%
\pgfpathlineto{\pgfqpoint{1.853792in}{3.515875in}}%
\pgfpathlineto{\pgfqpoint{1.854171in}{3.423633in}}%
\pgfpathlineto{\pgfqpoint{1.854455in}{3.366357in}}%
\pgfpathlineto{\pgfqpoint{1.854929in}{3.530269in}}%
\pgfpathlineto{\pgfqpoint{1.855403in}{3.648792in}}%
\pgfpathlineto{\pgfqpoint{1.855877in}{3.512510in}}%
\pgfpathlineto{\pgfqpoint{1.856161in}{3.421190in}}%
\pgfpathlineto{\pgfqpoint{1.856824in}{3.553820in}}%
\pgfpathlineto{\pgfqpoint{1.856919in}{3.560420in}}%
\pgfpathlineto{\pgfqpoint{1.857203in}{3.528083in}}%
\pgfpathlineto{\pgfqpoint{1.857677in}{3.468044in}}%
\pgfpathlineto{\pgfqpoint{1.858056in}{3.563988in}}%
\pgfpathlineto{\pgfqpoint{1.858435in}{3.716306in}}%
\pgfpathlineto{\pgfqpoint{1.859004in}{3.546652in}}%
\pgfpathlineto{\pgfqpoint{1.859383in}{3.439948in}}%
\pgfpathlineto{\pgfqpoint{1.859951in}{3.573662in}}%
\pgfpathlineto{\pgfqpoint{1.860141in}{3.606462in}}%
\pgfpathlineto{\pgfqpoint{1.860615in}{3.497650in}}%
\pgfpathlineto{\pgfqpoint{1.860710in}{3.487909in}}%
\pgfpathlineto{\pgfqpoint{1.860994in}{3.535264in}}%
\pgfpathlineto{\pgfqpoint{1.861657in}{3.713877in}}%
\pgfpathlineto{\pgfqpoint{1.862036in}{3.597242in}}%
\pgfpathlineto{\pgfqpoint{1.862510in}{3.433672in}}%
\pgfpathlineto{\pgfqpoint{1.863079in}{3.588208in}}%
\pgfpathlineto{\pgfqpoint{1.863932in}{3.554267in}}%
\pgfpathlineto{\pgfqpoint{1.864690in}{3.731942in}}%
\pgfpathlineto{\pgfqpoint{1.864784in}{3.737667in}}%
\pgfpathlineto{\pgfqpoint{1.864974in}{3.696511in}}%
\pgfpathlineto{\pgfqpoint{1.865543in}{3.483448in}}%
\pgfpathlineto{\pgfqpoint{1.866206in}{3.635023in}}%
\pgfpathlineto{\pgfqpoint{1.866490in}{3.683775in}}%
\pgfpathlineto{\pgfqpoint{1.867154in}{3.586650in}}%
\pgfpathlineto{\pgfqpoint{1.867627in}{3.694472in}}%
\pgfpathlineto{\pgfqpoint{1.867912in}{3.733011in}}%
\pgfpathlineto{\pgfqpoint{1.868291in}{3.626038in}}%
\pgfpathlineto{\pgfqpoint{1.868765in}{3.457493in}}%
\pgfpathlineto{\pgfqpoint{1.869238in}{3.636759in}}%
\pgfpathlineto{\pgfqpoint{1.869617in}{3.735329in}}%
\pgfpathlineto{\pgfqpoint{1.870281in}{3.636752in}}%
\pgfpathlineto{\pgfqpoint{1.870755in}{3.711915in}}%
\pgfpathlineto{\pgfqpoint{1.871039in}{3.754593in}}%
\pgfpathlineto{\pgfqpoint{1.871513in}{3.643360in}}%
\pgfpathlineto{\pgfqpoint{1.871892in}{3.554718in}}%
\pgfpathlineto{\pgfqpoint{1.872366in}{3.719957in}}%
\pgfpathlineto{\pgfqpoint{1.872745in}{3.833604in}}%
\pgfpathlineto{\pgfqpoint{1.873408in}{3.686169in}}%
\pgfpathlineto{\pgfqpoint{1.873503in}{3.681614in}}%
\pgfpathlineto{\pgfqpoint{1.873692in}{3.706077in}}%
\pgfpathlineto{\pgfqpoint{1.874261in}{3.816175in}}%
\pgfpathlineto{\pgfqpoint{1.874640in}{3.683751in}}%
\pgfpathlineto{\pgfqpoint{1.874924in}{3.618860in}}%
\pgfpathlineto{\pgfqpoint{1.875398in}{3.745478in}}%
\pgfpathlineto{\pgfqpoint{1.875872in}{3.898131in}}%
\pgfpathlineto{\pgfqpoint{1.876440in}{3.752036in}}%
\pgfpathlineto{\pgfqpoint{1.877293in}{3.778654in}}%
\pgfpathlineto{\pgfqpoint{1.878146in}{3.564734in}}%
\pgfpathlineto{\pgfqpoint{1.878810in}{3.773236in}}%
\pgfpathlineto{\pgfqpoint{1.878999in}{3.797536in}}%
\pgfpathlineto{\pgfqpoint{1.879378in}{3.687371in}}%
\pgfpathlineto{\pgfqpoint{1.880989in}{3.451629in}}%
\pgfpathlineto{\pgfqpoint{1.881273in}{3.406218in}}%
\pgfpathlineto{\pgfqpoint{1.881652in}{3.503209in}}%
\pgfpathlineto{\pgfqpoint{1.882126in}{3.626400in}}%
\pgfpathlineto{\pgfqpoint{1.882600in}{3.486959in}}%
\pgfpathlineto{\pgfqpoint{1.882979in}{3.416998in}}%
\pgfpathlineto{\pgfqpoint{1.883548in}{3.538483in}}%
\pgfpathlineto{\pgfqpoint{1.883737in}{3.559921in}}%
\pgfpathlineto{\pgfqpoint{1.884211in}{3.480323in}}%
\pgfpathlineto{\pgfqpoint{1.884401in}{3.466279in}}%
\pgfpathlineto{\pgfqpoint{1.884685in}{3.525427in}}%
\pgfpathlineto{\pgfqpoint{1.885159in}{3.639904in}}%
\pgfpathlineto{\pgfqpoint{1.885727in}{3.510306in}}%
\pgfpathlineto{\pgfqpoint{1.885822in}{3.496874in}}%
\pgfpathlineto{\pgfqpoint{1.886106in}{3.556499in}}%
\pgfpathlineto{\pgfqpoint{1.886864in}{3.960336in}}%
\pgfpathlineto{\pgfqpoint{1.887623in}{3.814422in}}%
\pgfpathlineto{\pgfqpoint{1.888002in}{3.871919in}}%
\pgfpathlineto{\pgfqpoint{1.888286in}{3.804072in}}%
\pgfpathlineto{\pgfqpoint{1.889044in}{3.325740in}}%
\pgfpathlineto{\pgfqpoint{1.889707in}{3.560127in}}%
\pgfpathlineto{\pgfqpoint{1.890655in}{3.536092in}}%
\pgfpathlineto{\pgfqpoint{1.891508in}{3.737580in}}%
\pgfpathlineto{\pgfqpoint{1.891792in}{3.679844in}}%
\pgfpathlineto{\pgfqpoint{1.892361in}{3.482457in}}%
\pgfpathlineto{\pgfqpoint{1.892929in}{3.640773in}}%
\pgfpathlineto{\pgfqpoint{1.893214in}{3.689927in}}%
\pgfpathlineto{\pgfqpoint{1.893877in}{3.606397in}}%
\pgfpathlineto{\pgfqpoint{1.893972in}{3.605565in}}%
\pgfpathlineto{\pgfqpoint{1.894635in}{3.767934in}}%
\pgfpathlineto{\pgfqpoint{1.895109in}{3.614091in}}%
\pgfpathlineto{\pgfqpoint{1.895488in}{3.519341in}}%
\pgfpathlineto{\pgfqpoint{1.895962in}{3.673402in}}%
\pgfpathlineto{\pgfqpoint{1.896341in}{3.748389in}}%
\pgfpathlineto{\pgfqpoint{1.897004in}{3.672041in}}%
\pgfpathlineto{\pgfqpoint{1.897573in}{3.773109in}}%
\pgfpathlineto{\pgfqpoint{1.897762in}{3.788030in}}%
\pgfpathlineto{\pgfqpoint{1.898047in}{3.723274in}}%
\pgfpathlineto{\pgfqpoint{1.898615in}{3.557147in}}%
\pgfpathlineto{\pgfqpoint{1.899089in}{3.723198in}}%
\pgfpathlineto{\pgfqpoint{1.899563in}{3.824997in}}%
\pgfpathlineto{\pgfqpoint{1.900131in}{3.698867in}}%
\pgfpathlineto{\pgfqpoint{1.900226in}{3.691314in}}%
\pgfpathlineto{\pgfqpoint{1.900510in}{3.745662in}}%
\pgfpathlineto{\pgfqpoint{1.900890in}{3.811914in}}%
\pgfpathlineto{\pgfqpoint{1.901363in}{3.697948in}}%
\pgfpathlineto{\pgfqpoint{1.901742in}{3.595415in}}%
\pgfpathlineto{\pgfqpoint{1.902216in}{3.761035in}}%
\pgfpathlineto{\pgfqpoint{1.902690in}{3.847716in}}%
\pgfpathlineto{\pgfqpoint{1.903164in}{3.722760in}}%
\pgfpathlineto{\pgfqpoint{1.904017in}{3.767116in}}%
\pgfpathlineto{\pgfqpoint{1.904870in}{3.570913in}}%
\pgfpathlineto{\pgfqpoint{1.905343in}{3.740234in}}%
\pgfpathlineto{\pgfqpoint{1.905723in}{3.830219in}}%
\pgfpathlineto{\pgfqpoint{1.906291in}{3.693959in}}%
\pgfpathlineto{\pgfqpoint{1.906575in}{3.619155in}}%
\pgfpathlineto{\pgfqpoint{1.907144in}{3.712824in}}%
\pgfpathlineto{\pgfqpoint{1.907428in}{3.677072in}}%
\pgfpathlineto{\pgfqpoint{1.907997in}{3.583718in}}%
\pgfpathlineto{\pgfqpoint{1.908376in}{3.680176in}}%
\pgfpathlineto{\pgfqpoint{1.908850in}{3.820289in}}%
\pgfpathlineto{\pgfqpoint{1.909324in}{3.664148in}}%
\pgfpathlineto{\pgfqpoint{1.909703in}{3.557573in}}%
\pgfpathlineto{\pgfqpoint{1.910461in}{3.638980in}}%
\pgfpathlineto{\pgfqpoint{1.911124in}{3.498127in}}%
\pgfpathlineto{\pgfqpoint{1.911503in}{3.605466in}}%
\pgfpathlineto{\pgfqpoint{1.911977in}{3.727788in}}%
\pgfpathlineto{\pgfqpoint{1.912451in}{3.563579in}}%
\pgfpathlineto{\pgfqpoint{1.912830in}{3.452190in}}%
\pgfpathlineto{\pgfqpoint{1.913493in}{3.561168in}}%
\pgfpathlineto{\pgfqpoint{1.913588in}{3.562562in}}%
\pgfpathlineto{\pgfqpoint{1.913683in}{3.556981in}}%
\pgfpathlineto{\pgfqpoint{1.914251in}{3.478206in}}%
\pgfpathlineto{\pgfqpoint{1.914630in}{3.562300in}}%
\pgfpathlineto{\pgfqpoint{1.915104in}{3.680523in}}%
\pgfpathlineto{\pgfqpoint{1.915578in}{3.518256in}}%
\pgfpathlineto{\pgfqpoint{1.915957in}{3.419505in}}%
\pgfpathlineto{\pgfqpoint{1.916526in}{3.544374in}}%
\pgfpathlineto{\pgfqpoint{1.916715in}{3.558986in}}%
\pgfpathlineto{\pgfqpoint{1.917094in}{3.504897in}}%
\pgfpathlineto{\pgfqpoint{1.917378in}{3.474963in}}%
\pgfpathlineto{\pgfqpoint{1.917758in}{3.582988in}}%
\pgfpathlineto{\pgfqpoint{1.918231in}{3.709504in}}%
\pgfpathlineto{\pgfqpoint{1.918705in}{3.536171in}}%
\pgfpathlineto{\pgfqpoint{1.919084in}{3.438560in}}%
\pgfpathlineto{\pgfqpoint{1.919558in}{3.580427in}}%
\pgfpathlineto{\pgfqpoint{1.919842in}{3.659357in}}%
\pgfpathlineto{\pgfqpoint{1.920695in}{3.596646in}}%
\pgfpathlineto{\pgfqpoint{1.921264in}{3.717481in}}%
\pgfpathlineto{\pgfqpoint{1.921738in}{3.610309in}}%
\pgfpathlineto{\pgfqpoint{1.922211in}{3.429714in}}%
\pgfpathlineto{\pgfqpoint{1.922780in}{3.569702in}}%
\pgfpathlineto{\pgfqpoint{1.923064in}{3.626297in}}%
\pgfpathlineto{\pgfqpoint{1.923728in}{3.531343in}}%
\pgfpathlineto{\pgfqpoint{1.924391in}{3.645658in}}%
\pgfpathlineto{\pgfqpoint{1.924865in}{3.540109in}}%
\pgfpathlineto{\pgfqpoint{1.925339in}{3.391729in}}%
\pgfpathlineto{\pgfqpoint{1.925813in}{3.545887in}}%
\pgfpathlineto{\pgfqpoint{1.926192in}{3.618511in}}%
\pgfpathlineto{\pgfqpoint{1.926760in}{3.521063in}}%
\pgfpathlineto{\pgfqpoint{1.926950in}{3.495731in}}%
\pgfpathlineto{\pgfqpoint{1.927518in}{3.584015in}}%
\pgfpathlineto{\pgfqpoint{1.927613in}{3.589789in}}%
\pgfpathlineto{\pgfqpoint{1.927897in}{3.546303in}}%
\pgfpathlineto{\pgfqpoint{1.928466in}{3.385338in}}%
\pgfpathlineto{\pgfqpoint{1.928940in}{3.526760in}}%
\pgfpathlineto{\pgfqpoint{1.930835in}{3.986434in}}%
\pgfpathlineto{\pgfqpoint{1.931025in}{3.950629in}}%
\pgfpathlineto{\pgfqpoint{1.933015in}{3.413538in}}%
\pgfpathlineto{\pgfqpoint{1.933299in}{3.464725in}}%
\pgfpathlineto{\pgfqpoint{1.933962in}{3.613274in}}%
\pgfpathlineto{\pgfqpoint{1.934436in}{3.481856in}}%
\pgfpathlineto{\pgfqpoint{1.934720in}{3.428521in}}%
\pgfpathlineto{\pgfqpoint{1.935194in}{3.578611in}}%
\pgfpathlineto{\pgfqpoint{1.935573in}{3.688561in}}%
\pgfpathlineto{\pgfqpoint{1.936142in}{3.507255in}}%
\pgfpathlineto{\pgfqpoint{1.936426in}{3.460746in}}%
\pgfpathlineto{\pgfqpoint{1.937089in}{3.544249in}}%
\pgfpathlineto{\pgfqpoint{1.937184in}{3.549224in}}%
\pgfpathlineto{\pgfqpoint{1.937374in}{3.526978in}}%
\pgfpathlineto{\pgfqpoint{1.937753in}{3.439764in}}%
\pgfpathlineto{\pgfqpoint{1.938227in}{3.555192in}}%
\pgfpathlineto{\pgfqpoint{1.938700in}{3.697458in}}%
\pgfpathlineto{\pgfqpoint{1.939174in}{3.555568in}}%
\pgfpathlineto{\pgfqpoint{1.939553in}{3.447760in}}%
\pgfpathlineto{\pgfqpoint{1.940217in}{3.565151in}}%
\pgfpathlineto{\pgfqpoint{1.940311in}{3.574318in}}%
\pgfpathlineto{\pgfqpoint{1.940690in}{3.512240in}}%
\pgfpathlineto{\pgfqpoint{1.940880in}{3.478374in}}%
\pgfpathlineto{\pgfqpoint{1.941354in}{3.602396in}}%
\pgfpathlineto{\pgfqpoint{1.941922in}{3.733828in}}%
\pgfpathlineto{\pgfqpoint{1.942301in}{3.609232in}}%
\pgfpathlineto{\pgfqpoint{1.942681in}{3.492739in}}%
\pgfpathlineto{\pgfqpoint{1.943249in}{3.648066in}}%
\pgfpathlineto{\pgfqpoint{1.944955in}{3.814069in}}%
\pgfpathlineto{\pgfqpoint{1.944007in}{3.623717in}}%
\pgfpathlineto{\pgfqpoint{1.945050in}{3.810881in}}%
\pgfpathlineto{\pgfqpoint{1.945808in}{3.554124in}}%
\pgfpathlineto{\pgfqpoint{1.946471in}{3.724557in}}%
\pgfpathlineto{\pgfqpoint{1.946661in}{3.749360in}}%
\pgfpathlineto{\pgfqpoint{1.947134in}{3.660114in}}%
\pgfpathlineto{\pgfqpoint{1.947324in}{3.647409in}}%
\pgfpathlineto{\pgfqpoint{1.947703in}{3.720031in}}%
\pgfpathlineto{\pgfqpoint{1.947987in}{3.743097in}}%
\pgfpathlineto{\pgfqpoint{1.948366in}{3.669253in}}%
\pgfpathlineto{\pgfqpoint{1.949030in}{3.411181in}}%
\pgfpathlineto{\pgfqpoint{1.949598in}{3.568521in}}%
\pgfpathlineto{\pgfqpoint{1.949693in}{3.577515in}}%
\pgfpathlineto{\pgfqpoint{1.950072in}{3.511083in}}%
\pgfpathlineto{\pgfqpoint{1.950641in}{3.435279in}}%
\pgfpathlineto{\pgfqpoint{1.951209in}{3.500686in}}%
\pgfpathlineto{\pgfqpoint{1.951683in}{3.328806in}}%
\pgfpathlineto{\pgfqpoint{1.952157in}{3.220401in}}%
\pgfpathlineto{\pgfqpoint{1.952631in}{3.366356in}}%
\pgfpathlineto{\pgfqpoint{1.953010in}{3.411587in}}%
\pgfpathlineto{\pgfqpoint{1.953484in}{3.300829in}}%
\pgfpathlineto{\pgfqpoint{1.955284in}{3.014880in}}%
\pgfpathlineto{\pgfqpoint{1.954147in}{3.327253in}}%
\pgfpathlineto{\pgfqpoint{1.955474in}{3.049871in}}%
\pgfpathlineto{\pgfqpoint{1.956042in}{3.232364in}}%
\pgfpathlineto{\pgfqpoint{1.956706in}{3.118365in}}%
\pgfpathlineto{\pgfqpoint{1.957085in}{3.168557in}}%
\pgfpathlineto{\pgfqpoint{1.957558in}{3.242565in}}%
\pgfpathlineto{\pgfqpoint{1.958032in}{3.162313in}}%
\pgfpathlineto{\pgfqpoint{1.958222in}{3.133536in}}%
\pgfpathlineto{\pgfqpoint{1.958601in}{3.207358in}}%
\pgfpathlineto{\pgfqpoint{1.959169in}{3.428083in}}%
\pgfpathlineto{\pgfqpoint{1.959833in}{3.298502in}}%
\pgfpathlineto{\pgfqpoint{1.960022in}{3.286174in}}%
\pgfpathlineto{\pgfqpoint{1.960307in}{3.338648in}}%
\pgfpathlineto{\pgfqpoint{1.960686in}{3.400530in}}%
\pgfpathlineto{\pgfqpoint{1.961254in}{3.301380in}}%
\pgfpathlineto{\pgfqpoint{1.961349in}{3.289485in}}%
\pgfpathlineto{\pgfqpoint{1.961633in}{3.366736in}}%
\pgfpathlineto{\pgfqpoint{1.962391in}{3.615286in}}%
\pgfpathlineto{\pgfqpoint{1.962865in}{3.501076in}}%
\pgfpathlineto{\pgfqpoint{1.963055in}{3.462985in}}%
\pgfpathlineto{\pgfqpoint{1.963623in}{3.569107in}}%
\pgfpathlineto{\pgfqpoint{1.963908in}{3.607633in}}%
\pgfpathlineto{\pgfqpoint{1.964381in}{3.516711in}}%
\pgfpathlineto{\pgfqpoint{1.964476in}{3.509111in}}%
\pgfpathlineto{\pgfqpoint{1.964761in}{3.572392in}}%
\pgfpathlineto{\pgfqpoint{1.965424in}{3.819169in}}%
\pgfpathlineto{\pgfqpoint{1.965992in}{3.643866in}}%
\pgfpathlineto{\pgfqpoint{1.966372in}{3.575106in}}%
\pgfpathlineto{\pgfqpoint{1.966940in}{3.674849in}}%
\pgfpathlineto{\pgfqpoint{1.967035in}{3.678105in}}%
\pgfpathlineto{\pgfqpoint{1.967224in}{3.659875in}}%
\pgfpathlineto{\pgfqpoint{1.967603in}{3.608274in}}%
\pgfpathlineto{\pgfqpoint{1.967983in}{3.685619in}}%
\pgfpathlineto{\pgfqpoint{1.968551in}{3.842294in}}%
\pgfpathlineto{\pgfqpoint{1.969120in}{3.705940in}}%
\pgfpathlineto{\pgfqpoint{1.969499in}{3.638659in}}%
\pgfpathlineto{\pgfqpoint{1.969973in}{3.743109in}}%
\pgfpathlineto{\pgfqpoint{1.970162in}{3.761928in}}%
\pgfpathlineto{\pgfqpoint{1.970731in}{3.708011in}}%
\pgfpathlineto{\pgfqpoint{1.970825in}{3.699471in}}%
\pgfpathlineto{\pgfqpoint{1.971110in}{3.735843in}}%
\pgfpathlineto{\pgfqpoint{1.971584in}{3.869780in}}%
\pgfpathlineto{\pgfqpoint{1.972152in}{3.734084in}}%
\pgfpathlineto{\pgfqpoint{1.972531in}{3.598745in}}%
\pgfpathlineto{\pgfqpoint{1.973195in}{3.722891in}}%
\pgfpathlineto{\pgfqpoint{1.973763in}{3.706528in}}%
\pgfpathlineto{\pgfqpoint{1.974332in}{3.991768in}}%
\pgfpathlineto{\pgfqpoint{1.974806in}{4.257286in}}%
\pgfpathlineto{\pgfqpoint{1.975469in}{4.041520in}}%
\pgfpathlineto{\pgfqpoint{1.976890in}{3.611560in}}%
\pgfpathlineto{\pgfqpoint{1.977269in}{3.709990in}}%
\pgfpathlineto{\pgfqpoint{1.977933in}{3.915057in}}%
\pgfpathlineto{\pgfqpoint{1.978407in}{3.782130in}}%
\pgfpathlineto{\pgfqpoint{1.978786in}{3.673878in}}%
\pgfpathlineto{\pgfqpoint{1.979354in}{3.839093in}}%
\pgfpathlineto{\pgfqpoint{1.979733in}{3.910555in}}%
\pgfpathlineto{\pgfqpoint{1.980397in}{3.824070in}}%
\pgfpathlineto{\pgfqpoint{1.980776in}{3.881370in}}%
\pgfpathlineto{\pgfqpoint{1.981060in}{3.926707in}}%
\pgfpathlineto{\pgfqpoint{1.981439in}{3.818944in}}%
\pgfpathlineto{\pgfqpoint{1.981913in}{3.706511in}}%
\pgfpathlineto{\pgfqpoint{1.982387in}{3.859260in}}%
\pgfpathlineto{\pgfqpoint{1.982860in}{3.972678in}}%
\pgfpathlineto{\pgfqpoint{1.983429in}{3.864806in}}%
\pgfpathlineto{\pgfqpoint{1.983524in}{3.855015in}}%
\pgfpathlineto{\pgfqpoint{1.983903in}{3.915400in}}%
\pgfpathlineto{\pgfqpoint{1.984187in}{3.950720in}}%
\pgfpathlineto{\pgfqpoint{1.984566in}{3.874198in}}%
\pgfpathlineto{\pgfqpoint{1.985040in}{3.760911in}}%
\pgfpathlineto{\pgfqpoint{1.985514in}{3.902512in}}%
\pgfpathlineto{\pgfqpoint{1.985893in}{4.037916in}}%
\pgfpathlineto{\pgfqpoint{1.986556in}{3.876702in}}%
\pgfpathlineto{\pgfqpoint{1.986651in}{3.864419in}}%
\pgfpathlineto{\pgfqpoint{1.987125in}{3.929694in}}%
\pgfpathlineto{\pgfqpoint{1.987409in}{3.957337in}}%
\pgfpathlineto{\pgfqpoint{1.987788in}{3.884738in}}%
\pgfpathlineto{\pgfqpoint{1.988167in}{3.776982in}}%
\pgfpathlineto{\pgfqpoint{1.988641in}{3.980508in}}%
\pgfpathlineto{\pgfqpoint{1.989020in}{4.082731in}}%
\pgfpathlineto{\pgfqpoint{1.989589in}{3.951425in}}%
\pgfpathlineto{\pgfqpoint{1.989968in}{3.895044in}}%
\pgfpathlineto{\pgfqpoint{1.990536in}{3.980568in}}%
\pgfpathlineto{\pgfqpoint{1.990821in}{3.946018in}}%
\pgfpathlineto{\pgfqpoint{1.991295in}{3.877000in}}%
\pgfpathlineto{\pgfqpoint{1.991674in}{3.954889in}}%
\pgfpathlineto{\pgfqpoint{1.992147in}{4.118512in}}%
\pgfpathlineto{\pgfqpoint{1.992716in}{3.963430in}}%
\pgfpathlineto{\pgfqpoint{1.993000in}{3.893572in}}%
\pgfpathlineto{\pgfqpoint{1.993758in}{3.971972in}}%
\pgfpathlineto{\pgfqpoint{1.994137in}{3.923863in}}%
\pgfpathlineto{\pgfqpoint{1.994422in}{3.879400in}}%
\pgfpathlineto{\pgfqpoint{1.994801in}{3.973889in}}%
\pgfpathlineto{\pgfqpoint{1.995275in}{4.109101in}}%
\pgfpathlineto{\pgfqpoint{1.995748in}{3.961693in}}%
\pgfpathlineto{\pgfqpoint{1.996127in}{3.842681in}}%
\pgfpathlineto{\pgfqpoint{1.996791in}{3.972467in}}%
\pgfpathlineto{\pgfqpoint{1.996886in}{3.977105in}}%
\pgfpathlineto{\pgfqpoint{1.997075in}{3.954244in}}%
\pgfpathlineto{\pgfqpoint{1.997454in}{3.874830in}}%
\pgfpathlineto{\pgfqpoint{1.998023in}{3.980957in}}%
\pgfpathlineto{\pgfqpoint{1.998402in}{4.057159in}}%
\pgfpathlineto{\pgfqpoint{1.998781in}{3.952777in}}%
\pgfpathlineto{\pgfqpoint{1.999255in}{3.781537in}}%
\pgfpathlineto{\pgfqpoint{1.999918in}{3.903598in}}%
\pgfpathlineto{\pgfqpoint{2.000108in}{3.924967in}}%
\pgfpathlineto{\pgfqpoint{2.000676in}{3.858877in}}%
\pgfpathlineto{\pgfqpoint{2.000771in}{3.854531in}}%
\pgfpathlineto{\pgfqpoint{2.000960in}{3.873554in}}%
\pgfpathlineto{\pgfqpoint{2.001529in}{3.998730in}}%
\pgfpathlineto{\pgfqpoint{2.001908in}{3.892475in}}%
\pgfpathlineto{\pgfqpoint{2.002382in}{3.728381in}}%
\pgfpathlineto{\pgfqpoint{2.002950in}{3.868282in}}%
\pgfpathlineto{\pgfqpoint{2.003235in}{3.918484in}}%
\pgfpathlineto{\pgfqpoint{2.003803in}{3.852938in}}%
\pgfpathlineto{\pgfqpoint{2.003993in}{3.862043in}}%
\pgfpathlineto{\pgfqpoint{2.004751in}{3.964446in}}%
\pgfpathlineto{\pgfqpoint{2.005035in}{3.886022in}}%
\pgfpathlineto{\pgfqpoint{2.005509in}{3.720065in}}%
\pgfpathlineto{\pgfqpoint{2.006078in}{3.892203in}}%
\pgfpathlineto{\pgfqpoint{2.006457in}{3.986105in}}%
\pgfpathlineto{\pgfqpoint{2.007025in}{3.887477in}}%
\pgfpathlineto{\pgfqpoint{2.007215in}{3.903690in}}%
\pgfpathlineto{\pgfqpoint{2.007783in}{4.012906in}}%
\pgfpathlineto{\pgfqpoint{2.008163in}{3.904517in}}%
\pgfpathlineto{\pgfqpoint{2.008636in}{3.771058in}}%
\pgfpathlineto{\pgfqpoint{2.009110in}{3.899457in}}%
\pgfpathlineto{\pgfqpoint{2.009584in}{4.021719in}}%
\pgfpathlineto{\pgfqpoint{2.010058in}{3.869229in}}%
\pgfpathlineto{\pgfqpoint{2.010816in}{3.891577in}}%
\pgfpathlineto{\pgfqpoint{2.011669in}{3.677886in}}%
\pgfpathlineto{\pgfqpoint{2.011764in}{3.676193in}}%
\pgfpathlineto{\pgfqpoint{2.011858in}{3.687365in}}%
\pgfpathlineto{\pgfqpoint{2.012616in}{3.924655in}}%
\pgfpathlineto{\pgfqpoint{2.013375in}{3.781889in}}%
\pgfpathlineto{\pgfqpoint{2.013469in}{3.773695in}}%
\pgfpathlineto{\pgfqpoint{2.013943in}{3.824807in}}%
\pgfpathlineto{\pgfqpoint{2.014133in}{3.835494in}}%
\pgfpathlineto{\pgfqpoint{2.014417in}{3.781770in}}%
\pgfpathlineto{\pgfqpoint{2.014891in}{3.662033in}}%
\pgfpathlineto{\pgfqpoint{2.015365in}{3.804635in}}%
\pgfpathlineto{\pgfqpoint{2.015744in}{3.932758in}}%
\pgfpathlineto{\pgfqpoint{2.016407in}{3.770660in}}%
\pgfpathlineto{\pgfqpoint{2.017165in}{3.793021in}}%
\pgfpathlineto{\pgfqpoint{2.017639in}{3.732453in}}%
\pgfpathlineto{\pgfqpoint{2.017828in}{3.752684in}}%
\pgfpathlineto{\pgfqpoint{2.018871in}{4.380349in}}%
\pgfpathlineto{\pgfqpoint{2.019819in}{4.075374in}}%
\pgfpathlineto{\pgfqpoint{2.020482in}{3.778408in}}%
\pgfpathlineto{\pgfqpoint{2.020861in}{3.644373in}}%
\pgfpathlineto{\pgfqpoint{2.021429in}{3.805250in}}%
\pgfpathlineto{\pgfqpoint{2.022093in}{3.985341in}}%
\pgfpathlineto{\pgfqpoint{2.022567in}{3.819720in}}%
\pgfpathlineto{\pgfqpoint{2.022946in}{3.751341in}}%
\pgfpathlineto{\pgfqpoint{2.023514in}{3.830334in}}%
\pgfpathlineto{\pgfqpoint{2.023704in}{3.854405in}}%
\pgfpathlineto{\pgfqpoint{2.024178in}{3.752244in}}%
\pgfpathlineto{\pgfqpoint{2.024272in}{3.752145in}}%
\pgfpathlineto{\pgfqpoint{2.025125in}{3.911314in}}%
\pgfpathlineto{\pgfqpoint{2.025504in}{3.786571in}}%
\pgfpathlineto{\pgfqpoint{2.026168in}{3.568945in}}%
\pgfpathlineto{\pgfqpoint{2.026736in}{3.689623in}}%
\pgfpathlineto{\pgfqpoint{2.027021in}{3.660205in}}%
\pgfpathlineto{\pgfqpoint{2.027589in}{3.567510in}}%
\pgfpathlineto{\pgfqpoint{2.027968in}{3.664904in}}%
\pgfpathlineto{\pgfqpoint{2.028158in}{3.703072in}}%
\pgfpathlineto{\pgfqpoint{2.028632in}{3.553666in}}%
\pgfpathlineto{\pgfqpoint{2.029295in}{3.375481in}}%
\pgfpathlineto{\pgfqpoint{2.029769in}{3.494459in}}%
\pgfpathlineto{\pgfqpoint{2.029958in}{3.514975in}}%
\pgfpathlineto{\pgfqpoint{2.030432in}{3.447788in}}%
\pgfpathlineto{\pgfqpoint{2.030716in}{3.397010in}}%
\pgfpathlineto{\pgfqpoint{2.031285in}{3.497449in}}%
\pgfpathlineto{\pgfqpoint{2.031380in}{3.491025in}}%
\pgfpathlineto{\pgfqpoint{2.032327in}{3.165732in}}%
\pgfpathlineto{\pgfqpoint{2.033180in}{3.364329in}}%
\pgfpathlineto{\pgfqpoint{2.033749in}{3.302981in}}%
\pgfpathlineto{\pgfqpoint{2.034128in}{3.374883in}}%
\pgfpathlineto{\pgfqpoint{2.034507in}{3.482135in}}%
\pgfpathlineto{\pgfqpoint{2.035076in}{3.366014in}}%
\pgfpathlineto{\pgfqpoint{2.035360in}{3.307222in}}%
\pgfpathlineto{\pgfqpoint{2.035834in}{3.435036in}}%
\pgfpathlineto{\pgfqpoint{2.036307in}{3.570449in}}%
\pgfpathlineto{\pgfqpoint{2.036971in}{3.445447in}}%
\pgfpathlineto{\pgfqpoint{2.037729in}{3.537686in}}%
\pgfpathlineto{\pgfqpoint{2.038108in}{3.449569in}}%
\pgfpathlineto{\pgfqpoint{2.038487in}{3.361475in}}%
\pgfpathlineto{\pgfqpoint{2.038961in}{3.489624in}}%
\pgfpathlineto{\pgfqpoint{2.039340in}{3.642757in}}%
\pgfpathlineto{\pgfqpoint{2.040098in}{3.507245in}}%
\pgfpathlineto{\pgfqpoint{2.040193in}{3.505642in}}%
\pgfpathlineto{\pgfqpoint{2.040382in}{3.519433in}}%
\pgfpathlineto{\pgfqpoint{2.040856in}{3.562170in}}%
\pgfpathlineto{\pgfqpoint{2.041140in}{3.513909in}}%
\pgfpathlineto{\pgfqpoint{2.041614in}{3.412819in}}%
\pgfpathlineto{\pgfqpoint{2.042088in}{3.541530in}}%
\pgfpathlineto{\pgfqpoint{2.042467in}{3.667975in}}%
\pgfpathlineto{\pgfqpoint{2.043130in}{3.506870in}}%
\pgfpathlineto{\pgfqpoint{2.043320in}{3.463540in}}%
\pgfpathlineto{\pgfqpoint{2.043983in}{3.546165in}}%
\pgfpathlineto{\pgfqpoint{2.044078in}{3.540637in}}%
\pgfpathlineto{\pgfqpoint{2.044741in}{3.407929in}}%
\pgfpathlineto{\pgfqpoint{2.045121in}{3.502467in}}%
\pgfpathlineto{\pgfqpoint{2.045689in}{3.659324in}}%
\pgfpathlineto{\pgfqpoint{2.046163in}{3.517799in}}%
\pgfpathlineto{\pgfqpoint{2.046447in}{3.436702in}}%
\pgfpathlineto{\pgfqpoint{2.047111in}{3.543552in}}%
\pgfpathlineto{\pgfqpoint{2.047205in}{3.540206in}}%
\pgfpathlineto{\pgfqpoint{2.047869in}{3.429061in}}%
\pgfpathlineto{\pgfqpoint{2.048248in}{3.536293in}}%
\pgfpathlineto{\pgfqpoint{2.048722in}{3.674100in}}%
\pgfpathlineto{\pgfqpoint{2.049290in}{3.540330in}}%
\pgfpathlineto{\pgfqpoint{2.049669in}{3.486036in}}%
\pgfpathlineto{\pgfqpoint{2.050143in}{3.577730in}}%
\pgfpathlineto{\pgfqpoint{2.050996in}{3.544992in}}%
\pgfpathlineto{\pgfqpoint{2.051849in}{3.779231in}}%
\pgfpathlineto{\pgfqpoint{2.052323in}{3.618955in}}%
\pgfpathlineto{\pgfqpoint{2.052702in}{3.501695in}}%
\pgfpathlineto{\pgfqpoint{2.053460in}{3.587597in}}%
\pgfpathlineto{\pgfqpoint{2.053649in}{3.580103in}}%
\pgfpathlineto{\pgfqpoint{2.054218in}{3.476405in}}%
\pgfpathlineto{\pgfqpoint{2.054692in}{3.572220in}}%
\pgfpathlineto{\pgfqpoint{2.054976in}{3.610486in}}%
\pgfpathlineto{\pgfqpoint{2.055355in}{3.515480in}}%
\pgfpathlineto{\pgfqpoint{2.055924in}{3.330247in}}%
\pgfpathlineto{\pgfqpoint{2.056492in}{3.471977in}}%
\pgfpathlineto{\pgfqpoint{2.056777in}{3.502664in}}%
\pgfpathlineto{\pgfqpoint{2.057345in}{3.426726in}}%
\pgfpathlineto{\pgfqpoint{2.057819in}{3.506737in}}%
\pgfpathlineto{\pgfqpoint{2.058103in}{3.551152in}}%
\pgfpathlineto{\pgfqpoint{2.058482in}{3.457944in}}%
\pgfpathlineto{\pgfqpoint{2.058956in}{3.299197in}}%
\pgfpathlineto{\pgfqpoint{2.059525in}{3.437363in}}%
\pgfpathlineto{\pgfqpoint{2.059809in}{3.489991in}}%
\pgfpathlineto{\pgfqpoint{2.060567in}{3.427386in}}%
\pgfpathlineto{\pgfqpoint{2.060757in}{3.421541in}}%
\pgfpathlineto{\pgfqpoint{2.061041in}{3.445470in}}%
\pgfpathlineto{\pgfqpoint{2.062557in}{3.806918in}}%
\pgfpathlineto{\pgfqpoint{2.062936in}{3.899762in}}%
\pgfpathlineto{\pgfqpoint{2.063505in}{3.784349in}}%
\pgfpathlineto{\pgfqpoint{2.065211in}{3.298250in}}%
\pgfpathlineto{\pgfqpoint{2.065590in}{3.374991in}}%
\pgfpathlineto{\pgfqpoint{2.066158in}{3.545511in}}%
\pgfpathlineto{\pgfqpoint{2.066727in}{3.432535in}}%
\pgfpathlineto{\pgfqpoint{2.066916in}{3.409372in}}%
\pgfpathlineto{\pgfqpoint{2.067485in}{3.486148in}}%
\pgfpathlineto{\pgfqpoint{2.067580in}{3.485347in}}%
\pgfpathlineto{\pgfqpoint{2.068338in}{3.304429in}}%
\pgfpathlineto{\pgfqpoint{2.068812in}{3.434683in}}%
\pgfpathlineto{\pgfqpoint{2.069285in}{3.580289in}}%
\pgfpathlineto{\pgfqpoint{2.069854in}{3.442545in}}%
\pgfpathlineto{\pgfqpoint{2.070707in}{3.473294in}}%
\pgfpathlineto{\pgfqpoint{2.071370in}{3.343387in}}%
\pgfpathlineto{\pgfqpoint{2.071465in}{3.334818in}}%
\pgfpathlineto{\pgfqpoint{2.071749in}{3.383238in}}%
\pgfpathlineto{\pgfqpoint{2.072413in}{3.573505in}}%
\pgfpathlineto{\pgfqpoint{2.072981in}{3.439018in}}%
\pgfpathlineto{\pgfqpoint{2.073265in}{3.408121in}}%
\pgfpathlineto{\pgfqpoint{2.073834in}{3.481376in}}%
\pgfpathlineto{\pgfqpoint{2.073929in}{3.482301in}}%
\pgfpathlineto{\pgfqpoint{2.074024in}{3.474137in}}%
\pgfpathlineto{\pgfqpoint{2.074592in}{3.380855in}}%
\pgfpathlineto{\pgfqpoint{2.074971in}{3.451565in}}%
\pgfpathlineto{\pgfqpoint{2.075445in}{3.618722in}}%
\pgfpathlineto{\pgfqpoint{2.076014in}{3.462781in}}%
\pgfpathlineto{\pgfqpoint{2.076393in}{3.377334in}}%
\pgfpathlineto{\pgfqpoint{2.077056in}{3.463999in}}%
\pgfpathlineto{\pgfqpoint{2.077151in}{3.464263in}}%
\pgfpathlineto{\pgfqpoint{2.077814in}{3.415387in}}%
\pgfpathlineto{\pgfqpoint{2.078098in}{3.465607in}}%
\pgfpathlineto{\pgfqpoint{2.078572in}{3.616754in}}%
\pgfpathlineto{\pgfqpoint{2.079141in}{3.443282in}}%
\pgfpathlineto{\pgfqpoint{2.079520in}{3.369068in}}%
\pgfpathlineto{\pgfqpoint{2.080088in}{3.479858in}}%
\pgfpathlineto{\pgfqpoint{2.080183in}{3.488197in}}%
\pgfpathlineto{\pgfqpoint{2.080657in}{3.435395in}}%
\pgfpathlineto{\pgfqpoint{2.080847in}{3.417591in}}%
\pgfpathlineto{\pgfqpoint{2.081226in}{3.478886in}}%
\pgfpathlineto{\pgfqpoint{2.081699in}{3.608142in}}%
\pgfpathlineto{\pgfqpoint{2.082173in}{3.465817in}}%
\pgfpathlineto{\pgfqpoint{2.082647in}{3.322717in}}%
\pgfpathlineto{\pgfqpoint{2.083216in}{3.461907in}}%
\pgfpathlineto{\pgfqpoint{2.083500in}{3.490414in}}%
\pgfpathlineto{\pgfqpoint{2.083974in}{3.421804in}}%
\pgfpathlineto{\pgfqpoint{2.084163in}{3.410776in}}%
\pgfpathlineto{\pgfqpoint{2.084448in}{3.475850in}}%
\pgfpathlineto{\pgfqpoint{2.084827in}{3.547366in}}%
\pgfpathlineto{\pgfqpoint{2.085301in}{3.397156in}}%
\pgfpathlineto{\pgfqpoint{2.085774in}{3.272564in}}%
\pgfpathlineto{\pgfqpoint{2.086343in}{3.381446in}}%
\pgfpathlineto{\pgfqpoint{2.086627in}{3.439985in}}%
\pgfpathlineto{\pgfqpoint{2.087291in}{3.345581in}}%
\pgfpathlineto{\pgfqpoint{2.087670in}{3.399828in}}%
\pgfpathlineto{\pgfqpoint{2.087954in}{3.437473in}}%
\pgfpathlineto{\pgfqpoint{2.088333in}{3.339663in}}%
\pgfpathlineto{\pgfqpoint{2.088902in}{3.182049in}}%
\pgfpathlineto{\pgfqpoint{2.089375in}{3.322128in}}%
\pgfpathlineto{\pgfqpoint{2.089754in}{3.385378in}}%
\pgfpathlineto{\pgfqpoint{2.090323in}{3.296512in}}%
\pgfpathlineto{\pgfqpoint{2.090418in}{3.289752in}}%
\pgfpathlineto{\pgfqpoint{2.090702in}{3.324490in}}%
\pgfpathlineto{\pgfqpoint{2.091081in}{3.384935in}}%
\pgfpathlineto{\pgfqpoint{2.091460in}{3.290631in}}%
\pgfpathlineto{\pgfqpoint{2.091934in}{3.168493in}}%
\pgfpathlineto{\pgfqpoint{2.092503in}{3.315790in}}%
\pgfpathlineto{\pgfqpoint{2.092882in}{3.406580in}}%
\pgfpathlineto{\pgfqpoint{2.093545in}{3.320983in}}%
\pgfpathlineto{\pgfqpoint{2.093829in}{3.343589in}}%
\pgfpathlineto{\pgfqpoint{2.094303in}{3.394755in}}%
\pgfpathlineto{\pgfqpoint{2.094682in}{3.312538in}}%
\pgfpathlineto{\pgfqpoint{2.094966in}{3.251696in}}%
\pgfpathlineto{\pgfqpoint{2.095535in}{3.356694in}}%
\pgfpathlineto{\pgfqpoint{2.096009in}{3.491712in}}%
\pgfpathlineto{\pgfqpoint{2.096577in}{3.363613in}}%
\pgfpathlineto{\pgfqpoint{2.098283in}{3.185394in}}%
\pgfpathlineto{\pgfqpoint{2.097336in}{3.364379in}}%
\pgfpathlineto{\pgfqpoint{2.098378in}{3.198621in}}%
\pgfpathlineto{\pgfqpoint{2.099136in}{3.425597in}}%
\pgfpathlineto{\pgfqpoint{2.099610in}{3.285792in}}%
\pgfpathlineto{\pgfqpoint{2.100937in}{3.215518in}}%
\pgfpathlineto{\pgfqpoint{2.101221in}{3.178637in}}%
\pgfpathlineto{\pgfqpoint{2.101695in}{3.250575in}}%
\pgfpathlineto{\pgfqpoint{2.102358in}{3.410893in}}%
\pgfpathlineto{\pgfqpoint{2.102737in}{3.274414in}}%
\pgfpathlineto{\pgfqpoint{2.103021in}{3.185136in}}%
\pgfpathlineto{\pgfqpoint{2.103780in}{3.268445in}}%
\pgfpathlineto{\pgfqpoint{2.103969in}{3.258692in}}%
\pgfpathlineto{\pgfqpoint{2.104443in}{3.185745in}}%
\pgfpathlineto{\pgfqpoint{2.104917in}{3.262172in}}%
\pgfpathlineto{\pgfqpoint{2.106907in}{3.691543in}}%
\pgfpathlineto{\pgfqpoint{2.107096in}{3.673255in}}%
\pgfpathlineto{\pgfqpoint{2.107760in}{3.413351in}}%
\pgfpathlineto{\pgfqpoint{2.108233in}{3.281062in}}%
\pgfpathlineto{\pgfqpoint{2.109086in}{3.297542in}}%
\pgfpathlineto{\pgfqpoint{2.109371in}{3.242165in}}%
\pgfpathlineto{\pgfqpoint{2.109844in}{3.362909in}}%
\pgfpathlineto{\pgfqpoint{2.110129in}{3.401301in}}%
\pgfpathlineto{\pgfqpoint{2.110697in}{3.323443in}}%
\pgfpathlineto{\pgfqpoint{2.110792in}{3.313300in}}%
\pgfpathlineto{\pgfqpoint{2.111171in}{3.372921in}}%
\pgfpathlineto{\pgfqpoint{2.111550in}{3.427865in}}%
\pgfpathlineto{\pgfqpoint{2.111929in}{3.336405in}}%
\pgfpathlineto{\pgfqpoint{2.112403in}{3.171671in}}%
\pgfpathlineto{\pgfqpoint{2.112972in}{3.344648in}}%
\pgfpathlineto{\pgfqpoint{2.114488in}{3.530816in}}%
\pgfpathlineto{\pgfqpoint{2.114677in}{3.542760in}}%
\pgfpathlineto{\pgfqpoint{2.115056in}{3.477587in}}%
\pgfpathlineto{\pgfqpoint{2.115625in}{3.291932in}}%
\pgfpathlineto{\pgfqpoint{2.116099in}{3.453455in}}%
\pgfpathlineto{\pgfqpoint{2.116573in}{3.517720in}}%
\pgfpathlineto{\pgfqpoint{2.117141in}{3.459430in}}%
\pgfpathlineto{\pgfqpoint{2.117236in}{3.457683in}}%
\pgfpathlineto{\pgfqpoint{2.117331in}{3.464667in}}%
\pgfpathlineto{\pgfqpoint{2.117805in}{3.545597in}}%
\pgfpathlineto{\pgfqpoint{2.118184in}{3.468698in}}%
\pgfpathlineto{\pgfqpoint{2.118657in}{3.320809in}}%
\pgfpathlineto{\pgfqpoint{2.119226in}{3.475370in}}%
\pgfpathlineto{\pgfqpoint{2.119605in}{3.587499in}}%
\pgfpathlineto{\pgfqpoint{2.120268in}{3.453972in}}%
\pgfpathlineto{\pgfqpoint{2.120363in}{3.448830in}}%
\pgfpathlineto{\pgfqpoint{2.120648in}{3.473078in}}%
\pgfpathlineto{\pgfqpoint{2.120932in}{3.508790in}}%
\pgfpathlineto{\pgfqpoint{2.121406in}{3.416712in}}%
\pgfpathlineto{\pgfqpoint{2.121785in}{3.334416in}}%
\pgfpathlineto{\pgfqpoint{2.122259in}{3.445401in}}%
\pgfpathlineto{\pgfqpoint{2.122732in}{3.562210in}}%
\pgfpathlineto{\pgfqpoint{2.123301in}{3.436156in}}%
\pgfpathlineto{\pgfqpoint{2.124059in}{3.445518in}}%
\pgfpathlineto{\pgfqpoint{2.125007in}{3.303307in}}%
\pgfpathlineto{\pgfqpoint{2.125386in}{3.408399in}}%
\pgfpathlineto{\pgfqpoint{2.125765in}{3.526424in}}%
\pgfpathlineto{\pgfqpoint{2.126428in}{3.397888in}}%
\pgfpathlineto{\pgfqpoint{2.127281in}{3.413760in}}%
\pgfpathlineto{\pgfqpoint{2.127944in}{3.287074in}}%
\pgfpathlineto{\pgfqpoint{2.128039in}{3.281949in}}%
\pgfpathlineto{\pgfqpoint{2.128323in}{3.320511in}}%
\pgfpathlineto{\pgfqpoint{2.128987in}{3.507607in}}%
\pgfpathlineto{\pgfqpoint{2.129461in}{3.332462in}}%
\pgfpathlineto{\pgfqpoint{2.129840in}{3.256540in}}%
\pgfpathlineto{\pgfqpoint{2.130693in}{3.280746in}}%
\pgfpathlineto{\pgfqpoint{2.131166in}{3.209146in}}%
\pgfpathlineto{\pgfqpoint{2.131640in}{3.300767in}}%
\pgfpathlineto{\pgfqpoint{2.132019in}{3.370387in}}%
\pgfpathlineto{\pgfqpoint{2.132493in}{3.255767in}}%
\pgfpathlineto{\pgfqpoint{2.132967in}{3.109071in}}%
\pgfpathlineto{\pgfqpoint{2.133725in}{3.196647in}}%
\pgfpathlineto{\pgfqpoint{2.134294in}{3.121289in}}%
\pgfpathlineto{\pgfqpoint{2.134673in}{3.185545in}}%
\pgfpathlineto{\pgfqpoint{2.135241in}{3.288101in}}%
\pgfpathlineto{\pgfqpoint{2.135620in}{3.176739in}}%
\pgfpathlineto{\pgfqpoint{2.136094in}{3.041748in}}%
\pgfpathlineto{\pgfqpoint{2.136663in}{3.154529in}}%
\pgfpathlineto{\pgfqpoint{2.138274in}{3.297698in}}%
\pgfpathlineto{\pgfqpoint{2.138463in}{3.275916in}}%
\pgfpathlineto{\pgfqpoint{2.139316in}{3.050307in}}%
\pgfpathlineto{\pgfqpoint{2.139790in}{3.176213in}}%
\pgfpathlineto{\pgfqpoint{2.140074in}{3.217965in}}%
\pgfpathlineto{\pgfqpoint{2.140832in}{3.162932in}}%
\pgfpathlineto{\pgfqpoint{2.141022in}{3.172563in}}%
\pgfpathlineto{\pgfqpoint{2.141401in}{3.224712in}}%
\pgfpathlineto{\pgfqpoint{2.141780in}{3.131522in}}%
\pgfpathlineto{\pgfqpoint{2.142348in}{2.960153in}}%
\pgfpathlineto{\pgfqpoint{2.142822in}{3.098208in}}%
\pgfpathlineto{\pgfqpoint{2.143201in}{3.172638in}}%
\pgfpathlineto{\pgfqpoint{2.143959in}{3.128450in}}%
\pgfpathlineto{\pgfqpoint{2.144623in}{3.237788in}}%
\pgfpathlineto{\pgfqpoint{2.145002in}{3.148091in}}%
\pgfpathlineto{\pgfqpoint{2.145476in}{3.044385in}}%
\pgfpathlineto{\pgfqpoint{2.145950in}{3.199946in}}%
\pgfpathlineto{\pgfqpoint{2.146423in}{3.305440in}}%
\pgfpathlineto{\pgfqpoint{2.146992in}{3.195403in}}%
\pgfpathlineto{\pgfqpoint{2.147276in}{3.241183in}}%
\pgfpathlineto{\pgfqpoint{2.147655in}{3.289261in}}%
\pgfpathlineto{\pgfqpoint{2.148129in}{3.183149in}}%
\pgfpathlineto{\pgfqpoint{2.148603in}{3.058624in}}%
\pgfpathlineto{\pgfqpoint{2.148982in}{3.236290in}}%
\pgfpathlineto{\pgfqpoint{2.149835in}{3.682649in}}%
\pgfpathlineto{\pgfqpoint{2.150498in}{3.601454in}}%
\pgfpathlineto{\pgfqpoint{2.150783in}{3.614176in}}%
\pgfpathlineto{\pgfqpoint{2.150972in}{3.594076in}}%
\pgfpathlineto{\pgfqpoint{2.151825in}{2.993776in}}%
\pgfpathlineto{\pgfqpoint{2.152583in}{3.375895in}}%
\pgfpathlineto{\pgfqpoint{2.152773in}{3.356868in}}%
\pgfpathlineto{\pgfqpoint{2.154763in}{3.113746in}}%
\pgfpathlineto{\pgfqpoint{2.154952in}{3.125219in}}%
\pgfpathlineto{\pgfqpoint{2.155615in}{3.370978in}}%
\pgfpathlineto{\pgfqpoint{2.156279in}{3.192104in}}%
\pgfpathlineto{\pgfqpoint{2.156658in}{3.124914in}}%
\pgfpathlineto{\pgfqpoint{2.157226in}{3.221596in}}%
\pgfpathlineto{\pgfqpoint{2.157511in}{3.183254in}}%
\pgfpathlineto{\pgfqpoint{2.157890in}{3.127442in}}%
\pgfpathlineto{\pgfqpoint{2.158364in}{3.239305in}}%
\pgfpathlineto{\pgfqpoint{2.158837in}{3.378352in}}%
\pgfpathlineto{\pgfqpoint{2.159406in}{3.210777in}}%
\pgfpathlineto{\pgfqpoint{2.159690in}{3.170514in}}%
\pgfpathlineto{\pgfqpoint{2.160164in}{3.261775in}}%
\pgfpathlineto{\pgfqpoint{2.161965in}{3.531088in}}%
\pgfpathlineto{\pgfqpoint{2.162154in}{3.507084in}}%
\pgfpathlineto{\pgfqpoint{2.162723in}{3.364128in}}%
\pgfpathlineto{\pgfqpoint{2.163291in}{3.485798in}}%
\pgfpathlineto{\pgfqpoint{2.164997in}{3.717226in}}%
\pgfpathlineto{\pgfqpoint{2.165376in}{3.648431in}}%
\pgfpathlineto{\pgfqpoint{2.165945in}{3.488455in}}%
\pgfpathlineto{\pgfqpoint{2.166419in}{3.612726in}}%
\pgfpathlineto{\pgfqpoint{2.167935in}{3.770131in}}%
\pgfpathlineto{\pgfqpoint{2.168124in}{3.785471in}}%
\pgfpathlineto{\pgfqpoint{2.168503in}{3.698393in}}%
\pgfpathlineto{\pgfqpoint{2.169072in}{3.567381in}}%
\pgfpathlineto{\pgfqpoint{2.169546in}{3.706789in}}%
\pgfpathlineto{\pgfqpoint{2.169925in}{3.769109in}}%
\pgfpathlineto{\pgfqpoint{2.170778in}{3.746978in}}%
\pgfpathlineto{\pgfqpoint{2.171252in}{3.824449in}}%
\pgfpathlineto{\pgfqpoint{2.171725in}{3.716787in}}%
\pgfpathlineto{\pgfqpoint{2.172104in}{3.614022in}}%
\pgfpathlineto{\pgfqpoint{2.172673in}{3.749301in}}%
\pgfpathlineto{\pgfqpoint{2.173052in}{3.858387in}}%
\pgfpathlineto{\pgfqpoint{2.173715in}{3.749434in}}%
\pgfpathlineto{\pgfqpoint{2.173810in}{3.744754in}}%
\pgfpathlineto{\pgfqpoint{2.174189in}{3.777771in}}%
\pgfpathlineto{\pgfqpoint{2.174474in}{3.810503in}}%
\pgfpathlineto{\pgfqpoint{2.174853in}{3.735413in}}%
\pgfpathlineto{\pgfqpoint{2.175232in}{3.612998in}}%
\pgfpathlineto{\pgfqpoint{2.175800in}{3.772230in}}%
\pgfpathlineto{\pgfqpoint{2.176274in}{3.865040in}}%
\pgfpathlineto{\pgfqpoint{2.176843in}{3.777806in}}%
\pgfpathlineto{\pgfqpoint{2.178359in}{3.644332in}}%
\pgfpathlineto{\pgfqpoint{2.178643in}{3.701498in}}%
\pgfpathlineto{\pgfqpoint{2.179306in}{3.893990in}}%
\pgfpathlineto{\pgfqpoint{2.179875in}{3.789286in}}%
\pgfpathlineto{\pgfqpoint{2.180159in}{3.763341in}}%
\pgfpathlineto{\pgfqpoint{2.180728in}{3.832485in}}%
\pgfpathlineto{\pgfqpoint{2.180823in}{3.841755in}}%
\pgfpathlineto{\pgfqpoint{2.181202in}{3.772104in}}%
\pgfpathlineto{\pgfqpoint{2.181486in}{3.735522in}}%
\pgfpathlineto{\pgfqpoint{2.181960in}{3.849269in}}%
\pgfpathlineto{\pgfqpoint{2.182528in}{4.007154in}}%
\pgfpathlineto{\pgfqpoint{2.183002in}{3.863531in}}%
\pgfpathlineto{\pgfqpoint{2.183381in}{3.800302in}}%
\pgfpathlineto{\pgfqpoint{2.184139in}{3.847439in}}%
\pgfpathlineto{\pgfqpoint{2.184613in}{3.750573in}}%
\pgfpathlineto{\pgfqpoint{2.185182in}{3.856243in}}%
\pgfpathlineto{\pgfqpoint{2.185561in}{3.929473in}}%
\pgfpathlineto{\pgfqpoint{2.186035in}{3.797069in}}%
\pgfpathlineto{\pgfqpoint{2.186509in}{3.699322in}}%
\pgfpathlineto{\pgfqpoint{2.187172in}{3.769326in}}%
\pgfpathlineto{\pgfqpoint{2.187741in}{3.691332in}}%
\pgfpathlineto{\pgfqpoint{2.188120in}{3.765684in}}%
\pgfpathlineto{\pgfqpoint{2.188688in}{3.899361in}}%
\pgfpathlineto{\pgfqpoint{2.189162in}{3.767303in}}%
\pgfpathlineto{\pgfqpoint{2.189636in}{3.656823in}}%
\pgfpathlineto{\pgfqpoint{2.190204in}{3.781344in}}%
\pgfpathlineto{\pgfqpoint{2.190394in}{3.797938in}}%
\pgfpathlineto{\pgfqpoint{2.190773in}{3.737943in}}%
\pgfpathlineto{\pgfqpoint{2.191057in}{3.702469in}}%
\pgfpathlineto{\pgfqpoint{2.191531in}{3.796975in}}%
\pgfpathlineto{\pgfqpoint{2.191721in}{3.817695in}}%
\pgfpathlineto{\pgfqpoint{2.192005in}{3.739743in}}%
\pgfpathlineto{\pgfqpoint{2.192479in}{3.557260in}}%
\pgfpathlineto{\pgfqpoint{2.192953in}{3.767511in}}%
\pgfpathlineto{\pgfqpoint{2.194469in}{4.117633in}}%
\pgfpathlineto{\pgfqpoint{2.194658in}{4.135628in}}%
\pgfpathlineto{\pgfqpoint{2.194943in}{4.008110in}}%
\pgfpathlineto{\pgfqpoint{2.195701in}{3.448290in}}%
\pgfpathlineto{\pgfqpoint{2.196364in}{3.709159in}}%
\pgfpathlineto{\pgfqpoint{2.196648in}{3.755462in}}%
\pgfpathlineto{\pgfqpoint{2.197312in}{3.685671in}}%
\pgfpathlineto{\pgfqpoint{2.197406in}{3.685650in}}%
\pgfpathlineto{\pgfqpoint{2.197975in}{3.753503in}}%
\pgfpathlineto{\pgfqpoint{2.198354in}{3.675255in}}%
\pgfpathlineto{\pgfqpoint{2.198923in}{3.495987in}}%
\pgfpathlineto{\pgfqpoint{2.199491in}{3.659357in}}%
\pgfpathlineto{\pgfqpoint{2.199870in}{3.708044in}}%
\pgfpathlineto{\pgfqpoint{2.200344in}{3.631157in}}%
\pgfpathlineto{\pgfqpoint{2.200439in}{3.623215in}}%
\pgfpathlineto{\pgfqpoint{2.200818in}{3.672376in}}%
\pgfpathlineto{\pgfqpoint{2.201102in}{3.696377in}}%
\pgfpathlineto{\pgfqpoint{2.201481in}{3.624463in}}%
\pgfpathlineto{\pgfqpoint{2.202050in}{3.464729in}}%
\pgfpathlineto{\pgfqpoint{2.202524in}{3.586611in}}%
\pgfpathlineto{\pgfqpoint{2.202903in}{3.688537in}}%
\pgfpathlineto{\pgfqpoint{2.203566in}{3.589835in}}%
\pgfpathlineto{\pgfqpoint{2.205082in}{3.489648in}}%
\pgfpathlineto{\pgfqpoint{2.204229in}{3.608741in}}%
\pgfpathlineto{\pgfqpoint{2.205177in}{3.490204in}}%
\pgfpathlineto{\pgfqpoint{2.206125in}{3.724164in}}%
\pgfpathlineto{\pgfqpoint{2.206788in}{3.574426in}}%
\pgfpathlineto{\pgfqpoint{2.206883in}{3.571163in}}%
\pgfpathlineto{\pgfqpoint{2.207072in}{3.592498in}}%
\pgfpathlineto{\pgfqpoint{2.207451in}{3.653172in}}%
\pgfpathlineto{\pgfqpoint{2.207925in}{3.555577in}}%
\pgfpathlineto{\pgfqpoint{2.208210in}{3.518360in}}%
\pgfpathlineto{\pgfqpoint{2.208683in}{3.624896in}}%
\pgfpathlineto{\pgfqpoint{2.209157in}{3.793764in}}%
\pgfpathlineto{\pgfqpoint{2.209726in}{3.649061in}}%
\pgfpathlineto{\pgfqpoint{2.210105in}{3.591995in}}%
\pgfpathlineto{\pgfqpoint{2.210958in}{3.608660in}}%
\pgfpathlineto{\pgfqpoint{2.211337in}{3.553106in}}%
\pgfpathlineto{\pgfqpoint{2.211716in}{3.623630in}}%
\pgfpathlineto{\pgfqpoint{2.212190in}{3.774039in}}%
\pgfpathlineto{\pgfqpoint{2.212758in}{3.667749in}}%
\pgfpathlineto{\pgfqpoint{2.213232in}{3.535749in}}%
\pgfpathlineto{\pgfqpoint{2.213990in}{3.612926in}}%
\pgfpathlineto{\pgfqpoint{2.214464in}{3.545731in}}%
\pgfpathlineto{\pgfqpoint{2.214843in}{3.617298in}}%
\pgfpathlineto{\pgfqpoint{2.215317in}{3.732760in}}%
\pgfpathlineto{\pgfqpoint{2.215791in}{3.610449in}}%
\pgfpathlineto{\pgfqpoint{2.216454in}{3.484274in}}%
\pgfpathlineto{\pgfqpoint{2.217023in}{3.548000in}}%
\pgfpathlineto{\pgfqpoint{2.217496in}{3.516487in}}%
\pgfpathlineto{\pgfqpoint{2.217781in}{3.500402in}}%
\pgfpathlineto{\pgfqpoint{2.218065in}{3.556368in}}%
\pgfpathlineto{\pgfqpoint{2.218444in}{3.636076in}}%
\pgfpathlineto{\pgfqpoint{2.218918in}{3.531253in}}%
\pgfpathlineto{\pgfqpoint{2.219486in}{3.372435in}}%
\pgfpathlineto{\pgfqpoint{2.220055in}{3.517208in}}%
\pgfpathlineto{\pgfqpoint{2.220245in}{3.538428in}}%
\pgfpathlineto{\pgfqpoint{2.220908in}{3.479496in}}%
\pgfpathlineto{\pgfqpoint{2.221003in}{3.477201in}}%
\pgfpathlineto{\pgfqpoint{2.221097in}{3.485241in}}%
\pgfpathlineto{\pgfqpoint{2.221666in}{3.605259in}}%
\pgfpathlineto{\pgfqpoint{2.222045in}{3.506232in}}%
\pgfpathlineto{\pgfqpoint{2.222519in}{3.363913in}}%
\pgfpathlineto{\pgfqpoint{2.223088in}{3.515338in}}%
\pgfpathlineto{\pgfqpoint{2.224699in}{3.647501in}}%
\pgfpathlineto{\pgfqpoint{2.224035in}{3.505070in}}%
\pgfpathlineto{\pgfqpoint{2.224793in}{3.643647in}}%
\pgfpathlineto{\pgfqpoint{2.225646in}{3.439288in}}%
\pgfpathlineto{\pgfqpoint{2.226120in}{3.605474in}}%
\pgfpathlineto{\pgfqpoint{2.226499in}{3.701039in}}%
\pgfpathlineto{\pgfqpoint{2.227257in}{3.622658in}}%
\pgfpathlineto{\pgfqpoint{2.227826in}{3.685658in}}%
\pgfpathlineto{\pgfqpoint{2.228205in}{3.623243in}}%
\pgfpathlineto{\pgfqpoint{2.228773in}{3.437751in}}%
\pgfpathlineto{\pgfqpoint{2.229342in}{3.583491in}}%
\pgfpathlineto{\pgfqpoint{2.229721in}{3.643650in}}%
\pgfpathlineto{\pgfqpoint{2.230290in}{3.552764in}}%
\pgfpathlineto{\pgfqpoint{2.231806in}{3.404135in}}%
\pgfpathlineto{\pgfqpoint{2.231048in}{3.574766in}}%
\pgfpathlineto{\pgfqpoint{2.232090in}{3.435759in}}%
\pgfpathlineto{\pgfqpoint{2.232753in}{3.654534in}}%
\pgfpathlineto{\pgfqpoint{2.233512in}{3.523737in}}%
\pgfpathlineto{\pgfqpoint{2.234933in}{3.402915in}}%
\pgfpathlineto{\pgfqpoint{2.234175in}{3.553810in}}%
\pgfpathlineto{\pgfqpoint{2.235217in}{3.445346in}}%
\pgfpathlineto{\pgfqpoint{2.237397in}{3.939558in}}%
\pgfpathlineto{\pgfqpoint{2.237586in}{3.919963in}}%
\pgfpathlineto{\pgfqpoint{2.239482in}{3.447117in}}%
\pgfpathlineto{\pgfqpoint{2.240335in}{3.548633in}}%
\pgfpathlineto{\pgfqpoint{2.242135in}{3.715844in}}%
\pgfpathlineto{\pgfqpoint{2.241187in}{3.528161in}}%
\pgfpathlineto{\pgfqpoint{2.242325in}{3.688251in}}%
\pgfpathlineto{\pgfqpoint{2.243083in}{3.507869in}}%
\pgfpathlineto{\pgfqpoint{2.243651in}{3.616945in}}%
\pgfpathlineto{\pgfqpoint{2.243841in}{3.624241in}}%
\pgfpathlineto{\pgfqpoint{2.244220in}{3.584154in}}%
\pgfpathlineto{\pgfqpoint{2.244315in}{3.576671in}}%
\pgfpathlineto{\pgfqpoint{2.244599in}{3.616192in}}%
\pgfpathlineto{\pgfqpoint{2.245357in}{3.756508in}}%
\pgfpathlineto{\pgfqpoint{2.245736in}{3.643147in}}%
\pgfpathlineto{\pgfqpoint{2.246115in}{3.542914in}}%
\pgfpathlineto{\pgfqpoint{2.246684in}{3.659248in}}%
\pgfpathlineto{\pgfqpoint{2.248390in}{3.812549in}}%
\pgfpathlineto{\pgfqpoint{2.247631in}{3.652708in}}%
\pgfpathlineto{\pgfqpoint{2.248484in}{3.807055in}}%
\pgfpathlineto{\pgfqpoint{2.249242in}{3.593499in}}%
\pgfpathlineto{\pgfqpoint{2.249906in}{3.747589in}}%
\pgfpathlineto{\pgfqpoint{2.251422in}{3.840218in}}%
\pgfpathlineto{\pgfqpoint{2.250569in}{3.740194in}}%
\pgfpathlineto{\pgfqpoint{2.251517in}{3.838191in}}%
\pgfpathlineto{\pgfqpoint{2.251896in}{3.773951in}}%
\pgfpathlineto{\pgfqpoint{2.252370in}{3.645261in}}%
\pgfpathlineto{\pgfqpoint{2.252843in}{3.777293in}}%
\pgfpathlineto{\pgfqpoint{2.253317in}{3.875401in}}%
\pgfpathlineto{\pgfqpoint{2.254075in}{3.821455in}}%
\pgfpathlineto{\pgfqpoint{2.254549in}{3.879092in}}%
\pgfpathlineto{\pgfqpoint{2.255023in}{3.796314in}}%
\pgfpathlineto{\pgfqpoint{2.255497in}{3.686485in}}%
\pgfpathlineto{\pgfqpoint{2.255971in}{3.802442in}}%
\pgfpathlineto{\pgfqpoint{2.256444in}{3.925639in}}%
\pgfpathlineto{\pgfqpoint{2.257013in}{3.823418in}}%
\pgfpathlineto{\pgfqpoint{2.258624in}{3.703619in}}%
\pgfpathlineto{\pgfqpoint{2.257676in}{3.856800in}}%
\pgfpathlineto{\pgfqpoint{2.258719in}{3.712756in}}%
\pgfpathlineto{\pgfqpoint{2.259477in}{3.921447in}}%
\pgfpathlineto{\pgfqpoint{2.260140in}{3.800910in}}%
\pgfpathlineto{\pgfqpoint{2.261751in}{3.672019in}}%
\pgfpathlineto{\pgfqpoint{2.261941in}{3.697666in}}%
\pgfpathlineto{\pgfqpoint{2.262604in}{3.908088in}}%
\pgfpathlineto{\pgfqpoint{2.263173in}{3.764105in}}%
\pgfpathlineto{\pgfqpoint{2.263457in}{3.709820in}}%
\pgfpathlineto{\pgfqpoint{2.264405in}{3.720088in}}%
\pgfpathlineto{\pgfqpoint{2.264784in}{3.648600in}}%
\pgfpathlineto{\pgfqpoint{2.265258in}{3.738643in}}%
\pgfpathlineto{\pgfqpoint{2.265731in}{3.876826in}}%
\pgfpathlineto{\pgfqpoint{2.266300in}{3.731751in}}%
\pgfpathlineto{\pgfqpoint{2.266584in}{3.678996in}}%
\pgfpathlineto{\pgfqpoint{2.267342in}{3.724191in}}%
\pgfpathlineto{\pgfqpoint{2.267437in}{3.724733in}}%
\pgfpathlineto{\pgfqpoint{2.267911in}{3.657767in}}%
\pgfpathlineto{\pgfqpoint{2.268290in}{3.729188in}}%
\pgfpathlineto{\pgfqpoint{2.268859in}{3.901368in}}%
\pgfpathlineto{\pgfqpoint{2.269427in}{3.752619in}}%
\pgfpathlineto{\pgfqpoint{2.269806in}{3.678180in}}%
\pgfpathlineto{\pgfqpoint{2.270659in}{3.711842in}}%
\pgfpathlineto{\pgfqpoint{2.271133in}{3.642844in}}%
\pgfpathlineto{\pgfqpoint{2.271607in}{3.722720in}}%
\pgfpathlineto{\pgfqpoint{2.271891in}{3.775920in}}%
\pgfpathlineto{\pgfqpoint{2.272365in}{3.652919in}}%
\pgfpathlineto{\pgfqpoint{2.272933in}{3.528023in}}%
\pgfpathlineto{\pgfqpoint{2.273407in}{3.665733in}}%
\pgfpathlineto{\pgfqpoint{2.274923in}{3.788697in}}%
\pgfpathlineto{\pgfqpoint{2.275113in}{3.811316in}}%
\pgfpathlineto{\pgfqpoint{2.275492in}{3.705208in}}%
\pgfpathlineto{\pgfqpoint{2.275966in}{3.559617in}}%
\pgfpathlineto{\pgfqpoint{2.276629in}{3.686137in}}%
\pgfpathlineto{\pgfqpoint{2.276914in}{3.707036in}}%
\pgfpathlineto{\pgfqpoint{2.277387in}{3.661362in}}%
\pgfpathlineto{\pgfqpoint{2.277482in}{3.656389in}}%
\pgfpathlineto{\pgfqpoint{2.277766in}{3.696400in}}%
\pgfpathlineto{\pgfqpoint{2.278145in}{3.757532in}}%
\pgfpathlineto{\pgfqpoint{2.278619in}{3.671627in}}%
\pgfpathlineto{\pgfqpoint{2.279188in}{3.552752in}}%
\pgfpathlineto{\pgfqpoint{2.279662in}{3.638832in}}%
\pgfpathlineto{\pgfqpoint{2.281367in}{4.180023in}}%
\pgfpathlineto{\pgfqpoint{2.281747in}{4.064803in}}%
\pgfpathlineto{\pgfqpoint{2.282884in}{3.622914in}}%
\pgfpathlineto{\pgfqpoint{2.283357in}{3.704006in}}%
\pgfpathlineto{\pgfqpoint{2.284400in}{3.776973in}}%
\pgfpathlineto{\pgfqpoint{2.284589in}{3.759118in}}%
\pgfpathlineto{\pgfqpoint{2.285348in}{3.593415in}}%
\pgfpathlineto{\pgfqpoint{2.285821in}{3.683299in}}%
\pgfpathlineto{\pgfqpoint{2.286390in}{3.807779in}}%
\pgfpathlineto{\pgfqpoint{2.286959in}{3.699672in}}%
\pgfpathlineto{\pgfqpoint{2.288380in}{3.579317in}}%
\pgfpathlineto{\pgfqpoint{2.287622in}{3.722713in}}%
\pgfpathlineto{\pgfqpoint{2.288570in}{3.594286in}}%
\pgfpathlineto{\pgfqpoint{2.289422in}{3.810282in}}%
\pgfpathlineto{\pgfqpoint{2.289991in}{3.691492in}}%
\pgfpathlineto{\pgfqpoint{2.290939in}{3.696710in}}%
\pgfpathlineto{\pgfqpoint{2.291602in}{3.591226in}}%
\pgfpathlineto{\pgfqpoint{2.291981in}{3.694401in}}%
\pgfpathlineto{\pgfqpoint{2.292360in}{3.808627in}}%
\pgfpathlineto{\pgfqpoint{2.293023in}{3.698191in}}%
\pgfpathlineto{\pgfqpoint{2.293497in}{3.628104in}}%
\pgfpathlineto{\pgfqpoint{2.294350in}{3.632173in}}%
\pgfpathlineto{\pgfqpoint{2.294634in}{3.610034in}}%
\pgfpathlineto{\pgfqpoint{2.295013in}{3.658457in}}%
\pgfpathlineto{\pgfqpoint{2.295487in}{3.800332in}}%
\pgfpathlineto{\pgfqpoint{2.296056in}{3.693393in}}%
\pgfpathlineto{\pgfqpoint{2.296530in}{3.583200in}}%
\pgfpathlineto{\pgfqpoint{2.297288in}{3.655174in}}%
\pgfpathlineto{\pgfqpoint{2.297762in}{3.615982in}}%
\pgfpathlineto{\pgfqpoint{2.298141in}{3.659526in}}%
\pgfpathlineto{\pgfqpoint{2.298709in}{3.760264in}}%
\pgfpathlineto{\pgfqpoint{2.299183in}{3.658371in}}%
\pgfpathlineto{\pgfqpoint{2.299657in}{3.531635in}}%
\pgfpathlineto{\pgfqpoint{2.300320in}{3.635354in}}%
\pgfpathlineto{\pgfqpoint{2.300415in}{3.641953in}}%
\pgfpathlineto{\pgfqpoint{2.300794in}{3.604424in}}%
\pgfpathlineto{\pgfqpoint{2.300984in}{3.595947in}}%
\pgfpathlineto{\pgfqpoint{2.301268in}{3.628415in}}%
\pgfpathlineto{\pgfqpoint{2.301742in}{3.750619in}}%
\pgfpathlineto{\pgfqpoint{2.302216in}{3.625675in}}%
\pgfpathlineto{\pgfqpoint{2.302784in}{3.477719in}}%
\pgfpathlineto{\pgfqpoint{2.303353in}{3.595108in}}%
\pgfpathlineto{\pgfqpoint{2.304206in}{3.584248in}}%
\pgfpathlineto{\pgfqpoint{2.304869in}{3.660237in}}%
\pgfpathlineto{\pgfqpoint{2.304964in}{3.661532in}}%
\pgfpathlineto{\pgfqpoint{2.305058in}{3.653110in}}%
\pgfpathlineto{\pgfqpoint{2.305911in}{3.414171in}}%
\pgfpathlineto{\pgfqpoint{2.306480in}{3.551279in}}%
\pgfpathlineto{\pgfqpoint{2.306764in}{3.605242in}}%
\pgfpathlineto{\pgfqpoint{2.307333in}{3.542360in}}%
\pgfpathlineto{\pgfqpoint{2.307617in}{3.565198in}}%
\pgfpathlineto{\pgfqpoint{2.307996in}{3.616168in}}%
\pgfpathlineto{\pgfqpoint{2.308470in}{3.524798in}}%
\pgfpathlineto{\pgfqpoint{2.309039in}{3.396765in}}%
\pgfpathlineto{\pgfqpoint{2.309512in}{3.518562in}}%
\pgfpathlineto{\pgfqpoint{2.309986in}{3.589466in}}%
\pgfpathlineto{\pgfqpoint{2.310555in}{3.523626in}}%
\pgfpathlineto{\pgfqpoint{2.310650in}{3.521700in}}%
\pgfpathlineto{\pgfqpoint{2.310839in}{3.532376in}}%
\pgfpathlineto{\pgfqpoint{2.311218in}{3.561509in}}%
\pgfpathlineto{\pgfqpoint{2.311597in}{3.514733in}}%
\pgfpathlineto{\pgfqpoint{2.312071in}{3.404411in}}%
\pgfpathlineto{\pgfqpoint{2.312545in}{3.534520in}}%
\pgfpathlineto{\pgfqpoint{2.313113in}{3.664032in}}%
\pgfpathlineto{\pgfqpoint{2.313682in}{3.561840in}}%
\pgfpathlineto{\pgfqpoint{2.313777in}{3.556517in}}%
\pgfpathlineto{\pgfqpoint{2.314156in}{3.587993in}}%
\pgfpathlineto{\pgfqpoint{2.314251in}{3.593780in}}%
\pgfpathlineto{\pgfqpoint{2.314535in}{3.562997in}}%
\pgfpathlineto{\pgfqpoint{2.315293in}{3.444145in}}%
\pgfpathlineto{\pgfqpoint{2.315672in}{3.536957in}}%
\pgfpathlineto{\pgfqpoint{2.316051in}{3.617227in}}%
\pgfpathlineto{\pgfqpoint{2.316620in}{3.505755in}}%
\pgfpathlineto{\pgfqpoint{2.318231in}{3.360667in}}%
\pgfpathlineto{\pgfqpoint{2.318325in}{3.357280in}}%
\pgfpathlineto{\pgfqpoint{2.318515in}{3.383867in}}%
\pgfpathlineto{\pgfqpoint{2.319273in}{3.593380in}}%
\pgfpathlineto{\pgfqpoint{2.319842in}{3.452654in}}%
\pgfpathlineto{\pgfqpoint{2.321168in}{3.408055in}}%
\pgfpathlineto{\pgfqpoint{2.321453in}{3.377430in}}%
\pgfpathlineto{\pgfqpoint{2.321832in}{3.462024in}}%
\pgfpathlineto{\pgfqpoint{2.322400in}{3.601374in}}%
\pgfpathlineto{\pgfqpoint{2.322874in}{3.463143in}}%
\pgfpathlineto{\pgfqpoint{2.323348in}{3.331683in}}%
\pgfpathlineto{\pgfqpoint{2.323727in}{3.486312in}}%
\pgfpathlineto{\pgfqpoint{2.325433in}{3.945620in}}%
\pgfpathlineto{\pgfqpoint{2.325528in}{3.941144in}}%
\pgfpathlineto{\pgfqpoint{2.326001in}{3.666897in}}%
\pgfpathlineto{\pgfqpoint{2.326570in}{3.233516in}}%
\pgfpathlineto{\pgfqpoint{2.327233in}{3.491313in}}%
\pgfpathlineto{\pgfqpoint{2.327802in}{3.467551in}}%
\pgfpathlineto{\pgfqpoint{2.328276in}{3.592017in}}%
\pgfpathlineto{\pgfqpoint{2.328465in}{3.631486in}}%
\pgfpathlineto{\pgfqpoint{2.329034in}{3.511127in}}%
\pgfpathlineto{\pgfqpoint{2.329508in}{3.392881in}}%
\pgfpathlineto{\pgfqpoint{2.330076in}{3.497434in}}%
\pgfpathlineto{\pgfqpoint{2.331687in}{3.606030in}}%
\pgfpathlineto{\pgfqpoint{2.332066in}{3.528195in}}%
\pgfpathlineto{\pgfqpoint{2.332635in}{3.395144in}}%
\pgfpathlineto{\pgfqpoint{2.333109in}{3.493837in}}%
\pgfpathlineto{\pgfqpoint{2.333582in}{3.572035in}}%
\pgfpathlineto{\pgfqpoint{2.334435in}{3.569523in}}%
\pgfpathlineto{\pgfqpoint{2.334814in}{3.607046in}}%
\pgfpathlineto{\pgfqpoint{2.335193in}{3.524734in}}%
\pgfpathlineto{\pgfqpoint{2.335667in}{3.394786in}}%
\pgfpathlineto{\pgfqpoint{2.336236in}{3.527131in}}%
\pgfpathlineto{\pgfqpoint{2.336710in}{3.605660in}}%
\pgfpathlineto{\pgfqpoint{2.337373in}{3.546459in}}%
\pgfpathlineto{\pgfqpoint{2.337942in}{3.600867in}}%
\pgfpathlineto{\pgfqpoint{2.338226in}{3.565625in}}%
\pgfpathlineto{\pgfqpoint{2.338794in}{3.414972in}}%
\pgfpathlineto{\pgfqpoint{2.339268in}{3.535235in}}%
\pgfpathlineto{\pgfqpoint{2.339837in}{3.627422in}}%
\pgfpathlineto{\pgfqpoint{2.340405in}{3.559711in}}%
\pgfpathlineto{\pgfqpoint{2.341827in}{3.427097in}}%
\pgfpathlineto{\pgfqpoint{2.340974in}{3.566746in}}%
\pgfpathlineto{\pgfqpoint{2.342206in}{3.482722in}}%
\pgfpathlineto{\pgfqpoint{2.342869in}{3.643378in}}%
\pgfpathlineto{\pgfqpoint{2.343438in}{3.550791in}}%
\pgfpathlineto{\pgfqpoint{2.345049in}{3.426123in}}%
\pgfpathlineto{\pgfqpoint{2.345712in}{3.601590in}}%
\pgfpathlineto{\pgfqpoint{2.346091in}{3.661675in}}%
\pgfpathlineto{\pgfqpoint{2.346470in}{3.539366in}}%
\pgfpathlineto{\pgfqpoint{2.347892in}{3.428784in}}%
\pgfpathlineto{\pgfqpoint{2.348271in}{3.390063in}}%
\pgfpathlineto{\pgfqpoint{2.348745in}{3.468658in}}%
\pgfpathlineto{\pgfqpoint{2.349029in}{3.527971in}}%
\pgfpathlineto{\pgfqpoint{2.349503in}{3.396690in}}%
\pgfpathlineto{\pgfqpoint{2.350071in}{3.258008in}}%
\pgfpathlineto{\pgfqpoint{2.350735in}{3.345874in}}%
\pgfpathlineto{\pgfqpoint{2.351209in}{3.345730in}}%
\pgfpathlineto{\pgfqpoint{2.351777in}{3.486883in}}%
\pgfpathlineto{\pgfqpoint{2.352156in}{3.571383in}}%
\pgfpathlineto{\pgfqpoint{2.352630in}{3.459218in}}%
\pgfpathlineto{\pgfqpoint{2.353104in}{3.357159in}}%
\pgfpathlineto{\pgfqpoint{2.353767in}{3.439741in}}%
\pgfpathlineto{\pgfqpoint{2.354336in}{3.406841in}}%
\pgfpathlineto{\pgfqpoint{2.354715in}{3.434106in}}%
\pgfpathlineto{\pgfqpoint{2.355283in}{3.575750in}}%
\pgfpathlineto{\pgfqpoint{2.355852in}{3.440671in}}%
\pgfpathlineto{\pgfqpoint{2.356326in}{3.360984in}}%
\pgfpathlineto{\pgfqpoint{2.356800in}{3.469415in}}%
\pgfpathlineto{\pgfqpoint{2.358505in}{3.598170in}}%
\pgfpathlineto{\pgfqpoint{2.359074in}{3.402245in}}%
\pgfpathlineto{\pgfqpoint{2.359453in}{3.352961in}}%
\pgfpathlineto{\pgfqpoint{2.359927in}{3.437372in}}%
\pgfpathlineto{\pgfqpoint{2.360116in}{3.464466in}}%
\pgfpathlineto{\pgfqpoint{2.360780in}{3.412280in}}%
\pgfpathlineto{\pgfqpoint{2.360875in}{3.414354in}}%
\pgfpathlineto{\pgfqpoint{2.361443in}{3.468421in}}%
\pgfpathlineto{\pgfqpoint{2.361917in}{3.409881in}}%
\pgfpathlineto{\pgfqpoint{2.362391in}{3.267041in}}%
\pgfpathlineto{\pgfqpoint{2.362959in}{3.396083in}}%
\pgfpathlineto{\pgfqpoint{2.363338in}{3.468444in}}%
\pgfpathlineto{\pgfqpoint{2.364097in}{3.423674in}}%
\pgfpathlineto{\pgfqpoint{2.364665in}{3.490853in}}%
\pgfpathlineto{\pgfqpoint{2.364949in}{3.434132in}}%
\pgfpathlineto{\pgfqpoint{2.365613in}{3.287806in}}%
\pgfpathlineto{\pgfqpoint{2.365992in}{3.401782in}}%
\pgfpathlineto{\pgfqpoint{2.366466in}{3.522420in}}%
\pgfpathlineto{\pgfqpoint{2.367034in}{3.411510in}}%
\pgfpathlineto{\pgfqpoint{2.367129in}{3.410039in}}%
\pgfpathlineto{\pgfqpoint{2.367224in}{3.422601in}}%
\pgfpathlineto{\pgfqpoint{2.369403in}{3.867763in}}%
\pgfpathlineto{\pgfqpoint{2.369593in}{3.842568in}}%
\pgfpathlineto{\pgfqpoint{2.370351in}{3.300208in}}%
\pgfpathlineto{\pgfqpoint{2.371583in}{3.380289in}}%
\pgfpathlineto{\pgfqpoint{2.371678in}{3.368002in}}%
\pgfpathlineto{\pgfqpoint{2.372057in}{3.436735in}}%
\pgfpathlineto{\pgfqpoint{2.372815in}{3.593191in}}%
\pgfpathlineto{\pgfqpoint{2.373289in}{3.491583in}}%
\pgfpathlineto{\pgfqpoint{2.373573in}{3.428258in}}%
\pgfpathlineto{\pgfqpoint{2.374426in}{3.468421in}}%
\pgfpathlineto{\pgfqpoint{2.374900in}{3.401361in}}%
\pgfpathlineto{\pgfqpoint{2.375279in}{3.492418in}}%
\pgfpathlineto{\pgfqpoint{2.375847in}{3.628348in}}%
\pgfpathlineto{\pgfqpoint{2.376321in}{3.514991in}}%
\pgfpathlineto{\pgfqpoint{2.376795in}{3.451114in}}%
\pgfpathlineto{\pgfqpoint{2.377458in}{3.495601in}}%
\pgfpathlineto{\pgfqpoint{2.377932in}{3.457672in}}%
\pgfpathlineto{\pgfqpoint{2.378216in}{3.494618in}}%
\pgfpathlineto{\pgfqpoint{2.378974in}{3.656473in}}%
\pgfpathlineto{\pgfqpoint{2.379448in}{3.532849in}}%
\pgfpathlineto{\pgfqpoint{2.379922in}{3.459080in}}%
\pgfpathlineto{\pgfqpoint{2.380491in}{3.542835in}}%
\pgfpathlineto{\pgfqpoint{2.380585in}{3.546850in}}%
\pgfpathlineto{\pgfqpoint{2.380965in}{3.526893in}}%
\pgfpathlineto{\pgfqpoint{2.381154in}{3.514753in}}%
\pgfpathlineto{\pgfqpoint{2.381438in}{3.553213in}}%
\pgfpathlineto{\pgfqpoint{2.382007in}{3.695458in}}%
\pgfpathlineto{\pgfqpoint{2.382576in}{3.563280in}}%
\pgfpathlineto{\pgfqpoint{2.382955in}{3.467233in}}%
\pgfpathlineto{\pgfqpoint{2.383523in}{3.582995in}}%
\pgfpathlineto{\pgfqpoint{2.385134in}{3.708378in}}%
\pgfpathlineto{\pgfqpoint{2.384376in}{3.578878in}}%
\pgfpathlineto{\pgfqpoint{2.385418in}{3.670550in}}%
\pgfpathlineto{\pgfqpoint{2.386177in}{3.501952in}}%
\pgfpathlineto{\pgfqpoint{2.386650in}{3.578164in}}%
\pgfpathlineto{\pgfqpoint{2.388072in}{3.668009in}}%
\pgfpathlineto{\pgfqpoint{2.388167in}{3.671036in}}%
\pgfpathlineto{\pgfqpoint{2.388451in}{3.652472in}}%
\pgfpathlineto{\pgfqpoint{2.389114in}{3.456018in}}%
\pgfpathlineto{\pgfqpoint{2.389778in}{3.599035in}}%
\pgfpathlineto{\pgfqpoint{2.390157in}{3.663882in}}%
\pgfpathlineto{\pgfqpoint{2.390820in}{3.589030in}}%
\pgfpathlineto{\pgfqpoint{2.391294in}{3.619232in}}%
\pgfpathlineto{\pgfqpoint{2.391578in}{3.589881in}}%
\pgfpathlineto{\pgfqpoint{2.392336in}{3.415492in}}%
\pgfpathlineto{\pgfqpoint{2.392810in}{3.540144in}}%
\pgfpathlineto{\pgfqpoint{2.393189in}{3.606661in}}%
\pgfpathlineto{\pgfqpoint{2.393852in}{3.531611in}}%
\pgfpathlineto{\pgfqpoint{2.395463in}{3.392668in}}%
\pgfpathlineto{\pgfqpoint{2.394516in}{3.546571in}}%
\pgfpathlineto{\pgfqpoint{2.395653in}{3.416788in}}%
\pgfpathlineto{\pgfqpoint{2.396316in}{3.575940in}}%
\pgfpathlineto{\pgfqpoint{2.396980in}{3.498478in}}%
\pgfpathlineto{\pgfqpoint{2.398496in}{3.395791in}}%
\pgfpathlineto{\pgfqpoint{2.398685in}{3.419594in}}%
\pgfpathlineto{\pgfqpoint{2.399538in}{3.646354in}}%
\pgfpathlineto{\pgfqpoint{2.400107in}{3.524997in}}%
\pgfpathlineto{\pgfqpoint{2.400296in}{3.510246in}}%
\pgfpathlineto{\pgfqpoint{2.400770in}{3.559557in}}%
\pgfpathlineto{\pgfqpoint{2.400865in}{3.563859in}}%
\pgfpathlineto{\pgfqpoint{2.401055in}{3.543127in}}%
\pgfpathlineto{\pgfqpoint{2.401434in}{3.484513in}}%
\pgfpathlineto{\pgfqpoint{2.402002in}{3.537545in}}%
\pgfpathlineto{\pgfqpoint{2.402476in}{3.673409in}}%
\pgfpathlineto{\pgfqpoint{2.403045in}{3.582177in}}%
\pgfpathlineto{\pgfqpoint{2.404466in}{3.426660in}}%
\pgfpathlineto{\pgfqpoint{2.404750in}{3.395094in}}%
\pgfpathlineto{\pgfqpoint{2.405129in}{3.480763in}}%
\pgfpathlineto{\pgfqpoint{2.405698in}{3.574078in}}%
\pgfpathlineto{\pgfqpoint{2.406077in}{3.484469in}}%
\pgfpathlineto{\pgfqpoint{2.406646in}{3.361680in}}%
\pgfpathlineto{\pgfqpoint{2.407309in}{3.424876in}}%
\pgfpathlineto{\pgfqpoint{2.407498in}{3.416592in}}%
\pgfpathlineto{\pgfqpoint{2.407783in}{3.390790in}}%
\pgfpathlineto{\pgfqpoint{2.408257in}{3.457486in}}%
\pgfpathlineto{\pgfqpoint{2.408730in}{3.559443in}}%
\pgfpathlineto{\pgfqpoint{2.409204in}{3.453777in}}%
\pgfpathlineto{\pgfqpoint{2.409773in}{3.333030in}}%
\pgfpathlineto{\pgfqpoint{2.410341in}{3.411754in}}%
\pgfpathlineto{\pgfqpoint{2.411005in}{3.376814in}}%
\pgfpathlineto{\pgfqpoint{2.411573in}{3.637978in}}%
\pgfpathlineto{\pgfqpoint{2.412047in}{3.863181in}}%
\pgfpathlineto{\pgfqpoint{2.412711in}{3.674136in}}%
\pgfpathlineto{\pgfqpoint{2.412805in}{3.669336in}}%
\pgfpathlineto{\pgfqpoint{2.412995in}{3.698105in}}%
\pgfpathlineto{\pgfqpoint{2.413374in}{3.765733in}}%
\pgfpathlineto{\pgfqpoint{2.413753in}{3.654653in}}%
\pgfpathlineto{\pgfqpoint{2.414321in}{3.312728in}}%
\pgfpathlineto{\pgfqpoint{2.414985in}{3.554996in}}%
\pgfpathlineto{\pgfqpoint{2.415174in}{3.544923in}}%
\pgfpathlineto{\pgfqpoint{2.415932in}{3.371875in}}%
\pgfpathlineto{\pgfqpoint{2.416501in}{3.497684in}}%
\pgfpathlineto{\pgfqpoint{2.416880in}{3.551907in}}%
\pgfpathlineto{\pgfqpoint{2.417828in}{3.549152in}}%
\pgfpathlineto{\pgfqpoint{2.418207in}{3.561601in}}%
\pgfpathlineto{\pgfqpoint{2.418491in}{3.532332in}}%
\pgfpathlineto{\pgfqpoint{2.419060in}{3.408491in}}%
\pgfpathlineto{\pgfqpoint{2.419534in}{3.497737in}}%
\pgfpathlineto{\pgfqpoint{2.419913in}{3.585336in}}%
\pgfpathlineto{\pgfqpoint{2.420671in}{3.528747in}}%
\pgfpathlineto{\pgfqpoint{2.420765in}{3.524341in}}%
\pgfpathlineto{\pgfqpoint{2.421145in}{3.557269in}}%
\pgfpathlineto{\pgfqpoint{2.421334in}{3.568280in}}%
\pgfpathlineto{\pgfqpoint{2.421618in}{3.509756in}}%
\pgfpathlineto{\pgfqpoint{2.422092in}{3.401710in}}%
\pgfpathlineto{\pgfqpoint{2.422566in}{3.508611in}}%
\pgfpathlineto{\pgfqpoint{2.423040in}{3.621816in}}%
\pgfpathlineto{\pgfqpoint{2.423798in}{3.555951in}}%
\pgfpathlineto{\pgfqpoint{2.425219in}{3.434560in}}%
\pgfpathlineto{\pgfqpoint{2.424366in}{3.572163in}}%
\pgfpathlineto{\pgfqpoint{2.425598in}{3.493286in}}%
\pgfpathlineto{\pgfqpoint{2.426167in}{3.608321in}}%
\pgfpathlineto{\pgfqpoint{2.426641in}{3.500042in}}%
\pgfpathlineto{\pgfqpoint{2.426925in}{3.440204in}}%
\pgfpathlineto{\pgfqpoint{2.427778in}{3.484675in}}%
\pgfpathlineto{\pgfqpoint{2.427873in}{3.487563in}}%
\pgfpathlineto{\pgfqpoint{2.428062in}{3.467575in}}%
\pgfpathlineto{\pgfqpoint{2.428252in}{3.441291in}}%
\pgfpathlineto{\pgfqpoint{2.428631in}{3.522928in}}%
\pgfpathlineto{\pgfqpoint{2.429389in}{3.733074in}}%
\pgfpathlineto{\pgfqpoint{2.429863in}{3.583999in}}%
\pgfpathlineto{\pgfqpoint{2.430052in}{3.548183in}}%
\pgfpathlineto{\pgfqpoint{2.431095in}{3.550384in}}%
\pgfpathlineto{\pgfqpoint{2.431569in}{3.518513in}}%
\pgfpathlineto{\pgfqpoint{2.431758in}{3.543270in}}%
\pgfpathlineto{\pgfqpoint{2.432327in}{3.712507in}}%
\pgfpathlineto{\pgfqpoint{2.432895in}{3.605658in}}%
\pgfpathlineto{\pgfqpoint{2.433369in}{3.501745in}}%
\pgfpathlineto{\pgfqpoint{2.434127in}{3.547246in}}%
\pgfpathlineto{\pgfqpoint{2.434696in}{3.504709in}}%
\pgfpathlineto{\pgfqpoint{2.434885in}{3.530468in}}%
\pgfpathlineto{\pgfqpoint{2.435454in}{3.662189in}}%
\pgfpathlineto{\pgfqpoint{2.435928in}{3.566143in}}%
\pgfpathlineto{\pgfqpoint{2.436591in}{3.430024in}}%
\pgfpathlineto{\pgfqpoint{2.437160in}{3.518124in}}%
\pgfpathlineto{\pgfqpoint{2.437254in}{3.520158in}}%
\pgfpathlineto{\pgfqpoint{2.437444in}{3.504910in}}%
\pgfpathlineto{\pgfqpoint{2.437728in}{3.488874in}}%
\pgfpathlineto{\pgfqpoint{2.438107in}{3.529002in}}%
\pgfpathlineto{\pgfqpoint{2.438581in}{3.628102in}}%
\pgfpathlineto{\pgfqpoint{2.439055in}{3.525615in}}%
\pgfpathlineto{\pgfqpoint{2.439529in}{3.395470in}}%
\pgfpathlineto{\pgfqpoint{2.440192in}{3.499596in}}%
\pgfpathlineto{\pgfqpoint{2.441708in}{3.608370in}}%
\pgfpathlineto{\pgfqpoint{2.442277in}{3.488750in}}%
\pgfpathlineto{\pgfqpoint{2.442656in}{3.400324in}}%
\pgfpathlineto{\pgfqpoint{2.443225in}{3.534534in}}%
\pgfpathlineto{\pgfqpoint{2.444741in}{3.649718in}}%
\pgfpathlineto{\pgfqpoint{2.444836in}{3.656219in}}%
\pgfpathlineto{\pgfqpoint{2.445120in}{3.626329in}}%
\pgfpathlineto{\pgfqpoint{2.445783in}{3.469433in}}%
\pgfpathlineto{\pgfqpoint{2.446352in}{3.589408in}}%
\pgfpathlineto{\pgfqpoint{2.446826in}{3.658654in}}%
\pgfpathlineto{\pgfqpoint{2.447205in}{3.579586in}}%
\pgfpathlineto{\pgfqpoint{2.448910in}{3.368329in}}%
\pgfpathlineto{\pgfqpoint{2.447773in}{3.583869in}}%
\pgfpathlineto{\pgfqpoint{2.449195in}{3.416912in}}%
\pgfpathlineto{\pgfqpoint{2.449858in}{3.545561in}}%
\pgfpathlineto{\pgfqpoint{2.450427in}{3.485235in}}%
\pgfpathlineto{\pgfqpoint{2.451943in}{3.349431in}}%
\pgfpathlineto{\pgfqpoint{2.452322in}{3.409339in}}%
\pgfpathlineto{\pgfqpoint{2.452890in}{3.560470in}}%
\pgfpathlineto{\pgfqpoint{2.453554in}{3.477616in}}%
\pgfpathlineto{\pgfqpoint{2.455070in}{3.355790in}}%
\pgfpathlineto{\pgfqpoint{2.455260in}{3.388995in}}%
\pgfpathlineto{\pgfqpoint{2.456207in}{3.930904in}}%
\pgfpathlineto{\pgfqpoint{2.457250in}{3.804424in}}%
\pgfpathlineto{\pgfqpoint{2.457344in}{3.810688in}}%
\pgfpathlineto{\pgfqpoint{2.457534in}{3.786135in}}%
\pgfpathlineto{\pgfqpoint{2.458387in}{3.299570in}}%
\pgfpathlineto{\pgfqpoint{2.459050in}{3.626167in}}%
\pgfpathlineto{\pgfqpoint{2.459145in}{3.632463in}}%
\pgfpathlineto{\pgfqpoint{2.459429in}{3.600636in}}%
\pgfpathlineto{\pgfqpoint{2.459998in}{3.480942in}}%
\pgfpathlineto{\pgfqpoint{2.460851in}{3.522257in}}%
\pgfpathlineto{\pgfqpoint{2.460945in}{3.523372in}}%
\pgfpathlineto{\pgfqpoint{2.461135in}{3.514726in}}%
\pgfpathlineto{\pgfqpoint{2.461230in}{3.511203in}}%
\pgfpathlineto{\pgfqpoint{2.461514in}{3.533594in}}%
\pgfpathlineto{\pgfqpoint{2.462272in}{3.678944in}}%
\pgfpathlineto{\pgfqpoint{2.462746in}{3.567983in}}%
\pgfpathlineto{\pgfqpoint{2.463220in}{3.493767in}}%
\pgfpathlineto{\pgfqpoint{2.463788in}{3.561145in}}%
\pgfpathlineto{\pgfqpoint{2.465399in}{3.678749in}}%
\pgfpathlineto{\pgfqpoint{2.464452in}{3.544288in}}%
\pgfpathlineto{\pgfqpoint{2.465589in}{3.659305in}}%
\pgfpathlineto{\pgfqpoint{2.466252in}{3.481273in}}%
\pgfpathlineto{\pgfqpoint{2.466916in}{3.595622in}}%
\pgfpathlineto{\pgfqpoint{2.468337in}{3.687966in}}%
\pgfpathlineto{\pgfqpoint{2.468716in}{3.660797in}}%
\pgfpathlineto{\pgfqpoint{2.469474in}{3.491475in}}%
\pgfpathlineto{\pgfqpoint{2.469948in}{3.586578in}}%
\pgfpathlineto{\pgfqpoint{2.471275in}{3.669018in}}%
\pgfpathlineto{\pgfqpoint{2.471369in}{3.670481in}}%
\pgfpathlineto{\pgfqpoint{2.471654in}{3.663549in}}%
\pgfpathlineto{\pgfqpoint{2.472128in}{3.575186in}}%
\pgfpathlineto{\pgfqpoint{2.472507in}{3.489950in}}%
\pgfpathlineto{\pgfqpoint{2.473075in}{3.616317in}}%
\pgfpathlineto{\pgfqpoint{2.473549in}{3.681145in}}%
\pgfpathlineto{\pgfqpoint{2.474307in}{3.644065in}}%
\pgfpathlineto{\pgfqpoint{2.474591in}{3.652204in}}%
\pgfpathlineto{\pgfqpoint{2.474876in}{3.624539in}}%
\pgfpathlineto{\pgfqpoint{2.475634in}{3.478488in}}%
\pgfpathlineto{\pgfqpoint{2.476108in}{3.597279in}}%
\pgfpathlineto{\pgfqpoint{2.476582in}{3.683188in}}%
\pgfpathlineto{\pgfqpoint{2.477245in}{3.618870in}}%
\pgfpathlineto{\pgfqpoint{2.478287in}{3.542785in}}%
\pgfpathlineto{\pgfqpoint{2.478761in}{3.477791in}}%
\pgfpathlineto{\pgfqpoint{2.479235in}{3.571992in}}%
\pgfpathlineto{\pgfqpoint{2.479614in}{3.636611in}}%
\pgfpathlineto{\pgfqpoint{2.480183in}{3.560834in}}%
\pgfpathlineto{\pgfqpoint{2.481794in}{3.399377in}}%
\pgfpathlineto{\pgfqpoint{2.481888in}{3.406825in}}%
\pgfpathlineto{\pgfqpoint{2.482836in}{3.586566in}}%
\pgfpathlineto{\pgfqpoint{2.483310in}{3.487516in}}%
\pgfpathlineto{\pgfqpoint{2.484731in}{3.423754in}}%
\pgfpathlineto{\pgfqpoint{2.484921in}{3.414247in}}%
\pgfpathlineto{\pgfqpoint{2.485300in}{3.454164in}}%
\pgfpathlineto{\pgfqpoint{2.485868in}{3.572795in}}%
\pgfpathlineto{\pgfqpoint{2.486342in}{3.485350in}}%
\pgfpathlineto{\pgfqpoint{2.486816in}{3.409079in}}%
\pgfpathlineto{\pgfqpoint{2.487574in}{3.442206in}}%
\pgfpathlineto{\pgfqpoint{2.487669in}{3.439918in}}%
\pgfpathlineto{\pgfqpoint{2.488048in}{3.451259in}}%
\pgfpathlineto{\pgfqpoint{2.488711in}{3.604889in}}%
\pgfpathlineto{\pgfqpoint{2.488996in}{3.641819in}}%
\pgfpathlineto{\pgfqpoint{2.489469in}{3.555086in}}%
\pgfpathlineto{\pgfqpoint{2.490038in}{3.479007in}}%
\pgfpathlineto{\pgfqpoint{2.490607in}{3.540094in}}%
\pgfpathlineto{\pgfqpoint{2.490701in}{3.543057in}}%
\pgfpathlineto{\pgfqpoint{2.491175in}{3.525724in}}%
\pgfpathlineto{\pgfqpoint{2.491270in}{3.524219in}}%
\pgfpathlineto{\pgfqpoint{2.491365in}{3.528839in}}%
\pgfpathlineto{\pgfqpoint{2.491933in}{3.610073in}}%
\pgfpathlineto{\pgfqpoint{2.492407in}{3.534773in}}%
\pgfpathlineto{\pgfqpoint{2.493070in}{3.361672in}}%
\pgfpathlineto{\pgfqpoint{2.493734in}{3.447326in}}%
\pgfpathlineto{\pgfqpoint{2.495250in}{3.538722in}}%
\pgfpathlineto{\pgfqpoint{2.494397in}{3.439699in}}%
\pgfpathlineto{\pgfqpoint{2.495534in}{3.506596in}}%
\pgfpathlineto{\pgfqpoint{2.496103in}{3.348224in}}%
\pgfpathlineto{\pgfqpoint{2.496766in}{3.458437in}}%
\pgfpathlineto{\pgfqpoint{2.498188in}{3.535829in}}%
\pgfpathlineto{\pgfqpoint{2.498377in}{3.532356in}}%
\pgfpathlineto{\pgfqpoint{2.498851in}{3.423337in}}%
\pgfpathlineto{\pgfqpoint{2.499230in}{3.319718in}}%
\pgfpathlineto{\pgfqpoint{2.499704in}{3.484095in}}%
\pgfpathlineto{\pgfqpoint{2.501315in}{3.885872in}}%
\pgfpathlineto{\pgfqpoint{2.501410in}{3.884014in}}%
\pgfpathlineto{\pgfqpoint{2.501884in}{3.740051in}}%
\pgfpathlineto{\pgfqpoint{2.502547in}{3.262477in}}%
\pgfpathlineto{\pgfqpoint{2.503210in}{3.541570in}}%
\pgfpathlineto{\pgfqpoint{2.503400in}{3.585761in}}%
\pgfpathlineto{\pgfqpoint{2.504063in}{3.511249in}}%
\pgfpathlineto{\pgfqpoint{2.504158in}{3.511823in}}%
\pgfpathlineto{\pgfqpoint{2.504347in}{3.514333in}}%
\pgfpathlineto{\pgfqpoint{2.504537in}{3.508656in}}%
\pgfpathlineto{\pgfqpoint{2.505485in}{3.346935in}}%
\pgfpathlineto{\pgfqpoint{2.506053in}{3.442844in}}%
\pgfpathlineto{\pgfqpoint{2.506432in}{3.495743in}}%
\pgfpathlineto{\pgfqpoint{2.507096in}{3.433950in}}%
\pgfpathlineto{\pgfqpoint{2.507285in}{3.441252in}}%
\pgfpathlineto{\pgfqpoint{2.507854in}{3.507899in}}%
\pgfpathlineto{\pgfqpoint{2.508422in}{3.460284in}}%
\pgfpathlineto{\pgfqpoint{2.508612in}{3.452989in}}%
\pgfpathlineto{\pgfqpoint{2.508801in}{3.475074in}}%
\pgfpathlineto{\pgfqpoint{2.509465in}{3.643014in}}%
\pgfpathlineto{\pgfqpoint{2.510128in}{3.553935in}}%
\pgfpathlineto{\pgfqpoint{2.511549in}{3.485676in}}%
\pgfpathlineto{\pgfqpoint{2.512023in}{3.551097in}}%
\pgfpathlineto{\pgfqpoint{2.512592in}{3.704431in}}%
\pgfpathlineto{\pgfqpoint{2.513160in}{3.594784in}}%
\pgfpathlineto{\pgfqpoint{2.513540in}{3.529100in}}%
\pgfpathlineto{\pgfqpoint{2.514487in}{3.548206in}}%
\pgfpathlineto{\pgfqpoint{2.514771in}{3.541908in}}%
\pgfpathlineto{\pgfqpoint{2.514866in}{3.545761in}}%
\pgfpathlineto{\pgfqpoint{2.515719in}{3.712083in}}%
\pgfpathlineto{\pgfqpoint{2.516193in}{3.621739in}}%
\pgfpathlineto{\pgfqpoint{2.516667in}{3.539380in}}%
\pgfpathlineto{\pgfqpoint{2.517330in}{3.590774in}}%
\pgfpathlineto{\pgfqpoint{2.517425in}{3.591637in}}%
\pgfpathlineto{\pgfqpoint{2.517614in}{3.586170in}}%
\pgfpathlineto{\pgfqpoint{2.517804in}{3.581307in}}%
\pgfpathlineto{\pgfqpoint{2.518183in}{3.599822in}}%
\pgfpathlineto{\pgfqpoint{2.518846in}{3.705716in}}%
\pgfpathlineto{\pgfqpoint{2.519225in}{3.620111in}}%
\pgfpathlineto{\pgfqpoint{2.519699in}{3.514376in}}%
\pgfpathlineto{\pgfqpoint{2.520363in}{3.579027in}}%
\pgfpathlineto{\pgfqpoint{2.521974in}{3.700177in}}%
\pgfpathlineto{\pgfqpoint{2.522068in}{3.687706in}}%
\pgfpathlineto{\pgfqpoint{2.522921in}{3.492141in}}%
\pgfpathlineto{\pgfqpoint{2.523490in}{3.581636in}}%
\pgfpathlineto{\pgfqpoint{2.524153in}{3.569604in}}%
\pgfpathlineto{\pgfqpoint{2.524911in}{3.617587in}}%
\pgfpathlineto{\pgfqpoint{2.525101in}{3.620531in}}%
\pgfpathlineto{\pgfqpoint{2.525290in}{3.605078in}}%
\pgfpathlineto{\pgfqpoint{2.525954in}{3.416201in}}%
\pgfpathlineto{\pgfqpoint{2.526617in}{3.513909in}}%
\pgfpathlineto{\pgfqpoint{2.526996in}{3.557587in}}%
\pgfpathlineto{\pgfqpoint{2.527565in}{3.491513in}}%
\pgfpathlineto{\pgfqpoint{2.528038in}{3.552069in}}%
\pgfpathlineto{\pgfqpoint{2.528417in}{3.488172in}}%
\pgfpathlineto{\pgfqpoint{2.529081in}{3.366279in}}%
\pgfpathlineto{\pgfqpoint{2.529555in}{3.442214in}}%
\pgfpathlineto{\pgfqpoint{2.530123in}{3.532704in}}%
\pgfpathlineto{\pgfqpoint{2.530787in}{3.486399in}}%
\pgfpathlineto{\pgfqpoint{2.532113in}{3.391282in}}%
\pgfpathlineto{\pgfqpoint{2.531355in}{3.491512in}}%
\pgfpathlineto{\pgfqpoint{2.532398in}{3.423639in}}%
\pgfpathlineto{\pgfqpoint{2.533250in}{3.588659in}}%
\pgfpathlineto{\pgfqpoint{2.533914in}{3.532364in}}%
\pgfpathlineto{\pgfqpoint{2.534009in}{3.531405in}}%
\pgfpathlineto{\pgfqpoint{2.534293in}{3.539095in}}%
\pgfpathlineto{\pgfqpoint{2.534388in}{3.541575in}}%
\pgfpathlineto{\pgfqpoint{2.534672in}{3.523655in}}%
\pgfpathlineto{\pgfqpoint{2.535240in}{3.437290in}}%
\pgfpathlineto{\pgfqpoint{2.535714in}{3.498150in}}%
\pgfpathlineto{\pgfqpoint{2.536093in}{3.575031in}}%
\pgfpathlineto{\pgfqpoint{2.536757in}{3.499208in}}%
\pgfpathlineto{\pgfqpoint{2.538368in}{3.364966in}}%
\pgfpathlineto{\pgfqpoint{2.538462in}{3.362728in}}%
\pgfpathlineto{\pgfqpoint{2.538557in}{3.370318in}}%
\pgfpathlineto{\pgfqpoint{2.539410in}{3.559454in}}%
\pgfpathlineto{\pgfqpoint{2.539979in}{3.452602in}}%
\pgfpathlineto{\pgfqpoint{2.540358in}{3.412766in}}%
\pgfpathlineto{\pgfqpoint{2.541211in}{3.432628in}}%
\pgfpathlineto{\pgfqpoint{2.541305in}{3.428414in}}%
\pgfpathlineto{\pgfqpoint{2.541684in}{3.455863in}}%
\pgfpathlineto{\pgfqpoint{2.542443in}{3.602720in}}%
\pgfpathlineto{\pgfqpoint{2.543011in}{3.501157in}}%
\pgfpathlineto{\pgfqpoint{2.543485in}{3.391757in}}%
\pgfpathlineto{\pgfqpoint{2.543959in}{3.522111in}}%
\pgfpathlineto{\pgfqpoint{2.545570in}{3.935863in}}%
\pgfpathlineto{\pgfqpoint{2.545759in}{3.917249in}}%
\pgfpathlineto{\pgfqpoint{2.546517in}{3.429191in}}%
\pgfpathlineto{\pgfqpoint{2.546802in}{3.297146in}}%
\pgfpathlineto{\pgfqpoint{2.547465in}{3.514155in}}%
\pgfpathlineto{\pgfqpoint{2.547749in}{3.510347in}}%
\pgfpathlineto{\pgfqpoint{2.548034in}{3.544700in}}%
\pgfpathlineto{\pgfqpoint{2.548602in}{3.621833in}}%
\pgfpathlineto{\pgfqpoint{2.549076in}{3.560290in}}%
\pgfpathlineto{\pgfqpoint{2.549645in}{3.441258in}}%
\pgfpathlineto{\pgfqpoint{2.550213in}{3.522366in}}%
\pgfpathlineto{\pgfqpoint{2.551540in}{3.612549in}}%
\pgfpathlineto{\pgfqpoint{2.551729in}{3.610643in}}%
\pgfpathlineto{\pgfqpoint{2.552203in}{3.548751in}}%
\pgfpathlineto{\pgfqpoint{2.552772in}{3.453533in}}%
\pgfpathlineto{\pgfqpoint{2.553246in}{3.544144in}}%
\pgfpathlineto{\pgfqpoint{2.554762in}{3.632154in}}%
\pgfpathlineto{\pgfqpoint{2.554857in}{3.634105in}}%
\pgfpathlineto{\pgfqpoint{2.554951in}{3.625708in}}%
\pgfpathlineto{\pgfqpoint{2.555804in}{3.502815in}}%
\pgfpathlineto{\pgfqpoint{2.556278in}{3.565501in}}%
\pgfpathlineto{\pgfqpoint{2.556847in}{3.660461in}}%
\pgfpathlineto{\pgfqpoint{2.557605in}{3.638274in}}%
\pgfpathlineto{\pgfqpoint{2.557984in}{3.659170in}}%
\pgfpathlineto{\pgfqpoint{2.558363in}{3.627576in}}%
\pgfpathlineto{\pgfqpoint{2.558932in}{3.548913in}}%
\pgfpathlineto{\pgfqpoint{2.559311in}{3.614967in}}%
\pgfpathlineto{\pgfqpoint{2.559974in}{3.723184in}}%
\pgfpathlineto{\pgfqpoint{2.560543in}{3.660521in}}%
\pgfpathlineto{\pgfqpoint{2.560637in}{3.660345in}}%
\pgfpathlineto{\pgfqpoint{2.560922in}{3.671118in}}%
\pgfpathlineto{\pgfqpoint{2.561301in}{3.651304in}}%
\pgfpathlineto{\pgfqpoint{2.561964in}{3.552046in}}%
\pgfpathlineto{\pgfqpoint{2.562533in}{3.636936in}}%
\pgfpathlineto{\pgfqpoint{2.562912in}{3.732584in}}%
\pgfpathlineto{\pgfqpoint{2.563670in}{3.647190in}}%
\pgfpathlineto{\pgfqpoint{2.565186in}{3.556726in}}%
\pgfpathlineto{\pgfqpoint{2.565375in}{3.588510in}}%
\pgfpathlineto{\pgfqpoint{2.566228in}{3.733009in}}%
\pgfpathlineto{\pgfqpoint{2.566607in}{3.653913in}}%
\pgfpathlineto{\pgfqpoint{2.568029in}{3.569582in}}%
\pgfpathlineto{\pgfqpoint{2.568218in}{3.564748in}}%
\pgfpathlineto{\pgfqpoint{2.568503in}{3.585741in}}%
\pgfpathlineto{\pgfqpoint{2.569166in}{3.680290in}}%
\pgfpathlineto{\pgfqpoint{2.569640in}{3.595174in}}%
\pgfpathlineto{\pgfqpoint{2.570114in}{3.517676in}}%
\pgfpathlineto{\pgfqpoint{2.570967in}{3.520047in}}%
\pgfpathlineto{\pgfqpoint{2.571535in}{3.543944in}}%
\pgfpathlineto{\pgfqpoint{2.572388in}{3.652239in}}%
\pgfpathlineto{\pgfqpoint{2.572767in}{3.582234in}}%
\pgfpathlineto{\pgfqpoint{2.573336in}{3.495316in}}%
\pgfpathlineto{\pgfqpoint{2.573904in}{3.542382in}}%
\pgfpathlineto{\pgfqpoint{2.575610in}{3.638407in}}%
\pgfpathlineto{\pgfqpoint{2.574378in}{3.531454in}}%
\pgfpathlineto{\pgfqpoint{2.575800in}{3.604619in}}%
\pgfpathlineto{\pgfqpoint{2.576273in}{3.504376in}}%
\pgfpathlineto{\pgfqpoint{2.576937in}{3.575876in}}%
\pgfpathlineto{\pgfqpoint{2.578358in}{3.716832in}}%
\pgfpathlineto{\pgfqpoint{2.578642in}{3.706753in}}%
\pgfpathlineto{\pgfqpoint{2.579495in}{3.558971in}}%
\pgfpathlineto{\pgfqpoint{2.580538in}{3.614371in}}%
\pgfpathlineto{\pgfqpoint{2.581106in}{3.617119in}}%
\pgfpathlineto{\pgfqpoint{2.581959in}{3.529852in}}%
\pgfpathlineto{\pgfqpoint{2.582528in}{3.395212in}}%
\pgfpathlineto{\pgfqpoint{2.583191in}{3.471116in}}%
\pgfpathlineto{\pgfqpoint{2.584802in}{3.554319in}}%
\pgfpathlineto{\pgfqpoint{2.584992in}{3.548225in}}%
\pgfpathlineto{\pgfqpoint{2.585655in}{3.496712in}}%
\pgfpathlineto{\pgfqpoint{2.585939in}{3.529571in}}%
\pgfpathlineto{\pgfqpoint{2.586603in}{3.678508in}}%
\pgfpathlineto{\pgfqpoint{2.587266in}{3.602276in}}%
\pgfpathlineto{\pgfqpoint{2.588119in}{3.571565in}}%
\pgfpathlineto{\pgfqpoint{2.588308in}{3.585768in}}%
\pgfpathlineto{\pgfqpoint{2.589730in}{4.060949in}}%
\pgfpathlineto{\pgfqpoint{2.590583in}{3.860706in}}%
\pgfpathlineto{\pgfqpoint{2.591341in}{3.408609in}}%
\pgfpathlineto{\pgfqpoint{2.592099in}{3.559358in}}%
\pgfpathlineto{\pgfqpoint{2.592762in}{3.695413in}}%
\pgfpathlineto{\pgfqpoint{2.593426in}{3.619821in}}%
\pgfpathlineto{\pgfqpoint{2.594847in}{3.556737in}}%
\pgfpathlineto{\pgfqpoint{2.594942in}{3.556448in}}%
\pgfpathlineto{\pgfqpoint{2.595321in}{3.618821in}}%
\pgfpathlineto{\pgfqpoint{2.595984in}{3.731176in}}%
\pgfpathlineto{\pgfqpoint{2.596458in}{3.630657in}}%
\pgfpathlineto{\pgfqpoint{2.597027in}{3.576427in}}%
\pgfpathlineto{\pgfqpoint{2.597690in}{3.610267in}}%
\pgfpathlineto{\pgfqpoint{2.597974in}{3.601756in}}%
\pgfpathlineto{\pgfqpoint{2.598259in}{3.622540in}}%
\pgfpathlineto{\pgfqpoint{2.599017in}{3.740726in}}%
\pgfpathlineto{\pgfqpoint{2.599585in}{3.679711in}}%
\pgfpathlineto{\pgfqpoint{2.600154in}{3.602273in}}%
\pgfpathlineto{\pgfqpoint{2.600722in}{3.662237in}}%
\pgfpathlineto{\pgfqpoint{2.600912in}{3.656239in}}%
\pgfpathlineto{\pgfqpoint{2.601196in}{3.675537in}}%
\pgfpathlineto{\pgfqpoint{2.602049in}{3.775066in}}%
\pgfpathlineto{\pgfqpoint{2.602618in}{3.708966in}}%
\pgfpathlineto{\pgfqpoint{2.603092in}{3.625044in}}%
\pgfpathlineto{\pgfqpoint{2.603755in}{3.686711in}}%
\pgfpathlineto{\pgfqpoint{2.605082in}{3.778407in}}%
\pgfpathlineto{\pgfqpoint{2.605461in}{3.745859in}}%
\pgfpathlineto{\pgfqpoint{2.606219in}{3.588516in}}%
\pgfpathlineto{\pgfqpoint{2.606882in}{3.687417in}}%
\pgfpathlineto{\pgfqpoint{2.608114in}{3.746475in}}%
\pgfpathlineto{\pgfqpoint{2.608398in}{3.722347in}}%
\pgfpathlineto{\pgfqpoint{2.609251in}{3.580420in}}%
\pgfpathlineto{\pgfqpoint{2.609915in}{3.667298in}}%
\pgfpathlineto{\pgfqpoint{2.610388in}{3.722404in}}%
\pgfpathlineto{\pgfqpoint{2.611147in}{3.696102in}}%
\pgfpathlineto{\pgfqpoint{2.611999in}{3.599784in}}%
\pgfpathlineto{\pgfqpoint{2.612378in}{3.523357in}}%
\pgfpathlineto{\pgfqpoint{2.612947in}{3.609249in}}%
\pgfpathlineto{\pgfqpoint{2.613326in}{3.646894in}}%
\pgfpathlineto{\pgfqpoint{2.613989in}{3.606105in}}%
\pgfpathlineto{\pgfqpoint{2.614084in}{3.606661in}}%
\pgfpathlineto{\pgfqpoint{2.614274in}{3.601675in}}%
\pgfpathlineto{\pgfqpoint{2.615600in}{3.471779in}}%
\pgfpathlineto{\pgfqpoint{2.615980in}{3.530522in}}%
\pgfpathlineto{\pgfqpoint{2.616453in}{3.612723in}}%
\pgfpathlineto{\pgfqpoint{2.617117in}{3.556609in}}%
\pgfpathlineto{\pgfqpoint{2.618443in}{3.457518in}}%
\pgfpathlineto{\pgfqpoint{2.618822in}{3.491270in}}%
\pgfpathlineto{\pgfqpoint{2.619486in}{3.620099in}}%
\pgfpathlineto{\pgfqpoint{2.620244in}{3.533275in}}%
\pgfpathlineto{\pgfqpoint{2.620433in}{3.514199in}}%
\pgfpathlineto{\pgfqpoint{2.621002in}{3.539778in}}%
\pgfpathlineto{\pgfqpoint{2.621286in}{3.536155in}}%
\pgfpathlineto{\pgfqpoint{2.621760in}{3.519568in}}%
\pgfpathlineto{\pgfqpoint{2.621950in}{3.541648in}}%
\pgfpathlineto{\pgfqpoint{2.622613in}{3.697127in}}%
\pgfpathlineto{\pgfqpoint{2.623276in}{3.604068in}}%
\pgfpathlineto{\pgfqpoint{2.624887in}{3.524749in}}%
\pgfpathlineto{\pgfqpoint{2.624982in}{3.526227in}}%
\pgfpathlineto{\pgfqpoint{2.625551in}{3.607778in}}%
\pgfpathlineto{\pgfqpoint{2.626119in}{3.555117in}}%
\pgfpathlineto{\pgfqpoint{2.626972in}{3.439682in}}%
\pgfpathlineto{\pgfqpoint{2.627541in}{3.475693in}}%
\pgfpathlineto{\pgfqpoint{2.627636in}{3.474598in}}%
\pgfpathlineto{\pgfqpoint{2.627920in}{3.481333in}}%
\pgfpathlineto{\pgfqpoint{2.628867in}{3.567128in}}%
\pgfpathlineto{\pgfqpoint{2.629246in}{3.513064in}}%
\pgfpathlineto{\pgfqpoint{2.629910in}{3.414022in}}%
\pgfpathlineto{\pgfqpoint{2.630478in}{3.473841in}}%
\pgfpathlineto{\pgfqpoint{2.631900in}{3.542652in}}%
\pgfpathlineto{\pgfqpoint{2.632279in}{3.516751in}}%
\pgfpathlineto{\pgfqpoint{2.632658in}{3.410166in}}%
\pgfpathlineto{\pgfqpoint{2.633037in}{3.551845in}}%
\pgfpathlineto{\pgfqpoint{2.633890in}{3.881218in}}%
\pgfpathlineto{\pgfqpoint{2.634648in}{3.878159in}}%
\pgfpathlineto{\pgfqpoint{2.634838in}{3.889714in}}%
\pgfpathlineto{\pgfqpoint{2.635027in}{3.855057in}}%
\pgfpathlineto{\pgfqpoint{2.635880in}{3.270737in}}%
\pgfpathlineto{\pgfqpoint{2.636733in}{3.528919in}}%
\pgfpathlineto{\pgfqpoint{2.637207in}{3.591048in}}%
\pgfpathlineto{\pgfqpoint{2.637965in}{3.569725in}}%
\pgfpathlineto{\pgfqpoint{2.638912in}{3.477842in}}%
\pgfpathlineto{\pgfqpoint{2.639197in}{3.451417in}}%
\pgfpathlineto{\pgfqpoint{2.639576in}{3.517666in}}%
\pgfpathlineto{\pgfqpoint{2.640239in}{3.598872in}}%
\pgfpathlineto{\pgfqpoint{2.640902in}{3.574099in}}%
\pgfpathlineto{\pgfqpoint{2.641187in}{3.576591in}}%
\pgfpathlineto{\pgfqpoint{2.641471in}{3.571144in}}%
\pgfpathlineto{\pgfqpoint{2.642134in}{3.502318in}}%
\pgfpathlineto{\pgfqpoint{2.642608in}{3.547075in}}%
\pgfpathlineto{\pgfqpoint{2.643177in}{3.651916in}}%
\pgfpathlineto{\pgfqpoint{2.643935in}{3.607214in}}%
\pgfpathlineto{\pgfqpoint{2.645262in}{3.495931in}}%
\pgfpathlineto{\pgfqpoint{2.645641in}{3.558176in}}%
\pgfpathlineto{\pgfqpoint{2.646399in}{3.671661in}}%
\pgfpathlineto{\pgfqpoint{2.646873in}{3.607680in}}%
\pgfpathlineto{\pgfqpoint{2.648294in}{3.547576in}}%
\pgfpathlineto{\pgfqpoint{2.648673in}{3.578617in}}%
\pgfpathlineto{\pgfqpoint{2.649431in}{3.668843in}}%
\pgfpathlineto{\pgfqpoint{2.649905in}{3.626220in}}%
\pgfpathlineto{\pgfqpoint{2.651327in}{3.547225in}}%
\pgfpathlineto{\pgfqpoint{2.651421in}{3.545892in}}%
\pgfpathlineto{\pgfqpoint{2.651611in}{3.553426in}}%
\pgfpathlineto{\pgfqpoint{2.652558in}{3.654994in}}%
\pgfpathlineto{\pgfqpoint{2.653032in}{3.586522in}}%
\pgfpathlineto{\pgfqpoint{2.653411in}{3.550457in}}%
\pgfpathlineto{\pgfqpoint{2.654169in}{3.573642in}}%
\pgfpathlineto{\pgfqpoint{2.654643in}{3.565283in}}%
\pgfpathlineto{\pgfqpoint{2.655117in}{3.616455in}}%
\pgfpathlineto{\pgfqpoint{2.655591in}{3.657980in}}%
\pgfpathlineto{\pgfqpoint{2.655970in}{3.612816in}}%
\pgfpathlineto{\pgfqpoint{2.656728in}{3.489625in}}%
\pgfpathlineto{\pgfqpoint{2.657581in}{3.497072in}}%
\pgfpathlineto{\pgfqpoint{2.657676in}{3.494373in}}%
\pgfpathlineto{\pgfqpoint{2.658055in}{3.508170in}}%
\pgfpathlineto{\pgfqpoint{2.658339in}{3.517325in}}%
\pgfpathlineto{\pgfqpoint{2.658718in}{3.498507in}}%
\pgfpathlineto{\pgfqpoint{2.659666in}{3.318295in}}%
\pgfpathlineto{\pgfqpoint{2.660329in}{3.409864in}}%
\pgfpathlineto{\pgfqpoint{2.661656in}{3.562508in}}%
\pgfpathlineto{\pgfqpoint{2.661940in}{3.540467in}}%
\pgfpathlineto{\pgfqpoint{2.662888in}{3.401402in}}%
\pgfpathlineto{\pgfqpoint{2.663362in}{3.481798in}}%
\pgfpathlineto{\pgfqpoint{2.663835in}{3.533701in}}%
\pgfpathlineto{\pgfqpoint{2.664688in}{3.525439in}}%
\pgfpathlineto{\pgfqpoint{2.665257in}{3.499111in}}%
\pgfpathlineto{\pgfqpoint{2.665825in}{3.450263in}}%
\pgfpathlineto{\pgfqpoint{2.666299in}{3.488871in}}%
\pgfpathlineto{\pgfqpoint{2.667057in}{3.609199in}}%
\pgfpathlineto{\pgfqpoint{2.667815in}{3.574796in}}%
\pgfpathlineto{\pgfqpoint{2.668195in}{3.543490in}}%
\pgfpathlineto{\pgfqpoint{2.669142in}{3.479127in}}%
\pgfpathlineto{\pgfqpoint{2.669426in}{3.514833in}}%
\pgfpathlineto{\pgfqpoint{2.669900in}{3.558311in}}%
\pgfpathlineto{\pgfqpoint{2.670469in}{3.507114in}}%
\pgfpathlineto{\pgfqpoint{2.672080in}{3.388898in}}%
\pgfpathlineto{\pgfqpoint{2.672459in}{3.432876in}}%
\pgfpathlineto{\pgfqpoint{2.673028in}{3.528241in}}%
\pgfpathlineto{\pgfqpoint{2.673786in}{3.481821in}}%
\pgfpathlineto{\pgfqpoint{2.674923in}{3.429016in}}%
\pgfpathlineto{\pgfqpoint{2.675302in}{3.446508in}}%
\pgfpathlineto{\pgfqpoint{2.676060in}{3.556239in}}%
\pgfpathlineto{\pgfqpoint{2.676629in}{3.494562in}}%
\pgfpathlineto{\pgfqpoint{2.676913in}{3.458151in}}%
\pgfpathlineto{\pgfqpoint{2.677292in}{3.569299in}}%
\pgfpathlineto{\pgfqpoint{2.679187in}{3.910598in}}%
\pgfpathlineto{\pgfqpoint{2.679566in}{3.731116in}}%
\pgfpathlineto{\pgfqpoint{2.680230in}{3.276135in}}%
\pgfpathlineto{\pgfqpoint{2.680988in}{3.467778in}}%
\pgfpathlineto{\pgfqpoint{2.681082in}{3.467234in}}%
\pgfpathlineto{\pgfqpoint{2.681177in}{3.468481in}}%
\pgfpathlineto{\pgfqpoint{2.682314in}{3.559497in}}%
\pgfpathlineto{\pgfqpoint{2.682978in}{3.512341in}}%
\pgfpathlineto{\pgfqpoint{2.683452in}{3.461991in}}%
\pgfpathlineto{\pgfqpoint{2.684020in}{3.502996in}}%
\pgfpathlineto{\pgfqpoint{2.685442in}{3.588086in}}%
\pgfpathlineto{\pgfqpoint{2.685821in}{3.554924in}}%
\pgfpathlineto{\pgfqpoint{2.686674in}{3.497627in}}%
\pgfpathlineto{\pgfqpoint{2.687053in}{3.536895in}}%
\pgfpathlineto{\pgfqpoint{2.688569in}{3.624629in}}%
\pgfpathlineto{\pgfqpoint{2.688664in}{3.618807in}}%
\pgfpathlineto{\pgfqpoint{2.689516in}{3.498456in}}%
\pgfpathlineto{\pgfqpoint{2.690085in}{3.571379in}}%
\pgfpathlineto{\pgfqpoint{2.691317in}{3.629014in}}%
\pgfpathlineto{\pgfqpoint{2.691412in}{3.625465in}}%
\pgfpathlineto{\pgfqpoint{2.692549in}{3.518114in}}%
\pgfpathlineto{\pgfqpoint{2.693118in}{3.554490in}}%
\pgfpathlineto{\pgfqpoint{2.693781in}{3.621292in}}%
\pgfpathlineto{\pgfqpoint{2.694444in}{3.605355in}}%
\pgfpathlineto{\pgfqpoint{2.695771in}{3.503931in}}%
\pgfpathlineto{\pgfqpoint{2.696245in}{3.557690in}}%
\pgfpathlineto{\pgfqpoint{2.696908in}{3.617038in}}%
\pgfpathlineto{\pgfqpoint{2.697477in}{3.575342in}}%
\pgfpathlineto{\pgfqpoint{2.698709in}{3.486759in}}%
\pgfpathlineto{\pgfqpoint{2.699088in}{3.512229in}}%
\pgfpathlineto{\pgfqpoint{2.699751in}{3.598667in}}%
\pgfpathlineto{\pgfqpoint{2.700320in}{3.562812in}}%
\pgfpathlineto{\pgfqpoint{2.701931in}{3.441509in}}%
\pgfpathlineto{\pgfqpoint{2.702025in}{3.439953in}}%
\pgfpathlineto{\pgfqpoint{2.702120in}{3.445839in}}%
\pgfpathlineto{\pgfqpoint{2.702878in}{3.555760in}}%
\pgfpathlineto{\pgfqpoint{2.703447in}{3.497500in}}%
\pgfpathlineto{\pgfqpoint{2.704584in}{3.417724in}}%
\pgfpathlineto{\pgfqpoint{2.704868in}{3.429271in}}%
\pgfpathlineto{\pgfqpoint{2.705342in}{3.477153in}}%
\pgfpathlineto{\pgfqpoint{2.706100in}{3.527811in}}%
\pgfpathlineto{\pgfqpoint{2.706479in}{3.493201in}}%
\pgfpathlineto{\pgfqpoint{2.707237in}{3.418857in}}%
\pgfpathlineto{\pgfqpoint{2.707901in}{3.444084in}}%
\pgfpathlineto{\pgfqpoint{2.707995in}{3.443572in}}%
\pgfpathlineto{\pgfqpoint{2.708090in}{3.444341in}}%
\pgfpathlineto{\pgfqpoint{2.709227in}{3.535393in}}%
\pgfpathlineto{\pgfqpoint{2.709701in}{3.483831in}}%
\pgfpathlineto{\pgfqpoint{2.710175in}{3.449103in}}%
\pgfpathlineto{\pgfqpoint{2.710649in}{3.489068in}}%
\pgfpathlineto{\pgfqpoint{2.712165in}{3.613076in}}%
\pgfpathlineto{\pgfqpoint{2.712544in}{3.556329in}}%
\pgfpathlineto{\pgfqpoint{2.713207in}{3.474047in}}%
\pgfpathlineto{\pgfqpoint{2.714060in}{3.488365in}}%
\pgfpathlineto{\pgfqpoint{2.715008in}{3.510534in}}%
\pgfpathlineto{\pgfqpoint{2.715198in}{3.498507in}}%
\pgfpathlineto{\pgfqpoint{2.716145in}{3.376868in}}%
\pgfpathlineto{\pgfqpoint{2.716809in}{3.419872in}}%
\pgfpathlineto{\pgfqpoint{2.718040in}{3.481319in}}%
\pgfpathlineto{\pgfqpoint{2.718420in}{3.473858in}}%
\pgfpathlineto{\pgfqpoint{2.718893in}{3.434954in}}%
\pgfpathlineto{\pgfqpoint{2.719367in}{3.367664in}}%
\pgfpathlineto{\pgfqpoint{2.719936in}{3.436991in}}%
\pgfpathlineto{\pgfqpoint{2.720504in}{3.482135in}}%
\pgfpathlineto{\pgfqpoint{2.721168in}{3.453117in}}%
\pgfpathlineto{\pgfqpoint{2.721357in}{3.447789in}}%
\pgfpathlineto{\pgfqpoint{2.721642in}{3.472614in}}%
\pgfpathlineto{\pgfqpoint{2.723632in}{3.825780in}}%
\pgfpathlineto{\pgfqpoint{2.723916in}{3.757818in}}%
\pgfpathlineto{\pgfqpoint{2.724769in}{3.276388in}}%
\pgfpathlineto{\pgfqpoint{2.725716in}{3.440182in}}%
\pgfpathlineto{\pgfqpoint{2.726664in}{3.532792in}}%
\pgfpathlineto{\pgfqpoint{2.727233in}{3.489971in}}%
\pgfpathlineto{\pgfqpoint{2.728559in}{3.423699in}}%
\pgfpathlineto{\pgfqpoint{2.728938in}{3.462832in}}%
\pgfpathlineto{\pgfqpoint{2.729696in}{3.504925in}}%
\pgfpathlineto{\pgfqpoint{2.730076in}{3.486840in}}%
\pgfpathlineto{\pgfqpoint{2.731592in}{3.329123in}}%
\pgfpathlineto{\pgfqpoint{2.731971in}{3.383704in}}%
\pgfpathlineto{\pgfqpoint{2.733487in}{3.515594in}}%
\pgfpathlineto{\pgfqpoint{2.733771in}{3.501996in}}%
\pgfpathlineto{\pgfqpoint{2.734435in}{3.494480in}}%
\pgfpathlineto{\pgfqpoint{2.734624in}{3.504072in}}%
\pgfpathlineto{\pgfqpoint{2.735951in}{3.604190in}}%
\pgfpathlineto{\pgfqpoint{2.736330in}{3.564337in}}%
\pgfpathlineto{\pgfqpoint{2.736899in}{3.524370in}}%
\pgfpathlineto{\pgfqpoint{2.737562in}{3.537054in}}%
\pgfpathlineto{\pgfqpoint{2.738983in}{3.649456in}}%
\pgfpathlineto{\pgfqpoint{2.739268in}{3.621074in}}%
\pgfpathlineto{\pgfqpoint{2.740310in}{3.522432in}}%
\pgfpathlineto{\pgfqpoint{2.740594in}{3.541395in}}%
\pgfpathlineto{\pgfqpoint{2.741921in}{3.636177in}}%
\pgfpathlineto{\pgfqpoint{2.742205in}{3.615042in}}%
\pgfpathlineto{\pgfqpoint{2.742774in}{3.555016in}}%
\pgfpathlineto{\pgfqpoint{2.743437in}{3.575384in}}%
\pgfpathlineto{\pgfqpoint{2.744385in}{3.681795in}}%
\pgfpathlineto{\pgfqpoint{2.745143in}{3.661109in}}%
\pgfpathlineto{\pgfqpoint{2.746185in}{3.547246in}}%
\pgfpathlineto{\pgfqpoint{2.746659in}{3.582855in}}%
\pgfpathlineto{\pgfqpoint{2.747417in}{3.620766in}}%
\pgfpathlineto{\pgfqpoint{2.747796in}{3.590303in}}%
\pgfpathlineto{\pgfqpoint{2.748365in}{3.550701in}}%
\pgfpathlineto{\pgfqpoint{2.749123in}{3.472662in}}%
\pgfpathlineto{\pgfqpoint{2.749692in}{3.517357in}}%
\pgfpathlineto{\pgfqpoint{2.749786in}{3.516824in}}%
\pgfpathlineto{\pgfqpoint{2.749881in}{3.520365in}}%
\pgfpathlineto{\pgfqpoint{2.750260in}{3.560784in}}%
\pgfpathlineto{\pgfqpoint{2.751113in}{3.535891in}}%
\pgfpathlineto{\pgfqpoint{2.752156in}{3.448764in}}%
\pgfpathlineto{\pgfqpoint{2.752535in}{3.490337in}}%
\pgfpathlineto{\pgfqpoint{2.753482in}{3.551114in}}%
\pgfpathlineto{\pgfqpoint{2.753956in}{3.532431in}}%
\pgfpathlineto{\pgfqpoint{2.755093in}{3.474855in}}%
\pgfpathlineto{\pgfqpoint{2.755472in}{3.496486in}}%
\pgfpathlineto{\pgfqpoint{2.756420in}{3.604120in}}%
\pgfpathlineto{\pgfqpoint{2.757178in}{3.559021in}}%
\pgfpathlineto{\pgfqpoint{2.758789in}{3.470539in}}%
\pgfpathlineto{\pgfqpoint{2.758884in}{3.469100in}}%
\pgfpathlineto{\pgfqpoint{2.759073in}{3.478716in}}%
\pgfpathlineto{\pgfqpoint{2.759263in}{3.489397in}}%
\pgfpathlineto{\pgfqpoint{2.759642in}{3.457329in}}%
\pgfpathlineto{\pgfqpoint{2.761253in}{3.299420in}}%
\pgfpathlineto{\pgfqpoint{2.761727in}{3.330543in}}%
\pgfpathlineto{\pgfqpoint{2.762390in}{3.382255in}}%
\pgfpathlineto{\pgfqpoint{2.763148in}{3.364042in}}%
\pgfpathlineto{\pgfqpoint{2.764191in}{3.312359in}}%
\pgfpathlineto{\pgfqpoint{2.764380in}{3.325326in}}%
\pgfpathlineto{\pgfqpoint{2.765612in}{3.425954in}}%
\pgfpathlineto{\pgfqpoint{2.765896in}{3.400114in}}%
\pgfpathlineto{\pgfqpoint{2.766465in}{3.336962in}}%
\pgfpathlineto{\pgfqpoint{2.766749in}{3.401198in}}%
\pgfpathlineto{\pgfqpoint{2.768550in}{3.765790in}}%
\pgfpathlineto{\pgfqpoint{2.768645in}{3.765579in}}%
\pgfpathlineto{\pgfqpoint{2.768929in}{3.717402in}}%
\pgfpathlineto{\pgfqpoint{2.769782in}{3.219986in}}%
\pgfpathlineto{\pgfqpoint{2.770540in}{3.431296in}}%
\pgfpathlineto{\pgfqpoint{2.771487in}{3.534924in}}%
\pgfpathlineto{\pgfqpoint{2.771961in}{3.486457in}}%
\pgfpathlineto{\pgfqpoint{2.772719in}{3.421759in}}%
\pgfpathlineto{\pgfqpoint{2.773572in}{3.455204in}}%
\pgfpathlineto{\pgfqpoint{2.774520in}{3.518064in}}%
\pgfpathlineto{\pgfqpoint{2.774899in}{3.496696in}}%
\pgfpathlineto{\pgfqpoint{2.776036in}{3.418924in}}%
\pgfpathlineto{\pgfqpoint{2.776320in}{3.443576in}}%
\pgfpathlineto{\pgfqpoint{2.777742in}{3.524744in}}%
\pgfpathlineto{\pgfqpoint{2.778121in}{3.489243in}}%
\pgfpathlineto{\pgfqpoint{2.778784in}{3.453284in}}%
\pgfpathlineto{\pgfqpoint{2.779353in}{3.471909in}}%
\pgfpathlineto{\pgfqpoint{2.780300in}{3.569006in}}%
\pgfpathlineto{\pgfqpoint{2.781059in}{3.535407in}}%
\pgfpathlineto{\pgfqpoint{2.782006in}{3.473681in}}%
\pgfpathlineto{\pgfqpoint{2.782385in}{3.512413in}}%
\pgfpathlineto{\pgfqpoint{2.783143in}{3.597640in}}%
\pgfpathlineto{\pgfqpoint{2.783902in}{3.568143in}}%
\pgfpathlineto{\pgfqpoint{2.784849in}{3.490249in}}%
\pgfpathlineto{\pgfqpoint{2.785228in}{3.518894in}}%
\pgfpathlineto{\pgfqpoint{2.786081in}{3.563403in}}%
\pgfpathlineto{\pgfqpoint{2.786650in}{3.538782in}}%
\pgfpathlineto{\pgfqpoint{2.787597in}{3.468277in}}%
\pgfpathlineto{\pgfqpoint{2.788450in}{3.505043in}}%
\pgfpathlineto{\pgfqpoint{2.789587in}{3.542472in}}%
\pgfpathlineto{\pgfqpoint{2.789777in}{3.528075in}}%
\pgfpathlineto{\pgfqpoint{2.790914in}{3.431954in}}%
\pgfpathlineto{\pgfqpoint{2.791293in}{3.464455in}}%
\pgfpathlineto{\pgfqpoint{2.791956in}{3.486391in}}%
\pgfpathlineto{\pgfqpoint{2.792336in}{3.463457in}}%
\pgfpathlineto{\pgfqpoint{2.793757in}{3.355443in}}%
\pgfpathlineto{\pgfqpoint{2.794705in}{3.405153in}}%
\pgfpathlineto{\pgfqpoint{2.794989in}{3.426502in}}%
\pgfpathlineto{\pgfqpoint{2.795558in}{3.395121in}}%
\pgfpathlineto{\pgfqpoint{2.795747in}{3.397248in}}%
\pgfpathlineto{\pgfqpoint{2.796979in}{3.326209in}}%
\pgfpathlineto{\pgfqpoint{2.797358in}{3.362564in}}%
\pgfpathlineto{\pgfqpoint{2.798211in}{3.430277in}}%
\pgfpathlineto{\pgfqpoint{2.798590in}{3.395123in}}%
\pgfpathlineto{\pgfqpoint{2.799443in}{3.323716in}}%
\pgfpathlineto{\pgfqpoint{2.799822in}{3.372876in}}%
\pgfpathlineto{\pgfqpoint{2.800959in}{3.449479in}}%
\pgfpathlineto{\pgfqpoint{2.801338in}{3.417362in}}%
\pgfpathlineto{\pgfqpoint{2.802760in}{3.291880in}}%
\pgfpathlineto{\pgfqpoint{2.803139in}{3.312694in}}%
\pgfpathlineto{\pgfqpoint{2.804086in}{3.423335in}}%
\pgfpathlineto{\pgfqpoint{2.804750in}{3.381688in}}%
\pgfpathlineto{\pgfqpoint{2.805413in}{3.342023in}}%
\pgfpathlineto{\pgfqpoint{2.806266in}{3.349276in}}%
\pgfpathlineto{\pgfqpoint{2.807308in}{3.397491in}}%
\pgfpathlineto{\pgfqpoint{2.807687in}{3.373330in}}%
\pgfpathlineto{\pgfqpoint{2.808445in}{3.293264in}}%
\pgfpathlineto{\pgfqpoint{2.809109in}{3.344982in}}%
\pgfpathlineto{\pgfqpoint{2.810246in}{3.415985in}}%
\pgfpathlineto{\pgfqpoint{2.810530in}{3.392514in}}%
\pgfpathlineto{\pgfqpoint{2.811573in}{3.301452in}}%
\pgfpathlineto{\pgfqpoint{2.811952in}{3.351400in}}%
\pgfpathlineto{\pgfqpoint{2.812994in}{3.778404in}}%
\pgfpathlineto{\pgfqpoint{2.813942in}{3.693743in}}%
\pgfpathlineto{\pgfqpoint{2.814510in}{3.443327in}}%
\pgfpathlineto{\pgfqpoint{2.815079in}{3.194214in}}%
\pgfpathlineto{\pgfqpoint{2.815647in}{3.418792in}}%
\pgfpathlineto{\pgfqpoint{2.815837in}{3.436585in}}%
\pgfpathlineto{\pgfqpoint{2.816690in}{3.417673in}}%
\pgfpathlineto{\pgfqpoint{2.817543in}{3.361089in}}%
\pgfpathlineto{\pgfqpoint{2.818017in}{3.393168in}}%
\pgfpathlineto{\pgfqpoint{2.819154in}{3.475049in}}%
\pgfpathlineto{\pgfqpoint{2.819438in}{3.444556in}}%
\pgfpathlineto{\pgfqpoint{2.820480in}{3.387575in}}%
\pgfpathlineto{\pgfqpoint{2.820765in}{3.394331in}}%
\pgfpathlineto{\pgfqpoint{2.821807in}{3.484970in}}%
\pgfpathlineto{\pgfqpoint{2.822565in}{3.458639in}}%
\pgfpathlineto{\pgfqpoint{2.823039in}{3.427516in}}%
\pgfpathlineto{\pgfqpoint{2.823797in}{3.411155in}}%
\pgfpathlineto{\pgfqpoint{2.823987in}{3.432591in}}%
\pgfpathlineto{\pgfqpoint{2.824840in}{3.518257in}}%
\pgfpathlineto{\pgfqpoint{2.825313in}{3.498026in}}%
\pgfpathlineto{\pgfqpoint{2.826545in}{3.453241in}}%
\pgfpathlineto{\pgfqpoint{2.826924in}{3.472388in}}%
\pgfpathlineto{\pgfqpoint{2.827967in}{3.554982in}}%
\pgfpathlineto{\pgfqpoint{2.828535in}{3.508182in}}%
\pgfpathlineto{\pgfqpoint{2.829388in}{3.456875in}}%
\pgfpathlineto{\pgfqpoint{2.829862in}{3.475064in}}%
\pgfpathlineto{\pgfqpoint{2.830241in}{3.481923in}}%
\pgfpathlineto{\pgfqpoint{2.830810in}{3.530207in}}%
\pgfpathlineto{\pgfqpoint{2.831473in}{3.496353in}}%
\pgfpathlineto{\pgfqpoint{2.832516in}{3.452522in}}%
\pgfpathlineto{\pgfqpoint{2.832705in}{3.464032in}}%
\pgfpathlineto{\pgfqpoint{2.833747in}{3.521495in}}%
\pgfpathlineto{\pgfqpoint{2.834221in}{3.517980in}}%
\pgfpathlineto{\pgfqpoint{2.834600in}{3.487611in}}%
\pgfpathlineto{\pgfqpoint{2.835358in}{3.444156in}}%
\pgfpathlineto{\pgfqpoint{2.835832in}{3.473110in}}%
\pgfpathlineto{\pgfqpoint{2.837064in}{3.497527in}}%
\pgfpathlineto{\pgfqpoint{2.837348in}{3.480802in}}%
\pgfpathlineto{\pgfqpoint{2.838201in}{3.406496in}}%
\pgfpathlineto{\pgfqpoint{2.838865in}{3.432187in}}%
\pgfpathlineto{\pgfqpoint{2.840002in}{3.470074in}}%
\pgfpathlineto{\pgfqpoint{2.840381in}{3.458651in}}%
\pgfpathlineto{\pgfqpoint{2.841139in}{3.384559in}}%
\pgfpathlineto{\pgfqpoint{2.841802in}{3.430295in}}%
\pgfpathlineto{\pgfqpoint{2.842561in}{3.461498in}}%
\pgfpathlineto{\pgfqpoint{2.843129in}{3.458772in}}%
\pgfpathlineto{\pgfqpoint{2.843319in}{3.469548in}}%
\pgfpathlineto{\pgfqpoint{2.843698in}{3.436555in}}%
\pgfpathlineto{\pgfqpoint{2.844077in}{3.403285in}}%
\pgfpathlineto{\pgfqpoint{2.844740in}{3.423659in}}%
\pgfpathlineto{\pgfqpoint{2.846067in}{3.531777in}}%
\pgfpathlineto{\pgfqpoint{2.846541in}{3.531542in}}%
\pgfpathlineto{\pgfqpoint{2.846635in}{3.534071in}}%
\pgfpathlineto{\pgfqpoint{2.846920in}{3.521016in}}%
\pgfpathlineto{\pgfqpoint{2.847109in}{3.509033in}}%
\pgfpathlineto{\pgfqpoint{2.847867in}{3.529781in}}%
\pgfpathlineto{\pgfqpoint{2.848531in}{3.592469in}}%
\pgfpathlineto{\pgfqpoint{2.848910in}{3.545436in}}%
\pgfpathlineto{\pgfqpoint{2.850521in}{3.435099in}}%
\pgfpathlineto{\pgfqpoint{2.850615in}{3.439189in}}%
\pgfpathlineto{\pgfqpoint{2.851658in}{3.540705in}}%
\pgfpathlineto{\pgfqpoint{2.852132in}{3.503917in}}%
\pgfpathlineto{\pgfqpoint{2.853743in}{3.466284in}}%
\pgfpathlineto{\pgfqpoint{2.853837in}{3.469397in}}%
\pgfpathlineto{\pgfqpoint{2.854975in}{3.537624in}}%
\pgfpathlineto{\pgfqpoint{2.855259in}{3.505199in}}%
\pgfpathlineto{\pgfqpoint{2.856017in}{3.526172in}}%
\pgfpathlineto{\pgfqpoint{2.856680in}{3.453538in}}%
\pgfpathlineto{\pgfqpoint{2.857059in}{3.541286in}}%
\pgfpathlineto{\pgfqpoint{2.858007in}{3.947516in}}%
\pgfpathlineto{\pgfqpoint{2.858860in}{3.888202in}}%
\pgfpathlineto{\pgfqpoint{2.859334in}{3.815581in}}%
\pgfpathlineto{\pgfqpoint{2.860092in}{3.396951in}}%
\pgfpathlineto{\pgfqpoint{2.860850in}{3.628353in}}%
\pgfpathlineto{\pgfqpoint{2.860945in}{3.629579in}}%
\pgfpathlineto{\pgfqpoint{2.861229in}{3.622998in}}%
\pgfpathlineto{\pgfqpoint{2.861798in}{3.559648in}}%
\pgfpathlineto{\pgfqpoint{2.862461in}{3.606923in}}%
\pgfpathlineto{\pgfqpoint{2.863977in}{3.694615in}}%
\pgfpathlineto{\pgfqpoint{2.864546in}{3.642450in}}%
\pgfpathlineto{\pgfqpoint{2.864830in}{3.627421in}}%
\pgfpathlineto{\pgfqpoint{2.865114in}{3.604556in}}%
\pgfpathlineto{\pgfqpoint{2.865588in}{3.661432in}}%
\pgfpathlineto{\pgfqpoint{2.865967in}{3.672331in}}%
\pgfpathlineto{\pgfqpoint{2.866915in}{3.710213in}}%
\pgfpathlineto{\pgfqpoint{2.867199in}{3.693321in}}%
\pgfpathlineto{\pgfqpoint{2.867863in}{3.630597in}}%
\pgfpathlineto{\pgfqpoint{2.868336in}{3.681511in}}%
\pgfpathlineto{\pgfqpoint{2.869094in}{3.727174in}}%
\pgfpathlineto{\pgfqpoint{2.869853in}{3.699522in}}%
\pgfpathlineto{\pgfqpoint{2.871085in}{3.586980in}}%
\pgfpathlineto{\pgfqpoint{2.871748in}{3.649681in}}%
\pgfpathlineto{\pgfqpoint{2.875159in}{3.831021in}}%
\pgfpathlineto{\pgfqpoint{2.875444in}{3.824527in}}%
\pgfpathlineto{\pgfqpoint{2.877149in}{3.772075in}}%
\pgfpathlineto{\pgfqpoint{2.877244in}{3.772351in}}%
\pgfpathlineto{\pgfqpoint{2.878287in}{3.869863in}}%
\pgfpathlineto{\pgfqpoint{2.879045in}{3.807531in}}%
\pgfpathlineto{\pgfqpoint{2.879519in}{3.832618in}}%
\pgfpathlineto{\pgfqpoint{2.879898in}{3.809773in}}%
\pgfpathlineto{\pgfqpoint{2.880371in}{3.790274in}}%
\pgfpathlineto{\pgfqpoint{2.880750in}{3.822131in}}%
\pgfpathlineto{\pgfqpoint{2.881319in}{3.868453in}}%
\pgfpathlineto{\pgfqpoint{2.881793in}{3.818289in}}%
\pgfpathlineto{\pgfqpoint{2.883499in}{3.733912in}}%
\pgfpathlineto{\pgfqpoint{2.884446in}{3.812791in}}%
\pgfpathlineto{\pgfqpoint{2.884825in}{3.772780in}}%
\pgfpathlineto{\pgfqpoint{2.885394in}{3.725443in}}%
\pgfpathlineto{\pgfqpoint{2.886247in}{3.736470in}}%
\pgfpathlineto{\pgfqpoint{2.886721in}{3.735083in}}%
\pgfpathlineto{\pgfqpoint{2.887005in}{3.755354in}}%
\pgfpathlineto{\pgfqpoint{2.887573in}{3.806861in}}%
\pgfpathlineto{\pgfqpoint{2.888047in}{3.755820in}}%
\pgfpathlineto{\pgfqpoint{2.888616in}{3.710014in}}%
\pgfpathlineto{\pgfqpoint{2.889090in}{3.755779in}}%
\pgfpathlineto{\pgfqpoint{2.889184in}{3.755476in}}%
\pgfpathlineto{\pgfqpoint{2.889564in}{3.724853in}}%
\pgfpathlineto{\pgfqpoint{2.890132in}{3.761028in}}%
\pgfpathlineto{\pgfqpoint{2.890606in}{3.809117in}}%
\pgfpathlineto{\pgfqpoint{2.891174in}{3.761563in}}%
\pgfpathlineto{\pgfqpoint{2.891269in}{3.761743in}}%
\pgfpathlineto{\pgfqpoint{2.891648in}{3.740504in}}%
\pgfpathlineto{\pgfqpoint{2.891933in}{3.768023in}}%
\pgfpathlineto{\pgfqpoint{2.892785in}{3.809559in}}%
\pgfpathlineto{\pgfqpoint{2.893259in}{3.801624in}}%
\pgfpathlineto{\pgfqpoint{2.893733in}{3.835862in}}%
\pgfpathlineto{\pgfqpoint{2.894112in}{3.793798in}}%
\pgfpathlineto{\pgfqpoint{2.894586in}{3.730712in}}%
\pgfpathlineto{\pgfqpoint{2.895344in}{3.757075in}}%
\pgfpathlineto{\pgfqpoint{2.895439in}{3.758572in}}%
\pgfpathlineto{\pgfqpoint{2.895628in}{3.747147in}}%
\pgfpathlineto{\pgfqpoint{2.896387in}{3.718451in}}%
\pgfpathlineto{\pgfqpoint{2.896671in}{3.745829in}}%
\pgfpathlineto{\pgfqpoint{2.896860in}{3.774597in}}%
\pgfpathlineto{\pgfqpoint{2.897429in}{3.708413in}}%
\pgfpathlineto{\pgfqpoint{2.897808in}{3.673299in}}%
\pgfpathlineto{\pgfqpoint{2.898282in}{3.729543in}}%
\pgfpathlineto{\pgfqpoint{2.898661in}{3.757859in}}%
\pgfpathlineto{\pgfqpoint{2.898850in}{3.779349in}}%
\pgfpathlineto{\pgfqpoint{2.899324in}{3.740353in}}%
\pgfpathlineto{\pgfqpoint{2.899798in}{3.765086in}}%
\pgfpathlineto{\pgfqpoint{2.900177in}{3.744636in}}%
\pgfpathlineto{\pgfqpoint{2.900840in}{3.696171in}}%
\pgfpathlineto{\pgfqpoint{2.901219in}{3.744551in}}%
\pgfpathlineto{\pgfqpoint{2.901883in}{3.802551in}}%
\pgfpathlineto{\pgfqpoint{2.902357in}{3.755994in}}%
\pgfpathlineto{\pgfqpoint{2.902830in}{3.723852in}}%
\pgfpathlineto{\pgfqpoint{2.903115in}{3.761235in}}%
\pgfpathlineto{\pgfqpoint{2.904726in}{4.232455in}}%
\pgfpathlineto{\pgfqpoint{2.905294in}{4.164574in}}%
\pgfpathlineto{\pgfqpoint{2.906242in}{3.662794in}}%
\pgfpathlineto{\pgfqpoint{2.907569in}{3.831666in}}%
\pgfpathlineto{\pgfqpoint{2.908043in}{3.890909in}}%
\pgfpathlineto{\pgfqpoint{2.908706in}{3.843781in}}%
\pgfpathlineto{\pgfqpoint{2.909369in}{3.848462in}}%
\pgfpathlineto{\pgfqpoint{2.910222in}{3.812576in}}%
\pgfpathlineto{\pgfqpoint{2.910601in}{3.852953in}}%
\pgfpathlineto{\pgfqpoint{2.911264in}{3.911762in}}%
\pgfpathlineto{\pgfqpoint{2.911738in}{3.875932in}}%
\pgfpathlineto{\pgfqpoint{2.912117in}{3.853080in}}%
\pgfpathlineto{\pgfqpoint{2.912591in}{3.892336in}}%
\pgfpathlineto{\pgfqpoint{2.912686in}{3.893593in}}%
\pgfpathlineto{\pgfqpoint{2.912781in}{3.888537in}}%
\pgfpathlineto{\pgfqpoint{2.913065in}{3.871773in}}%
\pgfpathlineto{\pgfqpoint{2.913728in}{3.898639in}}%
\pgfpathlineto{\pgfqpoint{2.914202in}{3.961395in}}%
\pgfpathlineto{\pgfqpoint{2.914866in}{3.904950in}}%
\pgfpathlineto{\pgfqpoint{2.915150in}{3.894299in}}%
\pgfpathlineto{\pgfqpoint{2.915624in}{3.916368in}}%
\pgfpathlineto{\pgfqpoint{2.917424in}{3.993613in}}%
\pgfpathlineto{\pgfqpoint{2.918372in}{3.907014in}}%
\pgfpathlineto{\pgfqpoint{2.918940in}{3.958883in}}%
\pgfpathlineto{\pgfqpoint{2.920457in}{4.025859in}}%
\pgfpathlineto{\pgfqpoint{2.919793in}{3.950427in}}%
\pgfpathlineto{\pgfqpoint{2.920646in}{4.014316in}}%
\pgfpathlineto{\pgfqpoint{2.921309in}{3.936999in}}%
\pgfpathlineto{\pgfqpoint{2.921973in}{3.981061in}}%
\pgfpathlineto{\pgfqpoint{2.922257in}{4.012329in}}%
\pgfpathlineto{\pgfqpoint{2.923110in}{3.992031in}}%
\pgfpathlineto{\pgfqpoint{2.923205in}{3.991543in}}%
\pgfpathlineto{\pgfqpoint{2.923300in}{3.994541in}}%
\pgfpathlineto{\pgfqpoint{2.923773in}{4.046852in}}%
\pgfpathlineto{\pgfqpoint{2.924152in}{3.976874in}}%
\pgfpathlineto{\pgfqpoint{2.924437in}{3.947597in}}%
\pgfpathlineto{\pgfqpoint{2.925100in}{4.008791in}}%
\pgfpathlineto{\pgfqpoint{2.925384in}{4.035321in}}%
\pgfpathlineto{\pgfqpoint{2.925953in}{3.985608in}}%
\pgfpathlineto{\pgfqpoint{2.926048in}{3.985645in}}%
\pgfpathlineto{\pgfqpoint{2.926806in}{4.005311in}}%
\pgfpathlineto{\pgfqpoint{2.926995in}{3.995103in}}%
\pgfpathlineto{\pgfqpoint{2.927943in}{3.920102in}}%
\pgfpathlineto{\pgfqpoint{2.928227in}{3.953387in}}%
\pgfpathlineto{\pgfqpoint{2.928512in}{3.978119in}}%
\pgfpathlineto{\pgfqpoint{2.929080in}{3.921171in}}%
\pgfpathlineto{\pgfqpoint{2.930691in}{3.833323in}}%
\pgfpathlineto{\pgfqpoint{2.931260in}{3.876004in}}%
\pgfpathlineto{\pgfqpoint{2.931734in}{3.911022in}}%
\pgfpathlineto{\pgfqpoint{2.932113in}{3.864328in}}%
\pgfpathlineto{\pgfqpoint{2.932681in}{3.844546in}}%
\pgfpathlineto{\pgfqpoint{2.933060in}{3.880501in}}%
\pgfpathlineto{\pgfqpoint{2.933155in}{3.885385in}}%
\pgfpathlineto{\pgfqpoint{2.933439in}{3.844621in}}%
\pgfpathlineto{\pgfqpoint{2.933724in}{3.819921in}}%
\pgfpathlineto{\pgfqpoint{2.934482in}{3.852570in}}%
\pgfpathlineto{\pgfqpoint{2.934671in}{3.862999in}}%
\pgfpathlineto{\pgfqpoint{2.935145in}{3.839479in}}%
\pgfpathlineto{\pgfqpoint{2.937040in}{3.648312in}}%
\pgfpathlineto{\pgfqpoint{2.937135in}{3.651892in}}%
\pgfpathlineto{\pgfqpoint{2.939504in}{3.888782in}}%
\pgfpathlineto{\pgfqpoint{2.940641in}{3.855904in}}%
\pgfpathlineto{\pgfqpoint{2.940736in}{3.858176in}}%
\pgfpathlineto{\pgfqpoint{2.941115in}{3.845494in}}%
\pgfpathlineto{\pgfqpoint{2.941684in}{3.728601in}}%
\pgfpathlineto{\pgfqpoint{2.941968in}{3.695042in}}%
\pgfpathlineto{\pgfqpoint{2.942916in}{3.696581in}}%
\pgfpathlineto{\pgfqpoint{2.943390in}{3.673976in}}%
\pgfpathlineto{\pgfqpoint{2.943769in}{3.700664in}}%
\pgfpathlineto{\pgfqpoint{2.944242in}{3.726741in}}%
\pgfpathlineto{\pgfqpoint{2.944527in}{3.692762in}}%
\pgfpathlineto{\pgfqpoint{2.945190in}{3.630849in}}%
\pgfpathlineto{\pgfqpoint{2.945664in}{3.670025in}}%
\pgfpathlineto{\pgfqpoint{2.945853in}{3.677225in}}%
\pgfpathlineto{\pgfqpoint{2.946232in}{3.648097in}}%
\pgfpathlineto{\pgfqpoint{2.946422in}{3.631989in}}%
\pgfpathlineto{\pgfqpoint{2.946896in}{3.680418in}}%
\pgfpathlineto{\pgfqpoint{2.947180in}{3.698280in}}%
\pgfpathlineto{\pgfqpoint{2.947654in}{3.661624in}}%
\pgfpathlineto{\pgfqpoint{2.948981in}{3.585079in}}%
\pgfpathlineto{\pgfqpoint{2.949170in}{3.600505in}}%
\pgfpathlineto{\pgfqpoint{2.950118in}{4.026345in}}%
\pgfpathlineto{\pgfqpoint{2.950402in}{4.105439in}}%
\pgfpathlineto{\pgfqpoint{2.951065in}{3.982158in}}%
\pgfpathlineto{\pgfqpoint{2.952013in}{3.662235in}}%
\pgfpathlineto{\pgfqpoint{2.952392in}{3.526820in}}%
\pgfpathlineto{\pgfqpoint{2.953055in}{3.665033in}}%
\pgfpathlineto{\pgfqpoint{2.953529in}{3.693643in}}%
\pgfpathlineto{\pgfqpoint{2.954003in}{3.653145in}}%
\pgfpathlineto{\pgfqpoint{2.954477in}{3.607527in}}%
\pgfpathlineto{\pgfqpoint{2.954856in}{3.668967in}}%
\pgfpathlineto{\pgfqpoint{2.955425in}{3.700739in}}%
\pgfpathlineto{\pgfqpoint{2.955804in}{3.667525in}}%
\pgfpathlineto{\pgfqpoint{2.955993in}{3.646867in}}%
\pgfpathlineto{\pgfqpoint{2.956562in}{3.694069in}}%
\pgfpathlineto{\pgfqpoint{2.956751in}{3.687563in}}%
\pgfpathlineto{\pgfqpoint{2.957130in}{3.650068in}}%
\pgfpathlineto{\pgfqpoint{2.957415in}{3.605949in}}%
\pgfpathlineto{\pgfqpoint{2.958078in}{3.674355in}}%
\pgfpathlineto{\pgfqpoint{2.958362in}{3.696137in}}%
\pgfpathlineto{\pgfqpoint{2.958836in}{3.658228in}}%
\pgfpathlineto{\pgfqpoint{2.959215in}{3.610316in}}%
\pgfpathlineto{\pgfqpoint{2.959689in}{3.661523in}}%
\pgfpathlineto{\pgfqpoint{2.959973in}{3.647588in}}%
\pgfpathlineto{\pgfqpoint{2.960637in}{3.568359in}}%
\pgfpathlineto{\pgfqpoint{2.961110in}{3.635392in}}%
\pgfpathlineto{\pgfqpoint{2.961679in}{3.679702in}}%
\pgfpathlineto{\pgfqpoint{2.962058in}{3.621398in}}%
\pgfpathlineto{\pgfqpoint{2.962342in}{3.588671in}}%
\pgfpathlineto{\pgfqpoint{2.963006in}{3.638270in}}%
\pgfpathlineto{\pgfqpoint{2.963385in}{3.646264in}}%
\pgfpathlineto{\pgfqpoint{2.963574in}{3.632542in}}%
\pgfpathlineto{\pgfqpoint{2.963859in}{3.603145in}}%
\pgfpathlineto{\pgfqpoint{2.964238in}{3.666804in}}%
\pgfpathlineto{\pgfqpoint{2.964522in}{3.708054in}}%
\pgfpathlineto{\pgfqpoint{2.965185in}{3.652632in}}%
\pgfpathlineto{\pgfqpoint{2.965470in}{3.642473in}}%
\pgfpathlineto{\pgfqpoint{2.965943in}{3.670516in}}%
\pgfpathlineto{\pgfqpoint{2.966133in}{3.676407in}}%
\pgfpathlineto{\pgfqpoint{2.966417in}{3.655206in}}%
\pgfpathlineto{\pgfqpoint{2.966891in}{3.613468in}}%
\pgfpathlineto{\pgfqpoint{2.967365in}{3.656628in}}%
\pgfpathlineto{\pgfqpoint{2.967649in}{3.690743in}}%
\pgfpathlineto{\pgfqpoint{2.968218in}{3.632725in}}%
\pgfpathlineto{\pgfqpoint{2.968692in}{3.582953in}}%
\pgfpathlineto{\pgfqpoint{2.969260in}{3.633492in}}%
\pgfpathlineto{\pgfqpoint{2.969355in}{3.638708in}}%
\pgfpathlineto{\pgfqpoint{2.969639in}{3.614705in}}%
\pgfpathlineto{\pgfqpoint{2.969923in}{3.574981in}}%
\pgfpathlineto{\pgfqpoint{2.970587in}{3.634035in}}%
\pgfpathlineto{\pgfqpoint{2.970871in}{3.682473in}}%
\pgfpathlineto{\pgfqpoint{2.971440in}{3.579569in}}%
\pgfpathlineto{\pgfqpoint{2.971819in}{3.549208in}}%
\pgfpathlineto{\pgfqpoint{2.972198in}{3.596312in}}%
\pgfpathlineto{\pgfqpoint{2.972482in}{3.633728in}}%
\pgfpathlineto{\pgfqpoint{2.973051in}{3.565817in}}%
\pgfpathlineto{\pgfqpoint{2.973335in}{3.560473in}}%
\pgfpathlineto{\pgfqpoint{2.973619in}{3.581514in}}%
\pgfpathlineto{\pgfqpoint{2.973998in}{3.606109in}}%
\pgfpathlineto{\pgfqpoint{2.974472in}{3.572484in}}%
\pgfpathlineto{\pgfqpoint{2.974946in}{3.489892in}}%
\pgfpathlineto{\pgfqpoint{2.975609in}{3.549776in}}%
\pgfpathlineto{\pgfqpoint{2.975894in}{3.536524in}}%
\pgfpathlineto{\pgfqpoint{2.978073in}{3.386712in}}%
\pgfpathlineto{\pgfqpoint{2.978263in}{3.403979in}}%
\pgfpathlineto{\pgfqpoint{2.978737in}{3.482187in}}%
\pgfpathlineto{\pgfqpoint{2.979305in}{3.420489in}}%
\pgfpathlineto{\pgfqpoint{2.979589in}{3.390400in}}%
\pgfpathlineto{\pgfqpoint{2.980063in}{3.466341in}}%
\pgfpathlineto{\pgfqpoint{2.980158in}{3.470484in}}%
\pgfpathlineto{\pgfqpoint{2.980537in}{3.440144in}}%
\pgfpathlineto{\pgfqpoint{2.981106in}{3.364760in}}%
\pgfpathlineto{\pgfqpoint{2.981674in}{3.415804in}}%
\pgfpathlineto{\pgfqpoint{2.981959in}{3.462205in}}%
\pgfpathlineto{\pgfqpoint{2.982527in}{3.413476in}}%
\pgfpathlineto{\pgfqpoint{2.982717in}{3.416298in}}%
\pgfpathlineto{\pgfqpoint{2.982906in}{3.413003in}}%
\pgfpathlineto{\pgfqpoint{2.983096in}{3.421925in}}%
\pgfpathlineto{\pgfqpoint{2.983570in}{3.475300in}}%
\pgfpathlineto{\pgfqpoint{2.984138in}{3.415535in}}%
\pgfpathlineto{\pgfqpoint{2.984517in}{3.451393in}}%
\pgfpathlineto{\pgfqpoint{2.985086in}{3.530816in}}%
\pgfpathlineto{\pgfqpoint{2.985654in}{3.487216in}}%
\pgfpathlineto{\pgfqpoint{2.985939in}{3.447372in}}%
\pgfpathlineto{\pgfqpoint{2.986412in}{3.519478in}}%
\pgfpathlineto{\pgfqpoint{2.986602in}{3.512077in}}%
\pgfpathlineto{\pgfqpoint{2.987360in}{3.409340in}}%
\pgfpathlineto{\pgfqpoint{2.988118in}{3.481406in}}%
\pgfpathlineto{\pgfqpoint{2.988308in}{3.489122in}}%
\pgfpathlineto{\pgfqpoint{2.988592in}{3.453314in}}%
\pgfpathlineto{\pgfqpoint{2.988971in}{3.420829in}}%
\pgfpathlineto{\pgfqpoint{2.989634in}{3.461156in}}%
\pgfpathlineto{\pgfqpoint{2.989824in}{3.467838in}}%
\pgfpathlineto{\pgfqpoint{2.990203in}{3.439959in}}%
\pgfpathlineto{\pgfqpoint{2.990393in}{3.430376in}}%
\pgfpathlineto{\pgfqpoint{2.990961in}{3.462013in}}%
\pgfpathlineto{\pgfqpoint{2.991435in}{3.509636in}}%
\pgfpathlineto{\pgfqpoint{2.991909in}{3.457183in}}%
\pgfpathlineto{\pgfqpoint{2.992288in}{3.418592in}}%
\pgfpathlineto{\pgfqpoint{2.992762in}{3.485138in}}%
\pgfpathlineto{\pgfqpoint{2.992856in}{3.491865in}}%
\pgfpathlineto{\pgfqpoint{2.993235in}{3.447779in}}%
\pgfpathlineto{\pgfqpoint{2.993804in}{3.407368in}}%
\pgfpathlineto{\pgfqpoint{2.994278in}{3.444863in}}%
\pgfpathlineto{\pgfqpoint{2.995225in}{3.594857in}}%
\pgfpathlineto{\pgfqpoint{2.995984in}{3.841563in}}%
\pgfpathlineto{\pgfqpoint{2.996647in}{3.760593in}}%
\pgfpathlineto{\pgfqpoint{2.997595in}{3.416681in}}%
\pgfpathlineto{\pgfqpoint{2.997974in}{3.302616in}}%
\pgfpathlineto{\pgfqpoint{2.998637in}{3.400862in}}%
\pgfpathlineto{\pgfqpoint{2.999206in}{3.473422in}}%
\pgfpathlineto{\pgfqpoint{2.999679in}{3.402827in}}%
\pgfpathlineto{\pgfqpoint{2.999869in}{3.388154in}}%
\pgfpathlineto{\pgfqpoint{3.000532in}{3.420582in}}%
\pgfpathlineto{\pgfqpoint{3.000627in}{3.424580in}}%
\pgfpathlineto{\pgfqpoint{3.000911in}{3.404165in}}%
\pgfpathlineto{\pgfqpoint{3.001669in}{3.260276in}}%
\pgfpathlineto{\pgfqpoint{3.002428in}{3.331381in}}%
\pgfpathlineto{\pgfqpoint{3.002901in}{3.305413in}}%
\pgfpathlineto{\pgfqpoint{3.003186in}{3.330791in}}%
\pgfpathlineto{\pgfqpoint{3.003944in}{3.483440in}}%
\pgfpathlineto{\pgfqpoint{3.004607in}{3.407234in}}%
\pgfpathlineto{\pgfqpoint{3.004702in}{3.403215in}}%
\pgfpathlineto{\pgfqpoint{3.004986in}{3.425347in}}%
\pgfpathlineto{\pgfqpoint{3.005555in}{3.507368in}}%
\pgfpathlineto{\pgfqpoint{3.006029in}{3.431167in}}%
\pgfpathlineto{\pgfqpoint{3.006313in}{3.406960in}}%
\pgfpathlineto{\pgfqpoint{3.006787in}{3.462743in}}%
\pgfpathlineto{\pgfqpoint{3.007166in}{3.493177in}}%
\pgfpathlineto{\pgfqpoint{3.007545in}{3.437412in}}%
\pgfpathlineto{\pgfqpoint{3.007829in}{3.413908in}}%
\pgfpathlineto{\pgfqpoint{3.008398in}{3.476461in}}%
\pgfpathlineto{\pgfqpoint{3.008777in}{3.529854in}}%
\pgfpathlineto{\pgfqpoint{3.009345in}{3.443442in}}%
\pgfpathlineto{\pgfqpoint{3.009440in}{3.438907in}}%
\pgfpathlineto{\pgfqpoint{3.009819in}{3.470373in}}%
\pgfpathlineto{\pgfqpoint{3.010293in}{3.525012in}}%
\pgfpathlineto{\pgfqpoint{3.010672in}{3.450049in}}%
\pgfpathlineto{\pgfqpoint{3.010862in}{3.417773in}}%
\pgfpathlineto{\pgfqpoint{3.011525in}{3.498333in}}%
\pgfpathlineto{\pgfqpoint{3.011809in}{3.525949in}}%
\pgfpathlineto{\pgfqpoint{3.012283in}{3.471371in}}%
\pgfpathlineto{\pgfqpoint{3.012567in}{3.444207in}}%
\pgfpathlineto{\pgfqpoint{3.013136in}{3.499125in}}%
\pgfpathlineto{\pgfqpoint{3.013325in}{3.514859in}}%
\pgfpathlineto{\pgfqpoint{3.013704in}{3.448060in}}%
\pgfpathlineto{\pgfqpoint{3.013989in}{3.408494in}}%
\pgfpathlineto{\pgfqpoint{3.014557in}{3.466043in}}%
\pgfpathlineto{\pgfqpoint{3.015031in}{3.532017in}}%
\pgfpathlineto{\pgfqpoint{3.015505in}{3.443269in}}%
\pgfpathlineto{\pgfqpoint{3.015695in}{3.426947in}}%
\pgfpathlineto{\pgfqpoint{3.016168in}{3.488799in}}%
\pgfpathlineto{\pgfqpoint{3.016453in}{3.506591in}}%
\pgfpathlineto{\pgfqpoint{3.016926in}{3.458403in}}%
\pgfpathlineto{\pgfqpoint{3.017306in}{3.434837in}}%
\pgfpathlineto{\pgfqpoint{3.017685in}{3.474287in}}%
\pgfpathlineto{\pgfqpoint{3.018064in}{3.519148in}}%
\pgfpathlineto{\pgfqpoint{3.018632in}{3.458699in}}%
\pgfpathlineto{\pgfqpoint{3.019011in}{3.417991in}}%
\pgfpathlineto{\pgfqpoint{3.019485in}{3.485241in}}%
\pgfpathlineto{\pgfqpoint{3.019580in}{3.491593in}}%
\pgfpathlineto{\pgfqpoint{3.019959in}{3.450621in}}%
\pgfpathlineto{\pgfqpoint{3.020433in}{3.398330in}}%
\pgfpathlineto{\pgfqpoint{3.021001in}{3.455620in}}%
\pgfpathlineto{\pgfqpoint{3.021380in}{3.466981in}}%
\pgfpathlineto{\pgfqpoint{3.021665in}{3.439712in}}%
\pgfpathlineto{\pgfqpoint{3.022138in}{3.366004in}}%
\pgfpathlineto{\pgfqpoint{3.022707in}{3.442180in}}%
\pgfpathlineto{\pgfqpoint{3.022802in}{3.442044in}}%
\pgfpathlineto{\pgfqpoint{3.023655in}{3.347825in}}%
\pgfpathlineto{\pgfqpoint{3.024129in}{3.422558in}}%
\pgfpathlineto{\pgfqpoint{3.024413in}{3.452290in}}%
\pgfpathlineto{\pgfqpoint{3.024792in}{3.369779in}}%
\pgfpathlineto{\pgfqpoint{3.025171in}{3.325882in}}%
\pgfpathlineto{\pgfqpoint{3.025740in}{3.401068in}}%
\pgfpathlineto{\pgfqpoint{3.025929in}{3.420008in}}%
\pgfpathlineto{\pgfqpoint{3.026498in}{3.354910in}}%
\pgfpathlineto{\pgfqpoint{3.026592in}{3.350743in}}%
\pgfpathlineto{\pgfqpoint{3.026971in}{3.372838in}}%
\pgfpathlineto{\pgfqpoint{3.027540in}{3.463453in}}%
\pgfpathlineto{\pgfqpoint{3.028014in}{3.386345in}}%
\pgfpathlineto{\pgfqpoint{3.028203in}{3.364896in}}%
\pgfpathlineto{\pgfqpoint{3.028677in}{3.430735in}}%
\pgfpathlineto{\pgfqpoint{3.029056in}{3.492849in}}%
\pgfpathlineto{\pgfqpoint{3.029720in}{3.424760in}}%
\pgfpathlineto{\pgfqpoint{3.029814in}{3.418964in}}%
\pgfpathlineto{\pgfqpoint{3.030193in}{3.457635in}}%
\pgfpathlineto{\pgfqpoint{3.030667in}{3.517309in}}%
\pgfpathlineto{\pgfqpoint{3.031141in}{3.436834in}}%
\pgfpathlineto{\pgfqpoint{3.031425in}{3.410566in}}%
\pgfpathlineto{\pgfqpoint{3.032089in}{3.458438in}}%
\pgfpathlineto{\pgfqpoint{3.032183in}{3.463638in}}%
\pgfpathlineto{\pgfqpoint{3.032468in}{3.421444in}}%
\pgfpathlineto{\pgfqpoint{3.033036in}{3.333351in}}%
\pgfpathlineto{\pgfqpoint{3.033605in}{3.399339in}}%
\pgfpathlineto{\pgfqpoint{3.033794in}{3.412684in}}%
\pgfpathlineto{\pgfqpoint{3.034174in}{3.351634in}}%
\pgfpathlineto{\pgfqpoint{3.034553in}{3.291144in}}%
\pgfpathlineto{\pgfqpoint{3.035121in}{3.365801in}}%
\pgfpathlineto{\pgfqpoint{3.035405in}{3.386491in}}%
\pgfpathlineto{\pgfqpoint{3.035879in}{3.339115in}}%
\pgfpathlineto{\pgfqpoint{3.036164in}{3.312365in}}%
\pgfpathlineto{\pgfqpoint{3.036637in}{3.372508in}}%
\pgfpathlineto{\pgfqpoint{3.036922in}{3.394469in}}%
\pgfpathlineto{\pgfqpoint{3.037396in}{3.340199in}}%
\pgfpathlineto{\pgfqpoint{3.037680in}{3.303736in}}%
\pgfpathlineto{\pgfqpoint{3.038248in}{3.361757in}}%
\pgfpathlineto{\pgfqpoint{3.038627in}{3.391217in}}%
\pgfpathlineto{\pgfqpoint{3.039007in}{3.352510in}}%
\pgfpathlineto{\pgfqpoint{3.039480in}{3.277458in}}%
\pgfpathlineto{\pgfqpoint{3.039859in}{3.362980in}}%
\pgfpathlineto{\pgfqpoint{3.041565in}{3.724713in}}%
\pgfpathlineto{\pgfqpoint{3.041849in}{3.719214in}}%
\pgfpathlineto{\pgfqpoint{3.042228in}{3.615113in}}%
\pgfpathlineto{\pgfqpoint{3.042987in}{3.223066in}}%
\pgfpathlineto{\pgfqpoint{3.043745in}{3.362484in}}%
\pgfpathlineto{\pgfqpoint{3.044124in}{3.344802in}}%
\pgfpathlineto{\pgfqpoint{3.044408in}{3.373616in}}%
\pgfpathlineto{\pgfqpoint{3.044882in}{3.444737in}}%
\pgfpathlineto{\pgfqpoint{3.045450in}{3.361887in}}%
\pgfpathlineto{\pgfqpoint{3.045640in}{3.351296in}}%
\pgfpathlineto{\pgfqpoint{3.046019in}{3.394988in}}%
\pgfpathlineto{\pgfqpoint{3.046398in}{3.456167in}}%
\pgfpathlineto{\pgfqpoint{3.046967in}{3.386086in}}%
\pgfpathlineto{\pgfqpoint{3.047251in}{3.364993in}}%
\pgfpathlineto{\pgfqpoint{3.047630in}{3.423395in}}%
\pgfpathlineto{\pgfqpoint{3.048009in}{3.478234in}}%
\pgfpathlineto{\pgfqpoint{3.048578in}{3.384439in}}%
\pgfpathlineto{\pgfqpoint{3.048767in}{3.362310in}}%
\pgfpathlineto{\pgfqpoint{3.049241in}{3.432578in}}%
\pgfpathlineto{\pgfqpoint{3.049431in}{3.449302in}}%
\pgfpathlineto{\pgfqpoint{3.049904in}{3.381020in}}%
\pgfpathlineto{\pgfqpoint{3.050094in}{3.367052in}}%
\pgfpathlineto{\pgfqpoint{3.050662in}{3.425009in}}%
\pgfpathlineto{\pgfqpoint{3.051136in}{3.509546in}}%
\pgfpathlineto{\pgfqpoint{3.051610in}{3.397199in}}%
\pgfpathlineto{\pgfqpoint{3.051800in}{3.381605in}}%
\pgfpathlineto{\pgfqpoint{3.052273in}{3.439400in}}%
\pgfpathlineto{\pgfqpoint{3.052653in}{3.494580in}}%
\pgfpathlineto{\pgfqpoint{3.053126in}{3.417484in}}%
\pgfpathlineto{\pgfqpoint{3.053411in}{3.391257in}}%
\pgfpathlineto{\pgfqpoint{3.053790in}{3.461506in}}%
\pgfpathlineto{\pgfqpoint{3.054169in}{3.513506in}}%
\pgfpathlineto{\pgfqpoint{3.054832in}{3.448818in}}%
\pgfpathlineto{\pgfqpoint{3.055116in}{3.408818in}}%
\pgfpathlineto{\pgfqpoint{3.055590in}{3.519047in}}%
\pgfpathlineto{\pgfqpoint{3.055780in}{3.537621in}}%
\pgfpathlineto{\pgfqpoint{3.056254in}{3.466978in}}%
\pgfpathlineto{\pgfqpoint{3.056633in}{3.437722in}}%
\pgfpathlineto{\pgfqpoint{3.057106in}{3.500619in}}%
\pgfpathlineto{\pgfqpoint{3.057391in}{3.543394in}}%
\pgfpathlineto{\pgfqpoint{3.057959in}{3.435607in}}%
\pgfpathlineto{\pgfqpoint{3.058244in}{3.405331in}}%
\pgfpathlineto{\pgfqpoint{3.058717in}{3.499786in}}%
\pgfpathlineto{\pgfqpoint{3.058812in}{3.505407in}}%
\pgfpathlineto{\pgfqpoint{3.059381in}{3.474305in}}%
\pgfpathlineto{\pgfqpoint{3.059760in}{3.430053in}}%
\pgfpathlineto{\pgfqpoint{3.060234in}{3.501896in}}%
\pgfpathlineto{\pgfqpoint{3.060518in}{3.528418in}}%
\pgfpathlineto{\pgfqpoint{3.060897in}{3.465201in}}%
\pgfpathlineto{\pgfqpoint{3.061181in}{3.413500in}}%
\pgfpathlineto{\pgfqpoint{3.061845in}{3.485134in}}%
\pgfpathlineto{\pgfqpoint{3.062224in}{3.531989in}}%
\pgfpathlineto{\pgfqpoint{3.062603in}{3.441558in}}%
\pgfpathlineto{\pgfqpoint{3.062887in}{3.408263in}}%
\pgfpathlineto{\pgfqpoint{3.063361in}{3.485729in}}%
\pgfpathlineto{\pgfqpoint{3.063550in}{3.514174in}}%
\pgfpathlineto{\pgfqpoint{3.064119in}{3.443818in}}%
\pgfpathlineto{\pgfqpoint{3.064498in}{3.369208in}}%
\pgfpathlineto{\pgfqpoint{3.065067in}{3.477411in}}%
\pgfpathlineto{\pgfqpoint{3.065161in}{3.479343in}}%
\pgfpathlineto{\pgfqpoint{3.065351in}{3.470466in}}%
\pgfpathlineto{\pgfqpoint{3.066014in}{3.359911in}}%
\pgfpathlineto{\pgfqpoint{3.066678in}{3.446801in}}%
\pgfpathlineto{\pgfqpoint{3.066772in}{3.453333in}}%
\pgfpathlineto{\pgfqpoint{3.067151in}{3.403533in}}%
\pgfpathlineto{\pgfqpoint{3.067531in}{3.363351in}}%
\pgfpathlineto{\pgfqpoint{3.068099in}{3.425713in}}%
\pgfpathlineto{\pgfqpoint{3.068383in}{3.464160in}}%
\pgfpathlineto{\pgfqpoint{3.068952in}{3.388470in}}%
\pgfpathlineto{\pgfqpoint{3.069236in}{3.361179in}}%
\pgfpathlineto{\pgfqpoint{3.069710in}{3.434065in}}%
\pgfpathlineto{\pgfqpoint{3.069994in}{3.451147in}}%
\pgfpathlineto{\pgfqpoint{3.070373in}{3.411407in}}%
\pgfpathlineto{\pgfqpoint{3.070752in}{3.363053in}}%
\pgfpathlineto{\pgfqpoint{3.071226in}{3.432246in}}%
\pgfpathlineto{\pgfqpoint{3.071511in}{3.480579in}}%
\pgfpathlineto{\pgfqpoint{3.071984in}{3.390528in}}%
\pgfpathlineto{\pgfqpoint{3.072458in}{3.328732in}}%
\pgfpathlineto{\pgfqpoint{3.072932in}{3.406671in}}%
\pgfpathlineto{\pgfqpoint{3.073027in}{3.417004in}}%
\pgfpathlineto{\pgfqpoint{3.073501in}{3.353479in}}%
\pgfpathlineto{\pgfqpoint{3.073880in}{3.302474in}}%
\pgfpathlineto{\pgfqpoint{3.074354in}{3.387561in}}%
\pgfpathlineto{\pgfqpoint{3.075965in}{3.527507in}}%
\pgfpathlineto{\pgfqpoint{3.076344in}{3.560171in}}%
\pgfpathlineto{\pgfqpoint{3.076723in}{3.492617in}}%
\pgfpathlineto{\pgfqpoint{3.077102in}{3.457604in}}%
\pgfpathlineto{\pgfqpoint{3.077765in}{3.499870in}}%
\pgfpathlineto{\pgfqpoint{3.078239in}{3.427486in}}%
\pgfpathlineto{\pgfqpoint{3.078713in}{3.356734in}}%
\pgfpathlineto{\pgfqpoint{3.079186in}{3.442389in}}%
\pgfpathlineto{\pgfqpoint{3.079376in}{3.450817in}}%
\pgfpathlineto{\pgfqpoint{3.079755in}{3.402765in}}%
\pgfpathlineto{\pgfqpoint{3.080229in}{3.347763in}}%
\pgfpathlineto{\pgfqpoint{3.080703in}{3.430637in}}%
\pgfpathlineto{\pgfqpoint{3.080987in}{3.470430in}}%
\pgfpathlineto{\pgfqpoint{3.081461in}{3.371358in}}%
\pgfpathlineto{\pgfqpoint{3.081840in}{3.329990in}}%
\pgfpathlineto{\pgfqpoint{3.082314in}{3.415509in}}%
\pgfpathlineto{\pgfqpoint{3.082598in}{3.457889in}}%
\pgfpathlineto{\pgfqpoint{3.083072in}{3.359568in}}%
\pgfpathlineto{\pgfqpoint{3.083356in}{3.329244in}}%
\pgfpathlineto{\pgfqpoint{3.083830in}{3.406578in}}%
\pgfpathlineto{\pgfqpoint{3.084209in}{3.448361in}}%
\pgfpathlineto{\pgfqpoint{3.084588in}{3.377950in}}%
\pgfpathlineto{\pgfqpoint{3.084967in}{3.296357in}}%
\pgfpathlineto{\pgfqpoint{3.085346in}{3.460275in}}%
\pgfpathlineto{\pgfqpoint{3.086010in}{3.768686in}}%
\pgfpathlineto{\pgfqpoint{3.086673in}{3.680533in}}%
\pgfpathlineto{\pgfqpoint{3.087241in}{3.788237in}}%
\pgfpathlineto{\pgfqpoint{3.087526in}{3.700799in}}%
\pgfpathlineto{\pgfqpoint{3.088284in}{3.203654in}}%
\pgfpathlineto{\pgfqpoint{3.088947in}{3.464765in}}%
\pgfpathlineto{\pgfqpoint{3.089042in}{3.464994in}}%
\pgfpathlineto{\pgfqpoint{3.089611in}{3.375460in}}%
\pgfpathlineto{\pgfqpoint{3.090084in}{3.460479in}}%
\pgfpathlineto{\pgfqpoint{3.090369in}{3.502564in}}%
\pgfpathlineto{\pgfqpoint{3.090937in}{3.415180in}}%
\pgfpathlineto{\pgfqpoint{3.091127in}{3.389231in}}%
\pgfpathlineto{\pgfqpoint{3.091695in}{3.470509in}}%
\pgfpathlineto{\pgfqpoint{3.092169in}{3.517527in}}%
\pgfpathlineto{\pgfqpoint{3.092548in}{3.438353in}}%
\pgfpathlineto{\pgfqpoint{3.092738in}{3.411975in}}%
\pgfpathlineto{\pgfqpoint{3.093306in}{3.487841in}}%
\pgfpathlineto{\pgfqpoint{3.093591in}{3.498560in}}%
\pgfpathlineto{\pgfqpoint{3.093875in}{3.481252in}}%
\pgfpathlineto{\pgfqpoint{3.094349in}{3.397397in}}%
\pgfpathlineto{\pgfqpoint{3.094917in}{3.486163in}}%
\pgfpathlineto{\pgfqpoint{3.095107in}{3.507224in}}%
\pgfpathlineto{\pgfqpoint{3.095581in}{3.439543in}}%
\pgfpathlineto{\pgfqpoint{3.095865in}{3.398933in}}%
\pgfpathlineto{\pgfqpoint{3.096434in}{3.494978in}}%
\pgfpathlineto{\pgfqpoint{3.096718in}{3.525501in}}%
\pgfpathlineto{\pgfqpoint{3.097192in}{3.440526in}}%
\pgfpathlineto{\pgfqpoint{3.097381in}{3.413707in}}%
\pgfpathlineto{\pgfqpoint{3.097950in}{3.501158in}}%
\pgfpathlineto{\pgfqpoint{3.098329in}{3.561389in}}%
\pgfpathlineto{\pgfqpoint{3.098803in}{3.473495in}}%
\pgfpathlineto{\pgfqpoint{3.099087in}{3.448400in}}%
\pgfpathlineto{\pgfqpoint{3.099561in}{3.527788in}}%
\pgfpathlineto{\pgfqpoint{3.099845in}{3.571588in}}%
\pgfpathlineto{\pgfqpoint{3.100319in}{3.478124in}}%
\pgfpathlineto{\pgfqpoint{3.100603in}{3.440620in}}%
\pgfpathlineto{\pgfqpoint{3.101077in}{3.522962in}}%
\pgfpathlineto{\pgfqpoint{3.101456in}{3.579462in}}%
\pgfpathlineto{\pgfqpoint{3.101930in}{3.496811in}}%
\pgfpathlineto{\pgfqpoint{3.102214in}{3.466347in}}%
\pgfpathlineto{\pgfqpoint{3.102688in}{3.540041in}}%
\pgfpathlineto{\pgfqpoint{3.103067in}{3.596752in}}%
\pgfpathlineto{\pgfqpoint{3.103541in}{3.494199in}}%
\pgfpathlineto{\pgfqpoint{3.103730in}{3.480315in}}%
\pgfpathlineto{\pgfqpoint{3.104204in}{3.524717in}}%
\pgfpathlineto{\pgfqpoint{3.104583in}{3.584981in}}%
\pgfpathlineto{\pgfqpoint{3.105057in}{3.498512in}}%
\pgfpathlineto{\pgfqpoint{3.105436in}{3.446934in}}%
\pgfpathlineto{\pgfqpoint{3.105815in}{3.536658in}}%
\pgfpathlineto{\pgfqpoint{3.106099in}{3.576681in}}%
\pgfpathlineto{\pgfqpoint{3.106573in}{3.485364in}}%
\pgfpathlineto{\pgfqpoint{3.106858in}{3.442243in}}%
\pgfpathlineto{\pgfqpoint{3.107331in}{3.538042in}}%
\pgfpathlineto{\pgfqpoint{3.107710in}{3.583482in}}%
\pgfpathlineto{\pgfqpoint{3.108184in}{3.502664in}}%
\pgfpathlineto{\pgfqpoint{3.108469in}{3.448690in}}%
\pgfpathlineto{\pgfqpoint{3.109037in}{3.557789in}}%
\pgfpathlineto{\pgfqpoint{3.109227in}{3.565999in}}%
\pgfpathlineto{\pgfqpoint{3.109606in}{3.519229in}}%
\pgfpathlineto{\pgfqpoint{3.110080in}{3.438810in}}%
\pgfpathlineto{\pgfqpoint{3.110648in}{3.527711in}}%
\pgfpathlineto{\pgfqpoint{3.110838in}{3.535312in}}%
\pgfpathlineto{\pgfqpoint{3.111122in}{3.505930in}}%
\pgfpathlineto{\pgfqpoint{3.111691in}{3.399703in}}%
\pgfpathlineto{\pgfqpoint{3.112259in}{3.487568in}}%
\pgfpathlineto{\pgfqpoint{3.112354in}{3.490250in}}%
\pgfpathlineto{\pgfqpoint{3.112733in}{3.470575in}}%
\pgfpathlineto{\pgfqpoint{3.113302in}{3.360038in}}%
\pgfpathlineto{\pgfqpoint{3.113775in}{3.463280in}}%
\pgfpathlineto{\pgfqpoint{3.113965in}{3.478155in}}%
\pgfpathlineto{\pgfqpoint{3.114344in}{3.411857in}}%
\pgfpathlineto{\pgfqpoint{3.114723in}{3.358652in}}%
\pgfpathlineto{\pgfqpoint{3.115292in}{3.445616in}}%
\pgfpathlineto{\pgfqpoint{3.115576in}{3.488534in}}%
\pgfpathlineto{\pgfqpoint{3.116050in}{3.394714in}}%
\pgfpathlineto{\pgfqpoint{3.116429in}{3.339460in}}%
\pgfpathlineto{\pgfqpoint{3.116808in}{3.433556in}}%
\pgfpathlineto{\pgfqpoint{3.116997in}{3.477947in}}%
\pgfpathlineto{\pgfqpoint{3.117661in}{3.368462in}}%
\pgfpathlineto{\pgfqpoint{3.117945in}{3.336523in}}%
\pgfpathlineto{\pgfqpoint{3.118324in}{3.415863in}}%
\pgfpathlineto{\pgfqpoint{3.118703in}{3.478357in}}%
\pgfpathlineto{\pgfqpoint{3.119272in}{3.399935in}}%
\pgfpathlineto{\pgfqpoint{3.119556in}{3.357185in}}%
\pgfpathlineto{\pgfqpoint{3.120030in}{3.471144in}}%
\pgfpathlineto{\pgfqpoint{3.120314in}{3.505706in}}%
\pgfpathlineto{\pgfqpoint{3.120883in}{3.416283in}}%
\pgfpathlineto{\pgfqpoint{3.121072in}{3.398386in}}%
\pgfpathlineto{\pgfqpoint{3.121546in}{3.470807in}}%
\pgfpathlineto{\pgfqpoint{3.121830in}{3.500345in}}%
\pgfpathlineto{\pgfqpoint{3.122304in}{3.440074in}}%
\pgfpathlineto{\pgfqpoint{3.122778in}{3.397686in}}%
\pgfpathlineto{\pgfqpoint{3.123252in}{3.451183in}}%
\pgfpathlineto{\pgfqpoint{3.123347in}{3.454446in}}%
\pgfpathlineto{\pgfqpoint{3.123536in}{3.429794in}}%
\pgfpathlineto{\pgfqpoint{3.124294in}{3.308048in}}%
\pgfpathlineto{\pgfqpoint{3.124673in}{3.384057in}}%
\pgfpathlineto{\pgfqpoint{3.124958in}{3.427060in}}%
\pgfpathlineto{\pgfqpoint{3.125431in}{3.342175in}}%
\pgfpathlineto{\pgfqpoint{3.125716in}{3.289517in}}%
\pgfpathlineto{\pgfqpoint{3.126284in}{3.394413in}}%
\pgfpathlineto{\pgfqpoint{3.126569in}{3.418359in}}%
\pgfpathlineto{\pgfqpoint{3.127042in}{3.336726in}}%
\pgfpathlineto{\pgfqpoint{3.127421in}{3.290274in}}%
\pgfpathlineto{\pgfqpoint{3.127800in}{3.375406in}}%
\pgfpathlineto{\pgfqpoint{3.128180in}{3.425732in}}%
\pgfpathlineto{\pgfqpoint{3.128653in}{3.343163in}}%
\pgfpathlineto{\pgfqpoint{3.128938in}{3.301903in}}%
\pgfpathlineto{\pgfqpoint{3.129411in}{3.396617in}}%
\pgfpathlineto{\pgfqpoint{3.129696in}{3.437848in}}%
\pgfpathlineto{\pgfqpoint{3.130170in}{3.337845in}}%
\pgfpathlineto{\pgfqpoint{3.130454in}{3.302696in}}%
\pgfpathlineto{\pgfqpoint{3.130833in}{3.434515in}}%
\pgfpathlineto{\pgfqpoint{3.131496in}{3.785537in}}%
\pgfpathlineto{\pgfqpoint{3.132254in}{3.664750in}}%
\pgfpathlineto{\pgfqpoint{3.132823in}{3.763109in}}%
\pgfpathlineto{\pgfqpoint{3.133107in}{3.631162in}}%
\pgfpathlineto{\pgfqpoint{3.133676in}{3.184795in}}%
\pgfpathlineto{\pgfqpoint{3.134339in}{3.451995in}}%
\pgfpathlineto{\pgfqpoint{3.134529in}{3.473629in}}%
\pgfpathlineto{\pgfqpoint{3.135003in}{3.396818in}}%
\pgfpathlineto{\pgfqpoint{3.135192in}{3.380344in}}%
\pgfpathlineto{\pgfqpoint{3.135571in}{3.442293in}}%
\pgfpathlineto{\pgfqpoint{3.136045in}{3.522208in}}%
\pgfpathlineto{\pgfqpoint{3.136519in}{3.420780in}}%
\pgfpathlineto{\pgfqpoint{3.136803in}{3.385799in}}%
\pgfpathlineto{\pgfqpoint{3.137277in}{3.477988in}}%
\pgfpathlineto{\pgfqpoint{3.137656in}{3.540816in}}%
\pgfpathlineto{\pgfqpoint{3.138130in}{3.445689in}}%
\pgfpathlineto{\pgfqpoint{3.138319in}{3.422450in}}%
\pgfpathlineto{\pgfqpoint{3.138793in}{3.512723in}}%
\pgfpathlineto{\pgfqpoint{3.139172in}{3.574255in}}%
\pgfpathlineto{\pgfqpoint{3.139646in}{3.480436in}}%
\pgfpathlineto{\pgfqpoint{3.139930in}{3.451970in}}%
\pgfpathlineto{\pgfqpoint{3.140404in}{3.531907in}}%
\pgfpathlineto{\pgfqpoint{3.140783in}{3.583959in}}%
\pgfpathlineto{\pgfqpoint{3.141257in}{3.492204in}}%
\pgfpathlineto{\pgfqpoint{3.141447in}{3.472564in}}%
\pgfpathlineto{\pgfqpoint{3.142015in}{3.549088in}}%
\pgfpathlineto{\pgfqpoint{3.142394in}{3.594203in}}%
\pgfpathlineto{\pgfqpoint{3.142773in}{3.511829in}}%
\pgfpathlineto{\pgfqpoint{3.143058in}{3.466190in}}%
\pgfpathlineto{\pgfqpoint{3.143626in}{3.540886in}}%
\pgfpathlineto{\pgfqpoint{3.143816in}{3.566974in}}%
\pgfpathlineto{\pgfqpoint{3.144289in}{3.482554in}}%
\pgfpathlineto{\pgfqpoint{3.144668in}{3.410383in}}%
\pgfpathlineto{\pgfqpoint{3.145237in}{3.515613in}}%
\pgfpathlineto{\pgfqpoint{3.145521in}{3.536243in}}%
\pgfpathlineto{\pgfqpoint{3.145900in}{3.480172in}}%
\pgfpathlineto{\pgfqpoint{3.146185in}{3.436732in}}%
\pgfpathlineto{\pgfqpoint{3.146564in}{3.535234in}}%
\pgfpathlineto{\pgfqpoint{3.147038in}{3.669400in}}%
\pgfpathlineto{\pgfqpoint{3.147701in}{3.569743in}}%
\pgfpathlineto{\pgfqpoint{3.147796in}{3.562603in}}%
\pgfpathlineto{\pgfqpoint{3.148080in}{3.596456in}}%
\pgfpathlineto{\pgfqpoint{3.148554in}{3.689651in}}%
\pgfpathlineto{\pgfqpoint{3.149122in}{3.609941in}}%
\pgfpathlineto{\pgfqpoint{3.149407in}{3.565971in}}%
\pgfpathlineto{\pgfqpoint{3.149881in}{3.655159in}}%
\pgfpathlineto{\pgfqpoint{3.150165in}{3.699044in}}%
\pgfpathlineto{\pgfqpoint{3.150733in}{3.605092in}}%
\pgfpathlineto{\pgfqpoint{3.150923in}{3.586369in}}%
\pgfpathlineto{\pgfqpoint{3.151302in}{3.646577in}}%
\pgfpathlineto{\pgfqpoint{3.151681in}{3.713574in}}%
\pgfpathlineto{\pgfqpoint{3.152250in}{3.614694in}}%
\pgfpathlineto{\pgfqpoint{3.152534in}{3.572522in}}%
\pgfpathlineto{\pgfqpoint{3.153008in}{3.688311in}}%
\pgfpathlineto{\pgfqpoint{3.153197in}{3.721369in}}%
\pgfpathlineto{\pgfqpoint{3.153766in}{3.618263in}}%
\pgfpathlineto{\pgfqpoint{3.154145in}{3.578149in}}%
\pgfpathlineto{\pgfqpoint{3.154524in}{3.651835in}}%
\pgfpathlineto{\pgfqpoint{3.154808in}{3.694157in}}%
\pgfpathlineto{\pgfqpoint{3.155282in}{3.605701in}}%
\pgfpathlineto{\pgfqpoint{3.155661in}{3.528486in}}%
\pgfpathlineto{\pgfqpoint{3.156230in}{3.642762in}}%
\pgfpathlineto{\pgfqpoint{3.156419in}{3.667391in}}%
\pgfpathlineto{\pgfqpoint{3.156893in}{3.559341in}}%
\pgfpathlineto{\pgfqpoint{3.157367in}{3.494028in}}%
\pgfpathlineto{\pgfqpoint{3.157841in}{3.589203in}}%
\pgfpathlineto{\pgfqpoint{3.158030in}{3.619178in}}%
\pgfpathlineto{\pgfqpoint{3.158504in}{3.530918in}}%
\pgfpathlineto{\pgfqpoint{3.158883in}{3.474150in}}%
\pgfpathlineto{\pgfqpoint{3.159262in}{3.566865in}}%
\pgfpathlineto{\pgfqpoint{3.159546in}{3.621651in}}%
\pgfpathlineto{\pgfqpoint{3.160115in}{3.518921in}}%
\pgfpathlineto{\pgfqpoint{3.160399in}{3.474123in}}%
\pgfpathlineto{\pgfqpoint{3.160873in}{3.570996in}}%
\pgfpathlineto{\pgfqpoint{3.161157in}{3.602535in}}%
\pgfpathlineto{\pgfqpoint{3.161631in}{3.525155in}}%
\pgfpathlineto{\pgfqpoint{3.162010in}{3.477904in}}%
\pgfpathlineto{\pgfqpoint{3.162484in}{3.571296in}}%
\pgfpathlineto{\pgfqpoint{3.162674in}{3.595411in}}%
\pgfpathlineto{\pgfqpoint{3.163147in}{3.524208in}}%
\pgfpathlineto{\pgfqpoint{3.163527in}{3.450205in}}%
\pgfpathlineto{\pgfqpoint{3.164000in}{3.554315in}}%
\pgfpathlineto{\pgfqpoint{3.164379in}{3.626786in}}%
\pgfpathlineto{\pgfqpoint{3.164853in}{3.515206in}}%
\pgfpathlineto{\pgfqpoint{3.165043in}{3.502447in}}%
\pgfpathlineto{\pgfqpoint{3.165517in}{3.560778in}}%
\pgfpathlineto{\pgfqpoint{3.165896in}{3.640874in}}%
\pgfpathlineto{\pgfqpoint{3.166464in}{3.535740in}}%
\pgfpathlineto{\pgfqpoint{3.166749in}{3.501346in}}%
\pgfpathlineto{\pgfqpoint{3.167222in}{3.599663in}}%
\pgfpathlineto{\pgfqpoint{3.167507in}{3.624701in}}%
\pgfpathlineto{\pgfqpoint{3.167886in}{3.555531in}}%
\pgfpathlineto{\pgfqpoint{3.168360in}{3.471135in}}%
\pgfpathlineto{\pgfqpoint{3.168928in}{3.552228in}}%
\pgfpathlineto{\pgfqpoint{3.169023in}{3.561996in}}%
\pgfpathlineto{\pgfqpoint{3.169307in}{3.511003in}}%
\pgfpathlineto{\pgfqpoint{3.169781in}{3.400962in}}%
\pgfpathlineto{\pgfqpoint{3.170350in}{3.520797in}}%
\pgfpathlineto{\pgfqpoint{3.170539in}{3.553302in}}%
\pgfpathlineto{\pgfqpoint{3.171108in}{3.458341in}}%
\pgfpathlineto{\pgfqpoint{3.171392in}{3.420148in}}%
\pgfpathlineto{\pgfqpoint{3.171866in}{3.523151in}}%
\pgfpathlineto{\pgfqpoint{3.172150in}{3.550023in}}%
\pgfpathlineto{\pgfqpoint{3.172624in}{3.470364in}}%
\pgfpathlineto{\pgfqpoint{3.172908in}{3.426018in}}%
\pgfpathlineto{\pgfqpoint{3.173382in}{3.509500in}}%
\pgfpathlineto{\pgfqpoint{3.173761in}{3.576794in}}%
\pgfpathlineto{\pgfqpoint{3.174330in}{3.466639in}}%
\pgfpathlineto{\pgfqpoint{3.174614in}{3.437454in}}%
\pgfpathlineto{\pgfqpoint{3.174993in}{3.535019in}}%
\pgfpathlineto{\pgfqpoint{3.175277in}{3.586004in}}%
\pgfpathlineto{\pgfqpoint{3.175751in}{3.463907in}}%
\pgfpathlineto{\pgfqpoint{3.175941in}{3.430247in}}%
\pgfpathlineto{\pgfqpoint{3.176320in}{3.589357in}}%
\pgfpathlineto{\pgfqpoint{3.176983in}{3.959665in}}%
\pgfpathlineto{\pgfqpoint{3.177646in}{3.788410in}}%
\pgfpathlineto{\pgfqpoint{3.177741in}{3.788019in}}%
\pgfpathlineto{\pgfqpoint{3.178215in}{3.870639in}}%
\pgfpathlineto{\pgfqpoint{3.178499in}{3.793108in}}%
\pgfpathlineto{\pgfqpoint{3.179163in}{3.298755in}}%
\pgfpathlineto{\pgfqpoint{3.179826in}{3.609534in}}%
\pgfpathlineto{\pgfqpoint{3.180110in}{3.653877in}}%
\pgfpathlineto{\pgfqpoint{3.180679in}{3.541991in}}%
\pgfpathlineto{\pgfqpoint{3.180868in}{3.527389in}}%
\pgfpathlineto{\pgfqpoint{3.181247in}{3.592490in}}%
\pgfpathlineto{\pgfqpoint{3.181626in}{3.666618in}}%
\pgfpathlineto{\pgfqpoint{3.182100in}{3.546766in}}%
\pgfpathlineto{\pgfqpoint{3.182290in}{3.519556in}}%
\pgfpathlineto{\pgfqpoint{3.182764in}{3.597020in}}%
\pgfpathlineto{\pgfqpoint{3.183237in}{3.687272in}}%
\pgfpathlineto{\pgfqpoint{3.183711in}{3.585675in}}%
\pgfpathlineto{\pgfqpoint{3.183996in}{3.546673in}}%
\pgfpathlineto{\pgfqpoint{3.184469in}{3.660297in}}%
\pgfpathlineto{\pgfqpoint{3.184659in}{3.691833in}}%
\pgfpathlineto{\pgfqpoint{3.185228in}{3.613111in}}%
\pgfpathlineto{\pgfqpoint{3.185607in}{3.551293in}}%
\pgfpathlineto{\pgfqpoint{3.185986in}{3.665108in}}%
\pgfpathlineto{\pgfqpoint{3.186270in}{3.719699in}}%
\pgfpathlineto{\pgfqpoint{3.186839in}{3.629956in}}%
\pgfpathlineto{\pgfqpoint{3.187123in}{3.582957in}}%
\pgfpathlineto{\pgfqpoint{3.187691in}{3.696865in}}%
\pgfpathlineto{\pgfqpoint{3.187976in}{3.727789in}}%
\pgfpathlineto{\pgfqpoint{3.188355in}{3.637580in}}%
\pgfpathlineto{\pgfqpoint{3.188544in}{3.608718in}}%
\pgfpathlineto{\pgfqpoint{3.189113in}{3.693422in}}%
\pgfpathlineto{\pgfqpoint{3.189492in}{3.778807in}}%
\pgfpathlineto{\pgfqpoint{3.189966in}{3.653936in}}%
\pgfpathlineto{\pgfqpoint{3.190155in}{3.618515in}}%
\pgfpathlineto{\pgfqpoint{3.190724in}{3.709036in}}%
\pgfpathlineto{\pgfqpoint{3.191103in}{3.760735in}}%
\pgfpathlineto{\pgfqpoint{3.191577in}{3.661426in}}%
\pgfpathlineto{\pgfqpoint{3.191861in}{3.623730in}}%
\pgfpathlineto{\pgfqpoint{3.192240in}{3.715714in}}%
\pgfpathlineto{\pgfqpoint{3.192524in}{3.798193in}}%
\pgfpathlineto{\pgfqpoint{3.193188in}{3.663364in}}%
\pgfpathlineto{\pgfqpoint{3.193472in}{3.628665in}}%
\pgfpathlineto{\pgfqpoint{3.193851in}{3.712437in}}%
\pgfpathlineto{\pgfqpoint{3.194230in}{3.781346in}}%
\pgfpathlineto{\pgfqpoint{3.194704in}{3.674959in}}%
\pgfpathlineto{\pgfqpoint{3.194988in}{3.636149in}}%
\pgfpathlineto{\pgfqpoint{3.195462in}{3.724275in}}%
\pgfpathlineto{\pgfqpoint{3.195746in}{3.782360in}}%
\pgfpathlineto{\pgfqpoint{3.196315in}{3.669720in}}%
\pgfpathlineto{\pgfqpoint{3.196599in}{3.640358in}}%
\pgfpathlineto{\pgfqpoint{3.196978in}{3.699980in}}%
\pgfpathlineto{\pgfqpoint{3.197357in}{3.765543in}}%
\pgfpathlineto{\pgfqpoint{3.197831in}{3.677655in}}%
\pgfpathlineto{\pgfqpoint{3.198115in}{3.640284in}}%
\pgfpathlineto{\pgfqpoint{3.198589in}{3.733352in}}%
\pgfpathlineto{\pgfqpoint{3.198874in}{3.769070in}}%
\pgfpathlineto{\pgfqpoint{3.199347in}{3.679632in}}%
\pgfpathlineto{\pgfqpoint{3.199726in}{3.624633in}}%
\pgfpathlineto{\pgfqpoint{3.200200in}{3.718543in}}%
\pgfpathlineto{\pgfqpoint{3.200390in}{3.760628in}}%
\pgfpathlineto{\pgfqpoint{3.200958in}{3.637160in}}%
\pgfpathlineto{\pgfqpoint{3.201337in}{3.600152in}}%
\pgfpathlineto{\pgfqpoint{3.201716in}{3.670407in}}%
\pgfpathlineto{\pgfqpoint{3.202001in}{3.720891in}}%
\pgfpathlineto{\pgfqpoint{3.202475in}{3.595672in}}%
\pgfpathlineto{\pgfqpoint{3.202759in}{3.535691in}}%
\pgfpathlineto{\pgfqpoint{3.203327in}{3.638615in}}%
\pgfpathlineto{\pgfqpoint{3.203707in}{3.718241in}}%
\pgfpathlineto{\pgfqpoint{3.204180in}{3.583395in}}%
\pgfpathlineto{\pgfqpoint{3.204370in}{3.568257in}}%
\pgfpathlineto{\pgfqpoint{3.204844in}{3.634726in}}%
\pgfpathlineto{\pgfqpoint{3.205128in}{3.703933in}}%
\pgfpathlineto{\pgfqpoint{3.205697in}{3.586015in}}%
\pgfpathlineto{\pgfqpoint{3.206076in}{3.543263in}}%
\pgfpathlineto{\pgfqpoint{3.206455in}{3.643154in}}%
\pgfpathlineto{\pgfqpoint{3.206739in}{3.694932in}}%
\pgfpathlineto{\pgfqpoint{3.207308in}{3.575543in}}%
\pgfpathlineto{\pgfqpoint{3.207592in}{3.531718in}}%
\pgfpathlineto{\pgfqpoint{3.208066in}{3.645124in}}%
\pgfpathlineto{\pgfqpoint{3.208255in}{3.672130in}}%
\pgfpathlineto{\pgfqpoint{3.208729in}{3.598630in}}%
\pgfpathlineto{\pgfqpoint{3.209108in}{3.508788in}}%
\pgfpathlineto{\pgfqpoint{3.209582in}{3.630116in}}%
\pgfpathlineto{\pgfqpoint{3.209961in}{3.704657in}}%
\pgfpathlineto{\pgfqpoint{3.210435in}{3.578135in}}%
\pgfpathlineto{\pgfqpoint{3.210624in}{3.538699in}}%
\pgfpathlineto{\pgfqpoint{3.211193in}{3.679602in}}%
\pgfpathlineto{\pgfqpoint{3.211477in}{3.739335in}}%
\pgfpathlineto{\pgfqpoint{3.211951in}{3.621943in}}%
\pgfpathlineto{\pgfqpoint{3.212330in}{3.555459in}}%
\pgfpathlineto{\pgfqpoint{3.212899in}{3.656379in}}%
\pgfpathlineto{\pgfqpoint{3.213088in}{3.681748in}}%
\pgfpathlineto{\pgfqpoint{3.213467in}{3.547817in}}%
\pgfpathlineto{\pgfqpoint{3.213846in}{3.420335in}}%
\pgfpathlineto{\pgfqpoint{3.214604in}{3.510335in}}%
\pgfpathlineto{\pgfqpoint{3.214889in}{3.477320in}}%
\pgfpathlineto{\pgfqpoint{3.215363in}{3.418496in}}%
\pgfpathlineto{\pgfqpoint{3.215742in}{3.493611in}}%
\pgfpathlineto{\pgfqpoint{3.216310in}{3.637427in}}%
\pgfpathlineto{\pgfqpoint{3.216879in}{3.513308in}}%
\pgfpathlineto{\pgfqpoint{3.216974in}{3.507667in}}%
\pgfpathlineto{\pgfqpoint{3.217258in}{3.547687in}}%
\pgfpathlineto{\pgfqpoint{3.217826in}{3.629550in}}%
\pgfpathlineto{\pgfqpoint{3.218205in}{3.552405in}}%
\pgfpathlineto{\pgfqpoint{3.218584in}{3.494270in}}%
\pgfpathlineto{\pgfqpoint{3.219058in}{3.608696in}}%
\pgfpathlineto{\pgfqpoint{3.219248in}{3.643444in}}%
\pgfpathlineto{\pgfqpoint{3.219816in}{3.565016in}}%
\pgfpathlineto{\pgfqpoint{3.220195in}{3.494310in}}%
\pgfpathlineto{\pgfqpoint{3.220669in}{3.619148in}}%
\pgfpathlineto{\pgfqpoint{3.220859in}{3.632429in}}%
\pgfpathlineto{\pgfqpoint{3.221238in}{3.571109in}}%
\pgfpathlineto{\pgfqpoint{3.221427in}{3.554025in}}%
\pgfpathlineto{\pgfqpoint{3.221712in}{3.636406in}}%
\pgfpathlineto{\pgfqpoint{3.222470in}{4.035120in}}%
\pgfpathlineto{\pgfqpoint{3.223133in}{3.839118in}}%
\pgfpathlineto{\pgfqpoint{3.223607in}{3.878363in}}%
\pgfpathlineto{\pgfqpoint{3.224081in}{3.698064in}}%
\pgfpathlineto{\pgfqpoint{3.224555in}{3.388686in}}%
\pgfpathlineto{\pgfqpoint{3.225218in}{3.604179in}}%
\pgfpathlineto{\pgfqpoint{3.225597in}{3.694385in}}%
\pgfpathlineto{\pgfqpoint{3.226166in}{3.571983in}}%
\pgfpathlineto{\pgfqpoint{3.226450in}{3.537306in}}%
\pgfpathlineto{\pgfqpoint{3.226829in}{3.614108in}}%
\pgfpathlineto{\pgfqpoint{3.227208in}{3.680925in}}%
\pgfpathlineto{\pgfqpoint{3.227777in}{3.581619in}}%
\pgfpathlineto{\pgfqpoint{3.227966in}{3.548382in}}%
\pgfpathlineto{\pgfqpoint{3.228440in}{3.654896in}}%
\pgfpathlineto{\pgfqpoint{3.228724in}{3.700462in}}%
\pgfpathlineto{\pgfqpoint{3.229198in}{3.608747in}}%
\pgfpathlineto{\pgfqpoint{3.229577in}{3.527034in}}%
\pgfpathlineto{\pgfqpoint{3.230051in}{3.655555in}}%
\pgfpathlineto{\pgfqpoint{3.230335in}{3.703258in}}%
\pgfpathlineto{\pgfqpoint{3.230904in}{3.582842in}}%
\pgfpathlineto{\pgfqpoint{3.231188in}{3.549862in}}%
\pgfpathlineto{\pgfqpoint{3.231567in}{3.637592in}}%
\pgfpathlineto{\pgfqpoint{3.231851in}{3.684900in}}%
\pgfpathlineto{\pgfqpoint{3.232420in}{3.595632in}}%
\pgfpathlineto{\pgfqpoint{3.232704in}{3.540292in}}%
\pgfpathlineto{\pgfqpoint{3.233178in}{3.653875in}}%
\pgfpathlineto{\pgfqpoint{3.233557in}{3.697643in}}%
\pgfpathlineto{\pgfqpoint{3.233936in}{3.621093in}}%
\pgfpathlineto{\pgfqpoint{3.234221in}{3.562057in}}%
\pgfpathlineto{\pgfqpoint{3.234789in}{3.658812in}}%
\pgfpathlineto{\pgfqpoint{3.235073in}{3.701501in}}%
\pgfpathlineto{\pgfqpoint{3.235547in}{3.587215in}}%
\pgfpathlineto{\pgfqpoint{3.235832in}{3.548681in}}%
\pgfpathlineto{\pgfqpoint{3.236305in}{3.652083in}}%
\pgfpathlineto{\pgfqpoint{3.236590in}{3.702378in}}%
\pgfpathlineto{\pgfqpoint{3.237158in}{3.599801in}}%
\pgfpathlineto{\pgfqpoint{3.237443in}{3.567791in}}%
\pgfpathlineto{\pgfqpoint{3.237916in}{3.652211in}}%
\pgfpathlineto{\pgfqpoint{3.238295in}{3.702300in}}%
\pgfpathlineto{\pgfqpoint{3.238674in}{3.606504in}}%
\pgfpathlineto{\pgfqpoint{3.238959in}{3.549856in}}%
\pgfpathlineto{\pgfqpoint{3.239527in}{3.690389in}}%
\pgfpathlineto{\pgfqpoint{3.239812in}{3.730018in}}%
\pgfpathlineto{\pgfqpoint{3.240191in}{3.634333in}}%
\pgfpathlineto{\pgfqpoint{3.240570in}{3.552645in}}%
\pgfpathlineto{\pgfqpoint{3.241138in}{3.659732in}}%
\pgfpathlineto{\pgfqpoint{3.241328in}{3.690984in}}%
\pgfpathlineto{\pgfqpoint{3.241802in}{3.588559in}}%
\pgfpathlineto{\pgfqpoint{3.242086in}{3.540852in}}%
\pgfpathlineto{\pgfqpoint{3.242560in}{3.644943in}}%
\pgfpathlineto{\pgfqpoint{3.242939in}{3.708143in}}%
\pgfpathlineto{\pgfqpoint{3.243413in}{3.579184in}}%
\pgfpathlineto{\pgfqpoint{3.243602in}{3.550028in}}%
\pgfpathlineto{\pgfqpoint{3.244266in}{3.635916in}}%
\pgfpathlineto{\pgfqpoint{3.244550in}{3.682597in}}%
\pgfpathlineto{\pgfqpoint{3.245024in}{3.567651in}}%
\pgfpathlineto{\pgfqpoint{3.245308in}{3.521703in}}%
\pgfpathlineto{\pgfqpoint{3.245782in}{3.632005in}}%
\pgfpathlineto{\pgfqpoint{3.246066in}{3.663194in}}%
\pgfpathlineto{\pgfqpoint{3.246445in}{3.587784in}}%
\pgfpathlineto{\pgfqpoint{3.246824in}{3.473894in}}%
\pgfpathlineto{\pgfqpoint{3.247393in}{3.599705in}}%
\pgfpathlineto{\pgfqpoint{3.247582in}{3.625350in}}%
\pgfpathlineto{\pgfqpoint{3.248056in}{3.527722in}}%
\pgfpathlineto{\pgfqpoint{3.248530in}{3.469937in}}%
\pgfpathlineto{\pgfqpoint{3.248909in}{3.576659in}}%
\pgfpathlineto{\pgfqpoint{3.249099in}{3.622340in}}%
\pgfpathlineto{\pgfqpoint{3.249667in}{3.514282in}}%
\pgfpathlineto{\pgfqpoint{3.250046in}{3.437332in}}%
\pgfpathlineto{\pgfqpoint{3.250520in}{3.574555in}}%
\pgfpathlineto{\pgfqpoint{3.250710in}{3.603308in}}%
\pgfpathlineto{\pgfqpoint{3.251278in}{3.518397in}}%
\pgfpathlineto{\pgfqpoint{3.251562in}{3.468222in}}%
\pgfpathlineto{\pgfqpoint{3.252036in}{3.559979in}}%
\pgfpathlineto{\pgfqpoint{3.252415in}{3.610811in}}%
\pgfpathlineto{\pgfqpoint{3.252794in}{3.519548in}}%
\pgfpathlineto{\pgfqpoint{3.253079in}{3.432870in}}%
\pgfpathlineto{\pgfqpoint{3.253647in}{3.572384in}}%
\pgfpathlineto{\pgfqpoint{3.253932in}{3.610302in}}%
\pgfpathlineto{\pgfqpoint{3.254405in}{3.525037in}}%
\pgfpathlineto{\pgfqpoint{3.254690in}{3.481122in}}%
\pgfpathlineto{\pgfqpoint{3.255163in}{3.621309in}}%
\pgfpathlineto{\pgfqpoint{3.255448in}{3.670044in}}%
\pgfpathlineto{\pgfqpoint{3.255922in}{3.564667in}}%
\pgfpathlineto{\pgfqpoint{3.256206in}{3.502435in}}%
\pgfpathlineto{\pgfqpoint{3.256774in}{3.625163in}}%
\pgfpathlineto{\pgfqpoint{3.257153in}{3.690849in}}%
\pgfpathlineto{\pgfqpoint{3.257627in}{3.560024in}}%
\pgfpathlineto{\pgfqpoint{3.257912in}{3.514424in}}%
\pgfpathlineto{\pgfqpoint{3.258385in}{3.617224in}}%
\pgfpathlineto{\pgfqpoint{3.258575in}{3.640698in}}%
\pgfpathlineto{\pgfqpoint{3.258954in}{3.564310in}}%
\pgfpathlineto{\pgfqpoint{3.259428in}{3.418002in}}%
\pgfpathlineto{\pgfqpoint{3.260091in}{3.551668in}}%
\pgfpathlineto{\pgfqpoint{3.260281in}{3.562704in}}%
\pgfpathlineto{\pgfqpoint{3.260565in}{3.501634in}}%
\pgfpathlineto{\pgfqpoint{3.260944in}{3.393446in}}%
\pgfpathlineto{\pgfqpoint{3.261607in}{3.512942in}}%
\pgfpathlineto{\pgfqpoint{3.261797in}{3.538839in}}%
\pgfpathlineto{\pgfqpoint{3.262176in}{3.449031in}}%
\pgfpathlineto{\pgfqpoint{3.262650in}{3.385132in}}%
\pgfpathlineto{\pgfqpoint{3.263029in}{3.470842in}}%
\pgfpathlineto{\pgfqpoint{3.263313in}{3.553713in}}%
\pgfpathlineto{\pgfqpoint{3.263977in}{3.429880in}}%
\pgfpathlineto{\pgfqpoint{3.264261in}{3.407151in}}%
\pgfpathlineto{\pgfqpoint{3.264545in}{3.452508in}}%
\pgfpathlineto{\pgfqpoint{3.264924in}{3.548549in}}%
\pgfpathlineto{\pgfqpoint{3.265493in}{3.401825in}}%
\pgfpathlineto{\pgfqpoint{3.265682in}{3.373387in}}%
\pgfpathlineto{\pgfqpoint{3.266156in}{3.485351in}}%
\pgfpathlineto{\pgfqpoint{3.268051in}{3.895936in}}%
\pgfpathlineto{\pgfqpoint{3.268525in}{3.779618in}}%
\pgfpathlineto{\pgfqpoint{3.269947in}{3.398098in}}%
\pgfpathlineto{\pgfqpoint{3.270705in}{3.483036in}}%
\pgfpathlineto{\pgfqpoint{3.271273in}{3.596411in}}%
\pgfpathlineto{\pgfqpoint{3.271747in}{3.486261in}}%
\pgfpathlineto{\pgfqpoint{3.271937in}{3.446393in}}%
\pgfpathlineto{\pgfqpoint{3.272505in}{3.555577in}}%
\pgfpathlineto{\pgfqpoint{3.272884in}{3.609253in}}%
\pgfpathlineto{\pgfqpoint{3.273263in}{3.510370in}}%
\pgfpathlineto{\pgfqpoint{3.273548in}{3.470269in}}%
\pgfpathlineto{\pgfqpoint{3.274116in}{3.552923in}}%
\pgfpathlineto{\pgfqpoint{3.274401in}{3.611200in}}%
\pgfpathlineto{\pgfqpoint{3.274874in}{3.499845in}}%
\pgfpathlineto{\pgfqpoint{3.275064in}{3.474570in}}%
\pgfpathlineto{\pgfqpoint{3.275538in}{3.547001in}}%
\pgfpathlineto{\pgfqpoint{3.276012in}{3.632434in}}%
\pgfpathlineto{\pgfqpoint{3.276485in}{3.509200in}}%
\pgfpathlineto{\pgfqpoint{3.276770in}{3.472227in}}%
\pgfpathlineto{\pgfqpoint{3.277243in}{3.582515in}}%
\pgfpathlineto{\pgfqpoint{3.277528in}{3.634061in}}%
\pgfpathlineto{\pgfqpoint{3.278096in}{3.505208in}}%
\pgfpathlineto{\pgfqpoint{3.278381in}{3.469545in}}%
\pgfpathlineto{\pgfqpoint{3.278854in}{3.569347in}}%
\pgfpathlineto{\pgfqpoint{3.279044in}{3.584360in}}%
\pgfpathlineto{\pgfqpoint{3.279423in}{3.518649in}}%
\pgfpathlineto{\pgfqpoint{3.279897in}{3.386355in}}%
\pgfpathlineto{\pgfqpoint{3.280465in}{3.519910in}}%
\pgfpathlineto{\pgfqpoint{3.281413in}{3.519172in}}%
\pgfpathlineto{\pgfqpoint{3.282171in}{3.715051in}}%
\pgfpathlineto{\pgfqpoint{3.282361in}{3.703044in}}%
\pgfpathlineto{\pgfqpoint{3.283024in}{3.570455in}}%
\pgfpathlineto{\pgfqpoint{3.283498in}{3.664640in}}%
\pgfpathlineto{\pgfqpoint{3.283877in}{3.722428in}}%
\pgfpathlineto{\pgfqpoint{3.284351in}{3.609114in}}%
\pgfpathlineto{\pgfqpoint{3.284540in}{3.587978in}}%
\pgfpathlineto{\pgfqpoint{3.285014in}{3.671759in}}%
\pgfpathlineto{\pgfqpoint{3.285393in}{3.729505in}}%
\pgfpathlineto{\pgfqpoint{3.285772in}{3.623463in}}%
\pgfpathlineto{\pgfqpoint{3.286151in}{3.555100in}}%
\pgfpathlineto{\pgfqpoint{3.286625in}{3.663370in}}%
\pgfpathlineto{\pgfqpoint{3.286909in}{3.725199in}}%
\pgfpathlineto{\pgfqpoint{3.287478in}{3.624184in}}%
\pgfpathlineto{\pgfqpoint{3.287762in}{3.589347in}}%
\pgfpathlineto{\pgfqpoint{3.288236in}{3.686882in}}%
\pgfpathlineto{\pgfqpoint{3.288520in}{3.724618in}}%
\pgfpathlineto{\pgfqpoint{3.288994in}{3.627854in}}%
\pgfpathlineto{\pgfqpoint{3.289373in}{3.564934in}}%
\pgfpathlineto{\pgfqpoint{3.289752in}{3.674680in}}%
\pgfpathlineto{\pgfqpoint{3.290131in}{3.723854in}}%
\pgfpathlineto{\pgfqpoint{3.290605in}{3.611091in}}%
\pgfpathlineto{\pgfqpoint{3.290890in}{3.556171in}}%
\pgfpathlineto{\pgfqpoint{3.291553in}{3.655511in}}%
\pgfpathlineto{\pgfqpoint{3.291648in}{3.658552in}}%
\pgfpathlineto{\pgfqpoint{3.291837in}{3.636583in}}%
\pgfpathlineto{\pgfqpoint{3.292501in}{3.499353in}}%
\pgfpathlineto{\pgfqpoint{3.292974in}{3.615048in}}%
\pgfpathlineto{\pgfqpoint{3.293259in}{3.666048in}}%
\pgfpathlineto{\pgfqpoint{3.293732in}{3.517605in}}%
\pgfpathlineto{\pgfqpoint{3.294017in}{3.457407in}}%
\pgfpathlineto{\pgfqpoint{3.294585in}{3.589907in}}%
\pgfpathlineto{\pgfqpoint{3.294775in}{3.624689in}}%
\pgfpathlineto{\pgfqpoint{3.295249in}{3.516098in}}%
\pgfpathlineto{\pgfqpoint{3.295628in}{3.455740in}}%
\pgfpathlineto{\pgfqpoint{3.296102in}{3.562903in}}%
\pgfpathlineto{\pgfqpoint{3.296386in}{3.619796in}}%
\pgfpathlineto{\pgfqpoint{3.296860in}{3.493667in}}%
\pgfpathlineto{\pgfqpoint{3.297239in}{3.430227in}}%
\pgfpathlineto{\pgfqpoint{3.297713in}{3.550271in}}%
\pgfpathlineto{\pgfqpoint{3.297902in}{3.585601in}}%
\pgfpathlineto{\pgfqpoint{3.298471in}{3.469272in}}%
\pgfpathlineto{\pgfqpoint{3.298755in}{3.431544in}}%
\pgfpathlineto{\pgfqpoint{3.299229in}{3.543334in}}%
\pgfpathlineto{\pgfqpoint{3.299513in}{3.593212in}}%
\pgfpathlineto{\pgfqpoint{3.300082in}{3.478640in}}%
\pgfpathlineto{\pgfqpoint{3.300271in}{3.453943in}}%
\pgfpathlineto{\pgfqpoint{3.300745in}{3.518645in}}%
\pgfpathlineto{\pgfqpoint{3.301124in}{3.600724in}}%
\pgfpathlineto{\pgfqpoint{3.301787in}{3.502826in}}%
\pgfpathlineto{\pgfqpoint{3.301977in}{3.495433in}}%
\pgfpathlineto{\pgfqpoint{3.302261in}{3.528222in}}%
\pgfpathlineto{\pgfqpoint{3.302640in}{3.616868in}}%
\pgfpathlineto{\pgfqpoint{3.303114in}{3.513967in}}%
\pgfpathlineto{\pgfqpoint{3.303493in}{3.425275in}}%
\pgfpathlineto{\pgfqpoint{3.304062in}{3.551294in}}%
\pgfpathlineto{\pgfqpoint{3.304156in}{3.557650in}}%
\pgfpathlineto{\pgfqpoint{3.304441in}{3.527850in}}%
\pgfpathlineto{\pgfqpoint{3.305104in}{3.382125in}}%
\pgfpathlineto{\pgfqpoint{3.305578in}{3.484419in}}%
\pgfpathlineto{\pgfqpoint{3.305862in}{3.532493in}}%
\pgfpathlineto{\pgfqpoint{3.306336in}{3.379702in}}%
\pgfpathlineto{\pgfqpoint{3.306620in}{3.356076in}}%
\pgfpathlineto{\pgfqpoint{3.306999in}{3.414881in}}%
\pgfpathlineto{\pgfqpoint{3.307378in}{3.509838in}}%
\pgfpathlineto{\pgfqpoint{3.307947in}{3.396064in}}%
\pgfpathlineto{\pgfqpoint{3.308231in}{3.345392in}}%
\pgfpathlineto{\pgfqpoint{3.308705in}{3.479212in}}%
\pgfpathlineto{\pgfqpoint{3.308895in}{3.491183in}}%
\pgfpathlineto{\pgfqpoint{3.309274in}{3.439822in}}%
\pgfpathlineto{\pgfqpoint{3.309748in}{3.328191in}}%
\pgfpathlineto{\pgfqpoint{3.310316in}{3.460036in}}%
\pgfpathlineto{\pgfqpoint{3.310600in}{3.509723in}}%
\pgfpathlineto{\pgfqpoint{3.311074in}{3.376610in}}%
\pgfpathlineto{\pgfqpoint{3.311453in}{3.330261in}}%
\pgfpathlineto{\pgfqpoint{3.311832in}{3.425165in}}%
\pgfpathlineto{\pgfqpoint{3.313633in}{3.843471in}}%
\pgfpathlineto{\pgfqpoint{3.313822in}{3.832546in}}%
\pgfpathlineto{\pgfqpoint{3.314486in}{3.458425in}}%
\pgfpathlineto{\pgfqpoint{3.314960in}{3.276363in}}%
\pgfpathlineto{\pgfqpoint{3.315623in}{3.417420in}}%
\pgfpathlineto{\pgfqpoint{3.316002in}{3.364367in}}%
\pgfpathlineto{\pgfqpoint{3.316476in}{3.454370in}}%
\pgfpathlineto{\pgfqpoint{3.316855in}{3.543223in}}%
\pgfpathlineto{\pgfqpoint{3.317423in}{3.405708in}}%
\pgfpathlineto{\pgfqpoint{3.317613in}{3.390554in}}%
\pgfpathlineto{\pgfqpoint{3.317992in}{3.440870in}}%
\pgfpathlineto{\pgfqpoint{3.318371in}{3.511325in}}%
\pgfpathlineto{\pgfqpoint{3.318940in}{3.408306in}}%
\pgfpathlineto{\pgfqpoint{3.319224in}{3.372833in}}%
\pgfpathlineto{\pgfqpoint{3.319603in}{3.468936in}}%
\pgfpathlineto{\pgfqpoint{3.319982in}{3.530287in}}%
\pgfpathlineto{\pgfqpoint{3.320456in}{3.433756in}}%
\pgfpathlineto{\pgfqpoint{3.320740in}{3.383332in}}%
\pgfpathlineto{\pgfqpoint{3.321214in}{3.481554in}}%
\pgfpathlineto{\pgfqpoint{3.321498in}{3.519912in}}%
\pgfpathlineto{\pgfqpoint{3.321972in}{3.435265in}}%
\pgfpathlineto{\pgfqpoint{3.322256in}{3.384661in}}%
\pgfpathlineto{\pgfqpoint{3.322730in}{3.488373in}}%
\pgfpathlineto{\pgfqpoint{3.323204in}{3.569254in}}%
\pgfpathlineto{\pgfqpoint{3.323678in}{3.442107in}}%
\pgfpathlineto{\pgfqpoint{3.323867in}{3.404986in}}%
\pgfpathlineto{\pgfqpoint{3.324436in}{3.525728in}}%
\pgfpathlineto{\pgfqpoint{3.324720in}{3.574527in}}%
\pgfpathlineto{\pgfqpoint{3.325194in}{3.464460in}}%
\pgfpathlineto{\pgfqpoint{3.325478in}{3.426849in}}%
\pgfpathlineto{\pgfqpoint{3.325857in}{3.513453in}}%
\pgfpathlineto{\pgfqpoint{3.326237in}{3.610688in}}%
\pgfpathlineto{\pgfqpoint{3.326805in}{3.478601in}}%
\pgfpathlineto{\pgfqpoint{3.327089in}{3.428809in}}%
\pgfpathlineto{\pgfqpoint{3.327563in}{3.540447in}}%
\pgfpathlineto{\pgfqpoint{3.327848in}{3.604009in}}%
\pgfpathlineto{\pgfqpoint{3.328416in}{3.465278in}}%
\pgfpathlineto{\pgfqpoint{3.328606in}{3.445678in}}%
\pgfpathlineto{\pgfqpoint{3.328985in}{3.526518in}}%
\pgfpathlineto{\pgfqpoint{3.329459in}{3.607734in}}%
\pgfpathlineto{\pgfqpoint{3.329932in}{3.488022in}}%
\pgfpathlineto{\pgfqpoint{3.330217in}{3.439722in}}%
\pgfpathlineto{\pgfqpoint{3.330690in}{3.549176in}}%
\pgfpathlineto{\pgfqpoint{3.331069in}{3.592693in}}%
\pgfpathlineto{\pgfqpoint{3.331449in}{3.515261in}}%
\pgfpathlineto{\pgfqpoint{3.331733in}{3.470158in}}%
\pgfpathlineto{\pgfqpoint{3.332207in}{3.559629in}}%
\pgfpathlineto{\pgfqpoint{3.332491in}{3.628680in}}%
\pgfpathlineto{\pgfqpoint{3.333060in}{3.496359in}}%
\pgfpathlineto{\pgfqpoint{3.333344in}{3.449289in}}%
\pgfpathlineto{\pgfqpoint{3.333818in}{3.548621in}}%
\pgfpathlineto{\pgfqpoint{3.334197in}{3.602768in}}%
\pgfpathlineto{\pgfqpoint{3.334671in}{3.506112in}}%
\pgfpathlineto{\pgfqpoint{3.334955in}{3.461800in}}%
\pgfpathlineto{\pgfqpoint{3.335429in}{3.560337in}}%
\pgfpathlineto{\pgfqpoint{3.335713in}{3.597761in}}%
\pgfpathlineto{\pgfqpoint{3.336092in}{3.499737in}}%
\pgfpathlineto{\pgfqpoint{3.336566in}{3.416888in}}%
\pgfpathlineto{\pgfqpoint{3.337040in}{3.530069in}}%
\pgfpathlineto{\pgfqpoint{3.337324in}{3.569083in}}%
\pgfpathlineto{\pgfqpoint{3.337798in}{3.449057in}}%
\pgfpathlineto{\pgfqpoint{3.338082in}{3.397462in}}%
\pgfpathlineto{\pgfqpoint{3.338556in}{3.511009in}}%
\pgfpathlineto{\pgfqpoint{3.338745in}{3.539327in}}%
\pgfpathlineto{\pgfqpoint{3.339219in}{3.434174in}}%
\pgfpathlineto{\pgfqpoint{3.339598in}{3.353507in}}%
\pgfpathlineto{\pgfqpoint{3.340072in}{3.468823in}}%
\pgfpathlineto{\pgfqpoint{3.340451in}{3.527697in}}%
\pgfpathlineto{\pgfqpoint{3.340925in}{3.407886in}}%
\pgfpathlineto{\pgfqpoint{3.341209in}{3.362736in}}%
\pgfpathlineto{\pgfqpoint{3.341683in}{3.469222in}}%
\pgfpathlineto{\pgfqpoint{3.341967in}{3.508474in}}%
\pgfpathlineto{\pgfqpoint{3.342441in}{3.409682in}}%
\pgfpathlineto{\pgfqpoint{3.342725in}{3.345154in}}%
\pgfpathlineto{\pgfqpoint{3.343294in}{3.471683in}}%
\pgfpathlineto{\pgfqpoint{3.343578in}{3.529942in}}%
\pgfpathlineto{\pgfqpoint{3.344147in}{3.396980in}}%
\pgfpathlineto{\pgfqpoint{3.344431in}{3.360266in}}%
\pgfpathlineto{\pgfqpoint{3.344810in}{3.431063in}}%
\pgfpathlineto{\pgfqpoint{3.345189in}{3.517632in}}%
\pgfpathlineto{\pgfqpoint{3.345758in}{3.369683in}}%
\pgfpathlineto{\pgfqpoint{3.345947in}{3.349253in}}%
\pgfpathlineto{\pgfqpoint{3.346232in}{3.438824in}}%
\pgfpathlineto{\pgfqpoint{3.346611in}{3.529041in}}%
\pgfpathlineto{\pgfqpoint{3.347179in}{3.431103in}}%
\pgfpathlineto{\pgfqpoint{3.347558in}{3.339895in}}%
\pgfpathlineto{\pgfqpoint{3.348127in}{3.463361in}}%
\pgfpathlineto{\pgfqpoint{3.348222in}{3.469403in}}%
\pgfpathlineto{\pgfqpoint{3.348506in}{3.429391in}}%
\pgfpathlineto{\pgfqpoint{3.348885in}{3.372498in}}%
\pgfpathlineto{\pgfqpoint{3.349359in}{3.465213in}}%
\pgfpathlineto{\pgfqpoint{3.349833in}{3.583455in}}%
\pgfpathlineto{\pgfqpoint{3.350307in}{3.458068in}}%
\pgfpathlineto{\pgfqpoint{3.350686in}{3.397003in}}%
\pgfpathlineto{\pgfqpoint{3.351254in}{3.501668in}}%
\pgfpathlineto{\pgfqpoint{3.351349in}{3.513138in}}%
\pgfpathlineto{\pgfqpoint{3.351728in}{3.459837in}}%
\pgfpathlineto{\pgfqpoint{3.352202in}{3.368038in}}%
\pgfpathlineto{\pgfqpoint{3.352676in}{3.486525in}}%
\pgfpathlineto{\pgfqpoint{3.353055in}{3.541872in}}%
\pgfpathlineto{\pgfqpoint{3.353434in}{3.449471in}}%
\pgfpathlineto{\pgfqpoint{3.353718in}{3.366132in}}%
\pgfpathlineto{\pgfqpoint{3.354381in}{3.489222in}}%
\pgfpathlineto{\pgfqpoint{3.354666in}{3.510064in}}%
\pgfpathlineto{\pgfqpoint{3.355045in}{3.445667in}}%
\pgfpathlineto{\pgfqpoint{3.355329in}{3.404239in}}%
\pgfpathlineto{\pgfqpoint{3.355803in}{3.501548in}}%
\pgfpathlineto{\pgfqpoint{3.356087in}{3.550163in}}%
\pgfpathlineto{\pgfqpoint{3.356656in}{3.452951in}}%
\pgfpathlineto{\pgfqpoint{3.357035in}{3.376146in}}%
\pgfpathlineto{\pgfqpoint{3.357414in}{3.493868in}}%
\pgfpathlineto{\pgfqpoint{3.359309in}{3.946709in}}%
\pgfpathlineto{\pgfqpoint{3.359783in}{3.687118in}}%
\pgfpathlineto{\pgfqpoint{3.360446in}{3.339437in}}%
\pgfpathlineto{\pgfqpoint{3.361015in}{3.565380in}}%
\pgfpathlineto{\pgfqpoint{3.361394in}{3.493621in}}%
\pgfpathlineto{\pgfqpoint{3.361584in}{3.469990in}}%
\pgfpathlineto{\pgfqpoint{3.362057in}{3.551329in}}%
\pgfpathlineto{\pgfqpoint{3.362436in}{3.619688in}}%
\pgfpathlineto{\pgfqpoint{3.362910in}{3.504936in}}%
\pgfpathlineto{\pgfqpoint{3.363195in}{3.455332in}}%
\pgfpathlineto{\pgfqpoint{3.363668in}{3.557438in}}%
\pgfpathlineto{\pgfqpoint{3.364047in}{3.637500in}}%
\pgfpathlineto{\pgfqpoint{3.364616in}{3.511143in}}%
\pgfpathlineto{\pgfqpoint{3.364711in}{3.501685in}}%
\pgfpathlineto{\pgfqpoint{3.365090in}{3.548712in}}%
\pgfpathlineto{\pgfqpoint{3.365564in}{3.620475in}}%
\pgfpathlineto{\pgfqpoint{3.365943in}{3.544580in}}%
\pgfpathlineto{\pgfqpoint{3.366322in}{3.471333in}}%
\pgfpathlineto{\pgfqpoint{3.366796in}{3.580698in}}%
\pgfpathlineto{\pgfqpoint{3.367175in}{3.634833in}}%
\pgfpathlineto{\pgfqpoint{3.367648in}{3.535732in}}%
\pgfpathlineto{\pgfqpoint{3.368027in}{3.484894in}}%
\pgfpathlineto{\pgfqpoint{3.368407in}{3.584623in}}%
\pgfpathlineto{\pgfqpoint{3.368691in}{3.640160in}}%
\pgfpathlineto{\pgfqpoint{3.369165in}{3.521125in}}%
\pgfpathlineto{\pgfqpoint{3.369449in}{3.469446in}}%
\pgfpathlineto{\pgfqpoint{3.369923in}{3.563529in}}%
\pgfpathlineto{\pgfqpoint{3.370397in}{3.650709in}}%
\pgfpathlineto{\pgfqpoint{3.370870in}{3.526195in}}%
\pgfpathlineto{\pgfqpoint{3.371060in}{3.500260in}}%
\pgfpathlineto{\pgfqpoint{3.371534in}{3.579477in}}%
\pgfpathlineto{\pgfqpoint{3.371913in}{3.629660in}}%
\pgfpathlineto{\pgfqpoint{3.372387in}{3.524246in}}%
\pgfpathlineto{\pgfqpoint{3.372576in}{3.481102in}}%
\pgfpathlineto{\pgfqpoint{3.373050in}{3.597025in}}%
\pgfpathlineto{\pgfqpoint{3.373429in}{3.665463in}}%
\pgfpathlineto{\pgfqpoint{3.373998in}{3.552077in}}%
\pgfpathlineto{\pgfqpoint{3.374282in}{3.510282in}}%
\pgfpathlineto{\pgfqpoint{3.374756in}{3.620292in}}%
\pgfpathlineto{\pgfqpoint{3.374945in}{3.639257in}}%
\pgfpathlineto{\pgfqpoint{3.375419in}{3.566901in}}%
\pgfpathlineto{\pgfqpoint{3.375798in}{3.484591in}}%
\pgfpathlineto{\pgfqpoint{3.376272in}{3.608869in}}%
\pgfpathlineto{\pgfqpoint{3.376556in}{3.664354in}}%
\pgfpathlineto{\pgfqpoint{3.377125in}{3.533061in}}%
\pgfpathlineto{\pgfqpoint{3.377409in}{3.495595in}}%
\pgfpathlineto{\pgfqpoint{3.377883in}{3.597164in}}%
\pgfpathlineto{\pgfqpoint{3.378167in}{3.634896in}}%
\pgfpathlineto{\pgfqpoint{3.378546in}{3.551350in}}%
\pgfpathlineto{\pgfqpoint{3.378831in}{3.487703in}}%
\pgfpathlineto{\pgfqpoint{3.379399in}{3.583095in}}%
\pgfpathlineto{\pgfqpoint{3.379778in}{3.674467in}}%
\pgfpathlineto{\pgfqpoint{3.380252in}{3.517389in}}%
\pgfpathlineto{\pgfqpoint{3.380442in}{3.487865in}}%
\pgfpathlineto{\pgfqpoint{3.381010in}{3.575595in}}%
\pgfpathlineto{\pgfqpoint{3.381294in}{3.610631in}}%
\pgfpathlineto{\pgfqpoint{3.381768in}{3.519875in}}%
\pgfpathlineto{\pgfqpoint{3.382053in}{3.473156in}}%
\pgfpathlineto{\pgfqpoint{3.382621in}{3.577968in}}%
\pgfpathlineto{\pgfqpoint{3.382811in}{3.598095in}}%
\pgfpathlineto{\pgfqpoint{3.383190in}{3.528365in}}%
\pgfpathlineto{\pgfqpoint{3.383664in}{3.396215in}}%
\pgfpathlineto{\pgfqpoint{3.384232in}{3.527727in}}%
\pgfpathlineto{\pgfqpoint{3.384327in}{3.532699in}}%
\pgfpathlineto{\pgfqpoint{3.384706in}{3.498330in}}%
\pgfpathlineto{\pgfqpoint{3.385275in}{3.389964in}}%
\pgfpathlineto{\pgfqpoint{3.385654in}{3.512712in}}%
\pgfpathlineto{\pgfqpoint{3.385938in}{3.551068in}}%
\pgfpathlineto{\pgfqpoint{3.386412in}{3.460650in}}%
\pgfpathlineto{\pgfqpoint{3.386791in}{3.360263in}}%
\pgfpathlineto{\pgfqpoint{3.387359in}{3.500031in}}%
\pgfpathlineto{\pgfqpoint{3.387549in}{3.526299in}}%
\pgfpathlineto{\pgfqpoint{3.388023in}{3.435889in}}%
\pgfpathlineto{\pgfqpoint{3.388402in}{3.365759in}}%
\pgfpathlineto{\pgfqpoint{3.388876in}{3.478374in}}%
\pgfpathlineto{\pgfqpoint{3.389160in}{3.525027in}}%
\pgfpathlineto{\pgfqpoint{3.389634in}{3.398574in}}%
\pgfpathlineto{\pgfqpoint{3.390013in}{3.348553in}}%
\pgfpathlineto{\pgfqpoint{3.390487in}{3.459006in}}%
\pgfpathlineto{\pgfqpoint{3.390771in}{3.492278in}}%
\pgfpathlineto{\pgfqpoint{3.391245in}{3.411877in}}%
\pgfpathlineto{\pgfqpoint{3.391434in}{3.370055in}}%
\pgfpathlineto{\pgfqpoint{3.392003in}{3.493936in}}%
\pgfpathlineto{\pgfqpoint{3.392287in}{3.528834in}}%
\pgfpathlineto{\pgfqpoint{3.392761in}{3.447615in}}%
\pgfpathlineto{\pgfqpoint{3.393045in}{3.390200in}}%
\pgfpathlineto{\pgfqpoint{3.393519in}{3.498113in}}%
\pgfpathlineto{\pgfqpoint{3.393898in}{3.571796in}}%
\pgfpathlineto{\pgfqpoint{3.394372in}{3.457599in}}%
\pgfpathlineto{\pgfqpoint{3.394561in}{3.428518in}}%
\pgfpathlineto{\pgfqpoint{3.395130in}{3.514685in}}%
\pgfpathlineto{\pgfqpoint{3.395414in}{3.548718in}}%
\pgfpathlineto{\pgfqpoint{3.395793in}{3.450026in}}%
\pgfpathlineto{\pgfqpoint{3.396267in}{3.356586in}}%
\pgfpathlineto{\pgfqpoint{3.396741in}{3.469644in}}%
\pgfpathlineto{\pgfqpoint{3.396931in}{3.502463in}}%
\pgfpathlineto{\pgfqpoint{3.397499in}{3.408998in}}%
\pgfpathlineto{\pgfqpoint{3.397878in}{3.357943in}}%
\pgfpathlineto{\pgfqpoint{3.398257in}{3.438131in}}%
\pgfpathlineto{\pgfqpoint{3.398542in}{3.496884in}}%
\pgfpathlineto{\pgfqpoint{3.399110in}{3.375854in}}%
\pgfpathlineto{\pgfqpoint{3.399394in}{3.340055in}}%
\pgfpathlineto{\pgfqpoint{3.399773in}{3.429044in}}%
\pgfpathlineto{\pgfqpoint{3.400153in}{3.527410in}}%
\pgfpathlineto{\pgfqpoint{3.400721in}{3.409984in}}%
\pgfpathlineto{\pgfqpoint{3.401005in}{3.369679in}}%
\pgfpathlineto{\pgfqpoint{3.401479in}{3.479692in}}%
\pgfpathlineto{\pgfqpoint{3.401764in}{3.502311in}}%
\pgfpathlineto{\pgfqpoint{3.402143in}{3.443907in}}%
\pgfpathlineto{\pgfqpoint{3.402616in}{3.362936in}}%
\pgfpathlineto{\pgfqpoint{3.403090in}{3.460259in}}%
\pgfpathlineto{\pgfqpoint{3.404796in}{3.942509in}}%
\pgfpathlineto{\pgfqpoint{3.405270in}{3.805478in}}%
\pgfpathlineto{\pgfqpoint{3.406596in}{3.411054in}}%
\pgfpathlineto{\pgfqpoint{3.407449in}{3.436718in}}%
\pgfpathlineto{\pgfqpoint{3.408018in}{3.565014in}}%
\pgfpathlineto{\pgfqpoint{3.408587in}{3.467503in}}%
\pgfpathlineto{\pgfqpoint{3.408871in}{3.446212in}}%
\pgfpathlineto{\pgfqpoint{3.409250in}{3.525534in}}%
\pgfpathlineto{\pgfqpoint{3.409629in}{3.619451in}}%
\pgfpathlineto{\pgfqpoint{3.410103in}{3.473857in}}%
\pgfpathlineto{\pgfqpoint{3.410387in}{3.428590in}}%
\pgfpathlineto{\pgfqpoint{3.410861in}{3.542094in}}%
\pgfpathlineto{\pgfqpoint{3.411145in}{3.599941in}}%
\pgfpathlineto{\pgfqpoint{3.411714in}{3.479936in}}%
\pgfpathlineto{\pgfqpoint{3.411903in}{3.443630in}}%
\pgfpathlineto{\pgfqpoint{3.412377in}{3.577432in}}%
\pgfpathlineto{\pgfqpoint{3.412661in}{3.635830in}}%
\pgfpathlineto{\pgfqpoint{3.413230in}{3.518820in}}%
\pgfpathlineto{\pgfqpoint{3.413609in}{3.453643in}}%
\pgfpathlineto{\pgfqpoint{3.413988in}{3.563455in}}%
\pgfpathlineto{\pgfqpoint{3.414272in}{3.613227in}}%
\pgfpathlineto{\pgfqpoint{3.414841in}{3.518852in}}%
\pgfpathlineto{\pgfqpoint{3.415125in}{3.481228in}}%
\pgfpathlineto{\pgfqpoint{3.415504in}{3.584368in}}%
\pgfpathlineto{\pgfqpoint{3.415789in}{3.632748in}}%
\pgfpathlineto{\pgfqpoint{3.416357in}{3.518658in}}%
\pgfpathlineto{\pgfqpoint{3.416641in}{3.452817in}}%
\pgfpathlineto{\pgfqpoint{3.417210in}{3.559916in}}%
\pgfpathlineto{\pgfqpoint{3.417494in}{3.593694in}}%
\pgfpathlineto{\pgfqpoint{3.417873in}{3.493681in}}%
\pgfpathlineto{\pgfqpoint{3.418252in}{3.437026in}}%
\pgfpathlineto{\pgfqpoint{3.418726in}{3.516757in}}%
\pgfpathlineto{\pgfqpoint{3.419105in}{3.601756in}}%
\pgfpathlineto{\pgfqpoint{3.419769in}{3.509326in}}%
\pgfpathlineto{\pgfqpoint{3.420622in}{3.724792in}}%
\pgfpathlineto{\pgfqpoint{3.421380in}{3.574290in}}%
\pgfpathlineto{\pgfqpoint{3.421474in}{3.572876in}}%
\pgfpathlineto{\pgfqpoint{3.421569in}{3.585695in}}%
\pgfpathlineto{\pgfqpoint{3.422138in}{3.724791in}}%
\pgfpathlineto{\pgfqpoint{3.422612in}{3.600509in}}%
\pgfpathlineto{\pgfqpoint{3.422896in}{3.538168in}}%
\pgfpathlineto{\pgfqpoint{3.423465in}{3.673764in}}%
\pgfpathlineto{\pgfqpoint{3.423749in}{3.736396in}}%
\pgfpathlineto{\pgfqpoint{3.424317in}{3.598600in}}%
\pgfpathlineto{\pgfqpoint{3.424507in}{3.570666in}}%
\pgfpathlineto{\pgfqpoint{3.424981in}{3.671983in}}%
\pgfpathlineto{\pgfqpoint{3.425265in}{3.704215in}}%
\pgfpathlineto{\pgfqpoint{3.425739in}{3.620289in}}%
\pgfpathlineto{\pgfqpoint{3.426023in}{3.559391in}}%
\pgfpathlineto{\pgfqpoint{3.426592in}{3.663131in}}%
\pgfpathlineto{\pgfqpoint{3.426971in}{3.726527in}}%
\pgfpathlineto{\pgfqpoint{3.427350in}{3.615409in}}%
\pgfpathlineto{\pgfqpoint{3.427729in}{3.549256in}}%
\pgfpathlineto{\pgfqpoint{3.428203in}{3.652731in}}%
\pgfpathlineto{\pgfqpoint{3.428487in}{3.685700in}}%
\pgfpathlineto{\pgfqpoint{3.428866in}{3.584807in}}%
\pgfpathlineto{\pgfqpoint{3.429245in}{3.509133in}}%
\pgfpathlineto{\pgfqpoint{3.429814in}{3.640637in}}%
\pgfpathlineto{\pgfqpoint{3.430003in}{3.680103in}}%
\pgfpathlineto{\pgfqpoint{3.430477in}{3.564142in}}%
\pgfpathlineto{\pgfqpoint{3.430856in}{3.508964in}}%
\pgfpathlineto{\pgfqpoint{3.431330in}{3.596916in}}%
\pgfpathlineto{\pgfqpoint{3.431614in}{3.635417in}}%
\pgfpathlineto{\pgfqpoint{3.432088in}{3.535257in}}%
\pgfpathlineto{\pgfqpoint{3.432372in}{3.486435in}}%
\pgfpathlineto{\pgfqpoint{3.432846in}{3.605505in}}%
\pgfpathlineto{\pgfqpoint{3.433225in}{3.658625in}}%
\pgfpathlineto{\pgfqpoint{3.433604in}{3.563613in}}%
\pgfpathlineto{\pgfqpoint{3.433983in}{3.489635in}}%
\pgfpathlineto{\pgfqpoint{3.434552in}{3.592129in}}%
\pgfpathlineto{\pgfqpoint{3.434741in}{3.619331in}}%
\pgfpathlineto{\pgfqpoint{3.435215in}{3.529451in}}%
\pgfpathlineto{\pgfqpoint{3.435500in}{3.477215in}}%
\pgfpathlineto{\pgfqpoint{3.435973in}{3.576942in}}%
\pgfpathlineto{\pgfqpoint{3.436352in}{3.640893in}}%
\pgfpathlineto{\pgfqpoint{3.436826in}{3.520907in}}%
\pgfpathlineto{\pgfqpoint{3.437111in}{3.469923in}}%
\pgfpathlineto{\pgfqpoint{3.437584in}{3.579262in}}%
\pgfpathlineto{\pgfqpoint{3.437963in}{3.635270in}}%
\pgfpathlineto{\pgfqpoint{3.438532in}{3.553179in}}%
\pgfpathlineto{\pgfqpoint{3.438627in}{3.546933in}}%
\pgfpathlineto{\pgfqpoint{3.438911in}{3.577966in}}%
\pgfpathlineto{\pgfqpoint{3.439385in}{3.677842in}}%
\pgfpathlineto{\pgfqpoint{3.439953in}{3.583822in}}%
\pgfpathlineto{\pgfqpoint{3.440238in}{3.524298in}}%
\pgfpathlineto{\pgfqpoint{3.440806in}{3.629432in}}%
\pgfpathlineto{\pgfqpoint{3.440996in}{3.651300in}}%
\pgfpathlineto{\pgfqpoint{3.441375in}{3.568022in}}%
\pgfpathlineto{\pgfqpoint{3.441849in}{3.477720in}}%
\pgfpathlineto{\pgfqpoint{3.442417in}{3.567944in}}%
\pgfpathlineto{\pgfqpoint{3.442607in}{3.582016in}}%
\pgfpathlineto{\pgfqpoint{3.442891in}{3.527101in}}%
\pgfpathlineto{\pgfqpoint{3.443365in}{3.400370in}}%
\pgfpathlineto{\pgfqpoint{3.443934in}{3.506636in}}%
\pgfpathlineto{\pgfqpoint{3.444218in}{3.546959in}}%
\pgfpathlineto{\pgfqpoint{3.444692in}{3.437140in}}%
\pgfpathlineto{\pgfqpoint{3.444976in}{3.391013in}}%
\pgfpathlineto{\pgfqpoint{3.445545in}{3.505042in}}%
\pgfpathlineto{\pgfqpoint{3.445829in}{3.527150in}}%
\pgfpathlineto{\pgfqpoint{3.446113in}{3.470733in}}%
\pgfpathlineto{\pgfqpoint{3.446492in}{3.358679in}}%
\pgfpathlineto{\pgfqpoint{3.447061in}{3.471692in}}%
\pgfpathlineto{\pgfqpoint{3.447440in}{3.541016in}}%
\pgfpathlineto{\pgfqpoint{3.447914in}{3.424831in}}%
\pgfpathlineto{\pgfqpoint{3.448103in}{3.399745in}}%
\pgfpathlineto{\pgfqpoint{3.448577in}{3.484734in}}%
\pgfpathlineto{\pgfqpoint{3.448767in}{3.502288in}}%
\pgfpathlineto{\pgfqpoint{3.449335in}{3.434352in}}%
\pgfpathlineto{\pgfqpoint{3.449430in}{3.435307in}}%
\pgfpathlineto{\pgfqpoint{3.449904in}{3.661088in}}%
\pgfpathlineto{\pgfqpoint{3.450567in}{4.009382in}}%
\pgfpathlineto{\pgfqpoint{3.451136in}{3.812961in}}%
\pgfpathlineto{\pgfqpoint{3.452368in}{3.374548in}}%
\pgfpathlineto{\pgfqpoint{3.452557in}{3.320398in}}%
\pgfpathlineto{\pgfqpoint{3.453126in}{3.514614in}}%
\pgfpathlineto{\pgfqpoint{3.453694in}{3.640492in}}%
\pgfpathlineto{\pgfqpoint{3.454168in}{3.526774in}}%
\pgfpathlineto{\pgfqpoint{3.454358in}{3.506634in}}%
\pgfpathlineto{\pgfqpoint{3.454831in}{3.593632in}}%
\pgfpathlineto{\pgfqpoint{3.455210in}{3.662358in}}%
\pgfpathlineto{\pgfqpoint{3.455684in}{3.535886in}}%
\pgfpathlineto{\pgfqpoint{3.455874in}{3.504276in}}%
\pgfpathlineto{\pgfqpoint{3.456348in}{3.589650in}}%
\pgfpathlineto{\pgfqpoint{3.456821in}{3.695981in}}%
\pgfpathlineto{\pgfqpoint{3.457295in}{3.551158in}}%
\pgfpathlineto{\pgfqpoint{3.457485in}{3.515373in}}%
\pgfpathlineto{\pgfqpoint{3.457959in}{3.624481in}}%
\pgfpathlineto{\pgfqpoint{3.458338in}{3.685100in}}%
\pgfpathlineto{\pgfqpoint{3.458906in}{3.588640in}}%
\pgfpathlineto{\pgfqpoint{3.459096in}{3.570862in}}%
\pgfpathlineto{\pgfqpoint{3.459475in}{3.641390in}}%
\pgfpathlineto{\pgfqpoint{3.459854in}{3.744475in}}%
\pgfpathlineto{\pgfqpoint{3.460423in}{3.627625in}}%
\pgfpathlineto{\pgfqpoint{3.460707in}{3.585953in}}%
\pgfpathlineto{\pgfqpoint{3.461181in}{3.700540in}}%
\pgfpathlineto{\pgfqpoint{3.461560in}{3.749647in}}%
\pgfpathlineto{\pgfqpoint{3.462033in}{3.652238in}}%
\pgfpathlineto{\pgfqpoint{3.462223in}{3.631426in}}%
\pgfpathlineto{\pgfqpoint{3.462602in}{3.703264in}}%
\pgfpathlineto{\pgfqpoint{3.462981in}{3.795463in}}%
\pgfpathlineto{\pgfqpoint{3.463550in}{3.692785in}}%
\pgfpathlineto{\pgfqpoint{3.463834in}{3.649762in}}%
\pgfpathlineto{\pgfqpoint{3.464308in}{3.739721in}}%
\pgfpathlineto{\pgfqpoint{3.464592in}{3.798738in}}%
\pgfpathlineto{\pgfqpoint{3.465255in}{3.714331in}}%
\pgfpathlineto{\pgfqpoint{3.465445in}{3.691810in}}%
\pgfpathlineto{\pgfqpoint{3.465824in}{3.784295in}}%
\pgfpathlineto{\pgfqpoint{3.466203in}{3.851995in}}%
\pgfpathlineto{\pgfqpoint{3.466772in}{3.728818in}}%
\pgfpathlineto{\pgfqpoint{3.466961in}{3.692226in}}%
\pgfpathlineto{\pgfqpoint{3.467435in}{3.790479in}}%
\pgfpathlineto{\pgfqpoint{3.467719in}{3.842633in}}%
\pgfpathlineto{\pgfqpoint{3.468288in}{3.753014in}}%
\pgfpathlineto{\pgfqpoint{3.468572in}{3.707652in}}%
\pgfpathlineto{\pgfqpoint{3.468951in}{3.810724in}}%
\pgfpathlineto{\pgfqpoint{3.469236in}{3.861747in}}%
\pgfpathlineto{\pgfqpoint{3.469804in}{3.772217in}}%
\pgfpathlineto{\pgfqpoint{3.470183in}{3.710167in}}%
\pgfpathlineto{\pgfqpoint{3.470657in}{3.815552in}}%
\pgfpathlineto{\pgfqpoint{3.470941in}{3.859191in}}%
\pgfpathlineto{\pgfqpoint{3.471415in}{3.770257in}}%
\pgfpathlineto{\pgfqpoint{3.471699in}{3.718906in}}%
\pgfpathlineto{\pgfqpoint{3.472268in}{3.822096in}}%
\pgfpathlineto{\pgfqpoint{3.472458in}{3.835281in}}%
\pgfpathlineto{\pgfqpoint{3.472837in}{3.788057in}}%
\pgfpathlineto{\pgfqpoint{3.473216in}{3.680554in}}%
\pgfpathlineto{\pgfqpoint{3.473879in}{3.789620in}}%
\pgfpathlineto{\pgfqpoint{3.474163in}{3.819923in}}%
\pgfpathlineto{\pgfqpoint{3.474542in}{3.737672in}}%
\pgfpathlineto{\pgfqpoint{3.474827in}{3.668251in}}%
\pgfpathlineto{\pgfqpoint{3.475395in}{3.788609in}}%
\pgfpathlineto{\pgfqpoint{3.475585in}{3.807944in}}%
\pgfpathlineto{\pgfqpoint{3.475964in}{3.717842in}}%
\pgfpathlineto{\pgfqpoint{3.476343in}{3.631637in}}%
\pgfpathlineto{\pgfqpoint{3.476911in}{3.723800in}}%
\pgfpathlineto{\pgfqpoint{3.477196in}{3.765060in}}%
\pgfpathlineto{\pgfqpoint{3.477764in}{3.674338in}}%
\pgfpathlineto{\pgfqpoint{3.478049in}{3.627913in}}%
\pgfpathlineto{\pgfqpoint{3.478522in}{3.742143in}}%
\pgfpathlineto{\pgfqpoint{3.478712in}{3.781586in}}%
\pgfpathlineto{\pgfqpoint{3.479186in}{3.668597in}}%
\pgfpathlineto{\pgfqpoint{3.479565in}{3.605660in}}%
\pgfpathlineto{\pgfqpoint{3.480039in}{3.713641in}}%
\pgfpathlineto{\pgfqpoint{3.480323in}{3.767735in}}%
\pgfpathlineto{\pgfqpoint{3.480892in}{3.650483in}}%
\pgfpathlineto{\pgfqpoint{3.481176in}{3.612817in}}%
\pgfpathlineto{\pgfqpoint{3.481650in}{3.718537in}}%
\pgfpathlineto{\pgfqpoint{3.481934in}{3.751393in}}%
\pgfpathlineto{\pgfqpoint{3.482313in}{3.672944in}}%
\pgfpathlineto{\pgfqpoint{3.482597in}{3.602659in}}%
\pgfpathlineto{\pgfqpoint{3.483166in}{3.712438in}}%
\pgfpathlineto{\pgfqpoint{3.483545in}{3.778525in}}%
\pgfpathlineto{\pgfqpoint{3.484114in}{3.672150in}}%
\pgfpathlineto{\pgfqpoint{3.484398in}{3.647178in}}%
\pgfpathlineto{\pgfqpoint{3.484682in}{3.708298in}}%
\pgfpathlineto{\pgfqpoint{3.485061in}{3.804195in}}%
\pgfpathlineto{\pgfqpoint{3.485630in}{3.676515in}}%
\pgfpathlineto{\pgfqpoint{3.485819in}{3.640193in}}%
\pgfpathlineto{\pgfqpoint{3.486388in}{3.747099in}}%
\pgfpathlineto{\pgfqpoint{3.486577in}{3.761490in}}%
\pgfpathlineto{\pgfqpoint{3.486862in}{3.705739in}}%
\pgfpathlineto{\pgfqpoint{3.487525in}{3.542827in}}%
\pgfpathlineto{\pgfqpoint{3.488283in}{3.587721in}}%
\pgfpathlineto{\pgfqpoint{3.488947in}{3.457270in}}%
\pgfpathlineto{\pgfqpoint{3.489420in}{3.579227in}}%
\pgfpathlineto{\pgfqpoint{3.489799in}{3.687053in}}%
\pgfpathlineto{\pgfqpoint{3.490368in}{3.563667in}}%
\pgfpathlineto{\pgfqpoint{3.490557in}{3.546483in}}%
\pgfpathlineto{\pgfqpoint{3.490937in}{3.608883in}}%
\pgfpathlineto{\pgfqpoint{3.491316in}{3.680428in}}%
\pgfpathlineto{\pgfqpoint{3.491884in}{3.568775in}}%
\pgfpathlineto{\pgfqpoint{3.492168in}{3.544931in}}%
\pgfpathlineto{\pgfqpoint{3.492548in}{3.625376in}}%
\pgfpathlineto{\pgfqpoint{3.492832in}{3.678250in}}%
\pgfpathlineto{\pgfqpoint{3.493400in}{3.562261in}}%
\pgfpathlineto{\pgfqpoint{3.493779in}{3.508698in}}%
\pgfpathlineto{\pgfqpoint{3.494253in}{3.603144in}}%
\pgfpathlineto{\pgfqpoint{3.494443in}{3.639922in}}%
\pgfpathlineto{\pgfqpoint{3.495011in}{3.518740in}}%
\pgfpathlineto{\pgfqpoint{3.495296in}{3.454880in}}%
\pgfpathlineto{\pgfqpoint{3.495675in}{3.566835in}}%
\pgfpathlineto{\pgfqpoint{3.497475in}{3.998168in}}%
\pgfpathlineto{\pgfqpoint{3.497665in}{3.990725in}}%
\pgfpathlineto{\pgfqpoint{3.498139in}{3.739573in}}%
\pgfpathlineto{\pgfqpoint{3.498707in}{3.410308in}}%
\pgfpathlineto{\pgfqpoint{3.499276in}{3.637425in}}%
\pgfpathlineto{\pgfqpoint{3.499371in}{3.643809in}}%
\pgfpathlineto{\pgfqpoint{3.499560in}{3.610732in}}%
\pgfpathlineto{\pgfqpoint{3.499939in}{3.519067in}}%
\pgfpathlineto{\pgfqpoint{3.500508in}{3.632657in}}%
\pgfpathlineto{\pgfqpoint{3.500792in}{3.667628in}}%
\pgfpathlineto{\pgfqpoint{3.501266in}{3.601322in}}%
\pgfpathlineto{\pgfqpoint{3.501550in}{3.567222in}}%
\pgfpathlineto{\pgfqpoint{3.502024in}{3.635709in}}%
\pgfpathlineto{\pgfqpoint{3.502403in}{3.697017in}}%
\pgfpathlineto{\pgfqpoint{3.502877in}{3.586891in}}%
\pgfpathlineto{\pgfqpoint{3.503066in}{3.552729in}}%
\pgfpathlineto{\pgfqpoint{3.503635in}{3.655427in}}%
\pgfpathlineto{\pgfqpoint{3.504014in}{3.702324in}}%
\pgfpathlineto{\pgfqpoint{3.504393in}{3.620576in}}%
\pgfpathlineto{\pgfqpoint{3.504677in}{3.560836in}}%
\pgfpathlineto{\pgfqpoint{3.505246in}{3.675978in}}%
\pgfpathlineto{\pgfqpoint{3.505530in}{3.705433in}}%
\pgfpathlineto{\pgfqpoint{3.505909in}{3.615742in}}%
\pgfpathlineto{\pgfqpoint{3.506288in}{3.543788in}}%
\pgfpathlineto{\pgfqpoint{3.506762in}{3.645706in}}%
\pgfpathlineto{\pgfqpoint{3.507141in}{3.716027in}}%
\pgfpathlineto{\pgfqpoint{3.507615in}{3.610722in}}%
\pgfpathlineto{\pgfqpoint{3.507805in}{3.591258in}}%
\pgfpathlineto{\pgfqpoint{3.508373in}{3.666797in}}%
\pgfpathlineto{\pgfqpoint{3.508657in}{3.698419in}}%
\pgfpathlineto{\pgfqpoint{3.509131in}{3.619248in}}%
\pgfpathlineto{\pgfqpoint{3.509416in}{3.562261in}}%
\pgfpathlineto{\pgfqpoint{3.509984in}{3.676271in}}%
\pgfpathlineto{\pgfqpoint{3.510268in}{3.702847in}}%
\pgfpathlineto{\pgfqpoint{3.510647in}{3.637124in}}%
\pgfpathlineto{\pgfqpoint{3.511027in}{3.574986in}}%
\pgfpathlineto{\pgfqpoint{3.511595in}{3.680616in}}%
\pgfpathlineto{\pgfqpoint{3.511785in}{3.699662in}}%
\pgfpathlineto{\pgfqpoint{3.512164in}{3.616839in}}%
\pgfpathlineto{\pgfqpoint{3.512638in}{3.566758in}}%
\pgfpathlineto{\pgfqpoint{3.513017in}{3.654252in}}%
\pgfpathlineto{\pgfqpoint{3.513301in}{3.712368in}}%
\pgfpathlineto{\pgfqpoint{3.513964in}{3.604943in}}%
\pgfpathlineto{\pgfqpoint{3.514154in}{3.572095in}}%
\pgfpathlineto{\pgfqpoint{3.514722in}{3.656407in}}%
\pgfpathlineto{\pgfqpoint{3.515007in}{3.691291in}}%
\pgfpathlineto{\pgfqpoint{3.515386in}{3.608798in}}%
\pgfpathlineto{\pgfqpoint{3.515670in}{3.561507in}}%
\pgfpathlineto{\pgfqpoint{3.516239in}{3.670402in}}%
\pgfpathlineto{\pgfqpoint{3.516523in}{3.706890in}}%
\pgfpathlineto{\pgfqpoint{3.516997in}{3.611113in}}%
\pgfpathlineto{\pgfqpoint{3.517376in}{3.560118in}}%
\pgfpathlineto{\pgfqpoint{3.517755in}{3.636464in}}%
\pgfpathlineto{\pgfqpoint{3.518039in}{3.692589in}}%
\pgfpathlineto{\pgfqpoint{3.518608in}{3.598060in}}%
\pgfpathlineto{\pgfqpoint{3.518892in}{3.554339in}}%
\pgfpathlineto{\pgfqpoint{3.519366in}{3.667055in}}%
\pgfpathlineto{\pgfqpoint{3.519650in}{3.688178in}}%
\pgfpathlineto{\pgfqpoint{3.520029in}{3.605656in}}%
\pgfpathlineto{\pgfqpoint{3.520408in}{3.511104in}}%
\pgfpathlineto{\pgfqpoint{3.521072in}{3.606087in}}%
\pgfpathlineto{\pgfqpoint{3.521261in}{3.622072in}}%
\pgfpathlineto{\pgfqpoint{3.521640in}{3.529930in}}%
\pgfpathlineto{\pgfqpoint{3.521924in}{3.484915in}}%
\pgfpathlineto{\pgfqpoint{3.522588in}{3.564285in}}%
\pgfpathlineto{\pgfqpoint{3.522777in}{3.582688in}}%
\pgfpathlineto{\pgfqpoint{3.523251in}{3.506086in}}%
\pgfpathlineto{\pgfqpoint{3.523630in}{3.413274in}}%
\pgfpathlineto{\pgfqpoint{3.524199in}{3.533620in}}%
\pgfpathlineto{\pgfqpoint{3.524294in}{3.539757in}}%
\pgfpathlineto{\pgfqpoint{3.524673in}{3.503983in}}%
\pgfpathlineto{\pgfqpoint{3.525241in}{3.400882in}}%
\pgfpathlineto{\pgfqpoint{3.525620in}{3.501520in}}%
\pgfpathlineto{\pgfqpoint{3.525905in}{3.550823in}}%
\pgfpathlineto{\pgfqpoint{3.526378in}{3.445099in}}%
\pgfpathlineto{\pgfqpoint{3.526757in}{3.397957in}}%
\pgfpathlineto{\pgfqpoint{3.527231in}{3.474770in}}%
\pgfpathlineto{\pgfqpoint{3.527515in}{3.545585in}}%
\pgfpathlineto{\pgfqpoint{3.528084in}{3.421642in}}%
\pgfpathlineto{\pgfqpoint{3.528179in}{3.413084in}}%
\pgfpathlineto{\pgfqpoint{3.528558in}{3.456853in}}%
\pgfpathlineto{\pgfqpoint{3.529126in}{3.531624in}}%
\pgfpathlineto{\pgfqpoint{3.529506in}{3.455653in}}%
\pgfpathlineto{\pgfqpoint{3.529885in}{3.381927in}}%
\pgfpathlineto{\pgfqpoint{3.530358in}{3.483159in}}%
\pgfpathlineto{\pgfqpoint{3.530737in}{3.563126in}}%
\pgfpathlineto{\pgfqpoint{3.531401in}{3.464373in}}%
\pgfpathlineto{\pgfqpoint{3.531496in}{3.463387in}}%
\pgfpathlineto{\pgfqpoint{3.531590in}{3.469671in}}%
\pgfpathlineto{\pgfqpoint{3.532159in}{3.575399in}}%
\pgfpathlineto{\pgfqpoint{3.532728in}{3.482595in}}%
\pgfpathlineto{\pgfqpoint{3.533012in}{3.445193in}}%
\pgfpathlineto{\pgfqpoint{3.533486in}{3.530486in}}%
\pgfpathlineto{\pgfqpoint{3.533770in}{3.576271in}}%
\pgfpathlineto{\pgfqpoint{3.534244in}{3.482115in}}%
\pgfpathlineto{\pgfqpoint{3.534718in}{3.416580in}}%
\pgfpathlineto{\pgfqpoint{3.535191in}{3.498865in}}%
\pgfpathlineto{\pgfqpoint{3.535286in}{3.504592in}}%
\pgfpathlineto{\pgfqpoint{3.535570in}{3.463077in}}%
\pgfpathlineto{\pgfqpoint{3.536234in}{3.350466in}}%
\pgfpathlineto{\pgfqpoint{3.536613in}{3.443769in}}%
\pgfpathlineto{\pgfqpoint{3.536897in}{3.493652in}}%
\pgfpathlineto{\pgfqpoint{3.537466in}{3.396773in}}%
\pgfpathlineto{\pgfqpoint{3.537750in}{3.368815in}}%
\pgfpathlineto{\pgfqpoint{3.538224in}{3.439653in}}%
\pgfpathlineto{\pgfqpoint{3.538508in}{3.483120in}}%
\pgfpathlineto{\pgfqpoint{3.538982in}{3.392142in}}%
\pgfpathlineto{\pgfqpoint{3.539266in}{3.361158in}}%
\pgfpathlineto{\pgfqpoint{3.539740in}{3.463412in}}%
\pgfpathlineto{\pgfqpoint{3.540024in}{3.523119in}}%
\pgfpathlineto{\pgfqpoint{3.540688in}{3.416314in}}%
\pgfpathlineto{\pgfqpoint{3.540972in}{3.384680in}}%
\pgfpathlineto{\pgfqpoint{3.541446in}{3.467835in}}%
\pgfpathlineto{\pgfqpoint{3.541541in}{3.471749in}}%
\pgfpathlineto{\pgfqpoint{3.541920in}{3.443967in}}%
\pgfpathlineto{\pgfqpoint{3.542109in}{3.433862in}}%
\pgfpathlineto{\pgfqpoint{3.542299in}{3.461445in}}%
\pgfpathlineto{\pgfqpoint{3.543246in}{3.972060in}}%
\pgfpathlineto{\pgfqpoint{3.544194in}{3.764910in}}%
\pgfpathlineto{\pgfqpoint{3.544478in}{3.736008in}}%
\pgfpathlineto{\pgfqpoint{3.545331in}{3.328657in}}%
\pgfpathlineto{\pgfqpoint{3.546089in}{3.550362in}}%
\pgfpathlineto{\pgfqpoint{3.546374in}{3.595191in}}%
\pgfpathlineto{\pgfqpoint{3.546942in}{3.486405in}}%
\pgfpathlineto{\pgfqpoint{3.547226in}{3.476097in}}%
\pgfpathlineto{\pgfqpoint{3.547511in}{3.512510in}}%
\pgfpathlineto{\pgfqpoint{3.547985in}{3.597235in}}%
\pgfpathlineto{\pgfqpoint{3.548553in}{3.521657in}}%
\pgfpathlineto{\pgfqpoint{3.548743in}{3.502373in}}%
\pgfpathlineto{\pgfqpoint{3.549122in}{3.572655in}}%
\pgfpathlineto{\pgfqpoint{3.549501in}{3.644645in}}%
\pgfpathlineto{\pgfqpoint{3.549975in}{3.539570in}}%
\pgfpathlineto{\pgfqpoint{3.550259in}{3.495108in}}%
\pgfpathlineto{\pgfqpoint{3.550827in}{3.591628in}}%
\pgfpathlineto{\pgfqpoint{3.551112in}{3.631338in}}%
\pgfpathlineto{\pgfqpoint{3.551680in}{3.537643in}}%
\pgfpathlineto{\pgfqpoint{3.551870in}{3.517297in}}%
\pgfpathlineto{\pgfqpoint{3.552344in}{3.601072in}}%
\pgfpathlineto{\pgfqpoint{3.552628in}{3.628319in}}%
\pgfpathlineto{\pgfqpoint{3.553102in}{3.553658in}}%
\pgfpathlineto{\pgfqpoint{3.553386in}{3.490157in}}%
\pgfpathlineto{\pgfqpoint{3.553955in}{3.601930in}}%
\pgfpathlineto{\pgfqpoint{3.554239in}{3.642675in}}%
\pgfpathlineto{\pgfqpoint{3.554713in}{3.555475in}}%
\pgfpathlineto{\pgfqpoint{3.554997in}{3.497811in}}%
\pgfpathlineto{\pgfqpoint{3.555660in}{3.587490in}}%
\pgfpathlineto{\pgfqpoint{3.555755in}{3.592812in}}%
\pgfpathlineto{\pgfqpoint{3.556039in}{3.550095in}}%
\pgfpathlineto{\pgfqpoint{3.556608in}{3.426509in}}%
\pgfpathlineto{\pgfqpoint{3.557177in}{3.538628in}}%
\pgfpathlineto{\pgfqpoint{3.557935in}{3.507277in}}%
\pgfpathlineto{\pgfqpoint{3.558882in}{3.702399in}}%
\pgfpathlineto{\pgfqpoint{3.558977in}{3.707251in}}%
\pgfpathlineto{\pgfqpoint{3.559167in}{3.686285in}}%
\pgfpathlineto{\pgfqpoint{3.559735in}{3.578344in}}%
\pgfpathlineto{\pgfqpoint{3.560304in}{3.675291in}}%
\pgfpathlineto{\pgfqpoint{3.560493in}{3.698708in}}%
\pgfpathlineto{\pgfqpoint{3.560967in}{3.634586in}}%
\pgfpathlineto{\pgfqpoint{3.561346in}{3.569795in}}%
\pgfpathlineto{\pgfqpoint{3.561820in}{3.659152in}}%
\pgfpathlineto{\pgfqpoint{3.562104in}{3.699464in}}%
\pgfpathlineto{\pgfqpoint{3.562578in}{3.616313in}}%
\pgfpathlineto{\pgfqpoint{3.562957in}{3.560811in}}%
\pgfpathlineto{\pgfqpoint{3.563336in}{3.646745in}}%
\pgfpathlineto{\pgfqpoint{3.563621in}{3.698683in}}%
\pgfpathlineto{\pgfqpoint{3.564189in}{3.607496in}}%
\pgfpathlineto{\pgfqpoint{3.564473in}{3.552337in}}%
\pgfpathlineto{\pgfqpoint{3.565042in}{3.661051in}}%
\pgfpathlineto{\pgfqpoint{3.565137in}{3.669749in}}%
\pgfpathlineto{\pgfqpoint{3.565516in}{3.618452in}}%
\pgfpathlineto{\pgfqpoint{3.566084in}{3.521724in}}%
\pgfpathlineto{\pgfqpoint{3.566464in}{3.621247in}}%
\pgfpathlineto{\pgfqpoint{3.566653in}{3.647738in}}%
\pgfpathlineto{\pgfqpoint{3.567222in}{3.568693in}}%
\pgfpathlineto{\pgfqpoint{3.567601in}{3.501530in}}%
\pgfpathlineto{\pgfqpoint{3.568264in}{3.584141in}}%
\pgfpathlineto{\pgfqpoint{3.568454in}{3.594162in}}%
\pgfpathlineto{\pgfqpoint{3.568738in}{3.541824in}}%
\pgfpathlineto{\pgfqpoint{3.569212in}{3.460264in}}%
\pgfpathlineto{\pgfqpoint{3.569686in}{3.539519in}}%
\pgfpathlineto{\pgfqpoint{3.569970in}{3.582888in}}%
\pgfpathlineto{\pgfqpoint{3.570444in}{3.471036in}}%
\pgfpathlineto{\pgfqpoint{3.570823in}{3.432130in}}%
\pgfpathlineto{\pgfqpoint{3.571297in}{3.511806in}}%
\pgfpathlineto{\pgfqpoint{3.571581in}{3.540122in}}%
\pgfpathlineto{\pgfqpoint{3.572055in}{3.464773in}}%
\pgfpathlineto{\pgfqpoint{3.572339in}{3.438747in}}%
\pgfpathlineto{\pgfqpoint{3.572813in}{3.516221in}}%
\pgfpathlineto{\pgfqpoint{3.573097in}{3.565671in}}%
\pgfpathlineto{\pgfqpoint{3.573666in}{3.460052in}}%
\pgfpathlineto{\pgfqpoint{3.573855in}{3.447025in}}%
\pgfpathlineto{\pgfqpoint{3.574329in}{3.506708in}}%
\pgfpathlineto{\pgfqpoint{3.574613in}{3.552269in}}%
\pgfpathlineto{\pgfqpoint{3.575182in}{3.470577in}}%
\pgfpathlineto{\pgfqpoint{3.575466in}{3.417662in}}%
\pgfpathlineto{\pgfqpoint{3.575940in}{3.543141in}}%
\pgfpathlineto{\pgfqpoint{3.576224in}{3.585453in}}%
\pgfpathlineto{\pgfqpoint{3.576698in}{3.483532in}}%
\pgfpathlineto{\pgfqpoint{3.576888in}{3.461691in}}%
\pgfpathlineto{\pgfqpoint{3.577456in}{3.531016in}}%
\pgfpathlineto{\pgfqpoint{3.577835in}{3.579494in}}%
\pgfpathlineto{\pgfqpoint{3.578404in}{3.501627in}}%
\pgfpathlineto{\pgfqpoint{3.578688in}{3.473442in}}%
\pgfpathlineto{\pgfqpoint{3.579067in}{3.554975in}}%
\pgfpathlineto{\pgfqpoint{3.579351in}{3.602682in}}%
\pgfpathlineto{\pgfqpoint{3.579825in}{3.510647in}}%
\pgfpathlineto{\pgfqpoint{3.580299in}{3.412825in}}%
\pgfpathlineto{\pgfqpoint{3.580868in}{3.507332in}}%
\pgfpathlineto{\pgfqpoint{3.580962in}{3.511280in}}%
\pgfpathlineto{\pgfqpoint{3.581152in}{3.494228in}}%
\pgfpathlineto{\pgfqpoint{3.581721in}{3.383497in}}%
\pgfpathlineto{\pgfqpoint{3.582289in}{3.458543in}}%
\pgfpathlineto{\pgfqpoint{3.582479in}{3.477107in}}%
\pgfpathlineto{\pgfqpoint{3.582953in}{3.416763in}}%
\pgfpathlineto{\pgfqpoint{3.583332in}{3.350384in}}%
\pgfpathlineto{\pgfqpoint{3.583805in}{3.445401in}}%
\pgfpathlineto{\pgfqpoint{3.584090in}{3.495758in}}%
\pgfpathlineto{\pgfqpoint{3.584658in}{3.403975in}}%
\pgfpathlineto{\pgfqpoint{3.584943in}{3.371310in}}%
\pgfpathlineto{\pgfqpoint{3.585322in}{3.445271in}}%
\pgfpathlineto{\pgfqpoint{3.585701in}{3.500826in}}%
\pgfpathlineto{\pgfqpoint{3.586174in}{3.403801in}}%
\pgfpathlineto{\pgfqpoint{3.586364in}{3.379177in}}%
\pgfpathlineto{\pgfqpoint{3.586933in}{3.453744in}}%
\pgfpathlineto{\pgfqpoint{3.587312in}{3.535539in}}%
\pgfpathlineto{\pgfqpoint{3.587785in}{3.417771in}}%
\pgfpathlineto{\pgfqpoint{3.588070in}{3.351555in}}%
\pgfpathlineto{\pgfqpoint{3.588544in}{3.514238in}}%
\pgfpathlineto{\pgfqpoint{3.590249in}{3.897656in}}%
\pgfpathlineto{\pgfqpoint{3.590439in}{3.886074in}}%
\pgfpathlineto{\pgfqpoint{3.591007in}{3.536998in}}%
\pgfpathlineto{\pgfqpoint{3.591387in}{3.289270in}}%
\pgfpathlineto{\pgfqpoint{3.592050in}{3.516576in}}%
\pgfpathlineto{\pgfqpoint{3.592145in}{3.524262in}}%
\pgfpathlineto{\pgfqpoint{3.592429in}{3.479416in}}%
\pgfpathlineto{\pgfqpoint{3.592713in}{3.435300in}}%
\pgfpathlineto{\pgfqpoint{3.593282in}{3.528229in}}%
\pgfpathlineto{\pgfqpoint{3.593566in}{3.576040in}}%
\pgfpathlineto{\pgfqpoint{3.594135in}{3.469218in}}%
\pgfpathlineto{\pgfqpoint{3.594324in}{3.451681in}}%
\pgfpathlineto{\pgfqpoint{3.594703in}{3.508972in}}%
\pgfpathlineto{\pgfqpoint{3.595082in}{3.572895in}}%
\pgfpathlineto{\pgfqpoint{3.595651in}{3.486252in}}%
\pgfpathlineto{\pgfqpoint{3.595935in}{3.454771in}}%
\pgfpathlineto{\pgfqpoint{3.596314in}{3.541025in}}%
\pgfpathlineto{\pgfqpoint{3.596693in}{3.592769in}}%
\pgfpathlineto{\pgfqpoint{3.597262in}{3.504033in}}%
\pgfpathlineto{\pgfqpoint{3.597451in}{3.479992in}}%
\pgfpathlineto{\pgfqpoint{3.598020in}{3.556295in}}%
\pgfpathlineto{\pgfqpoint{3.598304in}{3.585612in}}%
\pgfpathlineto{\pgfqpoint{3.598778in}{3.516227in}}%
\pgfpathlineto{\pgfqpoint{3.598968in}{3.496348in}}%
\pgfpathlineto{\pgfqpoint{3.599441in}{3.549455in}}%
\pgfpathlineto{\pgfqpoint{3.599821in}{3.618198in}}%
\pgfpathlineto{\pgfqpoint{3.600389in}{3.513685in}}%
\pgfpathlineto{\pgfqpoint{3.600579in}{3.490155in}}%
\pgfpathlineto{\pgfqpoint{3.601052in}{3.564352in}}%
\pgfpathlineto{\pgfqpoint{3.601432in}{3.613264in}}%
\pgfpathlineto{\pgfqpoint{3.602000in}{3.538827in}}%
\pgfpathlineto{\pgfqpoint{3.602190in}{3.530643in}}%
\pgfpathlineto{\pgfqpoint{3.602569in}{3.568108in}}%
\pgfpathlineto{\pgfqpoint{3.602948in}{3.642072in}}%
\pgfpathlineto{\pgfqpoint{3.603516in}{3.553973in}}%
\pgfpathlineto{\pgfqpoint{3.603895in}{3.522833in}}%
\pgfpathlineto{\pgfqpoint{3.604180in}{3.578971in}}%
\pgfpathlineto{\pgfqpoint{3.604559in}{3.645618in}}%
\pgfpathlineto{\pgfqpoint{3.605127in}{3.552992in}}%
\pgfpathlineto{\pgfqpoint{3.605317in}{3.532086in}}%
\pgfpathlineto{\pgfqpoint{3.605791in}{3.609470in}}%
\pgfpathlineto{\pgfqpoint{3.606075in}{3.648222in}}%
\pgfpathlineto{\pgfqpoint{3.606549in}{3.561600in}}%
\pgfpathlineto{\pgfqpoint{3.606833in}{3.528292in}}%
\pgfpathlineto{\pgfqpoint{3.607402in}{3.577099in}}%
\pgfpathlineto{\pgfqpoint{3.607781in}{3.638310in}}%
\pgfpathlineto{\pgfqpoint{3.608349in}{3.536475in}}%
\pgfpathlineto{\pgfqpoint{3.608444in}{3.533653in}}%
\pgfpathlineto{\pgfqpoint{3.608728in}{3.557629in}}%
\pgfpathlineto{\pgfqpoint{3.609202in}{3.636326in}}%
\pgfpathlineto{\pgfqpoint{3.609771in}{3.561771in}}%
\pgfpathlineto{\pgfqpoint{3.610055in}{3.529813in}}%
\pgfpathlineto{\pgfqpoint{3.610529in}{3.593099in}}%
\pgfpathlineto{\pgfqpoint{3.610813in}{3.644096in}}%
\pgfpathlineto{\pgfqpoint{3.611382in}{3.545201in}}%
\pgfpathlineto{\pgfqpoint{3.611666in}{3.525839in}}%
\pgfpathlineto{\pgfqpoint{3.612140in}{3.569975in}}%
\pgfpathlineto{\pgfqpoint{3.612424in}{3.596029in}}%
\pgfpathlineto{\pgfqpoint{3.612898in}{3.518164in}}%
\pgfpathlineto{\pgfqpoint{3.613277in}{3.474675in}}%
\pgfpathlineto{\pgfqpoint{3.613656in}{3.549015in}}%
\pgfpathlineto{\pgfqpoint{3.613846in}{3.582789in}}%
\pgfpathlineto{\pgfqpoint{3.614509in}{3.509331in}}%
\pgfpathlineto{\pgfqpoint{3.614888in}{3.443796in}}%
\pgfpathlineto{\pgfqpoint{3.615457in}{3.539284in}}%
\pgfpathlineto{\pgfqpoint{3.615646in}{3.535484in}}%
\pgfpathlineto{\pgfqpoint{3.616404in}{3.436036in}}%
\pgfpathlineto{\pgfqpoint{3.616878in}{3.515152in}}%
\pgfpathlineto{\pgfqpoint{3.617162in}{3.537748in}}%
\pgfpathlineto{\pgfqpoint{3.617636in}{3.483120in}}%
\pgfpathlineto{\pgfqpoint{3.617920in}{3.439686in}}%
\pgfpathlineto{\pgfqpoint{3.618489in}{3.531088in}}%
\pgfpathlineto{\pgfqpoint{3.618679in}{3.540167in}}%
\pgfpathlineto{\pgfqpoint{3.619058in}{3.499783in}}%
\pgfpathlineto{\pgfqpoint{3.619437in}{3.431493in}}%
\pgfpathlineto{\pgfqpoint{3.620005in}{3.511511in}}%
\pgfpathlineto{\pgfqpoint{3.620384in}{3.544898in}}%
\pgfpathlineto{\pgfqpoint{3.620763in}{3.485131in}}%
\pgfpathlineto{\pgfqpoint{3.621048in}{3.449407in}}%
\pgfpathlineto{\pgfqpoint{3.621521in}{3.512969in}}%
\pgfpathlineto{\pgfqpoint{3.621806in}{3.561452in}}%
\pgfpathlineto{\pgfqpoint{3.622374in}{3.466264in}}%
\pgfpathlineto{\pgfqpoint{3.622469in}{3.460806in}}%
\pgfpathlineto{\pgfqpoint{3.622753in}{3.489141in}}%
\pgfpathlineto{\pgfqpoint{3.623512in}{3.613176in}}%
\pgfpathlineto{\pgfqpoint{3.623891in}{3.528006in}}%
\pgfpathlineto{\pgfqpoint{3.624080in}{3.499970in}}%
\pgfpathlineto{\pgfqpoint{3.624649in}{3.570559in}}%
\pgfpathlineto{\pgfqpoint{3.625028in}{3.596509in}}%
\pgfpathlineto{\pgfqpoint{3.625407in}{3.547136in}}%
\pgfpathlineto{\pgfqpoint{3.625881in}{3.506084in}}%
\pgfpathlineto{\pgfqpoint{3.626260in}{3.555478in}}%
\pgfpathlineto{\pgfqpoint{3.626544in}{3.590548in}}%
\pgfpathlineto{\pgfqpoint{3.626923in}{3.500200in}}%
\pgfpathlineto{\pgfqpoint{3.627492in}{3.412641in}}%
\pgfpathlineto{\pgfqpoint{3.628155in}{3.461908in}}%
\pgfpathlineto{\pgfqpoint{3.630524in}{3.259797in}}%
\pgfpathlineto{\pgfqpoint{3.630714in}{3.281156in}}%
\pgfpathlineto{\pgfqpoint{3.631282in}{3.412015in}}%
\pgfpathlineto{\pgfqpoint{3.632135in}{3.368548in}}%
\pgfpathlineto{\pgfqpoint{3.632893in}{3.486607in}}%
\pgfpathlineto{\pgfqpoint{3.633367in}{3.399963in}}%
\pgfpathlineto{\pgfqpoint{3.633557in}{3.379370in}}%
\pgfpathlineto{\pgfqpoint{3.634409in}{3.409093in}}%
\pgfpathlineto{\pgfqpoint{3.634978in}{3.476355in}}%
\pgfpathlineto{\pgfqpoint{3.636020in}{3.913596in}}%
\pgfpathlineto{\pgfqpoint{3.636684in}{3.733898in}}%
\pgfpathlineto{\pgfqpoint{3.637347in}{3.561151in}}%
\pgfpathlineto{\pgfqpoint{3.637821in}{3.311409in}}%
\pgfpathlineto{\pgfqpoint{3.638579in}{3.435768in}}%
\pgfpathlineto{\pgfqpoint{3.639242in}{3.516696in}}%
\pgfpathlineto{\pgfqpoint{3.639621in}{3.448968in}}%
\pgfpathlineto{\pgfqpoint{3.639906in}{3.423337in}}%
\pgfpathlineto{\pgfqpoint{3.640380in}{3.472579in}}%
\pgfpathlineto{\pgfqpoint{3.640759in}{3.530418in}}%
\pgfpathlineto{\pgfqpoint{3.641232in}{3.452089in}}%
\pgfpathlineto{\pgfqpoint{3.641517in}{3.419529in}}%
\pgfpathlineto{\pgfqpoint{3.641991in}{3.494495in}}%
\pgfpathlineto{\pgfqpoint{3.642370in}{3.522730in}}%
\pgfpathlineto{\pgfqpoint{3.642749in}{3.467231in}}%
\pgfpathlineto{\pgfqpoint{3.643033in}{3.427945in}}%
\pgfpathlineto{\pgfqpoint{3.643602in}{3.513357in}}%
\pgfpathlineto{\pgfqpoint{3.643886in}{3.543581in}}%
\pgfpathlineto{\pgfqpoint{3.644360in}{3.469339in}}%
\pgfpathlineto{\pgfqpoint{3.644644in}{3.441356in}}%
\pgfpathlineto{\pgfqpoint{3.645118in}{3.523176in}}%
\pgfpathlineto{\pgfqpoint{3.645402in}{3.554484in}}%
\pgfpathlineto{\pgfqpoint{3.645971in}{3.492464in}}%
\pgfpathlineto{\pgfqpoint{3.646255in}{3.466235in}}%
\pgfpathlineto{\pgfqpoint{3.646634in}{3.527229in}}%
\pgfpathlineto{\pgfqpoint{3.647108in}{3.581269in}}%
\pgfpathlineto{\pgfqpoint{3.647582in}{3.523851in}}%
\pgfpathlineto{\pgfqpoint{3.647866in}{3.498538in}}%
\pgfpathlineto{\pgfqpoint{3.648245in}{3.557533in}}%
\pgfpathlineto{\pgfqpoint{3.648529in}{3.598138in}}%
\pgfpathlineto{\pgfqpoint{3.649098in}{3.533137in}}%
\pgfpathlineto{\pgfqpoint{3.649382in}{3.501483in}}%
\pgfpathlineto{\pgfqpoint{3.649856in}{3.580213in}}%
\pgfpathlineto{\pgfqpoint{3.650140in}{3.604232in}}%
\pgfpathlineto{\pgfqpoint{3.650614in}{3.554575in}}%
\pgfpathlineto{\pgfqpoint{3.650993in}{3.517412in}}%
\pgfpathlineto{\pgfqpoint{3.651467in}{3.572293in}}%
\pgfpathlineto{\pgfqpoint{3.651751in}{3.594043in}}%
\pgfpathlineto{\pgfqpoint{3.652225in}{3.527874in}}%
\pgfpathlineto{\pgfqpoint{3.652415in}{3.513747in}}%
\pgfpathlineto{\pgfqpoint{3.652983in}{3.560456in}}%
\pgfpathlineto{\pgfqpoint{3.653362in}{3.598023in}}%
\pgfpathlineto{\pgfqpoint{3.653741in}{3.533456in}}%
\pgfpathlineto{\pgfqpoint{3.654026in}{3.488284in}}%
\pgfpathlineto{\pgfqpoint{3.654594in}{3.567153in}}%
\pgfpathlineto{\pgfqpoint{3.654878in}{3.583744in}}%
\pgfpathlineto{\pgfqpoint{3.655352in}{3.536942in}}%
\pgfpathlineto{\pgfqpoint{3.655731in}{3.515297in}}%
\pgfpathlineto{\pgfqpoint{3.656110in}{3.551096in}}%
\pgfpathlineto{\pgfqpoint{3.656489in}{3.610753in}}%
\pgfpathlineto{\pgfqpoint{3.657058in}{3.525927in}}%
\pgfpathlineto{\pgfqpoint{3.657342in}{3.505548in}}%
\pgfpathlineto{\pgfqpoint{3.657721in}{3.559259in}}%
\pgfpathlineto{\pgfqpoint{3.658006in}{3.593267in}}%
\pgfpathlineto{\pgfqpoint{3.658479in}{3.529647in}}%
\pgfpathlineto{\pgfqpoint{3.658859in}{3.501998in}}%
\pgfpathlineto{\pgfqpoint{3.659332in}{3.541064in}}%
\pgfpathlineto{\pgfqpoint{3.659617in}{3.571172in}}%
\pgfpathlineto{\pgfqpoint{3.660090in}{3.487213in}}%
\pgfpathlineto{\pgfqpoint{3.660470in}{3.444412in}}%
\pgfpathlineto{\pgfqpoint{3.660943in}{3.513627in}}%
\pgfpathlineto{\pgfqpoint{3.661038in}{3.516237in}}%
\pgfpathlineto{\pgfqpoint{3.661322in}{3.498616in}}%
\pgfpathlineto{\pgfqpoint{3.661986in}{3.424201in}}%
\pgfpathlineto{\pgfqpoint{3.662554in}{3.484908in}}%
\pgfpathlineto{\pgfqpoint{3.662744in}{3.495233in}}%
\pgfpathlineto{\pgfqpoint{3.663123in}{3.457661in}}%
\pgfpathlineto{\pgfqpoint{3.663597in}{3.405242in}}%
\pgfpathlineto{\pgfqpoint{3.664071in}{3.469006in}}%
\pgfpathlineto{\pgfqpoint{3.664355in}{3.486797in}}%
\pgfpathlineto{\pgfqpoint{3.664734in}{3.439669in}}%
\pgfpathlineto{\pgfqpoint{3.665113in}{3.392436in}}%
\pgfpathlineto{\pgfqpoint{3.665587in}{3.473142in}}%
\pgfpathlineto{\pgfqpoint{3.665776in}{3.485765in}}%
\pgfpathlineto{\pgfqpoint{3.666250in}{3.427979in}}%
\pgfpathlineto{\pgfqpoint{3.666534in}{3.401830in}}%
\pgfpathlineto{\pgfqpoint{3.667103in}{3.450499in}}%
\pgfpathlineto{\pgfqpoint{3.667577in}{3.501215in}}%
\pgfpathlineto{\pgfqpoint{3.667956in}{3.438312in}}%
\pgfpathlineto{\pgfqpoint{3.668145in}{3.417194in}}%
\pgfpathlineto{\pgfqpoint{3.668714in}{3.474287in}}%
\pgfpathlineto{\pgfqpoint{3.669093in}{3.518098in}}%
\pgfpathlineto{\pgfqpoint{3.669662in}{3.462980in}}%
\pgfpathlineto{\pgfqpoint{3.669851in}{3.448204in}}%
\pgfpathlineto{\pgfqpoint{3.670135in}{3.494211in}}%
\pgfpathlineto{\pgfqpoint{3.670515in}{3.574234in}}%
\pgfpathlineto{\pgfqpoint{3.671178in}{3.510365in}}%
\pgfpathlineto{\pgfqpoint{3.671367in}{3.493275in}}%
\pgfpathlineto{\pgfqpoint{3.672031in}{3.535422in}}%
\pgfpathlineto{\pgfqpoint{3.672220in}{3.548840in}}%
\pgfpathlineto{\pgfqpoint{3.672599in}{3.494270in}}%
\pgfpathlineto{\pgfqpoint{3.673168in}{3.437643in}}%
\pgfpathlineto{\pgfqpoint{3.673642in}{3.489160in}}%
\pgfpathlineto{\pgfqpoint{3.673737in}{3.490452in}}%
\pgfpathlineto{\pgfqpoint{3.673831in}{3.482640in}}%
\pgfpathlineto{\pgfqpoint{3.674684in}{3.392031in}}%
\pgfpathlineto{\pgfqpoint{3.675063in}{3.450846in}}%
\pgfpathlineto{\pgfqpoint{3.675253in}{3.461823in}}%
\pgfpathlineto{\pgfqpoint{3.675727in}{3.423580in}}%
\pgfpathlineto{\pgfqpoint{3.676106in}{3.386439in}}%
\pgfpathlineto{\pgfqpoint{3.676579in}{3.443029in}}%
\pgfpathlineto{\pgfqpoint{3.676769in}{3.453139in}}%
\pgfpathlineto{\pgfqpoint{3.677338in}{3.432061in}}%
\pgfpathlineto{\pgfqpoint{3.677717in}{3.399241in}}%
\pgfpathlineto{\pgfqpoint{3.678190in}{3.456091in}}%
\pgfpathlineto{\pgfqpoint{3.678475in}{3.488887in}}%
\pgfpathlineto{\pgfqpoint{3.679043in}{3.429565in}}%
\pgfpathlineto{\pgfqpoint{3.679233in}{3.422679in}}%
\pgfpathlineto{\pgfqpoint{3.679612in}{3.452001in}}%
\pgfpathlineto{\pgfqpoint{3.679991in}{3.506406in}}%
\pgfpathlineto{\pgfqpoint{3.680560in}{3.428850in}}%
\pgfpathlineto{\pgfqpoint{3.680654in}{3.424522in}}%
\pgfpathlineto{\pgfqpoint{3.680844in}{3.452847in}}%
\pgfpathlineto{\pgfqpoint{3.681791in}{3.918061in}}%
\pgfpathlineto{\pgfqpoint{3.682929in}{3.855796in}}%
\pgfpathlineto{\pgfqpoint{3.683308in}{3.695041in}}%
\pgfpathlineto{\pgfqpoint{3.683876in}{3.317164in}}%
\pgfpathlineto{\pgfqpoint{3.684540in}{3.531022in}}%
\pgfpathlineto{\pgfqpoint{3.684824in}{3.581844in}}%
\pgfpathlineto{\pgfqpoint{3.685393in}{3.495373in}}%
\pgfpathlineto{\pgfqpoint{3.685487in}{3.497998in}}%
\pgfpathlineto{\pgfqpoint{3.686245in}{3.587206in}}%
\pgfpathlineto{\pgfqpoint{3.686909in}{3.553888in}}%
\pgfpathlineto{\pgfqpoint{3.687193in}{3.520914in}}%
\pgfpathlineto{\pgfqpoint{3.687667in}{3.595579in}}%
\pgfpathlineto{\pgfqpoint{3.687762in}{3.599488in}}%
\pgfpathlineto{\pgfqpoint{3.688235in}{3.579666in}}%
\pgfpathlineto{\pgfqpoint{3.688614in}{3.551573in}}%
\pgfpathlineto{\pgfqpoint{3.689088in}{3.584493in}}%
\pgfpathlineto{\pgfqpoint{3.689562in}{3.633672in}}%
\pgfpathlineto{\pgfqpoint{3.690036in}{3.579531in}}%
\pgfpathlineto{\pgfqpoint{3.690415in}{3.556395in}}%
\pgfpathlineto{\pgfqpoint{3.690699in}{3.601139in}}%
\pgfpathlineto{\pgfqpoint{3.690984in}{3.630312in}}%
\pgfpathlineto{\pgfqpoint{3.691647in}{3.579161in}}%
\pgfpathlineto{\pgfqpoint{3.691742in}{3.575559in}}%
\pgfpathlineto{\pgfqpoint{3.692026in}{3.598400in}}%
\pgfpathlineto{\pgfqpoint{3.692689in}{3.668330in}}%
\pgfpathlineto{\pgfqpoint{3.693163in}{3.613696in}}%
\pgfpathlineto{\pgfqpoint{3.693353in}{3.604865in}}%
\pgfpathlineto{\pgfqpoint{3.693732in}{3.630698in}}%
\pgfpathlineto{\pgfqpoint{3.694206in}{3.707841in}}%
\pgfpathlineto{\pgfqpoint{3.694869in}{3.641296in}}%
\pgfpathlineto{\pgfqpoint{3.695437in}{3.688683in}}%
\pgfpathlineto{\pgfqpoint{3.695722in}{3.713695in}}%
\pgfpathlineto{\pgfqpoint{3.696290in}{3.664544in}}%
\pgfpathlineto{\pgfqpoint{3.696575in}{3.656201in}}%
\pgfpathlineto{\pgfqpoint{3.696859in}{3.681612in}}%
\pgfpathlineto{\pgfqpoint{3.697333in}{3.743437in}}%
\pgfpathlineto{\pgfqpoint{3.697996in}{3.700882in}}%
\pgfpathlineto{\pgfqpoint{3.698091in}{3.700537in}}%
\pgfpathlineto{\pgfqpoint{3.698186in}{3.702822in}}%
\pgfpathlineto{\pgfqpoint{3.698944in}{3.754476in}}%
\pgfpathlineto{\pgfqpoint{3.699418in}{3.708067in}}%
\pgfpathlineto{\pgfqpoint{3.699607in}{3.700831in}}%
\pgfpathlineto{\pgfqpoint{3.700081in}{3.720287in}}%
\pgfpathlineto{\pgfqpoint{3.700460in}{3.770450in}}%
\pgfpathlineto{\pgfqpoint{3.701029in}{3.697615in}}%
\pgfpathlineto{\pgfqpoint{3.701408in}{3.674835in}}%
\pgfpathlineto{\pgfqpoint{3.701881in}{3.715978in}}%
\pgfpathlineto{\pgfqpoint{3.702166in}{3.698536in}}%
\pgfpathlineto{\pgfqpoint{3.704345in}{3.547681in}}%
\pgfpathlineto{\pgfqpoint{3.704440in}{3.547518in}}%
\pgfpathlineto{\pgfqpoint{3.706714in}{3.703377in}}%
\pgfpathlineto{\pgfqpoint{3.706904in}{3.693865in}}%
\pgfpathlineto{\pgfqpoint{3.707473in}{3.602711in}}%
\pgfpathlineto{\pgfqpoint{3.708136in}{3.655368in}}%
\pgfpathlineto{\pgfqpoint{3.708325in}{3.674295in}}%
\pgfpathlineto{\pgfqpoint{3.708894in}{3.618597in}}%
\pgfpathlineto{\pgfqpoint{3.709178in}{3.607449in}}%
\pgfpathlineto{\pgfqpoint{3.709463in}{3.634535in}}%
\pgfpathlineto{\pgfqpoint{3.709747in}{3.690515in}}%
\pgfpathlineto{\pgfqpoint{3.710505in}{3.619256in}}%
\pgfpathlineto{\pgfqpoint{3.710695in}{3.606160in}}%
\pgfpathlineto{\pgfqpoint{3.711168in}{3.649759in}}%
\pgfpathlineto{\pgfqpoint{3.711453in}{3.665101in}}%
\pgfpathlineto{\pgfqpoint{3.711832in}{3.633923in}}%
\pgfpathlineto{\pgfqpoint{3.712306in}{3.591241in}}%
\pgfpathlineto{\pgfqpoint{3.712779in}{3.636923in}}%
\pgfpathlineto{\pgfqpoint{3.712969in}{3.654466in}}%
\pgfpathlineto{\pgfqpoint{3.713537in}{3.602500in}}%
\pgfpathlineto{\pgfqpoint{3.713822in}{3.589151in}}%
\pgfpathlineto{\pgfqpoint{3.714296in}{3.625363in}}%
\pgfpathlineto{\pgfqpoint{3.715338in}{3.586028in}}%
\pgfpathlineto{\pgfqpoint{3.714959in}{3.630492in}}%
\pgfpathlineto{\pgfqpoint{3.715527in}{3.602498in}}%
\pgfpathlineto{\pgfqpoint{3.716191in}{3.675623in}}%
\pgfpathlineto{\pgfqpoint{3.716759in}{3.636134in}}%
\pgfpathlineto{\pgfqpoint{3.717138in}{3.634367in}}%
\pgfpathlineto{\pgfqpoint{3.717233in}{3.637341in}}%
\pgfpathlineto{\pgfqpoint{3.717707in}{3.692835in}}%
\pgfpathlineto{\pgfqpoint{3.718086in}{3.637694in}}%
\pgfpathlineto{\pgfqpoint{3.718749in}{3.606350in}}%
\pgfpathlineto{\pgfqpoint{3.719129in}{3.642515in}}%
\pgfpathlineto{\pgfqpoint{3.719223in}{3.648309in}}%
\pgfpathlineto{\pgfqpoint{3.719508in}{3.616932in}}%
\pgfpathlineto{\pgfqpoint{3.720171in}{3.528510in}}%
\pgfpathlineto{\pgfqpoint{3.720834in}{3.572781in}}%
\pgfpathlineto{\pgfqpoint{3.721024in}{3.568165in}}%
\pgfpathlineto{\pgfqpoint{3.721592in}{3.509882in}}%
\pgfpathlineto{\pgfqpoint{3.722351in}{3.550924in}}%
\pgfpathlineto{\pgfqpoint{3.722540in}{3.558788in}}%
\pgfpathlineto{\pgfqpoint{3.723014in}{3.531814in}}%
\pgfpathlineto{\pgfqpoint{3.723393in}{3.510380in}}%
\pgfpathlineto{\pgfqpoint{3.723772in}{3.547379in}}%
\pgfpathlineto{\pgfqpoint{3.723961in}{3.569615in}}%
\pgfpathlineto{\pgfqpoint{3.724435in}{3.503640in}}%
\pgfpathlineto{\pgfqpoint{3.724625in}{3.491635in}}%
\pgfpathlineto{\pgfqpoint{3.725288in}{3.523589in}}%
\pgfpathlineto{\pgfqpoint{3.725572in}{3.559834in}}%
\pgfpathlineto{\pgfqpoint{3.726141in}{3.502474in}}%
\pgfpathlineto{\pgfqpoint{3.726425in}{3.480041in}}%
\pgfpathlineto{\pgfqpoint{3.727089in}{3.501951in}}%
\pgfpathlineto{\pgfqpoint{3.728036in}{3.825920in}}%
\pgfpathlineto{\pgfqpoint{3.728700in}{3.988195in}}%
\pgfpathlineto{\pgfqpoint{3.729268in}{3.890928in}}%
\pgfpathlineto{\pgfqpoint{3.730121in}{3.493975in}}%
\pgfpathlineto{\pgfqpoint{3.730405in}{3.400101in}}%
\pgfpathlineto{\pgfqpoint{3.731164in}{3.517229in}}%
\pgfpathlineto{\pgfqpoint{3.732016in}{3.589150in}}%
\pgfpathlineto{\pgfqpoint{3.732490in}{3.539045in}}%
\pgfpathlineto{\pgfqpoint{3.732680in}{3.531829in}}%
\pgfpathlineto{\pgfqpoint{3.733154in}{3.560969in}}%
\pgfpathlineto{\pgfqpoint{3.733533in}{3.578779in}}%
\pgfpathlineto{\pgfqpoint{3.733817in}{3.554038in}}%
\pgfpathlineto{\pgfqpoint{3.734101in}{3.530040in}}%
\pgfpathlineto{\pgfqpoint{3.734765in}{3.565297in}}%
\pgfpathlineto{\pgfqpoint{3.734859in}{3.566218in}}%
\pgfpathlineto{\pgfqpoint{3.734954in}{3.561881in}}%
\pgfpathlineto{\pgfqpoint{3.735902in}{3.520425in}}%
\pgfpathlineto{\pgfqpoint{3.736091in}{3.537313in}}%
\pgfpathlineto{\pgfqpoint{3.736565in}{3.592133in}}%
\pgfpathlineto{\pgfqpoint{3.737228in}{3.550667in}}%
\pgfpathlineto{\pgfqpoint{3.737418in}{3.536696in}}%
\pgfpathlineto{\pgfqpoint{3.737987in}{3.578161in}}%
\pgfpathlineto{\pgfqpoint{3.738081in}{3.580200in}}%
\pgfpathlineto{\pgfqpoint{3.738460in}{3.567391in}}%
\pgfpathlineto{\pgfqpoint{3.739029in}{3.542134in}}%
\pgfpathlineto{\pgfqpoint{3.739408in}{3.572312in}}%
\pgfpathlineto{\pgfqpoint{3.739692in}{3.591764in}}%
\pgfpathlineto{\pgfqpoint{3.740450in}{3.567626in}}%
\pgfpathlineto{\pgfqpoint{3.740640in}{3.564259in}}%
\pgfpathlineto{\pgfqpoint{3.740830in}{3.570770in}}%
\pgfpathlineto{\pgfqpoint{3.742535in}{3.653740in}}%
\pgfpathlineto{\pgfqpoint{3.743104in}{3.674693in}}%
\pgfpathlineto{\pgfqpoint{3.743293in}{3.658467in}}%
\pgfpathlineto{\pgfqpoint{3.743672in}{3.631724in}}%
\pgfpathlineto{\pgfqpoint{3.744336in}{3.659141in}}%
\pgfpathlineto{\pgfqpoint{3.744525in}{3.676926in}}%
\pgfpathlineto{\pgfqpoint{3.745189in}{3.635040in}}%
\pgfpathlineto{\pgfqpoint{3.745473in}{3.626062in}}%
\pgfpathlineto{\pgfqpoint{3.745757in}{3.651222in}}%
\pgfpathlineto{\pgfqpoint{3.745947in}{3.667179in}}%
\pgfpathlineto{\pgfqpoint{3.746610in}{3.630665in}}%
\pgfpathlineto{\pgfqpoint{3.746894in}{3.612689in}}%
\pgfpathlineto{\pgfqpoint{3.747368in}{3.651099in}}%
\pgfpathlineto{\pgfqpoint{3.747463in}{3.651618in}}%
\pgfpathlineto{\pgfqpoint{3.747558in}{3.646952in}}%
\pgfpathlineto{\pgfqpoint{3.748505in}{3.609828in}}%
\pgfpathlineto{\pgfqpoint{3.748032in}{3.650375in}}%
\pgfpathlineto{\pgfqpoint{3.748979in}{3.628030in}}%
\pgfpathlineto{\pgfqpoint{3.749358in}{3.647186in}}%
\pgfpathlineto{\pgfqpoint{3.749737in}{3.609158in}}%
\pgfpathlineto{\pgfqpoint{3.749832in}{3.604677in}}%
\pgfpathlineto{\pgfqpoint{3.750495in}{3.622137in}}%
\pgfpathlineto{\pgfqpoint{3.750685in}{3.628173in}}%
\pgfpathlineto{\pgfqpoint{3.750969in}{3.611192in}}%
\pgfpathlineto{\pgfqpoint{3.751538in}{3.553319in}}%
\pgfpathlineto{\pgfqpoint{3.752201in}{3.593010in}}%
\pgfpathlineto{\pgfqpoint{3.752485in}{3.577241in}}%
\pgfpathlineto{\pgfqpoint{3.753433in}{3.523915in}}%
\pgfpathlineto{\pgfqpoint{3.754002in}{3.533961in}}%
\pgfpathlineto{\pgfqpoint{3.754855in}{3.487098in}}%
\pgfpathlineto{\pgfqpoint{3.755234in}{3.527458in}}%
\pgfpathlineto{\pgfqpoint{3.755328in}{3.529078in}}%
\pgfpathlineto{\pgfqpoint{3.755423in}{3.521884in}}%
\pgfpathlineto{\pgfqpoint{3.756371in}{3.429272in}}%
\pgfpathlineto{\pgfqpoint{3.756939in}{3.471449in}}%
\pgfpathlineto{\pgfqpoint{3.757508in}{3.448276in}}%
\pgfpathlineto{\pgfqpoint{3.757982in}{3.428947in}}%
\pgfpathlineto{\pgfqpoint{3.758361in}{3.455861in}}%
\pgfpathlineto{\pgfqpoint{3.758645in}{3.471108in}}%
\pgfpathlineto{\pgfqpoint{3.759119in}{3.431413in}}%
\pgfpathlineto{\pgfqpoint{3.759688in}{3.419882in}}%
\pgfpathlineto{\pgfqpoint{3.759877in}{3.431529in}}%
\pgfpathlineto{\pgfqpoint{3.760256in}{3.472132in}}%
\pgfpathlineto{\pgfqpoint{3.760825in}{3.416866in}}%
\pgfpathlineto{\pgfqpoint{3.760919in}{3.416778in}}%
\pgfpathlineto{\pgfqpoint{3.764900in}{3.529560in}}%
\pgfpathlineto{\pgfqpoint{3.765089in}{3.523306in}}%
\pgfpathlineto{\pgfqpoint{3.767174in}{3.419405in}}%
\pgfpathlineto{\pgfqpoint{3.768122in}{3.448329in}}%
\pgfpathlineto{\pgfqpoint{3.768690in}{3.434836in}}%
\pgfpathlineto{\pgfqpoint{3.769069in}{3.418380in}}%
\pgfpathlineto{\pgfqpoint{3.769543in}{3.449277in}}%
\pgfpathlineto{\pgfqpoint{3.769638in}{3.453101in}}%
\pgfpathlineto{\pgfqpoint{3.770017in}{3.426545in}}%
\pgfpathlineto{\pgfqpoint{3.770775in}{3.412724in}}%
\pgfpathlineto{\pgfqpoint{3.770964in}{3.428318in}}%
\pgfpathlineto{\pgfqpoint{3.771249in}{3.449033in}}%
\pgfpathlineto{\pgfqpoint{3.771723in}{3.386464in}}%
\pgfpathlineto{\pgfqpoint{3.773428in}{3.250193in}}%
\pgfpathlineto{\pgfqpoint{3.773713in}{3.302081in}}%
\pgfpathlineto{\pgfqpoint{3.775039in}{3.807848in}}%
\pgfpathlineto{\pgfqpoint{3.775892in}{3.757056in}}%
\pgfpathlineto{\pgfqpoint{3.776745in}{3.268755in}}%
\pgfpathlineto{\pgfqpoint{3.777693in}{3.478573in}}%
\pgfpathlineto{\pgfqpoint{3.778451in}{3.456611in}}%
\pgfpathlineto{\pgfqpoint{3.778640in}{3.472156in}}%
\pgfpathlineto{\pgfqpoint{3.778925in}{3.499087in}}%
\pgfpathlineto{\pgfqpoint{3.779778in}{3.475848in}}%
\pgfpathlineto{\pgfqpoint{3.780062in}{3.470633in}}%
\pgfpathlineto{\pgfqpoint{3.780441in}{3.485117in}}%
\pgfpathlineto{\pgfqpoint{3.780725in}{3.501370in}}%
\pgfpathlineto{\pgfqpoint{3.781199in}{3.472086in}}%
\pgfpathlineto{\pgfqpoint{3.781483in}{3.482219in}}%
\pgfpathlineto{\pgfqpoint{3.782336in}{3.518686in}}%
\pgfpathlineto{\pgfqpoint{3.782620in}{3.492048in}}%
\pgfpathlineto{\pgfqpoint{3.782810in}{3.485380in}}%
\pgfpathlineto{\pgfqpoint{3.783094in}{3.493069in}}%
\pgfpathlineto{\pgfqpoint{3.783568in}{3.492457in}}%
\pgfpathlineto{\pgfqpoint{3.783947in}{3.508010in}}%
\pgfpathlineto{\pgfqpoint{3.784800in}{3.502237in}}%
\pgfpathlineto{\pgfqpoint{3.784990in}{3.499504in}}%
\pgfpathlineto{\pgfqpoint{3.785179in}{3.506638in}}%
\pgfpathlineto{\pgfqpoint{3.785558in}{3.537980in}}%
\pgfpathlineto{\pgfqpoint{3.786316in}{3.511988in}}%
\pgfpathlineto{\pgfqpoint{3.787453in}{3.550621in}}%
\pgfpathlineto{\pgfqpoint{3.787738in}{3.539275in}}%
\pgfpathlineto{\pgfqpoint{3.787833in}{3.536387in}}%
\pgfpathlineto{\pgfqpoint{3.788212in}{3.558572in}}%
\pgfpathlineto{\pgfqpoint{3.788401in}{3.563889in}}%
\pgfpathlineto{\pgfqpoint{3.788780in}{3.553418in}}%
\pgfpathlineto{\pgfqpoint{3.789254in}{3.560334in}}%
\pgfpathlineto{\pgfqpoint{3.789349in}{3.558757in}}%
\pgfpathlineto{\pgfqpoint{3.789633in}{3.568189in}}%
\pgfpathlineto{\pgfqpoint{3.789823in}{3.578793in}}%
\pgfpathlineto{\pgfqpoint{3.790202in}{3.566585in}}%
\pgfpathlineto{\pgfqpoint{3.790675in}{3.573746in}}%
\pgfpathlineto{\pgfqpoint{3.790960in}{3.559990in}}%
\pgfpathlineto{\pgfqpoint{3.791434in}{3.593497in}}%
\pgfpathlineto{\pgfqpoint{3.791528in}{3.595705in}}%
\pgfpathlineto{\pgfqpoint{3.791907in}{3.580497in}}%
\pgfpathlineto{\pgfqpoint{3.793045in}{3.554481in}}%
\pgfpathlineto{\pgfqpoint{3.793234in}{3.560920in}}%
\pgfpathlineto{\pgfqpoint{3.794466in}{3.581432in}}%
\pgfpathlineto{\pgfqpoint{3.794656in}{3.577156in}}%
\pgfpathlineto{\pgfqpoint{3.795603in}{3.541196in}}%
\pgfpathlineto{\pgfqpoint{3.796361in}{3.557775in}}%
\pgfpathlineto{\pgfqpoint{3.796551in}{3.566296in}}%
\pgfpathlineto{\pgfqpoint{3.796930in}{3.531891in}}%
\pgfpathlineto{\pgfqpoint{3.797119in}{3.524155in}}%
\pgfpathlineto{\pgfqpoint{3.797498in}{3.542347in}}%
\pgfpathlineto{\pgfqpoint{3.798067in}{3.529543in}}%
\pgfpathlineto{\pgfqpoint{3.799962in}{3.490598in}}%
\pgfpathlineto{\pgfqpoint{3.800247in}{3.488137in}}%
\pgfpathlineto{\pgfqpoint{3.800531in}{3.491803in}}%
\pgfpathlineto{\pgfqpoint{3.800815in}{3.491406in}}%
\pgfpathlineto{\pgfqpoint{3.801194in}{3.503601in}}%
\pgfpathlineto{\pgfqpoint{3.801479in}{3.490598in}}%
\pgfpathlineto{\pgfqpoint{3.802047in}{3.467437in}}%
\pgfpathlineto{\pgfqpoint{3.802521in}{3.488819in}}%
\pgfpathlineto{\pgfqpoint{3.802805in}{3.497843in}}%
\pgfpathlineto{\pgfqpoint{3.803563in}{3.489120in}}%
\pgfpathlineto{\pgfqpoint{3.803848in}{3.482043in}}%
\pgfpathlineto{\pgfqpoint{3.804227in}{3.503580in}}%
\pgfpathlineto{\pgfqpoint{3.804416in}{3.497813in}}%
\pgfpathlineto{\pgfqpoint{3.804701in}{3.477885in}}%
\pgfpathlineto{\pgfqpoint{3.805553in}{3.486688in}}%
\pgfpathlineto{\pgfqpoint{3.807259in}{3.514034in}}%
\pgfpathlineto{\pgfqpoint{3.806217in}{3.477005in}}%
\pgfpathlineto{\pgfqpoint{3.807354in}{3.510913in}}%
\pgfpathlineto{\pgfqpoint{3.808112in}{3.486044in}}%
\pgfpathlineto{\pgfqpoint{3.808396in}{3.502458in}}%
\pgfpathlineto{\pgfqpoint{3.809818in}{3.543688in}}%
\pgfpathlineto{\pgfqpoint{3.810765in}{3.584157in}}%
\pgfpathlineto{\pgfqpoint{3.811144in}{3.553992in}}%
\pgfpathlineto{\pgfqpoint{3.811239in}{3.551836in}}%
\pgfpathlineto{\pgfqpoint{3.811524in}{3.567443in}}%
\pgfpathlineto{\pgfqpoint{3.811903in}{3.577931in}}%
\pgfpathlineto{\pgfqpoint{3.812282in}{3.560386in}}%
\pgfpathlineto{\pgfqpoint{3.814556in}{3.445188in}}%
\pgfpathlineto{\pgfqpoint{3.814840in}{3.457607in}}%
\pgfpathlineto{\pgfqpoint{3.816451in}{3.424294in}}%
\pgfpathlineto{\pgfqpoint{3.815504in}{3.462581in}}%
\pgfpathlineto{\pgfqpoint{3.816925in}{3.440632in}}%
\pgfpathlineto{\pgfqpoint{3.818536in}{3.474736in}}%
\pgfpathlineto{\pgfqpoint{3.817778in}{3.434419in}}%
\pgfpathlineto{\pgfqpoint{3.818726in}{3.465852in}}%
\pgfpathlineto{\pgfqpoint{3.819010in}{3.454363in}}%
\pgfpathlineto{\pgfqpoint{3.819673in}{3.473832in}}%
\pgfpathlineto{\pgfqpoint{3.819768in}{3.476490in}}%
\pgfpathlineto{\pgfqpoint{3.819958in}{3.463655in}}%
\pgfpathlineto{\pgfqpoint{3.820337in}{3.433902in}}%
\pgfpathlineto{\pgfqpoint{3.820810in}{3.481724in}}%
\pgfpathlineto{\pgfqpoint{3.821948in}{3.917592in}}%
\pgfpathlineto{\pgfqpoint{3.822706in}{3.834286in}}%
\pgfpathlineto{\pgfqpoint{3.823085in}{3.740638in}}%
\pgfpathlineto{\pgfqpoint{3.823748in}{3.354115in}}%
\pgfpathlineto{\pgfqpoint{3.824506in}{3.519685in}}%
\pgfpathlineto{\pgfqpoint{3.825170in}{3.533080in}}%
\pgfpathlineto{\pgfqpoint{3.825643in}{3.521798in}}%
\pgfpathlineto{\pgfqpoint{3.825738in}{3.521816in}}%
\pgfpathlineto{\pgfqpoint{3.825833in}{3.521028in}}%
\pgfpathlineto{\pgfqpoint{3.825928in}{3.519871in}}%
\pgfpathlineto{\pgfqpoint{3.826117in}{3.525272in}}%
\pgfpathlineto{\pgfqpoint{3.827160in}{3.550734in}}%
\pgfpathlineto{\pgfqpoint{3.827349in}{3.538305in}}%
\pgfpathlineto{\pgfqpoint{3.828297in}{3.522975in}}%
\pgfpathlineto{\pgfqpoint{3.827918in}{3.542538in}}%
\pgfpathlineto{\pgfqpoint{3.828486in}{3.530575in}}%
\pgfpathlineto{\pgfqpoint{3.829339in}{3.538097in}}%
\pgfpathlineto{\pgfqpoint{3.828960in}{3.525868in}}%
\pgfpathlineto{\pgfqpoint{3.829529in}{3.534294in}}%
\pgfpathlineto{\pgfqpoint{3.829813in}{3.526484in}}%
\pgfpathlineto{\pgfqpoint{3.830192in}{3.548254in}}%
\pgfpathlineto{\pgfqpoint{3.830287in}{3.550916in}}%
\pgfpathlineto{\pgfqpoint{3.830571in}{3.528323in}}%
\pgfpathlineto{\pgfqpoint{3.830761in}{3.518419in}}%
\pgfpathlineto{\pgfqpoint{3.831140in}{3.539894in}}%
\pgfpathlineto{\pgfqpoint{3.831519in}{3.530528in}}%
\pgfpathlineto{\pgfqpoint{3.832087in}{3.524184in}}%
\pgfpathlineto{\pgfqpoint{3.833225in}{3.560576in}}%
\pgfpathlineto{\pgfqpoint{3.833509in}{3.553330in}}%
\pgfpathlineto{\pgfqpoint{3.833793in}{3.569728in}}%
\pgfpathlineto{\pgfqpoint{3.835215in}{3.584526in}}%
\pgfpathlineto{\pgfqpoint{3.834267in}{3.560272in}}%
\pgfpathlineto{\pgfqpoint{3.835309in}{3.583592in}}%
\pgfpathlineto{\pgfqpoint{3.835404in}{3.583215in}}%
\pgfpathlineto{\pgfqpoint{3.835594in}{3.585756in}}%
\pgfpathlineto{\pgfqpoint{3.836257in}{3.603565in}}%
\pgfpathlineto{\pgfqpoint{3.836920in}{3.590346in}}%
\pgfpathlineto{\pgfqpoint{3.837110in}{3.581273in}}%
\pgfpathlineto{\pgfqpoint{3.837584in}{3.601827in}}%
\pgfpathlineto{\pgfqpoint{3.837773in}{3.599842in}}%
\pgfpathlineto{\pgfqpoint{3.840048in}{3.629026in}}%
\pgfpathlineto{\pgfqpoint{3.840142in}{3.631291in}}%
\pgfpathlineto{\pgfqpoint{3.840427in}{3.616027in}}%
\pgfpathlineto{\pgfqpoint{3.840711in}{3.594994in}}%
\pgfpathlineto{\pgfqpoint{3.841090in}{3.628040in}}%
\pgfpathlineto{\pgfqpoint{3.841469in}{3.621171in}}%
\pgfpathlineto{\pgfqpoint{3.841659in}{3.625024in}}%
\pgfpathlineto{\pgfqpoint{3.842038in}{3.616355in}}%
\pgfpathlineto{\pgfqpoint{3.842417in}{3.617837in}}%
\pgfpathlineto{\pgfqpoint{3.842511in}{3.616879in}}%
\pgfpathlineto{\pgfqpoint{3.842701in}{3.622698in}}%
\pgfpathlineto{\pgfqpoint{3.842890in}{3.631856in}}%
\pgfpathlineto{\pgfqpoint{3.843270in}{3.593866in}}%
\pgfpathlineto{\pgfqpoint{3.845544in}{3.492882in}}%
\pgfpathlineto{\pgfqpoint{3.845639in}{3.490411in}}%
\pgfpathlineto{\pgfqpoint{3.846018in}{3.504708in}}%
\pgfpathlineto{\pgfqpoint{3.847723in}{3.589819in}}%
\pgfpathlineto{\pgfqpoint{3.848482in}{3.582763in}}%
\pgfpathlineto{\pgfqpoint{3.849524in}{3.541358in}}%
\pgfpathlineto{\pgfqpoint{3.849808in}{3.568816in}}%
\pgfpathlineto{\pgfqpoint{3.849903in}{3.570228in}}%
\pgfpathlineto{\pgfqpoint{3.849998in}{3.563701in}}%
\pgfpathlineto{\pgfqpoint{3.850187in}{3.550722in}}%
\pgfpathlineto{\pgfqpoint{3.850566in}{3.597858in}}%
\pgfpathlineto{\pgfqpoint{3.850661in}{3.601993in}}%
\pgfpathlineto{\pgfqpoint{3.850945in}{3.567142in}}%
\pgfpathlineto{\pgfqpoint{3.852177in}{3.545556in}}%
\pgfpathlineto{\pgfqpoint{3.853125in}{3.527427in}}%
\pgfpathlineto{\pgfqpoint{3.852651in}{3.553149in}}%
\pgfpathlineto{\pgfqpoint{3.853409in}{3.537739in}}%
\pgfpathlineto{\pgfqpoint{3.853883in}{3.552970in}}%
\pgfpathlineto{\pgfqpoint{3.854357in}{3.538132in}}%
\pgfpathlineto{\pgfqpoint{3.855020in}{3.526733in}}%
\pgfpathlineto{\pgfqpoint{3.855305in}{3.542701in}}%
\pgfpathlineto{\pgfqpoint{3.856726in}{3.602421in}}%
\pgfpathlineto{\pgfqpoint{3.856916in}{3.594342in}}%
\pgfpathlineto{\pgfqpoint{3.858242in}{3.574375in}}%
\pgfpathlineto{\pgfqpoint{3.857768in}{3.600513in}}%
\pgfpathlineto{\pgfqpoint{3.858337in}{3.574800in}}%
\pgfpathlineto{\pgfqpoint{3.858621in}{3.580936in}}%
\pgfpathlineto{\pgfqpoint{3.858906in}{3.556982in}}%
\pgfpathlineto{\pgfqpoint{3.860517in}{3.486353in}}%
\pgfpathlineto{\pgfqpoint{3.860611in}{3.489718in}}%
\pgfpathlineto{\pgfqpoint{3.861180in}{3.476820in}}%
\pgfpathlineto{\pgfqpoint{3.861559in}{3.493358in}}%
\pgfpathlineto{\pgfqpoint{3.861938in}{3.460187in}}%
\pgfpathlineto{\pgfqpoint{3.862791in}{3.481297in}}%
\pgfpathlineto{\pgfqpoint{3.863075in}{3.505423in}}%
\pgfpathlineto{\pgfqpoint{3.863644in}{3.457522in}}%
\pgfpathlineto{\pgfqpoint{3.863928in}{3.463930in}}%
\pgfpathlineto{\pgfqpoint{3.864212in}{3.453786in}}%
\pgfpathlineto{\pgfqpoint{3.864307in}{3.451421in}}%
\pgfpathlineto{\pgfqpoint{3.864497in}{3.462236in}}%
\pgfpathlineto{\pgfqpoint{3.865444in}{3.486234in}}%
\pgfpathlineto{\pgfqpoint{3.865729in}{3.479880in}}%
\pgfpathlineto{\pgfqpoint{3.865823in}{3.479873in}}%
\pgfpathlineto{\pgfqpoint{3.866202in}{3.497774in}}%
\pgfpathlineto{\pgfqpoint{3.866487in}{3.465260in}}%
\pgfpathlineto{\pgfqpoint{3.866866in}{3.406383in}}%
\pgfpathlineto{\pgfqpoint{3.867340in}{3.500248in}}%
\pgfpathlineto{\pgfqpoint{3.868287in}{3.894460in}}%
\pgfpathlineto{\pgfqpoint{3.869140in}{3.837610in}}%
\pgfpathlineto{\pgfqpoint{3.869519in}{3.738424in}}%
\pgfpathlineto{\pgfqpoint{3.870277in}{3.304613in}}%
\pgfpathlineto{\pgfqpoint{3.870941in}{3.503489in}}%
\pgfpathlineto{\pgfqpoint{3.871225in}{3.531571in}}%
\pgfpathlineto{\pgfqpoint{3.871888in}{3.493659in}}%
\pgfpathlineto{\pgfqpoint{3.871983in}{3.495694in}}%
\pgfpathlineto{\pgfqpoint{3.872457in}{3.529152in}}%
\pgfpathlineto{\pgfqpoint{3.873215in}{3.507108in}}%
\pgfpathlineto{\pgfqpoint{3.873404in}{3.505910in}}%
\pgfpathlineto{\pgfqpoint{3.873499in}{3.508477in}}%
\pgfpathlineto{\pgfqpoint{3.874163in}{3.557259in}}%
\pgfpathlineto{\pgfqpoint{3.875015in}{3.535147in}}%
\pgfpathlineto{\pgfqpoint{3.875110in}{3.532798in}}%
\pgfpathlineto{\pgfqpoint{3.875395in}{3.545394in}}%
\pgfpathlineto{\pgfqpoint{3.875774in}{3.558474in}}%
\pgfpathlineto{\pgfqpoint{3.876153in}{3.536077in}}%
\pgfpathlineto{\pgfqpoint{3.876437in}{3.520334in}}%
\pgfpathlineto{\pgfqpoint{3.876911in}{3.564852in}}%
\pgfpathlineto{\pgfqpoint{3.877100in}{3.573468in}}%
\pgfpathlineto{\pgfqpoint{3.877858in}{3.555575in}}%
\pgfpathlineto{\pgfqpoint{3.878143in}{3.549888in}}%
\pgfpathlineto{\pgfqpoint{3.878427in}{3.562149in}}%
\pgfpathlineto{\pgfqpoint{3.878901in}{3.594002in}}%
\pgfpathlineto{\pgfqpoint{3.879564in}{3.575535in}}%
\pgfpathlineto{\pgfqpoint{3.879754in}{3.569744in}}%
\pgfpathlineto{\pgfqpoint{3.880038in}{3.600346in}}%
\pgfpathlineto{\pgfqpoint{3.880417in}{3.628075in}}%
\pgfpathlineto{\pgfqpoint{3.881080in}{3.605326in}}%
\pgfpathlineto{\pgfqpoint{3.881270in}{3.600090in}}%
\pgfpathlineto{\pgfqpoint{3.881649in}{3.623131in}}%
\pgfpathlineto{\pgfqpoint{3.881839in}{3.622035in}}%
\pgfpathlineto{\pgfqpoint{3.881933in}{3.621769in}}%
\pgfpathlineto{\pgfqpoint{3.882028in}{3.623956in}}%
\pgfpathlineto{\pgfqpoint{3.882218in}{3.629040in}}%
\pgfpathlineto{\pgfqpoint{3.882597in}{3.610156in}}%
\pgfpathlineto{\pgfqpoint{3.882786in}{3.604842in}}%
\pgfpathlineto{\pgfqpoint{3.883165in}{3.628656in}}%
\pgfpathlineto{\pgfqpoint{3.883544in}{3.655844in}}%
\pgfpathlineto{\pgfqpoint{3.884208in}{3.634778in}}%
\pgfpathlineto{\pgfqpoint{3.884681in}{3.610351in}}%
\pgfpathlineto{\pgfqpoint{3.885155in}{3.639769in}}%
\pgfpathlineto{\pgfqpoint{3.885819in}{3.625871in}}%
\pgfpathlineto{\pgfqpoint{3.886103in}{3.620037in}}%
\pgfpathlineto{\pgfqpoint{3.886482in}{3.631845in}}%
\pgfpathlineto{\pgfqpoint{3.886861in}{3.647183in}}%
\pgfpathlineto{\pgfqpoint{3.887240in}{3.611530in}}%
\pgfpathlineto{\pgfqpoint{3.887430in}{3.604760in}}%
\pgfpathlineto{\pgfqpoint{3.888093in}{3.621283in}}%
\pgfpathlineto{\pgfqpoint{3.888472in}{3.638771in}}%
\pgfpathlineto{\pgfqpoint{3.888946in}{3.610314in}}%
\pgfpathlineto{\pgfqpoint{3.889420in}{3.615988in}}%
\pgfpathlineto{\pgfqpoint{3.889893in}{3.652000in}}%
\pgfpathlineto{\pgfqpoint{3.890462in}{3.619508in}}%
\pgfpathlineto{\pgfqpoint{3.890936in}{3.589730in}}%
\pgfpathlineto{\pgfqpoint{3.891599in}{3.614408in}}%
\pgfpathlineto{\pgfqpoint{3.892168in}{3.594064in}}%
\pgfpathlineto{\pgfqpoint{3.892736in}{3.612881in}}%
\pgfpathlineto{\pgfqpoint{3.892926in}{3.608397in}}%
\pgfpathlineto{\pgfqpoint{3.893968in}{3.538051in}}%
\pgfpathlineto{\pgfqpoint{3.894537in}{3.565203in}}%
\pgfpathlineto{\pgfqpoint{3.894821in}{3.575922in}}%
\pgfpathlineto{\pgfqpoint{3.895200in}{3.557599in}}%
\pgfpathlineto{\pgfqpoint{3.895390in}{3.544627in}}%
\pgfpathlineto{\pgfqpoint{3.895864in}{3.582715in}}%
\pgfpathlineto{\pgfqpoint{3.896053in}{3.590910in}}%
\pgfpathlineto{\pgfqpoint{3.896527in}{3.559122in}}%
\pgfpathlineto{\pgfqpoint{3.897190in}{3.543253in}}%
\pgfpathlineto{\pgfqpoint{3.897380in}{3.553857in}}%
\pgfpathlineto{\pgfqpoint{3.897759in}{3.597249in}}%
\pgfpathlineto{\pgfqpoint{3.898422in}{3.558761in}}%
\pgfpathlineto{\pgfqpoint{3.898707in}{3.553290in}}%
\pgfpathlineto{\pgfqpoint{3.898991in}{3.571319in}}%
\pgfpathlineto{\pgfqpoint{3.899180in}{3.585300in}}%
\pgfpathlineto{\pgfqpoint{3.899749in}{3.549134in}}%
\pgfpathlineto{\pgfqpoint{3.899844in}{3.550902in}}%
\pgfpathlineto{\pgfqpoint{3.902402in}{3.664011in}}%
\pgfpathlineto{\pgfqpoint{3.902971in}{3.637654in}}%
\pgfpathlineto{\pgfqpoint{3.903445in}{3.613095in}}%
\pgfpathlineto{\pgfqpoint{3.903729in}{3.641123in}}%
\pgfpathlineto{\pgfqpoint{3.904013in}{3.662670in}}%
\pgfpathlineto{\pgfqpoint{3.904582in}{3.624912in}}%
\pgfpathlineto{\pgfqpoint{3.906477in}{3.518995in}}%
\pgfpathlineto{\pgfqpoint{3.906951in}{3.562415in}}%
\pgfpathlineto{\pgfqpoint{3.907425in}{3.585562in}}%
\pgfpathlineto{\pgfqpoint{3.907804in}{3.551933in}}%
\pgfpathlineto{\pgfqpoint{3.908088in}{3.524276in}}%
\pgfpathlineto{\pgfqpoint{3.908752in}{3.575144in}}%
\pgfpathlineto{\pgfqpoint{3.908846in}{3.578917in}}%
\pgfpathlineto{\pgfqpoint{3.909225in}{3.554230in}}%
\pgfpathlineto{\pgfqpoint{3.909699in}{3.533303in}}%
\pgfpathlineto{\pgfqpoint{3.910173in}{3.560929in}}%
\pgfpathlineto{\pgfqpoint{3.910552in}{3.579552in}}%
\pgfpathlineto{\pgfqpoint{3.910931in}{3.546798in}}%
\pgfpathlineto{\pgfqpoint{3.911121in}{3.537053in}}%
\pgfpathlineto{\pgfqpoint{3.911784in}{3.561320in}}%
\pgfpathlineto{\pgfqpoint{3.912068in}{3.581747in}}%
\pgfpathlineto{\pgfqpoint{3.912447in}{3.531175in}}%
\pgfpathlineto{\pgfqpoint{3.912921in}{3.498063in}}%
\pgfpathlineto{\pgfqpoint{3.913205in}{3.536027in}}%
\pgfpathlineto{\pgfqpoint{3.914722in}{3.911690in}}%
\pgfpathlineto{\pgfqpoint{3.915290in}{3.859784in}}%
\pgfpathlineto{\pgfqpoint{3.915764in}{3.656257in}}%
\pgfpathlineto{\pgfqpoint{3.916333in}{3.383826in}}%
\pgfpathlineto{\pgfqpoint{3.916996in}{3.544060in}}%
\pgfpathlineto{\pgfqpoint{3.917091in}{3.544263in}}%
\pgfpathlineto{\pgfqpoint{3.917186in}{3.543255in}}%
\pgfpathlineto{\pgfqpoint{3.917659in}{3.514916in}}%
\pgfpathlineto{\pgfqpoint{3.918038in}{3.546650in}}%
\pgfpathlineto{\pgfqpoint{3.918228in}{3.550478in}}%
\pgfpathlineto{\pgfqpoint{3.918607in}{3.525403in}}%
\pgfpathlineto{\pgfqpoint{3.919081in}{3.482677in}}%
\pgfpathlineto{\pgfqpoint{3.919839in}{3.505420in}}%
\pgfpathlineto{\pgfqpoint{3.920597in}{3.440692in}}%
\pgfpathlineto{\pgfqpoint{3.921071in}{3.491828in}}%
\pgfpathlineto{\pgfqpoint{3.922871in}{3.671099in}}%
\pgfpathlineto{\pgfqpoint{3.923061in}{3.654049in}}%
\pgfpathlineto{\pgfqpoint{3.923819in}{3.595468in}}%
\pgfpathlineto{\pgfqpoint{3.924198in}{3.628662in}}%
\pgfpathlineto{\pgfqpoint{3.924672in}{3.651815in}}%
\pgfpathlineto{\pgfqpoint{3.925146in}{3.616752in}}%
\pgfpathlineto{\pgfqpoint{3.925240in}{3.612585in}}%
\pgfpathlineto{\pgfqpoint{3.925620in}{3.640178in}}%
\pgfpathlineto{\pgfqpoint{3.925999in}{3.659929in}}%
\pgfpathlineto{\pgfqpoint{3.926472in}{3.623090in}}%
\pgfpathlineto{\pgfqpoint{3.926851in}{3.599357in}}%
\pgfpathlineto{\pgfqpoint{3.927325in}{3.646798in}}%
\pgfpathlineto{\pgfqpoint{3.927704in}{3.678424in}}%
\pgfpathlineto{\pgfqpoint{3.928178in}{3.620340in}}%
\pgfpathlineto{\pgfqpoint{3.928368in}{3.603251in}}%
\pgfpathlineto{\pgfqpoint{3.928936in}{3.659154in}}%
\pgfpathlineto{\pgfqpoint{3.929031in}{3.660177in}}%
\pgfpathlineto{\pgfqpoint{3.929315in}{3.652965in}}%
\pgfpathlineto{\pgfqpoint{3.929979in}{3.599369in}}%
\pgfpathlineto{\pgfqpoint{3.930452in}{3.649510in}}%
\pgfpathlineto{\pgfqpoint{3.930832in}{3.669561in}}%
\pgfpathlineto{\pgfqpoint{3.931400in}{3.643480in}}%
\pgfpathlineto{\pgfqpoint{3.931495in}{3.642158in}}%
\pgfpathlineto{\pgfqpoint{3.931969in}{3.648876in}}%
\pgfpathlineto{\pgfqpoint{3.932443in}{3.663434in}}%
\pgfpathlineto{\pgfqpoint{3.932727in}{3.648925in}}%
\pgfpathlineto{\pgfqpoint{3.933106in}{3.603143in}}%
\pgfpathlineto{\pgfqpoint{3.933580in}{3.664853in}}%
\pgfpathlineto{\pgfqpoint{3.933959in}{3.686230in}}%
\pgfpathlineto{\pgfqpoint{3.934338in}{3.643891in}}%
\pgfpathlineto{\pgfqpoint{3.934717in}{3.582883in}}%
\pgfpathlineto{\pgfqpoint{3.935380in}{3.635144in}}%
\pgfpathlineto{\pgfqpoint{3.935665in}{3.645363in}}%
\pgfpathlineto{\pgfqpoint{3.935854in}{3.626569in}}%
\pgfpathlineto{\pgfqpoint{3.936233in}{3.570800in}}%
\pgfpathlineto{\pgfqpoint{3.936896in}{3.626254in}}%
\pgfpathlineto{\pgfqpoint{3.937276in}{3.631593in}}%
\pgfpathlineto{\pgfqpoint{3.937465in}{3.616094in}}%
\pgfpathlineto{\pgfqpoint{3.937939in}{3.534468in}}%
\pgfpathlineto{\pgfqpoint{3.938792in}{3.566327in}}%
\pgfpathlineto{\pgfqpoint{3.939455in}{3.519804in}}%
\pgfpathlineto{\pgfqpoint{3.939929in}{3.552919in}}%
\pgfpathlineto{\pgfqpoint{3.940308in}{3.569363in}}%
\pgfpathlineto{\pgfqpoint{3.940687in}{3.531607in}}%
\pgfpathlineto{\pgfqpoint{3.941066in}{3.505249in}}%
\pgfpathlineto{\pgfqpoint{3.941540in}{3.546822in}}%
\pgfpathlineto{\pgfqpoint{3.941729in}{3.557437in}}%
\pgfpathlineto{\pgfqpoint{3.942393in}{3.524188in}}%
\pgfpathlineto{\pgfqpoint{3.942677in}{3.520639in}}%
\pgfpathlineto{\pgfqpoint{3.942867in}{3.529375in}}%
\pgfpathlineto{\pgfqpoint{3.943435in}{3.565721in}}%
\pgfpathlineto{\pgfqpoint{3.943719in}{3.534342in}}%
\pgfpathlineto{\pgfqpoint{3.944004in}{3.496297in}}%
\pgfpathlineto{\pgfqpoint{3.944762in}{3.539631in}}%
\pgfpathlineto{\pgfqpoint{3.944951in}{3.559417in}}%
\pgfpathlineto{\pgfqpoint{3.945520in}{3.508166in}}%
\pgfpathlineto{\pgfqpoint{3.945804in}{3.486916in}}%
\pgfpathlineto{\pgfqpoint{3.946278in}{3.536283in}}%
\pgfpathlineto{\pgfqpoint{3.946562in}{3.550186in}}%
\pgfpathlineto{\pgfqpoint{3.947036in}{3.505719in}}%
\pgfpathlineto{\pgfqpoint{3.947131in}{3.501197in}}%
\pgfpathlineto{\pgfqpoint{3.947605in}{3.529104in}}%
\pgfpathlineto{\pgfqpoint{3.948079in}{3.573061in}}%
\pgfpathlineto{\pgfqpoint{3.948837in}{3.553909in}}%
\pgfpathlineto{\pgfqpoint{3.949026in}{3.545361in}}%
\pgfpathlineto{\pgfqpoint{3.949405in}{3.586432in}}%
\pgfpathlineto{\pgfqpoint{3.949595in}{3.600855in}}%
\pgfpathlineto{\pgfqpoint{3.950069in}{3.545014in}}%
\pgfpathlineto{\pgfqpoint{3.950542in}{3.510402in}}%
\pgfpathlineto{\pgfqpoint{3.951016in}{3.557845in}}%
\pgfpathlineto{\pgfqpoint{3.951111in}{3.557702in}}%
\pgfpathlineto{\pgfqpoint{3.952248in}{3.466806in}}%
\pgfpathlineto{\pgfqpoint{3.952722in}{3.517360in}}%
\pgfpathlineto{\pgfqpoint{3.952817in}{3.522870in}}%
\pgfpathlineto{\pgfqpoint{3.953196in}{3.480296in}}%
\pgfpathlineto{\pgfqpoint{3.953575in}{3.429767in}}%
\pgfpathlineto{\pgfqpoint{3.954049in}{3.507416in}}%
\pgfpathlineto{\pgfqpoint{3.954333in}{3.537810in}}%
\pgfpathlineto{\pgfqpoint{3.954996in}{3.494835in}}%
\pgfpathlineto{\pgfqpoint{3.955281in}{3.485275in}}%
\pgfpathlineto{\pgfqpoint{3.955660in}{3.510610in}}%
\pgfpathlineto{\pgfqpoint{3.956039in}{3.538036in}}%
\pgfpathlineto{\pgfqpoint{3.956418in}{3.493628in}}%
\pgfpathlineto{\pgfqpoint{3.956702in}{3.467848in}}%
\pgfpathlineto{\pgfqpoint{3.957176in}{3.545759in}}%
\pgfpathlineto{\pgfqpoint{3.957271in}{3.549249in}}%
\pgfpathlineto{\pgfqpoint{3.957839in}{3.541150in}}%
\pgfpathlineto{\pgfqpoint{3.958408in}{3.475419in}}%
\pgfpathlineto{\pgfqpoint{3.958882in}{3.519136in}}%
\pgfpathlineto{\pgfqpoint{3.960682in}{3.911224in}}%
\pgfpathlineto{\pgfqpoint{3.961061in}{3.854617in}}%
\pgfpathlineto{\pgfqpoint{3.962009in}{3.382791in}}%
\pgfpathlineto{\pgfqpoint{3.962862in}{3.505167in}}%
\pgfpathlineto{\pgfqpoint{3.962957in}{3.504989in}}%
\pgfpathlineto{\pgfqpoint{3.963336in}{3.530156in}}%
\pgfpathlineto{\pgfqpoint{3.963809in}{3.606536in}}%
\pgfpathlineto{\pgfqpoint{3.964473in}{3.541642in}}%
\pgfpathlineto{\pgfqpoint{3.964662in}{3.535636in}}%
\pgfpathlineto{\pgfqpoint{3.964947in}{3.569059in}}%
\pgfpathlineto{\pgfqpoint{3.965515in}{3.628379in}}%
\pgfpathlineto{\pgfqpoint{3.966084in}{3.577790in}}%
\pgfpathlineto{\pgfqpoint{3.967126in}{3.663391in}}%
\pgfpathlineto{\pgfqpoint{3.967505in}{3.612743in}}%
\pgfpathlineto{\pgfqpoint{3.967695in}{3.591901in}}%
\pgfpathlineto{\pgfqpoint{3.968358in}{3.643306in}}%
\pgfpathlineto{\pgfqpoint{3.968642in}{3.658587in}}%
\pgfpathlineto{\pgfqpoint{3.969021in}{3.624321in}}%
\pgfpathlineto{\pgfqpoint{3.969401in}{3.594647in}}%
\pgfpathlineto{\pgfqpoint{3.969780in}{3.650387in}}%
\pgfpathlineto{\pgfqpoint{3.970064in}{3.673864in}}%
\pgfpathlineto{\pgfqpoint{3.970632in}{3.627653in}}%
\pgfpathlineto{\pgfqpoint{3.971012in}{3.588905in}}%
\pgfpathlineto{\pgfqpoint{3.971391in}{3.652456in}}%
\pgfpathlineto{\pgfqpoint{3.971675in}{3.676042in}}%
\pgfpathlineto{\pgfqpoint{3.972338in}{3.634911in}}%
\pgfpathlineto{\pgfqpoint{3.972433in}{3.630757in}}%
\pgfpathlineto{\pgfqpoint{3.972717in}{3.660405in}}%
\pgfpathlineto{\pgfqpoint{3.973286in}{3.742971in}}%
\pgfpathlineto{\pgfqpoint{3.973854in}{3.681840in}}%
\pgfpathlineto{\pgfqpoint{3.974044in}{3.675319in}}%
\pgfpathlineto{\pgfqpoint{3.974328in}{3.695140in}}%
\pgfpathlineto{\pgfqpoint{3.974802in}{3.744561in}}%
\pgfpathlineto{\pgfqpoint{3.975560in}{3.715393in}}%
\pgfpathlineto{\pgfqpoint{3.975750in}{3.722744in}}%
\pgfpathlineto{\pgfqpoint{3.976508in}{3.771052in}}%
\pgfpathlineto{\pgfqpoint{3.976792in}{3.732712in}}%
\pgfpathlineto{\pgfqpoint{3.977171in}{3.691208in}}%
\pgfpathlineto{\pgfqpoint{3.977740in}{3.751047in}}%
\pgfpathlineto{\pgfqpoint{3.978024in}{3.777765in}}%
\pgfpathlineto{\pgfqpoint{3.978498in}{3.718893in}}%
\pgfpathlineto{\pgfqpoint{3.978687in}{3.694309in}}%
\pgfpathlineto{\pgfqpoint{3.979446in}{3.744417in}}%
\pgfpathlineto{\pgfqpoint{3.979635in}{3.747528in}}%
\pgfpathlineto{\pgfqpoint{3.979825in}{3.733987in}}%
\pgfpathlineto{\pgfqpoint{3.980298in}{3.658716in}}%
\pgfpathlineto{\pgfqpoint{3.980867in}{3.731805in}}%
\pgfpathlineto{\pgfqpoint{3.981151in}{3.767134in}}%
\pgfpathlineto{\pgfqpoint{3.981720in}{3.699726in}}%
\pgfpathlineto{\pgfqpoint{3.981815in}{3.697471in}}%
\pgfpathlineto{\pgfqpoint{3.982194in}{3.707169in}}%
\pgfpathlineto{\pgfqpoint{3.982762in}{3.775610in}}%
\pgfpathlineto{\pgfqpoint{3.983141in}{3.711926in}}%
\pgfpathlineto{\pgfqpoint{3.983520in}{3.659760in}}%
\pgfpathlineto{\pgfqpoint{3.984089in}{3.740008in}}%
\pgfpathlineto{\pgfqpoint{3.984373in}{3.761258in}}%
\pgfpathlineto{\pgfqpoint{3.984847in}{3.704793in}}%
\pgfpathlineto{\pgfqpoint{3.985131in}{3.683537in}}%
\pgfpathlineto{\pgfqpoint{3.985700in}{3.737375in}}%
\pgfpathlineto{\pgfqpoint{3.985795in}{3.739839in}}%
\pgfpathlineto{\pgfqpoint{3.986079in}{3.723592in}}%
\pgfpathlineto{\pgfqpoint{3.986553in}{3.669037in}}%
\pgfpathlineto{\pgfqpoint{3.987121in}{3.723526in}}%
\pgfpathlineto{\pgfqpoint{3.987500in}{3.751357in}}%
\pgfpathlineto{\pgfqpoint{3.987974in}{3.699676in}}%
\pgfpathlineto{\pgfqpoint{3.988259in}{3.674010in}}%
\pgfpathlineto{\pgfqpoint{3.988827in}{3.728871in}}%
\pgfpathlineto{\pgfqpoint{3.988922in}{3.734428in}}%
\pgfpathlineto{\pgfqpoint{3.989206in}{3.705577in}}%
\pgfpathlineto{\pgfqpoint{3.989775in}{3.659090in}}%
\pgfpathlineto{\pgfqpoint{3.990154in}{3.707098in}}%
\pgfpathlineto{\pgfqpoint{3.990533in}{3.745311in}}%
\pgfpathlineto{\pgfqpoint{3.991007in}{3.679925in}}%
\pgfpathlineto{\pgfqpoint{3.991386in}{3.641115in}}%
\pgfpathlineto{\pgfqpoint{3.991860in}{3.700178in}}%
\pgfpathlineto{\pgfqpoint{3.992144in}{3.726703in}}%
\pgfpathlineto{\pgfqpoint{3.992713in}{3.671848in}}%
\pgfpathlineto{\pgfqpoint{3.992902in}{3.660644in}}%
\pgfpathlineto{\pgfqpoint{3.993281in}{3.706315in}}%
\pgfpathlineto{\pgfqpoint{3.993660in}{3.733251in}}%
\pgfpathlineto{\pgfqpoint{3.994134in}{3.682601in}}%
\pgfpathlineto{\pgfqpoint{3.994608in}{3.619613in}}%
\pgfpathlineto{\pgfqpoint{3.995366in}{3.655805in}}%
\pgfpathlineto{\pgfqpoint{3.996029in}{3.593508in}}%
\pgfpathlineto{\pgfqpoint{3.996314in}{3.642559in}}%
\pgfpathlineto{\pgfqpoint{3.996882in}{3.727663in}}%
\pgfpathlineto{\pgfqpoint{3.997451in}{3.678348in}}%
\pgfpathlineto{\pgfqpoint{3.997735in}{3.658996in}}%
\pgfpathlineto{\pgfqpoint{3.998114in}{3.702804in}}%
\pgfpathlineto{\pgfqpoint{3.998398in}{3.726599in}}%
\pgfpathlineto{\pgfqpoint{3.998872in}{3.669932in}}%
\pgfpathlineto{\pgfqpoint{3.999251in}{3.607775in}}%
\pgfpathlineto{\pgfqpoint{3.999820in}{3.681129in}}%
\pgfpathlineto{\pgfqpoint{3.999915in}{3.678319in}}%
\pgfpathlineto{\pgfqpoint{4.000578in}{3.602200in}}%
\pgfpathlineto{\pgfqpoint{4.000862in}{3.566809in}}%
\pgfpathlineto{\pgfqpoint{4.001431in}{3.644644in}}%
\pgfpathlineto{\pgfqpoint{4.001620in}{3.653356in}}%
\pgfpathlineto{\pgfqpoint{4.001905in}{3.618903in}}%
\pgfpathlineto{\pgfqpoint{4.002284in}{3.573578in}}%
\pgfpathlineto{\pgfqpoint{4.002947in}{3.618753in}}%
\pgfpathlineto{\pgfqpoint{4.003231in}{3.649063in}}%
\pgfpathlineto{\pgfqpoint{4.003610in}{3.589502in}}%
\pgfpathlineto{\pgfqpoint{4.003989in}{3.529081in}}%
\pgfpathlineto{\pgfqpoint{4.004463in}{3.619832in}}%
\pgfpathlineto{\pgfqpoint{4.005979in}{4.035841in}}%
\pgfpathlineto{\pgfqpoint{4.006359in}{4.018089in}}%
\pgfpathlineto{\pgfqpoint{4.007022in}{3.728573in}}%
\pgfpathlineto{\pgfqpoint{4.007496in}{3.481334in}}%
\pgfpathlineto{\pgfqpoint{4.008159in}{3.650858in}}%
\pgfpathlineto{\pgfqpoint{4.008728in}{3.616109in}}%
\pgfpathlineto{\pgfqpoint{4.009012in}{3.641855in}}%
\pgfpathlineto{\pgfqpoint{4.009391in}{3.696136in}}%
\pgfpathlineto{\pgfqpoint{4.009865in}{3.617091in}}%
\pgfpathlineto{\pgfqpoint{4.010339in}{3.548437in}}%
\pgfpathlineto{\pgfqpoint{4.010812in}{3.651161in}}%
\pgfpathlineto{\pgfqpoint{4.011002in}{3.665215in}}%
\pgfpathlineto{\pgfqpoint{4.011476in}{3.619943in}}%
\pgfpathlineto{\pgfqpoint{4.011855in}{3.568242in}}%
\pgfpathlineto{\pgfqpoint{4.012423in}{3.643397in}}%
\pgfpathlineto{\pgfqpoint{4.012518in}{3.648542in}}%
\pgfpathlineto{\pgfqpoint{4.012897in}{3.621314in}}%
\pgfpathlineto{\pgfqpoint{4.013371in}{3.572125in}}%
\pgfpathlineto{\pgfqpoint{4.013750in}{3.628734in}}%
\pgfpathlineto{\pgfqpoint{4.014129in}{3.688824in}}%
\pgfpathlineto{\pgfqpoint{4.014793in}{3.614990in}}%
\pgfpathlineto{\pgfqpoint{4.014887in}{3.609063in}}%
\pgfpathlineto{\pgfqpoint{4.015266in}{3.650683in}}%
\pgfpathlineto{\pgfqpoint{4.015645in}{3.691562in}}%
\pgfpathlineto{\pgfqpoint{4.016214in}{3.619506in}}%
\pgfpathlineto{\pgfqpoint{4.016404in}{3.596480in}}%
\pgfpathlineto{\pgfqpoint{4.016877in}{3.676060in}}%
\pgfpathlineto{\pgfqpoint{4.017351in}{3.702626in}}%
\pgfpathlineto{\pgfqpoint{4.017730in}{3.665469in}}%
\pgfpathlineto{\pgfqpoint{4.018204in}{3.608339in}}%
\pgfpathlineto{\pgfqpoint{4.018583in}{3.686177in}}%
\pgfpathlineto{\pgfqpoint{4.018773in}{3.711161in}}%
\pgfpathlineto{\pgfqpoint{4.019341in}{3.621483in}}%
\pgfpathlineto{\pgfqpoint{4.019626in}{3.598117in}}%
\pgfpathlineto{\pgfqpoint{4.020099in}{3.664839in}}%
\pgfpathlineto{\pgfqpoint{4.020478in}{3.732153in}}%
\pgfpathlineto{\pgfqpoint{4.020952in}{3.616823in}}%
\pgfpathlineto{\pgfqpoint{4.021237in}{3.580268in}}%
\pgfpathlineto{\pgfqpoint{4.021805in}{3.667793in}}%
\pgfpathlineto{\pgfqpoint{4.021995in}{3.674004in}}%
\pgfpathlineto{\pgfqpoint{4.022279in}{3.647985in}}%
\pgfpathlineto{\pgfqpoint{4.022753in}{3.588529in}}%
\pgfpathlineto{\pgfqpoint{4.023227in}{3.644538in}}%
\pgfpathlineto{\pgfqpoint{4.023606in}{3.697049in}}%
\pgfpathlineto{\pgfqpoint{4.024079in}{3.616070in}}%
\pgfpathlineto{\pgfqpoint{4.024269in}{3.592249in}}%
\pgfpathlineto{\pgfqpoint{4.024932in}{3.655053in}}%
\pgfpathlineto{\pgfqpoint{4.025122in}{3.664776in}}%
\pgfpathlineto{\pgfqpoint{4.025501in}{3.626009in}}%
\pgfpathlineto{\pgfqpoint{4.025880in}{3.588437in}}%
\pgfpathlineto{\pgfqpoint{4.026354in}{3.652443in}}%
\pgfpathlineto{\pgfqpoint{4.026733in}{3.698081in}}%
\pgfpathlineto{\pgfqpoint{4.027207in}{3.617835in}}%
\pgfpathlineto{\pgfqpoint{4.027586in}{3.572538in}}%
\pgfpathlineto{\pgfqpoint{4.028060in}{3.650953in}}%
\pgfpathlineto{\pgfqpoint{4.028249in}{3.679366in}}%
\pgfpathlineto{\pgfqpoint{4.028818in}{3.585893in}}%
\pgfpathlineto{\pgfqpoint{4.029007in}{3.578386in}}%
\pgfpathlineto{\pgfqpoint{4.029481in}{3.609979in}}%
\pgfpathlineto{\pgfqpoint{4.029860in}{3.669491in}}%
\pgfpathlineto{\pgfqpoint{4.030239in}{3.582369in}}%
\pgfpathlineto{\pgfqpoint{4.030713in}{3.511510in}}%
\pgfpathlineto{\pgfqpoint{4.031187in}{3.593390in}}%
\pgfpathlineto{\pgfqpoint{4.031376in}{3.611385in}}%
\pgfpathlineto{\pgfqpoint{4.031945in}{3.561182in}}%
\pgfpathlineto{\pgfqpoint{4.032324in}{3.517384in}}%
\pgfpathlineto{\pgfqpoint{4.032703in}{3.598162in}}%
\pgfpathlineto{\pgfqpoint{4.032892in}{3.619274in}}%
\pgfpathlineto{\pgfqpoint{4.033461in}{3.550960in}}%
\pgfpathlineto{\pgfqpoint{4.033840in}{3.505765in}}%
\pgfpathlineto{\pgfqpoint{4.034314in}{3.575357in}}%
\pgfpathlineto{\pgfqpoint{4.034693in}{3.607916in}}%
\pgfpathlineto{\pgfqpoint{4.034977in}{3.561715in}}%
\pgfpathlineto{\pgfqpoint{4.035262in}{3.505272in}}%
\pgfpathlineto{\pgfqpoint{4.035925in}{3.584106in}}%
\pgfpathlineto{\pgfqpoint{4.036209in}{3.603806in}}%
\pgfpathlineto{\pgfqpoint{4.036588in}{3.553322in}}%
\pgfpathlineto{\pgfqpoint{4.036873in}{3.507204in}}%
\pgfpathlineto{\pgfqpoint{4.037441in}{3.602589in}}%
\pgfpathlineto{\pgfqpoint{4.037820in}{3.671008in}}%
\pgfpathlineto{\pgfqpoint{4.038389in}{3.570833in}}%
\pgfpathlineto{\pgfqpoint{4.038673in}{3.590998in}}%
\pgfpathlineto{\pgfqpoint{4.039336in}{3.672231in}}%
\pgfpathlineto{\pgfqpoint{4.039905in}{3.608837in}}%
\pgfpathlineto{\pgfqpoint{4.040189in}{3.582653in}}%
\pgfpathlineto{\pgfqpoint{4.040568in}{3.657315in}}%
\pgfpathlineto{\pgfqpoint{4.040853in}{3.685222in}}%
\pgfpathlineto{\pgfqpoint{4.041327in}{3.619547in}}%
\pgfpathlineto{\pgfqpoint{4.041706in}{3.571130in}}%
\pgfpathlineto{\pgfqpoint{4.042369in}{3.617507in}}%
\pgfpathlineto{\pgfqpoint{4.042464in}{3.624479in}}%
\pgfpathlineto{\pgfqpoint{4.042748in}{3.582354in}}%
\pgfpathlineto{\pgfqpoint{4.043127in}{3.495788in}}%
\pgfpathlineto{\pgfqpoint{4.043790in}{3.589009in}}%
\pgfpathlineto{\pgfqpoint{4.044075in}{3.626292in}}%
\pgfpathlineto{\pgfqpoint{4.044643in}{3.540521in}}%
\pgfpathlineto{\pgfqpoint{4.044833in}{3.528399in}}%
\pgfpathlineto{\pgfqpoint{4.045212in}{3.567830in}}%
\pgfpathlineto{\pgfqpoint{4.045591in}{3.631887in}}%
\pgfpathlineto{\pgfqpoint{4.046065in}{3.544164in}}%
\pgfpathlineto{\pgfqpoint{4.046349in}{3.504803in}}%
\pgfpathlineto{\pgfqpoint{4.046823in}{3.603673in}}%
\pgfpathlineto{\pgfqpoint{4.047107in}{3.645261in}}%
\pgfpathlineto{\pgfqpoint{4.047676in}{3.551935in}}%
\pgfpathlineto{\pgfqpoint{4.048055in}{3.529680in}}%
\pgfpathlineto{\pgfqpoint{4.048434in}{3.575346in}}%
\pgfpathlineto{\pgfqpoint{4.048718in}{3.622247in}}%
\pgfpathlineto{\pgfqpoint{4.049192in}{3.530787in}}%
\pgfpathlineto{\pgfqpoint{4.049476in}{3.470484in}}%
\pgfpathlineto{\pgfqpoint{4.049950in}{3.623066in}}%
\pgfpathlineto{\pgfqpoint{4.050613in}{3.950904in}}%
\pgfpathlineto{\pgfqpoint{4.051466in}{3.909619in}}%
\pgfpathlineto{\pgfqpoint{4.051845in}{3.946953in}}%
\pgfpathlineto{\pgfqpoint{4.052130in}{3.875378in}}%
\pgfpathlineto{\pgfqpoint{4.052793in}{3.395422in}}%
\pgfpathlineto{\pgfqpoint{4.053551in}{3.634555in}}%
\pgfpathlineto{\pgfqpoint{4.053646in}{3.639954in}}%
\pgfpathlineto{\pgfqpoint{4.053930in}{3.598351in}}%
\pgfpathlineto{\pgfqpoint{4.054309in}{3.529146in}}%
\pgfpathlineto{\pgfqpoint{4.054878in}{3.618018in}}%
\pgfpathlineto{\pgfqpoint{4.055067in}{3.630748in}}%
\pgfpathlineto{\pgfqpoint{4.055446in}{3.572489in}}%
\pgfpathlineto{\pgfqpoint{4.055825in}{3.518419in}}%
\pgfpathlineto{\pgfqpoint{4.056299in}{3.608414in}}%
\pgfpathlineto{\pgfqpoint{4.056584in}{3.632079in}}%
\pgfpathlineto{\pgfqpoint{4.056963in}{3.580265in}}%
\pgfpathlineto{\pgfqpoint{4.057436in}{3.468265in}}%
\pgfpathlineto{\pgfqpoint{4.058005in}{3.567191in}}%
\pgfpathlineto{\pgfqpoint{4.058195in}{3.583017in}}%
\pgfpathlineto{\pgfqpoint{4.058574in}{3.536879in}}%
\pgfpathlineto{\pgfqpoint{4.058953in}{3.489481in}}%
\pgfpathlineto{\pgfqpoint{4.059426in}{3.559246in}}%
\pgfpathlineto{\pgfqpoint{4.059711in}{3.593584in}}%
\pgfpathlineto{\pgfqpoint{4.060185in}{3.516003in}}%
\pgfpathlineto{\pgfqpoint{4.060564in}{3.476324in}}%
\pgfpathlineto{\pgfqpoint{4.061037in}{3.550195in}}%
\pgfpathlineto{\pgfqpoint{4.061322in}{3.599920in}}%
\pgfpathlineto{\pgfqpoint{4.061890in}{3.512167in}}%
\pgfpathlineto{\pgfqpoint{4.062080in}{3.497209in}}%
\pgfpathlineto{\pgfqpoint{4.062554in}{3.555041in}}%
\pgfpathlineto{\pgfqpoint{4.062933in}{3.627249in}}%
\pgfpathlineto{\pgfqpoint{4.063407in}{3.528959in}}%
\pgfpathlineto{\pgfqpoint{4.063691in}{3.497399in}}%
\pgfpathlineto{\pgfqpoint{4.064165in}{3.587314in}}%
\pgfpathlineto{\pgfqpoint{4.064449in}{3.626032in}}%
\pgfpathlineto{\pgfqpoint{4.065018in}{3.541896in}}%
\pgfpathlineto{\pgfqpoint{4.065302in}{3.523206in}}%
\pgfpathlineto{\pgfqpoint{4.065681in}{3.580840in}}%
\pgfpathlineto{\pgfqpoint{4.065965in}{3.626088in}}%
\pgfpathlineto{\pgfqpoint{4.066534in}{3.537002in}}%
\pgfpathlineto{\pgfqpoint{4.066913in}{3.499822in}}%
\pgfpathlineto{\pgfqpoint{4.067197in}{3.568096in}}%
\pgfpathlineto{\pgfqpoint{4.067576in}{3.657096in}}%
\pgfpathlineto{\pgfqpoint{4.068145in}{3.552585in}}%
\pgfpathlineto{\pgfqpoint{4.068334in}{3.530251in}}%
\pgfpathlineto{\pgfqpoint{4.068903in}{3.597001in}}%
\pgfpathlineto{\pgfqpoint{4.069187in}{3.619358in}}%
\pgfpathlineto{\pgfqpoint{4.069566in}{3.566768in}}%
\pgfpathlineto{\pgfqpoint{4.069945in}{3.502355in}}%
\pgfpathlineto{\pgfqpoint{4.070514in}{3.589714in}}%
\pgfpathlineto{\pgfqpoint{4.070703in}{3.604456in}}%
\pgfpathlineto{\pgfqpoint{4.071082in}{3.550709in}}%
\pgfpathlineto{\pgfqpoint{4.071651in}{3.452024in}}%
\pgfpathlineto{\pgfqpoint{4.072220in}{3.520712in}}%
\pgfpathlineto{\pgfqpoint{4.072788in}{3.432452in}}%
\pgfpathlineto{\pgfqpoint{4.073072in}{3.409647in}}%
\pgfpathlineto{\pgfqpoint{4.073357in}{3.460390in}}%
\pgfpathlineto{\pgfqpoint{4.073925in}{3.620553in}}%
\pgfpathlineto{\pgfqpoint{4.074589in}{3.515229in}}%
\pgfpathlineto{\pgfqpoint{4.074778in}{3.490649in}}%
\pgfpathlineto{\pgfqpoint{4.075252in}{3.571123in}}%
\pgfpathlineto{\pgfqpoint{4.075442in}{3.586115in}}%
\pgfpathlineto{\pgfqpoint{4.075821in}{3.537450in}}%
\pgfpathlineto{\pgfqpoint{4.076294in}{3.459242in}}%
\pgfpathlineto{\pgfqpoint{4.076863in}{3.554361in}}%
\pgfpathlineto{\pgfqpoint{4.077053in}{3.577836in}}%
\pgfpathlineto{\pgfqpoint{4.077432in}{3.496629in}}%
\pgfpathlineto{\pgfqpoint{4.077811in}{3.429532in}}%
\pgfpathlineto{\pgfqpoint{4.078379in}{3.511037in}}%
\pgfpathlineto{\pgfqpoint{4.078664in}{3.547434in}}%
\pgfpathlineto{\pgfqpoint{4.079137in}{3.458964in}}%
\pgfpathlineto{\pgfqpoint{4.079327in}{3.452640in}}%
\pgfpathlineto{\pgfqpoint{4.079706in}{3.480043in}}%
\pgfpathlineto{\pgfqpoint{4.080180in}{3.582470in}}%
\pgfpathlineto{\pgfqpoint{4.080654in}{3.475312in}}%
\pgfpathlineto{\pgfqpoint{4.081033in}{3.425227in}}%
\pgfpathlineto{\pgfqpoint{4.081412in}{3.507844in}}%
\pgfpathlineto{\pgfqpoint{4.081791in}{3.558088in}}%
\pgfpathlineto{\pgfqpoint{4.082265in}{3.477024in}}%
\pgfpathlineto{\pgfqpoint{4.082454in}{3.455687in}}%
\pgfpathlineto{\pgfqpoint{4.082833in}{3.528178in}}%
\pgfpathlineto{\pgfqpoint{4.083307in}{3.644895in}}%
\pgfpathlineto{\pgfqpoint{4.083876in}{3.532269in}}%
\pgfpathlineto{\pgfqpoint{4.084160in}{3.504504in}}%
\pgfpathlineto{\pgfqpoint{4.084539in}{3.584463in}}%
\pgfpathlineto{\pgfqpoint{4.085013in}{3.678238in}}%
\pgfpathlineto{\pgfqpoint{4.085581in}{3.586381in}}%
\pgfpathlineto{\pgfqpoint{4.085676in}{3.583427in}}%
\pgfpathlineto{\pgfqpoint{4.085866in}{3.596342in}}%
\pgfpathlineto{\pgfqpoint{4.086529in}{3.709351in}}%
\pgfpathlineto{\pgfqpoint{4.086908in}{3.611194in}}%
\pgfpathlineto{\pgfqpoint{4.087287in}{3.524203in}}%
\pgfpathlineto{\pgfqpoint{4.087856in}{3.660204in}}%
\pgfpathlineto{\pgfqpoint{4.088045in}{3.681130in}}%
\pgfpathlineto{\pgfqpoint{4.088519in}{3.585937in}}%
\pgfpathlineto{\pgfqpoint{4.088803in}{3.562849in}}%
\pgfpathlineto{\pgfqpoint{4.089182in}{3.621143in}}%
\pgfpathlineto{\pgfqpoint{4.089561in}{3.668005in}}%
\pgfpathlineto{\pgfqpoint{4.090035in}{3.596790in}}%
\pgfpathlineto{\pgfqpoint{4.090414in}{3.548259in}}%
\pgfpathlineto{\pgfqpoint{4.090888in}{3.620111in}}%
\pgfpathlineto{\pgfqpoint{4.091172in}{3.679477in}}%
\pgfpathlineto{\pgfqpoint{4.091741in}{3.575739in}}%
\pgfpathlineto{\pgfqpoint{4.092025in}{3.536326in}}%
\pgfpathlineto{\pgfqpoint{4.092689in}{3.602755in}}%
\pgfpathlineto{\pgfqpoint{4.093162in}{3.525789in}}%
\pgfpathlineto{\pgfqpoint{4.093542in}{3.444947in}}%
\pgfpathlineto{\pgfqpoint{4.094205in}{3.534475in}}%
\pgfpathlineto{\pgfqpoint{4.094963in}{3.431467in}}%
\pgfpathlineto{\pgfqpoint{4.095247in}{3.514580in}}%
\pgfpathlineto{\pgfqpoint{4.095911in}{3.847177in}}%
\pgfpathlineto{\pgfqpoint{4.096574in}{3.682102in}}%
\pgfpathlineto{\pgfqpoint{4.097143in}{3.701540in}}%
\pgfpathlineto{\pgfqpoint{4.097522in}{3.577771in}}%
\pgfpathlineto{\pgfqpoint{4.098090in}{3.168407in}}%
\pgfpathlineto{\pgfqpoint{4.098754in}{3.374528in}}%
\pgfpathlineto{\pgfqpoint{4.099038in}{3.435269in}}%
\pgfpathlineto{\pgfqpoint{4.099512in}{3.323941in}}%
\pgfpathlineto{\pgfqpoint{4.099796in}{3.276232in}}%
\pgfpathlineto{\pgfqpoint{4.100270in}{3.379534in}}%
\pgfpathlineto{\pgfqpoint{4.100554in}{3.433919in}}%
\pgfpathlineto{\pgfqpoint{4.101123in}{3.344250in}}%
\pgfpathlineto{\pgfqpoint{4.101407in}{3.295697in}}%
\pgfpathlineto{\pgfqpoint{4.101976in}{3.389278in}}%
\pgfpathlineto{\pgfqpoint{4.102165in}{3.403817in}}%
\pgfpathlineto{\pgfqpoint{4.102639in}{3.344276in}}%
\pgfpathlineto{\pgfqpoint{4.102923in}{3.283263in}}%
\pgfpathlineto{\pgfqpoint{4.103492in}{3.414193in}}%
\pgfpathlineto{\pgfqpoint{4.103776in}{3.447179in}}%
\pgfpathlineto{\pgfqpoint{4.104155in}{3.378378in}}%
\pgfpathlineto{\pgfqpoint{4.104534in}{3.310990in}}%
\pgfpathlineto{\pgfqpoint{4.105103in}{3.426969in}}%
\pgfpathlineto{\pgfqpoint{4.105387in}{3.467775in}}%
\pgfpathlineto{\pgfqpoint{4.105956in}{3.374440in}}%
\pgfpathlineto{\pgfqpoint{4.106050in}{3.369936in}}%
\pgfpathlineto{\pgfqpoint{4.106335in}{3.403897in}}%
\pgfpathlineto{\pgfqpoint{4.106903in}{3.523364in}}%
\pgfpathlineto{\pgfqpoint{4.107472in}{3.428904in}}%
\pgfpathlineto{\pgfqpoint{4.107661in}{3.399632in}}%
\pgfpathlineto{\pgfqpoint{4.108230in}{3.484801in}}%
\pgfpathlineto{\pgfqpoint{4.108609in}{3.524626in}}%
\pgfpathlineto{\pgfqpoint{4.108988in}{3.450247in}}%
\pgfpathlineto{\pgfqpoint{4.109178in}{3.421424in}}%
\pgfpathlineto{\pgfqpoint{4.109651in}{3.513311in}}%
\pgfpathlineto{\pgfqpoint{4.110030in}{3.591869in}}%
\pgfpathlineto{\pgfqpoint{4.110504in}{3.478459in}}%
\pgfpathlineto{\pgfqpoint{4.110883in}{3.426914in}}%
\pgfpathlineto{\pgfqpoint{4.111357in}{3.526932in}}%
\pgfpathlineto{\pgfqpoint{4.111641in}{3.575326in}}%
\pgfpathlineto{\pgfqpoint{4.112210in}{3.484809in}}%
\pgfpathlineto{\pgfqpoint{4.112400in}{3.466800in}}%
\pgfpathlineto{\pgfqpoint{4.112779in}{3.547150in}}%
\pgfpathlineto{\pgfqpoint{4.113158in}{3.613681in}}%
\pgfpathlineto{\pgfqpoint{4.113632in}{3.520062in}}%
\pgfpathlineto{\pgfqpoint{4.113916in}{3.471979in}}%
\pgfpathlineto{\pgfqpoint{4.114484in}{3.582727in}}%
\pgfpathlineto{\pgfqpoint{4.114769in}{3.640449in}}%
\pgfpathlineto{\pgfqpoint{4.115432in}{3.560342in}}%
\pgfpathlineto{\pgfqpoint{4.115622in}{3.548307in}}%
\pgfpathlineto{\pgfqpoint{4.115906in}{3.593139in}}%
\pgfpathlineto{\pgfqpoint{4.116285in}{3.677357in}}%
\pgfpathlineto{\pgfqpoint{4.116948in}{3.576165in}}%
\pgfpathlineto{\pgfqpoint{4.117138in}{3.562110in}}%
\pgfpathlineto{\pgfqpoint{4.117422in}{3.619773in}}%
\pgfpathlineto{\pgfqpoint{4.117991in}{3.734921in}}%
\pgfpathlineto{\pgfqpoint{4.118559in}{3.630200in}}%
\pgfpathlineto{\pgfqpoint{4.118749in}{3.618265in}}%
\pgfpathlineto{\pgfqpoint{4.119128in}{3.694976in}}%
\pgfpathlineto{\pgfqpoint{4.119507in}{3.756211in}}%
\pgfpathlineto{\pgfqpoint{4.119981in}{3.645944in}}%
\pgfpathlineto{\pgfqpoint{4.120265in}{3.614350in}}%
\pgfpathlineto{\pgfqpoint{4.120739in}{3.709387in}}%
\pgfpathlineto{\pgfqpoint{4.121118in}{3.749701in}}%
\pgfpathlineto{\pgfqpoint{4.121592in}{3.673296in}}%
\pgfpathlineto{\pgfqpoint{4.121876in}{3.607610in}}%
\pgfpathlineto{\pgfqpoint{4.122539in}{3.716877in}}%
\pgfpathlineto{\pgfqpoint{4.122634in}{3.718041in}}%
\pgfpathlineto{\pgfqpoint{4.122729in}{3.710648in}}%
\pgfpathlineto{\pgfqpoint{4.123392in}{3.544156in}}%
\pgfpathlineto{\pgfqpoint{4.123961in}{3.683023in}}%
\pgfpathlineto{\pgfqpoint{4.124150in}{3.709556in}}%
\pgfpathlineto{\pgfqpoint{4.124624in}{3.630241in}}%
\pgfpathlineto{\pgfqpoint{4.125098in}{3.546163in}}%
\pgfpathlineto{\pgfqpoint{4.125572in}{3.649239in}}%
\pgfpathlineto{\pgfqpoint{4.125667in}{3.658241in}}%
\pgfpathlineto{\pgfqpoint{4.126046in}{3.602698in}}%
\pgfpathlineto{\pgfqpoint{4.126614in}{3.495294in}}%
\pgfpathlineto{\pgfqpoint{4.127088in}{3.602731in}}%
\pgfpathlineto{\pgfqpoint{4.127278in}{3.628715in}}%
\pgfpathlineto{\pgfqpoint{4.127751in}{3.515338in}}%
\pgfpathlineto{\pgfqpoint{4.128225in}{3.440332in}}%
\pgfpathlineto{\pgfqpoint{4.128604in}{3.533637in}}%
\pgfpathlineto{\pgfqpoint{4.128794in}{3.576769in}}%
\pgfpathlineto{\pgfqpoint{4.129362in}{3.475465in}}%
\pgfpathlineto{\pgfqpoint{4.129741in}{3.398576in}}%
\pgfpathlineto{\pgfqpoint{4.130215in}{3.530683in}}%
\pgfpathlineto{\pgfqpoint{4.130310in}{3.542802in}}%
\pgfpathlineto{\pgfqpoint{4.130784in}{3.495818in}}%
\pgfpathlineto{\pgfqpoint{4.131352in}{3.335621in}}%
\pgfpathlineto{\pgfqpoint{4.132016in}{3.457662in}}%
\pgfpathlineto{\pgfqpoint{4.132395in}{3.365802in}}%
\pgfpathlineto{\pgfqpoint{4.132869in}{3.250635in}}%
\pgfpathlineto{\pgfqpoint{4.133437in}{3.357088in}}%
\pgfpathlineto{\pgfqpoint{4.133627in}{3.369462in}}%
\pgfpathlineto{\pgfqpoint{4.133911in}{3.324523in}}%
\pgfpathlineto{\pgfqpoint{4.134480in}{3.197663in}}%
\pgfpathlineto{\pgfqpoint{4.135048in}{3.305067in}}%
\pgfpathlineto{\pgfqpoint{4.135332in}{3.337440in}}%
\pgfpathlineto{\pgfqpoint{4.135712in}{3.238612in}}%
\pgfpathlineto{\pgfqpoint{4.135996in}{3.193154in}}%
\pgfpathlineto{\pgfqpoint{4.136470in}{3.316397in}}%
\pgfpathlineto{\pgfqpoint{4.136754in}{3.362383in}}%
\pgfpathlineto{\pgfqpoint{4.137228in}{3.261232in}}%
\pgfpathlineto{\pgfqpoint{4.137607in}{3.168530in}}%
\pgfpathlineto{\pgfqpoint{4.138175in}{3.307870in}}%
\pgfpathlineto{\pgfqpoint{4.138365in}{3.325316in}}%
\pgfpathlineto{\pgfqpoint{4.138744in}{3.273583in}}%
\pgfpathlineto{\pgfqpoint{4.139123in}{3.199557in}}%
\pgfpathlineto{\pgfqpoint{4.139692in}{3.291942in}}%
\pgfpathlineto{\pgfqpoint{4.140545in}{3.373475in}}%
\pgfpathlineto{\pgfqpoint{4.141492in}{3.761624in}}%
\pgfpathlineto{\pgfqpoint{4.142250in}{3.607696in}}%
\pgfpathlineto{\pgfqpoint{4.143672in}{3.261829in}}%
\pgfpathlineto{\pgfqpoint{4.144240in}{3.361475in}}%
\pgfpathlineto{\pgfqpoint{4.144714in}{3.472494in}}%
\pgfpathlineto{\pgfqpoint{4.145283in}{3.366812in}}%
\pgfpathlineto{\pgfqpoint{4.145377in}{3.365638in}}%
\pgfpathlineto{\pgfqpoint{4.145567in}{3.373509in}}%
\pgfpathlineto{\pgfqpoint{4.146136in}{3.494540in}}%
\pgfpathlineto{\pgfqpoint{4.146704in}{3.393023in}}%
\pgfpathlineto{\pgfqpoint{4.146988in}{3.368796in}}%
\pgfpathlineto{\pgfqpoint{4.147368in}{3.443574in}}%
\pgfpathlineto{\pgfqpoint{4.147841in}{3.539178in}}%
\pgfpathlineto{\pgfqpoint{4.148410in}{3.431959in}}%
\pgfpathlineto{\pgfqpoint{4.148505in}{3.422561in}}%
\pgfpathlineto{\pgfqpoint{4.148884in}{3.467343in}}%
\pgfpathlineto{\pgfqpoint{4.149358in}{3.536886in}}%
\pgfpathlineto{\pgfqpoint{4.149737in}{3.448603in}}%
\pgfpathlineto{\pgfqpoint{4.150116in}{3.363823in}}%
\pgfpathlineto{\pgfqpoint{4.150684in}{3.485940in}}%
\pgfpathlineto{\pgfqpoint{4.150874in}{3.511978in}}%
\pgfpathlineto{\pgfqpoint{4.151348in}{3.433036in}}%
\pgfpathlineto{\pgfqpoint{4.151632in}{3.387462in}}%
\pgfpathlineto{\pgfqpoint{4.152106in}{3.497525in}}%
\pgfpathlineto{\pgfqpoint{4.153717in}{3.661146in}}%
\pgfpathlineto{\pgfqpoint{4.154096in}{3.735176in}}%
\pgfpathlineto{\pgfqpoint{4.154570in}{3.635028in}}%
\pgfpathlineto{\pgfqpoint{4.154759in}{3.609094in}}%
\pgfpathlineto{\pgfqpoint{4.155233in}{3.687008in}}%
\pgfpathlineto{\pgfqpoint{4.155612in}{3.748655in}}%
\pgfpathlineto{\pgfqpoint{4.156181in}{3.654469in}}%
\pgfpathlineto{\pgfqpoint{4.156370in}{3.634730in}}%
\pgfpathlineto{\pgfqpoint{4.156749in}{3.696147in}}%
\pgfpathlineto{\pgfqpoint{4.157223in}{3.820738in}}%
\pgfpathlineto{\pgfqpoint{4.157792in}{3.670545in}}%
\pgfpathlineto{\pgfqpoint{4.157981in}{3.629359in}}%
\pgfpathlineto{\pgfqpoint{4.158644in}{3.742546in}}%
\pgfpathlineto{\pgfqpoint{4.158834in}{3.759842in}}%
\pgfpathlineto{\pgfqpoint{4.159118in}{3.705200in}}%
\pgfpathlineto{\pgfqpoint{4.159497in}{3.615811in}}%
\pgfpathlineto{\pgfqpoint{4.160066in}{3.737155in}}%
\pgfpathlineto{\pgfqpoint{4.160350in}{3.781908in}}%
\pgfpathlineto{\pgfqpoint{4.160824in}{3.675001in}}%
\pgfpathlineto{\pgfqpoint{4.161108in}{3.620054in}}%
\pgfpathlineto{\pgfqpoint{4.161677in}{3.715195in}}%
\pgfpathlineto{\pgfqpoint{4.161961in}{3.732223in}}%
\pgfpathlineto{\pgfqpoint{4.162340in}{3.687886in}}%
\pgfpathlineto{\pgfqpoint{4.162719in}{3.590143in}}%
\pgfpathlineto{\pgfqpoint{4.163288in}{3.728829in}}%
\pgfpathlineto{\pgfqpoint{4.163383in}{3.738431in}}%
\pgfpathlineto{\pgfqpoint{4.163762in}{3.680685in}}%
\pgfpathlineto{\pgfqpoint{4.164330in}{3.507102in}}%
\pgfpathlineto{\pgfqpoint{4.164994in}{3.618291in}}%
\pgfpathlineto{\pgfqpoint{4.165088in}{3.620652in}}%
\pgfpathlineto{\pgfqpoint{4.165278in}{3.602595in}}%
\pgfpathlineto{\pgfqpoint{4.165847in}{3.431492in}}%
\pgfpathlineto{\pgfqpoint{4.166605in}{3.543016in}}%
\pgfpathlineto{\pgfqpoint{4.167173in}{3.388954in}}%
\pgfpathlineto{\pgfqpoint{4.167458in}{3.324976in}}%
\pgfpathlineto{\pgfqpoint{4.168026in}{3.435588in}}%
\pgfpathlineto{\pgfqpoint{4.168216in}{3.455905in}}%
\pgfpathlineto{\pgfqpoint{4.168595in}{3.392054in}}%
\pgfpathlineto{\pgfqpoint{4.168974in}{3.299413in}}%
\pgfpathlineto{\pgfqpoint{4.169637in}{3.415993in}}%
\pgfpathlineto{\pgfqpoint{4.169732in}{3.422349in}}%
\pgfpathlineto{\pgfqpoint{4.170016in}{3.374756in}}%
\pgfpathlineto{\pgfqpoint{4.170680in}{3.230671in}}%
\pgfpathlineto{\pgfqpoint{4.171153in}{3.343805in}}%
\pgfpathlineto{\pgfqpoint{4.171438in}{3.376262in}}%
\pgfpathlineto{\pgfqpoint{4.171911in}{3.275701in}}%
\pgfpathlineto{\pgfqpoint{4.172196in}{3.238759in}}%
\pgfpathlineto{\pgfqpoint{4.172670in}{3.331839in}}%
\pgfpathlineto{\pgfqpoint{4.172954in}{3.357313in}}%
\pgfpathlineto{\pgfqpoint{4.173333in}{3.278301in}}%
\pgfpathlineto{\pgfqpoint{4.173712in}{3.204475in}}%
\pgfpathlineto{\pgfqpoint{4.174186in}{3.329029in}}%
\pgfpathlineto{\pgfqpoint{4.174470in}{3.380819in}}%
\pgfpathlineto{\pgfqpoint{4.175133in}{3.293809in}}%
\pgfpathlineto{\pgfqpoint{4.175323in}{3.274263in}}%
\pgfpathlineto{\pgfqpoint{4.175702in}{3.332204in}}%
\pgfpathlineto{\pgfqpoint{4.176081in}{3.411495in}}%
\pgfpathlineto{\pgfqpoint{4.176555in}{3.285795in}}%
\pgfpathlineto{\pgfqpoint{4.176839in}{3.236378in}}%
\pgfpathlineto{\pgfqpoint{4.177313in}{3.332606in}}%
\pgfpathlineto{\pgfqpoint{4.177597in}{3.386742in}}%
\pgfpathlineto{\pgfqpoint{4.178166in}{3.265140in}}%
\pgfpathlineto{\pgfqpoint{4.178450in}{3.209061in}}%
\pgfpathlineto{\pgfqpoint{4.179114in}{3.299399in}}%
\pgfpathlineto{\pgfqpoint{4.179303in}{3.308872in}}%
\pgfpathlineto{\pgfqpoint{4.179587in}{3.253352in}}%
\pgfpathlineto{\pgfqpoint{4.179966in}{3.176385in}}%
\pgfpathlineto{\pgfqpoint{4.180440in}{3.272000in}}%
\pgfpathlineto{\pgfqpoint{4.180819in}{3.376390in}}%
\pgfpathlineto{\pgfqpoint{4.181388in}{3.247833in}}%
\pgfpathlineto{\pgfqpoint{4.181577in}{3.232938in}}%
\pgfpathlineto{\pgfqpoint{4.181956in}{3.293273in}}%
\pgfpathlineto{\pgfqpoint{4.182430in}{3.387972in}}%
\pgfpathlineto{\pgfqpoint{4.182904in}{3.266609in}}%
\pgfpathlineto{\pgfqpoint{4.183094in}{3.248615in}}%
\pgfpathlineto{\pgfqpoint{4.183473in}{3.307465in}}%
\pgfpathlineto{\pgfqpoint{4.183946in}{3.444776in}}%
\pgfpathlineto{\pgfqpoint{4.184515in}{3.295239in}}%
\pgfpathlineto{\pgfqpoint{4.184799in}{3.226360in}}%
\pgfpathlineto{\pgfqpoint{4.185273in}{3.366631in}}%
\pgfpathlineto{\pgfqpoint{4.186979in}{3.882426in}}%
\pgfpathlineto{\pgfqpoint{4.187358in}{3.807816in}}%
\pgfpathlineto{\pgfqpoint{4.188306in}{3.294406in}}%
\pgfpathlineto{\pgfqpoint{4.189253in}{3.443913in}}%
\pgfpathlineto{\pgfqpoint{4.189443in}{3.425430in}}%
\pgfpathlineto{\pgfqpoint{4.189822in}{3.526276in}}%
\pgfpathlineto{\pgfqpoint{4.190296in}{3.633445in}}%
\pgfpathlineto{\pgfqpoint{4.190770in}{3.519907in}}%
\pgfpathlineto{\pgfqpoint{4.191054in}{3.478098in}}%
\pgfpathlineto{\pgfqpoint{4.191433in}{3.577641in}}%
\pgfpathlineto{\pgfqpoint{4.191812in}{3.665074in}}%
\pgfpathlineto{\pgfqpoint{4.192475in}{3.553472in}}%
\pgfpathlineto{\pgfqpoint{4.192570in}{3.541221in}}%
\pgfpathlineto{\pgfqpoint{4.192854in}{3.598081in}}%
\pgfpathlineto{\pgfqpoint{4.193423in}{3.735159in}}%
\pgfpathlineto{\pgfqpoint{4.193897in}{3.616907in}}%
\pgfpathlineto{\pgfqpoint{4.194181in}{3.583323in}}%
\pgfpathlineto{\pgfqpoint{4.194655in}{3.695329in}}%
\pgfpathlineto{\pgfqpoint{4.194939in}{3.779872in}}%
\pgfpathlineto{\pgfqpoint{4.195602in}{3.632216in}}%
\pgfpathlineto{\pgfqpoint{4.195792in}{3.620842in}}%
\pgfpathlineto{\pgfqpoint{4.196076in}{3.681312in}}%
\pgfpathlineto{\pgfqpoint{4.196455in}{3.803834in}}%
\pgfpathlineto{\pgfqpoint{4.197024in}{3.654203in}}%
\pgfpathlineto{\pgfqpoint{4.197308in}{3.607950in}}%
\pgfpathlineto{\pgfqpoint{4.197782in}{3.744781in}}%
\pgfpathlineto{\pgfqpoint{4.198161in}{3.805676in}}%
\pgfpathlineto{\pgfqpoint{4.198635in}{3.690144in}}%
\pgfpathlineto{\pgfqpoint{4.198824in}{3.659696in}}%
\pgfpathlineto{\pgfqpoint{4.199393in}{3.763991in}}%
\pgfpathlineto{\pgfqpoint{4.199677in}{3.798850in}}%
\pgfpathlineto{\pgfqpoint{4.200056in}{3.712852in}}%
\pgfpathlineto{\pgfqpoint{4.200435in}{3.612667in}}%
\pgfpathlineto{\pgfqpoint{4.201004in}{3.724168in}}%
\pgfpathlineto{\pgfqpoint{4.201288in}{3.768994in}}%
\pgfpathlineto{\pgfqpoint{4.201667in}{3.664817in}}%
\pgfpathlineto{\pgfqpoint{4.202046in}{3.576460in}}%
\pgfpathlineto{\pgfqpoint{4.202615in}{3.684393in}}%
\pgfpathlineto{\pgfqpoint{4.202805in}{3.694744in}}%
\pgfpathlineto{\pgfqpoint{4.203089in}{3.644944in}}%
\pgfpathlineto{\pgfqpoint{4.203563in}{3.506600in}}%
\pgfpathlineto{\pgfqpoint{4.204226in}{3.624444in}}%
\pgfpathlineto{\pgfqpoint{4.204416in}{3.649304in}}%
\pgfpathlineto{\pgfqpoint{4.204795in}{3.548595in}}%
\pgfpathlineto{\pgfqpoint{4.205174in}{3.456757in}}%
\pgfpathlineto{\pgfqpoint{4.205742in}{3.563883in}}%
\pgfpathlineto{\pgfqpoint{4.205932in}{3.590696in}}%
\pgfpathlineto{\pgfqpoint{4.206311in}{3.486627in}}%
\pgfpathlineto{\pgfqpoint{4.206690in}{3.375174in}}%
\pgfpathlineto{\pgfqpoint{4.207258in}{3.510047in}}%
\pgfpathlineto{\pgfqpoint{4.207543in}{3.564321in}}%
\pgfpathlineto{\pgfqpoint{4.208017in}{3.444795in}}%
\pgfpathlineto{\pgfqpoint{4.208301in}{3.379759in}}%
\pgfpathlineto{\pgfqpoint{4.208964in}{3.488047in}}%
\pgfpathlineto{\pgfqpoint{4.209059in}{3.498043in}}%
\pgfpathlineto{\pgfqpoint{4.209438in}{3.430467in}}%
\pgfpathlineto{\pgfqpoint{4.209912in}{3.322680in}}%
\pgfpathlineto{\pgfqpoint{4.210386in}{3.436235in}}%
\pgfpathlineto{\pgfqpoint{4.210670in}{3.484800in}}%
\pgfpathlineto{\pgfqpoint{4.211144in}{3.370612in}}%
\pgfpathlineto{\pgfqpoint{4.211523in}{3.276458in}}%
\pgfpathlineto{\pgfqpoint{4.212091in}{3.414045in}}%
\pgfpathlineto{\pgfqpoint{4.212281in}{3.426137in}}%
\pgfpathlineto{\pgfqpoint{4.212565in}{3.367616in}}%
\pgfpathlineto{\pgfqpoint{4.212944in}{3.243210in}}%
\pgfpathlineto{\pgfqpoint{4.213513in}{3.376561in}}%
\pgfpathlineto{\pgfqpoint{4.213797in}{3.423011in}}%
\pgfpathlineto{\pgfqpoint{4.214271in}{3.318234in}}%
\pgfpathlineto{\pgfqpoint{4.214650in}{3.226558in}}%
\pgfpathlineto{\pgfqpoint{4.215124in}{3.359182in}}%
\pgfpathlineto{\pgfqpoint{4.215408in}{3.407763in}}%
\pgfpathlineto{\pgfqpoint{4.215882in}{3.289785in}}%
\pgfpathlineto{\pgfqpoint{4.216166in}{3.249133in}}%
\pgfpathlineto{\pgfqpoint{4.216640in}{3.363916in}}%
\pgfpathlineto{\pgfqpoint{4.216924in}{3.426782in}}%
\pgfpathlineto{\pgfqpoint{4.217493in}{3.290657in}}%
\pgfpathlineto{\pgfqpoint{4.217777in}{3.239462in}}%
\pgfpathlineto{\pgfqpoint{4.218251in}{3.366124in}}%
\pgfpathlineto{\pgfqpoint{4.218630in}{3.432362in}}%
\pgfpathlineto{\pgfqpoint{4.219199in}{3.316481in}}%
\pgfpathlineto{\pgfqpoint{4.219294in}{3.306029in}}%
\pgfpathlineto{\pgfqpoint{4.219578in}{3.362326in}}%
\pgfpathlineto{\pgfqpoint{4.220146in}{3.513042in}}%
\pgfpathlineto{\pgfqpoint{4.220620in}{3.381853in}}%
\pgfpathlineto{\pgfqpoint{4.220904in}{3.328720in}}%
\pgfpathlineto{\pgfqpoint{4.221378in}{3.438675in}}%
\pgfpathlineto{\pgfqpoint{4.221663in}{3.511916in}}%
\pgfpathlineto{\pgfqpoint{4.222231in}{3.366986in}}%
\pgfpathlineto{\pgfqpoint{4.222515in}{3.323128in}}%
\pgfpathlineto{\pgfqpoint{4.222989in}{3.426348in}}%
\pgfpathlineto{\pgfqpoint{4.223179in}{3.456042in}}%
\pgfpathlineto{\pgfqpoint{4.223653in}{3.342527in}}%
\pgfpathlineto{\pgfqpoint{4.224032in}{3.232763in}}%
\pgfpathlineto{\pgfqpoint{4.224600in}{3.410129in}}%
\pgfpathlineto{\pgfqpoint{4.224790in}{3.443835in}}%
\pgfpathlineto{\pgfqpoint{4.225358in}{3.336914in}}%
\pgfpathlineto{\pgfqpoint{4.225643in}{3.308081in}}%
\pgfpathlineto{\pgfqpoint{4.226022in}{3.393969in}}%
\pgfpathlineto{\pgfqpoint{4.226401in}{3.473097in}}%
\pgfpathlineto{\pgfqpoint{4.226875in}{3.352496in}}%
\pgfpathlineto{\pgfqpoint{4.227159in}{3.298250in}}%
\pgfpathlineto{\pgfqpoint{4.227633in}{3.410517in}}%
\pgfpathlineto{\pgfqpoint{4.228012in}{3.511109in}}%
\pgfpathlineto{\pgfqpoint{4.228580in}{3.358344in}}%
\pgfpathlineto{\pgfqpoint{4.228770in}{3.343092in}}%
\pgfpathlineto{\pgfqpoint{4.229149in}{3.421947in}}%
\pgfpathlineto{\pgfqpoint{4.229433in}{3.488774in}}%
\pgfpathlineto{\pgfqpoint{4.229907in}{3.359200in}}%
\pgfpathlineto{\pgfqpoint{4.230097in}{3.327077in}}%
\pgfpathlineto{\pgfqpoint{4.230476in}{3.420556in}}%
\pgfpathlineto{\pgfqpoint{4.231234in}{3.881645in}}%
\pgfpathlineto{\pgfqpoint{4.231992in}{3.696578in}}%
\pgfpathlineto{\pgfqpoint{4.232371in}{3.779269in}}%
\pgfpathlineto{\pgfqpoint{4.232750in}{3.635203in}}%
\pgfpathlineto{\pgfqpoint{4.233224in}{3.330901in}}%
\pgfpathlineto{\pgfqpoint{4.233887in}{3.577729in}}%
\pgfpathlineto{\pgfqpoint{4.234266in}{3.693341in}}%
\pgfpathlineto{\pgfqpoint{4.234835in}{3.528161in}}%
\pgfpathlineto{\pgfqpoint{4.235119in}{3.482595in}}%
\pgfpathlineto{\pgfqpoint{4.235593in}{3.597818in}}%
\pgfpathlineto{\pgfqpoint{4.235782in}{3.618635in}}%
\pgfpathlineto{\pgfqpoint{4.236256in}{3.531889in}}%
\pgfpathlineto{\pgfqpoint{4.236635in}{3.426323in}}%
\pgfpathlineto{\pgfqpoint{4.237204in}{3.581569in}}%
\pgfpathlineto{\pgfqpoint{4.237299in}{3.591664in}}%
\pgfpathlineto{\pgfqpoint{4.237678in}{3.516950in}}%
\pgfpathlineto{\pgfqpoint{4.238246in}{3.356441in}}%
\pgfpathlineto{\pgfqpoint{4.238815in}{3.483473in}}%
\pgfpathlineto{\pgfqpoint{4.238910in}{3.494047in}}%
\pgfpathlineto{\pgfqpoint{4.239289in}{3.423651in}}%
\pgfpathlineto{\pgfqpoint{4.239763in}{3.285138in}}%
\pgfpathlineto{\pgfqpoint{4.240331in}{3.426661in}}%
\pgfpathlineto{\pgfqpoint{4.240521in}{3.448987in}}%
\pgfpathlineto{\pgfqpoint{4.240900in}{3.356560in}}%
\pgfpathlineto{\pgfqpoint{4.241374in}{3.248888in}}%
\pgfpathlineto{\pgfqpoint{4.241847in}{3.355943in}}%
\pgfpathlineto{\pgfqpoint{4.242132in}{3.424185in}}%
\pgfpathlineto{\pgfqpoint{4.242700in}{3.286883in}}%
\pgfpathlineto{\pgfqpoint{4.242890in}{3.253542in}}%
\pgfpathlineto{\pgfqpoint{4.243364in}{3.378850in}}%
\pgfpathlineto{\pgfqpoint{4.243648in}{3.442610in}}%
\pgfpathlineto{\pgfqpoint{4.244216in}{3.323789in}}%
\pgfpathlineto{\pgfqpoint{4.244501in}{3.252496in}}%
\pgfpathlineto{\pgfqpoint{4.245069in}{3.399843in}}%
\pgfpathlineto{\pgfqpoint{4.245354in}{3.431139in}}%
\pgfpathlineto{\pgfqpoint{4.245733in}{3.339428in}}%
\pgfpathlineto{\pgfqpoint{4.246017in}{3.278202in}}%
\pgfpathlineto{\pgfqpoint{4.246586in}{3.402480in}}%
\pgfpathlineto{\pgfqpoint{4.246870in}{3.459756in}}%
\pgfpathlineto{\pgfqpoint{4.247344in}{3.342207in}}%
\pgfpathlineto{\pgfqpoint{4.247628in}{3.293694in}}%
\pgfpathlineto{\pgfqpoint{4.248102in}{3.389887in}}%
\pgfpathlineto{\pgfqpoint{4.248481in}{3.472840in}}%
\pgfpathlineto{\pgfqpoint{4.248955in}{3.353259in}}%
\pgfpathlineto{\pgfqpoint{4.249144in}{3.325958in}}%
\pgfpathlineto{\pgfqpoint{4.249618in}{3.416461in}}%
\pgfpathlineto{\pgfqpoint{4.249997in}{3.497533in}}%
\pgfpathlineto{\pgfqpoint{4.250471in}{3.376352in}}%
\pgfpathlineto{\pgfqpoint{4.250850in}{3.324912in}}%
\pgfpathlineto{\pgfqpoint{4.251229in}{3.424436in}}%
\pgfpathlineto{\pgfqpoint{4.251608in}{3.497025in}}%
\pgfpathlineto{\pgfqpoint{4.252082in}{3.364089in}}%
\pgfpathlineto{\pgfqpoint{4.252366in}{3.329896in}}%
\pgfpathlineto{\pgfqpoint{4.252840in}{3.427231in}}%
\pgfpathlineto{\pgfqpoint{4.253124in}{3.482898in}}%
\pgfpathlineto{\pgfqpoint{4.253598in}{3.343898in}}%
\pgfpathlineto{\pgfqpoint{4.253882in}{3.266615in}}%
\pgfpathlineto{\pgfqpoint{4.254451in}{3.398281in}}%
\pgfpathlineto{\pgfqpoint{4.254735in}{3.433502in}}%
\pgfpathlineto{\pgfqpoint{4.255114in}{3.344556in}}%
\pgfpathlineto{\pgfqpoint{4.255493in}{3.258234in}}%
\pgfpathlineto{\pgfqpoint{4.256062in}{3.385679in}}%
\pgfpathlineto{\pgfqpoint{4.256252in}{3.405434in}}%
\pgfpathlineto{\pgfqpoint{4.256631in}{3.318094in}}%
\pgfpathlineto{\pgfqpoint{4.257010in}{3.220058in}}%
\pgfpathlineto{\pgfqpoint{4.257578in}{3.366058in}}%
\pgfpathlineto{\pgfqpoint{4.257862in}{3.415188in}}%
\pgfpathlineto{\pgfqpoint{4.258336in}{3.317581in}}%
\pgfpathlineto{\pgfqpoint{4.258621in}{3.261132in}}%
\pgfpathlineto{\pgfqpoint{4.259189in}{3.389716in}}%
\pgfpathlineto{\pgfqpoint{4.259473in}{3.426441in}}%
\pgfpathlineto{\pgfqpoint{4.259853in}{3.304443in}}%
\pgfpathlineto{\pgfqpoint{4.260137in}{3.242498in}}%
\pgfpathlineto{\pgfqpoint{4.260611in}{3.356635in}}%
\pgfpathlineto{\pgfqpoint{4.260990in}{3.444299in}}%
\pgfpathlineto{\pgfqpoint{4.261558in}{3.335238in}}%
\pgfpathlineto{\pgfqpoint{4.261843in}{3.299192in}}%
\pgfpathlineto{\pgfqpoint{4.262222in}{3.400902in}}%
\pgfpathlineto{\pgfqpoint{4.262601in}{3.478367in}}%
\pgfpathlineto{\pgfqpoint{4.263169in}{3.344446in}}%
\pgfpathlineto{\pgfqpoint{4.263359in}{3.329527in}}%
\pgfpathlineto{\pgfqpoint{4.263643in}{3.391503in}}%
\pgfpathlineto{\pgfqpoint{4.264117in}{3.550892in}}%
\pgfpathlineto{\pgfqpoint{4.264780in}{3.419638in}}%
\pgfpathlineto{\pgfqpoint{4.264970in}{3.410215in}}%
\pgfpathlineto{\pgfqpoint{4.265254in}{3.450943in}}%
\pgfpathlineto{\pgfqpoint{4.265728in}{3.558961in}}%
\pgfpathlineto{\pgfqpoint{4.266202in}{3.427348in}}%
\pgfpathlineto{\pgfqpoint{4.266391in}{3.389539in}}%
\pgfpathlineto{\pgfqpoint{4.266960in}{3.508141in}}%
\pgfpathlineto{\pgfqpoint{4.267244in}{3.560353in}}%
\pgfpathlineto{\pgfqpoint{4.267718in}{3.423068in}}%
\pgfpathlineto{\pgfqpoint{4.268097in}{3.321317in}}%
\pgfpathlineto{\pgfqpoint{4.268666in}{3.480309in}}%
\pgfpathlineto{\pgfqpoint{4.268855in}{3.509398in}}%
\pgfpathlineto{\pgfqpoint{4.269329in}{3.379855in}}%
\pgfpathlineto{\pgfqpoint{4.269613in}{3.334114in}}%
\pgfpathlineto{\pgfqpoint{4.270087in}{3.446228in}}%
\pgfpathlineto{\pgfqpoint{4.270466in}{3.530276in}}%
\pgfpathlineto{\pgfqpoint{4.270940in}{3.381804in}}%
\pgfpathlineto{\pgfqpoint{4.271224in}{3.332565in}}%
\pgfpathlineto{\pgfqpoint{4.271793in}{3.459871in}}%
\pgfpathlineto{\pgfqpoint{4.272077in}{3.512343in}}%
\pgfpathlineto{\pgfqpoint{4.272551in}{3.375042in}}%
\pgfpathlineto{\pgfqpoint{4.272740in}{3.349380in}}%
\pgfpathlineto{\pgfqpoint{4.273120in}{3.428293in}}%
\pgfpathlineto{\pgfqpoint{4.273499in}{3.547081in}}%
\pgfpathlineto{\pgfqpoint{4.274067in}{3.409697in}}%
\pgfpathlineto{\pgfqpoint{4.274446in}{3.312679in}}%
\pgfpathlineto{\pgfqpoint{4.274920in}{3.475015in}}%
\pgfpathlineto{\pgfqpoint{4.276721in}{3.918658in}}%
\pgfpathlineto{\pgfqpoint{4.276815in}{3.909022in}}%
\pgfpathlineto{\pgfqpoint{4.277479in}{3.473407in}}%
\pgfpathlineto{\pgfqpoint{4.277858in}{3.320274in}}%
\pgfpathlineto{\pgfqpoint{4.278521in}{3.510818in}}%
\pgfpathlineto{\pgfqpoint{4.279090in}{3.423966in}}%
\pgfpathlineto{\pgfqpoint{4.279469in}{3.486104in}}%
\pgfpathlineto{\pgfqpoint{4.279848in}{3.597311in}}%
\pgfpathlineto{\pgfqpoint{4.280416in}{3.428873in}}%
\pgfpathlineto{\pgfqpoint{4.280606in}{3.390857in}}%
\pgfpathlineto{\pgfqpoint{4.281080in}{3.511934in}}%
\pgfpathlineto{\pgfqpoint{4.281459in}{3.613514in}}%
\pgfpathlineto{\pgfqpoint{4.282027in}{3.467763in}}%
\pgfpathlineto{\pgfqpoint{4.282217in}{3.449887in}}%
\pgfpathlineto{\pgfqpoint{4.282596in}{3.526776in}}%
\pgfpathlineto{\pgfqpoint{4.282975in}{3.623084in}}%
\pgfpathlineto{\pgfqpoint{4.283449in}{3.494456in}}%
\pgfpathlineto{\pgfqpoint{4.283828in}{3.410478in}}%
\pgfpathlineto{\pgfqpoint{4.284302in}{3.564390in}}%
\pgfpathlineto{\pgfqpoint{4.284586in}{3.631106in}}%
\pgfpathlineto{\pgfqpoint{4.285155in}{3.495637in}}%
\pgfpathlineto{\pgfqpoint{4.285439in}{3.450845in}}%
\pgfpathlineto{\pgfqpoint{4.285818in}{3.572532in}}%
\pgfpathlineto{\pgfqpoint{4.286102in}{3.630499in}}%
\pgfpathlineto{\pgfqpoint{4.286671in}{3.485215in}}%
\pgfpathlineto{\pgfqpoint{4.286955in}{3.444889in}}%
\pgfpathlineto{\pgfqpoint{4.287429in}{3.557081in}}%
\pgfpathlineto{\pgfqpoint{4.287713in}{3.648311in}}%
\pgfpathlineto{\pgfqpoint{4.288282in}{3.489554in}}%
\pgfpathlineto{\pgfqpoint{4.288471in}{3.462946in}}%
\pgfpathlineto{\pgfqpoint{4.288945in}{3.568923in}}%
\pgfpathlineto{\pgfqpoint{4.289229in}{3.633138in}}%
\pgfpathlineto{\pgfqpoint{4.289798in}{3.510703in}}%
\pgfpathlineto{\pgfqpoint{4.290082in}{3.458753in}}%
\pgfpathlineto{\pgfqpoint{4.290556in}{3.588735in}}%
\pgfpathlineto{\pgfqpoint{4.290935in}{3.663431in}}%
\pgfpathlineto{\pgfqpoint{4.291409in}{3.514351in}}%
\pgfpathlineto{\pgfqpoint{4.291599in}{3.472203in}}%
\pgfpathlineto{\pgfqpoint{4.292167in}{3.591805in}}%
\pgfpathlineto{\pgfqpoint{4.292451in}{3.646610in}}%
\pgfpathlineto{\pgfqpoint{4.292925in}{3.535769in}}%
\pgfpathlineto{\pgfqpoint{4.293210in}{3.490932in}}%
\pgfpathlineto{\pgfqpoint{4.293683in}{3.612577in}}%
\pgfpathlineto{\pgfqpoint{4.294062in}{3.687117in}}%
\pgfpathlineto{\pgfqpoint{4.294536in}{3.558231in}}%
\pgfpathlineto{\pgfqpoint{4.294726in}{3.512603in}}%
\pgfpathlineto{\pgfqpoint{4.295294in}{3.623350in}}%
\pgfpathlineto{\pgfqpoint{4.295579in}{3.686263in}}%
\pgfpathlineto{\pgfqpoint{4.296052in}{3.565047in}}%
\pgfpathlineto{\pgfqpoint{4.296337in}{3.521908in}}%
\pgfpathlineto{\pgfqpoint{4.296811in}{3.643430in}}%
\pgfpathlineto{\pgfqpoint{4.297190in}{3.733261in}}%
\pgfpathlineto{\pgfqpoint{4.297663in}{3.568691in}}%
\pgfpathlineto{\pgfqpoint{4.297948in}{3.511248in}}%
\pgfpathlineto{\pgfqpoint{4.298516in}{3.631788in}}%
\pgfpathlineto{\pgfqpoint{4.298706in}{3.659544in}}%
\pgfpathlineto{\pgfqpoint{4.299180in}{3.556196in}}%
\pgfpathlineto{\pgfqpoint{4.299464in}{3.501606in}}%
\pgfpathlineto{\pgfqpoint{4.300033in}{3.619127in}}%
\pgfpathlineto{\pgfqpoint{4.300317in}{3.677710in}}%
\pgfpathlineto{\pgfqpoint{4.300696in}{3.549514in}}%
\pgfpathlineto{\pgfqpoint{4.301075in}{3.435743in}}%
\pgfpathlineto{\pgfqpoint{4.301644in}{3.579795in}}%
\pgfpathlineto{\pgfqpoint{4.301928in}{3.610602in}}%
\pgfpathlineto{\pgfqpoint{4.302307in}{3.527232in}}%
\pgfpathlineto{\pgfqpoint{4.302686in}{3.442875in}}%
\pgfpathlineto{\pgfqpoint{4.303160in}{3.576318in}}%
\pgfpathlineto{\pgfqpoint{4.303349in}{3.604108in}}%
\pgfpathlineto{\pgfqpoint{4.303823in}{3.536834in}}%
\pgfpathlineto{\pgfqpoint{4.304202in}{3.417687in}}%
\pgfpathlineto{\pgfqpoint{4.304771in}{3.568301in}}%
\pgfpathlineto{\pgfqpoint{4.305055in}{3.610184in}}%
\pgfpathlineto{\pgfqpoint{4.305434in}{3.513006in}}%
\pgfpathlineto{\pgfqpoint{4.305718in}{3.445657in}}%
\pgfpathlineto{\pgfqpoint{4.306287in}{3.577476in}}%
\pgfpathlineto{\pgfqpoint{4.306571in}{3.638228in}}%
\pgfpathlineto{\pgfqpoint{4.307045in}{3.500463in}}%
\pgfpathlineto{\pgfqpoint{4.307424in}{3.418004in}}%
\pgfpathlineto{\pgfqpoint{4.307898in}{3.575354in}}%
\pgfpathlineto{\pgfqpoint{4.308182in}{3.634507in}}%
\pgfpathlineto{\pgfqpoint{4.308656in}{3.515897in}}%
\pgfpathlineto{\pgfqpoint{4.309035in}{3.445165in}}%
\pgfpathlineto{\pgfqpoint{4.309509in}{3.583603in}}%
\pgfpathlineto{\pgfqpoint{4.309698in}{3.606609in}}%
\pgfpathlineto{\pgfqpoint{4.310078in}{3.495578in}}%
\pgfpathlineto{\pgfqpoint{4.310457in}{3.384867in}}%
\pgfpathlineto{\pgfqpoint{4.311025in}{3.555442in}}%
\pgfpathlineto{\pgfqpoint{4.311404in}{3.662273in}}%
\pgfpathlineto{\pgfqpoint{4.311973in}{3.511522in}}%
\pgfpathlineto{\pgfqpoint{4.312162in}{3.494377in}}%
\pgfpathlineto{\pgfqpoint{4.312447in}{3.569086in}}%
\pgfpathlineto{\pgfqpoint{4.312826in}{3.653967in}}%
\pgfpathlineto{\pgfqpoint{4.313299in}{3.499363in}}%
\pgfpathlineto{\pgfqpoint{4.313679in}{3.411375in}}%
\pgfpathlineto{\pgfqpoint{4.314152in}{3.541914in}}%
\pgfpathlineto{\pgfqpoint{4.314531in}{3.630827in}}%
\pgfpathlineto{\pgfqpoint{4.315005in}{3.478592in}}%
\pgfpathlineto{\pgfqpoint{4.315290in}{3.440707in}}%
\pgfpathlineto{\pgfqpoint{4.315763in}{3.569045in}}%
\pgfpathlineto{\pgfqpoint{4.315953in}{3.596742in}}%
\pgfpathlineto{\pgfqpoint{4.316427in}{3.499810in}}%
\pgfpathlineto{\pgfqpoint{4.316806in}{3.419147in}}%
\pgfpathlineto{\pgfqpoint{4.317280in}{3.566964in}}%
\pgfpathlineto{\pgfqpoint{4.317659in}{3.637064in}}%
\pgfpathlineto{\pgfqpoint{4.318132in}{3.489856in}}%
\pgfpathlineto{\pgfqpoint{4.318417in}{3.448943in}}%
\pgfpathlineto{\pgfqpoint{4.318891in}{3.552602in}}%
\pgfpathlineto{\pgfqpoint{4.319080in}{3.589004in}}%
\pgfpathlineto{\pgfqpoint{4.319554in}{3.479060in}}%
\pgfpathlineto{\pgfqpoint{4.319743in}{3.448527in}}%
\pgfpathlineto{\pgfqpoint{4.320123in}{3.575451in}}%
\pgfpathlineto{\pgfqpoint{4.320786in}{4.039010in}}%
\pgfpathlineto{\pgfqpoint{4.321449in}{3.844567in}}%
\pgfpathlineto{\pgfqpoint{4.322018in}{3.893481in}}%
\pgfpathlineto{\pgfqpoint{4.322586in}{3.558792in}}%
\pgfpathlineto{\pgfqpoint{4.322965in}{3.336456in}}%
\pgfpathlineto{\pgfqpoint{4.323534in}{3.613223in}}%
\pgfpathlineto{\pgfqpoint{4.323913in}{3.718346in}}%
\pgfpathlineto{\pgfqpoint{4.324387in}{3.564170in}}%
\pgfpathlineto{\pgfqpoint{4.324671in}{3.511864in}}%
\pgfpathlineto{\pgfqpoint{4.325240in}{3.653641in}}%
\pgfpathlineto{\pgfqpoint{4.325524in}{3.704671in}}%
\pgfpathlineto{\pgfqpoint{4.325998in}{3.548805in}}%
\pgfpathlineto{\pgfqpoint{4.326187in}{3.531190in}}%
\pgfpathlineto{\pgfqpoint{4.326566in}{3.597709in}}%
\pgfpathlineto{\pgfqpoint{4.327040in}{3.739803in}}%
\pgfpathlineto{\pgfqpoint{4.327514in}{3.579448in}}%
\pgfpathlineto{\pgfqpoint{4.327798in}{3.503867in}}%
\pgfpathlineto{\pgfqpoint{4.328367in}{3.650529in}}%
\pgfpathlineto{\pgfqpoint{4.328651in}{3.704711in}}%
\pgfpathlineto{\pgfqpoint{4.329125in}{3.575259in}}%
\pgfpathlineto{\pgfqpoint{4.329409in}{3.532746in}}%
\pgfpathlineto{\pgfqpoint{4.329788in}{3.641515in}}%
\pgfpathlineto{\pgfqpoint{4.330168in}{3.730457in}}%
\pgfpathlineto{\pgfqpoint{4.330641in}{3.591467in}}%
\pgfpathlineto{\pgfqpoint{4.330926in}{3.511274in}}%
\pgfpathlineto{\pgfqpoint{4.331494in}{3.635686in}}%
\pgfpathlineto{\pgfqpoint{4.331779in}{3.709343in}}%
\pgfpathlineto{\pgfqpoint{4.332347in}{3.555768in}}%
\pgfpathlineto{\pgfqpoint{4.332537in}{3.540275in}}%
\pgfpathlineto{\pgfqpoint{4.332821in}{3.599803in}}%
\pgfpathlineto{\pgfqpoint{4.333295in}{3.726558in}}%
\pgfpathlineto{\pgfqpoint{4.333769in}{3.599637in}}%
\pgfpathlineto{\pgfqpoint{4.334053in}{3.518168in}}%
\pgfpathlineto{\pgfqpoint{4.334621in}{3.672169in}}%
\pgfpathlineto{\pgfqpoint{4.334906in}{3.728500in}}%
\pgfpathlineto{\pgfqpoint{4.335380in}{3.611294in}}%
\pgfpathlineto{\pgfqpoint{4.335664in}{3.568060in}}%
\pgfpathlineto{\pgfqpoint{4.336138in}{3.671456in}}%
\pgfpathlineto{\pgfqpoint{4.336422in}{3.728536in}}%
\pgfpathlineto{\pgfqpoint{4.336896in}{3.625117in}}%
\pgfpathlineto{\pgfqpoint{4.337180in}{3.536282in}}%
\pgfpathlineto{\pgfqpoint{4.337749in}{3.678010in}}%
\pgfpathlineto{\pgfqpoint{4.338128in}{3.752598in}}%
\pgfpathlineto{\pgfqpoint{4.338602in}{3.619672in}}%
\pgfpathlineto{\pgfqpoint{4.338886in}{3.584046in}}%
\pgfpathlineto{\pgfqpoint{4.339265in}{3.681387in}}%
\pgfpathlineto{\pgfqpoint{4.339549in}{3.753238in}}%
\pgfpathlineto{\pgfqpoint{4.340118in}{3.608106in}}%
\pgfpathlineto{\pgfqpoint{4.340402in}{3.551080in}}%
\pgfpathlineto{\pgfqpoint{4.340876in}{3.690391in}}%
\pgfpathlineto{\pgfqpoint{4.341255in}{3.765408in}}%
\pgfpathlineto{\pgfqpoint{4.341729in}{3.629654in}}%
\pgfpathlineto{\pgfqpoint{4.342013in}{3.580461in}}%
\pgfpathlineto{\pgfqpoint{4.342487in}{3.697241in}}%
\pgfpathlineto{\pgfqpoint{4.342771in}{3.738167in}}%
\pgfpathlineto{\pgfqpoint{4.343150in}{3.617732in}}%
\pgfpathlineto{\pgfqpoint{4.343529in}{3.527681in}}%
\pgfpathlineto{\pgfqpoint{4.344003in}{3.673437in}}%
\pgfpathlineto{\pgfqpoint{4.344287in}{3.738914in}}%
\pgfpathlineto{\pgfqpoint{4.344856in}{3.590890in}}%
\pgfpathlineto{\pgfqpoint{4.345140in}{3.515044in}}%
\pgfpathlineto{\pgfqpoint{4.345804in}{3.651274in}}%
\pgfpathlineto{\pgfqpoint{4.345898in}{3.651907in}}%
\pgfpathlineto{\pgfqpoint{4.345993in}{3.647624in}}%
\pgfpathlineto{\pgfqpoint{4.346656in}{3.443347in}}%
\pgfpathlineto{\pgfqpoint{4.347130in}{3.612484in}}%
\pgfpathlineto{\pgfqpoint{4.347415in}{3.676026in}}%
\pgfpathlineto{\pgfqpoint{4.347888in}{3.551046in}}%
\pgfpathlineto{\pgfqpoint{4.348267in}{3.469349in}}%
\pgfpathlineto{\pgfqpoint{4.348836in}{3.596963in}}%
\pgfpathlineto{\pgfqpoint{4.349120in}{3.632783in}}%
\pgfpathlineto{\pgfqpoint{4.349499in}{3.508675in}}%
\pgfpathlineto{\pgfqpoint{4.349784in}{3.449066in}}%
\pgfpathlineto{\pgfqpoint{4.350258in}{3.568056in}}%
\pgfpathlineto{\pgfqpoint{4.350637in}{3.659822in}}%
\pgfpathlineto{\pgfqpoint{4.351110in}{3.503276in}}%
\pgfpathlineto{\pgfqpoint{4.351395in}{3.453128in}}%
\pgfpathlineto{\pgfqpoint{4.351868in}{3.567469in}}%
\pgfpathlineto{\pgfqpoint{4.352153in}{3.628955in}}%
\pgfpathlineto{\pgfqpoint{4.352627in}{3.512328in}}%
\pgfpathlineto{\pgfqpoint{4.352911in}{3.459370in}}%
\pgfpathlineto{\pgfqpoint{4.353385in}{3.563722in}}%
\pgfpathlineto{\pgfqpoint{4.353764in}{3.660814in}}%
\pgfpathlineto{\pgfqpoint{4.354332in}{3.518459in}}%
\pgfpathlineto{\pgfqpoint{4.354617in}{3.477242in}}%
\pgfpathlineto{\pgfqpoint{4.354996in}{3.593675in}}%
\pgfpathlineto{\pgfqpoint{4.355375in}{3.682381in}}%
\pgfpathlineto{\pgfqpoint{4.355849in}{3.557744in}}%
\pgfpathlineto{\pgfqpoint{4.356038in}{3.522189in}}%
\pgfpathlineto{\pgfqpoint{4.356607in}{3.644739in}}%
\pgfpathlineto{\pgfqpoint{4.356891in}{3.714242in}}%
\pgfpathlineto{\pgfqpoint{4.357365in}{3.547210in}}%
\pgfpathlineto{\pgfqpoint{4.357649in}{3.452966in}}%
\pgfpathlineto{\pgfqpoint{4.358312in}{3.596498in}}%
\pgfpathlineto{\pgfqpoint{4.358502in}{3.611960in}}%
\pgfpathlineto{\pgfqpoint{4.358786in}{3.539205in}}%
\pgfpathlineto{\pgfqpoint{4.359260in}{3.425281in}}%
\pgfpathlineto{\pgfqpoint{4.359734in}{3.558687in}}%
\pgfpathlineto{\pgfqpoint{4.360018in}{3.617226in}}%
\pgfpathlineto{\pgfqpoint{4.360492in}{3.483394in}}%
\pgfpathlineto{\pgfqpoint{4.360871in}{3.389643in}}%
\pgfpathlineto{\pgfqpoint{4.361345in}{3.531274in}}%
\pgfpathlineto{\pgfqpoint{4.361629in}{3.610115in}}%
\pgfpathlineto{\pgfqpoint{4.362198in}{3.449588in}}%
\pgfpathlineto{\pgfqpoint{4.362387in}{3.423237in}}%
\pgfpathlineto{\pgfqpoint{4.362766in}{3.531238in}}%
\pgfpathlineto{\pgfqpoint{4.363145in}{3.618525in}}%
\pgfpathlineto{\pgfqpoint{4.363619in}{3.498293in}}%
\pgfpathlineto{\pgfqpoint{4.363998in}{3.419600in}}%
\pgfpathlineto{\pgfqpoint{4.364472in}{3.559597in}}%
\pgfpathlineto{\pgfqpoint{4.364756in}{3.638235in}}%
\pgfpathlineto{\pgfqpoint{4.365230in}{3.491599in}}%
\pgfpathlineto{\pgfqpoint{4.365515in}{3.403707in}}%
\pgfpathlineto{\pgfqpoint{4.365988in}{3.600230in}}%
\pgfpathlineto{\pgfqpoint{4.366652in}{3.917833in}}%
\pgfpathlineto{\pgfqpoint{4.367410in}{3.873293in}}%
\pgfpathlineto{\pgfqpoint{4.367884in}{3.978445in}}%
\pgfpathlineto{\pgfqpoint{4.368168in}{3.873818in}}%
\pgfpathlineto{\pgfqpoint{4.368831in}{3.340484in}}%
\pgfpathlineto{\pgfqpoint{4.369495in}{3.641167in}}%
\pgfpathlineto{\pgfqpoint{4.369589in}{3.642772in}}%
\pgfpathlineto{\pgfqpoint{4.369684in}{3.628498in}}%
\pgfpathlineto{\pgfqpoint{4.370253in}{3.452118in}}%
\pgfpathlineto{\pgfqpoint{4.370727in}{3.625531in}}%
\pgfpathlineto{\pgfqpoint{4.371106in}{3.687718in}}%
\pgfpathlineto{\pgfqpoint{4.371579in}{3.549469in}}%
\pgfpathlineto{\pgfqpoint{4.371864in}{3.498291in}}%
\pgfpathlineto{\pgfqpoint{4.372338in}{3.631007in}}%
\pgfpathlineto{\pgfqpoint{4.372622in}{3.675752in}}%
\pgfpathlineto{\pgfqpoint{4.373096in}{3.544867in}}%
\pgfpathlineto{\pgfqpoint{4.373380in}{3.503365in}}%
\pgfpathlineto{\pgfqpoint{4.373759in}{3.591967in}}%
\pgfpathlineto{\pgfqpoint{4.374233in}{3.692630in}}%
\pgfpathlineto{\pgfqpoint{4.374707in}{3.580218in}}%
\pgfpathlineto{\pgfqpoint{4.374991in}{3.531347in}}%
\pgfpathlineto{\pgfqpoint{4.375465in}{3.635419in}}%
\pgfpathlineto{\pgfqpoint{4.375844in}{3.703830in}}%
\pgfpathlineto{\pgfqpoint{4.376318in}{3.566515in}}%
\pgfpathlineto{\pgfqpoint{4.376507in}{3.527717in}}%
\pgfpathlineto{\pgfqpoint{4.376981in}{3.659920in}}%
\pgfpathlineto{\pgfqpoint{4.377360in}{3.743967in}}%
\pgfpathlineto{\pgfqpoint{4.377834in}{3.602197in}}%
\pgfpathlineto{\pgfqpoint{4.378118in}{3.527282in}}%
\pgfpathlineto{\pgfqpoint{4.378687in}{3.690569in}}%
\pgfpathlineto{\pgfqpoint{4.378876in}{3.723331in}}%
\pgfpathlineto{\pgfqpoint{4.379350in}{3.617511in}}%
\pgfpathlineto{\pgfqpoint{4.379634in}{3.558990in}}%
\pgfpathlineto{\pgfqpoint{4.380108in}{3.674455in}}%
\pgfpathlineto{\pgfqpoint{4.380487in}{3.782243in}}%
\pgfpathlineto{\pgfqpoint{4.381056in}{3.601658in}}%
\pgfpathlineto{\pgfqpoint{4.381340in}{3.558609in}}%
\pgfpathlineto{\pgfqpoint{4.381719in}{3.666869in}}%
\pgfpathlineto{\pgfqpoint{4.382098in}{3.759832in}}%
\pgfpathlineto{\pgfqpoint{4.382572in}{3.609616in}}%
\pgfpathlineto{\pgfqpoint{4.382762in}{3.580296in}}%
\pgfpathlineto{\pgfqpoint{4.383235in}{3.672780in}}%
\pgfpathlineto{\pgfqpoint{4.383614in}{3.768850in}}%
\pgfpathlineto{\pgfqpoint{4.384088in}{3.631174in}}%
\pgfpathlineto{\pgfqpoint{4.384467in}{3.561096in}}%
\pgfpathlineto{\pgfqpoint{4.384941in}{3.698384in}}%
\pgfpathlineto{\pgfqpoint{4.385225in}{3.756468in}}%
\pgfpathlineto{\pgfqpoint{4.385794in}{3.607056in}}%
\pgfpathlineto{\pgfqpoint{4.386078in}{3.568583in}}%
\pgfpathlineto{\pgfqpoint{4.386552in}{3.696692in}}%
\pgfpathlineto{\pgfqpoint{4.386742in}{3.724712in}}%
\pgfpathlineto{\pgfqpoint{4.387121in}{3.625114in}}%
\pgfpathlineto{\pgfqpoint{4.387595in}{3.474902in}}%
\pgfpathlineto{\pgfqpoint{4.388163in}{3.641965in}}%
\pgfpathlineto{\pgfqpoint{4.388353in}{3.670952in}}%
\pgfpathlineto{\pgfqpoint{4.388732in}{3.545200in}}%
\pgfpathlineto{\pgfqpoint{4.389016in}{3.476510in}}%
\pgfpathlineto{\pgfqpoint{4.389585in}{3.631338in}}%
\pgfpathlineto{\pgfqpoint{4.389964in}{3.728068in}}%
\pgfpathlineto{\pgfqpoint{4.390437in}{3.579276in}}%
\pgfpathlineto{\pgfqpoint{4.390722in}{3.529623in}}%
\pgfpathlineto{\pgfqpoint{4.391196in}{3.663172in}}%
\pgfpathlineto{\pgfqpoint{4.391480in}{3.730470in}}%
\pgfpathlineto{\pgfqpoint{4.391954in}{3.595951in}}%
\pgfpathlineto{\pgfqpoint{4.392238in}{3.524661in}}%
\pgfpathlineto{\pgfqpoint{4.392807in}{3.648766in}}%
\pgfpathlineto{\pgfqpoint{4.392996in}{3.679268in}}%
\pgfpathlineto{\pgfqpoint{4.393470in}{3.569350in}}%
\pgfpathlineto{\pgfqpoint{4.393849in}{3.461803in}}%
\pgfpathlineto{\pgfqpoint{4.394323in}{3.605921in}}%
\pgfpathlineto{\pgfqpoint{4.394607in}{3.678510in}}%
\pgfpathlineto{\pgfqpoint{4.395176in}{3.538173in}}%
\pgfpathlineto{\pgfqpoint{4.395460in}{3.468040in}}%
\pgfpathlineto{\pgfqpoint{4.396029in}{3.623374in}}%
\pgfpathlineto{\pgfqpoint{4.396218in}{3.635821in}}%
\pgfpathlineto{\pgfqpoint{4.396502in}{3.578891in}}%
\pgfpathlineto{\pgfqpoint{4.396976in}{3.433710in}}%
\pgfpathlineto{\pgfqpoint{4.397450in}{3.576320in}}%
\pgfpathlineto{\pgfqpoint{4.397829in}{3.673262in}}%
\pgfpathlineto{\pgfqpoint{4.398303in}{3.513704in}}%
\pgfpathlineto{\pgfqpoint{4.398587in}{3.449871in}}%
\pgfpathlineto{\pgfqpoint{4.399061in}{3.592909in}}%
\pgfpathlineto{\pgfqpoint{4.399345in}{3.649109in}}%
\pgfpathlineto{\pgfqpoint{4.399819in}{3.522628in}}%
\pgfpathlineto{\pgfqpoint{4.400103in}{3.463105in}}%
\pgfpathlineto{\pgfqpoint{4.400577in}{3.613071in}}%
\pgfpathlineto{\pgfqpoint{4.400956in}{3.696412in}}%
\pgfpathlineto{\pgfqpoint{4.401430in}{3.586366in}}%
\pgfpathlineto{\pgfqpoint{4.401714in}{3.506704in}}%
\pgfpathlineto{\pgfqpoint{4.402283in}{3.672258in}}%
\pgfpathlineto{\pgfqpoint{4.402473in}{3.698612in}}%
\pgfpathlineto{\pgfqpoint{4.402946in}{3.582734in}}%
\pgfpathlineto{\pgfqpoint{4.403325in}{3.490407in}}%
\pgfpathlineto{\pgfqpoint{4.403799in}{3.631997in}}%
\pgfpathlineto{\pgfqpoint{4.403989in}{3.668242in}}%
\pgfpathlineto{\pgfqpoint{4.404463in}{3.543338in}}%
\pgfpathlineto{\pgfqpoint{4.404936in}{3.439494in}}%
\pgfpathlineto{\pgfqpoint{4.405410in}{3.574439in}}%
\pgfpathlineto{\pgfqpoint{4.405600in}{3.611439in}}%
\pgfpathlineto{\pgfqpoint{4.406074in}{3.494047in}}%
\pgfpathlineto{\pgfqpoint{4.406453in}{3.438319in}}%
\pgfpathlineto{\pgfqpoint{4.406832in}{3.531788in}}%
\pgfpathlineto{\pgfqpoint{4.407211in}{3.642807in}}%
\pgfpathlineto{\pgfqpoint{4.407779in}{3.475548in}}%
\pgfpathlineto{\pgfqpoint{4.408064in}{3.433007in}}%
\pgfpathlineto{\pgfqpoint{4.408537in}{3.566105in}}%
\pgfpathlineto{\pgfqpoint{4.408822in}{3.617914in}}%
\pgfpathlineto{\pgfqpoint{4.409296in}{3.467237in}}%
\pgfpathlineto{\pgfqpoint{4.409485in}{3.425824in}}%
\pgfpathlineto{\pgfqpoint{4.410054in}{3.556721in}}%
\pgfpathlineto{\pgfqpoint{4.410338in}{3.609461in}}%
\pgfpathlineto{\pgfqpoint{4.410812in}{3.486049in}}%
\pgfpathlineto{\pgfqpoint{4.411191in}{3.349260in}}%
\pgfpathlineto{\pgfqpoint{4.411665in}{3.545613in}}%
\pgfpathlineto{\pgfqpoint{4.413370in}{3.986616in}}%
\pgfpathlineto{\pgfqpoint{4.413465in}{3.986408in}}%
\pgfpathlineto{\pgfqpoint{4.413844in}{3.843375in}}%
\pgfpathlineto{\pgfqpoint{4.414602in}{3.347121in}}%
\pgfpathlineto{\pgfqpoint{4.415171in}{3.578839in}}%
\pgfpathlineto{\pgfqpoint{4.415266in}{3.585689in}}%
\pgfpathlineto{\pgfqpoint{4.415550in}{3.534412in}}%
\pgfpathlineto{\pgfqpoint{4.415834in}{3.489853in}}%
\pgfpathlineto{\pgfqpoint{4.416308in}{3.603689in}}%
\pgfpathlineto{\pgfqpoint{4.416687in}{3.675208in}}%
\pgfpathlineto{\pgfqpoint{4.417161in}{3.519212in}}%
\pgfpathlineto{\pgfqpoint{4.417445in}{3.447934in}}%
\pgfpathlineto{\pgfqpoint{4.417919in}{3.602620in}}%
\pgfpathlineto{\pgfqpoint{4.418298in}{3.684468in}}%
\pgfpathlineto{\pgfqpoint{4.418772in}{3.537456in}}%
\pgfpathlineto{\pgfqpoint{4.418961in}{3.516380in}}%
\pgfpathlineto{\pgfqpoint{4.419341in}{3.588946in}}%
\pgfpathlineto{\pgfqpoint{4.419814in}{3.689017in}}%
\pgfpathlineto{\pgfqpoint{4.420288in}{3.553865in}}%
\pgfpathlineto{\pgfqpoint{4.420572in}{3.486035in}}%
\pgfpathlineto{\pgfqpoint{4.421046in}{3.624655in}}%
\pgfpathlineto{\pgfqpoint{4.421425in}{3.701326in}}%
\pgfpathlineto{\pgfqpoint{4.421899in}{3.579325in}}%
\pgfpathlineto{\pgfqpoint{4.422183in}{3.513767in}}%
\pgfpathlineto{\pgfqpoint{4.422752in}{3.665403in}}%
\pgfpathlineto{\pgfqpoint{4.422942in}{3.702540in}}%
\pgfpathlineto{\pgfqpoint{4.423415in}{3.573387in}}%
\pgfpathlineto{\pgfqpoint{4.423700in}{3.519814in}}%
\pgfpathlineto{\pgfqpoint{4.424174in}{3.663625in}}%
\pgfpathlineto{\pgfqpoint{4.424553in}{3.761741in}}%
\pgfpathlineto{\pgfqpoint{4.425026in}{3.615365in}}%
\pgfpathlineto{\pgfqpoint{4.425311in}{3.565840in}}%
\pgfpathlineto{\pgfqpoint{4.425784in}{3.713189in}}%
\pgfpathlineto{\pgfqpoint{4.426069in}{3.764249in}}%
\pgfpathlineto{\pgfqpoint{4.426543in}{3.623667in}}%
\pgfpathlineto{\pgfqpoint{4.426827in}{3.561632in}}%
\pgfpathlineto{\pgfqpoint{4.427301in}{3.698325in}}%
\pgfpathlineto{\pgfqpoint{4.427680in}{3.799316in}}%
\pgfpathlineto{\pgfqpoint{4.428154in}{3.659682in}}%
\pgfpathlineto{\pgfqpoint{4.428438in}{3.597467in}}%
\pgfpathlineto{\pgfqpoint{4.429006in}{3.729726in}}%
\pgfpathlineto{\pgfqpoint{4.429291in}{3.769679in}}%
\pgfpathlineto{\pgfqpoint{4.429670in}{3.649655in}}%
\pgfpathlineto{\pgfqpoint{4.429954in}{3.575209in}}%
\pgfpathlineto{\pgfqpoint{4.430523in}{3.740924in}}%
\pgfpathlineto{\pgfqpoint{4.430807in}{3.791888in}}%
\pgfpathlineto{\pgfqpoint{4.431281in}{3.667740in}}%
\pgfpathlineto{\pgfqpoint{4.431565in}{3.585666in}}%
\pgfpathlineto{\pgfqpoint{4.432134in}{3.727716in}}%
\pgfpathlineto{\pgfqpoint{4.432418in}{3.762222in}}%
\pgfpathlineto{\pgfqpoint{4.432797in}{3.656031in}}%
\pgfpathlineto{\pgfqpoint{4.433081in}{3.581319in}}%
\pgfpathlineto{\pgfqpoint{4.433650in}{3.737516in}}%
\pgfpathlineto{\pgfqpoint{4.433934in}{3.800776in}}%
\pgfpathlineto{\pgfqpoint{4.434408in}{3.661071in}}%
\pgfpathlineto{\pgfqpoint{4.434787in}{3.566088in}}%
\pgfpathlineto{\pgfqpoint{4.435356in}{3.733934in}}%
\pgfpathlineto{\pgfqpoint{4.435450in}{3.744251in}}%
\pgfpathlineto{\pgfqpoint{4.435829in}{3.682395in}}%
\pgfpathlineto{\pgfqpoint{4.436303in}{3.548097in}}%
\pgfpathlineto{\pgfqpoint{4.436777in}{3.675739in}}%
\pgfpathlineto{\pgfqpoint{4.437061in}{3.736618in}}%
\pgfpathlineto{\pgfqpoint{4.437535in}{3.594348in}}%
\pgfpathlineto{\pgfqpoint{4.437914in}{3.495241in}}%
\pgfpathlineto{\pgfqpoint{4.438483in}{3.658930in}}%
\pgfpathlineto{\pgfqpoint{4.438672in}{3.678663in}}%
\pgfpathlineto{\pgfqpoint{4.439051in}{3.596845in}}%
\pgfpathlineto{\pgfqpoint{4.439431in}{3.488861in}}%
\pgfpathlineto{\pgfqpoint{4.439904in}{3.635488in}}%
\pgfpathlineto{\pgfqpoint{4.440189in}{3.673082in}}%
\pgfpathlineto{\pgfqpoint{4.440662in}{3.567205in}}%
\pgfpathlineto{\pgfqpoint{4.441042in}{3.459165in}}%
\pgfpathlineto{\pgfqpoint{4.441610in}{3.621230in}}%
\pgfpathlineto{\pgfqpoint{4.441894in}{3.671623in}}%
\pgfpathlineto{\pgfqpoint{4.442273in}{3.541419in}}%
\pgfpathlineto{\pgfqpoint{4.442558in}{3.472507in}}%
\pgfpathlineto{\pgfqpoint{4.443126in}{3.609018in}}%
\pgfpathlineto{\pgfqpoint{4.443411in}{3.669891in}}%
\pgfpathlineto{\pgfqpoint{4.443884in}{3.494848in}}%
\pgfpathlineto{\pgfqpoint{4.444169in}{3.452274in}}%
\pgfpathlineto{\pgfqpoint{4.444643in}{3.573494in}}%
\pgfpathlineto{\pgfqpoint{4.444927in}{3.641079in}}%
\pgfpathlineto{\pgfqpoint{4.445495in}{3.498085in}}%
\pgfpathlineto{\pgfqpoint{4.445685in}{3.457266in}}%
\pgfpathlineto{\pgfqpoint{4.446159in}{3.594213in}}%
\pgfpathlineto{\pgfqpoint{4.446538in}{3.660532in}}%
\pgfpathlineto{\pgfqpoint{4.447012in}{3.544341in}}%
\pgfpathlineto{\pgfqpoint{4.447296in}{3.469915in}}%
\pgfpathlineto{\pgfqpoint{4.447770in}{3.619261in}}%
\pgfpathlineto{\pgfqpoint{4.448149in}{3.724295in}}%
\pgfpathlineto{\pgfqpoint{4.448623in}{3.554729in}}%
\pgfpathlineto{\pgfqpoint{4.448812in}{3.522974in}}%
\pgfpathlineto{\pgfqpoint{4.449381in}{3.643796in}}%
\pgfpathlineto{\pgfqpoint{4.449665in}{3.677959in}}%
\pgfpathlineto{\pgfqpoint{4.450044in}{3.554724in}}%
\pgfpathlineto{\pgfqpoint{4.450518in}{3.434651in}}%
\pgfpathlineto{\pgfqpoint{4.450992in}{3.599444in}}%
\pgfpathlineto{\pgfqpoint{4.451181in}{3.647949in}}%
\pgfpathlineto{\pgfqpoint{4.451750in}{3.500955in}}%
\pgfpathlineto{\pgfqpoint{4.452034in}{3.443702in}}%
\pgfpathlineto{\pgfqpoint{4.452508in}{3.551824in}}%
\pgfpathlineto{\pgfqpoint{4.452887in}{3.624126in}}%
\pgfpathlineto{\pgfqpoint{4.453266in}{3.497205in}}%
\pgfpathlineto{\pgfqpoint{4.453550in}{3.418763in}}%
\pgfpathlineto{\pgfqpoint{4.454119in}{3.577856in}}%
\pgfpathlineto{\pgfqpoint{4.454403in}{3.637546in}}%
\pgfpathlineto{\pgfqpoint{4.454877in}{3.513658in}}%
\pgfpathlineto{\pgfqpoint{4.455161in}{3.452331in}}%
\pgfpathlineto{\pgfqpoint{4.455730in}{3.590412in}}%
\pgfpathlineto{\pgfqpoint{4.456014in}{3.631069in}}%
\pgfpathlineto{\pgfqpoint{4.456393in}{3.516244in}}%
\pgfpathlineto{\pgfqpoint{4.456678in}{3.453032in}}%
\pgfpathlineto{\pgfqpoint{4.457246in}{3.559280in}}%
\pgfpathlineto{\pgfqpoint{4.459141in}{4.049336in}}%
\pgfpathlineto{\pgfqpoint{4.459426in}{3.954284in}}%
\pgfpathlineto{\pgfqpoint{4.461416in}{3.460230in}}%
\pgfpathlineto{\pgfqpoint{4.461700in}{3.513290in}}%
\pgfpathlineto{\pgfqpoint{4.462269in}{3.655269in}}%
\pgfpathlineto{\pgfqpoint{4.462743in}{3.532523in}}%
\pgfpathlineto{\pgfqpoint{4.463027in}{3.499314in}}%
\pgfpathlineto{\pgfqpoint{4.463406in}{3.599423in}}%
\pgfpathlineto{\pgfqpoint{4.463785in}{3.712466in}}%
\pgfpathlineto{\pgfqpoint{4.464353in}{3.534323in}}%
\pgfpathlineto{\pgfqpoint{4.464638in}{3.481081in}}%
\pgfpathlineto{\pgfqpoint{4.465112in}{3.606696in}}%
\pgfpathlineto{\pgfqpoint{4.465396in}{3.660073in}}%
\pgfpathlineto{\pgfqpoint{4.465870in}{3.530345in}}%
\pgfpathlineto{\pgfqpoint{4.466154in}{3.454821in}}%
\pgfpathlineto{\pgfqpoint{4.466723in}{3.595609in}}%
\pgfpathlineto{\pgfqpoint{4.466912in}{3.621321in}}%
\pgfpathlineto{\pgfqpoint{4.467291in}{3.519516in}}%
\pgfpathlineto{\pgfqpoint{4.467765in}{3.378833in}}%
\pgfpathlineto{\pgfqpoint{4.468239in}{3.538180in}}%
\pgfpathlineto{\pgfqpoint{4.468618in}{3.645761in}}%
\pgfpathlineto{\pgfqpoint{4.469281in}{3.507224in}}%
\pgfpathlineto{\pgfqpoint{4.470039in}{3.727585in}}%
\pgfpathlineto{\pgfqpoint{4.470703in}{3.560515in}}%
\pgfpathlineto{\pgfqpoint{4.470892in}{3.517025in}}%
\pgfpathlineto{\pgfqpoint{4.471366in}{3.646079in}}%
\pgfpathlineto{\pgfqpoint{4.471745in}{3.718419in}}%
\pgfpathlineto{\pgfqpoint{4.472219in}{3.591186in}}%
\pgfpathlineto{\pgfqpoint{4.472408in}{3.565141in}}%
\pgfpathlineto{\pgfqpoint{4.472882in}{3.663671in}}%
\pgfpathlineto{\pgfqpoint{4.473261in}{3.746632in}}%
\pgfpathlineto{\pgfqpoint{4.473735in}{3.583520in}}%
\pgfpathlineto{\pgfqpoint{4.474019in}{3.535961in}}%
\pgfpathlineto{\pgfqpoint{4.474493in}{3.667649in}}%
\pgfpathlineto{\pgfqpoint{4.474872in}{3.753831in}}%
\pgfpathlineto{\pgfqpoint{4.475346in}{3.619567in}}%
\pgfpathlineto{\pgfqpoint{4.475630in}{3.563402in}}%
\pgfpathlineto{\pgfqpoint{4.476104in}{3.694940in}}%
\pgfpathlineto{\pgfqpoint{4.476389in}{3.741294in}}%
\pgfpathlineto{\pgfqpoint{4.476862in}{3.603151in}}%
\pgfpathlineto{\pgfqpoint{4.477147in}{3.545236in}}%
\pgfpathlineto{\pgfqpoint{4.477620in}{3.666769in}}%
\pgfpathlineto{\pgfqpoint{4.478000in}{3.767190in}}%
\pgfpathlineto{\pgfqpoint{4.478568in}{3.608869in}}%
\pgfpathlineto{\pgfqpoint{4.478758in}{3.579051in}}%
\pgfpathlineto{\pgfqpoint{4.479231in}{3.698148in}}%
\pgfpathlineto{\pgfqpoint{4.479611in}{3.760421in}}%
\pgfpathlineto{\pgfqpoint{4.479990in}{3.623574in}}%
\pgfpathlineto{\pgfqpoint{4.480274in}{3.548050in}}%
\pgfpathlineto{\pgfqpoint{4.480842in}{3.697883in}}%
\pgfpathlineto{\pgfqpoint{4.481127in}{3.778624in}}%
\pgfpathlineto{\pgfqpoint{4.481601in}{3.612288in}}%
\pgfpathlineto{\pgfqpoint{4.481885in}{3.551640in}}%
\pgfpathlineto{\pgfqpoint{4.482453in}{3.673728in}}%
\pgfpathlineto{\pgfqpoint{4.482643in}{3.694089in}}%
\pgfpathlineto{\pgfqpoint{4.483022in}{3.630870in}}%
\pgfpathlineto{\pgfqpoint{4.483496in}{3.509054in}}%
\pgfpathlineto{\pgfqpoint{4.483970in}{3.647516in}}%
\pgfpathlineto{\pgfqpoint{4.484254in}{3.701605in}}%
\pgfpathlineto{\pgfqpoint{4.484728in}{3.575937in}}%
\pgfpathlineto{\pgfqpoint{4.485107in}{3.490136in}}%
\pgfpathlineto{\pgfqpoint{4.485675in}{3.626570in}}%
\pgfpathlineto{\pgfqpoint{4.485865in}{3.644967in}}%
\pgfpathlineto{\pgfqpoint{4.486244in}{3.525275in}}%
\pgfpathlineto{\pgfqpoint{4.486623in}{3.452903in}}%
\pgfpathlineto{\pgfqpoint{4.487097in}{3.597075in}}%
\pgfpathlineto{\pgfqpoint{4.487381in}{3.671415in}}%
\pgfpathlineto{\pgfqpoint{4.487855in}{3.520174in}}%
\pgfpathlineto{\pgfqpoint{4.488234in}{3.441656in}}%
\pgfpathlineto{\pgfqpoint{4.488708in}{3.554239in}}%
\pgfpathlineto{\pgfqpoint{4.488992in}{3.617274in}}%
\pgfpathlineto{\pgfqpoint{4.489466in}{3.466154in}}%
\pgfpathlineto{\pgfqpoint{4.489750in}{3.411068in}}%
\pgfpathlineto{\pgfqpoint{4.490224in}{3.562470in}}%
\pgfpathlineto{\pgfqpoint{4.490508in}{3.627989in}}%
\pgfpathlineto{\pgfqpoint{4.490982in}{3.501470in}}%
\pgfpathlineto{\pgfqpoint{4.491361in}{3.402077in}}%
\pgfpathlineto{\pgfqpoint{4.491835in}{3.534145in}}%
\pgfpathlineto{\pgfqpoint{4.492119in}{3.616566in}}%
\pgfpathlineto{\pgfqpoint{4.492688in}{3.469518in}}%
\pgfpathlineto{\pgfqpoint{4.492877in}{3.452108in}}%
\pgfpathlineto{\pgfqpoint{4.493162in}{3.516220in}}%
\pgfpathlineto{\pgfqpoint{4.493636in}{3.659056in}}%
\pgfpathlineto{\pgfqpoint{4.494204in}{3.497473in}}%
\pgfpathlineto{\pgfqpoint{4.494488in}{3.434616in}}%
\pgfpathlineto{\pgfqpoint{4.495057in}{3.586280in}}%
\pgfpathlineto{\pgfqpoint{4.495247in}{3.613240in}}%
\pgfpathlineto{\pgfqpoint{4.495626in}{3.512945in}}%
\pgfpathlineto{\pgfqpoint{4.496005in}{3.404978in}}%
\pgfpathlineto{\pgfqpoint{4.496573in}{3.546782in}}%
\pgfpathlineto{\pgfqpoint{4.496763in}{3.566059in}}%
\pgfpathlineto{\pgfqpoint{4.497142in}{3.488107in}}%
\pgfpathlineto{\pgfqpoint{4.497616in}{3.330636in}}%
\pgfpathlineto{\pgfqpoint{4.498184in}{3.487952in}}%
\pgfpathlineto{\pgfqpoint{4.498469in}{3.536149in}}%
\pgfpathlineto{\pgfqpoint{4.498942in}{3.410417in}}%
\pgfpathlineto{\pgfqpoint{4.499227in}{3.376358in}}%
\pgfpathlineto{\pgfqpoint{4.499606in}{3.466284in}}%
\pgfpathlineto{\pgfqpoint{4.499985in}{3.531801in}}%
\pgfpathlineto{\pgfqpoint{4.500459in}{3.393072in}}%
\pgfpathlineto{\pgfqpoint{4.500743in}{3.326322in}}%
\pgfpathlineto{\pgfqpoint{4.501217in}{3.459948in}}%
\pgfpathlineto{\pgfqpoint{4.501596in}{3.540507in}}%
\pgfpathlineto{\pgfqpoint{4.502164in}{3.405117in}}%
\pgfpathlineto{\pgfqpoint{4.502354in}{3.383722in}}%
\pgfpathlineto{\pgfqpoint{4.502733in}{3.472303in}}%
\pgfpathlineto{\pgfqpoint{4.503017in}{3.530852in}}%
\pgfpathlineto{\pgfqpoint{4.503491in}{3.411909in}}%
\pgfpathlineto{\pgfqpoint{4.503681in}{3.370738in}}%
\pgfpathlineto{\pgfqpoint{4.504060in}{3.488398in}}%
\pgfpathlineto{\pgfqpoint{4.504818in}{3.989029in}}%
\pgfpathlineto{\pgfqpoint{4.505481in}{3.779606in}}%
\pgfpathlineto{\pgfqpoint{4.505576in}{3.772601in}}%
\pgfpathlineto{\pgfqpoint{4.505860in}{3.832290in}}%
\pgfpathlineto{\pgfqpoint{4.505955in}{3.842308in}}%
\pgfpathlineto{\pgfqpoint{4.506144in}{3.796213in}}%
\pgfpathlineto{\pgfqpoint{4.506903in}{3.261243in}}%
\pgfpathlineto{\pgfqpoint{4.507566in}{3.546971in}}%
\pgfpathlineto{\pgfqpoint{4.507850in}{3.621293in}}%
\pgfpathlineto{\pgfqpoint{4.508419in}{3.455392in}}%
\pgfpathlineto{\pgfqpoint{4.508608in}{3.410649in}}%
\pgfpathlineto{\pgfqpoint{4.509177in}{3.566346in}}%
\pgfpathlineto{\pgfqpoint{4.509366in}{3.585983in}}%
\pgfpathlineto{\pgfqpoint{4.509745in}{3.503960in}}%
\pgfpathlineto{\pgfqpoint{4.510125in}{3.424786in}}%
\pgfpathlineto{\pgfqpoint{4.510598in}{3.541934in}}%
\pgfpathlineto{\pgfqpoint{4.511072in}{3.638053in}}%
\pgfpathlineto{\pgfqpoint{4.511546in}{3.507628in}}%
\pgfpathlineto{\pgfqpoint{4.511830in}{3.453156in}}%
\pgfpathlineto{\pgfqpoint{4.512304in}{3.564874in}}%
\pgfpathlineto{\pgfqpoint{4.512588in}{3.607912in}}%
\pgfpathlineto{\pgfqpoint{4.513062in}{3.484588in}}%
\pgfpathlineto{\pgfqpoint{4.513347in}{3.428231in}}%
\pgfpathlineto{\pgfqpoint{4.513820in}{3.573932in}}%
\pgfpathlineto{\pgfqpoint{4.514105in}{3.645015in}}%
\pgfpathlineto{\pgfqpoint{4.514673in}{3.496007in}}%
\pgfpathlineto{\pgfqpoint{4.514958in}{3.443909in}}%
\pgfpathlineto{\pgfqpoint{4.515431in}{3.570461in}}%
\pgfpathlineto{\pgfqpoint{4.515716in}{3.629040in}}%
\pgfpathlineto{\pgfqpoint{4.516189in}{3.517988in}}%
\pgfpathlineto{\pgfqpoint{4.516379in}{3.472665in}}%
\pgfpathlineto{\pgfqpoint{4.516948in}{3.602005in}}%
\pgfpathlineto{\pgfqpoint{4.517327in}{3.680006in}}%
\pgfpathlineto{\pgfqpoint{4.517800in}{3.523358in}}%
\pgfpathlineto{\pgfqpoint{4.518085in}{3.484782in}}%
\pgfpathlineto{\pgfqpoint{4.518559in}{3.593344in}}%
\pgfpathlineto{\pgfqpoint{4.518843in}{3.650560in}}%
\pgfpathlineto{\pgfqpoint{4.519317in}{3.539927in}}%
\pgfpathlineto{\pgfqpoint{4.519601in}{3.486126in}}%
\pgfpathlineto{\pgfqpoint{4.520075in}{3.603178in}}%
\pgfpathlineto{\pgfqpoint{4.520454in}{3.693114in}}%
\pgfpathlineto{\pgfqpoint{4.520928in}{3.536313in}}%
\pgfpathlineto{\pgfqpoint{4.521307in}{3.472391in}}%
\pgfpathlineto{\pgfqpoint{4.521686in}{3.585310in}}%
\pgfpathlineto{\pgfqpoint{4.522065in}{3.678753in}}%
\pgfpathlineto{\pgfqpoint{4.522633in}{3.517418in}}%
\pgfpathlineto{\pgfqpoint{4.522823in}{3.499855in}}%
\pgfpathlineto{\pgfqpoint{4.523202in}{3.606155in}}%
\pgfpathlineto{\pgfqpoint{4.523581in}{3.697128in}}%
\pgfpathlineto{\pgfqpoint{4.524055in}{3.530570in}}%
\pgfpathlineto{\pgfqpoint{4.524339in}{3.469852in}}%
\pgfpathlineto{\pgfqpoint{4.524813in}{3.588906in}}%
\pgfpathlineto{\pgfqpoint{4.525097in}{3.660266in}}%
\pgfpathlineto{\pgfqpoint{4.525666in}{3.528683in}}%
\pgfpathlineto{\pgfqpoint{4.525950in}{3.483269in}}%
\pgfpathlineto{\pgfqpoint{4.526424in}{3.605533in}}%
\pgfpathlineto{\pgfqpoint{4.526708in}{3.632599in}}%
\pgfpathlineto{\pgfqpoint{4.527087in}{3.557311in}}%
\pgfpathlineto{\pgfqpoint{4.527466in}{3.435170in}}%
\pgfpathlineto{\pgfqpoint{4.528035in}{3.572737in}}%
\pgfpathlineto{\pgfqpoint{4.528319in}{3.620251in}}%
\pgfpathlineto{\pgfqpoint{4.528793in}{3.494523in}}%
\pgfpathlineto{\pgfqpoint{4.529077in}{3.428212in}}%
\pgfpathlineto{\pgfqpoint{4.529646in}{3.576646in}}%
\pgfpathlineto{\pgfqpoint{4.529835in}{3.586703in}}%
\pgfpathlineto{\pgfqpoint{4.530120in}{3.519970in}}%
\pgfpathlineto{\pgfqpoint{4.530594in}{3.389263in}}%
\pgfpathlineto{\pgfqpoint{4.531162in}{3.518692in}}%
\pgfpathlineto{\pgfqpoint{4.531446in}{3.593069in}}%
\pgfpathlineto{\pgfqpoint{4.531920in}{3.457779in}}%
\pgfpathlineto{\pgfqpoint{4.532205in}{3.397224in}}%
\pgfpathlineto{\pgfqpoint{4.532773in}{3.517393in}}%
\pgfpathlineto{\pgfqpoint{4.532963in}{3.548643in}}%
\pgfpathlineto{\pgfqpoint{4.533437in}{3.428354in}}%
\pgfpathlineto{\pgfqpoint{4.533816in}{3.363896in}}%
\pgfpathlineto{\pgfqpoint{4.534289in}{3.502568in}}%
\pgfpathlineto{\pgfqpoint{4.534574in}{3.580579in}}%
\pgfpathlineto{\pgfqpoint{4.535142in}{3.432994in}}%
\pgfpathlineto{\pgfqpoint{4.535427in}{3.382545in}}%
\pgfpathlineto{\pgfqpoint{4.535900in}{3.534097in}}%
\pgfpathlineto{\pgfqpoint{4.536090in}{3.563171in}}%
\pgfpathlineto{\pgfqpoint{4.536564in}{3.433231in}}%
\pgfpathlineto{\pgfqpoint{4.536943in}{3.351703in}}%
\pgfpathlineto{\pgfqpoint{4.537417in}{3.509596in}}%
\pgfpathlineto{\pgfqpoint{4.537796in}{3.596799in}}%
\pgfpathlineto{\pgfqpoint{4.538269in}{3.453516in}}%
\pgfpathlineto{\pgfqpoint{4.538459in}{3.425530in}}%
\pgfpathlineto{\pgfqpoint{4.538838in}{3.523129in}}%
\pgfpathlineto{\pgfqpoint{4.539312in}{3.614154in}}%
\pgfpathlineto{\pgfqpoint{4.539786in}{3.520251in}}%
\pgfpathlineto{\pgfqpoint{4.540070in}{3.446274in}}%
\pgfpathlineto{\pgfqpoint{4.540544in}{3.597991in}}%
\pgfpathlineto{\pgfqpoint{4.540923in}{3.661307in}}%
\pgfpathlineto{\pgfqpoint{4.541397in}{3.514560in}}%
\pgfpathlineto{\pgfqpoint{4.541681in}{3.435205in}}%
\pgfpathlineto{\pgfqpoint{4.542344in}{3.568584in}}%
\pgfpathlineto{\pgfqpoint{4.542534in}{3.550047in}}%
\pgfpathlineto{\pgfqpoint{4.543292in}{3.346276in}}%
\pgfpathlineto{\pgfqpoint{4.543766in}{3.494769in}}%
\pgfpathlineto{\pgfqpoint{4.543955in}{3.518362in}}%
\pgfpathlineto{\pgfqpoint{4.544334in}{3.456358in}}%
\pgfpathlineto{\pgfqpoint{4.544808in}{3.295656in}}%
\pgfpathlineto{\pgfqpoint{4.545282in}{3.477709in}}%
\pgfpathlineto{\pgfqpoint{4.545661in}{3.563113in}}%
\pgfpathlineto{\pgfqpoint{4.546230in}{3.439745in}}%
\pgfpathlineto{\pgfqpoint{4.546324in}{3.436459in}}%
\pgfpathlineto{\pgfqpoint{4.546419in}{3.452574in}}%
\pgfpathlineto{\pgfqpoint{4.547083in}{3.658946in}}%
\pgfpathlineto{\pgfqpoint{4.547651in}{3.506213in}}%
\pgfpathlineto{\pgfqpoint{4.548030in}{3.435844in}}%
\pgfpathlineto{\pgfqpoint{4.548409in}{3.551365in}}%
\pgfpathlineto{\pgfqpoint{4.548694in}{3.623210in}}%
\pgfpathlineto{\pgfqpoint{4.549262in}{3.465521in}}%
\pgfpathlineto{\pgfqpoint{4.549452in}{3.414642in}}%
\pgfpathlineto{\pgfqpoint{4.549831in}{3.563336in}}%
\pgfpathlineto{\pgfqpoint{4.550589in}{3.981398in}}%
\pgfpathlineto{\pgfqpoint{4.551347in}{3.897186in}}%
\pgfpathlineto{\pgfqpoint{4.551821in}{3.994216in}}%
\pgfpathlineto{\pgfqpoint{4.552105in}{3.875920in}}%
\pgfpathlineto{\pgfqpoint{4.552768in}{3.372211in}}%
\pgfpathlineto{\pgfqpoint{4.553337in}{3.667651in}}%
\pgfpathlineto{\pgfqpoint{4.553527in}{3.691516in}}%
\pgfpathlineto{\pgfqpoint{4.553906in}{3.553545in}}%
\pgfpathlineto{\pgfqpoint{4.554190in}{3.483962in}}%
\pgfpathlineto{\pgfqpoint{4.554758in}{3.643401in}}%
\pgfpathlineto{\pgfqpoint{4.555043in}{3.708782in}}%
\pgfpathlineto{\pgfqpoint{4.555611in}{3.561138in}}%
\pgfpathlineto{\pgfqpoint{4.555801in}{3.539432in}}%
\pgfpathlineto{\pgfqpoint{4.556180in}{3.622543in}}%
\pgfpathlineto{\pgfqpoint{4.556559in}{3.714165in}}%
\pgfpathlineto{\pgfqpoint{4.557033in}{3.570243in}}%
\pgfpathlineto{\pgfqpoint{4.557317in}{3.516589in}}%
\pgfpathlineto{\pgfqpoint{4.557886in}{3.642162in}}%
\pgfpathlineto{\pgfqpoint{4.558170in}{3.708977in}}%
\pgfpathlineto{\pgfqpoint{4.558739in}{3.573700in}}%
\pgfpathlineto{\pgfqpoint{4.559023in}{3.531906in}}%
\pgfpathlineto{\pgfqpoint{4.559402in}{3.646120in}}%
\pgfpathlineto{\pgfqpoint{4.559686in}{3.700081in}}%
\pgfpathlineto{\pgfqpoint{4.560160in}{3.581649in}}%
\pgfpathlineto{\pgfqpoint{4.560539in}{3.508404in}}%
\pgfpathlineto{\pgfqpoint{4.561013in}{3.644209in}}%
\pgfpathlineto{\pgfqpoint{4.561392in}{3.729150in}}%
\pgfpathlineto{\pgfqpoint{4.561866in}{3.593196in}}%
\pgfpathlineto{\pgfqpoint{4.562150in}{3.556959in}}%
\pgfpathlineto{\pgfqpoint{4.562529in}{3.652908in}}%
\pgfpathlineto{\pgfqpoint{4.562813in}{3.715573in}}%
\pgfpathlineto{\pgfqpoint{4.563382in}{3.565197in}}%
\pgfpathlineto{\pgfqpoint{4.563666in}{3.517395in}}%
\pgfpathlineto{\pgfqpoint{4.564045in}{3.627171in}}%
\pgfpathlineto{\pgfqpoint{4.564424in}{3.734801in}}%
\pgfpathlineto{\pgfqpoint{4.564993in}{3.588019in}}%
\pgfpathlineto{\pgfqpoint{4.565277in}{3.524338in}}%
\pgfpathlineto{\pgfqpoint{4.565751in}{3.658987in}}%
\pgfpathlineto{\pgfqpoint{4.565941in}{3.693725in}}%
\pgfpathlineto{\pgfqpoint{4.566414in}{3.592689in}}%
\pgfpathlineto{\pgfqpoint{4.566888in}{3.491073in}}%
\pgfpathlineto{\pgfqpoint{4.567267in}{3.628960in}}%
\pgfpathlineto{\pgfqpoint{4.567552in}{3.713547in}}%
\pgfpathlineto{\pgfqpoint{4.568120in}{3.565987in}}%
\pgfpathlineto{\pgfqpoint{4.568404in}{3.509748in}}%
\pgfpathlineto{\pgfqpoint{4.568878in}{3.614393in}}%
\pgfpathlineto{\pgfqpoint{4.569163in}{3.675114in}}%
\pgfpathlineto{\pgfqpoint{4.569636in}{3.552389in}}%
\pgfpathlineto{\pgfqpoint{4.569921in}{3.497852in}}%
\pgfpathlineto{\pgfqpoint{4.570395in}{3.603897in}}%
\pgfpathlineto{\pgfqpoint{4.570774in}{3.694994in}}%
\pgfpathlineto{\pgfqpoint{4.571247in}{3.532388in}}%
\pgfpathlineto{\pgfqpoint{4.571532in}{3.471903in}}%
\pgfpathlineto{\pgfqpoint{4.572100in}{3.605547in}}%
\pgfpathlineto{\pgfqpoint{4.572385in}{3.651524in}}%
\pgfpathlineto{\pgfqpoint{4.572858in}{3.516636in}}%
\pgfpathlineto{\pgfqpoint{4.573048in}{3.487140in}}%
\pgfpathlineto{\pgfqpoint{4.573522in}{3.607029in}}%
\pgfpathlineto{\pgfqpoint{4.573806in}{3.665826in}}%
\pgfpathlineto{\pgfqpoint{4.574280in}{3.541692in}}%
\pgfpathlineto{\pgfqpoint{4.574754in}{3.428613in}}%
\pgfpathlineto{\pgfqpoint{4.575227in}{3.575942in}}%
\pgfpathlineto{\pgfqpoint{4.575417in}{3.615136in}}%
\pgfpathlineto{\pgfqpoint{4.575891in}{3.492363in}}%
\pgfpathlineto{\pgfqpoint{4.576175in}{3.423509in}}%
\pgfpathlineto{\pgfqpoint{4.576744in}{3.551361in}}%
\pgfpathlineto{\pgfqpoint{4.577028in}{3.605392in}}%
\pgfpathlineto{\pgfqpoint{4.577407in}{3.509350in}}%
\pgfpathlineto{\pgfqpoint{4.577786in}{3.380883in}}%
\pgfpathlineto{\pgfqpoint{4.578355in}{3.536226in}}%
\pgfpathlineto{\pgfqpoint{4.578544in}{3.562583in}}%
\pgfpathlineto{\pgfqpoint{4.579018in}{3.480839in}}%
\pgfpathlineto{\pgfqpoint{4.579397in}{3.403691in}}%
\pgfpathlineto{\pgfqpoint{4.579871in}{3.526561in}}%
\pgfpathlineto{\pgfqpoint{4.580155in}{3.586884in}}%
\pgfpathlineto{\pgfqpoint{4.580629in}{3.467106in}}%
\pgfpathlineto{\pgfqpoint{4.581008in}{3.382568in}}%
\pgfpathlineto{\pgfqpoint{4.581482in}{3.536881in}}%
\pgfpathlineto{\pgfqpoint{4.581766in}{3.591533in}}%
\pgfpathlineto{\pgfqpoint{4.582335in}{3.469600in}}%
\pgfpathlineto{\pgfqpoint{4.582524in}{3.432492in}}%
\pgfpathlineto{\pgfqpoint{4.582998in}{3.558033in}}%
\pgfpathlineto{\pgfqpoint{4.583282in}{3.599135in}}%
\pgfpathlineto{\pgfqpoint{4.583756in}{3.483094in}}%
\pgfpathlineto{\pgfqpoint{4.584135in}{3.391844in}}%
\pgfpathlineto{\pgfqpoint{4.584609in}{3.572635in}}%
\pgfpathlineto{\pgfqpoint{4.584893in}{3.628902in}}%
\pgfpathlineto{\pgfqpoint{4.585462in}{3.508997in}}%
\pgfpathlineto{\pgfqpoint{4.585652in}{3.480289in}}%
\pgfpathlineto{\pgfqpoint{4.586031in}{3.573593in}}%
\pgfpathlineto{\pgfqpoint{4.586410in}{3.644997in}}%
\pgfpathlineto{\pgfqpoint{4.586883in}{3.542204in}}%
\pgfpathlineto{\pgfqpoint{4.587263in}{3.448936in}}%
\pgfpathlineto{\pgfqpoint{4.587736in}{3.571064in}}%
\pgfpathlineto{\pgfqpoint{4.588021in}{3.636108in}}%
\pgfpathlineto{\pgfqpoint{4.588494in}{3.515409in}}%
\pgfpathlineto{\pgfqpoint{4.588874in}{3.417497in}}%
\pgfpathlineto{\pgfqpoint{4.589442in}{3.558879in}}%
\pgfpathlineto{\pgfqpoint{4.589537in}{3.566150in}}%
\pgfpathlineto{\pgfqpoint{4.589821in}{3.518709in}}%
\pgfpathlineto{\pgfqpoint{4.590390in}{3.368491in}}%
\pgfpathlineto{\pgfqpoint{4.590864in}{3.499395in}}%
\pgfpathlineto{\pgfqpoint{4.591243in}{3.572933in}}%
\pgfpathlineto{\pgfqpoint{4.591716in}{3.445983in}}%
\pgfpathlineto{\pgfqpoint{4.592001in}{3.393364in}}%
\pgfpathlineto{\pgfqpoint{4.592475in}{3.514184in}}%
\pgfpathlineto{\pgfqpoint{4.592664in}{3.543539in}}%
\pgfpathlineto{\pgfqpoint{4.593138in}{3.457601in}}%
\pgfpathlineto{\pgfqpoint{4.593517in}{3.365793in}}%
\pgfpathlineto{\pgfqpoint{4.593991in}{3.523607in}}%
\pgfpathlineto{\pgfqpoint{4.594370in}{3.588249in}}%
\pgfpathlineto{\pgfqpoint{4.594844in}{3.455277in}}%
\pgfpathlineto{\pgfqpoint{4.595128in}{3.401194in}}%
\pgfpathlineto{\pgfqpoint{4.595697in}{3.506840in}}%
\pgfpathlineto{\pgfqpoint{4.596928in}{3.831990in}}%
\pgfpathlineto{\pgfqpoint{4.597497in}{4.045944in}}%
\pgfpathlineto{\pgfqpoint{4.597971in}{3.855802in}}%
\pgfpathlineto{\pgfqpoint{4.599487in}{3.421553in}}%
\pgfpathlineto{\pgfqpoint{4.599866in}{3.474086in}}%
\pgfpathlineto{\pgfqpoint{4.600530in}{3.689892in}}%
\pgfpathlineto{\pgfqpoint{4.601098in}{3.552288in}}%
\pgfpathlineto{\pgfqpoint{4.601477in}{3.456142in}}%
\pgfpathlineto{\pgfqpoint{4.601951in}{3.617123in}}%
\pgfpathlineto{\pgfqpoint{4.602141in}{3.655235in}}%
\pgfpathlineto{\pgfqpoint{4.602709in}{3.549945in}}%
\pgfpathlineto{\pgfqpoint{4.602993in}{3.523629in}}%
\pgfpathlineto{\pgfqpoint{4.603372in}{3.594932in}}%
\pgfpathlineto{\pgfqpoint{4.603751in}{3.685039in}}%
\pgfpathlineto{\pgfqpoint{4.604225in}{3.531161in}}%
\pgfpathlineto{\pgfqpoint{4.604510in}{3.462018in}}%
\pgfpathlineto{\pgfqpoint{4.605078in}{3.579635in}}%
\pgfpathlineto{\pgfqpoint{4.605362in}{3.639138in}}%
\pgfpathlineto{\pgfqpoint{4.605931in}{3.505777in}}%
\pgfpathlineto{\pgfqpoint{4.606121in}{3.489411in}}%
\pgfpathlineto{\pgfqpoint{4.606500in}{3.584219in}}%
\pgfpathlineto{\pgfqpoint{4.606879in}{3.643603in}}%
\pgfpathlineto{\pgfqpoint{4.607353in}{3.526246in}}%
\pgfpathlineto{\pgfqpoint{4.607732in}{3.435803in}}%
\pgfpathlineto{\pgfqpoint{4.608205in}{3.589793in}}%
\pgfpathlineto{\pgfqpoint{4.608490in}{3.649768in}}%
\pgfpathlineto{\pgfqpoint{4.609058in}{3.517478in}}%
\pgfpathlineto{\pgfqpoint{4.609248in}{3.506006in}}%
\pgfpathlineto{\pgfqpoint{4.609532in}{3.541551in}}%
\pgfpathlineto{\pgfqpoint{4.610006in}{3.675190in}}%
\pgfpathlineto{\pgfqpoint{4.610575in}{3.521253in}}%
\pgfpathlineto{\pgfqpoint{4.610859in}{3.477294in}}%
\pgfpathlineto{\pgfqpoint{4.611238in}{3.573817in}}%
\pgfpathlineto{\pgfqpoint{4.611617in}{3.683995in}}%
\pgfpathlineto{\pgfqpoint{4.612186in}{3.545064in}}%
\pgfpathlineto{\pgfqpoint{4.612375in}{3.519191in}}%
\pgfpathlineto{\pgfqpoint{4.612849in}{3.602963in}}%
\pgfpathlineto{\pgfqpoint{4.613133in}{3.663389in}}%
\pgfpathlineto{\pgfqpoint{4.613607in}{3.560550in}}%
\pgfpathlineto{\pgfqpoint{4.613986in}{3.465954in}}%
\pgfpathlineto{\pgfqpoint{4.614460in}{3.591953in}}%
\pgfpathlineto{\pgfqpoint{4.614839in}{3.675183in}}%
\pgfpathlineto{\pgfqpoint{4.615313in}{3.530973in}}%
\pgfpathlineto{\pgfqpoint{4.615597in}{3.478254in}}%
\pgfpathlineto{\pgfqpoint{4.616166in}{3.595143in}}%
\pgfpathlineto{\pgfqpoint{4.616260in}{3.605131in}}%
\pgfpathlineto{\pgfqpoint{4.616545in}{3.542809in}}%
\pgfpathlineto{\pgfqpoint{4.617113in}{3.362994in}}%
\pgfpathlineto{\pgfqpoint{4.617682in}{3.518894in}}%
\pgfpathlineto{\pgfqpoint{4.617871in}{3.545846in}}%
\pgfpathlineto{\pgfqpoint{4.618440in}{3.460343in}}%
\pgfpathlineto{\pgfqpoint{4.618629in}{3.432862in}}%
\pgfpathlineto{\pgfqpoint{4.619009in}{3.516153in}}%
\pgfpathlineto{\pgfqpoint{4.619482in}{3.653525in}}%
\pgfpathlineto{\pgfqpoint{4.620051in}{3.519257in}}%
\pgfpathlineto{\pgfqpoint{4.620240in}{3.488306in}}%
\pgfpathlineto{\pgfqpoint{4.620714in}{3.611980in}}%
\pgfpathlineto{\pgfqpoint{4.620999in}{3.680916in}}%
\pgfpathlineto{\pgfqpoint{4.621567in}{3.528233in}}%
\pgfpathlineto{\pgfqpoint{4.621851in}{3.459172in}}%
\pgfpathlineto{\pgfqpoint{4.622420in}{3.580444in}}%
\pgfpathlineto{\pgfqpoint{4.622610in}{3.591824in}}%
\pgfpathlineto{\pgfqpoint{4.622989in}{3.526692in}}%
\pgfpathlineto{\pgfqpoint{4.623368in}{3.442674in}}%
\pgfpathlineto{\pgfqpoint{4.623841in}{3.562826in}}%
\pgfpathlineto{\pgfqpoint{4.624221in}{3.644461in}}%
\pgfpathlineto{\pgfqpoint{4.624694in}{3.507300in}}%
\pgfpathlineto{\pgfqpoint{4.625073in}{3.434353in}}%
\pgfpathlineto{\pgfqpoint{4.625547in}{3.568828in}}%
\pgfpathlineto{\pgfqpoint{4.625737in}{3.601645in}}%
\pgfpathlineto{\pgfqpoint{4.626211in}{3.506132in}}%
\pgfpathlineto{\pgfqpoint{4.626590in}{3.438932in}}%
\pgfpathlineto{\pgfqpoint{4.626969in}{3.563252in}}%
\pgfpathlineto{\pgfqpoint{4.627348in}{3.653281in}}%
\pgfpathlineto{\pgfqpoint{4.627822in}{3.500442in}}%
\pgfpathlineto{\pgfqpoint{4.628106in}{3.449864in}}%
\pgfpathlineto{\pgfqpoint{4.628674in}{3.553714in}}%
\pgfpathlineto{\pgfqpoint{4.628959in}{3.585858in}}%
\pgfpathlineto{\pgfqpoint{4.629433in}{3.501816in}}%
\pgfpathlineto{\pgfqpoint{4.629717in}{3.460404in}}%
\pgfpathlineto{\pgfqpoint{4.630191in}{3.581310in}}%
\pgfpathlineto{\pgfqpoint{4.630475in}{3.637088in}}%
\pgfpathlineto{\pgfqpoint{4.630949in}{3.513757in}}%
\pgfpathlineto{\pgfqpoint{4.631328in}{3.435818in}}%
\pgfpathlineto{\pgfqpoint{4.631802in}{3.592481in}}%
\pgfpathlineto{\pgfqpoint{4.632181in}{3.634519in}}%
\pgfpathlineto{\pgfqpoint{4.632560in}{3.539711in}}%
\pgfpathlineto{\pgfqpoint{4.632749in}{3.498371in}}%
\pgfpathlineto{\pgfqpoint{4.633318in}{3.612145in}}%
\pgfpathlineto{\pgfqpoint{4.633602in}{3.669190in}}%
\pgfpathlineto{\pgfqpoint{4.634076in}{3.526186in}}%
\pgfpathlineto{\pgfqpoint{4.634455in}{3.424587in}}%
\pgfpathlineto{\pgfqpoint{4.635024in}{3.551891in}}%
\pgfpathlineto{\pgfqpoint{4.635213in}{3.565993in}}%
\pgfpathlineto{\pgfqpoint{4.635592in}{3.491880in}}%
\pgfpathlineto{\pgfqpoint{4.635971in}{3.407535in}}%
\pgfpathlineto{\pgfqpoint{4.636540in}{3.544092in}}%
\pgfpathlineto{\pgfqpoint{4.636729in}{3.562770in}}%
\pgfpathlineto{\pgfqpoint{4.637108in}{3.492372in}}%
\pgfpathlineto{\pgfqpoint{4.637582in}{3.371872in}}%
\pgfpathlineto{\pgfqpoint{4.638056in}{3.513154in}}%
\pgfpathlineto{\pgfqpoint{4.638340in}{3.560569in}}%
\pgfpathlineto{\pgfqpoint{4.638814in}{3.464893in}}%
\pgfpathlineto{\pgfqpoint{4.639193in}{3.401935in}}%
\pgfpathlineto{\pgfqpoint{4.639667in}{3.521554in}}%
\pgfpathlineto{\pgfqpoint{4.639857in}{3.547262in}}%
\pgfpathlineto{\pgfqpoint{4.640330in}{3.429739in}}%
\pgfpathlineto{\pgfqpoint{4.640710in}{3.349757in}}%
\pgfpathlineto{\pgfqpoint{4.641183in}{3.484139in}}%
\pgfpathlineto{\pgfqpoint{4.641468in}{3.560813in}}%
\pgfpathlineto{\pgfqpoint{4.642036in}{3.387955in}}%
\pgfpathlineto{\pgfqpoint{4.642226in}{3.335554in}}%
\pgfpathlineto{\pgfqpoint{4.642700in}{3.544737in}}%
\pgfpathlineto{\pgfqpoint{4.643363in}{3.859907in}}%
\pgfpathlineto{\pgfqpoint{4.644216in}{3.836685in}}%
\pgfpathlineto{\pgfqpoint{4.644595in}{3.886200in}}%
\pgfpathlineto{\pgfqpoint{4.644879in}{3.795084in}}%
\pgfpathlineto{\pgfqpoint{4.645542in}{3.286705in}}%
\pgfpathlineto{\pgfqpoint{4.646206in}{3.566619in}}%
\pgfpathlineto{\pgfqpoint{4.646301in}{3.577912in}}%
\pgfpathlineto{\pgfqpoint{4.646585in}{3.523748in}}%
\pgfpathlineto{\pgfqpoint{4.646964in}{3.427348in}}%
\pgfpathlineto{\pgfqpoint{4.647438in}{3.559673in}}%
\pgfpathlineto{\pgfqpoint{4.647817in}{3.642815in}}%
\pgfpathlineto{\pgfqpoint{4.648291in}{3.506512in}}%
\pgfpathlineto{\pgfqpoint{4.648670in}{3.451508in}}%
\pgfpathlineto{\pgfqpoint{4.649144in}{3.551217in}}%
\pgfpathlineto{\pgfqpoint{4.649428in}{3.585927in}}%
\pgfpathlineto{\pgfqpoint{4.649807in}{3.482126in}}%
\pgfpathlineto{\pgfqpoint{4.650091in}{3.416757in}}%
\pgfpathlineto{\pgfqpoint{4.650565in}{3.557619in}}%
\pgfpathlineto{\pgfqpoint{4.650944in}{3.632643in}}%
\pgfpathlineto{\pgfqpoint{4.651418in}{3.507492in}}%
\pgfpathlineto{\pgfqpoint{4.651702in}{3.433602in}}%
\pgfpathlineto{\pgfqpoint{4.652271in}{3.561176in}}%
\pgfpathlineto{\pgfqpoint{4.652555in}{3.608475in}}%
\pgfpathlineto{\pgfqpoint{4.653029in}{3.473890in}}%
\pgfpathlineto{\pgfqpoint{4.653218in}{3.450440in}}%
\pgfpathlineto{\pgfqpoint{4.653597in}{3.537970in}}%
\pgfpathlineto{\pgfqpoint{4.654071in}{3.655003in}}%
\pgfpathlineto{\pgfqpoint{4.654545in}{3.521738in}}%
\pgfpathlineto{\pgfqpoint{4.654829in}{3.463692in}}%
\pgfpathlineto{\pgfqpoint{4.655398in}{3.598180in}}%
\pgfpathlineto{\pgfqpoint{4.655682in}{3.643948in}}%
\pgfpathlineto{\pgfqpoint{4.656156in}{3.532461in}}%
\pgfpathlineto{\pgfqpoint{4.656346in}{3.499286in}}%
\pgfpathlineto{\pgfqpoint{4.656819in}{3.613801in}}%
\pgfpathlineto{\pgfqpoint{4.657198in}{3.700112in}}%
\pgfpathlineto{\pgfqpoint{4.657672in}{3.571142in}}%
\pgfpathlineto{\pgfqpoint{4.658051in}{3.493101in}}%
\pgfpathlineto{\pgfqpoint{4.658525in}{3.632544in}}%
\pgfpathlineto{\pgfqpoint{4.658809in}{3.668698in}}%
\pgfpathlineto{\pgfqpoint{4.659283in}{3.588180in}}%
\pgfpathlineto{\pgfqpoint{4.659662in}{3.553145in}}%
\pgfpathlineto{\pgfqpoint{4.660041in}{3.635386in}}%
\pgfpathlineto{\pgfqpoint{4.660326in}{3.708884in}}%
\pgfpathlineto{\pgfqpoint{4.660894in}{3.565949in}}%
\pgfpathlineto{\pgfqpoint{4.661179in}{3.513395in}}%
\pgfpathlineto{\pgfqpoint{4.661652in}{3.617452in}}%
\pgfpathlineto{\pgfqpoint{4.662031in}{3.686722in}}%
\pgfpathlineto{\pgfqpoint{4.662505in}{3.555764in}}%
\pgfpathlineto{\pgfqpoint{4.662695in}{3.538406in}}%
\pgfpathlineto{\pgfqpoint{4.663074in}{3.617758in}}%
\pgfpathlineto{\pgfqpoint{4.663453in}{3.695990in}}%
\pgfpathlineto{\pgfqpoint{4.663927in}{3.575317in}}%
\pgfpathlineto{\pgfqpoint{4.664306in}{3.483500in}}%
\pgfpathlineto{\pgfqpoint{4.664874in}{3.633785in}}%
\pgfpathlineto{\pgfqpoint{4.665159in}{3.663722in}}%
\pgfpathlineto{\pgfqpoint{4.665538in}{3.580483in}}%
\pgfpathlineto{\pgfqpoint{4.665917in}{3.513721in}}%
\pgfpathlineto{\pgfqpoint{4.666391in}{3.615352in}}%
\pgfpathlineto{\pgfqpoint{4.666675in}{3.652323in}}%
\pgfpathlineto{\pgfqpoint{4.667054in}{3.534295in}}%
\pgfpathlineto{\pgfqpoint{4.667433in}{3.448894in}}%
\pgfpathlineto{\pgfqpoint{4.667907in}{3.562437in}}%
\pgfpathlineto{\pgfqpoint{4.668286in}{3.635790in}}%
\pgfpathlineto{\pgfqpoint{4.668760in}{3.517186in}}%
\pgfpathlineto{\pgfqpoint{4.669044in}{3.461729in}}%
\pgfpathlineto{\pgfqpoint{4.669518in}{3.573050in}}%
\pgfpathlineto{\pgfqpoint{4.669707in}{3.618965in}}%
\pgfpathlineto{\pgfqpoint{4.670181in}{3.484404in}}%
\pgfpathlineto{\pgfqpoint{4.670560in}{3.424840in}}%
\pgfpathlineto{\pgfqpoint{4.671034in}{3.545902in}}%
\pgfpathlineto{\pgfqpoint{4.671318in}{3.611277in}}%
\pgfpathlineto{\pgfqpoint{4.671887in}{3.501654in}}%
\pgfpathlineto{\pgfqpoint{4.672171in}{3.440495in}}%
\pgfpathlineto{\pgfqpoint{4.672740in}{3.561249in}}%
\pgfpathlineto{\pgfqpoint{4.672929in}{3.568505in}}%
\pgfpathlineto{\pgfqpoint{4.673214in}{3.522796in}}%
\pgfpathlineto{\pgfqpoint{4.673687in}{3.412723in}}%
\pgfpathlineto{\pgfqpoint{4.674161in}{3.527892in}}%
\pgfpathlineto{\pgfqpoint{4.674540in}{3.607748in}}%
\pgfpathlineto{\pgfqpoint{4.675014in}{3.496272in}}%
\pgfpathlineto{\pgfqpoint{4.675298in}{3.431398in}}%
\pgfpathlineto{\pgfqpoint{4.675867in}{3.552657in}}%
\pgfpathlineto{\pgfqpoint{4.676057in}{3.574095in}}%
\pgfpathlineto{\pgfqpoint{4.676530in}{3.478973in}}%
\pgfpathlineto{\pgfqpoint{4.676815in}{3.438192in}}%
\pgfpathlineto{\pgfqpoint{4.677288in}{3.547910in}}%
\pgfpathlineto{\pgfqpoint{4.677668in}{3.645050in}}%
\pgfpathlineto{\pgfqpoint{4.678236in}{3.519454in}}%
\pgfpathlineto{\pgfqpoint{4.678520in}{3.483680in}}%
\pgfpathlineto{\pgfqpoint{4.678994in}{3.591246in}}%
\pgfpathlineto{\pgfqpoint{4.679278in}{3.625699in}}%
\pgfpathlineto{\pgfqpoint{4.679752in}{3.527596in}}%
\pgfpathlineto{\pgfqpoint{4.680037in}{3.493838in}}%
\pgfpathlineto{\pgfqpoint{4.680416in}{3.598674in}}%
\pgfpathlineto{\pgfqpoint{4.680700in}{3.658428in}}%
\pgfpathlineto{\pgfqpoint{4.681269in}{3.521936in}}%
\pgfpathlineto{\pgfqpoint{4.681648in}{3.422096in}}%
\pgfpathlineto{\pgfqpoint{4.682216in}{3.537733in}}%
\pgfpathlineto{\pgfqpoint{4.682406in}{3.566205in}}%
\pgfpathlineto{\pgfqpoint{4.682880in}{3.454490in}}%
\pgfpathlineto{\pgfqpoint{4.683164in}{3.424295in}}%
\pgfpathlineto{\pgfqpoint{4.683543in}{3.503656in}}%
\pgfpathlineto{\pgfqpoint{4.683922in}{3.612705in}}%
\pgfpathlineto{\pgfqpoint{4.684396in}{3.455503in}}%
\pgfpathlineto{\pgfqpoint{4.684775in}{3.371063in}}%
\pgfpathlineto{\pgfqpoint{4.685343in}{3.506374in}}%
\pgfpathlineto{\pgfqpoint{4.685533in}{3.534344in}}%
\pgfpathlineto{\pgfqpoint{4.686007in}{3.436496in}}%
\pgfpathlineto{\pgfqpoint{4.686291in}{3.404766in}}%
\pgfpathlineto{\pgfqpoint{4.686670in}{3.481933in}}%
\pgfpathlineto{\pgfqpoint{4.687049in}{3.575905in}}%
\pgfpathlineto{\pgfqpoint{4.687618in}{3.431874in}}%
\pgfpathlineto{\pgfqpoint{4.687997in}{3.386840in}}%
\pgfpathlineto{\pgfqpoint{4.688565in}{3.462320in}}%
\pgfpathlineto{\pgfqpoint{4.689039in}{3.450577in}}%
\pgfpathlineto{\pgfqpoint{4.689608in}{3.650487in}}%
\pgfpathlineto{\pgfqpoint{4.690176in}{3.875009in}}%
\pgfpathlineto{\pgfqpoint{4.690745in}{3.679018in}}%
\pgfpathlineto{\pgfqpoint{4.692072in}{3.365334in}}%
\pgfpathlineto{\pgfqpoint{4.692451in}{3.399476in}}%
\pgfpathlineto{\pgfqpoint{4.693304in}{3.597356in}}%
\pgfpathlineto{\pgfqpoint{4.693872in}{3.454234in}}%
\pgfpathlineto{\pgfqpoint{4.694251in}{3.381584in}}%
\pgfpathlineto{\pgfqpoint{4.694630in}{3.487017in}}%
\pgfpathlineto{\pgfqpoint{4.695009in}{3.585491in}}%
\pgfpathlineto{\pgfqpoint{4.695578in}{3.445005in}}%
\pgfpathlineto{\pgfqpoint{4.695673in}{3.440021in}}%
\pgfpathlineto{\pgfqpoint{4.695957in}{3.464297in}}%
\pgfpathlineto{\pgfqpoint{4.696526in}{3.554677in}}%
\pgfpathlineto{\pgfqpoint{4.696905in}{3.464407in}}%
\pgfpathlineto{\pgfqpoint{4.697284in}{3.391391in}}%
\pgfpathlineto{\pgfqpoint{4.697757in}{3.523124in}}%
\pgfpathlineto{\pgfqpoint{4.698137in}{3.587797in}}%
\pgfpathlineto{\pgfqpoint{4.698705in}{3.474526in}}%
\pgfpathlineto{\pgfqpoint{4.698989in}{3.426788in}}%
\pgfpathlineto{\pgfqpoint{4.699463in}{3.545120in}}%
\pgfpathlineto{\pgfqpoint{4.699653in}{3.562852in}}%
\pgfpathlineto{\pgfqpoint{4.700032in}{3.485625in}}%
\pgfpathlineto{\pgfqpoint{4.700411in}{3.398113in}}%
\pgfpathlineto{\pgfqpoint{4.700885in}{3.517721in}}%
\pgfpathlineto{\pgfqpoint{4.701264in}{3.625154in}}%
\pgfpathlineto{\pgfqpoint{4.701832in}{3.496794in}}%
\pgfpathlineto{\pgfqpoint{4.702022in}{3.455590in}}%
\pgfpathlineto{\pgfqpoint{4.702496in}{3.578557in}}%
\pgfpathlineto{\pgfqpoint{4.702780in}{3.609935in}}%
\pgfpathlineto{\pgfqpoint{4.703254in}{3.534516in}}%
\pgfpathlineto{\pgfqpoint{4.703538in}{3.465005in}}%
\pgfpathlineto{\pgfqpoint{4.704012in}{3.618102in}}%
\pgfpathlineto{\pgfqpoint{4.704391in}{3.694206in}}%
\pgfpathlineto{\pgfqpoint{4.704865in}{3.575129in}}%
\pgfpathlineto{\pgfqpoint{4.705149in}{3.520252in}}%
\pgfpathlineto{\pgfqpoint{4.705718in}{3.630236in}}%
\pgfpathlineto{\pgfqpoint{4.706002in}{3.671156in}}%
\pgfpathlineto{\pgfqpoint{4.706476in}{3.563680in}}%
\pgfpathlineto{\pgfqpoint{4.706665in}{3.533117in}}%
\pgfpathlineto{\pgfqpoint{4.707139in}{3.654068in}}%
\pgfpathlineto{\pgfqpoint{4.707518in}{3.719389in}}%
\pgfpathlineto{\pgfqpoint{4.707992in}{3.617020in}}%
\pgfpathlineto{\pgfqpoint{4.708371in}{3.543953in}}%
\pgfpathlineto{\pgfqpoint{4.708940in}{3.672168in}}%
\pgfpathlineto{\pgfqpoint{4.709129in}{3.686547in}}%
\pgfpathlineto{\pgfqpoint{4.709508in}{3.616573in}}%
\pgfpathlineto{\pgfqpoint{4.709887in}{3.556945in}}%
\pgfpathlineto{\pgfqpoint{4.710361in}{3.671029in}}%
\pgfpathlineto{\pgfqpoint{4.710740in}{3.728350in}}%
\pgfpathlineto{\pgfqpoint{4.711119in}{3.622688in}}%
\pgfpathlineto{\pgfqpoint{4.711498in}{3.529915in}}%
\pgfpathlineto{\pgfqpoint{4.712162in}{3.647959in}}%
\pgfpathlineto{\pgfqpoint{4.712351in}{3.662150in}}%
\pgfpathlineto{\pgfqpoint{4.712635in}{3.589075in}}%
\pgfpathlineto{\pgfqpoint{4.712920in}{3.525811in}}%
\pgfpathlineto{\pgfqpoint{4.713488in}{3.619323in}}%
\pgfpathlineto{\pgfqpoint{4.713773in}{3.676566in}}%
\pgfpathlineto{\pgfqpoint{4.714246in}{3.556984in}}%
\pgfpathlineto{\pgfqpoint{4.714626in}{3.478584in}}%
\pgfpathlineto{\pgfqpoint{4.715099in}{3.585118in}}%
\pgfpathlineto{\pgfqpoint{4.715478in}{3.634241in}}%
\pgfpathlineto{\pgfqpoint{4.715952in}{3.530624in}}%
\pgfpathlineto{\pgfqpoint{4.716142in}{3.501019in}}%
\pgfpathlineto{\pgfqpoint{4.716616in}{3.621952in}}%
\pgfpathlineto{\pgfqpoint{4.716900in}{3.690297in}}%
\pgfpathlineto{\pgfqpoint{4.717468in}{3.533412in}}%
\pgfpathlineto{\pgfqpoint{4.717753in}{3.491391in}}%
\pgfpathlineto{\pgfqpoint{4.718227in}{3.602070in}}%
\pgfpathlineto{\pgfqpoint{4.718606in}{3.686657in}}%
\pgfpathlineto{\pgfqpoint{4.719174in}{3.563119in}}%
\pgfpathlineto{\pgfqpoint{4.719364in}{3.550232in}}%
\pgfpathlineto{\pgfqpoint{4.719648in}{3.597302in}}%
\pgfpathlineto{\pgfqpoint{4.720027in}{3.690687in}}%
\pgfpathlineto{\pgfqpoint{4.720501in}{3.566208in}}%
\pgfpathlineto{\pgfqpoint{4.720880in}{3.512693in}}%
\pgfpathlineto{\pgfqpoint{4.721354in}{3.605377in}}%
\pgfpathlineto{\pgfqpoint{4.721733in}{3.686179in}}%
\pgfpathlineto{\pgfqpoint{4.722301in}{3.577389in}}%
\pgfpathlineto{\pgfqpoint{4.722491in}{3.558154in}}%
\pgfpathlineto{\pgfqpoint{4.722870in}{3.639957in}}%
\pgfpathlineto{\pgfqpoint{4.723249in}{3.711854in}}%
\pgfpathlineto{\pgfqpoint{4.723818in}{3.596201in}}%
\pgfpathlineto{\pgfqpoint{4.724007in}{3.577525in}}%
\pgfpathlineto{\pgfqpoint{4.724386in}{3.672412in}}%
\pgfpathlineto{\pgfqpoint{4.724955in}{3.784304in}}%
\pgfpathlineto{\pgfqpoint{4.725429in}{3.692062in}}%
\pgfpathlineto{\pgfqpoint{4.725618in}{3.666086in}}%
\pgfpathlineto{\pgfqpoint{4.726092in}{3.742178in}}%
\pgfpathlineto{\pgfqpoint{4.726376in}{3.782831in}}%
\pgfpathlineto{\pgfqpoint{4.726755in}{3.690655in}}%
\pgfpathlineto{\pgfqpoint{4.727229in}{3.587688in}}%
\pgfpathlineto{\pgfqpoint{4.727703in}{3.723693in}}%
\pgfpathlineto{\pgfqpoint{4.727987in}{3.762019in}}%
\pgfpathlineto{\pgfqpoint{4.728461in}{3.651654in}}%
\pgfpathlineto{\pgfqpoint{4.728840in}{3.559892in}}%
\pgfpathlineto{\pgfqpoint{4.729503in}{3.673679in}}%
\pgfpathlineto{\pgfqpoint{4.729977in}{3.588315in}}%
\pgfpathlineto{\pgfqpoint{4.730262in}{3.533662in}}%
\pgfpathlineto{\pgfqpoint{4.730830in}{3.654881in}}%
\pgfpathlineto{\pgfqpoint{4.731209in}{3.700564in}}%
\pgfpathlineto{\pgfqpoint{4.731588in}{3.612001in}}%
\pgfpathlineto{\pgfqpoint{4.731873in}{3.534405in}}%
\pgfpathlineto{\pgfqpoint{4.732536in}{3.661329in}}%
\pgfpathlineto{\pgfqpoint{4.732725in}{3.681741in}}%
\pgfpathlineto{\pgfqpoint{4.733105in}{3.611992in}}%
\pgfpathlineto{\pgfqpoint{4.733389in}{3.567684in}}%
\pgfpathlineto{\pgfqpoint{4.733863in}{3.672278in}}%
\pgfpathlineto{\pgfqpoint{4.734242in}{3.790770in}}%
\pgfpathlineto{\pgfqpoint{4.734715in}{3.603818in}}%
\pgfpathlineto{\pgfqpoint{4.735000in}{3.527109in}}%
\pgfpathlineto{\pgfqpoint{4.735474in}{3.758123in}}%
\pgfpathlineto{\pgfqpoint{4.736042in}{4.090188in}}%
\pgfpathlineto{\pgfqpoint{4.736990in}{4.049997in}}%
\pgfpathlineto{\pgfqpoint{4.737274in}{4.103551in}}%
\pgfpathlineto{\pgfqpoint{4.737558in}{4.000329in}}%
\pgfpathlineto{\pgfqpoint{4.738222in}{3.433738in}}%
\pgfpathlineto{\pgfqpoint{4.738885in}{3.741157in}}%
\pgfpathlineto{\pgfqpoint{4.739075in}{3.774801in}}%
\pgfpathlineto{\pgfqpoint{4.739643in}{3.653637in}}%
\pgfpathlineto{\pgfqpoint{4.739738in}{3.651571in}}%
\pgfpathlineto{\pgfqpoint{4.739833in}{3.661575in}}%
\pgfpathlineto{\pgfqpoint{4.740591in}{3.825625in}}%
\pgfpathlineto{\pgfqpoint{4.740970in}{3.718290in}}%
\pgfpathlineto{\pgfqpoint{4.741349in}{3.633822in}}%
\pgfpathlineto{\pgfqpoint{4.741918in}{3.755722in}}%
\pgfpathlineto{\pgfqpoint{4.742202in}{3.797287in}}%
\pgfpathlineto{\pgfqpoint{4.742676in}{3.698426in}}%
\pgfpathlineto{\pgfqpoint{4.742865in}{3.677367in}}%
\pgfpathlineto{\pgfqpoint{4.743339in}{3.776249in}}%
\pgfpathlineto{\pgfqpoint{4.743718in}{3.832680in}}%
\pgfpathlineto{\pgfqpoint{4.744097in}{3.726111in}}%
\pgfpathlineto{\pgfqpoint{4.744381in}{3.657517in}}%
\pgfpathlineto{\pgfqpoint{4.745045in}{3.769959in}}%
\pgfpathlineto{\pgfqpoint{4.745329in}{3.826887in}}%
\pgfpathlineto{\pgfqpoint{4.745898in}{3.717293in}}%
\pgfpathlineto{\pgfqpoint{4.745992in}{3.713227in}}%
\pgfpathlineto{\pgfqpoint{4.746277in}{3.738823in}}%
\pgfpathlineto{\pgfqpoint{4.746751in}{3.839213in}}%
\pgfpathlineto{\pgfqpoint{4.747224in}{3.734347in}}%
\pgfpathlineto{\pgfqpoint{4.747603in}{3.655999in}}%
\pgfpathlineto{\pgfqpoint{4.748077in}{3.751431in}}%
\pgfpathlineto{\pgfqpoint{4.748456in}{3.835477in}}%
\pgfpathlineto{\pgfqpoint{4.749025in}{3.715612in}}%
\pgfpathlineto{\pgfqpoint{4.749120in}{3.710164in}}%
\pgfpathlineto{\pgfqpoint{4.749404in}{3.748601in}}%
\pgfpathlineto{\pgfqpoint{4.749878in}{3.835899in}}%
\pgfpathlineto{\pgfqpoint{4.750352in}{3.749820in}}%
\pgfpathlineto{\pgfqpoint{4.750825in}{3.671745in}}%
\pgfpathlineto{\pgfqpoint{4.751204in}{3.773935in}}%
\pgfpathlineto{\pgfqpoint{4.751584in}{3.883087in}}%
\pgfpathlineto{\pgfqpoint{4.752152in}{3.749944in}}%
\pgfpathlineto{\pgfqpoint{4.752342in}{3.724664in}}%
\pgfpathlineto{\pgfqpoint{4.752815in}{3.808649in}}%
\pgfpathlineto{\pgfqpoint{4.753100in}{3.836917in}}%
\pgfpathlineto{\pgfqpoint{4.753479in}{3.756841in}}%
\pgfpathlineto{\pgfqpoint{4.753858in}{3.661345in}}%
\pgfpathlineto{\pgfqpoint{4.754426in}{3.787451in}}%
\pgfpathlineto{\pgfqpoint{4.754805in}{3.872878in}}%
\pgfpathlineto{\pgfqpoint{4.755279in}{3.719952in}}%
\pgfpathlineto{\pgfqpoint{4.755469in}{3.697548in}}%
\pgfpathlineto{\pgfqpoint{4.756037in}{3.773716in}}%
\pgfpathlineto{\pgfqpoint{4.756322in}{3.796963in}}%
\pgfpathlineto{\pgfqpoint{4.756606in}{3.740012in}}%
\pgfpathlineto{\pgfqpoint{4.756985in}{3.653945in}}%
\pgfpathlineto{\pgfqpoint{4.757554in}{3.762565in}}%
\pgfpathlineto{\pgfqpoint{4.757838in}{3.820508in}}%
\pgfpathlineto{\pgfqpoint{4.758312in}{3.704407in}}%
\pgfpathlineto{\pgfqpoint{4.758691in}{3.623367in}}%
\pgfpathlineto{\pgfqpoint{4.759259in}{3.714790in}}%
\pgfpathlineto{\pgfqpoint{4.759354in}{3.712660in}}%
\pgfpathlineto{\pgfqpoint{4.761913in}{3.399669in}}%
\pgfpathlineto{\pgfqpoint{4.762292in}{3.472106in}}%
\pgfpathlineto{\pgfqpoint{4.764092in}{3.659778in}}%
\pgfpathlineto{\pgfqpoint{4.764187in}{3.659750in}}%
\pgfpathlineto{\pgfqpoint{4.764945in}{3.473029in}}%
\pgfpathlineto{\pgfqpoint{4.765703in}{3.590331in}}%
\pgfpathlineto{\pgfqpoint{4.765988in}{3.553436in}}%
\pgfpathlineto{\pgfqpoint{4.766461in}{3.458472in}}%
\pgfpathlineto{\pgfqpoint{4.766935in}{3.565120in}}%
\pgfpathlineto{\pgfqpoint{4.767220in}{3.614856in}}%
\pgfpathlineto{\pgfqpoint{4.767693in}{3.506163in}}%
\pgfpathlineto{\pgfqpoint{4.768072in}{3.416354in}}%
\pgfpathlineto{\pgfqpoint{4.768641in}{3.540953in}}%
\pgfpathlineto{\pgfqpoint{4.768831in}{3.570909in}}%
\pgfpathlineto{\pgfqpoint{4.769399in}{3.492244in}}%
\pgfpathlineto{\pgfqpoint{4.769494in}{3.492768in}}%
\pgfpathlineto{\pgfqpoint{4.770063in}{3.565714in}}%
\pgfpathlineto{\pgfqpoint{4.770442in}{3.641137in}}%
\pgfpathlineto{\pgfqpoint{4.770915in}{3.518471in}}%
\pgfpathlineto{\pgfqpoint{4.771294in}{3.448251in}}%
\pgfpathlineto{\pgfqpoint{4.771768in}{3.580423in}}%
\pgfpathlineto{\pgfqpoint{4.772053in}{3.624191in}}%
\pgfpathlineto{\pgfqpoint{4.772621in}{3.509552in}}%
\pgfpathlineto{\pgfqpoint{4.772716in}{3.503195in}}%
\pgfpathlineto{\pgfqpoint{4.773190in}{3.540891in}}%
\pgfpathlineto{\pgfqpoint{4.773474in}{3.580023in}}%
\pgfpathlineto{\pgfqpoint{4.773853in}{3.502018in}}%
\pgfpathlineto{\pgfqpoint{4.774422in}{3.357297in}}%
\pgfpathlineto{\pgfqpoint{4.774895in}{3.473478in}}%
\pgfpathlineto{\pgfqpoint{4.775180in}{3.510555in}}%
\pgfpathlineto{\pgfqpoint{4.775748in}{3.415727in}}%
\pgfpathlineto{\pgfqpoint{4.775938in}{3.395567in}}%
\pgfpathlineto{\pgfqpoint{4.776317in}{3.458355in}}%
\pgfpathlineto{\pgfqpoint{4.776601in}{3.516409in}}%
\pgfpathlineto{\pgfqpoint{4.777075in}{3.418087in}}%
\pgfpathlineto{\pgfqpoint{4.777454in}{3.318329in}}%
\pgfpathlineto{\pgfqpoint{4.777928in}{3.447370in}}%
\pgfpathlineto{\pgfqpoint{4.778307in}{3.511919in}}%
\pgfpathlineto{\pgfqpoint{4.778876in}{3.407920in}}%
\pgfpathlineto{\pgfqpoint{4.779065in}{3.389008in}}%
\pgfpathlineto{\pgfqpoint{4.779539in}{3.465388in}}%
\pgfpathlineto{\pgfqpoint{4.779823in}{3.510041in}}%
\pgfpathlineto{\pgfqpoint{4.780202in}{3.411367in}}%
\pgfpathlineto{\pgfqpoint{4.780676in}{3.289885in}}%
\pgfpathlineto{\pgfqpoint{4.781150in}{3.417988in}}%
\pgfpathlineto{\pgfqpoint{4.782856in}{3.850660in}}%
\pgfpathlineto{\pgfqpoint{4.783235in}{3.764546in}}%
\pgfpathlineto{\pgfqpoint{4.784182in}{3.256220in}}%
\pgfpathlineto{\pgfqpoint{4.785130in}{3.393284in}}%
\pgfpathlineto{\pgfqpoint{4.785414in}{3.352943in}}%
\pgfpathlineto{\pgfqpoint{4.785888in}{3.440792in}}%
\pgfpathlineto{\pgfqpoint{4.786078in}{3.464780in}}%
\pgfpathlineto{\pgfqpoint{4.786551in}{3.369368in}}%
\pgfpathlineto{\pgfqpoint{4.786741in}{3.339092in}}%
\pgfpathlineto{\pgfqpoint{4.787310in}{3.439631in}}%
\pgfpathlineto{\pgfqpoint{4.787783in}{3.551195in}}%
\pgfpathlineto{\pgfqpoint{4.788257in}{3.423489in}}%
\pgfpathlineto{\pgfqpoint{4.788542in}{3.374396in}}%
\pgfpathlineto{\pgfqpoint{4.789015in}{3.483917in}}%
\pgfpathlineto{\pgfqpoint{4.789300in}{3.521184in}}%
\pgfpathlineto{\pgfqpoint{4.789773in}{3.419727in}}%
\pgfpathlineto{\pgfqpoint{4.790058in}{3.382285in}}%
\pgfpathlineto{\pgfqpoint{4.790437in}{3.485308in}}%
\pgfpathlineto{\pgfqpoint{4.790816in}{3.581759in}}%
\pgfpathlineto{\pgfqpoint{4.791384in}{3.459596in}}%
\pgfpathlineto{\pgfqpoint{4.791763in}{3.411078in}}%
\pgfpathlineto{\pgfqpoint{4.792143in}{3.495337in}}%
\pgfpathlineto{\pgfqpoint{4.792427in}{3.533297in}}%
\pgfpathlineto{\pgfqpoint{4.792995in}{3.445521in}}%
\pgfpathlineto{\pgfqpoint{4.793185in}{3.433901in}}%
\pgfpathlineto{\pgfqpoint{4.793469in}{3.494202in}}%
\pgfpathlineto{\pgfqpoint{4.793943in}{3.598986in}}%
\pgfpathlineto{\pgfqpoint{4.794417in}{3.502267in}}%
\pgfpathlineto{\pgfqpoint{4.794796in}{3.411120in}}%
\pgfpathlineto{\pgfqpoint{4.795365in}{3.537043in}}%
\pgfpathlineto{\pgfqpoint{4.795649in}{3.576290in}}%
\pgfpathlineto{\pgfqpoint{4.796123in}{3.487710in}}%
\pgfpathlineto{\pgfqpoint{4.796312in}{3.467234in}}%
\pgfpathlineto{\pgfqpoint{4.796691in}{3.537373in}}%
\pgfpathlineto{\pgfqpoint{4.797070in}{3.615560in}}%
\pgfpathlineto{\pgfqpoint{4.797639in}{3.499387in}}%
\pgfpathlineto{\pgfqpoint{4.798018in}{3.422147in}}%
\pgfpathlineto{\pgfqpoint{4.798492in}{3.552086in}}%
\pgfpathlineto{\pgfqpoint{4.798776in}{3.590012in}}%
\pgfpathlineto{\pgfqpoint{4.799250in}{3.510037in}}%
\pgfpathlineto{\pgfqpoint{4.799439in}{3.472629in}}%
\pgfpathlineto{\pgfqpoint{4.800008in}{3.579703in}}%
\pgfpathlineto{\pgfqpoint{4.800197in}{3.594465in}}%
\pgfpathlineto{\pgfqpoint{4.800577in}{3.543693in}}%
\pgfpathlineto{\pgfqpoint{4.801050in}{3.419882in}}%
\pgfpathlineto{\pgfqpoint{4.801619in}{3.540321in}}%
\pgfpathlineto{\pgfqpoint{4.801903in}{3.594826in}}%
\pgfpathlineto{\pgfqpoint{4.802567in}{3.496739in}}%
\pgfpathlineto{\pgfqpoint{4.802661in}{3.493088in}}%
\pgfpathlineto{\pgfqpoint{4.802851in}{3.512370in}}%
\pgfpathlineto{\pgfqpoint{4.803325in}{3.593723in}}%
\pgfpathlineto{\pgfqpoint{4.803799in}{3.508093in}}%
\pgfpathlineto{\pgfqpoint{4.804178in}{3.426368in}}%
\pgfpathlineto{\pgfqpoint{4.804746in}{3.545707in}}%
\pgfpathlineto{\pgfqpoint{4.805125in}{3.598894in}}%
\pgfpathlineto{\pgfqpoint{4.805599in}{3.487770in}}%
\pgfpathlineto{\pgfqpoint{4.805789in}{3.461830in}}%
\pgfpathlineto{\pgfqpoint{4.806452in}{3.535929in}}%
\pgfpathlineto{\pgfqpoint{4.806547in}{3.538053in}}%
\pgfpathlineto{\pgfqpoint{4.806736in}{3.521437in}}%
\pgfpathlineto{\pgfqpoint{4.807305in}{3.360319in}}%
\pgfpathlineto{\pgfqpoint{4.807873in}{3.484540in}}%
\pgfpathlineto{\pgfqpoint{4.808252in}{3.524654in}}%
\pgfpathlineto{\pgfqpoint{4.808632in}{3.449449in}}%
\pgfpathlineto{\pgfqpoint{4.809011in}{3.390566in}}%
\pgfpathlineto{\pgfqpoint{4.809579in}{3.481553in}}%
\pgfpathlineto{\pgfqpoint{4.809674in}{3.490582in}}%
\pgfpathlineto{\pgfqpoint{4.810053in}{3.423171in}}%
\pgfpathlineto{\pgfqpoint{4.810432in}{3.334817in}}%
\pgfpathlineto{\pgfqpoint{4.811001in}{3.458158in}}%
\pgfpathlineto{\pgfqpoint{4.811380in}{3.519003in}}%
\pgfpathlineto{\pgfqpoint{4.811853in}{3.411258in}}%
\pgfpathlineto{\pgfqpoint{4.812233in}{3.362368in}}%
\pgfpathlineto{\pgfqpoint{4.812706in}{3.448554in}}%
\pgfpathlineto{\pgfqpoint{4.812896in}{3.461434in}}%
\pgfpathlineto{\pgfqpoint{4.813275in}{3.388907in}}%
\pgfpathlineto{\pgfqpoint{4.813654in}{3.343775in}}%
\pgfpathlineto{\pgfqpoint{4.814033in}{3.433798in}}%
\pgfpathlineto{\pgfqpoint{4.814412in}{3.539032in}}%
\pgfpathlineto{\pgfqpoint{4.814981in}{3.410057in}}%
\pgfpathlineto{\pgfqpoint{4.815265in}{3.360557in}}%
\pgfpathlineto{\pgfqpoint{4.815739in}{3.477068in}}%
\pgfpathlineto{\pgfqpoint{4.815928in}{3.504591in}}%
\pgfpathlineto{\pgfqpoint{4.816592in}{3.427159in}}%
\pgfpathlineto{\pgfqpoint{4.816686in}{3.420983in}}%
\pgfpathlineto{\pgfqpoint{4.816971in}{3.449320in}}%
\pgfpathlineto{\pgfqpoint{4.817634in}{3.601121in}}%
\pgfpathlineto{\pgfqpoint{4.818108in}{3.478661in}}%
\pgfpathlineto{\pgfqpoint{4.818487in}{3.417606in}}%
\pgfpathlineto{\pgfqpoint{4.819056in}{3.508537in}}%
\pgfpathlineto{\pgfqpoint{4.819150in}{3.516052in}}%
\pgfpathlineto{\pgfqpoint{4.819435in}{3.475041in}}%
\pgfpathlineto{\pgfqpoint{4.819908in}{3.401892in}}%
\pgfpathlineto{\pgfqpoint{4.820477in}{3.485409in}}%
\pgfpathlineto{\pgfqpoint{4.820667in}{3.510124in}}%
\pgfpathlineto{\pgfqpoint{4.821046in}{3.436510in}}%
\pgfpathlineto{\pgfqpoint{4.821614in}{3.306787in}}%
\pgfpathlineto{\pgfqpoint{4.822088in}{3.417280in}}%
\pgfpathlineto{\pgfqpoint{4.822278in}{3.444747in}}%
\pgfpathlineto{\pgfqpoint{4.822846in}{3.370537in}}%
\pgfpathlineto{\pgfqpoint{4.823036in}{3.360018in}}%
\pgfpathlineto{\pgfqpoint{4.823415in}{3.396452in}}%
\pgfpathlineto{\pgfqpoint{4.823889in}{3.508992in}}%
\pgfpathlineto{\pgfqpoint{4.824362in}{3.371091in}}%
\pgfpathlineto{\pgfqpoint{4.824647in}{3.318537in}}%
\pgfpathlineto{\pgfqpoint{4.825215in}{3.424127in}}%
\pgfpathlineto{\pgfqpoint{4.825500in}{3.473864in}}%
\pgfpathlineto{\pgfqpoint{4.826068in}{3.389254in}}%
\pgfpathlineto{\pgfqpoint{4.826258in}{3.380042in}}%
\pgfpathlineto{\pgfqpoint{4.826637in}{3.438623in}}%
\pgfpathlineto{\pgfqpoint{4.826826in}{3.450893in}}%
\pgfpathlineto{\pgfqpoint{4.827205in}{3.387256in}}%
\pgfpathlineto{\pgfqpoint{4.827490in}{3.355580in}}%
\pgfpathlineto{\pgfqpoint{4.827869in}{3.443641in}}%
\pgfpathlineto{\pgfqpoint{4.828627in}{3.820359in}}%
\pgfpathlineto{\pgfqpoint{4.829290in}{3.661427in}}%
\pgfpathlineto{\pgfqpoint{4.829669in}{3.664670in}}%
\pgfpathlineto{\pgfqpoint{4.830048in}{3.546102in}}%
\pgfpathlineto{\pgfqpoint{4.830522in}{3.307501in}}%
\pgfpathlineto{\pgfqpoint{4.831185in}{3.437751in}}%
\pgfpathlineto{\pgfqpoint{4.831849in}{3.587796in}}%
\pgfpathlineto{\pgfqpoint{4.832417in}{3.497347in}}%
\pgfpathlineto{\pgfqpoint{4.832512in}{3.487201in}}%
\pgfpathlineto{\pgfqpoint{4.832891in}{3.542286in}}%
\pgfpathlineto{\pgfqpoint{4.833270in}{3.589319in}}%
\pgfpathlineto{\pgfqpoint{4.833649in}{3.511129in}}%
\pgfpathlineto{\pgfqpoint{4.834028in}{3.432313in}}%
\pgfpathlineto{\pgfqpoint{4.834597in}{3.559037in}}%
\pgfpathlineto{\pgfqpoint{4.834976in}{3.626614in}}%
\pgfpathlineto{\pgfqpoint{4.835545in}{3.531485in}}%
\pgfpathlineto{\pgfqpoint{4.835734in}{3.514238in}}%
\pgfpathlineto{\pgfqpoint{4.836208in}{3.587579in}}%
\pgfpathlineto{\pgfqpoint{4.836397in}{3.602358in}}%
\pgfpathlineto{\pgfqpoint{4.836776in}{3.535831in}}%
\pgfpathlineto{\pgfqpoint{4.837156in}{3.477085in}}%
\pgfpathlineto{\pgfqpoint{4.837629in}{3.570662in}}%
\pgfpathlineto{\pgfqpoint{4.838008in}{3.665300in}}%
\pgfpathlineto{\pgfqpoint{4.838577in}{3.555511in}}%
\pgfpathlineto{\pgfqpoint{4.838861in}{3.505711in}}%
\pgfpathlineto{\pgfqpoint{4.839525in}{3.596856in}}%
\pgfpathlineto{\pgfqpoint{4.839904in}{3.559575in}}%
\pgfpathlineto{\pgfqpoint{4.840283in}{3.491388in}}%
\pgfpathlineto{\pgfqpoint{4.840757in}{3.604707in}}%
\pgfpathlineto{\pgfqpoint{4.841230in}{3.675379in}}%
\pgfpathlineto{\pgfqpoint{4.841704in}{3.582382in}}%
\pgfpathlineto{\pgfqpoint{4.841988in}{3.531129in}}%
\pgfpathlineto{\pgfqpoint{4.842557in}{3.616616in}}%
\pgfpathlineto{\pgfqpoint{4.842747in}{3.627679in}}%
\pgfpathlineto{\pgfqpoint{4.843126in}{3.581375in}}%
\pgfpathlineto{\pgfqpoint{4.843505in}{3.531397in}}%
\pgfpathlineto{\pgfqpoint{4.843979in}{3.628475in}}%
\pgfpathlineto{\pgfqpoint{4.844358in}{3.712405in}}%
\pgfpathlineto{\pgfqpoint{4.844831in}{3.587184in}}%
\pgfpathlineto{\pgfqpoint{4.845210in}{3.531853in}}%
\pgfpathlineto{\pgfqpoint{4.845684in}{3.613700in}}%
\pgfpathlineto{\pgfqpoint{4.845969in}{3.651534in}}%
\pgfpathlineto{\pgfqpoint{4.846537in}{3.558590in}}%
\pgfpathlineto{\pgfqpoint{4.846632in}{3.554659in}}%
\pgfpathlineto{\pgfqpoint{4.846821in}{3.578418in}}%
\pgfpathlineto{\pgfqpoint{4.847390in}{3.714235in}}%
\pgfpathlineto{\pgfqpoint{4.847959in}{3.617674in}}%
\pgfpathlineto{\pgfqpoint{4.848338in}{3.537258in}}%
\pgfpathlineto{\pgfqpoint{4.848906in}{3.648187in}}%
\pgfpathlineto{\pgfqpoint{4.849096in}{3.660641in}}%
\pgfpathlineto{\pgfqpoint{4.849475in}{3.608370in}}%
\pgfpathlineto{\pgfqpoint{4.849759in}{3.584062in}}%
\pgfpathlineto{\pgfqpoint{4.850138in}{3.644262in}}%
\pgfpathlineto{\pgfqpoint{4.850612in}{3.730139in}}%
\pgfpathlineto{\pgfqpoint{4.850991in}{3.616902in}}%
\pgfpathlineto{\pgfqpoint{4.851370in}{3.500403in}}%
\pgfpathlineto{\pgfqpoint{4.852033in}{3.631591in}}%
\pgfpathlineto{\pgfqpoint{4.852223in}{3.643965in}}%
\pgfpathlineto{\pgfqpoint{4.852602in}{3.589697in}}%
\pgfpathlineto{\pgfqpoint{4.852981in}{3.553234in}}%
\pgfpathlineto{\pgfqpoint{4.853360in}{3.611984in}}%
\pgfpathlineto{\pgfqpoint{4.853739in}{3.659493in}}%
\pgfpathlineto{\pgfqpoint{4.854118in}{3.567122in}}%
\pgfpathlineto{\pgfqpoint{4.854497in}{3.454913in}}%
\pgfpathlineto{\pgfqpoint{4.855161in}{3.572595in}}%
\pgfpathlineto{\pgfqpoint{4.855445in}{3.613081in}}%
\pgfpathlineto{\pgfqpoint{4.855919in}{3.516709in}}%
\pgfpathlineto{\pgfqpoint{4.856014in}{3.514786in}}%
\pgfpathlineto{\pgfqpoint{4.856298in}{3.526855in}}%
\pgfpathlineto{\pgfqpoint{4.856866in}{3.621594in}}%
\pgfpathlineto{\pgfqpoint{4.857245in}{3.537988in}}%
\pgfpathlineto{\pgfqpoint{4.857719in}{3.440718in}}%
\pgfpathlineto{\pgfqpoint{4.858193in}{3.557959in}}%
\pgfpathlineto{\pgfqpoint{4.858572in}{3.619299in}}%
\pgfpathlineto{\pgfqpoint{4.859141in}{3.516314in}}%
\pgfpathlineto{\pgfqpoint{4.859236in}{3.509702in}}%
\pgfpathlineto{\pgfqpoint{4.859615in}{3.556194in}}%
\pgfpathlineto{\pgfqpoint{4.859994in}{3.609412in}}%
\pgfpathlineto{\pgfqpoint{4.860373in}{3.525668in}}%
\pgfpathlineto{\pgfqpoint{4.860847in}{3.435634in}}%
\pgfpathlineto{\pgfqpoint{4.861320in}{3.562522in}}%
\pgfpathlineto{\pgfqpoint{4.861699in}{3.659536in}}%
\pgfpathlineto{\pgfqpoint{4.862363in}{3.553354in}}%
\pgfpathlineto{\pgfqpoint{4.862837in}{3.619564in}}%
\pgfpathlineto{\pgfqpoint{4.863121in}{3.651402in}}%
\pgfpathlineto{\pgfqpoint{4.863595in}{3.574670in}}%
\pgfpathlineto{\pgfqpoint{4.863974in}{3.518269in}}%
\pgfpathlineto{\pgfqpoint{4.864353in}{3.606895in}}%
\pgfpathlineto{\pgfqpoint{4.864827in}{3.705989in}}%
\pgfpathlineto{\pgfqpoint{4.865300in}{3.593426in}}%
\pgfpathlineto{\pgfqpoint{4.866817in}{3.475003in}}%
\pgfpathlineto{\pgfqpoint{4.867101in}{3.450548in}}%
\pgfpathlineto{\pgfqpoint{4.867480in}{3.525890in}}%
\pgfpathlineto{\pgfqpoint{4.867954in}{3.630318in}}%
\pgfpathlineto{\pgfqpoint{4.868428in}{3.516691in}}%
\pgfpathlineto{\pgfqpoint{4.868807in}{3.462954in}}%
\pgfpathlineto{\pgfqpoint{4.869375in}{3.541369in}}%
\pgfpathlineto{\pgfqpoint{4.869470in}{3.547062in}}%
\pgfpathlineto{\pgfqpoint{4.869754in}{3.503388in}}%
\pgfpathlineto{\pgfqpoint{4.870133in}{3.453462in}}%
\pgfpathlineto{\pgfqpoint{4.870607in}{3.527856in}}%
\pgfpathlineto{\pgfqpoint{4.870986in}{3.625090in}}%
\pgfpathlineto{\pgfqpoint{4.871555in}{3.511181in}}%
\pgfpathlineto{\pgfqpoint{4.871934in}{3.460799in}}%
\pgfpathlineto{\pgfqpoint{4.872408in}{3.536870in}}%
\pgfpathlineto{\pgfqpoint{4.872597in}{3.558210in}}%
\pgfpathlineto{\pgfqpoint{4.872976in}{3.462193in}}%
\pgfpathlineto{\pgfqpoint{4.873166in}{3.429217in}}%
\pgfpathlineto{\pgfqpoint{4.873545in}{3.558412in}}%
\pgfpathlineto{\pgfqpoint{4.874303in}{4.031308in}}%
\pgfpathlineto{\pgfqpoint{4.874966in}{3.840457in}}%
\pgfpathlineto{\pgfqpoint{4.875440in}{3.857807in}}%
\pgfpathlineto{\pgfqpoint{4.876009in}{3.599733in}}%
\pgfpathlineto{\pgfqpoint{4.876388in}{3.358515in}}%
\pgfpathlineto{\pgfqpoint{4.877051in}{3.621009in}}%
\pgfpathlineto{\pgfqpoint{4.877335in}{3.661651in}}%
\pgfpathlineto{\pgfqpoint{4.877809in}{3.558083in}}%
\pgfpathlineto{\pgfqpoint{4.878188in}{3.483164in}}%
\pgfpathlineto{\pgfqpoint{4.878757in}{3.612193in}}%
\pgfpathlineto{\pgfqpoint{4.878946in}{3.639614in}}%
\pgfpathlineto{\pgfqpoint{4.879610in}{3.567648in}}%
\pgfpathlineto{\pgfqpoint{4.879989in}{3.598850in}}%
\pgfpathlineto{\pgfqpoint{4.880463in}{3.694037in}}%
\pgfpathlineto{\pgfqpoint{4.880937in}{3.567648in}}%
\pgfpathlineto{\pgfqpoint{4.881221in}{3.486213in}}%
\pgfpathlineto{\pgfqpoint{4.881789in}{3.624125in}}%
\pgfpathlineto{\pgfqpoint{4.882168in}{3.667018in}}%
\pgfpathlineto{\pgfqpoint{4.882642in}{3.593235in}}%
\pgfpathlineto{\pgfqpoint{4.882737in}{3.587217in}}%
\pgfpathlineto{\pgfqpoint{4.883116in}{3.622288in}}%
\pgfpathlineto{\pgfqpoint{4.883590in}{3.678181in}}%
\pgfpathlineto{\pgfqpoint{4.883874in}{3.624399in}}%
\pgfpathlineto{\pgfqpoint{4.884443in}{3.501252in}}%
\pgfpathlineto{\pgfqpoint{4.884917in}{3.600159in}}%
\pgfpathlineto{\pgfqpoint{4.885296in}{3.669452in}}%
\pgfpathlineto{\pgfqpoint{4.885864in}{3.573072in}}%
\pgfpathlineto{\pgfqpoint{4.886054in}{3.563829in}}%
\pgfpathlineto{\pgfqpoint{4.886338in}{3.603030in}}%
\pgfpathlineto{\pgfqpoint{4.886622in}{3.645993in}}%
\pgfpathlineto{\pgfqpoint{4.887096in}{3.566638in}}%
\pgfpathlineto{\pgfqpoint{4.887475in}{3.496433in}}%
\pgfpathlineto{\pgfqpoint{4.888044in}{3.604487in}}%
\pgfpathlineto{\pgfqpoint{4.888423in}{3.669163in}}%
\pgfpathlineto{\pgfqpoint{4.888991in}{3.581402in}}%
\pgfpathlineto{\pgfqpoint{4.889276in}{3.555417in}}%
\pgfpathlineto{\pgfqpoint{4.889750in}{3.617662in}}%
\pgfpathlineto{\pgfqpoint{4.889844in}{3.627405in}}%
\pgfpathlineto{\pgfqpoint{4.890223in}{3.567277in}}%
\pgfpathlineto{\pgfqpoint{4.890697in}{3.471251in}}%
\pgfpathlineto{\pgfqpoint{4.891171in}{3.582060in}}%
\pgfpathlineto{\pgfqpoint{4.891550in}{3.637703in}}%
\pgfpathlineto{\pgfqpoint{4.892024in}{3.552169in}}%
\pgfpathlineto{\pgfqpoint{4.893824in}{3.434951in}}%
\pgfpathlineto{\pgfqpoint{4.894298in}{3.530528in}}%
\pgfpathlineto{\pgfqpoint{4.894677in}{3.603464in}}%
\pgfpathlineto{\pgfqpoint{4.895151in}{3.486878in}}%
\pgfpathlineto{\pgfqpoint{4.895435in}{3.429225in}}%
\pgfpathlineto{\pgfqpoint{4.896099in}{3.503482in}}%
\pgfpathlineto{\pgfqpoint{4.896194in}{3.496760in}}%
\pgfpathlineto{\pgfqpoint{4.896952in}{3.326183in}}%
\pgfpathlineto{\pgfqpoint{4.897520in}{3.444814in}}%
\pgfpathlineto{\pgfqpoint{4.897710in}{3.458125in}}%
\pgfpathlineto{\pgfqpoint{4.898089in}{3.399277in}}%
\pgfpathlineto{\pgfqpoint{4.898657in}{3.264887in}}%
\pgfpathlineto{\pgfqpoint{4.899131in}{3.382062in}}%
\pgfpathlineto{\pgfqpoint{4.899984in}{3.380330in}}%
\pgfpathlineto{\pgfqpoint{4.900932in}{3.552489in}}%
\pgfpathlineto{\pgfqpoint{4.901879in}{3.348585in}}%
\pgfpathlineto{\pgfqpoint{4.903206in}{3.377704in}}%
\pgfpathlineto{\pgfqpoint{4.903301in}{3.376740in}}%
\pgfpathlineto{\pgfqpoint{4.903396in}{3.379477in}}%
\pgfpathlineto{\pgfqpoint{4.903964in}{3.504225in}}%
\pgfpathlineto{\pgfqpoint{4.904438in}{3.409284in}}%
\pgfpathlineto{\pgfqpoint{4.904912in}{3.309834in}}%
\pgfpathlineto{\pgfqpoint{4.905480in}{3.421368in}}%
\pgfpathlineto{\pgfqpoint{4.905859in}{3.447966in}}%
\pgfpathlineto{\pgfqpoint{4.906239in}{3.405134in}}%
\pgfpathlineto{\pgfqpoint{4.906428in}{3.388066in}}%
\pgfpathlineto{\pgfqpoint{4.906807in}{3.463564in}}%
\pgfpathlineto{\pgfqpoint{4.907186in}{3.524568in}}%
\pgfpathlineto{\pgfqpoint{4.907660in}{3.435866in}}%
\pgfpathlineto{\pgfqpoint{4.908039in}{3.359823in}}%
\pgfpathlineto{\pgfqpoint{4.908513in}{3.493872in}}%
\pgfpathlineto{\pgfqpoint{4.908987in}{3.565615in}}%
\pgfpathlineto{\pgfqpoint{4.909745in}{3.536047in}}%
\pgfpathlineto{\pgfqpoint{4.910313in}{3.604953in}}%
\pgfpathlineto{\pgfqpoint{4.910692in}{3.534269in}}%
\pgfpathlineto{\pgfqpoint{4.911166in}{3.442188in}}%
\pgfpathlineto{\pgfqpoint{4.911735in}{3.550245in}}%
\pgfpathlineto{\pgfqpoint{4.912019in}{3.581415in}}%
\pgfpathlineto{\pgfqpoint{4.912493in}{3.505734in}}%
\pgfpathlineto{\pgfqpoint{4.912682in}{3.495678in}}%
\pgfpathlineto{\pgfqpoint{4.913062in}{3.545495in}}%
\pgfpathlineto{\pgfqpoint{4.913441in}{3.579042in}}%
\pgfpathlineto{\pgfqpoint{4.913820in}{3.515903in}}%
\pgfpathlineto{\pgfqpoint{4.914199in}{3.426549in}}%
\pgfpathlineto{\pgfqpoint{4.914767in}{3.545533in}}%
\pgfpathlineto{\pgfqpoint{4.915241in}{3.644900in}}%
\pgfpathlineto{\pgfqpoint{4.915810in}{3.531555in}}%
\pgfpathlineto{\pgfqpoint{4.915904in}{3.527687in}}%
\pgfpathlineto{\pgfqpoint{4.916189in}{3.557082in}}%
\pgfpathlineto{\pgfqpoint{4.916663in}{3.619474in}}%
\pgfpathlineto{\pgfqpoint{4.917042in}{3.539479in}}%
\pgfpathlineto{\pgfqpoint{4.917326in}{3.509766in}}%
\pgfpathlineto{\pgfqpoint{4.917800in}{3.563760in}}%
\pgfpathlineto{\pgfqpoint{4.918274in}{3.691010in}}%
\pgfpathlineto{\pgfqpoint{4.918747in}{3.566743in}}%
\pgfpathlineto{\pgfqpoint{4.919032in}{3.507450in}}%
\pgfpathlineto{\pgfqpoint{4.919506in}{3.647324in}}%
\pgfpathlineto{\pgfqpoint{4.921306in}{4.043906in}}%
\pgfpathlineto{\pgfqpoint{4.921496in}{4.021138in}}%
\pgfpathlineto{\pgfqpoint{4.922443in}{3.418359in}}%
\pgfpathlineto{\pgfqpoint{4.923580in}{3.564175in}}%
\pgfpathlineto{\pgfqpoint{4.923675in}{3.558271in}}%
\pgfpathlineto{\pgfqpoint{4.923959in}{3.606987in}}%
\pgfpathlineto{\pgfqpoint{4.924433in}{3.738582in}}%
\pgfpathlineto{\pgfqpoint{4.925097in}{3.618818in}}%
\pgfpathlineto{\pgfqpoint{4.925476in}{3.564970in}}%
\pgfpathlineto{\pgfqpoint{4.926044in}{3.637366in}}%
\pgfpathlineto{\pgfqpoint{4.926139in}{3.641351in}}%
\pgfpathlineto{\pgfqpoint{4.926423in}{3.613768in}}%
\pgfpathlineto{\pgfqpoint{4.926802in}{3.553371in}}%
\pgfpathlineto{\pgfqpoint{4.927276in}{3.667840in}}%
\pgfpathlineto{\pgfqpoint{4.927655in}{3.721374in}}%
\pgfpathlineto{\pgfqpoint{4.928129in}{3.615634in}}%
\pgfpathlineto{\pgfqpoint{4.928603in}{3.557480in}}%
\pgfpathlineto{\pgfqpoint{4.929077in}{3.624515in}}%
\pgfpathlineto{\pgfqpoint{4.929266in}{3.639623in}}%
\pgfpathlineto{\pgfqpoint{4.929740in}{3.588215in}}%
\pgfpathlineto{\pgfqpoint{4.929930in}{3.572477in}}%
\pgfpathlineto{\pgfqpoint{4.930309in}{3.631951in}}%
\pgfpathlineto{\pgfqpoint{4.930782in}{3.716042in}}%
\pgfpathlineto{\pgfqpoint{4.931256in}{3.608327in}}%
\pgfpathlineto{\pgfqpoint{4.931635in}{3.527416in}}%
\pgfpathlineto{\pgfqpoint{4.932204in}{3.628870in}}%
\pgfpathlineto{\pgfqpoint{4.932488in}{3.654504in}}%
\pgfpathlineto{\pgfqpoint{4.933152in}{3.610558in}}%
\pgfpathlineto{\pgfqpoint{4.933625in}{3.709392in}}%
\pgfpathlineto{\pgfqpoint{4.933815in}{3.735733in}}%
\pgfpathlineto{\pgfqpoint{4.934289in}{3.636645in}}%
\pgfpathlineto{\pgfqpoint{4.934763in}{3.563117in}}%
\pgfpathlineto{\pgfqpoint{4.935236in}{3.638816in}}%
\pgfpathlineto{\pgfqpoint{4.935615in}{3.705862in}}%
\pgfpathlineto{\pgfqpoint{4.936279in}{3.644423in}}%
\pgfpathlineto{\pgfqpoint{4.936374in}{3.643149in}}%
\pgfpathlineto{\pgfqpoint{4.936468in}{3.648961in}}%
\pgfpathlineto{\pgfqpoint{4.937037in}{3.712880in}}%
\pgfpathlineto{\pgfqpoint{4.937416in}{3.645414in}}%
\pgfpathlineto{\pgfqpoint{4.937795in}{3.566376in}}%
\pgfpathlineto{\pgfqpoint{4.938458in}{3.662739in}}%
\pgfpathlineto{\pgfqpoint{4.938837in}{3.707207in}}%
\pgfpathlineto{\pgfqpoint{4.939406in}{3.632737in}}%
\pgfpathlineto{\pgfqpoint{4.939880in}{3.694245in}}%
\pgfpathlineto{\pgfqpoint{4.940069in}{3.701352in}}%
\pgfpathlineto{\pgfqpoint{4.940354in}{3.675487in}}%
\pgfpathlineto{\pgfqpoint{4.941017in}{3.541553in}}%
\pgfpathlineto{\pgfqpoint{4.941491in}{3.624243in}}%
\pgfpathlineto{\pgfqpoint{4.941870in}{3.709481in}}%
\pgfpathlineto{\pgfqpoint{4.942438in}{3.605626in}}%
\pgfpathlineto{\pgfqpoint{4.942628in}{3.594097in}}%
\pgfpathlineto{\pgfqpoint{4.943102in}{3.644257in}}%
\pgfpathlineto{\pgfqpoint{4.943197in}{3.647934in}}%
\pgfpathlineto{\pgfqpoint{4.943386in}{3.631137in}}%
\pgfpathlineto{\pgfqpoint{4.944144in}{3.472384in}}%
\pgfpathlineto{\pgfqpoint{4.944713in}{3.585978in}}%
\pgfpathlineto{\pgfqpoint{4.944997in}{3.627347in}}%
\pgfpathlineto{\pgfqpoint{4.945471in}{3.531424in}}%
\pgfpathlineto{\pgfqpoint{4.945850in}{3.503621in}}%
\pgfpathlineto{\pgfqpoint{4.946419in}{3.555595in}}%
\pgfpathlineto{\pgfqpoint{4.946703in}{3.512668in}}%
\pgfpathlineto{\pgfqpoint{4.947271in}{3.422224in}}%
\pgfpathlineto{\pgfqpoint{4.947745in}{3.521676in}}%
\pgfpathlineto{\pgfqpoint{4.948219in}{3.611215in}}%
\pgfpathlineto{\pgfqpoint{4.948693in}{3.496973in}}%
\pgfpathlineto{\pgfqpoint{4.948882in}{3.465352in}}%
\pgfpathlineto{\pgfqpoint{4.949640in}{3.528222in}}%
\pgfpathlineto{\pgfqpoint{4.950114in}{3.467264in}}%
\pgfpathlineto{\pgfqpoint{4.950399in}{3.424589in}}%
\pgfpathlineto{\pgfqpoint{4.950778in}{3.504853in}}%
\pgfpathlineto{\pgfqpoint{4.951251in}{3.600541in}}%
\pgfpathlineto{\pgfqpoint{4.951820in}{3.484558in}}%
\pgfpathlineto{\pgfqpoint{4.952104in}{3.446057in}}%
\pgfpathlineto{\pgfqpoint{4.952768in}{3.510870in}}%
\pgfpathlineto{\pgfqpoint{4.952862in}{3.510826in}}%
\pgfpathlineto{\pgfqpoint{4.953242in}{3.478319in}}%
\pgfpathlineto{\pgfqpoint{4.953526in}{3.448923in}}%
\pgfpathlineto{\pgfqpoint{4.953905in}{3.537255in}}%
\pgfpathlineto{\pgfqpoint{4.954379in}{3.628135in}}%
\pgfpathlineto{\pgfqpoint{4.954947in}{3.534501in}}%
\pgfpathlineto{\pgfqpoint{4.955326in}{3.491814in}}%
\pgfpathlineto{\pgfqpoint{4.955800in}{3.563780in}}%
\pgfpathlineto{\pgfqpoint{4.955895in}{3.567929in}}%
\pgfpathlineto{\pgfqpoint{4.956179in}{3.542782in}}%
\pgfpathlineto{\pgfqpoint{4.956653in}{3.499842in}}%
\pgfpathlineto{\pgfqpoint{4.957032in}{3.544751in}}%
\pgfpathlineto{\pgfqpoint{4.957411in}{3.610918in}}%
\pgfpathlineto{\pgfqpoint{4.957885in}{3.515118in}}%
\pgfpathlineto{\pgfqpoint{4.958454in}{3.383983in}}%
\pgfpathlineto{\pgfqpoint{4.959022in}{3.481076in}}%
\pgfpathlineto{\pgfqpoint{4.960633in}{3.578850in}}%
\pgfpathlineto{\pgfqpoint{4.959780in}{3.457689in}}%
\pgfpathlineto{\pgfqpoint{4.960823in}{3.566597in}}%
\pgfpathlineto{\pgfqpoint{4.961486in}{3.408432in}}%
\pgfpathlineto{\pgfqpoint{4.962055in}{3.515173in}}%
\pgfpathlineto{\pgfqpoint{4.962339in}{3.567775in}}%
\pgfpathlineto{\pgfqpoint{4.962907in}{3.502417in}}%
\pgfpathlineto{\pgfqpoint{4.963192in}{3.526269in}}%
\pgfpathlineto{\pgfqpoint{4.963760in}{3.603262in}}%
\pgfpathlineto{\pgfqpoint{4.964139in}{3.533644in}}%
\pgfpathlineto{\pgfqpoint{4.964708in}{3.409806in}}%
\pgfpathlineto{\pgfqpoint{4.965182in}{3.525767in}}%
\pgfpathlineto{\pgfqpoint{4.966793in}{4.015283in}}%
\pgfpathlineto{\pgfqpoint{4.967077in}{3.960648in}}%
\pgfpathlineto{\pgfqpoint{4.968214in}{3.423471in}}%
\pgfpathlineto{\pgfqpoint{4.969162in}{3.583895in}}%
\pgfpathlineto{\pgfqpoint{4.970868in}{3.420962in}}%
\pgfpathlineto{\pgfqpoint{4.969825in}{3.612598in}}%
\pgfpathlineto{\pgfqpoint{4.971152in}{3.461464in}}%
\pgfpathlineto{\pgfqpoint{4.971721in}{3.586128in}}%
\pgfpathlineto{\pgfqpoint{4.972479in}{3.541978in}}%
\pgfpathlineto{\pgfqpoint{4.973142in}{3.658480in}}%
\pgfpathlineto{\pgfqpoint{4.973805in}{3.583581in}}%
\pgfpathlineto{\pgfqpoint{4.973995in}{3.568413in}}%
\pgfpathlineto{\pgfqpoint{4.974374in}{3.642206in}}%
\pgfpathlineto{\pgfqpoint{4.974848in}{3.751653in}}%
\pgfpathlineto{\pgfqpoint{4.975416in}{3.641147in}}%
\pgfpathlineto{\pgfqpoint{4.975701in}{3.596492in}}%
\pgfpathlineto{\pgfqpoint{4.976364in}{3.662935in}}%
\pgfpathlineto{\pgfqpoint{4.976459in}{3.656125in}}%
\pgfpathlineto{\pgfqpoint{4.977027in}{3.572545in}}%
\pgfpathlineto{\pgfqpoint{4.977501in}{3.651285in}}%
\pgfpathlineto{\pgfqpoint{4.977975in}{3.738387in}}%
\pgfpathlineto{\pgfqpoint{4.978544in}{3.645485in}}%
\pgfpathlineto{\pgfqpoint{4.978828in}{3.612306in}}%
\pgfpathlineto{\pgfqpoint{4.979396in}{3.676410in}}%
\pgfpathlineto{\pgfqpoint{4.979491in}{3.681136in}}%
\pgfpathlineto{\pgfqpoint{4.979775in}{3.655201in}}%
\pgfpathlineto{\pgfqpoint{4.980249in}{3.609826in}}%
\pgfpathlineto{\pgfqpoint{4.980628in}{3.665944in}}%
\pgfpathlineto{\pgfqpoint{4.981102in}{3.785763in}}%
\pgfpathlineto{\pgfqpoint{4.981671in}{3.663790in}}%
\pgfpathlineto{\pgfqpoint{4.982050in}{3.615612in}}%
\pgfpathlineto{\pgfqpoint{4.982618in}{3.693265in}}%
\pgfpathlineto{\pgfqpoint{4.982997in}{3.669758in}}%
\pgfpathlineto{\pgfqpoint{4.983377in}{3.635261in}}%
\pgfpathlineto{\pgfqpoint{4.983756in}{3.700574in}}%
\pgfpathlineto{\pgfqpoint{4.984229in}{3.785103in}}%
\pgfpathlineto{\pgfqpoint{4.984703in}{3.676920in}}%
\pgfpathlineto{\pgfqpoint{4.985177in}{3.583967in}}%
\pgfpathlineto{\pgfqpoint{4.985746in}{3.691911in}}%
\pgfpathlineto{\pgfqpoint{4.985935in}{3.686821in}}%
\pgfpathlineto{\pgfqpoint{4.986599in}{3.629578in}}%
\pgfpathlineto{\pgfqpoint{4.986883in}{3.668513in}}%
\pgfpathlineto{\pgfqpoint{4.987357in}{3.762207in}}%
\pgfpathlineto{\pgfqpoint{4.987830in}{3.653526in}}%
\pgfpathlineto{\pgfqpoint{4.988304in}{3.571574in}}%
\pgfpathlineto{\pgfqpoint{4.988873in}{3.668623in}}%
\pgfpathlineto{\pgfqpoint{4.989062in}{3.680579in}}%
\pgfpathlineto{\pgfqpoint{4.989441in}{3.620183in}}%
\pgfpathlineto{\pgfqpoint{4.989631in}{3.608032in}}%
\pgfpathlineto{\pgfqpoint{4.990105in}{3.651098in}}%
\pgfpathlineto{\pgfqpoint{4.990389in}{3.690555in}}%
\pgfpathlineto{\pgfqpoint{4.990768in}{3.619567in}}%
\pgfpathlineto{\pgfqpoint{4.991431in}{3.477570in}}%
\pgfpathlineto{\pgfqpoint{4.991905in}{3.577270in}}%
\pgfpathlineto{\pgfqpoint{4.992190in}{3.633257in}}%
\pgfpathlineto{\pgfqpoint{4.992853in}{3.529106in}}%
\pgfpathlineto{\pgfqpoint{4.993232in}{3.555337in}}%
\pgfpathlineto{\pgfqpoint{4.993611in}{3.603270in}}%
\pgfpathlineto{\pgfqpoint{4.993990in}{3.526786in}}%
\pgfpathlineto{\pgfqpoint{4.994464in}{3.443075in}}%
\pgfpathlineto{\pgfqpoint{4.995033in}{3.541503in}}%
\pgfpathlineto{\pgfqpoint{4.995317in}{3.592948in}}%
\pgfpathlineto{\pgfqpoint{4.995980in}{3.516604in}}%
\pgfpathlineto{\pgfqpoint{4.996170in}{3.519760in}}%
\pgfpathlineto{\pgfqpoint{4.996738in}{3.587899in}}%
\pgfpathlineto{\pgfqpoint{4.997023in}{3.526642in}}%
\pgfpathlineto{\pgfqpoint{4.997686in}{3.412457in}}%
\pgfpathlineto{\pgfqpoint{4.998065in}{3.501448in}}%
\pgfpathlineto{\pgfqpoint{4.998539in}{3.583656in}}%
\pgfpathlineto{\pgfqpoint{4.999202in}{3.510869in}}%
\pgfpathlineto{\pgfqpoint{4.999297in}{3.507816in}}%
\pgfpathlineto{\pgfqpoint{4.999581in}{3.532542in}}%
\pgfpathlineto{\pgfqpoint{4.999960in}{3.583054in}}%
\pgfpathlineto{\pgfqpoint{5.000434in}{3.519333in}}%
\pgfpathlineto{\pgfqpoint{5.000718in}{3.463454in}}%
\pgfpathlineto{\pgfqpoint{5.001192in}{3.594877in}}%
\pgfpathlineto{\pgfqpoint{5.001571in}{3.659534in}}%
\pgfpathlineto{\pgfqpoint{5.002140in}{3.572144in}}%
\pgfpathlineto{\pgfqpoint{5.003846in}{3.418373in}}%
\pgfpathlineto{\pgfqpoint{5.004035in}{3.435321in}}%
\pgfpathlineto{\pgfqpoint{5.004698in}{3.587212in}}%
\pgfpathlineto{\pgfqpoint{5.005267in}{3.475213in}}%
\pgfpathlineto{\pgfqpoint{5.005646in}{3.448241in}}%
\pgfpathlineto{\pgfqpoint{5.006215in}{3.497935in}}%
\pgfpathlineto{\pgfqpoint{5.006404in}{3.488917in}}%
\pgfpathlineto{\pgfqpoint{5.006973in}{3.420679in}}%
\pgfpathlineto{\pgfqpoint{5.007352in}{3.482650in}}%
\pgfpathlineto{\pgfqpoint{5.007826in}{3.603277in}}%
\pgfpathlineto{\pgfqpoint{5.008394in}{3.475654in}}%
\pgfpathlineto{\pgfqpoint{5.008868in}{3.429998in}}%
\pgfpathlineto{\pgfqpoint{5.009437in}{3.488886in}}%
\pgfpathlineto{\pgfqpoint{5.009910in}{3.459876in}}%
\pgfpathlineto{\pgfqpoint{5.010384in}{3.481360in}}%
\pgfpathlineto{\pgfqpoint{5.010858in}{3.576042in}}%
\pgfpathlineto{\pgfqpoint{5.011332in}{3.475045in}}%
\pgfpathlineto{\pgfqpoint{5.011521in}{3.449997in}}%
\pgfpathlineto{\pgfqpoint{5.011901in}{3.551884in}}%
\pgfpathlineto{\pgfqpoint{5.012753in}{3.917805in}}%
\pgfpathlineto{\pgfqpoint{5.013417in}{3.840130in}}%
\pgfpathlineto{\pgfqpoint{5.013701in}{3.867697in}}%
\pgfpathlineto{\pgfqpoint{5.013985in}{3.790951in}}%
\pgfpathlineto{\pgfqpoint{5.014743in}{3.310592in}}%
\pgfpathlineto{\pgfqpoint{5.015407in}{3.526242in}}%
\pgfpathlineto{\pgfqpoint{5.015786in}{3.583578in}}%
\pgfpathlineto{\pgfqpoint{5.016639in}{3.560115in}}%
\pgfpathlineto{\pgfqpoint{5.017207in}{3.647828in}}%
\pgfpathlineto{\pgfqpoint{5.017681in}{3.548432in}}%
\pgfpathlineto{\pgfqpoint{5.018060in}{3.471482in}}%
\pgfpathlineto{\pgfqpoint{5.018629in}{3.572227in}}%
\pgfpathlineto{\pgfqpoint{5.018913in}{3.627811in}}%
\pgfpathlineto{\pgfqpoint{5.019576in}{3.562240in}}%
\pgfpathlineto{\pgfqpoint{5.019671in}{3.563272in}}%
\pgfpathlineto{\pgfqpoint{5.020240in}{3.637103in}}%
\pgfpathlineto{\pgfqpoint{5.020714in}{3.580146in}}%
\pgfpathlineto{\pgfqpoint{5.021282in}{3.440173in}}%
\pgfpathlineto{\pgfqpoint{5.021756in}{3.564043in}}%
\pgfpathlineto{\pgfqpoint{5.022135in}{3.608852in}}%
\pgfpathlineto{\pgfqpoint{5.022704in}{3.547977in}}%
\pgfpathlineto{\pgfqpoint{5.022798in}{3.551575in}}%
\pgfpathlineto{\pgfqpoint{5.023367in}{3.610162in}}%
\pgfpathlineto{\pgfqpoint{5.023746in}{3.566004in}}%
\pgfpathlineto{\pgfqpoint{5.024315in}{3.468772in}}%
\pgfpathlineto{\pgfqpoint{5.024788in}{3.568656in}}%
\pgfpathlineto{\pgfqpoint{5.025262in}{3.660151in}}%
\pgfpathlineto{\pgfqpoint{5.025926in}{3.587381in}}%
\pgfpathlineto{\pgfqpoint{5.026020in}{3.587454in}}%
\pgfpathlineto{\pgfqpoint{5.026494in}{3.644396in}}%
\pgfpathlineto{\pgfqpoint{5.027063in}{3.582847in}}%
\pgfpathlineto{\pgfqpoint{5.027537in}{3.512673in}}%
\pgfpathlineto{\pgfqpoint{5.027916in}{3.623407in}}%
\pgfpathlineto{\pgfqpoint{5.028389in}{3.701778in}}%
\pgfpathlineto{\pgfqpoint{5.028958in}{3.605311in}}%
\pgfpathlineto{\pgfqpoint{5.030569in}{3.487124in}}%
\pgfpathlineto{\pgfqpoint{5.029621in}{3.613464in}}%
\pgfpathlineto{\pgfqpoint{5.030759in}{3.501858in}}%
\pgfpathlineto{\pgfqpoint{5.031517in}{3.678062in}}%
\pgfpathlineto{\pgfqpoint{5.031991in}{3.567324in}}%
\pgfpathlineto{\pgfqpoint{5.032464in}{3.536094in}}%
\pgfpathlineto{\pgfqpoint{5.032938in}{3.587647in}}%
\pgfpathlineto{\pgfqpoint{5.033033in}{3.587986in}}%
\pgfpathlineto{\pgfqpoint{5.033696in}{3.493147in}}%
\pgfpathlineto{\pgfqpoint{5.034075in}{3.565759in}}%
\pgfpathlineto{\pgfqpoint{5.034644in}{3.656292in}}%
\pgfpathlineto{\pgfqpoint{5.035023in}{3.574017in}}%
\pgfpathlineto{\pgfqpoint{5.035402in}{3.494448in}}%
\pgfpathlineto{\pgfqpoint{5.036160in}{3.548268in}}%
\pgfpathlineto{\pgfqpoint{5.036918in}{3.480427in}}%
\pgfpathlineto{\pgfqpoint{5.037203in}{3.534769in}}%
\pgfpathlineto{\pgfqpoint{5.037676in}{3.629831in}}%
\pgfpathlineto{\pgfqpoint{5.038245in}{3.511667in}}%
\pgfpathlineto{\pgfqpoint{5.038719in}{3.450648in}}%
\pgfpathlineto{\pgfqpoint{5.039382in}{3.500334in}}%
\pgfpathlineto{\pgfqpoint{5.039951in}{3.449660in}}%
\pgfpathlineto{\pgfqpoint{5.040330in}{3.491320in}}%
\pgfpathlineto{\pgfqpoint{5.040709in}{3.536732in}}%
\pgfpathlineto{\pgfqpoint{5.041088in}{3.482722in}}%
\pgfpathlineto{\pgfqpoint{5.041751in}{3.299360in}}%
\pgfpathlineto{\pgfqpoint{5.042509in}{3.367250in}}%
\pgfpathlineto{\pgfqpoint{5.042888in}{3.366880in}}%
\pgfpathlineto{\pgfqpoint{5.042983in}{3.367473in}}%
\pgfpathlineto{\pgfqpoint{5.043362in}{3.404454in}}%
\pgfpathlineto{\pgfqpoint{5.044026in}{3.557815in}}%
\pgfpathlineto{\pgfqpoint{5.044594in}{3.440915in}}%
\pgfpathlineto{\pgfqpoint{5.044689in}{3.434513in}}%
\pgfpathlineto{\pgfqpoint{5.045068in}{3.469496in}}%
\pgfpathlineto{\pgfqpoint{5.047248in}{3.736741in}}%
\pgfpathlineto{\pgfqpoint{5.047342in}{3.721046in}}%
\pgfpathlineto{\pgfqpoint{5.047911in}{3.578861in}}%
\pgfpathlineto{\pgfqpoint{5.048479in}{3.694950in}}%
\pgfpathlineto{\pgfqpoint{5.048764in}{3.727116in}}%
\pgfpathlineto{\pgfqpoint{5.049427in}{3.674664in}}%
\pgfpathlineto{\pgfqpoint{5.051038in}{3.479902in}}%
\pgfpathlineto{\pgfqpoint{5.050090in}{3.691370in}}%
\pgfpathlineto{\pgfqpoint{5.051417in}{3.530711in}}%
\pgfpathlineto{\pgfqpoint{5.051986in}{3.641929in}}%
\pgfpathlineto{\pgfqpoint{5.052649in}{3.567745in}}%
\pgfpathlineto{\pgfqpoint{5.052744in}{3.567275in}}%
\pgfpathlineto{\pgfqpoint{5.053312in}{3.599877in}}%
\pgfpathlineto{\pgfqpoint{5.053691in}{3.565799in}}%
\pgfpathlineto{\pgfqpoint{5.054165in}{3.473605in}}%
\pgfpathlineto{\pgfqpoint{5.054639in}{3.572707in}}%
\pgfpathlineto{\pgfqpoint{5.055018in}{3.641143in}}%
\pgfpathlineto{\pgfqpoint{5.055682in}{3.554813in}}%
\pgfpathlineto{\pgfqpoint{5.057293in}{3.391279in}}%
\pgfpathlineto{\pgfqpoint{5.056440in}{3.585122in}}%
\pgfpathlineto{\pgfqpoint{5.057482in}{3.424345in}}%
\pgfpathlineto{\pgfqpoint{5.058524in}{3.958527in}}%
\pgfpathlineto{\pgfqpoint{5.059472in}{3.857446in}}%
\pgfpathlineto{\pgfqpoint{5.059662in}{3.847381in}}%
\pgfpathlineto{\pgfqpoint{5.060230in}{3.490767in}}%
\pgfpathlineto{\pgfqpoint{5.060609in}{3.320915in}}%
\pgfpathlineto{\pgfqpoint{5.061178in}{3.596391in}}%
\pgfpathlineto{\pgfqpoint{5.061367in}{3.634153in}}%
\pgfpathlineto{\pgfqpoint{5.061936in}{3.520232in}}%
\pgfpathlineto{\pgfqpoint{5.062315in}{3.473756in}}%
\pgfpathlineto{\pgfqpoint{5.062789in}{3.521569in}}%
\pgfpathlineto{\pgfqpoint{5.063073in}{3.513006in}}%
\pgfpathlineto{\pgfqpoint{5.063642in}{3.449450in}}%
\pgfpathlineto{\pgfqpoint{5.063926in}{3.511542in}}%
\pgfpathlineto{\pgfqpoint{5.064400in}{3.595693in}}%
\pgfpathlineto{\pgfqpoint{5.064968in}{3.495610in}}%
\pgfpathlineto{\pgfqpoint{5.065347in}{3.428749in}}%
\pgfpathlineto{\pgfqpoint{5.066011in}{3.500049in}}%
\pgfpathlineto{\pgfqpoint{5.066674in}{3.466674in}}%
\pgfpathlineto{\pgfqpoint{5.066958in}{3.493523in}}%
\pgfpathlineto{\pgfqpoint{5.067432in}{3.583511in}}%
\pgfpathlineto{\pgfqpoint{5.068001in}{3.501313in}}%
\pgfpathlineto{\pgfqpoint{5.068475in}{3.407453in}}%
\pgfpathlineto{\pgfqpoint{5.069043in}{3.479925in}}%
\pgfpathlineto{\pgfqpoint{5.069801in}{3.477623in}}%
\pgfpathlineto{\pgfqpoint{5.070654in}{3.610756in}}%
\pgfpathlineto{\pgfqpoint{5.070749in}{3.609700in}}%
\pgfpathlineto{\pgfqpoint{5.071697in}{3.436903in}}%
\pgfpathlineto{\pgfqpoint{5.072170in}{3.538869in}}%
\pgfpathlineto{\pgfqpoint{5.073687in}{3.649790in}}%
\pgfpathlineto{\pgfqpoint{5.073781in}{3.648782in}}%
\pgfpathlineto{\pgfqpoint{5.074729in}{3.465404in}}%
\pgfpathlineto{\pgfqpoint{5.075203in}{3.580821in}}%
\pgfpathlineto{\pgfqpoint{5.075582in}{3.619988in}}%
\pgfpathlineto{\pgfqpoint{5.076151in}{3.577456in}}%
\pgfpathlineto{\pgfqpoint{5.076340in}{3.589216in}}%
\pgfpathlineto{\pgfqpoint{5.076814in}{3.649824in}}%
\pgfpathlineto{\pgfqpoint{5.077288in}{3.598419in}}%
\pgfpathlineto{\pgfqpoint{5.077856in}{3.490658in}}%
\pgfpathlineto{\pgfqpoint{5.078330in}{3.606165in}}%
\pgfpathlineto{\pgfqpoint{5.078709in}{3.679858in}}%
\pgfpathlineto{\pgfqpoint{5.079373in}{3.592946in}}%
\pgfpathlineto{\pgfqpoint{5.079467in}{3.584448in}}%
\pgfpathlineto{\pgfqpoint{5.079941in}{3.641409in}}%
\pgfpathlineto{\pgfqpoint{5.080036in}{3.646320in}}%
\pgfpathlineto{\pgfqpoint{5.080320in}{3.607024in}}%
\pgfpathlineto{\pgfqpoint{5.080889in}{3.495183in}}%
\pgfpathlineto{\pgfqpoint{5.081363in}{3.592588in}}%
\pgfpathlineto{\pgfqpoint{5.081742in}{3.678361in}}%
\pgfpathlineto{\pgfqpoint{5.082405in}{3.593306in}}%
\pgfpathlineto{\pgfqpoint{5.083921in}{3.504095in}}%
\pgfpathlineto{\pgfqpoint{5.083163in}{3.605435in}}%
\pgfpathlineto{\pgfqpoint{5.084111in}{3.516652in}}%
\pgfpathlineto{\pgfqpoint{5.084964in}{3.690869in}}%
\pgfpathlineto{\pgfqpoint{5.085532in}{3.600177in}}%
\pgfpathlineto{\pgfqpoint{5.087143in}{3.514177in}}%
\pgfpathlineto{\pgfqpoint{5.087428in}{3.544100in}}%
\pgfpathlineto{\pgfqpoint{5.088091in}{3.664992in}}%
\pgfpathlineto{\pgfqpoint{5.088565in}{3.579525in}}%
\pgfpathlineto{\pgfqpoint{5.088944in}{3.505303in}}%
\pgfpathlineto{\pgfqpoint{5.089891in}{3.511052in}}%
\pgfpathlineto{\pgfqpoint{5.090270in}{3.455307in}}%
\pgfpathlineto{\pgfqpoint{5.090649in}{3.530257in}}%
\pgfpathlineto{\pgfqpoint{5.091123in}{3.629860in}}%
\pgfpathlineto{\pgfqpoint{5.091692in}{3.543377in}}%
\pgfpathlineto{\pgfqpoint{5.091976in}{3.486297in}}%
\pgfpathlineto{\pgfqpoint{5.092734in}{3.555663in}}%
\pgfpathlineto{\pgfqpoint{5.094251in}{3.687138in}}%
\pgfpathlineto{\pgfqpoint{5.093303in}{3.546651in}}%
\pgfpathlineto{\pgfqpoint{5.094535in}{3.646733in}}%
\pgfpathlineto{\pgfqpoint{5.095482in}{3.482101in}}%
\pgfpathlineto{\pgfqpoint{5.095956in}{3.520745in}}%
\pgfpathlineto{\pgfqpoint{5.096525in}{3.467880in}}%
\pgfpathlineto{\pgfqpoint{5.097093in}{3.516407in}}%
\pgfpathlineto{\pgfqpoint{5.097378in}{3.532356in}}%
\pgfpathlineto{\pgfqpoint{5.097757in}{3.473247in}}%
\pgfpathlineto{\pgfqpoint{5.098325in}{3.354690in}}%
\pgfpathlineto{\pgfqpoint{5.098894in}{3.442420in}}%
\pgfpathlineto{\pgfqpoint{5.099747in}{3.434474in}}%
\pgfpathlineto{\pgfqpoint{5.100410in}{3.521495in}}%
\pgfpathlineto{\pgfqpoint{5.100600in}{3.513548in}}%
\pgfpathlineto{\pgfqpoint{5.101453in}{3.330757in}}%
\pgfpathlineto{\pgfqpoint{5.102021in}{3.455668in}}%
\pgfpathlineto{\pgfqpoint{5.102305in}{3.480647in}}%
\pgfpathlineto{\pgfqpoint{5.102874in}{3.427864in}}%
\pgfpathlineto{\pgfqpoint{5.103158in}{3.405018in}}%
\pgfpathlineto{\pgfqpoint{5.103632in}{3.470269in}}%
\pgfpathlineto{\pgfqpoint{5.105243in}{3.877805in}}%
\pgfpathlineto{\pgfqpoint{5.105907in}{3.742023in}}%
\pgfpathlineto{\pgfqpoint{5.106759in}{3.327841in}}%
\pgfpathlineto{\pgfqpoint{5.107707in}{3.383734in}}%
\pgfpathlineto{\pgfqpoint{5.108655in}{3.572921in}}%
\pgfpathlineto{\pgfqpoint{5.109887in}{3.523996in}}%
\pgfpathlineto{\pgfqpoint{5.110076in}{3.514999in}}%
\pgfpathlineto{\pgfqpoint{5.110739in}{3.412486in}}%
\pgfpathlineto{\pgfqpoint{5.111213in}{3.480566in}}%
\pgfpathlineto{\pgfqpoint{5.111592in}{3.570120in}}%
\pgfpathlineto{\pgfqpoint{5.112161in}{3.453890in}}%
\pgfpathlineto{\pgfqpoint{5.113677in}{3.369938in}}%
\pgfpathlineto{\pgfqpoint{5.114246in}{3.473040in}}%
\pgfpathlineto{\pgfqpoint{5.114909in}{3.630527in}}%
\pgfpathlineto{\pgfqpoint{5.115478in}{3.535790in}}%
\pgfpathlineto{\pgfqpoint{5.115762in}{3.497193in}}%
\pgfpathlineto{\pgfqpoint{5.116236in}{3.556071in}}%
\pgfpathlineto{\pgfqpoint{5.116615in}{3.528742in}}%
\pgfpathlineto{\pgfqpoint{5.116994in}{3.493199in}}%
\pgfpathlineto{\pgfqpoint{5.117278in}{3.536868in}}%
\pgfpathlineto{\pgfqpoint{5.117942in}{3.660914in}}%
\pgfpathlineto{\pgfqpoint{5.118415in}{3.575940in}}%
\pgfpathlineto{\pgfqpoint{5.118889in}{3.514321in}}%
\pgfpathlineto{\pgfqpoint{5.119553in}{3.566959in}}%
\pgfpathlineto{\pgfqpoint{5.119932in}{3.546661in}}%
\pgfpathlineto{\pgfqpoint{5.120216in}{3.528420in}}%
\pgfpathlineto{\pgfqpoint{5.120500in}{3.587202in}}%
\pgfpathlineto{\pgfqpoint{5.121069in}{3.683038in}}%
\pgfpathlineto{\pgfqpoint{5.121543in}{3.576606in}}%
\pgfpathlineto{\pgfqpoint{5.121827in}{3.515147in}}%
\pgfpathlineto{\pgfqpoint{5.122585in}{3.585076in}}%
\pgfpathlineto{\pgfqpoint{5.124196in}{3.695169in}}%
\pgfpathlineto{\pgfqpoint{5.123248in}{3.563716in}}%
\pgfpathlineto{\pgfqpoint{5.124386in}{3.656529in}}%
\pgfpathlineto{\pgfqpoint{5.125049in}{3.501538in}}%
\pgfpathlineto{\pgfqpoint{5.125617in}{3.600028in}}%
\pgfpathlineto{\pgfqpoint{5.127228in}{3.676194in}}%
\pgfpathlineto{\pgfqpoint{5.126470in}{3.587476in}}%
\pgfpathlineto{\pgfqpoint{5.127323in}{3.673768in}}%
\pgfpathlineto{\pgfqpoint{5.128176in}{3.473803in}}%
\pgfpathlineto{\pgfqpoint{5.128934in}{3.599235in}}%
\pgfpathlineto{\pgfqpoint{5.129218in}{3.584689in}}%
\pgfpathlineto{\pgfqpoint{5.129408in}{3.576853in}}%
\pgfpathlineto{\pgfqpoint{5.130166in}{3.593268in}}%
\pgfpathlineto{\pgfqpoint{5.130261in}{3.595220in}}%
\pgfpathlineto{\pgfqpoint{5.130450in}{3.582698in}}%
\pgfpathlineto{\pgfqpoint{5.131303in}{3.414345in}}%
\pgfpathlineto{\pgfqpoint{5.131872in}{3.531204in}}%
\pgfpathlineto{\pgfqpoint{5.132156in}{3.569136in}}%
\pgfpathlineto{\pgfqpoint{5.132820in}{3.512348in}}%
\pgfpathlineto{\pgfqpoint{5.133388in}{3.549731in}}%
\pgfpathlineto{\pgfqpoint{5.133672in}{3.519783in}}%
\pgfpathlineto{\pgfqpoint{5.134431in}{3.384998in}}%
\pgfpathlineto{\pgfqpoint{5.134810in}{3.460878in}}%
\pgfpathlineto{\pgfqpoint{5.135189in}{3.558610in}}%
\pgfpathlineto{\pgfqpoint{5.135947in}{3.483101in}}%
\pgfpathlineto{\pgfqpoint{5.137368in}{3.385435in}}%
\pgfpathlineto{\pgfqpoint{5.136705in}{3.499407in}}%
\pgfpathlineto{\pgfqpoint{5.137652in}{3.418475in}}%
\pgfpathlineto{\pgfqpoint{5.138505in}{3.596023in}}%
\pgfpathlineto{\pgfqpoint{5.139169in}{3.514762in}}%
\pgfpathlineto{\pgfqpoint{5.139548in}{3.521258in}}%
\pgfpathlineto{\pgfqpoint{5.139927in}{3.539759in}}%
\pgfpathlineto{\pgfqpoint{5.140116in}{3.515292in}}%
\pgfpathlineto{\pgfqpoint{5.140590in}{3.427853in}}%
\pgfpathlineto{\pgfqpoint{5.141064in}{3.531022in}}%
\pgfpathlineto{\pgfqpoint{5.141538in}{3.592929in}}%
\pgfpathlineto{\pgfqpoint{5.142012in}{3.501094in}}%
\pgfpathlineto{\pgfqpoint{5.143433in}{3.410873in}}%
\pgfpathlineto{\pgfqpoint{5.143717in}{3.379964in}}%
\pgfpathlineto{\pgfqpoint{5.144096in}{3.436464in}}%
\pgfpathlineto{\pgfqpoint{5.144570in}{3.557563in}}%
\pgfpathlineto{\pgfqpoint{5.145139in}{3.465246in}}%
\pgfpathlineto{\pgfqpoint{5.145518in}{3.404740in}}%
\pgfpathlineto{\pgfqpoint{5.146371in}{3.431611in}}%
\pgfpathlineto{\pgfqpoint{5.146750in}{3.390174in}}%
\pgfpathlineto{\pgfqpoint{5.147129in}{3.437067in}}%
\pgfpathlineto{\pgfqpoint{5.147792in}{3.562912in}}%
\pgfpathlineto{\pgfqpoint{5.148266in}{3.451173in}}%
\pgfpathlineto{\pgfqpoint{5.148740in}{3.352228in}}%
\pgfpathlineto{\pgfqpoint{5.149308in}{3.422909in}}%
\pgfpathlineto{\pgfqpoint{5.150256in}{3.820352in}}%
\pgfpathlineto{\pgfqpoint{5.150730in}{3.978717in}}%
\pgfpathlineto{\pgfqpoint{5.151299in}{3.832257in}}%
\pgfpathlineto{\pgfqpoint{5.152530in}{3.316598in}}%
\pgfpathlineto{\pgfqpoint{5.153194in}{3.478348in}}%
\pgfpathlineto{\pgfqpoint{5.153952in}{3.595764in}}%
\pgfpathlineto{\pgfqpoint{5.154426in}{3.494207in}}%
\pgfpathlineto{\pgfqpoint{5.154994in}{3.409072in}}%
\pgfpathlineto{\pgfqpoint{5.155468in}{3.482124in}}%
\pgfpathlineto{\pgfqpoint{5.156890in}{3.567880in}}%
\pgfpathlineto{\pgfqpoint{5.157174in}{3.588960in}}%
\pgfpathlineto{\pgfqpoint{5.157458in}{3.539217in}}%
\pgfpathlineto{\pgfqpoint{5.158027in}{3.401035in}}%
\pgfpathlineto{\pgfqpoint{5.158501in}{3.513448in}}%
\pgfpathlineto{\pgfqpoint{5.159069in}{3.563634in}}%
\pgfpathlineto{\pgfqpoint{5.159638in}{3.523831in}}%
\pgfpathlineto{\pgfqpoint{5.160112in}{3.575986in}}%
\pgfpathlineto{\pgfqpoint{5.160585in}{3.528177in}}%
\pgfpathlineto{\pgfqpoint{5.161059in}{3.422047in}}%
\pgfpathlineto{\pgfqpoint{5.161628in}{3.542712in}}%
\pgfpathlineto{\pgfqpoint{5.162196in}{3.619752in}}%
\pgfpathlineto{\pgfqpoint{5.162670in}{3.542659in}}%
\pgfpathlineto{\pgfqpoint{5.162765in}{3.533192in}}%
\pgfpathlineto{\pgfqpoint{5.163239in}{3.588319in}}%
\pgfpathlineto{\pgfqpoint{5.163334in}{3.587830in}}%
\pgfpathlineto{\pgfqpoint{5.164186in}{3.476106in}}%
\pgfpathlineto{\pgfqpoint{5.164660in}{3.552398in}}%
\pgfpathlineto{\pgfqpoint{5.165229in}{3.671644in}}%
\pgfpathlineto{\pgfqpoint{5.165797in}{3.571429in}}%
\pgfpathlineto{\pgfqpoint{5.167314in}{3.507213in}}%
\pgfpathlineto{\pgfqpoint{5.166556in}{3.589934in}}%
\pgfpathlineto{\pgfqpoint{5.167408in}{3.514138in}}%
\pgfpathlineto{\pgfqpoint{5.168356in}{3.688167in}}%
\pgfpathlineto{\pgfqpoint{5.168925in}{3.581311in}}%
\pgfpathlineto{\pgfqpoint{5.170441in}{3.511085in}}%
\pgfpathlineto{\pgfqpoint{5.170536in}{3.510312in}}%
\pgfpathlineto{\pgfqpoint{5.171389in}{3.691331in}}%
\pgfpathlineto{\pgfqpoint{5.172052in}{3.566096in}}%
\pgfpathlineto{\pgfqpoint{5.173284in}{3.531748in}}%
\pgfpathlineto{\pgfqpoint{5.173568in}{3.517793in}}%
\pgfpathlineto{\pgfqpoint{5.173852in}{3.542506in}}%
\pgfpathlineto{\pgfqpoint{5.174421in}{3.660096in}}%
\pgfpathlineto{\pgfqpoint{5.174895in}{3.570367in}}%
\pgfpathlineto{\pgfqpoint{5.175463in}{3.469102in}}%
\pgfpathlineto{\pgfqpoint{5.176127in}{3.510631in}}%
\pgfpathlineto{\pgfqpoint{5.176316in}{3.516094in}}%
\pgfpathlineto{\pgfqpoint{5.176790in}{3.496131in}}%
\pgfpathlineto{\pgfqpoint{5.176885in}{3.496718in}}%
\pgfpathlineto{\pgfqpoint{5.177548in}{3.640021in}}%
\pgfpathlineto{\pgfqpoint{5.178117in}{3.510261in}}%
\pgfpathlineto{\pgfqpoint{5.178591in}{3.408799in}}%
\pgfpathlineto{\pgfqpoint{5.179254in}{3.498780in}}%
\pgfpathlineto{\pgfqpoint{5.179728in}{3.498464in}}%
\pgfpathlineto{\pgfqpoint{5.180202in}{3.537941in}}%
\pgfpathlineto{\pgfqpoint{5.180770in}{3.591503in}}%
\pgfpathlineto{\pgfqpoint{5.181054in}{3.533717in}}%
\pgfpathlineto{\pgfqpoint{5.181718in}{3.400680in}}%
\pgfpathlineto{\pgfqpoint{5.182192in}{3.492364in}}%
\pgfpathlineto{\pgfqpoint{5.182381in}{3.512320in}}%
\pgfpathlineto{\pgfqpoint{5.183234in}{3.483741in}}%
\pgfpathlineto{\pgfqpoint{5.183708in}{3.530118in}}%
\pgfpathlineto{\pgfqpoint{5.184087in}{3.472860in}}%
\pgfpathlineto{\pgfqpoint{5.184750in}{3.326127in}}%
\pgfpathlineto{\pgfqpoint{5.185224in}{3.421154in}}%
\pgfpathlineto{\pgfqpoint{5.186835in}{3.574675in}}%
\pgfpathlineto{\pgfqpoint{5.186930in}{3.574801in}}%
\pgfpathlineto{\pgfqpoint{5.187404in}{3.508059in}}%
\pgfpathlineto{\pgfqpoint{5.187972in}{3.399598in}}%
\pgfpathlineto{\pgfqpoint{5.188446in}{3.492075in}}%
\pgfpathlineto{\pgfqpoint{5.188730in}{3.536026in}}%
\pgfpathlineto{\pgfqpoint{5.189394in}{3.466664in}}%
\pgfpathlineto{\pgfqpoint{5.190910in}{3.355808in}}%
\pgfpathlineto{\pgfqpoint{5.191289in}{3.418965in}}%
\pgfpathlineto{\pgfqpoint{5.191952in}{3.562022in}}%
\pgfpathlineto{\pgfqpoint{5.192521in}{3.485342in}}%
\pgfpathlineto{\pgfqpoint{5.194132in}{3.387507in}}%
\pgfpathlineto{\pgfqpoint{5.193089in}{3.499036in}}%
\pgfpathlineto{\pgfqpoint{5.194227in}{3.397804in}}%
\pgfpathlineto{\pgfqpoint{5.196406in}{3.925145in}}%
\pgfpathlineto{\pgfqpoint{5.197164in}{3.818572in}}%
\pgfpathlineto{\pgfqpoint{5.197922in}{3.668250in}}%
\pgfpathlineto{\pgfqpoint{5.198396in}{3.458496in}}%
\pgfpathlineto{\pgfqpoint{5.199249in}{3.534365in}}%
\pgfpathlineto{\pgfqpoint{5.201239in}{3.694682in}}%
\pgfpathlineto{\pgfqpoint{5.201618in}{3.638992in}}%
\pgfpathlineto{\pgfqpoint{5.202187in}{3.526646in}}%
\pgfpathlineto{\pgfqpoint{5.202755in}{3.617395in}}%
\pgfpathlineto{\pgfqpoint{5.203040in}{3.615036in}}%
\pgfpathlineto{\pgfqpoint{5.203893in}{3.674022in}}%
\pgfpathlineto{\pgfqpoint{5.204272in}{3.737815in}}%
\pgfpathlineto{\pgfqpoint{5.204840in}{3.636941in}}%
\pgfpathlineto{\pgfqpoint{5.205314in}{3.556312in}}%
\pgfpathlineto{\pgfqpoint{5.205883in}{3.650267in}}%
\pgfpathlineto{\pgfqpoint{5.207399in}{3.778291in}}%
\pgfpathlineto{\pgfqpoint{5.207588in}{3.756686in}}%
\pgfpathlineto{\pgfqpoint{5.208536in}{3.588223in}}%
\pgfpathlineto{\pgfqpoint{5.208915in}{3.671722in}}%
\pgfpathlineto{\pgfqpoint{5.210337in}{3.771520in}}%
\pgfpathlineto{\pgfqpoint{5.210621in}{3.800404in}}%
\pgfpathlineto{\pgfqpoint{5.211000in}{3.714966in}}%
\pgfpathlineto{\pgfqpoint{5.211474in}{3.630281in}}%
\pgfpathlineto{\pgfqpoint{5.212042in}{3.726525in}}%
\pgfpathlineto{\pgfqpoint{5.213559in}{3.795696in}}%
\pgfpathlineto{\pgfqpoint{5.213653in}{3.799703in}}%
\pgfpathlineto{\pgfqpoint{5.213938in}{3.771023in}}%
\pgfpathlineto{\pgfqpoint{5.214506in}{3.627297in}}%
\pgfpathlineto{\pgfqpoint{5.215075in}{3.722470in}}%
\pgfpathlineto{\pgfqpoint{5.215643in}{3.816577in}}%
\pgfpathlineto{\pgfqpoint{5.216307in}{3.753923in}}%
\pgfpathlineto{\pgfqpoint{5.216686in}{3.767165in}}%
\pgfpathlineto{\pgfqpoint{5.216970in}{3.751031in}}%
\pgfpathlineto{\pgfqpoint{5.217633in}{3.610584in}}%
\pgfpathlineto{\pgfqpoint{5.218202in}{3.720758in}}%
\pgfpathlineto{\pgfqpoint{5.218676in}{3.807419in}}%
\pgfpathlineto{\pgfqpoint{5.219244in}{3.717481in}}%
\pgfpathlineto{\pgfqpoint{5.220855in}{3.562141in}}%
\pgfpathlineto{\pgfqpoint{5.221140in}{3.612547in}}%
\pgfpathlineto{\pgfqpoint{5.221708in}{3.732055in}}%
\pgfpathlineto{\pgfqpoint{5.222277in}{3.639159in}}%
\pgfpathlineto{\pgfqpoint{5.223983in}{3.528556in}}%
\pgfpathlineto{\pgfqpoint{5.224267in}{3.571560in}}%
\pgfpathlineto{\pgfqpoint{5.224835in}{3.708793in}}%
\pgfpathlineto{\pgfqpoint{5.225404in}{3.601681in}}%
\pgfpathlineto{\pgfqpoint{5.227015in}{3.539662in}}%
\pgfpathlineto{\pgfqpoint{5.227110in}{3.536547in}}%
\pgfpathlineto{\pgfqpoint{5.227205in}{3.547459in}}%
\pgfpathlineto{\pgfqpoint{5.227963in}{3.707884in}}%
\pgfpathlineto{\pgfqpoint{5.228437in}{3.606304in}}%
\pgfpathlineto{\pgfqpoint{5.228910in}{3.559152in}}%
\pgfpathlineto{\pgfqpoint{5.229479in}{3.597284in}}%
\pgfpathlineto{\pgfqpoint{5.230521in}{3.678026in}}%
\pgfpathlineto{\pgfqpoint{5.230995in}{3.757863in}}%
\pgfpathlineto{\pgfqpoint{5.231564in}{3.666860in}}%
\pgfpathlineto{\pgfqpoint{5.231943in}{3.576069in}}%
\pgfpathlineto{\pgfqpoint{5.232701in}{3.635098in}}%
\pgfpathlineto{\pgfqpoint{5.232796in}{3.638642in}}%
\pgfpathlineto{\pgfqpoint{5.233080in}{3.615921in}}%
\pgfpathlineto{\pgfqpoint{5.233364in}{3.593472in}}%
\pgfpathlineto{\pgfqpoint{5.233838in}{3.638872in}}%
\pgfpathlineto{\pgfqpoint{5.234122in}{3.660915in}}%
\pgfpathlineto{\pgfqpoint{5.234501in}{3.605258in}}%
\pgfpathlineto{\pgfqpoint{5.235070in}{3.474777in}}%
\pgfpathlineto{\pgfqpoint{5.235733in}{3.562767in}}%
\pgfpathlineto{\pgfqpoint{5.237155in}{3.640063in}}%
\pgfpathlineto{\pgfqpoint{5.236491in}{3.562515in}}%
\pgfpathlineto{\pgfqpoint{5.237250in}{3.632542in}}%
\pgfpathlineto{\pgfqpoint{5.238197in}{3.442134in}}%
\pgfpathlineto{\pgfqpoint{5.238766in}{3.562729in}}%
\pgfpathlineto{\pgfqpoint{5.239240in}{3.618294in}}%
\pgfpathlineto{\pgfqpoint{5.240092in}{3.617130in}}%
\pgfpathlineto{\pgfqpoint{5.240187in}{3.624482in}}%
\pgfpathlineto{\pgfqpoint{5.240472in}{3.592279in}}%
\pgfpathlineto{\pgfqpoint{5.240945in}{3.487850in}}%
\pgfpathlineto{\pgfqpoint{5.241324in}{3.578836in}}%
\pgfpathlineto{\pgfqpoint{5.242272in}{4.042663in}}%
\pgfpathlineto{\pgfqpoint{5.242935in}{3.929918in}}%
\pgfpathlineto{\pgfqpoint{5.243599in}{3.667546in}}%
\pgfpathlineto{\pgfqpoint{5.244167in}{3.348510in}}%
\pgfpathlineto{\pgfqpoint{5.244736in}{3.540807in}}%
\pgfpathlineto{\pgfqpoint{5.245399in}{3.668224in}}%
\pgfpathlineto{\pgfqpoint{5.245968in}{3.593526in}}%
\pgfpathlineto{\pgfqpoint{5.246442in}{3.606135in}}%
\pgfpathlineto{\pgfqpoint{5.247105in}{3.548672in}}%
\pgfpathlineto{\pgfqpoint{5.247484in}{3.509924in}}%
\pgfpathlineto{\pgfqpoint{5.247958in}{3.566823in}}%
\pgfpathlineto{\pgfqpoint{5.248527in}{3.701331in}}%
\pgfpathlineto{\pgfqpoint{5.249095in}{3.587003in}}%
\pgfpathlineto{\pgfqpoint{5.249190in}{3.586664in}}%
\pgfpathlineto{\pgfqpoint{5.249285in}{3.589177in}}%
\pgfpathlineto{\pgfqpoint{5.249379in}{3.590716in}}%
\pgfpathlineto{\pgfqpoint{5.249664in}{3.580272in}}%
\pgfpathlineto{\pgfqpoint{5.250517in}{3.526178in}}%
\pgfpathlineto{\pgfqpoint{5.250896in}{3.566042in}}%
\pgfpathlineto{\pgfqpoint{5.251654in}{3.738699in}}%
\pgfpathlineto{\pgfqpoint{5.252317in}{3.621873in}}%
\pgfpathlineto{\pgfqpoint{5.253075in}{3.633228in}}%
\pgfpathlineto{\pgfqpoint{5.253833in}{3.575348in}}%
\pgfpathlineto{\pgfqpoint{5.254118in}{3.605358in}}%
\pgfpathlineto{\pgfqpoint{5.254686in}{3.714559in}}%
\pgfpathlineto{\pgfqpoint{5.255160in}{3.595281in}}%
\pgfpathlineto{\pgfqpoint{5.255823in}{3.507468in}}%
\pgfpathlineto{\pgfqpoint{5.256392in}{3.547744in}}%
\pgfpathlineto{\pgfqpoint{5.256581in}{3.543913in}}%
\pgfpathlineto{\pgfqpoint{5.256771in}{3.559864in}}%
\pgfpathlineto{\pgfqpoint{5.257813in}{3.809833in}}%
\pgfpathlineto{\pgfqpoint{5.258382in}{3.691321in}}%
\pgfpathlineto{\pgfqpoint{5.258761in}{3.614124in}}%
\pgfpathlineto{\pgfqpoint{5.259519in}{3.665670in}}%
\pgfpathlineto{\pgfqpoint{5.259709in}{3.651282in}}%
\pgfpathlineto{\pgfqpoint{5.260277in}{3.698667in}}%
\pgfpathlineto{\pgfqpoint{5.260846in}{3.789928in}}%
\pgfpathlineto{\pgfqpoint{5.261320in}{3.708428in}}%
\pgfpathlineto{\pgfqpoint{5.261793in}{3.574975in}}%
\pgfpathlineto{\pgfqpoint{5.262552in}{3.667838in}}%
\pgfpathlineto{\pgfqpoint{5.263973in}{3.755688in}}%
\pgfpathlineto{\pgfqpoint{5.264163in}{3.735133in}}%
\pgfpathlineto{\pgfqpoint{5.265110in}{3.537644in}}%
\pgfpathlineto{\pgfqpoint{5.265584in}{3.625299in}}%
\pgfpathlineto{\pgfqpoint{5.267006in}{3.661515in}}%
\pgfpathlineto{\pgfqpoint{5.266532in}{3.617722in}}%
\pgfpathlineto{\pgfqpoint{5.267100in}{3.658441in}}%
\pgfpathlineto{\pgfqpoint{5.267764in}{3.494112in}}%
\pgfpathlineto{\pgfqpoint{5.268048in}{3.455355in}}%
\pgfpathlineto{\pgfqpoint{5.268616in}{3.539933in}}%
\pgfpathlineto{\pgfqpoint{5.268901in}{3.596359in}}%
\pgfpathlineto{\pgfqpoint{5.269754in}{3.556771in}}%
\pgfpathlineto{\pgfqpoint{5.270227in}{3.568506in}}%
\pgfpathlineto{\pgfqpoint{5.270417in}{3.558819in}}%
\pgfpathlineto{\pgfqpoint{5.271175in}{3.407671in}}%
\pgfpathlineto{\pgfqpoint{5.271649in}{3.496773in}}%
\pgfpathlineto{\pgfqpoint{5.272028in}{3.581153in}}%
\pgfpathlineto{\pgfqpoint{5.272786in}{3.512405in}}%
\pgfpathlineto{\pgfqpoint{5.274208in}{3.386554in}}%
\pgfpathlineto{\pgfqpoint{5.274492in}{3.426477in}}%
\pgfpathlineto{\pgfqpoint{5.275250in}{3.606209in}}%
\pgfpathlineto{\pgfqpoint{5.275913in}{3.529324in}}%
\pgfpathlineto{\pgfqpoint{5.276292in}{3.543541in}}%
\pgfpathlineto{\pgfqpoint{5.276577in}{3.512041in}}%
\pgfpathlineto{\pgfqpoint{5.277430in}{3.407251in}}%
\pgfpathlineto{\pgfqpoint{5.277809in}{3.479466in}}%
\pgfpathlineto{\pgfqpoint{5.278282in}{3.575688in}}%
\pgfpathlineto{\pgfqpoint{5.278756in}{3.463416in}}%
\pgfpathlineto{\pgfqpoint{5.280367in}{3.313663in}}%
\pgfpathlineto{\pgfqpoint{5.280462in}{3.307803in}}%
\pgfpathlineto{\pgfqpoint{5.280746in}{3.339516in}}%
\pgfpathlineto{\pgfqpoint{5.281504in}{3.477902in}}%
\pgfpathlineto{\pgfqpoint{5.281978in}{3.371757in}}%
\pgfpathlineto{\pgfqpoint{5.282547in}{3.312840in}}%
\pgfpathlineto{\pgfqpoint{5.283210in}{3.350087in}}%
\pgfpathlineto{\pgfqpoint{5.283589in}{3.334975in}}%
\pgfpathlineto{\pgfqpoint{5.283779in}{3.345820in}}%
\pgfpathlineto{\pgfqpoint{5.284537in}{3.483847in}}%
\pgfpathlineto{\pgfqpoint{5.285011in}{3.413807in}}%
\pgfpathlineto{\pgfqpoint{5.285485in}{3.318532in}}%
\pgfpathlineto{\pgfqpoint{5.286243in}{3.348233in}}%
\pgfpathlineto{\pgfqpoint{5.287664in}{3.933905in}}%
\pgfpathlineto{\pgfqpoint{5.288517in}{3.680148in}}%
\pgfpathlineto{\pgfqpoint{5.289559in}{3.298877in}}%
\pgfpathlineto{\pgfqpoint{5.290128in}{3.486636in}}%
\pgfpathlineto{\pgfqpoint{5.290791in}{3.558034in}}%
\pgfpathlineto{\pgfqpoint{5.291076in}{3.504116in}}%
\pgfpathlineto{\pgfqpoint{5.291739in}{3.364277in}}%
\pgfpathlineto{\pgfqpoint{5.292308in}{3.446847in}}%
\pgfpathlineto{\pgfqpoint{5.293729in}{3.559380in}}%
\pgfpathlineto{\pgfqpoint{5.294013in}{3.543338in}}%
\pgfpathlineto{\pgfqpoint{5.294677in}{3.382781in}}%
\pgfpathlineto{\pgfqpoint{5.295340in}{3.494599in}}%
\pgfpathlineto{\pgfqpoint{5.295909in}{3.556686in}}%
\pgfpathlineto{\pgfqpoint{5.296477in}{3.505408in}}%
\pgfpathlineto{\pgfqpoint{5.296951in}{3.537417in}}%
\pgfpathlineto{\pgfqpoint{5.297235in}{3.514476in}}%
\pgfpathlineto{\pgfqpoint{5.297899in}{3.384814in}}%
\pgfpathlineto{\pgfqpoint{5.298467in}{3.486184in}}%
\pgfpathlineto{\pgfqpoint{5.298941in}{3.584919in}}%
\pgfpathlineto{\pgfqpoint{5.299604in}{3.519361in}}%
\pgfpathlineto{\pgfqpoint{5.299699in}{3.517614in}}%
\pgfpathlineto{\pgfqpoint{5.299983in}{3.530344in}}%
\pgfpathlineto{\pgfqpoint{5.300078in}{3.532255in}}%
\pgfpathlineto{\pgfqpoint{5.300268in}{3.520705in}}%
\pgfpathlineto{\pgfqpoint{5.301026in}{3.426292in}}%
\pgfpathlineto{\pgfqpoint{5.301405in}{3.494363in}}%
\pgfpathlineto{\pgfqpoint{5.301879in}{3.592840in}}%
\pgfpathlineto{\pgfqpoint{5.302637in}{3.542516in}}%
\pgfpathlineto{\pgfqpoint{5.304058in}{3.443316in}}%
\pgfpathlineto{\pgfqpoint{5.304343in}{3.477634in}}%
\pgfpathlineto{\pgfqpoint{5.305101in}{3.653312in}}%
\pgfpathlineto{\pgfqpoint{5.305574in}{3.553796in}}%
\pgfpathlineto{\pgfqpoint{5.306996in}{3.494020in}}%
\pgfpathlineto{\pgfqpoint{5.307185in}{3.489027in}}%
\pgfpathlineto{\pgfqpoint{5.307470in}{3.504338in}}%
\pgfpathlineto{\pgfqpoint{5.308133in}{3.664000in}}%
\pgfpathlineto{\pgfqpoint{5.308702in}{3.559167in}}%
\pgfpathlineto{\pgfqpoint{5.310028in}{3.489831in}}%
\pgfpathlineto{\pgfqpoint{5.310407in}{3.472603in}}%
\pgfpathlineto{\pgfqpoint{5.310692in}{3.502679in}}%
\pgfpathlineto{\pgfqpoint{5.311260in}{3.612624in}}%
\pgfpathlineto{\pgfqpoint{5.311734in}{3.522814in}}%
\pgfpathlineto{\pgfqpoint{5.312303in}{3.415185in}}%
\pgfpathlineto{\pgfqpoint{5.313061in}{3.451577in}}%
\pgfpathlineto{\pgfqpoint{5.313440in}{3.446940in}}%
\pgfpathlineto{\pgfqpoint{5.313629in}{3.453945in}}%
\pgfpathlineto{\pgfqpoint{5.314293in}{3.582070in}}%
\pgfpathlineto{\pgfqpoint{5.314767in}{3.487952in}}%
\pgfpathlineto{\pgfqpoint{5.315335in}{3.366006in}}%
\pgfpathlineto{\pgfqpoint{5.315999in}{3.432878in}}%
\pgfpathlineto{\pgfqpoint{5.317420in}{3.550784in}}%
\pgfpathlineto{\pgfqpoint{5.317799in}{3.511047in}}%
\pgfpathlineto{\pgfqpoint{5.318462in}{3.357273in}}%
\pgfpathlineto{\pgfqpoint{5.319031in}{3.450544in}}%
\pgfpathlineto{\pgfqpoint{5.320642in}{3.584746in}}%
\pgfpathlineto{\pgfqpoint{5.321021in}{3.514548in}}%
\pgfpathlineto{\pgfqpoint{5.321495in}{3.448829in}}%
\pgfpathlineto{\pgfqpoint{5.322063in}{3.525680in}}%
\pgfpathlineto{\pgfqpoint{5.322443in}{3.588008in}}%
\pgfpathlineto{\pgfqpoint{5.323295in}{3.570204in}}%
\pgfpathlineto{\pgfqpoint{5.323674in}{3.588125in}}%
\pgfpathlineto{\pgfqpoint{5.323864in}{3.569816in}}%
\pgfpathlineto{\pgfqpoint{5.324717in}{3.377471in}}%
\pgfpathlineto{\pgfqpoint{5.325191in}{3.489081in}}%
\pgfpathlineto{\pgfqpoint{5.325570in}{3.537865in}}%
\pgfpathlineto{\pgfqpoint{5.326233in}{3.481087in}}%
\pgfpathlineto{\pgfqpoint{5.327749in}{3.278659in}}%
\pgfpathlineto{\pgfqpoint{5.328318in}{3.380547in}}%
\pgfpathlineto{\pgfqpoint{5.328886in}{3.474851in}}%
\pgfpathlineto{\pgfqpoint{5.329834in}{3.470337in}}%
\pgfpathlineto{\pgfqpoint{5.330118in}{3.481697in}}%
\pgfpathlineto{\pgfqpoint{5.330497in}{3.438483in}}%
\pgfpathlineto{\pgfqpoint{5.330782in}{3.413505in}}%
\pgfpathlineto{\pgfqpoint{5.331350in}{3.456764in}}%
\pgfpathlineto{\pgfqpoint{5.333056in}{3.892296in}}%
\pgfpathlineto{\pgfqpoint{5.333909in}{3.728693in}}%
\pgfpathlineto{\pgfqpoint{5.334762in}{3.450672in}}%
\pgfpathlineto{\pgfqpoint{5.335899in}{3.481092in}}%
\pgfpathlineto{\pgfqpoint{5.335994in}{3.480460in}}%
\pgfpathlineto{\pgfqpoint{5.336183in}{3.486494in}}%
\pgfpathlineto{\pgfqpoint{5.336562in}{3.503910in}}%
\pgfpathlineto{\pgfqpoint{5.336847in}{3.479025in}}%
\pgfpathlineto{\pgfqpoint{5.337036in}{3.463417in}}%
\pgfpathlineto{\pgfqpoint{5.337415in}{3.528667in}}%
\pgfpathlineto{\pgfqpoint{5.338079in}{3.668901in}}%
\pgfpathlineto{\pgfqpoint{5.338552in}{3.561046in}}%
\pgfpathlineto{\pgfqpoint{5.339026in}{3.476097in}}%
\pgfpathlineto{\pgfqpoint{5.339784in}{3.515777in}}%
\pgfpathlineto{\pgfqpoint{5.340637in}{3.594498in}}%
\pgfpathlineto{\pgfqpoint{5.341016in}{3.663877in}}%
\pgfpathlineto{\pgfqpoint{5.341585in}{3.579487in}}%
\pgfpathlineto{\pgfqpoint{5.342059in}{3.455704in}}%
\pgfpathlineto{\pgfqpoint{5.342722in}{3.539502in}}%
\pgfpathlineto{\pgfqpoint{5.344143in}{3.662076in}}%
\pgfpathlineto{\pgfqpoint{5.344617in}{3.589637in}}%
\pgfpathlineto{\pgfqpoint{5.345186in}{3.483859in}}%
\pgfpathlineto{\pgfqpoint{5.345754in}{3.561307in}}%
\pgfpathlineto{\pgfqpoint{5.347271in}{3.654189in}}%
\pgfpathlineto{\pgfqpoint{5.347365in}{3.651723in}}%
\pgfpathlineto{\pgfqpoint{5.348408in}{3.452808in}}%
\pgfpathlineto{\pgfqpoint{5.348976in}{3.595155in}}%
\pgfpathlineto{\pgfqpoint{5.350019in}{3.620212in}}%
\pgfpathlineto{\pgfqpoint{5.350208in}{3.614658in}}%
\pgfpathlineto{\pgfqpoint{5.350872in}{3.561614in}}%
\pgfpathlineto{\pgfqpoint{5.351251in}{3.444747in}}%
\pgfpathlineto{\pgfqpoint{5.351914in}{3.551367in}}%
\pgfpathlineto{\pgfqpoint{5.352293in}{3.632197in}}%
\pgfpathlineto{\pgfqpoint{5.353051in}{3.576414in}}%
\pgfpathlineto{\pgfqpoint{5.353525in}{3.580890in}}%
\pgfpathlineto{\pgfqpoint{5.353809in}{3.552600in}}%
\pgfpathlineto{\pgfqpoint{5.354473in}{3.432723in}}%
\pgfpathlineto{\pgfqpoint{5.354947in}{3.520995in}}%
\pgfpathlineto{\pgfqpoint{5.355420in}{3.602255in}}%
\pgfpathlineto{\pgfqpoint{5.356084in}{3.537301in}}%
\pgfpathlineto{\pgfqpoint{5.357505in}{3.391764in}}%
\pgfpathlineto{\pgfqpoint{5.356747in}{3.540371in}}%
\pgfpathlineto{\pgfqpoint{5.357790in}{3.418684in}}%
\pgfpathlineto{\pgfqpoint{5.358548in}{3.580955in}}%
\pgfpathlineto{\pgfqpoint{5.359116in}{3.505551in}}%
\pgfpathlineto{\pgfqpoint{5.360538in}{3.386084in}}%
\pgfpathlineto{\pgfqpoint{5.360822in}{3.400103in}}%
\pgfpathlineto{\pgfqpoint{5.361675in}{3.559707in}}%
\pgfpathlineto{\pgfqpoint{5.362243in}{3.475497in}}%
\pgfpathlineto{\pgfqpoint{5.363570in}{3.417778in}}%
\pgfpathlineto{\pgfqpoint{5.363665in}{3.414710in}}%
\pgfpathlineto{\pgfqpoint{5.363949in}{3.437388in}}%
\pgfpathlineto{\pgfqpoint{5.364802in}{3.591030in}}%
\pgfpathlineto{\pgfqpoint{5.365371in}{3.495004in}}%
\pgfpathlineto{\pgfqpoint{5.365844in}{3.439488in}}%
\pgfpathlineto{\pgfqpoint{5.366413in}{3.502876in}}%
\pgfpathlineto{\pgfqpoint{5.366887in}{3.490684in}}%
\pgfpathlineto{\pgfqpoint{5.367076in}{3.507258in}}%
\pgfpathlineto{\pgfqpoint{5.367740in}{3.653030in}}%
\pgfpathlineto{\pgfqpoint{5.368308in}{3.541586in}}%
\pgfpathlineto{\pgfqpoint{5.368782in}{3.456714in}}%
\pgfpathlineto{\pgfqpoint{5.369540in}{3.480124in}}%
\pgfpathlineto{\pgfqpoint{5.369730in}{3.487367in}}%
\pgfpathlineto{\pgfqpoint{5.370109in}{3.445030in}}%
\pgfpathlineto{\pgfqpoint{5.370204in}{3.442874in}}%
\pgfpathlineto{\pgfqpoint{5.370298in}{3.450644in}}%
\pgfpathlineto{\pgfqpoint{5.370867in}{3.530525in}}%
\pgfpathlineto{\pgfqpoint{5.371341in}{3.458245in}}%
\pgfpathlineto{\pgfqpoint{5.372004in}{3.329993in}}%
\pgfpathlineto{\pgfqpoint{5.372478in}{3.412127in}}%
\pgfpathlineto{\pgfqpoint{5.373994in}{3.518672in}}%
\pgfpathlineto{\pgfqpoint{5.374278in}{3.504627in}}%
\pgfpathlineto{\pgfqpoint{5.374942in}{3.328477in}}%
\pgfpathlineto{\pgfqpoint{5.375700in}{3.444702in}}%
\pgfpathlineto{\pgfqpoint{5.375984in}{3.473696in}}%
\pgfpathlineto{\pgfqpoint{5.376837in}{3.449069in}}%
\pgfpathlineto{\pgfqpoint{5.377690in}{3.569496in}}%
\pgfpathlineto{\pgfqpoint{5.378827in}{3.899034in}}%
\pgfpathlineto{\pgfqpoint{5.379396in}{3.808894in}}%
\pgfpathlineto{\pgfqpoint{5.380343in}{3.356096in}}%
\pgfpathlineto{\pgfqpoint{5.381575in}{3.429919in}}%
\pgfpathlineto{\pgfqpoint{5.382333in}{3.580951in}}%
\pgfpathlineto{\pgfqpoint{5.382807in}{3.512220in}}%
\pgfpathlineto{\pgfqpoint{5.384323in}{3.410184in}}%
\pgfpathlineto{\pgfqpoint{5.383281in}{3.512377in}}%
\pgfpathlineto{\pgfqpoint{5.384608in}{3.461026in}}%
\pgfpathlineto{\pgfqpoint{5.385271in}{3.613435in}}%
\pgfpathlineto{\pgfqpoint{5.385934in}{3.552014in}}%
\pgfpathlineto{\pgfqpoint{5.387545in}{3.431229in}}%
\pgfpathlineto{\pgfqpoint{5.387640in}{3.444012in}}%
\pgfpathlineto{\pgfqpoint{5.388493in}{3.609087in}}%
\pgfpathlineto{\pgfqpoint{5.388967in}{3.535791in}}%
\pgfpathlineto{\pgfqpoint{5.389630in}{3.499726in}}%
\pgfpathlineto{\pgfqpoint{5.390009in}{3.541628in}}%
\pgfpathlineto{\pgfqpoint{5.390199in}{3.529444in}}%
\pgfpathlineto{\pgfqpoint{5.390483in}{3.503788in}}%
\pgfpathlineto{\pgfqpoint{5.390957in}{3.569060in}}%
\pgfpathlineto{\pgfqpoint{5.391526in}{3.684600in}}%
\pgfpathlineto{\pgfqpoint{5.391999in}{3.593091in}}%
\pgfpathlineto{\pgfqpoint{5.392378in}{3.518925in}}%
\pgfpathlineto{\pgfqpoint{5.393137in}{3.553272in}}%
\pgfpathlineto{\pgfqpoint{5.393610in}{3.542597in}}%
\pgfpathlineto{\pgfqpoint{5.394084in}{3.620030in}}%
\pgfpathlineto{\pgfqpoint{5.394463in}{3.681283in}}%
\pgfpathlineto{\pgfqpoint{5.395032in}{3.614722in}}%
\pgfpathlineto{\pgfqpoint{5.395506in}{3.492088in}}%
\pgfpathlineto{\pgfqpoint{5.396264in}{3.554418in}}%
\pgfpathlineto{\pgfqpoint{5.397685in}{3.713956in}}%
\pgfpathlineto{\pgfqpoint{5.398064in}{3.640035in}}%
\pgfpathlineto{\pgfqpoint{5.398633in}{3.510938in}}%
\pgfpathlineto{\pgfqpoint{5.399296in}{3.571434in}}%
\pgfpathlineto{\pgfqpoint{5.400718in}{3.620772in}}%
\pgfpathlineto{\pgfqpoint{5.400812in}{3.620565in}}%
\pgfpathlineto{\pgfqpoint{5.401191in}{3.538864in}}%
\pgfpathlineto{\pgfqpoint{5.401855in}{3.366981in}}%
\pgfpathlineto{\pgfqpoint{5.402423in}{3.480080in}}%
\pgfpathlineto{\pgfqpoint{5.403940in}{3.596994in}}%
\pgfpathlineto{\pgfqpoint{5.404129in}{3.582340in}}%
\pgfpathlineto{\pgfqpoint{5.404887in}{3.457976in}}%
\pgfpathlineto{\pgfqpoint{5.405361in}{3.533108in}}%
\pgfpathlineto{\pgfqpoint{5.405740in}{3.590231in}}%
\pgfpathlineto{\pgfqpoint{5.406593in}{3.576668in}}%
\pgfpathlineto{\pgfqpoint{5.406783in}{3.592036in}}%
\pgfpathlineto{\pgfqpoint{5.407256in}{3.543006in}}%
\pgfpathlineto{\pgfqpoint{5.407920in}{3.415328in}}%
\pgfpathlineto{\pgfqpoint{5.408394in}{3.514716in}}%
\pgfpathlineto{\pgfqpoint{5.408962in}{3.604644in}}%
\pgfpathlineto{\pgfqpoint{5.409625in}{3.546717in}}%
\pgfpathlineto{\pgfqpoint{5.410005in}{3.555981in}}%
\pgfpathlineto{\pgfqpoint{5.410668in}{3.497158in}}%
\pgfpathlineto{\pgfqpoint{5.411047in}{3.454729in}}%
\pgfpathlineto{\pgfqpoint{5.411426in}{3.529809in}}%
\pgfpathlineto{\pgfqpoint{5.412089in}{3.681011in}}%
\pgfpathlineto{\pgfqpoint{5.412658in}{3.612957in}}%
\pgfpathlineto{\pgfqpoint{5.413985in}{3.493670in}}%
\pgfpathlineto{\pgfqpoint{5.414269in}{3.502310in}}%
\pgfpathlineto{\pgfqpoint{5.415122in}{3.641397in}}%
\pgfpathlineto{\pgfqpoint{5.415690in}{3.539140in}}%
\pgfpathlineto{\pgfqpoint{5.417112in}{3.431573in}}%
\pgfpathlineto{\pgfqpoint{5.417301in}{3.423974in}}%
\pgfpathlineto{\pgfqpoint{5.417586in}{3.454518in}}%
\pgfpathlineto{\pgfqpoint{5.418249in}{3.603185in}}%
\pgfpathlineto{\pgfqpoint{5.418818in}{3.501801in}}%
\pgfpathlineto{\pgfqpoint{5.419197in}{3.440031in}}%
\pgfpathlineto{\pgfqpoint{5.420050in}{3.458764in}}%
\pgfpathlineto{\pgfqpoint{5.420239in}{3.453777in}}%
\pgfpathlineto{\pgfqpoint{5.420523in}{3.474239in}}%
\pgfpathlineto{\pgfqpoint{5.421281in}{3.622079in}}%
\pgfpathlineto{\pgfqpoint{5.421755in}{3.535204in}}%
\pgfpathlineto{\pgfqpoint{5.422324in}{3.367633in}}%
\pgfpathlineto{\pgfqpoint{5.422798in}{3.551552in}}%
\pgfpathlineto{\pgfqpoint{5.424124in}{3.991796in}}%
\pgfpathlineto{\pgfqpoint{5.424503in}{3.956000in}}%
\pgfpathlineto{\pgfqpoint{5.424977in}{3.704677in}}%
\pgfpathlineto{\pgfqpoint{5.425641in}{3.279604in}}%
\pgfpathlineto{\pgfqpoint{5.426209in}{3.509911in}}%
\pgfpathlineto{\pgfqpoint{5.427441in}{3.634724in}}%
\pgfpathlineto{\pgfqpoint{5.427725in}{3.592733in}}%
\pgfpathlineto{\pgfqpoint{5.428484in}{3.432291in}}%
\pgfpathlineto{\pgfqpoint{5.429147in}{3.521643in}}%
\pgfpathlineto{\pgfqpoint{5.430568in}{3.634332in}}%
\pgfpathlineto{\pgfqpoint{5.430758in}{3.624182in}}%
\pgfpathlineto{\pgfqpoint{5.431611in}{3.476031in}}%
\pgfpathlineto{\pgfqpoint{5.432085in}{3.548041in}}%
\pgfpathlineto{\pgfqpoint{5.433222in}{3.632891in}}%
\pgfpathlineto{\pgfqpoint{5.433411in}{3.626621in}}%
\pgfpathlineto{\pgfqpoint{5.433790in}{3.632709in}}%
\pgfpathlineto{\pgfqpoint{5.434075in}{3.586159in}}%
\pgfpathlineto{\pgfqpoint{5.434643in}{3.482516in}}%
\pgfpathlineto{\pgfqpoint{5.435117in}{3.557853in}}%
\pgfpathlineto{\pgfqpoint{5.435686in}{3.674869in}}%
\pgfpathlineto{\pgfqpoint{5.436349in}{3.636194in}}%
\pgfpathlineto{\pgfqpoint{5.436444in}{3.634830in}}%
\pgfpathlineto{\pgfqpoint{5.436633in}{3.646588in}}%
\pgfpathlineto{\pgfqpoint{5.436823in}{3.654459in}}%
\pgfpathlineto{\pgfqpoint{5.437107in}{3.624369in}}%
\pgfpathlineto{\pgfqpoint{5.437865in}{3.546160in}}%
\pgfpathlineto{\pgfqpoint{5.438244in}{3.619011in}}%
\pgfpathlineto{\pgfqpoint{5.438718in}{3.730826in}}%
\pgfpathlineto{\pgfqpoint{5.439476in}{3.679819in}}%
\pgfpathlineto{\pgfqpoint{5.440803in}{3.584085in}}%
\pgfpathlineto{\pgfqpoint{5.441087in}{3.613107in}}%
\pgfpathlineto{\pgfqpoint{5.441940in}{3.761327in}}%
\pgfpathlineto{\pgfqpoint{5.442414in}{3.689704in}}%
\pgfpathlineto{\pgfqpoint{5.444025in}{3.591103in}}%
\pgfpathlineto{\pgfqpoint{5.444120in}{3.594813in}}%
\pgfpathlineto{\pgfqpoint{5.444973in}{3.768847in}}%
\pgfpathlineto{\pgfqpoint{5.445541in}{3.635974in}}%
\pgfpathlineto{\pgfqpoint{5.447152in}{3.545486in}}%
\pgfpathlineto{\pgfqpoint{5.447247in}{3.542117in}}%
\pgfpathlineto{\pgfqpoint{5.447436in}{3.554911in}}%
\pgfpathlineto{\pgfqpoint{5.448005in}{3.685250in}}%
\pgfpathlineto{\pgfqpoint{5.448574in}{3.581960in}}%
\pgfpathlineto{\pgfqpoint{5.449142in}{3.475599in}}%
\pgfpathlineto{\pgfqpoint{5.449995in}{3.489699in}}%
\pgfpathlineto{\pgfqpoint{5.450753in}{3.574258in}}%
\pgfpathlineto{\pgfqpoint{5.451132in}{3.633518in}}%
\pgfpathlineto{\pgfqpoint{5.451606in}{3.545974in}}%
\pgfpathlineto{\pgfqpoint{5.452269in}{3.433515in}}%
\pgfpathlineto{\pgfqpoint{5.452743in}{3.502724in}}%
\pgfpathlineto{\pgfqpoint{5.454165in}{3.653591in}}%
\pgfpathlineto{\pgfqpoint{5.454449in}{3.619642in}}%
\pgfpathlineto{\pgfqpoint{5.455207in}{3.464522in}}%
\pgfpathlineto{\pgfqpoint{5.455776in}{3.560163in}}%
\pgfpathlineto{\pgfqpoint{5.457481in}{3.705412in}}%
\pgfpathlineto{\pgfqpoint{5.457576in}{3.701746in}}%
\pgfpathlineto{\pgfqpoint{5.458334in}{3.541663in}}%
\pgfpathlineto{\pgfqpoint{5.459092in}{3.628669in}}%
\pgfpathlineto{\pgfqpoint{5.459377in}{3.650741in}}%
\pgfpathlineto{\pgfqpoint{5.459850in}{3.622995in}}%
\pgfpathlineto{\pgfqpoint{5.460230in}{3.633363in}}%
\pgfpathlineto{\pgfqpoint{5.460324in}{3.633493in}}%
\pgfpathlineto{\pgfqpoint{5.460703in}{3.584312in}}%
\pgfpathlineto{\pgfqpoint{5.461461in}{3.417335in}}%
\pgfpathlineto{\pgfqpoint{5.461935in}{3.513063in}}%
\pgfpathlineto{\pgfqpoint{5.462599in}{3.580553in}}%
\pgfpathlineto{\pgfqpoint{5.463167in}{3.541417in}}%
\pgfpathlineto{\pgfqpoint{5.463452in}{3.552843in}}%
\pgfpathlineto{\pgfqpoint{5.463736in}{3.532916in}}%
\pgfpathlineto{\pgfqpoint{5.464494in}{3.389272in}}%
\pgfpathlineto{\pgfqpoint{5.465062in}{3.485019in}}%
\pgfpathlineto{\pgfqpoint{5.465536in}{3.609516in}}%
\pgfpathlineto{\pgfqpoint{5.466294in}{3.529789in}}%
\pgfpathlineto{\pgfqpoint{5.467526in}{3.345474in}}%
\pgfpathlineto{\pgfqpoint{5.467905in}{3.453190in}}%
\pgfpathlineto{\pgfqpoint{5.468758in}{3.975299in}}%
\pgfpathlineto{\pgfqpoint{5.469516in}{3.859798in}}%
\pgfpathlineto{\pgfqpoint{5.470843in}{3.281657in}}%
\pgfpathlineto{\pgfqpoint{5.471506in}{3.592631in}}%
\pgfpathlineto{\pgfqpoint{5.471791in}{3.632076in}}%
\pgfpathlineto{\pgfqpoint{5.472265in}{3.521544in}}%
\pgfpathlineto{\pgfqpoint{5.473781in}{3.354486in}}%
\pgfpathlineto{\pgfqpoint{5.473876in}{3.362515in}}%
\pgfpathlineto{\pgfqpoint{5.474918in}{3.525949in}}%
\pgfpathlineto{\pgfqpoint{5.475487in}{3.441658in}}%
\pgfpathlineto{\pgfqpoint{5.475676in}{3.424259in}}%
\pgfpathlineto{\pgfqpoint{5.476245in}{3.482684in}}%
\pgfpathlineto{\pgfqpoint{5.477856in}{3.659342in}}%
\pgfpathlineto{\pgfqpoint{5.478140in}{3.623828in}}%
\pgfpathlineto{\pgfqpoint{5.478898in}{3.451920in}}%
\pgfpathlineto{\pgfqpoint{5.479656in}{3.524625in}}%
\pgfpathlineto{\pgfqpoint{5.480983in}{3.658644in}}%
\pgfpathlineto{\pgfqpoint{5.481362in}{3.588000in}}%
\pgfpathlineto{\pgfqpoint{5.482025in}{3.478815in}}%
\pgfpathlineto{\pgfqpoint{5.482499in}{3.532301in}}%
\pgfpathlineto{\pgfqpoint{5.484015in}{3.657311in}}%
\pgfpathlineto{\pgfqpoint{5.484110in}{3.653385in}}%
\pgfpathlineto{\pgfqpoint{5.485152in}{3.461740in}}%
\pgfpathlineto{\pgfqpoint{5.485721in}{3.566590in}}%
\pgfpathlineto{\pgfqpoint{5.486953in}{3.633229in}}%
\pgfpathlineto{\pgfqpoint{5.487143in}{3.628428in}}%
\pgfpathlineto{\pgfqpoint{5.487522in}{3.594106in}}%
\pgfpathlineto{\pgfqpoint{5.488280in}{3.465558in}}%
\pgfpathlineto{\pgfqpoint{5.488659in}{3.560371in}}%
\pgfpathlineto{\pgfqpoint{5.489322in}{3.630031in}}%
\pgfpathlineto{\pgfqpoint{5.489891in}{3.613074in}}%
\pgfpathlineto{\pgfqpoint{5.490080in}{3.622731in}}%
\pgfpathlineto{\pgfqpoint{5.490554in}{3.593673in}}%
\pgfpathlineto{\pgfqpoint{5.491217in}{3.448203in}}%
\pgfpathlineto{\pgfqpoint{5.491691in}{3.540838in}}%
\pgfpathlineto{\pgfqpoint{5.492260in}{3.637572in}}%
\pgfpathlineto{\pgfqpoint{5.492828in}{3.578867in}}%
\pgfpathlineto{\pgfqpoint{5.494345in}{3.404923in}}%
\pgfpathlineto{\pgfqpoint{5.494724in}{3.463136in}}%
\pgfpathlineto{\pgfqpoint{5.495387in}{3.576713in}}%
\pgfpathlineto{\pgfqpoint{5.495956in}{3.524859in}}%
\pgfpathlineto{\pgfqpoint{5.497377in}{3.389629in}}%
\pgfpathlineto{\pgfqpoint{5.497661in}{3.414128in}}%
\pgfpathlineto{\pgfqpoint{5.498419in}{3.574551in}}%
\pgfpathlineto{\pgfqpoint{5.498988in}{3.501002in}}%
\pgfpathlineto{\pgfqpoint{5.500315in}{3.413705in}}%
\pgfpathlineto{\pgfqpoint{5.500410in}{3.414211in}}%
\pgfpathlineto{\pgfqpoint{5.500694in}{3.422829in}}%
\pgfpathlineto{\pgfqpoint{5.501547in}{3.579432in}}%
\pgfpathlineto{\pgfqpoint{5.502210in}{3.464557in}}%
\pgfpathlineto{\pgfqpoint{5.502494in}{3.432814in}}%
\pgfpathlineto{\pgfqpoint{5.503158in}{3.495098in}}%
\pgfpathlineto{\pgfqpoint{5.503537in}{3.483410in}}%
\pgfpathlineto{\pgfqpoint{5.503726in}{3.495192in}}%
\pgfpathlineto{\pgfqpoint{5.504579in}{3.658340in}}%
\pgfpathlineto{\pgfqpoint{5.505053in}{3.541194in}}%
\pgfpathlineto{\pgfqpoint{5.505716in}{3.394453in}}%
\pgfpathlineto{\pgfqpoint{5.506569in}{3.434829in}}%
\pgfpathlineto{\pgfqpoint{5.506759in}{3.430288in}}%
\pgfpathlineto{\pgfqpoint{5.507043in}{3.448050in}}%
\pgfpathlineto{\pgfqpoint{5.507801in}{3.535442in}}%
\pgfpathlineto{\pgfqpoint{5.508180in}{3.462303in}}%
\pgfpathlineto{\pgfqpoint{5.508749in}{3.328950in}}%
\pgfpathlineto{\pgfqpoint{5.509317in}{3.435352in}}%
\pgfpathlineto{\pgfqpoint{5.510739in}{3.555355in}}%
\pgfpathlineto{\pgfqpoint{5.511118in}{3.509872in}}%
\pgfpathlineto{\pgfqpoint{5.511876in}{3.366507in}}%
\pgfpathlineto{\pgfqpoint{5.512445in}{3.459321in}}%
\pgfpathlineto{\pgfqpoint{5.513392in}{3.450402in}}%
\pgfpathlineto{\pgfqpoint{5.515382in}{3.835642in}}%
\pgfpathlineto{\pgfqpoint{5.515477in}{3.835032in}}%
\pgfpathlineto{\pgfqpoint{5.515667in}{3.840653in}}%
\pgfpathlineto{\pgfqpoint{5.515856in}{3.846950in}}%
\pgfpathlineto{\pgfqpoint{5.516046in}{3.795412in}}%
\pgfpathlineto{\pgfqpoint{5.516709in}{3.382132in}}%
\pgfpathlineto{\pgfqpoint{5.517657in}{3.455377in}}%
\pgfpathlineto{\pgfqpoint{5.517941in}{3.408709in}}%
\pgfpathlineto{\pgfqpoint{5.518509in}{3.508546in}}%
\pgfpathlineto{\pgfqpoint{5.519078in}{3.602875in}}%
\pgfpathlineto{\pgfqpoint{5.519741in}{3.542933in}}%
\pgfpathlineto{\pgfqpoint{5.521068in}{3.429350in}}%
\pgfpathlineto{\pgfqpoint{5.521447in}{3.493415in}}%
\pgfpathlineto{\pgfqpoint{5.522205in}{3.638467in}}%
\pgfpathlineto{\pgfqpoint{5.522679in}{3.562600in}}%
\pgfpathlineto{\pgfqpoint{5.524101in}{3.474426in}}%
\pgfpathlineto{\pgfqpoint{5.524385in}{3.507503in}}%
\pgfpathlineto{\pgfqpoint{5.525143in}{3.680812in}}%
\pgfpathlineto{\pgfqpoint{5.525806in}{3.587234in}}%
\pgfpathlineto{\pgfqpoint{5.526280in}{3.555847in}}%
\pgfpathlineto{\pgfqpoint{5.527133in}{3.526003in}}%
\pgfpathlineto{\pgfqpoint{5.526754in}{3.558405in}}%
\pgfpathlineto{\pgfqpoint{5.527417in}{3.545188in}}%
\pgfpathlineto{\pgfqpoint{5.528270in}{3.717161in}}%
\pgfpathlineto{\pgfqpoint{5.528839in}{3.619127in}}%
\pgfpathlineto{\pgfqpoint{5.529313in}{3.539670in}}%
\pgfpathlineto{\pgfqpoint{5.530165in}{3.558832in}}%
\pgfpathlineto{\pgfqpoint{5.530260in}{3.558997in}}%
\pgfpathlineto{\pgfqpoint{5.531018in}{3.664220in}}%
\pgfpathlineto{\pgfqpoint{5.531397in}{3.710763in}}%
\pgfpathlineto{\pgfqpoint{5.531871in}{3.624894in}}%
\pgfpathlineto{\pgfqpoint{5.532440in}{3.525992in}}%
\pgfpathlineto{\pgfqpoint{5.533103in}{3.581059in}}%
\pgfpathlineto{\pgfqpoint{5.534051in}{3.659220in}}%
\pgfpathlineto{\pgfqpoint{5.534430in}{3.707269in}}%
\pgfpathlineto{\pgfqpoint{5.534904in}{3.622633in}}%
\pgfpathlineto{\pgfqpoint{5.535472in}{3.507653in}}%
\pgfpathlineto{\pgfqpoint{5.536041in}{3.576507in}}%
\pgfpathlineto{\pgfqpoint{5.537652in}{3.651325in}}%
\pgfpathlineto{\pgfqpoint{5.537747in}{3.650061in}}%
\pgfpathlineto{\pgfqpoint{5.538599in}{3.473538in}}%
\pgfpathlineto{\pgfqpoint{5.539358in}{3.584676in}}%
\pgfpathlineto{\pgfqpoint{5.540779in}{3.622701in}}%
\pgfpathlineto{\pgfqpoint{5.540874in}{3.612736in}}%
\pgfpathlineto{\pgfqpoint{5.541727in}{3.382283in}}%
\pgfpathlineto{\pgfqpoint{5.542485in}{3.474243in}}%
\pgfpathlineto{\pgfqpoint{5.542769in}{3.493996in}}%
\pgfpathlineto{\pgfqpoint{5.543243in}{3.459344in}}%
\pgfpathlineto{\pgfqpoint{5.543432in}{3.462102in}}%
\pgfpathlineto{\pgfqpoint{5.544380in}{3.378313in}}%
\pgfpathlineto{\pgfqpoint{5.544570in}{3.368426in}}%
\pgfpathlineto{\pgfqpoint{5.544949in}{3.405149in}}%
\pgfpathlineto{\pgfqpoint{5.545707in}{3.618933in}}%
\pgfpathlineto{\pgfqpoint{5.546654in}{3.552535in}}%
\pgfpathlineto{\pgfqpoint{5.547507in}{3.489596in}}%
\pgfpathlineto{\pgfqpoint{5.547886in}{3.447990in}}%
\pgfpathlineto{\pgfqpoint{5.548265in}{3.532367in}}%
\pgfpathlineto{\pgfqpoint{5.548834in}{3.647397in}}%
\pgfpathlineto{\pgfqpoint{5.549497in}{3.579256in}}%
\pgfpathlineto{\pgfqpoint{5.551014in}{3.449529in}}%
\pgfpathlineto{\pgfqpoint{5.551108in}{3.455735in}}%
\pgfpathlineto{\pgfqpoint{5.551961in}{3.599680in}}%
\pgfpathlineto{\pgfqpoint{5.552435in}{3.513626in}}%
\pgfpathlineto{\pgfqpoint{5.553951in}{3.417603in}}%
\pgfpathlineto{\pgfqpoint{5.554520in}{3.506722in}}%
\pgfpathlineto{\pgfqpoint{5.554994in}{3.615605in}}%
\pgfpathlineto{\pgfqpoint{5.555562in}{3.510171in}}%
\pgfpathlineto{\pgfqpoint{5.555941in}{3.437197in}}%
\pgfpathlineto{\pgfqpoint{5.556794in}{3.453576in}}%
\pgfpathlineto{\pgfqpoint{5.557078in}{3.451014in}}%
\pgfpathlineto{\pgfqpoint{5.557363in}{3.475437in}}%
\pgfpathlineto{\pgfqpoint{5.558121in}{3.638530in}}%
\pgfpathlineto{\pgfqpoint{5.558500in}{3.534227in}}%
\pgfpathlineto{\pgfqpoint{5.558974in}{3.409108in}}%
\pgfpathlineto{\pgfqpoint{5.559448in}{3.565284in}}%
\pgfpathlineto{\pgfqpoint{5.561059in}{3.980947in}}%
\pgfpathlineto{\pgfqpoint{5.561153in}{3.972711in}}%
\pgfpathlineto{\pgfqpoint{5.561722in}{3.620376in}}%
\pgfpathlineto{\pgfqpoint{5.562196in}{3.275817in}}%
\pgfpathlineto{\pgfqpoint{5.562859in}{3.498191in}}%
\pgfpathlineto{\pgfqpoint{5.564375in}{3.611482in}}%
\pgfpathlineto{\pgfqpoint{5.564470in}{3.607443in}}%
\pgfpathlineto{\pgfqpoint{5.565323in}{3.429896in}}%
\pgfpathlineto{\pgfqpoint{5.565986in}{3.533762in}}%
\pgfpathlineto{\pgfqpoint{5.567313in}{3.594677in}}%
\pgfpathlineto{\pgfqpoint{5.567502in}{3.583378in}}%
\pgfpathlineto{\pgfqpoint{5.568450in}{3.412821in}}%
\pgfpathlineto{\pgfqpoint{5.568924in}{3.508878in}}%
\pgfpathlineto{\pgfqpoint{5.570345in}{3.593420in}}%
\pgfpathlineto{\pgfqpoint{5.570440in}{3.594055in}}%
\pgfpathlineto{\pgfqpoint{5.570535in}{3.587906in}}%
\pgfpathlineto{\pgfqpoint{5.571388in}{3.453860in}}%
\pgfpathlineto{\pgfqpoint{5.571956in}{3.537658in}}%
\pgfpathlineto{\pgfqpoint{5.572904in}{3.636477in}}%
\pgfpathlineto{\pgfqpoint{5.573283in}{3.604493in}}%
\pgfpathlineto{\pgfqpoint{5.574610in}{3.481015in}}%
\pgfpathlineto{\pgfqpoint{5.574989in}{3.554177in}}%
\pgfpathlineto{\pgfqpoint{5.575557in}{3.673598in}}%
\pgfpathlineto{\pgfqpoint{5.576221in}{3.635231in}}%
\pgfpathlineto{\pgfqpoint{5.577642in}{3.521754in}}%
\pgfpathlineto{\pgfqpoint{5.578021in}{3.564509in}}%
\pgfpathlineto{\pgfqpoint{5.578685in}{3.721938in}}%
\pgfpathlineto{\pgfqpoint{5.579348in}{3.633038in}}%
\pgfpathlineto{\pgfqpoint{5.580390in}{3.566364in}}%
\pgfpathlineto{\pgfqpoint{5.580769in}{3.568488in}}%
\pgfpathlineto{\pgfqpoint{5.581338in}{3.667659in}}%
\pgfpathlineto{\pgfqpoint{5.581812in}{3.761822in}}%
\pgfpathlineto{\pgfqpoint{5.582286in}{3.660093in}}%
\pgfpathlineto{\pgfqpoint{5.583328in}{3.594535in}}%
\pgfpathlineto{\pgfqpoint{5.583612in}{3.598476in}}%
\pgfpathlineto{\pgfqpoint{5.583897in}{3.594178in}}%
\pgfpathlineto{\pgfqpoint{5.584086in}{3.600702in}}%
\pgfpathlineto{\pgfqpoint{5.584844in}{3.726251in}}%
\pgfpathlineto{\pgfqpoint{5.585318in}{3.647246in}}%
\pgfpathlineto{\pgfqpoint{5.585982in}{3.515840in}}%
\pgfpathlineto{\pgfqpoint{5.586740in}{3.554645in}}%
\pgfpathlineto{\pgfqpoint{5.587119in}{3.572532in}}%
\pgfpathlineto{\pgfqpoint{5.587687in}{3.680168in}}%
\pgfpathlineto{\pgfqpoint{5.588351in}{3.613850in}}%
\pgfpathlineto{\pgfqpoint{5.589014in}{3.486819in}}%
\pgfpathlineto{\pgfqpoint{5.589583in}{3.565565in}}%
\pgfpathlineto{\pgfqpoint{5.591004in}{3.664752in}}%
\pgfpathlineto{\pgfqpoint{5.591194in}{3.645595in}}%
\pgfpathlineto{\pgfqpoint{5.592046in}{3.459393in}}%
\pgfpathlineto{\pgfqpoint{5.592710in}{3.559934in}}%
\pgfpathlineto{\pgfqpoint{5.594131in}{3.684194in}}%
\pgfpathlineto{\pgfqpoint{5.594510in}{3.640397in}}%
\pgfpathlineto{\pgfqpoint{5.595079in}{3.522936in}}%
\pgfpathlineto{\pgfqpoint{5.595742in}{3.600666in}}%
\pgfpathlineto{\pgfqpoint{5.596216in}{3.679352in}}%
\pgfpathlineto{\pgfqpoint{5.596879in}{3.623015in}}%
\pgfpathlineto{\pgfqpoint{5.597732in}{3.502653in}}%
\pgfpathlineto{\pgfqpoint{5.598206in}{3.404277in}}%
\pgfpathlineto{\pgfqpoint{5.598775in}{3.513345in}}%
\pgfpathlineto{\pgfqpoint{5.599154in}{3.563525in}}%
\pgfpathlineto{\pgfqpoint{5.600007in}{3.544107in}}%
\pgfpathlineto{\pgfqpoint{5.600101in}{3.544840in}}%
\pgfpathlineto{\pgfqpoint{5.600196in}{3.539528in}}%
\pgfpathlineto{\pgfqpoint{5.601239in}{3.395440in}}%
\pgfpathlineto{\pgfqpoint{5.601807in}{3.468730in}}%
\pgfpathlineto{\pgfqpoint{5.602376in}{3.578494in}}%
\pgfpathlineto{\pgfqpoint{5.603039in}{3.523707in}}%
\pgfpathlineto{\pgfqpoint{5.603892in}{3.459631in}}%
\pgfpathlineto{\pgfqpoint{5.604366in}{3.352264in}}%
\pgfpathlineto{\pgfqpoint{5.604745in}{3.465835in}}%
\pgfpathlineto{\pgfqpoint{5.605692in}{4.001472in}}%
\pgfpathlineto{\pgfqpoint{5.606451in}{3.881771in}}%
\pgfpathlineto{\pgfqpoint{5.607209in}{3.596465in}}%
\pgfpathlineto{\pgfqpoint{5.607682in}{3.322142in}}%
\pgfpathlineto{\pgfqpoint{5.608251in}{3.556310in}}%
\pgfpathlineto{\pgfqpoint{5.608535in}{3.621807in}}%
\pgfpathlineto{\pgfqpoint{5.609104in}{3.488382in}}%
\pgfpathlineto{\pgfqpoint{5.610431in}{3.392601in}}%
\pgfpathlineto{\pgfqpoint{5.610620in}{3.402235in}}%
\pgfpathlineto{\pgfqpoint{5.611663in}{3.654762in}}%
\pgfpathlineto{\pgfqpoint{5.612421in}{3.529822in}}%
\pgfpathlineto{\pgfqpoint{5.612705in}{3.491463in}}%
\pgfpathlineto{\pgfqpoint{5.613463in}{3.528109in}}%
\pgfpathlineto{\pgfqpoint{5.614316in}{3.628940in}}%
\pgfpathlineto{\pgfqpoint{5.614600in}{3.668821in}}%
\pgfpathlineto{\pgfqpoint{5.615169in}{3.597668in}}%
\pgfpathlineto{\pgfqpoint{5.615737in}{3.493815in}}%
\pgfpathlineto{\pgfqpoint{5.616306in}{3.563796in}}%
\pgfpathlineto{\pgfqpoint{5.617917in}{3.710541in}}%
\pgfpathlineto{\pgfqpoint{5.618012in}{3.702819in}}%
\pgfpathlineto{\pgfqpoint{5.618770in}{3.520594in}}%
\pgfpathlineto{\pgfqpoint{5.619433in}{3.631172in}}%
\pgfpathlineto{\pgfqpoint{5.620760in}{3.732997in}}%
\pgfpathlineto{\pgfqpoint{5.621139in}{3.718393in}}%
\pgfpathlineto{\pgfqpoint{5.621897in}{3.556142in}}%
\pgfpathlineto{\pgfqpoint{5.622466in}{3.665251in}}%
\pgfpathlineto{\pgfqpoint{5.623792in}{3.733957in}}%
\pgfpathlineto{\pgfqpoint{5.623982in}{3.728325in}}%
\pgfpathlineto{\pgfqpoint{5.624930in}{3.559433in}}%
\pgfpathlineto{\pgfqpoint{5.625593in}{3.665415in}}%
\pgfpathlineto{\pgfqpoint{5.626067in}{3.724328in}}%
\pgfpathlineto{\pgfqpoint{5.626825in}{3.713302in}}%
\pgfpathlineto{\pgfqpoint{5.626920in}{3.715026in}}%
\pgfpathlineto{\pgfqpoint{5.627109in}{3.708232in}}%
\pgfpathlineto{\pgfqpoint{5.628152in}{3.555377in}}%
\pgfpathlineto{\pgfqpoint{5.628531in}{3.631407in}}%
\pgfpathlineto{\pgfqpoint{5.629099in}{3.730780in}}%
\pgfpathlineto{\pgfqpoint{5.629668in}{3.647471in}}%
\pgfpathlineto{\pgfqpoint{5.631089in}{3.488928in}}%
\pgfpathlineto{\pgfqpoint{5.631563in}{3.571234in}}%
\pgfpathlineto{\pgfqpoint{5.632226in}{3.662146in}}%
\pgfpathlineto{\pgfqpoint{5.632700in}{3.594874in}}%
\pgfpathlineto{\pgfqpoint{5.634122in}{3.477292in}}%
\pgfpathlineto{\pgfqpoint{5.634501in}{3.510110in}}%
\pgfpathlineto{\pgfqpoint{5.635164in}{3.658016in}}%
\pgfpathlineto{\pgfqpoint{5.635827in}{3.583172in}}%
\pgfpathlineto{\pgfqpoint{5.636965in}{3.502496in}}%
\pgfpathlineto{\pgfqpoint{5.637249in}{3.506127in}}%
\pgfpathlineto{\pgfqpoint{5.637628in}{3.544273in}}%
\pgfpathlineto{\pgfqpoint{5.638386in}{3.675712in}}%
\pgfpathlineto{\pgfqpoint{5.638860in}{3.605741in}}%
\pgfpathlineto{\pgfqpoint{5.639239in}{3.542190in}}%
\pgfpathlineto{\pgfqpoint{5.639997in}{3.586987in}}%
\pgfpathlineto{\pgfqpoint{5.640187in}{3.585362in}}%
\pgfpathlineto{\pgfqpoint{5.640376in}{3.589860in}}%
\pgfpathlineto{\pgfqpoint{5.641419in}{3.754200in}}%
\pgfpathlineto{\pgfqpoint{5.641892in}{3.648359in}}%
\pgfpathlineto{\pgfqpoint{5.643409in}{3.521596in}}%
\pgfpathlineto{\pgfqpoint{5.643693in}{3.539241in}}%
\pgfpathlineto{\pgfqpoint{5.644546in}{3.631572in}}%
\pgfpathlineto{\pgfqpoint{5.644925in}{3.565252in}}%
\pgfpathlineto{\pgfqpoint{5.645493in}{3.445153in}}%
\pgfpathlineto{\pgfqpoint{5.646157in}{3.497017in}}%
\pgfpathlineto{\pgfqpoint{5.647673in}{3.618107in}}%
\pgfpathlineto{\pgfqpoint{5.647862in}{3.600863in}}%
\pgfpathlineto{\pgfqpoint{5.648621in}{3.447212in}}%
\pgfpathlineto{\pgfqpoint{5.649189in}{3.542904in}}%
\pgfpathlineto{\pgfqpoint{5.649758in}{3.598498in}}%
\pgfpathlineto{\pgfqpoint{5.650232in}{3.543074in}}%
\pgfpathlineto{\pgfqpoint{5.650326in}{3.537899in}}%
\pgfpathlineto{\pgfqpoint{5.650611in}{3.581376in}}%
\pgfpathlineto{\pgfqpoint{5.652695in}{3.939441in}}%
\pgfpathlineto{\pgfqpoint{5.652790in}{3.931481in}}%
\pgfpathlineto{\pgfqpoint{5.653548in}{3.491318in}}%
\pgfpathlineto{\pgfqpoint{5.653738in}{3.442746in}}%
\pgfpathlineto{\pgfqpoint{5.654212in}{3.517463in}}%
\pgfpathlineto{\pgfqpoint{5.654685in}{3.481888in}}%
\pgfpathlineto{\pgfqpoint{5.654780in}{3.481073in}}%
\pgfpathlineto{\pgfqpoint{5.654875in}{3.486978in}}%
\pgfpathlineto{\pgfqpoint{5.655917in}{3.665402in}}%
\pgfpathlineto{\pgfqpoint{5.656770in}{3.608790in}}%
\pgfpathlineto{\pgfqpoint{5.657149in}{3.579450in}}%
\pgfpathlineto{\pgfqpoint{5.657907in}{3.463270in}}%
\pgfpathlineto{\pgfqpoint{5.658381in}{3.521971in}}%
\pgfpathlineto{\pgfqpoint{5.658855in}{3.639310in}}%
\pgfpathlineto{\pgfqpoint{5.659708in}{3.586264in}}%
\pgfpathlineto{\pgfqpoint{5.660940in}{3.485647in}}%
\pgfpathlineto{\pgfqpoint{5.661129in}{3.506054in}}%
\pgfpathlineto{\pgfqpoint{5.661982in}{3.696984in}}%
\pgfpathlineto{\pgfqpoint{5.662646in}{3.589598in}}%
\pgfpathlineto{\pgfqpoint{5.664067in}{3.505656in}}%
\pgfpathlineto{\pgfqpoint{5.664257in}{3.514221in}}%
\pgfpathlineto{\pgfqpoint{5.665015in}{3.685130in}}%
\pgfpathlineto{\pgfqpoint{5.665773in}{3.575163in}}%
\pgfpathlineto{\pgfqpoint{5.667005in}{3.527165in}}%
\pgfpathlineto{\pgfqpoint{5.667194in}{3.535813in}}%
\pgfpathlineto{\pgfqpoint{5.668142in}{3.668766in}}%
\pgfpathlineto{\pgfqpoint{5.668616in}{3.584368in}}%
\pgfpathlineto{\pgfqpoint{5.669563in}{3.478721in}}%
\pgfpathlineto{\pgfqpoint{5.669848in}{3.517359in}}%
\pgfpathlineto{\pgfqpoint{5.671174in}{3.662822in}}%
\pgfpathlineto{\pgfqpoint{5.671553in}{3.628090in}}%
\pgfpathlineto{\pgfqpoint{5.672217in}{3.488438in}}%
\pgfpathlineto{\pgfqpoint{5.672975in}{3.537849in}}%
\pgfpathlineto{\pgfqpoint{5.674207in}{3.617924in}}%
\pgfpathlineto{\pgfqpoint{5.674396in}{3.601790in}}%
\pgfpathlineto{\pgfqpoint{5.675439in}{3.349073in}}%
\pgfpathlineto{\pgfqpoint{5.676481in}{3.442952in}}%
\pgfpathlineto{\pgfqpoint{5.677524in}{3.523893in}}%
\pgfpathlineto{\pgfqpoint{5.677997in}{3.460824in}}%
\pgfpathlineto{\pgfqpoint{5.678471in}{3.405363in}}%
\pgfpathlineto{\pgfqpoint{5.678945in}{3.467437in}}%
\pgfpathlineto{\pgfqpoint{5.679703in}{3.536817in}}%
\pgfpathlineto{\pgfqpoint{5.680272in}{3.513331in}}%
\pgfpathlineto{\pgfqpoint{5.680461in}{3.515113in}}%
\pgfpathlineto{\pgfqpoint{5.680651in}{3.503835in}}%
\pgfpathlineto{\pgfqpoint{5.681598in}{3.355499in}}%
\pgfpathlineto{\pgfqpoint{5.682072in}{3.465224in}}%
\pgfpathlineto{\pgfqpoint{5.682357in}{3.518992in}}%
\pgfpathlineto{\pgfqpoint{5.683209in}{3.494920in}}%
\pgfpathlineto{\pgfqpoint{5.684536in}{3.344793in}}%
\pgfpathlineto{\pgfqpoint{5.685010in}{3.426948in}}%
\pgfpathlineto{\pgfqpoint{5.685768in}{3.602157in}}%
\pgfpathlineto{\pgfqpoint{5.686526in}{3.527747in}}%
\pgfpathlineto{\pgfqpoint{5.687663in}{3.464723in}}%
\pgfpathlineto{\pgfqpoint{5.687948in}{3.480432in}}%
\pgfpathlineto{\pgfqpoint{5.688801in}{3.607717in}}%
\pgfpathlineto{\pgfqpoint{5.689180in}{3.539135in}}%
\pgfpathlineto{\pgfqpoint{5.690696in}{3.412201in}}%
\pgfpathlineto{\pgfqpoint{5.691075in}{3.459236in}}%
\pgfpathlineto{\pgfqpoint{5.691928in}{3.587321in}}%
\pgfpathlineto{\pgfqpoint{5.692307in}{3.513352in}}%
\pgfpathlineto{\pgfqpoint{5.693065in}{3.423530in}}%
\pgfpathlineto{\pgfqpoint{5.693634in}{3.437323in}}%
\pgfpathlineto{\pgfqpoint{5.693728in}{3.435503in}}%
\pgfpathlineto{\pgfqpoint{5.693918in}{3.446712in}}%
\pgfpathlineto{\pgfqpoint{5.694865in}{3.613180in}}%
\pgfpathlineto{\pgfqpoint{5.695434in}{3.499761in}}%
\pgfpathlineto{\pgfqpoint{5.696571in}{3.378270in}}%
\pgfpathlineto{\pgfqpoint{5.696761in}{3.413134in}}%
\pgfpathlineto{\pgfqpoint{5.697993in}{3.997351in}}%
\pgfpathlineto{\pgfqpoint{5.698751in}{3.790202in}}%
\pgfpathlineto{\pgfqpoint{5.699888in}{3.344682in}}%
\pgfpathlineto{\pgfqpoint{5.700551in}{3.541388in}}%
\pgfpathlineto{\pgfqpoint{5.700930in}{3.611747in}}%
\pgfpathlineto{\pgfqpoint{5.701499in}{3.520277in}}%
\pgfpathlineto{\pgfqpoint{5.702162in}{3.405921in}}%
\pgfpathlineto{\pgfqpoint{5.702636in}{3.492537in}}%
\pgfpathlineto{\pgfqpoint{5.704058in}{3.603984in}}%
\pgfpathlineto{\pgfqpoint{5.704247in}{3.592928in}}%
\pgfpathlineto{\pgfqpoint{5.705195in}{3.443815in}}%
\pgfpathlineto{\pgfqpoint{5.705763in}{3.544730in}}%
\pgfpathlineto{\pgfqpoint{5.706521in}{3.608105in}}%
\pgfpathlineto{\pgfqpoint{5.707090in}{3.597452in}}%
\pgfpathlineto{\pgfqpoint{5.707280in}{3.604385in}}%
\pgfpathlineto{\pgfqpoint{5.707564in}{3.582062in}}%
\pgfpathlineto{\pgfqpoint{5.708322in}{3.449590in}}%
\pgfpathlineto{\pgfqpoint{5.708796in}{3.551900in}}%
\pgfpathlineto{\pgfqpoint{5.709270in}{3.637352in}}%
\pgfpathlineto{\pgfqpoint{5.710122in}{3.615921in}}%
\pgfpathlineto{\pgfqpoint{5.711354in}{3.502209in}}%
\pgfpathlineto{\pgfqpoint{5.711733in}{3.547498in}}%
\pgfpathlineto{\pgfqpoint{5.712397in}{3.703054in}}%
\pgfpathlineto{\pgfqpoint{5.713060in}{3.628398in}}%
\pgfpathlineto{\pgfqpoint{5.713344in}{3.633724in}}%
\pgfpathlineto{\pgfqpoint{5.713439in}{3.628166in}}%
\pgfpathlineto{\pgfqpoint{5.714482in}{3.525297in}}%
\pgfpathlineto{\pgfqpoint{5.714766in}{3.575701in}}%
\pgfpathlineto{\pgfqpoint{5.715714in}{3.714558in}}%
\pgfpathlineto{\pgfqpoint{5.715998in}{3.658689in}}%
\pgfpathlineto{\pgfqpoint{5.717514in}{3.560051in}}%
\pgfpathlineto{\pgfqpoint{5.717704in}{3.554822in}}%
\pgfpathlineto{\pgfqpoint{5.717893in}{3.578845in}}%
\pgfpathlineto{\pgfqpoint{5.718462in}{3.721691in}}%
\pgfpathlineto{\pgfqpoint{5.719125in}{3.620847in}}%
\pgfpathlineto{\pgfqpoint{5.720262in}{3.546860in}}%
\pgfpathlineto{\pgfqpoint{5.720452in}{3.554581in}}%
\pgfpathlineto{\pgfqpoint{5.721589in}{3.657724in}}%
\pgfpathlineto{\pgfqpoint{5.722063in}{3.604434in}}%
\pgfpathlineto{\pgfqpoint{5.722726in}{3.456057in}}%
\pgfpathlineto{\pgfqpoint{5.723579in}{3.485034in}}%
\pgfpathlineto{\pgfqpoint{5.723674in}{3.483624in}}%
\pgfpathlineto{\pgfqpoint{5.723769in}{3.488377in}}%
\pgfpathlineto{\pgfqpoint{5.724716in}{3.618639in}}%
\pgfpathlineto{\pgfqpoint{5.725190in}{3.535727in}}%
\pgfpathlineto{\pgfqpoint{5.726043in}{3.439184in}}%
\pgfpathlineto{\pgfqpoint{5.726422in}{3.483274in}}%
\pgfpathlineto{\pgfqpoint{5.727749in}{3.610901in}}%
\pgfpathlineto{\pgfqpoint{5.728033in}{3.579384in}}%
\pgfpathlineto{\pgfqpoint{5.728886in}{3.423069in}}%
\pgfpathlineto{\pgfqpoint{5.729454in}{3.501916in}}%
\pgfpathlineto{\pgfqpoint{5.730971in}{3.613668in}}%
\pgfpathlineto{\pgfqpoint{5.731255in}{3.574351in}}%
\pgfpathlineto{\pgfqpoint{5.731918in}{3.469804in}}%
\pgfpathlineto{\pgfqpoint{5.732487in}{3.533072in}}%
\pgfpathlineto{\pgfqpoint{5.733908in}{3.648869in}}%
\pgfpathlineto{\pgfqpoint{5.734003in}{3.646755in}}%
\pgfpathlineto{\pgfqpoint{5.734951in}{3.425454in}}%
\pgfpathlineto{\pgfqpoint{5.735898in}{3.551360in}}%
\pgfpathlineto{\pgfqpoint{5.736183in}{3.586137in}}%
\pgfpathlineto{\pgfqpoint{5.736846in}{3.531817in}}%
\pgfpathlineto{\pgfqpoint{5.737604in}{3.462717in}}%
\pgfpathlineto{\pgfqpoint{5.738078in}{3.369812in}}%
\pgfpathlineto{\pgfqpoint{5.738646in}{3.462935in}}%
\pgfpathlineto{\pgfqpoint{5.739215in}{3.558959in}}%
\pgfpathlineto{\pgfqpoint{5.739784in}{3.502615in}}%
\pgfpathlineto{\pgfqpoint{5.741205in}{3.340276in}}%
\pgfpathlineto{\pgfqpoint{5.741774in}{3.423642in}}%
\pgfpathlineto{\pgfqpoint{5.742058in}{3.489763in}}%
\pgfpathlineto{\pgfqpoint{5.742627in}{3.383472in}}%
\pgfpathlineto{\pgfqpoint{5.742911in}{3.431918in}}%
\pgfpathlineto{\pgfqpoint{5.744996in}{3.928427in}}%
\pgfpathlineto{\pgfqpoint{5.745375in}{3.806540in}}%
\pgfpathlineto{\pgfqpoint{5.746038in}{3.373617in}}%
\pgfpathlineto{\pgfqpoint{5.746796in}{3.499451in}}%
\pgfpathlineto{\pgfqpoint{5.747270in}{3.495905in}}%
\pgfpathlineto{\pgfqpoint{5.747649in}{3.525700in}}%
\pgfpathlineto{\pgfqpoint{5.748407in}{3.669109in}}%
\pgfpathlineto{\pgfqpoint{5.748881in}{3.582192in}}%
\pgfpathlineto{\pgfqpoint{5.749450in}{3.496856in}}%
\pgfpathlineto{\pgfqpoint{5.750302in}{3.520251in}}%
\pgfpathlineto{\pgfqpoint{5.751534in}{3.679925in}}%
\pgfpathlineto{\pgfqpoint{5.752008in}{3.591409in}}%
\pgfpathlineto{\pgfqpoint{5.752672in}{3.511445in}}%
\pgfpathlineto{\pgfqpoint{5.753145in}{3.566340in}}%
\pgfpathlineto{\pgfqpoint{5.754567in}{3.691421in}}%
\pgfpathlineto{\pgfqpoint{5.754946in}{3.641689in}}%
\pgfpathlineto{\pgfqpoint{5.755609in}{3.517557in}}%
\pgfpathlineto{\pgfqpoint{5.756178in}{3.588431in}}%
\pgfpathlineto{\pgfqpoint{5.757505in}{3.716422in}}%
\pgfpathlineto{\pgfqpoint{5.757884in}{3.683395in}}%
\pgfpathlineto{\pgfqpoint{5.758642in}{3.539921in}}%
\pgfpathlineto{\pgfqpoint{5.759210in}{3.626484in}}%
\pgfpathlineto{\pgfqpoint{5.760442in}{3.729309in}}%
\pgfpathlineto{\pgfqpoint{5.760727in}{3.705111in}}%
\pgfpathlineto{\pgfqpoint{5.761674in}{3.540814in}}%
\pgfpathlineto{\pgfqpoint{5.762338in}{3.631772in}}%
\pgfpathlineto{\pgfqpoint{5.763096in}{3.700158in}}%
\pgfpathlineto{\pgfqpoint{5.763475in}{3.660276in}}%
\pgfpathlineto{\pgfqpoint{5.764801in}{3.491251in}}%
\pgfpathlineto{\pgfqpoint{5.765275in}{3.561798in}}%
\pgfpathlineto{\pgfqpoint{5.765939in}{3.669737in}}%
\pgfpathlineto{\pgfqpoint{5.766602in}{3.615443in}}%
\pgfpathlineto{\pgfqpoint{5.768023in}{3.461469in}}%
\pgfpathlineto{\pgfqpoint{5.768402in}{3.533391in}}%
\pgfpathlineto{\pgfqpoint{5.768876in}{3.634035in}}%
\pgfpathlineto{\pgfqpoint{5.769540in}{3.553917in}}%
\pgfpathlineto{\pgfqpoint{5.770961in}{3.426290in}}%
\pgfpathlineto{\pgfqpoint{5.771340in}{3.476431in}}%
\pgfpathlineto{\pgfqpoint{5.772098in}{3.609171in}}%
\pgfpathlineto{\pgfqpoint{5.772572in}{3.521697in}}%
\pgfpathlineto{\pgfqpoint{5.774183in}{3.433926in}}%
\pgfpathlineto{\pgfqpoint{5.774562in}{3.485942in}}%
\pgfpathlineto{\pgfqpoint{5.775131in}{3.581692in}}%
\pgfpathlineto{\pgfqpoint{5.775699in}{3.506687in}}%
\pgfpathlineto{\pgfqpoint{5.776268in}{3.425512in}}%
\pgfpathlineto{\pgfqpoint{5.777026in}{3.454633in}}%
\pgfpathlineto{\pgfqpoint{5.777784in}{3.567607in}}%
\pgfpathlineto{\pgfqpoint{5.778163in}{3.638823in}}%
\pgfpathlineto{\pgfqpoint{5.778826in}{3.553040in}}%
\pgfpathlineto{\pgfqpoint{5.779300in}{3.478298in}}%
\pgfpathlineto{\pgfqpoint{5.779964in}{3.517152in}}%
\pgfpathlineto{\pgfqpoint{5.781290in}{3.606598in}}%
\pgfpathlineto{\pgfqpoint{5.781575in}{3.558748in}}%
\pgfpathlineto{\pgfqpoint{5.782333in}{3.381602in}}%
\pgfpathlineto{\pgfqpoint{5.782996in}{3.466098in}}%
\pgfpathlineto{\pgfqpoint{5.784323in}{3.562963in}}%
\pgfpathlineto{\pgfqpoint{5.784702in}{3.516526in}}%
\pgfpathlineto{\pgfqpoint{5.785365in}{3.369264in}}%
\pgfpathlineto{\pgfqpoint{5.785934in}{3.479173in}}%
\pgfpathlineto{\pgfqpoint{5.787545in}{3.563173in}}%
\pgfpathlineto{\pgfqpoint{5.788019in}{3.437776in}}%
\pgfpathlineto{\pgfqpoint{5.788492in}{3.331892in}}%
\pgfpathlineto{\pgfqpoint{5.788871in}{3.483034in}}%
\pgfpathlineto{\pgfqpoint{5.789914in}{3.973214in}}%
\pgfpathlineto{\pgfqpoint{5.790482in}{3.903854in}}%
\pgfpathlineto{\pgfqpoint{5.791051in}{3.729423in}}%
\pgfpathlineto{\pgfqpoint{5.791714in}{3.287784in}}%
\pgfpathlineto{\pgfqpoint{5.792283in}{3.537218in}}%
\pgfpathlineto{\pgfqpoint{5.792757in}{3.610499in}}%
\pgfpathlineto{\pgfqpoint{5.793420in}{3.555349in}}%
\pgfpathlineto{\pgfqpoint{5.793799in}{3.569694in}}%
\pgfpathlineto{\pgfqpoint{5.793989in}{3.554892in}}%
\pgfpathlineto{\pgfqpoint{5.794557in}{3.461953in}}%
\pgfpathlineto{\pgfqpoint{5.795221in}{3.531083in}}%
\pgfpathlineto{\pgfqpoint{5.795694in}{3.638401in}}%
\pgfpathlineto{\pgfqpoint{5.796453in}{3.571359in}}%
\pgfpathlineto{\pgfqpoint{5.797779in}{3.492167in}}%
\pgfpathlineto{\pgfqpoint{5.797969in}{3.497308in}}%
\pgfpathlineto{\pgfqpoint{5.798822in}{3.647638in}}%
\pgfpathlineto{\pgfqpoint{5.799485in}{3.577913in}}%
\pgfpathlineto{\pgfqpoint{5.800717in}{3.505034in}}%
\pgfpathlineto{\pgfqpoint{5.801001in}{3.522291in}}%
\pgfpathlineto{\pgfqpoint{5.801949in}{3.691945in}}%
\pgfpathlineto{\pgfqpoint{5.802423in}{3.598978in}}%
\pgfpathlineto{\pgfqpoint{5.803844in}{3.531458in}}%
\pgfpathlineto{\pgfqpoint{5.804318in}{3.572448in}}%
\pgfpathlineto{\pgfqpoint{5.804887in}{3.652171in}}%
\pgfpathlineto{\pgfqpoint{5.805360in}{3.563922in}}%
\pgfpathlineto{\pgfqpoint{5.805929in}{3.429070in}}%
\pgfpathlineto{\pgfqpoint{5.806687in}{3.474315in}}%
\pgfpathlineto{\pgfqpoint{5.807350in}{3.569238in}}%
\pgfpathlineto{\pgfqpoint{5.808109in}{3.709495in}}%
\pgfpathlineto{\pgfqpoint{5.808582in}{3.624552in}}%
\pgfpathlineto{\pgfqpoint{5.809056in}{3.527087in}}%
\pgfpathlineto{\pgfqpoint{5.809814in}{3.585814in}}%
\pgfpathlineto{\pgfqpoint{5.811141in}{3.687548in}}%
\pgfpathlineto{\pgfqpoint{5.811520in}{3.632065in}}%
\pgfpathlineto{\pgfqpoint{5.812183in}{3.484285in}}%
\pgfpathlineto{\pgfqpoint{5.812752in}{3.569658in}}%
\pgfpathlineto{\pgfqpoint{5.814079in}{3.625357in}}%
\pgfpathlineto{\pgfqpoint{5.814173in}{3.621282in}}%
\pgfpathlineto{\pgfqpoint{5.814932in}{3.481574in}}%
\pgfpathlineto{\pgfqpoint{5.815216in}{3.449502in}}%
\pgfpathlineto{\pgfqpoint{5.815784in}{3.514735in}}%
\pgfpathlineto{\pgfqpoint{5.816258in}{3.579242in}}%
\pgfpathlineto{\pgfqpoint{5.817016in}{3.549424in}}%
\pgfpathlineto{\pgfqpoint{5.817585in}{3.512747in}}%
\pgfpathlineto{\pgfqpoint{5.818343in}{3.407925in}}%
\pgfpathlineto{\pgfqpoint{5.818817in}{3.475768in}}%
\pgfpathlineto{\pgfqpoint{5.819386in}{3.563300in}}%
\pgfpathlineto{\pgfqpoint{5.820144in}{3.537661in}}%
\pgfpathlineto{\pgfqpoint{5.821281in}{3.393780in}}%
\pgfpathlineto{\pgfqpoint{5.821849in}{3.470692in}}%
\pgfpathlineto{\pgfqpoint{5.822513in}{3.589382in}}%
\pgfpathlineto{\pgfqpoint{5.823081in}{3.531337in}}%
\pgfpathlineto{\pgfqpoint{5.824408in}{3.444913in}}%
\pgfpathlineto{\pgfqpoint{5.823745in}{3.535568in}}%
\pgfpathlineto{\pgfqpoint{5.824598in}{3.454972in}}%
\pgfpathlineto{\pgfqpoint{5.825545in}{3.631167in}}%
\pgfpathlineto{\pgfqpoint{5.826114in}{3.524965in}}%
\pgfpathlineto{\pgfqpoint{5.827630in}{3.362583in}}%
\pgfpathlineto{\pgfqpoint{5.827914in}{3.392988in}}%
\pgfpathlineto{\pgfqpoint{5.828578in}{3.522947in}}%
\pgfpathlineto{\pgfqpoint{5.829146in}{3.432653in}}%
\pgfpathlineto{\pgfqpoint{5.830473in}{3.364178in}}%
\pgfpathlineto{\pgfqpoint{5.830662in}{3.356500in}}%
\pgfpathlineto{\pgfqpoint{5.830947in}{3.398649in}}%
\pgfpathlineto{\pgfqpoint{5.831705in}{3.531774in}}%
\pgfpathlineto{\pgfqpoint{5.832179in}{3.460454in}}%
\pgfpathlineto{\pgfqpoint{5.832842in}{3.365257in}}%
\pgfpathlineto{\pgfqpoint{5.833505in}{3.400158in}}%
\pgfpathlineto{\pgfqpoint{5.833790in}{3.370641in}}%
\pgfpathlineto{\pgfqpoint{5.834169in}{3.430387in}}%
\pgfpathlineto{\pgfqpoint{5.835116in}{3.906371in}}%
\pgfpathlineto{\pgfqpoint{5.835969in}{3.750424in}}%
\pgfpathlineto{\pgfqpoint{5.836064in}{3.748384in}}%
\pgfpathlineto{\pgfqpoint{5.836254in}{3.764808in}}%
\pgfpathlineto{\pgfqpoint{5.836348in}{3.770078in}}%
\pgfpathlineto{\pgfqpoint{5.836538in}{3.733138in}}%
\pgfpathlineto{\pgfqpoint{5.837296in}{3.346291in}}%
\pgfpathlineto{\pgfqpoint{5.837959in}{3.550887in}}%
\pgfpathlineto{\pgfqpoint{5.838338in}{3.488537in}}%
\pgfpathlineto{\pgfqpoint{5.838812in}{3.395176in}}%
\pgfpathlineto{\pgfqpoint{5.839475in}{3.456485in}}%
\pgfpathlineto{\pgfqpoint{5.840992in}{3.572098in}}%
\pgfpathlineto{\pgfqpoint{5.841181in}{3.559310in}}%
\pgfpathlineto{\pgfqpoint{5.841939in}{3.410171in}}%
\pgfpathlineto{\pgfqpoint{5.842508in}{3.496676in}}%
\pgfpathlineto{\pgfqpoint{5.843361in}{3.577965in}}%
\pgfpathlineto{\pgfqpoint{5.844024in}{3.575628in}}%
\pgfpathlineto{\pgfqpoint{5.844782in}{3.483045in}}%
\pgfpathlineto{\pgfqpoint{5.845161in}{3.431359in}}%
\pgfpathlineto{\pgfqpoint{5.845540in}{3.503978in}}%
\pgfpathlineto{\pgfqpoint{5.846204in}{3.626434in}}%
\pgfpathlineto{\pgfqpoint{5.846962in}{3.596058in}}%
\pgfpathlineto{\pgfqpoint{5.847815in}{3.489621in}}%
\pgfpathlineto{\pgfqpoint{5.848099in}{3.452336in}}%
\pgfpathlineto{\pgfqpoint{5.848573in}{3.534477in}}%
\pgfpathlineto{\pgfqpoint{5.849236in}{3.657544in}}%
\pgfpathlineto{\pgfqpoint{5.849900in}{3.602935in}}%
\pgfpathlineto{\pgfqpoint{5.851037in}{3.486432in}}%
\pgfpathlineto{\pgfqpoint{5.851605in}{3.550830in}}%
\pgfpathlineto{\pgfqpoint{5.852269in}{3.678549in}}%
\pgfpathlineto{\pgfqpoint{5.852932in}{3.600130in}}%
\pgfpathlineto{\pgfqpoint{5.854353in}{3.517498in}}%
\pgfpathlineto{\pgfqpoint{5.854448in}{3.526978in}}%
\pgfpathlineto{\pgfqpoint{5.855396in}{3.674512in}}%
\pgfpathlineto{\pgfqpoint{5.855870in}{3.608401in}}%
\pgfpathlineto{\pgfqpoint{5.857291in}{3.516939in}}%
\pgfpathlineto{\pgfqpoint{5.857386in}{3.517064in}}%
\pgfpathlineto{\pgfqpoint{5.857670in}{3.530527in}}%
\pgfpathlineto{\pgfqpoint{5.858428in}{3.658949in}}%
\pgfpathlineto{\pgfqpoint{5.858902in}{3.568745in}}%
\pgfpathlineto{\pgfqpoint{5.859565in}{3.458978in}}%
\pgfpathlineto{\pgfqpoint{5.860418in}{3.486461in}}%
\pgfpathlineto{\pgfqpoint{5.860703in}{3.476719in}}%
\pgfpathlineto{\pgfqpoint{5.860892in}{3.491346in}}%
\pgfpathlineto{\pgfqpoint{5.861461in}{3.583789in}}%
\pgfpathlineto{\pgfqpoint{5.862029in}{3.504925in}}%
\pgfpathlineto{\pgfqpoint{5.862598in}{3.397665in}}%
\pgfpathlineto{\pgfqpoint{5.863261in}{3.449687in}}%
\pgfpathlineto{\pgfqpoint{5.864588in}{3.559021in}}%
\pgfpathlineto{\pgfqpoint{5.864872in}{3.521575in}}%
\pgfpathlineto{\pgfqpoint{5.865725in}{3.373973in}}%
\pgfpathlineto{\pgfqpoint{5.866294in}{3.435199in}}%
\pgfpathlineto{\pgfqpoint{5.866389in}{3.435330in}}%
\pgfpathlineto{\pgfqpoint{5.867336in}{3.514074in}}%
\pgfpathlineto{\pgfqpoint{5.867620in}{3.542214in}}%
\pgfpathlineto{\pgfqpoint{5.868094in}{3.474509in}}%
\pgfpathlineto{\pgfqpoint{5.868758in}{3.338137in}}%
\pgfpathlineto{\pgfqpoint{5.869326in}{3.419458in}}%
\pgfpathlineto{\pgfqpoint{5.870653in}{3.483101in}}%
\pgfpathlineto{\pgfqpoint{5.870937in}{3.504789in}}%
\pgfpathlineto{\pgfqpoint{5.871316in}{3.459111in}}%
\pgfpathlineto{\pgfqpoint{5.871885in}{3.417256in}}%
\pgfpathlineto{\pgfqpoint{5.872264in}{3.475956in}}%
\pgfpathlineto{\pgfqpoint{5.872643in}{3.532120in}}%
\pgfpathlineto{\pgfqpoint{5.873401in}{3.506352in}}%
\pgfpathlineto{\pgfqpoint{5.874538in}{3.396749in}}%
\pgfpathlineto{\pgfqpoint{5.874823in}{3.365720in}}%
\pgfpathlineto{\pgfqpoint{5.875391in}{3.440686in}}%
\pgfpathlineto{\pgfqpoint{5.875960in}{3.537469in}}%
\pgfpathlineto{\pgfqpoint{5.876623in}{3.473253in}}%
\pgfpathlineto{\pgfqpoint{5.877002in}{3.475988in}}%
\pgfpathlineto{\pgfqpoint{5.877665in}{3.396506in}}%
\pgfpathlineto{\pgfqpoint{5.877855in}{3.382827in}}%
\pgfpathlineto{\pgfqpoint{5.878329in}{3.432164in}}%
\pgfpathlineto{\pgfqpoint{5.878992in}{3.551301in}}%
\pgfpathlineto{\pgfqpoint{5.879466in}{3.465068in}}%
\pgfpathlineto{\pgfqpoint{5.879655in}{3.443501in}}%
\pgfpathlineto{\pgfqpoint{5.880035in}{3.534997in}}%
\pgfpathlineto{\pgfqpoint{5.881930in}{3.908313in}}%
\pgfpathlineto{\pgfqpoint{5.882025in}{3.905268in}}%
\pgfpathlineto{\pgfqpoint{5.882688in}{3.399731in}}%
\pgfpathlineto{\pgfqpoint{5.882877in}{3.323594in}}%
\pgfpathlineto{\pgfqpoint{5.883541in}{3.471661in}}%
\pgfpathlineto{\pgfqpoint{5.883636in}{3.471016in}}%
\pgfpathlineto{\pgfqpoint{5.883730in}{3.470136in}}%
\pgfpathlineto{\pgfqpoint{5.884015in}{3.476906in}}%
\pgfpathlineto{\pgfqpoint{5.884583in}{3.538291in}}%
\pgfpathlineto{\pgfqpoint{5.885152in}{3.633195in}}%
\pgfpathlineto{\pgfqpoint{5.885720in}{3.556920in}}%
\pgfpathlineto{\pgfqpoint{5.886384in}{3.479529in}}%
\pgfpathlineto{\pgfqpoint{5.886952in}{3.520119in}}%
\pgfpathlineto{\pgfqpoint{5.887047in}{3.519960in}}%
\pgfpathlineto{\pgfqpoint{5.887237in}{3.521730in}}%
\pgfpathlineto{\pgfqpoint{5.887710in}{3.575824in}}%
\pgfpathlineto{\pgfqpoint{5.888279in}{3.646699in}}%
\pgfpathlineto{\pgfqpoint{5.888753in}{3.570289in}}%
\pgfpathlineto{\pgfqpoint{5.889321in}{3.471591in}}%
\pgfpathlineto{\pgfqpoint{5.889985in}{3.534474in}}%
\pgfpathlineto{\pgfqpoint{5.890174in}{3.530445in}}%
\pgfpathlineto{\pgfqpoint{5.890459in}{3.542904in}}%
\pgfpathlineto{\pgfqpoint{5.891406in}{3.641556in}}%
\pgfpathlineto{\pgfqpoint{5.891691in}{3.593443in}}%
\pgfpathlineto{\pgfqpoint{5.892354in}{3.460260in}}%
\pgfpathlineto{\pgfqpoint{5.892922in}{3.522247in}}%
\pgfpathlineto{\pgfqpoint{5.894344in}{3.642495in}}%
\pgfpathlineto{\pgfqpoint{5.894723in}{3.627372in}}%
\pgfpathlineto{\pgfqpoint{5.895386in}{3.465001in}}%
\pgfpathlineto{\pgfqpoint{5.896144in}{3.569465in}}%
\pgfpathlineto{\pgfqpoint{5.897471in}{3.649707in}}%
\pgfpathlineto{\pgfqpoint{5.897755in}{3.626348in}}%
\pgfpathlineto{\pgfqpoint{5.898514in}{3.475253in}}%
\pgfpathlineto{\pgfqpoint{5.899082in}{3.574307in}}%
\pgfpathlineto{\pgfqpoint{5.899745in}{3.640236in}}%
\pgfpathlineto{\pgfqpoint{5.900409in}{3.617349in}}%
\pgfpathlineto{\pgfqpoint{5.900598in}{3.620864in}}%
\pgfpathlineto{\pgfqpoint{5.900788in}{3.603598in}}%
\pgfpathlineto{\pgfqpoint{5.901641in}{3.485252in}}%
\pgfpathlineto{\pgfqpoint{5.902020in}{3.544204in}}%
\pgfpathlineto{\pgfqpoint{5.902778in}{3.639978in}}%
\pgfpathlineto{\pgfqpoint{5.903347in}{3.604912in}}%
\pgfpathlineto{\pgfqpoint{5.904673in}{3.461450in}}%
\pgfpathlineto{\pgfqpoint{5.905242in}{3.535491in}}%
\pgfpathlineto{\pgfqpoint{5.905716in}{3.648705in}}%
\pgfpathlineto{\pgfqpoint{5.906379in}{3.553696in}}%
\pgfpathlineto{\pgfqpoint{5.907800in}{3.460806in}}%
\pgfpathlineto{\pgfqpoint{5.907990in}{3.463848in}}%
\pgfpathlineto{\pgfqpoint{5.908748in}{3.609301in}}%
\pgfpathlineto{\pgfqpoint{5.909317in}{3.558535in}}%
\pgfusepath{stroke}%
\end{pgfscope}%
\begin{pgfscope}%
\pgfsetrectcap%
\pgfsetmiterjoin%
\pgfsetlinewidth{0.803000pt}%
\definecolor{currentstroke}{rgb}{0.000000,0.000000,0.000000}%
\pgfsetstrokecolor{currentstroke}%
\pgfsetdash{}{0pt}%
\pgfpathmoveto{\pgfqpoint{0.934300in}{2.889143in}}%
\pgfpathlineto{\pgfqpoint{0.934300in}{4.451359in}}%
\pgfusepath{stroke}%
\end{pgfscope}%
\begin{pgfscope}%
\pgfsetrectcap%
\pgfsetmiterjoin%
\pgfsetlinewidth{0.803000pt}%
\definecolor{currentstroke}{rgb}{0.000000,0.000000,0.000000}%
\pgfsetstrokecolor{currentstroke}%
\pgfsetdash{}{0pt}%
\pgfpathmoveto{\pgfqpoint{6.146222in}{2.889143in}}%
\pgfpathlineto{\pgfqpoint{6.146222in}{4.451359in}}%
\pgfusepath{stroke}%
\end{pgfscope}%
\begin{pgfscope}%
\pgfsetrectcap%
\pgfsetmiterjoin%
\pgfsetlinewidth{0.803000pt}%
\definecolor{currentstroke}{rgb}{0.000000,0.000000,0.000000}%
\pgfsetstrokecolor{currentstroke}%
\pgfsetdash{}{0pt}%
\pgfpathmoveto{\pgfqpoint{0.934300in}{2.889143in}}%
\pgfpathlineto{\pgfqpoint{6.146222in}{2.889143in}}%
\pgfusepath{stroke}%
\end{pgfscope}%
\begin{pgfscope}%
\pgfsetrectcap%
\pgfsetmiterjoin%
\pgfsetlinewidth{0.803000pt}%
\definecolor{currentstroke}{rgb}{0.000000,0.000000,0.000000}%
\pgfsetstrokecolor{currentstroke}%
\pgfsetdash{}{0pt}%
\pgfpathmoveto{\pgfqpoint{0.934300in}{4.451359in}}%
\pgfpathlineto{\pgfqpoint{6.146222in}{4.451359in}}%
\pgfusepath{stroke}%
\end{pgfscope}%
\begin{pgfscope}%
\definecolor{textcolor}{rgb}{0.000000,0.000000,0.000000}%
\pgfsetstrokecolor{textcolor}%
\pgfsetfillcolor{textcolor}%
\pgftext[x=3.540261in,y=4.534692in,,base]{\color{textcolor}\rmfamily\fontsize{12.000000}{14.400000}\selectfont Noisy ECG Signal}%
\end{pgfscope}%
\begin{pgfscope}%
\pgfsetbuttcap%
\pgfsetmiterjoin%
\definecolor{currentfill}{rgb}{1.000000,1.000000,1.000000}%
\pgfsetfillcolor{currentfill}%
\pgfsetlinewidth{0.000000pt}%
\definecolor{currentstroke}{rgb}{0.000000,0.000000,0.000000}%
\pgfsetstrokecolor{currentstroke}%
\pgfsetstrokeopacity{0.000000}%
\pgfsetdash{}{0pt}%
\pgfpathmoveto{\pgfqpoint{0.934300in}{0.564143in}}%
\pgfpathlineto{\pgfqpoint{6.146222in}{0.564143in}}%
\pgfpathlineto{\pgfqpoint{6.146222in}{2.126359in}}%
\pgfpathlineto{\pgfqpoint{0.934300in}{2.126359in}}%
\pgfpathclose%
\pgfusepath{fill}%
\end{pgfscope}%
\begin{pgfscope}%
\pgfsetbuttcap%
\pgfsetroundjoin%
\definecolor{currentfill}{rgb}{0.000000,0.000000,0.000000}%
\pgfsetfillcolor{currentfill}%
\pgfsetlinewidth{0.803000pt}%
\definecolor{currentstroke}{rgb}{0.000000,0.000000,0.000000}%
\pgfsetstrokecolor{currentstroke}%
\pgfsetdash{}{0pt}%
\pgfsys@defobject{currentmarker}{\pgfqpoint{0.000000in}{-0.048611in}}{\pgfqpoint{0.000000in}{0.000000in}}{%
\pgfpathmoveto{\pgfqpoint{0.000000in}{0.000000in}}%
\pgfpathlineto{\pgfqpoint{0.000000in}{-0.048611in}}%
\pgfusepath{stroke,fill}%
}%
\begin{pgfscope}%
\pgfsys@transformshift{0.934300in}{0.564143in}%
\pgfsys@useobject{currentmarker}{}%
\end{pgfscope}%
\end{pgfscope}%
\begin{pgfscope}%
\definecolor{textcolor}{rgb}{0.000000,0.000000,0.000000}%
\pgfsetstrokecolor{textcolor}%
\pgfsetfillcolor{textcolor}%
\pgftext[x=0.934300in,y=0.466921in,,top]{\color{textcolor}\rmfamily\fontsize{10.000000}{12.000000}\selectfont \(\displaystyle -100\)}%
\end{pgfscope}%
\begin{pgfscope}%
\pgfsetbuttcap%
\pgfsetroundjoin%
\definecolor{currentfill}{rgb}{0.000000,0.000000,0.000000}%
\pgfsetfillcolor{currentfill}%
\pgfsetlinewidth{0.803000pt}%
\definecolor{currentstroke}{rgb}{0.000000,0.000000,0.000000}%
\pgfsetstrokecolor{currentstroke}%
\pgfsetdash{}{0pt}%
\pgfsys@defobject{currentmarker}{\pgfqpoint{0.000000in}{-0.048611in}}{\pgfqpoint{0.000000in}{0.000000in}}{%
\pgfpathmoveto{\pgfqpoint{0.000000in}{0.000000in}}%
\pgfpathlineto{\pgfqpoint{0.000000in}{-0.048611in}}%
\pgfusepath{stroke,fill}%
}%
\begin{pgfscope}%
\pgfsys@transformshift{1.585791in}{0.564143in}%
\pgfsys@useobject{currentmarker}{}%
\end{pgfscope}%
\end{pgfscope}%
\begin{pgfscope}%
\definecolor{textcolor}{rgb}{0.000000,0.000000,0.000000}%
\pgfsetstrokecolor{textcolor}%
\pgfsetfillcolor{textcolor}%
\pgftext[x=1.585791in,y=0.466921in,,top]{\color{textcolor}\rmfamily\fontsize{10.000000}{12.000000}\selectfont \(\displaystyle -75\)}%
\end{pgfscope}%
\begin{pgfscope}%
\pgfsetbuttcap%
\pgfsetroundjoin%
\definecolor{currentfill}{rgb}{0.000000,0.000000,0.000000}%
\pgfsetfillcolor{currentfill}%
\pgfsetlinewidth{0.803000pt}%
\definecolor{currentstroke}{rgb}{0.000000,0.000000,0.000000}%
\pgfsetstrokecolor{currentstroke}%
\pgfsetdash{}{0pt}%
\pgfsys@defobject{currentmarker}{\pgfqpoint{0.000000in}{-0.048611in}}{\pgfqpoint{0.000000in}{0.000000in}}{%
\pgfpathmoveto{\pgfqpoint{0.000000in}{0.000000in}}%
\pgfpathlineto{\pgfqpoint{0.000000in}{-0.048611in}}%
\pgfusepath{stroke,fill}%
}%
\begin{pgfscope}%
\pgfsys@transformshift{2.237281in}{0.564143in}%
\pgfsys@useobject{currentmarker}{}%
\end{pgfscope}%
\end{pgfscope}%
\begin{pgfscope}%
\definecolor{textcolor}{rgb}{0.000000,0.000000,0.000000}%
\pgfsetstrokecolor{textcolor}%
\pgfsetfillcolor{textcolor}%
\pgftext[x=2.237281in,y=0.466921in,,top]{\color{textcolor}\rmfamily\fontsize{10.000000}{12.000000}\selectfont \(\displaystyle -50\)}%
\end{pgfscope}%
\begin{pgfscope}%
\pgfsetbuttcap%
\pgfsetroundjoin%
\definecolor{currentfill}{rgb}{0.000000,0.000000,0.000000}%
\pgfsetfillcolor{currentfill}%
\pgfsetlinewidth{0.803000pt}%
\definecolor{currentstroke}{rgb}{0.000000,0.000000,0.000000}%
\pgfsetstrokecolor{currentstroke}%
\pgfsetdash{}{0pt}%
\pgfsys@defobject{currentmarker}{\pgfqpoint{0.000000in}{-0.048611in}}{\pgfqpoint{0.000000in}{0.000000in}}{%
\pgfpathmoveto{\pgfqpoint{0.000000in}{0.000000in}}%
\pgfpathlineto{\pgfqpoint{0.000000in}{-0.048611in}}%
\pgfusepath{stroke,fill}%
}%
\begin{pgfscope}%
\pgfsys@transformshift{2.888771in}{0.564143in}%
\pgfsys@useobject{currentmarker}{}%
\end{pgfscope}%
\end{pgfscope}%
\begin{pgfscope}%
\definecolor{textcolor}{rgb}{0.000000,0.000000,0.000000}%
\pgfsetstrokecolor{textcolor}%
\pgfsetfillcolor{textcolor}%
\pgftext[x=2.888771in,y=0.466921in,,top]{\color{textcolor}\rmfamily\fontsize{10.000000}{12.000000}\selectfont \(\displaystyle -25\)}%
\end{pgfscope}%
\begin{pgfscope}%
\pgfsetbuttcap%
\pgfsetroundjoin%
\definecolor{currentfill}{rgb}{0.000000,0.000000,0.000000}%
\pgfsetfillcolor{currentfill}%
\pgfsetlinewidth{0.803000pt}%
\definecolor{currentstroke}{rgb}{0.000000,0.000000,0.000000}%
\pgfsetstrokecolor{currentstroke}%
\pgfsetdash{}{0pt}%
\pgfsys@defobject{currentmarker}{\pgfqpoint{0.000000in}{-0.048611in}}{\pgfqpoint{0.000000in}{0.000000in}}{%
\pgfpathmoveto{\pgfqpoint{0.000000in}{0.000000in}}%
\pgfpathlineto{\pgfqpoint{0.000000in}{-0.048611in}}%
\pgfusepath{stroke,fill}%
}%
\begin{pgfscope}%
\pgfsys@transformshift{3.540261in}{0.564143in}%
\pgfsys@useobject{currentmarker}{}%
\end{pgfscope}%
\end{pgfscope}%
\begin{pgfscope}%
\definecolor{textcolor}{rgb}{0.000000,0.000000,0.000000}%
\pgfsetstrokecolor{textcolor}%
\pgfsetfillcolor{textcolor}%
\pgftext[x=3.540261in,y=0.466921in,,top]{\color{textcolor}\rmfamily\fontsize{10.000000}{12.000000}\selectfont \(\displaystyle 0\)}%
\end{pgfscope}%
\begin{pgfscope}%
\pgfsetbuttcap%
\pgfsetroundjoin%
\definecolor{currentfill}{rgb}{0.000000,0.000000,0.000000}%
\pgfsetfillcolor{currentfill}%
\pgfsetlinewidth{0.803000pt}%
\definecolor{currentstroke}{rgb}{0.000000,0.000000,0.000000}%
\pgfsetstrokecolor{currentstroke}%
\pgfsetdash{}{0pt}%
\pgfsys@defobject{currentmarker}{\pgfqpoint{0.000000in}{-0.048611in}}{\pgfqpoint{0.000000in}{0.000000in}}{%
\pgfpathmoveto{\pgfqpoint{0.000000in}{0.000000in}}%
\pgfpathlineto{\pgfqpoint{0.000000in}{-0.048611in}}%
\pgfusepath{stroke,fill}%
}%
\begin{pgfscope}%
\pgfsys@transformshift{4.191751in}{0.564143in}%
\pgfsys@useobject{currentmarker}{}%
\end{pgfscope}%
\end{pgfscope}%
\begin{pgfscope}%
\definecolor{textcolor}{rgb}{0.000000,0.000000,0.000000}%
\pgfsetstrokecolor{textcolor}%
\pgfsetfillcolor{textcolor}%
\pgftext[x=4.191751in,y=0.466921in,,top]{\color{textcolor}\rmfamily\fontsize{10.000000}{12.000000}\selectfont \(\displaystyle 25\)}%
\end{pgfscope}%
\begin{pgfscope}%
\pgfsetbuttcap%
\pgfsetroundjoin%
\definecolor{currentfill}{rgb}{0.000000,0.000000,0.000000}%
\pgfsetfillcolor{currentfill}%
\pgfsetlinewidth{0.803000pt}%
\definecolor{currentstroke}{rgb}{0.000000,0.000000,0.000000}%
\pgfsetstrokecolor{currentstroke}%
\pgfsetdash{}{0pt}%
\pgfsys@defobject{currentmarker}{\pgfqpoint{0.000000in}{-0.048611in}}{\pgfqpoint{0.000000in}{0.000000in}}{%
\pgfpathmoveto{\pgfqpoint{0.000000in}{0.000000in}}%
\pgfpathlineto{\pgfqpoint{0.000000in}{-0.048611in}}%
\pgfusepath{stroke,fill}%
}%
\begin{pgfscope}%
\pgfsys@transformshift{4.843242in}{0.564143in}%
\pgfsys@useobject{currentmarker}{}%
\end{pgfscope}%
\end{pgfscope}%
\begin{pgfscope}%
\definecolor{textcolor}{rgb}{0.000000,0.000000,0.000000}%
\pgfsetstrokecolor{textcolor}%
\pgfsetfillcolor{textcolor}%
\pgftext[x=4.843242in,y=0.466921in,,top]{\color{textcolor}\rmfamily\fontsize{10.000000}{12.000000}\selectfont \(\displaystyle 50\)}%
\end{pgfscope}%
\begin{pgfscope}%
\pgfsetbuttcap%
\pgfsetroundjoin%
\definecolor{currentfill}{rgb}{0.000000,0.000000,0.000000}%
\pgfsetfillcolor{currentfill}%
\pgfsetlinewidth{0.803000pt}%
\definecolor{currentstroke}{rgb}{0.000000,0.000000,0.000000}%
\pgfsetstrokecolor{currentstroke}%
\pgfsetdash{}{0pt}%
\pgfsys@defobject{currentmarker}{\pgfqpoint{0.000000in}{-0.048611in}}{\pgfqpoint{0.000000in}{0.000000in}}{%
\pgfpathmoveto{\pgfqpoint{0.000000in}{0.000000in}}%
\pgfpathlineto{\pgfqpoint{0.000000in}{-0.048611in}}%
\pgfusepath{stroke,fill}%
}%
\begin{pgfscope}%
\pgfsys@transformshift{5.494732in}{0.564143in}%
\pgfsys@useobject{currentmarker}{}%
\end{pgfscope}%
\end{pgfscope}%
\begin{pgfscope}%
\definecolor{textcolor}{rgb}{0.000000,0.000000,0.000000}%
\pgfsetstrokecolor{textcolor}%
\pgfsetfillcolor{textcolor}%
\pgftext[x=5.494732in,y=0.466921in,,top]{\color{textcolor}\rmfamily\fontsize{10.000000}{12.000000}\selectfont \(\displaystyle 75\)}%
\end{pgfscope}%
\begin{pgfscope}%
\pgfsetbuttcap%
\pgfsetroundjoin%
\definecolor{currentfill}{rgb}{0.000000,0.000000,0.000000}%
\pgfsetfillcolor{currentfill}%
\pgfsetlinewidth{0.803000pt}%
\definecolor{currentstroke}{rgb}{0.000000,0.000000,0.000000}%
\pgfsetstrokecolor{currentstroke}%
\pgfsetdash{}{0pt}%
\pgfsys@defobject{currentmarker}{\pgfqpoint{0.000000in}{-0.048611in}}{\pgfqpoint{0.000000in}{0.000000in}}{%
\pgfpathmoveto{\pgfqpoint{0.000000in}{0.000000in}}%
\pgfpathlineto{\pgfqpoint{0.000000in}{-0.048611in}}%
\pgfusepath{stroke,fill}%
}%
\begin{pgfscope}%
\pgfsys@transformshift{6.146222in}{0.564143in}%
\pgfsys@useobject{currentmarker}{}%
\end{pgfscope}%
\end{pgfscope}%
\begin{pgfscope}%
\definecolor{textcolor}{rgb}{0.000000,0.000000,0.000000}%
\pgfsetstrokecolor{textcolor}%
\pgfsetfillcolor{textcolor}%
\pgftext[x=6.146222in,y=0.466921in,,top]{\color{textcolor}\rmfamily\fontsize{10.000000}{12.000000}\selectfont \(\displaystyle 100\)}%
\end{pgfscope}%
\begin{pgfscope}%
\definecolor{textcolor}{rgb}{0.000000,0.000000,0.000000}%
\pgfsetstrokecolor{textcolor}%
\pgfsetfillcolor{textcolor}%
\pgftext[x=3.540261in,y=0.287909in,,top]{\color{textcolor}\rmfamily\fontsize{10.000000}{12.000000}\selectfont Frequency (Hz)}%
\end{pgfscope}%
\begin{pgfscope}%
\pgfsetbuttcap%
\pgfsetroundjoin%
\definecolor{currentfill}{rgb}{0.000000,0.000000,0.000000}%
\pgfsetfillcolor{currentfill}%
\pgfsetlinewidth{0.803000pt}%
\definecolor{currentstroke}{rgb}{0.000000,0.000000,0.000000}%
\pgfsetstrokecolor{currentstroke}%
\pgfsetdash{}{0pt}%
\pgfsys@defobject{currentmarker}{\pgfqpoint{-0.048611in}{0.000000in}}{\pgfqpoint{0.000000in}{0.000000in}}{%
\pgfpathmoveto{\pgfqpoint{0.000000in}{0.000000in}}%
\pgfpathlineto{\pgfqpoint{-0.048611in}{0.000000in}}%
\pgfusepath{stroke,fill}%
}%
\begin{pgfscope}%
\pgfsys@transformshift{0.934300in}{0.635109in}%
\pgfsys@useobject{currentmarker}{}%
\end{pgfscope}%
\end{pgfscope}%
\begin{pgfscope}%
\definecolor{textcolor}{rgb}{0.000000,0.000000,0.000000}%
\pgfsetstrokecolor{textcolor}%
\pgfsetfillcolor{textcolor}%
\pgftext[x=0.767633in,y=0.586884in,left,base]{\color{textcolor}\rmfamily\fontsize{10.000000}{12.000000}\selectfont \(\displaystyle 0\)}%
\end{pgfscope}%
\begin{pgfscope}%
\pgfsetbuttcap%
\pgfsetroundjoin%
\definecolor{currentfill}{rgb}{0.000000,0.000000,0.000000}%
\pgfsetfillcolor{currentfill}%
\pgfsetlinewidth{0.803000pt}%
\definecolor{currentstroke}{rgb}{0.000000,0.000000,0.000000}%
\pgfsetstrokecolor{currentstroke}%
\pgfsetdash{}{0pt}%
\pgfsys@defobject{currentmarker}{\pgfqpoint{-0.048611in}{0.000000in}}{\pgfqpoint{0.000000in}{0.000000in}}{%
\pgfpathmoveto{\pgfqpoint{0.000000in}{0.000000in}}%
\pgfpathlineto{\pgfqpoint{-0.048611in}{0.000000in}}%
\pgfusepath{stroke,fill}%
}%
\begin{pgfscope}%
\pgfsys@transformshift{0.934300in}{1.246702in}%
\pgfsys@useobject{currentmarker}{}%
\end{pgfscope}%
\end{pgfscope}%
\begin{pgfscope}%
\definecolor{textcolor}{rgb}{0.000000,0.000000,0.000000}%
\pgfsetstrokecolor{textcolor}%
\pgfsetfillcolor{textcolor}%
\pgftext[x=0.350966in,y=1.198476in,left,base]{\color{textcolor}\rmfamily\fontsize{10.000000}{12.000000}\selectfont \(\displaystyle 1000000\)}%
\end{pgfscope}%
\begin{pgfscope}%
\pgfsetbuttcap%
\pgfsetroundjoin%
\definecolor{currentfill}{rgb}{0.000000,0.000000,0.000000}%
\pgfsetfillcolor{currentfill}%
\pgfsetlinewidth{0.803000pt}%
\definecolor{currentstroke}{rgb}{0.000000,0.000000,0.000000}%
\pgfsetstrokecolor{currentstroke}%
\pgfsetdash{}{0pt}%
\pgfsys@defobject{currentmarker}{\pgfqpoint{-0.048611in}{0.000000in}}{\pgfqpoint{0.000000in}{0.000000in}}{%
\pgfpathmoveto{\pgfqpoint{0.000000in}{0.000000in}}%
\pgfpathlineto{\pgfqpoint{-0.048611in}{0.000000in}}%
\pgfusepath{stroke,fill}%
}%
\begin{pgfscope}%
\pgfsys@transformshift{0.934300in}{1.858294in}%
\pgfsys@useobject{currentmarker}{}%
\end{pgfscope}%
\end{pgfscope}%
\begin{pgfscope}%
\definecolor{textcolor}{rgb}{0.000000,0.000000,0.000000}%
\pgfsetstrokecolor{textcolor}%
\pgfsetfillcolor{textcolor}%
\pgftext[x=0.350966in,y=1.810069in,left,base]{\color{textcolor}\rmfamily\fontsize{10.000000}{12.000000}\selectfont \(\displaystyle 2000000\)}%
\end{pgfscope}%
\begin{pgfscope}%
\definecolor{textcolor}{rgb}{0.000000,0.000000,0.000000}%
\pgfsetstrokecolor{textcolor}%
\pgfsetfillcolor{textcolor}%
\pgftext[x=0.295410in,y=1.345251in,,bottom,rotate=90.000000]{\color{textcolor}\rmfamily\fontsize{10.000000}{12.000000}\selectfont abs(X(f)) (\(\displaystyle \mu V^2\))}%
\end{pgfscope}%
\begin{pgfscope}%
\pgfpathrectangle{\pgfqpoint{0.934300in}{0.564143in}}{\pgfqpoint{5.211922in}{1.562215in}}%
\pgfusepath{clip}%
\pgfsetrectcap%
\pgfsetroundjoin%
\pgfsetlinewidth{1.505625pt}%
\definecolor{currentstroke}{rgb}{0.121569,0.466667,0.705882}%
\pgfsetstrokecolor{currentstroke}%
\pgfsetdash{}{0pt}%
\pgfpathmoveto{\pgfqpoint{3.540261in}{0.692992in}}%
\pgfpathlineto{\pgfqpoint{3.540795in}{0.681478in}}%
\pgfpathlineto{\pgfqpoint{3.541329in}{0.757387in}}%
\pgfpathlineto{\pgfqpoint{3.541862in}{0.748295in}}%
\pgfpathlineto{\pgfqpoint{3.542396in}{0.693540in}}%
\pgfpathlineto{\pgfqpoint{3.542930in}{0.910344in}}%
\pgfpathlineto{\pgfqpoint{3.543463in}{0.793365in}}%
\pgfpathlineto{\pgfqpoint{3.543997in}{0.679389in}}%
\pgfpathlineto{\pgfqpoint{3.545065in}{1.221637in}}%
\pgfpathlineto{\pgfqpoint{3.545598in}{1.079498in}}%
\pgfpathlineto{\pgfqpoint{3.546132in}{0.660007in}}%
\pgfpathlineto{\pgfqpoint{3.548267in}{1.733996in}}%
\pgfpathlineto{\pgfqpoint{3.548800in}{2.055349in}}%
\pgfpathlineto{\pgfqpoint{3.549334in}{0.897930in}}%
\pgfpathlineto{\pgfqpoint{3.549868in}{0.996736in}}%
\pgfpathlineto{\pgfqpoint{3.550402in}{1.662308in}}%
\pgfpathlineto{\pgfqpoint{3.550935in}{1.159759in}}%
\pgfpathlineto{\pgfqpoint{3.552536in}{0.948857in}}%
\pgfpathlineto{\pgfqpoint{3.553070in}{0.711344in}}%
\pgfpathlineto{\pgfqpoint{3.553604in}{0.854957in}}%
\pgfpathlineto{\pgfqpoint{3.554671in}{1.216794in}}%
\pgfpathlineto{\pgfqpoint{3.555205in}{1.177348in}}%
\pgfpathlineto{\pgfqpoint{3.555739in}{0.701277in}}%
\pgfpathlineto{\pgfqpoint{3.556272in}{0.836953in}}%
\pgfpathlineto{\pgfqpoint{3.557340in}{0.826055in}}%
\pgfpathlineto{\pgfqpoint{3.557873in}{1.167222in}}%
\pgfpathlineto{\pgfqpoint{3.558941in}{0.829117in}}%
\pgfpathlineto{\pgfqpoint{3.559474in}{0.838546in}}%
\pgfpathlineto{\pgfqpoint{3.560008in}{1.173789in}}%
\pgfpathlineto{\pgfqpoint{3.560542in}{0.930970in}}%
\pgfpathlineto{\pgfqpoint{3.561609in}{0.833676in}}%
\pgfpathlineto{\pgfqpoint{3.562143in}{1.077489in}}%
\pgfpathlineto{\pgfqpoint{3.562677in}{0.669806in}}%
\pgfpathlineto{\pgfqpoint{3.563210in}{0.737341in}}%
\pgfpathlineto{\pgfqpoint{3.565879in}{1.279587in}}%
\pgfpathlineto{\pgfqpoint{3.567480in}{0.906588in}}%
\pgfpathlineto{\pgfqpoint{3.568014in}{1.005451in}}%
\pgfpathlineto{\pgfqpoint{3.568547in}{0.743178in}}%
\pgfpathlineto{\pgfqpoint{3.569081in}{0.781352in}}%
\pgfpathlineto{\pgfqpoint{3.569615in}{0.876048in}}%
\pgfpathlineto{\pgfqpoint{3.570149in}{0.859217in}}%
\pgfpathlineto{\pgfqpoint{3.570682in}{0.793772in}}%
\pgfpathlineto{\pgfqpoint{3.571216in}{0.863504in}}%
\pgfpathlineto{\pgfqpoint{3.571750in}{0.861028in}}%
\pgfpathlineto{\pgfqpoint{3.572817in}{0.712356in}}%
\pgfpathlineto{\pgfqpoint{3.573351in}{0.731172in}}%
\pgfpathlineto{\pgfqpoint{3.573884in}{0.711654in}}%
\pgfpathlineto{\pgfqpoint{3.574418in}{0.727648in}}%
\pgfpathlineto{\pgfqpoint{3.574952in}{0.737995in}}%
\pgfpathlineto{\pgfqpoint{3.575486in}{0.805059in}}%
\pgfpathlineto{\pgfqpoint{3.576019in}{0.666013in}}%
\pgfpathlineto{\pgfqpoint{3.576553in}{0.720688in}}%
\pgfpathlineto{\pgfqpoint{3.577087in}{0.967087in}}%
\pgfpathlineto{\pgfqpoint{3.577620in}{0.866458in}}%
\pgfpathlineto{\pgfqpoint{3.578154in}{0.938380in}}%
\pgfpathlineto{\pgfqpoint{3.578688in}{0.879471in}}%
\pgfpathlineto{\pgfqpoint{3.579221in}{0.856511in}}%
\pgfpathlineto{\pgfqpoint{3.579755in}{0.888958in}}%
\pgfpathlineto{\pgfqpoint{3.580289in}{0.701780in}}%
\pgfpathlineto{\pgfqpoint{3.580823in}{0.934572in}}%
\pgfpathlineto{\pgfqpoint{3.581356in}{0.857373in}}%
\pgfpathlineto{\pgfqpoint{3.582424in}{0.925650in}}%
\pgfpathlineto{\pgfqpoint{3.582957in}{0.675064in}}%
\pgfpathlineto{\pgfqpoint{3.584025in}{0.677552in}}%
\pgfpathlineto{\pgfqpoint{3.585092in}{0.783163in}}%
\pgfpathlineto{\pgfqpoint{3.586693in}{0.673145in}}%
\pgfpathlineto{\pgfqpoint{3.587227in}{0.815423in}}%
\pgfpathlineto{\pgfqpoint{3.587761in}{0.786723in}}%
\pgfpathlineto{\pgfqpoint{3.589362in}{0.696863in}}%
\pgfpathlineto{\pgfqpoint{3.589895in}{0.772274in}}%
\pgfpathlineto{\pgfqpoint{3.590429in}{0.727488in}}%
\pgfpathlineto{\pgfqpoint{3.591497in}{0.653176in}}%
\pgfpathlineto{\pgfqpoint{3.592030in}{0.874099in}}%
\pgfpathlineto{\pgfqpoint{3.593098in}{0.861548in}}%
\pgfpathlineto{\pgfqpoint{3.593631in}{0.674808in}}%
\pgfpathlineto{\pgfqpoint{3.594165in}{0.841183in}}%
\pgfpathlineto{\pgfqpoint{3.594699in}{0.761392in}}%
\pgfpathlineto{\pgfqpoint{3.595232in}{1.352518in}}%
\pgfpathlineto{\pgfqpoint{3.595766in}{1.207660in}}%
\pgfpathlineto{\pgfqpoint{3.596834in}{0.943916in}}%
\pgfpathlineto{\pgfqpoint{3.597367in}{1.040537in}}%
\pgfpathlineto{\pgfqpoint{3.598435in}{0.706986in}}%
\pgfpathlineto{\pgfqpoint{3.598968in}{0.822940in}}%
\pgfpathlineto{\pgfqpoint{3.599502in}{0.679641in}}%
\pgfpathlineto{\pgfqpoint{3.600036in}{0.816524in}}%
\pgfpathlineto{\pgfqpoint{3.600569in}{0.831190in}}%
\pgfpathlineto{\pgfqpoint{3.601103in}{0.785290in}}%
\pgfpathlineto{\pgfqpoint{3.601637in}{0.830879in}}%
\pgfpathlineto{\pgfqpoint{3.602171in}{0.824117in}}%
\pgfpathlineto{\pgfqpoint{3.602704in}{1.070616in}}%
\pgfpathlineto{\pgfqpoint{3.603238in}{0.675217in}}%
\pgfpathlineto{\pgfqpoint{3.603772in}{0.965560in}}%
\pgfpathlineto{\pgfqpoint{3.604305in}{0.777734in}}%
\pgfpathlineto{\pgfqpoint{3.604839in}{0.869342in}}%
\pgfpathlineto{\pgfqpoint{3.605373in}{1.009309in}}%
\pgfpathlineto{\pgfqpoint{3.605906in}{0.811315in}}%
\pgfpathlineto{\pgfqpoint{3.607508in}{1.199479in}}%
\pgfpathlineto{\pgfqpoint{3.608041in}{0.832937in}}%
\pgfpathlineto{\pgfqpoint{3.608575in}{1.179134in}}%
\pgfpathlineto{\pgfqpoint{3.609109in}{0.929724in}}%
\pgfpathlineto{\pgfqpoint{3.609642in}{0.984101in}}%
\pgfpathlineto{\pgfqpoint{3.610176in}{1.005663in}}%
\pgfpathlineto{\pgfqpoint{3.610710in}{0.871213in}}%
\pgfpathlineto{\pgfqpoint{3.612311in}{1.181330in}}%
\pgfpathlineto{\pgfqpoint{3.614446in}{0.896023in}}%
\pgfpathlineto{\pgfqpoint{3.614979in}{0.852655in}}%
\pgfpathlineto{\pgfqpoint{3.615513in}{0.692226in}}%
\pgfpathlineto{\pgfqpoint{3.616047in}{0.944442in}}%
\pgfpathlineto{\pgfqpoint{3.616580in}{0.739121in}}%
\pgfpathlineto{\pgfqpoint{3.617114in}{0.754372in}}%
\pgfpathlineto{\pgfqpoint{3.618715in}{0.705081in}}%
\pgfpathlineto{\pgfqpoint{3.619249in}{0.770886in}}%
\pgfpathlineto{\pgfqpoint{3.619783in}{0.744665in}}%
\pgfpathlineto{\pgfqpoint{3.620316in}{0.656309in}}%
\pgfpathlineto{\pgfqpoint{3.620850in}{0.743243in}}%
\pgfpathlineto{\pgfqpoint{3.621384in}{0.774322in}}%
\pgfpathlineto{\pgfqpoint{3.622451in}{0.695860in}}%
\pgfpathlineto{\pgfqpoint{3.622985in}{0.704906in}}%
\pgfpathlineto{\pgfqpoint{3.623519in}{0.748307in}}%
\pgfpathlineto{\pgfqpoint{3.625120in}{0.651300in}}%
\pgfpathlineto{\pgfqpoint{3.625653in}{0.676906in}}%
\pgfpathlineto{\pgfqpoint{3.626187in}{0.769855in}}%
\pgfpathlineto{\pgfqpoint{3.626721in}{0.745730in}}%
\pgfpathlineto{\pgfqpoint{3.627254in}{0.658667in}}%
\pgfpathlineto{\pgfqpoint{3.627788in}{0.671881in}}%
\pgfpathlineto{\pgfqpoint{3.628856in}{0.716081in}}%
\pgfpathlineto{\pgfqpoint{3.629389in}{0.697692in}}%
\pgfpathlineto{\pgfqpoint{3.629923in}{0.690101in}}%
\pgfpathlineto{\pgfqpoint{3.630457in}{0.644547in}}%
\pgfpathlineto{\pgfqpoint{3.630990in}{0.686691in}}%
\pgfpathlineto{\pgfqpoint{3.631524in}{0.692393in}}%
\pgfpathlineto{\pgfqpoint{3.632058in}{0.761942in}}%
\pgfpathlineto{\pgfqpoint{3.632592in}{0.664871in}}%
\pgfpathlineto{\pgfqpoint{3.633125in}{0.719205in}}%
\pgfpathlineto{\pgfqpoint{3.634193in}{0.703231in}}%
\pgfpathlineto{\pgfqpoint{3.635260in}{0.752633in}}%
\pgfpathlineto{\pgfqpoint{3.635794in}{0.715438in}}%
\pgfpathlineto{\pgfqpoint{3.636327in}{0.754744in}}%
\pgfpathlineto{\pgfqpoint{3.636861in}{0.791057in}}%
\pgfpathlineto{\pgfqpoint{3.638462in}{0.679622in}}%
\pgfpathlineto{\pgfqpoint{3.638996in}{0.647526in}}%
\pgfpathlineto{\pgfqpoint{3.641131in}{0.795147in}}%
\pgfpathlineto{\pgfqpoint{3.642732in}{0.725464in}}%
\pgfpathlineto{\pgfqpoint{3.643266in}{0.743619in}}%
\pgfpathlineto{\pgfqpoint{3.643799in}{0.739247in}}%
\pgfpathlineto{\pgfqpoint{3.644333in}{0.745188in}}%
\pgfpathlineto{\pgfqpoint{3.645400in}{0.707580in}}%
\pgfpathlineto{\pgfqpoint{3.645934in}{0.754848in}}%
\pgfpathlineto{\pgfqpoint{3.646468in}{0.681107in}}%
\pgfpathlineto{\pgfqpoint{3.647001in}{0.763971in}}%
\pgfpathlineto{\pgfqpoint{3.647535in}{0.760962in}}%
\pgfpathlineto{\pgfqpoint{3.648603in}{0.706833in}}%
\pgfpathlineto{\pgfqpoint{3.649136in}{0.746908in}}%
\pgfpathlineto{\pgfqpoint{3.649670in}{0.686064in}}%
\pgfpathlineto{\pgfqpoint{3.650204in}{0.755478in}}%
\pgfpathlineto{\pgfqpoint{3.650737in}{1.052081in}}%
\pgfpathlineto{\pgfqpoint{3.651271in}{0.737152in}}%
\pgfpathlineto{\pgfqpoint{3.651805in}{0.811558in}}%
\pgfpathlineto{\pgfqpoint{3.652338in}{0.779759in}}%
\pgfpathlineto{\pgfqpoint{3.652872in}{0.751124in}}%
\pgfpathlineto{\pgfqpoint{3.653406in}{0.756138in}}%
\pgfpathlineto{\pgfqpoint{3.654473in}{0.791317in}}%
\pgfpathlineto{\pgfqpoint{3.655007in}{0.734562in}}%
\pgfpathlineto{\pgfqpoint{3.655541in}{0.759390in}}%
\pgfpathlineto{\pgfqpoint{3.658743in}{0.652424in}}%
\pgfpathlineto{\pgfqpoint{3.660344in}{0.696364in}}%
\pgfpathlineto{\pgfqpoint{3.660878in}{0.705428in}}%
\pgfpathlineto{\pgfqpoint{3.661945in}{0.660161in}}%
\pgfpathlineto{\pgfqpoint{3.662479in}{0.699980in}}%
\pgfpathlineto{\pgfqpoint{3.663012in}{0.670799in}}%
\pgfpathlineto{\pgfqpoint{3.664080in}{0.721186in}}%
\pgfpathlineto{\pgfqpoint{3.664614in}{0.669506in}}%
\pgfpathlineto{\pgfqpoint{3.665147in}{0.698793in}}%
\pgfpathlineto{\pgfqpoint{3.667816in}{0.661738in}}%
\pgfpathlineto{\pgfqpoint{3.668349in}{0.707963in}}%
\pgfpathlineto{\pgfqpoint{3.668883in}{0.679144in}}%
\pgfpathlineto{\pgfqpoint{3.669417in}{0.687390in}}%
\pgfpathlineto{\pgfqpoint{3.669951in}{0.677219in}}%
\pgfpathlineto{\pgfqpoint{3.670484in}{0.740566in}}%
\pgfpathlineto{\pgfqpoint{3.671018in}{0.657850in}}%
\pgfpathlineto{\pgfqpoint{3.671552in}{0.711734in}}%
\pgfpathlineto{\pgfqpoint{3.673153in}{0.645571in}}%
\pgfpathlineto{\pgfqpoint{3.675288in}{0.749810in}}%
\pgfpathlineto{\pgfqpoint{3.675821in}{0.700628in}}%
\pgfpathlineto{\pgfqpoint{3.676355in}{0.778380in}}%
\pgfpathlineto{\pgfqpoint{3.676889in}{0.691742in}}%
\pgfpathlineto{\pgfqpoint{3.677422in}{0.720960in}}%
\pgfpathlineto{\pgfqpoint{3.679023in}{0.668473in}}%
\pgfpathlineto{\pgfqpoint{3.679557in}{0.673688in}}%
\pgfpathlineto{\pgfqpoint{3.681158in}{0.733251in}}%
\pgfpathlineto{\pgfqpoint{3.681692in}{0.658921in}}%
\pgfpathlineto{\pgfqpoint{3.682226in}{0.737161in}}%
\pgfpathlineto{\pgfqpoint{3.684360in}{0.640059in}}%
\pgfpathlineto{\pgfqpoint{3.685962in}{0.696829in}}%
\pgfpathlineto{\pgfqpoint{3.686495in}{0.644220in}}%
\pgfpathlineto{\pgfqpoint{3.687029in}{0.650388in}}%
\pgfpathlineto{\pgfqpoint{3.688096in}{0.678474in}}%
\pgfpathlineto{\pgfqpoint{3.688630in}{0.650952in}}%
\pgfpathlineto{\pgfqpoint{3.689164in}{0.656273in}}%
\pgfpathlineto{\pgfqpoint{3.690231in}{0.666447in}}%
\pgfpathlineto{\pgfqpoint{3.690765in}{0.685026in}}%
\pgfpathlineto{\pgfqpoint{3.691299in}{0.641493in}}%
\pgfpathlineto{\pgfqpoint{3.691832in}{0.668543in}}%
\pgfpathlineto{\pgfqpoint{3.692900in}{0.678972in}}%
\pgfpathlineto{\pgfqpoint{3.693433in}{0.706379in}}%
\pgfpathlineto{\pgfqpoint{3.693967in}{0.686487in}}%
\pgfpathlineto{\pgfqpoint{3.694501in}{0.641835in}}%
\pgfpathlineto{\pgfqpoint{3.695568in}{0.707831in}}%
\pgfpathlineto{\pgfqpoint{3.696102in}{0.652775in}}%
\pgfpathlineto{\pgfqpoint{3.697169in}{0.653762in}}%
\pgfpathlineto{\pgfqpoint{3.697703in}{0.653992in}}%
\pgfpathlineto{\pgfqpoint{3.699304in}{0.682818in}}%
\pgfpathlineto{\pgfqpoint{3.699838in}{0.670972in}}%
\pgfpathlineto{\pgfqpoint{3.700372in}{0.773743in}}%
\pgfpathlineto{\pgfqpoint{3.700905in}{0.642160in}}%
\pgfpathlineto{\pgfqpoint{3.701439in}{0.731605in}}%
\pgfpathlineto{\pgfqpoint{3.701973in}{0.662495in}}%
\pgfpathlineto{\pgfqpoint{3.702506in}{0.681888in}}%
\pgfpathlineto{\pgfqpoint{3.703040in}{0.697389in}}%
\pgfpathlineto{\pgfqpoint{3.703574in}{0.925873in}}%
\pgfpathlineto{\pgfqpoint{3.704107in}{0.734617in}}%
\pgfpathlineto{\pgfqpoint{3.705175in}{0.815785in}}%
\pgfpathlineto{\pgfqpoint{3.705709in}{1.178039in}}%
\pgfpathlineto{\pgfqpoint{3.706242in}{0.865274in}}%
\pgfpathlineto{\pgfqpoint{3.707310in}{0.688765in}}%
\pgfpathlineto{\pgfqpoint{3.707843in}{0.901801in}}%
\pgfpathlineto{\pgfqpoint{3.708377in}{0.786005in}}%
\pgfpathlineto{\pgfqpoint{3.708911in}{0.828383in}}%
\pgfpathlineto{\pgfqpoint{3.709444in}{0.812282in}}%
\pgfpathlineto{\pgfqpoint{3.711046in}{0.708808in}}%
\pgfpathlineto{\pgfqpoint{3.712647in}{0.890035in}}%
\pgfpathlineto{\pgfqpoint{3.714781in}{0.701937in}}%
\pgfpathlineto{\pgfqpoint{3.715315in}{0.712678in}}%
\pgfpathlineto{\pgfqpoint{3.716916in}{0.679607in}}%
\pgfpathlineto{\pgfqpoint{3.717450in}{0.692561in}}%
\pgfpathlineto{\pgfqpoint{3.717984in}{0.684604in}}%
\pgfpathlineto{\pgfqpoint{3.719585in}{0.647653in}}%
\pgfpathlineto{\pgfqpoint{3.721186in}{0.678788in}}%
\pgfpathlineto{\pgfqpoint{3.722253in}{0.642592in}}%
\pgfpathlineto{\pgfqpoint{3.722787in}{0.650700in}}%
\pgfpathlineto{\pgfqpoint{3.723321in}{0.676689in}}%
\pgfpathlineto{\pgfqpoint{3.723854in}{0.645629in}}%
\pgfpathlineto{\pgfqpoint{3.724388in}{0.671551in}}%
\pgfpathlineto{\pgfqpoint{3.724922in}{0.701345in}}%
\pgfpathlineto{\pgfqpoint{3.725455in}{0.646356in}}%
\pgfpathlineto{\pgfqpoint{3.725989in}{0.665834in}}%
\pgfpathlineto{\pgfqpoint{3.727057in}{0.651350in}}%
\pgfpathlineto{\pgfqpoint{3.727590in}{0.671232in}}%
\pgfpathlineto{\pgfqpoint{3.728124in}{0.647011in}}%
\pgfpathlineto{\pgfqpoint{3.728658in}{0.659727in}}%
\pgfpathlineto{\pgfqpoint{3.729191in}{0.657238in}}%
\pgfpathlineto{\pgfqpoint{3.729725in}{0.666396in}}%
\pgfpathlineto{\pgfqpoint{3.730792in}{0.649301in}}%
\pgfpathlineto{\pgfqpoint{3.731326in}{0.692439in}}%
\pgfpathlineto{\pgfqpoint{3.731860in}{0.655457in}}%
\pgfpathlineto{\pgfqpoint{3.732394in}{0.660166in}}%
\pgfpathlineto{\pgfqpoint{3.732927in}{0.654851in}}%
\pgfpathlineto{\pgfqpoint{3.733461in}{0.709786in}}%
\pgfpathlineto{\pgfqpoint{3.733995in}{0.692286in}}%
\pgfpathlineto{\pgfqpoint{3.736129in}{0.652390in}}%
\pgfpathlineto{\pgfqpoint{3.736663in}{0.677319in}}%
\pgfpathlineto{\pgfqpoint{3.737197in}{0.668708in}}%
\pgfpathlineto{\pgfqpoint{3.737731in}{0.667709in}}%
\pgfpathlineto{\pgfqpoint{3.738264in}{0.680156in}}%
\pgfpathlineto{\pgfqpoint{3.738798in}{0.645439in}}%
\pgfpathlineto{\pgfqpoint{3.739332in}{0.653052in}}%
\pgfpathlineto{\pgfqpoint{3.741466in}{0.674375in}}%
\pgfpathlineto{\pgfqpoint{3.743068in}{0.650566in}}%
\pgfpathlineto{\pgfqpoint{3.743601in}{0.661463in}}%
\pgfpathlineto{\pgfqpoint{3.744135in}{0.646360in}}%
\pgfpathlineto{\pgfqpoint{3.744669in}{0.646898in}}%
\pgfpathlineto{\pgfqpoint{3.745736in}{0.692920in}}%
\pgfpathlineto{\pgfqpoint{3.746270in}{0.670885in}}%
\pgfpathlineto{\pgfqpoint{3.748938in}{0.718236in}}%
\pgfpathlineto{\pgfqpoint{3.751073in}{0.651066in}}%
\pgfpathlineto{\pgfqpoint{3.751607in}{0.682376in}}%
\pgfpathlineto{\pgfqpoint{3.752140in}{0.675159in}}%
\pgfpathlineto{\pgfqpoint{3.752674in}{0.650170in}}%
\pgfpathlineto{\pgfqpoint{3.754275in}{0.706507in}}%
\pgfpathlineto{\pgfqpoint{3.754809in}{0.685188in}}%
\pgfpathlineto{\pgfqpoint{3.755343in}{0.757550in}}%
\pgfpathlineto{\pgfqpoint{3.755876in}{0.650935in}}%
\pgfpathlineto{\pgfqpoint{3.756410in}{0.745829in}}%
\pgfpathlineto{\pgfqpoint{3.756944in}{0.663508in}}%
\pgfpathlineto{\pgfqpoint{3.758545in}{1.024803in}}%
\pgfpathlineto{\pgfqpoint{3.759079in}{0.740294in}}%
\pgfpathlineto{\pgfqpoint{3.759612in}{0.790533in}}%
\pgfpathlineto{\pgfqpoint{3.760146in}{0.765002in}}%
\pgfpathlineto{\pgfqpoint{3.760680in}{0.951717in}}%
\pgfpathlineto{\pgfqpoint{3.761213in}{0.941365in}}%
\pgfpathlineto{\pgfqpoint{3.761747in}{0.702648in}}%
\pgfpathlineto{\pgfqpoint{3.762281in}{0.884028in}}%
\pgfpathlineto{\pgfqpoint{3.762815in}{0.997517in}}%
\pgfpathlineto{\pgfqpoint{3.763348in}{0.903362in}}%
\pgfpathlineto{\pgfqpoint{3.764416in}{0.754725in}}%
\pgfpathlineto{\pgfqpoint{3.764949in}{0.815808in}}%
\pgfpathlineto{\pgfqpoint{3.766017in}{0.704604in}}%
\pgfpathlineto{\pgfqpoint{3.766550in}{0.764894in}}%
\pgfpathlineto{\pgfqpoint{3.767084in}{0.823257in}}%
\pgfpathlineto{\pgfqpoint{3.767618in}{0.767881in}}%
\pgfpathlineto{\pgfqpoint{3.768152in}{0.660200in}}%
\pgfpathlineto{\pgfqpoint{3.768685in}{0.768806in}}%
\pgfpathlineto{\pgfqpoint{3.769219in}{0.761226in}}%
\pgfpathlineto{\pgfqpoint{3.770286in}{0.887250in}}%
\pgfpathlineto{\pgfqpoint{3.770820in}{0.862549in}}%
\pgfpathlineto{\pgfqpoint{3.774022in}{0.663350in}}%
\pgfpathlineto{\pgfqpoint{3.774556in}{0.648463in}}%
\pgfpathlineto{\pgfqpoint{3.775090in}{0.658860in}}%
\pgfpathlineto{\pgfqpoint{3.775623in}{0.682837in}}%
\pgfpathlineto{\pgfqpoint{3.776157in}{0.648469in}}%
\pgfpathlineto{\pgfqpoint{3.776691in}{0.668154in}}%
\pgfpathlineto{\pgfqpoint{3.777224in}{0.662620in}}%
\pgfpathlineto{\pgfqpoint{3.777758in}{0.670822in}}%
\pgfpathlineto{\pgfqpoint{3.778292in}{0.638061in}}%
\pgfpathlineto{\pgfqpoint{3.778826in}{0.671911in}}%
\pgfpathlineto{\pgfqpoint{3.780427in}{0.643287in}}%
\pgfpathlineto{\pgfqpoint{3.781494in}{0.679540in}}%
\pgfpathlineto{\pgfqpoint{3.782028in}{0.659037in}}%
\pgfpathlineto{\pgfqpoint{3.783095in}{0.672045in}}%
\pgfpathlineto{\pgfqpoint{3.783629in}{0.644517in}}%
\pgfpathlineto{\pgfqpoint{3.784163in}{0.668174in}}%
\pgfpathlineto{\pgfqpoint{3.785230in}{0.661061in}}%
\pgfpathlineto{\pgfqpoint{3.785764in}{0.684189in}}%
\pgfpathlineto{\pgfqpoint{3.786297in}{0.664805in}}%
\pgfpathlineto{\pgfqpoint{3.786831in}{0.669225in}}%
\pgfpathlineto{\pgfqpoint{3.788966in}{0.644689in}}%
\pgfpathlineto{\pgfqpoint{3.789500in}{0.693007in}}%
\pgfpathlineto{\pgfqpoint{3.790033in}{0.646004in}}%
\pgfpathlineto{\pgfqpoint{3.790567in}{0.651769in}}%
\pgfpathlineto{\pgfqpoint{3.791101in}{0.672723in}}%
\pgfpathlineto{\pgfqpoint{3.791634in}{0.642856in}}%
\pgfpathlineto{\pgfqpoint{3.792168in}{0.683728in}}%
\pgfpathlineto{\pgfqpoint{3.792702in}{0.663811in}}%
\pgfpathlineto{\pgfqpoint{3.793235in}{0.640023in}}%
\pgfpathlineto{\pgfqpoint{3.793769in}{0.647462in}}%
\pgfpathlineto{\pgfqpoint{3.795370in}{0.688047in}}%
\pgfpathlineto{\pgfqpoint{3.795904in}{0.638197in}}%
\pgfpathlineto{\pgfqpoint{3.796438in}{0.673717in}}%
\pgfpathlineto{\pgfqpoint{3.796971in}{0.679076in}}%
\pgfpathlineto{\pgfqpoint{3.797505in}{0.698275in}}%
\pgfpathlineto{\pgfqpoint{3.798039in}{0.693196in}}%
\pgfpathlineto{\pgfqpoint{3.799106in}{0.647786in}}%
\pgfpathlineto{\pgfqpoint{3.799640in}{0.650047in}}%
\pgfpathlineto{\pgfqpoint{3.800174in}{0.666166in}}%
\pgfpathlineto{\pgfqpoint{3.800707in}{0.649363in}}%
\pgfpathlineto{\pgfqpoint{3.801775in}{0.640280in}}%
\pgfpathlineto{\pgfqpoint{3.802308in}{0.641788in}}%
\pgfpathlineto{\pgfqpoint{3.803376in}{0.705573in}}%
\pgfpathlineto{\pgfqpoint{3.803909in}{0.666110in}}%
\pgfpathlineto{\pgfqpoint{3.804443in}{0.672320in}}%
\pgfpathlineto{\pgfqpoint{3.804977in}{0.687423in}}%
\pgfpathlineto{\pgfqpoint{3.805511in}{0.659522in}}%
\pgfpathlineto{\pgfqpoint{3.806044in}{0.667440in}}%
\pgfpathlineto{\pgfqpoint{3.806578in}{0.684656in}}%
\pgfpathlineto{\pgfqpoint{3.807112in}{0.650539in}}%
\pgfpathlineto{\pgfqpoint{3.807645in}{0.662817in}}%
\pgfpathlineto{\pgfqpoint{3.808713in}{0.677027in}}%
\pgfpathlineto{\pgfqpoint{3.809246in}{0.661519in}}%
\pgfpathlineto{\pgfqpoint{3.809780in}{0.669932in}}%
\pgfpathlineto{\pgfqpoint{3.811381in}{0.744182in}}%
\pgfpathlineto{\pgfqpoint{3.811915in}{0.652526in}}%
\pgfpathlineto{\pgfqpoint{3.813516in}{0.872115in}}%
\pgfpathlineto{\pgfqpoint{3.814050in}{0.855916in}}%
\pgfpathlineto{\pgfqpoint{3.815117in}{0.651223in}}%
\pgfpathlineto{\pgfqpoint{3.815651in}{0.742045in}}%
\pgfpathlineto{\pgfqpoint{3.816185in}{0.965027in}}%
\pgfpathlineto{\pgfqpoint{3.816718in}{0.771053in}}%
\pgfpathlineto{\pgfqpoint{3.817252in}{0.853968in}}%
\pgfpathlineto{\pgfqpoint{3.817786in}{0.757945in}}%
\pgfpathlineto{\pgfqpoint{3.818319in}{0.842182in}}%
\pgfpathlineto{\pgfqpoint{3.819920in}{0.759079in}}%
\pgfpathlineto{\pgfqpoint{3.820454in}{0.774190in}}%
\pgfpathlineto{\pgfqpoint{3.820988in}{0.765046in}}%
\pgfpathlineto{\pgfqpoint{3.822589in}{0.689190in}}%
\pgfpathlineto{\pgfqpoint{3.823123in}{0.671817in}}%
\pgfpathlineto{\pgfqpoint{3.824190in}{0.746460in}}%
\pgfpathlineto{\pgfqpoint{3.825257in}{0.732056in}}%
\pgfpathlineto{\pgfqpoint{3.825791in}{0.674885in}}%
\pgfpathlineto{\pgfqpoint{3.826325in}{0.723324in}}%
\pgfpathlineto{\pgfqpoint{3.827926in}{0.757344in}}%
\pgfpathlineto{\pgfqpoint{3.828460in}{0.748096in}}%
\pgfpathlineto{\pgfqpoint{3.828993in}{0.775625in}}%
\pgfpathlineto{\pgfqpoint{3.830595in}{0.699063in}}%
\pgfpathlineto{\pgfqpoint{3.831662in}{0.648786in}}%
\pgfpathlineto{\pgfqpoint{3.832196in}{0.668411in}}%
\pgfpathlineto{\pgfqpoint{3.833263in}{0.647189in}}%
\pgfpathlineto{\pgfqpoint{3.833797in}{0.654253in}}%
\pgfpathlineto{\pgfqpoint{3.834864in}{0.667410in}}%
\pgfpathlineto{\pgfqpoint{3.835398in}{0.636209in}}%
\pgfpathlineto{\pgfqpoint{3.835932in}{0.650431in}}%
\pgfpathlineto{\pgfqpoint{3.836465in}{0.667827in}}%
\pgfpathlineto{\pgfqpoint{3.836999in}{0.650109in}}%
\pgfpathlineto{\pgfqpoint{3.837533in}{0.650175in}}%
\pgfpathlineto{\pgfqpoint{3.838066in}{0.641042in}}%
\pgfpathlineto{\pgfqpoint{3.838600in}{0.665223in}}%
\pgfpathlineto{\pgfqpoint{3.839667in}{0.664418in}}%
\pgfpathlineto{\pgfqpoint{3.840201in}{0.656704in}}%
\pgfpathlineto{\pgfqpoint{3.840735in}{0.672788in}}%
\pgfpathlineto{\pgfqpoint{3.841269in}{0.660951in}}%
\pgfpathlineto{\pgfqpoint{3.841802in}{0.646485in}}%
\pgfpathlineto{\pgfqpoint{3.842336in}{0.659841in}}%
\pgfpathlineto{\pgfqpoint{3.842870in}{0.668250in}}%
\pgfpathlineto{\pgfqpoint{3.843403in}{0.645539in}}%
\pgfpathlineto{\pgfqpoint{3.843937in}{0.685253in}}%
\pgfpathlineto{\pgfqpoint{3.844471in}{0.663915in}}%
\pgfpathlineto{\pgfqpoint{3.846606in}{0.647879in}}%
\pgfpathlineto{\pgfqpoint{3.847673in}{0.679212in}}%
\pgfpathlineto{\pgfqpoint{3.848207in}{0.672525in}}%
\pgfpathlineto{\pgfqpoint{3.848740in}{0.680480in}}%
\pgfpathlineto{\pgfqpoint{3.849274in}{0.648645in}}%
\pgfpathlineto{\pgfqpoint{3.849808in}{0.665750in}}%
\pgfpathlineto{\pgfqpoint{3.850341in}{0.679221in}}%
\pgfpathlineto{\pgfqpoint{3.850875in}{0.652451in}}%
\pgfpathlineto{\pgfqpoint{3.851409in}{0.672033in}}%
\pgfpathlineto{\pgfqpoint{3.851943in}{0.659363in}}%
\pgfpathlineto{\pgfqpoint{3.852476in}{0.671017in}}%
\pgfpathlineto{\pgfqpoint{3.853010in}{0.685160in}}%
\pgfpathlineto{\pgfqpoint{3.854077in}{0.643001in}}%
\pgfpathlineto{\pgfqpoint{3.854611in}{0.651779in}}%
\pgfpathlineto{\pgfqpoint{3.855678in}{0.655552in}}%
\pgfpathlineto{\pgfqpoint{3.856212in}{0.685716in}}%
\pgfpathlineto{\pgfqpoint{3.857280in}{0.649734in}}%
\pgfpathlineto{\pgfqpoint{3.858881in}{0.682062in}}%
\pgfpathlineto{\pgfqpoint{3.859414in}{0.639399in}}%
\pgfpathlineto{\pgfqpoint{3.859948in}{0.669943in}}%
\pgfpathlineto{\pgfqpoint{3.861015in}{0.647279in}}%
\pgfpathlineto{\pgfqpoint{3.861549in}{0.708072in}}%
\pgfpathlineto{\pgfqpoint{3.862083in}{0.669146in}}%
\pgfpathlineto{\pgfqpoint{3.862617in}{0.656018in}}%
\pgfpathlineto{\pgfqpoint{3.863150in}{0.668250in}}%
\pgfpathlineto{\pgfqpoint{3.863684in}{0.679608in}}%
\pgfpathlineto{\pgfqpoint{3.864751in}{0.655971in}}%
\pgfpathlineto{\pgfqpoint{3.865285in}{0.658840in}}%
\pgfpathlineto{\pgfqpoint{3.865819in}{0.740210in}}%
\pgfpathlineto{\pgfqpoint{3.866886in}{0.737216in}}%
\pgfpathlineto{\pgfqpoint{3.868487in}{0.709366in}}%
\pgfpathlineto{\pgfqpoint{3.869021in}{0.902721in}}%
\pgfpathlineto{\pgfqpoint{3.869555in}{0.742523in}}%
\pgfpathlineto{\pgfqpoint{3.870088in}{0.675986in}}%
\pgfpathlineto{\pgfqpoint{3.870622in}{0.741145in}}%
\pgfpathlineto{\pgfqpoint{3.871156in}{0.801318in}}%
\pgfpathlineto{\pgfqpoint{3.871689in}{0.691548in}}%
\pgfpathlineto{\pgfqpoint{3.872223in}{0.736034in}}%
\pgfpathlineto{\pgfqpoint{3.872757in}{0.723970in}}%
\pgfpathlineto{\pgfqpoint{3.873291in}{0.850337in}}%
\pgfpathlineto{\pgfqpoint{3.873824in}{0.749289in}}%
\pgfpathlineto{\pgfqpoint{3.874358in}{0.769102in}}%
\pgfpathlineto{\pgfqpoint{3.874892in}{0.676000in}}%
\pgfpathlineto{\pgfqpoint{3.875425in}{0.748987in}}%
\pgfpathlineto{\pgfqpoint{3.875959in}{0.790538in}}%
\pgfpathlineto{\pgfqpoint{3.876493in}{0.670061in}}%
\pgfpathlineto{\pgfqpoint{3.877026in}{0.696731in}}%
\pgfpathlineto{\pgfqpoint{3.877560in}{0.693813in}}%
\pgfpathlineto{\pgfqpoint{3.878628in}{0.674386in}}%
\pgfpathlineto{\pgfqpoint{3.879161in}{0.756148in}}%
\pgfpathlineto{\pgfqpoint{3.879695in}{0.687611in}}%
\pgfpathlineto{\pgfqpoint{3.880229in}{0.686809in}}%
\pgfpathlineto{\pgfqpoint{3.880762in}{0.679496in}}%
\pgfpathlineto{\pgfqpoint{3.882363in}{0.756439in}}%
\pgfpathlineto{\pgfqpoint{3.883431in}{0.670275in}}%
\pgfpathlineto{\pgfqpoint{3.883965in}{0.693169in}}%
\pgfpathlineto{\pgfqpoint{3.885032in}{0.752871in}}%
\pgfpathlineto{\pgfqpoint{3.885566in}{0.748389in}}%
\pgfpathlineto{\pgfqpoint{3.887700in}{0.689745in}}%
\pgfpathlineto{\pgfqpoint{3.888234in}{0.694561in}}%
\pgfpathlineto{\pgfqpoint{3.888768in}{0.662242in}}%
\pgfpathlineto{\pgfqpoint{3.889835in}{0.664595in}}%
\pgfpathlineto{\pgfqpoint{3.890369in}{0.655588in}}%
\pgfpathlineto{\pgfqpoint{3.890903in}{0.669376in}}%
\pgfpathlineto{\pgfqpoint{3.891436in}{0.668371in}}%
\pgfpathlineto{\pgfqpoint{3.891970in}{0.657436in}}%
\pgfpathlineto{\pgfqpoint{3.892504in}{0.679801in}}%
\pgfpathlineto{\pgfqpoint{3.893037in}{0.637062in}}%
\pgfpathlineto{\pgfqpoint{3.893571in}{0.673436in}}%
\pgfpathlineto{\pgfqpoint{3.894105in}{0.649360in}}%
\pgfpathlineto{\pgfqpoint{3.894639in}{0.663023in}}%
\pgfpathlineto{\pgfqpoint{3.895172in}{0.664686in}}%
\pgfpathlineto{\pgfqpoint{3.895706in}{0.662599in}}%
\pgfpathlineto{\pgfqpoint{3.896240in}{0.679410in}}%
\pgfpathlineto{\pgfqpoint{3.896773in}{0.642011in}}%
\pgfpathlineto{\pgfqpoint{3.897307in}{0.671409in}}%
\pgfpathlineto{\pgfqpoint{3.899442in}{0.643337in}}%
\pgfpathlineto{\pgfqpoint{3.900509in}{0.662810in}}%
\pgfpathlineto{\pgfqpoint{3.901043in}{0.642899in}}%
\pgfpathlineto{\pgfqpoint{3.901577in}{0.650655in}}%
\pgfpathlineto{\pgfqpoint{3.902110in}{0.662098in}}%
\pgfpathlineto{\pgfqpoint{3.902644in}{0.651227in}}%
\pgfpathlineto{\pgfqpoint{3.904245in}{0.664915in}}%
\pgfpathlineto{\pgfqpoint{3.904779in}{0.661093in}}%
\pgfpathlineto{\pgfqpoint{3.905846in}{0.644400in}}%
\pgfpathlineto{\pgfqpoint{3.906380in}{0.653945in}}%
\pgfpathlineto{\pgfqpoint{3.906914in}{0.652701in}}%
\pgfpathlineto{\pgfqpoint{3.907447in}{0.652445in}}%
\pgfpathlineto{\pgfqpoint{3.907981in}{0.645841in}}%
\pgfpathlineto{\pgfqpoint{3.908515in}{0.649328in}}%
\pgfpathlineto{\pgfqpoint{3.909049in}{0.655699in}}%
\pgfpathlineto{\pgfqpoint{3.909582in}{0.645880in}}%
\pgfpathlineto{\pgfqpoint{3.910116in}{0.648803in}}%
\pgfpathlineto{\pgfqpoint{3.910650in}{0.665216in}}%
\pgfpathlineto{\pgfqpoint{3.911183in}{0.662493in}}%
\pgfpathlineto{\pgfqpoint{3.911717in}{0.658674in}}%
\pgfpathlineto{\pgfqpoint{3.913852in}{0.677933in}}%
\pgfpathlineto{\pgfqpoint{3.914386in}{0.637561in}}%
\pgfpathlineto{\pgfqpoint{3.914919in}{0.675045in}}%
\pgfpathlineto{\pgfqpoint{3.915453in}{0.664978in}}%
\pgfpathlineto{\pgfqpoint{3.915987in}{0.678068in}}%
\pgfpathlineto{\pgfqpoint{3.916520in}{0.648597in}}%
\pgfpathlineto{\pgfqpoint{3.917054in}{0.669495in}}%
\pgfpathlineto{\pgfqpoint{3.918655in}{0.690133in}}%
\pgfpathlineto{\pgfqpoint{3.919189in}{0.643941in}}%
\pgfpathlineto{\pgfqpoint{3.920790in}{0.752079in}}%
\pgfpathlineto{\pgfqpoint{3.921324in}{0.733114in}}%
\pgfpathlineto{\pgfqpoint{3.921857in}{0.794663in}}%
\pgfpathlineto{\pgfqpoint{3.922391in}{0.663075in}}%
\pgfpathlineto{\pgfqpoint{3.922925in}{0.723150in}}%
\pgfpathlineto{\pgfqpoint{3.923458in}{0.741959in}}%
\pgfpathlineto{\pgfqpoint{3.923992in}{0.844590in}}%
\pgfpathlineto{\pgfqpoint{3.924526in}{0.780023in}}%
\pgfpathlineto{\pgfqpoint{3.926127in}{0.655712in}}%
\pgfpathlineto{\pgfqpoint{3.927728in}{0.821897in}}%
\pgfpathlineto{\pgfqpoint{3.929863in}{0.696079in}}%
\pgfpathlineto{\pgfqpoint{3.930930in}{0.770145in}}%
\pgfpathlineto{\pgfqpoint{3.931998in}{0.673110in}}%
\pgfpathlineto{\pgfqpoint{3.932531in}{0.700822in}}%
\pgfpathlineto{\pgfqpoint{3.933065in}{0.744656in}}%
\pgfpathlineto{\pgfqpoint{3.933599in}{0.660988in}}%
\pgfpathlineto{\pgfqpoint{3.934132in}{0.720420in}}%
\pgfpathlineto{\pgfqpoint{3.935200in}{0.655347in}}%
\pgfpathlineto{\pgfqpoint{3.935734in}{0.661996in}}%
\pgfpathlineto{\pgfqpoint{3.936267in}{0.751875in}}%
\pgfpathlineto{\pgfqpoint{3.936801in}{0.693462in}}%
\pgfpathlineto{\pgfqpoint{3.937868in}{0.661966in}}%
\pgfpathlineto{\pgfqpoint{3.939469in}{0.741231in}}%
\pgfpathlineto{\pgfqpoint{3.940537in}{0.665304in}}%
\pgfpathlineto{\pgfqpoint{3.941071in}{0.673311in}}%
\pgfpathlineto{\pgfqpoint{3.941604in}{0.672942in}}%
\pgfpathlineto{\pgfqpoint{3.943205in}{0.725477in}}%
\pgfpathlineto{\pgfqpoint{3.943739in}{0.761397in}}%
\pgfpathlineto{\pgfqpoint{3.945340in}{0.690009in}}%
\pgfpathlineto{\pgfqpoint{3.945874in}{0.706416in}}%
\pgfpathlineto{\pgfqpoint{3.946408in}{0.692406in}}%
\pgfpathlineto{\pgfqpoint{3.947475in}{0.665034in}}%
\pgfpathlineto{\pgfqpoint{3.948009in}{0.670439in}}%
\pgfpathlineto{\pgfqpoint{3.948542in}{0.673709in}}%
\pgfpathlineto{\pgfqpoint{3.949076in}{0.645623in}}%
\pgfpathlineto{\pgfqpoint{3.949610in}{0.657780in}}%
\pgfpathlineto{\pgfqpoint{3.950677in}{0.657282in}}%
\pgfpathlineto{\pgfqpoint{3.951211in}{0.673087in}}%
\pgfpathlineto{\pgfqpoint{3.951745in}{0.643675in}}%
\pgfpathlineto{\pgfqpoint{3.952278in}{0.652355in}}%
\pgfpathlineto{\pgfqpoint{3.953879in}{0.659670in}}%
\pgfpathlineto{\pgfqpoint{3.954413in}{0.647171in}}%
\pgfpathlineto{\pgfqpoint{3.955480in}{0.672889in}}%
\pgfpathlineto{\pgfqpoint{3.956014in}{0.671930in}}%
\pgfpathlineto{\pgfqpoint{3.956548in}{0.675223in}}%
\pgfpathlineto{\pgfqpoint{3.957082in}{0.646562in}}%
\pgfpathlineto{\pgfqpoint{3.957615in}{0.652634in}}%
\pgfpathlineto{\pgfqpoint{3.958149in}{0.670460in}}%
\pgfpathlineto{\pgfqpoint{3.958683in}{0.654713in}}%
\pgfpathlineto{\pgfqpoint{3.959216in}{0.658075in}}%
\pgfpathlineto{\pgfqpoint{3.959750in}{0.670249in}}%
\pgfpathlineto{\pgfqpoint{3.960284in}{0.643397in}}%
\pgfpathlineto{\pgfqpoint{3.960817in}{0.655836in}}%
\pgfpathlineto{\pgfqpoint{3.961351in}{0.662194in}}%
\pgfpathlineto{\pgfqpoint{3.961885in}{0.644016in}}%
\pgfpathlineto{\pgfqpoint{3.962419in}{0.659603in}}%
\pgfpathlineto{\pgfqpoint{3.962952in}{0.690555in}}%
\pgfpathlineto{\pgfqpoint{3.963486in}{0.661876in}}%
\pgfpathlineto{\pgfqpoint{3.964020in}{0.666396in}}%
\pgfpathlineto{\pgfqpoint{3.964553in}{0.662711in}}%
\pgfpathlineto{\pgfqpoint{3.965087in}{0.654755in}}%
\pgfpathlineto{\pgfqpoint{3.965621in}{0.673716in}}%
\pgfpathlineto{\pgfqpoint{3.966155in}{0.644345in}}%
\pgfpathlineto{\pgfqpoint{3.966688in}{0.675808in}}%
\pgfpathlineto{\pgfqpoint{3.967756in}{0.671523in}}%
\pgfpathlineto{\pgfqpoint{3.968289in}{0.676422in}}%
\pgfpathlineto{\pgfqpoint{3.969357in}{0.644907in}}%
\pgfpathlineto{\pgfqpoint{3.969890in}{0.689415in}}%
\pgfpathlineto{\pgfqpoint{3.970424in}{0.657913in}}%
\pgfpathlineto{\pgfqpoint{3.970958in}{0.669301in}}%
\pgfpathlineto{\pgfqpoint{3.971492in}{0.650687in}}%
\pgfpathlineto{\pgfqpoint{3.972025in}{0.667334in}}%
\pgfpathlineto{\pgfqpoint{3.974160in}{0.690159in}}%
\pgfpathlineto{\pgfqpoint{3.974694in}{0.763725in}}%
\pgfpathlineto{\pgfqpoint{3.975227in}{0.689462in}}%
\pgfpathlineto{\pgfqpoint{3.976829in}{0.791954in}}%
\pgfpathlineto{\pgfqpoint{3.978430in}{0.693843in}}%
\pgfpathlineto{\pgfqpoint{3.979497in}{0.916135in}}%
\pgfpathlineto{\pgfqpoint{3.980564in}{0.664345in}}%
\pgfpathlineto{\pgfqpoint{3.981098in}{0.769783in}}%
\pgfpathlineto{\pgfqpoint{3.982166in}{0.769394in}}%
\pgfpathlineto{\pgfqpoint{3.982699in}{0.801435in}}%
\pgfpathlineto{\pgfqpoint{3.983233in}{0.655524in}}%
\pgfpathlineto{\pgfqpoint{3.983767in}{0.846225in}}%
\pgfpathlineto{\pgfqpoint{3.984300in}{0.725095in}}%
\pgfpathlineto{\pgfqpoint{3.984834in}{0.822752in}}%
\pgfpathlineto{\pgfqpoint{3.985368in}{0.750758in}}%
\pgfpathlineto{\pgfqpoint{3.986435in}{0.847386in}}%
\pgfpathlineto{\pgfqpoint{3.986969in}{0.697292in}}%
\pgfpathlineto{\pgfqpoint{3.987503in}{0.719625in}}%
\pgfpathlineto{\pgfqpoint{3.988036in}{0.762942in}}%
\pgfpathlineto{\pgfqpoint{3.988570in}{0.722787in}}%
\pgfpathlineto{\pgfqpoint{3.989104in}{0.673667in}}%
\pgfpathlineto{\pgfqpoint{3.989637in}{0.753024in}}%
\pgfpathlineto{\pgfqpoint{3.990171in}{0.676269in}}%
\pgfpathlineto{\pgfqpoint{3.990705in}{0.651071in}}%
\pgfpathlineto{\pgfqpoint{3.991238in}{0.778116in}}%
\pgfpathlineto{\pgfqpoint{3.991772in}{0.700000in}}%
\pgfpathlineto{\pgfqpoint{3.992306in}{0.657430in}}%
\pgfpathlineto{\pgfqpoint{3.992840in}{0.687877in}}%
\pgfpathlineto{\pgfqpoint{3.994441in}{0.730175in}}%
\pgfpathlineto{\pgfqpoint{3.994974in}{0.675846in}}%
\pgfpathlineto{\pgfqpoint{3.995508in}{0.683470in}}%
\pgfpathlineto{\pgfqpoint{3.996575in}{0.781667in}}%
\pgfpathlineto{\pgfqpoint{3.998710in}{0.652310in}}%
\pgfpathlineto{\pgfqpoint{3.999244in}{0.673014in}}%
\pgfpathlineto{\pgfqpoint{4.000845in}{0.758973in}}%
\pgfpathlineto{\pgfqpoint{4.001379in}{0.757903in}}%
\pgfpathlineto{\pgfqpoint{4.002980in}{0.716612in}}%
\pgfpathlineto{\pgfqpoint{4.003514in}{0.698225in}}%
\pgfpathlineto{\pgfqpoint{4.004047in}{0.739242in}}%
\pgfpathlineto{\pgfqpoint{4.004581in}{0.719452in}}%
\pgfpathlineto{\pgfqpoint{4.006182in}{0.645660in}}%
\pgfpathlineto{\pgfqpoint{4.006716in}{0.650829in}}%
\pgfpathlineto{\pgfqpoint{4.007249in}{0.645979in}}%
\pgfpathlineto{\pgfqpoint{4.008851in}{0.662747in}}%
\pgfpathlineto{\pgfqpoint{4.009384in}{0.645995in}}%
\pgfpathlineto{\pgfqpoint{4.009918in}{0.653736in}}%
\pgfpathlineto{\pgfqpoint{4.010452in}{0.653515in}}%
\pgfpathlineto{\pgfqpoint{4.011519in}{0.659161in}}%
\pgfpathlineto{\pgfqpoint{4.012053in}{0.650522in}}%
\pgfpathlineto{\pgfqpoint{4.012586in}{0.671753in}}%
\pgfpathlineto{\pgfqpoint{4.013120in}{0.644052in}}%
\pgfpathlineto{\pgfqpoint{4.013654in}{0.650969in}}%
\pgfpathlineto{\pgfqpoint{4.014188in}{0.652277in}}%
\pgfpathlineto{\pgfqpoint{4.015789in}{0.641399in}}%
\pgfpathlineto{\pgfqpoint{4.016856in}{0.669865in}}%
\pgfpathlineto{\pgfqpoint{4.018457in}{0.649905in}}%
\pgfpathlineto{\pgfqpoint{4.020058in}{0.678959in}}%
\pgfpathlineto{\pgfqpoint{4.020592in}{0.640133in}}%
\pgfpathlineto{\pgfqpoint{4.021126in}{0.671830in}}%
\pgfpathlineto{\pgfqpoint{4.021659in}{0.678243in}}%
\pgfpathlineto{\pgfqpoint{4.022193in}{0.673721in}}%
\pgfpathlineto{\pgfqpoint{4.023260in}{0.674860in}}%
\pgfpathlineto{\pgfqpoint{4.023794in}{0.641426in}}%
\pgfpathlineto{\pgfqpoint{4.025395in}{0.676755in}}%
\pgfpathlineto{\pgfqpoint{4.025929in}{0.662811in}}%
\pgfpathlineto{\pgfqpoint{4.026996in}{0.692786in}}%
\pgfpathlineto{\pgfqpoint{4.027530in}{0.648457in}}%
\pgfpathlineto{\pgfqpoint{4.029131in}{0.756391in}}%
\pgfpathlineto{\pgfqpoint{4.030732in}{0.693420in}}%
\pgfpathlineto{\pgfqpoint{4.032333in}{0.837458in}}%
\pgfpathlineto{\pgfqpoint{4.033401in}{0.700811in}}%
\pgfpathlineto{\pgfqpoint{4.034468in}{0.902493in}}%
\pgfpathlineto{\pgfqpoint{4.035536in}{0.670812in}}%
\pgfpathlineto{\pgfqpoint{4.036069in}{0.679347in}}%
\pgfpathlineto{\pgfqpoint{4.036603in}{0.682698in}}%
\pgfpathlineto{\pgfqpoint{4.037137in}{0.677362in}}%
\pgfpathlineto{\pgfqpoint{4.038738in}{0.815056in}}%
\pgfpathlineto{\pgfqpoint{4.040339in}{0.688205in}}%
\pgfpathlineto{\pgfqpoint{4.041406in}{0.835427in}}%
\pgfpathlineto{\pgfqpoint{4.042474in}{0.680779in}}%
\pgfpathlineto{\pgfqpoint{4.043007in}{0.685146in}}%
\pgfpathlineto{\pgfqpoint{4.043541in}{0.788812in}}%
\pgfpathlineto{\pgfqpoint{4.044075in}{0.653997in}}%
\pgfpathlineto{\pgfqpoint{4.044609in}{0.743846in}}%
\pgfpathlineto{\pgfqpoint{4.045142in}{0.675680in}}%
\pgfpathlineto{\pgfqpoint{4.045676in}{0.692342in}}%
\pgfpathlineto{\pgfqpoint{4.046210in}{0.691741in}}%
\pgfpathlineto{\pgfqpoint{4.046743in}{0.732189in}}%
\pgfpathlineto{\pgfqpoint{4.047811in}{0.650480in}}%
\pgfpathlineto{\pgfqpoint{4.048344in}{0.733515in}}%
\pgfpathlineto{\pgfqpoint{4.048878in}{0.715130in}}%
\pgfpathlineto{\pgfqpoint{4.049946in}{0.651023in}}%
\pgfpathlineto{\pgfqpoint{4.051547in}{0.736179in}}%
\pgfpathlineto{\pgfqpoint{4.052614in}{0.644465in}}%
\pgfpathlineto{\pgfqpoint{4.053148in}{0.672762in}}%
\pgfpathlineto{\pgfqpoint{4.054215in}{0.741802in}}%
\pgfpathlineto{\pgfqpoint{4.054749in}{0.729690in}}%
\pgfpathlineto{\pgfqpoint{4.056350in}{0.639448in}}%
\pgfpathlineto{\pgfqpoint{4.058485in}{0.724334in}}%
\pgfpathlineto{\pgfqpoint{4.059018in}{0.708678in}}%
\pgfpathlineto{\pgfqpoint{4.059552in}{0.713278in}}%
\pgfpathlineto{\pgfqpoint{4.060086in}{0.727802in}}%
\pgfpathlineto{\pgfqpoint{4.061153in}{0.686793in}}%
\pgfpathlineto{\pgfqpoint{4.062221in}{0.687223in}}%
\pgfpathlineto{\pgfqpoint{4.065423in}{0.650029in}}%
\pgfpathlineto{\pgfqpoint{4.065957in}{0.655835in}}%
\pgfpathlineto{\pgfqpoint{4.066490in}{0.637706in}}%
\pgfpathlineto{\pgfqpoint{4.067558in}{0.638424in}}%
\pgfpathlineto{\pgfqpoint{4.068091in}{0.639027in}}%
\pgfpathlineto{\pgfqpoint{4.069159in}{0.653320in}}%
\pgfpathlineto{\pgfqpoint{4.069692in}{0.643383in}}%
\pgfpathlineto{\pgfqpoint{4.070760in}{0.663329in}}%
\pgfpathlineto{\pgfqpoint{4.071294in}{0.646077in}}%
\pgfpathlineto{\pgfqpoint{4.071827in}{0.651414in}}%
\pgfpathlineto{\pgfqpoint{4.072361in}{0.664576in}}%
\pgfpathlineto{\pgfqpoint{4.072895in}{0.651727in}}%
\pgfpathlineto{\pgfqpoint{4.073428in}{0.645490in}}%
\pgfpathlineto{\pgfqpoint{4.075029in}{0.664496in}}%
\pgfpathlineto{\pgfqpoint{4.075563in}{0.662515in}}%
\pgfpathlineto{\pgfqpoint{4.076097in}{0.663582in}}%
\pgfpathlineto{\pgfqpoint{4.077698in}{0.647274in}}%
\pgfpathlineto{\pgfqpoint{4.079299in}{0.692694in}}%
\pgfpathlineto{\pgfqpoint{4.080900in}{0.654230in}}%
\pgfpathlineto{\pgfqpoint{4.081968in}{0.705383in}}%
\pgfpathlineto{\pgfqpoint{4.082501in}{0.651561in}}%
\pgfpathlineto{\pgfqpoint{4.084102in}{0.727398in}}%
\pgfpathlineto{\pgfqpoint{4.084636in}{0.679382in}}%
\pgfpathlineto{\pgfqpoint{4.085170in}{0.753041in}}%
\pgfpathlineto{\pgfqpoint{4.085703in}{0.655865in}}%
\pgfpathlineto{\pgfqpoint{4.086237in}{0.732834in}}%
\pgfpathlineto{\pgfqpoint{4.086771in}{0.721280in}}%
\pgfpathlineto{\pgfqpoint{4.087305in}{0.827672in}}%
\pgfpathlineto{\pgfqpoint{4.087838in}{0.741493in}}%
\pgfpathlineto{\pgfqpoint{4.088372in}{0.737395in}}%
\pgfpathlineto{\pgfqpoint{4.088906in}{0.724124in}}%
\pgfpathlineto{\pgfqpoint{4.089973in}{0.780990in}}%
\pgfpathlineto{\pgfqpoint{4.091040in}{0.680216in}}%
\pgfpathlineto{\pgfqpoint{4.091574in}{0.741130in}}%
\pgfpathlineto{\pgfqpoint{4.092108in}{0.650827in}}%
\pgfpathlineto{\pgfqpoint{4.093175in}{0.788995in}}%
\pgfpathlineto{\pgfqpoint{4.093709in}{0.673185in}}%
\pgfpathlineto{\pgfqpoint{4.094776in}{0.680518in}}%
\pgfpathlineto{\pgfqpoint{4.096377in}{0.780150in}}%
\pgfpathlineto{\pgfqpoint{4.097979in}{0.685000in}}%
\pgfpathlineto{\pgfqpoint{4.098512in}{0.785794in}}%
\pgfpathlineto{\pgfqpoint{4.099046in}{0.692700in}}%
\pgfpathlineto{\pgfqpoint{4.100647in}{0.716260in}}%
\pgfpathlineto{\pgfqpoint{4.101181in}{0.648705in}}%
\pgfpathlineto{\pgfqpoint{4.101715in}{0.738912in}}%
\pgfpathlineto{\pgfqpoint{4.102248in}{0.656847in}}%
\pgfpathlineto{\pgfqpoint{4.103849in}{0.696571in}}%
\pgfpathlineto{\pgfqpoint{4.104383in}{0.660248in}}%
\pgfpathlineto{\pgfqpoint{4.104917in}{0.667799in}}%
\pgfpathlineto{\pgfqpoint{4.105984in}{0.735066in}}%
\pgfpathlineto{\pgfqpoint{4.106518in}{0.699643in}}%
\pgfpathlineto{\pgfqpoint{4.107052in}{0.648374in}}%
\pgfpathlineto{\pgfqpoint{4.107585in}{0.686278in}}%
\pgfpathlineto{\pgfqpoint{4.108653in}{0.714000in}}%
\pgfpathlineto{\pgfqpoint{4.110254in}{0.649356in}}%
\pgfpathlineto{\pgfqpoint{4.111855in}{0.715248in}}%
\pgfpathlineto{\pgfqpoint{4.113456in}{0.644303in}}%
\pgfpathlineto{\pgfqpoint{4.112922in}{0.715620in}}%
\pgfpathlineto{\pgfqpoint{4.113990in}{0.655439in}}%
\pgfpathlineto{\pgfqpoint{4.114523in}{0.668560in}}%
\pgfpathlineto{\pgfqpoint{4.115057in}{0.712502in}}%
\pgfpathlineto{\pgfqpoint{4.115591in}{0.701309in}}%
\pgfpathlineto{\pgfqpoint{4.116124in}{0.699541in}}%
\pgfpathlineto{\pgfqpoint{4.117192in}{0.733915in}}%
\pgfpathlineto{\pgfqpoint{4.118793in}{0.701375in}}%
\pgfpathlineto{\pgfqpoint{4.119327in}{0.705757in}}%
\pgfpathlineto{\pgfqpoint{4.120394in}{0.704342in}}%
\pgfpathlineto{\pgfqpoint{4.121461in}{0.660952in}}%
\pgfpathlineto{\pgfqpoint{4.121995in}{0.662278in}}%
\pgfpathlineto{\pgfqpoint{4.122529in}{0.650934in}}%
\pgfpathlineto{\pgfqpoint{4.124130in}{0.672646in}}%
\pgfpathlineto{\pgfqpoint{4.125197in}{0.639323in}}%
\pgfpathlineto{\pgfqpoint{4.126265in}{0.649626in}}%
\pgfpathlineto{\pgfqpoint{4.126798in}{0.650075in}}%
\pgfpathlineto{\pgfqpoint{4.127332in}{0.673458in}}%
\pgfpathlineto{\pgfqpoint{4.127866in}{0.654941in}}%
\pgfpathlineto{\pgfqpoint{4.128400in}{0.670447in}}%
\pgfpathlineto{\pgfqpoint{4.128933in}{0.661123in}}%
\pgfpathlineto{\pgfqpoint{4.129467in}{0.658217in}}%
\pgfpathlineto{\pgfqpoint{4.130534in}{0.671325in}}%
\pgfpathlineto{\pgfqpoint{4.131602in}{0.654352in}}%
\pgfpathlineto{\pgfqpoint{4.133203in}{0.698613in}}%
\pgfpathlineto{\pgfqpoint{4.133737in}{0.647427in}}%
\pgfpathlineto{\pgfqpoint{4.134270in}{0.665146in}}%
\pgfpathlineto{\pgfqpoint{4.135338in}{0.668951in}}%
\pgfpathlineto{\pgfqpoint{4.135871in}{0.637775in}}%
\pgfpathlineto{\pgfqpoint{4.136405in}{0.662374in}}%
\pgfpathlineto{\pgfqpoint{4.136939in}{0.671462in}}%
\pgfpathlineto{\pgfqpoint{4.137472in}{0.706898in}}%
\pgfpathlineto{\pgfqpoint{4.138006in}{0.691522in}}%
\pgfpathlineto{\pgfqpoint{4.138540in}{0.692926in}}%
\pgfpathlineto{\pgfqpoint{4.139074in}{0.688292in}}%
\pgfpathlineto{\pgfqpoint{4.140141in}{0.745086in}}%
\pgfpathlineto{\pgfqpoint{4.141208in}{0.714111in}}%
\pgfpathlineto{\pgfqpoint{4.142809in}{0.775672in}}%
\pgfpathlineto{\pgfqpoint{4.143877in}{0.697630in}}%
\pgfpathlineto{\pgfqpoint{4.144944in}{0.775111in}}%
\pgfpathlineto{\pgfqpoint{4.146545in}{0.679351in}}%
\pgfpathlineto{\pgfqpoint{4.147079in}{0.684967in}}%
\pgfpathlineto{\pgfqpoint{4.148680in}{0.759785in}}%
\pgfpathlineto{\pgfqpoint{4.149214in}{0.706232in}}%
\pgfpathlineto{\pgfqpoint{4.149748in}{0.742408in}}%
\pgfpathlineto{\pgfqpoint{4.150281in}{0.772541in}}%
\pgfpathlineto{\pgfqpoint{4.150815in}{0.675284in}}%
\pgfpathlineto{\pgfqpoint{4.151349in}{0.748472in}}%
\pgfpathlineto{\pgfqpoint{4.152416in}{0.728500in}}%
\pgfpathlineto{\pgfqpoint{4.152950in}{0.658841in}}%
\pgfpathlineto{\pgfqpoint{4.154017in}{0.764972in}}%
\pgfpathlineto{\pgfqpoint{4.154551in}{0.659322in}}%
\pgfpathlineto{\pgfqpoint{4.155085in}{0.750221in}}%
\pgfpathlineto{\pgfqpoint{4.156152in}{0.651743in}}%
\pgfpathlineto{\pgfqpoint{4.156686in}{0.733707in}}%
\pgfpathlineto{\pgfqpoint{4.157219in}{0.687758in}}%
\pgfpathlineto{\pgfqpoint{4.157753in}{0.692045in}}%
\pgfpathlineto{\pgfqpoint{4.158287in}{0.648064in}}%
\pgfpathlineto{\pgfqpoint{4.158820in}{0.700325in}}%
\pgfpathlineto{\pgfqpoint{4.159354in}{0.683340in}}%
\pgfpathlineto{\pgfqpoint{4.159888in}{0.680234in}}%
\pgfpathlineto{\pgfqpoint{4.160422in}{0.689334in}}%
\pgfpathlineto{\pgfqpoint{4.160955in}{0.736892in}}%
\pgfpathlineto{\pgfqpoint{4.162023in}{0.650541in}}%
\pgfpathlineto{\pgfqpoint{4.163090in}{0.710188in}}%
\pgfpathlineto{\pgfqpoint{4.163624in}{0.688134in}}%
\pgfpathlineto{\pgfqpoint{4.164157in}{0.688121in}}%
\pgfpathlineto{\pgfqpoint{4.165225in}{0.657754in}}%
\pgfpathlineto{\pgfqpoint{4.165759in}{0.734533in}}%
\pgfpathlineto{\pgfqpoint{4.166292in}{0.700816in}}%
\pgfpathlineto{\pgfqpoint{4.167893in}{0.659531in}}%
\pgfpathlineto{\pgfqpoint{4.168427in}{0.671819in}}%
\pgfpathlineto{\pgfqpoint{4.169495in}{0.733907in}}%
\pgfpathlineto{\pgfqpoint{4.170028in}{0.727946in}}%
\pgfpathlineto{\pgfqpoint{4.171629in}{0.653247in}}%
\pgfpathlineto{\pgfqpoint{4.172163in}{0.662311in}}%
\pgfpathlineto{\pgfqpoint{4.173764in}{0.716733in}}%
\pgfpathlineto{\pgfqpoint{4.174298in}{0.717397in}}%
\pgfpathlineto{\pgfqpoint{4.174832in}{0.690206in}}%
\pgfpathlineto{\pgfqpoint{4.175365in}{0.714047in}}%
\pgfpathlineto{\pgfqpoint{4.175899in}{0.711192in}}%
\pgfpathlineto{\pgfqpoint{4.176433in}{0.731773in}}%
\pgfpathlineto{\pgfqpoint{4.177500in}{0.691623in}}%
\pgfpathlineto{\pgfqpoint{4.178034in}{0.697190in}}%
\pgfpathlineto{\pgfqpoint{4.180169in}{0.648965in}}%
\pgfpathlineto{\pgfqpoint{4.180702in}{0.650651in}}%
\pgfpathlineto{\pgfqpoint{4.181236in}{0.672414in}}%
\pgfpathlineto{\pgfqpoint{4.181770in}{0.638854in}}%
\pgfpathlineto{\pgfqpoint{4.182303in}{0.654499in}}%
\pgfpathlineto{\pgfqpoint{4.182837in}{0.645725in}}%
\pgfpathlineto{\pgfqpoint{4.183371in}{0.667607in}}%
\pgfpathlineto{\pgfqpoint{4.183904in}{0.642184in}}%
\pgfpathlineto{\pgfqpoint{4.184438in}{0.654992in}}%
\pgfpathlineto{\pgfqpoint{4.184972in}{0.647954in}}%
\pgfpathlineto{\pgfqpoint{4.185506in}{0.669570in}}%
\pgfpathlineto{\pgfqpoint{4.186039in}{0.650720in}}%
\pgfpathlineto{\pgfqpoint{4.186573in}{0.639476in}}%
\pgfpathlineto{\pgfqpoint{4.187107in}{0.673157in}}%
\pgfpathlineto{\pgfqpoint{4.187640in}{0.654416in}}%
\pgfpathlineto{\pgfqpoint{4.188708in}{0.669697in}}%
\pgfpathlineto{\pgfqpoint{4.189775in}{0.653783in}}%
\pgfpathlineto{\pgfqpoint{4.190309in}{0.722107in}}%
\pgfpathlineto{\pgfqpoint{4.190843in}{0.663092in}}%
\pgfpathlineto{\pgfqpoint{4.191376in}{0.683807in}}%
\pgfpathlineto{\pgfqpoint{4.191910in}{0.652617in}}%
\pgfpathlineto{\pgfqpoint{4.192444in}{0.717949in}}%
\pgfpathlineto{\pgfqpoint{4.192977in}{0.710479in}}%
\pgfpathlineto{\pgfqpoint{4.194045in}{0.678029in}}%
\pgfpathlineto{\pgfqpoint{4.195646in}{0.770930in}}%
\pgfpathlineto{\pgfqpoint{4.196713in}{0.692115in}}%
\pgfpathlineto{\pgfqpoint{4.197781in}{0.818159in}}%
\pgfpathlineto{\pgfqpoint{4.198314in}{0.677201in}}%
\pgfpathlineto{\pgfqpoint{4.198848in}{0.706754in}}%
\pgfpathlineto{\pgfqpoint{4.199382in}{0.754327in}}%
\pgfpathlineto{\pgfqpoint{4.199915in}{0.730515in}}%
\pgfpathlineto{\pgfqpoint{4.200449in}{0.660211in}}%
\pgfpathlineto{\pgfqpoint{4.200983in}{0.735727in}}%
\pgfpathlineto{\pgfqpoint{4.202584in}{0.654627in}}%
\pgfpathlineto{\pgfqpoint{4.203651in}{0.775876in}}%
\pgfpathlineto{\pgfqpoint{4.204185in}{0.671730in}}%
\pgfpathlineto{\pgfqpoint{4.204719in}{0.717614in}}%
\pgfpathlineto{\pgfqpoint{4.205252in}{0.720635in}}%
\pgfpathlineto{\pgfqpoint{4.206854in}{0.748225in}}%
\pgfpathlineto{\pgfqpoint{4.208455in}{0.671925in}}%
\pgfpathlineto{\pgfqpoint{4.208988in}{0.818897in}}%
\pgfpathlineto{\pgfqpoint{4.209522in}{0.656410in}}%
\pgfpathlineto{\pgfqpoint{4.210056in}{0.730188in}}%
\pgfpathlineto{\pgfqpoint{4.211657in}{0.675583in}}%
\pgfpathlineto{\pgfqpoint{4.212191in}{0.736792in}}%
\pgfpathlineto{\pgfqpoint{4.212724in}{0.672620in}}%
\pgfpathlineto{\pgfqpoint{4.213258in}{0.651206in}}%
\pgfpathlineto{\pgfqpoint{4.213792in}{0.701749in}}%
\pgfpathlineto{\pgfqpoint{4.214325in}{0.691357in}}%
\pgfpathlineto{\pgfqpoint{4.214859in}{0.691546in}}%
\pgfpathlineto{\pgfqpoint{4.215393in}{0.647589in}}%
\pgfpathlineto{\pgfqpoint{4.215926in}{0.697307in}}%
\pgfpathlineto{\pgfqpoint{4.216460in}{0.669718in}}%
\pgfpathlineto{\pgfqpoint{4.216994in}{0.653706in}}%
\pgfpathlineto{\pgfqpoint{4.217528in}{0.662108in}}%
\pgfpathlineto{\pgfqpoint{4.218061in}{0.709879in}}%
\pgfpathlineto{\pgfqpoint{4.218595in}{0.677459in}}%
\pgfpathlineto{\pgfqpoint{4.219129in}{0.654870in}}%
\pgfpathlineto{\pgfqpoint{4.220196in}{0.697632in}}%
\pgfpathlineto{\pgfqpoint{4.220730in}{0.685696in}}%
\pgfpathlineto{\pgfqpoint{4.221263in}{0.708494in}}%
\pgfpathlineto{\pgfqpoint{4.222331in}{0.641335in}}%
\pgfpathlineto{\pgfqpoint{4.223932in}{0.711108in}}%
\pgfpathlineto{\pgfqpoint{4.225533in}{0.662546in}}%
\pgfpathlineto{\pgfqpoint{4.226067in}{0.673806in}}%
\pgfpathlineto{\pgfqpoint{4.226600in}{0.696556in}}%
\pgfpathlineto{\pgfqpoint{4.227134in}{0.695053in}}%
\pgfpathlineto{\pgfqpoint{4.228735in}{0.674070in}}%
\pgfpathlineto{\pgfqpoint{4.229269in}{0.656183in}}%
\pgfpathlineto{\pgfqpoint{4.229803in}{0.667642in}}%
\pgfpathlineto{\pgfqpoint{4.230336in}{0.672341in}}%
\pgfpathlineto{\pgfqpoint{4.230870in}{0.670971in}}%
\pgfpathlineto{\pgfqpoint{4.232471in}{0.718257in}}%
\pgfpathlineto{\pgfqpoint{4.233005in}{0.719133in}}%
\pgfpathlineto{\pgfqpoint{4.234072in}{0.690683in}}%
\pgfpathlineto{\pgfqpoint{4.234606in}{0.716219in}}%
\pgfpathlineto{\pgfqpoint{4.235140in}{0.689485in}}%
\pgfpathlineto{\pgfqpoint{4.235673in}{0.686481in}}%
\pgfpathlineto{\pgfqpoint{4.236207in}{0.661713in}}%
\pgfpathlineto{\pgfqpoint{4.236741in}{0.666068in}}%
\pgfpathlineto{\pgfqpoint{4.237275in}{0.690325in}}%
\pgfpathlineto{\pgfqpoint{4.237808in}{0.687246in}}%
\pgfpathlineto{\pgfqpoint{4.238876in}{0.645563in}}%
\pgfpathlineto{\pgfqpoint{4.239409in}{0.681156in}}%
\pgfpathlineto{\pgfqpoint{4.239943in}{0.660150in}}%
\pgfpathlineto{\pgfqpoint{4.240477in}{0.676979in}}%
\pgfpathlineto{\pgfqpoint{4.241010in}{0.651048in}}%
\pgfpathlineto{\pgfqpoint{4.241544in}{0.689799in}}%
\pgfpathlineto{\pgfqpoint{4.242078in}{0.651948in}}%
\pgfpathlineto{\pgfqpoint{4.243145in}{0.651639in}}%
\pgfpathlineto{\pgfqpoint{4.243679in}{0.679598in}}%
\pgfpathlineto{\pgfqpoint{4.244213in}{0.667447in}}%
\pgfpathlineto{\pgfqpoint{4.245280in}{0.718610in}}%
\pgfpathlineto{\pgfqpoint{4.245814in}{0.644526in}}%
\pgfpathlineto{\pgfqpoint{4.246881in}{0.648537in}}%
\pgfpathlineto{\pgfqpoint{4.248482in}{0.727975in}}%
\pgfpathlineto{\pgfqpoint{4.249016in}{0.673259in}}%
\pgfpathlineto{\pgfqpoint{4.249550in}{0.709791in}}%
\pgfpathlineto{\pgfqpoint{4.250083in}{0.700362in}}%
\pgfpathlineto{\pgfqpoint{4.250617in}{0.746517in}}%
\pgfpathlineto{\pgfqpoint{4.251151in}{0.699791in}}%
\pgfpathlineto{\pgfqpoint{4.251684in}{0.699952in}}%
\pgfpathlineto{\pgfqpoint{4.252218in}{0.717951in}}%
\pgfpathlineto{\pgfqpoint{4.252752in}{0.778770in}}%
\pgfpathlineto{\pgfqpoint{4.253286in}{0.719366in}}%
\pgfpathlineto{\pgfqpoint{4.253819in}{0.716279in}}%
\pgfpathlineto{\pgfqpoint{4.254353in}{0.703234in}}%
\pgfpathlineto{\pgfqpoint{4.256488in}{0.766518in}}%
\pgfpathlineto{\pgfqpoint{4.257555in}{0.657367in}}%
\pgfpathlineto{\pgfqpoint{4.258089in}{0.692917in}}%
\pgfpathlineto{\pgfqpoint{4.258623in}{0.682222in}}%
\pgfpathlineto{\pgfqpoint{4.259156in}{0.704631in}}%
\pgfpathlineto{\pgfqpoint{4.259690in}{0.657454in}}%
\pgfpathlineto{\pgfqpoint{4.260224in}{0.747304in}}%
\pgfpathlineto{\pgfqpoint{4.260757in}{0.747214in}}%
\pgfpathlineto{\pgfqpoint{4.261291in}{0.671531in}}%
\pgfpathlineto{\pgfqpoint{4.262358in}{0.671675in}}%
\pgfpathlineto{\pgfqpoint{4.262892in}{0.750954in}}%
\pgfpathlineto{\pgfqpoint{4.263426in}{0.649254in}}%
\pgfpathlineto{\pgfqpoint{4.263960in}{0.790400in}}%
\pgfpathlineto{\pgfqpoint{4.264493in}{0.678966in}}%
\pgfpathlineto{\pgfqpoint{4.265561in}{0.682814in}}%
\pgfpathlineto{\pgfqpoint{4.266094in}{0.731395in}}%
\pgfpathlineto{\pgfqpoint{4.266628in}{0.642719in}}%
\pgfpathlineto{\pgfqpoint{4.267162in}{0.730014in}}%
\pgfpathlineto{\pgfqpoint{4.267695in}{0.662034in}}%
\pgfpathlineto{\pgfqpoint{4.268229in}{0.665431in}}%
\pgfpathlineto{\pgfqpoint{4.268763in}{0.668108in}}%
\pgfpathlineto{\pgfqpoint{4.269830in}{0.697728in}}%
\pgfpathlineto{\pgfqpoint{4.270898in}{0.686446in}}%
\pgfpathlineto{\pgfqpoint{4.271431in}{0.701733in}}%
\pgfpathlineto{\pgfqpoint{4.272499in}{0.647415in}}%
\pgfpathlineto{\pgfqpoint{4.273032in}{0.692932in}}%
\pgfpathlineto{\pgfqpoint{4.273566in}{0.679325in}}%
\pgfpathlineto{\pgfqpoint{4.274100in}{0.644509in}}%
\pgfpathlineto{\pgfqpoint{4.274634in}{0.666808in}}%
\pgfpathlineto{\pgfqpoint{4.275167in}{0.697733in}}%
\pgfpathlineto{\pgfqpoint{4.276768in}{0.649193in}}%
\pgfpathlineto{\pgfqpoint{4.278903in}{0.690910in}}%
\pgfpathlineto{\pgfqpoint{4.279437in}{0.661070in}}%
\pgfpathlineto{\pgfqpoint{4.279971in}{0.667195in}}%
\pgfpathlineto{\pgfqpoint{4.280504in}{0.667916in}}%
\pgfpathlineto{\pgfqpoint{4.281038in}{0.720526in}}%
\pgfpathlineto{\pgfqpoint{4.281572in}{0.693231in}}%
\pgfpathlineto{\pgfqpoint{4.283173in}{0.636654in}}%
\pgfpathlineto{\pgfqpoint{4.283706in}{0.654068in}}%
\pgfpathlineto{\pgfqpoint{4.284240in}{0.713937in}}%
\pgfpathlineto{\pgfqpoint{4.285308in}{0.709617in}}%
\pgfpathlineto{\pgfqpoint{4.286909in}{0.653161in}}%
\pgfpathlineto{\pgfqpoint{4.287976in}{0.662816in}}%
\pgfpathlineto{\pgfqpoint{4.289043in}{0.694191in}}%
\pgfpathlineto{\pgfqpoint{4.289577in}{0.686406in}}%
\pgfpathlineto{\pgfqpoint{4.290111in}{0.688457in}}%
\pgfpathlineto{\pgfqpoint{4.290645in}{0.717711in}}%
\pgfpathlineto{\pgfqpoint{4.291178in}{0.703589in}}%
\pgfpathlineto{\pgfqpoint{4.291712in}{0.715260in}}%
\pgfpathlineto{\pgfqpoint{4.292246in}{0.670291in}}%
\pgfpathlineto{\pgfqpoint{4.292779in}{0.672169in}}%
\pgfpathlineto{\pgfqpoint{4.293847in}{0.704597in}}%
\pgfpathlineto{\pgfqpoint{4.294914in}{0.669416in}}%
\pgfpathlineto{\pgfqpoint{4.295448in}{0.672571in}}%
\pgfpathlineto{\pgfqpoint{4.295982in}{0.660001in}}%
\pgfpathlineto{\pgfqpoint{4.296515in}{0.681983in}}%
\pgfpathlineto{\pgfqpoint{4.297049in}{0.656471in}}%
\pgfpathlineto{\pgfqpoint{4.297583in}{0.671171in}}%
\pgfpathlineto{\pgfqpoint{4.298116in}{0.668034in}}%
\pgfpathlineto{\pgfqpoint{4.298650in}{0.676129in}}%
\pgfpathlineto{\pgfqpoint{4.299184in}{0.655693in}}%
\pgfpathlineto{\pgfqpoint{4.299718in}{0.659792in}}%
\pgfpathlineto{\pgfqpoint{4.300785in}{0.693185in}}%
\pgfpathlineto{\pgfqpoint{4.301319in}{0.668261in}}%
\pgfpathlineto{\pgfqpoint{4.301852in}{0.675068in}}%
\pgfpathlineto{\pgfqpoint{4.302386in}{0.673603in}}%
\pgfpathlineto{\pgfqpoint{4.302920in}{0.724030in}}%
\pgfpathlineto{\pgfqpoint{4.303453in}{0.723068in}}%
\pgfpathlineto{\pgfqpoint{4.304521in}{0.686026in}}%
\pgfpathlineto{\pgfqpoint{4.306122in}{0.722085in}}%
\pgfpathlineto{\pgfqpoint{4.307189in}{0.665439in}}%
\pgfpathlineto{\pgfqpoint{4.307723in}{0.759317in}}%
\pgfpathlineto{\pgfqpoint{4.308257in}{0.758251in}}%
\pgfpathlineto{\pgfqpoint{4.309324in}{0.677412in}}%
\pgfpathlineto{\pgfqpoint{4.309858in}{0.713445in}}%
\pgfpathlineto{\pgfqpoint{4.310392in}{0.682575in}}%
\pgfpathlineto{\pgfqpoint{4.311459in}{0.732893in}}%
\pgfpathlineto{\pgfqpoint{4.313060in}{0.649641in}}%
\pgfpathlineto{\pgfqpoint{4.314127in}{0.723789in}}%
\pgfpathlineto{\pgfqpoint{4.314661in}{0.656065in}}%
\pgfpathlineto{\pgfqpoint{4.315195in}{0.730528in}}%
\pgfpathlineto{\pgfqpoint{4.315729in}{0.715245in}}%
\pgfpathlineto{\pgfqpoint{4.316262in}{0.674333in}}%
\pgfpathlineto{\pgfqpoint{4.316796in}{0.695773in}}%
\pgfpathlineto{\pgfqpoint{4.317330in}{0.700178in}}%
\pgfpathlineto{\pgfqpoint{4.317863in}{0.744700in}}%
\pgfpathlineto{\pgfqpoint{4.318397in}{0.684835in}}%
\pgfpathlineto{\pgfqpoint{4.318931in}{0.709576in}}%
\pgfpathlineto{\pgfqpoint{4.319464in}{0.758860in}}%
\pgfpathlineto{\pgfqpoint{4.319998in}{0.688106in}}%
\pgfpathlineto{\pgfqpoint{4.320532in}{0.721958in}}%
\pgfpathlineto{\pgfqpoint{4.321066in}{0.718558in}}%
\pgfpathlineto{\pgfqpoint{4.321599in}{0.668637in}}%
\pgfpathlineto{\pgfqpoint{4.322133in}{0.688877in}}%
\pgfpathlineto{\pgfqpoint{4.322667in}{0.700580in}}%
\pgfpathlineto{\pgfqpoint{4.323734in}{0.641763in}}%
\pgfpathlineto{\pgfqpoint{4.325335in}{0.714272in}}%
\pgfpathlineto{\pgfqpoint{4.325869in}{0.656542in}}%
\pgfpathlineto{\pgfqpoint{4.326403in}{0.694975in}}%
\pgfpathlineto{\pgfqpoint{4.328004in}{0.663039in}}%
\pgfpathlineto{\pgfqpoint{4.328537in}{0.686866in}}%
\pgfpathlineto{\pgfqpoint{4.329605in}{0.646744in}}%
\pgfpathlineto{\pgfqpoint{4.330138in}{0.709672in}}%
\pgfpathlineto{\pgfqpoint{4.330672in}{0.676247in}}%
\pgfpathlineto{\pgfqpoint{4.331206in}{0.674420in}}%
\pgfpathlineto{\pgfqpoint{4.332807in}{0.643383in}}%
\pgfpathlineto{\pgfqpoint{4.333341in}{0.674273in}}%
\pgfpathlineto{\pgfqpoint{4.333874in}{0.657564in}}%
\pgfpathlineto{\pgfqpoint{4.334408in}{0.651208in}}%
\pgfpathlineto{\pgfqpoint{4.335475in}{0.707108in}}%
\pgfpathlineto{\pgfqpoint{4.336009in}{0.699939in}}%
\pgfpathlineto{\pgfqpoint{4.337610in}{0.652924in}}%
\pgfpathlineto{\pgfqpoint{4.339211in}{0.702914in}}%
\pgfpathlineto{\pgfqpoint{4.340812in}{0.656413in}}%
\pgfpathlineto{\pgfqpoint{4.342414in}{0.704826in}}%
\pgfpathlineto{\pgfqpoint{4.342947in}{0.695575in}}%
\pgfpathlineto{\pgfqpoint{4.343481in}{0.686189in}}%
\pgfpathlineto{\pgfqpoint{4.345082in}{0.642170in}}%
\pgfpathlineto{\pgfqpoint{4.346149in}{0.664967in}}%
\pgfpathlineto{\pgfqpoint{4.347751in}{0.706308in}}%
\pgfpathlineto{\pgfqpoint{4.348818in}{0.669352in}}%
\pgfpathlineto{\pgfqpoint{4.349352in}{0.704125in}}%
\pgfpathlineto{\pgfqpoint{4.349885in}{0.693300in}}%
\pgfpathlineto{\pgfqpoint{4.350419in}{0.684854in}}%
\pgfpathlineto{\pgfqpoint{4.350953in}{0.694993in}}%
\pgfpathlineto{\pgfqpoint{4.352020in}{0.658756in}}%
\pgfpathlineto{\pgfqpoint{4.352554in}{0.691915in}}%
\pgfpathlineto{\pgfqpoint{4.354155in}{0.642245in}}%
\pgfpathlineto{\pgfqpoint{4.354689in}{0.646042in}}%
\pgfpathlineto{\pgfqpoint{4.355222in}{0.641152in}}%
\pgfpathlineto{\pgfqpoint{4.355756in}{0.734915in}}%
\pgfpathlineto{\pgfqpoint{4.356290in}{0.655611in}}%
\pgfpathlineto{\pgfqpoint{4.356823in}{0.693631in}}%
\pgfpathlineto{\pgfqpoint{4.357357in}{0.644658in}}%
\pgfpathlineto{\pgfqpoint{4.358958in}{0.711649in}}%
\pgfpathlineto{\pgfqpoint{4.359492in}{0.655340in}}%
\pgfpathlineto{\pgfqpoint{4.360026in}{0.695427in}}%
\pgfpathlineto{\pgfqpoint{4.360559in}{0.701770in}}%
\pgfpathlineto{\pgfqpoint{4.361093in}{0.731440in}}%
\pgfpathlineto{\pgfqpoint{4.362160in}{0.674578in}}%
\pgfpathlineto{\pgfqpoint{4.362694in}{0.746045in}}%
\pgfpathlineto{\pgfqpoint{4.363228in}{0.739435in}}%
\pgfpathlineto{\pgfqpoint{4.363762in}{0.650169in}}%
\pgfpathlineto{\pgfqpoint{4.364295in}{0.701755in}}%
\pgfpathlineto{\pgfqpoint{4.365363in}{0.651571in}}%
\pgfpathlineto{\pgfqpoint{4.365896in}{0.659553in}}%
\pgfpathlineto{\pgfqpoint{4.366964in}{0.758197in}}%
\pgfpathlineto{\pgfqpoint{4.368031in}{0.663190in}}%
\pgfpathlineto{\pgfqpoint{4.368565in}{0.689886in}}%
\pgfpathlineto{\pgfqpoint{4.369632in}{0.664526in}}%
\pgfpathlineto{\pgfqpoint{4.370166in}{0.679399in}}%
\pgfpathlineto{\pgfqpoint{4.370700in}{0.682392in}}%
\pgfpathlineto{\pgfqpoint{4.371233in}{0.698174in}}%
\pgfpathlineto{\pgfqpoint{4.371767in}{0.665690in}}%
\pgfpathlineto{\pgfqpoint{4.372301in}{0.695069in}}%
\pgfpathlineto{\pgfqpoint{4.372835in}{0.677533in}}%
\pgfpathlineto{\pgfqpoint{4.373368in}{0.718671in}}%
\pgfpathlineto{\pgfqpoint{4.373902in}{0.650603in}}%
\pgfpathlineto{\pgfqpoint{4.374436in}{0.777882in}}%
\pgfpathlineto{\pgfqpoint{4.374969in}{0.666221in}}%
\pgfpathlineto{\pgfqpoint{4.375503in}{0.699417in}}%
\pgfpathlineto{\pgfqpoint{4.376037in}{0.654116in}}%
\pgfpathlineto{\pgfqpoint{4.376570in}{0.713005in}}%
\pgfpathlineto{\pgfqpoint{4.377104in}{0.651811in}}%
\pgfpathlineto{\pgfqpoint{4.377638in}{0.703632in}}%
\pgfpathlineto{\pgfqpoint{4.378172in}{0.653319in}}%
\pgfpathlineto{\pgfqpoint{4.378705in}{0.646018in}}%
\pgfpathlineto{\pgfqpoint{4.379773in}{0.701824in}}%
\pgfpathlineto{\pgfqpoint{4.380306in}{0.679140in}}%
\pgfpathlineto{\pgfqpoint{4.380840in}{0.669883in}}%
\pgfpathlineto{\pgfqpoint{4.381374in}{0.698596in}}%
\pgfpathlineto{\pgfqpoint{4.381907in}{0.661902in}}%
\pgfpathlineto{\pgfqpoint{4.382441in}{0.703110in}}%
\pgfpathlineto{\pgfqpoint{4.382975in}{0.648154in}}%
\pgfpathlineto{\pgfqpoint{4.383509in}{0.669612in}}%
\pgfpathlineto{\pgfqpoint{4.384042in}{0.664413in}}%
\pgfpathlineto{\pgfqpoint{4.384576in}{0.640902in}}%
\pgfpathlineto{\pgfqpoint{4.385110in}{0.655637in}}%
\pgfpathlineto{\pgfqpoint{4.385643in}{0.697462in}}%
\pgfpathlineto{\pgfqpoint{4.386177in}{0.654614in}}%
\pgfpathlineto{\pgfqpoint{4.387244in}{0.714360in}}%
\pgfpathlineto{\pgfqpoint{4.388312in}{0.647881in}}%
\pgfpathlineto{\pgfqpoint{4.390447in}{1.630931in}}%
\pgfpathlineto{\pgfqpoint{4.390980in}{0.755313in}}%
\pgfpathlineto{\pgfqpoint{4.391514in}{1.099046in}}%
\pgfpathlineto{\pgfqpoint{4.393115in}{0.764362in}}%
\pgfpathlineto{\pgfqpoint{4.395784in}{0.644204in}}%
\pgfpathlineto{\pgfqpoint{4.396851in}{0.701500in}}%
\pgfpathlineto{\pgfqpoint{4.397385in}{0.693654in}}%
\pgfpathlineto{\pgfqpoint{4.398986in}{0.653646in}}%
\pgfpathlineto{\pgfqpoint{4.400053in}{0.677116in}}%
\pgfpathlineto{\pgfqpoint{4.400587in}{0.675985in}}%
\pgfpathlineto{\pgfqpoint{4.402188in}{0.642684in}}%
\pgfpathlineto{\pgfqpoint{4.403789in}{0.680293in}}%
\pgfpathlineto{\pgfqpoint{4.404323in}{0.679939in}}%
\pgfpathlineto{\pgfqpoint{4.405390in}{0.674603in}}%
\pgfpathlineto{\pgfqpoint{4.406991in}{0.712191in}}%
\pgfpathlineto{\pgfqpoint{4.407525in}{0.660052in}}%
\pgfpathlineto{\pgfqpoint{4.408059in}{0.660182in}}%
\pgfpathlineto{\pgfqpoint{4.408592in}{0.700586in}}%
\pgfpathlineto{\pgfqpoint{4.409126in}{0.687314in}}%
\pgfpathlineto{\pgfqpoint{4.409660in}{0.694065in}}%
\pgfpathlineto{\pgfqpoint{4.410727in}{0.695166in}}%
\pgfpathlineto{\pgfqpoint{4.411261in}{0.670728in}}%
\pgfpathlineto{\pgfqpoint{4.411795in}{0.702512in}}%
\pgfpathlineto{\pgfqpoint{4.412328in}{0.655440in}}%
\pgfpathlineto{\pgfqpoint{4.412862in}{0.676715in}}%
\pgfpathlineto{\pgfqpoint{4.413396in}{0.681983in}}%
\pgfpathlineto{\pgfqpoint{4.413929in}{0.712478in}}%
\pgfpathlineto{\pgfqpoint{4.414997in}{0.653189in}}%
\pgfpathlineto{\pgfqpoint{4.416064in}{0.703340in}}%
\pgfpathlineto{\pgfqpoint{4.416598in}{0.682904in}}%
\pgfpathlineto{\pgfqpoint{4.417132in}{0.695255in}}%
\pgfpathlineto{\pgfqpoint{4.417665in}{0.695471in}}%
\pgfpathlineto{\pgfqpoint{4.418199in}{0.716608in}}%
\pgfpathlineto{\pgfqpoint{4.419800in}{0.655954in}}%
\pgfpathlineto{\pgfqpoint{4.421935in}{0.731644in}}%
\pgfpathlineto{\pgfqpoint{4.423002in}{0.647879in}}%
\pgfpathlineto{\pgfqpoint{4.423536in}{0.677290in}}%
\pgfpathlineto{\pgfqpoint{4.424070in}{0.674670in}}%
\pgfpathlineto{\pgfqpoint{4.424603in}{0.687459in}}%
\pgfpathlineto{\pgfqpoint{4.425137in}{0.641407in}}%
\pgfpathlineto{\pgfqpoint{4.425671in}{0.693053in}}%
\pgfpathlineto{\pgfqpoint{4.426205in}{0.681741in}}%
\pgfpathlineto{\pgfqpoint{4.426738in}{0.647469in}}%
\pgfpathlineto{\pgfqpoint{4.427272in}{0.655689in}}%
\pgfpathlineto{\pgfqpoint{4.428339in}{0.713993in}}%
\pgfpathlineto{\pgfqpoint{4.428873in}{0.649444in}}%
\pgfpathlineto{\pgfqpoint{4.429407in}{0.706045in}}%
\pgfpathlineto{\pgfqpoint{4.429940in}{0.671670in}}%
\pgfpathlineto{\pgfqpoint{4.430474in}{0.687797in}}%
\pgfpathlineto{\pgfqpoint{4.431008in}{0.675022in}}%
\pgfpathlineto{\pgfqpoint{4.431542in}{0.719192in}}%
\pgfpathlineto{\pgfqpoint{4.432075in}{0.651840in}}%
\pgfpathlineto{\pgfqpoint{4.432609in}{0.670719in}}%
\pgfpathlineto{\pgfqpoint{4.433676in}{0.652097in}}%
\pgfpathlineto{\pgfqpoint{4.435811in}{0.682662in}}%
\pgfpathlineto{\pgfqpoint{4.436879in}{0.664951in}}%
\pgfpathlineto{\pgfqpoint{4.437412in}{0.678232in}}%
\pgfpathlineto{\pgfqpoint{4.438480in}{0.655972in}}%
\pgfpathlineto{\pgfqpoint{4.439013in}{0.667239in}}%
\pgfpathlineto{\pgfqpoint{4.439547in}{0.655681in}}%
\pgfpathlineto{\pgfqpoint{4.440081in}{0.640286in}}%
\pgfpathlineto{\pgfqpoint{4.440615in}{0.690978in}}%
\pgfpathlineto{\pgfqpoint{4.441148in}{0.650778in}}%
\pgfpathlineto{\pgfqpoint{4.441682in}{0.673273in}}%
\pgfpathlineto{\pgfqpoint{4.442216in}{0.658539in}}%
\pgfpathlineto{\pgfqpoint{4.442749in}{0.666467in}}%
\pgfpathlineto{\pgfqpoint{4.444350in}{0.644749in}}%
\pgfpathlineto{\pgfqpoint{4.444884in}{0.669186in}}%
\pgfpathlineto{\pgfqpoint{4.445418in}{0.668167in}}%
\pgfpathlineto{\pgfqpoint{4.445952in}{0.648441in}}%
\pgfpathlineto{\pgfqpoint{4.446485in}{0.660275in}}%
\pgfpathlineto{\pgfqpoint{4.447553in}{0.660694in}}%
\pgfpathlineto{\pgfqpoint{4.448086in}{0.642149in}}%
\pgfpathlineto{\pgfqpoint{4.449687in}{0.663316in}}%
\pgfpathlineto{\pgfqpoint{4.450221in}{0.666737in}}%
\pgfpathlineto{\pgfqpoint{4.450755in}{0.678609in}}%
\pgfpathlineto{\pgfqpoint{4.452890in}{0.643580in}}%
\pgfpathlineto{\pgfqpoint{4.453957in}{0.688847in}}%
\pgfpathlineto{\pgfqpoint{4.454491in}{0.687841in}}%
\pgfpathlineto{\pgfqpoint{4.455558in}{0.640820in}}%
\pgfpathlineto{\pgfqpoint{4.456626in}{0.648003in}}%
\pgfpathlineto{\pgfqpoint{4.458760in}{0.681022in}}%
\pgfpathlineto{\pgfqpoint{4.460361in}{0.641196in}}%
\pgfpathlineto{\pgfqpoint{4.460895in}{0.649808in}}%
\pgfpathlineto{\pgfqpoint{4.461429in}{0.646786in}}%
\pgfpathlineto{\pgfqpoint{4.463030in}{0.701421in}}%
\pgfpathlineto{\pgfqpoint{4.463564in}{0.649271in}}%
\pgfpathlineto{\pgfqpoint{4.464097in}{0.678080in}}%
\pgfpathlineto{\pgfqpoint{4.465165in}{0.672545in}}%
\pgfpathlineto{\pgfqpoint{4.465698in}{0.678942in}}%
\pgfpathlineto{\pgfqpoint{4.466232in}{0.715219in}}%
\pgfpathlineto{\pgfqpoint{4.466766in}{0.647055in}}%
\pgfpathlineto{\pgfqpoint{4.467300in}{0.667573in}}%
\pgfpathlineto{\pgfqpoint{4.467833in}{0.664088in}}%
\pgfpathlineto{\pgfqpoint{4.468901in}{0.696597in}}%
\pgfpathlineto{\pgfqpoint{4.469434in}{0.690585in}}%
\pgfpathlineto{\pgfqpoint{4.469968in}{0.636615in}}%
\pgfpathlineto{\pgfqpoint{4.470502in}{0.677759in}}%
\pgfpathlineto{\pgfqpoint{4.471569in}{0.690408in}}%
\pgfpathlineto{\pgfqpoint{4.472103in}{0.673765in}}%
\pgfpathlineto{\pgfqpoint{4.472637in}{0.674174in}}%
\pgfpathlineto{\pgfqpoint{4.473170in}{0.711668in}}%
\pgfpathlineto{\pgfqpoint{4.473704in}{0.686099in}}%
\pgfpathlineto{\pgfqpoint{4.475305in}{0.638692in}}%
\pgfpathlineto{\pgfqpoint{4.475839in}{0.660520in}}%
\pgfpathlineto{\pgfqpoint{4.476372in}{0.655751in}}%
\pgfpathlineto{\pgfqpoint{4.477440in}{0.709302in}}%
\pgfpathlineto{\pgfqpoint{4.477974in}{0.666262in}}%
\pgfpathlineto{\pgfqpoint{4.478507in}{0.673542in}}%
\pgfpathlineto{\pgfqpoint{4.479041in}{0.688769in}}%
\pgfpathlineto{\pgfqpoint{4.480108in}{0.641375in}}%
\pgfpathlineto{\pgfqpoint{4.481709in}{0.677786in}}%
\pgfpathlineto{\pgfqpoint{4.482243in}{0.648328in}}%
\pgfpathlineto{\pgfqpoint{4.482777in}{0.690917in}}%
\pgfpathlineto{\pgfqpoint{4.483311in}{0.684497in}}%
\pgfpathlineto{\pgfqpoint{4.484378in}{0.651589in}}%
\pgfpathlineto{\pgfqpoint{4.484912in}{0.701116in}}%
\pgfpathlineto{\pgfqpoint{4.485445in}{0.672579in}}%
\pgfpathlineto{\pgfqpoint{4.485979in}{0.690556in}}%
\pgfpathlineto{\pgfqpoint{4.486513in}{0.683384in}}%
\pgfpathlineto{\pgfqpoint{4.487046in}{0.684042in}}%
\pgfpathlineto{\pgfqpoint{4.487580in}{0.647121in}}%
\pgfpathlineto{\pgfqpoint{4.488114in}{0.676213in}}%
\pgfpathlineto{\pgfqpoint{4.489181in}{0.652805in}}%
\pgfpathlineto{\pgfqpoint{4.490782in}{0.679528in}}%
\pgfpathlineto{\pgfqpoint{4.491316in}{0.646178in}}%
\pgfpathlineto{\pgfqpoint{4.491850in}{0.665462in}}%
\pgfpathlineto{\pgfqpoint{4.492383in}{0.665777in}}%
\pgfpathlineto{\pgfqpoint{4.492917in}{0.672153in}}%
\pgfpathlineto{\pgfqpoint{4.494518in}{0.640890in}}%
\pgfpathlineto{\pgfqpoint{4.495586in}{0.676783in}}%
\pgfpathlineto{\pgfqpoint{4.497187in}{0.635874in}}%
\pgfpathlineto{\pgfqpoint{4.498254in}{0.658656in}}%
\pgfpathlineto{\pgfqpoint{4.499322in}{0.641187in}}%
\pgfpathlineto{\pgfqpoint{4.500923in}{0.662796in}}%
\pgfpathlineto{\pgfqpoint{4.501456in}{0.662490in}}%
\pgfpathlineto{\pgfqpoint{4.504125in}{0.636045in}}%
\pgfpathlineto{\pgfqpoint{4.504659in}{0.671124in}}%
\pgfpathlineto{\pgfqpoint{4.505192in}{0.651409in}}%
\pgfpathlineto{\pgfqpoint{4.505726in}{0.662613in}}%
\pgfpathlineto{\pgfqpoint{4.506260in}{0.644298in}}%
\pgfpathlineto{\pgfqpoint{4.506793in}{0.651149in}}%
\pgfpathlineto{\pgfqpoint{4.508928in}{0.672068in}}%
\pgfpathlineto{\pgfqpoint{4.509462in}{0.639987in}}%
\pgfpathlineto{\pgfqpoint{4.509996in}{0.647783in}}%
\pgfpathlineto{\pgfqpoint{4.511063in}{0.681635in}}%
\pgfpathlineto{\pgfqpoint{4.511597in}{0.663958in}}%
\pgfpathlineto{\pgfqpoint{4.512130in}{0.646805in}}%
\pgfpathlineto{\pgfqpoint{4.512664in}{0.666706in}}%
\pgfpathlineto{\pgfqpoint{4.513198in}{0.648863in}}%
\pgfpathlineto{\pgfqpoint{4.514265in}{0.658555in}}%
\pgfpathlineto{\pgfqpoint{4.514799in}{0.672818in}}%
\pgfpathlineto{\pgfqpoint{4.515333in}{0.668136in}}%
\pgfpathlineto{\pgfqpoint{4.516934in}{0.651119in}}%
\pgfpathlineto{\pgfqpoint{4.518001in}{0.658444in}}%
\pgfpathlineto{\pgfqpoint{4.518535in}{0.652711in}}%
\pgfpathlineto{\pgfqpoint{4.519069in}{0.680420in}}%
\pgfpathlineto{\pgfqpoint{4.519602in}{0.648492in}}%
\pgfpathlineto{\pgfqpoint{4.520136in}{0.677670in}}%
\pgfpathlineto{\pgfqpoint{4.520670in}{0.663547in}}%
\pgfpathlineto{\pgfqpoint{4.521203in}{0.670246in}}%
\pgfpathlineto{\pgfqpoint{4.522271in}{0.700827in}}%
\pgfpathlineto{\pgfqpoint{4.523338in}{0.662178in}}%
\pgfpathlineto{\pgfqpoint{4.523872in}{0.697101in}}%
\pgfpathlineto{\pgfqpoint{4.525473in}{0.651588in}}%
\pgfpathlineto{\pgfqpoint{4.526007in}{0.721285in}}%
\pgfpathlineto{\pgfqpoint{4.526540in}{0.666084in}}%
\pgfpathlineto{\pgfqpoint{4.527074in}{0.660269in}}%
\pgfpathlineto{\pgfqpoint{4.527608in}{0.664131in}}%
\pgfpathlineto{\pgfqpoint{4.528141in}{0.697275in}}%
\pgfpathlineto{\pgfqpoint{4.528675in}{0.673506in}}%
\pgfpathlineto{\pgfqpoint{4.529209in}{0.640693in}}%
\pgfpathlineto{\pgfqpoint{4.529743in}{0.677480in}}%
\pgfpathlineto{\pgfqpoint{4.530276in}{0.659484in}}%
\pgfpathlineto{\pgfqpoint{4.531877in}{0.670101in}}%
\pgfpathlineto{\pgfqpoint{4.532411in}{0.701712in}}%
\pgfpathlineto{\pgfqpoint{4.533478in}{0.648631in}}%
\pgfpathlineto{\pgfqpoint{4.534012in}{0.676788in}}%
\pgfpathlineto{\pgfqpoint{4.534546in}{0.670009in}}%
\pgfpathlineto{\pgfqpoint{4.535613in}{0.649680in}}%
\pgfpathlineto{\pgfqpoint{4.536681in}{0.667595in}}%
\pgfpathlineto{\pgfqpoint{4.537214in}{0.654446in}}%
\pgfpathlineto{\pgfqpoint{4.537748in}{0.674307in}}%
\pgfpathlineto{\pgfqpoint{4.539349in}{0.639960in}}%
\pgfpathlineto{\pgfqpoint{4.539883in}{0.678884in}}%
\pgfpathlineto{\pgfqpoint{4.540950in}{0.678210in}}%
\pgfpathlineto{\pgfqpoint{4.541484in}{0.646168in}}%
\pgfpathlineto{\pgfqpoint{4.542018in}{0.693374in}}%
\pgfpathlineto{\pgfqpoint{4.542551in}{0.647533in}}%
\pgfpathlineto{\pgfqpoint{4.543085in}{0.673335in}}%
\pgfpathlineto{\pgfqpoint{4.543619in}{0.645632in}}%
\pgfpathlineto{\pgfqpoint{4.544152in}{0.649119in}}%
\pgfpathlineto{\pgfqpoint{4.545754in}{0.662705in}}%
\pgfpathlineto{\pgfqpoint{4.546287in}{0.647062in}}%
\pgfpathlineto{\pgfqpoint{4.546821in}{0.659631in}}%
\pgfpathlineto{\pgfqpoint{4.547355in}{0.661162in}}%
\pgfpathlineto{\pgfqpoint{4.547888in}{0.672506in}}%
\pgfpathlineto{\pgfqpoint{4.549489in}{0.636336in}}%
\pgfpathlineto{\pgfqpoint{4.550023in}{0.666545in}}%
\pgfpathlineto{\pgfqpoint{4.550557in}{0.653844in}}%
\pgfpathlineto{\pgfqpoint{4.551624in}{0.647752in}}%
\pgfpathlineto{\pgfqpoint{4.552692in}{0.659449in}}%
\pgfpathlineto{\pgfqpoint{4.553225in}{0.654736in}}%
\pgfpathlineto{\pgfqpoint{4.555360in}{0.642187in}}%
\pgfpathlineto{\pgfqpoint{4.555894in}{0.652908in}}%
\pgfpathlineto{\pgfqpoint{4.556428in}{0.645019in}}%
\pgfpathlineto{\pgfqpoint{4.556961in}{0.650993in}}%
\pgfpathlineto{\pgfqpoint{4.557495in}{0.646269in}}%
\pgfpathlineto{\pgfqpoint{4.558029in}{0.648431in}}%
\pgfpathlineto{\pgfqpoint{4.559096in}{0.636423in}}%
\pgfpathlineto{\pgfqpoint{4.560163in}{0.665889in}}%
\pgfpathlineto{\pgfqpoint{4.561231in}{0.639401in}}%
\pgfpathlineto{\pgfqpoint{4.562832in}{0.655193in}}%
\pgfpathlineto{\pgfqpoint{4.563366in}{0.639286in}}%
\pgfpathlineto{\pgfqpoint{4.563899in}{0.644086in}}%
\pgfpathlineto{\pgfqpoint{4.564433in}{0.646976in}}%
\pgfpathlineto{\pgfqpoint{4.566034in}{0.675067in}}%
\pgfpathlineto{\pgfqpoint{4.567635in}{0.638848in}}%
\pgfpathlineto{\pgfqpoint{4.569236in}{0.660643in}}%
\pgfpathlineto{\pgfqpoint{4.569770in}{0.656526in}}%
\pgfpathlineto{\pgfqpoint{4.570304in}{0.662277in}}%
\pgfpathlineto{\pgfqpoint{4.571905in}{0.649055in}}%
\pgfpathlineto{\pgfqpoint{4.573506in}{0.640899in}}%
\pgfpathlineto{\pgfqpoint{4.574040in}{0.694682in}}%
\pgfpathlineto{\pgfqpoint{4.574573in}{0.639561in}}%
\pgfpathlineto{\pgfqpoint{4.576175in}{0.666023in}}%
\pgfpathlineto{\pgfqpoint{4.576708in}{0.672269in}}%
\pgfpathlineto{\pgfqpoint{4.578309in}{0.649527in}}%
\pgfpathlineto{\pgfqpoint{4.579910in}{0.679596in}}%
\pgfpathlineto{\pgfqpoint{4.580444in}{0.656417in}}%
\pgfpathlineto{\pgfqpoint{4.580978in}{0.678302in}}%
\pgfpathlineto{\pgfqpoint{4.581512in}{0.680985in}}%
\pgfpathlineto{\pgfqpoint{4.583113in}{0.644286in}}%
\pgfpathlineto{\pgfqpoint{4.583646in}{0.674682in}}%
\pgfpathlineto{\pgfqpoint{4.584180in}{0.642336in}}%
\pgfpathlineto{\pgfqpoint{4.585247in}{0.679767in}}%
\pgfpathlineto{\pgfqpoint{4.586849in}{0.648704in}}%
\pgfpathlineto{\pgfqpoint{4.587916in}{0.666557in}}%
\pgfpathlineto{\pgfqpoint{4.588450in}{0.648093in}}%
\pgfpathlineto{\pgfqpoint{4.588983in}{0.666618in}}%
\pgfpathlineto{\pgfqpoint{4.589517in}{0.671844in}}%
\pgfpathlineto{\pgfqpoint{4.590051in}{0.644623in}}%
\pgfpathlineto{\pgfqpoint{4.590584in}{0.646821in}}%
\pgfpathlineto{\pgfqpoint{4.591652in}{0.658794in}}%
\pgfpathlineto{\pgfqpoint{4.592719in}{0.635934in}}%
\pgfpathlineto{\pgfqpoint{4.593787in}{0.659544in}}%
\pgfpathlineto{\pgfqpoint{4.594320in}{0.640575in}}%
\pgfpathlineto{\pgfqpoint{4.594854in}{0.643826in}}%
\pgfpathlineto{\pgfqpoint{4.595921in}{0.660279in}}%
\pgfpathlineto{\pgfqpoint{4.596455in}{0.659524in}}%
\pgfpathlineto{\pgfqpoint{4.596989in}{0.670892in}}%
\pgfpathlineto{\pgfqpoint{4.598056in}{0.650004in}}%
\pgfpathlineto{\pgfqpoint{4.598590in}{0.658483in}}%
\pgfpathlineto{\pgfqpoint{4.599124in}{0.659606in}}%
\pgfpathlineto{\pgfqpoint{4.599657in}{0.657034in}}%
\pgfpathlineto{\pgfqpoint{4.600191in}{0.644720in}}%
\pgfpathlineto{\pgfqpoint{4.600725in}{0.649338in}}%
\pgfpathlineto{\pgfqpoint{4.601258in}{0.655439in}}%
\pgfpathlineto{\pgfqpoint{4.601792in}{0.651124in}}%
\pgfpathlineto{\pgfqpoint{4.602326in}{0.642509in}}%
\pgfpathlineto{\pgfqpoint{4.602860in}{0.659731in}}%
\pgfpathlineto{\pgfqpoint{4.603393in}{0.658472in}}%
\pgfpathlineto{\pgfqpoint{4.603927in}{0.642524in}}%
\pgfpathlineto{\pgfqpoint{4.604461in}{0.645931in}}%
\pgfpathlineto{\pgfqpoint{4.606062in}{0.652498in}}%
\pgfpathlineto{\pgfqpoint{4.606595in}{0.638212in}}%
\pgfpathlineto{\pgfqpoint{4.607129in}{0.656245in}}%
\pgfpathlineto{\pgfqpoint{4.607663in}{0.642980in}}%
\pgfpathlineto{\pgfqpoint{4.609264in}{0.657868in}}%
\pgfpathlineto{\pgfqpoint{4.609798in}{0.636312in}}%
\pgfpathlineto{\pgfqpoint{4.610331in}{0.653456in}}%
\pgfpathlineto{\pgfqpoint{4.611399in}{0.643056in}}%
\pgfpathlineto{\pgfqpoint{4.611932in}{0.650512in}}%
\pgfpathlineto{\pgfqpoint{4.612466in}{0.644341in}}%
\pgfpathlineto{\pgfqpoint{4.613534in}{0.635738in}}%
\pgfpathlineto{\pgfqpoint{4.615135in}{0.657040in}}%
\pgfpathlineto{\pgfqpoint{4.615668in}{0.638430in}}%
\pgfpathlineto{\pgfqpoint{4.616202in}{0.648098in}}%
\pgfpathlineto{\pgfqpoint{4.616736in}{0.652720in}}%
\pgfpathlineto{\pgfqpoint{4.618337in}{0.637315in}}%
\pgfpathlineto{\pgfqpoint{4.618871in}{0.641694in}}%
\pgfpathlineto{\pgfqpoint{4.619404in}{0.639541in}}%
\pgfpathlineto{\pgfqpoint{4.619938in}{0.659878in}}%
\pgfpathlineto{\pgfqpoint{4.621005in}{0.659702in}}%
\pgfpathlineto{\pgfqpoint{4.621539in}{0.640042in}}%
\pgfpathlineto{\pgfqpoint{4.622073in}{0.650730in}}%
\pgfpathlineto{\pgfqpoint{4.622606in}{0.654550in}}%
\pgfpathlineto{\pgfqpoint{4.623140in}{0.638340in}}%
\pgfpathlineto{\pgfqpoint{4.623674in}{0.650181in}}%
\pgfpathlineto{\pgfqpoint{4.624208in}{0.650116in}}%
\pgfpathlineto{\pgfqpoint{4.624741in}{0.638334in}}%
\pgfpathlineto{\pgfqpoint{4.625275in}{0.646612in}}%
\pgfpathlineto{\pgfqpoint{4.625809in}{0.649784in}}%
\pgfpathlineto{\pgfqpoint{4.626342in}{0.659903in}}%
\pgfpathlineto{\pgfqpoint{4.626876in}{0.650683in}}%
\pgfpathlineto{\pgfqpoint{4.627410in}{0.652958in}}%
\pgfpathlineto{\pgfqpoint{4.628477in}{0.641486in}}%
\pgfpathlineto{\pgfqpoint{4.629011in}{0.654813in}}%
\pgfpathlineto{\pgfqpoint{4.629545in}{0.651223in}}%
\pgfpathlineto{\pgfqpoint{4.630078in}{0.651167in}}%
\pgfpathlineto{\pgfqpoint{4.630612in}{0.641624in}}%
\pgfpathlineto{\pgfqpoint{4.631146in}{0.644708in}}%
\pgfpathlineto{\pgfqpoint{4.631679in}{0.672957in}}%
\pgfpathlineto{\pgfqpoint{4.632213in}{0.643093in}}%
\pgfpathlineto{\pgfqpoint{4.632747in}{0.643068in}}%
\pgfpathlineto{\pgfqpoint{4.633280in}{0.641753in}}%
\pgfpathlineto{\pgfqpoint{4.633814in}{0.652266in}}%
\pgfpathlineto{\pgfqpoint{4.634348in}{0.646393in}}%
\pgfpathlineto{\pgfqpoint{4.635415in}{0.653092in}}%
\pgfpathlineto{\pgfqpoint{4.635949in}{0.640510in}}%
\pgfpathlineto{\pgfqpoint{4.636483in}{0.685641in}}%
\pgfpathlineto{\pgfqpoint{4.637016in}{0.643017in}}%
\pgfpathlineto{\pgfqpoint{4.637550in}{0.654557in}}%
\pgfpathlineto{\pgfqpoint{4.638084in}{0.647034in}}%
\pgfpathlineto{\pgfqpoint{4.639151in}{0.660139in}}%
\pgfpathlineto{\pgfqpoint{4.639685in}{0.647652in}}%
\pgfpathlineto{\pgfqpoint{4.640219in}{0.661945in}}%
\pgfpathlineto{\pgfqpoint{4.640752in}{0.660482in}}%
\pgfpathlineto{\pgfqpoint{4.641820in}{0.642550in}}%
\pgfpathlineto{\pgfqpoint{4.642353in}{0.642695in}}%
\pgfpathlineto{\pgfqpoint{4.642887in}{0.664535in}}%
\pgfpathlineto{\pgfqpoint{4.643421in}{0.645846in}}%
\pgfpathlineto{\pgfqpoint{4.643955in}{0.652540in}}%
\pgfpathlineto{\pgfqpoint{4.644488in}{0.650081in}}%
\pgfpathlineto{\pgfqpoint{4.645556in}{0.640997in}}%
\pgfpathlineto{\pgfqpoint{4.646089in}{0.641698in}}%
\pgfpathlineto{\pgfqpoint{4.646623in}{0.655251in}}%
\pgfpathlineto{\pgfqpoint{4.647157in}{0.651654in}}%
\pgfpathlineto{\pgfqpoint{4.648758in}{0.640964in}}%
\pgfpathlineto{\pgfqpoint{4.649292in}{0.641461in}}%
\pgfpathlineto{\pgfqpoint{4.649825in}{0.638578in}}%
\pgfpathlineto{\pgfqpoint{4.651426in}{0.651611in}}%
\pgfpathlineto{\pgfqpoint{4.651960in}{0.639986in}}%
\pgfpathlineto{\pgfqpoint{4.652494in}{0.659314in}}%
\pgfpathlineto{\pgfqpoint{4.653027in}{0.649247in}}%
\pgfpathlineto{\pgfqpoint{4.653561in}{0.654827in}}%
\pgfpathlineto{\pgfqpoint{4.654095in}{0.644639in}}%
\pgfpathlineto{\pgfqpoint{4.654629in}{0.658878in}}%
\pgfpathlineto{\pgfqpoint{4.655162in}{0.643590in}}%
\pgfpathlineto{\pgfqpoint{4.655696in}{0.637006in}}%
\pgfpathlineto{\pgfqpoint{4.656230in}{0.639950in}}%
\pgfpathlineto{\pgfqpoint{4.656763in}{0.638938in}}%
\pgfpathlineto{\pgfqpoint{4.658364in}{0.660654in}}%
\pgfpathlineto{\pgfqpoint{4.658898in}{0.640645in}}%
\pgfpathlineto{\pgfqpoint{4.659432in}{0.645099in}}%
\pgfpathlineto{\pgfqpoint{4.659966in}{0.645226in}}%
\pgfpathlineto{\pgfqpoint{4.660499in}{0.647449in}}%
\pgfpathlineto{\pgfqpoint{4.661033in}{0.646047in}}%
\pgfpathlineto{\pgfqpoint{4.661567in}{0.638924in}}%
\pgfpathlineto{\pgfqpoint{4.662100in}{0.643756in}}%
\pgfpathlineto{\pgfqpoint{4.662634in}{0.642839in}}%
\pgfpathlineto{\pgfqpoint{4.663168in}{0.643596in}}%
\pgfpathlineto{\pgfqpoint{4.664235in}{0.647353in}}%
\pgfpathlineto{\pgfqpoint{4.664769in}{0.644847in}}%
\pgfpathlineto{\pgfqpoint{4.665303in}{0.641210in}}%
\pgfpathlineto{\pgfqpoint{4.665836in}{0.641629in}}%
\pgfpathlineto{\pgfqpoint{4.666370in}{0.647211in}}%
\pgfpathlineto{\pgfqpoint{4.666904in}{0.645855in}}%
\pgfpathlineto{\pgfqpoint{4.668505in}{0.639042in}}%
\pgfpathlineto{\pgfqpoint{4.669038in}{0.640856in}}%
\pgfpathlineto{\pgfqpoint{4.669572in}{0.639594in}}%
\pgfpathlineto{\pgfqpoint{4.670106in}{0.637470in}}%
\pgfpathlineto{\pgfqpoint{4.670640in}{0.637692in}}%
\pgfpathlineto{\pgfqpoint{4.671173in}{0.649890in}}%
\pgfpathlineto{\pgfqpoint{4.671707in}{0.640679in}}%
\pgfpathlineto{\pgfqpoint{4.672241in}{0.644541in}}%
\pgfpathlineto{\pgfqpoint{4.673842in}{0.637314in}}%
\pgfpathlineto{\pgfqpoint{4.674909in}{0.645842in}}%
\pgfpathlineto{\pgfqpoint{4.675443in}{0.644587in}}%
\pgfpathlineto{\pgfqpoint{4.676510in}{0.636164in}}%
\pgfpathlineto{\pgfqpoint{4.677044in}{0.649595in}}%
\pgfpathlineto{\pgfqpoint{4.677578in}{0.648483in}}%
\pgfpathlineto{\pgfqpoint{4.678645in}{0.638525in}}%
\pgfpathlineto{\pgfqpoint{4.679179in}{0.638960in}}%
\pgfpathlineto{\pgfqpoint{4.679712in}{0.640245in}}%
\pgfpathlineto{\pgfqpoint{4.681314in}{0.650275in}}%
\pgfpathlineto{\pgfqpoint{4.682381in}{0.639090in}}%
\pgfpathlineto{\pgfqpoint{4.683448in}{0.647938in}}%
\pgfpathlineto{\pgfqpoint{4.683982in}{0.645561in}}%
\pgfpathlineto{\pgfqpoint{4.684516in}{0.643934in}}%
\pgfpathlineto{\pgfqpoint{4.685049in}{0.646346in}}%
\pgfpathlineto{\pgfqpoint{4.685583in}{0.642972in}}%
\pgfpathlineto{\pgfqpoint{4.686117in}{0.643082in}}%
\pgfpathlineto{\pgfqpoint{4.686651in}{0.649821in}}%
\pgfpathlineto{\pgfqpoint{4.687184in}{0.649431in}}%
\pgfpathlineto{\pgfqpoint{4.688252in}{0.641995in}}%
\pgfpathlineto{\pgfqpoint{4.688785in}{0.644316in}}%
\pgfpathlineto{\pgfqpoint{4.689853in}{0.645580in}}%
\pgfpathlineto{\pgfqpoint{4.690920in}{0.640398in}}%
\pgfpathlineto{\pgfqpoint{4.691454in}{0.654087in}}%
\pgfpathlineto{\pgfqpoint{4.691988in}{0.643274in}}%
\pgfpathlineto{\pgfqpoint{4.692521in}{0.646334in}}%
\pgfpathlineto{\pgfqpoint{4.693055in}{0.645465in}}%
\pgfpathlineto{\pgfqpoint{4.693589in}{0.638802in}}%
\pgfpathlineto{\pgfqpoint{4.694122in}{0.648189in}}%
\pgfpathlineto{\pgfqpoint{4.694656in}{0.641315in}}%
\pgfpathlineto{\pgfqpoint{4.695190in}{0.643094in}}%
\pgfpathlineto{\pgfqpoint{4.695723in}{0.660702in}}%
\pgfpathlineto{\pgfqpoint{4.696257in}{0.639201in}}%
\pgfpathlineto{\pgfqpoint{4.696791in}{0.657931in}}%
\pgfpathlineto{\pgfqpoint{4.698392in}{0.638745in}}%
\pgfpathlineto{\pgfqpoint{4.698926in}{0.641047in}}%
\pgfpathlineto{\pgfqpoint{4.699459in}{0.637828in}}%
\pgfpathlineto{\pgfqpoint{4.699993in}{0.655083in}}%
\pgfpathlineto{\pgfqpoint{4.700527in}{0.645433in}}%
\pgfpathlineto{\pgfqpoint{4.701060in}{0.640395in}}%
\pgfpathlineto{\pgfqpoint{4.701594in}{0.644741in}}%
\pgfpathlineto{\pgfqpoint{4.703195in}{0.640735in}}%
\pgfpathlineto{\pgfqpoint{4.703729in}{0.645288in}}%
\pgfpathlineto{\pgfqpoint{4.704263in}{0.638074in}}%
\pgfpathlineto{\pgfqpoint{4.704796in}{0.640256in}}%
\pgfpathlineto{\pgfqpoint{4.705330in}{0.639896in}}%
\pgfpathlineto{\pgfqpoint{4.706398in}{0.636987in}}%
\pgfpathlineto{\pgfqpoint{4.706931in}{0.644959in}}%
\pgfpathlineto{\pgfqpoint{4.707465in}{0.644878in}}%
\pgfpathlineto{\pgfqpoint{4.709066in}{0.641411in}}%
\pgfpathlineto{\pgfqpoint{4.709600in}{0.644999in}}%
\pgfpathlineto{\pgfqpoint{4.710133in}{0.643066in}}%
\pgfpathlineto{\pgfqpoint{4.710667in}{0.636084in}}%
\pgfpathlineto{\pgfqpoint{4.711201in}{0.637407in}}%
\pgfpathlineto{\pgfqpoint{4.711735in}{0.637873in}}%
\pgfpathlineto{\pgfqpoint{4.712268in}{0.647258in}}%
\pgfpathlineto{\pgfqpoint{4.712802in}{0.636890in}}%
\pgfpathlineto{\pgfqpoint{4.713336in}{0.643543in}}%
\pgfpathlineto{\pgfqpoint{4.713869in}{0.641530in}}%
\pgfpathlineto{\pgfqpoint{4.714403in}{0.642884in}}%
\pgfpathlineto{\pgfqpoint{4.714937in}{0.642329in}}%
\pgfpathlineto{\pgfqpoint{4.716538in}{0.638744in}}%
\pgfpathlineto{\pgfqpoint{4.717072in}{0.640025in}}%
\pgfpathlineto{\pgfqpoint{4.717605in}{0.636694in}}%
\pgfpathlineto{\pgfqpoint{4.718139in}{0.644335in}}%
\pgfpathlineto{\pgfqpoint{4.718673in}{0.639312in}}%
\pgfpathlineto{\pgfqpoint{4.720274in}{0.642446in}}%
\pgfpathlineto{\pgfqpoint{4.721875in}{0.638368in}}%
\pgfpathlineto{\pgfqpoint{4.723476in}{0.639560in}}%
\pgfpathlineto{\pgfqpoint{4.724010in}{0.638513in}}%
\pgfpathlineto{\pgfqpoint{4.724543in}{0.648066in}}%
\pgfpathlineto{\pgfqpoint{4.725077in}{0.637006in}}%
\pgfpathlineto{\pgfqpoint{4.725611in}{0.638887in}}%
\pgfpathlineto{\pgfqpoint{4.726144in}{0.645712in}}%
\pgfpathlineto{\pgfqpoint{4.726678in}{0.644150in}}%
\pgfpathlineto{\pgfqpoint{4.727212in}{0.640335in}}%
\pgfpathlineto{\pgfqpoint{4.727746in}{0.641238in}}%
\pgfpathlineto{\pgfqpoint{4.728279in}{0.643793in}}%
\pgfpathlineto{\pgfqpoint{4.728813in}{0.642579in}}%
\pgfpathlineto{\pgfqpoint{4.729880in}{0.638662in}}%
\pgfpathlineto{\pgfqpoint{4.730414in}{0.643843in}}%
\pgfpathlineto{\pgfqpoint{4.730948in}{0.635604in}}%
\pgfpathlineto{\pgfqpoint{4.731481in}{0.636106in}}%
\pgfpathlineto{\pgfqpoint{4.733083in}{0.639201in}}%
\pgfpathlineto{\pgfqpoint{4.733616in}{0.646481in}}%
\pgfpathlineto{\pgfqpoint{4.734150in}{0.642316in}}%
\pgfpathlineto{\pgfqpoint{4.734684in}{0.637543in}}%
\pgfpathlineto{\pgfqpoint{4.735217in}{0.638626in}}%
\pgfpathlineto{\pgfqpoint{4.735751in}{0.639808in}}%
\pgfpathlineto{\pgfqpoint{4.736285in}{0.637950in}}%
\pgfpathlineto{\pgfqpoint{4.737352in}{0.644114in}}%
\pgfpathlineto{\pgfqpoint{4.737886in}{0.639316in}}%
\pgfpathlineto{\pgfqpoint{4.738420in}{0.644347in}}%
\pgfpathlineto{\pgfqpoint{4.738953in}{0.643314in}}%
\pgfpathlineto{\pgfqpoint{4.739487in}{0.645622in}}%
\pgfpathlineto{\pgfqpoint{4.740554in}{0.639154in}}%
\pgfpathlineto{\pgfqpoint{4.741088in}{0.639728in}}%
\pgfpathlineto{\pgfqpoint{4.741622in}{0.640842in}}%
\pgfpathlineto{\pgfqpoint{4.742689in}{0.638760in}}%
\pgfpathlineto{\pgfqpoint{4.744290in}{0.648514in}}%
\pgfpathlineto{\pgfqpoint{4.745891in}{0.635645in}}%
\pgfpathlineto{\pgfqpoint{4.747492in}{0.648367in}}%
\pgfpathlineto{\pgfqpoint{4.748560in}{0.636343in}}%
\pgfpathlineto{\pgfqpoint{4.749627in}{0.645089in}}%
\pgfpathlineto{\pgfqpoint{4.750161in}{0.637271in}}%
\pgfpathlineto{\pgfqpoint{4.750695in}{0.643006in}}%
\pgfpathlineto{\pgfqpoint{4.751228in}{0.646798in}}%
\pgfpathlineto{\pgfqpoint{4.752296in}{0.636975in}}%
\pgfpathlineto{\pgfqpoint{4.753897in}{0.648942in}}%
\pgfpathlineto{\pgfqpoint{4.754964in}{0.640043in}}%
\pgfpathlineto{\pgfqpoint{4.755498in}{0.647464in}}%
\pgfpathlineto{\pgfqpoint{4.756032in}{0.645446in}}%
\pgfpathlineto{\pgfqpoint{4.756565in}{0.636016in}}%
\pgfpathlineto{\pgfqpoint{4.757099in}{0.645641in}}%
\pgfpathlineto{\pgfqpoint{4.758166in}{0.636337in}}%
\pgfpathlineto{\pgfqpoint{4.759768in}{0.642624in}}%
\pgfpathlineto{\pgfqpoint{4.761369in}{0.640494in}}%
\pgfpathlineto{\pgfqpoint{4.761902in}{0.646054in}}%
\pgfpathlineto{\pgfqpoint{4.762436in}{0.640136in}}%
\pgfpathlineto{\pgfqpoint{4.762970in}{0.643058in}}%
\pgfpathlineto{\pgfqpoint{4.763503in}{0.640222in}}%
\pgfpathlineto{\pgfqpoint{4.764037in}{0.642168in}}%
\pgfpathlineto{\pgfqpoint{4.765105in}{0.648770in}}%
\pgfpathlineto{\pgfqpoint{4.765638in}{0.637640in}}%
\pgfpathlineto{\pgfqpoint{4.766172in}{0.641850in}}%
\pgfpathlineto{\pgfqpoint{4.766706in}{0.639414in}}%
\pgfpathlineto{\pgfqpoint{4.768307in}{0.644640in}}%
\pgfpathlineto{\pgfqpoint{4.769374in}{0.646558in}}%
\pgfpathlineto{\pgfqpoint{4.769908in}{0.642869in}}%
\pgfpathlineto{\pgfqpoint{4.770442in}{0.649453in}}%
\pgfpathlineto{\pgfqpoint{4.771509in}{0.638457in}}%
\pgfpathlineto{\pgfqpoint{4.772043in}{0.639773in}}%
\pgfpathlineto{\pgfqpoint{4.772576in}{0.645242in}}%
\pgfpathlineto{\pgfqpoint{4.773110in}{0.638607in}}%
\pgfpathlineto{\pgfqpoint{4.773644in}{0.642316in}}%
\pgfpathlineto{\pgfqpoint{4.774178in}{0.639686in}}%
\pgfpathlineto{\pgfqpoint{4.774711in}{0.645679in}}%
\pgfpathlineto{\pgfqpoint{4.775245in}{0.637568in}}%
\pgfpathlineto{\pgfqpoint{4.775779in}{0.638875in}}%
\pgfpathlineto{\pgfqpoint{4.776312in}{0.638651in}}%
\pgfpathlineto{\pgfqpoint{4.776846in}{0.640022in}}%
\pgfpathlineto{\pgfqpoint{4.777380in}{0.636138in}}%
\pgfpathlineto{\pgfqpoint{4.777913in}{0.643491in}}%
\pgfpathlineto{\pgfqpoint{4.778447in}{0.637154in}}%
\pgfpathlineto{\pgfqpoint{4.779515in}{0.646517in}}%
\pgfpathlineto{\pgfqpoint{4.780048in}{0.643961in}}%
\pgfpathlineto{\pgfqpoint{4.782183in}{0.637113in}}%
\pgfpathlineto{\pgfqpoint{4.783250in}{0.646286in}}%
\pgfpathlineto{\pgfqpoint{4.783784in}{0.640489in}}%
\pgfpathlineto{\pgfqpoint{4.784852in}{0.640877in}}%
\pgfpathlineto{\pgfqpoint{4.785385in}{0.639482in}}%
\pgfpathlineto{\pgfqpoint{4.785919in}{0.639742in}}%
\pgfpathlineto{\pgfqpoint{4.786453in}{0.641814in}}%
\pgfpathlineto{\pgfqpoint{4.786986in}{0.640895in}}%
\pgfpathlineto{\pgfqpoint{4.788054in}{0.636339in}}%
\pgfpathlineto{\pgfqpoint{4.789121in}{0.643420in}}%
\pgfpathlineto{\pgfqpoint{4.789655in}{0.636972in}}%
\pgfpathlineto{\pgfqpoint{4.790189in}{0.641438in}}%
\pgfpathlineto{\pgfqpoint{4.790722in}{0.641197in}}%
\pgfpathlineto{\pgfqpoint{4.791790in}{0.637225in}}%
\pgfpathlineto{\pgfqpoint{4.792323in}{0.654258in}}%
\pgfpathlineto{\pgfqpoint{4.792857in}{0.649681in}}%
\pgfpathlineto{\pgfqpoint{4.793924in}{0.641011in}}%
\pgfpathlineto{\pgfqpoint{4.794458in}{0.645458in}}%
\pgfpathlineto{\pgfqpoint{4.794992in}{0.647746in}}%
\pgfpathlineto{\pgfqpoint{4.796593in}{0.642234in}}%
\pgfpathlineto{\pgfqpoint{4.797127in}{0.648390in}}%
\pgfpathlineto{\pgfqpoint{4.797660in}{0.635947in}}%
\pgfpathlineto{\pgfqpoint{4.798194in}{0.640352in}}%
\pgfpathlineto{\pgfqpoint{4.799795in}{0.651718in}}%
\pgfpathlineto{\pgfqpoint{4.800863in}{0.640198in}}%
\pgfpathlineto{\pgfqpoint{4.801396in}{0.643612in}}%
\pgfpathlineto{\pgfqpoint{4.801930in}{0.645868in}}%
\pgfpathlineto{\pgfqpoint{4.803531in}{0.637241in}}%
\pgfpathlineto{\pgfqpoint{4.804065in}{0.640489in}}%
\pgfpathlineto{\pgfqpoint{4.804598in}{0.656242in}}%
\pgfpathlineto{\pgfqpoint{4.805132in}{0.642265in}}%
\pgfpathlineto{\pgfqpoint{4.805666in}{0.651058in}}%
\pgfpathlineto{\pgfqpoint{4.806200in}{0.649217in}}%
\pgfpathlineto{\pgfqpoint{4.806733in}{0.640342in}}%
\pgfpathlineto{\pgfqpoint{4.807267in}{0.643473in}}%
\pgfpathlineto{\pgfqpoint{4.808334in}{0.639429in}}%
\pgfpathlineto{\pgfqpoint{4.810469in}{0.650792in}}%
\pgfpathlineto{\pgfqpoint{4.811537in}{0.639567in}}%
\pgfpathlineto{\pgfqpoint{4.813138in}{0.649887in}}%
\pgfpathlineto{\pgfqpoint{4.813671in}{0.643827in}}%
\pgfpathlineto{\pgfqpoint{4.814205in}{0.646720in}}%
\pgfpathlineto{\pgfqpoint{4.814739in}{0.648052in}}%
\pgfpathlineto{\pgfqpoint{4.815806in}{0.641133in}}%
\pgfpathlineto{\pgfqpoint{4.816340in}{0.642123in}}%
\pgfpathlineto{\pgfqpoint{4.816874in}{0.639253in}}%
\pgfpathlineto{\pgfqpoint{4.817941in}{0.657130in}}%
\pgfpathlineto{\pgfqpoint{4.818475in}{0.646586in}}%
\pgfpathlineto{\pgfqpoint{4.820076in}{0.659229in}}%
\pgfpathlineto{\pgfqpoint{4.820609in}{0.638879in}}%
\pgfpathlineto{\pgfqpoint{4.821143in}{0.643479in}}%
\pgfpathlineto{\pgfqpoint{4.821677in}{0.640941in}}%
\pgfpathlineto{\pgfqpoint{4.822744in}{0.659411in}}%
\pgfpathlineto{\pgfqpoint{4.823278in}{0.642092in}}%
\pgfpathlineto{\pgfqpoint{4.823812in}{0.649802in}}%
\pgfpathlineto{\pgfqpoint{4.824879in}{0.646572in}}%
\pgfpathlineto{\pgfqpoint{4.825413in}{0.656795in}}%
\pgfpathlineto{\pgfqpoint{4.825946in}{0.646334in}}%
\pgfpathlineto{\pgfqpoint{4.827548in}{0.638809in}}%
\pgfpathlineto{\pgfqpoint{4.828615in}{0.643252in}}%
\pgfpathlineto{\pgfqpoint{4.829149in}{0.638274in}}%
\pgfpathlineto{\pgfqpoint{4.829682in}{0.644118in}}%
\pgfpathlineto{\pgfqpoint{4.830216in}{0.640928in}}%
\pgfpathlineto{\pgfqpoint{4.831817in}{0.645448in}}%
\pgfpathlineto{\pgfqpoint{4.832351in}{0.641515in}}%
\pgfpathlineto{\pgfqpoint{4.832885in}{0.642667in}}%
\pgfpathlineto{\pgfqpoint{4.833418in}{0.643312in}}%
\pgfpathlineto{\pgfqpoint{4.833952in}{0.639125in}}%
\pgfpathlineto{\pgfqpoint{4.834486in}{0.648572in}}%
\pgfpathlineto{\pgfqpoint{4.835019in}{0.647142in}}%
\pgfpathlineto{\pgfqpoint{4.835553in}{0.637040in}}%
\pgfpathlineto{\pgfqpoint{4.836087in}{0.638875in}}%
\pgfpathlineto{\pgfqpoint{4.836621in}{0.647316in}}%
\pgfpathlineto{\pgfqpoint{4.837154in}{0.640790in}}%
\pgfpathlineto{\pgfqpoint{4.837688in}{0.639510in}}%
\pgfpathlineto{\pgfqpoint{4.838755in}{0.651446in}}%
\pgfpathlineto{\pgfqpoint{4.839289in}{0.635266in}}%
\pgfpathlineto{\pgfqpoint{4.839823in}{0.640332in}}%
\pgfpathlineto{\pgfqpoint{4.840356in}{0.637261in}}%
\pgfpathlineto{\pgfqpoint{4.842491in}{0.649358in}}%
\pgfpathlineto{\pgfqpoint{4.844092in}{0.641959in}}%
\pgfpathlineto{\pgfqpoint{4.844626in}{0.645401in}}%
\pgfpathlineto{\pgfqpoint{4.845693in}{0.636925in}}%
\pgfpathlineto{\pgfqpoint{4.846227in}{0.639170in}}%
\pgfpathlineto{\pgfqpoint{4.846761in}{0.637755in}}%
\pgfpathlineto{\pgfqpoint{4.847828in}{0.652348in}}%
\pgfpathlineto{\pgfqpoint{4.848362in}{0.638293in}}%
\pgfpathlineto{\pgfqpoint{4.848896in}{0.640631in}}%
\pgfpathlineto{\pgfqpoint{4.849963in}{0.666707in}}%
\pgfpathlineto{\pgfqpoint{4.850497in}{0.640214in}}%
\pgfpathlineto{\pgfqpoint{4.851564in}{0.640578in}}%
\pgfpathlineto{\pgfqpoint{4.852632in}{0.650370in}}%
\pgfpathlineto{\pgfqpoint{4.853699in}{0.637842in}}%
\pgfpathlineto{\pgfqpoint{4.854233in}{0.639523in}}%
\pgfpathlineto{\pgfqpoint{4.854766in}{0.659337in}}%
\pgfpathlineto{\pgfqpoint{4.855300in}{0.638938in}}%
\pgfpathlineto{\pgfqpoint{4.855834in}{0.638109in}}%
\pgfpathlineto{\pgfqpoint{4.857435in}{0.650547in}}%
\pgfpathlineto{\pgfqpoint{4.858502in}{0.638624in}}%
\pgfpathlineto{\pgfqpoint{4.859570in}{0.651898in}}%
\pgfpathlineto{\pgfqpoint{4.860103in}{0.647153in}}%
\pgfpathlineto{\pgfqpoint{4.860637in}{0.648889in}}%
\pgfpathlineto{\pgfqpoint{4.861171in}{0.655816in}}%
\pgfpathlineto{\pgfqpoint{4.861704in}{0.636569in}}%
\pgfpathlineto{\pgfqpoint{4.862238in}{0.648382in}}%
\pgfpathlineto{\pgfqpoint{4.862772in}{0.651975in}}%
\pgfpathlineto{\pgfqpoint{4.863306in}{0.651508in}}%
\pgfpathlineto{\pgfqpoint{4.863839in}{0.641383in}}%
\pgfpathlineto{\pgfqpoint{4.864907in}{0.641809in}}%
\pgfpathlineto{\pgfqpoint{4.865974in}{0.652325in}}%
\pgfpathlineto{\pgfqpoint{4.867041in}{0.636663in}}%
\pgfpathlineto{\pgfqpoint{4.867575in}{0.646298in}}%
\pgfpathlineto{\pgfqpoint{4.868109in}{0.640486in}}%
\pgfpathlineto{\pgfqpoint{4.868643in}{0.637727in}}%
\pgfpathlineto{\pgfqpoint{4.869176in}{0.639322in}}%
\pgfpathlineto{\pgfqpoint{4.870244in}{0.638320in}}%
\pgfpathlineto{\pgfqpoint{4.870777in}{0.640205in}}%
\pgfpathlineto{\pgfqpoint{4.871311in}{0.638899in}}%
\pgfpathlineto{\pgfqpoint{4.872378in}{0.637615in}}%
\pgfpathlineto{\pgfqpoint{4.872912in}{0.639262in}}%
\pgfpathlineto{\pgfqpoint{4.873446in}{0.638429in}}%
\pgfpathlineto{\pgfqpoint{4.875581in}{0.636789in}}%
\pgfpathlineto{\pgfqpoint{4.877715in}{0.639483in}}%
\pgfpathlineto{\pgfqpoint{4.878783in}{0.638729in}}%
\pgfpathlineto{\pgfqpoint{4.880918in}{0.648423in}}%
\pgfpathlineto{\pgfqpoint{4.881985in}{0.640687in}}%
\pgfpathlineto{\pgfqpoint{4.883052in}{0.650901in}}%
\pgfpathlineto{\pgfqpoint{4.883586in}{0.638187in}}%
\pgfpathlineto{\pgfqpoint{4.884120in}{0.639286in}}%
\pgfpathlineto{\pgfqpoint{4.885187in}{0.651957in}}%
\pgfpathlineto{\pgfqpoint{4.885721in}{0.648729in}}%
\pgfpathlineto{\pgfqpoint{4.886255in}{0.642076in}}%
\pgfpathlineto{\pgfqpoint{4.886788in}{0.650264in}}%
\pgfpathlineto{\pgfqpoint{4.887322in}{0.640276in}}%
\pgfpathlineto{\pgfqpoint{4.887856in}{0.646913in}}%
\pgfpathlineto{\pgfqpoint{4.888923in}{0.640297in}}%
\pgfpathlineto{\pgfqpoint{4.889991in}{0.647183in}}%
\pgfpathlineto{\pgfqpoint{4.890524in}{0.646655in}}%
\pgfpathlineto{\pgfqpoint{4.891058in}{0.638515in}}%
\pgfpathlineto{\pgfqpoint{4.891592in}{0.648147in}}%
\pgfpathlineto{\pgfqpoint{4.892125in}{0.646902in}}%
\pgfpathlineto{\pgfqpoint{4.892659in}{0.640936in}}%
\pgfpathlineto{\pgfqpoint{4.893193in}{0.645314in}}%
\pgfpathlineto{\pgfqpoint{4.893726in}{0.646184in}}%
\pgfpathlineto{\pgfqpoint{4.894260in}{0.642063in}}%
\pgfpathlineto{\pgfqpoint{4.895328in}{0.642273in}}%
\pgfpathlineto{\pgfqpoint{4.895861in}{0.638773in}}%
\pgfpathlineto{\pgfqpoint{4.896395in}{0.642491in}}%
\pgfpathlineto{\pgfqpoint{4.896929in}{0.647935in}}%
\pgfpathlineto{\pgfqpoint{4.897996in}{0.638502in}}%
\pgfpathlineto{\pgfqpoint{4.899597in}{0.646119in}}%
\pgfpathlineto{\pgfqpoint{4.901198in}{0.637830in}}%
\pgfpathlineto{\pgfqpoint{4.902266in}{0.643120in}}%
\pgfpathlineto{\pgfqpoint{4.902799in}{0.655819in}}%
\pgfpathlineto{\pgfqpoint{4.903333in}{0.648615in}}%
\pgfpathlineto{\pgfqpoint{4.903867in}{0.644155in}}%
\pgfpathlineto{\pgfqpoint{4.905468in}{0.660066in}}%
\pgfpathlineto{\pgfqpoint{4.906002in}{0.641449in}}%
\pgfpathlineto{\pgfqpoint{4.906535in}{0.649341in}}%
\pgfpathlineto{\pgfqpoint{4.907069in}{0.649304in}}%
\pgfpathlineto{\pgfqpoint{4.908136in}{0.652761in}}%
\pgfpathlineto{\pgfqpoint{4.909204in}{0.644647in}}%
\pgfpathlineto{\pgfqpoint{4.909738in}{0.661499in}}%
\pgfpathlineto{\pgfqpoint{4.910271in}{0.653182in}}%
\pgfpathlineto{\pgfqpoint{4.911872in}{0.648434in}}%
\pgfpathlineto{\pgfqpoint{4.912406in}{0.663721in}}%
\pgfpathlineto{\pgfqpoint{4.914007in}{0.636314in}}%
\pgfpathlineto{\pgfqpoint{4.914541in}{0.665246in}}%
\pgfpathlineto{\pgfqpoint{4.915075in}{0.652615in}}%
\pgfpathlineto{\pgfqpoint{4.915608in}{0.650789in}}%
\pgfpathlineto{\pgfqpoint{4.916142in}{0.641494in}}%
\pgfpathlineto{\pgfqpoint{4.917209in}{0.641938in}}%
\pgfpathlineto{\pgfqpoint{4.917743in}{0.646330in}}%
\pgfpathlineto{\pgfqpoint{4.918810in}{0.641029in}}%
\pgfpathlineto{\pgfqpoint{4.919344in}{0.656475in}}%
\pgfpathlineto{\pgfqpoint{4.919878in}{0.653237in}}%
\pgfpathlineto{\pgfqpoint{4.921479in}{0.640553in}}%
\pgfpathlineto{\pgfqpoint{4.922013in}{0.661269in}}%
\pgfpathlineto{\pgfqpoint{4.922546in}{0.652129in}}%
\pgfpathlineto{\pgfqpoint{4.924147in}{0.636177in}}%
\pgfpathlineto{\pgfqpoint{4.924681in}{0.652438in}}%
\pgfpathlineto{\pgfqpoint{4.925215in}{0.643254in}}%
\pgfpathlineto{\pgfqpoint{4.926282in}{0.639734in}}%
\pgfpathlineto{\pgfqpoint{4.927350in}{0.655028in}}%
\pgfpathlineto{\pgfqpoint{4.927883in}{0.647709in}}%
\pgfpathlineto{\pgfqpoint{4.928951in}{0.666489in}}%
\pgfpathlineto{\pgfqpoint{4.929484in}{0.646402in}}%
\pgfpathlineto{\pgfqpoint{4.930018in}{0.648262in}}%
\pgfpathlineto{\pgfqpoint{4.930552in}{0.661621in}}%
\pgfpathlineto{\pgfqpoint{4.931086in}{0.642226in}}%
\pgfpathlineto{\pgfqpoint{4.931619in}{0.655942in}}%
\pgfpathlineto{\pgfqpoint{4.932153in}{0.657555in}}%
\pgfpathlineto{\pgfqpoint{4.932687in}{0.655593in}}%
\pgfpathlineto{\pgfqpoint{4.933220in}{0.658390in}}%
\pgfpathlineto{\pgfqpoint{4.933754in}{0.644275in}}%
\pgfpathlineto{\pgfqpoint{4.934288in}{0.645678in}}%
\pgfpathlineto{\pgfqpoint{4.934821in}{0.661320in}}%
\pgfpathlineto{\pgfqpoint{4.935355in}{0.641558in}}%
\pgfpathlineto{\pgfqpoint{4.935889in}{0.661861in}}%
\pgfpathlineto{\pgfqpoint{4.936956in}{0.639336in}}%
\pgfpathlineto{\pgfqpoint{4.937490in}{0.660240in}}%
\pgfpathlineto{\pgfqpoint{4.938024in}{0.655382in}}%
\pgfpathlineto{\pgfqpoint{4.939625in}{0.637491in}}%
\pgfpathlineto{\pgfqpoint{4.940158in}{0.650917in}}%
\pgfpathlineto{\pgfqpoint{4.940692in}{0.646116in}}%
\pgfpathlineto{\pgfqpoint{4.941226in}{0.647736in}}%
\pgfpathlineto{\pgfqpoint{4.941760in}{0.656528in}}%
\pgfpathlineto{\pgfqpoint{4.942827in}{0.638752in}}%
\pgfpathlineto{\pgfqpoint{4.943361in}{0.641504in}}%
\pgfpathlineto{\pgfqpoint{4.945495in}{0.653494in}}%
\pgfpathlineto{\pgfqpoint{4.946029in}{0.642701in}}%
\pgfpathlineto{\pgfqpoint{4.946563in}{0.647905in}}%
\pgfpathlineto{\pgfqpoint{4.947097in}{0.649356in}}%
\pgfpathlineto{\pgfqpoint{4.947630in}{0.638516in}}%
\pgfpathlineto{\pgfqpoint{4.948164in}{0.646435in}}%
\pgfpathlineto{\pgfqpoint{4.948698in}{0.657659in}}%
\pgfpathlineto{\pgfqpoint{4.949231in}{0.648098in}}%
\pgfpathlineto{\pgfqpoint{4.949765in}{0.646078in}}%
\pgfpathlineto{\pgfqpoint{4.950299in}{0.648766in}}%
\pgfpathlineto{\pgfqpoint{4.950832in}{0.641275in}}%
\pgfpathlineto{\pgfqpoint{4.951366in}{0.648234in}}%
\pgfpathlineto{\pgfqpoint{4.951900in}{0.647827in}}%
\pgfpathlineto{\pgfqpoint{4.952434in}{0.637269in}}%
\pgfpathlineto{\pgfqpoint{4.952967in}{0.640689in}}%
\pgfpathlineto{\pgfqpoint{4.953501in}{0.648463in}}%
\pgfpathlineto{\pgfqpoint{4.954035in}{0.645458in}}%
\pgfpathlineto{\pgfqpoint{4.954568in}{0.639876in}}%
\pgfpathlineto{\pgfqpoint{4.956169in}{0.649037in}}%
\pgfpathlineto{\pgfqpoint{4.957237in}{0.637880in}}%
\pgfpathlineto{\pgfqpoint{4.958304in}{0.663143in}}%
\pgfpathlineto{\pgfqpoint{4.959372in}{0.640530in}}%
\pgfpathlineto{\pgfqpoint{4.959905in}{0.643823in}}%
\pgfpathlineto{\pgfqpoint{4.960439in}{0.667676in}}%
\pgfpathlineto{\pgfqpoint{4.960973in}{0.640455in}}%
\pgfpathlineto{\pgfqpoint{4.961506in}{0.659651in}}%
\pgfpathlineto{\pgfqpoint{4.962040in}{0.667330in}}%
\pgfpathlineto{\pgfqpoint{4.962574in}{0.643311in}}%
\pgfpathlineto{\pgfqpoint{4.963108in}{0.655222in}}%
\pgfpathlineto{\pgfqpoint{4.963641in}{0.657983in}}%
\pgfpathlineto{\pgfqpoint{4.964175in}{0.643165in}}%
\pgfpathlineto{\pgfqpoint{4.964709in}{0.676469in}}%
\pgfpathlineto{\pgfqpoint{4.965242in}{0.646554in}}%
\pgfpathlineto{\pgfqpoint{4.965776in}{0.668296in}}%
\pgfpathlineto{\pgfqpoint{4.966310in}{0.637624in}}%
\pgfpathlineto{\pgfqpoint{4.966843in}{0.644675in}}%
\pgfpathlineto{\pgfqpoint{4.967377in}{0.645389in}}%
\pgfpathlineto{\pgfqpoint{4.967911in}{0.656803in}}%
\pgfpathlineto{\pgfqpoint{4.968445in}{0.637933in}}%
\pgfpathlineto{\pgfqpoint{4.968978in}{0.667915in}}%
\pgfpathlineto{\pgfqpoint{4.970046in}{0.667292in}}%
\pgfpathlineto{\pgfqpoint{4.971647in}{0.650366in}}%
\pgfpathlineto{\pgfqpoint{4.972181in}{0.648374in}}%
\pgfpathlineto{\pgfqpoint{4.972714in}{0.661562in}}%
\pgfpathlineto{\pgfqpoint{4.973248in}{0.650039in}}%
\pgfpathlineto{\pgfqpoint{4.974315in}{0.645504in}}%
\pgfpathlineto{\pgfqpoint{4.974849in}{0.664530in}}%
\pgfpathlineto{\pgfqpoint{4.975383in}{0.639708in}}%
\pgfpathlineto{\pgfqpoint{4.975916in}{0.655207in}}%
\pgfpathlineto{\pgfqpoint{4.977518in}{0.642451in}}%
\pgfpathlineto{\pgfqpoint{4.978051in}{0.646183in}}%
\pgfpathlineto{\pgfqpoint{4.978585in}{0.664705in}}%
\pgfpathlineto{\pgfqpoint{4.979119in}{0.656592in}}%
\pgfpathlineto{\pgfqpoint{4.980186in}{0.639673in}}%
\pgfpathlineto{\pgfqpoint{4.980720in}{0.639994in}}%
\pgfpathlineto{\pgfqpoint{4.981253in}{0.652273in}}%
\pgfpathlineto{\pgfqpoint{4.981787in}{0.640606in}}%
\pgfpathlineto{\pgfqpoint{4.983388in}{0.651206in}}%
\pgfpathlineto{\pgfqpoint{4.983922in}{0.653020in}}%
\pgfpathlineto{\pgfqpoint{4.984456in}{0.664220in}}%
\pgfpathlineto{\pgfqpoint{4.984989in}{0.650341in}}%
\pgfpathlineto{\pgfqpoint{4.985523in}{0.677732in}}%
\pgfpathlineto{\pgfqpoint{4.986057in}{0.663880in}}%
\pgfpathlineto{\pgfqpoint{4.987124in}{0.639251in}}%
\pgfpathlineto{\pgfqpoint{4.987658in}{0.667419in}}%
\pgfpathlineto{\pgfqpoint{4.988192in}{0.666236in}}%
\pgfpathlineto{\pgfqpoint{4.988725in}{0.648532in}}%
\pgfpathlineto{\pgfqpoint{4.989259in}{0.651121in}}%
\pgfpathlineto{\pgfqpoint{4.989793in}{0.652985in}}%
\pgfpathlineto{\pgfqpoint{4.990326in}{0.671272in}}%
\pgfpathlineto{\pgfqpoint{4.990860in}{0.661363in}}%
\pgfpathlineto{\pgfqpoint{4.991394in}{0.664292in}}%
\pgfpathlineto{\pgfqpoint{4.992461in}{0.648958in}}%
\pgfpathlineto{\pgfqpoint{4.992995in}{0.667110in}}%
\pgfpathlineto{\pgfqpoint{4.993529in}{0.663570in}}%
\pgfpathlineto{\pgfqpoint{4.995130in}{0.643274in}}%
\pgfpathlineto{\pgfqpoint{4.995663in}{0.640709in}}%
\pgfpathlineto{\pgfqpoint{4.996197in}{0.658791in}}%
\pgfpathlineto{\pgfqpoint{4.997264in}{0.657699in}}%
\pgfpathlineto{\pgfqpoint{4.997798in}{0.640340in}}%
\pgfpathlineto{\pgfqpoint{4.998332in}{0.641147in}}%
\pgfpathlineto{\pgfqpoint{4.999399in}{0.648678in}}%
\pgfpathlineto{\pgfqpoint{4.999933in}{0.642905in}}%
\pgfpathlineto{\pgfqpoint{5.000467in}{0.653069in}}%
\pgfpathlineto{\pgfqpoint{5.001000in}{0.650724in}}%
\pgfpathlineto{\pgfqpoint{5.001534in}{0.642213in}}%
\pgfpathlineto{\pgfqpoint{5.002068in}{0.662370in}}%
\pgfpathlineto{\pgfqpoint{5.002601in}{0.650414in}}%
\pgfpathlineto{\pgfqpoint{5.003135in}{0.658149in}}%
\pgfpathlineto{\pgfqpoint{5.003669in}{0.649072in}}%
\pgfpathlineto{\pgfqpoint{5.004203in}{0.660313in}}%
\pgfpathlineto{\pgfqpoint{5.004736in}{0.650094in}}%
\pgfpathlineto{\pgfqpoint{5.006337in}{0.639052in}}%
\pgfpathlineto{\pgfqpoint{5.007405in}{0.652568in}}%
\pgfpathlineto{\pgfqpoint{5.007938in}{0.651663in}}%
\pgfpathlineto{\pgfqpoint{5.008472in}{0.650033in}}%
\pgfpathlineto{\pgfqpoint{5.009006in}{0.656945in}}%
\pgfpathlineto{\pgfqpoint{5.009540in}{0.638667in}}%
\pgfpathlineto{\pgfqpoint{5.010073in}{0.653366in}}%
\pgfpathlineto{\pgfqpoint{5.011674in}{0.641323in}}%
\pgfpathlineto{\pgfqpoint{5.012208in}{0.642412in}}%
\pgfpathlineto{\pgfqpoint{5.013275in}{0.656550in}}%
\pgfpathlineto{\pgfqpoint{5.013809in}{0.656400in}}%
\pgfpathlineto{\pgfqpoint{5.014343in}{0.642181in}}%
\pgfpathlineto{\pgfqpoint{5.014877in}{0.651177in}}%
\pgfpathlineto{\pgfqpoint{5.015410in}{0.655555in}}%
\pgfpathlineto{\pgfqpoint{5.015944in}{0.646877in}}%
\pgfpathlineto{\pgfqpoint{5.016478in}{0.647423in}}%
\pgfpathlineto{\pgfqpoint{5.017011in}{0.669765in}}%
\pgfpathlineto{\pgfqpoint{5.017545in}{0.659162in}}%
\pgfpathlineto{\pgfqpoint{5.018079in}{0.660742in}}%
\pgfpathlineto{\pgfqpoint{5.018612in}{0.652987in}}%
\pgfpathlineto{\pgfqpoint{5.020214in}{0.676037in}}%
\pgfpathlineto{\pgfqpoint{5.021815in}{0.641962in}}%
\pgfpathlineto{\pgfqpoint{5.023416in}{0.669554in}}%
\pgfpathlineto{\pgfqpoint{5.023949in}{0.665018in}}%
\pgfpathlineto{\pgfqpoint{5.024483in}{0.651455in}}%
\pgfpathlineto{\pgfqpoint{5.025017in}{0.657072in}}%
\pgfpathlineto{\pgfqpoint{5.025551in}{0.674822in}}%
\pgfpathlineto{\pgfqpoint{5.026084in}{0.645791in}}%
\pgfpathlineto{\pgfqpoint{5.026618in}{0.658431in}}%
\pgfpathlineto{\pgfqpoint{5.027685in}{0.646104in}}%
\pgfpathlineto{\pgfqpoint{5.028219in}{0.646547in}}%
\pgfpathlineto{\pgfqpoint{5.030888in}{0.665124in}}%
\pgfpathlineto{\pgfqpoint{5.031421in}{0.642617in}}%
\pgfpathlineto{\pgfqpoint{5.031955in}{0.666703in}}%
\pgfpathlineto{\pgfqpoint{5.033556in}{0.650215in}}%
\pgfpathlineto{\pgfqpoint{5.034623in}{0.640239in}}%
\pgfpathlineto{\pgfqpoint{5.035157in}{0.645175in}}%
\pgfpathlineto{\pgfqpoint{5.035691in}{0.672650in}}%
\pgfpathlineto{\pgfqpoint{5.036225in}{0.662517in}}%
\pgfpathlineto{\pgfqpoint{5.037292in}{0.647726in}}%
\pgfpathlineto{\pgfqpoint{5.037826in}{0.650632in}}%
\pgfpathlineto{\pgfqpoint{5.038359in}{0.658379in}}%
\pgfpathlineto{\pgfqpoint{5.038893in}{0.648480in}}%
\pgfpathlineto{\pgfqpoint{5.039427in}{0.666969in}}%
\pgfpathlineto{\pgfqpoint{5.039961in}{0.652983in}}%
\pgfpathlineto{\pgfqpoint{5.041562in}{0.657352in}}%
\pgfpathlineto{\pgfqpoint{5.042095in}{0.673108in}}%
\pgfpathlineto{\pgfqpoint{5.042629in}{0.661223in}}%
\pgfpathlineto{\pgfqpoint{5.043163in}{0.663327in}}%
\pgfpathlineto{\pgfqpoint{5.043696in}{0.645404in}}%
\pgfpathlineto{\pgfqpoint{5.044230in}{0.663556in}}%
\pgfpathlineto{\pgfqpoint{5.044764in}{0.664885in}}%
\pgfpathlineto{\pgfqpoint{5.045298in}{0.659849in}}%
\pgfpathlineto{\pgfqpoint{5.045831in}{0.643767in}}%
\pgfpathlineto{\pgfqpoint{5.046365in}{0.670121in}}%
\pgfpathlineto{\pgfqpoint{5.046899in}{0.665377in}}%
\pgfpathlineto{\pgfqpoint{5.047432in}{0.657328in}}%
\pgfpathlineto{\pgfqpoint{5.047966in}{0.659141in}}%
\pgfpathlineto{\pgfqpoint{5.048500in}{0.666230in}}%
\pgfpathlineto{\pgfqpoint{5.049033in}{0.648562in}}%
\pgfpathlineto{\pgfqpoint{5.049567in}{0.654030in}}%
\pgfpathlineto{\pgfqpoint{5.050101in}{0.655645in}}%
\pgfpathlineto{\pgfqpoint{5.050635in}{0.647077in}}%
\pgfpathlineto{\pgfqpoint{5.051168in}{0.652659in}}%
\pgfpathlineto{\pgfqpoint{5.053303in}{0.643131in}}%
\pgfpathlineto{\pgfqpoint{5.053837in}{0.644177in}}%
\pgfpathlineto{\pgfqpoint{5.054370in}{0.652581in}}%
\pgfpathlineto{\pgfqpoint{5.054904in}{0.640872in}}%
\pgfpathlineto{\pgfqpoint{5.055438in}{0.652362in}}%
\pgfpathlineto{\pgfqpoint{5.055972in}{0.642435in}}%
\pgfpathlineto{\pgfqpoint{5.056505in}{0.647617in}}%
\pgfpathlineto{\pgfqpoint{5.057039in}{0.658024in}}%
\pgfpathlineto{\pgfqpoint{5.057573in}{0.651834in}}%
\pgfpathlineto{\pgfqpoint{5.058640in}{0.641599in}}%
\pgfpathlineto{\pgfqpoint{5.059174in}{0.662778in}}%
\pgfpathlineto{\pgfqpoint{5.059707in}{0.657983in}}%
\pgfpathlineto{\pgfqpoint{5.060241in}{0.658326in}}%
\pgfpathlineto{\pgfqpoint{5.061309in}{0.642858in}}%
\pgfpathlineto{\pgfqpoint{5.061842in}{0.648802in}}%
\pgfpathlineto{\pgfqpoint{5.063443in}{0.635667in}}%
\pgfpathlineto{\pgfqpoint{5.065044in}{0.659689in}}%
\pgfpathlineto{\pgfqpoint{5.065578in}{0.644582in}}%
\pgfpathlineto{\pgfqpoint{5.066112in}{0.661253in}}%
\pgfpathlineto{\pgfqpoint{5.066646in}{0.643500in}}%
\pgfpathlineto{\pgfqpoint{5.067179in}{0.642841in}}%
\pgfpathlineto{\pgfqpoint{5.067713in}{0.645074in}}%
\pgfpathlineto{\pgfqpoint{5.068247in}{0.664285in}}%
\pgfpathlineto{\pgfqpoint{5.068780in}{0.650094in}}%
\pgfpathlineto{\pgfqpoint{5.069314in}{0.650011in}}%
\pgfpathlineto{\pgfqpoint{5.069848in}{0.645507in}}%
\pgfpathlineto{\pgfqpoint{5.070381in}{0.655431in}}%
\pgfpathlineto{\pgfqpoint{5.071449in}{0.638786in}}%
\pgfpathlineto{\pgfqpoint{5.072516in}{0.666277in}}%
\pgfpathlineto{\pgfqpoint{5.073050in}{0.664883in}}%
\pgfpathlineto{\pgfqpoint{5.074651in}{0.642851in}}%
\pgfpathlineto{\pgfqpoint{5.075185in}{0.664263in}}%
\pgfpathlineto{\pgfqpoint{5.075718in}{0.661461in}}%
\pgfpathlineto{\pgfqpoint{5.076786in}{0.648809in}}%
\pgfpathlineto{\pgfqpoint{5.077320in}{0.655709in}}%
\pgfpathlineto{\pgfqpoint{5.077853in}{0.638477in}}%
\pgfpathlineto{\pgfqpoint{5.078387in}{0.667673in}}%
\pgfpathlineto{\pgfqpoint{5.078921in}{0.658649in}}%
\pgfpathlineto{\pgfqpoint{5.079454in}{0.654037in}}%
\pgfpathlineto{\pgfqpoint{5.080522in}{0.663072in}}%
\pgfpathlineto{\pgfqpoint{5.081055in}{0.660972in}}%
\pgfpathlineto{\pgfqpoint{5.081589in}{0.662716in}}%
\pgfpathlineto{\pgfqpoint{5.082123in}{0.640145in}}%
\pgfpathlineto{\pgfqpoint{5.082657in}{0.670434in}}%
\pgfpathlineto{\pgfqpoint{5.083190in}{0.639174in}}%
\pgfpathlineto{\pgfqpoint{5.084791in}{0.656940in}}%
\pgfpathlineto{\pgfqpoint{5.085325in}{0.642494in}}%
\pgfpathlineto{\pgfqpoint{5.086392in}{0.643169in}}%
\pgfpathlineto{\pgfqpoint{5.087994in}{0.661170in}}%
\pgfpathlineto{\pgfqpoint{5.088527in}{0.645433in}}%
\pgfpathlineto{\pgfqpoint{5.089595in}{0.646227in}}%
\pgfpathlineto{\pgfqpoint{5.090128in}{0.652607in}}%
\pgfpathlineto{\pgfqpoint{5.090662in}{0.648622in}}%
\pgfpathlineto{\pgfqpoint{5.091196in}{0.640304in}}%
\pgfpathlineto{\pgfqpoint{5.091729in}{0.643932in}}%
\pgfpathlineto{\pgfqpoint{5.092263in}{0.641079in}}%
\pgfpathlineto{\pgfqpoint{5.092797in}{0.661351in}}%
\pgfpathlineto{\pgfqpoint{5.093331in}{0.655548in}}%
\pgfpathlineto{\pgfqpoint{5.094398in}{0.643042in}}%
\pgfpathlineto{\pgfqpoint{5.095465in}{0.659759in}}%
\pgfpathlineto{\pgfqpoint{5.095999in}{0.658893in}}%
\pgfpathlineto{\pgfqpoint{5.097066in}{0.661578in}}%
\pgfpathlineto{\pgfqpoint{5.098668in}{0.650968in}}%
\pgfpathlineto{\pgfqpoint{5.101336in}{0.670940in}}%
\pgfpathlineto{\pgfqpoint{5.102403in}{0.645432in}}%
\pgfpathlineto{\pgfqpoint{5.102937in}{0.651282in}}%
\pgfpathlineto{\pgfqpoint{5.103471in}{0.692311in}}%
\pgfpathlineto{\pgfqpoint{5.104005in}{0.674466in}}%
\pgfpathlineto{\pgfqpoint{5.104538in}{0.649053in}}%
\pgfpathlineto{\pgfqpoint{5.105072in}{0.671983in}}%
\pgfpathlineto{\pgfqpoint{5.106673in}{0.644569in}}%
\pgfpathlineto{\pgfqpoint{5.107207in}{0.652633in}}%
\pgfpathlineto{\pgfqpoint{5.107741in}{0.648459in}}%
\pgfpathlineto{\pgfqpoint{5.108274in}{0.644783in}}%
\pgfpathlineto{\pgfqpoint{5.108808in}{0.658819in}}%
\pgfpathlineto{\pgfqpoint{5.109342in}{0.637984in}}%
\pgfpathlineto{\pgfqpoint{5.109875in}{0.660053in}}%
\pgfpathlineto{\pgfqpoint{5.110409in}{0.660541in}}%
\pgfpathlineto{\pgfqpoint{5.111476in}{0.651879in}}%
\pgfpathlineto{\pgfqpoint{5.112010in}{0.660755in}}%
\pgfpathlineto{\pgfqpoint{5.113611in}{0.642575in}}%
\pgfpathlineto{\pgfqpoint{5.114145in}{0.638768in}}%
\pgfpathlineto{\pgfqpoint{5.114679in}{0.670830in}}%
\pgfpathlineto{\pgfqpoint{5.115212in}{0.646639in}}%
\pgfpathlineto{\pgfqpoint{5.115746in}{0.650788in}}%
\pgfpathlineto{\pgfqpoint{5.116280in}{0.664211in}}%
\pgfpathlineto{\pgfqpoint{5.116813in}{0.644786in}}%
\pgfpathlineto{\pgfqpoint{5.117347in}{0.650672in}}%
\pgfpathlineto{\pgfqpoint{5.117881in}{0.648670in}}%
\pgfpathlineto{\pgfqpoint{5.119482in}{0.659270in}}%
\pgfpathlineto{\pgfqpoint{5.120016in}{0.645772in}}%
\pgfpathlineto{\pgfqpoint{5.120549in}{0.669504in}}%
\pgfpathlineto{\pgfqpoint{5.121083in}{0.657850in}}%
\pgfpathlineto{\pgfqpoint{5.121617in}{0.666210in}}%
\pgfpathlineto{\pgfqpoint{5.122150in}{0.661166in}}%
\pgfpathlineto{\pgfqpoint{5.122684in}{0.657424in}}%
\pgfpathlineto{\pgfqpoint{5.123218in}{0.642677in}}%
\pgfpathlineto{\pgfqpoint{5.123752in}{0.668568in}}%
\pgfpathlineto{\pgfqpoint{5.124285in}{0.663861in}}%
\pgfpathlineto{\pgfqpoint{5.125353in}{0.645053in}}%
\pgfpathlineto{\pgfqpoint{5.125886in}{0.669096in}}%
\pgfpathlineto{\pgfqpoint{5.126420in}{0.655298in}}%
\pgfpathlineto{\pgfqpoint{5.126954in}{0.660435in}}%
\pgfpathlineto{\pgfqpoint{5.127487in}{0.649293in}}%
\pgfpathlineto{\pgfqpoint{5.128021in}{0.681871in}}%
\pgfpathlineto{\pgfqpoint{5.128555in}{0.672215in}}%
\pgfpathlineto{\pgfqpoint{5.129089in}{0.657242in}}%
\pgfpathlineto{\pgfqpoint{5.129622in}{0.679293in}}%
\pgfpathlineto{\pgfqpoint{5.130690in}{0.677606in}}%
\pgfpathlineto{\pgfqpoint{5.131757in}{0.639342in}}%
\pgfpathlineto{\pgfqpoint{5.132291in}{0.660054in}}%
\pgfpathlineto{\pgfqpoint{5.133358in}{0.682098in}}%
\pgfpathlineto{\pgfqpoint{5.133892in}{0.654505in}}%
\pgfpathlineto{\pgfqpoint{5.134426in}{0.667637in}}%
\pgfpathlineto{\pgfqpoint{5.134959in}{0.680454in}}%
\pgfpathlineto{\pgfqpoint{5.135493in}{0.667420in}}%
\pgfpathlineto{\pgfqpoint{5.136560in}{0.665944in}}%
\pgfpathlineto{\pgfqpoint{5.137628in}{0.662826in}}%
\pgfpathlineto{\pgfqpoint{5.138161in}{0.712461in}}%
\pgfpathlineto{\pgfqpoint{5.138695in}{0.686506in}}%
\pgfpathlineto{\pgfqpoint{5.140296in}{0.702828in}}%
\pgfpathlineto{\pgfqpoint{5.140830in}{0.691341in}}%
\pgfpathlineto{\pgfqpoint{5.141897in}{0.727580in}}%
\pgfpathlineto{\pgfqpoint{5.142431in}{0.721863in}}%
\pgfpathlineto{\pgfqpoint{5.142965in}{0.705053in}}%
\pgfpathlineto{\pgfqpoint{5.143498in}{0.718574in}}%
\pgfpathlineto{\pgfqpoint{5.144032in}{0.718375in}}%
\pgfpathlineto{\pgfqpoint{5.144566in}{0.726251in}}%
\pgfpathlineto{\pgfqpoint{5.146167in}{0.846556in}}%
\pgfpathlineto{\pgfqpoint{5.146701in}{1.007364in}}%
\pgfpathlineto{\pgfqpoint{5.147234in}{0.766644in}}%
\pgfpathlineto{\pgfqpoint{5.147768in}{1.862432in}}%
\pgfpathlineto{\pgfqpoint{5.148302in}{1.054242in}}%
\pgfpathlineto{\pgfqpoint{5.148835in}{1.239425in}}%
\pgfpathlineto{\pgfqpoint{5.149369in}{1.895647in}}%
\pgfpathlineto{\pgfqpoint{5.149903in}{0.832177in}}%
\pgfpathlineto{\pgfqpoint{5.150970in}{0.849312in}}%
\pgfpathlineto{\pgfqpoint{5.153639in}{0.722116in}}%
\pgfpathlineto{\pgfqpoint{5.155240in}{0.693877in}}%
\pgfpathlineto{\pgfqpoint{5.155774in}{0.723561in}}%
\pgfpathlineto{\pgfqpoint{5.156307in}{0.690088in}}%
\pgfpathlineto{\pgfqpoint{5.156841in}{0.717371in}}%
\pgfpathlineto{\pgfqpoint{5.157375in}{0.711947in}}%
\pgfpathlineto{\pgfqpoint{5.157908in}{0.653003in}}%
\pgfpathlineto{\pgfqpoint{5.158442in}{0.703145in}}%
\pgfpathlineto{\pgfqpoint{5.158976in}{0.671439in}}%
\pgfpathlineto{\pgfqpoint{5.159509in}{0.698677in}}%
\pgfpathlineto{\pgfqpoint{5.160043in}{0.702731in}}%
\pgfpathlineto{\pgfqpoint{5.160577in}{0.643044in}}%
\pgfpathlineto{\pgfqpoint{5.161111in}{0.680667in}}%
\pgfpathlineto{\pgfqpoint{5.163245in}{0.654161in}}%
\pgfpathlineto{\pgfqpoint{5.164313in}{0.675107in}}%
\pgfpathlineto{\pgfqpoint{5.166981in}{0.650505in}}%
\pgfpathlineto{\pgfqpoint{5.168049in}{0.666598in}}%
\pgfpathlineto{\pgfqpoint{5.168582in}{0.657914in}}%
\pgfpathlineto{\pgfqpoint{5.169116in}{0.664677in}}%
\pgfpathlineto{\pgfqpoint{5.169650in}{0.676190in}}%
\pgfpathlineto{\pgfqpoint{5.170183in}{0.650329in}}%
\pgfpathlineto{\pgfqpoint{5.170717in}{0.671520in}}%
\pgfpathlineto{\pgfqpoint{5.171251in}{0.671787in}}%
\pgfpathlineto{\pgfqpoint{5.171785in}{0.646453in}}%
\pgfpathlineto{\pgfqpoint{5.172318in}{0.661135in}}%
\pgfpathlineto{\pgfqpoint{5.173386in}{0.661895in}}%
\pgfpathlineto{\pgfqpoint{5.173919in}{0.660240in}}%
\pgfpathlineto{\pgfqpoint{5.174453in}{0.660787in}}%
\pgfpathlineto{\pgfqpoint{5.174987in}{0.650973in}}%
\pgfpathlineto{\pgfqpoint{5.175521in}{0.677678in}}%
\pgfpathlineto{\pgfqpoint{5.176054in}{0.642838in}}%
\pgfpathlineto{\pgfqpoint{5.176588in}{0.657084in}}%
\pgfpathlineto{\pgfqpoint{5.177122in}{0.664942in}}%
\pgfpathlineto{\pgfqpoint{5.177655in}{0.640992in}}%
\pgfpathlineto{\pgfqpoint{5.178189in}{0.644275in}}%
\pgfpathlineto{\pgfqpoint{5.178723in}{0.667177in}}%
\pgfpathlineto{\pgfqpoint{5.179256in}{0.658406in}}%
\pgfpathlineto{\pgfqpoint{5.180324in}{0.642463in}}%
\pgfpathlineto{\pgfqpoint{5.180858in}{0.651023in}}%
\pgfpathlineto{\pgfqpoint{5.181391in}{0.650627in}}%
\pgfpathlineto{\pgfqpoint{5.181925in}{0.643766in}}%
\pgfpathlineto{\pgfqpoint{5.183526in}{0.687136in}}%
\pgfpathlineto{\pgfqpoint{5.184060in}{0.640623in}}%
\pgfpathlineto{\pgfqpoint{5.184593in}{0.664774in}}%
\pgfpathlineto{\pgfqpoint{5.185127in}{0.664504in}}%
\pgfpathlineto{\pgfqpoint{5.186195in}{0.642570in}}%
\pgfpathlineto{\pgfqpoint{5.186728in}{0.657279in}}%
\pgfpathlineto{\pgfqpoint{5.187262in}{0.651724in}}%
\pgfpathlineto{\pgfqpoint{5.187796in}{0.653768in}}%
\pgfpathlineto{\pgfqpoint{5.188329in}{0.672442in}}%
\pgfpathlineto{\pgfqpoint{5.188863in}{0.648387in}}%
\pgfpathlineto{\pgfqpoint{5.189930in}{0.648506in}}%
\pgfpathlineto{\pgfqpoint{5.190464in}{0.658230in}}%
\pgfpathlineto{\pgfqpoint{5.190998in}{0.647709in}}%
\pgfpathlineto{\pgfqpoint{5.191532in}{0.643979in}}%
\pgfpathlineto{\pgfqpoint{5.193133in}{0.679712in}}%
\pgfpathlineto{\pgfqpoint{5.194734in}{0.646131in}}%
\pgfpathlineto{\pgfqpoint{5.195801in}{0.656734in}}%
\pgfpathlineto{\pgfqpoint{5.196869in}{0.642478in}}%
\pgfpathlineto{\pgfqpoint{5.197402in}{0.653919in}}%
\pgfpathlineto{\pgfqpoint{5.197936in}{0.643816in}}%
\pgfpathlineto{\pgfqpoint{5.200071in}{0.665854in}}%
\pgfpathlineto{\pgfqpoint{5.200604in}{0.646986in}}%
\pgfpathlineto{\pgfqpoint{5.201138in}{0.659815in}}%
\pgfpathlineto{\pgfqpoint{5.201672in}{0.658346in}}%
\pgfpathlineto{\pgfqpoint{5.202206in}{0.641644in}}%
\pgfpathlineto{\pgfqpoint{5.203273in}{0.642389in}}%
\pgfpathlineto{\pgfqpoint{5.203807in}{0.659412in}}%
\pgfpathlineto{\pgfqpoint{5.204874in}{0.638172in}}%
\pgfpathlineto{\pgfqpoint{5.205408in}{0.671903in}}%
\pgfpathlineto{\pgfqpoint{5.205941in}{0.643587in}}%
\pgfpathlineto{\pgfqpoint{5.206475in}{0.640881in}}%
\pgfpathlineto{\pgfqpoint{5.207009in}{0.642597in}}%
\pgfpathlineto{\pgfqpoint{5.207543in}{0.666941in}}%
\pgfpathlineto{\pgfqpoint{5.208076in}{0.656901in}}%
\pgfpathlineto{\pgfqpoint{5.208610in}{0.646962in}}%
\pgfpathlineto{\pgfqpoint{5.209144in}{0.649356in}}%
\pgfpathlineto{\pgfqpoint{5.210745in}{0.662197in}}%
\pgfpathlineto{\pgfqpoint{5.211278in}{0.646174in}}%
\pgfpathlineto{\pgfqpoint{5.211812in}{0.649507in}}%
\pgfpathlineto{\pgfqpoint{5.212880in}{0.662309in}}%
\pgfpathlineto{\pgfqpoint{5.213413in}{0.654394in}}%
\pgfpathlineto{\pgfqpoint{5.213947in}{0.686217in}}%
\pgfpathlineto{\pgfqpoint{5.214481in}{0.642143in}}%
\pgfpathlineto{\pgfqpoint{5.215014in}{0.650865in}}%
\pgfpathlineto{\pgfqpoint{5.215548in}{0.654819in}}%
\pgfpathlineto{\pgfqpoint{5.216082in}{0.652709in}}%
\pgfpathlineto{\pgfqpoint{5.216615in}{0.680878in}}%
\pgfpathlineto{\pgfqpoint{5.217149in}{0.653155in}}%
\pgfpathlineto{\pgfqpoint{5.218750in}{0.661711in}}%
\pgfpathlineto{\pgfqpoint{5.219284in}{0.662112in}}%
\pgfpathlineto{\pgfqpoint{5.220351in}{0.640538in}}%
\pgfpathlineto{\pgfqpoint{5.220885in}{0.657081in}}%
\pgfpathlineto{\pgfqpoint{5.221419in}{0.654836in}}%
\pgfpathlineto{\pgfqpoint{5.221952in}{0.655180in}}%
\pgfpathlineto{\pgfqpoint{5.224087in}{0.644473in}}%
\pgfpathlineto{\pgfqpoint{5.225688in}{0.661838in}}%
\pgfpathlineto{\pgfqpoint{5.226222in}{0.639735in}}%
\pgfpathlineto{\pgfqpoint{5.226756in}{0.654153in}}%
\pgfpathlineto{\pgfqpoint{5.228891in}{0.636292in}}%
\pgfpathlineto{\pgfqpoint{5.229424in}{0.658348in}}%
\pgfpathlineto{\pgfqpoint{5.229958in}{0.651464in}}%
\pgfpathlineto{\pgfqpoint{5.230492in}{0.650381in}}%
\pgfpathlineto{\pgfqpoint{5.231559in}{0.658933in}}%
\pgfpathlineto{\pgfqpoint{5.232093in}{0.648993in}}%
\pgfpathlineto{\pgfqpoint{5.233160in}{0.649542in}}%
\pgfpathlineto{\pgfqpoint{5.233694in}{0.661913in}}%
\pgfpathlineto{\pgfqpoint{5.234228in}{0.649896in}}%
\pgfpathlineto{\pgfqpoint{5.234761in}{0.650201in}}%
\pgfpathlineto{\pgfqpoint{5.235295in}{0.639901in}}%
\pgfpathlineto{\pgfqpoint{5.235829in}{0.653607in}}%
\pgfpathlineto{\pgfqpoint{5.236362in}{0.646708in}}%
\pgfpathlineto{\pgfqpoint{5.236896in}{0.645960in}}%
\pgfpathlineto{\pgfqpoint{5.238497in}{0.662605in}}%
\pgfpathlineto{\pgfqpoint{5.239031in}{0.658338in}}%
\pgfpathlineto{\pgfqpoint{5.240098in}{0.643137in}}%
\pgfpathlineto{\pgfqpoint{5.240632in}{0.651034in}}%
\pgfpathlineto{\pgfqpoint{5.241699in}{0.659748in}}%
\pgfpathlineto{\pgfqpoint{5.242767in}{0.644444in}}%
\pgfpathlineto{\pgfqpoint{5.244368in}{0.666824in}}%
\pgfpathlineto{\pgfqpoint{5.245435in}{0.639093in}}%
\pgfpathlineto{\pgfqpoint{5.245969in}{0.639203in}}%
\pgfpathlineto{\pgfqpoint{5.246503in}{0.651048in}}%
\pgfpathlineto{\pgfqpoint{5.247036in}{0.648601in}}%
\pgfpathlineto{\pgfqpoint{5.247570in}{0.648099in}}%
\pgfpathlineto{\pgfqpoint{5.248104in}{0.653804in}}%
\pgfpathlineto{\pgfqpoint{5.248638in}{0.642391in}}%
\pgfpathlineto{\pgfqpoint{5.249171in}{0.652328in}}%
\pgfpathlineto{\pgfqpoint{5.249705in}{0.654426in}}%
\pgfpathlineto{\pgfqpoint{5.250239in}{0.664985in}}%
\pgfpathlineto{\pgfqpoint{5.250772in}{0.642608in}}%
\pgfpathlineto{\pgfqpoint{5.251306in}{0.653192in}}%
\pgfpathlineto{\pgfqpoint{5.252373in}{0.641045in}}%
\pgfpathlineto{\pgfqpoint{5.253441in}{0.658717in}}%
\pgfpathlineto{\pgfqpoint{5.253975in}{0.651473in}}%
\pgfpathlineto{\pgfqpoint{5.255042in}{0.651648in}}%
\pgfpathlineto{\pgfqpoint{5.255576in}{0.637492in}}%
\pgfpathlineto{\pgfqpoint{5.257177in}{0.654340in}}%
\pgfpathlineto{\pgfqpoint{5.258778in}{0.641648in}}%
\pgfpathlineto{\pgfqpoint{5.259312in}{0.650760in}}%
\pgfpathlineto{\pgfqpoint{5.259845in}{0.648034in}}%
\pgfpathlineto{\pgfqpoint{5.260379in}{0.645418in}}%
\pgfpathlineto{\pgfqpoint{5.260913in}{0.659132in}}%
\pgfpathlineto{\pgfqpoint{5.261446in}{0.642501in}}%
\pgfpathlineto{\pgfqpoint{5.261980in}{0.645749in}}%
\pgfpathlineto{\pgfqpoint{5.262514in}{0.668012in}}%
\pgfpathlineto{\pgfqpoint{5.263047in}{0.647554in}}%
\pgfpathlineto{\pgfqpoint{5.264115in}{0.637402in}}%
\pgfpathlineto{\pgfqpoint{5.264649in}{0.655221in}}%
\pgfpathlineto{\pgfqpoint{5.265182in}{0.640462in}}%
\pgfpathlineto{\pgfqpoint{5.266783in}{0.652510in}}%
\pgfpathlineto{\pgfqpoint{5.267317in}{0.648098in}}%
\pgfpathlineto{\pgfqpoint{5.267851in}{0.655507in}}%
\pgfpathlineto{\pgfqpoint{5.268384in}{0.640514in}}%
\pgfpathlineto{\pgfqpoint{5.268918in}{0.665725in}}%
\pgfpathlineto{\pgfqpoint{5.269452in}{0.661240in}}%
\pgfpathlineto{\pgfqpoint{5.269986in}{0.637670in}}%
\pgfpathlineto{\pgfqpoint{5.270519in}{0.671469in}}%
\pgfpathlineto{\pgfqpoint{5.271053in}{0.649682in}}%
\pgfpathlineto{\pgfqpoint{5.271587in}{0.650631in}}%
\pgfpathlineto{\pgfqpoint{5.272120in}{0.640677in}}%
\pgfpathlineto{\pgfqpoint{5.272654in}{0.644570in}}%
\pgfpathlineto{\pgfqpoint{5.273188in}{0.654965in}}%
\pgfpathlineto{\pgfqpoint{5.273721in}{0.653845in}}%
\pgfpathlineto{\pgfqpoint{5.274255in}{0.640905in}}%
\pgfpathlineto{\pgfqpoint{5.275323in}{0.674393in}}%
\pgfpathlineto{\pgfqpoint{5.276924in}{0.640604in}}%
\pgfpathlineto{\pgfqpoint{5.277457in}{0.649986in}}%
\pgfpathlineto{\pgfqpoint{5.277991in}{0.643893in}}%
\pgfpathlineto{\pgfqpoint{5.279058in}{0.650114in}}%
\pgfpathlineto{\pgfqpoint{5.279592in}{0.647094in}}%
\pgfpathlineto{\pgfqpoint{5.280126in}{0.650258in}}%
\pgfpathlineto{\pgfqpoint{5.280660in}{0.654226in}}%
\pgfpathlineto{\pgfqpoint{5.281193in}{0.646560in}}%
\pgfpathlineto{\pgfqpoint{5.282261in}{0.646922in}}%
\pgfpathlineto{\pgfqpoint{5.282794in}{0.652496in}}%
\pgfpathlineto{\pgfqpoint{5.283328in}{0.636135in}}%
\pgfpathlineto{\pgfqpoint{5.283862in}{0.643641in}}%
\pgfpathlineto{\pgfqpoint{5.284395in}{0.645632in}}%
\pgfpathlineto{\pgfqpoint{5.285463in}{0.639144in}}%
\pgfpathlineto{\pgfqpoint{5.287064in}{0.660830in}}%
\pgfpathlineto{\pgfqpoint{5.288131in}{0.643390in}}%
\pgfpathlineto{\pgfqpoint{5.288665in}{0.658897in}}%
\pgfpathlineto{\pgfqpoint{5.289199in}{0.653906in}}%
\pgfpathlineto{\pgfqpoint{5.289732in}{0.644558in}}%
\pgfpathlineto{\pgfqpoint{5.290266in}{0.647535in}}%
\pgfpathlineto{\pgfqpoint{5.290800in}{0.647097in}}%
\pgfpathlineto{\pgfqpoint{5.291334in}{0.657718in}}%
\pgfpathlineto{\pgfqpoint{5.292935in}{0.641177in}}%
\pgfpathlineto{\pgfqpoint{5.293468in}{0.658490in}}%
\pgfpathlineto{\pgfqpoint{5.294002in}{0.652553in}}%
\pgfpathlineto{\pgfqpoint{5.294536in}{0.644241in}}%
\pgfpathlineto{\pgfqpoint{5.295069in}{0.652759in}}%
\pgfpathlineto{\pgfqpoint{5.296137in}{0.652759in}}%
\pgfpathlineto{\pgfqpoint{5.296671in}{0.639818in}}%
\pgfpathlineto{\pgfqpoint{5.297204in}{0.652385in}}%
\pgfpathlineto{\pgfqpoint{5.297738in}{0.653381in}}%
\pgfpathlineto{\pgfqpoint{5.298272in}{0.661638in}}%
\pgfpathlineto{\pgfqpoint{5.298805in}{0.656250in}}%
\pgfpathlineto{\pgfqpoint{5.299339in}{0.644562in}}%
\pgfpathlineto{\pgfqpoint{5.299873in}{0.660976in}}%
\pgfpathlineto{\pgfqpoint{5.300406in}{0.645398in}}%
\pgfpathlineto{\pgfqpoint{5.301474in}{0.647144in}}%
\pgfpathlineto{\pgfqpoint{5.302008in}{0.643866in}}%
\pgfpathlineto{\pgfqpoint{5.302541in}{0.649996in}}%
\pgfpathlineto{\pgfqpoint{5.303075in}{0.645906in}}%
\pgfpathlineto{\pgfqpoint{5.303609in}{0.645101in}}%
\pgfpathlineto{\pgfqpoint{5.304676in}{0.661651in}}%
\pgfpathlineto{\pgfqpoint{5.305210in}{0.654888in}}%
\pgfpathlineto{\pgfqpoint{5.305743in}{0.648200in}}%
\pgfpathlineto{\pgfqpoint{5.306277in}{0.648260in}}%
\pgfpathlineto{\pgfqpoint{5.306811in}{0.650543in}}%
\pgfpathlineto{\pgfqpoint{5.307345in}{0.643201in}}%
\pgfpathlineto{\pgfqpoint{5.307878in}{0.646948in}}%
\pgfpathlineto{\pgfqpoint{5.309479in}{0.639005in}}%
\pgfpathlineto{\pgfqpoint{5.310013in}{0.656690in}}%
\pgfpathlineto{\pgfqpoint{5.310547in}{0.644594in}}%
\pgfpathlineto{\pgfqpoint{5.311614in}{0.656019in}}%
\pgfpathlineto{\pgfqpoint{5.312682in}{0.637733in}}%
\pgfpathlineto{\pgfqpoint{5.313215in}{0.642817in}}%
\pgfpathlineto{\pgfqpoint{5.314816in}{0.652632in}}%
\pgfpathlineto{\pgfqpoint{5.315884in}{0.639810in}}%
\pgfpathlineto{\pgfqpoint{5.316418in}{0.651327in}}%
\pgfpathlineto{\pgfqpoint{5.316951in}{0.643770in}}%
\pgfpathlineto{\pgfqpoint{5.318019in}{0.651779in}}%
\pgfpathlineto{\pgfqpoint{5.318552in}{0.644221in}}%
\pgfpathlineto{\pgfqpoint{5.319086in}{0.644697in}}%
\pgfpathlineto{\pgfqpoint{5.319620in}{0.658145in}}%
\pgfpathlineto{\pgfqpoint{5.320153in}{0.645166in}}%
\pgfpathlineto{\pgfqpoint{5.321221in}{0.659404in}}%
\pgfpathlineto{\pgfqpoint{5.322288in}{0.646162in}}%
\pgfpathlineto{\pgfqpoint{5.323889in}{0.652645in}}%
\pgfpathlineto{\pgfqpoint{5.324423in}{0.650641in}}%
\pgfpathlineto{\pgfqpoint{5.324957in}{0.637612in}}%
\pgfpathlineto{\pgfqpoint{5.326024in}{0.663853in}}%
\pgfpathlineto{\pgfqpoint{5.326558in}{0.647715in}}%
\pgfpathlineto{\pgfqpoint{5.327625in}{0.647781in}}%
\pgfpathlineto{\pgfqpoint{5.328159in}{0.645951in}}%
\pgfpathlineto{\pgfqpoint{5.328693in}{0.653365in}}%
\pgfpathlineto{\pgfqpoint{5.329760in}{0.654149in}}%
\pgfpathlineto{\pgfqpoint{5.330294in}{0.638423in}}%
\pgfpathlineto{\pgfqpoint{5.331895in}{0.658745in}}%
\pgfpathlineto{\pgfqpoint{5.332429in}{0.653953in}}%
\pgfpathlineto{\pgfqpoint{5.332962in}{0.663075in}}%
\pgfpathlineto{\pgfqpoint{5.333496in}{0.646211in}}%
\pgfpathlineto{\pgfqpoint{5.334030in}{0.648477in}}%
\pgfpathlineto{\pgfqpoint{5.334563in}{0.649383in}}%
\pgfpathlineto{\pgfqpoint{5.335097in}{0.640642in}}%
\pgfpathlineto{\pgfqpoint{5.335631in}{0.643513in}}%
\pgfpathlineto{\pgfqpoint{5.336164in}{0.654451in}}%
\pgfpathlineto{\pgfqpoint{5.336698in}{0.638638in}}%
\pgfpathlineto{\pgfqpoint{5.337232in}{0.657210in}}%
\pgfpathlineto{\pgfqpoint{5.337766in}{0.648899in}}%
\pgfpathlineto{\pgfqpoint{5.338833in}{0.637918in}}%
\pgfpathlineto{\pgfqpoint{5.339367in}{0.641658in}}%
\pgfpathlineto{\pgfqpoint{5.340434in}{0.636955in}}%
\pgfpathlineto{\pgfqpoint{5.341501in}{0.658574in}}%
\pgfpathlineto{\pgfqpoint{5.342035in}{0.649553in}}%
\pgfpathlineto{\pgfqpoint{5.343103in}{0.641276in}}%
\pgfpathlineto{\pgfqpoint{5.343636in}{0.657249in}}%
\pgfpathlineto{\pgfqpoint{5.344170in}{0.650382in}}%
\pgfpathlineto{\pgfqpoint{5.344704in}{0.655714in}}%
\pgfpathlineto{\pgfqpoint{5.346838in}{0.642061in}}%
\pgfpathlineto{\pgfqpoint{5.347372in}{0.641805in}}%
\pgfpathlineto{\pgfqpoint{5.348440in}{0.645776in}}%
\pgfpathlineto{\pgfqpoint{5.348973in}{0.661224in}}%
\pgfpathlineto{\pgfqpoint{5.349507in}{0.640387in}}%
\pgfpathlineto{\pgfqpoint{5.350041in}{0.644182in}}%
\pgfpathlineto{\pgfqpoint{5.352709in}{0.661519in}}%
\pgfpathlineto{\pgfqpoint{5.353243in}{0.641682in}}%
\pgfpathlineto{\pgfqpoint{5.353777in}{0.659066in}}%
\pgfpathlineto{\pgfqpoint{5.355911in}{0.640177in}}%
\pgfpathlineto{\pgfqpoint{5.356445in}{0.641514in}}%
\pgfpathlineto{\pgfqpoint{5.356979in}{0.648828in}}%
\pgfpathlineto{\pgfqpoint{5.357512in}{0.639069in}}%
\pgfpathlineto{\pgfqpoint{5.358046in}{0.642985in}}%
\pgfpathlineto{\pgfqpoint{5.358580in}{0.645467in}}%
\pgfpathlineto{\pgfqpoint{5.359114in}{0.640447in}}%
\pgfpathlineto{\pgfqpoint{5.359647in}{0.646950in}}%
\pgfpathlineto{\pgfqpoint{5.360181in}{0.636692in}}%
\pgfpathlineto{\pgfqpoint{5.361782in}{0.659825in}}%
\pgfpathlineto{\pgfqpoint{5.362849in}{0.642282in}}%
\pgfpathlineto{\pgfqpoint{5.363383in}{0.644729in}}%
\pgfpathlineto{\pgfqpoint{5.363917in}{0.653683in}}%
\pgfpathlineto{\pgfqpoint{5.364451in}{0.637949in}}%
\pgfpathlineto{\pgfqpoint{5.364984in}{0.657872in}}%
\pgfpathlineto{\pgfqpoint{5.365518in}{0.646641in}}%
\pgfpathlineto{\pgfqpoint{5.366052in}{0.642666in}}%
\pgfpathlineto{\pgfqpoint{5.366585in}{0.645562in}}%
\pgfpathlineto{\pgfqpoint{5.367119in}{0.654676in}}%
\pgfpathlineto{\pgfqpoint{5.368186in}{0.643183in}}%
\pgfpathlineto{\pgfqpoint{5.368720in}{0.670652in}}%
\pgfpathlineto{\pgfqpoint{5.369254in}{0.649857in}}%
\pgfpathlineto{\pgfqpoint{5.370321in}{0.646849in}}%
\pgfpathlineto{\pgfqpoint{5.370855in}{0.648783in}}%
\pgfpathlineto{\pgfqpoint{5.371389in}{0.641970in}}%
\pgfpathlineto{\pgfqpoint{5.371922in}{0.646832in}}%
\pgfpathlineto{\pgfqpoint{5.372456in}{0.649566in}}%
\pgfpathlineto{\pgfqpoint{5.372990in}{0.642279in}}%
\pgfpathlineto{\pgfqpoint{5.373523in}{0.649496in}}%
\pgfpathlineto{\pgfqpoint{5.374591in}{0.642571in}}%
\pgfpathlineto{\pgfqpoint{5.375125in}{0.653382in}}%
\pgfpathlineto{\pgfqpoint{5.375658in}{0.646526in}}%
\pgfpathlineto{\pgfqpoint{5.376192in}{0.651628in}}%
\pgfpathlineto{\pgfqpoint{5.376726in}{0.647245in}}%
\pgfpathlineto{\pgfqpoint{5.377793in}{0.642598in}}%
\pgfpathlineto{\pgfqpoint{5.379394in}{0.648927in}}%
\pgfpathlineto{\pgfqpoint{5.379928in}{0.636915in}}%
\pgfpathlineto{\pgfqpoint{5.380462in}{0.649590in}}%
\pgfpathlineto{\pgfqpoint{5.380995in}{0.655002in}}%
\pgfpathlineto{\pgfqpoint{5.381529in}{0.639358in}}%
\pgfpathlineto{\pgfqpoint{5.382063in}{0.654466in}}%
\pgfpathlineto{\pgfqpoint{5.382596in}{0.647674in}}%
\pgfpathlineto{\pgfqpoint{5.383130in}{0.661104in}}%
\pgfpathlineto{\pgfqpoint{5.383664in}{0.643732in}}%
\pgfpathlineto{\pgfqpoint{5.384198in}{0.651518in}}%
\pgfpathlineto{\pgfqpoint{5.384731in}{0.651465in}}%
\pgfpathlineto{\pgfqpoint{5.386332in}{0.639920in}}%
\pgfpathlineto{\pgfqpoint{5.387933in}{0.646183in}}%
\pgfpathlineto{\pgfqpoint{5.388467in}{0.641173in}}%
\pgfpathlineto{\pgfqpoint{5.390068in}{0.655014in}}%
\pgfpathlineto{\pgfqpoint{5.390602in}{0.654457in}}%
\pgfpathlineto{\pgfqpoint{5.391136in}{0.655081in}}%
\pgfpathlineto{\pgfqpoint{5.392203in}{0.641807in}}%
\pgfpathlineto{\pgfqpoint{5.392737in}{0.650140in}}%
\pgfpathlineto{\pgfqpoint{5.393270in}{0.656281in}}%
\pgfpathlineto{\pgfqpoint{5.393804in}{0.647300in}}%
\pgfpathlineto{\pgfqpoint{5.394338in}{0.651617in}}%
\pgfpathlineto{\pgfqpoint{5.394872in}{0.648295in}}%
\pgfpathlineto{\pgfqpoint{5.395405in}{0.637864in}}%
\pgfpathlineto{\pgfqpoint{5.395939in}{0.640360in}}%
\pgfpathlineto{\pgfqpoint{5.397006in}{0.655920in}}%
\pgfpathlineto{\pgfqpoint{5.397540in}{0.651002in}}%
\pgfpathlineto{\pgfqpoint{5.398607in}{0.638961in}}%
\pgfpathlineto{\pgfqpoint{5.399141in}{0.654609in}}%
\pgfpathlineto{\pgfqpoint{5.399675in}{0.646109in}}%
\pgfpathlineto{\pgfqpoint{5.400209in}{0.648622in}}%
\pgfpathlineto{\pgfqpoint{5.400742in}{0.639092in}}%
\pgfpathlineto{\pgfqpoint{5.401276in}{0.642454in}}%
\pgfpathlineto{\pgfqpoint{5.402343in}{0.641276in}}%
\pgfpathlineto{\pgfqpoint{5.403944in}{0.649762in}}%
\pgfpathlineto{\pgfqpoint{5.404478in}{0.648870in}}%
\pgfpathlineto{\pgfqpoint{5.405546in}{0.643899in}}%
\pgfpathlineto{\pgfqpoint{5.407680in}{0.656138in}}%
\pgfpathlineto{\pgfqpoint{5.408214in}{0.654873in}}%
\pgfpathlineto{\pgfqpoint{5.408748in}{0.655586in}}%
\pgfpathlineto{\pgfqpoint{5.409815in}{0.639187in}}%
\pgfpathlineto{\pgfqpoint{5.410349in}{0.657221in}}%
\pgfpathlineto{\pgfqpoint{5.410883in}{0.647541in}}%
\pgfpathlineto{\pgfqpoint{5.411416in}{0.650184in}}%
\pgfpathlineto{\pgfqpoint{5.412484in}{0.641582in}}%
\pgfpathlineto{\pgfqpoint{5.413017in}{0.643764in}}%
\pgfpathlineto{\pgfqpoint{5.413551in}{0.649409in}}%
\pgfpathlineto{\pgfqpoint{5.414085in}{0.644649in}}%
\pgfpathlineto{\pgfqpoint{5.414618in}{0.636719in}}%
\pgfpathlineto{\pgfqpoint{5.415152in}{0.637541in}}%
\pgfpathlineto{\pgfqpoint{5.416220in}{0.653532in}}%
\pgfpathlineto{\pgfqpoint{5.417287in}{0.637688in}}%
\pgfpathlineto{\pgfqpoint{5.417821in}{0.640343in}}%
\pgfpathlineto{\pgfqpoint{5.418354in}{0.649195in}}%
\pgfpathlineto{\pgfqpoint{5.418888in}{0.641375in}}%
\pgfpathlineto{\pgfqpoint{5.419422in}{0.641392in}}%
\pgfpathlineto{\pgfqpoint{5.420489in}{0.648475in}}%
\pgfpathlineto{\pgfqpoint{5.421023in}{0.637831in}}%
\pgfpathlineto{\pgfqpoint{5.421557in}{0.645013in}}%
\pgfpathlineto{\pgfqpoint{5.422090in}{0.646568in}}%
\pgfpathlineto{\pgfqpoint{5.422624in}{0.651708in}}%
\pgfpathlineto{\pgfqpoint{5.423158in}{0.644284in}}%
\pgfpathlineto{\pgfqpoint{5.423691in}{0.655205in}}%
\pgfpathlineto{\pgfqpoint{5.424225in}{0.647015in}}%
\pgfpathlineto{\pgfqpoint{5.424759in}{0.637545in}}%
\pgfpathlineto{\pgfqpoint{5.425292in}{0.640660in}}%
\pgfpathlineto{\pgfqpoint{5.426360in}{0.648170in}}%
\pgfpathlineto{\pgfqpoint{5.427427in}{0.637558in}}%
\pgfpathlineto{\pgfqpoint{5.428495in}{0.649265in}}%
\pgfpathlineto{\pgfqpoint{5.430629in}{0.641173in}}%
\pgfpathlineto{\pgfqpoint{5.432231in}{0.649668in}}%
\pgfpathlineto{\pgfqpoint{5.432764in}{0.645613in}}%
\pgfpathlineto{\pgfqpoint{5.433298in}{0.655177in}}%
\pgfpathlineto{\pgfqpoint{5.433832in}{0.645083in}}%
\pgfpathlineto{\pgfqpoint{5.434365in}{0.649545in}}%
\pgfpathlineto{\pgfqpoint{5.434899in}{0.644924in}}%
\pgfpathlineto{\pgfqpoint{5.435433in}{0.645950in}}%
\pgfpathlineto{\pgfqpoint{5.435966in}{0.657710in}}%
\pgfpathlineto{\pgfqpoint{5.436500in}{0.639879in}}%
\pgfpathlineto{\pgfqpoint{5.437568in}{0.640287in}}%
\pgfpathlineto{\pgfqpoint{5.439169in}{0.655459in}}%
\pgfpathlineto{\pgfqpoint{5.440236in}{0.644002in}}%
\pgfpathlineto{\pgfqpoint{5.440770in}{0.649364in}}%
\pgfpathlineto{\pgfqpoint{5.442371in}{0.639741in}}%
\pgfpathlineto{\pgfqpoint{5.443438in}{0.643284in}}%
\pgfpathlineto{\pgfqpoint{5.444506in}{0.640844in}}%
\pgfpathlineto{\pgfqpoint{5.445039in}{0.649728in}}%
\pgfpathlineto{\pgfqpoint{5.445573in}{0.638008in}}%
\pgfpathlineto{\pgfqpoint{5.446107in}{0.647092in}}%
\pgfpathlineto{\pgfqpoint{5.446641in}{0.647051in}}%
\pgfpathlineto{\pgfqpoint{5.447174in}{0.642072in}}%
\pgfpathlineto{\pgfqpoint{5.447708in}{0.653343in}}%
\pgfpathlineto{\pgfqpoint{5.448242in}{0.648667in}}%
\pgfpathlineto{\pgfqpoint{5.448775in}{0.648946in}}%
\pgfpathlineto{\pgfqpoint{5.449309in}{0.640441in}}%
\pgfpathlineto{\pgfqpoint{5.449843in}{0.641211in}}%
\pgfpathlineto{\pgfqpoint{5.451444in}{0.657049in}}%
\pgfpathlineto{\pgfqpoint{5.453579in}{0.642105in}}%
\pgfpathlineto{\pgfqpoint{5.455180in}{0.649168in}}%
\pgfpathlineto{\pgfqpoint{5.455713in}{0.638673in}}%
\pgfpathlineto{\pgfqpoint{5.456247in}{0.642862in}}%
\pgfpathlineto{\pgfqpoint{5.456781in}{0.646779in}}%
\pgfpathlineto{\pgfqpoint{5.458382in}{0.636407in}}%
\pgfpathlineto{\pgfqpoint{5.459449in}{0.650856in}}%
\pgfpathlineto{\pgfqpoint{5.459983in}{0.645119in}}%
\pgfpathlineto{\pgfqpoint{5.460517in}{0.639685in}}%
\pgfpathlineto{\pgfqpoint{5.462118in}{0.652604in}}%
\pgfpathlineto{\pgfqpoint{5.462652in}{0.647103in}}%
\pgfpathlineto{\pgfqpoint{5.463185in}{0.653854in}}%
\pgfpathlineto{\pgfqpoint{5.464786in}{0.637829in}}%
\pgfpathlineto{\pgfqpoint{5.465854in}{0.646282in}}%
\pgfpathlineto{\pgfqpoint{5.466387in}{0.641205in}}%
\pgfpathlineto{\pgfqpoint{5.466921in}{0.646723in}}%
\pgfpathlineto{\pgfqpoint{5.467455in}{0.647102in}}%
\pgfpathlineto{\pgfqpoint{5.467989in}{0.637178in}}%
\pgfpathlineto{\pgfqpoint{5.468522in}{0.641031in}}%
\pgfpathlineto{\pgfqpoint{5.469056in}{0.647225in}}%
\pgfpathlineto{\pgfqpoint{5.469590in}{0.641768in}}%
\pgfpathlineto{\pgfqpoint{5.470123in}{0.636407in}}%
\pgfpathlineto{\pgfqpoint{5.470657in}{0.647555in}}%
\pgfpathlineto{\pgfqpoint{5.471191in}{0.646626in}}%
\pgfpathlineto{\pgfqpoint{5.472792in}{0.638151in}}%
\pgfpathlineto{\pgfqpoint{5.473326in}{0.657866in}}%
\pgfpathlineto{\pgfqpoint{5.473859in}{0.648316in}}%
\pgfpathlineto{\pgfqpoint{5.474927in}{0.638651in}}%
\pgfpathlineto{\pgfqpoint{5.475460in}{0.646829in}}%
\pgfpathlineto{\pgfqpoint{5.475994in}{0.642866in}}%
\pgfpathlineto{\pgfqpoint{5.477595in}{0.646410in}}%
\pgfpathlineto{\pgfqpoint{5.478129in}{0.636143in}}%
\pgfpathlineto{\pgfqpoint{5.478663in}{0.644617in}}%
\pgfpathlineto{\pgfqpoint{5.479730in}{0.641265in}}%
\pgfpathlineto{\pgfqpoint{5.480264in}{0.650710in}}%
\pgfpathlineto{\pgfqpoint{5.480797in}{0.647950in}}%
\pgfpathlineto{\pgfqpoint{5.482398in}{0.641614in}}%
\pgfpathlineto{\pgfqpoint{5.482932in}{0.639750in}}%
\pgfpathlineto{\pgfqpoint{5.483466in}{0.648961in}}%
\pgfpathlineto{\pgfqpoint{5.484000in}{0.640576in}}%
\pgfpathlineto{\pgfqpoint{5.484533in}{0.642222in}}%
\pgfpathlineto{\pgfqpoint{5.485067in}{0.640788in}}%
\pgfpathlineto{\pgfqpoint{5.485601in}{0.641757in}}%
\pgfpathlineto{\pgfqpoint{5.486668in}{0.649443in}}%
\pgfpathlineto{\pgfqpoint{5.487735in}{0.640984in}}%
\pgfpathlineto{\pgfqpoint{5.488269in}{0.650418in}}%
\pgfpathlineto{\pgfqpoint{5.488803in}{0.640486in}}%
\pgfpathlineto{\pgfqpoint{5.489337in}{0.638976in}}%
\pgfpathlineto{\pgfqpoint{5.489870in}{0.645170in}}%
\pgfpathlineto{\pgfqpoint{5.490404in}{0.637127in}}%
\pgfpathlineto{\pgfqpoint{5.490938in}{0.654643in}}%
\pgfpathlineto{\pgfqpoint{5.491471in}{0.648033in}}%
\pgfpathlineto{\pgfqpoint{5.492005in}{0.646120in}}%
\pgfpathlineto{\pgfqpoint{5.492539in}{0.639969in}}%
\pgfpathlineto{\pgfqpoint{5.493606in}{0.650301in}}%
\pgfpathlineto{\pgfqpoint{5.495207in}{0.637587in}}%
\pgfpathlineto{\pgfqpoint{5.495741in}{0.646067in}}%
\pgfpathlineto{\pgfqpoint{5.496275in}{0.635634in}}%
\pgfpathlineto{\pgfqpoint{5.496808in}{0.646370in}}%
\pgfpathlineto{\pgfqpoint{5.497342in}{0.646743in}}%
\pgfpathlineto{\pgfqpoint{5.497876in}{0.648963in}}%
\pgfpathlineto{\pgfqpoint{5.499477in}{0.639942in}}%
\pgfpathlineto{\pgfqpoint{5.500011in}{0.644524in}}%
\pgfpathlineto{\pgfqpoint{5.500544in}{0.639269in}}%
\pgfpathlineto{\pgfqpoint{5.501078in}{0.644578in}}%
\pgfpathlineto{\pgfqpoint{5.501612in}{0.652384in}}%
\pgfpathlineto{\pgfqpoint{5.502145in}{0.640407in}}%
\pgfpathlineto{\pgfqpoint{5.502679in}{0.643908in}}%
\pgfpathlineto{\pgfqpoint{5.503213in}{0.644343in}}%
\pgfpathlineto{\pgfqpoint{5.503746in}{0.651096in}}%
\pgfpathlineto{\pgfqpoint{5.504280in}{0.645908in}}%
\pgfpathlineto{\pgfqpoint{5.505881in}{0.638967in}}%
\pgfpathlineto{\pgfqpoint{5.507482in}{0.658078in}}%
\pgfpathlineto{\pgfqpoint{5.508550in}{0.639514in}}%
\pgfpathlineto{\pgfqpoint{5.509084in}{0.640945in}}%
\pgfpathlineto{\pgfqpoint{5.509617in}{0.644849in}}%
\pgfpathlineto{\pgfqpoint{5.510151in}{0.636959in}}%
\pgfpathlineto{\pgfqpoint{5.510685in}{0.642198in}}%
\pgfpathlineto{\pgfqpoint{5.511218in}{0.638450in}}%
\pgfpathlineto{\pgfqpoint{5.512286in}{0.644969in}}%
\pgfpathlineto{\pgfqpoint{5.512819in}{0.637292in}}%
\pgfpathlineto{\pgfqpoint{5.513353in}{0.642249in}}%
\pgfpathlineto{\pgfqpoint{5.513887in}{0.646507in}}%
\pgfpathlineto{\pgfqpoint{5.514421in}{0.641885in}}%
\pgfpathlineto{\pgfqpoint{5.514954in}{0.640814in}}%
\pgfpathlineto{\pgfqpoint{5.515488in}{0.642588in}}%
\pgfpathlineto{\pgfqpoint{5.516022in}{0.639539in}}%
\pgfpathlineto{\pgfqpoint{5.518156in}{0.653286in}}%
\pgfpathlineto{\pgfqpoint{5.519758in}{0.637440in}}%
\pgfpathlineto{\pgfqpoint{5.520291in}{0.639838in}}%
\pgfpathlineto{\pgfqpoint{5.520825in}{0.644050in}}%
\pgfpathlineto{\pgfqpoint{5.521359in}{0.639301in}}%
\pgfpathlineto{\pgfqpoint{5.521892in}{0.649860in}}%
\pgfpathlineto{\pgfqpoint{5.522426in}{0.641883in}}%
\pgfpathlineto{\pgfqpoint{5.522960in}{0.642753in}}%
\pgfpathlineto{\pgfqpoint{5.523493in}{0.638143in}}%
\pgfpathlineto{\pgfqpoint{5.524027in}{0.640512in}}%
\pgfpathlineto{\pgfqpoint{5.524561in}{0.641982in}}%
\pgfpathlineto{\pgfqpoint{5.525095in}{0.637237in}}%
\pgfpathlineto{\pgfqpoint{5.525628in}{0.646203in}}%
\pgfpathlineto{\pgfqpoint{5.526162in}{0.639594in}}%
\pgfpathlineto{\pgfqpoint{5.527763in}{0.648616in}}%
\pgfpathlineto{\pgfqpoint{5.528297in}{0.646954in}}%
\pgfpathlineto{\pgfqpoint{5.529364in}{0.636085in}}%
\pgfpathlineto{\pgfqpoint{5.530432in}{0.639524in}}%
\pgfpathlineto{\pgfqpoint{5.530965in}{0.646134in}}%
\pgfpathlineto{\pgfqpoint{5.531499in}{0.643650in}}%
\pgfpathlineto{\pgfqpoint{5.532033in}{0.637603in}}%
\pgfpathlineto{\pgfqpoint{5.532566in}{0.639727in}}%
\pgfpathlineto{\pgfqpoint{5.533100in}{0.639827in}}%
\pgfpathlineto{\pgfqpoint{5.533634in}{0.636020in}}%
\pgfpathlineto{\pgfqpoint{5.534701in}{0.647621in}}%
\pgfpathlineto{\pgfqpoint{5.535235in}{0.645129in}}%
\pgfpathlineto{\pgfqpoint{5.536836in}{0.638405in}}%
\pgfpathlineto{\pgfqpoint{5.537903in}{0.638958in}}%
\pgfpathlineto{\pgfqpoint{5.538437in}{0.648308in}}%
\pgfpathlineto{\pgfqpoint{5.538971in}{0.640914in}}%
\pgfpathlineto{\pgfqpoint{5.539504in}{0.636899in}}%
\pgfpathlineto{\pgfqpoint{5.541106in}{0.645805in}}%
\pgfpathlineto{\pgfqpoint{5.542173in}{0.639145in}}%
\pgfpathlineto{\pgfqpoint{5.543774in}{0.645566in}}%
\pgfpathlineto{\pgfqpoint{5.544308in}{0.643158in}}%
\pgfpathlineto{\pgfqpoint{5.544841in}{0.650782in}}%
\pgfpathlineto{\pgfqpoint{5.545375in}{0.644164in}}%
\pgfpathlineto{\pgfqpoint{5.545909in}{0.637503in}}%
\pgfpathlineto{\pgfqpoint{5.546443in}{0.648953in}}%
\pgfpathlineto{\pgfqpoint{5.546976in}{0.642470in}}%
\pgfpathlineto{\pgfqpoint{5.549111in}{0.652289in}}%
\pgfpathlineto{\pgfqpoint{5.549645in}{0.643672in}}%
\pgfpathlineto{\pgfqpoint{5.550178in}{0.644597in}}%
\pgfpathlineto{\pgfqpoint{5.550712in}{0.645066in}}%
\pgfpathlineto{\pgfqpoint{5.552313in}{0.640343in}}%
\pgfpathlineto{\pgfqpoint{5.552847in}{0.643410in}}%
\pgfpathlineto{\pgfqpoint{5.553914in}{0.638401in}}%
\pgfpathlineto{\pgfqpoint{5.554448in}{0.644701in}}%
\pgfpathlineto{\pgfqpoint{5.554982in}{0.635966in}}%
\pgfpathlineto{\pgfqpoint{5.555515in}{0.640074in}}%
\pgfpathlineto{\pgfqpoint{5.558184in}{0.643630in}}%
\pgfpathlineto{\pgfqpoint{5.558718in}{0.643005in}}%
\pgfpathlineto{\pgfqpoint{5.560319in}{0.640378in}}%
\pgfpathlineto{\pgfqpoint{5.560852in}{0.639698in}}%
\pgfpathlineto{\pgfqpoint{5.562454in}{0.647079in}}%
\pgfpathlineto{\pgfqpoint{5.562987in}{0.644742in}}%
\pgfpathlineto{\pgfqpoint{5.564055in}{0.636442in}}%
\pgfpathlineto{\pgfqpoint{5.565656in}{0.650461in}}%
\pgfpathlineto{\pgfqpoint{5.566189in}{0.639631in}}%
\pgfpathlineto{\pgfqpoint{5.566723in}{0.641829in}}%
\pgfpathlineto{\pgfqpoint{5.567257in}{0.642267in}}%
\pgfpathlineto{\pgfqpoint{5.567791in}{0.636751in}}%
\pgfpathlineto{\pgfqpoint{5.568324in}{0.641544in}}%
\pgfpathlineto{\pgfqpoint{5.568858in}{0.644138in}}%
\pgfpathlineto{\pgfqpoint{5.569392in}{0.643787in}}%
\pgfpathlineto{\pgfqpoint{5.570459in}{0.639670in}}%
\pgfpathlineto{\pgfqpoint{5.570993in}{0.641239in}}%
\pgfpathlineto{\pgfqpoint{5.571526in}{0.643176in}}%
\pgfpathlineto{\pgfqpoint{5.572060in}{0.651622in}}%
\pgfpathlineto{\pgfqpoint{5.572594in}{0.646371in}}%
\pgfpathlineto{\pgfqpoint{5.573128in}{0.642789in}}%
\pgfpathlineto{\pgfqpoint{5.573661in}{0.650284in}}%
\pgfpathlineto{\pgfqpoint{5.575262in}{0.637726in}}%
\pgfpathlineto{\pgfqpoint{5.575796in}{0.641168in}}%
\pgfpathlineto{\pgfqpoint{5.576330in}{0.638979in}}%
\pgfpathlineto{\pgfqpoint{5.576864in}{0.637398in}}%
\pgfpathlineto{\pgfqpoint{5.577397in}{0.645255in}}%
\pgfpathlineto{\pgfqpoint{5.577931in}{0.636150in}}%
\pgfpathlineto{\pgfqpoint{5.578465in}{0.637070in}}%
\pgfpathlineto{\pgfqpoint{5.578998in}{0.643293in}}%
\pgfpathlineto{\pgfqpoint{5.579532in}{0.641974in}}%
\pgfpathlineto{\pgfqpoint{5.580599in}{0.638641in}}%
\pgfpathlineto{\pgfqpoint{5.581133in}{0.643242in}}%
\pgfpathlineto{\pgfqpoint{5.582734in}{0.636725in}}%
\pgfpathlineto{\pgfqpoint{5.583802in}{0.644856in}}%
\pgfpathlineto{\pgfqpoint{5.584335in}{0.638359in}}%
\pgfpathlineto{\pgfqpoint{5.584869in}{0.639400in}}%
\pgfpathlineto{\pgfqpoint{5.585403in}{0.640709in}}%
\pgfpathlineto{\pgfqpoint{5.585936in}{0.639905in}}%
\pgfpathlineto{\pgfqpoint{5.587538in}{0.640163in}}%
\pgfpathlineto{\pgfqpoint{5.588071in}{0.642418in}}%
\pgfpathlineto{\pgfqpoint{5.588605in}{0.636355in}}%
\pgfpathlineto{\pgfqpoint{5.589139in}{0.649220in}}%
\pgfpathlineto{\pgfqpoint{5.589672in}{0.646076in}}%
\pgfpathlineto{\pgfqpoint{5.590740in}{0.637866in}}%
\pgfpathlineto{\pgfqpoint{5.592341in}{0.644102in}}%
\pgfpathlineto{\pgfqpoint{5.593408in}{0.636052in}}%
\pgfpathlineto{\pgfqpoint{5.595543in}{0.652392in}}%
\pgfpathlineto{\pgfqpoint{5.596610in}{0.637192in}}%
\pgfpathlineto{\pgfqpoint{5.597678in}{0.642547in}}%
\pgfpathlineto{\pgfqpoint{5.598212in}{0.639194in}}%
\pgfpathlineto{\pgfqpoint{5.598745in}{0.643507in}}%
\pgfpathlineto{\pgfqpoint{5.600346in}{0.637081in}}%
\pgfpathlineto{\pgfqpoint{5.600880in}{0.637506in}}%
\pgfpathlineto{\pgfqpoint{5.601414in}{0.643169in}}%
\pgfpathlineto{\pgfqpoint{5.601947in}{0.637292in}}%
\pgfpathlineto{\pgfqpoint{5.602481in}{0.635554in}}%
\pgfpathlineto{\pgfqpoint{5.603549in}{0.646415in}}%
\pgfpathlineto{\pgfqpoint{5.604082in}{0.645794in}}%
\pgfpathlineto{\pgfqpoint{5.604616in}{0.644461in}}%
\pgfpathlineto{\pgfqpoint{5.605150in}{0.638744in}}%
\pgfpathlineto{\pgfqpoint{5.606217in}{0.638940in}}%
\pgfpathlineto{\pgfqpoint{5.606751in}{0.637033in}}%
\pgfpathlineto{\pgfqpoint{5.607818in}{0.641524in}}%
\pgfpathlineto{\pgfqpoint{5.608352in}{0.639063in}}%
\pgfpathlineto{\pgfqpoint{5.608886in}{0.644581in}}%
\pgfpathlineto{\pgfqpoint{5.609419in}{0.644357in}}%
\pgfpathlineto{\pgfqpoint{5.609953in}{0.637525in}}%
\pgfpathlineto{\pgfqpoint{5.610487in}{0.639159in}}%
\pgfpathlineto{\pgfqpoint{5.612088in}{0.641381in}}%
\pgfpathlineto{\pgfqpoint{5.612621in}{0.637616in}}%
\pgfpathlineto{\pgfqpoint{5.613155in}{0.642771in}}%
\pgfpathlineto{\pgfqpoint{5.613689in}{0.641199in}}%
\pgfpathlineto{\pgfqpoint{5.614223in}{0.636716in}}%
\pgfpathlineto{\pgfqpoint{5.614756in}{0.638928in}}%
\pgfpathlineto{\pgfqpoint{5.615290in}{0.644941in}}%
\pgfpathlineto{\pgfqpoint{5.615824in}{0.641012in}}%
\pgfpathlineto{\pgfqpoint{5.616357in}{0.639966in}}%
\pgfpathlineto{\pgfqpoint{5.617958in}{0.648300in}}%
\pgfpathlineto{\pgfqpoint{5.619560in}{0.639826in}}%
\pgfpathlineto{\pgfqpoint{5.620627in}{0.636489in}}%
\pgfpathlineto{\pgfqpoint{5.622228in}{0.644739in}}%
\pgfpathlineto{\pgfqpoint{5.622762in}{0.645235in}}%
\pgfpathlineto{\pgfqpoint{5.623295in}{0.639370in}}%
\pgfpathlineto{\pgfqpoint{5.623829in}{0.644809in}}%
\pgfpathlineto{\pgfqpoint{5.624363in}{0.646771in}}%
\pgfpathlineto{\pgfqpoint{5.625430in}{0.638618in}}%
\pgfpathlineto{\pgfqpoint{5.626498in}{0.640207in}}%
\pgfpathlineto{\pgfqpoint{5.627031in}{0.641742in}}%
\pgfpathlineto{\pgfqpoint{5.627565in}{0.641430in}}%
\pgfpathlineto{\pgfqpoint{5.628632in}{0.652397in}}%
\pgfpathlineto{\pgfqpoint{5.629166in}{0.636375in}}%
\pgfpathlineto{\pgfqpoint{5.630234in}{0.637001in}}%
\pgfpathlineto{\pgfqpoint{5.631301in}{0.637540in}}%
\pgfpathlineto{\pgfqpoint{5.632368in}{0.645212in}}%
\pgfpathlineto{\pgfqpoint{5.632902in}{0.637589in}}%
\pgfpathlineto{\pgfqpoint{5.633436in}{0.639390in}}%
\pgfpathlineto{\pgfqpoint{5.633969in}{0.639934in}}%
\pgfpathlineto{\pgfqpoint{5.635571in}{0.636504in}}%
\pgfpathlineto{\pgfqpoint{5.637172in}{0.642568in}}%
\pgfpathlineto{\pgfqpoint{5.638239in}{0.639672in}}%
\pgfpathlineto{\pgfqpoint{5.638773in}{0.646956in}}%
\pgfpathlineto{\pgfqpoint{5.639306in}{0.641213in}}%
\pgfpathlineto{\pgfqpoint{5.640374in}{0.635790in}}%
\pgfpathlineto{\pgfqpoint{5.640908in}{0.641089in}}%
\pgfpathlineto{\pgfqpoint{5.641441in}{0.639124in}}%
\pgfpathlineto{\pgfqpoint{5.642509in}{0.637095in}}%
\pgfpathlineto{\pgfqpoint{5.643042in}{0.644345in}}%
\pgfpathlineto{\pgfqpoint{5.644110in}{0.643833in}}%
\pgfpathlineto{\pgfqpoint{5.645711in}{0.639216in}}%
\pgfpathlineto{\pgfqpoint{5.646245in}{0.643709in}}%
\pgfpathlineto{\pgfqpoint{5.647312in}{0.643406in}}%
\pgfpathlineto{\pgfqpoint{5.648379in}{0.641000in}}%
\pgfpathlineto{\pgfqpoint{5.648913in}{0.642987in}}%
\pgfpathlineto{\pgfqpoint{5.649447in}{0.638021in}}%
\pgfpathlineto{\pgfqpoint{5.649981in}{0.644056in}}%
\pgfpathlineto{\pgfqpoint{5.650514in}{0.635877in}}%
\pgfpathlineto{\pgfqpoint{5.651048in}{0.636099in}}%
\pgfpathlineto{\pgfqpoint{5.651582in}{0.638086in}}%
\pgfpathlineto{\pgfqpoint{5.652115in}{0.646629in}}%
\pgfpathlineto{\pgfqpoint{5.652649in}{0.641898in}}%
\pgfpathlineto{\pgfqpoint{5.653183in}{0.637789in}}%
\pgfpathlineto{\pgfqpoint{5.653716in}{0.640956in}}%
\pgfpathlineto{\pgfqpoint{5.654250in}{0.642495in}}%
\pgfpathlineto{\pgfqpoint{5.654784in}{0.635653in}}%
\pgfpathlineto{\pgfqpoint{5.655318in}{0.640171in}}%
\pgfpathlineto{\pgfqpoint{5.655851in}{0.637423in}}%
\pgfpathlineto{\pgfqpoint{5.656385in}{0.644366in}}%
\pgfpathlineto{\pgfqpoint{5.656919in}{0.635841in}}%
\pgfpathlineto{\pgfqpoint{5.657452in}{0.643767in}}%
\pgfpathlineto{\pgfqpoint{5.657986in}{0.640720in}}%
\pgfpathlineto{\pgfqpoint{5.658520in}{0.643683in}}%
\pgfpathlineto{\pgfqpoint{5.659053in}{0.642163in}}%
\pgfpathlineto{\pgfqpoint{5.659587in}{0.643685in}}%
\pgfpathlineto{\pgfqpoint{5.660121in}{0.644926in}}%
\pgfpathlineto{\pgfqpoint{5.661722in}{0.637887in}}%
\pgfpathlineto{\pgfqpoint{5.662256in}{0.638173in}}%
\pgfpathlineto{\pgfqpoint{5.663857in}{0.642813in}}%
\pgfpathlineto{\pgfqpoint{5.665992in}{0.635981in}}%
\pgfpathlineto{\pgfqpoint{5.666525in}{0.642566in}}%
\pgfpathlineto{\pgfqpoint{5.667059in}{0.638690in}}%
\pgfpathlineto{\pgfqpoint{5.668126in}{0.644075in}}%
\pgfpathlineto{\pgfqpoint{5.669194in}{0.637987in}}%
\pgfpathlineto{\pgfqpoint{5.670795in}{0.644661in}}%
\pgfpathlineto{\pgfqpoint{5.672396in}{0.637557in}}%
\pgfpathlineto{\pgfqpoint{5.673463in}{0.649702in}}%
\pgfpathlineto{\pgfqpoint{5.674531in}{0.637650in}}%
\pgfpathlineto{\pgfqpoint{5.675598in}{0.640972in}}%
\pgfpathlineto{\pgfqpoint{5.676132in}{0.638990in}}%
\pgfpathlineto{\pgfqpoint{5.676666in}{0.639785in}}%
\pgfpathlineto{\pgfqpoint{5.677199in}{0.640769in}}%
\pgfpathlineto{\pgfqpoint{5.677733in}{0.639327in}}%
\pgfpathlineto{\pgfqpoint{5.678267in}{0.642384in}}%
\pgfpathlineto{\pgfqpoint{5.678800in}{0.639848in}}%
\pgfpathlineto{\pgfqpoint{5.679334in}{0.639035in}}%
\pgfpathlineto{\pgfqpoint{5.680401in}{0.648492in}}%
\pgfpathlineto{\pgfqpoint{5.682003in}{0.637358in}}%
\pgfpathlineto{\pgfqpoint{5.682536in}{0.649569in}}%
\pgfpathlineto{\pgfqpoint{5.683070in}{0.636985in}}%
\pgfpathlineto{\pgfqpoint{5.683604in}{0.643627in}}%
\pgfpathlineto{\pgfqpoint{5.684137in}{0.641709in}}%
\pgfpathlineto{\pgfqpoint{5.684671in}{0.639436in}}%
\pgfpathlineto{\pgfqpoint{5.685738in}{0.644308in}}%
\pgfpathlineto{\pgfqpoint{5.686806in}{0.639008in}}%
\pgfpathlineto{\pgfqpoint{5.687340in}{0.641680in}}%
\pgfpathlineto{\pgfqpoint{5.687873in}{0.639030in}}%
\pgfpathlineto{\pgfqpoint{5.688407in}{0.639292in}}%
\pgfpathlineto{\pgfqpoint{5.688941in}{0.636141in}}%
\pgfpathlineto{\pgfqpoint{5.689474in}{0.643346in}}%
\pgfpathlineto{\pgfqpoint{5.690008in}{0.637267in}}%
\pgfpathlineto{\pgfqpoint{5.690542in}{0.641798in}}%
\pgfpathlineto{\pgfqpoint{5.691075in}{0.638616in}}%
\pgfpathlineto{\pgfqpoint{5.692143in}{0.638963in}}%
\pgfpathlineto{\pgfqpoint{5.692677in}{0.643015in}}%
\pgfpathlineto{\pgfqpoint{5.693210in}{0.638854in}}%
\pgfpathlineto{\pgfqpoint{5.693744in}{0.637751in}}%
\pgfpathlineto{\pgfqpoint{5.694278in}{0.638947in}}%
\pgfpathlineto{\pgfqpoint{5.694811in}{0.642771in}}%
\pgfpathlineto{\pgfqpoint{5.695345in}{0.639621in}}%
\pgfpathlineto{\pgfqpoint{5.695879in}{0.638920in}}%
\pgfpathlineto{\pgfqpoint{5.696412in}{0.640739in}}%
\pgfpathlineto{\pgfqpoint{5.696946in}{0.639951in}}%
\pgfpathlineto{\pgfqpoint{5.697480in}{0.636903in}}%
\pgfpathlineto{\pgfqpoint{5.698014in}{0.640268in}}%
\pgfpathlineto{\pgfqpoint{5.698547in}{0.637158in}}%
\pgfpathlineto{\pgfqpoint{5.700682in}{0.641494in}}%
\pgfpathlineto{\pgfqpoint{5.701216in}{0.639476in}}%
\pgfpathlineto{\pgfqpoint{5.701749in}{0.647034in}}%
\pgfpathlineto{\pgfqpoint{5.702283in}{0.642430in}}%
\pgfpathlineto{\pgfqpoint{5.703884in}{0.638956in}}%
\pgfpathlineto{\pgfqpoint{5.704418in}{0.641984in}}%
\pgfpathlineto{\pgfqpoint{5.704952in}{0.638439in}}%
\pgfpathlineto{\pgfqpoint{5.705485in}{0.640219in}}%
\pgfpathlineto{\pgfqpoint{5.706553in}{0.645715in}}%
\pgfpathlineto{\pgfqpoint{5.708154in}{0.640433in}}%
\pgfpathlineto{\pgfqpoint{5.708688in}{0.638057in}}%
\pgfpathlineto{\pgfqpoint{5.709221in}{0.645237in}}%
\pgfpathlineto{\pgfqpoint{5.709755in}{0.639469in}}%
\pgfpathlineto{\pgfqpoint{5.710289in}{0.644444in}}%
\pgfpathlineto{\pgfqpoint{5.710822in}{0.643016in}}%
\pgfpathlineto{\pgfqpoint{5.711890in}{0.637360in}}%
\pgfpathlineto{\pgfqpoint{5.712424in}{0.638569in}}%
\pgfpathlineto{\pgfqpoint{5.712957in}{0.639766in}}%
\pgfpathlineto{\pgfqpoint{5.713491in}{0.637621in}}%
\pgfpathlineto{\pgfqpoint{5.714025in}{0.639160in}}%
\pgfpathlineto{\pgfqpoint{5.714558in}{0.638621in}}%
\pgfpathlineto{\pgfqpoint{5.715092in}{0.644377in}}%
\pgfpathlineto{\pgfqpoint{5.715626in}{0.642634in}}%
\pgfpathlineto{\pgfqpoint{5.716159in}{0.642573in}}%
\pgfpathlineto{\pgfqpoint{5.716693in}{0.636074in}}%
\pgfpathlineto{\pgfqpoint{5.717761in}{0.644288in}}%
\pgfpathlineto{\pgfqpoint{5.719362in}{0.638734in}}%
\pgfpathlineto{\pgfqpoint{5.719895in}{0.640023in}}%
\pgfpathlineto{\pgfqpoint{5.720429in}{0.636157in}}%
\pgfpathlineto{\pgfqpoint{5.720963in}{0.639118in}}%
\pgfpathlineto{\pgfqpoint{5.722030in}{0.642673in}}%
\pgfpathlineto{\pgfqpoint{5.722564in}{0.637601in}}%
\pgfpathlineto{\pgfqpoint{5.723098in}{0.640746in}}%
\pgfpathlineto{\pgfqpoint{5.723631in}{0.640993in}}%
\pgfpathlineto{\pgfqpoint{5.724165in}{0.637904in}}%
\pgfpathlineto{\pgfqpoint{5.724699in}{0.640070in}}%
\pgfpathlineto{\pgfqpoint{5.725232in}{0.646388in}}%
\pgfpathlineto{\pgfqpoint{5.725766in}{0.642846in}}%
\pgfpathlineto{\pgfqpoint{5.726300in}{0.645660in}}%
\pgfpathlineto{\pgfqpoint{5.726833in}{0.639579in}}%
\pgfpathlineto{\pgfqpoint{5.727367in}{0.640518in}}%
\pgfpathlineto{\pgfqpoint{5.727901in}{0.643238in}}%
\pgfpathlineto{\pgfqpoint{5.729502in}{0.636046in}}%
\pgfpathlineto{\pgfqpoint{5.731103in}{0.644810in}}%
\pgfpathlineto{\pgfqpoint{5.732170in}{0.637467in}}%
\pgfpathlineto{\pgfqpoint{5.732704in}{0.644021in}}%
\pgfpathlineto{\pgfqpoint{5.733238in}{0.637327in}}%
\pgfpathlineto{\pgfqpoint{5.733772in}{0.640302in}}%
\pgfpathlineto{\pgfqpoint{5.734305in}{0.638169in}}%
\pgfpathlineto{\pgfqpoint{5.734839in}{0.637964in}}%
\pgfpathlineto{\pgfqpoint{5.735906in}{0.642994in}}%
\pgfpathlineto{\pgfqpoint{5.736440in}{0.649580in}}%
\pgfpathlineto{\pgfqpoint{5.736974in}{0.646525in}}%
\pgfpathlineto{\pgfqpoint{5.737507in}{0.639517in}}%
\pgfpathlineto{\pgfqpoint{5.738041in}{0.643976in}}%
\pgfpathlineto{\pgfqpoint{5.738575in}{0.640219in}}%
\pgfpathlineto{\pgfqpoint{5.739109in}{0.653221in}}%
\pgfpathlineto{\pgfqpoint{5.739642in}{0.641925in}}%
\pgfpathlineto{\pgfqpoint{5.740710in}{0.637302in}}%
\pgfpathlineto{\pgfqpoint{5.741243in}{0.640996in}}%
\pgfpathlineto{\pgfqpoint{5.742311in}{0.643763in}}%
\pgfpathlineto{\pgfqpoint{5.742844in}{0.641380in}}%
\pgfpathlineto{\pgfqpoint{5.743378in}{0.643077in}}%
\pgfpathlineto{\pgfqpoint{5.746047in}{0.636174in}}%
\pgfpathlineto{\pgfqpoint{5.747648in}{0.640005in}}%
\pgfpathlineto{\pgfqpoint{5.748181in}{0.639479in}}%
\pgfpathlineto{\pgfqpoint{5.748715in}{0.641929in}}%
\pgfpathlineto{\pgfqpoint{5.749249in}{0.639762in}}%
\pgfpathlineto{\pgfqpoint{5.749783in}{0.636323in}}%
\pgfpathlineto{\pgfqpoint{5.750316in}{0.639764in}}%
\pgfpathlineto{\pgfqpoint{5.750850in}{0.638831in}}%
\pgfpathlineto{\pgfqpoint{5.751384in}{0.639447in}}%
\pgfpathlineto{\pgfqpoint{5.751917in}{0.642230in}}%
\pgfpathlineto{\pgfqpoint{5.752451in}{0.640572in}}%
\pgfpathlineto{\pgfqpoint{5.753518in}{0.636303in}}%
\pgfpathlineto{\pgfqpoint{5.754052in}{0.639560in}}%
\pgfpathlineto{\pgfqpoint{5.755120in}{0.641288in}}%
\pgfpathlineto{\pgfqpoint{5.755653in}{0.643700in}}%
\pgfpathlineto{\pgfqpoint{5.756187in}{0.638275in}}%
\pgfpathlineto{\pgfqpoint{5.756721in}{0.640950in}}%
\pgfpathlineto{\pgfqpoint{5.757254in}{0.644921in}}%
\pgfpathlineto{\pgfqpoint{5.758855in}{0.637683in}}%
\pgfpathlineto{\pgfqpoint{5.759389in}{0.636251in}}%
\pgfpathlineto{\pgfqpoint{5.759923in}{0.638916in}}%
\pgfpathlineto{\pgfqpoint{5.760457in}{0.637460in}}%
\pgfpathlineto{\pgfqpoint{5.761524in}{0.638392in}}%
\pgfpathlineto{\pgfqpoint{5.762058in}{0.643717in}}%
\pgfpathlineto{\pgfqpoint{5.762591in}{0.639537in}}%
\pgfpathlineto{\pgfqpoint{5.763125in}{0.636654in}}%
\pgfpathlineto{\pgfqpoint{5.763659in}{0.644085in}}%
\pgfpathlineto{\pgfqpoint{5.764192in}{0.643786in}}%
\pgfpathlineto{\pgfqpoint{5.765260in}{0.639989in}}%
\pgfpathlineto{\pgfqpoint{5.765794in}{0.643158in}}%
\pgfpathlineto{\pgfqpoint{5.766327in}{0.640244in}}%
\pgfpathlineto{\pgfqpoint{5.766861in}{0.641852in}}%
\pgfpathlineto{\pgfqpoint{5.767395in}{0.637926in}}%
\pgfpathlineto{\pgfqpoint{5.767928in}{0.642970in}}%
\pgfpathlineto{\pgfqpoint{5.768462in}{0.638218in}}%
\pgfpathlineto{\pgfqpoint{5.770063in}{0.641013in}}%
\pgfpathlineto{\pgfqpoint{5.770597in}{0.638821in}}%
\pgfpathlineto{\pgfqpoint{5.771131in}{0.643490in}}%
\pgfpathlineto{\pgfqpoint{5.771664in}{0.642599in}}%
\pgfpathlineto{\pgfqpoint{5.772198in}{0.640698in}}%
\pgfpathlineto{\pgfqpoint{5.772732in}{0.645609in}}%
\pgfpathlineto{\pgfqpoint{5.773265in}{0.640591in}}%
\pgfpathlineto{\pgfqpoint{5.774866in}{0.636584in}}%
\pgfpathlineto{\pgfqpoint{5.775400in}{0.638051in}}%
\pgfpathlineto{\pgfqpoint{5.777001in}{0.643247in}}%
\pgfpathlineto{\pgfqpoint{5.778069in}{0.643551in}}%
\pgfpathlineto{\pgfqpoint{5.778602in}{0.636360in}}%
\pgfpathlineto{\pgfqpoint{5.779136in}{0.640806in}}%
\pgfpathlineto{\pgfqpoint{5.779670in}{0.639409in}}%
\pgfpathlineto{\pgfqpoint{5.780204in}{0.638795in}}%
\pgfpathlineto{\pgfqpoint{5.781271in}{0.647007in}}%
\pgfpathlineto{\pgfqpoint{5.781805in}{0.645903in}}%
\pgfpathlineto{\pgfqpoint{5.783406in}{0.641782in}}%
\pgfpathlineto{\pgfqpoint{5.783939in}{0.641845in}}%
\pgfpathlineto{\pgfqpoint{5.785007in}{0.637306in}}%
\pgfpathlineto{\pgfqpoint{5.785541in}{0.642951in}}%
\pgfpathlineto{\pgfqpoint{5.786074in}{0.637009in}}%
\pgfpathlineto{\pgfqpoint{5.787142in}{0.642977in}}%
\pgfpathlineto{\pgfqpoint{5.787675in}{0.640847in}}%
\pgfpathlineto{\pgfqpoint{5.788209in}{0.640556in}}%
\pgfpathlineto{\pgfqpoint{5.788743in}{0.642378in}}%
\pgfpathlineto{\pgfqpoint{5.789810in}{0.637604in}}%
\pgfpathlineto{\pgfqpoint{5.790344in}{0.645778in}}%
\pgfpathlineto{\pgfqpoint{5.790878in}{0.638453in}}%
\pgfpathlineto{\pgfqpoint{5.791945in}{0.642869in}}%
\pgfpathlineto{\pgfqpoint{5.792479in}{0.637282in}}%
\pgfpathlineto{\pgfqpoint{5.793012in}{0.640012in}}%
\pgfpathlineto{\pgfqpoint{5.794080in}{0.639876in}}%
\pgfpathlineto{\pgfqpoint{5.794613in}{0.636876in}}%
\pgfpathlineto{\pgfqpoint{5.795681in}{0.648021in}}%
\pgfpathlineto{\pgfqpoint{5.796215in}{0.643469in}}%
\pgfpathlineto{\pgfqpoint{5.797282in}{0.640521in}}%
\pgfpathlineto{\pgfqpoint{5.797816in}{0.643850in}}%
\pgfpathlineto{\pgfqpoint{5.798349in}{0.637317in}}%
\pgfpathlineto{\pgfqpoint{5.798883in}{0.645045in}}%
\pgfpathlineto{\pgfqpoint{5.799417in}{0.643921in}}%
\pgfpathlineto{\pgfqpoint{5.801018in}{0.640765in}}%
\pgfpathlineto{\pgfqpoint{5.802085in}{0.635504in}}%
\pgfpathlineto{\pgfqpoint{5.802619in}{0.638835in}}%
\pgfpathlineto{\pgfqpoint{5.803153in}{0.642289in}}%
\pgfpathlineto{\pgfqpoint{5.803686in}{0.641253in}}%
\pgfpathlineto{\pgfqpoint{5.804220in}{0.640345in}}%
\pgfpathlineto{\pgfqpoint{5.804754in}{0.642985in}}%
\pgfpathlineto{\pgfqpoint{5.805287in}{0.637053in}}%
\pgfpathlineto{\pgfqpoint{5.805821in}{0.641397in}}%
\pgfpathlineto{\pgfqpoint{5.806355in}{0.638006in}}%
\pgfpathlineto{\pgfqpoint{5.806889in}{0.639951in}}%
\pgfpathlineto{\pgfqpoint{5.807422in}{0.643606in}}%
\pgfpathlineto{\pgfqpoint{5.808490in}{0.636987in}}%
\pgfpathlineto{\pgfqpoint{5.809023in}{0.638787in}}%
\pgfpathlineto{\pgfqpoint{5.809557in}{0.637060in}}%
\pgfpathlineto{\pgfqpoint{5.810091in}{0.637616in}}%
\pgfpathlineto{\pgfqpoint{5.811158in}{0.639197in}}%
\pgfpathlineto{\pgfqpoint{5.811692in}{0.637398in}}%
\pgfpathlineto{\pgfqpoint{5.813293in}{0.642276in}}%
\pgfpathlineto{\pgfqpoint{5.813827in}{0.642656in}}%
\pgfpathlineto{\pgfqpoint{5.814360in}{0.639950in}}%
\pgfpathlineto{\pgfqpoint{5.814894in}{0.643388in}}%
\pgfpathlineto{\pgfqpoint{5.815428in}{0.639677in}}%
\pgfpathlineto{\pgfqpoint{5.815961in}{0.642630in}}%
\pgfpathlineto{\pgfqpoint{5.817029in}{0.638756in}}%
\pgfpathlineto{\pgfqpoint{5.817563in}{0.644010in}}%
\pgfpathlineto{\pgfqpoint{5.818096in}{0.641029in}}%
\pgfpathlineto{\pgfqpoint{5.820231in}{0.639308in}}%
\pgfpathlineto{\pgfqpoint{5.820765in}{0.642227in}}%
\pgfpathlineto{\pgfqpoint{5.821298in}{0.639568in}}%
\pgfpathlineto{\pgfqpoint{5.821832in}{0.636940in}}%
\pgfpathlineto{\pgfqpoint{5.822366in}{0.638363in}}%
\pgfpathlineto{\pgfqpoint{5.822900in}{0.640625in}}%
\pgfpathlineto{\pgfqpoint{5.823433in}{0.637570in}}%
\pgfpathlineto{\pgfqpoint{5.823967in}{0.639745in}}%
\pgfpathlineto{\pgfqpoint{5.824501in}{0.637993in}}%
\pgfpathlineto{\pgfqpoint{5.825034in}{0.649009in}}%
\pgfpathlineto{\pgfqpoint{5.825568in}{0.639314in}}%
\pgfpathlineto{\pgfqpoint{5.826102in}{0.644199in}}%
\pgfpathlineto{\pgfqpoint{5.827703in}{0.637554in}}%
\pgfpathlineto{\pgfqpoint{5.828770in}{0.645606in}}%
\pgfpathlineto{\pgfqpoint{5.829304in}{0.635666in}}%
\pgfpathlineto{\pgfqpoint{5.829838in}{0.638085in}}%
\pgfpathlineto{\pgfqpoint{5.830371in}{0.640543in}}%
\pgfpathlineto{\pgfqpoint{5.830905in}{0.640314in}}%
\pgfpathlineto{\pgfqpoint{5.832506in}{0.636941in}}%
\pgfpathlineto{\pgfqpoint{5.833574in}{0.640297in}}%
\pgfpathlineto{\pgfqpoint{5.835175in}{0.637426in}}%
\pgfpathlineto{\pgfqpoint{5.836776in}{0.646212in}}%
\pgfpathlineto{\pgfqpoint{5.837309in}{0.638344in}}%
\pgfpathlineto{\pgfqpoint{5.838377in}{0.639020in}}%
\pgfpathlineto{\pgfqpoint{5.838911in}{0.644273in}}%
\pgfpathlineto{\pgfqpoint{5.839444in}{0.635489in}}%
\pgfpathlineto{\pgfqpoint{5.839978in}{0.639325in}}%
\pgfpathlineto{\pgfqpoint{5.841045in}{0.636337in}}%
\pgfpathlineto{\pgfqpoint{5.841579in}{0.640189in}}%
\pgfpathlineto{\pgfqpoint{5.842113in}{0.636456in}}%
\pgfpathlineto{\pgfqpoint{5.844248in}{0.644444in}}%
\pgfpathlineto{\pgfqpoint{5.844781in}{0.642888in}}%
\pgfpathlineto{\pgfqpoint{5.845315in}{0.643471in}}%
\pgfpathlineto{\pgfqpoint{5.845849in}{0.639659in}}%
\pgfpathlineto{\pgfqpoint{5.846382in}{0.642466in}}%
\pgfpathlineto{\pgfqpoint{5.846916in}{0.648359in}}%
\pgfpathlineto{\pgfqpoint{5.847984in}{0.637343in}}%
\pgfpathlineto{\pgfqpoint{5.849585in}{0.644541in}}%
\pgfpathlineto{\pgfqpoint{5.851186in}{0.637715in}}%
\pgfpathlineto{\pgfqpoint{5.851719in}{0.639039in}}%
\pgfpathlineto{\pgfqpoint{5.852253in}{0.645195in}}%
\pgfpathlineto{\pgfqpoint{5.852787in}{0.639853in}}%
\pgfpathlineto{\pgfqpoint{5.853321in}{0.644449in}}%
\pgfpathlineto{\pgfqpoint{5.853854in}{0.636832in}}%
\pgfpathlineto{\pgfqpoint{5.854388in}{0.641578in}}%
\pgfpathlineto{\pgfqpoint{5.855455in}{0.639096in}}%
\pgfpathlineto{\pgfqpoint{5.857056in}{0.640388in}}%
\pgfpathlineto{\pgfqpoint{5.857590in}{0.639890in}}%
\pgfpathlineto{\pgfqpoint{5.858124in}{0.642930in}}%
\pgfpathlineto{\pgfqpoint{5.858658in}{0.638313in}}%
\pgfpathlineto{\pgfqpoint{5.859191in}{0.641723in}}%
\pgfpathlineto{\pgfqpoint{5.860259in}{0.642295in}}%
\pgfpathlineto{\pgfqpoint{5.861326in}{0.638744in}}%
\pgfpathlineto{\pgfqpoint{5.862393in}{0.642560in}}%
\pgfpathlineto{\pgfqpoint{5.862927in}{0.637605in}}%
\pgfpathlineto{\pgfqpoint{5.863461in}{0.644464in}}%
\pgfpathlineto{\pgfqpoint{5.863995in}{0.642893in}}%
\pgfpathlineto{\pgfqpoint{5.864528in}{0.638363in}}%
\pgfpathlineto{\pgfqpoint{5.865062in}{0.639114in}}%
\pgfpathlineto{\pgfqpoint{5.865596in}{0.639834in}}%
\pgfpathlineto{\pgfqpoint{5.866129in}{0.636661in}}%
\pgfpathlineto{\pgfqpoint{5.867197in}{0.644360in}}%
\pgfpathlineto{\pgfqpoint{5.868264in}{0.638157in}}%
\pgfpathlineto{\pgfqpoint{5.868798in}{0.638244in}}%
\pgfpathlineto{\pgfqpoint{5.869332in}{0.640911in}}%
\pgfpathlineto{\pgfqpoint{5.869865in}{0.638639in}}%
\pgfpathlineto{\pgfqpoint{5.870399in}{0.637035in}}%
\pgfpathlineto{\pgfqpoint{5.870933in}{0.641833in}}%
\pgfpathlineto{\pgfqpoint{5.871466in}{0.640927in}}%
\pgfpathlineto{\pgfqpoint{5.872000in}{0.639274in}}%
\pgfpathlineto{\pgfqpoint{5.873067in}{0.641678in}}%
\pgfpathlineto{\pgfqpoint{5.873601in}{0.639611in}}%
\pgfpathlineto{\pgfqpoint{5.874669in}{0.642728in}}%
\pgfpathlineto{\pgfqpoint{5.875202in}{0.638443in}}%
\pgfpathlineto{\pgfqpoint{5.875736in}{0.642449in}}%
\pgfpathlineto{\pgfqpoint{5.877337in}{0.636535in}}%
\pgfpathlineto{\pgfqpoint{5.878404in}{0.643437in}}%
\pgfpathlineto{\pgfqpoint{5.878938in}{0.641406in}}%
\pgfpathlineto{\pgfqpoint{5.880539in}{0.637186in}}%
\pgfpathlineto{\pgfqpoint{5.881607in}{0.647235in}}%
\pgfpathlineto{\pgfqpoint{5.882674in}{0.644328in}}%
\pgfpathlineto{\pgfqpoint{5.883208in}{0.641959in}}%
\pgfpathlineto{\pgfqpoint{5.883741in}{0.642361in}}%
\pgfpathlineto{\pgfqpoint{5.884275in}{0.645010in}}%
\pgfpathlineto{\pgfqpoint{5.885343in}{0.638983in}}%
\pgfpathlineto{\pgfqpoint{5.885876in}{0.639635in}}%
\pgfpathlineto{\pgfqpoint{5.886410in}{0.637136in}}%
\pgfpathlineto{\pgfqpoint{5.887477in}{0.647136in}}%
\pgfpathlineto{\pgfqpoint{5.888011in}{0.643168in}}%
\pgfpathlineto{\pgfqpoint{5.888545in}{0.642460in}}%
\pgfpathlineto{\pgfqpoint{5.889612in}{0.635496in}}%
\pgfpathlineto{\pgfqpoint{5.890146in}{0.636995in}}%
\pgfpathlineto{\pgfqpoint{5.891213in}{0.647367in}}%
\pgfpathlineto{\pgfqpoint{5.891747in}{0.637553in}}%
\pgfpathlineto{\pgfqpoint{5.892281in}{0.637734in}}%
\pgfpathlineto{\pgfqpoint{5.893348in}{0.639221in}}%
\pgfpathlineto{\pgfqpoint{5.893882in}{0.642376in}}%
\pgfpathlineto{\pgfqpoint{5.894415in}{0.637072in}}%
\pgfpathlineto{\pgfqpoint{5.894949in}{0.639083in}}%
\pgfpathlineto{\pgfqpoint{5.895483in}{0.637977in}}%
\pgfpathlineto{\pgfqpoint{5.896017in}{0.639105in}}%
\pgfpathlineto{\pgfqpoint{5.896550in}{0.640899in}}%
\pgfpathlineto{\pgfqpoint{5.897084in}{0.637626in}}%
\pgfpathlineto{\pgfqpoint{5.897618in}{0.638692in}}%
\pgfpathlineto{\pgfqpoint{5.899219in}{0.639764in}}%
\pgfpathlineto{\pgfqpoint{5.899752in}{0.637467in}}%
\pgfpathlineto{\pgfqpoint{5.900820in}{0.642867in}}%
\pgfpathlineto{\pgfqpoint{5.901354in}{0.639338in}}%
\pgfpathlineto{\pgfqpoint{5.901887in}{0.646511in}}%
\pgfpathlineto{\pgfqpoint{5.902421in}{0.639452in}}%
\pgfpathlineto{\pgfqpoint{5.902955in}{0.639663in}}%
\pgfpathlineto{\pgfqpoint{5.903488in}{0.641313in}}%
\pgfpathlineto{\pgfqpoint{5.904022in}{0.636735in}}%
\pgfpathlineto{\pgfqpoint{5.904556in}{0.640749in}}%
\pgfpathlineto{\pgfqpoint{5.905089in}{0.644848in}}%
\pgfpathlineto{\pgfqpoint{5.905623in}{0.644694in}}%
\pgfpathlineto{\pgfqpoint{5.907224in}{0.638136in}}%
\pgfpathlineto{\pgfqpoint{5.907758in}{0.639396in}}%
\pgfpathlineto{\pgfqpoint{5.908292in}{0.644482in}}%
\pgfpathlineto{\pgfqpoint{5.908825in}{0.636814in}}%
\pgfpathlineto{\pgfqpoint{5.909359in}{0.645514in}}%
\pgfpathlineto{\pgfqpoint{5.909893in}{0.643531in}}%
\pgfpathlineto{\pgfqpoint{5.910426in}{0.644254in}}%
\pgfpathlineto{\pgfqpoint{5.910960in}{0.643413in}}%
\pgfpathlineto{\pgfqpoint{5.912561in}{0.638390in}}%
\pgfpathlineto{\pgfqpoint{5.913629in}{0.644540in}}%
\pgfpathlineto{\pgfqpoint{5.914162in}{0.639844in}}%
\pgfpathlineto{\pgfqpoint{5.914696in}{0.641083in}}%
\pgfpathlineto{\pgfqpoint{5.915764in}{0.640752in}}%
\pgfpathlineto{\pgfqpoint{5.916297in}{0.637207in}}%
\pgfpathlineto{\pgfqpoint{5.916831in}{0.643482in}}%
\pgfpathlineto{\pgfqpoint{5.917365in}{0.638783in}}%
\pgfpathlineto{\pgfqpoint{5.917898in}{0.636457in}}%
\pgfpathlineto{\pgfqpoint{5.918966in}{0.639935in}}%
\pgfpathlineto{\pgfqpoint{5.919499in}{0.636534in}}%
\pgfpathlineto{\pgfqpoint{5.920033in}{0.637810in}}%
\pgfpathlineto{\pgfqpoint{5.920567in}{0.638907in}}%
\pgfpathlineto{\pgfqpoint{5.921101in}{0.636825in}}%
\pgfpathlineto{\pgfqpoint{5.922702in}{0.642727in}}%
\pgfpathlineto{\pgfqpoint{5.923235in}{0.642013in}}%
\pgfpathlineto{\pgfqpoint{5.923769in}{0.636775in}}%
\pgfpathlineto{\pgfqpoint{5.924303in}{0.642436in}}%
\pgfpathlineto{\pgfqpoint{5.925370in}{0.639937in}}%
\pgfpathlineto{\pgfqpoint{5.925904in}{0.641895in}}%
\pgfpathlineto{\pgfqpoint{5.926438in}{0.640726in}}%
\pgfpathlineto{\pgfqpoint{5.926971in}{0.641538in}}%
\pgfpathlineto{\pgfqpoint{5.927505in}{0.645933in}}%
\pgfpathlineto{\pgfqpoint{5.928039in}{0.639806in}}%
\pgfpathlineto{\pgfqpoint{5.928572in}{0.648143in}}%
\pgfpathlineto{\pgfqpoint{5.929106in}{0.640237in}}%
\pgfpathlineto{\pgfqpoint{5.929640in}{0.642196in}}%
\pgfpathlineto{\pgfqpoint{5.930707in}{0.637435in}}%
\pgfpathlineto{\pgfqpoint{5.931241in}{0.643061in}}%
\pgfpathlineto{\pgfqpoint{5.931775in}{0.637660in}}%
\pgfpathlineto{\pgfqpoint{5.932308in}{0.637876in}}%
\pgfpathlineto{\pgfqpoint{5.933376in}{0.649326in}}%
\pgfpathlineto{\pgfqpoint{5.933909in}{0.639859in}}%
\pgfpathlineto{\pgfqpoint{5.934977in}{0.640592in}}%
\pgfpathlineto{\pgfqpoint{5.936044in}{0.643241in}}%
\pgfpathlineto{\pgfqpoint{5.936578in}{0.651730in}}%
\pgfpathlineto{\pgfqpoint{5.937112in}{0.640967in}}%
\pgfpathlineto{\pgfqpoint{5.937645in}{0.645333in}}%
\pgfpathlineto{\pgfqpoint{5.938713in}{0.638893in}}%
\pgfpathlineto{\pgfqpoint{5.939246in}{0.647779in}}%
\pgfpathlineto{\pgfqpoint{5.939780in}{0.639997in}}%
\pgfpathlineto{\pgfqpoint{5.940314in}{0.645952in}}%
\pgfpathlineto{\pgfqpoint{5.940847in}{0.641850in}}%
\pgfpathlineto{\pgfqpoint{5.941381in}{0.643299in}}%
\pgfpathlineto{\pgfqpoint{5.941915in}{0.637334in}}%
\pgfpathlineto{\pgfqpoint{5.942449in}{0.638600in}}%
\pgfpathlineto{\pgfqpoint{5.943516in}{0.638921in}}%
\pgfpathlineto{\pgfqpoint{5.944050in}{0.642220in}}%
\pgfpathlineto{\pgfqpoint{5.944583in}{0.639939in}}%
\pgfpathlineto{\pgfqpoint{5.945117in}{0.639043in}}%
\pgfpathlineto{\pgfqpoint{5.945651in}{0.639496in}}%
\pgfpathlineto{\pgfqpoint{5.946184in}{0.650285in}}%
\pgfpathlineto{\pgfqpoint{5.946718in}{0.638848in}}%
\pgfpathlineto{\pgfqpoint{5.947252in}{0.638836in}}%
\pgfpathlineto{\pgfqpoint{5.947786in}{0.640357in}}%
\pgfpathlineto{\pgfqpoint{5.948319in}{0.637415in}}%
\pgfpathlineto{\pgfqpoint{5.948853in}{0.643914in}}%
\pgfpathlineto{\pgfqpoint{5.949387in}{0.635668in}}%
\pgfpathlineto{\pgfqpoint{5.949920in}{0.642035in}}%
\pgfpathlineto{\pgfqpoint{5.952055in}{0.636690in}}%
\pgfpathlineto{\pgfqpoint{5.953123in}{0.639957in}}%
\pgfpathlineto{\pgfqpoint{5.954190in}{0.637283in}}%
\pgfpathlineto{\pgfqpoint{5.955257in}{0.642720in}}%
\pgfpathlineto{\pgfqpoint{5.956325in}{0.638084in}}%
\pgfpathlineto{\pgfqpoint{5.956858in}{0.649184in}}%
\pgfpathlineto{\pgfqpoint{5.957392in}{0.642546in}}%
\pgfpathlineto{\pgfqpoint{5.957926in}{0.635232in}}%
\pgfpathlineto{\pgfqpoint{5.958460in}{0.637116in}}%
\pgfpathlineto{\pgfqpoint{5.958993in}{0.636723in}}%
\pgfpathlineto{\pgfqpoint{5.960061in}{0.641319in}}%
\pgfpathlineto{\pgfqpoint{5.960594in}{0.638346in}}%
\pgfpathlineto{\pgfqpoint{5.961128in}{0.644354in}}%
\pgfpathlineto{\pgfqpoint{5.961662in}{0.640096in}}%
\pgfpathlineto{\pgfqpoint{5.962195in}{0.640532in}}%
\pgfpathlineto{\pgfqpoint{5.963263in}{0.637715in}}%
\pgfpathlineto{\pgfqpoint{5.963797in}{0.641771in}}%
\pgfpathlineto{\pgfqpoint{5.964330in}{0.639121in}}%
\pgfpathlineto{\pgfqpoint{5.964864in}{0.640147in}}%
\pgfpathlineto{\pgfqpoint{5.965398in}{0.638109in}}%
\pgfpathlineto{\pgfqpoint{5.966999in}{0.641997in}}%
\pgfpathlineto{\pgfqpoint{5.968066in}{0.637427in}}%
\pgfpathlineto{\pgfqpoint{5.968600in}{0.638836in}}%
\pgfpathlineto{\pgfqpoint{5.969134in}{0.642693in}}%
\pgfpathlineto{\pgfqpoint{5.969667in}{0.642521in}}%
\pgfpathlineto{\pgfqpoint{5.970735in}{0.640017in}}%
\pgfpathlineto{\pgfqpoint{5.971268in}{0.644492in}}%
\pgfpathlineto{\pgfqpoint{5.971802in}{0.640048in}}%
\pgfpathlineto{\pgfqpoint{5.972869in}{0.638298in}}%
\pgfpathlineto{\pgfqpoint{5.973937in}{0.642361in}}%
\pgfpathlineto{\pgfqpoint{5.975538in}{0.637672in}}%
\pgfpathlineto{\pgfqpoint{5.976072in}{0.639349in}}%
\pgfpathlineto{\pgfqpoint{5.976605in}{0.637153in}}%
\pgfpathlineto{\pgfqpoint{5.977139in}{0.639028in}}%
\pgfpathlineto{\pgfqpoint{5.977673in}{0.641873in}}%
\pgfpathlineto{\pgfqpoint{5.979274in}{0.636604in}}%
\pgfpathlineto{\pgfqpoint{5.979808in}{0.637345in}}%
\pgfpathlineto{\pgfqpoint{5.981409in}{0.640870in}}%
\pgfpathlineto{\pgfqpoint{5.981942in}{0.638305in}}%
\pgfpathlineto{\pgfqpoint{5.982476in}{0.643744in}}%
\pgfpathlineto{\pgfqpoint{5.983010in}{0.642697in}}%
\pgfpathlineto{\pgfqpoint{5.984611in}{0.636951in}}%
\pgfpathlineto{\pgfqpoint{5.985678in}{0.643092in}}%
\pgfpathlineto{\pgfqpoint{5.986212in}{0.639712in}}%
\pgfpathlineto{\pgfqpoint{5.987279in}{0.637273in}}%
\pgfpathlineto{\pgfqpoint{5.987813in}{0.648025in}}%
\pgfpathlineto{\pgfqpoint{5.988347in}{0.643422in}}%
\pgfpathlineto{\pgfqpoint{5.988881in}{0.636413in}}%
\pgfpathlineto{\pgfqpoint{5.989414in}{0.639734in}}%
\pgfpathlineto{\pgfqpoint{5.989948in}{0.639488in}}%
\pgfpathlineto{\pgfqpoint{5.990482in}{0.637806in}}%
\pgfpathlineto{\pgfqpoint{5.992616in}{0.642974in}}%
\pgfpathlineto{\pgfqpoint{5.993684in}{0.639112in}}%
\pgfpathlineto{\pgfqpoint{5.994218in}{0.642654in}}%
\pgfpathlineto{\pgfqpoint{5.994751in}{0.636010in}}%
\pgfpathlineto{\pgfqpoint{5.995285in}{0.637696in}}%
\pgfpathlineto{\pgfqpoint{5.996886in}{0.641157in}}%
\pgfpathlineto{\pgfqpoint{5.997953in}{0.637295in}}%
\pgfpathlineto{\pgfqpoint{5.999555in}{0.639623in}}%
\pgfpathlineto{\pgfqpoint{6.000088in}{0.638056in}}%
\pgfpathlineto{\pgfqpoint{6.001156in}{0.644045in}}%
\pgfpathlineto{\pgfqpoint{6.001689in}{0.642710in}}%
\pgfpathlineto{\pgfqpoint{6.002223in}{0.642799in}}%
\pgfpathlineto{\pgfqpoint{6.002757in}{0.638371in}}%
\pgfpathlineto{\pgfqpoint{6.003290in}{0.639963in}}%
\pgfpathlineto{\pgfqpoint{6.003824in}{0.639509in}}%
\pgfpathlineto{\pgfqpoint{6.004358in}{0.641586in}}%
\pgfpathlineto{\pgfqpoint{6.004892in}{0.636092in}}%
\pgfpathlineto{\pgfqpoint{6.005425in}{0.636715in}}%
\pgfpathlineto{\pgfqpoint{6.008627in}{0.640605in}}%
\pgfpathlineto{\pgfqpoint{6.009161in}{0.639515in}}%
\pgfpathlineto{\pgfqpoint{6.009695in}{0.640698in}}%
\pgfpathlineto{\pgfqpoint{6.010762in}{0.639730in}}%
\pgfpathlineto{\pgfqpoint{6.011296in}{0.638152in}}%
\pgfpathlineto{\pgfqpoint{6.011830in}{0.638913in}}%
\pgfpathlineto{\pgfqpoint{6.012897in}{0.639918in}}%
\pgfpathlineto{\pgfqpoint{6.013964in}{0.641684in}}%
\pgfpathlineto{\pgfqpoint{6.014498in}{0.637308in}}%
\pgfpathlineto{\pgfqpoint{6.015032in}{0.640715in}}%
\pgfpathlineto{\pgfqpoint{6.017167in}{0.638064in}}%
\pgfpathlineto{\pgfqpoint{6.018234in}{0.642850in}}%
\pgfpathlineto{\pgfqpoint{6.018768in}{0.640556in}}%
\pgfpathlineto{\pgfqpoint{6.019835in}{0.637539in}}%
\pgfpathlineto{\pgfqpoint{6.020903in}{0.641087in}}%
\pgfpathlineto{\pgfqpoint{6.021436in}{0.636488in}}%
\pgfpathlineto{\pgfqpoint{6.021970in}{0.636633in}}%
\pgfpathlineto{\pgfqpoint{6.023037in}{0.642165in}}%
\pgfpathlineto{\pgfqpoint{6.023571in}{0.635859in}}%
\pgfpathlineto{\pgfqpoint{6.024105in}{0.642399in}}%
\pgfpathlineto{\pgfqpoint{6.025706in}{0.636342in}}%
\pgfpathlineto{\pgfqpoint{6.026773in}{0.643081in}}%
\pgfpathlineto{\pgfqpoint{6.027307in}{0.637050in}}%
\pgfpathlineto{\pgfqpoint{6.027841in}{0.637174in}}%
\pgfpathlineto{\pgfqpoint{6.029442in}{0.642549in}}%
\pgfpathlineto{\pgfqpoint{6.029975in}{0.636361in}}%
\pgfpathlineto{\pgfqpoint{6.030509in}{0.640340in}}%
\pgfpathlineto{\pgfqpoint{6.031043in}{0.637453in}}%
\pgfpathlineto{\pgfqpoint{6.031577in}{0.644462in}}%
\pgfpathlineto{\pgfqpoint{6.032110in}{0.639431in}}%
\pgfpathlineto{\pgfqpoint{6.033711in}{0.636266in}}%
\pgfpathlineto{\pgfqpoint{6.034245in}{0.644975in}}%
\pgfpathlineto{\pgfqpoint{6.034779in}{0.641972in}}%
\pgfpathlineto{\pgfqpoint{6.035312in}{0.636597in}}%
\pgfpathlineto{\pgfqpoint{6.036380in}{0.636928in}}%
\pgfpathlineto{\pgfqpoint{6.037447in}{0.638051in}}%
\pgfpathlineto{\pgfqpoint{6.037981in}{0.639822in}}%
\pgfpathlineto{\pgfqpoint{6.039582in}{0.636059in}}%
\pgfpathlineto{\pgfqpoint{6.040116in}{0.642054in}}%
\pgfpathlineto{\pgfqpoint{6.040649in}{0.640055in}}%
\pgfpathlineto{\pgfqpoint{6.042251in}{0.635923in}}%
\pgfpathlineto{\pgfqpoint{6.043852in}{0.641524in}}%
\pgfpathlineto{\pgfqpoint{6.045453in}{0.638044in}}%
\pgfpathlineto{\pgfqpoint{6.045987in}{0.637637in}}%
\pgfpathlineto{\pgfqpoint{6.046520in}{0.639759in}}%
\pgfpathlineto{\pgfqpoint{6.047054in}{0.637906in}}%
\pgfpathlineto{\pgfqpoint{6.047588in}{0.637874in}}%
\pgfpathlineto{\pgfqpoint{6.048121in}{0.642581in}}%
\pgfpathlineto{\pgfqpoint{6.048655in}{0.639389in}}%
\pgfpathlineto{\pgfqpoint{6.050256in}{0.641474in}}%
\pgfpathlineto{\pgfqpoint{6.051324in}{0.635538in}}%
\pgfpathlineto{\pgfqpoint{6.051857in}{0.639368in}}%
\pgfpathlineto{\pgfqpoint{6.052391in}{0.637563in}}%
\pgfpathlineto{\pgfqpoint{6.052925in}{0.646208in}}%
\pgfpathlineto{\pgfqpoint{6.053458in}{0.641942in}}%
\pgfpathlineto{\pgfqpoint{6.053992in}{0.637262in}}%
\pgfpathlineto{\pgfqpoint{6.054526in}{0.639956in}}%
\pgfpathlineto{\pgfqpoint{6.055593in}{0.637331in}}%
\pgfpathlineto{\pgfqpoint{6.056127in}{0.638498in}}%
\pgfpathlineto{\pgfqpoint{6.056661in}{0.636910in}}%
\pgfpathlineto{\pgfqpoint{6.057194in}{0.639711in}}%
\pgfpathlineto{\pgfqpoint{6.057728in}{0.638107in}}%
\pgfpathlineto{\pgfqpoint{6.058262in}{0.637520in}}%
\pgfpathlineto{\pgfqpoint{6.058795in}{0.640722in}}%
\pgfpathlineto{\pgfqpoint{6.059863in}{0.635743in}}%
\pgfpathlineto{\pgfqpoint{6.061464in}{0.641808in}}%
\pgfpathlineto{\pgfqpoint{6.061998in}{0.637320in}}%
\pgfpathlineto{\pgfqpoint{6.062531in}{0.639590in}}%
\pgfpathlineto{\pgfqpoint{6.064132in}{0.643219in}}%
\pgfpathlineto{\pgfqpoint{6.064666in}{0.639917in}}%
\pgfpathlineto{\pgfqpoint{6.065200in}{0.643889in}}%
\pgfpathlineto{\pgfqpoint{6.065733in}{0.642734in}}%
\pgfpathlineto{\pgfqpoint{6.066267in}{0.637600in}}%
\pgfpathlineto{\pgfqpoint{6.066801in}{0.638372in}}%
\pgfpathlineto{\pgfqpoint{6.067335in}{0.643885in}}%
\pgfpathlineto{\pgfqpoint{6.067868in}{0.641156in}}%
\pgfpathlineto{\pgfqpoint{6.068402in}{0.641899in}}%
\pgfpathlineto{\pgfqpoint{6.069469in}{0.636358in}}%
\pgfpathlineto{\pgfqpoint{6.070003in}{0.638004in}}%
\pgfpathlineto{\pgfqpoint{6.071604in}{0.641544in}}%
\pgfpathlineto{\pgfqpoint{6.072672in}{0.637379in}}%
\pgfpathlineto{\pgfqpoint{6.073205in}{0.640416in}}%
\pgfpathlineto{\pgfqpoint{6.073739in}{0.637990in}}%
\pgfpathlineto{\pgfqpoint{6.074273in}{0.638838in}}%
\pgfpathlineto{\pgfqpoint{6.074806in}{0.638103in}}%
\pgfpathlineto{\pgfqpoint{6.075340in}{0.636565in}}%
\pgfpathlineto{\pgfqpoint{6.076407in}{0.640769in}}%
\pgfpathlineto{\pgfqpoint{6.076941in}{0.640259in}}%
\pgfpathlineto{\pgfqpoint{6.077475in}{0.640418in}}%
\pgfpathlineto{\pgfqpoint{6.078009in}{0.636175in}}%
\pgfpathlineto{\pgfqpoint{6.078542in}{0.638264in}}%
\pgfpathlineto{\pgfqpoint{6.079076in}{0.638451in}}%
\pgfpathlineto{\pgfqpoint{6.080677in}{0.636144in}}%
\pgfpathlineto{\pgfqpoint{6.081744in}{0.640760in}}%
\pgfpathlineto{\pgfqpoint{6.082278in}{0.637579in}}%
\pgfpathlineto{\pgfqpoint{6.083346in}{0.642125in}}%
\pgfpathlineto{\pgfqpoint{6.084947in}{0.637694in}}%
\pgfpathlineto{\pgfqpoint{6.086014in}{0.645067in}}%
\pgfpathlineto{\pgfqpoint{6.087081in}{0.636889in}}%
\pgfpathlineto{\pgfqpoint{6.087615in}{0.642380in}}%
\pgfpathlineto{\pgfqpoint{6.088149in}{0.638576in}}%
\pgfpathlineto{\pgfqpoint{6.088683in}{0.637827in}}%
\pgfpathlineto{\pgfqpoint{6.090284in}{0.640078in}}%
\pgfpathlineto{\pgfqpoint{6.090817in}{0.640869in}}%
\pgfpathlineto{\pgfqpoint{6.091885in}{0.636206in}}%
\pgfpathlineto{\pgfqpoint{6.092418in}{0.637511in}}%
\pgfpathlineto{\pgfqpoint{6.092952in}{0.637099in}}%
\pgfpathlineto{\pgfqpoint{6.093486in}{0.642526in}}%
\pgfpathlineto{\pgfqpoint{6.094020in}{0.637026in}}%
\pgfpathlineto{\pgfqpoint{6.094553in}{0.636057in}}%
\pgfpathlineto{\pgfqpoint{6.095621in}{0.644601in}}%
\pgfpathlineto{\pgfqpoint{6.096154in}{0.636713in}}%
\pgfpathlineto{\pgfqpoint{6.096688in}{0.638189in}}%
\pgfpathlineto{\pgfqpoint{6.097222in}{0.639058in}}%
\pgfpathlineto{\pgfqpoint{6.097755in}{0.637784in}}%
\pgfpathlineto{\pgfqpoint{6.098823in}{0.645607in}}%
\pgfpathlineto{\pgfqpoint{6.100424in}{0.638535in}}%
\pgfpathlineto{\pgfqpoint{6.100958in}{0.644506in}}%
\pgfpathlineto{\pgfqpoint{6.102559in}{0.637336in}}%
\pgfpathlineto{\pgfqpoint{6.103092in}{0.636463in}}%
\pgfpathlineto{\pgfqpoint{6.103626in}{0.643034in}}%
\pgfpathlineto{\pgfqpoint{6.104160in}{0.640059in}}%
\pgfpathlineto{\pgfqpoint{6.105227in}{0.642772in}}%
\pgfpathlineto{\pgfqpoint{6.105761in}{0.647370in}}%
\pgfpathlineto{\pgfqpoint{6.106828in}{0.638224in}}%
\pgfpathlineto{\pgfqpoint{6.107362in}{0.646543in}}%
\pgfpathlineto{\pgfqpoint{6.107896in}{0.641254in}}%
\pgfpathlineto{\pgfqpoint{6.108429in}{0.640438in}}%
\pgfpathlineto{\pgfqpoint{6.108963in}{0.637131in}}%
\pgfpathlineto{\pgfqpoint{6.109497in}{0.646277in}}%
\pgfpathlineto{\pgfqpoint{6.110031in}{0.640997in}}%
\pgfpathlineto{\pgfqpoint{6.110564in}{0.636494in}}%
\pgfpathlineto{\pgfqpoint{6.111098in}{0.639197in}}%
\pgfpathlineto{\pgfqpoint{6.112165in}{0.643126in}}%
\pgfpathlineto{\pgfqpoint{6.112699in}{0.637604in}}%
\pgfpathlineto{\pgfqpoint{6.113233in}{0.639476in}}%
\pgfpathlineto{\pgfqpoint{6.115368in}{0.640897in}}%
\pgfpathlineto{\pgfqpoint{6.115901in}{0.640342in}}%
\pgfpathlineto{\pgfqpoint{6.116435in}{0.636876in}}%
\pgfpathlineto{\pgfqpoint{6.116969in}{0.639377in}}%
\pgfpathlineto{\pgfqpoint{6.117502in}{0.640068in}}%
\pgfpathlineto{\pgfqpoint{6.118036in}{0.638223in}}%
\pgfpathlineto{\pgfqpoint{6.118570in}{0.641425in}}%
\pgfpathlineto{\pgfqpoint{6.119104in}{0.640633in}}%
\pgfpathlineto{\pgfqpoint{6.119637in}{0.636006in}}%
\pgfpathlineto{\pgfqpoint{6.120171in}{0.638068in}}%
\pgfpathlineto{\pgfqpoint{6.121772in}{0.642210in}}%
\pgfpathlineto{\pgfqpoint{6.122306in}{0.641373in}}%
\pgfpathlineto{\pgfqpoint{6.124441in}{0.636266in}}%
\pgfpathlineto{\pgfqpoint{6.125508in}{0.645096in}}%
\pgfpathlineto{\pgfqpoint{6.126042in}{0.640229in}}%
\pgfpathlineto{\pgfqpoint{6.126575in}{0.635498in}}%
\pgfpathlineto{\pgfqpoint{6.127109in}{0.637206in}}%
\pgfpathlineto{\pgfqpoint{6.127643in}{0.637269in}}%
\pgfpathlineto{\pgfqpoint{6.128710in}{0.640683in}}%
\pgfpathlineto{\pgfqpoint{6.129244in}{0.637820in}}%
\pgfpathlineto{\pgfqpoint{6.129778in}{0.640466in}}%
\pgfpathlineto{\pgfqpoint{6.131379in}{0.638000in}}%
\pgfpathlineto{\pgfqpoint{6.131912in}{0.640887in}}%
\pgfpathlineto{\pgfqpoint{6.132446in}{0.637874in}}%
\pgfpathlineto{\pgfqpoint{6.132980in}{0.640115in}}%
\pgfpathlineto{\pgfqpoint{6.133513in}{0.639353in}}%
\pgfpathlineto{\pgfqpoint{6.134047in}{0.638178in}}%
\pgfpathlineto{\pgfqpoint{6.134581in}{0.639113in}}%
\pgfpathlineto{\pgfqpoint{6.135648in}{0.639181in}}%
\pgfpathlineto{\pgfqpoint{6.136182in}{0.641375in}}%
\pgfpathlineto{\pgfqpoint{6.136716in}{0.639465in}}%
\pgfpathlineto{\pgfqpoint{6.137783in}{0.642201in}}%
\pgfpathlineto{\pgfqpoint{6.138317in}{0.636699in}}%
\pgfpathlineto{\pgfqpoint{6.138850in}{0.640348in}}%
\pgfpathlineto{\pgfqpoint{6.140452in}{0.637108in}}%
\pgfpathlineto{\pgfqpoint{6.141519in}{0.639438in}}%
\pgfpathlineto{\pgfqpoint{6.142053in}{0.636049in}}%
\pgfpathlineto{\pgfqpoint{6.142586in}{0.641121in}}%
\pgfpathlineto{\pgfqpoint{6.143120in}{0.637375in}}%
\pgfpathlineto{\pgfqpoint{6.143654in}{0.637218in}}%
\pgfpathlineto{\pgfqpoint{6.144187in}{0.641624in}}%
\pgfpathlineto{\pgfqpoint{6.144721in}{0.637087in}}%
\pgfpathlineto{\pgfqpoint{6.147390in}{0.640785in}}%
\pgfpathlineto{\pgfqpoint{6.148991in}{0.637401in}}%
\pgfpathlineto{\pgfqpoint{6.150058in}{0.645634in}}%
\pgfpathlineto{\pgfqpoint{6.151659in}{0.637764in}}%
\pgfpathlineto{\pgfqpoint{6.152193in}{0.637592in}}%
\pgfpathlineto{\pgfqpoint{6.152727in}{0.640548in}}%
\pgfpathlineto{\pgfqpoint{6.153260in}{0.639187in}}%
\pgfpathlineto{\pgfqpoint{6.154328in}{0.637962in}}%
\pgfpathlineto{\pgfqpoint{6.155929in}{0.639928in}}%
\pgfpathlineto{\pgfqpoint{6.156222in}{0.638653in}}%
\pgfpathmoveto{\pgfqpoint{6.156222in}{0.635242in}}%
\pgfpathlineto{\pgfqpoint{0.924300in}{0.635246in}}%
\pgfpathmoveto{\pgfqpoint{0.924300in}{0.638653in}}%
\pgfpathlineto{\pgfqpoint{0.925661in}{0.639782in}}%
\pgfpathlineto{\pgfqpoint{0.926195in}{0.637962in}}%
\pgfpathlineto{\pgfqpoint{0.926728in}{0.638456in}}%
\pgfpathlineto{\pgfqpoint{0.927796in}{0.640548in}}%
\pgfpathlineto{\pgfqpoint{0.928330in}{0.637592in}}%
\pgfpathlineto{\pgfqpoint{0.928863in}{0.637764in}}%
\pgfpathlineto{\pgfqpoint{0.930464in}{0.645634in}}%
\pgfpathlineto{\pgfqpoint{0.930998in}{0.641620in}}%
\pgfpathlineto{\pgfqpoint{0.931532in}{0.637401in}}%
\pgfpathlineto{\pgfqpoint{0.932065in}{0.637756in}}%
\pgfpathlineto{\pgfqpoint{0.933133in}{0.640785in}}%
\pgfpathlineto{\pgfqpoint{0.934200in}{0.639511in}}%
\pgfpathlineto{\pgfqpoint{0.935801in}{0.637087in}}%
\pgfpathlineto{\pgfqpoint{0.936335in}{0.641624in}}%
\pgfpathlineto{\pgfqpoint{0.936869in}{0.637218in}}%
\pgfpathlineto{\pgfqpoint{0.937402in}{0.637375in}}%
\pgfpathlineto{\pgfqpoint{0.937936in}{0.641121in}}%
\pgfpathlineto{\pgfqpoint{0.938470in}{0.636049in}}%
\pgfpathlineto{\pgfqpoint{0.939004in}{0.639438in}}%
\pgfpathlineto{\pgfqpoint{0.940605in}{0.637233in}}%
\pgfpathlineto{\pgfqpoint{0.941138in}{0.638589in}}%
\pgfpathlineto{\pgfqpoint{0.941672in}{0.640348in}}%
\pgfpathlineto{\pgfqpoint{0.942206in}{0.636699in}}%
\pgfpathlineto{\pgfqpoint{0.942739in}{0.642201in}}%
\pgfpathlineto{\pgfqpoint{0.943273in}{0.640713in}}%
\pgfpathlineto{\pgfqpoint{0.943807in}{0.639465in}}%
\pgfpathlineto{\pgfqpoint{0.944341in}{0.641375in}}%
\pgfpathlineto{\pgfqpoint{0.944874in}{0.639181in}}%
\pgfpathlineto{\pgfqpoint{0.947009in}{0.639353in}}%
\pgfpathlineto{\pgfqpoint{0.947543in}{0.640115in}}%
\pgfpathlineto{\pgfqpoint{0.948077in}{0.637874in}}%
\pgfpathlineto{\pgfqpoint{0.948610in}{0.640887in}}%
\pgfpathlineto{\pgfqpoint{0.949678in}{0.640655in}}%
\pgfpathlineto{\pgfqpoint{0.951279in}{0.637820in}}%
\pgfpathlineto{\pgfqpoint{0.951812in}{0.640683in}}%
\pgfpathlineto{\pgfqpoint{0.952346in}{0.640263in}}%
\pgfpathlineto{\pgfqpoint{0.953947in}{0.635498in}}%
\pgfpathlineto{\pgfqpoint{0.955015in}{0.645096in}}%
\pgfpathlineto{\pgfqpoint{0.956082in}{0.636266in}}%
\pgfpathlineto{\pgfqpoint{0.956616in}{0.639161in}}%
\pgfpathlineto{\pgfqpoint{0.957683in}{0.639902in}}%
\pgfpathlineto{\pgfqpoint{0.958751in}{0.642210in}}%
\pgfpathlineto{\pgfqpoint{0.959284in}{0.641593in}}%
\pgfpathlineto{\pgfqpoint{0.960885in}{0.636006in}}%
\pgfpathlineto{\pgfqpoint{0.961953in}{0.641425in}}%
\pgfpathlineto{\pgfqpoint{0.962486in}{0.638223in}}%
\pgfpathlineto{\pgfqpoint{0.963020in}{0.640068in}}%
\pgfpathlineto{\pgfqpoint{0.964088in}{0.636876in}}%
\pgfpathlineto{\pgfqpoint{0.965155in}{0.640897in}}%
\pgfpathlineto{\pgfqpoint{0.966222in}{0.639513in}}%
\pgfpathlineto{\pgfqpoint{0.966756in}{0.640075in}}%
\pgfpathlineto{\pgfqpoint{0.967823in}{0.637604in}}%
\pgfpathlineto{\pgfqpoint{0.968357in}{0.643126in}}%
\pgfpathlineto{\pgfqpoint{0.968891in}{0.640361in}}%
\pgfpathlineto{\pgfqpoint{0.969958in}{0.636494in}}%
\pgfpathlineto{\pgfqpoint{0.971026in}{0.646277in}}%
\pgfpathlineto{\pgfqpoint{0.971559in}{0.637131in}}%
\pgfpathlineto{\pgfqpoint{0.972093in}{0.640438in}}%
\pgfpathlineto{\pgfqpoint{0.972627in}{0.641254in}}%
\pgfpathlineto{\pgfqpoint{0.973160in}{0.646543in}}%
\pgfpathlineto{\pgfqpoint{0.973694in}{0.638224in}}%
\pgfpathlineto{\pgfqpoint{0.974228in}{0.641795in}}%
\pgfpathlineto{\pgfqpoint{0.974762in}{0.647370in}}%
\pgfpathlineto{\pgfqpoint{0.975295in}{0.642772in}}%
\pgfpathlineto{\pgfqpoint{0.976363in}{0.640059in}}%
\pgfpathlineto{\pgfqpoint{0.976896in}{0.643034in}}%
\pgfpathlineto{\pgfqpoint{0.977430in}{0.636463in}}%
\pgfpathlineto{\pgfqpoint{0.977964in}{0.637336in}}%
\pgfpathlineto{\pgfqpoint{0.979565in}{0.644506in}}%
\pgfpathlineto{\pgfqpoint{0.980099in}{0.638535in}}%
\pgfpathlineto{\pgfqpoint{0.980632in}{0.640483in}}%
\pgfpathlineto{\pgfqpoint{0.981166in}{0.640147in}}%
\pgfpathlineto{\pgfqpoint{0.981700in}{0.645607in}}%
\pgfpathlineto{\pgfqpoint{0.982233in}{0.640741in}}%
\pgfpathlineto{\pgfqpoint{0.982767in}{0.637784in}}%
\pgfpathlineto{\pgfqpoint{0.983301in}{0.639058in}}%
\pgfpathlineto{\pgfqpoint{0.984368in}{0.636713in}}%
\pgfpathlineto{\pgfqpoint{0.984902in}{0.644601in}}%
\pgfpathlineto{\pgfqpoint{0.985436in}{0.638778in}}%
\pgfpathlineto{\pgfqpoint{0.985969in}{0.636057in}}%
\pgfpathlineto{\pgfqpoint{0.986503in}{0.637026in}}%
\pgfpathlineto{\pgfqpoint{0.987037in}{0.642526in}}%
\pgfpathlineto{\pgfqpoint{0.987570in}{0.637099in}}%
\pgfpathlineto{\pgfqpoint{0.988104in}{0.637511in}}%
\pgfpathlineto{\pgfqpoint{0.988638in}{0.636206in}}%
\pgfpathlineto{\pgfqpoint{0.989705in}{0.640869in}}%
\pgfpathlineto{\pgfqpoint{0.990239in}{0.640078in}}%
\pgfpathlineto{\pgfqpoint{0.991840in}{0.637827in}}%
\pgfpathlineto{\pgfqpoint{0.992374in}{0.638576in}}%
\pgfpathlineto{\pgfqpoint{0.992907in}{0.642380in}}%
\pgfpathlineto{\pgfqpoint{0.993441in}{0.636889in}}%
\pgfpathlineto{\pgfqpoint{0.993975in}{0.639004in}}%
\pgfpathlineto{\pgfqpoint{0.994508in}{0.645067in}}%
\pgfpathlineto{\pgfqpoint{0.995042in}{0.641703in}}%
\pgfpathlineto{\pgfqpoint{0.996643in}{0.637684in}}%
\pgfpathlineto{\pgfqpoint{0.997177in}{0.642125in}}%
\pgfpathlineto{\pgfqpoint{0.997711in}{0.641428in}}%
\pgfpathlineto{\pgfqpoint{0.998244in}{0.637579in}}%
\pgfpathlineto{\pgfqpoint{0.998778in}{0.640760in}}%
\pgfpathlineto{\pgfqpoint{0.999312in}{0.640421in}}%
\pgfpathlineto{\pgfqpoint{1.000913in}{0.636070in}}%
\pgfpathlineto{\pgfqpoint{1.001447in}{0.638451in}}%
\pgfpathlineto{\pgfqpoint{1.001980in}{0.638264in}}%
\pgfpathlineto{\pgfqpoint{1.002514in}{0.636175in}}%
\pgfpathlineto{\pgfqpoint{1.004115in}{0.640769in}}%
\pgfpathlineto{\pgfqpoint{1.004649in}{0.639966in}}%
\pgfpathlineto{\pgfqpoint{1.005182in}{0.636565in}}%
\pgfpathlineto{\pgfqpoint{1.005716in}{0.638103in}}%
\pgfpathlineto{\pgfqpoint{1.007317in}{0.640416in}}%
\pgfpathlineto{\pgfqpoint{1.007851in}{0.637379in}}%
\pgfpathlineto{\pgfqpoint{1.008385in}{0.638696in}}%
\pgfpathlineto{\pgfqpoint{1.008918in}{0.641544in}}%
\pgfpathlineto{\pgfqpoint{1.009452in}{0.640824in}}%
\pgfpathlineto{\pgfqpoint{1.009986in}{0.640723in}}%
\pgfpathlineto{\pgfqpoint{1.011053in}{0.636358in}}%
\pgfpathlineto{\pgfqpoint{1.013188in}{0.643885in}}%
\pgfpathlineto{\pgfqpoint{1.014255in}{0.637600in}}%
\pgfpathlineto{\pgfqpoint{1.015323in}{0.643889in}}%
\pgfpathlineto{\pgfqpoint{1.015857in}{0.639917in}}%
\pgfpathlineto{\pgfqpoint{1.016390in}{0.643219in}}%
\pgfpathlineto{\pgfqpoint{1.017991in}{0.639590in}}%
\pgfpathlineto{\pgfqpoint{1.018525in}{0.637320in}}%
\pgfpathlineto{\pgfqpoint{1.019059in}{0.641808in}}%
\pgfpathlineto{\pgfqpoint{1.019592in}{0.640779in}}%
\pgfpathlineto{\pgfqpoint{1.020660in}{0.635743in}}%
\pgfpathlineto{\pgfqpoint{1.021194in}{0.636100in}}%
\pgfpathlineto{\pgfqpoint{1.021727in}{0.640722in}}%
\pgfpathlineto{\pgfqpoint{1.022261in}{0.637520in}}%
\pgfpathlineto{\pgfqpoint{1.023328in}{0.639711in}}%
\pgfpathlineto{\pgfqpoint{1.023862in}{0.636910in}}%
\pgfpathlineto{\pgfqpoint{1.024396in}{0.638498in}}%
\pgfpathlineto{\pgfqpoint{1.024929in}{0.637331in}}%
\pgfpathlineto{\pgfqpoint{1.025463in}{0.637953in}}%
\pgfpathlineto{\pgfqpoint{1.025997in}{0.639956in}}%
\pgfpathlineto{\pgfqpoint{1.026531in}{0.637262in}}%
\pgfpathlineto{\pgfqpoint{1.027598in}{0.646208in}}%
\pgfpathlineto{\pgfqpoint{1.029199in}{0.635538in}}%
\pgfpathlineto{\pgfqpoint{1.030266in}{0.641474in}}%
\pgfpathlineto{\pgfqpoint{1.030800in}{0.639848in}}%
\pgfpathlineto{\pgfqpoint{1.031334in}{0.640999in}}%
\pgfpathlineto{\pgfqpoint{1.031868in}{0.639389in}}%
\pgfpathlineto{\pgfqpoint{1.032401in}{0.642581in}}%
\pgfpathlineto{\pgfqpoint{1.033469in}{0.637906in}}%
\pgfpathlineto{\pgfqpoint{1.034002in}{0.639759in}}%
\pgfpathlineto{\pgfqpoint{1.034536in}{0.637637in}}%
\pgfpathlineto{\pgfqpoint{1.036137in}{0.638735in}}%
\pgfpathlineto{\pgfqpoint{1.036671in}{0.641524in}}%
\pgfpathlineto{\pgfqpoint{1.037205in}{0.639378in}}%
\pgfpathlineto{\pgfqpoint{1.037738in}{0.640997in}}%
\pgfpathlineto{\pgfqpoint{1.038272in}{0.635923in}}%
\pgfpathlineto{\pgfqpoint{1.038806in}{0.636339in}}%
\pgfpathlineto{\pgfqpoint{1.040407in}{0.642054in}}%
\pgfpathlineto{\pgfqpoint{1.040940in}{0.636059in}}%
\pgfpathlineto{\pgfqpoint{1.041474in}{0.637546in}}%
\pgfpathlineto{\pgfqpoint{1.042008in}{0.637152in}}%
\pgfpathlineto{\pgfqpoint{1.042542in}{0.639822in}}%
\pgfpathlineto{\pgfqpoint{1.043075in}{0.638051in}}%
\pgfpathlineto{\pgfqpoint{1.044143in}{0.636928in}}%
\pgfpathlineto{\pgfqpoint{1.044676in}{0.638421in}}%
\pgfpathlineto{\pgfqpoint{1.045210in}{0.636597in}}%
\pgfpathlineto{\pgfqpoint{1.046277in}{0.644975in}}%
\pgfpathlineto{\pgfqpoint{1.047345in}{0.636136in}}%
\pgfpathlineto{\pgfqpoint{1.048946in}{0.644462in}}%
\pgfpathlineto{\pgfqpoint{1.050547in}{0.636361in}}%
\pgfpathlineto{\pgfqpoint{1.051081in}{0.642549in}}%
\pgfpathlineto{\pgfqpoint{1.051614in}{0.641512in}}%
\pgfpathlineto{\pgfqpoint{1.053216in}{0.637050in}}%
\pgfpathlineto{\pgfqpoint{1.053749in}{0.643081in}}%
\pgfpathlineto{\pgfqpoint{1.054283in}{0.639355in}}%
\pgfpathlineto{\pgfqpoint{1.054817in}{0.636342in}}%
\pgfpathlineto{\pgfqpoint{1.055350in}{0.637697in}}%
\pgfpathlineto{\pgfqpoint{1.056418in}{0.642399in}}%
\pgfpathlineto{\pgfqpoint{1.056951in}{0.635859in}}%
\pgfpathlineto{\pgfqpoint{1.057485in}{0.642165in}}%
\pgfpathlineto{\pgfqpoint{1.059086in}{0.636488in}}%
\pgfpathlineto{\pgfqpoint{1.059620in}{0.641087in}}%
\pgfpathlineto{\pgfqpoint{1.060154in}{0.639951in}}%
\pgfpathlineto{\pgfqpoint{1.060687in}{0.637539in}}%
\pgfpathlineto{\pgfqpoint{1.061221in}{0.637732in}}%
\pgfpathlineto{\pgfqpoint{1.062288in}{0.642850in}}%
\pgfpathlineto{\pgfqpoint{1.063356in}{0.638064in}}%
\pgfpathlineto{\pgfqpoint{1.063890in}{0.638300in}}%
\pgfpathlineto{\pgfqpoint{1.065491in}{0.640715in}}%
\pgfpathlineto{\pgfqpoint{1.066024in}{0.637308in}}%
\pgfpathlineto{\pgfqpoint{1.067092in}{0.641813in}}%
\pgfpathlineto{\pgfqpoint{1.069227in}{0.638152in}}%
\pgfpathlineto{\pgfqpoint{1.070828in}{0.640698in}}%
\pgfpathlineto{\pgfqpoint{1.071361in}{0.639515in}}%
\pgfpathlineto{\pgfqpoint{1.071895in}{0.640605in}}%
\pgfpathlineto{\pgfqpoint{1.072962in}{0.640695in}}%
\pgfpathlineto{\pgfqpoint{1.074030in}{0.640134in}}%
\pgfpathlineto{\pgfqpoint{1.075631in}{0.636092in}}%
\pgfpathlineto{\pgfqpoint{1.076165in}{0.641586in}}%
\pgfpathlineto{\pgfqpoint{1.076698in}{0.639509in}}%
\pgfpathlineto{\pgfqpoint{1.077232in}{0.639963in}}%
\pgfpathlineto{\pgfqpoint{1.077766in}{0.638371in}}%
\pgfpathlineto{\pgfqpoint{1.079367in}{0.644045in}}%
\pgfpathlineto{\pgfqpoint{1.080434in}{0.638056in}}%
\pgfpathlineto{\pgfqpoint{1.080968in}{0.639623in}}%
\pgfpathlineto{\pgfqpoint{1.082569in}{0.637295in}}%
\pgfpathlineto{\pgfqpoint{1.083103in}{0.638097in}}%
\pgfpathlineto{\pgfqpoint{1.083637in}{0.641157in}}%
\pgfpathlineto{\pgfqpoint{1.084170in}{0.637943in}}%
\pgfpathlineto{\pgfqpoint{1.084704in}{0.640658in}}%
\pgfpathlineto{\pgfqpoint{1.085238in}{0.637696in}}%
\pgfpathlineto{\pgfqpoint{1.085771in}{0.636010in}}%
\pgfpathlineto{\pgfqpoint{1.086305in}{0.642654in}}%
\pgfpathlineto{\pgfqpoint{1.086839in}{0.639112in}}%
\pgfpathlineto{\pgfqpoint{1.087906in}{0.642974in}}%
\pgfpathlineto{\pgfqpoint{1.090041in}{0.637806in}}%
\pgfpathlineto{\pgfqpoint{1.091108in}{0.639734in}}%
\pgfpathlineto{\pgfqpoint{1.091642in}{0.636413in}}%
\pgfpathlineto{\pgfqpoint{1.092709in}{0.648025in}}%
\pgfpathlineto{\pgfqpoint{1.093243in}{0.637273in}}%
\pgfpathlineto{\pgfqpoint{1.093777in}{0.638184in}}%
\pgfpathlineto{\pgfqpoint{1.094844in}{0.643092in}}%
\pgfpathlineto{\pgfqpoint{1.095378in}{0.639790in}}%
\pgfpathlineto{\pgfqpoint{1.095912in}{0.636951in}}%
\pgfpathlineto{\pgfqpoint{1.096445in}{0.639119in}}%
\pgfpathlineto{\pgfqpoint{1.096979in}{0.638811in}}%
\pgfpathlineto{\pgfqpoint{1.098046in}{0.643744in}}%
\pgfpathlineto{\pgfqpoint{1.098580in}{0.638305in}}%
\pgfpathlineto{\pgfqpoint{1.099114in}{0.640870in}}%
\pgfpathlineto{\pgfqpoint{1.101782in}{0.636748in}}%
\pgfpathlineto{\pgfqpoint{1.102316in}{0.637242in}}%
\pgfpathlineto{\pgfqpoint{1.102850in}{0.641873in}}%
\pgfpathlineto{\pgfqpoint{1.103383in}{0.639028in}}%
\pgfpathlineto{\pgfqpoint{1.103917in}{0.637153in}}%
\pgfpathlineto{\pgfqpoint{1.105518in}{0.639655in}}%
\pgfpathlineto{\pgfqpoint{1.106052in}{0.639164in}}%
\pgfpathlineto{\pgfqpoint{1.106586in}{0.642361in}}%
\pgfpathlineto{\pgfqpoint{1.107119in}{0.640526in}}%
\pgfpathlineto{\pgfqpoint{1.107653in}{0.638298in}}%
\pgfpathlineto{\pgfqpoint{1.108187in}{0.639680in}}%
\pgfpathlineto{\pgfqpoint{1.108720in}{0.640048in}}%
\pgfpathlineto{\pgfqpoint{1.109254in}{0.644492in}}%
\pgfpathlineto{\pgfqpoint{1.109788in}{0.640017in}}%
\pgfpathlineto{\pgfqpoint{1.111389in}{0.642693in}}%
\pgfpathlineto{\pgfqpoint{1.112456in}{0.637427in}}%
\pgfpathlineto{\pgfqpoint{1.112990in}{0.638677in}}%
\pgfpathlineto{\pgfqpoint{1.113524in}{0.641997in}}%
\pgfpathlineto{\pgfqpoint{1.114057in}{0.641609in}}%
\pgfpathlineto{\pgfqpoint{1.115125in}{0.638109in}}%
\pgfpathlineto{\pgfqpoint{1.116192in}{0.639121in}}%
\pgfpathlineto{\pgfqpoint{1.116726in}{0.641771in}}%
\pgfpathlineto{\pgfqpoint{1.117260in}{0.637715in}}%
\pgfpathlineto{\pgfqpoint{1.117793in}{0.639092in}}%
\pgfpathlineto{\pgfqpoint{1.119394in}{0.644354in}}%
\pgfpathlineto{\pgfqpoint{1.119928in}{0.638346in}}%
\pgfpathlineto{\pgfqpoint{1.120462in}{0.641319in}}%
\pgfpathlineto{\pgfqpoint{1.120996in}{0.640378in}}%
\pgfpathlineto{\pgfqpoint{1.122597in}{0.635232in}}%
\pgfpathlineto{\pgfqpoint{1.123664in}{0.649184in}}%
\pgfpathlineto{\pgfqpoint{1.124198in}{0.638084in}}%
\pgfpathlineto{\pgfqpoint{1.124731in}{0.640395in}}%
\pgfpathlineto{\pgfqpoint{1.125265in}{0.642720in}}%
\pgfpathlineto{\pgfqpoint{1.125799in}{0.641751in}}%
\pgfpathlineto{\pgfqpoint{1.126333in}{0.637283in}}%
\pgfpathlineto{\pgfqpoint{1.126866in}{0.637469in}}%
\pgfpathlineto{\pgfqpoint{1.127934in}{0.639934in}}%
\pgfpathlineto{\pgfqpoint{1.128467in}{0.636690in}}%
\pgfpathlineto{\pgfqpoint{1.129001in}{0.638708in}}%
\pgfpathlineto{\pgfqpoint{1.130602in}{0.642035in}}%
\pgfpathlineto{\pgfqpoint{1.131136in}{0.635668in}}%
\pgfpathlineto{\pgfqpoint{1.131670in}{0.643914in}}%
\pgfpathlineto{\pgfqpoint{1.132203in}{0.637415in}}%
\pgfpathlineto{\pgfqpoint{1.132737in}{0.640357in}}%
\pgfpathlineto{\pgfqpoint{1.133271in}{0.638836in}}%
\pgfpathlineto{\pgfqpoint{1.133804in}{0.638848in}}%
\pgfpathlineto{\pgfqpoint{1.134338in}{0.650285in}}%
\pgfpathlineto{\pgfqpoint{1.134872in}{0.639496in}}%
\pgfpathlineto{\pgfqpoint{1.135405in}{0.639043in}}%
\pgfpathlineto{\pgfqpoint{1.135939in}{0.639939in}}%
\pgfpathlineto{\pgfqpoint{1.136473in}{0.642220in}}%
\pgfpathlineto{\pgfqpoint{1.138074in}{0.638600in}}%
\pgfpathlineto{\pgfqpoint{1.138608in}{0.637334in}}%
\pgfpathlineto{\pgfqpoint{1.140209in}{0.645952in}}%
\pgfpathlineto{\pgfqpoint{1.140742in}{0.639997in}}%
\pgfpathlineto{\pgfqpoint{1.141276in}{0.647779in}}%
\pgfpathlineto{\pgfqpoint{1.141810in}{0.638893in}}%
\pgfpathlineto{\pgfqpoint{1.142344in}{0.643128in}}%
\pgfpathlineto{\pgfqpoint{1.142877in}{0.645333in}}%
\pgfpathlineto{\pgfqpoint{1.143411in}{0.640967in}}%
\pgfpathlineto{\pgfqpoint{1.143945in}{0.651730in}}%
\pgfpathlineto{\pgfqpoint{1.144478in}{0.643241in}}%
\pgfpathlineto{\pgfqpoint{1.146613in}{0.639859in}}%
\pgfpathlineto{\pgfqpoint{1.147147in}{0.649326in}}%
\pgfpathlineto{\pgfqpoint{1.147681in}{0.641188in}}%
\pgfpathlineto{\pgfqpoint{1.148748in}{0.637660in}}%
\pgfpathlineto{\pgfqpoint{1.149282in}{0.643061in}}%
\pgfpathlineto{\pgfqpoint{1.149815in}{0.637435in}}%
\pgfpathlineto{\pgfqpoint{1.150883in}{0.642196in}}%
\pgfpathlineto{\pgfqpoint{1.151417in}{0.640237in}}%
\pgfpathlineto{\pgfqpoint{1.151950in}{0.648143in}}%
\pgfpathlineto{\pgfqpoint{1.152484in}{0.639806in}}%
\pgfpathlineto{\pgfqpoint{1.153018in}{0.645933in}}%
\pgfpathlineto{\pgfqpoint{1.153551in}{0.641538in}}%
\pgfpathlineto{\pgfqpoint{1.155152in}{0.639937in}}%
\pgfpathlineto{\pgfqpoint{1.156220in}{0.642436in}}%
\pgfpathlineto{\pgfqpoint{1.156754in}{0.636775in}}%
\pgfpathlineto{\pgfqpoint{1.157287in}{0.642013in}}%
\pgfpathlineto{\pgfqpoint{1.157821in}{0.642727in}}%
\pgfpathlineto{\pgfqpoint{1.159422in}{0.636825in}}%
\pgfpathlineto{\pgfqpoint{1.159956in}{0.638907in}}%
\pgfpathlineto{\pgfqpoint{1.160489in}{0.637810in}}%
\pgfpathlineto{\pgfqpoint{1.161023in}{0.636534in}}%
\pgfpathlineto{\pgfqpoint{1.161557in}{0.639935in}}%
\pgfpathlineto{\pgfqpoint{1.162091in}{0.639536in}}%
\pgfpathlineto{\pgfqpoint{1.162624in}{0.636457in}}%
\pgfpathlineto{\pgfqpoint{1.163158in}{0.638783in}}%
\pgfpathlineto{\pgfqpoint{1.163692in}{0.643482in}}%
\pgfpathlineto{\pgfqpoint{1.164225in}{0.637207in}}%
\pgfpathlineto{\pgfqpoint{1.164759in}{0.640752in}}%
\pgfpathlineto{\pgfqpoint{1.166360in}{0.639844in}}%
\pgfpathlineto{\pgfqpoint{1.166894in}{0.644540in}}%
\pgfpathlineto{\pgfqpoint{1.167428in}{0.641113in}}%
\pgfpathlineto{\pgfqpoint{1.167961in}{0.638390in}}%
\pgfpathlineto{\pgfqpoint{1.168495in}{0.640266in}}%
\pgfpathlineto{\pgfqpoint{1.169029in}{0.640102in}}%
\pgfpathlineto{\pgfqpoint{1.170096in}{0.644254in}}%
\pgfpathlineto{\pgfqpoint{1.170630in}{0.643531in}}%
\pgfpathlineto{\pgfqpoint{1.171163in}{0.645514in}}%
\pgfpathlineto{\pgfqpoint{1.171697in}{0.636814in}}%
\pgfpathlineto{\pgfqpoint{1.172231in}{0.644482in}}%
\pgfpathlineto{\pgfqpoint{1.173298in}{0.638136in}}%
\pgfpathlineto{\pgfqpoint{1.173832in}{0.638253in}}%
\pgfpathlineto{\pgfqpoint{1.175433in}{0.644848in}}%
\pgfpathlineto{\pgfqpoint{1.176500in}{0.636735in}}%
\pgfpathlineto{\pgfqpoint{1.177034in}{0.641313in}}%
\pgfpathlineto{\pgfqpoint{1.177568in}{0.639663in}}%
\pgfpathlineto{\pgfqpoint{1.178102in}{0.639452in}}%
\pgfpathlineto{\pgfqpoint{1.178635in}{0.646511in}}%
\pgfpathlineto{\pgfqpoint{1.179169in}{0.639338in}}%
\pgfpathlineto{\pgfqpoint{1.179703in}{0.642867in}}%
\pgfpathlineto{\pgfqpoint{1.180236in}{0.641552in}}%
\pgfpathlineto{\pgfqpoint{1.180770in}{0.637467in}}%
\pgfpathlineto{\pgfqpoint{1.181304in}{0.639764in}}%
\pgfpathlineto{\pgfqpoint{1.182905in}{0.638692in}}%
\pgfpathlineto{\pgfqpoint{1.183439in}{0.637626in}}%
\pgfpathlineto{\pgfqpoint{1.183972in}{0.640899in}}%
\pgfpathlineto{\pgfqpoint{1.184506in}{0.639105in}}%
\pgfpathlineto{\pgfqpoint{1.186107in}{0.637072in}}%
\pgfpathlineto{\pgfqpoint{1.186641in}{0.642376in}}%
\pgfpathlineto{\pgfqpoint{1.187174in}{0.639221in}}%
\pgfpathlineto{\pgfqpoint{1.188776in}{0.637553in}}%
\pgfpathlineto{\pgfqpoint{1.189309in}{0.647367in}}%
\pgfpathlineto{\pgfqpoint{1.189843in}{0.640413in}}%
\pgfpathlineto{\pgfqpoint{1.190910in}{0.635496in}}%
\pgfpathlineto{\pgfqpoint{1.193045in}{0.647136in}}%
\pgfpathlineto{\pgfqpoint{1.194113in}{0.637136in}}%
\pgfpathlineto{\pgfqpoint{1.195180in}{0.638983in}}%
\pgfpathlineto{\pgfqpoint{1.195714in}{0.639990in}}%
\pgfpathlineto{\pgfqpoint{1.196247in}{0.645010in}}%
\pgfpathlineto{\pgfqpoint{1.196781in}{0.642361in}}%
\pgfpathlineto{\pgfqpoint{1.197315in}{0.641959in}}%
\pgfpathlineto{\pgfqpoint{1.198916in}{0.647235in}}%
\pgfpathlineto{\pgfqpoint{1.199983in}{0.637186in}}%
\pgfpathlineto{\pgfqpoint{1.200517in}{0.639147in}}%
\pgfpathlineto{\pgfqpoint{1.202118in}{0.643437in}}%
\pgfpathlineto{\pgfqpoint{1.203185in}{0.636535in}}%
\pgfpathlineto{\pgfqpoint{1.204787in}{0.642449in}}%
\pgfpathlineto{\pgfqpoint{1.205320in}{0.638443in}}%
\pgfpathlineto{\pgfqpoint{1.205854in}{0.642728in}}%
\pgfpathlineto{\pgfqpoint{1.206388in}{0.642399in}}%
\pgfpathlineto{\pgfqpoint{1.206921in}{0.639611in}}%
\pgfpathlineto{\pgfqpoint{1.207455in}{0.641678in}}%
\pgfpathlineto{\pgfqpoint{1.207989in}{0.641692in}}%
\pgfpathlineto{\pgfqpoint{1.208522in}{0.639274in}}%
\pgfpathlineto{\pgfqpoint{1.209056in}{0.640927in}}%
\pgfpathlineto{\pgfqpoint{1.209590in}{0.641833in}}%
\pgfpathlineto{\pgfqpoint{1.210124in}{0.637035in}}%
\pgfpathlineto{\pgfqpoint{1.210657in}{0.638639in}}%
\pgfpathlineto{\pgfqpoint{1.211191in}{0.640911in}}%
\pgfpathlineto{\pgfqpoint{1.212258in}{0.638157in}}%
\pgfpathlineto{\pgfqpoint{1.212792in}{0.638334in}}%
\pgfpathlineto{\pgfqpoint{1.213326in}{0.644360in}}%
\pgfpathlineto{\pgfqpoint{1.213860in}{0.640765in}}%
\pgfpathlineto{\pgfqpoint{1.214393in}{0.636661in}}%
\pgfpathlineto{\pgfqpoint{1.214927in}{0.639834in}}%
\pgfpathlineto{\pgfqpoint{1.215994in}{0.638363in}}%
\pgfpathlineto{\pgfqpoint{1.217062in}{0.644464in}}%
\pgfpathlineto{\pgfqpoint{1.217595in}{0.637605in}}%
\pgfpathlineto{\pgfqpoint{1.218129in}{0.642560in}}%
\pgfpathlineto{\pgfqpoint{1.218663in}{0.642214in}}%
\pgfpathlineto{\pgfqpoint{1.219197in}{0.638744in}}%
\pgfpathlineto{\pgfqpoint{1.219730in}{0.638835in}}%
\pgfpathlineto{\pgfqpoint{1.220264in}{0.642295in}}%
\pgfpathlineto{\pgfqpoint{1.220798in}{0.641767in}}%
\pgfpathlineto{\pgfqpoint{1.221331in}{0.641723in}}%
\pgfpathlineto{\pgfqpoint{1.221865in}{0.638313in}}%
\pgfpathlineto{\pgfqpoint{1.222399in}{0.642930in}}%
\pgfpathlineto{\pgfqpoint{1.222932in}{0.639890in}}%
\pgfpathlineto{\pgfqpoint{1.223466in}{0.640388in}}%
\pgfpathlineto{\pgfqpoint{1.225067in}{0.639096in}}%
\pgfpathlineto{\pgfqpoint{1.225601in}{0.639556in}}%
\pgfpathlineto{\pgfqpoint{1.226135in}{0.641578in}}%
\pgfpathlineto{\pgfqpoint{1.226668in}{0.636832in}}%
\pgfpathlineto{\pgfqpoint{1.228269in}{0.645195in}}%
\pgfpathlineto{\pgfqpoint{1.229337in}{0.637715in}}%
\pgfpathlineto{\pgfqpoint{1.230938in}{0.644541in}}%
\pgfpathlineto{\pgfqpoint{1.232539in}{0.637343in}}%
\pgfpathlineto{\pgfqpoint{1.233606in}{0.648359in}}%
\pgfpathlineto{\pgfqpoint{1.234140in}{0.642466in}}%
\pgfpathlineto{\pgfqpoint{1.234674in}{0.639659in}}%
\pgfpathlineto{\pgfqpoint{1.236275in}{0.644444in}}%
\pgfpathlineto{\pgfqpoint{1.238410in}{0.636456in}}%
\pgfpathlineto{\pgfqpoint{1.238943in}{0.640189in}}%
\pgfpathlineto{\pgfqpoint{1.239477in}{0.636337in}}%
\pgfpathlineto{\pgfqpoint{1.240545in}{0.639325in}}%
\pgfpathlineto{\pgfqpoint{1.241078in}{0.635489in}}%
\pgfpathlineto{\pgfqpoint{1.241612in}{0.644273in}}%
\pgfpathlineto{\pgfqpoint{1.242146in}{0.639020in}}%
\pgfpathlineto{\pgfqpoint{1.242679in}{0.640265in}}%
\pgfpathlineto{\pgfqpoint{1.243213in}{0.638344in}}%
\pgfpathlineto{\pgfqpoint{1.243747in}{0.646212in}}%
\pgfpathlineto{\pgfqpoint{1.244280in}{0.642603in}}%
\pgfpathlineto{\pgfqpoint{1.244814in}{0.644584in}}%
\pgfpathlineto{\pgfqpoint{1.245348in}{0.637426in}}%
\pgfpathlineto{\pgfqpoint{1.245882in}{0.639551in}}%
\pgfpathlineto{\pgfqpoint{1.246415in}{0.638980in}}%
\pgfpathlineto{\pgfqpoint{1.246949in}{0.640297in}}%
\pgfpathlineto{\pgfqpoint{1.247483in}{0.639513in}}%
\pgfpathlineto{\pgfqpoint{1.248016in}{0.636941in}}%
\pgfpathlineto{\pgfqpoint{1.249617in}{0.640314in}}%
\pgfpathlineto{\pgfqpoint{1.250151in}{0.640543in}}%
\pgfpathlineto{\pgfqpoint{1.251219in}{0.635666in}}%
\pgfpathlineto{\pgfqpoint{1.251752in}{0.645606in}}%
\pgfpathlineto{\pgfqpoint{1.252286in}{0.640921in}}%
\pgfpathlineto{\pgfqpoint{1.253887in}{0.637570in}}%
\pgfpathlineto{\pgfqpoint{1.255488in}{0.649009in}}%
\pgfpathlineto{\pgfqpoint{1.257089in}{0.637570in}}%
\pgfpathlineto{\pgfqpoint{1.257623in}{0.640625in}}%
\pgfpathlineto{\pgfqpoint{1.258157in}{0.638363in}}%
\pgfpathlineto{\pgfqpoint{1.258690in}{0.636940in}}%
\pgfpathlineto{\pgfqpoint{1.259758in}{0.642227in}}%
\pgfpathlineto{\pgfqpoint{1.260291in}{0.639308in}}%
\pgfpathlineto{\pgfqpoint{1.261359in}{0.641438in}}%
\pgfpathlineto{\pgfqpoint{1.261893in}{0.640732in}}%
\pgfpathlineto{\pgfqpoint{1.262426in}{0.641029in}}%
\pgfpathlineto{\pgfqpoint{1.262960in}{0.644010in}}%
\pgfpathlineto{\pgfqpoint{1.264027in}{0.638721in}}%
\pgfpathlineto{\pgfqpoint{1.265628in}{0.643388in}}%
\pgfpathlineto{\pgfqpoint{1.266162in}{0.639950in}}%
\pgfpathlineto{\pgfqpoint{1.266696in}{0.642656in}}%
\pgfpathlineto{\pgfqpoint{1.268297in}{0.641776in}}%
\pgfpathlineto{\pgfqpoint{1.268831in}{0.637398in}}%
\pgfpathlineto{\pgfqpoint{1.269364in}{0.639197in}}%
\pgfpathlineto{\pgfqpoint{1.269898in}{0.639289in}}%
\pgfpathlineto{\pgfqpoint{1.270965in}{0.637060in}}%
\pgfpathlineto{\pgfqpoint{1.271499in}{0.638787in}}%
\pgfpathlineto{\pgfqpoint{1.272033in}{0.636987in}}%
\pgfpathlineto{\pgfqpoint{1.272567in}{0.638364in}}%
\pgfpathlineto{\pgfqpoint{1.273100in}{0.643606in}}%
\pgfpathlineto{\pgfqpoint{1.273634in}{0.639951in}}%
\pgfpathlineto{\pgfqpoint{1.274168in}{0.638006in}}%
\pgfpathlineto{\pgfqpoint{1.274701in}{0.641397in}}%
\pgfpathlineto{\pgfqpoint{1.275235in}{0.637053in}}%
\pgfpathlineto{\pgfqpoint{1.275769in}{0.642985in}}%
\pgfpathlineto{\pgfqpoint{1.276302in}{0.640345in}}%
\pgfpathlineto{\pgfqpoint{1.277370in}{0.642289in}}%
\pgfpathlineto{\pgfqpoint{1.278437in}{0.635504in}}%
\pgfpathlineto{\pgfqpoint{1.279505in}{0.640765in}}%
\pgfpathlineto{\pgfqpoint{1.280038in}{0.640693in}}%
\pgfpathlineto{\pgfqpoint{1.280572in}{0.640907in}}%
\pgfpathlineto{\pgfqpoint{1.281640in}{0.645045in}}%
\pgfpathlineto{\pgfqpoint{1.282173in}{0.637317in}}%
\pgfpathlineto{\pgfqpoint{1.282707in}{0.643850in}}%
\pgfpathlineto{\pgfqpoint{1.283241in}{0.640521in}}%
\pgfpathlineto{\pgfqpoint{1.283774in}{0.642791in}}%
\pgfpathlineto{\pgfqpoint{1.284308in}{0.643469in}}%
\pgfpathlineto{\pgfqpoint{1.284842in}{0.648021in}}%
\pgfpathlineto{\pgfqpoint{1.285909in}{0.636876in}}%
\pgfpathlineto{\pgfqpoint{1.286443in}{0.639876in}}%
\pgfpathlineto{\pgfqpoint{1.287510in}{0.640012in}}%
\pgfpathlineto{\pgfqpoint{1.288044in}{0.637282in}}%
\pgfpathlineto{\pgfqpoint{1.288578in}{0.642869in}}%
\pgfpathlineto{\pgfqpoint{1.289111in}{0.641305in}}%
\pgfpathlineto{\pgfqpoint{1.289645in}{0.638453in}}%
\pgfpathlineto{\pgfqpoint{1.290179in}{0.645778in}}%
\pgfpathlineto{\pgfqpoint{1.290712in}{0.637604in}}%
\pgfpathlineto{\pgfqpoint{1.291246in}{0.637834in}}%
\pgfpathlineto{\pgfqpoint{1.291780in}{0.642378in}}%
\pgfpathlineto{\pgfqpoint{1.292314in}{0.640556in}}%
\pgfpathlineto{\pgfqpoint{1.292847in}{0.640847in}}%
\pgfpathlineto{\pgfqpoint{1.293381in}{0.642977in}}%
\pgfpathlineto{\pgfqpoint{1.294448in}{0.637009in}}%
\pgfpathlineto{\pgfqpoint{1.294982in}{0.642951in}}%
\pgfpathlineto{\pgfqpoint{1.295516in}{0.637306in}}%
\pgfpathlineto{\pgfqpoint{1.297651in}{0.642335in}}%
\pgfpathlineto{\pgfqpoint{1.298184in}{0.643005in}}%
\pgfpathlineto{\pgfqpoint{1.299252in}{0.647007in}}%
\pgfpathlineto{\pgfqpoint{1.300319in}{0.638795in}}%
\pgfpathlineto{\pgfqpoint{1.300853in}{0.639409in}}%
\pgfpathlineto{\pgfqpoint{1.301386in}{0.640806in}}%
\pgfpathlineto{\pgfqpoint{1.301920in}{0.636360in}}%
\pgfpathlineto{\pgfqpoint{1.302454in}{0.643551in}}%
\pgfpathlineto{\pgfqpoint{1.303521in}{0.643247in}}%
\pgfpathlineto{\pgfqpoint{1.305656in}{0.636584in}}%
\pgfpathlineto{\pgfqpoint{1.307791in}{0.645609in}}%
\pgfpathlineto{\pgfqpoint{1.308325in}{0.640698in}}%
\pgfpathlineto{\pgfqpoint{1.308858in}{0.642599in}}%
\pgfpathlineto{\pgfqpoint{1.309392in}{0.643490in}}%
\pgfpathlineto{\pgfqpoint{1.310993in}{0.638834in}}%
\pgfpathlineto{\pgfqpoint{1.311527in}{0.640576in}}%
\pgfpathlineto{\pgfqpoint{1.312060in}{0.638218in}}%
\pgfpathlineto{\pgfqpoint{1.312594in}{0.642970in}}%
\pgfpathlineto{\pgfqpoint{1.313128in}{0.637926in}}%
\pgfpathlineto{\pgfqpoint{1.314729in}{0.643158in}}%
\pgfpathlineto{\pgfqpoint{1.315263in}{0.639989in}}%
\pgfpathlineto{\pgfqpoint{1.315796in}{0.640740in}}%
\pgfpathlineto{\pgfqpoint{1.316864in}{0.644085in}}%
\pgfpathlineto{\pgfqpoint{1.317397in}{0.636654in}}%
\pgfpathlineto{\pgfqpoint{1.317931in}{0.639537in}}%
\pgfpathlineto{\pgfqpoint{1.318465in}{0.643717in}}%
\pgfpathlineto{\pgfqpoint{1.320066in}{0.637460in}}%
\pgfpathlineto{\pgfqpoint{1.320600in}{0.638916in}}%
\pgfpathlineto{\pgfqpoint{1.321133in}{0.636251in}}%
\pgfpathlineto{\pgfqpoint{1.321667in}{0.637683in}}%
\pgfpathlineto{\pgfqpoint{1.323268in}{0.644921in}}%
\pgfpathlineto{\pgfqpoint{1.324336in}{0.638275in}}%
\pgfpathlineto{\pgfqpoint{1.324869in}{0.643700in}}%
\pgfpathlineto{\pgfqpoint{1.325403in}{0.641288in}}%
\pgfpathlineto{\pgfqpoint{1.327004in}{0.636303in}}%
\pgfpathlineto{\pgfqpoint{1.328605in}{0.642230in}}%
\pgfpathlineto{\pgfqpoint{1.330740in}{0.636323in}}%
\pgfpathlineto{\pgfqpoint{1.331807in}{0.641929in}}%
\pgfpathlineto{\pgfqpoint{1.332341in}{0.639479in}}%
\pgfpathlineto{\pgfqpoint{1.332875in}{0.640005in}}%
\pgfpathlineto{\pgfqpoint{1.334476in}{0.636174in}}%
\pgfpathlineto{\pgfqpoint{1.336077in}{0.640905in}}%
\pgfpathlineto{\pgfqpoint{1.338212in}{0.643763in}}%
\pgfpathlineto{\pgfqpoint{1.339813in}{0.637302in}}%
\pgfpathlineto{\pgfqpoint{1.340347in}{0.640765in}}%
\pgfpathlineto{\pgfqpoint{1.340880in}{0.641925in}}%
\pgfpathlineto{\pgfqpoint{1.341414in}{0.653221in}}%
\pgfpathlineto{\pgfqpoint{1.343015in}{0.639517in}}%
\pgfpathlineto{\pgfqpoint{1.344082in}{0.649580in}}%
\pgfpathlineto{\pgfqpoint{1.345684in}{0.637964in}}%
\pgfpathlineto{\pgfqpoint{1.346217in}{0.638169in}}%
\pgfpathlineto{\pgfqpoint{1.346751in}{0.640302in}}%
\pgfpathlineto{\pgfqpoint{1.347285in}{0.637327in}}%
\pgfpathlineto{\pgfqpoint{1.347818in}{0.644021in}}%
\pgfpathlineto{\pgfqpoint{1.348352in}{0.637467in}}%
\pgfpathlineto{\pgfqpoint{1.349420in}{0.644810in}}%
\pgfpathlineto{\pgfqpoint{1.349953in}{0.642000in}}%
\pgfpathlineto{\pgfqpoint{1.350487in}{0.642294in}}%
\pgfpathlineto{\pgfqpoint{1.351021in}{0.636046in}}%
\pgfpathlineto{\pgfqpoint{1.351554in}{0.637439in}}%
\pgfpathlineto{\pgfqpoint{1.352088in}{0.638669in}}%
\pgfpathlineto{\pgfqpoint{1.352622in}{0.643238in}}%
\pgfpathlineto{\pgfqpoint{1.353155in}{0.640518in}}%
\pgfpathlineto{\pgfqpoint{1.353689in}{0.639579in}}%
\pgfpathlineto{\pgfqpoint{1.355290in}{0.646388in}}%
\pgfpathlineto{\pgfqpoint{1.356358in}{0.637904in}}%
\pgfpathlineto{\pgfqpoint{1.356891in}{0.640993in}}%
\pgfpathlineto{\pgfqpoint{1.357425in}{0.640746in}}%
\pgfpathlineto{\pgfqpoint{1.357959in}{0.637601in}}%
\pgfpathlineto{\pgfqpoint{1.358492in}{0.642673in}}%
\pgfpathlineto{\pgfqpoint{1.359026in}{0.641711in}}%
\pgfpathlineto{\pgfqpoint{1.360094in}{0.636157in}}%
\pgfpathlineto{\pgfqpoint{1.361695in}{0.640493in}}%
\pgfpathlineto{\pgfqpoint{1.362228in}{0.640741in}}%
\pgfpathlineto{\pgfqpoint{1.362762in}{0.644288in}}%
\pgfpathlineto{\pgfqpoint{1.363296in}{0.643858in}}%
\pgfpathlineto{\pgfqpoint{1.363829in}{0.636074in}}%
\pgfpathlineto{\pgfqpoint{1.364363in}{0.642573in}}%
\pgfpathlineto{\pgfqpoint{1.364897in}{0.642634in}}%
\pgfpathlineto{\pgfqpoint{1.365431in}{0.644377in}}%
\pgfpathlineto{\pgfqpoint{1.367032in}{0.637621in}}%
\pgfpathlineto{\pgfqpoint{1.367565in}{0.639766in}}%
\pgfpathlineto{\pgfqpoint{1.368099in}{0.638569in}}%
\pgfpathlineto{\pgfqpoint{1.368633in}{0.637360in}}%
\pgfpathlineto{\pgfqpoint{1.370234in}{0.644444in}}%
\pgfpathlineto{\pgfqpoint{1.370768in}{0.639469in}}%
\pgfpathlineto{\pgfqpoint{1.371301in}{0.645237in}}%
\pgfpathlineto{\pgfqpoint{1.371835in}{0.638057in}}%
\pgfpathlineto{\pgfqpoint{1.372369in}{0.640433in}}%
\pgfpathlineto{\pgfqpoint{1.373436in}{0.640579in}}%
\pgfpathlineto{\pgfqpoint{1.373970in}{0.645715in}}%
\pgfpathlineto{\pgfqpoint{1.374503in}{0.642816in}}%
\pgfpathlineto{\pgfqpoint{1.375571in}{0.638439in}}%
\pgfpathlineto{\pgfqpoint{1.376105in}{0.641984in}}%
\pgfpathlineto{\pgfqpoint{1.376638in}{0.638956in}}%
\pgfpathlineto{\pgfqpoint{1.378773in}{0.647034in}}%
\pgfpathlineto{\pgfqpoint{1.380374in}{0.638610in}}%
\pgfpathlineto{\pgfqpoint{1.381442in}{0.638658in}}%
\pgfpathlineto{\pgfqpoint{1.381975in}{0.637158in}}%
\pgfpathlineto{\pgfqpoint{1.382509in}{0.640268in}}%
\pgfpathlineto{\pgfqpoint{1.383043in}{0.636903in}}%
\pgfpathlineto{\pgfqpoint{1.383576in}{0.639951in}}%
\pgfpathlineto{\pgfqpoint{1.384110in}{0.640739in}}%
\pgfpathlineto{\pgfqpoint{1.384644in}{0.638920in}}%
\pgfpathlineto{\pgfqpoint{1.385177in}{0.639621in}}%
\pgfpathlineto{\pgfqpoint{1.385711in}{0.642771in}}%
\pgfpathlineto{\pgfqpoint{1.386779in}{0.637751in}}%
\pgfpathlineto{\pgfqpoint{1.387312in}{0.638854in}}%
\pgfpathlineto{\pgfqpoint{1.387846in}{0.643015in}}%
\pgfpathlineto{\pgfqpoint{1.388380in}{0.638963in}}%
\pgfpathlineto{\pgfqpoint{1.389447in}{0.638616in}}%
\pgfpathlineto{\pgfqpoint{1.389981in}{0.641798in}}%
\pgfpathlineto{\pgfqpoint{1.390514in}{0.637267in}}%
\pgfpathlineto{\pgfqpoint{1.391048in}{0.643346in}}%
\pgfpathlineto{\pgfqpoint{1.391582in}{0.636141in}}%
\pgfpathlineto{\pgfqpoint{1.392116in}{0.639292in}}%
\pgfpathlineto{\pgfqpoint{1.392649in}{0.639030in}}%
\pgfpathlineto{\pgfqpoint{1.393183in}{0.641680in}}%
\pgfpathlineto{\pgfqpoint{1.393717in}{0.639008in}}%
\pgfpathlineto{\pgfqpoint{1.394784in}{0.644308in}}%
\pgfpathlineto{\pgfqpoint{1.395318in}{0.643850in}}%
\pgfpathlineto{\pgfqpoint{1.395851in}{0.639436in}}%
\pgfpathlineto{\pgfqpoint{1.396385in}{0.641709in}}%
\pgfpathlineto{\pgfqpoint{1.396919in}{0.643627in}}%
\pgfpathlineto{\pgfqpoint{1.397453in}{0.636985in}}%
\pgfpathlineto{\pgfqpoint{1.397986in}{0.649569in}}%
\pgfpathlineto{\pgfqpoint{1.398520in}{0.637358in}}%
\pgfpathlineto{\pgfqpoint{1.400121in}{0.648492in}}%
\pgfpathlineto{\pgfqpoint{1.401188in}{0.639035in}}%
\pgfpathlineto{\pgfqpoint{1.401722in}{0.639848in}}%
\pgfpathlineto{\pgfqpoint{1.402256in}{0.642384in}}%
\pgfpathlineto{\pgfqpoint{1.402790in}{0.639327in}}%
\pgfpathlineto{\pgfqpoint{1.403323in}{0.640769in}}%
\pgfpathlineto{\pgfqpoint{1.404391in}{0.638990in}}%
\pgfpathlineto{\pgfqpoint{1.405458in}{0.640920in}}%
\pgfpathlineto{\pgfqpoint{1.405992in}{0.637650in}}%
\pgfpathlineto{\pgfqpoint{1.406525in}{0.638672in}}%
\pgfpathlineto{\pgfqpoint{1.407059in}{0.649702in}}%
\pgfpathlineto{\pgfqpoint{1.407593in}{0.648643in}}%
\pgfpathlineto{\pgfqpoint{1.408127in}{0.637557in}}%
\pgfpathlineto{\pgfqpoint{1.408660in}{0.641128in}}%
\pgfpathlineto{\pgfqpoint{1.409194in}{0.640215in}}%
\pgfpathlineto{\pgfqpoint{1.409728in}{0.644661in}}%
\pgfpathlineto{\pgfqpoint{1.410261in}{0.644401in}}%
\pgfpathlineto{\pgfqpoint{1.411329in}{0.637987in}}%
\pgfpathlineto{\pgfqpoint{1.411862in}{0.640512in}}%
\pgfpathlineto{\pgfqpoint{1.412396in}{0.644075in}}%
\pgfpathlineto{\pgfqpoint{1.412930in}{0.641291in}}%
\pgfpathlineto{\pgfqpoint{1.413464in}{0.638690in}}%
\pgfpathlineto{\pgfqpoint{1.413997in}{0.642566in}}%
\pgfpathlineto{\pgfqpoint{1.414531in}{0.635981in}}%
\pgfpathlineto{\pgfqpoint{1.415065in}{0.639904in}}%
\pgfpathlineto{\pgfqpoint{1.415598in}{0.638729in}}%
\pgfpathlineto{\pgfqpoint{1.416132in}{0.639761in}}%
\pgfpathlineto{\pgfqpoint{1.416666in}{0.642813in}}%
\pgfpathlineto{\pgfqpoint{1.417200in}{0.641463in}}%
\pgfpathlineto{\pgfqpoint{1.418801in}{0.637887in}}%
\pgfpathlineto{\pgfqpoint{1.419334in}{0.638084in}}%
\pgfpathlineto{\pgfqpoint{1.420402in}{0.644926in}}%
\pgfpathlineto{\pgfqpoint{1.420935in}{0.643685in}}%
\pgfpathlineto{\pgfqpoint{1.422537in}{0.640720in}}%
\pgfpathlineto{\pgfqpoint{1.423070in}{0.643767in}}%
\pgfpathlineto{\pgfqpoint{1.423604in}{0.635841in}}%
\pgfpathlineto{\pgfqpoint{1.424138in}{0.644366in}}%
\pgfpathlineto{\pgfqpoint{1.425739in}{0.635653in}}%
\pgfpathlineto{\pgfqpoint{1.426272in}{0.642495in}}%
\pgfpathlineto{\pgfqpoint{1.426806in}{0.640956in}}%
\pgfpathlineto{\pgfqpoint{1.427340in}{0.637789in}}%
\pgfpathlineto{\pgfqpoint{1.428407in}{0.646629in}}%
\pgfpathlineto{\pgfqpoint{1.430008in}{0.635877in}}%
\pgfpathlineto{\pgfqpoint{1.430542in}{0.644056in}}%
\pgfpathlineto{\pgfqpoint{1.431076in}{0.638021in}}%
\pgfpathlineto{\pgfqpoint{1.431609in}{0.642987in}}%
\pgfpathlineto{\pgfqpoint{1.432143in}{0.641000in}}%
\pgfpathlineto{\pgfqpoint{1.433211in}{0.643406in}}%
\pgfpathlineto{\pgfqpoint{1.434278in}{0.643709in}}%
\pgfpathlineto{\pgfqpoint{1.434812in}{0.639216in}}%
\pgfpathlineto{\pgfqpoint{1.436413in}{0.643833in}}%
\pgfpathlineto{\pgfqpoint{1.436946in}{0.640960in}}%
\pgfpathlineto{\pgfqpoint{1.437480in}{0.644345in}}%
\pgfpathlineto{\pgfqpoint{1.438014in}{0.637095in}}%
\pgfpathlineto{\pgfqpoint{1.438548in}{0.638330in}}%
\pgfpathlineto{\pgfqpoint{1.439615in}{0.641089in}}%
\pgfpathlineto{\pgfqpoint{1.440149in}{0.635790in}}%
\pgfpathlineto{\pgfqpoint{1.440682in}{0.639700in}}%
\pgfpathlineto{\pgfqpoint{1.441216in}{0.641213in}}%
\pgfpathlineto{\pgfqpoint{1.441750in}{0.646956in}}%
\pgfpathlineto{\pgfqpoint{1.442283in}{0.639672in}}%
\pgfpathlineto{\pgfqpoint{1.442817in}{0.639961in}}%
\pgfpathlineto{\pgfqpoint{1.443351in}{0.642568in}}%
\pgfpathlineto{\pgfqpoint{1.443885in}{0.636377in}}%
\pgfpathlineto{\pgfqpoint{1.444952in}{0.636504in}}%
\pgfpathlineto{\pgfqpoint{1.446553in}{0.639934in}}%
\pgfpathlineto{\pgfqpoint{1.447087in}{0.639390in}}%
\pgfpathlineto{\pgfqpoint{1.447620in}{0.637589in}}%
\pgfpathlineto{\pgfqpoint{1.448154in}{0.645212in}}%
\pgfpathlineto{\pgfqpoint{1.448688in}{0.642927in}}%
\pgfpathlineto{\pgfqpoint{1.450289in}{0.637001in}}%
\pgfpathlineto{\pgfqpoint{1.450823in}{0.638393in}}%
\pgfpathlineto{\pgfqpoint{1.451356in}{0.636375in}}%
\pgfpathlineto{\pgfqpoint{1.451890in}{0.652397in}}%
\pgfpathlineto{\pgfqpoint{1.452424in}{0.651746in}}%
\pgfpathlineto{\pgfqpoint{1.454025in}{0.640207in}}%
\pgfpathlineto{\pgfqpoint{1.454559in}{0.643141in}}%
\pgfpathlineto{\pgfqpoint{1.455092in}{0.638618in}}%
\pgfpathlineto{\pgfqpoint{1.456160in}{0.646771in}}%
\pgfpathlineto{\pgfqpoint{1.456693in}{0.644809in}}%
\pgfpathlineto{\pgfqpoint{1.457227in}{0.639370in}}%
\pgfpathlineto{\pgfqpoint{1.457761in}{0.645235in}}%
\pgfpathlineto{\pgfqpoint{1.458294in}{0.644739in}}%
\pgfpathlineto{\pgfqpoint{1.459896in}{0.636489in}}%
\pgfpathlineto{\pgfqpoint{1.461497in}{0.640556in}}%
\pgfpathlineto{\pgfqpoint{1.462030in}{0.640186in}}%
\pgfpathlineto{\pgfqpoint{1.462564in}{0.648300in}}%
\pgfpathlineto{\pgfqpoint{1.463098in}{0.643039in}}%
\pgfpathlineto{\pgfqpoint{1.463631in}{0.643692in}}%
\pgfpathlineto{\pgfqpoint{1.464165in}{0.639966in}}%
\pgfpathlineto{\pgfqpoint{1.464699in}{0.641012in}}%
\pgfpathlineto{\pgfqpoint{1.465233in}{0.644941in}}%
\pgfpathlineto{\pgfqpoint{1.466300in}{0.636716in}}%
\pgfpathlineto{\pgfqpoint{1.467367in}{0.642771in}}%
\pgfpathlineto{\pgfqpoint{1.467901in}{0.637616in}}%
\pgfpathlineto{\pgfqpoint{1.468435in}{0.641381in}}%
\pgfpathlineto{\pgfqpoint{1.470036in}{0.639159in}}%
\pgfpathlineto{\pgfqpoint{1.470570in}{0.637525in}}%
\pgfpathlineto{\pgfqpoint{1.471637in}{0.644581in}}%
\pgfpathlineto{\pgfqpoint{1.472171in}{0.639063in}}%
\pgfpathlineto{\pgfqpoint{1.472704in}{0.641524in}}%
\pgfpathlineto{\pgfqpoint{1.473238in}{0.641584in}}%
\pgfpathlineto{\pgfqpoint{1.473772in}{0.637033in}}%
\pgfpathlineto{\pgfqpoint{1.474305in}{0.638940in}}%
\pgfpathlineto{\pgfqpoint{1.475373in}{0.638744in}}%
\pgfpathlineto{\pgfqpoint{1.475907in}{0.644461in}}%
\pgfpathlineto{\pgfqpoint{1.476974in}{0.646415in}}%
\pgfpathlineto{\pgfqpoint{1.478041in}{0.635554in}}%
\pgfpathlineto{\pgfqpoint{1.478575in}{0.637292in}}%
\pgfpathlineto{\pgfqpoint{1.479109in}{0.643169in}}%
\pgfpathlineto{\pgfqpoint{1.479642in}{0.637506in}}%
\pgfpathlineto{\pgfqpoint{1.480176in}{0.637081in}}%
\pgfpathlineto{\pgfqpoint{1.481777in}{0.643507in}}%
\pgfpathlineto{\pgfqpoint{1.482311in}{0.639194in}}%
\pgfpathlineto{\pgfqpoint{1.482845in}{0.642547in}}%
\pgfpathlineto{\pgfqpoint{1.483378in}{0.641507in}}%
\pgfpathlineto{\pgfqpoint{1.483912in}{0.637192in}}%
\pgfpathlineto{\pgfqpoint{1.484446in}{0.640131in}}%
\pgfpathlineto{\pgfqpoint{1.484980in}{0.652392in}}%
\pgfpathlineto{\pgfqpoint{1.485513in}{0.643814in}}%
\pgfpathlineto{\pgfqpoint{1.487114in}{0.636052in}}%
\pgfpathlineto{\pgfqpoint{1.487648in}{0.636996in}}%
\pgfpathlineto{\pgfqpoint{1.488182in}{0.644102in}}%
\pgfpathlineto{\pgfqpoint{1.488715in}{0.639682in}}%
\pgfpathlineto{\pgfqpoint{1.489249in}{0.639547in}}%
\pgfpathlineto{\pgfqpoint{1.489783in}{0.637866in}}%
\pgfpathlineto{\pgfqpoint{1.490317in}{0.638632in}}%
\pgfpathlineto{\pgfqpoint{1.491384in}{0.649220in}}%
\pgfpathlineto{\pgfqpoint{1.491918in}{0.636355in}}%
\pgfpathlineto{\pgfqpoint{1.492451in}{0.642418in}}%
\pgfpathlineto{\pgfqpoint{1.494052in}{0.639895in}}%
\pgfpathlineto{\pgfqpoint{1.495654in}{0.639400in}}%
\pgfpathlineto{\pgfqpoint{1.496187in}{0.638359in}}%
\pgfpathlineto{\pgfqpoint{1.496721in}{0.644856in}}%
\pgfpathlineto{\pgfqpoint{1.497255in}{0.639735in}}%
\pgfpathlineto{\pgfqpoint{1.497788in}{0.636725in}}%
\pgfpathlineto{\pgfqpoint{1.499389in}{0.643242in}}%
\pgfpathlineto{\pgfqpoint{1.500457in}{0.638634in}}%
\pgfpathlineto{\pgfqpoint{1.501524in}{0.643293in}}%
\pgfpathlineto{\pgfqpoint{1.502592in}{0.636150in}}%
\pgfpathlineto{\pgfqpoint{1.503125in}{0.645255in}}%
\pgfpathlineto{\pgfqpoint{1.503659in}{0.637398in}}%
\pgfpathlineto{\pgfqpoint{1.504726in}{0.641168in}}%
\pgfpathlineto{\pgfqpoint{1.505260in}{0.637726in}}%
\pgfpathlineto{\pgfqpoint{1.505794in}{0.638698in}}%
\pgfpathlineto{\pgfqpoint{1.506328in}{0.638407in}}%
\pgfpathlineto{\pgfqpoint{1.506861in}{0.650284in}}%
\pgfpathlineto{\pgfqpoint{1.507395in}{0.642789in}}%
\pgfpathlineto{\pgfqpoint{1.508462in}{0.651622in}}%
\pgfpathlineto{\pgfqpoint{1.510063in}{0.639670in}}%
\pgfpathlineto{\pgfqpoint{1.511665in}{0.644138in}}%
\pgfpathlineto{\pgfqpoint{1.512732in}{0.636751in}}%
\pgfpathlineto{\pgfqpoint{1.513266in}{0.642267in}}%
\pgfpathlineto{\pgfqpoint{1.513799in}{0.641829in}}%
\pgfpathlineto{\pgfqpoint{1.514333in}{0.639631in}}%
\pgfpathlineto{\pgfqpoint{1.514867in}{0.650461in}}%
\pgfpathlineto{\pgfqpoint{1.515400in}{0.643268in}}%
\pgfpathlineto{\pgfqpoint{1.516468in}{0.636442in}}%
\pgfpathlineto{\pgfqpoint{1.517002in}{0.639947in}}%
\pgfpathlineto{\pgfqpoint{1.518069in}{0.647079in}}%
\pgfpathlineto{\pgfqpoint{1.518603in}{0.645494in}}%
\pgfpathlineto{\pgfqpoint{1.519670in}{0.639698in}}%
\pgfpathlineto{\pgfqpoint{1.520204in}{0.640378in}}%
\pgfpathlineto{\pgfqpoint{1.522339in}{0.643630in}}%
\pgfpathlineto{\pgfqpoint{1.524473in}{0.640612in}}%
\pgfpathlineto{\pgfqpoint{1.525007in}{0.640074in}}%
\pgfpathlineto{\pgfqpoint{1.525541in}{0.635966in}}%
\pgfpathlineto{\pgfqpoint{1.526074in}{0.644701in}}%
\pgfpathlineto{\pgfqpoint{1.526608in}{0.638401in}}%
\pgfpathlineto{\pgfqpoint{1.527142in}{0.638555in}}%
\pgfpathlineto{\pgfqpoint{1.528743in}{0.643898in}}%
\pgfpathlineto{\pgfqpoint{1.529277in}{0.642522in}}%
\pgfpathlineto{\pgfqpoint{1.529810in}{0.645066in}}%
\pgfpathlineto{\pgfqpoint{1.530344in}{0.644597in}}%
\pgfpathlineto{\pgfqpoint{1.530878in}{0.643672in}}%
\pgfpathlineto{\pgfqpoint{1.531411in}{0.652289in}}%
\pgfpathlineto{\pgfqpoint{1.531945in}{0.646839in}}%
\pgfpathlineto{\pgfqpoint{1.533013in}{0.647280in}}%
\pgfpathlineto{\pgfqpoint{1.533546in}{0.642470in}}%
\pgfpathlineto{\pgfqpoint{1.534080in}{0.648953in}}%
\pgfpathlineto{\pgfqpoint{1.534614in}{0.637503in}}%
\pgfpathlineto{\pgfqpoint{1.535147in}{0.644164in}}%
\pgfpathlineto{\pgfqpoint{1.535681in}{0.650782in}}%
\pgfpathlineto{\pgfqpoint{1.536215in}{0.643158in}}%
\pgfpathlineto{\pgfqpoint{1.537282in}{0.643269in}}%
\pgfpathlineto{\pgfqpoint{1.537816in}{0.643736in}}%
\pgfpathlineto{\pgfqpoint{1.538350in}{0.639145in}}%
\pgfpathlineto{\pgfqpoint{1.538883in}{0.641242in}}%
\pgfpathlineto{\pgfqpoint{1.539417in}{0.645805in}}%
\pgfpathlineto{\pgfqpoint{1.539951in}{0.645058in}}%
\pgfpathlineto{\pgfqpoint{1.541018in}{0.636899in}}%
\pgfpathlineto{\pgfqpoint{1.541552in}{0.640914in}}%
\pgfpathlineto{\pgfqpoint{1.542085in}{0.648308in}}%
\pgfpathlineto{\pgfqpoint{1.543687in}{0.638405in}}%
\pgfpathlineto{\pgfqpoint{1.545821in}{0.647621in}}%
\pgfpathlineto{\pgfqpoint{1.546355in}{0.645809in}}%
\pgfpathlineto{\pgfqpoint{1.546889in}{0.636020in}}%
\pgfpathlineto{\pgfqpoint{1.547423in}{0.639827in}}%
\pgfpathlineto{\pgfqpoint{1.547956in}{0.639727in}}%
\pgfpathlineto{\pgfqpoint{1.548490in}{0.637603in}}%
\pgfpathlineto{\pgfqpoint{1.549557in}{0.646134in}}%
\pgfpathlineto{\pgfqpoint{1.551158in}{0.636085in}}%
\pgfpathlineto{\pgfqpoint{1.552760in}{0.648616in}}%
\pgfpathlineto{\pgfqpoint{1.554361in}{0.639594in}}%
\pgfpathlineto{\pgfqpoint{1.554894in}{0.646203in}}%
\pgfpathlineto{\pgfqpoint{1.555428in}{0.637237in}}%
\pgfpathlineto{\pgfqpoint{1.555962in}{0.641982in}}%
\pgfpathlineto{\pgfqpoint{1.557029in}{0.638143in}}%
\pgfpathlineto{\pgfqpoint{1.558630in}{0.649860in}}%
\pgfpathlineto{\pgfqpoint{1.559164in}{0.639301in}}%
\pgfpathlineto{\pgfqpoint{1.560231in}{0.639838in}}%
\pgfpathlineto{\pgfqpoint{1.560765in}{0.637440in}}%
\pgfpathlineto{\pgfqpoint{1.562366in}{0.653286in}}%
\pgfpathlineto{\pgfqpoint{1.562900in}{0.648424in}}%
\pgfpathlineto{\pgfqpoint{1.564501in}{0.639539in}}%
\pgfpathlineto{\pgfqpoint{1.565035in}{0.642588in}}%
\pgfpathlineto{\pgfqpoint{1.565568in}{0.640814in}}%
\pgfpathlineto{\pgfqpoint{1.566102in}{0.641885in}}%
\pgfpathlineto{\pgfqpoint{1.566636in}{0.646507in}}%
\pgfpathlineto{\pgfqpoint{1.567169in}{0.642249in}}%
\pgfpathlineto{\pgfqpoint{1.567703in}{0.637292in}}%
\pgfpathlineto{\pgfqpoint{1.568237in}{0.644969in}}%
\pgfpathlineto{\pgfqpoint{1.568771in}{0.643175in}}%
\pgfpathlineto{\pgfqpoint{1.570372in}{0.636959in}}%
\pgfpathlineto{\pgfqpoint{1.570905in}{0.644849in}}%
\pgfpathlineto{\pgfqpoint{1.571439in}{0.640945in}}%
\pgfpathlineto{\pgfqpoint{1.571973in}{0.639514in}}%
\pgfpathlineto{\pgfqpoint{1.573040in}{0.658078in}}%
\pgfpathlineto{\pgfqpoint{1.574641in}{0.638967in}}%
\pgfpathlineto{\pgfqpoint{1.576776in}{0.651096in}}%
\pgfpathlineto{\pgfqpoint{1.578377in}{0.640407in}}%
\pgfpathlineto{\pgfqpoint{1.578911in}{0.652384in}}%
\pgfpathlineto{\pgfqpoint{1.579445in}{0.644578in}}%
\pgfpathlineto{\pgfqpoint{1.579978in}{0.639269in}}%
\pgfpathlineto{\pgfqpoint{1.580512in}{0.644524in}}%
\pgfpathlineto{\pgfqpoint{1.581046in}{0.639942in}}%
\pgfpathlineto{\pgfqpoint{1.581579in}{0.644386in}}%
\pgfpathlineto{\pgfqpoint{1.582113in}{0.640644in}}%
\pgfpathlineto{\pgfqpoint{1.582647in}{0.648963in}}%
\pgfpathlineto{\pgfqpoint{1.583180in}{0.646743in}}%
\pgfpathlineto{\pgfqpoint{1.583714in}{0.646370in}}%
\pgfpathlineto{\pgfqpoint{1.584248in}{0.635634in}}%
\pgfpathlineto{\pgfqpoint{1.584782in}{0.646067in}}%
\pgfpathlineto{\pgfqpoint{1.585315in}{0.637587in}}%
\pgfpathlineto{\pgfqpoint{1.586916in}{0.650301in}}%
\pgfpathlineto{\pgfqpoint{1.587450in}{0.650239in}}%
\pgfpathlineto{\pgfqpoint{1.587984in}{0.639969in}}%
\pgfpathlineto{\pgfqpoint{1.588517in}{0.646120in}}%
\pgfpathlineto{\pgfqpoint{1.589051in}{0.648033in}}%
\pgfpathlineto{\pgfqpoint{1.589585in}{0.654643in}}%
\pgfpathlineto{\pgfqpoint{1.590119in}{0.637127in}}%
\pgfpathlineto{\pgfqpoint{1.590652in}{0.645170in}}%
\pgfpathlineto{\pgfqpoint{1.591186in}{0.638976in}}%
\pgfpathlineto{\pgfqpoint{1.591720in}{0.640486in}}%
\pgfpathlineto{\pgfqpoint{1.592253in}{0.650418in}}%
\pgfpathlineto{\pgfqpoint{1.592787in}{0.640984in}}%
\pgfpathlineto{\pgfqpoint{1.593854in}{0.649443in}}%
\pgfpathlineto{\pgfqpoint{1.594388in}{0.646940in}}%
\pgfpathlineto{\pgfqpoint{1.595456in}{0.640788in}}%
\pgfpathlineto{\pgfqpoint{1.595989in}{0.642222in}}%
\pgfpathlineto{\pgfqpoint{1.596523in}{0.640576in}}%
\pgfpathlineto{\pgfqpoint{1.597057in}{0.648961in}}%
\pgfpathlineto{\pgfqpoint{1.597590in}{0.639750in}}%
\pgfpathlineto{\pgfqpoint{1.598124in}{0.641614in}}%
\pgfpathlineto{\pgfqpoint{1.598658in}{0.641699in}}%
\pgfpathlineto{\pgfqpoint{1.599725in}{0.647950in}}%
\pgfpathlineto{\pgfqpoint{1.600259in}{0.650710in}}%
\pgfpathlineto{\pgfqpoint{1.600793in}{0.641265in}}%
\pgfpathlineto{\pgfqpoint{1.601326in}{0.643808in}}%
\pgfpathlineto{\pgfqpoint{1.601860in}{0.644617in}}%
\pgfpathlineto{\pgfqpoint{1.602394in}{0.636143in}}%
\pgfpathlineto{\pgfqpoint{1.602927in}{0.646410in}}%
\pgfpathlineto{\pgfqpoint{1.603461in}{0.642831in}}%
\pgfpathlineto{\pgfqpoint{1.605062in}{0.646829in}}%
\pgfpathlineto{\pgfqpoint{1.605596in}{0.638651in}}%
\pgfpathlineto{\pgfqpoint{1.606130in}{0.642427in}}%
\pgfpathlineto{\pgfqpoint{1.607197in}{0.657866in}}%
\pgfpathlineto{\pgfqpoint{1.607731in}{0.638151in}}%
\pgfpathlineto{\pgfqpoint{1.608798in}{0.638240in}}%
\pgfpathlineto{\pgfqpoint{1.609865in}{0.647555in}}%
\pgfpathlineto{\pgfqpoint{1.610399in}{0.636407in}}%
\pgfpathlineto{\pgfqpoint{1.610933in}{0.641768in}}%
\pgfpathlineto{\pgfqpoint{1.611467in}{0.647225in}}%
\pgfpathlineto{\pgfqpoint{1.612534in}{0.637178in}}%
\pgfpathlineto{\pgfqpoint{1.613068in}{0.647102in}}%
\pgfpathlineto{\pgfqpoint{1.613601in}{0.646723in}}%
\pgfpathlineto{\pgfqpoint{1.614135in}{0.641205in}}%
\pgfpathlineto{\pgfqpoint{1.614669in}{0.646282in}}%
\pgfpathlineto{\pgfqpoint{1.615736in}{0.637829in}}%
\pgfpathlineto{\pgfqpoint{1.617337in}{0.653854in}}%
\pgfpathlineto{\pgfqpoint{1.618938in}{0.643171in}}%
\pgfpathlineto{\pgfqpoint{1.619472in}{0.649330in}}%
\pgfpathlineto{\pgfqpoint{1.620006in}{0.639685in}}%
\pgfpathlineto{\pgfqpoint{1.620540in}{0.645119in}}%
\pgfpathlineto{\pgfqpoint{1.621073in}{0.650856in}}%
\pgfpathlineto{\pgfqpoint{1.621607in}{0.645214in}}%
\pgfpathlineto{\pgfqpoint{1.622141in}{0.636407in}}%
\pgfpathlineto{\pgfqpoint{1.622674in}{0.638187in}}%
\pgfpathlineto{\pgfqpoint{1.623742in}{0.646779in}}%
\pgfpathlineto{\pgfqpoint{1.624275in}{0.642862in}}%
\pgfpathlineto{\pgfqpoint{1.624809in}{0.638673in}}%
\pgfpathlineto{\pgfqpoint{1.625343in}{0.649168in}}%
\pgfpathlineto{\pgfqpoint{1.625877in}{0.645480in}}%
\pgfpathlineto{\pgfqpoint{1.626944in}{0.642105in}}%
\pgfpathlineto{\pgfqpoint{1.629079in}{0.657049in}}%
\pgfpathlineto{\pgfqpoint{1.630680in}{0.641211in}}%
\pgfpathlineto{\pgfqpoint{1.631214in}{0.640441in}}%
\pgfpathlineto{\pgfqpoint{1.632815in}{0.653343in}}%
\pgfpathlineto{\pgfqpoint{1.633348in}{0.642072in}}%
\pgfpathlineto{\pgfqpoint{1.633882in}{0.647051in}}%
\pgfpathlineto{\pgfqpoint{1.634416in}{0.647092in}}%
\pgfpathlineto{\pgfqpoint{1.634949in}{0.638008in}}%
\pgfpathlineto{\pgfqpoint{1.635483in}{0.649728in}}%
\pgfpathlineto{\pgfqpoint{1.636017in}{0.640844in}}%
\pgfpathlineto{\pgfqpoint{1.636551in}{0.641307in}}%
\pgfpathlineto{\pgfqpoint{1.637084in}{0.643284in}}%
\pgfpathlineto{\pgfqpoint{1.637618in}{0.642923in}}%
\pgfpathlineto{\pgfqpoint{1.638152in}{0.639741in}}%
\pgfpathlineto{\pgfqpoint{1.639753in}{0.649364in}}%
\pgfpathlineto{\pgfqpoint{1.640286in}{0.644002in}}%
\pgfpathlineto{\pgfqpoint{1.640820in}{0.646675in}}%
\pgfpathlineto{\pgfqpoint{1.641354in}{0.655459in}}%
\pgfpathlineto{\pgfqpoint{1.641888in}{0.646856in}}%
\pgfpathlineto{\pgfqpoint{1.642421in}{0.651438in}}%
\pgfpathlineto{\pgfqpoint{1.643489in}{0.652335in}}%
\pgfpathlineto{\pgfqpoint{1.644022in}{0.639879in}}%
\pgfpathlineto{\pgfqpoint{1.644556in}{0.657710in}}%
\pgfpathlineto{\pgfqpoint{1.645090in}{0.645950in}}%
\pgfpathlineto{\pgfqpoint{1.645623in}{0.644924in}}%
\pgfpathlineto{\pgfqpoint{1.647225in}{0.655177in}}%
\pgfpathlineto{\pgfqpoint{1.647758in}{0.645613in}}%
\pgfpathlineto{\pgfqpoint{1.648826in}{0.645841in}}%
\pgfpathlineto{\pgfqpoint{1.649359in}{0.645516in}}%
\pgfpathlineto{\pgfqpoint{1.649893in}{0.641173in}}%
\pgfpathlineto{\pgfqpoint{1.650427in}{0.641979in}}%
\pgfpathlineto{\pgfqpoint{1.652028in}{0.649265in}}%
\pgfpathlineto{\pgfqpoint{1.652562in}{0.648589in}}%
\pgfpathlineto{\pgfqpoint{1.653095in}{0.637558in}}%
\pgfpathlineto{\pgfqpoint{1.653629in}{0.641541in}}%
\pgfpathlineto{\pgfqpoint{1.654163in}{0.648170in}}%
\pgfpathlineto{\pgfqpoint{1.654696in}{0.646800in}}%
\pgfpathlineto{\pgfqpoint{1.655764in}{0.637545in}}%
\pgfpathlineto{\pgfqpoint{1.656831in}{0.655205in}}%
\pgfpathlineto{\pgfqpoint{1.657365in}{0.644284in}}%
\pgfpathlineto{\pgfqpoint{1.657899in}{0.651708in}}%
\pgfpathlineto{\pgfqpoint{1.659500in}{0.637831in}}%
\pgfpathlineto{\pgfqpoint{1.660033in}{0.648475in}}%
\pgfpathlineto{\pgfqpoint{1.660567in}{0.644953in}}%
\pgfpathlineto{\pgfqpoint{1.661634in}{0.641375in}}%
\pgfpathlineto{\pgfqpoint{1.662168in}{0.649195in}}%
\pgfpathlineto{\pgfqpoint{1.663236in}{0.637688in}}%
\pgfpathlineto{\pgfqpoint{1.664303in}{0.653532in}}%
\pgfpathlineto{\pgfqpoint{1.664837in}{0.650916in}}%
\pgfpathlineto{\pgfqpoint{1.665904in}{0.636719in}}%
\pgfpathlineto{\pgfqpoint{1.666971in}{0.649409in}}%
\pgfpathlineto{\pgfqpoint{1.668039in}{0.641582in}}%
\pgfpathlineto{\pgfqpoint{1.670174in}{0.657221in}}%
\pgfpathlineto{\pgfqpoint{1.670707in}{0.639187in}}%
\pgfpathlineto{\pgfqpoint{1.671241in}{0.646614in}}%
\pgfpathlineto{\pgfqpoint{1.672842in}{0.656138in}}%
\pgfpathlineto{\pgfqpoint{1.674977in}{0.643899in}}%
\pgfpathlineto{\pgfqpoint{1.676578in}{0.649762in}}%
\pgfpathlineto{\pgfqpoint{1.678179in}{0.641276in}}%
\pgfpathlineto{\pgfqpoint{1.678713in}{0.641105in}}%
\pgfpathlineto{\pgfqpoint{1.679247in}{0.642454in}}%
\pgfpathlineto{\pgfqpoint{1.679780in}{0.639092in}}%
\pgfpathlineto{\pgfqpoint{1.681381in}{0.654609in}}%
\pgfpathlineto{\pgfqpoint{1.681915in}{0.638961in}}%
\pgfpathlineto{\pgfqpoint{1.682449in}{0.643019in}}%
\pgfpathlineto{\pgfqpoint{1.683516in}{0.655920in}}%
\pgfpathlineto{\pgfqpoint{1.685117in}{0.637864in}}%
\pgfpathlineto{\pgfqpoint{1.686185in}{0.651617in}}%
\pgfpathlineto{\pgfqpoint{1.686718in}{0.647300in}}%
\pgfpathlineto{\pgfqpoint{1.687252in}{0.656281in}}%
\pgfpathlineto{\pgfqpoint{1.687786in}{0.650140in}}%
\pgfpathlineto{\pgfqpoint{1.688320in}{0.641807in}}%
\pgfpathlineto{\pgfqpoint{1.688853in}{0.650277in}}%
\pgfpathlineto{\pgfqpoint{1.690454in}{0.655014in}}%
\pgfpathlineto{\pgfqpoint{1.691522in}{0.650777in}}%
\pgfpathlineto{\pgfqpoint{1.692055in}{0.641173in}}%
\pgfpathlineto{\pgfqpoint{1.692589in}{0.646183in}}%
\pgfpathlineto{\pgfqpoint{1.694190in}{0.639920in}}%
\pgfpathlineto{\pgfqpoint{1.695791in}{0.651465in}}%
\pgfpathlineto{\pgfqpoint{1.696325in}{0.651518in}}%
\pgfpathlineto{\pgfqpoint{1.696859in}{0.643732in}}%
\pgfpathlineto{\pgfqpoint{1.697392in}{0.661104in}}%
\pgfpathlineto{\pgfqpoint{1.697926in}{0.647674in}}%
\pgfpathlineto{\pgfqpoint{1.698460in}{0.654466in}}%
\pgfpathlineto{\pgfqpoint{1.698994in}{0.639358in}}%
\pgfpathlineto{\pgfqpoint{1.699527in}{0.655002in}}%
\pgfpathlineto{\pgfqpoint{1.700595in}{0.636915in}}%
\pgfpathlineto{\pgfqpoint{1.701662in}{0.643425in}}%
\pgfpathlineto{\pgfqpoint{1.702196in}{0.648013in}}%
\pgfpathlineto{\pgfqpoint{1.702729in}{0.642598in}}%
\pgfpathlineto{\pgfqpoint{1.703263in}{0.643337in}}%
\pgfpathlineto{\pgfqpoint{1.704331in}{0.651628in}}%
\pgfpathlineto{\pgfqpoint{1.704864in}{0.646526in}}%
\pgfpathlineto{\pgfqpoint{1.705398in}{0.653382in}}%
\pgfpathlineto{\pgfqpoint{1.705932in}{0.642571in}}%
\pgfpathlineto{\pgfqpoint{1.706465in}{0.644526in}}%
\pgfpathlineto{\pgfqpoint{1.706999in}{0.649496in}}%
\pgfpathlineto{\pgfqpoint{1.707533in}{0.642279in}}%
\pgfpathlineto{\pgfqpoint{1.708066in}{0.649566in}}%
\pgfpathlineto{\pgfqpoint{1.709134in}{0.641970in}}%
\pgfpathlineto{\pgfqpoint{1.710735in}{0.648809in}}%
\pgfpathlineto{\pgfqpoint{1.711269in}{0.649857in}}%
\pgfpathlineto{\pgfqpoint{1.711802in}{0.670652in}}%
\pgfpathlineto{\pgfqpoint{1.712336in}{0.643183in}}%
\pgfpathlineto{\pgfqpoint{1.712870in}{0.644121in}}%
\pgfpathlineto{\pgfqpoint{1.713403in}{0.654676in}}%
\pgfpathlineto{\pgfqpoint{1.713937in}{0.645562in}}%
\pgfpathlineto{\pgfqpoint{1.714471in}{0.642666in}}%
\pgfpathlineto{\pgfqpoint{1.715538in}{0.657872in}}%
\pgfpathlineto{\pgfqpoint{1.716072in}{0.637949in}}%
\pgfpathlineto{\pgfqpoint{1.716606in}{0.653683in}}%
\pgfpathlineto{\pgfqpoint{1.717673in}{0.642282in}}%
\pgfpathlineto{\pgfqpoint{1.718207in}{0.644396in}}%
\pgfpathlineto{\pgfqpoint{1.718740in}{0.659825in}}%
\pgfpathlineto{\pgfqpoint{1.719274in}{0.653489in}}%
\pgfpathlineto{\pgfqpoint{1.719808in}{0.652411in}}%
\pgfpathlineto{\pgfqpoint{1.720342in}{0.636692in}}%
\pgfpathlineto{\pgfqpoint{1.720875in}{0.646950in}}%
\pgfpathlineto{\pgfqpoint{1.721409in}{0.640447in}}%
\pgfpathlineto{\pgfqpoint{1.721943in}{0.645467in}}%
\pgfpathlineto{\pgfqpoint{1.723010in}{0.639069in}}%
\pgfpathlineto{\pgfqpoint{1.723544in}{0.648828in}}%
\pgfpathlineto{\pgfqpoint{1.724077in}{0.641514in}}%
\pgfpathlineto{\pgfqpoint{1.724611in}{0.640177in}}%
\pgfpathlineto{\pgfqpoint{1.726212in}{0.652820in}}%
\pgfpathlineto{\pgfqpoint{1.726746in}{0.659066in}}%
\pgfpathlineto{\pgfqpoint{1.727280in}{0.641682in}}%
\pgfpathlineto{\pgfqpoint{1.727813in}{0.661519in}}%
\pgfpathlineto{\pgfqpoint{1.728347in}{0.660293in}}%
\pgfpathlineto{\pgfqpoint{1.729414in}{0.645819in}}%
\pgfpathlineto{\pgfqpoint{1.729948in}{0.647492in}}%
\pgfpathlineto{\pgfqpoint{1.731016in}{0.640387in}}%
\pgfpathlineto{\pgfqpoint{1.731549in}{0.661224in}}%
\pgfpathlineto{\pgfqpoint{1.732083in}{0.645776in}}%
\pgfpathlineto{\pgfqpoint{1.733150in}{0.641805in}}%
\pgfpathlineto{\pgfqpoint{1.733684in}{0.642061in}}%
\pgfpathlineto{\pgfqpoint{1.734751in}{0.643522in}}%
\pgfpathlineto{\pgfqpoint{1.736886in}{0.657249in}}%
\pgfpathlineto{\pgfqpoint{1.737420in}{0.641276in}}%
\pgfpathlineto{\pgfqpoint{1.737954in}{0.644899in}}%
\pgfpathlineto{\pgfqpoint{1.739021in}{0.658574in}}%
\pgfpathlineto{\pgfqpoint{1.739555in}{0.649623in}}%
\pgfpathlineto{\pgfqpoint{1.740088in}{0.636955in}}%
\pgfpathlineto{\pgfqpoint{1.740622in}{0.637095in}}%
\pgfpathlineto{\pgfqpoint{1.741156in}{0.641658in}}%
\pgfpathlineto{\pgfqpoint{1.741690in}{0.637918in}}%
\pgfpathlineto{\pgfqpoint{1.742223in}{0.638757in}}%
\pgfpathlineto{\pgfqpoint{1.743291in}{0.657210in}}%
\pgfpathlineto{\pgfqpoint{1.743824in}{0.638638in}}%
\pgfpathlineto{\pgfqpoint{1.744358in}{0.654451in}}%
\pgfpathlineto{\pgfqpoint{1.745425in}{0.640642in}}%
\pgfpathlineto{\pgfqpoint{1.745959in}{0.649383in}}%
\pgfpathlineto{\pgfqpoint{1.746493in}{0.648477in}}%
\pgfpathlineto{\pgfqpoint{1.747027in}{0.646211in}}%
\pgfpathlineto{\pgfqpoint{1.747560in}{0.663075in}}%
\pgfpathlineto{\pgfqpoint{1.748094in}{0.653953in}}%
\pgfpathlineto{\pgfqpoint{1.748628in}{0.658745in}}%
\pgfpathlineto{\pgfqpoint{1.750229in}{0.638423in}}%
\pgfpathlineto{\pgfqpoint{1.750763in}{0.654149in}}%
\pgfpathlineto{\pgfqpoint{1.751830in}{0.653365in}}%
\pgfpathlineto{\pgfqpoint{1.752364in}{0.645951in}}%
\pgfpathlineto{\pgfqpoint{1.752897in}{0.647781in}}%
\pgfpathlineto{\pgfqpoint{1.753965in}{0.647715in}}%
\pgfpathlineto{\pgfqpoint{1.754498in}{0.663853in}}%
\pgfpathlineto{\pgfqpoint{1.755032in}{0.659790in}}%
\pgfpathlineto{\pgfqpoint{1.755566in}{0.637612in}}%
\pgfpathlineto{\pgfqpoint{1.756100in}{0.650641in}}%
\pgfpathlineto{\pgfqpoint{1.756633in}{0.652645in}}%
\pgfpathlineto{\pgfqpoint{1.758234in}{0.646162in}}%
\pgfpathlineto{\pgfqpoint{1.758768in}{0.646752in}}%
\pgfpathlineto{\pgfqpoint{1.759302in}{0.659404in}}%
\pgfpathlineto{\pgfqpoint{1.759835in}{0.649733in}}%
\pgfpathlineto{\pgfqpoint{1.760369in}{0.645166in}}%
\pgfpathlineto{\pgfqpoint{1.760903in}{0.658145in}}%
\pgfpathlineto{\pgfqpoint{1.761437in}{0.644697in}}%
\pgfpathlineto{\pgfqpoint{1.761970in}{0.644221in}}%
\pgfpathlineto{\pgfqpoint{1.762504in}{0.651779in}}%
\pgfpathlineto{\pgfqpoint{1.763038in}{0.650050in}}%
\pgfpathlineto{\pgfqpoint{1.763571in}{0.643770in}}%
\pgfpathlineto{\pgfqpoint{1.764105in}{0.651327in}}%
\pgfpathlineto{\pgfqpoint{1.764639in}{0.639810in}}%
\pgfpathlineto{\pgfqpoint{1.765172in}{0.640480in}}%
\pgfpathlineto{\pgfqpoint{1.765706in}{0.652632in}}%
\pgfpathlineto{\pgfqpoint{1.766240in}{0.642836in}}%
\pgfpathlineto{\pgfqpoint{1.766774in}{0.649351in}}%
\pgfpathlineto{\pgfqpoint{1.767307in}{0.642817in}}%
\pgfpathlineto{\pgfqpoint{1.767841in}{0.637733in}}%
\pgfpathlineto{\pgfqpoint{1.768375in}{0.642792in}}%
\pgfpathlineto{\pgfqpoint{1.768908in}{0.656019in}}%
\pgfpathlineto{\pgfqpoint{1.769442in}{0.650681in}}%
\pgfpathlineto{\pgfqpoint{1.769976in}{0.644594in}}%
\pgfpathlineto{\pgfqpoint{1.770509in}{0.656690in}}%
\pgfpathlineto{\pgfqpoint{1.771043in}{0.639005in}}%
\pgfpathlineto{\pgfqpoint{1.771577in}{0.643011in}}%
\pgfpathlineto{\pgfqpoint{1.772644in}{0.646948in}}%
\pgfpathlineto{\pgfqpoint{1.773178in}{0.643201in}}%
\pgfpathlineto{\pgfqpoint{1.773712in}{0.650543in}}%
\pgfpathlineto{\pgfqpoint{1.774245in}{0.648260in}}%
\pgfpathlineto{\pgfqpoint{1.774779in}{0.648200in}}%
\pgfpathlineto{\pgfqpoint{1.775846in}{0.661651in}}%
\pgfpathlineto{\pgfqpoint{1.776914in}{0.645101in}}%
\pgfpathlineto{\pgfqpoint{1.777448in}{0.645906in}}%
\pgfpathlineto{\pgfqpoint{1.777981in}{0.649996in}}%
\pgfpathlineto{\pgfqpoint{1.778515in}{0.643866in}}%
\pgfpathlineto{\pgfqpoint{1.779049in}{0.647144in}}%
\pgfpathlineto{\pgfqpoint{1.780116in}{0.645398in}}%
\pgfpathlineto{\pgfqpoint{1.780650in}{0.660976in}}%
\pgfpathlineto{\pgfqpoint{1.781183in}{0.644562in}}%
\pgfpathlineto{\pgfqpoint{1.782251in}{0.661638in}}%
\pgfpathlineto{\pgfqpoint{1.783852in}{0.639818in}}%
\pgfpathlineto{\pgfqpoint{1.785453in}{0.652759in}}%
\pgfpathlineto{\pgfqpoint{1.785987in}{0.644241in}}%
\pgfpathlineto{\pgfqpoint{1.786520in}{0.652553in}}%
\pgfpathlineto{\pgfqpoint{1.787054in}{0.658490in}}%
\pgfpathlineto{\pgfqpoint{1.787588in}{0.641177in}}%
\pgfpathlineto{\pgfqpoint{1.788122in}{0.646785in}}%
\pgfpathlineto{\pgfqpoint{1.788655in}{0.646049in}}%
\pgfpathlineto{\pgfqpoint{1.789189in}{0.657718in}}%
\pgfpathlineto{\pgfqpoint{1.789723in}{0.647097in}}%
\pgfpathlineto{\pgfqpoint{1.790256in}{0.647535in}}%
\pgfpathlineto{\pgfqpoint{1.790790in}{0.644558in}}%
\pgfpathlineto{\pgfqpoint{1.791857in}{0.658897in}}%
\pgfpathlineto{\pgfqpoint{1.792391in}{0.643390in}}%
\pgfpathlineto{\pgfqpoint{1.792925in}{0.643808in}}%
\pgfpathlineto{\pgfqpoint{1.793459in}{0.660830in}}%
\pgfpathlineto{\pgfqpoint{1.793992in}{0.644226in}}%
\pgfpathlineto{\pgfqpoint{1.794526in}{0.649216in}}%
\pgfpathlineto{\pgfqpoint{1.795060in}{0.639144in}}%
\pgfpathlineto{\pgfqpoint{1.795593in}{0.642111in}}%
\pgfpathlineto{\pgfqpoint{1.796127in}{0.645632in}}%
\pgfpathlineto{\pgfqpoint{1.796661in}{0.643641in}}%
\pgfpathlineto{\pgfqpoint{1.797194in}{0.636135in}}%
\pgfpathlineto{\pgfqpoint{1.797728in}{0.652496in}}%
\pgfpathlineto{\pgfqpoint{1.798796in}{0.652421in}}%
\pgfpathlineto{\pgfqpoint{1.799329in}{0.646560in}}%
\pgfpathlineto{\pgfqpoint{1.799863in}{0.654226in}}%
\pgfpathlineto{\pgfqpoint{1.800397in}{0.650258in}}%
\pgfpathlineto{\pgfqpoint{1.800930in}{0.647094in}}%
\pgfpathlineto{\pgfqpoint{1.801464in}{0.650114in}}%
\pgfpathlineto{\pgfqpoint{1.801998in}{0.649082in}}%
\pgfpathlineto{\pgfqpoint{1.802531in}{0.643893in}}%
\pgfpathlineto{\pgfqpoint{1.803065in}{0.649986in}}%
\pgfpathlineto{\pgfqpoint{1.803599in}{0.640604in}}%
\pgfpathlineto{\pgfqpoint{1.804133in}{0.650746in}}%
\pgfpathlineto{\pgfqpoint{1.804666in}{0.652849in}}%
\pgfpathlineto{\pgfqpoint{1.805200in}{0.674393in}}%
\pgfpathlineto{\pgfqpoint{1.805734in}{0.656188in}}%
\pgfpathlineto{\pgfqpoint{1.806267in}{0.640905in}}%
\pgfpathlineto{\pgfqpoint{1.806801in}{0.653845in}}%
\pgfpathlineto{\pgfqpoint{1.807335in}{0.654965in}}%
\pgfpathlineto{\pgfqpoint{1.808402in}{0.640677in}}%
\pgfpathlineto{\pgfqpoint{1.810003in}{0.671469in}}%
\pgfpathlineto{\pgfqpoint{1.810537in}{0.637670in}}%
\pgfpathlineto{\pgfqpoint{1.811071in}{0.661240in}}%
\pgfpathlineto{\pgfqpoint{1.811604in}{0.665725in}}%
\pgfpathlineto{\pgfqpoint{1.812138in}{0.640514in}}%
\pgfpathlineto{\pgfqpoint{1.812672in}{0.655507in}}%
\pgfpathlineto{\pgfqpoint{1.813205in}{0.648098in}}%
\pgfpathlineto{\pgfqpoint{1.813739in}{0.652510in}}%
\pgfpathlineto{\pgfqpoint{1.814807in}{0.649676in}}%
\pgfpathlineto{\pgfqpoint{1.815340in}{0.640462in}}%
\pgfpathlineto{\pgfqpoint{1.815874in}{0.655221in}}%
\pgfpathlineto{\pgfqpoint{1.816408in}{0.637402in}}%
\pgfpathlineto{\pgfqpoint{1.816941in}{0.640708in}}%
\pgfpathlineto{\pgfqpoint{1.818009in}{0.668012in}}%
\pgfpathlineto{\pgfqpoint{1.819076in}{0.642501in}}%
\pgfpathlineto{\pgfqpoint{1.819610in}{0.659132in}}%
\pgfpathlineto{\pgfqpoint{1.820144in}{0.645418in}}%
\pgfpathlineto{\pgfqpoint{1.821211in}{0.650760in}}%
\pgfpathlineto{\pgfqpoint{1.821745in}{0.641648in}}%
\pgfpathlineto{\pgfqpoint{1.822278in}{0.646219in}}%
\pgfpathlineto{\pgfqpoint{1.822812in}{0.648082in}}%
\pgfpathlineto{\pgfqpoint{1.823346in}{0.654340in}}%
\pgfpathlineto{\pgfqpoint{1.823880in}{0.651100in}}%
\pgfpathlineto{\pgfqpoint{1.824413in}{0.650077in}}%
\pgfpathlineto{\pgfqpoint{1.824947in}{0.637492in}}%
\pgfpathlineto{\pgfqpoint{1.825481in}{0.651648in}}%
\pgfpathlineto{\pgfqpoint{1.826548in}{0.651473in}}%
\pgfpathlineto{\pgfqpoint{1.827082in}{0.658717in}}%
\pgfpathlineto{\pgfqpoint{1.827615in}{0.652303in}}%
\pgfpathlineto{\pgfqpoint{1.828149in}{0.641045in}}%
\pgfpathlineto{\pgfqpoint{1.828683in}{0.646033in}}%
\pgfpathlineto{\pgfqpoint{1.829217in}{0.653192in}}%
\pgfpathlineto{\pgfqpoint{1.829750in}{0.642608in}}%
\pgfpathlineto{\pgfqpoint{1.830284in}{0.664985in}}%
\pgfpathlineto{\pgfqpoint{1.830818in}{0.654426in}}%
\pgfpathlineto{\pgfqpoint{1.831351in}{0.652328in}}%
\pgfpathlineto{\pgfqpoint{1.831885in}{0.642391in}}%
\pgfpathlineto{\pgfqpoint{1.832419in}{0.653804in}}%
\pgfpathlineto{\pgfqpoint{1.832952in}{0.648099in}}%
\pgfpathlineto{\pgfqpoint{1.833486in}{0.648601in}}%
\pgfpathlineto{\pgfqpoint{1.834020in}{0.651048in}}%
\pgfpathlineto{\pgfqpoint{1.835087in}{0.639093in}}%
\pgfpathlineto{\pgfqpoint{1.836155in}{0.666824in}}%
\pgfpathlineto{\pgfqpoint{1.836688in}{0.661107in}}%
\pgfpathlineto{\pgfqpoint{1.837222in}{0.660517in}}%
\pgfpathlineto{\pgfqpoint{1.837756in}{0.644444in}}%
\pgfpathlineto{\pgfqpoint{1.838289in}{0.654446in}}%
\pgfpathlineto{\pgfqpoint{1.838823in}{0.659748in}}%
\pgfpathlineto{\pgfqpoint{1.839357in}{0.656921in}}%
\pgfpathlineto{\pgfqpoint{1.840424in}{0.643137in}}%
\pgfpathlineto{\pgfqpoint{1.842025in}{0.662605in}}%
\pgfpathlineto{\pgfqpoint{1.843626in}{0.645960in}}%
\pgfpathlineto{\pgfqpoint{1.844160in}{0.646708in}}%
\pgfpathlineto{\pgfqpoint{1.844694in}{0.653607in}}%
\pgfpathlineto{\pgfqpoint{1.845228in}{0.639901in}}%
\pgfpathlineto{\pgfqpoint{1.845761in}{0.650201in}}%
\pgfpathlineto{\pgfqpoint{1.846295in}{0.649896in}}%
\pgfpathlineto{\pgfqpoint{1.846829in}{0.661913in}}%
\pgfpathlineto{\pgfqpoint{1.847362in}{0.649542in}}%
\pgfpathlineto{\pgfqpoint{1.847896in}{0.651627in}}%
\pgfpathlineto{\pgfqpoint{1.848430in}{0.648993in}}%
\pgfpathlineto{\pgfqpoint{1.848963in}{0.658933in}}%
\pgfpathlineto{\pgfqpoint{1.849497in}{0.653370in}}%
\pgfpathlineto{\pgfqpoint{1.850031in}{0.650381in}}%
\pgfpathlineto{\pgfqpoint{1.850565in}{0.651464in}}%
\pgfpathlineto{\pgfqpoint{1.851098in}{0.658348in}}%
\pgfpathlineto{\pgfqpoint{1.851632in}{0.636292in}}%
\pgfpathlineto{\pgfqpoint{1.852166in}{0.639589in}}%
\pgfpathlineto{\pgfqpoint{1.853767in}{0.654153in}}%
\pgfpathlineto{\pgfqpoint{1.854300in}{0.639735in}}%
\pgfpathlineto{\pgfqpoint{1.854834in}{0.661838in}}%
\pgfpathlineto{\pgfqpoint{1.855368in}{0.651218in}}%
\pgfpathlineto{\pgfqpoint{1.855902in}{0.655067in}}%
\pgfpathlineto{\pgfqpoint{1.856435in}{0.644473in}}%
\pgfpathlineto{\pgfqpoint{1.856969in}{0.647996in}}%
\pgfpathlineto{\pgfqpoint{1.857503in}{0.647901in}}%
\pgfpathlineto{\pgfqpoint{1.858570in}{0.655180in}}%
\pgfpathlineto{\pgfqpoint{1.859104in}{0.654836in}}%
\pgfpathlineto{\pgfqpoint{1.859637in}{0.657081in}}%
\pgfpathlineto{\pgfqpoint{1.860171in}{0.640538in}}%
\pgfpathlineto{\pgfqpoint{1.860705in}{0.643246in}}%
\pgfpathlineto{\pgfqpoint{1.861239in}{0.662112in}}%
\pgfpathlineto{\pgfqpoint{1.861772in}{0.661711in}}%
\pgfpathlineto{\pgfqpoint{1.863373in}{0.653155in}}%
\pgfpathlineto{\pgfqpoint{1.863907in}{0.680878in}}%
\pgfpathlineto{\pgfqpoint{1.864441in}{0.652709in}}%
\pgfpathlineto{\pgfqpoint{1.864974in}{0.654819in}}%
\pgfpathlineto{\pgfqpoint{1.866042in}{0.642143in}}%
\pgfpathlineto{\pgfqpoint{1.866576in}{0.686217in}}%
\pgfpathlineto{\pgfqpoint{1.867109in}{0.654394in}}%
\pgfpathlineto{\pgfqpoint{1.867643in}{0.662309in}}%
\pgfpathlineto{\pgfqpoint{1.868177in}{0.654650in}}%
\pgfpathlineto{\pgfqpoint{1.869244in}{0.646174in}}%
\pgfpathlineto{\pgfqpoint{1.869778in}{0.662197in}}%
\pgfpathlineto{\pgfqpoint{1.870311in}{0.657401in}}%
\pgfpathlineto{\pgfqpoint{1.871913in}{0.646962in}}%
\pgfpathlineto{\pgfqpoint{1.872980in}{0.666941in}}%
\pgfpathlineto{\pgfqpoint{1.874047in}{0.640881in}}%
\pgfpathlineto{\pgfqpoint{1.874581in}{0.643587in}}%
\pgfpathlineto{\pgfqpoint{1.875115in}{0.671903in}}%
\pgfpathlineto{\pgfqpoint{1.875648in}{0.638172in}}%
\pgfpathlineto{\pgfqpoint{1.876182in}{0.639515in}}%
\pgfpathlineto{\pgfqpoint{1.876716in}{0.659412in}}%
\pgfpathlineto{\pgfqpoint{1.877250in}{0.642389in}}%
\pgfpathlineto{\pgfqpoint{1.878317in}{0.641644in}}%
\pgfpathlineto{\pgfqpoint{1.878851in}{0.658346in}}%
\pgfpathlineto{\pgfqpoint{1.879384in}{0.659815in}}%
\pgfpathlineto{\pgfqpoint{1.879918in}{0.646986in}}%
\pgfpathlineto{\pgfqpoint{1.880452in}{0.665854in}}%
\pgfpathlineto{\pgfqpoint{1.880985in}{0.655320in}}%
\pgfpathlineto{\pgfqpoint{1.881519in}{0.661874in}}%
\pgfpathlineto{\pgfqpoint{1.882587in}{0.643816in}}%
\pgfpathlineto{\pgfqpoint{1.883120in}{0.653919in}}%
\pgfpathlineto{\pgfqpoint{1.883654in}{0.642478in}}%
\pgfpathlineto{\pgfqpoint{1.884188in}{0.643830in}}%
\pgfpathlineto{\pgfqpoint{1.884721in}{0.656734in}}%
\pgfpathlineto{\pgfqpoint{1.885255in}{0.650169in}}%
\pgfpathlineto{\pgfqpoint{1.885789in}{0.646131in}}%
\pgfpathlineto{\pgfqpoint{1.887390in}{0.679712in}}%
\pgfpathlineto{\pgfqpoint{1.888991in}{0.643979in}}%
\pgfpathlineto{\pgfqpoint{1.891126in}{0.665891in}}%
\pgfpathlineto{\pgfqpoint{1.891660in}{0.648387in}}%
\pgfpathlineto{\pgfqpoint{1.892193in}{0.672442in}}%
\pgfpathlineto{\pgfqpoint{1.892727in}{0.653768in}}%
\pgfpathlineto{\pgfqpoint{1.893261in}{0.651724in}}%
\pgfpathlineto{\pgfqpoint{1.893794in}{0.657279in}}%
\pgfpathlineto{\pgfqpoint{1.894328in}{0.642570in}}%
\pgfpathlineto{\pgfqpoint{1.894862in}{0.656401in}}%
\pgfpathlineto{\pgfqpoint{1.895929in}{0.664774in}}%
\pgfpathlineto{\pgfqpoint{1.896463in}{0.640623in}}%
\pgfpathlineto{\pgfqpoint{1.896997in}{0.687136in}}%
\pgfpathlineto{\pgfqpoint{1.897530in}{0.663737in}}%
\pgfpathlineto{\pgfqpoint{1.898064in}{0.658974in}}%
\pgfpathlineto{\pgfqpoint{1.898598in}{0.643766in}}%
\pgfpathlineto{\pgfqpoint{1.899131in}{0.650627in}}%
\pgfpathlineto{\pgfqpoint{1.899665in}{0.651023in}}%
\pgfpathlineto{\pgfqpoint{1.900199in}{0.642463in}}%
\pgfpathlineto{\pgfqpoint{1.900732in}{0.645688in}}%
\pgfpathlineto{\pgfqpoint{1.901800in}{0.667177in}}%
\pgfpathlineto{\pgfqpoint{1.902867in}{0.640992in}}%
\pgfpathlineto{\pgfqpoint{1.903401in}{0.664942in}}%
\pgfpathlineto{\pgfqpoint{1.903935in}{0.657084in}}%
\pgfpathlineto{\pgfqpoint{1.904468in}{0.642838in}}%
\pgfpathlineto{\pgfqpoint{1.905002in}{0.677678in}}%
\pgfpathlineto{\pgfqpoint{1.905536in}{0.650973in}}%
\pgfpathlineto{\pgfqpoint{1.907137in}{0.661895in}}%
\pgfpathlineto{\pgfqpoint{1.908204in}{0.661135in}}%
\pgfpathlineto{\pgfqpoint{1.908738in}{0.646453in}}%
\pgfpathlineto{\pgfqpoint{1.909272in}{0.671787in}}%
\pgfpathlineto{\pgfqpoint{1.909805in}{0.671520in}}%
\pgfpathlineto{\pgfqpoint{1.910339in}{0.650329in}}%
\pgfpathlineto{\pgfqpoint{1.910873in}{0.676190in}}%
\pgfpathlineto{\pgfqpoint{1.911406in}{0.664677in}}%
\pgfpathlineto{\pgfqpoint{1.911940in}{0.657914in}}%
\pgfpathlineto{\pgfqpoint{1.912474in}{0.666598in}}%
\pgfpathlineto{\pgfqpoint{1.913008in}{0.661984in}}%
\pgfpathlineto{\pgfqpoint{1.914609in}{0.650413in}}%
\pgfpathlineto{\pgfqpoint{1.916210in}{0.675107in}}%
\pgfpathlineto{\pgfqpoint{1.916743in}{0.674008in}}%
\pgfpathlineto{\pgfqpoint{1.917277in}{0.654161in}}%
\pgfpathlineto{\pgfqpoint{1.917811in}{0.660683in}}%
\pgfpathlineto{\pgfqpoint{1.919412in}{0.680667in}}%
\pgfpathlineto{\pgfqpoint{1.919946in}{0.643044in}}%
\pgfpathlineto{\pgfqpoint{1.920479in}{0.702731in}}%
\pgfpathlineto{\pgfqpoint{1.921013in}{0.698677in}}%
\pgfpathlineto{\pgfqpoint{1.921547in}{0.671439in}}%
\pgfpathlineto{\pgfqpoint{1.922080in}{0.703145in}}%
\pgfpathlineto{\pgfqpoint{1.922614in}{0.653003in}}%
\pgfpathlineto{\pgfqpoint{1.923682in}{0.717371in}}%
\pgfpathlineto{\pgfqpoint{1.924215in}{0.690088in}}%
\pgfpathlineto{\pgfqpoint{1.924749in}{0.723561in}}%
\pgfpathlineto{\pgfqpoint{1.925283in}{0.693877in}}%
\pgfpathlineto{\pgfqpoint{1.927417in}{0.738860in}}%
\pgfpathlineto{\pgfqpoint{1.929552in}{0.849312in}}%
\pgfpathlineto{\pgfqpoint{1.930620in}{0.832177in}}%
\pgfpathlineto{\pgfqpoint{1.931153in}{1.895647in}}%
\pgfpathlineto{\pgfqpoint{1.932221in}{1.054242in}}%
\pgfpathlineto{\pgfqpoint{1.932754in}{1.862432in}}%
\pgfpathlineto{\pgfqpoint{1.933288in}{0.766644in}}%
\pgfpathlineto{\pgfqpoint{1.934356in}{0.846556in}}%
\pgfpathlineto{\pgfqpoint{1.936490in}{0.718375in}}%
\pgfpathlineto{\pgfqpoint{1.937024in}{0.718574in}}%
\pgfpathlineto{\pgfqpoint{1.937558in}{0.705053in}}%
\pgfpathlineto{\pgfqpoint{1.938625in}{0.727580in}}%
\pgfpathlineto{\pgfqpoint{1.939693in}{0.691341in}}%
\pgfpathlineto{\pgfqpoint{1.940226in}{0.702828in}}%
\pgfpathlineto{\pgfqpoint{1.941827in}{0.686506in}}%
\pgfpathlineto{\pgfqpoint{1.942361in}{0.712461in}}%
\pgfpathlineto{\pgfqpoint{1.942895in}{0.662826in}}%
\pgfpathlineto{\pgfqpoint{1.943962in}{0.665944in}}%
\pgfpathlineto{\pgfqpoint{1.945030in}{0.667420in}}%
\pgfpathlineto{\pgfqpoint{1.945563in}{0.680454in}}%
\pgfpathlineto{\pgfqpoint{1.946097in}{0.667637in}}%
\pgfpathlineto{\pgfqpoint{1.946631in}{0.654505in}}%
\pgfpathlineto{\pgfqpoint{1.947164in}{0.682098in}}%
\pgfpathlineto{\pgfqpoint{1.947698in}{0.677768in}}%
\pgfpathlineto{\pgfqpoint{1.948765in}{0.639342in}}%
\pgfpathlineto{\pgfqpoint{1.949833in}{0.677606in}}%
\pgfpathlineto{\pgfqpoint{1.950367in}{0.674322in}}%
\pgfpathlineto{\pgfqpoint{1.950900in}{0.679293in}}%
\pgfpathlineto{\pgfqpoint{1.951434in}{0.657242in}}%
\pgfpathlineto{\pgfqpoint{1.951968in}{0.672215in}}%
\pgfpathlineto{\pgfqpoint{1.952501in}{0.681871in}}%
\pgfpathlineto{\pgfqpoint{1.953035in}{0.649293in}}%
\pgfpathlineto{\pgfqpoint{1.953569in}{0.660435in}}%
\pgfpathlineto{\pgfqpoint{1.954103in}{0.655298in}}%
\pgfpathlineto{\pgfqpoint{1.954636in}{0.669096in}}%
\pgfpathlineto{\pgfqpoint{1.955170in}{0.645053in}}%
\pgfpathlineto{\pgfqpoint{1.955704in}{0.655447in}}%
\pgfpathlineto{\pgfqpoint{1.956771in}{0.668568in}}%
\pgfpathlineto{\pgfqpoint{1.957305in}{0.642677in}}%
\pgfpathlineto{\pgfqpoint{1.957838in}{0.657424in}}%
\pgfpathlineto{\pgfqpoint{1.958906in}{0.666210in}}%
\pgfpathlineto{\pgfqpoint{1.959440in}{0.657850in}}%
\pgfpathlineto{\pgfqpoint{1.959973in}{0.669504in}}%
\pgfpathlineto{\pgfqpoint{1.960507in}{0.645772in}}%
\pgfpathlineto{\pgfqpoint{1.961041in}{0.659270in}}%
\pgfpathlineto{\pgfqpoint{1.962642in}{0.648670in}}%
\pgfpathlineto{\pgfqpoint{1.963175in}{0.650672in}}%
\pgfpathlineto{\pgfqpoint{1.963709in}{0.644786in}}%
\pgfpathlineto{\pgfqpoint{1.964243in}{0.664211in}}%
\pgfpathlineto{\pgfqpoint{1.964777in}{0.650788in}}%
\pgfpathlineto{\pgfqpoint{1.965310in}{0.646639in}}%
\pgfpathlineto{\pgfqpoint{1.965844in}{0.670830in}}%
\pgfpathlineto{\pgfqpoint{1.966378in}{0.638768in}}%
\pgfpathlineto{\pgfqpoint{1.966911in}{0.642575in}}%
\pgfpathlineto{\pgfqpoint{1.968512in}{0.660755in}}%
\pgfpathlineto{\pgfqpoint{1.969046in}{0.651879in}}%
\pgfpathlineto{\pgfqpoint{1.969580in}{0.653398in}}%
\pgfpathlineto{\pgfqpoint{1.970114in}{0.660541in}}%
\pgfpathlineto{\pgfqpoint{1.970647in}{0.660053in}}%
\pgfpathlineto{\pgfqpoint{1.971181in}{0.637984in}}%
\pgfpathlineto{\pgfqpoint{1.971715in}{0.658819in}}%
\pgfpathlineto{\pgfqpoint{1.972248in}{0.644783in}}%
\pgfpathlineto{\pgfqpoint{1.972782in}{0.648459in}}%
\pgfpathlineto{\pgfqpoint{1.973316in}{0.652633in}}%
\pgfpathlineto{\pgfqpoint{1.973849in}{0.644569in}}%
\pgfpathlineto{\pgfqpoint{1.975451in}{0.671983in}}%
\pgfpathlineto{\pgfqpoint{1.975984in}{0.649053in}}%
\pgfpathlineto{\pgfqpoint{1.977052in}{0.692311in}}%
\pgfpathlineto{\pgfqpoint{1.978119in}{0.645432in}}%
\pgfpathlineto{\pgfqpoint{1.978653in}{0.646312in}}%
\pgfpathlineto{\pgfqpoint{1.979186in}{0.670940in}}%
\pgfpathlineto{\pgfqpoint{1.979720in}{0.670812in}}%
\pgfpathlineto{\pgfqpoint{1.980788in}{0.652354in}}%
\pgfpathlineto{\pgfqpoint{1.981321in}{0.657473in}}%
\pgfpathlineto{\pgfqpoint{1.981855in}{0.650968in}}%
\pgfpathlineto{\pgfqpoint{1.983456in}{0.661578in}}%
\pgfpathlineto{\pgfqpoint{1.985591in}{0.652427in}}%
\pgfpathlineto{\pgfqpoint{1.986125in}{0.643042in}}%
\pgfpathlineto{\pgfqpoint{1.986658in}{0.649785in}}%
\pgfpathlineto{\pgfqpoint{1.987726in}{0.661351in}}%
\pgfpathlineto{\pgfqpoint{1.989327in}{0.640304in}}%
\pgfpathlineto{\pgfqpoint{1.990394in}{0.652607in}}%
\pgfpathlineto{\pgfqpoint{1.991995in}{0.645433in}}%
\pgfpathlineto{\pgfqpoint{1.992529in}{0.661170in}}%
\pgfpathlineto{\pgfqpoint{1.993063in}{0.657319in}}%
\pgfpathlineto{\pgfqpoint{1.994130in}{0.643169in}}%
\pgfpathlineto{\pgfqpoint{1.995197in}{0.642494in}}%
\pgfpathlineto{\pgfqpoint{1.995731in}{0.656940in}}%
\pgfpathlineto{\pgfqpoint{1.997332in}{0.639174in}}%
\pgfpathlineto{\pgfqpoint{1.997866in}{0.670434in}}%
\pgfpathlineto{\pgfqpoint{1.998400in}{0.640145in}}%
\pgfpathlineto{\pgfqpoint{2.000001in}{0.663072in}}%
\pgfpathlineto{\pgfqpoint{2.001068in}{0.654037in}}%
\pgfpathlineto{\pgfqpoint{2.001602in}{0.658649in}}%
\pgfpathlineto{\pgfqpoint{2.002136in}{0.667673in}}%
\pgfpathlineto{\pgfqpoint{2.002669in}{0.638477in}}%
\pgfpathlineto{\pgfqpoint{2.003203in}{0.655709in}}%
\pgfpathlineto{\pgfqpoint{2.003737in}{0.648809in}}%
\pgfpathlineto{\pgfqpoint{2.004270in}{0.652898in}}%
\pgfpathlineto{\pgfqpoint{2.005338in}{0.664263in}}%
\pgfpathlineto{\pgfqpoint{2.005871in}{0.642851in}}%
\pgfpathlineto{\pgfqpoint{2.006405in}{0.662615in}}%
\pgfpathlineto{\pgfqpoint{2.006939in}{0.645311in}}%
\pgfpathlineto{\pgfqpoint{2.008006in}{0.666277in}}%
\pgfpathlineto{\pgfqpoint{2.008540in}{0.660099in}}%
\pgfpathlineto{\pgfqpoint{2.009074in}{0.638786in}}%
\pgfpathlineto{\pgfqpoint{2.009607in}{0.642417in}}%
\pgfpathlineto{\pgfqpoint{2.010141in}{0.655431in}}%
\pgfpathlineto{\pgfqpoint{2.010675in}{0.645507in}}%
\pgfpathlineto{\pgfqpoint{2.012276in}{0.664285in}}%
\pgfpathlineto{\pgfqpoint{2.013343in}{0.642841in}}%
\pgfpathlineto{\pgfqpoint{2.013877in}{0.643500in}}%
\pgfpathlineto{\pgfqpoint{2.014411in}{0.661253in}}%
\pgfpathlineto{\pgfqpoint{2.014944in}{0.644582in}}%
\pgfpathlineto{\pgfqpoint{2.015478in}{0.659689in}}%
\pgfpathlineto{\pgfqpoint{2.016012in}{0.652357in}}%
\pgfpathlineto{\pgfqpoint{2.016545in}{0.650670in}}%
\pgfpathlineto{\pgfqpoint{2.017079in}{0.635667in}}%
\pgfpathlineto{\pgfqpoint{2.017613in}{0.642174in}}%
\pgfpathlineto{\pgfqpoint{2.018147in}{0.642312in}}%
\pgfpathlineto{\pgfqpoint{2.018680in}{0.648802in}}%
\pgfpathlineto{\pgfqpoint{2.019214in}{0.642858in}}%
\pgfpathlineto{\pgfqpoint{2.019748in}{0.644498in}}%
\pgfpathlineto{\pgfqpoint{2.021349in}{0.662778in}}%
\pgfpathlineto{\pgfqpoint{2.021883in}{0.641599in}}%
\pgfpathlineto{\pgfqpoint{2.022416in}{0.646225in}}%
\pgfpathlineto{\pgfqpoint{2.023484in}{0.658024in}}%
\pgfpathlineto{\pgfqpoint{2.024551in}{0.642435in}}%
\pgfpathlineto{\pgfqpoint{2.025085in}{0.652362in}}%
\pgfpathlineto{\pgfqpoint{2.025618in}{0.640872in}}%
\pgfpathlineto{\pgfqpoint{2.026152in}{0.652581in}}%
\pgfpathlineto{\pgfqpoint{2.027220in}{0.643131in}}%
\pgfpathlineto{\pgfqpoint{2.029354in}{0.652659in}}%
\pgfpathlineto{\pgfqpoint{2.029888in}{0.647077in}}%
\pgfpathlineto{\pgfqpoint{2.030422in}{0.655645in}}%
\pgfpathlineto{\pgfqpoint{2.030955in}{0.654030in}}%
\pgfpathlineto{\pgfqpoint{2.031489in}{0.648562in}}%
\pgfpathlineto{\pgfqpoint{2.032023in}{0.666230in}}%
\pgfpathlineto{\pgfqpoint{2.032557in}{0.659141in}}%
\pgfpathlineto{\pgfqpoint{2.033090in}{0.657328in}}%
\pgfpathlineto{\pgfqpoint{2.034158in}{0.670121in}}%
\pgfpathlineto{\pgfqpoint{2.034691in}{0.643767in}}%
\pgfpathlineto{\pgfqpoint{2.035225in}{0.659849in}}%
\pgfpathlineto{\pgfqpoint{2.035759in}{0.664885in}}%
\pgfpathlineto{\pgfqpoint{2.036292in}{0.663556in}}%
\pgfpathlineto{\pgfqpoint{2.036826in}{0.645404in}}%
\pgfpathlineto{\pgfqpoint{2.037360in}{0.663327in}}%
\pgfpathlineto{\pgfqpoint{2.037894in}{0.661223in}}%
\pgfpathlineto{\pgfqpoint{2.038427in}{0.673108in}}%
\pgfpathlineto{\pgfqpoint{2.039495in}{0.653990in}}%
\pgfpathlineto{\pgfqpoint{2.040028in}{0.655503in}}%
\pgfpathlineto{\pgfqpoint{2.040562in}{0.652983in}}%
\pgfpathlineto{\pgfqpoint{2.041096in}{0.666969in}}%
\pgfpathlineto{\pgfqpoint{2.041629in}{0.648480in}}%
\pgfpathlineto{\pgfqpoint{2.042163in}{0.658379in}}%
\pgfpathlineto{\pgfqpoint{2.043231in}{0.647726in}}%
\pgfpathlineto{\pgfqpoint{2.044298in}{0.662517in}}%
\pgfpathlineto{\pgfqpoint{2.044832in}{0.672650in}}%
\pgfpathlineto{\pgfqpoint{2.045899in}{0.640239in}}%
\pgfpathlineto{\pgfqpoint{2.046433in}{0.645403in}}%
\pgfpathlineto{\pgfqpoint{2.048568in}{0.666703in}}%
\pgfpathlineto{\pgfqpoint{2.049101in}{0.642617in}}%
\pgfpathlineto{\pgfqpoint{2.049635in}{0.665124in}}%
\pgfpathlineto{\pgfqpoint{2.052303in}{0.646547in}}%
\pgfpathlineto{\pgfqpoint{2.052837in}{0.646104in}}%
\pgfpathlineto{\pgfqpoint{2.053905in}{0.658431in}}%
\pgfpathlineto{\pgfqpoint{2.054438in}{0.645791in}}%
\pgfpathlineto{\pgfqpoint{2.054972in}{0.674822in}}%
\pgfpathlineto{\pgfqpoint{2.055506in}{0.657072in}}%
\pgfpathlineto{\pgfqpoint{2.056039in}{0.651455in}}%
\pgfpathlineto{\pgfqpoint{2.057107in}{0.669554in}}%
\pgfpathlineto{\pgfqpoint{2.058708in}{0.641962in}}%
\pgfpathlineto{\pgfqpoint{2.059775in}{0.653122in}}%
\pgfpathlineto{\pgfqpoint{2.060309in}{0.676037in}}%
\pgfpathlineto{\pgfqpoint{2.060843in}{0.655078in}}%
\pgfpathlineto{\pgfqpoint{2.061376in}{0.664883in}}%
\pgfpathlineto{\pgfqpoint{2.061910in}{0.652987in}}%
\pgfpathlineto{\pgfqpoint{2.062444in}{0.660742in}}%
\pgfpathlineto{\pgfqpoint{2.062977in}{0.659162in}}%
\pgfpathlineto{\pgfqpoint{2.063511in}{0.669765in}}%
\pgfpathlineto{\pgfqpoint{2.064579in}{0.646877in}}%
\pgfpathlineto{\pgfqpoint{2.065112in}{0.655555in}}%
\pgfpathlineto{\pgfqpoint{2.065646in}{0.651177in}}%
\pgfpathlineto{\pgfqpoint{2.066180in}{0.642181in}}%
\pgfpathlineto{\pgfqpoint{2.067247in}{0.656550in}}%
\pgfpathlineto{\pgfqpoint{2.068848in}{0.641323in}}%
\pgfpathlineto{\pgfqpoint{2.070449in}{0.653366in}}%
\pgfpathlineto{\pgfqpoint{2.070983in}{0.638667in}}%
\pgfpathlineto{\pgfqpoint{2.071517in}{0.656945in}}%
\pgfpathlineto{\pgfqpoint{2.072050in}{0.650033in}}%
\pgfpathlineto{\pgfqpoint{2.073118in}{0.652568in}}%
\pgfpathlineto{\pgfqpoint{2.074185in}{0.639052in}}%
\pgfpathlineto{\pgfqpoint{2.076320in}{0.660313in}}%
\pgfpathlineto{\pgfqpoint{2.076854in}{0.649072in}}%
\pgfpathlineto{\pgfqpoint{2.077387in}{0.658149in}}%
\pgfpathlineto{\pgfqpoint{2.077921in}{0.650414in}}%
\pgfpathlineto{\pgfqpoint{2.078455in}{0.662370in}}%
\pgfpathlineto{\pgfqpoint{2.078988in}{0.642213in}}%
\pgfpathlineto{\pgfqpoint{2.079522in}{0.650724in}}%
\pgfpathlineto{\pgfqpoint{2.080056in}{0.653069in}}%
\pgfpathlineto{\pgfqpoint{2.080590in}{0.642905in}}%
\pgfpathlineto{\pgfqpoint{2.081657in}{0.643257in}}%
\pgfpathlineto{\pgfqpoint{2.082724in}{0.640340in}}%
\pgfpathlineto{\pgfqpoint{2.084326in}{0.658791in}}%
\pgfpathlineto{\pgfqpoint{2.084859in}{0.640709in}}%
\pgfpathlineto{\pgfqpoint{2.085393in}{0.643274in}}%
\pgfpathlineto{\pgfqpoint{2.087528in}{0.667110in}}%
\pgfpathlineto{\pgfqpoint{2.088061in}{0.648958in}}%
\pgfpathlineto{\pgfqpoint{2.088595in}{0.652331in}}%
\pgfpathlineto{\pgfqpoint{2.090196in}{0.671272in}}%
\pgfpathlineto{\pgfqpoint{2.091797in}{0.648532in}}%
\pgfpathlineto{\pgfqpoint{2.092865in}{0.667419in}}%
\pgfpathlineto{\pgfqpoint{2.093398in}{0.639251in}}%
\pgfpathlineto{\pgfqpoint{2.093932in}{0.643289in}}%
\pgfpathlineto{\pgfqpoint{2.095000in}{0.677732in}}%
\pgfpathlineto{\pgfqpoint{2.095533in}{0.650341in}}%
\pgfpathlineto{\pgfqpoint{2.096067in}{0.664220in}}%
\pgfpathlineto{\pgfqpoint{2.097668in}{0.647271in}}%
\pgfpathlineto{\pgfqpoint{2.098202in}{0.647481in}}%
\pgfpathlineto{\pgfqpoint{2.098735in}{0.640606in}}%
\pgfpathlineto{\pgfqpoint{2.099269in}{0.652273in}}%
\pgfpathlineto{\pgfqpoint{2.099803in}{0.639994in}}%
\pgfpathlineto{\pgfqpoint{2.100337in}{0.639673in}}%
\pgfpathlineto{\pgfqpoint{2.101938in}{0.664705in}}%
\pgfpathlineto{\pgfqpoint{2.103005in}{0.642451in}}%
\pgfpathlineto{\pgfqpoint{2.103539in}{0.643442in}}%
\pgfpathlineto{\pgfqpoint{2.104606in}{0.655207in}}%
\pgfpathlineto{\pgfqpoint{2.105140in}{0.639708in}}%
\pgfpathlineto{\pgfqpoint{2.105674in}{0.664530in}}%
\pgfpathlineto{\pgfqpoint{2.106207in}{0.645504in}}%
\pgfpathlineto{\pgfqpoint{2.106741in}{0.645970in}}%
\pgfpathlineto{\pgfqpoint{2.107808in}{0.661562in}}%
\pgfpathlineto{\pgfqpoint{2.108342in}{0.648374in}}%
\pgfpathlineto{\pgfqpoint{2.108876in}{0.650366in}}%
\pgfpathlineto{\pgfqpoint{2.111544in}{0.667915in}}%
\pgfpathlineto{\pgfqpoint{2.112078in}{0.637933in}}%
\pgfpathlineto{\pgfqpoint{2.112612in}{0.656803in}}%
\pgfpathlineto{\pgfqpoint{2.114213in}{0.637624in}}%
\pgfpathlineto{\pgfqpoint{2.115814in}{0.676469in}}%
\pgfpathlineto{\pgfqpoint{2.116348in}{0.643165in}}%
\pgfpathlineto{\pgfqpoint{2.116881in}{0.657983in}}%
\pgfpathlineto{\pgfqpoint{2.117415in}{0.655222in}}%
\pgfpathlineto{\pgfqpoint{2.117949in}{0.643311in}}%
\pgfpathlineto{\pgfqpoint{2.118482in}{0.667330in}}%
\pgfpathlineto{\pgfqpoint{2.119016in}{0.659651in}}%
\pgfpathlineto{\pgfqpoint{2.119550in}{0.640455in}}%
\pgfpathlineto{\pgfqpoint{2.120083in}{0.667676in}}%
\pgfpathlineto{\pgfqpoint{2.120617in}{0.643823in}}%
\pgfpathlineto{\pgfqpoint{2.121151in}{0.640530in}}%
\pgfpathlineto{\pgfqpoint{2.122218in}{0.663143in}}%
\pgfpathlineto{\pgfqpoint{2.122752in}{0.655144in}}%
\pgfpathlineto{\pgfqpoint{2.123286in}{0.637880in}}%
\pgfpathlineto{\pgfqpoint{2.123819in}{0.644353in}}%
\pgfpathlineto{\pgfqpoint{2.124353in}{0.649037in}}%
\pgfpathlineto{\pgfqpoint{2.124887in}{0.644788in}}%
\pgfpathlineto{\pgfqpoint{2.125420in}{0.647449in}}%
\pgfpathlineto{\pgfqpoint{2.125954in}{0.639876in}}%
\pgfpathlineto{\pgfqpoint{2.126488in}{0.645458in}}%
\pgfpathlineto{\pgfqpoint{2.127022in}{0.648463in}}%
\pgfpathlineto{\pgfqpoint{2.128089in}{0.637269in}}%
\pgfpathlineto{\pgfqpoint{2.129156in}{0.648234in}}%
\pgfpathlineto{\pgfqpoint{2.129690in}{0.641275in}}%
\pgfpathlineto{\pgfqpoint{2.130224in}{0.648766in}}%
\pgfpathlineto{\pgfqpoint{2.130757in}{0.646078in}}%
\pgfpathlineto{\pgfqpoint{2.131291in}{0.648098in}}%
\pgfpathlineto{\pgfqpoint{2.131825in}{0.657659in}}%
\pgfpathlineto{\pgfqpoint{2.132892in}{0.638516in}}%
\pgfpathlineto{\pgfqpoint{2.133426in}{0.649356in}}%
\pgfpathlineto{\pgfqpoint{2.133960in}{0.647905in}}%
\pgfpathlineto{\pgfqpoint{2.134493in}{0.642701in}}%
\pgfpathlineto{\pgfqpoint{2.135027in}{0.653494in}}%
\pgfpathlineto{\pgfqpoint{2.135561in}{0.648351in}}%
\pgfpathlineto{\pgfqpoint{2.137696in}{0.638752in}}%
\pgfpathlineto{\pgfqpoint{2.138763in}{0.656528in}}%
\pgfpathlineto{\pgfqpoint{2.139297in}{0.647736in}}%
\pgfpathlineto{\pgfqpoint{2.139830in}{0.646116in}}%
\pgfpathlineto{\pgfqpoint{2.140364in}{0.650917in}}%
\pgfpathlineto{\pgfqpoint{2.140898in}{0.637491in}}%
\pgfpathlineto{\pgfqpoint{2.141431in}{0.638809in}}%
\pgfpathlineto{\pgfqpoint{2.143033in}{0.660240in}}%
\pgfpathlineto{\pgfqpoint{2.143566in}{0.639336in}}%
\pgfpathlineto{\pgfqpoint{2.144100in}{0.648128in}}%
\pgfpathlineto{\pgfqpoint{2.144634in}{0.661861in}}%
\pgfpathlineto{\pgfqpoint{2.145167in}{0.641558in}}%
\pgfpathlineto{\pgfqpoint{2.145701in}{0.661320in}}%
\pgfpathlineto{\pgfqpoint{2.146768in}{0.644275in}}%
\pgfpathlineto{\pgfqpoint{2.147302in}{0.658390in}}%
\pgfpathlineto{\pgfqpoint{2.148370in}{0.657555in}}%
\pgfpathlineto{\pgfqpoint{2.148903in}{0.655942in}}%
\pgfpathlineto{\pgfqpoint{2.149437in}{0.642226in}}%
\pgfpathlineto{\pgfqpoint{2.149971in}{0.661621in}}%
\pgfpathlineto{\pgfqpoint{2.150504in}{0.648262in}}%
\pgfpathlineto{\pgfqpoint{2.151038in}{0.646402in}}%
\pgfpathlineto{\pgfqpoint{2.151572in}{0.666489in}}%
\pgfpathlineto{\pgfqpoint{2.152106in}{0.664923in}}%
\pgfpathlineto{\pgfqpoint{2.152639in}{0.647709in}}%
\pgfpathlineto{\pgfqpoint{2.153173in}{0.655028in}}%
\pgfpathlineto{\pgfqpoint{2.153707in}{0.653325in}}%
\pgfpathlineto{\pgfqpoint{2.154240in}{0.639734in}}%
\pgfpathlineto{\pgfqpoint{2.154774in}{0.641461in}}%
\pgfpathlineto{\pgfqpoint{2.155308in}{0.643254in}}%
\pgfpathlineto{\pgfqpoint{2.155841in}{0.652438in}}%
\pgfpathlineto{\pgfqpoint{2.156375in}{0.636177in}}%
\pgfpathlineto{\pgfqpoint{2.156909in}{0.642737in}}%
\pgfpathlineto{\pgfqpoint{2.157443in}{0.640553in}}%
\pgfpathlineto{\pgfqpoint{2.158510in}{0.661269in}}%
\pgfpathlineto{\pgfqpoint{2.159044in}{0.640553in}}%
\pgfpathlineto{\pgfqpoint{2.159577in}{0.649817in}}%
\pgfpathlineto{\pgfqpoint{2.160111in}{0.645597in}}%
\pgfpathlineto{\pgfqpoint{2.161178in}{0.656475in}}%
\pgfpathlineto{\pgfqpoint{2.161712in}{0.641029in}}%
\pgfpathlineto{\pgfqpoint{2.162246in}{0.641167in}}%
\pgfpathlineto{\pgfqpoint{2.163847in}{0.646595in}}%
\pgfpathlineto{\pgfqpoint{2.164381in}{0.641494in}}%
\pgfpathlineto{\pgfqpoint{2.165982in}{0.665246in}}%
\pgfpathlineto{\pgfqpoint{2.166515in}{0.636314in}}%
\pgfpathlineto{\pgfqpoint{2.167049in}{0.652930in}}%
\pgfpathlineto{\pgfqpoint{2.167583in}{0.641973in}}%
\pgfpathlineto{\pgfqpoint{2.168117in}{0.663721in}}%
\pgfpathlineto{\pgfqpoint{2.168650in}{0.648434in}}%
\pgfpathlineto{\pgfqpoint{2.170785in}{0.661499in}}%
\pgfpathlineto{\pgfqpoint{2.171319in}{0.644647in}}%
\pgfpathlineto{\pgfqpoint{2.171852in}{0.648168in}}%
\pgfpathlineto{\pgfqpoint{2.172386in}{0.652761in}}%
\pgfpathlineto{\pgfqpoint{2.172920in}{0.650973in}}%
\pgfpathlineto{\pgfqpoint{2.173987in}{0.649341in}}%
\pgfpathlineto{\pgfqpoint{2.174521in}{0.641449in}}%
\pgfpathlineto{\pgfqpoint{2.175055in}{0.660066in}}%
\pgfpathlineto{\pgfqpoint{2.175588in}{0.654490in}}%
\pgfpathlineto{\pgfqpoint{2.176122in}{0.652827in}}%
\pgfpathlineto{\pgfqpoint{2.176656in}{0.644155in}}%
\pgfpathlineto{\pgfqpoint{2.177189in}{0.648615in}}%
\pgfpathlineto{\pgfqpoint{2.177723in}{0.655819in}}%
\pgfpathlineto{\pgfqpoint{2.179324in}{0.637830in}}%
\pgfpathlineto{\pgfqpoint{2.180925in}{0.646119in}}%
\pgfpathlineto{\pgfqpoint{2.182526in}{0.638502in}}%
\pgfpathlineto{\pgfqpoint{2.183060in}{0.639349in}}%
\pgfpathlineto{\pgfqpoint{2.183594in}{0.647935in}}%
\pgfpathlineto{\pgfqpoint{2.184128in}{0.642491in}}%
\pgfpathlineto{\pgfqpoint{2.184661in}{0.638773in}}%
\pgfpathlineto{\pgfqpoint{2.185195in}{0.642273in}}%
\pgfpathlineto{\pgfqpoint{2.185729in}{0.642879in}}%
\pgfpathlineto{\pgfqpoint{2.186262in}{0.642063in}}%
\pgfpathlineto{\pgfqpoint{2.186796in}{0.646184in}}%
\pgfpathlineto{\pgfqpoint{2.187330in}{0.645314in}}%
\pgfpathlineto{\pgfqpoint{2.187863in}{0.640936in}}%
\pgfpathlineto{\pgfqpoint{2.188931in}{0.648147in}}%
\pgfpathlineto{\pgfqpoint{2.189465in}{0.638515in}}%
\pgfpathlineto{\pgfqpoint{2.189998in}{0.646655in}}%
\pgfpathlineto{\pgfqpoint{2.190532in}{0.647183in}}%
\pgfpathlineto{\pgfqpoint{2.191599in}{0.640297in}}%
\pgfpathlineto{\pgfqpoint{2.192133in}{0.643469in}}%
\pgfpathlineto{\pgfqpoint{2.192667in}{0.646913in}}%
\pgfpathlineto{\pgfqpoint{2.193200in}{0.640276in}}%
\pgfpathlineto{\pgfqpoint{2.193734in}{0.650264in}}%
\pgfpathlineto{\pgfqpoint{2.194268in}{0.642076in}}%
\pgfpathlineto{\pgfqpoint{2.195335in}{0.651957in}}%
\pgfpathlineto{\pgfqpoint{2.196936in}{0.638187in}}%
\pgfpathlineto{\pgfqpoint{2.197470in}{0.650901in}}%
\pgfpathlineto{\pgfqpoint{2.198004in}{0.648950in}}%
\pgfpathlineto{\pgfqpoint{2.198537in}{0.640687in}}%
\pgfpathlineto{\pgfqpoint{2.199071in}{0.642850in}}%
\pgfpathlineto{\pgfqpoint{2.199605in}{0.648423in}}%
\pgfpathlineto{\pgfqpoint{2.200139in}{0.642720in}}%
\pgfpathlineto{\pgfqpoint{2.201206in}{0.642556in}}%
\pgfpathlineto{\pgfqpoint{2.201740in}{0.638729in}}%
\pgfpathlineto{\pgfqpoint{2.202273in}{0.639618in}}%
\pgfpathlineto{\pgfqpoint{2.203341in}{0.638208in}}%
\pgfpathlineto{\pgfqpoint{2.204942in}{0.636789in}}%
\pgfpathlineto{\pgfqpoint{2.206009in}{0.637450in}}%
\pgfpathlineto{\pgfqpoint{2.209745in}{0.640205in}}%
\pgfpathlineto{\pgfqpoint{2.210813in}{0.638435in}}%
\pgfpathlineto{\pgfqpoint{2.211346in}{0.639322in}}%
\pgfpathlineto{\pgfqpoint{2.211880in}{0.637727in}}%
\pgfpathlineto{\pgfqpoint{2.212947in}{0.646298in}}%
\pgfpathlineto{\pgfqpoint{2.213481in}{0.636663in}}%
\pgfpathlineto{\pgfqpoint{2.214015in}{0.641789in}}%
\pgfpathlineto{\pgfqpoint{2.214548in}{0.652325in}}%
\pgfpathlineto{\pgfqpoint{2.215082in}{0.647077in}}%
\pgfpathlineto{\pgfqpoint{2.216683in}{0.641383in}}%
\pgfpathlineto{\pgfqpoint{2.217751in}{0.651975in}}%
\pgfpathlineto{\pgfqpoint{2.218284in}{0.648382in}}%
\pgfpathlineto{\pgfqpoint{2.218818in}{0.636569in}}%
\pgfpathlineto{\pgfqpoint{2.219352in}{0.655816in}}%
\pgfpathlineto{\pgfqpoint{2.219886in}{0.648889in}}%
\pgfpathlineto{\pgfqpoint{2.220419in}{0.647153in}}%
\pgfpathlineto{\pgfqpoint{2.220953in}{0.651898in}}%
\pgfpathlineto{\pgfqpoint{2.222020in}{0.638624in}}%
\pgfpathlineto{\pgfqpoint{2.222554in}{0.644466in}}%
\pgfpathlineto{\pgfqpoint{2.223088in}{0.650547in}}%
\pgfpathlineto{\pgfqpoint{2.223621in}{0.645376in}}%
\pgfpathlineto{\pgfqpoint{2.224155in}{0.646744in}}%
\pgfpathlineto{\pgfqpoint{2.224689in}{0.638109in}}%
\pgfpathlineto{\pgfqpoint{2.225223in}{0.638938in}}%
\pgfpathlineto{\pgfqpoint{2.225756in}{0.659337in}}%
\pgfpathlineto{\pgfqpoint{2.226290in}{0.639523in}}%
\pgfpathlineto{\pgfqpoint{2.226824in}{0.637842in}}%
\pgfpathlineto{\pgfqpoint{2.227891in}{0.650370in}}%
\pgfpathlineto{\pgfqpoint{2.228425in}{0.645519in}}%
\pgfpathlineto{\pgfqpoint{2.228958in}{0.640578in}}%
\pgfpathlineto{\pgfqpoint{2.230026in}{0.640214in}}%
\pgfpathlineto{\pgfqpoint{2.230560in}{0.666707in}}%
\pgfpathlineto{\pgfqpoint{2.232161in}{0.638293in}}%
\pgfpathlineto{\pgfqpoint{2.232694in}{0.652348in}}%
\pgfpathlineto{\pgfqpoint{2.233228in}{0.644764in}}%
\pgfpathlineto{\pgfqpoint{2.234829in}{0.636925in}}%
\pgfpathlineto{\pgfqpoint{2.235363in}{0.638897in}}%
\pgfpathlineto{\pgfqpoint{2.235897in}{0.645401in}}%
\pgfpathlineto{\pgfqpoint{2.236430in}{0.641959in}}%
\pgfpathlineto{\pgfqpoint{2.237498in}{0.645487in}}%
\pgfpathlineto{\pgfqpoint{2.238031in}{0.649358in}}%
\pgfpathlineto{\pgfqpoint{2.238565in}{0.641069in}}%
\pgfpathlineto{\pgfqpoint{2.239632in}{0.641388in}}%
\pgfpathlineto{\pgfqpoint{2.241234in}{0.635266in}}%
\pgfpathlineto{\pgfqpoint{2.241767in}{0.651446in}}%
\pgfpathlineto{\pgfqpoint{2.242301in}{0.645800in}}%
\pgfpathlineto{\pgfqpoint{2.242835in}{0.639510in}}%
\pgfpathlineto{\pgfqpoint{2.243368in}{0.640790in}}%
\pgfpathlineto{\pgfqpoint{2.243902in}{0.647316in}}%
\pgfpathlineto{\pgfqpoint{2.244969in}{0.637040in}}%
\pgfpathlineto{\pgfqpoint{2.246037in}{0.648572in}}%
\pgfpathlineto{\pgfqpoint{2.246571in}{0.639125in}}%
\pgfpathlineto{\pgfqpoint{2.247104in}{0.643312in}}%
\pgfpathlineto{\pgfqpoint{2.248172in}{0.641515in}}%
\pgfpathlineto{\pgfqpoint{2.248705in}{0.645448in}}%
\pgfpathlineto{\pgfqpoint{2.249239in}{0.644250in}}%
\pgfpathlineto{\pgfqpoint{2.250306in}{0.640928in}}%
\pgfpathlineto{\pgfqpoint{2.250840in}{0.644118in}}%
\pgfpathlineto{\pgfqpoint{2.251374in}{0.638274in}}%
\pgfpathlineto{\pgfqpoint{2.251908in}{0.643252in}}%
\pgfpathlineto{\pgfqpoint{2.252441in}{0.643302in}}%
\pgfpathlineto{\pgfqpoint{2.252975in}{0.638809in}}%
\pgfpathlineto{\pgfqpoint{2.253509in}{0.642105in}}%
\pgfpathlineto{\pgfqpoint{2.254042in}{0.642086in}}%
\pgfpathlineto{\pgfqpoint{2.255110in}{0.656795in}}%
\pgfpathlineto{\pgfqpoint{2.255643in}{0.646572in}}%
\pgfpathlineto{\pgfqpoint{2.256177in}{0.647201in}}%
\pgfpathlineto{\pgfqpoint{2.256711in}{0.649802in}}%
\pgfpathlineto{\pgfqpoint{2.257245in}{0.642092in}}%
\pgfpathlineto{\pgfqpoint{2.257778in}{0.659411in}}%
\pgfpathlineto{\pgfqpoint{2.258312in}{0.647686in}}%
\pgfpathlineto{\pgfqpoint{2.259913in}{0.638879in}}%
\pgfpathlineto{\pgfqpoint{2.260447in}{0.659229in}}%
\pgfpathlineto{\pgfqpoint{2.260980in}{0.648504in}}%
\pgfpathlineto{\pgfqpoint{2.261514in}{0.650885in}}%
\pgfpathlineto{\pgfqpoint{2.262048in}{0.646586in}}%
\pgfpathlineto{\pgfqpoint{2.262582in}{0.657130in}}%
\pgfpathlineto{\pgfqpoint{2.263115in}{0.646164in}}%
\pgfpathlineto{\pgfqpoint{2.263649in}{0.639253in}}%
\pgfpathlineto{\pgfqpoint{2.264183in}{0.642123in}}%
\pgfpathlineto{\pgfqpoint{2.264716in}{0.641133in}}%
\pgfpathlineto{\pgfqpoint{2.265784in}{0.648052in}}%
\pgfpathlineto{\pgfqpoint{2.266317in}{0.646720in}}%
\pgfpathlineto{\pgfqpoint{2.266851in}{0.643827in}}%
\pgfpathlineto{\pgfqpoint{2.267385in}{0.649887in}}%
\pgfpathlineto{\pgfqpoint{2.268986in}{0.639567in}}%
\pgfpathlineto{\pgfqpoint{2.270053in}{0.650792in}}%
\pgfpathlineto{\pgfqpoint{2.270587in}{0.645543in}}%
\pgfpathlineto{\pgfqpoint{2.271121in}{0.646668in}}%
\pgfpathlineto{\pgfqpoint{2.272188in}{0.639429in}}%
\pgfpathlineto{\pgfqpoint{2.272722in}{0.641178in}}%
\pgfpathlineto{\pgfqpoint{2.273256in}{0.643473in}}%
\pgfpathlineto{\pgfqpoint{2.273789in}{0.640342in}}%
\pgfpathlineto{\pgfqpoint{2.274857in}{0.651058in}}%
\pgfpathlineto{\pgfqpoint{2.275390in}{0.642265in}}%
\pgfpathlineto{\pgfqpoint{2.275924in}{0.656242in}}%
\pgfpathlineto{\pgfqpoint{2.276991in}{0.637241in}}%
\pgfpathlineto{\pgfqpoint{2.277525in}{0.639569in}}%
\pgfpathlineto{\pgfqpoint{2.278059in}{0.639780in}}%
\pgfpathlineto{\pgfqpoint{2.278593in}{0.645868in}}%
\pgfpathlineto{\pgfqpoint{2.279126in}{0.643612in}}%
\pgfpathlineto{\pgfqpoint{2.279660in}{0.640198in}}%
\pgfpathlineto{\pgfqpoint{2.280727in}{0.651718in}}%
\pgfpathlineto{\pgfqpoint{2.281261in}{0.644846in}}%
\pgfpathlineto{\pgfqpoint{2.282862in}{0.635947in}}%
\pgfpathlineto{\pgfqpoint{2.283396in}{0.648390in}}%
\pgfpathlineto{\pgfqpoint{2.283930in}{0.642234in}}%
\pgfpathlineto{\pgfqpoint{2.285531in}{0.647746in}}%
\pgfpathlineto{\pgfqpoint{2.286598in}{0.641011in}}%
\pgfpathlineto{\pgfqpoint{2.287132in}{0.645277in}}%
\pgfpathlineto{\pgfqpoint{2.288199in}{0.654258in}}%
\pgfpathlineto{\pgfqpoint{2.288733in}{0.637225in}}%
\pgfpathlineto{\pgfqpoint{2.289267in}{0.638672in}}%
\pgfpathlineto{\pgfqpoint{2.290334in}{0.641438in}}%
\pgfpathlineto{\pgfqpoint{2.290868in}{0.636972in}}%
\pgfpathlineto{\pgfqpoint{2.291401in}{0.643420in}}%
\pgfpathlineto{\pgfqpoint{2.291935in}{0.640735in}}%
\pgfpathlineto{\pgfqpoint{2.292469in}{0.636339in}}%
\pgfpathlineto{\pgfqpoint{2.293003in}{0.638525in}}%
\pgfpathlineto{\pgfqpoint{2.294070in}{0.641814in}}%
\pgfpathlineto{\pgfqpoint{2.295137in}{0.639482in}}%
\pgfpathlineto{\pgfqpoint{2.296205in}{0.644881in}}%
\pgfpathlineto{\pgfqpoint{2.296738in}{0.640489in}}%
\pgfpathlineto{\pgfqpoint{2.297272in}{0.646286in}}%
\pgfpathlineto{\pgfqpoint{2.297806in}{0.645123in}}%
\pgfpathlineto{\pgfqpoint{2.298340in}{0.637113in}}%
\pgfpathlineto{\pgfqpoint{2.298873in}{0.639795in}}%
\pgfpathlineto{\pgfqpoint{2.301008in}{0.646517in}}%
\pgfpathlineto{\pgfqpoint{2.302075in}{0.637154in}}%
\pgfpathlineto{\pgfqpoint{2.302609in}{0.643491in}}%
\pgfpathlineto{\pgfqpoint{2.303143in}{0.636138in}}%
\pgfpathlineto{\pgfqpoint{2.303677in}{0.640022in}}%
\pgfpathlineto{\pgfqpoint{2.305278in}{0.637568in}}%
\pgfpathlineto{\pgfqpoint{2.305811in}{0.645679in}}%
\pgfpathlineto{\pgfqpoint{2.306345in}{0.639686in}}%
\pgfpathlineto{\pgfqpoint{2.306879in}{0.642316in}}%
\pgfpathlineto{\pgfqpoint{2.307412in}{0.638607in}}%
\pgfpathlineto{\pgfqpoint{2.307946in}{0.645242in}}%
\pgfpathlineto{\pgfqpoint{2.308480in}{0.639773in}}%
\pgfpathlineto{\pgfqpoint{2.309014in}{0.638457in}}%
\pgfpathlineto{\pgfqpoint{2.309547in}{0.640636in}}%
\pgfpathlineto{\pgfqpoint{2.310081in}{0.649453in}}%
\pgfpathlineto{\pgfqpoint{2.310615in}{0.642869in}}%
\pgfpathlineto{\pgfqpoint{2.311148in}{0.646558in}}%
\pgfpathlineto{\pgfqpoint{2.311682in}{0.645511in}}%
\pgfpathlineto{\pgfqpoint{2.313283in}{0.644183in}}%
\pgfpathlineto{\pgfqpoint{2.314884in}{0.637640in}}%
\pgfpathlineto{\pgfqpoint{2.315418in}{0.648770in}}%
\pgfpathlineto{\pgfqpoint{2.315952in}{0.648590in}}%
\pgfpathlineto{\pgfqpoint{2.317019in}{0.640222in}}%
\pgfpathlineto{\pgfqpoint{2.318620in}{0.646054in}}%
\pgfpathlineto{\pgfqpoint{2.320221in}{0.640318in}}%
\pgfpathlineto{\pgfqpoint{2.320755in}{0.642624in}}%
\pgfpathlineto{\pgfqpoint{2.321289in}{0.640457in}}%
\pgfpathlineto{\pgfqpoint{2.322356in}{0.636337in}}%
\pgfpathlineto{\pgfqpoint{2.322890in}{0.637893in}}%
\pgfpathlineto{\pgfqpoint{2.323423in}{0.645641in}}%
\pgfpathlineto{\pgfqpoint{2.323957in}{0.636016in}}%
\pgfpathlineto{\pgfqpoint{2.324491in}{0.645446in}}%
\pgfpathlineto{\pgfqpoint{2.325025in}{0.647464in}}%
\pgfpathlineto{\pgfqpoint{2.325558in}{0.640043in}}%
\pgfpathlineto{\pgfqpoint{2.326092in}{0.643481in}}%
\pgfpathlineto{\pgfqpoint{2.326626in}{0.648942in}}%
\pgfpathlineto{\pgfqpoint{2.327159in}{0.647286in}}%
\pgfpathlineto{\pgfqpoint{2.327693in}{0.648054in}}%
\pgfpathlineto{\pgfqpoint{2.328227in}{0.636975in}}%
\pgfpathlineto{\pgfqpoint{2.328760in}{0.642340in}}%
\pgfpathlineto{\pgfqpoint{2.329294in}{0.646798in}}%
\pgfpathlineto{\pgfqpoint{2.329828in}{0.643006in}}%
\pgfpathlineto{\pgfqpoint{2.330362in}{0.637271in}}%
\pgfpathlineto{\pgfqpoint{2.330895in}{0.645089in}}%
\pgfpathlineto{\pgfqpoint{2.331429in}{0.639315in}}%
\pgfpathlineto{\pgfqpoint{2.331963in}{0.636343in}}%
\pgfpathlineto{\pgfqpoint{2.332496in}{0.637598in}}%
\pgfpathlineto{\pgfqpoint{2.333030in}{0.648367in}}%
\pgfpathlineto{\pgfqpoint{2.333564in}{0.639752in}}%
\pgfpathlineto{\pgfqpoint{2.334097in}{0.646468in}}%
\pgfpathlineto{\pgfqpoint{2.334631in}{0.635645in}}%
\pgfpathlineto{\pgfqpoint{2.335165in}{0.645693in}}%
\pgfpathlineto{\pgfqpoint{2.335699in}{0.637053in}}%
\pgfpathlineto{\pgfqpoint{2.336232in}{0.648514in}}%
\pgfpathlineto{\pgfqpoint{2.336766in}{0.642050in}}%
\pgfpathlineto{\pgfqpoint{2.337300in}{0.643403in}}%
\pgfpathlineto{\pgfqpoint{2.337833in}{0.638760in}}%
\pgfpathlineto{\pgfqpoint{2.338367in}{0.638866in}}%
\pgfpathlineto{\pgfqpoint{2.338901in}{0.640842in}}%
\pgfpathlineto{\pgfqpoint{2.339434in}{0.639728in}}%
\pgfpathlineto{\pgfqpoint{2.339968in}{0.639154in}}%
\pgfpathlineto{\pgfqpoint{2.341036in}{0.645622in}}%
\pgfpathlineto{\pgfqpoint{2.341569in}{0.643314in}}%
\pgfpathlineto{\pgfqpoint{2.342103in}{0.644347in}}%
\pgfpathlineto{\pgfqpoint{2.342637in}{0.639316in}}%
\pgfpathlineto{\pgfqpoint{2.343170in}{0.644114in}}%
\pgfpathlineto{\pgfqpoint{2.343704in}{0.643616in}}%
\pgfpathlineto{\pgfqpoint{2.344238in}{0.637950in}}%
\pgfpathlineto{\pgfqpoint{2.344771in}{0.639808in}}%
\pgfpathlineto{\pgfqpoint{2.345839in}{0.637543in}}%
\pgfpathlineto{\pgfqpoint{2.346906in}{0.646481in}}%
\pgfpathlineto{\pgfqpoint{2.348507in}{0.637710in}}%
\pgfpathlineto{\pgfqpoint{2.349575in}{0.635604in}}%
\pgfpathlineto{\pgfqpoint{2.350108in}{0.643843in}}%
\pgfpathlineto{\pgfqpoint{2.350642in}{0.638662in}}%
\pgfpathlineto{\pgfqpoint{2.352243in}{0.643793in}}%
\pgfpathlineto{\pgfqpoint{2.353311in}{0.640335in}}%
\pgfpathlineto{\pgfqpoint{2.354378in}{0.645712in}}%
\pgfpathlineto{\pgfqpoint{2.355446in}{0.637006in}}%
\pgfpathlineto{\pgfqpoint{2.355979in}{0.648066in}}%
\pgfpathlineto{\pgfqpoint{2.356513in}{0.638513in}}%
\pgfpathlineto{\pgfqpoint{2.358114in}{0.639522in}}%
\pgfpathlineto{\pgfqpoint{2.358648in}{0.638368in}}%
\pgfpathlineto{\pgfqpoint{2.359181in}{0.639474in}}%
\pgfpathlineto{\pgfqpoint{2.360249in}{0.642446in}}%
\pgfpathlineto{\pgfqpoint{2.360783in}{0.640991in}}%
\pgfpathlineto{\pgfqpoint{2.361850in}{0.639312in}}%
\pgfpathlineto{\pgfqpoint{2.362384in}{0.644335in}}%
\pgfpathlineto{\pgfqpoint{2.362917in}{0.636694in}}%
\pgfpathlineto{\pgfqpoint{2.363451in}{0.640025in}}%
\pgfpathlineto{\pgfqpoint{2.363985in}{0.638744in}}%
\pgfpathlineto{\pgfqpoint{2.364518in}{0.639940in}}%
\pgfpathlineto{\pgfqpoint{2.365052in}{0.639786in}}%
\pgfpathlineto{\pgfqpoint{2.366120in}{0.642884in}}%
\pgfpathlineto{\pgfqpoint{2.366653in}{0.641530in}}%
\pgfpathlineto{\pgfqpoint{2.367187in}{0.643543in}}%
\pgfpathlineto{\pgfqpoint{2.367721in}{0.636890in}}%
\pgfpathlineto{\pgfqpoint{2.368254in}{0.647258in}}%
\pgfpathlineto{\pgfqpoint{2.368788in}{0.637873in}}%
\pgfpathlineto{\pgfqpoint{2.369855in}{0.636084in}}%
\pgfpathlineto{\pgfqpoint{2.370923in}{0.644999in}}%
\pgfpathlineto{\pgfqpoint{2.371457in}{0.641411in}}%
\pgfpathlineto{\pgfqpoint{2.371990in}{0.644518in}}%
\pgfpathlineto{\pgfqpoint{2.372524in}{0.643366in}}%
\pgfpathlineto{\pgfqpoint{2.373591in}{0.644959in}}%
\pgfpathlineto{\pgfqpoint{2.374125in}{0.636987in}}%
\pgfpathlineto{\pgfqpoint{2.374659in}{0.638052in}}%
\pgfpathlineto{\pgfqpoint{2.375726in}{0.640256in}}%
\pgfpathlineto{\pgfqpoint{2.376260in}{0.638074in}}%
\pgfpathlineto{\pgfqpoint{2.376794in}{0.645288in}}%
\pgfpathlineto{\pgfqpoint{2.377327in}{0.640735in}}%
\pgfpathlineto{\pgfqpoint{2.378928in}{0.644741in}}%
\pgfpathlineto{\pgfqpoint{2.379462in}{0.640395in}}%
\pgfpathlineto{\pgfqpoint{2.380529in}{0.655083in}}%
\pgfpathlineto{\pgfqpoint{2.381063in}{0.637828in}}%
\pgfpathlineto{\pgfqpoint{2.382131in}{0.638745in}}%
\pgfpathlineto{\pgfqpoint{2.383732in}{0.657931in}}%
\pgfpathlineto{\pgfqpoint{2.384265in}{0.639201in}}%
\pgfpathlineto{\pgfqpoint{2.384799in}{0.660702in}}%
\pgfpathlineto{\pgfqpoint{2.385333in}{0.643094in}}%
\pgfpathlineto{\pgfqpoint{2.385866in}{0.641315in}}%
\pgfpathlineto{\pgfqpoint{2.386400in}{0.648189in}}%
\pgfpathlineto{\pgfqpoint{2.386934in}{0.638802in}}%
\pgfpathlineto{\pgfqpoint{2.387468in}{0.645465in}}%
\pgfpathlineto{\pgfqpoint{2.388001in}{0.646334in}}%
\pgfpathlineto{\pgfqpoint{2.388535in}{0.643274in}}%
\pgfpathlineto{\pgfqpoint{2.389069in}{0.654087in}}%
\pgfpathlineto{\pgfqpoint{2.389602in}{0.640398in}}%
\pgfpathlineto{\pgfqpoint{2.390136in}{0.641080in}}%
\pgfpathlineto{\pgfqpoint{2.390670in}{0.645580in}}%
\pgfpathlineto{\pgfqpoint{2.391203in}{0.645232in}}%
\pgfpathlineto{\pgfqpoint{2.392271in}{0.641995in}}%
\pgfpathlineto{\pgfqpoint{2.393872in}{0.649821in}}%
\pgfpathlineto{\pgfqpoint{2.394939in}{0.642972in}}%
\pgfpathlineto{\pgfqpoint{2.395473in}{0.646346in}}%
\pgfpathlineto{\pgfqpoint{2.396007in}{0.643934in}}%
\pgfpathlineto{\pgfqpoint{2.397074in}{0.647938in}}%
\pgfpathlineto{\pgfqpoint{2.398142in}{0.639090in}}%
\pgfpathlineto{\pgfqpoint{2.398675in}{0.640514in}}%
\pgfpathlineto{\pgfqpoint{2.399209in}{0.650275in}}%
\pgfpathlineto{\pgfqpoint{2.399743in}{0.648591in}}%
\pgfpathlineto{\pgfqpoint{2.400276in}{0.647800in}}%
\pgfpathlineto{\pgfqpoint{2.401877in}{0.638525in}}%
\pgfpathlineto{\pgfqpoint{2.402411in}{0.639900in}}%
\pgfpathlineto{\pgfqpoint{2.403479in}{0.649595in}}%
\pgfpathlineto{\pgfqpoint{2.404012in}{0.636164in}}%
\pgfpathlineto{\pgfqpoint{2.404546in}{0.640433in}}%
\pgfpathlineto{\pgfqpoint{2.405613in}{0.645842in}}%
\pgfpathlineto{\pgfqpoint{2.406681in}{0.637314in}}%
\pgfpathlineto{\pgfqpoint{2.407214in}{0.640422in}}%
\pgfpathlineto{\pgfqpoint{2.407748in}{0.638603in}}%
\pgfpathlineto{\pgfqpoint{2.409349in}{0.649890in}}%
\pgfpathlineto{\pgfqpoint{2.410417in}{0.637470in}}%
\pgfpathlineto{\pgfqpoint{2.411484in}{0.640856in}}%
\pgfpathlineto{\pgfqpoint{2.412018in}{0.639042in}}%
\pgfpathlineto{\pgfqpoint{2.412551in}{0.640771in}}%
\pgfpathlineto{\pgfqpoint{2.414153in}{0.647211in}}%
\pgfpathlineto{\pgfqpoint{2.415220in}{0.641210in}}%
\pgfpathlineto{\pgfqpoint{2.416287in}{0.647353in}}%
\pgfpathlineto{\pgfqpoint{2.417888in}{0.642839in}}%
\pgfpathlineto{\pgfqpoint{2.418422in}{0.643756in}}%
\pgfpathlineto{\pgfqpoint{2.418956in}{0.638924in}}%
\pgfpathlineto{\pgfqpoint{2.420023in}{0.647449in}}%
\pgfpathlineto{\pgfqpoint{2.421624in}{0.640645in}}%
\pgfpathlineto{\pgfqpoint{2.422158in}{0.660654in}}%
\pgfpathlineto{\pgfqpoint{2.422692in}{0.643847in}}%
\pgfpathlineto{\pgfqpoint{2.423226in}{0.645245in}}%
\pgfpathlineto{\pgfqpoint{2.424827in}{0.637006in}}%
\pgfpathlineto{\pgfqpoint{2.425894in}{0.658878in}}%
\pgfpathlineto{\pgfqpoint{2.426961in}{0.654827in}}%
\pgfpathlineto{\pgfqpoint{2.427495in}{0.649247in}}%
\pgfpathlineto{\pgfqpoint{2.428029in}{0.659314in}}%
\pgfpathlineto{\pgfqpoint{2.428563in}{0.639986in}}%
\pgfpathlineto{\pgfqpoint{2.429096in}{0.651611in}}%
\pgfpathlineto{\pgfqpoint{2.430164in}{0.647815in}}%
\pgfpathlineto{\pgfqpoint{2.430697in}{0.638578in}}%
\pgfpathlineto{\pgfqpoint{2.431231in}{0.641461in}}%
\pgfpathlineto{\pgfqpoint{2.431765in}{0.640964in}}%
\pgfpathlineto{\pgfqpoint{2.433366in}{0.651654in}}%
\pgfpathlineto{\pgfqpoint{2.433900in}{0.655251in}}%
\pgfpathlineto{\pgfqpoint{2.434967in}{0.640997in}}%
\pgfpathlineto{\pgfqpoint{2.436568in}{0.652540in}}%
\pgfpathlineto{\pgfqpoint{2.437102in}{0.645846in}}%
\pgfpathlineto{\pgfqpoint{2.437635in}{0.664535in}}%
\pgfpathlineto{\pgfqpoint{2.438703in}{0.642550in}}%
\pgfpathlineto{\pgfqpoint{2.440304in}{0.661945in}}%
\pgfpathlineto{\pgfqpoint{2.440838in}{0.647652in}}%
\pgfpathlineto{\pgfqpoint{2.441371in}{0.660139in}}%
\pgfpathlineto{\pgfqpoint{2.442439in}{0.647034in}}%
\pgfpathlineto{\pgfqpoint{2.442972in}{0.654557in}}%
\pgfpathlineto{\pgfqpoint{2.443506in}{0.643017in}}%
\pgfpathlineto{\pgfqpoint{2.444040in}{0.685641in}}%
\pgfpathlineto{\pgfqpoint{2.444574in}{0.640510in}}%
\pgfpathlineto{\pgfqpoint{2.445107in}{0.653092in}}%
\pgfpathlineto{\pgfqpoint{2.445641in}{0.649952in}}%
\pgfpathlineto{\pgfqpoint{2.446175in}{0.646393in}}%
\pgfpathlineto{\pgfqpoint{2.446708in}{0.652266in}}%
\pgfpathlineto{\pgfqpoint{2.447242in}{0.641753in}}%
\pgfpathlineto{\pgfqpoint{2.447776in}{0.643068in}}%
\pgfpathlineto{\pgfqpoint{2.448309in}{0.643093in}}%
\pgfpathlineto{\pgfqpoint{2.448843in}{0.672957in}}%
\pgfpathlineto{\pgfqpoint{2.449377in}{0.644708in}}%
\pgfpathlineto{\pgfqpoint{2.449911in}{0.641624in}}%
\pgfpathlineto{\pgfqpoint{2.451512in}{0.654813in}}%
\pgfpathlineto{\pgfqpoint{2.452045in}{0.641486in}}%
\pgfpathlineto{\pgfqpoint{2.452579in}{0.643294in}}%
\pgfpathlineto{\pgfqpoint{2.454180in}{0.659903in}}%
\pgfpathlineto{\pgfqpoint{2.455781in}{0.638334in}}%
\pgfpathlineto{\pgfqpoint{2.456849in}{0.650181in}}%
\pgfpathlineto{\pgfqpoint{2.457382in}{0.638340in}}%
\pgfpathlineto{\pgfqpoint{2.457916in}{0.654550in}}%
\pgfpathlineto{\pgfqpoint{2.458450in}{0.650730in}}%
\pgfpathlineto{\pgfqpoint{2.458983in}{0.640042in}}%
\pgfpathlineto{\pgfqpoint{2.460585in}{0.659878in}}%
\pgfpathlineto{\pgfqpoint{2.462186in}{0.637315in}}%
\pgfpathlineto{\pgfqpoint{2.463787in}{0.652720in}}%
\pgfpathlineto{\pgfqpoint{2.464854in}{0.638430in}}%
\pgfpathlineto{\pgfqpoint{2.465388in}{0.657040in}}%
\pgfpathlineto{\pgfqpoint{2.465922in}{0.638755in}}%
\pgfpathlineto{\pgfqpoint{2.466455in}{0.641870in}}%
\pgfpathlineto{\pgfqpoint{2.466989in}{0.635738in}}%
\pgfpathlineto{\pgfqpoint{2.467523in}{0.639928in}}%
\pgfpathlineto{\pgfqpoint{2.468590in}{0.650512in}}%
\pgfpathlineto{\pgfqpoint{2.469124in}{0.643056in}}%
\pgfpathlineto{\pgfqpoint{2.469657in}{0.649674in}}%
\pgfpathlineto{\pgfqpoint{2.470191in}{0.653456in}}%
\pgfpathlineto{\pgfqpoint{2.470725in}{0.636312in}}%
\pgfpathlineto{\pgfqpoint{2.471259in}{0.657868in}}%
\pgfpathlineto{\pgfqpoint{2.471792in}{0.654996in}}%
\pgfpathlineto{\pgfqpoint{2.472326in}{0.652232in}}%
\pgfpathlineto{\pgfqpoint{2.472860in}{0.642980in}}%
\pgfpathlineto{\pgfqpoint{2.473393in}{0.656245in}}%
\pgfpathlineto{\pgfqpoint{2.473927in}{0.638212in}}%
\pgfpathlineto{\pgfqpoint{2.474461in}{0.652498in}}%
\pgfpathlineto{\pgfqpoint{2.474994in}{0.645754in}}%
\pgfpathlineto{\pgfqpoint{2.476062in}{0.645931in}}%
\pgfpathlineto{\pgfqpoint{2.476596in}{0.642524in}}%
\pgfpathlineto{\pgfqpoint{2.477663in}{0.659731in}}%
\pgfpathlineto{\pgfqpoint{2.478197in}{0.642509in}}%
\pgfpathlineto{\pgfqpoint{2.478730in}{0.651124in}}%
\pgfpathlineto{\pgfqpoint{2.479264in}{0.655439in}}%
\pgfpathlineto{\pgfqpoint{2.480331in}{0.644720in}}%
\pgfpathlineto{\pgfqpoint{2.481399in}{0.659606in}}%
\pgfpathlineto{\pgfqpoint{2.481933in}{0.658483in}}%
\pgfpathlineto{\pgfqpoint{2.482466in}{0.650004in}}%
\pgfpathlineto{\pgfqpoint{2.483000in}{0.658350in}}%
\pgfpathlineto{\pgfqpoint{2.483534in}{0.670892in}}%
\pgfpathlineto{\pgfqpoint{2.484067in}{0.659524in}}%
\pgfpathlineto{\pgfqpoint{2.484601in}{0.660279in}}%
\pgfpathlineto{\pgfqpoint{2.485135in}{0.658793in}}%
\pgfpathlineto{\pgfqpoint{2.486202in}{0.640575in}}%
\pgfpathlineto{\pgfqpoint{2.486736in}{0.659544in}}%
\pgfpathlineto{\pgfqpoint{2.487270in}{0.655196in}}%
\pgfpathlineto{\pgfqpoint{2.487803in}{0.635934in}}%
\pgfpathlineto{\pgfqpoint{2.488337in}{0.649056in}}%
\pgfpathlineto{\pgfqpoint{2.488871in}{0.658794in}}%
\pgfpathlineto{\pgfqpoint{2.489404in}{0.651061in}}%
\pgfpathlineto{\pgfqpoint{2.490472in}{0.644623in}}%
\pgfpathlineto{\pgfqpoint{2.491006in}{0.671844in}}%
\pgfpathlineto{\pgfqpoint{2.491539in}{0.666618in}}%
\pgfpathlineto{\pgfqpoint{2.492073in}{0.648093in}}%
\pgfpathlineto{\pgfqpoint{2.492607in}{0.666557in}}%
\pgfpathlineto{\pgfqpoint{2.493674in}{0.648704in}}%
\pgfpathlineto{\pgfqpoint{2.495275in}{0.679767in}}%
\pgfpathlineto{\pgfqpoint{2.496343in}{0.642336in}}%
\pgfpathlineto{\pgfqpoint{2.497944in}{0.677107in}}%
\pgfpathlineto{\pgfqpoint{2.498477in}{0.666874in}}%
\pgfpathlineto{\pgfqpoint{2.499011in}{0.680985in}}%
\pgfpathlineto{\pgfqpoint{2.499545in}{0.678302in}}%
\pgfpathlineto{\pgfqpoint{2.500078in}{0.656417in}}%
\pgfpathlineto{\pgfqpoint{2.500612in}{0.679596in}}%
\pgfpathlineto{\pgfqpoint{2.502213in}{0.649527in}}%
\pgfpathlineto{\pgfqpoint{2.502747in}{0.653008in}}%
\pgfpathlineto{\pgfqpoint{2.503281in}{0.650218in}}%
\pgfpathlineto{\pgfqpoint{2.503814in}{0.672269in}}%
\pgfpathlineto{\pgfqpoint{2.504348in}{0.666023in}}%
\pgfpathlineto{\pgfqpoint{2.505949in}{0.639561in}}%
\pgfpathlineto{\pgfqpoint{2.506483in}{0.694682in}}%
\pgfpathlineto{\pgfqpoint{2.507017in}{0.640899in}}%
\pgfpathlineto{\pgfqpoint{2.509151in}{0.653336in}}%
\pgfpathlineto{\pgfqpoint{2.509685in}{0.651826in}}%
\pgfpathlineto{\pgfqpoint{2.510219in}{0.662277in}}%
\pgfpathlineto{\pgfqpoint{2.510752in}{0.656526in}}%
\pgfpathlineto{\pgfqpoint{2.511286in}{0.660643in}}%
\pgfpathlineto{\pgfqpoint{2.512887in}{0.638848in}}%
\pgfpathlineto{\pgfqpoint{2.513421in}{0.642112in}}%
\pgfpathlineto{\pgfqpoint{2.514488in}{0.675067in}}%
\pgfpathlineto{\pgfqpoint{2.515022in}{0.658570in}}%
\pgfpathlineto{\pgfqpoint{2.515556in}{0.657932in}}%
\pgfpathlineto{\pgfqpoint{2.517157in}{0.639286in}}%
\pgfpathlineto{\pgfqpoint{2.517691in}{0.655193in}}%
\pgfpathlineto{\pgfqpoint{2.518224in}{0.645157in}}%
\pgfpathlineto{\pgfqpoint{2.518758in}{0.651947in}}%
\pgfpathlineto{\pgfqpoint{2.519292in}{0.639401in}}%
\pgfpathlineto{\pgfqpoint{2.519825in}{0.647531in}}%
\pgfpathlineto{\pgfqpoint{2.520359in}{0.665889in}}%
\pgfpathlineto{\pgfqpoint{2.520893in}{0.648702in}}%
\pgfpathlineto{\pgfqpoint{2.521426in}{0.636423in}}%
\pgfpathlineto{\pgfqpoint{2.521960in}{0.641741in}}%
\pgfpathlineto{\pgfqpoint{2.523561in}{0.650993in}}%
\pgfpathlineto{\pgfqpoint{2.524095in}{0.645019in}}%
\pgfpathlineto{\pgfqpoint{2.524629in}{0.652908in}}%
\pgfpathlineto{\pgfqpoint{2.525162in}{0.642187in}}%
\pgfpathlineto{\pgfqpoint{2.525696in}{0.651791in}}%
\pgfpathlineto{\pgfqpoint{2.526230in}{0.646849in}}%
\pgfpathlineto{\pgfqpoint{2.526763in}{0.651881in}}%
\pgfpathlineto{\pgfqpoint{2.527831in}{0.659449in}}%
\pgfpathlineto{\pgfqpoint{2.528365in}{0.654970in}}%
\pgfpathlineto{\pgfqpoint{2.528898in}{0.647752in}}%
\pgfpathlineto{\pgfqpoint{2.529432in}{0.650805in}}%
\pgfpathlineto{\pgfqpoint{2.529966in}{0.653844in}}%
\pgfpathlineto{\pgfqpoint{2.530499in}{0.666545in}}%
\pgfpathlineto{\pgfqpoint{2.531033in}{0.636336in}}%
\pgfpathlineto{\pgfqpoint{2.531567in}{0.657089in}}%
\pgfpathlineto{\pgfqpoint{2.532100in}{0.649534in}}%
\pgfpathlineto{\pgfqpoint{2.532634in}{0.672506in}}%
\pgfpathlineto{\pgfqpoint{2.533168in}{0.661162in}}%
\pgfpathlineto{\pgfqpoint{2.533702in}{0.659631in}}%
\pgfpathlineto{\pgfqpoint{2.534235in}{0.647062in}}%
\pgfpathlineto{\pgfqpoint{2.534769in}{0.662705in}}%
\pgfpathlineto{\pgfqpoint{2.535836in}{0.662251in}}%
\pgfpathlineto{\pgfqpoint{2.536904in}{0.645632in}}%
\pgfpathlineto{\pgfqpoint{2.538505in}{0.693374in}}%
\pgfpathlineto{\pgfqpoint{2.539039in}{0.646168in}}%
\pgfpathlineto{\pgfqpoint{2.539572in}{0.678210in}}%
\pgfpathlineto{\pgfqpoint{2.540640in}{0.678884in}}%
\pgfpathlineto{\pgfqpoint{2.541173in}{0.639960in}}%
\pgfpathlineto{\pgfqpoint{2.542774in}{0.674307in}}%
\pgfpathlineto{\pgfqpoint{2.543308in}{0.654446in}}%
\pgfpathlineto{\pgfqpoint{2.543842in}{0.667595in}}%
\pgfpathlineto{\pgfqpoint{2.544909in}{0.649680in}}%
\pgfpathlineto{\pgfqpoint{2.545443in}{0.658284in}}%
\pgfpathlineto{\pgfqpoint{2.546510in}{0.676788in}}%
\pgfpathlineto{\pgfqpoint{2.547044in}{0.648631in}}%
\pgfpathlineto{\pgfqpoint{2.547578in}{0.663875in}}%
\pgfpathlineto{\pgfqpoint{2.548111in}{0.701712in}}%
\pgfpathlineto{\pgfqpoint{2.548645in}{0.670101in}}%
\pgfpathlineto{\pgfqpoint{2.549179in}{0.668835in}}%
\pgfpathlineto{\pgfqpoint{2.550246in}{0.659484in}}%
\pgfpathlineto{\pgfqpoint{2.550780in}{0.677480in}}%
\pgfpathlineto{\pgfqpoint{2.551314in}{0.640693in}}%
\pgfpathlineto{\pgfqpoint{2.551847in}{0.673506in}}%
\pgfpathlineto{\pgfqpoint{2.552381in}{0.697275in}}%
\pgfpathlineto{\pgfqpoint{2.553448in}{0.660269in}}%
\pgfpathlineto{\pgfqpoint{2.553982in}{0.666084in}}%
\pgfpathlineto{\pgfqpoint{2.554516in}{0.721285in}}%
\pgfpathlineto{\pgfqpoint{2.555050in}{0.651588in}}%
\pgfpathlineto{\pgfqpoint{2.555583in}{0.659085in}}%
\pgfpathlineto{\pgfqpoint{2.556117in}{0.658639in}}%
\pgfpathlineto{\pgfqpoint{2.556651in}{0.697101in}}%
\pgfpathlineto{\pgfqpoint{2.557184in}{0.662178in}}%
\pgfpathlineto{\pgfqpoint{2.557718in}{0.664431in}}%
\pgfpathlineto{\pgfqpoint{2.558786in}{0.700821in}}%
\pgfpathlineto{\pgfqpoint{2.559853in}{0.663547in}}%
\pgfpathlineto{\pgfqpoint{2.560387in}{0.677670in}}%
\pgfpathlineto{\pgfqpoint{2.560920in}{0.648492in}}%
\pgfpathlineto{\pgfqpoint{2.561454in}{0.680420in}}%
\pgfpathlineto{\pgfqpoint{2.561988in}{0.652711in}}%
\pgfpathlineto{\pgfqpoint{2.562521in}{0.658444in}}%
\pgfpathlineto{\pgfqpoint{2.563055in}{0.653656in}}%
\pgfpathlineto{\pgfqpoint{2.563589in}{0.651119in}}%
\pgfpathlineto{\pgfqpoint{2.564123in}{0.653825in}}%
\pgfpathlineto{\pgfqpoint{2.565724in}{0.672818in}}%
\pgfpathlineto{\pgfqpoint{2.567325in}{0.648863in}}%
\pgfpathlineto{\pgfqpoint{2.567858in}{0.666706in}}%
\pgfpathlineto{\pgfqpoint{2.568392in}{0.646805in}}%
\pgfpathlineto{\pgfqpoint{2.568926in}{0.663958in}}%
\pgfpathlineto{\pgfqpoint{2.569460in}{0.681635in}}%
\pgfpathlineto{\pgfqpoint{2.571061in}{0.639987in}}%
\pgfpathlineto{\pgfqpoint{2.571594in}{0.672068in}}%
\pgfpathlineto{\pgfqpoint{2.572128in}{0.657570in}}%
\pgfpathlineto{\pgfqpoint{2.574263in}{0.644298in}}%
\pgfpathlineto{\pgfqpoint{2.575864in}{0.671124in}}%
\pgfpathlineto{\pgfqpoint{2.576398in}{0.636045in}}%
\pgfpathlineto{\pgfqpoint{2.576931in}{0.639218in}}%
\pgfpathlineto{\pgfqpoint{2.578532in}{0.656047in}}%
\pgfpathlineto{\pgfqpoint{2.579600in}{0.662796in}}%
\pgfpathlineto{\pgfqpoint{2.581201in}{0.641187in}}%
\pgfpathlineto{\pgfqpoint{2.582268in}{0.658656in}}%
\pgfpathlineto{\pgfqpoint{2.582802in}{0.658233in}}%
\pgfpathlineto{\pgfqpoint{2.583336in}{0.635874in}}%
\pgfpathlineto{\pgfqpoint{2.584937in}{0.676783in}}%
\pgfpathlineto{\pgfqpoint{2.586004in}{0.640890in}}%
\pgfpathlineto{\pgfqpoint{2.587605in}{0.672153in}}%
\pgfpathlineto{\pgfqpoint{2.589206in}{0.646178in}}%
\pgfpathlineto{\pgfqpoint{2.589740in}{0.679528in}}%
\pgfpathlineto{\pgfqpoint{2.590274in}{0.663508in}}%
\pgfpathlineto{\pgfqpoint{2.590808in}{0.669682in}}%
\pgfpathlineto{\pgfqpoint{2.591341in}{0.652805in}}%
\pgfpathlineto{\pgfqpoint{2.591875in}{0.660662in}}%
\pgfpathlineto{\pgfqpoint{2.592409in}{0.676213in}}%
\pgfpathlineto{\pgfqpoint{2.592942in}{0.647121in}}%
\pgfpathlineto{\pgfqpoint{2.594543in}{0.690556in}}%
\pgfpathlineto{\pgfqpoint{2.595077in}{0.672579in}}%
\pgfpathlineto{\pgfqpoint{2.595611in}{0.701116in}}%
\pgfpathlineto{\pgfqpoint{2.596145in}{0.651589in}}%
\pgfpathlineto{\pgfqpoint{2.596678in}{0.660406in}}%
\pgfpathlineto{\pgfqpoint{2.597746in}{0.690917in}}%
\pgfpathlineto{\pgfqpoint{2.599347in}{0.642466in}}%
\pgfpathlineto{\pgfqpoint{2.599880in}{0.654800in}}%
\pgfpathlineto{\pgfqpoint{2.600414in}{0.641375in}}%
\pgfpathlineto{\pgfqpoint{2.600948in}{0.650774in}}%
\pgfpathlineto{\pgfqpoint{2.601482in}{0.688769in}}%
\pgfpathlineto{\pgfqpoint{2.602015in}{0.673542in}}%
\pgfpathlineto{\pgfqpoint{2.602549in}{0.666262in}}%
\pgfpathlineto{\pgfqpoint{2.603083in}{0.709302in}}%
\pgfpathlineto{\pgfqpoint{2.603616in}{0.680761in}}%
\pgfpathlineto{\pgfqpoint{2.605217in}{0.638692in}}%
\pgfpathlineto{\pgfqpoint{2.607352in}{0.711668in}}%
\pgfpathlineto{\pgfqpoint{2.608420in}{0.673765in}}%
\pgfpathlineto{\pgfqpoint{2.608953in}{0.690408in}}%
\pgfpathlineto{\pgfqpoint{2.609487in}{0.684433in}}%
\pgfpathlineto{\pgfqpoint{2.610021in}{0.677759in}}%
\pgfpathlineto{\pgfqpoint{2.610554in}{0.636615in}}%
\pgfpathlineto{\pgfqpoint{2.611622in}{0.696597in}}%
\pgfpathlineto{\pgfqpoint{2.612156in}{0.690000in}}%
\pgfpathlineto{\pgfqpoint{2.613757in}{0.647055in}}%
\pgfpathlineto{\pgfqpoint{2.614290in}{0.715219in}}%
\pgfpathlineto{\pgfqpoint{2.614824in}{0.678942in}}%
\pgfpathlineto{\pgfqpoint{2.615358in}{0.672545in}}%
\pgfpathlineto{\pgfqpoint{2.615891in}{0.673201in}}%
\pgfpathlineto{\pgfqpoint{2.616425in}{0.678080in}}%
\pgfpathlineto{\pgfqpoint{2.616959in}{0.649271in}}%
\pgfpathlineto{\pgfqpoint{2.617493in}{0.701421in}}%
\pgfpathlineto{\pgfqpoint{2.618026in}{0.676944in}}%
\pgfpathlineto{\pgfqpoint{2.620161in}{0.641196in}}%
\pgfpathlineto{\pgfqpoint{2.621762in}{0.681022in}}%
\pgfpathlineto{\pgfqpoint{2.623897in}{0.648003in}}%
\pgfpathlineto{\pgfqpoint{2.624431in}{0.658012in}}%
\pgfpathlineto{\pgfqpoint{2.624964in}{0.640820in}}%
\pgfpathlineto{\pgfqpoint{2.626566in}{0.688847in}}%
\pgfpathlineto{\pgfqpoint{2.627633in}{0.643580in}}%
\pgfpathlineto{\pgfqpoint{2.628167in}{0.650921in}}%
\pgfpathlineto{\pgfqpoint{2.628700in}{0.651884in}}%
\pgfpathlineto{\pgfqpoint{2.629768in}{0.678609in}}%
\pgfpathlineto{\pgfqpoint{2.630301in}{0.666737in}}%
\pgfpathlineto{\pgfqpoint{2.630835in}{0.663316in}}%
\pgfpathlineto{\pgfqpoint{2.632436in}{0.642149in}}%
\pgfpathlineto{\pgfqpoint{2.632970in}{0.660694in}}%
\pgfpathlineto{\pgfqpoint{2.634037in}{0.660275in}}%
\pgfpathlineto{\pgfqpoint{2.634571in}{0.648441in}}%
\pgfpathlineto{\pgfqpoint{2.635638in}{0.669186in}}%
\pgfpathlineto{\pgfqpoint{2.636172in}{0.644749in}}%
\pgfpathlineto{\pgfqpoint{2.636706in}{0.654853in}}%
\pgfpathlineto{\pgfqpoint{2.637240in}{0.652668in}}%
\pgfpathlineto{\pgfqpoint{2.638841in}{0.673273in}}%
\pgfpathlineto{\pgfqpoint{2.639374in}{0.650778in}}%
\pgfpathlineto{\pgfqpoint{2.639908in}{0.690978in}}%
\pgfpathlineto{\pgfqpoint{2.640442in}{0.640286in}}%
\pgfpathlineto{\pgfqpoint{2.640975in}{0.655681in}}%
\pgfpathlineto{\pgfqpoint{2.641509in}{0.667239in}}%
\pgfpathlineto{\pgfqpoint{2.642043in}{0.655972in}}%
\pgfpathlineto{\pgfqpoint{2.642577in}{0.659295in}}%
\pgfpathlineto{\pgfqpoint{2.643110in}{0.678232in}}%
\pgfpathlineto{\pgfqpoint{2.643644in}{0.664951in}}%
\pgfpathlineto{\pgfqpoint{2.644711in}{0.682662in}}%
\pgfpathlineto{\pgfqpoint{2.646846in}{0.652097in}}%
\pgfpathlineto{\pgfqpoint{2.647914in}{0.670719in}}%
\pgfpathlineto{\pgfqpoint{2.648447in}{0.651840in}}%
\pgfpathlineto{\pgfqpoint{2.648981in}{0.719192in}}%
\pgfpathlineto{\pgfqpoint{2.649515in}{0.675022in}}%
\pgfpathlineto{\pgfqpoint{2.650048in}{0.687797in}}%
\pgfpathlineto{\pgfqpoint{2.650582in}{0.671670in}}%
\pgfpathlineto{\pgfqpoint{2.651116in}{0.706045in}}%
\pgfpathlineto{\pgfqpoint{2.651649in}{0.649444in}}%
\pgfpathlineto{\pgfqpoint{2.652183in}{0.713993in}}%
\pgfpathlineto{\pgfqpoint{2.652717in}{0.675388in}}%
\pgfpathlineto{\pgfqpoint{2.653784in}{0.647469in}}%
\pgfpathlineto{\pgfqpoint{2.654852in}{0.693053in}}%
\pgfpathlineto{\pgfqpoint{2.655385in}{0.641407in}}%
\pgfpathlineto{\pgfqpoint{2.655919in}{0.687459in}}%
\pgfpathlineto{\pgfqpoint{2.657520in}{0.647879in}}%
\pgfpathlineto{\pgfqpoint{2.658588in}{0.731644in}}%
\pgfpathlineto{\pgfqpoint{2.659121in}{0.697386in}}%
\pgfpathlineto{\pgfqpoint{2.660722in}{0.655954in}}%
\pgfpathlineto{\pgfqpoint{2.662323in}{0.716608in}}%
\pgfpathlineto{\pgfqpoint{2.663925in}{0.682904in}}%
\pgfpathlineto{\pgfqpoint{2.664458in}{0.703340in}}%
\pgfpathlineto{\pgfqpoint{2.664992in}{0.702996in}}%
\pgfpathlineto{\pgfqpoint{2.665526in}{0.653189in}}%
\pgfpathlineto{\pgfqpoint{2.666059in}{0.658421in}}%
\pgfpathlineto{\pgfqpoint{2.666593in}{0.712478in}}%
\pgfpathlineto{\pgfqpoint{2.667127in}{0.681983in}}%
\pgfpathlineto{\pgfqpoint{2.667660in}{0.676715in}}%
\pgfpathlineto{\pgfqpoint{2.668194in}{0.655440in}}%
\pgfpathlineto{\pgfqpoint{2.668728in}{0.702512in}}%
\pgfpathlineto{\pgfqpoint{2.669262in}{0.670728in}}%
\pgfpathlineto{\pgfqpoint{2.669795in}{0.695166in}}%
\pgfpathlineto{\pgfqpoint{2.670863in}{0.694065in}}%
\pgfpathlineto{\pgfqpoint{2.671396in}{0.687314in}}%
\pgfpathlineto{\pgfqpoint{2.671930in}{0.700586in}}%
\pgfpathlineto{\pgfqpoint{2.672997in}{0.660052in}}%
\pgfpathlineto{\pgfqpoint{2.673531in}{0.712191in}}%
\pgfpathlineto{\pgfqpoint{2.674065in}{0.691005in}}%
\pgfpathlineto{\pgfqpoint{2.674599in}{0.690810in}}%
\pgfpathlineto{\pgfqpoint{2.675132in}{0.674603in}}%
\pgfpathlineto{\pgfqpoint{2.675666in}{0.675324in}}%
\pgfpathlineto{\pgfqpoint{2.676733in}{0.680293in}}%
\pgfpathlineto{\pgfqpoint{2.678334in}{0.642684in}}%
\pgfpathlineto{\pgfqpoint{2.679936in}{0.675985in}}%
\pgfpathlineto{\pgfqpoint{2.680469in}{0.677116in}}%
\pgfpathlineto{\pgfqpoint{2.681003in}{0.676643in}}%
\pgfpathlineto{\pgfqpoint{2.681537in}{0.653646in}}%
\pgfpathlineto{\pgfqpoint{2.682070in}{0.661152in}}%
\pgfpathlineto{\pgfqpoint{2.682604in}{0.658227in}}%
\pgfpathlineto{\pgfqpoint{2.683671in}{0.701500in}}%
\pgfpathlineto{\pgfqpoint{2.684739in}{0.644204in}}%
\pgfpathlineto{\pgfqpoint{2.685273in}{0.684080in}}%
\pgfpathlineto{\pgfqpoint{2.685806in}{0.664497in}}%
\pgfpathlineto{\pgfqpoint{2.686340in}{0.679281in}}%
\pgfpathlineto{\pgfqpoint{2.688475in}{0.855118in}}%
\pgfpathlineto{\pgfqpoint{2.689009in}{1.099046in}}%
\pgfpathlineto{\pgfqpoint{2.689542in}{0.755313in}}%
\pgfpathlineto{\pgfqpoint{2.690076in}{1.630931in}}%
\pgfpathlineto{\pgfqpoint{2.690610in}{1.336096in}}%
\pgfpathlineto{\pgfqpoint{2.692211in}{0.647881in}}%
\pgfpathlineto{\pgfqpoint{2.693278in}{0.714360in}}%
\pgfpathlineto{\pgfqpoint{2.693812in}{0.679913in}}%
\pgfpathlineto{\pgfqpoint{2.694346in}{0.654614in}}%
\pgfpathlineto{\pgfqpoint{2.694879in}{0.697462in}}%
\pgfpathlineto{\pgfqpoint{2.695413in}{0.655637in}}%
\pgfpathlineto{\pgfqpoint{2.695947in}{0.640902in}}%
\pgfpathlineto{\pgfqpoint{2.697014in}{0.669612in}}%
\pgfpathlineto{\pgfqpoint{2.697548in}{0.648154in}}%
\pgfpathlineto{\pgfqpoint{2.698081in}{0.703110in}}%
\pgfpathlineto{\pgfqpoint{2.698615in}{0.661902in}}%
\pgfpathlineto{\pgfqpoint{2.699149in}{0.698596in}}%
\pgfpathlineto{\pgfqpoint{2.699683in}{0.669883in}}%
\pgfpathlineto{\pgfqpoint{2.700750in}{0.701824in}}%
\pgfpathlineto{\pgfqpoint{2.701817in}{0.646018in}}%
\pgfpathlineto{\pgfqpoint{2.702351in}{0.653319in}}%
\pgfpathlineto{\pgfqpoint{2.703418in}{0.651811in}}%
\pgfpathlineto{\pgfqpoint{2.703952in}{0.713005in}}%
\pgfpathlineto{\pgfqpoint{2.704486in}{0.654116in}}%
\pgfpathlineto{\pgfqpoint{2.705020in}{0.699417in}}%
\pgfpathlineto{\pgfqpoint{2.705553in}{0.666221in}}%
\pgfpathlineto{\pgfqpoint{2.706087in}{0.777882in}}%
\pgfpathlineto{\pgfqpoint{2.706621in}{0.650603in}}%
\pgfpathlineto{\pgfqpoint{2.707154in}{0.718671in}}%
\pgfpathlineto{\pgfqpoint{2.708755in}{0.665690in}}%
\pgfpathlineto{\pgfqpoint{2.709289in}{0.698174in}}%
\pgfpathlineto{\pgfqpoint{2.709823in}{0.682392in}}%
\pgfpathlineto{\pgfqpoint{2.710357in}{0.679399in}}%
\pgfpathlineto{\pgfqpoint{2.710890in}{0.664526in}}%
\pgfpathlineto{\pgfqpoint{2.711424in}{0.679309in}}%
\pgfpathlineto{\pgfqpoint{2.711958in}{0.689886in}}%
\pgfpathlineto{\pgfqpoint{2.712491in}{0.663190in}}%
\pgfpathlineto{\pgfqpoint{2.713025in}{0.691368in}}%
\pgfpathlineto{\pgfqpoint{2.713559in}{0.758197in}}%
\pgfpathlineto{\pgfqpoint{2.714092in}{0.732528in}}%
\pgfpathlineto{\pgfqpoint{2.715160in}{0.651571in}}%
\pgfpathlineto{\pgfqpoint{2.716227in}{0.701755in}}%
\pgfpathlineto{\pgfqpoint{2.716761in}{0.650169in}}%
\pgfpathlineto{\pgfqpoint{2.717828in}{0.746045in}}%
\pgfpathlineto{\pgfqpoint{2.718362in}{0.674578in}}%
\pgfpathlineto{\pgfqpoint{2.718896in}{0.690712in}}%
\pgfpathlineto{\pgfqpoint{2.719429in}{0.731440in}}%
\pgfpathlineto{\pgfqpoint{2.719963in}{0.701770in}}%
\pgfpathlineto{\pgfqpoint{2.720497in}{0.695427in}}%
\pgfpathlineto{\pgfqpoint{2.721031in}{0.655340in}}%
\pgfpathlineto{\pgfqpoint{2.721564in}{0.711649in}}%
\pgfpathlineto{\pgfqpoint{2.722632in}{0.708487in}}%
\pgfpathlineto{\pgfqpoint{2.723165in}{0.644658in}}%
\pgfpathlineto{\pgfqpoint{2.723699in}{0.693631in}}%
\pgfpathlineto{\pgfqpoint{2.724233in}{0.655611in}}%
\pgfpathlineto{\pgfqpoint{2.724766in}{0.734915in}}%
\pgfpathlineto{\pgfqpoint{2.725300in}{0.641152in}}%
\pgfpathlineto{\pgfqpoint{2.726368in}{0.642245in}}%
\pgfpathlineto{\pgfqpoint{2.727969in}{0.691915in}}%
\pgfpathlineto{\pgfqpoint{2.728502in}{0.658756in}}%
\pgfpathlineto{\pgfqpoint{2.729036in}{0.674558in}}%
\pgfpathlineto{\pgfqpoint{2.729570in}{0.694993in}}%
\pgfpathlineto{\pgfqpoint{2.730103in}{0.684854in}}%
\pgfpathlineto{\pgfqpoint{2.731171in}{0.704125in}}%
\pgfpathlineto{\pgfqpoint{2.731705in}{0.669352in}}%
\pgfpathlineto{\pgfqpoint{2.732238in}{0.690297in}}%
\pgfpathlineto{\pgfqpoint{2.732772in}{0.706308in}}%
\pgfpathlineto{\pgfqpoint{2.733306in}{0.695883in}}%
\pgfpathlineto{\pgfqpoint{2.733839in}{0.693532in}}%
\pgfpathlineto{\pgfqpoint{2.735440in}{0.642170in}}%
\pgfpathlineto{\pgfqpoint{2.735974in}{0.642296in}}%
\pgfpathlineto{\pgfqpoint{2.736508in}{0.652999in}}%
\pgfpathlineto{\pgfqpoint{2.738109in}{0.704826in}}%
\pgfpathlineto{\pgfqpoint{2.739710in}{0.656413in}}%
\pgfpathlineto{\pgfqpoint{2.740244in}{0.658491in}}%
\pgfpathlineto{\pgfqpoint{2.741311in}{0.702914in}}%
\pgfpathlineto{\pgfqpoint{2.741845in}{0.690782in}}%
\pgfpathlineto{\pgfqpoint{2.742912in}{0.652924in}}%
\pgfpathlineto{\pgfqpoint{2.743980in}{0.657064in}}%
\pgfpathlineto{\pgfqpoint{2.745047in}{0.707108in}}%
\pgfpathlineto{\pgfqpoint{2.746114in}{0.651208in}}%
\pgfpathlineto{\pgfqpoint{2.746648in}{0.657564in}}%
\pgfpathlineto{\pgfqpoint{2.747182in}{0.674273in}}%
\pgfpathlineto{\pgfqpoint{2.747716in}{0.643383in}}%
\pgfpathlineto{\pgfqpoint{2.748249in}{0.671712in}}%
\pgfpathlineto{\pgfqpoint{2.748783in}{0.656262in}}%
\pgfpathlineto{\pgfqpoint{2.750384in}{0.709672in}}%
\pgfpathlineto{\pgfqpoint{2.750918in}{0.646744in}}%
\pgfpathlineto{\pgfqpoint{2.751451in}{0.650057in}}%
\pgfpathlineto{\pgfqpoint{2.751985in}{0.686866in}}%
\pgfpathlineto{\pgfqpoint{2.752519in}{0.663039in}}%
\pgfpathlineto{\pgfqpoint{2.753053in}{0.664510in}}%
\pgfpathlineto{\pgfqpoint{2.754120in}{0.694975in}}%
\pgfpathlineto{\pgfqpoint{2.754654in}{0.656542in}}%
\pgfpathlineto{\pgfqpoint{2.755187in}{0.714272in}}%
\pgfpathlineto{\pgfqpoint{2.756255in}{0.709904in}}%
\pgfpathlineto{\pgfqpoint{2.756789in}{0.641763in}}%
\pgfpathlineto{\pgfqpoint{2.757322in}{0.656267in}}%
\pgfpathlineto{\pgfqpoint{2.757856in}{0.700580in}}%
\pgfpathlineto{\pgfqpoint{2.758390in}{0.688877in}}%
\pgfpathlineto{\pgfqpoint{2.758923in}{0.668637in}}%
\pgfpathlineto{\pgfqpoint{2.759991in}{0.721958in}}%
\pgfpathlineto{\pgfqpoint{2.760524in}{0.688106in}}%
\pgfpathlineto{\pgfqpoint{2.761058in}{0.758860in}}%
\pgfpathlineto{\pgfqpoint{2.761592in}{0.709576in}}%
\pgfpathlineto{\pgfqpoint{2.762126in}{0.684835in}}%
\pgfpathlineto{\pgfqpoint{2.762659in}{0.744700in}}%
\pgfpathlineto{\pgfqpoint{2.763193in}{0.700178in}}%
\pgfpathlineto{\pgfqpoint{2.763727in}{0.695773in}}%
\pgfpathlineto{\pgfqpoint{2.764260in}{0.674333in}}%
\pgfpathlineto{\pgfqpoint{2.765328in}{0.730528in}}%
\pgfpathlineto{\pgfqpoint{2.765861in}{0.656065in}}%
\pgfpathlineto{\pgfqpoint{2.766395in}{0.723789in}}%
\pgfpathlineto{\pgfqpoint{2.767463in}{0.649641in}}%
\pgfpathlineto{\pgfqpoint{2.769064in}{0.732893in}}%
\pgfpathlineto{\pgfqpoint{2.770131in}{0.682575in}}%
\pgfpathlineto{\pgfqpoint{2.770665in}{0.713445in}}%
\pgfpathlineto{\pgfqpoint{2.771198in}{0.677412in}}%
\pgfpathlineto{\pgfqpoint{2.771732in}{0.696266in}}%
\pgfpathlineto{\pgfqpoint{2.772800in}{0.759317in}}%
\pgfpathlineto{\pgfqpoint{2.773333in}{0.665439in}}%
\pgfpathlineto{\pgfqpoint{2.773867in}{0.697937in}}%
\pgfpathlineto{\pgfqpoint{2.774401in}{0.722085in}}%
\pgfpathlineto{\pgfqpoint{2.774934in}{0.703124in}}%
\pgfpathlineto{\pgfqpoint{2.775468in}{0.711329in}}%
\pgfpathlineto{\pgfqpoint{2.776002in}{0.686026in}}%
\pgfpathlineto{\pgfqpoint{2.776535in}{0.687259in}}%
\pgfpathlineto{\pgfqpoint{2.777603in}{0.724030in}}%
\pgfpathlineto{\pgfqpoint{2.779204in}{0.668261in}}%
\pgfpathlineto{\pgfqpoint{2.779738in}{0.693185in}}%
\pgfpathlineto{\pgfqpoint{2.780271in}{0.685914in}}%
\pgfpathlineto{\pgfqpoint{2.781339in}{0.655693in}}%
\pgfpathlineto{\pgfqpoint{2.781872in}{0.676129in}}%
\pgfpathlineto{\pgfqpoint{2.782406in}{0.668034in}}%
\pgfpathlineto{\pgfqpoint{2.782940in}{0.671171in}}%
\pgfpathlineto{\pgfqpoint{2.783474in}{0.656471in}}%
\pgfpathlineto{\pgfqpoint{2.784007in}{0.681983in}}%
\pgfpathlineto{\pgfqpoint{2.784541in}{0.660001in}}%
\pgfpathlineto{\pgfqpoint{2.786676in}{0.704597in}}%
\pgfpathlineto{\pgfqpoint{2.787209in}{0.699351in}}%
\pgfpathlineto{\pgfqpoint{2.788277in}{0.670291in}}%
\pgfpathlineto{\pgfqpoint{2.789878in}{0.717711in}}%
\pgfpathlineto{\pgfqpoint{2.790945in}{0.686406in}}%
\pgfpathlineto{\pgfqpoint{2.791479in}{0.694191in}}%
\pgfpathlineto{\pgfqpoint{2.792013in}{0.687284in}}%
\pgfpathlineto{\pgfqpoint{2.793614in}{0.653161in}}%
\pgfpathlineto{\pgfqpoint{2.794148in}{0.654122in}}%
\pgfpathlineto{\pgfqpoint{2.796282in}{0.713937in}}%
\pgfpathlineto{\pgfqpoint{2.797350in}{0.636654in}}%
\pgfpathlineto{\pgfqpoint{2.799485in}{0.720526in}}%
\pgfpathlineto{\pgfqpoint{2.801086in}{0.661070in}}%
\pgfpathlineto{\pgfqpoint{2.801619in}{0.690910in}}%
\pgfpathlineto{\pgfqpoint{2.802153in}{0.688245in}}%
\pgfpathlineto{\pgfqpoint{2.803754in}{0.649193in}}%
\pgfpathlineto{\pgfqpoint{2.805355in}{0.697733in}}%
\pgfpathlineto{\pgfqpoint{2.806423in}{0.644509in}}%
\pgfpathlineto{\pgfqpoint{2.807490in}{0.692932in}}%
\pgfpathlineto{\pgfqpoint{2.808024in}{0.647415in}}%
\pgfpathlineto{\pgfqpoint{2.808557in}{0.661782in}}%
\pgfpathlineto{\pgfqpoint{2.809091in}{0.701733in}}%
\pgfpathlineto{\pgfqpoint{2.809625in}{0.686446in}}%
\pgfpathlineto{\pgfqpoint{2.810692in}{0.697728in}}%
\pgfpathlineto{\pgfqpoint{2.811226in}{0.693992in}}%
\pgfpathlineto{\pgfqpoint{2.812827in}{0.662034in}}%
\pgfpathlineto{\pgfqpoint{2.813361in}{0.730014in}}%
\pgfpathlineto{\pgfqpoint{2.813894in}{0.642719in}}%
\pgfpathlineto{\pgfqpoint{2.814428in}{0.731395in}}%
\pgfpathlineto{\pgfqpoint{2.816029in}{0.678966in}}%
\pgfpathlineto{\pgfqpoint{2.816563in}{0.790400in}}%
\pgfpathlineto{\pgfqpoint{2.817097in}{0.649254in}}%
\pgfpathlineto{\pgfqpoint{2.817630in}{0.750954in}}%
\pgfpathlineto{\pgfqpoint{2.819231in}{0.671531in}}%
\pgfpathlineto{\pgfqpoint{2.820299in}{0.747304in}}%
\pgfpathlineto{\pgfqpoint{2.820833in}{0.657454in}}%
\pgfpathlineto{\pgfqpoint{2.821366in}{0.704631in}}%
\pgfpathlineto{\pgfqpoint{2.822967in}{0.657367in}}%
\pgfpathlineto{\pgfqpoint{2.824035in}{0.766518in}}%
\pgfpathlineto{\pgfqpoint{2.824569in}{0.757074in}}%
\pgfpathlineto{\pgfqpoint{2.826170in}{0.703234in}}%
\pgfpathlineto{\pgfqpoint{2.827237in}{0.719366in}}%
\pgfpathlineto{\pgfqpoint{2.827771in}{0.778770in}}%
\pgfpathlineto{\pgfqpoint{2.828304in}{0.717951in}}%
\pgfpathlineto{\pgfqpoint{2.829372in}{0.699791in}}%
\pgfpathlineto{\pgfqpoint{2.829906in}{0.746517in}}%
\pgfpathlineto{\pgfqpoint{2.830439in}{0.700362in}}%
\pgfpathlineto{\pgfqpoint{2.830973in}{0.709791in}}%
\pgfpathlineto{\pgfqpoint{2.831507in}{0.673259in}}%
\pgfpathlineto{\pgfqpoint{2.832040in}{0.727975in}}%
\pgfpathlineto{\pgfqpoint{2.832574in}{0.698315in}}%
\pgfpathlineto{\pgfqpoint{2.833108in}{0.700279in}}%
\pgfpathlineto{\pgfqpoint{2.834709in}{0.644526in}}%
\pgfpathlineto{\pgfqpoint{2.835243in}{0.718610in}}%
\pgfpathlineto{\pgfqpoint{2.835776in}{0.682428in}}%
\pgfpathlineto{\pgfqpoint{2.837377in}{0.651639in}}%
\pgfpathlineto{\pgfqpoint{2.838978in}{0.689799in}}%
\pgfpathlineto{\pgfqpoint{2.839512in}{0.651048in}}%
\pgfpathlineto{\pgfqpoint{2.840046in}{0.676979in}}%
\pgfpathlineto{\pgfqpoint{2.840580in}{0.660150in}}%
\pgfpathlineto{\pgfqpoint{2.841113in}{0.681156in}}%
\pgfpathlineto{\pgfqpoint{2.841647in}{0.645563in}}%
\pgfpathlineto{\pgfqpoint{2.842181in}{0.653084in}}%
\pgfpathlineto{\pgfqpoint{2.843248in}{0.690325in}}%
\pgfpathlineto{\pgfqpoint{2.844315in}{0.661713in}}%
\pgfpathlineto{\pgfqpoint{2.845917in}{0.716219in}}%
\pgfpathlineto{\pgfqpoint{2.846450in}{0.690683in}}%
\pgfpathlineto{\pgfqpoint{2.846984in}{0.710120in}}%
\pgfpathlineto{\pgfqpoint{2.847518in}{0.719133in}}%
\pgfpathlineto{\pgfqpoint{2.848051in}{0.718257in}}%
\pgfpathlineto{\pgfqpoint{2.849652in}{0.670971in}}%
\pgfpathlineto{\pgfqpoint{2.850186in}{0.672341in}}%
\pgfpathlineto{\pgfqpoint{2.851254in}{0.656183in}}%
\pgfpathlineto{\pgfqpoint{2.853388in}{0.695053in}}%
\pgfpathlineto{\pgfqpoint{2.853922in}{0.696556in}}%
\pgfpathlineto{\pgfqpoint{2.854989in}{0.662546in}}%
\pgfpathlineto{\pgfqpoint{2.855523in}{0.671722in}}%
\pgfpathlineto{\pgfqpoint{2.856591in}{0.711108in}}%
\pgfpathlineto{\pgfqpoint{2.858192in}{0.641335in}}%
\pgfpathlineto{\pgfqpoint{2.858725in}{0.647045in}}%
\pgfpathlineto{\pgfqpoint{2.859259in}{0.708494in}}%
\pgfpathlineto{\pgfqpoint{2.859793in}{0.685696in}}%
\pgfpathlineto{\pgfqpoint{2.860326in}{0.697632in}}%
\pgfpathlineto{\pgfqpoint{2.860860in}{0.688483in}}%
\pgfpathlineto{\pgfqpoint{2.861394in}{0.654870in}}%
\pgfpathlineto{\pgfqpoint{2.861928in}{0.677459in}}%
\pgfpathlineto{\pgfqpoint{2.862461in}{0.709879in}}%
\pgfpathlineto{\pgfqpoint{2.863529in}{0.653706in}}%
\pgfpathlineto{\pgfqpoint{2.864596in}{0.697307in}}%
\pgfpathlineto{\pgfqpoint{2.865130in}{0.647589in}}%
\pgfpathlineto{\pgfqpoint{2.865663in}{0.691546in}}%
\pgfpathlineto{\pgfqpoint{2.866197in}{0.691357in}}%
\pgfpathlineto{\pgfqpoint{2.866731in}{0.701749in}}%
\pgfpathlineto{\pgfqpoint{2.867265in}{0.651206in}}%
\pgfpathlineto{\pgfqpoint{2.867798in}{0.672620in}}%
\pgfpathlineto{\pgfqpoint{2.868332in}{0.736792in}}%
\pgfpathlineto{\pgfqpoint{2.868866in}{0.675583in}}%
\pgfpathlineto{\pgfqpoint{2.870467in}{0.730188in}}%
\pgfpathlineto{\pgfqpoint{2.871000in}{0.656410in}}%
\pgfpathlineto{\pgfqpoint{2.871534in}{0.818897in}}%
\pgfpathlineto{\pgfqpoint{2.872068in}{0.671925in}}%
\pgfpathlineto{\pgfqpoint{2.873669in}{0.748225in}}%
\pgfpathlineto{\pgfqpoint{2.874203in}{0.741361in}}%
\pgfpathlineto{\pgfqpoint{2.874736in}{0.746337in}}%
\pgfpathlineto{\pgfqpoint{2.876337in}{0.671730in}}%
\pgfpathlineto{\pgfqpoint{2.876871in}{0.775876in}}%
\pgfpathlineto{\pgfqpoint{2.877405in}{0.723552in}}%
\pgfpathlineto{\pgfqpoint{2.877939in}{0.654627in}}%
\pgfpathlineto{\pgfqpoint{2.878472in}{0.727286in}}%
\pgfpathlineto{\pgfqpoint{2.879006in}{0.707811in}}%
\pgfpathlineto{\pgfqpoint{2.879540in}{0.735727in}}%
\pgfpathlineto{\pgfqpoint{2.880073in}{0.660211in}}%
\pgfpathlineto{\pgfqpoint{2.880607in}{0.730515in}}%
\pgfpathlineto{\pgfqpoint{2.881141in}{0.754327in}}%
\pgfpathlineto{\pgfqpoint{2.882208in}{0.677201in}}%
\pgfpathlineto{\pgfqpoint{2.882742in}{0.818159in}}%
\pgfpathlineto{\pgfqpoint{2.883276in}{0.760540in}}%
\pgfpathlineto{\pgfqpoint{2.883809in}{0.692115in}}%
\pgfpathlineto{\pgfqpoint{2.884343in}{0.694542in}}%
\pgfpathlineto{\pgfqpoint{2.884877in}{0.770930in}}%
\pgfpathlineto{\pgfqpoint{2.885410in}{0.701086in}}%
\pgfpathlineto{\pgfqpoint{2.885944in}{0.727026in}}%
\pgfpathlineto{\pgfqpoint{2.886478in}{0.678029in}}%
\pgfpathlineto{\pgfqpoint{2.887011in}{0.701479in}}%
\pgfpathlineto{\pgfqpoint{2.888079in}{0.717949in}}%
\pgfpathlineto{\pgfqpoint{2.888613in}{0.652617in}}%
\pgfpathlineto{\pgfqpoint{2.889146in}{0.683807in}}%
\pgfpathlineto{\pgfqpoint{2.889680in}{0.663092in}}%
\pgfpathlineto{\pgfqpoint{2.890214in}{0.722107in}}%
\pgfpathlineto{\pgfqpoint{2.890747in}{0.653783in}}%
\pgfpathlineto{\pgfqpoint{2.891281in}{0.660076in}}%
\pgfpathlineto{\pgfqpoint{2.891815in}{0.669697in}}%
\pgfpathlineto{\pgfqpoint{2.892349in}{0.666208in}}%
\pgfpathlineto{\pgfqpoint{2.892882in}{0.654416in}}%
\pgfpathlineto{\pgfqpoint{2.893416in}{0.673157in}}%
\pgfpathlineto{\pgfqpoint{2.893950in}{0.639476in}}%
\pgfpathlineto{\pgfqpoint{2.894483in}{0.650720in}}%
\pgfpathlineto{\pgfqpoint{2.895017in}{0.669570in}}%
\pgfpathlineto{\pgfqpoint{2.896618in}{0.642184in}}%
\pgfpathlineto{\pgfqpoint{2.897152in}{0.667607in}}%
\pgfpathlineto{\pgfqpoint{2.897686in}{0.645725in}}%
\pgfpathlineto{\pgfqpoint{2.898219in}{0.654499in}}%
\pgfpathlineto{\pgfqpoint{2.898753in}{0.638854in}}%
\pgfpathlineto{\pgfqpoint{2.899287in}{0.672414in}}%
\pgfpathlineto{\pgfqpoint{2.899820in}{0.650651in}}%
\pgfpathlineto{\pgfqpoint{2.900354in}{0.648965in}}%
\pgfpathlineto{\pgfqpoint{2.902489in}{0.697190in}}%
\pgfpathlineto{\pgfqpoint{2.903023in}{0.691623in}}%
\pgfpathlineto{\pgfqpoint{2.903556in}{0.699874in}}%
\pgfpathlineto{\pgfqpoint{2.904090in}{0.731773in}}%
\pgfpathlineto{\pgfqpoint{2.904624in}{0.711192in}}%
\pgfpathlineto{\pgfqpoint{2.905157in}{0.714047in}}%
\pgfpathlineto{\pgfqpoint{2.905691in}{0.690206in}}%
\pgfpathlineto{\pgfqpoint{2.906225in}{0.717397in}}%
\pgfpathlineto{\pgfqpoint{2.906758in}{0.716733in}}%
\pgfpathlineto{\pgfqpoint{2.907826in}{0.695459in}}%
\pgfpathlineto{\pgfqpoint{2.908893in}{0.653247in}}%
\pgfpathlineto{\pgfqpoint{2.909427in}{0.661769in}}%
\pgfpathlineto{\pgfqpoint{2.909961in}{0.659523in}}%
\pgfpathlineto{\pgfqpoint{2.911028in}{0.733907in}}%
\pgfpathlineto{\pgfqpoint{2.912629in}{0.659531in}}%
\pgfpathlineto{\pgfqpoint{2.914764in}{0.734533in}}%
\pgfpathlineto{\pgfqpoint{2.915298in}{0.657754in}}%
\pgfpathlineto{\pgfqpoint{2.915831in}{0.659220in}}%
\pgfpathlineto{\pgfqpoint{2.917432in}{0.710188in}}%
\pgfpathlineto{\pgfqpoint{2.918500in}{0.650541in}}%
\pgfpathlineto{\pgfqpoint{2.919034in}{0.663372in}}%
\pgfpathlineto{\pgfqpoint{2.919567in}{0.736892in}}%
\pgfpathlineto{\pgfqpoint{2.920101in}{0.689334in}}%
\pgfpathlineto{\pgfqpoint{2.920635in}{0.680234in}}%
\pgfpathlineto{\pgfqpoint{2.921168in}{0.683340in}}%
\pgfpathlineto{\pgfqpoint{2.921702in}{0.700325in}}%
\pgfpathlineto{\pgfqpoint{2.922236in}{0.648064in}}%
\pgfpathlineto{\pgfqpoint{2.922769in}{0.692045in}}%
\pgfpathlineto{\pgfqpoint{2.923303in}{0.687758in}}%
\pgfpathlineto{\pgfqpoint{2.923837in}{0.733707in}}%
\pgfpathlineto{\pgfqpoint{2.924371in}{0.651743in}}%
\pgfpathlineto{\pgfqpoint{2.924904in}{0.698362in}}%
\pgfpathlineto{\pgfqpoint{2.925438in}{0.750221in}}%
\pgfpathlineto{\pgfqpoint{2.925972in}{0.659322in}}%
\pgfpathlineto{\pgfqpoint{2.926505in}{0.764972in}}%
\pgfpathlineto{\pgfqpoint{2.927039in}{0.742959in}}%
\pgfpathlineto{\pgfqpoint{2.927573in}{0.658841in}}%
\pgfpathlineto{\pgfqpoint{2.928106in}{0.728500in}}%
\pgfpathlineto{\pgfqpoint{2.929174in}{0.748472in}}%
\pgfpathlineto{\pgfqpoint{2.929708in}{0.675284in}}%
\pgfpathlineto{\pgfqpoint{2.930241in}{0.772541in}}%
\pgfpathlineto{\pgfqpoint{2.930775in}{0.742408in}}%
\pgfpathlineto{\pgfqpoint{2.931309in}{0.706232in}}%
\pgfpathlineto{\pgfqpoint{2.931842in}{0.759785in}}%
\pgfpathlineto{\pgfqpoint{2.932376in}{0.706993in}}%
\pgfpathlineto{\pgfqpoint{2.932910in}{0.710323in}}%
\pgfpathlineto{\pgfqpoint{2.933977in}{0.679351in}}%
\pgfpathlineto{\pgfqpoint{2.935578in}{0.775111in}}%
\pgfpathlineto{\pgfqpoint{2.936112in}{0.752218in}}%
\pgfpathlineto{\pgfqpoint{2.936646in}{0.697630in}}%
\pgfpathlineto{\pgfqpoint{2.937179in}{0.727561in}}%
\pgfpathlineto{\pgfqpoint{2.937713in}{0.775672in}}%
\pgfpathlineto{\pgfqpoint{2.938247in}{0.757617in}}%
\pgfpathlineto{\pgfqpoint{2.939314in}{0.714111in}}%
\pgfpathlineto{\pgfqpoint{2.939848in}{0.717830in}}%
\pgfpathlineto{\pgfqpoint{2.940382in}{0.745086in}}%
\pgfpathlineto{\pgfqpoint{2.940915in}{0.730933in}}%
\pgfpathlineto{\pgfqpoint{2.941449in}{0.688292in}}%
\pgfpathlineto{\pgfqpoint{2.942516in}{0.691522in}}%
\pgfpathlineto{\pgfqpoint{2.943050in}{0.706898in}}%
\pgfpathlineto{\pgfqpoint{2.944651in}{0.637775in}}%
\pgfpathlineto{\pgfqpoint{2.945185in}{0.668951in}}%
\pgfpathlineto{\pgfqpoint{2.945719in}{0.666252in}}%
\pgfpathlineto{\pgfqpoint{2.946252in}{0.665146in}}%
\pgfpathlineto{\pgfqpoint{2.946786in}{0.647427in}}%
\pgfpathlineto{\pgfqpoint{2.947320in}{0.698613in}}%
\pgfpathlineto{\pgfqpoint{2.947853in}{0.661422in}}%
\pgfpathlineto{\pgfqpoint{2.948387in}{0.669421in}}%
\pgfpathlineto{\pgfqpoint{2.948921in}{0.654352in}}%
\pgfpathlineto{\pgfqpoint{2.949454in}{0.664711in}}%
\pgfpathlineto{\pgfqpoint{2.949988in}{0.671325in}}%
\pgfpathlineto{\pgfqpoint{2.950522in}{0.669137in}}%
\pgfpathlineto{\pgfqpoint{2.951056in}{0.658217in}}%
\pgfpathlineto{\pgfqpoint{2.951589in}{0.661123in}}%
\pgfpathlineto{\pgfqpoint{2.952123in}{0.670447in}}%
\pgfpathlineto{\pgfqpoint{2.952657in}{0.654941in}}%
\pgfpathlineto{\pgfqpoint{2.953190in}{0.673458in}}%
\pgfpathlineto{\pgfqpoint{2.954791in}{0.649270in}}%
\pgfpathlineto{\pgfqpoint{2.955325in}{0.639323in}}%
\pgfpathlineto{\pgfqpoint{2.956393in}{0.672646in}}%
\pgfpathlineto{\pgfqpoint{2.956926in}{0.666645in}}%
\pgfpathlineto{\pgfqpoint{2.957460in}{0.665337in}}%
\pgfpathlineto{\pgfqpoint{2.957994in}{0.650934in}}%
\pgfpathlineto{\pgfqpoint{2.958527in}{0.662278in}}%
\pgfpathlineto{\pgfqpoint{2.959061in}{0.660952in}}%
\pgfpathlineto{\pgfqpoint{2.959595in}{0.666874in}}%
\pgfpathlineto{\pgfqpoint{2.961196in}{0.705757in}}%
\pgfpathlineto{\pgfqpoint{2.961730in}{0.701375in}}%
\pgfpathlineto{\pgfqpoint{2.962263in}{0.704101in}}%
\pgfpathlineto{\pgfqpoint{2.963331in}{0.733915in}}%
\pgfpathlineto{\pgfqpoint{2.963864in}{0.727524in}}%
\pgfpathlineto{\pgfqpoint{2.964398in}{0.699541in}}%
\pgfpathlineto{\pgfqpoint{2.964932in}{0.701309in}}%
\pgfpathlineto{\pgfqpoint{2.965466in}{0.712502in}}%
\pgfpathlineto{\pgfqpoint{2.967067in}{0.644303in}}%
\pgfpathlineto{\pgfqpoint{2.967600in}{0.715620in}}%
\pgfpathlineto{\pgfqpoint{2.968668in}{0.715248in}}%
\pgfpathlineto{\pgfqpoint{2.970269in}{0.649356in}}%
\pgfpathlineto{\pgfqpoint{2.971870in}{0.714000in}}%
\pgfpathlineto{\pgfqpoint{2.972404in}{0.710868in}}%
\pgfpathlineto{\pgfqpoint{2.973471in}{0.648374in}}%
\pgfpathlineto{\pgfqpoint{2.974538in}{0.735066in}}%
\pgfpathlineto{\pgfqpoint{2.975072in}{0.699382in}}%
\pgfpathlineto{\pgfqpoint{2.976140in}{0.660248in}}%
\pgfpathlineto{\pgfqpoint{2.976673in}{0.696571in}}%
\pgfpathlineto{\pgfqpoint{2.977207in}{0.678467in}}%
\pgfpathlineto{\pgfqpoint{2.977741in}{0.688737in}}%
\pgfpathlineto{\pgfqpoint{2.978274in}{0.656847in}}%
\pgfpathlineto{\pgfqpoint{2.978808in}{0.738912in}}%
\pgfpathlineto{\pgfqpoint{2.979342in}{0.648705in}}%
\pgfpathlineto{\pgfqpoint{2.979875in}{0.716260in}}%
\pgfpathlineto{\pgfqpoint{2.981477in}{0.692700in}}%
\pgfpathlineto{\pgfqpoint{2.982010in}{0.785794in}}%
\pgfpathlineto{\pgfqpoint{2.982544in}{0.685000in}}%
\pgfpathlineto{\pgfqpoint{2.983078in}{0.726825in}}%
\pgfpathlineto{\pgfqpoint{2.983611in}{0.737679in}}%
\pgfpathlineto{\pgfqpoint{2.984145in}{0.780150in}}%
\pgfpathlineto{\pgfqpoint{2.984679in}{0.749330in}}%
\pgfpathlineto{\pgfqpoint{2.985212in}{0.774933in}}%
\pgfpathlineto{\pgfqpoint{2.986814in}{0.673185in}}%
\pgfpathlineto{\pgfqpoint{2.987347in}{0.788995in}}%
\pgfpathlineto{\pgfqpoint{2.987881in}{0.763281in}}%
\pgfpathlineto{\pgfqpoint{2.988415in}{0.650827in}}%
\pgfpathlineto{\pgfqpoint{2.988948in}{0.741130in}}%
\pgfpathlineto{\pgfqpoint{2.989482in}{0.680216in}}%
\pgfpathlineto{\pgfqpoint{2.990016in}{0.722928in}}%
\pgfpathlineto{\pgfqpoint{2.990549in}{0.780990in}}%
\pgfpathlineto{\pgfqpoint{2.991083in}{0.768369in}}%
\pgfpathlineto{\pgfqpoint{2.991617in}{0.724124in}}%
\pgfpathlineto{\pgfqpoint{2.992151in}{0.737395in}}%
\pgfpathlineto{\pgfqpoint{2.992684in}{0.741493in}}%
\pgfpathlineto{\pgfqpoint{2.993218in}{0.827672in}}%
\pgfpathlineto{\pgfqpoint{2.994819in}{0.655865in}}%
\pgfpathlineto{\pgfqpoint{2.995353in}{0.753041in}}%
\pgfpathlineto{\pgfqpoint{2.995886in}{0.679382in}}%
\pgfpathlineto{\pgfqpoint{2.996420in}{0.727398in}}%
\pgfpathlineto{\pgfqpoint{2.998021in}{0.651561in}}%
\pgfpathlineto{\pgfqpoint{2.998555in}{0.705383in}}%
\pgfpathlineto{\pgfqpoint{2.999089in}{0.678354in}}%
\pgfpathlineto{\pgfqpoint{2.999622in}{0.654230in}}%
\pgfpathlineto{\pgfqpoint{3.000156in}{0.663763in}}%
\pgfpathlineto{\pgfqpoint{3.001223in}{0.692694in}}%
\pgfpathlineto{\pgfqpoint{3.002825in}{0.647274in}}%
\pgfpathlineto{\pgfqpoint{3.005493in}{0.664496in}}%
\pgfpathlineto{\pgfqpoint{3.007094in}{0.645490in}}%
\pgfpathlineto{\pgfqpoint{3.007628in}{0.651727in}}%
\pgfpathlineto{\pgfqpoint{3.008162in}{0.664576in}}%
\pgfpathlineto{\pgfqpoint{3.008695in}{0.651414in}}%
\pgfpathlineto{\pgfqpoint{3.009229in}{0.646077in}}%
\pgfpathlineto{\pgfqpoint{3.009763in}{0.663329in}}%
\pgfpathlineto{\pgfqpoint{3.010296in}{0.660622in}}%
\pgfpathlineto{\pgfqpoint{3.010830in}{0.643383in}}%
\pgfpathlineto{\pgfqpoint{3.011364in}{0.653320in}}%
\pgfpathlineto{\pgfqpoint{3.014032in}{0.637706in}}%
\pgfpathlineto{\pgfqpoint{3.015633in}{0.663932in}}%
\pgfpathlineto{\pgfqpoint{3.016167in}{0.651139in}}%
\pgfpathlineto{\pgfqpoint{3.018836in}{0.704819in}}%
\pgfpathlineto{\pgfqpoint{3.019369in}{0.686793in}}%
\pgfpathlineto{\pgfqpoint{3.020437in}{0.727802in}}%
\pgfpathlineto{\pgfqpoint{3.020970in}{0.713278in}}%
\pgfpathlineto{\pgfqpoint{3.021504in}{0.708678in}}%
\pgfpathlineto{\pgfqpoint{3.022038in}{0.724334in}}%
\pgfpathlineto{\pgfqpoint{3.022571in}{0.722557in}}%
\pgfpathlineto{\pgfqpoint{3.024173in}{0.639448in}}%
\pgfpathlineto{\pgfqpoint{3.026307in}{0.741802in}}%
\pgfpathlineto{\pgfqpoint{3.027909in}{0.644465in}}%
\pgfpathlineto{\pgfqpoint{3.028976in}{0.736179in}}%
\pgfpathlineto{\pgfqpoint{3.029510in}{0.716807in}}%
\pgfpathlineto{\pgfqpoint{3.030043in}{0.711769in}}%
\pgfpathlineto{\pgfqpoint{3.030577in}{0.651023in}}%
\pgfpathlineto{\pgfqpoint{3.031111in}{0.663625in}}%
\pgfpathlineto{\pgfqpoint{3.032178in}{0.733515in}}%
\pgfpathlineto{\pgfqpoint{3.032712in}{0.650480in}}%
\pgfpathlineto{\pgfqpoint{3.033246in}{0.653792in}}%
\pgfpathlineto{\pgfqpoint{3.033779in}{0.732189in}}%
\pgfpathlineto{\pgfqpoint{3.034313in}{0.691741in}}%
\pgfpathlineto{\pgfqpoint{3.034847in}{0.692342in}}%
\pgfpathlineto{\pgfqpoint{3.035380in}{0.675680in}}%
\pgfpathlineto{\pgfqpoint{3.035914in}{0.743846in}}%
\pgfpathlineto{\pgfqpoint{3.036448in}{0.653997in}}%
\pgfpathlineto{\pgfqpoint{3.036981in}{0.788812in}}%
\pgfpathlineto{\pgfqpoint{3.037515in}{0.685146in}}%
\pgfpathlineto{\pgfqpoint{3.038049in}{0.680779in}}%
\pgfpathlineto{\pgfqpoint{3.039116in}{0.835427in}}%
\pgfpathlineto{\pgfqpoint{3.039650in}{0.799046in}}%
\pgfpathlineto{\pgfqpoint{3.040184in}{0.688205in}}%
\pgfpathlineto{\pgfqpoint{3.040717in}{0.783491in}}%
\pgfpathlineto{\pgfqpoint{3.041251in}{0.749688in}}%
\pgfpathlineto{\pgfqpoint{3.041785in}{0.815056in}}%
\pgfpathlineto{\pgfqpoint{3.042318in}{0.797104in}}%
\pgfpathlineto{\pgfqpoint{3.042852in}{0.772445in}}%
\pgfpathlineto{\pgfqpoint{3.043386in}{0.677362in}}%
\pgfpathlineto{\pgfqpoint{3.044453in}{0.679347in}}%
\pgfpathlineto{\pgfqpoint{3.044987in}{0.670812in}}%
\pgfpathlineto{\pgfqpoint{3.045521in}{0.697470in}}%
\pgfpathlineto{\pgfqpoint{3.046054in}{0.902493in}}%
\pgfpathlineto{\pgfqpoint{3.046588in}{0.799498in}}%
\pgfpathlineto{\pgfqpoint{3.047122in}{0.700811in}}%
\pgfpathlineto{\pgfqpoint{3.047655in}{0.740916in}}%
\pgfpathlineto{\pgfqpoint{3.048189in}{0.837458in}}%
\pgfpathlineto{\pgfqpoint{3.049790in}{0.693420in}}%
\pgfpathlineto{\pgfqpoint{3.051391in}{0.756391in}}%
\pgfpathlineto{\pgfqpoint{3.052992in}{0.648457in}}%
\pgfpathlineto{\pgfqpoint{3.053526in}{0.692786in}}%
\pgfpathlineto{\pgfqpoint{3.054060in}{0.680004in}}%
\pgfpathlineto{\pgfqpoint{3.054594in}{0.662811in}}%
\pgfpathlineto{\pgfqpoint{3.055127in}{0.676755in}}%
\pgfpathlineto{\pgfqpoint{3.056728in}{0.641426in}}%
\pgfpathlineto{\pgfqpoint{3.057262in}{0.674860in}}%
\pgfpathlineto{\pgfqpoint{3.058329in}{0.673721in}}%
\pgfpathlineto{\pgfqpoint{3.058863in}{0.678243in}}%
\pgfpathlineto{\pgfqpoint{3.059397in}{0.671830in}}%
\pgfpathlineto{\pgfqpoint{3.059931in}{0.640133in}}%
\pgfpathlineto{\pgfqpoint{3.060464in}{0.678959in}}%
\pgfpathlineto{\pgfqpoint{3.060998in}{0.654573in}}%
\pgfpathlineto{\pgfqpoint{3.061532in}{0.671605in}}%
\pgfpathlineto{\pgfqpoint{3.062065in}{0.649905in}}%
\pgfpathlineto{\pgfqpoint{3.062599in}{0.650248in}}%
\pgfpathlineto{\pgfqpoint{3.063133in}{0.653408in}}%
\pgfpathlineto{\pgfqpoint{3.063666in}{0.669865in}}%
\pgfpathlineto{\pgfqpoint{3.064200in}{0.655695in}}%
\pgfpathlineto{\pgfqpoint{3.064734in}{0.641399in}}%
\pgfpathlineto{\pgfqpoint{3.065268in}{0.641430in}}%
\pgfpathlineto{\pgfqpoint{3.065801in}{0.643413in}}%
\pgfpathlineto{\pgfqpoint{3.066335in}{0.652277in}}%
\pgfpathlineto{\pgfqpoint{3.066869in}{0.650969in}}%
\pgfpathlineto{\pgfqpoint{3.067402in}{0.644052in}}%
\pgfpathlineto{\pgfqpoint{3.067936in}{0.671753in}}%
\pgfpathlineto{\pgfqpoint{3.068470in}{0.650522in}}%
\pgfpathlineto{\pgfqpoint{3.069003in}{0.659161in}}%
\pgfpathlineto{\pgfqpoint{3.069537in}{0.655730in}}%
\pgfpathlineto{\pgfqpoint{3.071138in}{0.645995in}}%
\pgfpathlineto{\pgfqpoint{3.071672in}{0.662747in}}%
\pgfpathlineto{\pgfqpoint{3.072206in}{0.649259in}}%
\pgfpathlineto{\pgfqpoint{3.072739in}{0.660483in}}%
\pgfpathlineto{\pgfqpoint{3.074340in}{0.645660in}}%
\pgfpathlineto{\pgfqpoint{3.076475in}{0.739242in}}%
\pgfpathlineto{\pgfqpoint{3.077009in}{0.698225in}}%
\pgfpathlineto{\pgfqpoint{3.077543in}{0.716612in}}%
\pgfpathlineto{\pgfqpoint{3.079144in}{0.757903in}}%
\pgfpathlineto{\pgfqpoint{3.079677in}{0.758973in}}%
\pgfpathlineto{\pgfqpoint{3.081812in}{0.652310in}}%
\pgfpathlineto{\pgfqpoint{3.082346in}{0.665834in}}%
\pgfpathlineto{\pgfqpoint{3.083947in}{0.781667in}}%
\pgfpathlineto{\pgfqpoint{3.085548in}{0.675846in}}%
\pgfpathlineto{\pgfqpoint{3.086082in}{0.730175in}}%
\pgfpathlineto{\pgfqpoint{3.086616in}{0.716504in}}%
\pgfpathlineto{\pgfqpoint{3.087149in}{0.714157in}}%
\pgfpathlineto{\pgfqpoint{3.088217in}{0.657430in}}%
\pgfpathlineto{\pgfqpoint{3.089284in}{0.778116in}}%
\pgfpathlineto{\pgfqpoint{3.089818in}{0.651071in}}%
\pgfpathlineto{\pgfqpoint{3.090351in}{0.676269in}}%
\pgfpathlineto{\pgfqpoint{3.090885in}{0.753024in}}%
\pgfpathlineto{\pgfqpoint{3.091419in}{0.673667in}}%
\pgfpathlineto{\pgfqpoint{3.092486in}{0.762942in}}%
\pgfpathlineto{\pgfqpoint{3.093020in}{0.719625in}}%
\pgfpathlineto{\pgfqpoint{3.093554in}{0.697292in}}%
\pgfpathlineto{\pgfqpoint{3.094087in}{0.847386in}}%
\pgfpathlineto{\pgfqpoint{3.094621in}{0.799491in}}%
\pgfpathlineto{\pgfqpoint{3.095155in}{0.750758in}}%
\pgfpathlineto{\pgfqpoint{3.095689in}{0.822752in}}%
\pgfpathlineto{\pgfqpoint{3.096222in}{0.725095in}}%
\pgfpathlineto{\pgfqpoint{3.096756in}{0.846225in}}%
\pgfpathlineto{\pgfqpoint{3.097290in}{0.655524in}}%
\pgfpathlineto{\pgfqpoint{3.097823in}{0.801435in}}%
\pgfpathlineto{\pgfqpoint{3.099958in}{0.664345in}}%
\pgfpathlineto{\pgfqpoint{3.101026in}{0.916135in}}%
\pgfpathlineto{\pgfqpoint{3.101559in}{0.797747in}}%
\pgfpathlineto{\pgfqpoint{3.102093in}{0.693843in}}%
\pgfpathlineto{\pgfqpoint{3.102627in}{0.745930in}}%
\pgfpathlineto{\pgfqpoint{3.103160in}{0.755357in}}%
\pgfpathlineto{\pgfqpoint{3.103694in}{0.791954in}}%
\pgfpathlineto{\pgfqpoint{3.104228in}{0.775308in}}%
\pgfpathlineto{\pgfqpoint{3.104761in}{0.755560in}}%
\pgfpathlineto{\pgfqpoint{3.105295in}{0.689462in}}%
\pgfpathlineto{\pgfqpoint{3.105829in}{0.763725in}}%
\pgfpathlineto{\pgfqpoint{3.106363in}{0.690159in}}%
\pgfpathlineto{\pgfqpoint{3.109031in}{0.650687in}}%
\pgfpathlineto{\pgfqpoint{3.110632in}{0.689415in}}%
\pgfpathlineto{\pgfqpoint{3.111166in}{0.644907in}}%
\pgfpathlineto{\pgfqpoint{3.111700in}{0.657283in}}%
\pgfpathlineto{\pgfqpoint{3.112233in}{0.676422in}}%
\pgfpathlineto{\pgfqpoint{3.112767in}{0.671523in}}%
\pgfpathlineto{\pgfqpoint{3.113301in}{0.672320in}}%
\pgfpathlineto{\pgfqpoint{3.113834in}{0.675808in}}%
\pgfpathlineto{\pgfqpoint{3.114368in}{0.644345in}}%
\pgfpathlineto{\pgfqpoint{3.114902in}{0.673716in}}%
\pgfpathlineto{\pgfqpoint{3.115435in}{0.654755in}}%
\pgfpathlineto{\pgfqpoint{3.115969in}{0.662711in}}%
\pgfpathlineto{\pgfqpoint{3.116503in}{0.666396in}}%
\pgfpathlineto{\pgfqpoint{3.117037in}{0.661876in}}%
\pgfpathlineto{\pgfqpoint{3.117570in}{0.690555in}}%
\pgfpathlineto{\pgfqpoint{3.118104in}{0.659603in}}%
\pgfpathlineto{\pgfqpoint{3.118638in}{0.644016in}}%
\pgfpathlineto{\pgfqpoint{3.119171in}{0.662194in}}%
\pgfpathlineto{\pgfqpoint{3.119705in}{0.655836in}}%
\pgfpathlineto{\pgfqpoint{3.120239in}{0.643397in}}%
\pgfpathlineto{\pgfqpoint{3.120772in}{0.670249in}}%
\pgfpathlineto{\pgfqpoint{3.121306in}{0.658075in}}%
\pgfpathlineto{\pgfqpoint{3.121840in}{0.654713in}}%
\pgfpathlineto{\pgfqpoint{3.122374in}{0.670460in}}%
\pgfpathlineto{\pgfqpoint{3.123441in}{0.646562in}}%
\pgfpathlineto{\pgfqpoint{3.123975in}{0.675223in}}%
\pgfpathlineto{\pgfqpoint{3.125042in}{0.672889in}}%
\pgfpathlineto{\pgfqpoint{3.125576in}{0.669432in}}%
\pgfpathlineto{\pgfqpoint{3.126109in}{0.647171in}}%
\pgfpathlineto{\pgfqpoint{3.126643in}{0.659670in}}%
\pgfpathlineto{\pgfqpoint{3.128778in}{0.643675in}}%
\pgfpathlineto{\pgfqpoint{3.129312in}{0.673087in}}%
\pgfpathlineto{\pgfqpoint{3.129845in}{0.657282in}}%
\pgfpathlineto{\pgfqpoint{3.130379in}{0.667929in}}%
\pgfpathlineto{\pgfqpoint{3.130913in}{0.657780in}}%
\pgfpathlineto{\pgfqpoint{3.131446in}{0.645623in}}%
\pgfpathlineto{\pgfqpoint{3.131980in}{0.673709in}}%
\pgfpathlineto{\pgfqpoint{3.132514in}{0.670439in}}%
\pgfpathlineto{\pgfqpoint{3.133048in}{0.665034in}}%
\pgfpathlineto{\pgfqpoint{3.135716in}{0.720134in}}%
\pgfpathlineto{\pgfqpoint{3.136250in}{0.713279in}}%
\pgfpathlineto{\pgfqpoint{3.136783in}{0.761397in}}%
\pgfpathlineto{\pgfqpoint{3.137317in}{0.725477in}}%
\pgfpathlineto{\pgfqpoint{3.139986in}{0.665304in}}%
\pgfpathlineto{\pgfqpoint{3.140519in}{0.670412in}}%
\pgfpathlineto{\pgfqpoint{3.141053in}{0.741231in}}%
\pgfpathlineto{\pgfqpoint{3.141587in}{0.734248in}}%
\pgfpathlineto{\pgfqpoint{3.142654in}{0.661966in}}%
\pgfpathlineto{\pgfqpoint{3.143188in}{0.682468in}}%
\pgfpathlineto{\pgfqpoint{3.143722in}{0.693462in}}%
\pgfpathlineto{\pgfqpoint{3.144255in}{0.751875in}}%
\pgfpathlineto{\pgfqpoint{3.145323in}{0.655347in}}%
\pgfpathlineto{\pgfqpoint{3.146390in}{0.720420in}}%
\pgfpathlineto{\pgfqpoint{3.146924in}{0.660988in}}%
\pgfpathlineto{\pgfqpoint{3.147457in}{0.744656in}}%
\pgfpathlineto{\pgfqpoint{3.147991in}{0.700822in}}%
\pgfpathlineto{\pgfqpoint{3.148525in}{0.673110in}}%
\pgfpathlineto{\pgfqpoint{3.149592in}{0.770145in}}%
\pgfpathlineto{\pgfqpoint{3.150126in}{0.746712in}}%
\pgfpathlineto{\pgfqpoint{3.150660in}{0.696079in}}%
\pgfpathlineto{\pgfqpoint{3.151193in}{0.702404in}}%
\pgfpathlineto{\pgfqpoint{3.152794in}{0.821897in}}%
\pgfpathlineto{\pgfqpoint{3.153862in}{0.730879in}}%
\pgfpathlineto{\pgfqpoint{3.154396in}{0.655712in}}%
\pgfpathlineto{\pgfqpoint{3.154929in}{0.670330in}}%
\pgfpathlineto{\pgfqpoint{3.155463in}{0.682603in}}%
\pgfpathlineto{\pgfqpoint{3.156530in}{0.844590in}}%
\pgfpathlineto{\pgfqpoint{3.158131in}{0.663075in}}%
\pgfpathlineto{\pgfqpoint{3.158665in}{0.794663in}}%
\pgfpathlineto{\pgfqpoint{3.159199in}{0.733114in}}%
\pgfpathlineto{\pgfqpoint{3.159733in}{0.752079in}}%
\pgfpathlineto{\pgfqpoint{3.161334in}{0.643941in}}%
\pgfpathlineto{\pgfqpoint{3.161867in}{0.690133in}}%
\pgfpathlineto{\pgfqpoint{3.162935in}{0.688042in}}%
\pgfpathlineto{\pgfqpoint{3.164002in}{0.648597in}}%
\pgfpathlineto{\pgfqpoint{3.164536in}{0.678068in}}%
\pgfpathlineto{\pgfqpoint{3.165070in}{0.664978in}}%
\pgfpathlineto{\pgfqpoint{3.165603in}{0.675045in}}%
\pgfpathlineto{\pgfqpoint{3.166137in}{0.637561in}}%
\pgfpathlineto{\pgfqpoint{3.166671in}{0.677933in}}%
\pgfpathlineto{\pgfqpoint{3.167738in}{0.658732in}}%
\pgfpathlineto{\pgfqpoint{3.168272in}{0.663015in}}%
\pgfpathlineto{\pgfqpoint{3.168806in}{0.658674in}}%
\pgfpathlineto{\pgfqpoint{3.169339in}{0.662493in}}%
\pgfpathlineto{\pgfqpoint{3.169873in}{0.665216in}}%
\pgfpathlineto{\pgfqpoint{3.170940in}{0.645880in}}%
\pgfpathlineto{\pgfqpoint{3.171474in}{0.655699in}}%
\pgfpathlineto{\pgfqpoint{3.172008in}{0.649328in}}%
\pgfpathlineto{\pgfqpoint{3.172541in}{0.645841in}}%
\pgfpathlineto{\pgfqpoint{3.174143in}{0.653945in}}%
\pgfpathlineto{\pgfqpoint{3.174676in}{0.644400in}}%
\pgfpathlineto{\pgfqpoint{3.175210in}{0.646747in}}%
\pgfpathlineto{\pgfqpoint{3.176277in}{0.664915in}}%
\pgfpathlineto{\pgfqpoint{3.177878in}{0.651227in}}%
\pgfpathlineto{\pgfqpoint{3.178412in}{0.662098in}}%
\pgfpathlineto{\pgfqpoint{3.178946in}{0.650655in}}%
\pgfpathlineto{\pgfqpoint{3.179480in}{0.642899in}}%
\pgfpathlineto{\pgfqpoint{3.180013in}{0.662810in}}%
\pgfpathlineto{\pgfqpoint{3.180547in}{0.658467in}}%
\pgfpathlineto{\pgfqpoint{3.181081in}{0.643337in}}%
\pgfpathlineto{\pgfqpoint{3.182148in}{0.644332in}}%
\pgfpathlineto{\pgfqpoint{3.183215in}{0.671409in}}%
\pgfpathlineto{\pgfqpoint{3.183749in}{0.642011in}}%
\pgfpathlineto{\pgfqpoint{3.184283in}{0.679410in}}%
\pgfpathlineto{\pgfqpoint{3.184817in}{0.662599in}}%
\pgfpathlineto{\pgfqpoint{3.185350in}{0.664686in}}%
\pgfpathlineto{\pgfqpoint{3.185884in}{0.663023in}}%
\pgfpathlineto{\pgfqpoint{3.186418in}{0.649360in}}%
\pgfpathlineto{\pgfqpoint{3.186951in}{0.673436in}}%
\pgfpathlineto{\pgfqpoint{3.187485in}{0.637062in}}%
\pgfpathlineto{\pgfqpoint{3.188019in}{0.679801in}}%
\pgfpathlineto{\pgfqpoint{3.188552in}{0.657436in}}%
\pgfpathlineto{\pgfqpoint{3.189620in}{0.669376in}}%
\pgfpathlineto{\pgfqpoint{3.190154in}{0.655588in}}%
\pgfpathlineto{\pgfqpoint{3.190687in}{0.664595in}}%
\pgfpathlineto{\pgfqpoint{3.191221in}{0.687265in}}%
\pgfpathlineto{\pgfqpoint{3.191755in}{0.662242in}}%
\pgfpathlineto{\pgfqpoint{3.193889in}{0.729367in}}%
\pgfpathlineto{\pgfqpoint{3.194423in}{0.724056in}}%
\pgfpathlineto{\pgfqpoint{3.195491in}{0.752871in}}%
\pgfpathlineto{\pgfqpoint{3.197092in}{0.670275in}}%
\pgfpathlineto{\pgfqpoint{3.197625in}{0.690850in}}%
\pgfpathlineto{\pgfqpoint{3.198159in}{0.756439in}}%
\pgfpathlineto{\pgfqpoint{3.198693in}{0.724225in}}%
\pgfpathlineto{\pgfqpoint{3.199760in}{0.679496in}}%
\pgfpathlineto{\pgfqpoint{3.200294in}{0.686809in}}%
\pgfpathlineto{\pgfqpoint{3.200828in}{0.687611in}}%
\pgfpathlineto{\pgfqpoint{3.201361in}{0.756148in}}%
\pgfpathlineto{\pgfqpoint{3.201895in}{0.674386in}}%
\pgfpathlineto{\pgfqpoint{3.202429in}{0.677866in}}%
\pgfpathlineto{\pgfqpoint{3.203496in}{0.696731in}}%
\pgfpathlineto{\pgfqpoint{3.204030in}{0.670061in}}%
\pgfpathlineto{\pgfqpoint{3.204563in}{0.790538in}}%
\pgfpathlineto{\pgfqpoint{3.205097in}{0.748987in}}%
\pgfpathlineto{\pgfqpoint{3.205631in}{0.676000in}}%
\pgfpathlineto{\pgfqpoint{3.207232in}{0.850337in}}%
\pgfpathlineto{\pgfqpoint{3.208833in}{0.691548in}}%
\pgfpathlineto{\pgfqpoint{3.209367in}{0.801318in}}%
\pgfpathlineto{\pgfqpoint{3.209900in}{0.741145in}}%
\pgfpathlineto{\pgfqpoint{3.210434in}{0.675986in}}%
\pgfpathlineto{\pgfqpoint{3.210968in}{0.742523in}}%
\pgfpathlineto{\pgfqpoint{3.211502in}{0.902721in}}%
\pgfpathlineto{\pgfqpoint{3.212035in}{0.709366in}}%
\pgfpathlineto{\pgfqpoint{3.213103in}{0.716896in}}%
\pgfpathlineto{\pgfqpoint{3.213636in}{0.737216in}}%
\pgfpathlineto{\pgfqpoint{3.214704in}{0.740210in}}%
\pgfpathlineto{\pgfqpoint{3.215237in}{0.658840in}}%
\pgfpathlineto{\pgfqpoint{3.215771in}{0.655971in}}%
\pgfpathlineto{\pgfqpoint{3.216305in}{0.656983in}}%
\pgfpathlineto{\pgfqpoint{3.216839in}{0.679608in}}%
\pgfpathlineto{\pgfqpoint{3.217372in}{0.668250in}}%
\pgfpathlineto{\pgfqpoint{3.217906in}{0.656018in}}%
\pgfpathlineto{\pgfqpoint{3.218440in}{0.669146in}}%
\pgfpathlineto{\pgfqpoint{3.218973in}{0.708072in}}%
\pgfpathlineto{\pgfqpoint{3.219507in}{0.647279in}}%
\pgfpathlineto{\pgfqpoint{3.220041in}{0.656429in}}%
\pgfpathlineto{\pgfqpoint{3.220574in}{0.669943in}}%
\pgfpathlineto{\pgfqpoint{3.221108in}{0.639399in}}%
\pgfpathlineto{\pgfqpoint{3.221642in}{0.682062in}}%
\pgfpathlineto{\pgfqpoint{3.222709in}{0.679705in}}%
\pgfpathlineto{\pgfqpoint{3.223243in}{0.649734in}}%
\pgfpathlineto{\pgfqpoint{3.223777in}{0.651744in}}%
\pgfpathlineto{\pgfqpoint{3.224310in}{0.685716in}}%
\pgfpathlineto{\pgfqpoint{3.224844in}{0.655552in}}%
\pgfpathlineto{\pgfqpoint{3.225912in}{0.651779in}}%
\pgfpathlineto{\pgfqpoint{3.226445in}{0.643001in}}%
\pgfpathlineto{\pgfqpoint{3.227513in}{0.685160in}}%
\pgfpathlineto{\pgfqpoint{3.228046in}{0.671017in}}%
\pgfpathlineto{\pgfqpoint{3.228580in}{0.659363in}}%
\pgfpathlineto{\pgfqpoint{3.229114in}{0.672033in}}%
\pgfpathlineto{\pgfqpoint{3.229647in}{0.652451in}}%
\pgfpathlineto{\pgfqpoint{3.230181in}{0.679221in}}%
\pgfpathlineto{\pgfqpoint{3.230715in}{0.665750in}}%
\pgfpathlineto{\pgfqpoint{3.231249in}{0.648645in}}%
\pgfpathlineto{\pgfqpoint{3.231782in}{0.680480in}}%
\pgfpathlineto{\pgfqpoint{3.232850in}{0.679212in}}%
\pgfpathlineto{\pgfqpoint{3.233383in}{0.678862in}}%
\pgfpathlineto{\pgfqpoint{3.233917in}{0.647879in}}%
\pgfpathlineto{\pgfqpoint{3.234451in}{0.655024in}}%
\pgfpathlineto{\pgfqpoint{3.234984in}{0.655624in}}%
\pgfpathlineto{\pgfqpoint{3.236052in}{0.663915in}}%
\pgfpathlineto{\pgfqpoint{3.236586in}{0.685253in}}%
\pgfpathlineto{\pgfqpoint{3.237119in}{0.645539in}}%
\pgfpathlineto{\pgfqpoint{3.237653in}{0.668250in}}%
\pgfpathlineto{\pgfqpoint{3.238720in}{0.646485in}}%
\pgfpathlineto{\pgfqpoint{3.239788in}{0.672788in}}%
\pgfpathlineto{\pgfqpoint{3.241389in}{0.645455in}}%
\pgfpathlineto{\pgfqpoint{3.241923in}{0.665223in}}%
\pgfpathlineto{\pgfqpoint{3.242456in}{0.641042in}}%
\pgfpathlineto{\pgfqpoint{3.242990in}{0.650175in}}%
\pgfpathlineto{\pgfqpoint{3.243524in}{0.650109in}}%
\pgfpathlineto{\pgfqpoint{3.244057in}{0.667827in}}%
\pgfpathlineto{\pgfqpoint{3.244591in}{0.650431in}}%
\pgfpathlineto{\pgfqpoint{3.245125in}{0.636209in}}%
\pgfpathlineto{\pgfqpoint{3.245658in}{0.667410in}}%
\pgfpathlineto{\pgfqpoint{3.246192in}{0.663939in}}%
\pgfpathlineto{\pgfqpoint{3.247260in}{0.647189in}}%
\pgfpathlineto{\pgfqpoint{3.248327in}{0.668411in}}%
\pgfpathlineto{\pgfqpoint{3.248861in}{0.648786in}}%
\pgfpathlineto{\pgfqpoint{3.249394in}{0.668295in}}%
\pgfpathlineto{\pgfqpoint{3.251529in}{0.775625in}}%
\pgfpathlineto{\pgfqpoint{3.253130in}{0.737319in}}%
\pgfpathlineto{\pgfqpoint{3.253664in}{0.738714in}}%
\pgfpathlineto{\pgfqpoint{3.254198in}{0.723324in}}%
\pgfpathlineto{\pgfqpoint{3.254731in}{0.674885in}}%
\pgfpathlineto{\pgfqpoint{3.256332in}{0.746460in}}%
\pgfpathlineto{\pgfqpoint{3.257400in}{0.671817in}}%
\pgfpathlineto{\pgfqpoint{3.257934in}{0.689190in}}%
\pgfpathlineto{\pgfqpoint{3.259535in}{0.765046in}}%
\pgfpathlineto{\pgfqpoint{3.260068in}{0.774190in}}%
\pgfpathlineto{\pgfqpoint{3.260602in}{0.759079in}}%
\pgfpathlineto{\pgfqpoint{3.261136in}{0.766840in}}%
\pgfpathlineto{\pgfqpoint{3.261669in}{0.781969in}}%
\pgfpathlineto{\pgfqpoint{3.262203in}{0.842182in}}%
\pgfpathlineto{\pgfqpoint{3.262737in}{0.757945in}}%
\pgfpathlineto{\pgfqpoint{3.264338in}{0.965027in}}%
\pgfpathlineto{\pgfqpoint{3.265405in}{0.651223in}}%
\pgfpathlineto{\pgfqpoint{3.267006in}{0.872115in}}%
\pgfpathlineto{\pgfqpoint{3.268608in}{0.652526in}}%
\pgfpathlineto{\pgfqpoint{3.269141in}{0.744182in}}%
\pgfpathlineto{\pgfqpoint{3.269675in}{0.712804in}}%
\pgfpathlineto{\pgfqpoint{3.270209in}{0.713255in}}%
\pgfpathlineto{\pgfqpoint{3.271276in}{0.661519in}}%
\pgfpathlineto{\pgfqpoint{3.271810in}{0.677027in}}%
\pgfpathlineto{\pgfqpoint{3.272343in}{0.675354in}}%
\pgfpathlineto{\pgfqpoint{3.273411in}{0.650539in}}%
\pgfpathlineto{\pgfqpoint{3.273945in}{0.684656in}}%
\pgfpathlineto{\pgfqpoint{3.274478in}{0.667440in}}%
\pgfpathlineto{\pgfqpoint{3.275012in}{0.659522in}}%
\pgfpathlineto{\pgfqpoint{3.275546in}{0.687423in}}%
\pgfpathlineto{\pgfqpoint{3.276079in}{0.672320in}}%
\pgfpathlineto{\pgfqpoint{3.276613in}{0.666110in}}%
\pgfpathlineto{\pgfqpoint{3.277147in}{0.705573in}}%
\pgfpathlineto{\pgfqpoint{3.277680in}{0.671965in}}%
\pgfpathlineto{\pgfqpoint{3.278748in}{0.640280in}}%
\pgfpathlineto{\pgfqpoint{3.280349in}{0.666166in}}%
\pgfpathlineto{\pgfqpoint{3.281416in}{0.647786in}}%
\pgfpathlineto{\pgfqpoint{3.283017in}{0.698275in}}%
\pgfpathlineto{\pgfqpoint{3.284619in}{0.638197in}}%
\pgfpathlineto{\pgfqpoint{3.285152in}{0.688047in}}%
\pgfpathlineto{\pgfqpoint{3.285686in}{0.658404in}}%
\pgfpathlineto{\pgfqpoint{3.286220in}{0.661407in}}%
\pgfpathlineto{\pgfqpoint{3.287287in}{0.640023in}}%
\pgfpathlineto{\pgfqpoint{3.288354in}{0.683728in}}%
\pgfpathlineto{\pgfqpoint{3.288888in}{0.642856in}}%
\pgfpathlineto{\pgfqpoint{3.289422in}{0.672723in}}%
\pgfpathlineto{\pgfqpoint{3.290489in}{0.646004in}}%
\pgfpathlineto{\pgfqpoint{3.291023in}{0.693007in}}%
\pgfpathlineto{\pgfqpoint{3.291557in}{0.644689in}}%
\pgfpathlineto{\pgfqpoint{3.293692in}{0.669225in}}%
\pgfpathlineto{\pgfqpoint{3.294225in}{0.664805in}}%
\pgfpathlineto{\pgfqpoint{3.294759in}{0.684189in}}%
\pgfpathlineto{\pgfqpoint{3.295293in}{0.661061in}}%
\pgfpathlineto{\pgfqpoint{3.295826in}{0.663459in}}%
\pgfpathlineto{\pgfqpoint{3.296360in}{0.668174in}}%
\pgfpathlineto{\pgfqpoint{3.296894in}{0.644517in}}%
\pgfpathlineto{\pgfqpoint{3.297427in}{0.672045in}}%
\pgfpathlineto{\pgfqpoint{3.297961in}{0.667189in}}%
\pgfpathlineto{\pgfqpoint{3.298495in}{0.659037in}}%
\pgfpathlineto{\pgfqpoint{3.299029in}{0.679540in}}%
\pgfpathlineto{\pgfqpoint{3.299562in}{0.659779in}}%
\pgfpathlineto{\pgfqpoint{3.300096in}{0.643287in}}%
\pgfpathlineto{\pgfqpoint{3.301697in}{0.671911in}}%
\pgfpathlineto{\pgfqpoint{3.302231in}{0.638061in}}%
\pgfpathlineto{\pgfqpoint{3.302764in}{0.670822in}}%
\pgfpathlineto{\pgfqpoint{3.304366in}{0.648469in}}%
\pgfpathlineto{\pgfqpoint{3.304899in}{0.682837in}}%
\pgfpathlineto{\pgfqpoint{3.305433in}{0.658860in}}%
\pgfpathlineto{\pgfqpoint{3.305967in}{0.648463in}}%
\pgfpathlineto{\pgfqpoint{3.307034in}{0.697588in}}%
\pgfpathlineto{\pgfqpoint{3.308101in}{0.729840in}}%
\pgfpathlineto{\pgfqpoint{3.308635in}{0.717120in}}%
\pgfpathlineto{\pgfqpoint{3.310236in}{0.887250in}}%
\pgfpathlineto{\pgfqpoint{3.312371in}{0.660200in}}%
\pgfpathlineto{\pgfqpoint{3.313438in}{0.823257in}}%
\pgfpathlineto{\pgfqpoint{3.313972in}{0.764894in}}%
\pgfpathlineto{\pgfqpoint{3.314506in}{0.704604in}}%
\pgfpathlineto{\pgfqpoint{3.315040in}{0.769628in}}%
\pgfpathlineto{\pgfqpoint{3.315573in}{0.815808in}}%
\pgfpathlineto{\pgfqpoint{3.316107in}{0.754725in}}%
\pgfpathlineto{\pgfqpoint{3.317708in}{0.997517in}}%
\pgfpathlineto{\pgfqpoint{3.318775in}{0.702648in}}%
\pgfpathlineto{\pgfqpoint{3.319843in}{0.951717in}}%
\pgfpathlineto{\pgfqpoint{3.321444in}{0.740294in}}%
\pgfpathlineto{\pgfqpoint{3.321978in}{1.024803in}}%
\pgfpathlineto{\pgfqpoint{3.322511in}{0.780180in}}%
\pgfpathlineto{\pgfqpoint{3.323045in}{0.781482in}}%
\pgfpathlineto{\pgfqpoint{3.324646in}{0.650935in}}%
\pgfpathlineto{\pgfqpoint{3.325180in}{0.757550in}}%
\pgfpathlineto{\pgfqpoint{3.325714in}{0.685188in}}%
\pgfpathlineto{\pgfqpoint{3.326247in}{0.706507in}}%
\pgfpathlineto{\pgfqpoint{3.327848in}{0.650170in}}%
\pgfpathlineto{\pgfqpoint{3.328916in}{0.682376in}}%
\pgfpathlineto{\pgfqpoint{3.329449in}{0.651066in}}%
\pgfpathlineto{\pgfqpoint{3.330517in}{0.653382in}}%
\pgfpathlineto{\pgfqpoint{3.331584in}{0.718236in}}%
\pgfpathlineto{\pgfqpoint{3.332118in}{0.702858in}}%
\pgfpathlineto{\pgfqpoint{3.332652in}{0.708142in}}%
\pgfpathlineto{\pgfqpoint{3.334253in}{0.670885in}}%
\pgfpathlineto{\pgfqpoint{3.334786in}{0.692920in}}%
\pgfpathlineto{\pgfqpoint{3.335320in}{0.669136in}}%
\pgfpathlineto{\pgfqpoint{3.336388in}{0.646360in}}%
\pgfpathlineto{\pgfqpoint{3.337989in}{0.662171in}}%
\pgfpathlineto{\pgfqpoint{3.338522in}{0.653498in}}%
\pgfpathlineto{\pgfqpoint{3.339056in}{0.674375in}}%
\pgfpathlineto{\pgfqpoint{3.339590in}{0.665520in}}%
\pgfpathlineto{\pgfqpoint{3.341725in}{0.645439in}}%
\pgfpathlineto{\pgfqpoint{3.342258in}{0.680156in}}%
\pgfpathlineto{\pgfqpoint{3.342792in}{0.667709in}}%
\pgfpathlineto{\pgfqpoint{3.343326in}{0.668708in}}%
\pgfpathlineto{\pgfqpoint{3.343859in}{0.677319in}}%
\pgfpathlineto{\pgfqpoint{3.344393in}{0.652390in}}%
\pgfpathlineto{\pgfqpoint{3.344927in}{0.666069in}}%
\pgfpathlineto{\pgfqpoint{3.346528in}{0.692286in}}%
\pgfpathlineto{\pgfqpoint{3.347062in}{0.709786in}}%
\pgfpathlineto{\pgfqpoint{3.347595in}{0.654851in}}%
\pgfpathlineto{\pgfqpoint{3.348663in}{0.655457in}}%
\pgfpathlineto{\pgfqpoint{3.349196in}{0.692439in}}%
\pgfpathlineto{\pgfqpoint{3.349730in}{0.649301in}}%
\pgfpathlineto{\pgfqpoint{3.350264in}{0.656398in}}%
\pgfpathlineto{\pgfqpoint{3.350797in}{0.666396in}}%
\pgfpathlineto{\pgfqpoint{3.351331in}{0.657238in}}%
\pgfpathlineto{\pgfqpoint{3.351865in}{0.659727in}}%
\pgfpathlineto{\pgfqpoint{3.352399in}{0.647011in}}%
\pgfpathlineto{\pgfqpoint{3.352932in}{0.671232in}}%
\pgfpathlineto{\pgfqpoint{3.353466in}{0.651350in}}%
\pgfpathlineto{\pgfqpoint{3.354000in}{0.653717in}}%
\pgfpathlineto{\pgfqpoint{3.354533in}{0.665834in}}%
\pgfpathlineto{\pgfqpoint{3.355067in}{0.646356in}}%
\pgfpathlineto{\pgfqpoint{3.355601in}{0.701345in}}%
\pgfpathlineto{\pgfqpoint{3.356134in}{0.671551in}}%
\pgfpathlineto{\pgfqpoint{3.356668in}{0.645629in}}%
\pgfpathlineto{\pgfqpoint{3.357202in}{0.676689in}}%
\pgfpathlineto{\pgfqpoint{3.357736in}{0.650700in}}%
\pgfpathlineto{\pgfqpoint{3.358269in}{0.642592in}}%
\pgfpathlineto{\pgfqpoint{3.358803in}{0.645172in}}%
\pgfpathlineto{\pgfqpoint{3.359337in}{0.678788in}}%
\pgfpathlineto{\pgfqpoint{3.359870in}{0.663999in}}%
\pgfpathlineto{\pgfqpoint{3.360404in}{0.671879in}}%
\pgfpathlineto{\pgfqpoint{3.360938in}{0.647653in}}%
\pgfpathlineto{\pgfqpoint{3.362539in}{0.684604in}}%
\pgfpathlineto{\pgfqpoint{3.363073in}{0.692561in}}%
\pgfpathlineto{\pgfqpoint{3.363606in}{0.679607in}}%
\pgfpathlineto{\pgfqpoint{3.365207in}{0.712678in}}%
\pgfpathlineto{\pgfqpoint{3.365741in}{0.701937in}}%
\pgfpathlineto{\pgfqpoint{3.367876in}{0.890035in}}%
\pgfpathlineto{\pgfqpoint{3.369477in}{0.708808in}}%
\pgfpathlineto{\pgfqpoint{3.370011in}{0.731198in}}%
\pgfpathlineto{\pgfqpoint{3.371612in}{0.828383in}}%
\pgfpathlineto{\pgfqpoint{3.372146in}{0.786005in}}%
\pgfpathlineto{\pgfqpoint{3.372679in}{0.901801in}}%
\pgfpathlineto{\pgfqpoint{3.373213in}{0.688765in}}%
\pgfpathlineto{\pgfqpoint{3.373747in}{0.816193in}}%
\pgfpathlineto{\pgfqpoint{3.374280in}{0.865274in}}%
\pgfpathlineto{\pgfqpoint{3.374814in}{1.178039in}}%
\pgfpathlineto{\pgfqpoint{3.376415in}{0.734617in}}%
\pgfpathlineto{\pgfqpoint{3.376949in}{0.925873in}}%
\pgfpathlineto{\pgfqpoint{3.378550in}{0.662495in}}%
\pgfpathlineto{\pgfqpoint{3.379084in}{0.731605in}}%
\pgfpathlineto{\pgfqpoint{3.379617in}{0.642160in}}%
\pgfpathlineto{\pgfqpoint{3.380151in}{0.773743in}}%
\pgfpathlineto{\pgfqpoint{3.380685in}{0.670972in}}%
\pgfpathlineto{\pgfqpoint{3.381218in}{0.682818in}}%
\pgfpathlineto{\pgfqpoint{3.382820in}{0.653992in}}%
\pgfpathlineto{\pgfqpoint{3.383353in}{0.653762in}}%
\pgfpathlineto{\pgfqpoint{3.383887in}{0.658480in}}%
\pgfpathlineto{\pgfqpoint{3.384421in}{0.652775in}}%
\pgfpathlineto{\pgfqpoint{3.384954in}{0.707831in}}%
\pgfpathlineto{\pgfqpoint{3.385488in}{0.691431in}}%
\pgfpathlineto{\pgfqpoint{3.386022in}{0.641835in}}%
\pgfpathlineto{\pgfqpoint{3.386555in}{0.686487in}}%
\pgfpathlineto{\pgfqpoint{3.387089in}{0.706379in}}%
\pgfpathlineto{\pgfqpoint{3.389224in}{0.641493in}}%
\pgfpathlineto{\pgfqpoint{3.389758in}{0.685026in}}%
\pgfpathlineto{\pgfqpoint{3.390291in}{0.666447in}}%
\pgfpathlineto{\pgfqpoint{3.391892in}{0.650952in}}%
\pgfpathlineto{\pgfqpoint{3.392426in}{0.678474in}}%
\pgfpathlineto{\pgfqpoint{3.392960in}{0.663583in}}%
\pgfpathlineto{\pgfqpoint{3.394027in}{0.644220in}}%
\pgfpathlineto{\pgfqpoint{3.394561in}{0.696829in}}%
\pgfpathlineto{\pgfqpoint{3.395095in}{0.658526in}}%
\pgfpathlineto{\pgfqpoint{3.395628in}{0.690114in}}%
\pgfpathlineto{\pgfqpoint{3.396162in}{0.640059in}}%
\pgfpathlineto{\pgfqpoint{3.396696in}{0.665114in}}%
\pgfpathlineto{\pgfqpoint{3.398297in}{0.737161in}}%
\pgfpathlineto{\pgfqpoint{3.398831in}{0.658921in}}%
\pgfpathlineto{\pgfqpoint{3.399364in}{0.733251in}}%
\pgfpathlineto{\pgfqpoint{3.401499in}{0.668473in}}%
\pgfpathlineto{\pgfqpoint{3.402033in}{0.698114in}}%
\pgfpathlineto{\pgfqpoint{3.403100in}{0.720960in}}%
\pgfpathlineto{\pgfqpoint{3.403634in}{0.691742in}}%
\pgfpathlineto{\pgfqpoint{3.404168in}{0.778380in}}%
\pgfpathlineto{\pgfqpoint{3.404701in}{0.700628in}}%
\pgfpathlineto{\pgfqpoint{3.405235in}{0.749810in}}%
\pgfpathlineto{\pgfqpoint{3.405769in}{0.705844in}}%
\pgfpathlineto{\pgfqpoint{3.406302in}{0.700690in}}%
\pgfpathlineto{\pgfqpoint{3.407370in}{0.645571in}}%
\pgfpathlineto{\pgfqpoint{3.408971in}{0.711734in}}%
\pgfpathlineto{\pgfqpoint{3.409505in}{0.657850in}}%
\pgfpathlineto{\pgfqpoint{3.410038in}{0.740566in}}%
\pgfpathlineto{\pgfqpoint{3.410572in}{0.677219in}}%
\pgfpathlineto{\pgfqpoint{3.412173in}{0.707963in}}%
\pgfpathlineto{\pgfqpoint{3.412707in}{0.661738in}}%
\pgfpathlineto{\pgfqpoint{3.413774in}{0.663769in}}%
\pgfpathlineto{\pgfqpoint{3.415375in}{0.698793in}}%
\pgfpathlineto{\pgfqpoint{3.415909in}{0.669506in}}%
\pgfpathlineto{\pgfqpoint{3.416443in}{0.721186in}}%
\pgfpathlineto{\pgfqpoint{3.416976in}{0.687312in}}%
\pgfpathlineto{\pgfqpoint{3.417510in}{0.670799in}}%
\pgfpathlineto{\pgfqpoint{3.418044in}{0.699980in}}%
\pgfpathlineto{\pgfqpoint{3.418577in}{0.660161in}}%
\pgfpathlineto{\pgfqpoint{3.419111in}{0.692446in}}%
\pgfpathlineto{\pgfqpoint{3.419645in}{0.705428in}}%
\pgfpathlineto{\pgfqpoint{3.420179in}{0.696364in}}%
\pgfpathlineto{\pgfqpoint{3.421780in}{0.652424in}}%
\pgfpathlineto{\pgfqpoint{3.423381in}{0.696312in}}%
\pgfpathlineto{\pgfqpoint{3.424448in}{0.743306in}}%
\pgfpathlineto{\pgfqpoint{3.424982in}{0.759390in}}%
\pgfpathlineto{\pgfqpoint{3.425516in}{0.734562in}}%
\pgfpathlineto{\pgfqpoint{3.426049in}{0.791317in}}%
\pgfpathlineto{\pgfqpoint{3.426583in}{0.766346in}}%
\pgfpathlineto{\pgfqpoint{3.427650in}{0.751124in}}%
\pgfpathlineto{\pgfqpoint{3.428718in}{0.811558in}}%
\pgfpathlineto{\pgfqpoint{3.429252in}{0.737152in}}%
\pgfpathlineto{\pgfqpoint{3.429785in}{1.052081in}}%
\pgfpathlineto{\pgfqpoint{3.430319in}{0.755478in}}%
\pgfpathlineto{\pgfqpoint{3.430853in}{0.686064in}}%
\pgfpathlineto{\pgfqpoint{3.431386in}{0.746908in}}%
\pgfpathlineto{\pgfqpoint{3.431920in}{0.706833in}}%
\pgfpathlineto{\pgfqpoint{3.432454in}{0.714145in}}%
\pgfpathlineto{\pgfqpoint{3.433521in}{0.763971in}}%
\pgfpathlineto{\pgfqpoint{3.434055in}{0.681107in}}%
\pgfpathlineto{\pgfqpoint{3.434589in}{0.754848in}}%
\pgfpathlineto{\pgfqpoint{3.435122in}{0.707580in}}%
\pgfpathlineto{\pgfqpoint{3.435656in}{0.735064in}}%
\pgfpathlineto{\pgfqpoint{3.436190in}{0.745188in}}%
\pgfpathlineto{\pgfqpoint{3.436723in}{0.739247in}}%
\pgfpathlineto{\pgfqpoint{3.437257in}{0.743619in}}%
\pgfpathlineto{\pgfqpoint{3.437791in}{0.725464in}}%
\pgfpathlineto{\pgfqpoint{3.439392in}{0.795147in}}%
\pgfpathlineto{\pgfqpoint{3.440459in}{0.678959in}}%
\pgfpathlineto{\pgfqpoint{3.440993in}{0.693928in}}%
\pgfpathlineto{\pgfqpoint{3.441527in}{0.647526in}}%
\pgfpathlineto{\pgfqpoint{3.442060in}{0.679622in}}%
\pgfpathlineto{\pgfqpoint{3.443661in}{0.791057in}}%
\pgfpathlineto{\pgfqpoint{3.444729in}{0.715438in}}%
\pgfpathlineto{\pgfqpoint{3.445796in}{0.720207in}}%
\pgfpathlineto{\pgfqpoint{3.446330in}{0.703231in}}%
\pgfpathlineto{\pgfqpoint{3.446864in}{0.706623in}}%
\pgfpathlineto{\pgfqpoint{3.447397in}{0.719205in}}%
\pgfpathlineto{\pgfqpoint{3.447931in}{0.664871in}}%
\pgfpathlineto{\pgfqpoint{3.448465in}{0.761942in}}%
\pgfpathlineto{\pgfqpoint{3.448998in}{0.692393in}}%
\pgfpathlineto{\pgfqpoint{3.449532in}{0.686691in}}%
\pgfpathlineto{\pgfqpoint{3.450066in}{0.644547in}}%
\pgfpathlineto{\pgfqpoint{3.450600in}{0.690101in}}%
\pgfpathlineto{\pgfqpoint{3.451667in}{0.716081in}}%
\pgfpathlineto{\pgfqpoint{3.452201in}{0.698186in}}%
\pgfpathlineto{\pgfqpoint{3.453268in}{0.658667in}}%
\pgfpathlineto{\pgfqpoint{3.454335in}{0.769855in}}%
\pgfpathlineto{\pgfqpoint{3.455403in}{0.651300in}}%
\pgfpathlineto{\pgfqpoint{3.457004in}{0.748307in}}%
\pgfpathlineto{\pgfqpoint{3.458071in}{0.695860in}}%
\pgfpathlineto{\pgfqpoint{3.459139in}{0.774322in}}%
\pgfpathlineto{\pgfqpoint{3.459672in}{0.743243in}}%
\pgfpathlineto{\pgfqpoint{3.460206in}{0.656309in}}%
\pgfpathlineto{\pgfqpoint{3.460740in}{0.744665in}}%
\pgfpathlineto{\pgfqpoint{3.461274in}{0.770886in}}%
\pgfpathlineto{\pgfqpoint{3.461807in}{0.705081in}}%
\pgfpathlineto{\pgfqpoint{3.462341in}{0.708818in}}%
\pgfpathlineto{\pgfqpoint{3.462875in}{0.712057in}}%
\pgfpathlineto{\pgfqpoint{3.463408in}{0.754372in}}%
\pgfpathlineto{\pgfqpoint{3.463942in}{0.739121in}}%
\pgfpathlineto{\pgfqpoint{3.464476in}{0.944442in}}%
\pgfpathlineto{\pgfqpoint{3.465009in}{0.692226in}}%
\pgfpathlineto{\pgfqpoint{3.465543in}{0.852655in}}%
\pgfpathlineto{\pgfqpoint{3.466611in}{0.972856in}}%
\pgfpathlineto{\pgfqpoint{3.468212in}{1.181330in}}%
\pgfpathlineto{\pgfqpoint{3.469813in}{0.871213in}}%
\pgfpathlineto{\pgfqpoint{3.470346in}{1.005663in}}%
\pgfpathlineto{\pgfqpoint{3.470880in}{0.984101in}}%
\pgfpathlineto{\pgfqpoint{3.471414in}{0.929724in}}%
\pgfpathlineto{\pgfqpoint{3.471948in}{1.179134in}}%
\pgfpathlineto{\pgfqpoint{3.472481in}{0.832937in}}%
\pgfpathlineto{\pgfqpoint{3.473015in}{1.199479in}}%
\pgfpathlineto{\pgfqpoint{3.474616in}{0.811315in}}%
\pgfpathlineto{\pgfqpoint{3.475150in}{1.009309in}}%
\pgfpathlineto{\pgfqpoint{3.475683in}{0.869342in}}%
\pgfpathlineto{\pgfqpoint{3.476217in}{0.777734in}}%
\pgfpathlineto{\pgfqpoint{3.476751in}{0.965560in}}%
\pgfpathlineto{\pgfqpoint{3.477285in}{0.675217in}}%
\pgfpathlineto{\pgfqpoint{3.477818in}{1.070616in}}%
\pgfpathlineto{\pgfqpoint{3.478352in}{0.824117in}}%
\pgfpathlineto{\pgfqpoint{3.478886in}{0.830879in}}%
\pgfpathlineto{\pgfqpoint{3.479419in}{0.785290in}}%
\pgfpathlineto{\pgfqpoint{3.479953in}{0.831190in}}%
\pgfpathlineto{\pgfqpoint{3.480487in}{0.816524in}}%
\pgfpathlineto{\pgfqpoint{3.481020in}{0.679641in}}%
\pgfpathlineto{\pgfqpoint{3.481554in}{0.822940in}}%
\pgfpathlineto{\pgfqpoint{3.482088in}{0.706986in}}%
\pgfpathlineto{\pgfqpoint{3.482622in}{0.788045in}}%
\pgfpathlineto{\pgfqpoint{3.483155in}{1.040537in}}%
\pgfpathlineto{\pgfqpoint{3.483689in}{0.943916in}}%
\pgfpathlineto{\pgfqpoint{3.484223in}{0.956382in}}%
\pgfpathlineto{\pgfqpoint{3.485290in}{1.352518in}}%
\pgfpathlineto{\pgfqpoint{3.486891in}{0.674808in}}%
\pgfpathlineto{\pgfqpoint{3.488492in}{0.874099in}}%
\pgfpathlineto{\pgfqpoint{3.489026in}{0.653176in}}%
\pgfpathlineto{\pgfqpoint{3.489560in}{0.680045in}}%
\pgfpathlineto{\pgfqpoint{3.490627in}{0.772274in}}%
\pgfpathlineto{\pgfqpoint{3.491161in}{0.696863in}}%
\pgfpathlineto{\pgfqpoint{3.491694in}{0.750718in}}%
\pgfpathlineto{\pgfqpoint{3.492228in}{0.747431in}}%
\pgfpathlineto{\pgfqpoint{3.493296in}{0.815423in}}%
\pgfpathlineto{\pgfqpoint{3.493829in}{0.673145in}}%
\pgfpathlineto{\pgfqpoint{3.494363in}{0.770206in}}%
\pgfpathlineto{\pgfqpoint{3.494897in}{0.734796in}}%
\pgfpathlineto{\pgfqpoint{3.495430in}{0.783163in}}%
\pgfpathlineto{\pgfqpoint{3.495964in}{0.752901in}}%
\pgfpathlineto{\pgfqpoint{3.496498in}{0.677552in}}%
\pgfpathlineto{\pgfqpoint{3.497565in}{0.675064in}}%
\pgfpathlineto{\pgfqpoint{3.498099in}{0.925650in}}%
\pgfpathlineto{\pgfqpoint{3.499166in}{0.857373in}}%
\pgfpathlineto{\pgfqpoint{3.499700in}{0.934572in}}%
\pgfpathlineto{\pgfqpoint{3.500234in}{0.701780in}}%
\pgfpathlineto{\pgfqpoint{3.500767in}{0.888958in}}%
\pgfpathlineto{\pgfqpoint{3.501301in}{0.856511in}}%
\pgfpathlineto{\pgfqpoint{3.501835in}{0.879471in}}%
\pgfpathlineto{\pgfqpoint{3.502369in}{0.938380in}}%
\pgfpathlineto{\pgfqpoint{3.502902in}{0.866458in}}%
\pgfpathlineto{\pgfqpoint{3.503436in}{0.967087in}}%
\pgfpathlineto{\pgfqpoint{3.504503in}{0.666013in}}%
\pgfpathlineto{\pgfqpoint{3.505037in}{0.805059in}}%
\pgfpathlineto{\pgfqpoint{3.505571in}{0.737995in}}%
\pgfpathlineto{\pgfqpoint{3.506638in}{0.711654in}}%
\pgfpathlineto{\pgfqpoint{3.508239in}{0.774249in}}%
\pgfpathlineto{\pgfqpoint{3.509307in}{0.863504in}}%
\pgfpathlineto{\pgfqpoint{3.509840in}{0.793772in}}%
\pgfpathlineto{\pgfqpoint{3.510374in}{0.859217in}}%
\pgfpathlineto{\pgfqpoint{3.510908in}{0.876048in}}%
\pgfpathlineto{\pgfqpoint{3.511975in}{0.743178in}}%
\pgfpathlineto{\pgfqpoint{3.513576in}{1.015338in}}%
\pgfpathlineto{\pgfqpoint{3.514110in}{0.939491in}}%
\pgfpathlineto{\pgfqpoint{3.514644in}{1.279587in}}%
\pgfpathlineto{\pgfqpoint{3.515177in}{1.235358in}}%
\pgfpathlineto{\pgfqpoint{3.516778in}{0.844391in}}%
\pgfpathlineto{\pgfqpoint{3.517846in}{0.669806in}}%
\pgfpathlineto{\pgfqpoint{3.518380in}{1.077489in}}%
\pgfpathlineto{\pgfqpoint{3.518913in}{0.833676in}}%
\pgfpathlineto{\pgfqpoint{3.519447in}{0.854347in}}%
\pgfpathlineto{\pgfqpoint{3.519981in}{0.930970in}}%
\pgfpathlineto{\pgfqpoint{3.520514in}{1.173789in}}%
\pgfpathlineto{\pgfqpoint{3.521582in}{0.829117in}}%
\pgfpathlineto{\pgfqpoint{3.522115in}{0.901466in}}%
\pgfpathlineto{\pgfqpoint{3.522649in}{1.167222in}}%
\pgfpathlineto{\pgfqpoint{3.523183in}{0.826055in}}%
\pgfpathlineto{\pgfqpoint{3.524250in}{0.836953in}}%
\pgfpathlineto{\pgfqpoint{3.524784in}{0.701277in}}%
\pgfpathlineto{\pgfqpoint{3.525851in}{1.216794in}}%
\pgfpathlineto{\pgfqpoint{3.527452in}{0.711344in}}%
\pgfpathlineto{\pgfqpoint{3.527986in}{0.948857in}}%
\pgfpathlineto{\pgfqpoint{3.529587in}{1.159759in}}%
\pgfpathlineto{\pgfqpoint{3.530121in}{1.662308in}}%
\pgfpathlineto{\pgfqpoint{3.531188in}{0.897930in}}%
\pgfpathlineto{\pgfqpoint{3.531722in}{2.055349in}}%
\pgfpathlineto{\pgfqpoint{3.532256in}{1.733996in}}%
\pgfpathlineto{\pgfqpoint{3.532789in}{1.654769in}}%
\pgfpathlineto{\pgfqpoint{3.534391in}{0.660007in}}%
\pgfpathlineto{\pgfqpoint{3.535458in}{1.221637in}}%
\pgfpathlineto{\pgfqpoint{3.536525in}{0.679389in}}%
\pgfpathlineto{\pgfqpoint{3.537059in}{0.793365in}}%
\pgfpathlineto{\pgfqpoint{3.537593in}{0.910344in}}%
\pgfpathlineto{\pgfqpoint{3.538126in}{0.693540in}}%
\pgfpathlineto{\pgfqpoint{3.538660in}{0.748295in}}%
\pgfpathlineto{\pgfqpoint{3.539194in}{0.757387in}}%
\pgfpathlineto{\pgfqpoint{3.539728in}{0.681478in}}%
\pgfpathlineto{\pgfqpoint{3.539728in}{0.681478in}}%
\pgfusepath{stroke}%
\end{pgfscope}%
\begin{pgfscope}%
\pgfsetrectcap%
\pgfsetmiterjoin%
\pgfsetlinewidth{0.803000pt}%
\definecolor{currentstroke}{rgb}{0.000000,0.000000,0.000000}%
\pgfsetstrokecolor{currentstroke}%
\pgfsetdash{}{0pt}%
\pgfpathmoveto{\pgfqpoint{0.934300in}{0.564143in}}%
\pgfpathlineto{\pgfqpoint{0.934300in}{2.126359in}}%
\pgfusepath{stroke}%
\end{pgfscope}%
\begin{pgfscope}%
\pgfsetrectcap%
\pgfsetmiterjoin%
\pgfsetlinewidth{0.803000pt}%
\definecolor{currentstroke}{rgb}{0.000000,0.000000,0.000000}%
\pgfsetstrokecolor{currentstroke}%
\pgfsetdash{}{0pt}%
\pgfpathmoveto{\pgfqpoint{6.146222in}{0.564143in}}%
\pgfpathlineto{\pgfqpoint{6.146222in}{2.126359in}}%
\pgfusepath{stroke}%
\end{pgfscope}%
\begin{pgfscope}%
\pgfsetrectcap%
\pgfsetmiterjoin%
\pgfsetlinewidth{0.803000pt}%
\definecolor{currentstroke}{rgb}{0.000000,0.000000,0.000000}%
\pgfsetstrokecolor{currentstroke}%
\pgfsetdash{}{0pt}%
\pgfpathmoveto{\pgfqpoint{0.934300in}{0.564143in}}%
\pgfpathlineto{\pgfqpoint{6.146222in}{0.564143in}}%
\pgfusepath{stroke}%
\end{pgfscope}%
\begin{pgfscope}%
\pgfsetrectcap%
\pgfsetmiterjoin%
\pgfsetlinewidth{0.803000pt}%
\definecolor{currentstroke}{rgb}{0.000000,0.000000,0.000000}%
\pgfsetstrokecolor{currentstroke}%
\pgfsetdash{}{0pt}%
\pgfpathmoveto{\pgfqpoint{0.934300in}{2.126359in}}%
\pgfpathlineto{\pgfqpoint{6.146222in}{2.126359in}}%
\pgfusepath{stroke}%
\end{pgfscope}%
\begin{pgfscope}%
\definecolor{textcolor}{rgb}{0.000000,0.000000,0.000000}%
\pgfsetstrokecolor{textcolor}%
\pgfsetfillcolor{textcolor}%
\pgftext[x=3.540261in,y=2.209692in,,base]{\color{textcolor}\rmfamily\fontsize{12.000000}{14.400000}\selectfont Noisy ECG Spectrum}%
\end{pgfscope}%
\end{pgfpicture}%
\makeatother%
\endgroup%

    }
    \caption{Sampled time domain ECG signal with additive noise at 32.6 Hz and 61.7 Hz (top) and its spectrum (bottom).}
    \label{fig:noisy}
\end{figure}

\subsection{Window FIR Filter}
\begin{figure}[H]
    \centering
    \adjustbox{max width=0.75\textwidth}{
    %% Creator: Matplotlib, PGF backend
%%
%% To include the figure in your LaTeX document, write
%%   \input{<filename>.pgf}
%%
%% Make sure the required packages are loaded in your preamble
%%   \usepackage{pgf}
%%
%% and, on pdftex
%%   \usepackage[utf8]{inputenc}\DeclareUnicodeCharacter{2212}{-}
%%
%% or, on luatex and xetex
%%   \usepackage{unicode-math}
%%
%% Figures using additional raster images can only be included by \input if
%% they are in the same directory as the main LaTeX file. For loading figures
%% from other directories you can use the `import` package
%%   \usepackage{import}
%%
%% and then include the figures with
%%   \import{<path to file>}{<filename>.pgf}
%%
%% Matplotlib used the following preamble
%%
\begingroup%
\makeatletter%
\begin{pgfpicture}%
\pgfpathrectangle{\pgfpointorigin}{\pgfqpoint{6.400000in}{4.800000in}}%
\pgfusepath{use as bounding box, clip}%
\begin{pgfscope}%
\pgfsetbuttcap%
\pgfsetmiterjoin%
\definecolor{currentfill}{rgb}{1.000000,1.000000,1.000000}%
\pgfsetfillcolor{currentfill}%
\pgfsetlinewidth{0.000000pt}%
\definecolor{currentstroke}{rgb}{1.000000,1.000000,1.000000}%
\pgfsetstrokecolor{currentstroke}%
\pgfsetdash{}{0pt}%
\pgfpathmoveto{\pgfqpoint{0.000000in}{0.000000in}}%
\pgfpathlineto{\pgfqpoint{6.400000in}{0.000000in}}%
\pgfpathlineto{\pgfqpoint{6.400000in}{4.800000in}}%
\pgfpathlineto{\pgfqpoint{0.000000in}{4.800000in}}%
\pgfpathclose%
\pgfusepath{fill}%
\end{pgfscope}%
\begin{pgfscope}%
\pgfsetbuttcap%
\pgfsetmiterjoin%
\definecolor{currentfill}{rgb}{1.000000,1.000000,1.000000}%
\pgfsetfillcolor{currentfill}%
\pgfsetlinewidth{0.000000pt}%
\definecolor{currentstroke}{rgb}{0.000000,0.000000,0.000000}%
\pgfsetstrokecolor{currentstroke}%
\pgfsetstrokeopacity{0.000000}%
\pgfsetdash{}{0pt}%
\pgfpathmoveto{\pgfqpoint{0.717889in}{3.664143in}}%
\pgfpathlineto{\pgfqpoint{6.146222in}{3.664143in}}%
\pgfpathlineto{\pgfqpoint{6.146222in}{4.451359in}}%
\pgfpathlineto{\pgfqpoint{0.717889in}{4.451359in}}%
\pgfpathclose%
\pgfusepath{fill}%
\end{pgfscope}%
\begin{pgfscope}%
\pgfsetbuttcap%
\pgfsetroundjoin%
\definecolor{currentfill}{rgb}{0.000000,0.000000,0.000000}%
\pgfsetfillcolor{currentfill}%
\pgfsetlinewidth{0.803000pt}%
\definecolor{currentstroke}{rgb}{0.000000,0.000000,0.000000}%
\pgfsetstrokecolor{currentstroke}%
\pgfsetdash{}{0pt}%
\pgfsys@defobject{currentmarker}{\pgfqpoint{0.000000in}{-0.048611in}}{\pgfqpoint{0.000000in}{0.000000in}}{%
\pgfpathmoveto{\pgfqpoint{0.000000in}{0.000000in}}%
\pgfpathlineto{\pgfqpoint{0.000000in}{-0.048611in}}%
\pgfusepath{stroke,fill}%
}%
\begin{pgfscope}%
\pgfsys@transformshift{0.717889in}{3.664143in}%
\pgfsys@useobject{currentmarker}{}%
\end{pgfscope}%
\end{pgfscope}%
\begin{pgfscope}%
\definecolor{textcolor}{rgb}{0.000000,0.000000,0.000000}%
\pgfsetstrokecolor{textcolor}%
\pgfsetfillcolor{textcolor}%
\pgftext[x=0.717889in,y=3.566921in,,top]{\color{textcolor}\rmfamily\fontsize{10.000000}{12.000000}\selectfont \(\displaystyle {0}\)}%
\end{pgfscope}%
\begin{pgfscope}%
\pgfsetbuttcap%
\pgfsetroundjoin%
\definecolor{currentfill}{rgb}{0.000000,0.000000,0.000000}%
\pgfsetfillcolor{currentfill}%
\pgfsetlinewidth{0.803000pt}%
\definecolor{currentstroke}{rgb}{0.000000,0.000000,0.000000}%
\pgfsetstrokecolor{currentstroke}%
\pgfsetdash{}{0pt}%
\pgfsys@defobject{currentmarker}{\pgfqpoint{0.000000in}{-0.048611in}}{\pgfqpoint{0.000000in}{0.000000in}}{%
\pgfpathmoveto{\pgfqpoint{0.000000in}{0.000000in}}%
\pgfpathlineto{\pgfqpoint{0.000000in}{-0.048611in}}%
\pgfusepath{stroke,fill}%
}%
\begin{pgfscope}%
\pgfsys@transformshift{1.803555in}{3.664143in}%
\pgfsys@useobject{currentmarker}{}%
\end{pgfscope}%
\end{pgfscope}%
\begin{pgfscope}%
\definecolor{textcolor}{rgb}{0.000000,0.000000,0.000000}%
\pgfsetstrokecolor{textcolor}%
\pgfsetfillcolor{textcolor}%
\pgftext[x=1.803555in,y=3.566921in,,top]{\color{textcolor}\rmfamily\fontsize{10.000000}{12.000000}\selectfont \(\displaystyle {20}\)}%
\end{pgfscope}%
\begin{pgfscope}%
\pgfsetbuttcap%
\pgfsetroundjoin%
\definecolor{currentfill}{rgb}{0.000000,0.000000,0.000000}%
\pgfsetfillcolor{currentfill}%
\pgfsetlinewidth{0.803000pt}%
\definecolor{currentstroke}{rgb}{0.000000,0.000000,0.000000}%
\pgfsetstrokecolor{currentstroke}%
\pgfsetdash{}{0pt}%
\pgfsys@defobject{currentmarker}{\pgfqpoint{0.000000in}{-0.048611in}}{\pgfqpoint{0.000000in}{0.000000in}}{%
\pgfpathmoveto{\pgfqpoint{0.000000in}{0.000000in}}%
\pgfpathlineto{\pgfqpoint{0.000000in}{-0.048611in}}%
\pgfusepath{stroke,fill}%
}%
\begin{pgfscope}%
\pgfsys@transformshift{2.889222in}{3.664143in}%
\pgfsys@useobject{currentmarker}{}%
\end{pgfscope}%
\end{pgfscope}%
\begin{pgfscope}%
\definecolor{textcolor}{rgb}{0.000000,0.000000,0.000000}%
\pgfsetstrokecolor{textcolor}%
\pgfsetfillcolor{textcolor}%
\pgftext[x=2.889222in,y=3.566921in,,top]{\color{textcolor}\rmfamily\fontsize{10.000000}{12.000000}\selectfont \(\displaystyle {40}\)}%
\end{pgfscope}%
\begin{pgfscope}%
\pgfsetbuttcap%
\pgfsetroundjoin%
\definecolor{currentfill}{rgb}{0.000000,0.000000,0.000000}%
\pgfsetfillcolor{currentfill}%
\pgfsetlinewidth{0.803000pt}%
\definecolor{currentstroke}{rgb}{0.000000,0.000000,0.000000}%
\pgfsetstrokecolor{currentstroke}%
\pgfsetdash{}{0pt}%
\pgfsys@defobject{currentmarker}{\pgfqpoint{0.000000in}{-0.048611in}}{\pgfqpoint{0.000000in}{0.000000in}}{%
\pgfpathmoveto{\pgfqpoint{0.000000in}{0.000000in}}%
\pgfpathlineto{\pgfqpoint{0.000000in}{-0.048611in}}%
\pgfusepath{stroke,fill}%
}%
\begin{pgfscope}%
\pgfsys@transformshift{3.974889in}{3.664143in}%
\pgfsys@useobject{currentmarker}{}%
\end{pgfscope}%
\end{pgfscope}%
\begin{pgfscope}%
\definecolor{textcolor}{rgb}{0.000000,0.000000,0.000000}%
\pgfsetstrokecolor{textcolor}%
\pgfsetfillcolor{textcolor}%
\pgftext[x=3.974889in,y=3.566921in,,top]{\color{textcolor}\rmfamily\fontsize{10.000000}{12.000000}\selectfont \(\displaystyle {60}\)}%
\end{pgfscope}%
\begin{pgfscope}%
\pgfsetbuttcap%
\pgfsetroundjoin%
\definecolor{currentfill}{rgb}{0.000000,0.000000,0.000000}%
\pgfsetfillcolor{currentfill}%
\pgfsetlinewidth{0.803000pt}%
\definecolor{currentstroke}{rgb}{0.000000,0.000000,0.000000}%
\pgfsetstrokecolor{currentstroke}%
\pgfsetdash{}{0pt}%
\pgfsys@defobject{currentmarker}{\pgfqpoint{0.000000in}{-0.048611in}}{\pgfqpoint{0.000000in}{0.000000in}}{%
\pgfpathmoveto{\pgfqpoint{0.000000in}{0.000000in}}%
\pgfpathlineto{\pgfqpoint{0.000000in}{-0.048611in}}%
\pgfusepath{stroke,fill}%
}%
\begin{pgfscope}%
\pgfsys@transformshift{5.060555in}{3.664143in}%
\pgfsys@useobject{currentmarker}{}%
\end{pgfscope}%
\end{pgfscope}%
\begin{pgfscope}%
\definecolor{textcolor}{rgb}{0.000000,0.000000,0.000000}%
\pgfsetstrokecolor{textcolor}%
\pgfsetfillcolor{textcolor}%
\pgftext[x=5.060555in,y=3.566921in,,top]{\color{textcolor}\rmfamily\fontsize{10.000000}{12.000000}\selectfont \(\displaystyle {80}\)}%
\end{pgfscope}%
\begin{pgfscope}%
\pgfsetbuttcap%
\pgfsetroundjoin%
\definecolor{currentfill}{rgb}{0.000000,0.000000,0.000000}%
\pgfsetfillcolor{currentfill}%
\pgfsetlinewidth{0.803000pt}%
\definecolor{currentstroke}{rgb}{0.000000,0.000000,0.000000}%
\pgfsetstrokecolor{currentstroke}%
\pgfsetdash{}{0pt}%
\pgfsys@defobject{currentmarker}{\pgfqpoint{0.000000in}{-0.048611in}}{\pgfqpoint{0.000000in}{0.000000in}}{%
\pgfpathmoveto{\pgfqpoint{0.000000in}{0.000000in}}%
\pgfpathlineto{\pgfqpoint{0.000000in}{-0.048611in}}%
\pgfusepath{stroke,fill}%
}%
\begin{pgfscope}%
\pgfsys@transformshift{6.146222in}{3.664143in}%
\pgfsys@useobject{currentmarker}{}%
\end{pgfscope}%
\end{pgfscope}%
\begin{pgfscope}%
\definecolor{textcolor}{rgb}{0.000000,0.000000,0.000000}%
\pgfsetstrokecolor{textcolor}%
\pgfsetfillcolor{textcolor}%
\pgftext[x=6.146222in,y=3.566921in,,top]{\color{textcolor}\rmfamily\fontsize{10.000000}{12.000000}\selectfont \(\displaystyle {100}\)}%
\end{pgfscope}%
\begin{pgfscope}%
\definecolor{textcolor}{rgb}{0.000000,0.000000,0.000000}%
\pgfsetstrokecolor{textcolor}%
\pgfsetfillcolor{textcolor}%
\pgftext[x=3.432055in,y=3.387909in,,top]{\color{textcolor}\rmfamily\fontsize{10.000000}{12.000000}\selectfont Frequency (Hz)}%
\end{pgfscope}%
\begin{pgfscope}%
\pgfsetbuttcap%
\pgfsetroundjoin%
\definecolor{currentfill}{rgb}{0.000000,0.000000,0.000000}%
\pgfsetfillcolor{currentfill}%
\pgfsetlinewidth{0.803000pt}%
\definecolor{currentstroke}{rgb}{0.000000,0.000000,0.000000}%
\pgfsetstrokecolor{currentstroke}%
\pgfsetdash{}{0pt}%
\pgfsys@defobject{currentmarker}{\pgfqpoint{-0.048611in}{0.000000in}}{\pgfqpoint{0.000000in}{0.000000in}}{%
\pgfpathmoveto{\pgfqpoint{0.000000in}{0.000000in}}%
\pgfpathlineto{\pgfqpoint{-0.048611in}{0.000000in}}%
\pgfusepath{stroke,fill}%
}%
\begin{pgfscope}%
\pgfsys@transformshift{0.717889in}{3.929497in}%
\pgfsys@useobject{currentmarker}{}%
\end{pgfscope}%
\end{pgfscope}%
\begin{pgfscope}%
\definecolor{textcolor}{rgb}{0.000000,0.000000,0.000000}%
\pgfsetstrokecolor{textcolor}%
\pgfsetfillcolor{textcolor}%
\pgftext[x=0.373752in, y=3.881272in, left, base]{\color{textcolor}\rmfamily\fontsize{10.000000}{12.000000}\selectfont \(\displaystyle {-25}\)}%
\end{pgfscope}%
\begin{pgfscope}%
\pgfsetbuttcap%
\pgfsetroundjoin%
\definecolor{currentfill}{rgb}{0.000000,0.000000,0.000000}%
\pgfsetfillcolor{currentfill}%
\pgfsetlinewidth{0.803000pt}%
\definecolor{currentstroke}{rgb}{0.000000,0.000000,0.000000}%
\pgfsetstrokecolor{currentstroke}%
\pgfsetdash{}{0pt}%
\pgfsys@defobject{currentmarker}{\pgfqpoint{-0.048611in}{0.000000in}}{\pgfqpoint{0.000000in}{0.000000in}}{%
\pgfpathmoveto{\pgfqpoint{0.000000in}{0.000000in}}%
\pgfpathlineto{\pgfqpoint{-0.048611in}{0.000000in}}%
\pgfusepath{stroke,fill}%
}%
\begin{pgfscope}%
\pgfsys@transformshift{0.717889in}{4.415389in}%
\pgfsys@useobject{currentmarker}{}%
\end{pgfscope}%
\end{pgfscope}%
\begin{pgfscope}%
\definecolor{textcolor}{rgb}{0.000000,0.000000,0.000000}%
\pgfsetstrokecolor{textcolor}%
\pgfsetfillcolor{textcolor}%
\pgftext[x=0.551222in, y=4.367164in, left, base]{\color{textcolor}\rmfamily\fontsize{10.000000}{12.000000}\selectfont \(\displaystyle {0}\)}%
\end{pgfscope}%
\begin{pgfscope}%
\definecolor{textcolor}{rgb}{0.000000,0.000000,0.000000}%
\pgfsetstrokecolor{textcolor}%
\pgfsetfillcolor{textcolor}%
\pgftext[x=0.318197in,y=4.057751in,,bottom,rotate=90.000000]{\color{textcolor}\rmfamily\fontsize{10.000000}{12.000000}\selectfont Amplitude (dB)}%
\end{pgfscope}%
\begin{pgfscope}%
\pgfpathrectangle{\pgfqpoint{0.717889in}{3.664143in}}{\pgfqpoint{5.428334in}{0.787215in}}%
\pgfusepath{clip}%
\pgfsetrectcap%
\pgfsetroundjoin%
\pgfsetlinewidth{1.505625pt}%
\definecolor{currentstroke}{rgb}{0.121569,0.466667,0.705882}%
\pgfsetstrokecolor{currentstroke}%
\pgfsetdash{}{0pt}%
\pgfpathmoveto{\pgfqpoint{0.717889in}{4.415389in}}%
\pgfpathlineto{\pgfqpoint{0.935022in}{4.415079in}}%
\pgfpathlineto{\pgfqpoint{1.043589in}{4.415222in}}%
\pgfpathlineto{\pgfqpoint{1.206439in}{4.415003in}}%
\pgfpathlineto{\pgfqpoint{1.315005in}{4.415301in}}%
\pgfpathlineto{\pgfqpoint{1.477855in}{4.414914in}}%
\pgfpathlineto{\pgfqpoint{1.586422in}{4.415388in}}%
\pgfpathlineto{\pgfqpoint{1.749272in}{4.414815in}}%
\pgfpathlineto{\pgfqpoint{1.857839in}{4.415390in}}%
\pgfpathlineto{\pgfqpoint{2.020689in}{4.415026in}}%
\pgfpathlineto{\pgfqpoint{2.074972in}{4.412373in}}%
\pgfpathlineto{\pgfqpoint{2.129255in}{4.402037in}}%
\pgfpathlineto{\pgfqpoint{2.183539in}{4.378160in}}%
\pgfpathlineto{\pgfqpoint{2.237822in}{4.335088in}}%
\pgfpathlineto{\pgfqpoint{2.292105in}{4.267010in}}%
\pgfpathlineto{\pgfqpoint{2.346389in}{4.165732in}}%
\pgfpathlineto{\pgfqpoint{2.400672in}{4.014085in}}%
\pgfpathlineto{\pgfqpoint{2.454955in}{3.774443in}}%
\pgfpathlineto{\pgfqpoint{2.509239in}{3.720850in}}%
\pgfpathlineto{\pgfqpoint{2.563522in}{3.975673in}}%
\pgfpathlineto{\pgfqpoint{2.617805in}{4.140485in}}%
\pgfpathlineto{\pgfqpoint{2.672089in}{4.249641in}}%
\pgfpathlineto{\pgfqpoint{2.726372in}{4.323525in}}%
\pgfpathlineto{\pgfqpoint{2.780655in}{4.371221in}}%
\pgfpathlineto{\pgfqpoint{2.834939in}{4.398620in}}%
\pgfpathlineto{\pgfqpoint{2.889222in}{4.411213in}}%
\pgfpathlineto{\pgfqpoint{2.943505in}{4.414836in}}%
\pgfpathlineto{\pgfqpoint{3.540622in}{4.414911in}}%
\pgfpathlineto{\pgfqpoint{3.594905in}{4.414921in}}%
\pgfpathlineto{\pgfqpoint{3.649189in}{4.412858in}}%
\pgfpathlineto{\pgfqpoint{3.703472in}{4.403725in}}%
\pgfpathlineto{\pgfqpoint{3.757755in}{4.381491in}}%
\pgfpathlineto{\pgfqpoint{3.812039in}{4.340384in}}%
\pgfpathlineto{\pgfqpoint{3.866322in}{4.274766in}}%
\pgfpathlineto{\pgfqpoint{3.920605in}{4.177269in}}%
\pgfpathlineto{\pgfqpoint{3.974889in}{4.033241in}}%
\pgfpathlineto{\pgfqpoint{4.029172in}{3.809595in}}%
\pgfpathlineto{\pgfqpoint{4.083455in}{3.699926in}}%
\pgfpathlineto{\pgfqpoint{4.137739in}{3.952179in}}%
\pgfpathlineto{\pgfqpoint{4.192022in}{4.126022in}}%
\pgfpathlineto{\pgfqpoint{4.246305in}{4.240602in}}%
\pgfpathlineto{\pgfqpoint{4.300589in}{4.317819in}}%
\pgfpathlineto{\pgfqpoint{4.354872in}{4.367633in}}%
\pgfpathlineto{\pgfqpoint{4.409155in}{4.396484in}}%
\pgfpathlineto{\pgfqpoint{4.463439in}{4.410195in}}%
\pgfpathlineto{\pgfqpoint{4.517722in}{4.414675in}}%
\pgfpathlineto{\pgfqpoint{4.680572in}{4.415260in}}%
\pgfpathlineto{\pgfqpoint{6.156222in}{4.415069in}}%
\pgfpathlineto{\pgfqpoint{6.156222in}{4.415069in}}%
\pgfusepath{stroke}%
\end{pgfscope}%
\begin{pgfscope}%
\pgfsetrectcap%
\pgfsetmiterjoin%
\pgfsetlinewidth{0.803000pt}%
\definecolor{currentstroke}{rgb}{0.000000,0.000000,0.000000}%
\pgfsetstrokecolor{currentstroke}%
\pgfsetdash{}{0pt}%
\pgfpathmoveto{\pgfqpoint{0.717889in}{3.664143in}}%
\pgfpathlineto{\pgfqpoint{0.717889in}{4.451359in}}%
\pgfusepath{stroke}%
\end{pgfscope}%
\begin{pgfscope}%
\pgfsetrectcap%
\pgfsetmiterjoin%
\pgfsetlinewidth{0.803000pt}%
\definecolor{currentstroke}{rgb}{0.000000,0.000000,0.000000}%
\pgfsetstrokecolor{currentstroke}%
\pgfsetdash{}{0pt}%
\pgfpathmoveto{\pgfqpoint{6.146222in}{3.664143in}}%
\pgfpathlineto{\pgfqpoint{6.146222in}{4.451359in}}%
\pgfusepath{stroke}%
\end{pgfscope}%
\begin{pgfscope}%
\pgfsetrectcap%
\pgfsetmiterjoin%
\pgfsetlinewidth{0.803000pt}%
\definecolor{currentstroke}{rgb}{0.000000,0.000000,0.000000}%
\pgfsetstrokecolor{currentstroke}%
\pgfsetdash{}{0pt}%
\pgfpathmoveto{\pgfqpoint{0.717889in}{3.664143in}}%
\pgfpathlineto{\pgfqpoint{6.146222in}{3.664143in}}%
\pgfusepath{stroke}%
\end{pgfscope}%
\begin{pgfscope}%
\pgfsetrectcap%
\pgfsetmiterjoin%
\pgfsetlinewidth{0.803000pt}%
\definecolor{currentstroke}{rgb}{0.000000,0.000000,0.000000}%
\pgfsetstrokecolor{currentstroke}%
\pgfsetdash{}{0pt}%
\pgfpathmoveto{\pgfqpoint{0.717889in}{4.451359in}}%
\pgfpathlineto{\pgfqpoint{6.146222in}{4.451359in}}%
\pgfusepath{stroke}%
\end{pgfscope}%
\begin{pgfscope}%
\definecolor{textcolor}{rgb}{0.000000,0.000000,0.000000}%
\pgfsetstrokecolor{textcolor}%
\pgfsetfillcolor{textcolor}%
\pgftext[x=3.432055in,y=4.534692in,,base]{\color{textcolor}\rmfamily\fontsize{12.000000}{14.400000}\selectfont Filter Frequency Response}%
\end{pgfscope}%
\begin{pgfscope}%
\pgfsetbuttcap%
\pgfsetmiterjoin%
\definecolor{currentfill}{rgb}{1.000000,1.000000,1.000000}%
\pgfsetfillcolor{currentfill}%
\pgfsetlinewidth{0.000000pt}%
\definecolor{currentstroke}{rgb}{0.000000,0.000000,0.000000}%
\pgfsetstrokecolor{currentstroke}%
\pgfsetstrokeopacity{0.000000}%
\pgfsetdash{}{0pt}%
\pgfpathmoveto{\pgfqpoint{0.717889in}{2.114143in}}%
\pgfpathlineto{\pgfqpoint{6.146222in}{2.114143in}}%
\pgfpathlineto{\pgfqpoint{6.146222in}{2.901359in}}%
\pgfpathlineto{\pgfqpoint{0.717889in}{2.901359in}}%
\pgfpathclose%
\pgfusepath{fill}%
\end{pgfscope}%
\begin{pgfscope}%
\pgfsetbuttcap%
\pgfsetroundjoin%
\definecolor{currentfill}{rgb}{0.000000,0.000000,0.000000}%
\pgfsetfillcolor{currentfill}%
\pgfsetlinewidth{0.803000pt}%
\definecolor{currentstroke}{rgb}{0.000000,0.000000,0.000000}%
\pgfsetstrokecolor{currentstroke}%
\pgfsetdash{}{0pt}%
\pgfsys@defobject{currentmarker}{\pgfqpoint{0.000000in}{-0.048611in}}{\pgfqpoint{0.000000in}{0.000000in}}{%
\pgfpathmoveto{\pgfqpoint{0.000000in}{0.000000in}}%
\pgfpathlineto{\pgfqpoint{0.000000in}{-0.048611in}}%
\pgfusepath{stroke,fill}%
}%
\begin{pgfscope}%
\pgfsys@transformshift{0.964631in}{2.114143in}%
\pgfsys@useobject{currentmarker}{}%
\end{pgfscope}%
\end{pgfscope}%
\begin{pgfscope}%
\definecolor{textcolor}{rgb}{0.000000,0.000000,0.000000}%
\pgfsetstrokecolor{textcolor}%
\pgfsetfillcolor{textcolor}%
\pgftext[x=0.964631in,y=2.016921in,,top]{\color{textcolor}\rmfamily\fontsize{10.000000}{12.000000}\selectfont \(\displaystyle {0}\)}%
\end{pgfscope}%
\begin{pgfscope}%
\pgfsetbuttcap%
\pgfsetroundjoin%
\definecolor{currentfill}{rgb}{0.000000,0.000000,0.000000}%
\pgfsetfillcolor{currentfill}%
\pgfsetlinewidth{0.803000pt}%
\definecolor{currentstroke}{rgb}{0.000000,0.000000,0.000000}%
\pgfsetstrokecolor{currentstroke}%
\pgfsetdash{}{0pt}%
\pgfsys@defobject{currentmarker}{\pgfqpoint{0.000000in}{-0.048611in}}{\pgfqpoint{0.000000in}{0.000000in}}{%
\pgfpathmoveto{\pgfqpoint{0.000000in}{0.000000in}}%
\pgfpathlineto{\pgfqpoint{0.000000in}{-0.048611in}}%
\pgfusepath{stroke,fill}%
}%
\begin{pgfscope}%
\pgfsys@transformshift{1.975308in}{2.114143in}%
\pgfsys@useobject{currentmarker}{}%
\end{pgfscope}%
\end{pgfscope}%
\begin{pgfscope}%
\definecolor{textcolor}{rgb}{0.000000,0.000000,0.000000}%
\pgfsetstrokecolor{textcolor}%
\pgfsetfillcolor{textcolor}%
\pgftext[x=1.975308in,y=2.016921in,,top]{\color{textcolor}\rmfamily\fontsize{10.000000}{12.000000}\selectfont \(\displaystyle {10}\)}%
\end{pgfscope}%
\begin{pgfscope}%
\pgfsetbuttcap%
\pgfsetroundjoin%
\definecolor{currentfill}{rgb}{0.000000,0.000000,0.000000}%
\pgfsetfillcolor{currentfill}%
\pgfsetlinewidth{0.803000pt}%
\definecolor{currentstroke}{rgb}{0.000000,0.000000,0.000000}%
\pgfsetstrokecolor{currentstroke}%
\pgfsetdash{}{0pt}%
\pgfsys@defobject{currentmarker}{\pgfqpoint{0.000000in}{-0.048611in}}{\pgfqpoint{0.000000in}{0.000000in}}{%
\pgfpathmoveto{\pgfqpoint{0.000000in}{0.000000in}}%
\pgfpathlineto{\pgfqpoint{0.000000in}{-0.048611in}}%
\pgfusepath{stroke,fill}%
}%
\begin{pgfscope}%
\pgfsys@transformshift{2.985986in}{2.114143in}%
\pgfsys@useobject{currentmarker}{}%
\end{pgfscope}%
\end{pgfscope}%
\begin{pgfscope}%
\definecolor{textcolor}{rgb}{0.000000,0.000000,0.000000}%
\pgfsetstrokecolor{textcolor}%
\pgfsetfillcolor{textcolor}%
\pgftext[x=2.985986in,y=2.016921in,,top]{\color{textcolor}\rmfamily\fontsize{10.000000}{12.000000}\selectfont \(\displaystyle {20}\)}%
\end{pgfscope}%
\begin{pgfscope}%
\pgfsetbuttcap%
\pgfsetroundjoin%
\definecolor{currentfill}{rgb}{0.000000,0.000000,0.000000}%
\pgfsetfillcolor{currentfill}%
\pgfsetlinewidth{0.803000pt}%
\definecolor{currentstroke}{rgb}{0.000000,0.000000,0.000000}%
\pgfsetstrokecolor{currentstroke}%
\pgfsetdash{}{0pt}%
\pgfsys@defobject{currentmarker}{\pgfqpoint{0.000000in}{-0.048611in}}{\pgfqpoint{0.000000in}{0.000000in}}{%
\pgfpathmoveto{\pgfqpoint{0.000000in}{0.000000in}}%
\pgfpathlineto{\pgfqpoint{0.000000in}{-0.048611in}}%
\pgfusepath{stroke,fill}%
}%
\begin{pgfscope}%
\pgfsys@transformshift{3.996663in}{2.114143in}%
\pgfsys@useobject{currentmarker}{}%
\end{pgfscope}%
\end{pgfscope}%
\begin{pgfscope}%
\definecolor{textcolor}{rgb}{0.000000,0.000000,0.000000}%
\pgfsetstrokecolor{textcolor}%
\pgfsetfillcolor{textcolor}%
\pgftext[x=3.996663in,y=2.016921in,,top]{\color{textcolor}\rmfamily\fontsize{10.000000}{12.000000}\selectfont \(\displaystyle {30}\)}%
\end{pgfscope}%
\begin{pgfscope}%
\pgfsetbuttcap%
\pgfsetroundjoin%
\definecolor{currentfill}{rgb}{0.000000,0.000000,0.000000}%
\pgfsetfillcolor{currentfill}%
\pgfsetlinewidth{0.803000pt}%
\definecolor{currentstroke}{rgb}{0.000000,0.000000,0.000000}%
\pgfsetstrokecolor{currentstroke}%
\pgfsetdash{}{0pt}%
\pgfsys@defobject{currentmarker}{\pgfqpoint{0.000000in}{-0.048611in}}{\pgfqpoint{0.000000in}{0.000000in}}{%
\pgfpathmoveto{\pgfqpoint{0.000000in}{0.000000in}}%
\pgfpathlineto{\pgfqpoint{0.000000in}{-0.048611in}}%
\pgfusepath{stroke,fill}%
}%
\begin{pgfscope}%
\pgfsys@transformshift{5.007340in}{2.114143in}%
\pgfsys@useobject{currentmarker}{}%
\end{pgfscope}%
\end{pgfscope}%
\begin{pgfscope}%
\definecolor{textcolor}{rgb}{0.000000,0.000000,0.000000}%
\pgfsetstrokecolor{textcolor}%
\pgfsetfillcolor{textcolor}%
\pgftext[x=5.007340in,y=2.016921in,,top]{\color{textcolor}\rmfamily\fontsize{10.000000}{12.000000}\selectfont \(\displaystyle {40}\)}%
\end{pgfscope}%
\begin{pgfscope}%
\pgfsetbuttcap%
\pgfsetroundjoin%
\definecolor{currentfill}{rgb}{0.000000,0.000000,0.000000}%
\pgfsetfillcolor{currentfill}%
\pgfsetlinewidth{0.803000pt}%
\definecolor{currentstroke}{rgb}{0.000000,0.000000,0.000000}%
\pgfsetstrokecolor{currentstroke}%
\pgfsetdash{}{0pt}%
\pgfsys@defobject{currentmarker}{\pgfqpoint{0.000000in}{-0.048611in}}{\pgfqpoint{0.000000in}{0.000000in}}{%
\pgfpathmoveto{\pgfqpoint{0.000000in}{0.000000in}}%
\pgfpathlineto{\pgfqpoint{0.000000in}{-0.048611in}}%
\pgfusepath{stroke,fill}%
}%
\begin{pgfscope}%
\pgfsys@transformshift{6.018017in}{2.114143in}%
\pgfsys@useobject{currentmarker}{}%
\end{pgfscope}%
\end{pgfscope}%
\begin{pgfscope}%
\definecolor{textcolor}{rgb}{0.000000,0.000000,0.000000}%
\pgfsetstrokecolor{textcolor}%
\pgfsetfillcolor{textcolor}%
\pgftext[x=6.018017in,y=2.016921in,,top]{\color{textcolor}\rmfamily\fontsize{10.000000}{12.000000}\selectfont \(\displaystyle {50}\)}%
\end{pgfscope}%
\begin{pgfscope}%
\definecolor{textcolor}{rgb}{0.000000,0.000000,0.000000}%
\pgfsetstrokecolor{textcolor}%
\pgfsetfillcolor{textcolor}%
\pgftext[x=3.432055in,y=1.837909in,,top]{\color{textcolor}\rmfamily\fontsize{10.000000}{12.000000}\selectfont Time (s)}%
\end{pgfscope}%
\begin{pgfscope}%
\pgfsetbuttcap%
\pgfsetroundjoin%
\definecolor{currentfill}{rgb}{0.000000,0.000000,0.000000}%
\pgfsetfillcolor{currentfill}%
\pgfsetlinewidth{0.803000pt}%
\definecolor{currentstroke}{rgb}{0.000000,0.000000,0.000000}%
\pgfsetstrokecolor{currentstroke}%
\pgfsetdash{}{0pt}%
\pgfsys@defobject{currentmarker}{\pgfqpoint{-0.048611in}{0.000000in}}{\pgfqpoint{0.000000in}{0.000000in}}{%
\pgfpathmoveto{\pgfqpoint{0.000000in}{0.000000in}}%
\pgfpathlineto{\pgfqpoint{-0.048611in}{0.000000in}}%
\pgfusepath{stroke,fill}%
}%
\begin{pgfscope}%
\pgfsys@transformshift{0.717889in}{2.454909in}%
\pgfsys@useobject{currentmarker}{}%
\end{pgfscope}%
\end{pgfscope}%
\begin{pgfscope}%
\definecolor{textcolor}{rgb}{0.000000,0.000000,0.000000}%
\pgfsetstrokecolor{textcolor}%
\pgfsetfillcolor{textcolor}%
\pgftext[x=0.551222in, y=2.406684in, left, base]{\color{textcolor}\rmfamily\fontsize{10.000000}{12.000000}\selectfont \(\displaystyle {0}\)}%
\end{pgfscope}%
\begin{pgfscope}%
\pgfsetbuttcap%
\pgfsetroundjoin%
\definecolor{currentfill}{rgb}{0.000000,0.000000,0.000000}%
\pgfsetfillcolor{currentfill}%
\pgfsetlinewidth{0.803000pt}%
\definecolor{currentstroke}{rgb}{0.000000,0.000000,0.000000}%
\pgfsetstrokecolor{currentstroke}%
\pgfsetdash{}{0pt}%
\pgfsys@defobject{currentmarker}{\pgfqpoint{-0.048611in}{0.000000in}}{\pgfqpoint{0.000000in}{0.000000in}}{%
\pgfpathmoveto{\pgfqpoint{0.000000in}{0.000000in}}%
\pgfpathlineto{\pgfqpoint{-0.048611in}{0.000000in}}%
\pgfusepath{stroke,fill}%
}%
\begin{pgfscope}%
\pgfsys@transformshift{0.717889in}{2.807082in}%
\pgfsys@useobject{currentmarker}{}%
\end{pgfscope}%
\end{pgfscope}%
\begin{pgfscope}%
\definecolor{textcolor}{rgb}{0.000000,0.000000,0.000000}%
\pgfsetstrokecolor{textcolor}%
\pgfsetfillcolor{textcolor}%
\pgftext[x=0.342888in, y=2.758857in, left, base]{\color{textcolor}\rmfamily\fontsize{10.000000}{12.000000}\selectfont \(\displaystyle {1000}\)}%
\end{pgfscope}%
\begin{pgfscope}%
\definecolor{textcolor}{rgb}{0.000000,0.000000,0.000000}%
\pgfsetstrokecolor{textcolor}%
\pgfsetfillcolor{textcolor}%
\pgftext[x=0.287332in,y=2.507751in,,bottom,rotate=90.000000]{\color{textcolor}\rmfamily\fontsize{10.000000}{12.000000}\selectfont ECG Voltage (\(\displaystyle \mu V\))}%
\end{pgfscope}%
\begin{pgfscope}%
\pgfpathrectangle{\pgfqpoint{0.717889in}{2.114143in}}{\pgfqpoint{5.428334in}{0.787215in}}%
\pgfusepath{clip}%
\pgfsetrectcap%
\pgfsetroundjoin%
\pgfsetlinewidth{1.505625pt}%
\definecolor{currentstroke}{rgb}{0.121569,0.466667,0.705882}%
\pgfsetstrokecolor{currentstroke}%
\pgfsetdash{}{0pt}%
\pgfpathmoveto{\pgfqpoint{0.964631in}{2.454929in}}%
\pgfpathlineto{\pgfqpoint{0.967000in}{2.454624in}}%
\pgfpathlineto{\pgfqpoint{0.967395in}{2.454210in}}%
\pgfpathlineto{\pgfqpoint{0.967888in}{2.455264in}}%
\pgfpathlineto{\pgfqpoint{0.968184in}{2.455675in}}%
\pgfpathlineto{\pgfqpoint{0.968678in}{2.454661in}}%
\pgfpathlineto{\pgfqpoint{0.968974in}{2.454270in}}%
\pgfpathlineto{\pgfqpoint{0.969467in}{2.455630in}}%
\pgfpathlineto{\pgfqpoint{0.969763in}{2.456127in}}%
\pgfpathlineto{\pgfqpoint{0.970158in}{2.454803in}}%
\pgfpathlineto{\pgfqpoint{0.970553in}{2.453471in}}%
\pgfpathlineto{\pgfqpoint{0.971046in}{2.455208in}}%
\pgfpathlineto{\pgfqpoint{0.971441in}{2.456210in}}%
\pgfpathlineto{\pgfqpoint{0.971935in}{2.454225in}}%
\pgfpathlineto{\pgfqpoint{0.972231in}{2.453623in}}%
\pgfpathlineto{\pgfqpoint{0.972724in}{2.455189in}}%
\pgfpathlineto{\pgfqpoint{0.974007in}{2.455987in}}%
\pgfpathlineto{\pgfqpoint{0.974106in}{2.455885in}}%
\pgfpathlineto{\pgfqpoint{0.974896in}{2.452272in}}%
\pgfpathlineto{\pgfqpoint{0.975389in}{2.455407in}}%
\pgfpathlineto{\pgfqpoint{0.975587in}{2.456288in}}%
\pgfpathlineto{\pgfqpoint{0.975981in}{2.453639in}}%
\pgfpathlineto{\pgfqpoint{0.976278in}{2.451363in}}%
\pgfpathlineto{\pgfqpoint{0.976672in}{2.456915in}}%
\pgfpathlineto{\pgfqpoint{0.977166in}{2.465934in}}%
\pgfpathlineto{\pgfqpoint{0.977561in}{2.453913in}}%
\pgfpathlineto{\pgfqpoint{0.978054in}{2.439059in}}%
\pgfpathlineto{\pgfqpoint{0.978548in}{2.459783in}}%
\pgfpathlineto{\pgfqpoint{0.978844in}{2.468006in}}%
\pgfpathlineto{\pgfqpoint{0.979337in}{2.446847in}}%
\pgfpathlineto{\pgfqpoint{0.979535in}{2.439840in}}%
\pgfpathlineto{\pgfqpoint{0.980028in}{2.464669in}}%
\pgfpathlineto{\pgfqpoint{0.980423in}{2.484843in}}%
\pgfpathlineto{\pgfqpoint{0.980818in}{2.451689in}}%
\pgfpathlineto{\pgfqpoint{0.981212in}{2.414449in}}%
\pgfpathlineto{\pgfqpoint{0.981706in}{2.457163in}}%
\pgfpathlineto{\pgfqpoint{0.982101in}{2.491637in}}%
\pgfpathlineto{\pgfqpoint{0.982594in}{2.439718in}}%
\pgfpathlineto{\pgfqpoint{0.982890in}{2.419437in}}%
\pgfpathlineto{\pgfqpoint{0.983285in}{2.465884in}}%
\pgfpathlineto{\pgfqpoint{0.984470in}{2.579042in}}%
\pgfpathlineto{\pgfqpoint{0.984173in}{2.428562in}}%
\pgfpathlineto{\pgfqpoint{0.984766in}{2.546128in}}%
\pgfpathlineto{\pgfqpoint{0.985358in}{2.475488in}}%
\pgfpathlineto{\pgfqpoint{0.985753in}{2.539599in}}%
\pgfpathlineto{\pgfqpoint{0.986049in}{2.561838in}}%
\pgfpathlineto{\pgfqpoint{0.986641in}{2.507165in}}%
\pgfpathlineto{\pgfqpoint{0.986937in}{2.481013in}}%
\pgfpathlineto{\pgfqpoint{0.987628in}{2.521771in}}%
\pgfpathlineto{\pgfqpoint{0.987727in}{2.524248in}}%
\pgfpathlineto{\pgfqpoint{0.988023in}{2.514105in}}%
\pgfpathlineto{\pgfqpoint{0.988615in}{2.480755in}}%
\pgfpathlineto{\pgfqpoint{0.989207in}{2.503443in}}%
\pgfpathlineto{\pgfqpoint{0.989404in}{2.507115in}}%
\pgfpathlineto{\pgfqpoint{0.989799in}{2.489566in}}%
\pgfpathlineto{\pgfqpoint{0.990490in}{2.478395in}}%
\pgfpathlineto{\pgfqpoint{0.990885in}{2.487224in}}%
\pgfpathlineto{\pgfqpoint{0.990984in}{2.487608in}}%
\pgfpathlineto{\pgfqpoint{0.991280in}{2.484455in}}%
\pgfpathlineto{\pgfqpoint{0.996116in}{2.404182in}}%
\pgfpathlineto{\pgfqpoint{0.996412in}{2.419006in}}%
\pgfpathlineto{\pgfqpoint{0.996610in}{2.423731in}}%
\pgfpathlineto{\pgfqpoint{0.997103in}{2.409013in}}%
\pgfpathlineto{\pgfqpoint{0.997399in}{2.402000in}}%
\pgfpathlineto{\pgfqpoint{0.997794in}{2.411926in}}%
\pgfpathlineto{\pgfqpoint{0.998287in}{2.403839in}}%
\pgfpathlineto{\pgfqpoint{0.999768in}{2.417048in}}%
\pgfpathlineto{\pgfqpoint{0.999965in}{2.416534in}}%
\pgfpathlineto{\pgfqpoint{1.000261in}{2.418274in}}%
\pgfpathlineto{\pgfqpoint{1.000755in}{2.433043in}}%
\pgfpathlineto{\pgfqpoint{1.001841in}{2.431785in}}%
\pgfpathlineto{\pgfqpoint{1.001939in}{2.431716in}}%
\pgfpathlineto{\pgfqpoint{1.002334in}{2.420339in}}%
\pgfpathlineto{\pgfqpoint{1.003222in}{2.425436in}}%
\pgfpathlineto{\pgfqpoint{1.006973in}{2.475660in}}%
\pgfpathlineto{\pgfqpoint{1.003617in}{2.422144in}}%
\pgfpathlineto{\pgfqpoint{1.007368in}{2.472751in}}%
\pgfpathlineto{\pgfqpoint{1.007466in}{2.473002in}}%
\pgfpathlineto{\pgfqpoint{1.007565in}{2.472126in}}%
\pgfpathlineto{\pgfqpoint{1.012105in}{2.384082in}}%
\pgfpathlineto{\pgfqpoint{1.012204in}{2.384567in}}%
\pgfpathlineto{\pgfqpoint{1.013487in}{2.438506in}}%
\pgfpathlineto{\pgfqpoint{1.013783in}{2.461437in}}%
\pgfpathlineto{\pgfqpoint{1.014277in}{2.409771in}}%
\pgfpathlineto{\pgfqpoint{1.014968in}{2.389827in}}%
\pgfpathlineto{\pgfqpoint{1.015560in}{2.391419in}}%
\pgfpathlineto{\pgfqpoint{1.016053in}{2.407567in}}%
\pgfpathlineto{\pgfqpoint{1.017534in}{2.574680in}}%
\pgfpathlineto{\pgfqpoint{1.018126in}{2.670860in}}%
\pgfpathlineto{\pgfqpoint{1.018817in}{2.641699in}}%
\pgfpathlineto{\pgfqpoint{1.019705in}{2.494395in}}%
\pgfpathlineto{\pgfqpoint{1.020199in}{2.390214in}}%
\pgfpathlineto{\pgfqpoint{1.020988in}{2.430396in}}%
\pgfpathlineto{\pgfqpoint{1.021284in}{2.420844in}}%
\pgfpathlineto{\pgfqpoint{1.021778in}{2.441622in}}%
\pgfpathlineto{\pgfqpoint{1.023357in}{2.493909in}}%
\pgfpathlineto{\pgfqpoint{1.023752in}{2.471481in}}%
\pgfpathlineto{\pgfqpoint{1.024541in}{2.431455in}}%
\pgfpathlineto{\pgfqpoint{1.025035in}{2.449131in}}%
\pgfpathlineto{\pgfqpoint{1.026318in}{2.476709in}}%
\pgfpathlineto{\pgfqpoint{1.027009in}{2.462916in}}%
\pgfpathlineto{\pgfqpoint{1.027996in}{2.435106in}}%
\pgfpathlineto{\pgfqpoint{1.028292in}{2.449327in}}%
\pgfpathlineto{\pgfqpoint{1.028489in}{2.454102in}}%
\pgfpathlineto{\pgfqpoint{1.029378in}{2.449496in}}%
\pgfpathlineto{\pgfqpoint{1.029871in}{2.463790in}}%
\pgfpathlineto{\pgfqpoint{1.030562in}{2.451366in}}%
\pgfpathlineto{\pgfqpoint{1.030661in}{2.450622in}}%
\pgfpathlineto{\pgfqpoint{1.030858in}{2.454488in}}%
\pgfpathlineto{\pgfqpoint{1.031845in}{2.471752in}}%
\pgfpathlineto{\pgfqpoint{1.032141in}{2.463726in}}%
\pgfpathlineto{\pgfqpoint{1.032437in}{2.458774in}}%
\pgfpathlineto{\pgfqpoint{1.032931in}{2.470268in}}%
\pgfpathlineto{\pgfqpoint{1.033029in}{2.470183in}}%
\pgfpathlineto{\pgfqpoint{1.033424in}{2.460562in}}%
\pgfpathlineto{\pgfqpoint{1.033720in}{2.475548in}}%
\pgfpathlineto{\pgfqpoint{1.033819in}{2.478742in}}%
\pgfpathlineto{\pgfqpoint{1.034411in}{2.465253in}}%
\pgfpathlineto{\pgfqpoint{1.034707in}{2.459123in}}%
\pgfpathlineto{\pgfqpoint{1.035102in}{2.471361in}}%
\pgfpathlineto{\pgfqpoint{1.035201in}{2.474337in}}%
\pgfpathlineto{\pgfqpoint{1.035596in}{2.467203in}}%
\pgfpathlineto{\pgfqpoint{1.036188in}{2.471485in}}%
\pgfpathlineto{\pgfqpoint{1.037471in}{2.437387in}}%
\pgfpathlineto{\pgfqpoint{1.039050in}{2.374420in}}%
\pgfpathlineto{\pgfqpoint{1.039247in}{2.378054in}}%
\pgfpathlineto{\pgfqpoint{1.041814in}{2.458336in}}%
\pgfpathlineto{\pgfqpoint{1.041912in}{2.457807in}}%
\pgfpathlineto{\pgfqpoint{1.044380in}{2.401894in}}%
\pgfpathlineto{\pgfqpoint{1.044479in}{2.403071in}}%
\pgfpathlineto{\pgfqpoint{1.044676in}{2.406426in}}%
\pgfpathlineto{\pgfqpoint{1.045169in}{2.395970in}}%
\pgfpathlineto{\pgfqpoint{1.046156in}{2.389143in}}%
\pgfpathlineto{\pgfqpoint{1.045663in}{2.401026in}}%
\pgfpathlineto{\pgfqpoint{1.046354in}{2.392696in}}%
\pgfpathlineto{\pgfqpoint{1.046551in}{2.394875in}}%
\pgfpathlineto{\pgfqpoint{1.047143in}{2.389928in}}%
\pgfpathlineto{\pgfqpoint{1.048130in}{2.377263in}}%
\pgfpathlineto{\pgfqpoint{1.048328in}{2.381394in}}%
\pgfpathlineto{\pgfqpoint{1.049611in}{2.393951in}}%
\pgfpathlineto{\pgfqpoint{1.049710in}{2.391884in}}%
\pgfpathlineto{\pgfqpoint{1.050400in}{2.382127in}}%
\pgfpathlineto{\pgfqpoint{1.051091in}{2.382411in}}%
\pgfpathlineto{\pgfqpoint{1.051190in}{2.381247in}}%
\pgfpathlineto{\pgfqpoint{1.051585in}{2.388481in}}%
\pgfpathlineto{\pgfqpoint{1.051684in}{2.389308in}}%
\pgfpathlineto{\pgfqpoint{1.051980in}{2.383288in}}%
\pgfpathlineto{\pgfqpoint{1.052276in}{2.376489in}}%
\pgfpathlineto{\pgfqpoint{1.052671in}{2.384045in}}%
\pgfpathlineto{\pgfqpoint{1.053065in}{2.382479in}}%
\pgfpathlineto{\pgfqpoint{1.056322in}{2.452077in}}%
\pgfpathlineto{\pgfqpoint{1.056520in}{2.447230in}}%
\pgfpathlineto{\pgfqpoint{1.057112in}{2.440305in}}%
\pgfpathlineto{\pgfqpoint{1.057507in}{2.450039in}}%
\pgfpathlineto{\pgfqpoint{1.057803in}{2.440475in}}%
\pgfpathlineto{\pgfqpoint{1.058889in}{2.414044in}}%
\pgfpathlineto{\pgfqpoint{1.059086in}{2.420136in}}%
\pgfpathlineto{\pgfqpoint{1.059283in}{2.429364in}}%
\pgfpathlineto{\pgfqpoint{1.059777in}{2.417211in}}%
\pgfpathlineto{\pgfqpoint{1.060073in}{2.418559in}}%
\pgfpathlineto{\pgfqpoint{1.061652in}{2.387011in}}%
\pgfpathlineto{\pgfqpoint{1.061850in}{2.391817in}}%
\pgfpathlineto{\pgfqpoint{1.062738in}{2.452709in}}%
\pgfpathlineto{\pgfqpoint{1.063330in}{2.414375in}}%
\pgfpathlineto{\pgfqpoint{1.064909in}{2.369449in}}%
\pgfpathlineto{\pgfqpoint{1.065107in}{2.378962in}}%
\pgfpathlineto{\pgfqpoint{1.066390in}{2.488754in}}%
\pgfpathlineto{\pgfqpoint{1.067377in}{2.643315in}}%
\pgfpathlineto{\pgfqpoint{1.067969in}{2.613061in}}%
\pgfpathlineto{\pgfqpoint{1.068758in}{2.514427in}}%
\pgfpathlineto{\pgfqpoint{1.069252in}{2.372026in}}%
\pgfpathlineto{\pgfqpoint{1.070140in}{2.397881in}}%
\pgfpathlineto{\pgfqpoint{1.070338in}{2.394158in}}%
\pgfpathlineto{\pgfqpoint{1.070831in}{2.405690in}}%
\pgfpathlineto{\pgfqpoint{1.072608in}{2.457006in}}%
\pgfpathlineto{\pgfqpoint{1.072706in}{2.456996in}}%
\pgfpathlineto{\pgfqpoint{1.073299in}{2.398221in}}%
\pgfpathlineto{\pgfqpoint{1.073397in}{2.393012in}}%
\pgfpathlineto{\pgfqpoint{1.073891in}{2.410576in}}%
\pgfpathlineto{\pgfqpoint{1.074088in}{2.409031in}}%
\pgfpathlineto{\pgfqpoint{1.075470in}{2.441085in}}%
\pgfpathlineto{\pgfqpoint{1.076161in}{2.427868in}}%
\pgfpathlineto{\pgfqpoint{1.076556in}{2.408700in}}%
\pgfpathlineto{\pgfqpoint{1.077345in}{2.418219in}}%
\pgfpathlineto{\pgfqpoint{1.078234in}{2.438411in}}%
\pgfpathlineto{\pgfqpoint{1.078727in}{2.433710in}}%
\pgfpathlineto{\pgfqpoint{1.079122in}{2.447557in}}%
\pgfpathlineto{\pgfqpoint{1.079517in}{2.433085in}}%
\pgfpathlineto{\pgfqpoint{1.079813in}{2.433566in}}%
\pgfpathlineto{\pgfqpoint{1.080109in}{2.428126in}}%
\pgfpathlineto{\pgfqpoint{1.080405in}{2.438838in}}%
\pgfpathlineto{\pgfqpoint{1.081589in}{2.450233in}}%
\pgfpathlineto{\pgfqpoint{1.081195in}{2.438308in}}%
\pgfpathlineto{\pgfqpoint{1.081688in}{2.447596in}}%
\pgfpathlineto{\pgfqpoint{1.081984in}{2.433549in}}%
\pgfpathlineto{\pgfqpoint{1.082379in}{2.459817in}}%
\pgfpathlineto{\pgfqpoint{1.082774in}{2.444190in}}%
\pgfpathlineto{\pgfqpoint{1.082872in}{2.444465in}}%
\pgfpathlineto{\pgfqpoint{1.083267in}{2.458863in}}%
\pgfpathlineto{\pgfqpoint{1.083761in}{2.439958in}}%
\pgfpathlineto{\pgfqpoint{1.083859in}{2.438603in}}%
\pgfpathlineto{\pgfqpoint{1.084057in}{2.447771in}}%
\pgfpathlineto{\pgfqpoint{1.085142in}{2.470275in}}%
\pgfpathlineto{\pgfqpoint{1.085439in}{2.468758in}}%
\pgfpathlineto{\pgfqpoint{1.090275in}{2.553412in}}%
\pgfpathlineto{\pgfqpoint{1.092150in}{2.531607in}}%
\pgfpathlineto{\pgfqpoint{1.091064in}{2.559656in}}%
\pgfpathlineto{\pgfqpoint{1.092249in}{2.531707in}}%
\pgfpathlineto{\pgfqpoint{1.092446in}{2.535188in}}%
\pgfpathlineto{\pgfqpoint{1.092841in}{2.520335in}}%
\pgfpathlineto{\pgfqpoint{1.093828in}{2.500798in}}%
\pgfpathlineto{\pgfqpoint{1.095210in}{2.465557in}}%
\pgfpathlineto{\pgfqpoint{1.095407in}{2.468508in}}%
\pgfpathlineto{\pgfqpoint{1.095506in}{2.469344in}}%
\pgfpathlineto{\pgfqpoint{1.095703in}{2.463840in}}%
\pgfpathlineto{\pgfqpoint{1.096690in}{2.443549in}}%
\pgfpathlineto{\pgfqpoint{1.096888in}{2.451900in}}%
\pgfpathlineto{\pgfqpoint{1.097677in}{2.457889in}}%
\pgfpathlineto{\pgfqpoint{1.097381in}{2.449878in}}%
\pgfpathlineto{\pgfqpoint{1.097875in}{2.453256in}}%
\pgfpathlineto{\pgfqpoint{1.098072in}{2.446412in}}%
\pgfpathlineto{\pgfqpoint{1.098862in}{2.455889in}}%
\pgfpathlineto{\pgfqpoint{1.099355in}{2.466169in}}%
\pgfpathlineto{\pgfqpoint{1.099947in}{2.456850in}}%
\pgfpathlineto{\pgfqpoint{1.100243in}{2.464066in}}%
\pgfpathlineto{\pgfqpoint{1.101230in}{2.472559in}}%
\pgfpathlineto{\pgfqpoint{1.100836in}{2.463732in}}%
\pgfpathlineto{\pgfqpoint{1.101329in}{2.470399in}}%
\pgfpathlineto{\pgfqpoint{1.102119in}{2.449139in}}%
\pgfpathlineto{\pgfqpoint{1.102612in}{2.462382in}}%
\pgfpathlineto{\pgfqpoint{1.102908in}{2.458111in}}%
\pgfpathlineto{\pgfqpoint{1.103599in}{2.463880in}}%
\pgfpathlineto{\pgfqpoint{1.105178in}{2.480828in}}%
\pgfpathlineto{\pgfqpoint{1.105376in}{2.477505in}}%
\pgfpathlineto{\pgfqpoint{1.107448in}{2.426962in}}%
\pgfpathlineto{\pgfqpoint{1.107942in}{2.429985in}}%
\pgfpathlineto{\pgfqpoint{1.108337in}{2.448971in}}%
\pgfpathlineto{\pgfqpoint{1.108929in}{2.428034in}}%
\pgfpathlineto{\pgfqpoint{1.110508in}{2.403659in}}%
\pgfpathlineto{\pgfqpoint{1.111594in}{2.448740in}}%
\pgfpathlineto{\pgfqpoint{1.111989in}{2.418665in}}%
\pgfpathlineto{\pgfqpoint{1.113666in}{2.337475in}}%
\pgfpathlineto{\pgfqpoint{1.113963in}{2.342953in}}%
\pgfpathlineto{\pgfqpoint{1.114752in}{2.371003in}}%
\pgfpathlineto{\pgfqpoint{1.116430in}{2.594854in}}%
\pgfpathlineto{\pgfqpoint{1.117121in}{2.579032in}}%
\pgfpathlineto{\pgfqpoint{1.118009in}{2.374632in}}%
\pgfpathlineto{\pgfqpoint{1.119687in}{2.414021in}}%
\pgfpathlineto{\pgfqpoint{1.119885in}{2.409316in}}%
\pgfpathlineto{\pgfqpoint{1.120181in}{2.428069in}}%
\pgfpathlineto{\pgfqpoint{1.121069in}{2.483605in}}%
\pgfpathlineto{\pgfqpoint{1.121661in}{2.462193in}}%
\pgfpathlineto{\pgfqpoint{1.122253in}{2.426170in}}%
\pgfpathlineto{\pgfqpoint{1.123043in}{2.450134in}}%
\pgfpathlineto{\pgfqpoint{1.124819in}{2.508624in}}%
\pgfpathlineto{\pgfqpoint{1.125116in}{2.494656in}}%
\pgfpathlineto{\pgfqpoint{1.125510in}{2.476183in}}%
\pgfpathlineto{\pgfqpoint{1.126004in}{2.497293in}}%
\pgfpathlineto{\pgfqpoint{1.126201in}{2.495025in}}%
\pgfpathlineto{\pgfqpoint{1.126300in}{2.494651in}}%
\pgfpathlineto{\pgfqpoint{1.126399in}{2.495867in}}%
\pgfpathlineto{\pgfqpoint{1.127484in}{2.518832in}}%
\pgfpathlineto{\pgfqpoint{1.127879in}{2.507943in}}%
\pgfpathlineto{\pgfqpoint{1.128077in}{2.512586in}}%
\pgfpathlineto{\pgfqpoint{1.129261in}{2.542543in}}%
\pgfpathlineto{\pgfqpoint{1.129557in}{2.536691in}}%
\pgfpathlineto{\pgfqpoint{1.129656in}{2.535334in}}%
\pgfpathlineto{\pgfqpoint{1.129952in}{2.544428in}}%
\pgfpathlineto{\pgfqpoint{1.130741in}{2.562639in}}%
\pgfpathlineto{\pgfqpoint{1.131334in}{2.557893in}}%
\pgfpathlineto{\pgfqpoint{1.132024in}{2.558836in}}%
\pgfpathlineto{\pgfqpoint{1.132222in}{2.552328in}}%
\pgfpathlineto{\pgfqpoint{1.132419in}{2.546182in}}%
\pgfpathlineto{\pgfqpoint{1.132814in}{2.555266in}}%
\pgfpathlineto{\pgfqpoint{1.133406in}{2.548793in}}%
\pgfpathlineto{\pgfqpoint{1.134393in}{2.530654in}}%
\pgfpathlineto{\pgfqpoint{1.134591in}{2.538559in}}%
\pgfpathlineto{\pgfqpoint{1.134788in}{2.549792in}}%
\pgfpathlineto{\pgfqpoint{1.135183in}{2.524993in}}%
\pgfpathlineto{\pgfqpoint{1.135578in}{2.539268in}}%
\pgfpathlineto{\pgfqpoint{1.137946in}{2.504234in}}%
\pgfpathlineto{\pgfqpoint{1.138243in}{2.517634in}}%
\pgfpathlineto{\pgfqpoint{1.138736in}{2.492948in}}%
\pgfpathlineto{\pgfqpoint{1.139032in}{2.498109in}}%
\pgfpathlineto{\pgfqpoint{1.139229in}{2.492960in}}%
\pgfpathlineto{\pgfqpoint{1.141796in}{2.404216in}}%
\pgfpathlineto{\pgfqpoint{1.141993in}{2.405926in}}%
\pgfpathlineto{\pgfqpoint{1.142388in}{2.399897in}}%
\pgfpathlineto{\pgfqpoint{1.145250in}{2.363519in}}%
\pgfpathlineto{\pgfqpoint{1.145448in}{2.364244in}}%
\pgfpathlineto{\pgfqpoint{1.145546in}{2.364561in}}%
\pgfpathlineto{\pgfqpoint{1.145645in}{2.363210in}}%
\pgfpathlineto{\pgfqpoint{1.147323in}{2.346271in}}%
\pgfpathlineto{\pgfqpoint{1.147422in}{2.347239in}}%
\pgfpathlineto{\pgfqpoint{1.148310in}{2.350881in}}%
\pgfpathlineto{\pgfqpoint{1.148014in}{2.344421in}}%
\pgfpathlineto{\pgfqpoint{1.148507in}{2.348113in}}%
\pgfpathlineto{\pgfqpoint{1.148902in}{2.332785in}}%
\pgfpathlineto{\pgfqpoint{1.149692in}{2.342965in}}%
\pgfpathlineto{\pgfqpoint{1.151764in}{2.381182in}}%
\pgfpathlineto{\pgfqpoint{1.150382in}{2.338351in}}%
\pgfpathlineto{\pgfqpoint{1.152060in}{2.362871in}}%
\pgfpathlineto{\pgfqpoint{1.152159in}{2.360892in}}%
\pgfpathlineto{\pgfqpoint{1.152356in}{2.375509in}}%
\pgfpathlineto{\pgfqpoint{1.153245in}{2.394224in}}%
\pgfpathlineto{\pgfqpoint{1.153541in}{2.387599in}}%
\pgfpathlineto{\pgfqpoint{1.154232in}{2.377334in}}%
\pgfpathlineto{\pgfqpoint{1.154429in}{2.384581in}}%
\pgfpathlineto{\pgfqpoint{1.154627in}{2.393259in}}%
\pgfpathlineto{\pgfqpoint{1.155021in}{2.380756in}}%
\pgfpathlineto{\pgfqpoint{1.155416in}{2.384580in}}%
\pgfpathlineto{\pgfqpoint{1.155712in}{2.376452in}}%
\pgfpathlineto{\pgfqpoint{1.156107in}{2.394955in}}%
\pgfpathlineto{\pgfqpoint{1.156897in}{2.402388in}}%
\pgfpathlineto{\pgfqpoint{1.156403in}{2.393942in}}%
\pgfpathlineto{\pgfqpoint{1.157193in}{2.395545in}}%
\pgfpathlineto{\pgfqpoint{1.158180in}{2.385726in}}%
\pgfpathlineto{\pgfqpoint{1.158673in}{2.388940in}}%
\pgfpathlineto{\pgfqpoint{1.158772in}{2.388790in}}%
\pgfpathlineto{\pgfqpoint{1.158871in}{2.389708in}}%
\pgfpathlineto{\pgfqpoint{1.159463in}{2.432118in}}%
\pgfpathlineto{\pgfqpoint{1.159858in}{2.463039in}}%
\pgfpathlineto{\pgfqpoint{1.160548in}{2.431551in}}%
\pgfpathlineto{\pgfqpoint{1.162029in}{2.400073in}}%
\pgfpathlineto{\pgfqpoint{1.162128in}{2.400753in}}%
\pgfpathlineto{\pgfqpoint{1.162917in}{2.441773in}}%
\pgfpathlineto{\pgfqpoint{1.164595in}{2.683561in}}%
\pgfpathlineto{\pgfqpoint{1.165483in}{2.627466in}}%
\pgfpathlineto{\pgfqpoint{1.165878in}{2.560158in}}%
\pgfpathlineto{\pgfqpoint{1.166470in}{2.401347in}}%
\pgfpathlineto{\pgfqpoint{1.167260in}{2.437672in}}%
\pgfpathlineto{\pgfqpoint{1.167457in}{2.427701in}}%
\pgfpathlineto{\pgfqpoint{1.168050in}{2.454221in}}%
\pgfpathlineto{\pgfqpoint{1.168543in}{2.452980in}}%
\pgfpathlineto{\pgfqpoint{1.169037in}{2.475255in}}%
\pgfpathlineto{\pgfqpoint{1.169530in}{2.491299in}}%
\pgfpathlineto{\pgfqpoint{1.169925in}{2.476941in}}%
\pgfpathlineto{\pgfqpoint{1.170616in}{2.423542in}}%
\pgfpathlineto{\pgfqpoint{1.171405in}{2.430561in}}%
\pgfpathlineto{\pgfqpoint{1.172688in}{2.456901in}}%
\pgfpathlineto{\pgfqpoint{1.172886in}{2.449824in}}%
\pgfpathlineto{\pgfqpoint{1.174366in}{2.397404in}}%
\pgfpathlineto{\pgfqpoint{1.174662in}{2.408843in}}%
\pgfpathlineto{\pgfqpoint{1.175748in}{2.414329in}}%
\pgfpathlineto{\pgfqpoint{1.175255in}{2.399743in}}%
\pgfpathlineto{\pgfqpoint{1.175847in}{2.413986in}}%
\pgfpathlineto{\pgfqpoint{1.175946in}{2.413417in}}%
\pgfpathlineto{\pgfqpoint{1.176143in}{2.417143in}}%
\pgfpathlineto{\pgfqpoint{1.176340in}{2.420466in}}%
\pgfpathlineto{\pgfqpoint{1.176636in}{2.406118in}}%
\pgfpathlineto{\pgfqpoint{1.177327in}{2.394101in}}%
\pgfpathlineto{\pgfqpoint{1.177623in}{2.406005in}}%
\pgfpathlineto{\pgfqpoint{1.177722in}{2.408936in}}%
\pgfpathlineto{\pgfqpoint{1.178117in}{2.398759in}}%
\pgfpathlineto{\pgfqpoint{1.178512in}{2.401304in}}%
\pgfpathlineto{\pgfqpoint{1.178808in}{2.389585in}}%
\pgfpathlineto{\pgfqpoint{1.179203in}{2.404340in}}%
\pgfpathlineto{\pgfqpoint{1.179597in}{2.395477in}}%
\pgfpathlineto{\pgfqpoint{1.180091in}{2.418084in}}%
\pgfpathlineto{\pgfqpoint{1.181078in}{2.411500in}}%
\pgfpathlineto{\pgfqpoint{1.181374in}{2.400761in}}%
\pgfpathlineto{\pgfqpoint{1.181867in}{2.419181in}}%
\pgfpathlineto{\pgfqpoint{1.182065in}{2.414289in}}%
\pgfpathlineto{\pgfqpoint{1.182164in}{2.413239in}}%
\pgfpathlineto{\pgfqpoint{1.182361in}{2.420840in}}%
\pgfpathlineto{\pgfqpoint{1.182953in}{2.413198in}}%
\pgfpathlineto{\pgfqpoint{1.183545in}{2.429980in}}%
\pgfpathlineto{\pgfqpoint{1.185618in}{2.442234in}}%
\pgfpathlineto{\pgfqpoint{1.186408in}{2.456131in}}%
\pgfpathlineto{\pgfqpoint{1.186802in}{2.445811in}}%
\pgfpathlineto{\pgfqpoint{1.189566in}{2.389195in}}%
\pgfpathlineto{\pgfqpoint{1.187395in}{2.455239in}}%
\pgfpathlineto{\pgfqpoint{1.189665in}{2.390288in}}%
\pgfpathlineto{\pgfqpoint{1.191046in}{2.453424in}}%
\pgfpathlineto{\pgfqpoint{1.192527in}{2.513065in}}%
\pgfpathlineto{\pgfqpoint{1.193119in}{2.503443in}}%
\pgfpathlineto{\pgfqpoint{1.197067in}{2.550231in}}%
\pgfpathlineto{\pgfqpoint{1.197166in}{2.550057in}}%
\pgfpathlineto{\pgfqpoint{1.197561in}{2.538182in}}%
\pgfpathlineto{\pgfqpoint{1.198350in}{2.546398in}}%
\pgfpathlineto{\pgfqpoint{1.200028in}{2.581274in}}%
\pgfpathlineto{\pgfqpoint{1.200127in}{2.578148in}}%
\pgfpathlineto{\pgfqpoint{1.200423in}{2.568893in}}%
\pgfpathlineto{\pgfqpoint{1.200916in}{2.590323in}}%
\pgfpathlineto{\pgfqpoint{1.201015in}{2.590640in}}%
\pgfpathlineto{\pgfqpoint{1.201114in}{2.588858in}}%
\pgfpathlineto{\pgfqpoint{1.201509in}{2.578154in}}%
\pgfpathlineto{\pgfqpoint{1.201903in}{2.589510in}}%
\pgfpathlineto{\pgfqpoint{1.202298in}{2.582741in}}%
\pgfpathlineto{\pgfqpoint{1.202594in}{2.594773in}}%
\pgfpathlineto{\pgfqpoint{1.202989in}{2.572947in}}%
\pgfpathlineto{\pgfqpoint{1.203285in}{2.579165in}}%
\pgfpathlineto{\pgfqpoint{1.204272in}{2.541463in}}%
\pgfpathlineto{\pgfqpoint{1.204371in}{2.537164in}}%
\pgfpathlineto{\pgfqpoint{1.204963in}{2.557400in}}%
\pgfpathlineto{\pgfqpoint{1.205259in}{2.566353in}}%
\pgfpathlineto{\pgfqpoint{1.205654in}{2.551186in}}%
\pgfpathlineto{\pgfqpoint{1.207628in}{2.522591in}}%
\pgfpathlineto{\pgfqpoint{1.207825in}{2.527604in}}%
\pgfpathlineto{\pgfqpoint{1.208220in}{2.551122in}}%
\pgfpathlineto{\pgfqpoint{1.209010in}{2.535847in}}%
\pgfpathlineto{\pgfqpoint{1.210885in}{2.457116in}}%
\pgfpathlineto{\pgfqpoint{1.210984in}{2.453470in}}%
\pgfpathlineto{\pgfqpoint{1.211378in}{2.476500in}}%
\pgfpathlineto{\pgfqpoint{1.212464in}{2.606877in}}%
\pgfpathlineto{\pgfqpoint{1.212958in}{2.676374in}}%
\pgfpathlineto{\pgfqpoint{1.213649in}{2.648241in}}%
\pgfpathlineto{\pgfqpoint{1.214833in}{2.416909in}}%
\pgfpathlineto{\pgfqpoint{1.215129in}{2.375529in}}%
\pgfpathlineto{\pgfqpoint{1.215919in}{2.401312in}}%
\pgfpathlineto{\pgfqpoint{1.216017in}{2.401408in}}%
\pgfpathlineto{\pgfqpoint{1.216116in}{2.401131in}}%
\pgfpathlineto{\pgfqpoint{1.216511in}{2.385952in}}%
\pgfpathlineto{\pgfqpoint{1.217004in}{2.402565in}}%
\pgfpathlineto{\pgfqpoint{1.217202in}{2.400521in}}%
\pgfpathlineto{\pgfqpoint{1.217991in}{2.437486in}}%
\pgfpathlineto{\pgfqpoint{1.218287in}{2.443535in}}%
\pgfpathlineto{\pgfqpoint{1.218682in}{2.433293in}}%
\pgfpathlineto{\pgfqpoint{1.219274in}{2.383925in}}%
\pgfpathlineto{\pgfqpoint{1.219965in}{2.411331in}}%
\pgfpathlineto{\pgfqpoint{1.221446in}{2.445244in}}%
\pgfpathlineto{\pgfqpoint{1.221643in}{2.435620in}}%
\pgfpathlineto{\pgfqpoint{1.222531in}{2.412303in}}%
\pgfpathlineto{\pgfqpoint{1.222828in}{2.419154in}}%
\pgfpathlineto{\pgfqpoint{1.224012in}{2.442711in}}%
\pgfpathlineto{\pgfqpoint{1.223518in}{2.414974in}}%
\pgfpathlineto{\pgfqpoint{1.224209in}{2.433622in}}%
\pgfpathlineto{\pgfqpoint{1.224407in}{2.427667in}}%
\pgfpathlineto{\pgfqpoint{1.224900in}{2.448408in}}%
\pgfpathlineto{\pgfqpoint{1.224999in}{2.450555in}}%
\pgfpathlineto{\pgfqpoint{1.225394in}{2.436616in}}%
\pgfpathlineto{\pgfqpoint{1.225591in}{2.433213in}}%
\pgfpathlineto{\pgfqpoint{1.226183in}{2.443423in}}%
\pgfpathlineto{\pgfqpoint{1.227861in}{2.468529in}}%
\pgfpathlineto{\pgfqpoint{1.233684in}{2.545132in}}%
\pgfpathlineto{\pgfqpoint{1.233980in}{2.543909in}}%
\pgfpathlineto{\pgfqpoint{1.234967in}{2.537357in}}%
\pgfpathlineto{\pgfqpoint{1.235066in}{2.539519in}}%
\pgfpathlineto{\pgfqpoint{1.235362in}{2.549563in}}%
\pgfpathlineto{\pgfqpoint{1.235757in}{2.533050in}}%
\pgfpathlineto{\pgfqpoint{1.236152in}{2.539743in}}%
\pgfpathlineto{\pgfqpoint{1.237830in}{2.499185in}}%
\pgfpathlineto{\pgfqpoint{1.238027in}{2.501698in}}%
\pgfpathlineto{\pgfqpoint{1.238225in}{2.506737in}}%
\pgfpathlineto{\pgfqpoint{1.238619in}{2.486174in}}%
\pgfpathlineto{\pgfqpoint{1.240100in}{2.452913in}}%
\pgfpathlineto{\pgfqpoint{1.241580in}{2.417428in}}%
\pgfpathlineto{\pgfqpoint{1.241975in}{2.427961in}}%
\pgfpathlineto{\pgfqpoint{1.242567in}{2.434984in}}%
\pgfpathlineto{\pgfqpoint{1.242863in}{2.426988in}}%
\pgfpathlineto{\pgfqpoint{1.243159in}{2.413323in}}%
\pgfpathlineto{\pgfqpoint{1.243949in}{2.421460in}}%
\pgfpathlineto{\pgfqpoint{1.244146in}{2.430670in}}%
\pgfpathlineto{\pgfqpoint{1.244640in}{2.414577in}}%
\pgfpathlineto{\pgfqpoint{1.244936in}{2.419493in}}%
\pgfpathlineto{\pgfqpoint{1.246417in}{2.397487in}}%
\pgfpathlineto{\pgfqpoint{1.246515in}{2.398414in}}%
\pgfpathlineto{\pgfqpoint{1.246811in}{2.403965in}}%
\pgfpathlineto{\pgfqpoint{1.247305in}{2.391204in}}%
\pgfpathlineto{\pgfqpoint{1.248292in}{2.387179in}}%
\pgfpathlineto{\pgfqpoint{1.247897in}{2.396389in}}%
\pgfpathlineto{\pgfqpoint{1.248391in}{2.388621in}}%
\pgfpathlineto{\pgfqpoint{1.248489in}{2.389720in}}%
\pgfpathlineto{\pgfqpoint{1.248687in}{2.381904in}}%
\pgfpathlineto{\pgfqpoint{1.250167in}{2.348712in}}%
\pgfpathlineto{\pgfqpoint{1.252931in}{2.217620in}}%
\pgfpathlineto{\pgfqpoint{1.253424in}{2.221101in}}%
\pgfpathlineto{\pgfqpoint{1.255201in}{2.170267in}}%
\pgfpathlineto{\pgfqpoint{1.255497in}{2.184446in}}%
\pgfpathlineto{\pgfqpoint{1.256484in}{2.245134in}}%
\pgfpathlineto{\pgfqpoint{1.257076in}{2.229501in}}%
\pgfpathlineto{\pgfqpoint{1.257471in}{2.204642in}}%
\pgfpathlineto{\pgfqpoint{1.258359in}{2.211801in}}%
\pgfpathlineto{\pgfqpoint{1.258655in}{2.204125in}}%
\pgfpathlineto{\pgfqpoint{1.259050in}{2.220132in}}%
\pgfpathlineto{\pgfqpoint{1.260531in}{2.436250in}}%
\pgfpathlineto{\pgfqpoint{1.261024in}{2.506111in}}%
\pgfpathlineto{\pgfqpoint{1.261715in}{2.472684in}}%
\pgfpathlineto{\pgfqpoint{1.262702in}{2.280797in}}%
\pgfpathlineto{\pgfqpoint{1.263097in}{2.207563in}}%
\pgfpathlineto{\pgfqpoint{1.263886in}{2.235425in}}%
\pgfpathlineto{\pgfqpoint{1.263985in}{2.233440in}}%
\pgfpathlineto{\pgfqpoint{1.264182in}{2.244918in}}%
\pgfpathlineto{\pgfqpoint{1.266452in}{2.429526in}}%
\pgfpathlineto{\pgfqpoint{1.266749in}{2.421461in}}%
\pgfpathlineto{\pgfqpoint{1.267341in}{2.389856in}}%
\pgfpathlineto{\pgfqpoint{1.267736in}{2.417138in}}%
\pgfpathlineto{\pgfqpoint{1.270006in}{2.507537in}}%
\pgfpathlineto{\pgfqpoint{1.270104in}{2.504373in}}%
\pgfpathlineto{\pgfqpoint{1.270400in}{2.494250in}}%
\pgfpathlineto{\pgfqpoint{1.270795in}{2.515996in}}%
\pgfpathlineto{\pgfqpoint{1.273361in}{2.598441in}}%
\pgfpathlineto{\pgfqpoint{1.273460in}{2.597787in}}%
\pgfpathlineto{\pgfqpoint{1.273657in}{2.598992in}}%
\pgfpathlineto{\pgfqpoint{1.274842in}{2.630247in}}%
\pgfpathlineto{\pgfqpoint{1.275237in}{2.613156in}}%
\pgfpathlineto{\pgfqpoint{1.275533in}{2.621658in}}%
\pgfpathlineto{\pgfqpoint{1.275631in}{2.623157in}}%
\pgfpathlineto{\pgfqpoint{1.276026in}{2.612289in}}%
\pgfpathlineto{\pgfqpoint{1.276520in}{2.621210in}}%
\pgfpathlineto{\pgfqpoint{1.278198in}{2.579399in}}%
\pgfpathlineto{\pgfqpoint{1.279086in}{2.584293in}}%
\pgfpathlineto{\pgfqpoint{1.280862in}{2.567706in}}%
\pgfpathlineto{\pgfqpoint{1.281257in}{2.578669in}}%
\pgfpathlineto{\pgfqpoint{1.281553in}{2.566724in}}%
\pgfpathlineto{\pgfqpoint{1.283034in}{2.536221in}}%
\pgfpathlineto{\pgfqpoint{1.283133in}{2.536856in}}%
\pgfpathlineto{\pgfqpoint{1.283429in}{2.544804in}}%
\pgfpathlineto{\pgfqpoint{1.283725in}{2.528863in}}%
\pgfpathlineto{\pgfqpoint{1.288857in}{2.266973in}}%
\pgfpathlineto{\pgfqpoint{1.289943in}{2.228238in}}%
\pgfpathlineto{\pgfqpoint{1.291226in}{2.191522in}}%
\pgfpathlineto{\pgfqpoint{1.291325in}{2.193832in}}%
\pgfpathlineto{\pgfqpoint{1.291423in}{2.196214in}}%
\pgfpathlineto{\pgfqpoint{1.291917in}{2.186144in}}%
\pgfpathlineto{\pgfqpoint{1.292213in}{2.189756in}}%
\pgfpathlineto{\pgfqpoint{1.292509in}{2.182913in}}%
\pgfpathlineto{\pgfqpoint{1.292706in}{2.178872in}}%
\pgfpathlineto{\pgfqpoint{1.293101in}{2.194806in}}%
\pgfpathlineto{\pgfqpoint{1.294384in}{2.210898in}}%
\pgfpathlineto{\pgfqpoint{1.294582in}{2.209091in}}%
\pgfpathlineto{\pgfqpoint{1.294680in}{2.209238in}}%
\pgfpathlineto{\pgfqpoint{1.297543in}{2.274726in}}%
\pgfpathlineto{\pgfqpoint{1.297839in}{2.259676in}}%
\pgfpathlineto{\pgfqpoint{1.298036in}{2.253973in}}%
\pgfpathlineto{\pgfqpoint{1.298332in}{2.263752in}}%
\pgfpathlineto{\pgfqpoint{1.298826in}{2.260961in}}%
\pgfpathlineto{\pgfqpoint{1.299023in}{2.266669in}}%
\pgfpathlineto{\pgfqpoint{1.299517in}{2.254259in}}%
\pgfpathlineto{\pgfqpoint{1.299714in}{2.256471in}}%
\pgfpathlineto{\pgfqpoint{1.300207in}{2.237339in}}%
\pgfpathlineto{\pgfqpoint{1.300898in}{2.252303in}}%
\pgfpathlineto{\pgfqpoint{1.301688in}{2.270407in}}%
\pgfpathlineto{\pgfqpoint{1.301984in}{2.260710in}}%
\pgfpathlineto{\pgfqpoint{1.302280in}{2.248038in}}%
\pgfpathlineto{\pgfqpoint{1.302971in}{2.261978in}}%
\pgfpathlineto{\pgfqpoint{1.303761in}{2.301235in}}%
\pgfpathlineto{\pgfqpoint{1.304649in}{2.360889in}}%
\pgfpathlineto{\pgfqpoint{1.305241in}{2.334969in}}%
\pgfpathlineto{\pgfqpoint{1.305340in}{2.334011in}}%
\pgfpathlineto{\pgfqpoint{1.305537in}{2.340697in}}%
\pgfpathlineto{\pgfqpoint{1.307807in}{2.538792in}}%
\pgfpathlineto{\pgfqpoint{1.308992in}{2.764023in}}%
\pgfpathlineto{\pgfqpoint{1.309683in}{2.722756in}}%
\pgfpathlineto{\pgfqpoint{1.310670in}{2.526416in}}%
\pgfpathlineto{\pgfqpoint{1.310966in}{2.466032in}}%
\pgfpathlineto{\pgfqpoint{1.311755in}{2.512666in}}%
\pgfpathlineto{\pgfqpoint{1.311953in}{2.506650in}}%
\pgfpathlineto{\pgfqpoint{1.312347in}{2.526343in}}%
\pgfpathlineto{\pgfqpoint{1.314025in}{2.646866in}}%
\pgfpathlineto{\pgfqpoint{1.314716in}{2.614228in}}%
\pgfpathlineto{\pgfqpoint{1.315308in}{2.580107in}}%
\pgfpathlineto{\pgfqpoint{1.316197in}{2.599661in}}%
\pgfpathlineto{\pgfqpoint{1.317282in}{2.611211in}}%
\pgfpathlineto{\pgfqpoint{1.317381in}{2.610311in}}%
\pgfpathlineto{\pgfqpoint{1.318467in}{2.546350in}}%
\pgfpathlineto{\pgfqpoint{1.318862in}{2.579146in}}%
\pgfpathlineto{\pgfqpoint{1.318960in}{2.580539in}}%
\pgfpathlineto{\pgfqpoint{1.319059in}{2.573420in}}%
\pgfpathlineto{\pgfqpoint{1.319355in}{2.552501in}}%
\pgfpathlineto{\pgfqpoint{1.320145in}{2.565067in}}%
\pgfpathlineto{\pgfqpoint{1.321033in}{2.566924in}}%
\pgfpathlineto{\pgfqpoint{1.320638in}{2.561282in}}%
\pgfpathlineto{\pgfqpoint{1.321132in}{2.564392in}}%
\pgfpathlineto{\pgfqpoint{1.322513in}{2.540647in}}%
\pgfpathlineto{\pgfqpoint{1.322612in}{2.541360in}}%
\pgfpathlineto{\pgfqpoint{1.323007in}{2.550920in}}%
\pgfpathlineto{\pgfqpoint{1.323698in}{2.542642in}}%
\pgfpathlineto{\pgfqpoint{1.324685in}{2.536485in}}%
\pgfpathlineto{\pgfqpoint{1.324191in}{2.544683in}}%
\pgfpathlineto{\pgfqpoint{1.324882in}{2.540539in}}%
\pgfpathlineto{\pgfqpoint{1.325080in}{2.536580in}}%
\pgfpathlineto{\pgfqpoint{1.327547in}{2.450223in}}%
\pgfpathlineto{\pgfqpoint{1.327843in}{2.445211in}}%
\pgfpathlineto{\pgfqpoint{1.330607in}{2.357569in}}%
\pgfpathlineto{\pgfqpoint{1.331100in}{2.341234in}}%
\pgfpathlineto{\pgfqpoint{1.333173in}{2.293127in}}%
\pgfpathlineto{\pgfqpoint{1.335937in}{2.237498in}}%
\pgfpathlineto{\pgfqpoint{1.336331in}{2.247886in}}%
\pgfpathlineto{\pgfqpoint{1.336529in}{2.254475in}}%
\pgfpathlineto{\pgfqpoint{1.337022in}{2.242469in}}%
\pgfpathlineto{\pgfqpoint{1.337318in}{2.245361in}}%
\pgfpathlineto{\pgfqpoint{1.337516in}{2.244538in}}%
\pgfpathlineto{\pgfqpoint{1.337713in}{2.246907in}}%
\pgfpathlineto{\pgfqpoint{1.338799in}{2.264004in}}%
\pgfpathlineto{\pgfqpoint{1.339095in}{2.256674in}}%
\pgfpathlineto{\pgfqpoint{1.339983in}{2.225448in}}%
\pgfpathlineto{\pgfqpoint{1.340575in}{2.240549in}}%
\pgfpathlineto{\pgfqpoint{1.341365in}{2.264140in}}%
\pgfpathlineto{\pgfqpoint{1.342944in}{2.424172in}}%
\pgfpathlineto{\pgfqpoint{1.345313in}{2.634057in}}%
\pgfpathlineto{\pgfqpoint{1.345412in}{2.633804in}}%
\pgfpathlineto{\pgfqpoint{1.345609in}{2.632066in}}%
\pgfpathlineto{\pgfqpoint{1.345806in}{2.639214in}}%
\pgfpathlineto{\pgfqpoint{1.346102in}{2.651149in}}%
\pgfpathlineto{\pgfqpoint{1.346596in}{2.626904in}}%
\pgfpathlineto{\pgfqpoint{1.348471in}{2.586322in}}%
\pgfpathlineto{\pgfqpoint{1.348767in}{2.588034in}}%
\pgfpathlineto{\pgfqpoint{1.349261in}{2.584743in}}%
\pgfpathlineto{\pgfqpoint{1.349656in}{2.571400in}}%
\pgfpathlineto{\pgfqpoint{1.350248in}{2.589015in}}%
\pgfpathlineto{\pgfqpoint{1.350643in}{2.582406in}}%
\pgfpathlineto{\pgfqpoint{1.351037in}{2.589992in}}%
\pgfpathlineto{\pgfqpoint{1.351827in}{2.649859in}}%
\pgfpathlineto{\pgfqpoint{1.352123in}{2.661031in}}%
\pgfpathlineto{\pgfqpoint{1.352617in}{2.634059in}}%
\pgfpathlineto{\pgfqpoint{1.354196in}{2.585346in}}%
\pgfpathlineto{\pgfqpoint{1.354295in}{2.585875in}}%
\pgfpathlineto{\pgfqpoint{1.355676in}{2.680153in}}%
\pgfpathlineto{\pgfqpoint{1.356861in}{2.802323in}}%
\pgfpathlineto{\pgfqpoint{1.357157in}{2.774571in}}%
\pgfpathlineto{\pgfqpoint{1.359822in}{2.452766in}}%
\pgfpathlineto{\pgfqpoint{1.360019in}{2.459315in}}%
\pgfpathlineto{\pgfqpoint{1.361500in}{2.489772in}}%
\pgfpathlineto{\pgfqpoint{1.361598in}{2.488330in}}%
\pgfpathlineto{\pgfqpoint{1.367323in}{2.306917in}}%
\pgfpathlineto{\pgfqpoint{1.369889in}{2.252046in}}%
\pgfpathlineto{\pgfqpoint{1.370679in}{2.253519in}}%
\pgfpathlineto{\pgfqpoint{1.371764in}{2.262922in}}%
\pgfpathlineto{\pgfqpoint{1.371369in}{2.250053in}}%
\pgfpathlineto{\pgfqpoint{1.371962in}{2.258424in}}%
\pgfpathlineto{\pgfqpoint{1.372159in}{2.253455in}}%
\pgfpathlineto{\pgfqpoint{1.372653in}{2.271628in}}%
\pgfpathlineto{\pgfqpoint{1.373146in}{2.253509in}}%
\pgfpathlineto{\pgfqpoint{1.375021in}{2.276312in}}%
\pgfpathlineto{\pgfqpoint{1.375317in}{2.269585in}}%
\pgfpathlineto{\pgfqpoint{1.375712in}{2.284012in}}%
\pgfpathlineto{\pgfqpoint{1.377982in}{2.320118in}}%
\pgfpathlineto{\pgfqpoint{1.378180in}{2.315701in}}%
\pgfpathlineto{\pgfqpoint{1.378377in}{2.309897in}}%
\pgfpathlineto{\pgfqpoint{1.378772in}{2.322475in}}%
\pgfpathlineto{\pgfqpoint{1.379167in}{2.315916in}}%
\pgfpathlineto{\pgfqpoint{1.381338in}{2.359568in}}%
\pgfpathlineto{\pgfqpoint{1.381437in}{2.359091in}}%
\pgfpathlineto{\pgfqpoint{1.381535in}{2.359099in}}%
\pgfpathlineto{\pgfqpoint{1.382621in}{2.398604in}}%
\pgfpathlineto{\pgfqpoint{1.386273in}{2.576708in}}%
\pgfpathlineto{\pgfqpoint{1.386470in}{2.571477in}}%
\pgfpathlineto{\pgfqpoint{1.388050in}{2.549410in}}%
\pgfpathlineto{\pgfqpoint{1.388148in}{2.549609in}}%
\pgfpathlineto{\pgfqpoint{1.388346in}{2.548573in}}%
\pgfpathlineto{\pgfqpoint{1.389333in}{2.533599in}}%
\pgfpathlineto{\pgfqpoint{1.389629in}{2.542663in}}%
\pgfpathlineto{\pgfqpoint{1.392590in}{2.600203in}}%
\pgfpathlineto{\pgfqpoint{1.389925in}{2.542599in}}%
\pgfpathlineto{\pgfqpoint{1.392886in}{2.588145in}}%
\pgfpathlineto{\pgfqpoint{1.395551in}{2.473503in}}%
\pgfpathlineto{\pgfqpoint{1.395649in}{2.473870in}}%
\pgfpathlineto{\pgfqpoint{1.395945in}{2.478309in}}%
\pgfpathlineto{\pgfqpoint{1.396242in}{2.467761in}}%
\pgfpathlineto{\pgfqpoint{1.398413in}{2.393458in}}%
\pgfpathlineto{\pgfqpoint{1.398808in}{2.407175in}}%
\pgfpathlineto{\pgfqpoint{1.399400in}{2.439375in}}%
\pgfpathlineto{\pgfqpoint{1.399795in}{2.409342in}}%
\pgfpathlineto{\pgfqpoint{1.401571in}{2.342683in}}%
\pgfpathlineto{\pgfqpoint{1.401670in}{2.344488in}}%
\pgfpathlineto{\pgfqpoint{1.403052in}{2.472675in}}%
\pgfpathlineto{\pgfqpoint{1.403841in}{2.579591in}}%
\pgfpathlineto{\pgfqpoint{1.404631in}{2.544457in}}%
\pgfpathlineto{\pgfqpoint{1.405519in}{2.368869in}}%
\pgfpathlineto{\pgfqpoint{1.406901in}{2.262817in}}%
\pgfpathlineto{\pgfqpoint{1.407493in}{2.256076in}}%
\pgfpathlineto{\pgfqpoint{1.407888in}{2.264041in}}%
\pgfpathlineto{\pgfqpoint{1.408184in}{2.260807in}}%
\pgfpathlineto{\pgfqpoint{1.408283in}{2.263342in}}%
\pgfpathlineto{\pgfqpoint{1.409171in}{2.303262in}}%
\pgfpathlineto{\pgfqpoint{1.409566in}{2.276050in}}%
\pgfpathlineto{\pgfqpoint{1.410356in}{2.253459in}}%
\pgfpathlineto{\pgfqpoint{1.410750in}{2.265011in}}%
\pgfpathlineto{\pgfqpoint{1.411836in}{2.291254in}}%
\pgfpathlineto{\pgfqpoint{1.412231in}{2.311688in}}%
\pgfpathlineto{\pgfqpoint{1.412922in}{2.296298in}}%
\pgfpathlineto{\pgfqpoint{1.413218in}{2.289033in}}%
\pgfpathlineto{\pgfqpoint{1.413711in}{2.302732in}}%
\pgfpathlineto{\pgfqpoint{1.415192in}{2.339412in}}%
\pgfpathlineto{\pgfqpoint{1.415290in}{2.338763in}}%
\pgfpathlineto{\pgfqpoint{1.416870in}{2.306001in}}%
\pgfpathlineto{\pgfqpoint{1.417067in}{2.315767in}}%
\pgfpathlineto{\pgfqpoint{1.420127in}{2.510985in}}%
\pgfpathlineto{\pgfqpoint{1.422101in}{2.589382in}}%
\pgfpathlineto{\pgfqpoint{1.422397in}{2.584136in}}%
\pgfpathlineto{\pgfqpoint{1.422495in}{2.583592in}}%
\pgfpathlineto{\pgfqpoint{1.422693in}{2.587155in}}%
\pgfpathlineto{\pgfqpoint{1.422792in}{2.588434in}}%
\pgfpathlineto{\pgfqpoint{1.423088in}{2.577991in}}%
\pgfpathlineto{\pgfqpoint{1.424075in}{2.567176in}}%
\pgfpathlineto{\pgfqpoint{1.423581in}{2.579610in}}%
\pgfpathlineto{\pgfqpoint{1.424371in}{2.570823in}}%
\pgfpathlineto{\pgfqpoint{1.424568in}{2.571776in}}%
\pgfpathlineto{\pgfqpoint{1.425062in}{2.569849in}}%
\pgfpathlineto{\pgfqpoint{1.425555in}{2.552037in}}%
\pgfpathlineto{\pgfqpoint{1.426246in}{2.566986in}}%
\pgfpathlineto{\pgfqpoint{1.427233in}{2.582172in}}%
\pgfpathlineto{\pgfqpoint{1.427529in}{2.570124in}}%
\pgfpathlineto{\pgfqpoint{1.429010in}{2.548308in}}%
\pgfpathlineto{\pgfqpoint{1.429207in}{2.554440in}}%
\pgfpathlineto{\pgfqpoint{1.429306in}{2.557584in}}%
\pgfpathlineto{\pgfqpoint{1.429701in}{2.534986in}}%
\pgfpathlineto{\pgfqpoint{1.430885in}{2.510945in}}%
\pgfpathlineto{\pgfqpoint{1.430194in}{2.541994in}}%
\pgfpathlineto{\pgfqpoint{1.431082in}{2.515894in}}%
\pgfpathlineto{\pgfqpoint{1.431280in}{2.519500in}}%
\pgfpathlineto{\pgfqpoint{1.431576in}{2.510108in}}%
\pgfpathlineto{\pgfqpoint{1.432069in}{2.518273in}}%
\pgfpathlineto{\pgfqpoint{1.433846in}{2.483631in}}%
\pgfpathlineto{\pgfqpoint{1.434142in}{2.486109in}}%
\pgfpathlineto{\pgfqpoint{1.434339in}{2.483639in}}%
\pgfpathlineto{\pgfqpoint{1.435326in}{2.473341in}}%
\pgfpathlineto{\pgfqpoint{1.435721in}{2.476273in}}%
\pgfpathlineto{\pgfqpoint{1.436708in}{2.482134in}}%
\pgfpathlineto{\pgfqpoint{1.436313in}{2.467824in}}%
\pgfpathlineto{\pgfqpoint{1.436906in}{2.478883in}}%
\pgfpathlineto{\pgfqpoint{1.437103in}{2.472516in}}%
\pgfpathlineto{\pgfqpoint{1.437596in}{2.481649in}}%
\pgfpathlineto{\pgfqpoint{1.437893in}{2.480393in}}%
\pgfpathlineto{\pgfqpoint{1.438090in}{2.483020in}}%
\pgfpathlineto{\pgfqpoint{1.438386in}{2.471287in}}%
\pgfpathlineto{\pgfqpoint{1.438583in}{2.463797in}}%
\pgfpathlineto{\pgfqpoint{1.439274in}{2.475600in}}%
\pgfpathlineto{\pgfqpoint{1.439669in}{2.487919in}}%
\pgfpathlineto{\pgfqpoint{1.440261in}{2.472355in}}%
\pgfpathlineto{\pgfqpoint{1.440656in}{2.461397in}}%
\pgfpathlineto{\pgfqpoint{1.442729in}{2.403714in}}%
\pgfpathlineto{\pgfqpoint{1.444407in}{2.371637in}}%
\pgfpathlineto{\pgfqpoint{1.443420in}{2.408243in}}%
\pgfpathlineto{\pgfqpoint{1.444505in}{2.372236in}}%
\pgfpathlineto{\pgfqpoint{1.444801in}{2.376747in}}%
\pgfpathlineto{\pgfqpoint{1.445098in}{2.366290in}}%
\pgfpathlineto{\pgfqpoint{1.445394in}{2.359345in}}%
\pgfpathlineto{\pgfqpoint{1.445887in}{2.373033in}}%
\pgfpathlineto{\pgfqpoint{1.446677in}{2.415993in}}%
\pgfpathlineto{\pgfqpoint{1.447170in}{2.388577in}}%
\pgfpathlineto{\pgfqpoint{1.448453in}{2.354072in}}%
\pgfpathlineto{\pgfqpoint{1.448749in}{2.357143in}}%
\pgfpathlineto{\pgfqpoint{1.449736in}{2.410283in}}%
\pgfpathlineto{\pgfqpoint{1.451316in}{2.641645in}}%
\pgfpathlineto{\pgfqpoint{1.452006in}{2.599838in}}%
\pgfpathlineto{\pgfqpoint{1.452500in}{2.543156in}}%
\pgfpathlineto{\pgfqpoint{1.453092in}{2.373135in}}%
\pgfpathlineto{\pgfqpoint{1.453980in}{2.411909in}}%
\pgfpathlineto{\pgfqpoint{1.454277in}{2.413400in}}%
\pgfpathlineto{\pgfqpoint{1.456349in}{2.516437in}}%
\pgfpathlineto{\pgfqpoint{1.456843in}{2.489702in}}%
\pgfpathlineto{\pgfqpoint{1.457435in}{2.462891in}}%
\pgfpathlineto{\pgfqpoint{1.458027in}{2.477994in}}%
\pgfpathlineto{\pgfqpoint{1.459606in}{2.531995in}}%
\pgfpathlineto{\pgfqpoint{1.460100in}{2.518704in}}%
\pgfpathlineto{\pgfqpoint{1.460297in}{2.516907in}}%
\pgfpathlineto{\pgfqpoint{1.460692in}{2.505193in}}%
\pgfpathlineto{\pgfqpoint{1.461087in}{2.524659in}}%
\pgfpathlineto{\pgfqpoint{1.461284in}{2.528030in}}%
\pgfpathlineto{\pgfqpoint{1.461778in}{2.516333in}}%
\pgfpathlineto{\pgfqpoint{1.462172in}{2.525162in}}%
\pgfpathlineto{\pgfqpoint{1.462962in}{2.531242in}}%
\pgfpathlineto{\pgfqpoint{1.463159in}{2.527940in}}%
\pgfpathlineto{\pgfqpoint{1.463456in}{2.521434in}}%
\pgfpathlineto{\pgfqpoint{1.464048in}{2.533435in}}%
\pgfpathlineto{\pgfqpoint{1.464245in}{2.534609in}}%
\pgfpathlineto{\pgfqpoint{1.465133in}{2.549949in}}%
\pgfpathlineto{\pgfqpoint{1.465627in}{2.540097in}}%
\pgfpathlineto{\pgfqpoint{1.466022in}{2.537755in}}%
\pgfpathlineto{\pgfqpoint{1.466219in}{2.542808in}}%
\pgfpathlineto{\pgfqpoint{1.466811in}{2.542182in}}%
\pgfpathlineto{\pgfqpoint{1.467601in}{2.553529in}}%
\pgfpathlineto{\pgfqpoint{1.467798in}{2.553575in}}%
\pgfpathlineto{\pgfqpoint{1.467897in}{2.552875in}}%
\pgfpathlineto{\pgfqpoint{1.468292in}{2.549128in}}%
\pgfpathlineto{\pgfqpoint{1.468489in}{2.553204in}}%
\pgfpathlineto{\pgfqpoint{1.468785in}{2.563941in}}%
\pgfpathlineto{\pgfqpoint{1.469476in}{2.550727in}}%
\pgfpathlineto{\pgfqpoint{1.471648in}{2.524286in}}%
\pgfpathlineto{\pgfqpoint{1.470068in}{2.555450in}}%
\pgfpathlineto{\pgfqpoint{1.472141in}{2.529082in}}%
\pgfpathlineto{\pgfqpoint{1.472338in}{2.530210in}}%
\pgfpathlineto{\pgfqpoint{1.472536in}{2.525701in}}%
\pgfpathlineto{\pgfqpoint{1.473622in}{2.510980in}}%
\pgfpathlineto{\pgfqpoint{1.473227in}{2.526029in}}%
\pgfpathlineto{\pgfqpoint{1.473819in}{2.513970in}}%
\pgfpathlineto{\pgfqpoint{1.474115in}{2.518415in}}%
\pgfpathlineto{\pgfqpoint{1.474707in}{2.512142in}}%
\pgfpathlineto{\pgfqpoint{1.476681in}{2.487710in}}%
\pgfpathlineto{\pgfqpoint{1.476879in}{2.489770in}}%
\pgfpathlineto{\pgfqpoint{1.476977in}{2.491219in}}%
\pgfpathlineto{\pgfqpoint{1.477372in}{2.483166in}}%
\pgfpathlineto{\pgfqpoint{1.478260in}{2.471633in}}%
\pgfpathlineto{\pgfqpoint{1.477767in}{2.486635in}}%
\pgfpathlineto{\pgfqpoint{1.478556in}{2.480653in}}%
\pgfpathlineto{\pgfqpoint{1.479445in}{2.486105in}}%
\pgfpathlineto{\pgfqpoint{1.479050in}{2.476992in}}%
\pgfpathlineto{\pgfqpoint{1.479543in}{2.483662in}}%
\pgfpathlineto{\pgfqpoint{1.480827in}{2.467813in}}%
\pgfpathlineto{\pgfqpoint{1.480925in}{2.470059in}}%
\pgfpathlineto{\pgfqpoint{1.481221in}{2.478367in}}%
\pgfpathlineto{\pgfqpoint{1.482011in}{2.470641in}}%
\pgfpathlineto{\pgfqpoint{1.482208in}{2.465679in}}%
\pgfpathlineto{\pgfqpoint{1.482603in}{2.479468in}}%
\pgfpathlineto{\pgfqpoint{1.482998in}{2.473770in}}%
\pgfpathlineto{\pgfqpoint{1.483590in}{2.477502in}}%
\pgfpathlineto{\pgfqpoint{1.484577in}{2.489977in}}%
\pgfpathlineto{\pgfqpoint{1.484182in}{2.467854in}}%
\pgfpathlineto{\pgfqpoint{1.484873in}{2.482590in}}%
\pgfpathlineto{\pgfqpoint{1.485169in}{2.494514in}}%
\pgfpathlineto{\pgfqpoint{1.486847in}{2.529667in}}%
\pgfpathlineto{\pgfqpoint{1.489216in}{2.461903in}}%
\pgfpathlineto{\pgfqpoint{1.489315in}{2.456017in}}%
\pgfpathlineto{\pgfqpoint{1.489709in}{2.479077in}}%
\pgfpathlineto{\pgfqpoint{1.490203in}{2.463427in}}%
\pgfpathlineto{\pgfqpoint{1.490400in}{2.466761in}}%
\pgfpathlineto{\pgfqpoint{1.490993in}{2.455041in}}%
\pgfpathlineto{\pgfqpoint{1.491387in}{2.437707in}}%
\pgfpathlineto{\pgfqpoint{1.492276in}{2.443328in}}%
\pgfpathlineto{\pgfqpoint{1.493164in}{2.464226in}}%
\pgfpathlineto{\pgfqpoint{1.493657in}{2.481238in}}%
\pgfpathlineto{\pgfqpoint{1.493855in}{2.466330in}}%
\pgfpathlineto{\pgfqpoint{1.495533in}{2.393073in}}%
\pgfpathlineto{\pgfqpoint{1.496026in}{2.435428in}}%
\pgfpathlineto{\pgfqpoint{1.497211in}{2.616148in}}%
\pgfpathlineto{\pgfqpoint{1.498198in}{2.752360in}}%
\pgfpathlineto{\pgfqpoint{1.498691in}{2.709152in}}%
\pgfpathlineto{\pgfqpoint{1.499481in}{2.597395in}}%
\pgfpathlineto{\pgfqpoint{1.500073in}{2.457981in}}%
\pgfpathlineto{\pgfqpoint{1.500862in}{2.485448in}}%
\pgfpathlineto{\pgfqpoint{1.501060in}{2.479979in}}%
\pgfpathlineto{\pgfqpoint{1.501652in}{2.495873in}}%
\pgfpathlineto{\pgfqpoint{1.503034in}{2.564726in}}%
\pgfpathlineto{\pgfqpoint{1.503823in}{2.537457in}}%
\pgfpathlineto{\pgfqpoint{1.504317in}{2.511963in}}%
\pgfpathlineto{\pgfqpoint{1.504909in}{2.538302in}}%
\pgfpathlineto{\pgfqpoint{1.505205in}{2.526533in}}%
\pgfpathlineto{\pgfqpoint{1.505600in}{2.545836in}}%
\pgfpathlineto{\pgfqpoint{1.506686in}{2.556156in}}%
\pgfpathlineto{\pgfqpoint{1.506883in}{2.552437in}}%
\pgfpathlineto{\pgfqpoint{1.507278in}{2.525990in}}%
\pgfpathlineto{\pgfqpoint{1.508067in}{2.546723in}}%
\pgfpathlineto{\pgfqpoint{1.510140in}{2.574025in}}%
\pgfpathlineto{\pgfqpoint{1.510239in}{2.573236in}}%
\pgfpathlineto{\pgfqpoint{1.511028in}{2.568884in}}%
\pgfpathlineto{\pgfqpoint{1.511226in}{2.574480in}}%
\pgfpathlineto{\pgfqpoint{1.512114in}{2.582555in}}%
\pgfpathlineto{\pgfqpoint{1.512410in}{2.579242in}}%
\pgfpathlineto{\pgfqpoint{1.512509in}{2.578950in}}%
\pgfpathlineto{\pgfqpoint{1.512706in}{2.581731in}}%
\pgfpathlineto{\pgfqpoint{1.514088in}{2.588611in}}%
\pgfpathlineto{\pgfqpoint{1.515075in}{2.574041in}}%
\pgfpathlineto{\pgfqpoint{1.515371in}{2.580412in}}%
\pgfpathlineto{\pgfqpoint{1.516457in}{2.591275in}}%
\pgfpathlineto{\pgfqpoint{1.515963in}{2.568360in}}%
\pgfpathlineto{\pgfqpoint{1.516556in}{2.587267in}}%
\pgfpathlineto{\pgfqpoint{1.517444in}{2.568571in}}%
\pgfpathlineto{\pgfqpoint{1.517740in}{2.574612in}}%
\pgfpathlineto{\pgfqpoint{1.517839in}{2.576489in}}%
\pgfpathlineto{\pgfqpoint{1.518233in}{2.568077in}}%
\pgfpathlineto{\pgfqpoint{1.518628in}{2.572297in}}%
\pgfpathlineto{\pgfqpoint{1.520109in}{2.562468in}}%
\pgfpathlineto{\pgfqpoint{1.520306in}{2.561286in}}%
\pgfpathlineto{\pgfqpoint{1.520602in}{2.566258in}}%
\pgfpathlineto{\pgfqpoint{1.520898in}{2.575526in}}%
\pgfpathlineto{\pgfqpoint{1.521293in}{2.554296in}}%
\pgfpathlineto{\pgfqpoint{1.521787in}{2.558697in}}%
\pgfpathlineto{\pgfqpoint{1.523168in}{2.534388in}}%
\pgfpathlineto{\pgfqpoint{1.526327in}{2.494949in}}%
\pgfpathlineto{\pgfqpoint{1.526524in}{2.499695in}}%
\pgfpathlineto{\pgfqpoint{1.527314in}{2.499026in}}%
\pgfpathlineto{\pgfqpoint{1.527906in}{2.510773in}}%
\pgfpathlineto{\pgfqpoint{1.529485in}{2.528777in}}%
\pgfpathlineto{\pgfqpoint{1.529584in}{2.527615in}}%
\pgfpathlineto{\pgfqpoint{1.529880in}{2.523868in}}%
\pgfpathlineto{\pgfqpoint{1.530373in}{2.532958in}}%
\pgfpathlineto{\pgfqpoint{1.530867in}{2.536677in}}%
\pgfpathlineto{\pgfqpoint{1.533236in}{2.602423in}}%
\pgfpathlineto{\pgfqpoint{1.534223in}{2.590939in}}%
\pgfpathlineto{\pgfqpoint{1.536591in}{2.519974in}}%
\pgfpathlineto{\pgfqpoint{1.536690in}{2.522191in}}%
\pgfpathlineto{\pgfqpoint{1.536986in}{2.532409in}}%
\pgfpathlineto{\pgfqpoint{1.537480in}{2.509783in}}%
\pgfpathlineto{\pgfqpoint{1.537578in}{2.510031in}}%
\pgfpathlineto{\pgfqpoint{1.537677in}{2.508385in}}%
\pgfpathlineto{\pgfqpoint{1.538862in}{2.469573in}}%
\pgfpathlineto{\pgfqpoint{1.539454in}{2.473474in}}%
\pgfpathlineto{\pgfqpoint{1.540243in}{2.524286in}}%
\pgfpathlineto{\pgfqpoint{1.540737in}{2.490947in}}%
\pgfpathlineto{\pgfqpoint{1.542415in}{2.430977in}}%
\pgfpathlineto{\pgfqpoint{1.543599in}{2.507763in}}%
\pgfpathlineto{\pgfqpoint{1.544981in}{2.688712in}}%
\pgfpathlineto{\pgfqpoint{1.545573in}{2.652151in}}%
\pgfpathlineto{\pgfqpoint{1.546067in}{2.572878in}}%
\pgfpathlineto{\pgfqpoint{1.546757in}{2.402228in}}%
\pgfpathlineto{\pgfqpoint{1.547646in}{2.434058in}}%
\pgfpathlineto{\pgfqpoint{1.547843in}{2.431382in}}%
\pgfpathlineto{\pgfqpoint{1.548337in}{2.443658in}}%
\pgfpathlineto{\pgfqpoint{1.548929in}{2.453220in}}%
\pgfpathlineto{\pgfqpoint{1.549324in}{2.470691in}}%
\pgfpathlineto{\pgfqpoint{1.549718in}{2.512649in}}%
\pgfpathlineto{\pgfqpoint{1.550409in}{2.467603in}}%
\pgfpathlineto{\pgfqpoint{1.551002in}{2.432298in}}%
\pgfpathlineto{\pgfqpoint{1.551890in}{2.444243in}}%
\pgfpathlineto{\pgfqpoint{1.552087in}{2.445655in}}%
\pgfpathlineto{\pgfqpoint{1.553074in}{2.468484in}}%
\pgfpathlineto{\pgfqpoint{1.553469in}{2.458424in}}%
\pgfpathlineto{\pgfqpoint{1.554456in}{2.444682in}}%
\pgfpathlineto{\pgfqpoint{1.554752in}{2.453812in}}%
\pgfpathlineto{\pgfqpoint{1.555936in}{2.481721in}}%
\pgfpathlineto{\pgfqpoint{1.556134in}{2.476651in}}%
\pgfpathlineto{\pgfqpoint{1.557022in}{2.467946in}}%
\pgfpathlineto{\pgfqpoint{1.556529in}{2.480736in}}%
\pgfpathlineto{\pgfqpoint{1.557417in}{2.473646in}}%
\pgfpathlineto{\pgfqpoint{1.559095in}{2.496873in}}%
\pgfpathlineto{\pgfqpoint{1.560279in}{2.514036in}}%
\pgfpathlineto{\pgfqpoint{1.559687in}{2.493781in}}%
\pgfpathlineto{\pgfqpoint{1.560575in}{2.503503in}}%
\pgfpathlineto{\pgfqpoint{1.561168in}{2.496388in}}%
\pgfpathlineto{\pgfqpoint{1.561464in}{2.504523in}}%
\pgfpathlineto{\pgfqpoint{1.561661in}{2.511423in}}%
\pgfpathlineto{\pgfqpoint{1.562056in}{2.494570in}}%
\pgfpathlineto{\pgfqpoint{1.562451in}{2.504470in}}%
\pgfpathlineto{\pgfqpoint{1.563536in}{2.496321in}}%
\pgfpathlineto{\pgfqpoint{1.563734in}{2.499161in}}%
\pgfpathlineto{\pgfqpoint{1.564030in}{2.508130in}}%
\pgfpathlineto{\pgfqpoint{1.564819in}{2.499972in}}%
\pgfpathlineto{\pgfqpoint{1.565115in}{2.490293in}}%
\pgfpathlineto{\pgfqpoint{1.566201in}{2.490802in}}%
\pgfpathlineto{\pgfqpoint{1.568767in}{2.456396in}}%
\pgfpathlineto{\pgfqpoint{1.569063in}{2.462432in}}%
\pgfpathlineto{\pgfqpoint{1.569261in}{2.467033in}}%
\pgfpathlineto{\pgfqpoint{1.569656in}{2.444069in}}%
\pgfpathlineto{\pgfqpoint{1.572913in}{2.317225in}}%
\pgfpathlineto{\pgfqpoint{1.575775in}{2.401280in}}%
\pgfpathlineto{\pgfqpoint{1.577058in}{2.387289in}}%
\pgfpathlineto{\pgfqpoint{1.577552in}{2.395025in}}%
\pgfpathlineto{\pgfqpoint{1.579920in}{2.431658in}}%
\pgfpathlineto{\pgfqpoint{1.580216in}{2.429353in}}%
\pgfpathlineto{\pgfqpoint{1.582585in}{2.380422in}}%
\pgfpathlineto{\pgfqpoint{1.583276in}{2.394205in}}%
\pgfpathlineto{\pgfqpoint{1.583375in}{2.395193in}}%
\pgfpathlineto{\pgfqpoint{1.583671in}{2.389524in}}%
\pgfpathlineto{\pgfqpoint{1.584658in}{2.363484in}}%
\pgfpathlineto{\pgfqpoint{1.585349in}{2.377038in}}%
\pgfpathlineto{\pgfqpoint{1.585447in}{2.376887in}}%
\pgfpathlineto{\pgfqpoint{1.585941in}{2.405743in}}%
\pgfpathlineto{\pgfqpoint{1.586731in}{2.474232in}}%
\pgfpathlineto{\pgfqpoint{1.587323in}{2.443133in}}%
\pgfpathlineto{\pgfqpoint{1.587520in}{2.442348in}}%
\pgfpathlineto{\pgfqpoint{1.587915in}{2.422316in}}%
\pgfpathlineto{\pgfqpoint{1.588705in}{2.434273in}}%
\pgfpathlineto{\pgfqpoint{1.589988in}{2.518839in}}%
\pgfpathlineto{\pgfqpoint{1.591369in}{2.720451in}}%
\pgfpathlineto{\pgfqpoint{1.591863in}{2.687608in}}%
\pgfpathlineto{\pgfqpoint{1.592751in}{2.542590in}}%
\pgfpathlineto{\pgfqpoint{1.593245in}{2.430004in}}%
\pgfpathlineto{\pgfqpoint{1.594133in}{2.454263in}}%
\pgfpathlineto{\pgfqpoint{1.594330in}{2.448537in}}%
\pgfpathlineto{\pgfqpoint{1.594824in}{2.467358in}}%
\pgfpathlineto{\pgfqpoint{1.596206in}{2.495513in}}%
\pgfpathlineto{\pgfqpoint{1.596600in}{2.482651in}}%
\pgfpathlineto{\pgfqpoint{1.598278in}{2.424263in}}%
\pgfpathlineto{\pgfqpoint{1.598574in}{2.430059in}}%
\pgfpathlineto{\pgfqpoint{1.599265in}{2.452172in}}%
\pgfpathlineto{\pgfqpoint{1.599857in}{2.435859in}}%
\pgfpathlineto{\pgfqpoint{1.600647in}{2.396037in}}%
\pgfpathlineto{\pgfqpoint{1.601535in}{2.419810in}}%
\pgfpathlineto{\pgfqpoint{1.601733in}{2.426609in}}%
\pgfpathlineto{\pgfqpoint{1.602226in}{2.408473in}}%
\pgfpathlineto{\pgfqpoint{1.602720in}{2.424540in}}%
\pgfpathlineto{\pgfqpoint{1.602818in}{2.424859in}}%
\pgfpathlineto{\pgfqpoint{1.602917in}{2.423398in}}%
\pgfpathlineto{\pgfqpoint{1.604694in}{2.397727in}}%
\pgfpathlineto{\pgfqpoint{1.604792in}{2.399039in}}%
\pgfpathlineto{\pgfqpoint{1.605089in}{2.413970in}}%
\pgfpathlineto{\pgfqpoint{1.605878in}{2.402111in}}%
\pgfpathlineto{\pgfqpoint{1.605977in}{2.399984in}}%
\pgfpathlineto{\pgfqpoint{1.606273in}{2.412473in}}%
\pgfpathlineto{\pgfqpoint{1.607260in}{2.420782in}}%
\pgfpathlineto{\pgfqpoint{1.606766in}{2.407568in}}%
\pgfpathlineto{\pgfqpoint{1.607359in}{2.418790in}}%
\pgfpathlineto{\pgfqpoint{1.608543in}{2.405712in}}%
\pgfpathlineto{\pgfqpoint{1.608642in}{2.406126in}}%
\pgfpathlineto{\pgfqpoint{1.610122in}{2.415905in}}%
\pgfpathlineto{\pgfqpoint{1.610320in}{2.413347in}}%
\pgfpathlineto{\pgfqpoint{1.610714in}{2.401895in}}%
\pgfpathlineto{\pgfqpoint{1.611109in}{2.414804in}}%
\pgfpathlineto{\pgfqpoint{1.611504in}{2.404738in}}%
\pgfpathlineto{\pgfqpoint{1.611997in}{2.429797in}}%
\pgfpathlineto{\pgfqpoint{1.613083in}{2.424690in}}%
\pgfpathlineto{\pgfqpoint{1.613281in}{2.425684in}}%
\pgfpathlineto{\pgfqpoint{1.613478in}{2.422743in}}%
\pgfpathlineto{\pgfqpoint{1.614761in}{2.403000in}}%
\pgfpathlineto{\pgfqpoint{1.614169in}{2.424446in}}%
\pgfpathlineto{\pgfqpoint{1.614958in}{2.406801in}}%
\pgfpathlineto{\pgfqpoint{1.615156in}{2.410287in}}%
\pgfpathlineto{\pgfqpoint{1.615551in}{2.398435in}}%
\pgfpathlineto{\pgfqpoint{1.615847in}{2.401688in}}%
\pgfpathlineto{\pgfqpoint{1.618512in}{2.364129in}}%
\pgfpathlineto{\pgfqpoint{1.618709in}{2.367557in}}%
\pgfpathlineto{\pgfqpoint{1.618906in}{2.369500in}}%
\pgfpathlineto{\pgfqpoint{1.619499in}{2.364331in}}%
\pgfpathlineto{\pgfqpoint{1.619597in}{2.364106in}}%
\pgfpathlineto{\pgfqpoint{1.619696in}{2.365082in}}%
\pgfpathlineto{\pgfqpoint{1.620683in}{2.369046in}}%
\pgfpathlineto{\pgfqpoint{1.620288in}{2.364048in}}%
\pgfpathlineto{\pgfqpoint{1.620782in}{2.366438in}}%
\pgfpathlineto{\pgfqpoint{1.621769in}{2.352353in}}%
\pgfpathlineto{\pgfqpoint{1.621966in}{2.357906in}}%
\pgfpathlineto{\pgfqpoint{1.622163in}{2.365341in}}%
\pgfpathlineto{\pgfqpoint{1.623150in}{2.361531in}}%
\pgfpathlineto{\pgfqpoint{1.626210in}{2.410814in}}%
\pgfpathlineto{\pgfqpoint{1.626408in}{2.412699in}}%
\pgfpathlineto{\pgfqpoint{1.626704in}{2.407461in}}%
\pgfpathlineto{\pgfqpoint{1.628480in}{2.349374in}}%
\pgfpathlineto{\pgfqpoint{1.628678in}{2.354558in}}%
\pgfpathlineto{\pgfqpoint{1.629763in}{2.365823in}}%
\pgfpathlineto{\pgfqpoint{1.629862in}{2.364262in}}%
\pgfpathlineto{\pgfqpoint{1.631145in}{2.331924in}}%
\pgfpathlineto{\pgfqpoint{1.631935in}{2.333337in}}%
\pgfpathlineto{\pgfqpoint{1.632428in}{2.366958in}}%
\pgfpathlineto{\pgfqpoint{1.632823in}{2.395777in}}%
\pgfpathlineto{\pgfqpoint{1.633415in}{2.365939in}}%
\pgfpathlineto{\pgfqpoint{1.635093in}{2.333160in}}%
\pgfpathlineto{\pgfqpoint{1.635883in}{2.364993in}}%
\pgfpathlineto{\pgfqpoint{1.637067in}{2.581913in}}%
\pgfpathlineto{\pgfqpoint{1.637758in}{2.617938in}}%
\pgfpathlineto{\pgfqpoint{1.638251in}{2.593423in}}%
\pgfpathlineto{\pgfqpoint{1.638449in}{2.591973in}}%
\pgfpathlineto{\pgfqpoint{1.638942in}{2.486621in}}%
\pgfpathlineto{\pgfqpoint{1.639534in}{2.378695in}}%
\pgfpathlineto{\pgfqpoint{1.640225in}{2.415100in}}%
\pgfpathlineto{\pgfqpoint{1.640719in}{2.402138in}}%
\pgfpathlineto{\pgfqpoint{1.641015in}{2.412765in}}%
\pgfpathlineto{\pgfqpoint{1.642397in}{2.469728in}}%
\pgfpathlineto{\pgfqpoint{1.642693in}{2.463263in}}%
\pgfpathlineto{\pgfqpoint{1.643779in}{2.401032in}}%
\pgfpathlineto{\pgfqpoint{1.644272in}{2.428174in}}%
\pgfpathlineto{\pgfqpoint{1.644469in}{2.432621in}}%
\pgfpathlineto{\pgfqpoint{1.644963in}{2.418175in}}%
\pgfpathlineto{\pgfqpoint{1.645160in}{2.420969in}}%
\pgfpathlineto{\pgfqpoint{1.645259in}{2.422209in}}%
\pgfpathlineto{\pgfqpoint{1.645654in}{2.416169in}}%
\pgfpathlineto{\pgfqpoint{1.645851in}{2.418078in}}%
\pgfpathlineto{\pgfqpoint{1.646147in}{2.401828in}}%
\pgfpathlineto{\pgfqpoint{1.647036in}{2.328390in}}%
\pgfpathlineto{\pgfqpoint{1.647924in}{2.330701in}}%
\pgfpathlineto{\pgfqpoint{1.648220in}{2.325454in}}%
\pgfpathlineto{\pgfqpoint{1.648516in}{2.335378in}}%
\pgfpathlineto{\pgfqpoint{1.650095in}{2.381008in}}%
\pgfpathlineto{\pgfqpoint{1.650293in}{2.375811in}}%
\pgfpathlineto{\pgfqpoint{1.651872in}{2.353507in}}%
\pgfpathlineto{\pgfqpoint{1.654833in}{2.306441in}}%
\pgfpathlineto{\pgfqpoint{1.656412in}{2.276267in}}%
\pgfpathlineto{\pgfqpoint{1.657892in}{2.260199in}}%
\pgfpathlineto{\pgfqpoint{1.657103in}{2.276741in}}%
\pgfpathlineto{\pgfqpoint{1.658090in}{2.262535in}}%
\pgfpathlineto{\pgfqpoint{1.659373in}{2.265970in}}%
\pgfpathlineto{\pgfqpoint{1.658485in}{2.261535in}}%
\pgfpathlineto{\pgfqpoint{1.659472in}{2.265123in}}%
\pgfpathlineto{\pgfqpoint{1.661643in}{2.229855in}}%
\pgfpathlineto{\pgfqpoint{1.662038in}{2.234562in}}%
\pgfpathlineto{\pgfqpoint{1.662334in}{2.227571in}}%
\pgfpathlineto{\pgfqpoint{1.663124in}{2.218469in}}%
\pgfpathlineto{\pgfqpoint{1.663617in}{2.221468in}}%
\pgfpathlineto{\pgfqpoint{1.665295in}{2.197529in}}%
\pgfpathlineto{\pgfqpoint{1.665591in}{2.204313in}}%
\pgfpathlineto{\pgfqpoint{1.665788in}{2.208022in}}%
\pgfpathlineto{\pgfqpoint{1.666282in}{2.191610in}}%
\pgfpathlineto{\pgfqpoint{1.667762in}{2.216669in}}%
\pgfpathlineto{\pgfqpoint{1.668552in}{2.214372in}}%
\pgfpathlineto{\pgfqpoint{1.669440in}{2.239886in}}%
\pgfpathlineto{\pgfqpoint{1.671710in}{2.321073in}}%
\pgfpathlineto{\pgfqpoint{1.672697in}{2.363661in}}%
\pgfpathlineto{\pgfqpoint{1.672993in}{2.343222in}}%
\pgfpathlineto{\pgfqpoint{1.673191in}{2.336194in}}%
\pgfpathlineto{\pgfqpoint{1.673684in}{2.352783in}}%
\pgfpathlineto{\pgfqpoint{1.674178in}{2.337044in}}%
\pgfpathlineto{\pgfqpoint{1.675856in}{2.319338in}}%
\pgfpathlineto{\pgfqpoint{1.675954in}{2.319645in}}%
\pgfpathlineto{\pgfqpoint{1.676152in}{2.317191in}}%
\pgfpathlineto{\pgfqpoint{1.677928in}{2.279212in}}%
\pgfpathlineto{\pgfqpoint{1.678126in}{2.282919in}}%
\pgfpathlineto{\pgfqpoint{1.679014in}{2.328370in}}%
\pgfpathlineto{\pgfqpoint{1.679508in}{2.294628in}}%
\pgfpathlineto{\pgfqpoint{1.680297in}{2.244345in}}%
\pgfpathlineto{\pgfqpoint{1.681284in}{2.252441in}}%
\pgfpathlineto{\pgfqpoint{1.682863in}{2.424543in}}%
\pgfpathlineto{\pgfqpoint{1.683653in}{2.535839in}}%
\pgfpathlineto{\pgfqpoint{1.684344in}{2.513612in}}%
\pgfpathlineto{\pgfqpoint{1.684936in}{2.448682in}}%
\pgfpathlineto{\pgfqpoint{1.685627in}{2.275335in}}%
\pgfpathlineto{\pgfqpoint{1.686416in}{2.324671in}}%
\pgfpathlineto{\pgfqpoint{1.687305in}{2.339757in}}%
\pgfpathlineto{\pgfqpoint{1.688687in}{2.401767in}}%
\pgfpathlineto{\pgfqpoint{1.689081in}{2.386823in}}%
\pgfpathlineto{\pgfqpoint{1.689871in}{2.338169in}}%
\pgfpathlineto{\pgfqpoint{1.690562in}{2.363024in}}%
\pgfpathlineto{\pgfqpoint{1.692240in}{2.402344in}}%
\pgfpathlineto{\pgfqpoint{1.692634in}{2.382852in}}%
\pgfpathlineto{\pgfqpoint{1.692931in}{2.374132in}}%
\pgfpathlineto{\pgfqpoint{1.693424in}{2.392810in}}%
\pgfpathlineto{\pgfqpoint{1.694411in}{2.406458in}}%
\pgfpathlineto{\pgfqpoint{1.694016in}{2.391940in}}%
\pgfpathlineto{\pgfqpoint{1.694608in}{2.402648in}}%
\pgfpathlineto{\pgfqpoint{1.694707in}{2.401205in}}%
\pgfpathlineto{\pgfqpoint{1.695102in}{2.410405in}}%
\pgfpathlineto{\pgfqpoint{1.695201in}{2.411588in}}%
\pgfpathlineto{\pgfqpoint{1.695595in}{2.403371in}}%
\pgfpathlineto{\pgfqpoint{1.695793in}{2.400991in}}%
\pgfpathlineto{\pgfqpoint{1.696089in}{2.411879in}}%
\pgfpathlineto{\pgfqpoint{1.697273in}{2.430197in}}%
\pgfpathlineto{\pgfqpoint{1.696780in}{2.409079in}}%
\pgfpathlineto{\pgfqpoint{1.697372in}{2.427879in}}%
\pgfpathlineto{\pgfqpoint{1.697569in}{2.423194in}}%
\pgfpathlineto{\pgfqpoint{1.697964in}{2.439098in}}%
\pgfpathlineto{\pgfqpoint{1.699149in}{2.454588in}}%
\pgfpathlineto{\pgfqpoint{1.698556in}{2.438233in}}%
\pgfpathlineto{\pgfqpoint{1.699346in}{2.450506in}}%
\pgfpathlineto{\pgfqpoint{1.699543in}{2.442392in}}%
\pgfpathlineto{\pgfqpoint{1.700037in}{2.458630in}}%
\pgfpathlineto{\pgfqpoint{1.700333in}{2.453965in}}%
\pgfpathlineto{\pgfqpoint{1.700629in}{2.456372in}}%
\pgfpathlineto{\pgfqpoint{1.700925in}{2.452680in}}%
\pgfpathlineto{\pgfqpoint{1.701320in}{2.453792in}}%
\pgfpathlineto{\pgfqpoint{1.701517in}{2.451937in}}%
\pgfpathlineto{\pgfqpoint{1.701715in}{2.459401in}}%
\pgfpathlineto{\pgfqpoint{1.702800in}{2.465345in}}%
\pgfpathlineto{\pgfqpoint{1.702307in}{2.448104in}}%
\pgfpathlineto{\pgfqpoint{1.702899in}{2.463591in}}%
\pgfpathlineto{\pgfqpoint{1.704084in}{2.455412in}}%
\pgfpathlineto{\pgfqpoint{1.703689in}{2.470066in}}%
\pgfpathlineto{\pgfqpoint{1.704182in}{2.456496in}}%
\pgfpathlineto{\pgfqpoint{1.705761in}{2.480545in}}%
\pgfpathlineto{\pgfqpoint{1.706748in}{2.489031in}}%
\pgfpathlineto{\pgfqpoint{1.706354in}{2.478582in}}%
\pgfpathlineto{\pgfqpoint{1.706946in}{2.482724in}}%
\pgfpathlineto{\pgfqpoint{1.707143in}{2.476583in}}%
\pgfpathlineto{\pgfqpoint{1.707538in}{2.501205in}}%
\pgfpathlineto{\pgfqpoint{1.707933in}{2.486017in}}%
\pgfpathlineto{\pgfqpoint{1.708130in}{2.488634in}}%
\pgfpathlineto{\pgfqpoint{1.709117in}{2.493078in}}%
\pgfpathlineto{\pgfqpoint{1.708624in}{2.485497in}}%
\pgfpathlineto{\pgfqpoint{1.709315in}{2.492685in}}%
\pgfpathlineto{\pgfqpoint{1.710203in}{2.503589in}}%
\pgfpathlineto{\pgfqpoint{1.710499in}{2.495248in}}%
\pgfpathlineto{\pgfqpoint{1.710696in}{2.487514in}}%
\pgfpathlineto{\pgfqpoint{1.711289in}{2.507438in}}%
\pgfpathlineto{\pgfqpoint{1.711486in}{2.506074in}}%
\pgfpathlineto{\pgfqpoint{1.711881in}{2.511628in}}%
\pgfpathlineto{\pgfqpoint{1.713361in}{2.530485in}}%
\pgfpathlineto{\pgfqpoint{1.714743in}{2.560723in}}%
\pgfpathlineto{\pgfqpoint{1.715039in}{2.557288in}}%
\pgfpathlineto{\pgfqpoint{1.715434in}{2.557129in}}%
\pgfpathlineto{\pgfqpoint{1.715631in}{2.555887in}}%
\pgfpathlineto{\pgfqpoint{1.716026in}{2.562651in}}%
\pgfpathlineto{\pgfqpoint{1.718592in}{2.595411in}}%
\pgfpathlineto{\pgfqpoint{1.718790in}{2.593171in}}%
\pgfpathlineto{\pgfqpoint{1.721158in}{2.519405in}}%
\pgfpathlineto{\pgfqpoint{1.721455in}{2.523426in}}%
\pgfpathlineto{\pgfqpoint{1.721553in}{2.524079in}}%
\pgfpathlineto{\pgfqpoint{1.721948in}{2.520328in}}%
\pgfpathlineto{\pgfqpoint{1.724119in}{2.397968in}}%
\pgfpathlineto{\pgfqpoint{1.724810in}{2.427479in}}%
\pgfpathlineto{\pgfqpoint{1.725403in}{2.444231in}}%
\pgfpathlineto{\pgfqpoint{1.725797in}{2.427898in}}%
\pgfpathlineto{\pgfqpoint{1.726192in}{2.414785in}}%
\pgfpathlineto{\pgfqpoint{1.726686in}{2.428311in}}%
\pgfpathlineto{\pgfqpoint{1.726982in}{2.423298in}}%
\pgfpathlineto{\pgfqpoint{1.727574in}{2.421045in}}%
\pgfpathlineto{\pgfqpoint{1.727771in}{2.425811in}}%
\pgfpathlineto{\pgfqpoint{1.728857in}{2.544344in}}%
\pgfpathlineto{\pgfqpoint{1.729844in}{2.668907in}}%
\pgfpathlineto{\pgfqpoint{1.730337in}{2.637326in}}%
\pgfpathlineto{\pgfqpoint{1.731127in}{2.545963in}}%
\pgfpathlineto{\pgfqpoint{1.731719in}{2.388063in}}%
\pgfpathlineto{\pgfqpoint{1.732509in}{2.438943in}}%
\pgfpathlineto{\pgfqpoint{1.732805in}{2.426524in}}%
\pgfpathlineto{\pgfqpoint{1.733397in}{2.448821in}}%
\pgfpathlineto{\pgfqpoint{1.734285in}{2.492750in}}%
\pgfpathlineto{\pgfqpoint{1.734779in}{2.536867in}}%
\pgfpathlineto{\pgfqpoint{1.735371in}{2.501426in}}%
\pgfpathlineto{\pgfqpoint{1.735766in}{2.467933in}}%
\pgfpathlineto{\pgfqpoint{1.736457in}{2.491449in}}%
\pgfpathlineto{\pgfqpoint{1.738135in}{2.535528in}}%
\pgfpathlineto{\pgfqpoint{1.738233in}{2.535861in}}%
\pgfpathlineto{\pgfqpoint{1.738332in}{2.534642in}}%
\pgfpathlineto{\pgfqpoint{1.739418in}{2.516310in}}%
\pgfpathlineto{\pgfqpoint{1.739714in}{2.526251in}}%
\pgfpathlineto{\pgfqpoint{1.741194in}{2.550311in}}%
\pgfpathlineto{\pgfqpoint{1.741293in}{2.549597in}}%
\pgfpathlineto{\pgfqpoint{1.741589in}{2.545729in}}%
\pgfpathlineto{\pgfqpoint{1.741984in}{2.552234in}}%
\pgfpathlineto{\pgfqpoint{1.743168in}{2.571365in}}%
\pgfpathlineto{\pgfqpoint{1.743366in}{2.562793in}}%
\pgfpathlineto{\pgfqpoint{1.743563in}{2.555778in}}%
\pgfpathlineto{\pgfqpoint{1.744155in}{2.575524in}}%
\pgfpathlineto{\pgfqpoint{1.744353in}{2.574387in}}%
\pgfpathlineto{\pgfqpoint{1.744550in}{2.577418in}}%
\pgfpathlineto{\pgfqpoint{1.744846in}{2.583420in}}%
\pgfpathlineto{\pgfqpoint{1.745241in}{2.568866in}}%
\pgfpathlineto{\pgfqpoint{1.745636in}{2.577220in}}%
\pgfpathlineto{\pgfqpoint{1.746031in}{2.562493in}}%
\pgfpathlineto{\pgfqpoint{1.747018in}{2.565448in}}%
\pgfpathlineto{\pgfqpoint{1.747116in}{2.565766in}}%
\pgfpathlineto{\pgfqpoint{1.747314in}{2.564066in}}%
\pgfpathlineto{\pgfqpoint{1.749386in}{2.540540in}}%
\pgfpathlineto{\pgfqpoint{1.749584in}{2.547837in}}%
\pgfpathlineto{\pgfqpoint{1.749781in}{2.552961in}}%
\pgfpathlineto{\pgfqpoint{1.750176in}{2.530887in}}%
\pgfpathlineto{\pgfqpoint{1.750275in}{2.529598in}}%
\pgfpathlineto{\pgfqpoint{1.750571in}{2.539923in}}%
\pgfpathlineto{\pgfqpoint{1.750669in}{2.540477in}}%
\pgfpathlineto{\pgfqpoint{1.750768in}{2.538282in}}%
\pgfpathlineto{\pgfqpoint{1.751459in}{2.546676in}}%
\pgfpathlineto{\pgfqpoint{1.752051in}{2.531226in}}%
\pgfpathlineto{\pgfqpoint{1.753038in}{2.515030in}}%
\pgfpathlineto{\pgfqpoint{1.754914in}{2.474818in}}%
\pgfpathlineto{\pgfqpoint{1.755012in}{2.474956in}}%
\pgfpathlineto{\pgfqpoint{1.755210in}{2.475886in}}%
\pgfpathlineto{\pgfqpoint{1.755407in}{2.471400in}}%
\pgfpathlineto{\pgfqpoint{1.757282in}{2.447107in}}%
\pgfpathlineto{\pgfqpoint{1.757480in}{2.444743in}}%
\pgfpathlineto{\pgfqpoint{1.757875in}{2.453717in}}%
\pgfpathlineto{\pgfqpoint{1.757973in}{2.453853in}}%
\pgfpathlineto{\pgfqpoint{1.759355in}{2.437182in}}%
\pgfpathlineto{\pgfqpoint{1.759750in}{2.438964in}}%
\pgfpathlineto{\pgfqpoint{1.760243in}{2.450890in}}%
\pgfpathlineto{\pgfqpoint{1.760737in}{2.436672in}}%
\pgfpathlineto{\pgfqpoint{1.761230in}{2.430605in}}%
\pgfpathlineto{\pgfqpoint{1.761625in}{2.441466in}}%
\pgfpathlineto{\pgfqpoint{1.762415in}{2.464614in}}%
\pgfpathlineto{\pgfqpoint{1.763106in}{2.461768in}}%
\pgfpathlineto{\pgfqpoint{1.764685in}{2.503505in}}%
\pgfpathlineto{\pgfqpoint{1.765080in}{2.488184in}}%
\pgfpathlineto{\pgfqpoint{1.767054in}{2.445711in}}%
\pgfpathlineto{\pgfqpoint{1.767843in}{2.462821in}}%
\pgfpathlineto{\pgfqpoint{1.768139in}{2.466705in}}%
\pgfpathlineto{\pgfqpoint{1.768534in}{2.457349in}}%
\pgfpathlineto{\pgfqpoint{1.769817in}{2.421002in}}%
\pgfpathlineto{\pgfqpoint{1.770113in}{2.432263in}}%
\pgfpathlineto{\pgfqpoint{1.771100in}{2.474156in}}%
\pgfpathlineto{\pgfqpoint{1.771495in}{2.449417in}}%
\pgfpathlineto{\pgfqpoint{1.772975in}{2.407284in}}%
\pgfpathlineto{\pgfqpoint{1.773074in}{2.407534in}}%
\pgfpathlineto{\pgfqpoint{1.773469in}{2.410693in}}%
\pgfpathlineto{\pgfqpoint{1.774851in}{2.551682in}}%
\pgfpathlineto{\pgfqpoint{1.775542in}{2.674169in}}%
\pgfpathlineto{\pgfqpoint{1.776430in}{2.658859in}}%
\pgfpathlineto{\pgfqpoint{1.777022in}{2.563991in}}%
\pgfpathlineto{\pgfqpoint{1.777614in}{2.412736in}}%
\pgfpathlineto{\pgfqpoint{1.778503in}{2.432016in}}%
\pgfpathlineto{\pgfqpoint{1.778700in}{2.435703in}}%
\pgfpathlineto{\pgfqpoint{1.778897in}{2.440757in}}%
\pgfpathlineto{\pgfqpoint{1.779490in}{2.429216in}}%
\pgfpathlineto{\pgfqpoint{1.779588in}{2.429659in}}%
\pgfpathlineto{\pgfqpoint{1.779687in}{2.429693in}}%
\pgfpathlineto{\pgfqpoint{1.780082in}{2.446770in}}%
\pgfpathlineto{\pgfqpoint{1.780674in}{2.476843in}}%
\pgfpathlineto{\pgfqpoint{1.781167in}{2.455094in}}%
\pgfpathlineto{\pgfqpoint{1.781858in}{2.403121in}}%
\pgfpathlineto{\pgfqpoint{1.782648in}{2.431340in}}%
\pgfpathlineto{\pgfqpoint{1.783931in}{2.466939in}}%
\pgfpathlineto{\pgfqpoint{1.784227in}{2.460959in}}%
\pgfpathlineto{\pgfqpoint{1.785115in}{2.434352in}}%
\pgfpathlineto{\pgfqpoint{1.785609in}{2.447757in}}%
\pgfpathlineto{\pgfqpoint{1.785806in}{2.444775in}}%
\pgfpathlineto{\pgfqpoint{1.786102in}{2.456425in}}%
\pgfpathlineto{\pgfqpoint{1.787287in}{2.498399in}}%
\pgfpathlineto{\pgfqpoint{1.787583in}{2.485151in}}%
\pgfpathlineto{\pgfqpoint{1.787682in}{2.484950in}}%
\pgfpathlineto{\pgfqpoint{1.789458in}{2.533881in}}%
\pgfpathlineto{\pgfqpoint{1.789656in}{2.529957in}}%
\pgfpathlineto{\pgfqpoint{1.789853in}{2.526050in}}%
\pgfpathlineto{\pgfqpoint{1.790248in}{2.537379in}}%
\pgfpathlineto{\pgfqpoint{1.790544in}{2.547434in}}%
\pgfpathlineto{\pgfqpoint{1.791432in}{2.543206in}}%
\pgfpathlineto{\pgfqpoint{1.791926in}{2.547092in}}%
\pgfpathlineto{\pgfqpoint{1.792123in}{2.543852in}}%
\pgfpathlineto{\pgfqpoint{1.802092in}{2.169226in}}%
\pgfpathlineto{\pgfqpoint{1.802190in}{2.171276in}}%
\pgfpathlineto{\pgfqpoint{1.805645in}{2.305486in}}%
\pgfpathlineto{\pgfqpoint{1.807323in}{2.359070in}}%
\pgfpathlineto{\pgfqpoint{1.809198in}{2.444580in}}%
\pgfpathlineto{\pgfqpoint{1.811468in}{2.529099in}}%
\pgfpathlineto{\pgfqpoint{1.811567in}{2.529817in}}%
\pgfpathlineto{\pgfqpoint{1.811764in}{2.525060in}}%
\pgfpathlineto{\pgfqpoint{1.812652in}{2.510074in}}%
\pgfpathlineto{\pgfqpoint{1.812949in}{2.520997in}}%
\pgfpathlineto{\pgfqpoint{1.814232in}{2.554712in}}%
\pgfpathlineto{\pgfqpoint{1.814528in}{2.546480in}}%
\pgfpathlineto{\pgfqpoint{1.814824in}{2.539157in}}%
\pgfpathlineto{\pgfqpoint{1.815219in}{2.553862in}}%
\pgfpathlineto{\pgfqpoint{1.816896in}{2.618464in}}%
\pgfpathlineto{\pgfqpoint{1.815811in}{2.552751in}}%
\pgfpathlineto{\pgfqpoint{1.817094in}{2.609394in}}%
\pgfpathlineto{\pgfqpoint{1.818870in}{2.531865in}}%
\pgfpathlineto{\pgfqpoint{1.818969in}{2.532586in}}%
\pgfpathlineto{\pgfqpoint{1.820252in}{2.621802in}}%
\pgfpathlineto{\pgfqpoint{1.821831in}{2.809764in}}%
\pgfpathlineto{\pgfqpoint{1.822226in}{2.775060in}}%
\pgfpathlineto{\pgfqpoint{1.823016in}{2.625844in}}%
\pgfpathlineto{\pgfqpoint{1.823411in}{2.533041in}}%
\pgfpathlineto{\pgfqpoint{1.824200in}{2.581401in}}%
\pgfpathlineto{\pgfqpoint{1.825088in}{2.607448in}}%
\pgfpathlineto{\pgfqpoint{1.825483in}{2.607257in}}%
\pgfpathlineto{\pgfqpoint{1.826273in}{2.658489in}}%
\pgfpathlineto{\pgfqpoint{1.826569in}{2.669380in}}%
\pgfpathlineto{\pgfqpoint{1.827062in}{2.644714in}}%
\pgfpathlineto{\pgfqpoint{1.827556in}{2.611694in}}%
\pgfpathlineto{\pgfqpoint{1.828247in}{2.621070in}}%
\pgfpathlineto{\pgfqpoint{1.829925in}{2.686603in}}%
\pgfpathlineto{\pgfqpoint{1.830023in}{2.683006in}}%
\pgfpathlineto{\pgfqpoint{1.830912in}{2.659061in}}%
\pgfpathlineto{\pgfqpoint{1.831307in}{2.664056in}}%
\pgfpathlineto{\pgfqpoint{1.832787in}{2.687034in}}%
\pgfpathlineto{\pgfqpoint{1.833280in}{2.678340in}}%
\pgfpathlineto{\pgfqpoint{1.833478in}{2.675627in}}%
\pgfpathlineto{\pgfqpoint{1.833971in}{2.685083in}}%
\pgfpathlineto{\pgfqpoint{1.834169in}{2.684279in}}%
\pgfpathlineto{\pgfqpoint{1.834267in}{2.684969in}}%
\pgfpathlineto{\pgfqpoint{1.834662in}{2.700566in}}%
\pgfpathlineto{\pgfqpoint{1.835057in}{2.676859in}}%
\pgfpathlineto{\pgfqpoint{1.835156in}{2.676244in}}%
\pgfpathlineto{\pgfqpoint{1.835254in}{2.681065in}}%
\pgfpathlineto{\pgfqpoint{1.836735in}{2.716686in}}%
\pgfpathlineto{\pgfqpoint{1.836834in}{2.718412in}}%
\pgfpathlineto{\pgfqpoint{1.837130in}{2.707241in}}%
\pgfpathlineto{\pgfqpoint{1.837228in}{2.704892in}}%
\pgfpathlineto{\pgfqpoint{1.837722in}{2.715623in}}%
\pgfpathlineto{\pgfqpoint{1.837821in}{2.715472in}}%
\pgfpathlineto{\pgfqpoint{1.838018in}{2.717860in}}%
\pgfpathlineto{\pgfqpoint{1.838314in}{2.728177in}}%
\pgfpathlineto{\pgfqpoint{1.838808in}{2.713179in}}%
\pgfpathlineto{\pgfqpoint{1.839104in}{2.719305in}}%
\pgfpathlineto{\pgfqpoint{1.839795in}{2.727086in}}%
\pgfpathlineto{\pgfqpoint{1.840486in}{2.725711in}}%
\pgfpathlineto{\pgfqpoint{1.840979in}{2.712818in}}%
\pgfpathlineto{\pgfqpoint{1.841966in}{2.716543in}}%
\pgfpathlineto{\pgfqpoint{1.842756in}{2.718510in}}%
\pgfpathlineto{\pgfqpoint{1.842361in}{2.714733in}}%
\pgfpathlineto{\pgfqpoint{1.842854in}{2.716178in}}%
\pgfpathlineto{\pgfqpoint{1.843348in}{2.705324in}}%
\pgfpathlineto{\pgfqpoint{1.843940in}{2.714228in}}%
\pgfpathlineto{\pgfqpoint{1.844039in}{2.715206in}}%
\pgfpathlineto{\pgfqpoint{1.844335in}{2.708263in}}%
\pgfpathlineto{\pgfqpoint{1.845618in}{2.686633in}}%
\pgfpathlineto{\pgfqpoint{1.845815in}{2.689034in}}%
\pgfpathlineto{\pgfqpoint{1.846013in}{2.690949in}}%
\pgfpathlineto{\pgfqpoint{1.846309in}{2.682420in}}%
\pgfpathlineto{\pgfqpoint{1.847493in}{2.666069in}}%
\pgfpathlineto{\pgfqpoint{1.847789in}{2.671982in}}%
\pgfpathlineto{\pgfqpoint{1.847987in}{2.673990in}}%
\pgfpathlineto{\pgfqpoint{1.848381in}{2.665040in}}%
\pgfpathlineto{\pgfqpoint{1.849763in}{2.650780in}}%
\pgfpathlineto{\pgfqpoint{1.849862in}{2.651176in}}%
\pgfpathlineto{\pgfqpoint{1.850257in}{2.658879in}}%
\pgfpathlineto{\pgfqpoint{1.850750in}{2.648388in}}%
\pgfpathlineto{\pgfqpoint{1.851046in}{2.647512in}}%
\pgfpathlineto{\pgfqpoint{1.851342in}{2.651313in}}%
\pgfpathlineto{\pgfqpoint{1.852625in}{2.666201in}}%
\pgfpathlineto{\pgfqpoint{1.852132in}{2.639627in}}%
\pgfpathlineto{\pgfqpoint{1.852724in}{2.664004in}}%
\pgfpathlineto{\pgfqpoint{1.853020in}{2.649497in}}%
\pgfpathlineto{\pgfqpoint{1.853810in}{2.658797in}}%
\pgfpathlineto{\pgfqpoint{1.855685in}{2.697440in}}%
\pgfpathlineto{\pgfqpoint{1.855883in}{2.691592in}}%
\pgfpathlineto{\pgfqpoint{1.857067in}{2.679782in}}%
\pgfpathlineto{\pgfqpoint{1.857166in}{2.680226in}}%
\pgfpathlineto{\pgfqpoint{1.857363in}{2.681897in}}%
\pgfpathlineto{\pgfqpoint{1.857659in}{2.675107in}}%
\pgfpathlineto{\pgfqpoint{1.858251in}{2.631318in}}%
\pgfpathlineto{\pgfqpoint{1.859140in}{2.642271in}}%
\pgfpathlineto{\pgfqpoint{1.860028in}{2.623173in}}%
\pgfpathlineto{\pgfqpoint{1.860916in}{2.594764in}}%
\pgfpathlineto{\pgfqpoint{1.861508in}{2.597437in}}%
\pgfpathlineto{\pgfqpoint{1.861607in}{2.596075in}}%
\pgfpathlineto{\pgfqpoint{1.861903in}{2.603628in}}%
\pgfpathlineto{\pgfqpoint{1.862594in}{2.647703in}}%
\pgfpathlineto{\pgfqpoint{1.863088in}{2.614368in}}%
\pgfpathlineto{\pgfqpoint{1.864272in}{2.578120in}}%
\pgfpathlineto{\pgfqpoint{1.864469in}{2.579167in}}%
\pgfpathlineto{\pgfqpoint{1.865259in}{2.607397in}}%
\pgfpathlineto{\pgfqpoint{1.867036in}{2.865576in}}%
\pgfpathlineto{\pgfqpoint{1.868319in}{2.795219in}}%
\pgfpathlineto{\pgfqpoint{1.869108in}{2.584045in}}%
\pgfpathlineto{\pgfqpoint{1.870293in}{2.598413in}}%
\pgfpathlineto{\pgfqpoint{1.872168in}{2.674864in}}%
\pgfpathlineto{\pgfqpoint{1.872365in}{2.666611in}}%
\pgfpathlineto{\pgfqpoint{1.874339in}{2.551588in}}%
\pgfpathlineto{\pgfqpoint{1.876116in}{2.485263in}}%
\pgfpathlineto{\pgfqpoint{1.877300in}{2.427937in}}%
\pgfpathlineto{\pgfqpoint{1.877498in}{2.438948in}}%
\pgfpathlineto{\pgfqpoint{1.877596in}{2.443191in}}%
\pgfpathlineto{\pgfqpoint{1.877991in}{2.416610in}}%
\pgfpathlineto{\pgfqpoint{1.881248in}{2.289685in}}%
\pgfpathlineto{\pgfqpoint{1.881643in}{2.293631in}}%
\pgfpathlineto{\pgfqpoint{1.881939in}{2.285948in}}%
\pgfpathlineto{\pgfqpoint{1.882136in}{2.280984in}}%
\pgfpathlineto{\pgfqpoint{1.882531in}{2.296408in}}%
\pgfpathlineto{\pgfqpoint{1.885394in}{2.404461in}}%
\pgfpathlineto{\pgfqpoint{1.885492in}{2.402889in}}%
\pgfpathlineto{\pgfqpoint{1.885788in}{2.390328in}}%
\pgfpathlineto{\pgfqpoint{1.886677in}{2.397046in}}%
\pgfpathlineto{\pgfqpoint{1.888453in}{2.437028in}}%
\pgfpathlineto{\pgfqpoint{1.888848in}{2.424509in}}%
\pgfpathlineto{\pgfqpoint{1.889342in}{2.441954in}}%
\pgfpathlineto{\pgfqpoint{1.889440in}{2.441878in}}%
\pgfpathlineto{\pgfqpoint{1.889934in}{2.428942in}}%
\pgfpathlineto{\pgfqpoint{1.890625in}{2.436070in}}%
\pgfpathlineto{\pgfqpoint{1.891612in}{2.445507in}}%
\pgfpathlineto{\pgfqpoint{1.891217in}{2.435455in}}%
\pgfpathlineto{\pgfqpoint{1.891809in}{2.438930in}}%
\pgfpathlineto{\pgfqpoint{1.892697in}{2.426078in}}%
\pgfpathlineto{\pgfqpoint{1.892204in}{2.444376in}}%
\pgfpathlineto{\pgfqpoint{1.893388in}{2.430354in}}%
\pgfpathlineto{\pgfqpoint{1.894276in}{2.432372in}}%
\pgfpathlineto{\pgfqpoint{1.893882in}{2.426664in}}%
\pgfpathlineto{\pgfqpoint{1.894375in}{2.431461in}}%
\pgfpathlineto{\pgfqpoint{1.894671in}{2.426388in}}%
\pgfpathlineto{\pgfqpoint{1.895263in}{2.434515in}}%
\pgfpathlineto{\pgfqpoint{1.895362in}{2.436006in}}%
\pgfpathlineto{\pgfqpoint{1.895658in}{2.427181in}}%
\pgfpathlineto{\pgfqpoint{1.895856in}{2.420706in}}%
\pgfpathlineto{\pgfqpoint{1.896349in}{2.437470in}}%
\pgfpathlineto{\pgfqpoint{1.896645in}{2.432266in}}%
\pgfpathlineto{\pgfqpoint{1.897237in}{2.430375in}}%
\pgfpathlineto{\pgfqpoint{1.897435in}{2.432556in}}%
\pgfpathlineto{\pgfqpoint{1.899113in}{2.468040in}}%
\pgfpathlineto{\pgfqpoint{1.899409in}{2.457721in}}%
\pgfpathlineto{\pgfqpoint{1.899507in}{2.456654in}}%
\pgfpathlineto{\pgfqpoint{1.899606in}{2.459651in}}%
\pgfpathlineto{\pgfqpoint{1.901383in}{2.512938in}}%
\pgfpathlineto{\pgfqpoint{1.901975in}{2.492566in}}%
\pgfpathlineto{\pgfqpoint{1.904541in}{2.418271in}}%
\pgfpathlineto{\pgfqpoint{1.902567in}{2.493987in}}%
\pgfpathlineto{\pgfqpoint{1.904936in}{2.424978in}}%
\pgfpathlineto{\pgfqpoint{1.905331in}{2.410105in}}%
\pgfpathlineto{\pgfqpoint{1.907206in}{2.366659in}}%
\pgfpathlineto{\pgfqpoint{1.908193in}{2.422635in}}%
\pgfpathlineto{\pgfqpoint{1.908588in}{2.386325in}}%
\pgfpathlineto{\pgfqpoint{1.910068in}{2.339623in}}%
\pgfpathlineto{\pgfqpoint{1.910167in}{2.342402in}}%
\pgfpathlineto{\pgfqpoint{1.911845in}{2.476917in}}%
\pgfpathlineto{\pgfqpoint{1.912832in}{2.607206in}}%
\pgfpathlineto{\pgfqpoint{1.913325in}{2.569451in}}%
\pgfpathlineto{\pgfqpoint{1.913523in}{2.571215in}}%
\pgfpathlineto{\pgfqpoint{1.913720in}{2.563922in}}%
\pgfpathlineto{\pgfqpoint{1.914707in}{2.318480in}}%
\pgfpathlineto{\pgfqpoint{1.915990in}{2.348109in}}%
\pgfpathlineto{\pgfqpoint{1.917767in}{2.420621in}}%
\pgfpathlineto{\pgfqpoint{1.918063in}{2.407308in}}%
\pgfpathlineto{\pgfqpoint{1.918951in}{2.348711in}}%
\pgfpathlineto{\pgfqpoint{1.919445in}{2.371711in}}%
\pgfpathlineto{\pgfqpoint{1.921024in}{2.404333in}}%
\pgfpathlineto{\pgfqpoint{1.922504in}{2.369778in}}%
\pgfpathlineto{\pgfqpoint{1.923294in}{2.381631in}}%
\pgfpathlineto{\pgfqpoint{1.924380in}{2.402126in}}%
\pgfpathlineto{\pgfqpoint{1.924676in}{2.400038in}}%
\pgfpathlineto{\pgfqpoint{1.925564in}{2.391216in}}%
\pgfpathlineto{\pgfqpoint{1.925761in}{2.397580in}}%
\pgfpathlineto{\pgfqpoint{1.925959in}{2.404393in}}%
\pgfpathlineto{\pgfqpoint{1.926650in}{2.394647in}}%
\pgfpathlineto{\pgfqpoint{1.927735in}{2.387257in}}%
\pgfpathlineto{\pgfqpoint{1.927341in}{2.397325in}}%
\pgfpathlineto{\pgfqpoint{1.927834in}{2.388701in}}%
\pgfpathlineto{\pgfqpoint{1.928328in}{2.410393in}}%
\pgfpathlineto{\pgfqpoint{1.929216in}{2.402462in}}%
\pgfpathlineto{\pgfqpoint{1.929315in}{2.402895in}}%
\pgfpathlineto{\pgfqpoint{1.929413in}{2.399890in}}%
\pgfpathlineto{\pgfqpoint{1.929709in}{2.386333in}}%
\pgfpathlineto{\pgfqpoint{1.930203in}{2.401703in}}%
\pgfpathlineto{\pgfqpoint{1.930499in}{2.400021in}}%
\pgfpathlineto{\pgfqpoint{1.930795in}{2.403246in}}%
\pgfpathlineto{\pgfqpoint{1.931091in}{2.396650in}}%
\pgfpathlineto{\pgfqpoint{1.931387in}{2.391937in}}%
\pgfpathlineto{\pgfqpoint{1.932078in}{2.395923in}}%
\pgfpathlineto{\pgfqpoint{1.932473in}{2.403455in}}%
\pgfpathlineto{\pgfqpoint{1.932868in}{2.392325in}}%
\pgfpathlineto{\pgfqpoint{1.933657in}{2.381362in}}%
\pgfpathlineto{\pgfqpoint{1.934348in}{2.387209in}}%
\pgfpathlineto{\pgfqpoint{1.934644in}{2.390175in}}%
\pgfpathlineto{\pgfqpoint{1.934842in}{2.386467in}}%
\pgfpathlineto{\pgfqpoint{1.936520in}{2.352458in}}%
\pgfpathlineto{\pgfqpoint{1.936816in}{2.357806in}}%
\pgfpathlineto{\pgfqpoint{1.937013in}{2.344683in}}%
\pgfpathlineto{\pgfqpoint{1.938296in}{2.325758in}}%
\pgfpathlineto{\pgfqpoint{1.940270in}{2.295910in}}%
\pgfpathlineto{\pgfqpoint{1.940468in}{2.301079in}}%
\pgfpathlineto{\pgfqpoint{1.940566in}{2.303383in}}%
\pgfpathlineto{\pgfqpoint{1.941060in}{2.290491in}}%
\pgfpathlineto{\pgfqpoint{1.941257in}{2.291366in}}%
\pgfpathlineto{\pgfqpoint{1.941455in}{2.288178in}}%
\pgfpathlineto{\pgfqpoint{1.941652in}{2.284286in}}%
\pgfpathlineto{\pgfqpoint{1.942244in}{2.293573in}}%
\pgfpathlineto{\pgfqpoint{1.942343in}{2.295201in}}%
\pgfpathlineto{\pgfqpoint{1.942738in}{2.283748in}}%
\pgfpathlineto{\pgfqpoint{1.942836in}{2.280598in}}%
\pgfpathlineto{\pgfqpoint{1.943231in}{2.292315in}}%
\pgfpathlineto{\pgfqpoint{1.943725in}{2.287145in}}%
\pgfpathlineto{\pgfqpoint{1.945995in}{2.325088in}}%
\pgfpathlineto{\pgfqpoint{1.946093in}{2.324367in}}%
\pgfpathlineto{\pgfqpoint{1.946291in}{2.328347in}}%
\pgfpathlineto{\pgfqpoint{1.946686in}{2.348090in}}%
\pgfpathlineto{\pgfqpoint{1.947475in}{2.339435in}}%
\pgfpathlineto{\pgfqpoint{1.947870in}{2.333545in}}%
\pgfpathlineto{\pgfqpoint{1.949844in}{2.284382in}}%
\pgfpathlineto{\pgfqpoint{1.950337in}{2.297037in}}%
\pgfpathlineto{\pgfqpoint{1.950535in}{2.300255in}}%
\pgfpathlineto{\pgfqpoint{1.950930in}{2.290146in}}%
\pgfpathlineto{\pgfqpoint{1.952608in}{2.237551in}}%
\pgfpathlineto{\pgfqpoint{1.952706in}{2.239151in}}%
\pgfpathlineto{\pgfqpoint{1.954088in}{2.302781in}}%
\pgfpathlineto{\pgfqpoint{1.954384in}{2.282399in}}%
\pgfpathlineto{\pgfqpoint{1.955766in}{2.237191in}}%
\pgfpathlineto{\pgfqpoint{1.955963in}{2.238600in}}%
\pgfpathlineto{\pgfqpoint{1.956654in}{2.277198in}}%
\pgfpathlineto{\pgfqpoint{1.957246in}{2.353699in}}%
\pgfpathlineto{\pgfqpoint{1.958332in}{2.522768in}}%
\pgfpathlineto{\pgfqpoint{1.959122in}{2.492801in}}%
\pgfpathlineto{\pgfqpoint{1.959418in}{2.475081in}}%
\pgfpathlineto{\pgfqpoint{1.960306in}{2.257943in}}%
\pgfpathlineto{\pgfqpoint{1.961392in}{2.298199in}}%
\pgfpathlineto{\pgfqpoint{1.963366in}{2.343822in}}%
\pgfpathlineto{\pgfqpoint{1.963563in}{2.335412in}}%
\pgfpathlineto{\pgfqpoint{1.964451in}{2.265582in}}%
\pgfpathlineto{\pgfqpoint{1.964945in}{2.298463in}}%
\pgfpathlineto{\pgfqpoint{1.966623in}{2.406263in}}%
\pgfpathlineto{\pgfqpoint{1.966919in}{2.396902in}}%
\pgfpathlineto{\pgfqpoint{1.967906in}{2.366268in}}%
\pgfpathlineto{\pgfqpoint{1.968301in}{2.384682in}}%
\pgfpathlineto{\pgfqpoint{1.969386in}{2.394192in}}%
\pgfpathlineto{\pgfqpoint{1.968794in}{2.372803in}}%
\pgfpathlineto{\pgfqpoint{1.969682in}{2.392488in}}%
\pgfpathlineto{\pgfqpoint{1.970275in}{2.405072in}}%
\pgfpathlineto{\pgfqpoint{1.970669in}{2.394779in}}%
\pgfpathlineto{\pgfqpoint{1.971459in}{2.389861in}}%
\pgfpathlineto{\pgfqpoint{1.971656in}{2.393529in}}%
\pgfpathlineto{\pgfqpoint{1.972051in}{2.404322in}}%
\pgfpathlineto{\pgfqpoint{1.972545in}{2.389751in}}%
\pgfpathlineto{\pgfqpoint{1.972643in}{2.389846in}}%
\pgfpathlineto{\pgfqpoint{1.974223in}{2.395807in}}%
\pgfpathlineto{\pgfqpoint{1.974420in}{2.394067in}}%
\pgfpathlineto{\pgfqpoint{1.976986in}{2.363501in}}%
\pgfpathlineto{\pgfqpoint{1.977085in}{2.364934in}}%
\pgfpathlineto{\pgfqpoint{1.977381in}{2.370629in}}%
\pgfpathlineto{\pgfqpoint{1.977776in}{2.356549in}}%
\pgfpathlineto{\pgfqpoint{1.978763in}{2.343169in}}%
\pgfpathlineto{\pgfqpoint{1.978269in}{2.356794in}}%
\pgfpathlineto{\pgfqpoint{1.978960in}{2.349164in}}%
\pgfpathlineto{\pgfqpoint{1.979947in}{2.358020in}}%
\pgfpathlineto{\pgfqpoint{1.979552in}{2.344286in}}%
\pgfpathlineto{\pgfqpoint{1.980046in}{2.357077in}}%
\pgfpathlineto{\pgfqpoint{1.981428in}{2.324161in}}%
\pgfpathlineto{\pgfqpoint{1.981625in}{2.326887in}}%
\pgfpathlineto{\pgfqpoint{1.981921in}{2.335447in}}%
\pgfpathlineto{\pgfqpoint{1.982217in}{2.312694in}}%
\pgfpathlineto{\pgfqpoint{1.983303in}{2.274268in}}%
\pgfpathlineto{\pgfqpoint{1.983599in}{2.282198in}}%
\pgfpathlineto{\pgfqpoint{1.984685in}{2.262761in}}%
\pgfpathlineto{\pgfqpoint{1.987448in}{2.198617in}}%
\pgfpathlineto{\pgfqpoint{1.987843in}{2.204477in}}%
\pgfpathlineto{\pgfqpoint{1.988633in}{2.202179in}}%
\pgfpathlineto{\pgfqpoint{1.990113in}{2.190877in}}%
\pgfpathlineto{\pgfqpoint{1.990212in}{2.190595in}}%
\pgfpathlineto{\pgfqpoint{1.990311in}{2.192910in}}%
\pgfpathlineto{\pgfqpoint{1.991495in}{2.212319in}}%
\pgfpathlineto{\pgfqpoint{1.991692in}{2.210174in}}%
\pgfpathlineto{\pgfqpoint{1.992778in}{2.201772in}}%
\pgfpathlineto{\pgfqpoint{1.992285in}{2.220611in}}%
\pgfpathlineto{\pgfqpoint{1.992975in}{2.206023in}}%
\pgfpathlineto{\pgfqpoint{1.993173in}{2.210768in}}%
\pgfpathlineto{\pgfqpoint{1.993568in}{2.203222in}}%
\pgfpathlineto{\pgfqpoint{1.993962in}{2.206917in}}%
\pgfpathlineto{\pgfqpoint{1.995147in}{2.159170in}}%
\pgfpathlineto{\pgfqpoint{1.995838in}{2.168715in}}%
\pgfpathlineto{\pgfqpoint{1.996331in}{2.181099in}}%
\pgfpathlineto{\pgfqpoint{1.996627in}{2.164925in}}%
\pgfpathlineto{\pgfqpoint{1.996923in}{2.149926in}}%
\pgfpathlineto{\pgfqpoint{1.997614in}{2.168849in}}%
\pgfpathlineto{\pgfqpoint{1.997910in}{2.172996in}}%
\pgfpathlineto{\pgfqpoint{1.999391in}{2.256133in}}%
\pgfpathlineto{\pgfqpoint{2.000279in}{2.220023in}}%
\pgfpathlineto{\pgfqpoint{2.000575in}{2.204067in}}%
\pgfpathlineto{\pgfqpoint{2.000970in}{2.226677in}}%
\pgfpathlineto{\pgfqpoint{2.001464in}{2.214304in}}%
\pgfpathlineto{\pgfqpoint{2.002549in}{2.291327in}}%
\pgfpathlineto{\pgfqpoint{2.003832in}{2.492101in}}%
\pgfpathlineto{\pgfqpoint{2.004523in}{2.445408in}}%
\pgfpathlineto{\pgfqpoint{2.004721in}{2.441054in}}%
\pgfpathlineto{\pgfqpoint{2.005411in}{2.269172in}}%
\pgfpathlineto{\pgfqpoint{2.005806in}{2.193546in}}%
\pgfpathlineto{\pgfqpoint{2.006596in}{2.224347in}}%
\pgfpathlineto{\pgfqpoint{2.006793in}{2.218604in}}%
\pgfpathlineto{\pgfqpoint{2.007583in}{2.228011in}}%
\pgfpathlineto{\pgfqpoint{2.008767in}{2.271056in}}%
\pgfpathlineto{\pgfqpoint{2.009359in}{2.256897in}}%
\pgfpathlineto{\pgfqpoint{2.010050in}{2.200756in}}%
\pgfpathlineto{\pgfqpoint{2.010939in}{2.216977in}}%
\pgfpathlineto{\pgfqpoint{2.012123in}{2.248093in}}%
\pgfpathlineto{\pgfqpoint{2.012419in}{2.242738in}}%
\pgfpathlineto{\pgfqpoint{2.013209in}{2.230047in}}%
\pgfpathlineto{\pgfqpoint{2.013406in}{2.236760in}}%
\pgfpathlineto{\pgfqpoint{2.015676in}{2.334117in}}%
\pgfpathlineto{\pgfqpoint{2.020019in}{2.499939in}}%
\pgfpathlineto{\pgfqpoint{2.020216in}{2.494802in}}%
\pgfpathlineto{\pgfqpoint{2.020414in}{2.489689in}}%
\pgfpathlineto{\pgfqpoint{2.020907in}{2.508497in}}%
\pgfpathlineto{\pgfqpoint{2.022486in}{2.542403in}}%
\pgfpathlineto{\pgfqpoint{2.022585in}{2.539135in}}%
\pgfpathlineto{\pgfqpoint{2.022881in}{2.528986in}}%
\pgfpathlineto{\pgfqpoint{2.023375in}{2.550028in}}%
\pgfpathlineto{\pgfqpoint{2.023473in}{2.549443in}}%
\pgfpathlineto{\pgfqpoint{2.023671in}{2.546093in}}%
\pgfpathlineto{\pgfqpoint{2.024066in}{2.560142in}}%
\pgfpathlineto{\pgfqpoint{2.024164in}{2.559957in}}%
\pgfpathlineto{\pgfqpoint{2.024559in}{2.547093in}}%
\pgfpathlineto{\pgfqpoint{2.025053in}{2.567417in}}%
\pgfpathlineto{\pgfqpoint{2.025250in}{2.567381in}}%
\pgfpathlineto{\pgfqpoint{2.027027in}{2.583713in}}%
\pgfpathlineto{\pgfqpoint{2.027125in}{2.582803in}}%
\pgfpathlineto{\pgfqpoint{2.027717in}{2.590672in}}%
\pgfpathlineto{\pgfqpoint{2.028704in}{2.573908in}}%
\pgfpathlineto{\pgfqpoint{2.028803in}{2.573370in}}%
\pgfpathlineto{\pgfqpoint{2.029001in}{2.576590in}}%
\pgfpathlineto{\pgfqpoint{2.029494in}{2.594045in}}%
\pgfpathlineto{\pgfqpoint{2.029988in}{2.572224in}}%
\pgfpathlineto{\pgfqpoint{2.031665in}{2.599628in}}%
\pgfpathlineto{\pgfqpoint{2.031961in}{2.593083in}}%
\pgfpathlineto{\pgfqpoint{2.032356in}{2.575588in}}%
\pgfpathlineto{\pgfqpoint{2.032850in}{2.593119in}}%
\pgfpathlineto{\pgfqpoint{2.033146in}{2.585477in}}%
\pgfpathlineto{\pgfqpoint{2.033245in}{2.584652in}}%
\pgfpathlineto{\pgfqpoint{2.033343in}{2.587686in}}%
\pgfpathlineto{\pgfqpoint{2.033738in}{2.605147in}}%
\pgfpathlineto{\pgfqpoint{2.034528in}{2.597262in}}%
\pgfpathlineto{\pgfqpoint{2.034922in}{2.603781in}}%
\pgfpathlineto{\pgfqpoint{2.036798in}{2.641794in}}%
\pgfpathlineto{\pgfqpoint{2.036995in}{2.641128in}}%
\pgfpathlineto{\pgfqpoint{2.037390in}{2.655398in}}%
\pgfpathlineto{\pgfqpoint{2.037686in}{2.666227in}}%
\pgfpathlineto{\pgfqpoint{2.038377in}{2.654279in}}%
\pgfpathlineto{\pgfqpoint{2.040252in}{2.613690in}}%
\pgfpathlineto{\pgfqpoint{2.040548in}{2.599351in}}%
\pgfpathlineto{\pgfqpoint{2.041239in}{2.615637in}}%
\pgfpathlineto{\pgfqpoint{2.041437in}{2.618718in}}%
\pgfpathlineto{\pgfqpoint{2.041930in}{2.609064in}}%
\pgfpathlineto{\pgfqpoint{2.043016in}{2.564834in}}%
\pgfpathlineto{\pgfqpoint{2.043805in}{2.575243in}}%
\pgfpathlineto{\pgfqpoint{2.044595in}{2.629185in}}%
\pgfpathlineto{\pgfqpoint{2.045187in}{2.598096in}}%
\pgfpathlineto{\pgfqpoint{2.046766in}{2.522802in}}%
\pgfpathlineto{\pgfqpoint{2.047260in}{2.537901in}}%
\pgfpathlineto{\pgfqpoint{2.047951in}{2.567050in}}%
\pgfpathlineto{\pgfqpoint{2.049431in}{2.780234in}}%
\pgfpathlineto{\pgfqpoint{2.050122in}{2.763077in}}%
\pgfpathlineto{\pgfqpoint{2.050714in}{2.621880in}}%
\pgfpathlineto{\pgfqpoint{2.051208in}{2.510417in}}%
\pgfpathlineto{\pgfqpoint{2.052096in}{2.519464in}}%
\pgfpathlineto{\pgfqpoint{2.052293in}{2.512736in}}%
\pgfpathlineto{\pgfqpoint{2.052787in}{2.522903in}}%
\pgfpathlineto{\pgfqpoint{2.053182in}{2.516100in}}%
\pgfpathlineto{\pgfqpoint{2.053774in}{2.533283in}}%
\pgfpathlineto{\pgfqpoint{2.054169in}{2.553569in}}%
\pgfpathlineto{\pgfqpoint{2.054761in}{2.525232in}}%
\pgfpathlineto{\pgfqpoint{2.055452in}{2.473892in}}%
\pgfpathlineto{\pgfqpoint{2.056439in}{2.495858in}}%
\pgfpathlineto{\pgfqpoint{2.057426in}{2.514096in}}%
\pgfpathlineto{\pgfqpoint{2.057623in}{2.508781in}}%
\pgfpathlineto{\pgfqpoint{2.058610in}{2.461424in}}%
\pgfpathlineto{\pgfqpoint{2.059301in}{2.476521in}}%
\pgfpathlineto{\pgfqpoint{2.059400in}{2.477081in}}%
\pgfpathlineto{\pgfqpoint{2.059498in}{2.474405in}}%
\pgfpathlineto{\pgfqpoint{2.059893in}{2.461349in}}%
\pgfpathlineto{\pgfqpoint{2.060485in}{2.470909in}}%
\pgfpathlineto{\pgfqpoint{2.061374in}{2.493949in}}%
\pgfpathlineto{\pgfqpoint{2.061769in}{2.484743in}}%
\pgfpathlineto{\pgfqpoint{2.061966in}{2.480225in}}%
\pgfpathlineto{\pgfqpoint{2.062361in}{2.486995in}}%
\pgfpathlineto{\pgfqpoint{2.062854in}{2.480900in}}%
\pgfpathlineto{\pgfqpoint{2.065322in}{2.521331in}}%
\pgfpathlineto{\pgfqpoint{2.065914in}{2.513746in}}%
\pgfpathlineto{\pgfqpoint{2.067789in}{2.505706in}}%
\pgfpathlineto{\pgfqpoint{2.068085in}{2.513240in}}%
\pgfpathlineto{\pgfqpoint{2.068283in}{2.516545in}}%
\pgfpathlineto{\pgfqpoint{2.068677in}{2.503184in}}%
\pgfpathlineto{\pgfqpoint{2.069171in}{2.513470in}}%
\pgfpathlineto{\pgfqpoint{2.070750in}{2.483815in}}%
\pgfpathlineto{\pgfqpoint{2.070849in}{2.485764in}}%
\pgfpathlineto{\pgfqpoint{2.071145in}{2.493435in}}%
\pgfpathlineto{\pgfqpoint{2.071737in}{2.482338in}}%
\pgfpathlineto{\pgfqpoint{2.074501in}{2.428175in}}%
\pgfpathlineto{\pgfqpoint{2.074698in}{2.433325in}}%
\pgfpathlineto{\pgfqpoint{2.074797in}{2.434603in}}%
\pgfpathlineto{\pgfqpoint{2.074994in}{2.426200in}}%
\pgfpathlineto{\pgfqpoint{2.076179in}{2.412312in}}%
\pgfpathlineto{\pgfqpoint{2.075586in}{2.426298in}}%
\pgfpathlineto{\pgfqpoint{2.076277in}{2.413428in}}%
\pgfpathlineto{\pgfqpoint{2.077264in}{2.425719in}}%
\pgfpathlineto{\pgfqpoint{2.077462in}{2.419718in}}%
\pgfpathlineto{\pgfqpoint{2.077856in}{2.397693in}}%
\pgfpathlineto{\pgfqpoint{2.078547in}{2.414408in}}%
\pgfpathlineto{\pgfqpoint{2.079633in}{2.434095in}}%
\pgfpathlineto{\pgfqpoint{2.079929in}{2.420425in}}%
\pgfpathlineto{\pgfqpoint{2.080028in}{2.417759in}}%
\pgfpathlineto{\pgfqpoint{2.080423in}{2.432906in}}%
\pgfpathlineto{\pgfqpoint{2.080817in}{2.423668in}}%
\pgfpathlineto{\pgfqpoint{2.083186in}{2.504333in}}%
\pgfpathlineto{\pgfqpoint{2.083285in}{2.502131in}}%
\pgfpathlineto{\pgfqpoint{2.083482in}{2.498054in}}%
\pgfpathlineto{\pgfqpoint{2.084272in}{2.502482in}}%
\pgfpathlineto{\pgfqpoint{2.084371in}{2.503712in}}%
\pgfpathlineto{\pgfqpoint{2.084667in}{2.495132in}}%
\pgfpathlineto{\pgfqpoint{2.086246in}{2.449440in}}%
\pgfpathlineto{\pgfqpoint{2.086641in}{2.455578in}}%
\pgfpathlineto{\pgfqpoint{2.087134in}{2.469807in}}%
\pgfpathlineto{\pgfqpoint{2.087430in}{2.456907in}}%
\pgfpathlineto{\pgfqpoint{2.088911in}{2.418441in}}%
\pgfpathlineto{\pgfqpoint{2.089009in}{2.418595in}}%
\pgfpathlineto{\pgfqpoint{2.089404in}{2.416431in}}%
\pgfpathlineto{\pgfqpoint{2.089602in}{2.428734in}}%
\pgfpathlineto{\pgfqpoint{2.089996in}{2.466879in}}%
\pgfpathlineto{\pgfqpoint{2.090786in}{2.452986in}}%
\pgfpathlineto{\pgfqpoint{2.092168in}{2.399219in}}%
\pgfpathlineto{\pgfqpoint{2.092267in}{2.399428in}}%
\pgfpathlineto{\pgfqpoint{2.093254in}{2.447552in}}%
\pgfpathlineto{\pgfqpoint{2.094931in}{2.670190in}}%
\pgfpathlineto{\pgfqpoint{2.095918in}{2.617422in}}%
\pgfpathlineto{\pgfqpoint{2.096807in}{2.407306in}}%
\pgfpathlineto{\pgfqpoint{2.097892in}{2.430393in}}%
\pgfpathlineto{\pgfqpoint{2.098485in}{2.454925in}}%
\pgfpathlineto{\pgfqpoint{2.100064in}{2.515297in}}%
\pgfpathlineto{\pgfqpoint{2.100360in}{2.499946in}}%
\pgfpathlineto{\pgfqpoint{2.100853in}{2.465364in}}%
\pgfpathlineto{\pgfqpoint{2.101544in}{2.482502in}}%
\pgfpathlineto{\pgfqpoint{2.103222in}{2.537344in}}%
\pgfpathlineto{\pgfqpoint{2.103617in}{2.518645in}}%
\pgfpathlineto{\pgfqpoint{2.104012in}{2.508135in}}%
\pgfpathlineto{\pgfqpoint{2.104604in}{2.518946in}}%
\pgfpathlineto{\pgfqpoint{2.107367in}{2.563365in}}%
\pgfpathlineto{\pgfqpoint{2.107664in}{2.562186in}}%
\pgfpathlineto{\pgfqpoint{2.108552in}{2.575291in}}%
\pgfpathlineto{\pgfqpoint{2.109835in}{2.599156in}}%
\pgfpathlineto{\pgfqpoint{2.109934in}{2.598922in}}%
\pgfpathlineto{\pgfqpoint{2.110230in}{2.593251in}}%
\pgfpathlineto{\pgfqpoint{2.110625in}{2.609316in}}%
\pgfpathlineto{\pgfqpoint{2.111414in}{2.614660in}}%
\pgfpathlineto{\pgfqpoint{2.111019in}{2.609051in}}%
\pgfpathlineto{\pgfqpoint{2.111612in}{2.610269in}}%
\pgfpathlineto{\pgfqpoint{2.111809in}{2.605152in}}%
\pgfpathlineto{\pgfqpoint{2.112204in}{2.623810in}}%
\pgfpathlineto{\pgfqpoint{2.112302in}{2.625536in}}%
\pgfpathlineto{\pgfqpoint{2.112599in}{2.619767in}}%
\pgfpathlineto{\pgfqpoint{2.113092in}{2.623498in}}%
\pgfpathlineto{\pgfqpoint{2.114079in}{2.617305in}}%
\pgfpathlineto{\pgfqpoint{2.114276in}{2.621306in}}%
\pgfpathlineto{\pgfqpoint{2.114671in}{2.631337in}}%
\pgfpathlineto{\pgfqpoint{2.115263in}{2.618948in}}%
\pgfpathlineto{\pgfqpoint{2.115461in}{2.614601in}}%
\pgfpathlineto{\pgfqpoint{2.115856in}{2.623348in}}%
\pgfpathlineto{\pgfqpoint{2.116250in}{2.621259in}}%
\pgfpathlineto{\pgfqpoint{2.117237in}{2.630075in}}%
\pgfpathlineto{\pgfqpoint{2.117435in}{2.626992in}}%
\pgfpathlineto{\pgfqpoint{2.119211in}{2.599126in}}%
\pgfpathlineto{\pgfqpoint{2.119310in}{2.598614in}}%
\pgfpathlineto{\pgfqpoint{2.119606in}{2.601977in}}%
\pgfpathlineto{\pgfqpoint{2.119804in}{2.603703in}}%
\pgfpathlineto{\pgfqpoint{2.120198in}{2.595757in}}%
\pgfpathlineto{\pgfqpoint{2.120396in}{2.593906in}}%
\pgfpathlineto{\pgfqpoint{2.120791in}{2.600723in}}%
\pgfpathlineto{\pgfqpoint{2.121087in}{2.598904in}}%
\pgfpathlineto{\pgfqpoint{2.121383in}{2.592698in}}%
\pgfpathlineto{\pgfqpoint{2.122468in}{2.575793in}}%
\pgfpathlineto{\pgfqpoint{2.122666in}{2.578943in}}%
\pgfpathlineto{\pgfqpoint{2.122765in}{2.580940in}}%
\pgfpathlineto{\pgfqpoint{2.123258in}{2.571427in}}%
\pgfpathlineto{\pgfqpoint{2.123357in}{2.572270in}}%
\pgfpathlineto{\pgfqpoint{2.123455in}{2.572386in}}%
\pgfpathlineto{\pgfqpoint{2.123554in}{2.571142in}}%
\pgfpathlineto{\pgfqpoint{2.123752in}{2.566996in}}%
\pgfpathlineto{\pgfqpoint{2.124442in}{2.575717in}}%
\pgfpathlineto{\pgfqpoint{2.124541in}{2.575780in}}%
\pgfpathlineto{\pgfqpoint{2.124640in}{2.574916in}}%
\pgfpathlineto{\pgfqpoint{2.125725in}{2.559706in}}%
\pgfpathlineto{\pgfqpoint{2.125331in}{2.580851in}}%
\pgfpathlineto{\pgfqpoint{2.126022in}{2.568809in}}%
\pgfpathlineto{\pgfqpoint{2.126120in}{2.569702in}}%
\pgfpathlineto{\pgfqpoint{2.126318in}{2.562904in}}%
\pgfpathlineto{\pgfqpoint{2.126416in}{2.561079in}}%
\pgfpathlineto{\pgfqpoint{2.126811in}{2.572394in}}%
\pgfpathlineto{\pgfqpoint{2.126910in}{2.571764in}}%
\pgfpathlineto{\pgfqpoint{2.127009in}{2.571680in}}%
\pgfpathlineto{\pgfqpoint{2.128193in}{2.597521in}}%
\pgfpathlineto{\pgfqpoint{2.128588in}{2.586581in}}%
\pgfpathlineto{\pgfqpoint{2.131055in}{2.487995in}}%
\pgfpathlineto{\pgfqpoint{2.131845in}{2.503404in}}%
\pgfpathlineto{\pgfqpoint{2.132634in}{2.559828in}}%
\pgfpathlineto{\pgfqpoint{2.133424in}{2.545638in}}%
\pgfpathlineto{\pgfqpoint{2.134707in}{2.512786in}}%
\pgfpathlineto{\pgfqpoint{2.134904in}{2.519843in}}%
\pgfpathlineto{\pgfqpoint{2.135497in}{2.560318in}}%
\pgfpathlineto{\pgfqpoint{2.136188in}{2.546590in}}%
\pgfpathlineto{\pgfqpoint{2.137569in}{2.498966in}}%
\pgfpathlineto{\pgfqpoint{2.137767in}{2.502128in}}%
\pgfpathlineto{\pgfqpoint{2.139050in}{2.586630in}}%
\pgfpathlineto{\pgfqpoint{2.140629in}{2.788717in}}%
\pgfpathlineto{\pgfqpoint{2.141123in}{2.733928in}}%
\pgfpathlineto{\pgfqpoint{2.141813in}{2.601774in}}%
\pgfpathlineto{\pgfqpoint{2.142307in}{2.509542in}}%
\pgfpathlineto{\pgfqpoint{2.143096in}{2.529943in}}%
\pgfpathlineto{\pgfqpoint{2.143294in}{2.522996in}}%
\pgfpathlineto{\pgfqpoint{2.143689in}{2.540007in}}%
\pgfpathlineto{\pgfqpoint{2.144083in}{2.537057in}}%
\pgfpathlineto{\pgfqpoint{2.145465in}{2.579441in}}%
\pgfpathlineto{\pgfqpoint{2.146057in}{2.558877in}}%
\pgfpathlineto{\pgfqpoint{2.146551in}{2.513092in}}%
\pgfpathlineto{\pgfqpoint{2.147341in}{2.534554in}}%
\pgfpathlineto{\pgfqpoint{2.147637in}{2.533166in}}%
\pgfpathlineto{\pgfqpoint{2.147735in}{2.534959in}}%
\pgfpathlineto{\pgfqpoint{2.148525in}{2.564784in}}%
\pgfpathlineto{\pgfqpoint{2.149117in}{2.550090in}}%
\pgfpathlineto{\pgfqpoint{2.149611in}{2.530128in}}%
\pgfpathlineto{\pgfqpoint{2.150400in}{2.539805in}}%
\pgfpathlineto{\pgfqpoint{2.151289in}{2.534939in}}%
\pgfpathlineto{\pgfqpoint{2.151782in}{2.557372in}}%
\pgfpathlineto{\pgfqpoint{2.153361in}{2.529450in}}%
\pgfpathlineto{\pgfqpoint{2.153657in}{2.534979in}}%
\pgfpathlineto{\pgfqpoint{2.153756in}{2.535209in}}%
\pgfpathlineto{\pgfqpoint{2.153855in}{2.532667in}}%
\pgfpathlineto{\pgfqpoint{2.154052in}{2.527772in}}%
\pgfpathlineto{\pgfqpoint{2.154447in}{2.533065in}}%
\pgfpathlineto{\pgfqpoint{2.154842in}{2.530412in}}%
\pgfpathlineto{\pgfqpoint{2.155039in}{2.537013in}}%
\pgfpathlineto{\pgfqpoint{2.155533in}{2.520614in}}%
\pgfpathlineto{\pgfqpoint{2.155927in}{2.530934in}}%
\pgfpathlineto{\pgfqpoint{2.156026in}{2.530956in}}%
\pgfpathlineto{\pgfqpoint{2.158592in}{2.504682in}}%
\pgfpathlineto{\pgfqpoint{2.158691in}{2.507222in}}%
\pgfpathlineto{\pgfqpoint{2.158987in}{2.517057in}}%
\pgfpathlineto{\pgfqpoint{2.159678in}{2.504746in}}%
\pgfpathlineto{\pgfqpoint{2.161257in}{2.491825in}}%
\pgfpathlineto{\pgfqpoint{2.161356in}{2.493519in}}%
\pgfpathlineto{\pgfqpoint{2.161652in}{2.506856in}}%
\pgfpathlineto{\pgfqpoint{2.162145in}{2.481020in}}%
\pgfpathlineto{\pgfqpoint{2.162343in}{2.484601in}}%
\pgfpathlineto{\pgfqpoint{2.162540in}{2.488464in}}%
\pgfpathlineto{\pgfqpoint{2.162935in}{2.480275in}}%
\pgfpathlineto{\pgfqpoint{2.163330in}{2.482032in}}%
\pgfpathlineto{\pgfqpoint{2.165008in}{2.458977in}}%
\pgfpathlineto{\pgfqpoint{2.165106in}{2.459325in}}%
\pgfpathlineto{\pgfqpoint{2.165205in}{2.459568in}}%
\pgfpathlineto{\pgfqpoint{2.165304in}{2.458694in}}%
\pgfpathlineto{\pgfqpoint{2.166488in}{2.436486in}}%
\pgfpathlineto{\pgfqpoint{2.166784in}{2.446106in}}%
\pgfpathlineto{\pgfqpoint{2.166982in}{2.448955in}}%
\pgfpathlineto{\pgfqpoint{2.167475in}{2.436197in}}%
\pgfpathlineto{\pgfqpoint{2.167574in}{2.437651in}}%
\pgfpathlineto{\pgfqpoint{2.168166in}{2.445034in}}%
\pgfpathlineto{\pgfqpoint{2.168561in}{2.438668in}}%
\pgfpathlineto{\pgfqpoint{2.170140in}{2.423196in}}%
\pgfpathlineto{\pgfqpoint{2.170337in}{2.425353in}}%
\pgfpathlineto{\pgfqpoint{2.170732in}{2.428994in}}%
\pgfpathlineto{\pgfqpoint{2.171324in}{2.424507in}}%
\pgfpathlineto{\pgfqpoint{2.171423in}{2.424434in}}%
\pgfpathlineto{\pgfqpoint{2.171620in}{2.425552in}}%
\pgfpathlineto{\pgfqpoint{2.174680in}{2.467852in}}%
\pgfpathlineto{\pgfqpoint{2.174878in}{2.463811in}}%
\pgfpathlineto{\pgfqpoint{2.177049in}{2.400800in}}%
\pgfpathlineto{\pgfqpoint{2.177148in}{2.401583in}}%
\pgfpathlineto{\pgfqpoint{2.178135in}{2.414455in}}%
\pgfpathlineto{\pgfqpoint{2.178431in}{2.403574in}}%
\pgfpathlineto{\pgfqpoint{2.179813in}{2.374601in}}%
\pgfpathlineto{\pgfqpoint{2.180109in}{2.375498in}}%
\pgfpathlineto{\pgfqpoint{2.180207in}{2.374995in}}%
\pgfpathlineto{\pgfqpoint{2.180405in}{2.379696in}}%
\pgfpathlineto{\pgfqpoint{2.181194in}{2.428568in}}%
\pgfpathlineto{\pgfqpoint{2.181688in}{2.395436in}}%
\pgfpathlineto{\pgfqpoint{2.182477in}{2.368090in}}%
\pgfpathlineto{\pgfqpoint{2.183168in}{2.370080in}}%
\pgfpathlineto{\pgfqpoint{2.183267in}{2.369191in}}%
\pgfpathlineto{\pgfqpoint{2.183563in}{2.375764in}}%
\pgfpathlineto{\pgfqpoint{2.184945in}{2.524139in}}%
\pgfpathlineto{\pgfqpoint{2.185932in}{2.629596in}}%
\pgfpathlineto{\pgfqpoint{2.186327in}{2.600183in}}%
\pgfpathlineto{\pgfqpoint{2.187018in}{2.538198in}}%
\pgfpathlineto{\pgfqpoint{2.187708in}{2.349040in}}%
\pgfpathlineto{\pgfqpoint{2.188597in}{2.388447in}}%
\pgfpathlineto{\pgfqpoint{2.188893in}{2.381210in}}%
\pgfpathlineto{\pgfqpoint{2.189288in}{2.402855in}}%
\pgfpathlineto{\pgfqpoint{2.190867in}{2.463593in}}%
\pgfpathlineto{\pgfqpoint{2.191064in}{2.455965in}}%
\pgfpathlineto{\pgfqpoint{2.192051in}{2.394933in}}%
\pgfpathlineto{\pgfqpoint{2.192643in}{2.410988in}}%
\pgfpathlineto{\pgfqpoint{2.193926in}{2.446883in}}%
\pgfpathlineto{\pgfqpoint{2.194617in}{2.431517in}}%
\pgfpathlineto{\pgfqpoint{2.195210in}{2.421688in}}%
\pgfpathlineto{\pgfqpoint{2.195703in}{2.429187in}}%
\pgfpathlineto{\pgfqpoint{2.196394in}{2.436588in}}%
\pgfpathlineto{\pgfqpoint{2.196789in}{2.429920in}}%
\pgfpathlineto{\pgfqpoint{2.196887in}{2.429092in}}%
\pgfpathlineto{\pgfqpoint{2.197085in}{2.434173in}}%
\pgfpathlineto{\pgfqpoint{2.197381in}{2.445343in}}%
\pgfpathlineto{\pgfqpoint{2.198269in}{2.438597in}}%
\pgfpathlineto{\pgfqpoint{2.199157in}{2.434151in}}%
\pgfpathlineto{\pgfqpoint{2.199256in}{2.436531in}}%
\pgfpathlineto{\pgfqpoint{2.199651in}{2.455244in}}%
\pgfpathlineto{\pgfqpoint{2.200441in}{2.442771in}}%
\pgfpathlineto{\pgfqpoint{2.201822in}{2.455671in}}%
\pgfpathlineto{\pgfqpoint{2.201329in}{2.438896in}}%
\pgfpathlineto{\pgfqpoint{2.201921in}{2.454051in}}%
\pgfpathlineto{\pgfqpoint{2.203204in}{2.430687in}}%
\pgfpathlineto{\pgfqpoint{2.203402in}{2.436009in}}%
\pgfpathlineto{\pgfqpoint{2.204191in}{2.450715in}}%
\pgfpathlineto{\pgfqpoint{2.204487in}{2.439851in}}%
\pgfpathlineto{\pgfqpoint{2.204586in}{2.437376in}}%
\pgfpathlineto{\pgfqpoint{2.204981in}{2.446050in}}%
\pgfpathlineto{\pgfqpoint{2.205474in}{2.442941in}}%
\pgfpathlineto{\pgfqpoint{2.205770in}{2.445596in}}%
\pgfpathlineto{\pgfqpoint{2.206264in}{2.439940in}}%
\pgfpathlineto{\pgfqpoint{2.207547in}{2.427020in}}%
\pgfpathlineto{\pgfqpoint{2.207744in}{2.432351in}}%
\pgfpathlineto{\pgfqpoint{2.208040in}{2.446128in}}%
\pgfpathlineto{\pgfqpoint{2.208435in}{2.425448in}}%
\pgfpathlineto{\pgfqpoint{2.208830in}{2.432393in}}%
\pgfpathlineto{\pgfqpoint{2.212186in}{2.310345in}}%
\pgfpathlineto{\pgfqpoint{2.212679in}{2.321793in}}%
\pgfpathlineto{\pgfqpoint{2.214160in}{2.382355in}}%
\pgfpathlineto{\pgfqpoint{2.214555in}{2.374273in}}%
\pgfpathlineto{\pgfqpoint{2.214653in}{2.373360in}}%
\pgfpathlineto{\pgfqpoint{2.214949in}{2.379802in}}%
\pgfpathlineto{\pgfqpoint{2.215739in}{2.389623in}}%
\pgfpathlineto{\pgfqpoint{2.216035in}{2.380135in}}%
\pgfpathlineto{\pgfqpoint{2.216331in}{2.370836in}}%
\pgfpathlineto{\pgfqpoint{2.217121in}{2.374541in}}%
\pgfpathlineto{\pgfqpoint{2.218305in}{2.403677in}}%
\pgfpathlineto{\pgfqpoint{2.218601in}{2.395628in}}%
\pgfpathlineto{\pgfqpoint{2.218700in}{2.393613in}}%
\pgfpathlineto{\pgfqpoint{2.219095in}{2.405386in}}%
\pgfpathlineto{\pgfqpoint{2.219786in}{2.415642in}}%
\pgfpathlineto{\pgfqpoint{2.220378in}{2.410609in}}%
\pgfpathlineto{\pgfqpoint{2.222845in}{2.359689in}}%
\pgfpathlineto{\pgfqpoint{2.220970in}{2.419890in}}%
\pgfpathlineto{\pgfqpoint{2.223043in}{2.364405in}}%
\pgfpathlineto{\pgfqpoint{2.223536in}{2.373475in}}%
\pgfpathlineto{\pgfqpoint{2.223832in}{2.358356in}}%
\pgfpathlineto{\pgfqpoint{2.224819in}{2.336193in}}%
\pgfpathlineto{\pgfqpoint{2.225115in}{2.341990in}}%
\pgfpathlineto{\pgfqpoint{2.225609in}{2.337300in}}%
\pgfpathlineto{\pgfqpoint{2.226300in}{2.371058in}}%
\pgfpathlineto{\pgfqpoint{2.226695in}{2.394524in}}%
\pgfpathlineto{\pgfqpoint{2.227385in}{2.372139in}}%
\pgfpathlineto{\pgfqpoint{2.228372in}{2.339652in}}%
\pgfpathlineto{\pgfqpoint{2.228668in}{2.345012in}}%
\pgfpathlineto{\pgfqpoint{2.229359in}{2.373684in}}%
\pgfpathlineto{\pgfqpoint{2.230149in}{2.445022in}}%
\pgfpathlineto{\pgfqpoint{2.231531in}{2.617211in}}%
\pgfpathlineto{\pgfqpoint{2.232123in}{2.582366in}}%
\pgfpathlineto{\pgfqpoint{2.232419in}{2.560822in}}%
\pgfpathlineto{\pgfqpoint{2.233209in}{2.348512in}}%
\pgfpathlineto{\pgfqpoint{2.234393in}{2.367875in}}%
\pgfpathlineto{\pgfqpoint{2.234492in}{2.367833in}}%
\pgfpathlineto{\pgfqpoint{2.235874in}{2.421925in}}%
\pgfpathlineto{\pgfqpoint{2.236170in}{2.448368in}}%
\pgfpathlineto{\pgfqpoint{2.236860in}{2.420274in}}%
\pgfpathlineto{\pgfqpoint{2.237354in}{2.389231in}}%
\pgfpathlineto{\pgfqpoint{2.238144in}{2.399570in}}%
\pgfpathlineto{\pgfqpoint{2.239723in}{2.443268in}}%
\pgfpathlineto{\pgfqpoint{2.239821in}{2.440901in}}%
\pgfpathlineto{\pgfqpoint{2.240611in}{2.412776in}}%
\pgfpathlineto{\pgfqpoint{2.241203in}{2.425637in}}%
\pgfpathlineto{\pgfqpoint{2.242388in}{2.442790in}}%
\pgfpathlineto{\pgfqpoint{2.242585in}{2.441969in}}%
\pgfpathlineto{\pgfqpoint{2.244263in}{2.452864in}}%
\pgfpathlineto{\pgfqpoint{2.244559in}{2.448854in}}%
\pgfpathlineto{\pgfqpoint{2.244855in}{2.457609in}}%
\pgfpathlineto{\pgfqpoint{2.245151in}{2.470431in}}%
\pgfpathlineto{\pgfqpoint{2.246039in}{2.469059in}}%
\pgfpathlineto{\pgfqpoint{2.246336in}{2.462085in}}%
\pgfpathlineto{\pgfqpoint{2.246730in}{2.479524in}}%
\pgfpathlineto{\pgfqpoint{2.246829in}{2.482067in}}%
\pgfpathlineto{\pgfqpoint{2.247224in}{2.473537in}}%
\pgfpathlineto{\pgfqpoint{2.247717in}{2.477155in}}%
\pgfpathlineto{\pgfqpoint{2.247915in}{2.473362in}}%
\pgfpathlineto{\pgfqpoint{2.248507in}{2.481131in}}%
\pgfpathlineto{\pgfqpoint{2.249593in}{2.495201in}}%
\pgfpathlineto{\pgfqpoint{2.249691in}{2.490720in}}%
\pgfpathlineto{\pgfqpoint{2.250086in}{2.468518in}}%
\pgfpathlineto{\pgfqpoint{2.250876in}{2.480534in}}%
\pgfpathlineto{\pgfqpoint{2.251468in}{2.481453in}}%
\pgfpathlineto{\pgfqpoint{2.251665in}{2.478424in}}%
\pgfpathlineto{\pgfqpoint{2.252652in}{2.464254in}}%
\pgfpathlineto{\pgfqpoint{2.252258in}{2.481814in}}%
\pgfpathlineto{\pgfqpoint{2.252850in}{2.470931in}}%
\pgfpathlineto{\pgfqpoint{2.253837in}{2.483084in}}%
\pgfpathlineto{\pgfqpoint{2.253935in}{2.478929in}}%
\pgfpathlineto{\pgfqpoint{2.255515in}{2.444416in}}%
\pgfpathlineto{\pgfqpoint{2.258772in}{2.416259in}}%
\pgfpathlineto{\pgfqpoint{2.258969in}{2.418940in}}%
\pgfpathlineto{\pgfqpoint{2.259166in}{2.423206in}}%
\pgfpathlineto{\pgfqpoint{2.259660in}{2.411633in}}%
\pgfpathlineto{\pgfqpoint{2.259857in}{2.413764in}}%
\pgfpathlineto{\pgfqpoint{2.260351in}{2.400480in}}%
\pgfpathlineto{\pgfqpoint{2.261042in}{2.411614in}}%
\pgfpathlineto{\pgfqpoint{2.261239in}{2.412852in}}%
\pgfpathlineto{\pgfqpoint{2.261535in}{2.407109in}}%
\pgfpathlineto{\pgfqpoint{2.261634in}{2.405436in}}%
\pgfpathlineto{\pgfqpoint{2.262226in}{2.413200in}}%
\pgfpathlineto{\pgfqpoint{2.262424in}{2.414328in}}%
\pgfpathlineto{\pgfqpoint{2.264694in}{2.452350in}}%
\pgfpathlineto{\pgfqpoint{2.265878in}{2.469377in}}%
\pgfpathlineto{\pgfqpoint{2.265384in}{2.451481in}}%
\pgfpathlineto{\pgfqpoint{2.265977in}{2.464853in}}%
\pgfpathlineto{\pgfqpoint{2.267753in}{2.409680in}}%
\pgfpathlineto{\pgfqpoint{2.267951in}{2.415986in}}%
\pgfpathlineto{\pgfqpoint{2.268444in}{2.398742in}}%
\pgfpathlineto{\pgfqpoint{2.268642in}{2.403926in}}%
\pgfpathlineto{\pgfqpoint{2.269332in}{2.414035in}}%
\pgfpathlineto{\pgfqpoint{2.268938in}{2.401803in}}%
\pgfpathlineto{\pgfqpoint{2.269530in}{2.408670in}}%
\pgfpathlineto{\pgfqpoint{2.271208in}{2.361061in}}%
\pgfpathlineto{\pgfqpoint{2.271405in}{2.356216in}}%
\pgfpathlineto{\pgfqpoint{2.271800in}{2.372775in}}%
\pgfpathlineto{\pgfqpoint{2.272491in}{2.410249in}}%
\pgfpathlineto{\pgfqpoint{2.273083in}{2.381706in}}%
\pgfpathlineto{\pgfqpoint{2.274662in}{2.338730in}}%
\pgfpathlineto{\pgfqpoint{2.274761in}{2.340832in}}%
\pgfpathlineto{\pgfqpoint{2.276044in}{2.462908in}}%
\pgfpathlineto{\pgfqpoint{2.277722in}{2.588795in}}%
\pgfpathlineto{\pgfqpoint{2.278018in}{2.576161in}}%
\pgfpathlineto{\pgfqpoint{2.278709in}{2.405514in}}%
\pgfpathlineto{\pgfqpoint{2.279005in}{2.342554in}}%
\pgfpathlineto{\pgfqpoint{2.279893in}{2.374419in}}%
\pgfpathlineto{\pgfqpoint{2.281571in}{2.420409in}}%
\pgfpathlineto{\pgfqpoint{2.282262in}{2.456704in}}%
\pgfpathlineto{\pgfqpoint{2.282657in}{2.431343in}}%
\pgfpathlineto{\pgfqpoint{2.283052in}{2.399547in}}%
\pgfpathlineto{\pgfqpoint{2.283742in}{2.421380in}}%
\pgfpathlineto{\pgfqpoint{2.285420in}{2.451209in}}%
\pgfpathlineto{\pgfqpoint{2.284433in}{2.417864in}}%
\pgfpathlineto{\pgfqpoint{2.285618in}{2.445082in}}%
\pgfpathlineto{\pgfqpoint{2.286901in}{2.414407in}}%
\pgfpathlineto{\pgfqpoint{2.287000in}{2.417131in}}%
\pgfpathlineto{\pgfqpoint{2.288776in}{2.461118in}}%
\pgfpathlineto{\pgfqpoint{2.288875in}{2.459551in}}%
\pgfpathlineto{\pgfqpoint{2.292033in}{2.389903in}}%
\pgfpathlineto{\pgfqpoint{2.292823in}{2.399435in}}%
\pgfpathlineto{\pgfqpoint{2.294797in}{2.492343in}}%
\pgfpathlineto{\pgfqpoint{2.294895in}{2.490509in}}%
\pgfpathlineto{\pgfqpoint{2.295784in}{2.493327in}}%
\pgfpathlineto{\pgfqpoint{2.296277in}{2.471363in}}%
\pgfpathlineto{\pgfqpoint{2.296475in}{2.474491in}}%
\pgfpathlineto{\pgfqpoint{2.296672in}{2.481581in}}%
\pgfpathlineto{\pgfqpoint{2.297166in}{2.462453in}}%
\pgfpathlineto{\pgfqpoint{2.297560in}{2.477612in}}%
\pgfpathlineto{\pgfqpoint{2.298745in}{2.467930in}}%
\pgfpathlineto{\pgfqpoint{2.298942in}{2.471491in}}%
\pgfpathlineto{\pgfqpoint{2.299041in}{2.473065in}}%
\pgfpathlineto{\pgfqpoint{2.299337in}{2.462549in}}%
\pgfpathlineto{\pgfqpoint{2.300423in}{2.443698in}}%
\pgfpathlineto{\pgfqpoint{2.300817in}{2.445670in}}%
\pgfpathlineto{\pgfqpoint{2.303877in}{2.420433in}}%
\pgfpathlineto{\pgfqpoint{2.303976in}{2.421922in}}%
\pgfpathlineto{\pgfqpoint{2.304272in}{2.429188in}}%
\pgfpathlineto{\pgfqpoint{2.305061in}{2.423187in}}%
\pgfpathlineto{\pgfqpoint{2.305851in}{2.417125in}}%
\pgfpathlineto{\pgfqpoint{2.306048in}{2.421129in}}%
\pgfpathlineto{\pgfqpoint{2.307332in}{2.432577in}}%
\pgfpathlineto{\pgfqpoint{2.307726in}{2.426647in}}%
\pgfpathlineto{\pgfqpoint{2.308121in}{2.434983in}}%
\pgfpathlineto{\pgfqpoint{2.308516in}{2.430124in}}%
\pgfpathlineto{\pgfqpoint{2.310391in}{2.464400in}}%
\pgfpathlineto{\pgfqpoint{2.311280in}{2.483280in}}%
\pgfpathlineto{\pgfqpoint{2.311477in}{2.485973in}}%
\pgfpathlineto{\pgfqpoint{2.311773in}{2.478443in}}%
\pgfpathlineto{\pgfqpoint{2.312168in}{2.484077in}}%
\pgfpathlineto{\pgfqpoint{2.314339in}{2.408782in}}%
\pgfpathlineto{\pgfqpoint{2.314635in}{2.412457in}}%
\pgfpathlineto{\pgfqpoint{2.315030in}{2.419788in}}%
\pgfpathlineto{\pgfqpoint{2.315524in}{2.413113in}}%
\pgfpathlineto{\pgfqpoint{2.317201in}{2.373475in}}%
\pgfpathlineto{\pgfqpoint{2.317300in}{2.373730in}}%
\pgfpathlineto{\pgfqpoint{2.318386in}{2.423564in}}%
\pgfpathlineto{\pgfqpoint{2.318879in}{2.401600in}}%
\pgfpathlineto{\pgfqpoint{2.320360in}{2.368719in}}%
\pgfpathlineto{\pgfqpoint{2.320557in}{2.373820in}}%
\pgfpathlineto{\pgfqpoint{2.321347in}{2.416865in}}%
\pgfpathlineto{\pgfqpoint{2.323123in}{2.618467in}}%
\pgfpathlineto{\pgfqpoint{2.324110in}{2.581573in}}%
\pgfpathlineto{\pgfqpoint{2.324999in}{2.370121in}}%
\pgfpathlineto{\pgfqpoint{2.325986in}{2.405065in}}%
\pgfpathlineto{\pgfqpoint{2.328157in}{2.493767in}}%
\pgfpathlineto{\pgfqpoint{2.328354in}{2.482300in}}%
\pgfpathlineto{\pgfqpoint{2.329243in}{2.432086in}}%
\pgfpathlineto{\pgfqpoint{2.329736in}{2.455898in}}%
\pgfpathlineto{\pgfqpoint{2.331513in}{2.480886in}}%
\pgfpathlineto{\pgfqpoint{2.331611in}{2.479277in}}%
\pgfpathlineto{\pgfqpoint{2.332105in}{2.458017in}}%
\pgfpathlineto{\pgfqpoint{2.332993in}{2.465074in}}%
\pgfpathlineto{\pgfqpoint{2.333191in}{2.467974in}}%
\pgfpathlineto{\pgfqpoint{2.334671in}{2.487802in}}%
\pgfpathlineto{\pgfqpoint{2.334967in}{2.495262in}}%
\pgfpathlineto{\pgfqpoint{2.335461in}{2.480249in}}%
\pgfpathlineto{\pgfqpoint{2.335658in}{2.484913in}}%
\pgfpathlineto{\pgfqpoint{2.335757in}{2.486726in}}%
\pgfpathlineto{\pgfqpoint{2.336448in}{2.481794in}}%
\pgfpathlineto{\pgfqpoint{2.336744in}{2.477844in}}%
\pgfpathlineto{\pgfqpoint{2.337040in}{2.486562in}}%
\pgfpathlineto{\pgfqpoint{2.338224in}{2.496531in}}%
\pgfpathlineto{\pgfqpoint{2.338323in}{2.493637in}}%
\pgfpathlineto{\pgfqpoint{2.338619in}{2.482343in}}%
\pgfpathlineto{\pgfqpoint{2.339113in}{2.497433in}}%
\pgfpathlineto{\pgfqpoint{2.339606in}{2.484450in}}%
\pgfpathlineto{\pgfqpoint{2.339705in}{2.484202in}}%
\pgfpathlineto{\pgfqpoint{2.339803in}{2.485740in}}%
\pgfpathlineto{\pgfqpoint{2.340790in}{2.493005in}}%
\pgfpathlineto{\pgfqpoint{2.340988in}{2.488122in}}%
\pgfpathlineto{\pgfqpoint{2.342172in}{2.479848in}}%
\pgfpathlineto{\pgfqpoint{2.341481in}{2.491502in}}%
\pgfpathlineto{\pgfqpoint{2.342271in}{2.481310in}}%
\pgfpathlineto{\pgfqpoint{2.342567in}{2.490147in}}%
\pgfpathlineto{\pgfqpoint{2.342962in}{2.470560in}}%
\pgfpathlineto{\pgfqpoint{2.343455in}{2.487074in}}%
\pgfpathlineto{\pgfqpoint{2.345035in}{2.480230in}}%
\pgfpathlineto{\pgfqpoint{2.345627in}{2.471872in}}%
\pgfpathlineto{\pgfqpoint{2.348489in}{2.426296in}}%
\pgfpathlineto{\pgfqpoint{2.348588in}{2.426306in}}%
\pgfpathlineto{\pgfqpoint{2.348785in}{2.429507in}}%
\pgfpathlineto{\pgfqpoint{2.349081in}{2.414965in}}%
\pgfpathlineto{\pgfqpoint{2.350167in}{2.405731in}}%
\pgfpathlineto{\pgfqpoint{2.349673in}{2.420258in}}%
\pgfpathlineto{\pgfqpoint{2.350266in}{2.406743in}}%
\pgfpathlineto{\pgfqpoint{2.350364in}{2.407811in}}%
\pgfpathlineto{\pgfqpoint{2.350759in}{2.400578in}}%
\pgfpathlineto{\pgfqpoint{2.351647in}{2.391597in}}%
\pgfpathlineto{\pgfqpoint{2.351154in}{2.403735in}}%
\pgfpathlineto{\pgfqpoint{2.351943in}{2.396955in}}%
\pgfpathlineto{\pgfqpoint{2.352338in}{2.405125in}}%
\pgfpathlineto{\pgfqpoint{2.352832in}{2.393833in}}%
\pgfpathlineto{\pgfqpoint{2.353128in}{2.389238in}}%
\pgfpathlineto{\pgfqpoint{2.353720in}{2.397986in}}%
\pgfpathlineto{\pgfqpoint{2.353917in}{2.394428in}}%
\pgfpathlineto{\pgfqpoint{2.354214in}{2.391982in}}%
\pgfpathlineto{\pgfqpoint{2.354608in}{2.397711in}}%
\pgfpathlineto{\pgfqpoint{2.355299in}{2.395445in}}%
\pgfpathlineto{\pgfqpoint{2.356582in}{2.436133in}}%
\pgfpathlineto{\pgfqpoint{2.357668in}{2.456742in}}%
\pgfpathlineto{\pgfqpoint{2.358655in}{2.454323in}}%
\pgfpathlineto{\pgfqpoint{2.359247in}{2.438834in}}%
\pgfpathlineto{\pgfqpoint{2.359938in}{2.413109in}}%
\pgfpathlineto{\pgfqpoint{2.361122in}{2.399436in}}%
\pgfpathlineto{\pgfqpoint{2.361419in}{2.404945in}}%
\pgfpathlineto{\pgfqpoint{2.361715in}{2.393283in}}%
\pgfpathlineto{\pgfqpoint{2.363195in}{2.365151in}}%
\pgfpathlineto{\pgfqpoint{2.363294in}{2.367426in}}%
\pgfpathlineto{\pgfqpoint{2.364577in}{2.425844in}}%
\pgfpathlineto{\pgfqpoint{2.364972in}{2.404832in}}%
\pgfpathlineto{\pgfqpoint{2.366551in}{2.366963in}}%
\pgfpathlineto{\pgfqpoint{2.367538in}{2.415101in}}%
\pgfpathlineto{\pgfqpoint{2.368920in}{2.618220in}}%
\pgfpathlineto{\pgfqpoint{2.370005in}{2.590139in}}%
\pgfpathlineto{\pgfqpoint{2.370499in}{2.500268in}}%
\pgfpathlineto{\pgfqpoint{2.370992in}{2.364367in}}%
\pgfpathlineto{\pgfqpoint{2.371782in}{2.407640in}}%
\pgfpathlineto{\pgfqpoint{2.373262in}{2.380220in}}%
\pgfpathlineto{\pgfqpoint{2.373559in}{2.388877in}}%
\pgfpathlineto{\pgfqpoint{2.374052in}{2.412209in}}%
\pgfpathlineto{\pgfqpoint{2.374546in}{2.383212in}}%
\pgfpathlineto{\pgfqpoint{2.375236in}{2.337986in}}%
\pgfpathlineto{\pgfqpoint{2.375730in}{2.365327in}}%
\pgfpathlineto{\pgfqpoint{2.377704in}{2.467850in}}%
\pgfpathlineto{\pgfqpoint{2.377901in}{2.461995in}}%
\pgfpathlineto{\pgfqpoint{2.378296in}{2.433965in}}%
\pgfpathlineto{\pgfqpoint{2.379283in}{2.437154in}}%
\pgfpathlineto{\pgfqpoint{2.379382in}{2.435589in}}%
\pgfpathlineto{\pgfqpoint{2.379678in}{2.446740in}}%
\pgfpathlineto{\pgfqpoint{2.381257in}{2.480552in}}%
\pgfpathlineto{\pgfqpoint{2.381356in}{2.479798in}}%
\pgfpathlineto{\pgfqpoint{2.381652in}{2.468051in}}%
\pgfpathlineto{\pgfqpoint{2.382540in}{2.472055in}}%
\pgfpathlineto{\pgfqpoint{2.383527in}{2.476569in}}%
\pgfpathlineto{\pgfqpoint{2.383132in}{2.465127in}}%
\pgfpathlineto{\pgfqpoint{2.383725in}{2.474769in}}%
\pgfpathlineto{\pgfqpoint{2.384218in}{2.494082in}}%
\pgfpathlineto{\pgfqpoint{2.385402in}{2.486399in}}%
\pgfpathlineto{\pgfqpoint{2.385600in}{2.489419in}}%
\pgfpathlineto{\pgfqpoint{2.386389in}{2.485893in}}%
\pgfpathlineto{\pgfqpoint{2.386587in}{2.485577in}}%
\pgfpathlineto{\pgfqpoint{2.386784in}{2.486915in}}%
\pgfpathlineto{\pgfqpoint{2.386883in}{2.487276in}}%
\pgfpathlineto{\pgfqpoint{2.387080in}{2.484533in}}%
\pgfpathlineto{\pgfqpoint{2.387475in}{2.471745in}}%
\pgfpathlineto{\pgfqpoint{2.387969in}{2.486017in}}%
\pgfpathlineto{\pgfqpoint{2.388462in}{2.472334in}}%
\pgfpathlineto{\pgfqpoint{2.388758in}{2.481323in}}%
\pgfpathlineto{\pgfqpoint{2.388956in}{2.486594in}}%
\pgfpathlineto{\pgfqpoint{2.389350in}{2.476073in}}%
\pgfpathlineto{\pgfqpoint{2.389844in}{2.483084in}}%
\pgfpathlineto{\pgfqpoint{2.390732in}{2.476367in}}%
\pgfpathlineto{\pgfqpoint{2.390337in}{2.494062in}}%
\pgfpathlineto{\pgfqpoint{2.390930in}{2.481282in}}%
\pgfpathlineto{\pgfqpoint{2.391127in}{2.488835in}}%
\pgfpathlineto{\pgfqpoint{2.391522in}{2.477718in}}%
\pgfpathlineto{\pgfqpoint{2.391917in}{2.482241in}}%
\pgfpathlineto{\pgfqpoint{2.393397in}{2.453143in}}%
\pgfpathlineto{\pgfqpoint{2.393693in}{2.462066in}}%
\pgfpathlineto{\pgfqpoint{2.393792in}{2.463118in}}%
\pgfpathlineto{\pgfqpoint{2.393989in}{2.455876in}}%
\pgfpathlineto{\pgfqpoint{2.395174in}{2.428423in}}%
\pgfpathlineto{\pgfqpoint{2.394581in}{2.457467in}}%
\pgfpathlineto{\pgfqpoint{2.395470in}{2.433591in}}%
\pgfpathlineto{\pgfqpoint{2.396654in}{2.412886in}}%
\pgfpathlineto{\pgfqpoint{2.397345in}{2.416979in}}%
\pgfpathlineto{\pgfqpoint{2.397937in}{2.394667in}}%
\pgfpathlineto{\pgfqpoint{2.398233in}{2.401617in}}%
\pgfpathlineto{\pgfqpoint{2.398924in}{2.391967in}}%
\pgfpathlineto{\pgfqpoint{2.399220in}{2.391881in}}%
\pgfpathlineto{\pgfqpoint{2.399319in}{2.392829in}}%
\pgfpathlineto{\pgfqpoint{2.400010in}{2.406748in}}%
\pgfpathlineto{\pgfqpoint{2.400306in}{2.396892in}}%
\pgfpathlineto{\pgfqpoint{2.400602in}{2.384504in}}%
\pgfpathlineto{\pgfqpoint{2.400997in}{2.398013in}}%
\pgfpathlineto{\pgfqpoint{2.401392in}{2.396112in}}%
\pgfpathlineto{\pgfqpoint{2.403958in}{2.442332in}}%
\pgfpathlineto{\pgfqpoint{2.404254in}{2.445283in}}%
\pgfpathlineto{\pgfqpoint{2.404451in}{2.439049in}}%
\pgfpathlineto{\pgfqpoint{2.406228in}{2.386349in}}%
\pgfpathlineto{\pgfqpoint{2.406327in}{2.389562in}}%
\pgfpathlineto{\pgfqpoint{2.407314in}{2.405965in}}%
\pgfpathlineto{\pgfqpoint{2.406919in}{2.387462in}}%
\pgfpathlineto{\pgfqpoint{2.407511in}{2.397160in}}%
\pgfpathlineto{\pgfqpoint{2.408399in}{2.365449in}}%
\pgfpathlineto{\pgfqpoint{2.408893in}{2.371601in}}%
\pgfpathlineto{\pgfqpoint{2.409090in}{2.369459in}}%
\pgfpathlineto{\pgfqpoint{2.409386in}{2.373491in}}%
\pgfpathlineto{\pgfqpoint{2.409682in}{2.373336in}}%
\pgfpathlineto{\pgfqpoint{2.410669in}{2.428766in}}%
\pgfpathlineto{\pgfqpoint{2.411163in}{2.399885in}}%
\pgfpathlineto{\pgfqpoint{2.412051in}{2.389689in}}%
\pgfpathlineto{\pgfqpoint{2.412347in}{2.393888in}}%
\pgfpathlineto{\pgfqpoint{2.412446in}{2.393959in}}%
\pgfpathlineto{\pgfqpoint{2.412742in}{2.389231in}}%
\pgfpathlineto{\pgfqpoint{2.413038in}{2.405748in}}%
\pgfpathlineto{\pgfqpoint{2.414321in}{2.555758in}}%
\pgfpathlineto{\pgfqpoint{2.414815in}{2.620012in}}%
\pgfpathlineto{\pgfqpoint{2.415604in}{2.614744in}}%
\pgfpathlineto{\pgfqpoint{2.416196in}{2.588844in}}%
\pgfpathlineto{\pgfqpoint{2.417085in}{2.371028in}}%
\pgfpathlineto{\pgfqpoint{2.418269in}{2.411072in}}%
\pgfpathlineto{\pgfqpoint{2.419947in}{2.466550in}}%
\pgfpathlineto{\pgfqpoint{2.420243in}{2.476473in}}%
\pgfpathlineto{\pgfqpoint{2.420737in}{2.454402in}}%
\pgfpathlineto{\pgfqpoint{2.421329in}{2.417793in}}%
\pgfpathlineto{\pgfqpoint{2.422118in}{2.437699in}}%
\pgfpathlineto{\pgfqpoint{2.422217in}{2.437770in}}%
\pgfpathlineto{\pgfqpoint{2.423599in}{2.472230in}}%
\pgfpathlineto{\pgfqpoint{2.423994in}{2.455079in}}%
\pgfpathlineto{\pgfqpoint{2.424685in}{2.444091in}}%
\pgfpathlineto{\pgfqpoint{2.425079in}{2.455753in}}%
\pgfpathlineto{\pgfqpoint{2.426461in}{2.484252in}}%
\pgfpathlineto{\pgfqpoint{2.425770in}{2.454325in}}%
\pgfpathlineto{\pgfqpoint{2.426560in}{2.483079in}}%
\pgfpathlineto{\pgfqpoint{2.426856in}{2.479044in}}%
\pgfpathlineto{\pgfqpoint{2.427251in}{2.484841in}}%
\pgfpathlineto{\pgfqpoint{2.427547in}{2.483966in}}%
\pgfpathlineto{\pgfqpoint{2.428929in}{2.501556in}}%
\pgfpathlineto{\pgfqpoint{2.429916in}{2.513569in}}%
\pgfpathlineto{\pgfqpoint{2.430212in}{2.510415in}}%
\pgfpathlineto{\pgfqpoint{2.430310in}{2.510226in}}%
\pgfpathlineto{\pgfqpoint{2.430409in}{2.510739in}}%
\pgfpathlineto{\pgfqpoint{2.431594in}{2.521548in}}%
\pgfpathlineto{\pgfqpoint{2.431791in}{2.517630in}}%
\pgfpathlineto{\pgfqpoint{2.433271in}{2.498441in}}%
\pgfpathlineto{\pgfqpoint{2.433666in}{2.516092in}}%
\pgfpathlineto{\pgfqpoint{2.434752in}{2.509847in}}%
\pgfpathlineto{\pgfqpoint{2.434949in}{2.508388in}}%
\pgfpathlineto{\pgfqpoint{2.435443in}{2.511419in}}%
\pgfpathlineto{\pgfqpoint{2.435739in}{2.519306in}}%
\pgfpathlineto{\pgfqpoint{2.436035in}{2.502417in}}%
\pgfpathlineto{\pgfqpoint{2.436232in}{2.497112in}}%
\pgfpathlineto{\pgfqpoint{2.436627in}{2.514816in}}%
\pgfpathlineto{\pgfqpoint{2.437022in}{2.501172in}}%
\pgfpathlineto{\pgfqpoint{2.437318in}{2.514423in}}%
\pgfpathlineto{\pgfqpoint{2.437910in}{2.495477in}}%
\pgfpathlineto{\pgfqpoint{2.438404in}{2.487963in}}%
\pgfpathlineto{\pgfqpoint{2.439095in}{2.493761in}}%
\pgfpathlineto{\pgfqpoint{2.439489in}{2.489536in}}%
\pgfpathlineto{\pgfqpoint{2.441069in}{2.459657in}}%
\pgfpathlineto{\pgfqpoint{2.441463in}{2.467410in}}%
\pgfpathlineto{\pgfqpoint{2.441562in}{2.470297in}}%
\pgfpathlineto{\pgfqpoint{2.441957in}{2.452336in}}%
\pgfpathlineto{\pgfqpoint{2.442056in}{2.448571in}}%
\pgfpathlineto{\pgfqpoint{2.442747in}{2.457116in}}%
\pgfpathlineto{\pgfqpoint{2.443043in}{2.468189in}}%
\pgfpathlineto{\pgfqpoint{2.443437in}{2.453259in}}%
\pgfpathlineto{\pgfqpoint{2.443931in}{2.466521in}}%
\pgfpathlineto{\pgfqpoint{2.444918in}{2.451159in}}%
\pgfpathlineto{\pgfqpoint{2.445214in}{2.457772in}}%
\pgfpathlineto{\pgfqpoint{2.445313in}{2.459324in}}%
\pgfpathlineto{\pgfqpoint{2.445707in}{2.448544in}}%
\pgfpathlineto{\pgfqpoint{2.445806in}{2.448660in}}%
\pgfpathlineto{\pgfqpoint{2.447089in}{2.473498in}}%
\pgfpathlineto{\pgfqpoint{2.446694in}{2.444048in}}%
\pgfpathlineto{\pgfqpoint{2.447287in}{2.467649in}}%
\pgfpathlineto{\pgfqpoint{2.447484in}{2.458217in}}%
\pgfpathlineto{\pgfqpoint{2.447978in}{2.480583in}}%
\pgfpathlineto{\pgfqpoint{2.448274in}{2.474352in}}%
\pgfpathlineto{\pgfqpoint{2.450445in}{2.520960in}}%
\pgfpathlineto{\pgfqpoint{2.450939in}{2.509954in}}%
\pgfpathlineto{\pgfqpoint{2.453702in}{2.439547in}}%
\pgfpathlineto{\pgfqpoint{2.455676in}{2.367057in}}%
\pgfpathlineto{\pgfqpoint{2.455775in}{2.368126in}}%
\pgfpathlineto{\pgfqpoint{2.456268in}{2.422190in}}%
\pgfpathlineto{\pgfqpoint{2.457058in}{2.489591in}}%
\pgfpathlineto{\pgfqpoint{2.457650in}{2.462531in}}%
\pgfpathlineto{\pgfqpoint{2.457847in}{2.458950in}}%
\pgfpathlineto{\pgfqpoint{2.458242in}{2.476284in}}%
\pgfpathlineto{\pgfqpoint{2.458341in}{2.476594in}}%
\pgfpathlineto{\pgfqpoint{2.459229in}{2.449738in}}%
\pgfpathlineto{\pgfqpoint{2.459525in}{2.473861in}}%
\pgfpathlineto{\pgfqpoint{2.461499in}{2.708483in}}%
\pgfpathlineto{\pgfqpoint{2.462190in}{2.690801in}}%
\pgfpathlineto{\pgfqpoint{2.462486in}{2.671850in}}%
\pgfpathlineto{\pgfqpoint{2.463375in}{2.427804in}}%
\pgfpathlineto{\pgfqpoint{2.464559in}{2.459263in}}%
\pgfpathlineto{\pgfqpoint{2.466434in}{2.529169in}}%
\pgfpathlineto{\pgfqpoint{2.466730in}{2.516946in}}%
\pgfpathlineto{\pgfqpoint{2.467816in}{2.472959in}}%
\pgfpathlineto{\pgfqpoint{2.468112in}{2.488899in}}%
\pgfpathlineto{\pgfqpoint{2.468606in}{2.487538in}}%
\pgfpathlineto{\pgfqpoint{2.469593in}{2.523053in}}%
\pgfpathlineto{\pgfqpoint{2.469790in}{2.530193in}}%
\pgfpathlineto{\pgfqpoint{2.470382in}{2.505682in}}%
\pgfpathlineto{\pgfqpoint{2.470580in}{2.501522in}}%
\pgfpathlineto{\pgfqpoint{2.471073in}{2.512486in}}%
\pgfpathlineto{\pgfqpoint{2.471369in}{2.508730in}}%
\pgfpathlineto{\pgfqpoint{2.473047in}{2.542809in}}%
\pgfpathlineto{\pgfqpoint{2.473343in}{2.532882in}}%
\pgfpathlineto{\pgfqpoint{2.473442in}{2.532678in}}%
\pgfpathlineto{\pgfqpoint{2.473738in}{2.541788in}}%
\pgfpathlineto{\pgfqpoint{2.474133in}{2.532246in}}%
\pgfpathlineto{\pgfqpoint{2.474626in}{2.537149in}}%
\pgfpathlineto{\pgfqpoint{2.474725in}{2.536259in}}%
\pgfpathlineto{\pgfqpoint{2.474922in}{2.542293in}}%
\pgfpathlineto{\pgfqpoint{2.475811in}{2.551082in}}%
\pgfpathlineto{\pgfqpoint{2.476107in}{2.547766in}}%
\pgfpathlineto{\pgfqpoint{2.476403in}{2.548550in}}%
\pgfpathlineto{\pgfqpoint{2.476699in}{2.544477in}}%
\pgfpathlineto{\pgfqpoint{2.476995in}{2.538625in}}%
\pgfpathlineto{\pgfqpoint{2.477489in}{2.552895in}}%
\pgfpathlineto{\pgfqpoint{2.477587in}{2.553260in}}%
\pgfpathlineto{\pgfqpoint{2.477686in}{2.551218in}}%
\pgfpathlineto{\pgfqpoint{2.478772in}{2.530214in}}%
\pgfpathlineto{\pgfqpoint{2.479166in}{2.537339in}}%
\pgfpathlineto{\pgfqpoint{2.480153in}{2.545700in}}%
\pgfpathlineto{\pgfqpoint{2.479759in}{2.534498in}}%
\pgfpathlineto{\pgfqpoint{2.480351in}{2.539591in}}%
\pgfpathlineto{\pgfqpoint{2.481239in}{2.524800in}}%
\pgfpathlineto{\pgfqpoint{2.480844in}{2.542002in}}%
\pgfpathlineto{\pgfqpoint{2.481535in}{2.534560in}}%
\pgfpathlineto{\pgfqpoint{2.481634in}{2.535826in}}%
\pgfpathlineto{\pgfqpoint{2.482029in}{2.526858in}}%
\pgfpathlineto{\pgfqpoint{2.482226in}{2.529317in}}%
\pgfpathlineto{\pgfqpoint{2.482325in}{2.529774in}}%
\pgfpathlineto{\pgfqpoint{2.482522in}{2.526943in}}%
\pgfpathlineto{\pgfqpoint{2.484397in}{2.501265in}}%
\pgfpathlineto{\pgfqpoint{2.483213in}{2.530572in}}%
\pgfpathlineto{\pgfqpoint{2.484694in}{2.508290in}}%
\pgfpathlineto{\pgfqpoint{2.484792in}{2.510378in}}%
\pgfpathlineto{\pgfqpoint{2.485088in}{2.498948in}}%
\pgfpathlineto{\pgfqpoint{2.486766in}{2.462828in}}%
\pgfpathlineto{\pgfqpoint{2.487161in}{2.473519in}}%
\pgfpathlineto{\pgfqpoint{2.487457in}{2.459771in}}%
\pgfpathlineto{\pgfqpoint{2.488148in}{2.462946in}}%
\pgfpathlineto{\pgfqpoint{2.488938in}{2.438772in}}%
\pgfpathlineto{\pgfqpoint{2.489332in}{2.446411in}}%
\pgfpathlineto{\pgfqpoint{2.490319in}{2.443252in}}%
\pgfpathlineto{\pgfqpoint{2.491405in}{2.425632in}}%
\pgfpathlineto{\pgfqpoint{2.491602in}{2.431224in}}%
\pgfpathlineto{\pgfqpoint{2.492589in}{2.448722in}}%
\pgfpathlineto{\pgfqpoint{2.492787in}{2.441507in}}%
\pgfpathlineto{\pgfqpoint{2.492984in}{2.436552in}}%
\pgfpathlineto{\pgfqpoint{2.493379in}{2.442494in}}%
\pgfpathlineto{\pgfqpoint{2.493774in}{2.440643in}}%
\pgfpathlineto{\pgfqpoint{2.495847in}{2.492615in}}%
\pgfpathlineto{\pgfqpoint{2.496044in}{2.490106in}}%
\pgfpathlineto{\pgfqpoint{2.496340in}{2.485645in}}%
\pgfpathlineto{\pgfqpoint{2.496834in}{2.495683in}}%
\pgfpathlineto{\pgfqpoint{2.497130in}{2.488353in}}%
\pgfpathlineto{\pgfqpoint{2.499202in}{2.425588in}}%
\pgfpathlineto{\pgfqpoint{2.499597in}{2.435567in}}%
\pgfpathlineto{\pgfqpoint{2.500091in}{2.445417in}}%
\pgfpathlineto{\pgfqpoint{2.500288in}{2.432939in}}%
\pgfpathlineto{\pgfqpoint{2.501176in}{2.405408in}}%
\pgfpathlineto{\pgfqpoint{2.501571in}{2.408493in}}%
\pgfpathlineto{\pgfqpoint{2.502065in}{2.392588in}}%
\pgfpathlineto{\pgfqpoint{2.502558in}{2.408192in}}%
\pgfpathlineto{\pgfqpoint{2.503249in}{2.439313in}}%
\pgfpathlineto{\pgfqpoint{2.503742in}{2.413015in}}%
\pgfpathlineto{\pgfqpoint{2.505223in}{2.374354in}}%
\pgfpathlineto{\pgfqpoint{2.505519in}{2.379802in}}%
\pgfpathlineto{\pgfqpoint{2.506605in}{2.469548in}}%
\pgfpathlineto{\pgfqpoint{2.507690in}{2.626911in}}%
\pgfpathlineto{\pgfqpoint{2.508184in}{2.609507in}}%
\pgfpathlineto{\pgfqpoint{2.508579in}{2.614803in}}%
\pgfpathlineto{\pgfqpoint{2.508776in}{2.608807in}}%
\pgfpathlineto{\pgfqpoint{2.509467in}{2.407949in}}%
\pgfpathlineto{\pgfqpoint{2.509763in}{2.365315in}}%
\pgfpathlineto{\pgfqpoint{2.510553in}{2.400779in}}%
\pgfpathlineto{\pgfqpoint{2.510651in}{2.398936in}}%
\pgfpathlineto{\pgfqpoint{2.510849in}{2.407078in}}%
\pgfpathlineto{\pgfqpoint{2.511244in}{2.432709in}}%
\pgfpathlineto{\pgfqpoint{2.512132in}{2.426550in}}%
\pgfpathlineto{\pgfqpoint{2.512823in}{2.484561in}}%
\pgfpathlineto{\pgfqpoint{2.513415in}{2.445040in}}%
\pgfpathlineto{\pgfqpoint{2.514007in}{2.417033in}}%
\pgfpathlineto{\pgfqpoint{2.514599in}{2.439279in}}%
\pgfpathlineto{\pgfqpoint{2.515389in}{2.453949in}}%
\pgfpathlineto{\pgfqpoint{2.516179in}{2.483947in}}%
\pgfpathlineto{\pgfqpoint{2.516672in}{2.467844in}}%
\pgfpathlineto{\pgfqpoint{2.517659in}{2.454371in}}%
\pgfpathlineto{\pgfqpoint{2.517856in}{2.462146in}}%
\pgfpathlineto{\pgfqpoint{2.518152in}{2.475796in}}%
\pgfpathlineto{\pgfqpoint{2.519041in}{2.470157in}}%
\pgfpathlineto{\pgfqpoint{2.520028in}{2.475537in}}%
\pgfpathlineto{\pgfqpoint{2.519534in}{2.462000in}}%
\pgfpathlineto{\pgfqpoint{2.520126in}{2.473969in}}%
\pgfpathlineto{\pgfqpoint{2.520324in}{2.469773in}}%
\pgfpathlineto{\pgfqpoint{2.520719in}{2.481547in}}%
\pgfpathlineto{\pgfqpoint{2.521212in}{2.473874in}}%
\pgfpathlineto{\pgfqpoint{2.521607in}{2.487335in}}%
\pgfpathlineto{\pgfqpoint{2.522100in}{2.472003in}}%
\pgfpathlineto{\pgfqpoint{2.522397in}{2.477121in}}%
\pgfpathlineto{\pgfqpoint{2.523285in}{2.488417in}}%
\pgfpathlineto{\pgfqpoint{2.523581in}{2.484027in}}%
\pgfpathlineto{\pgfqpoint{2.524765in}{2.468573in}}%
\pgfpathlineto{\pgfqpoint{2.524272in}{2.488802in}}%
\pgfpathlineto{\pgfqpoint{2.524864in}{2.472849in}}%
\pgfpathlineto{\pgfqpoint{2.525160in}{2.486102in}}%
\pgfpathlineto{\pgfqpoint{2.525950in}{2.472087in}}%
\pgfpathlineto{\pgfqpoint{2.526246in}{2.473289in}}%
\pgfpathlineto{\pgfqpoint{2.526443in}{2.471086in}}%
\pgfpathlineto{\pgfqpoint{2.526838in}{2.458046in}}%
\pgfpathlineto{\pgfqpoint{2.527332in}{2.473638in}}%
\pgfpathlineto{\pgfqpoint{2.527529in}{2.470300in}}%
\pgfpathlineto{\pgfqpoint{2.527628in}{2.469166in}}%
\pgfpathlineto{\pgfqpoint{2.527924in}{2.474993in}}%
\pgfpathlineto{\pgfqpoint{2.528220in}{2.481861in}}%
\pgfpathlineto{\pgfqpoint{2.529009in}{2.480264in}}%
\pgfpathlineto{\pgfqpoint{2.530095in}{2.467458in}}%
\pgfpathlineto{\pgfqpoint{2.530292in}{2.470716in}}%
\pgfpathlineto{\pgfqpoint{2.530490in}{2.471947in}}%
\pgfpathlineto{\pgfqpoint{2.530983in}{2.467706in}}%
\pgfpathlineto{\pgfqpoint{2.533846in}{2.361737in}}%
\pgfpathlineto{\pgfqpoint{2.534240in}{2.343602in}}%
\pgfpathlineto{\pgfqpoint{2.535030in}{2.355382in}}%
\pgfpathlineto{\pgfqpoint{2.537399in}{2.433030in}}%
\pgfpathlineto{\pgfqpoint{2.537596in}{2.426728in}}%
\pgfpathlineto{\pgfqpoint{2.537991in}{2.403234in}}%
\pgfpathlineto{\pgfqpoint{2.538781in}{2.416642in}}%
\pgfpathlineto{\pgfqpoint{2.539274in}{2.405285in}}%
\pgfpathlineto{\pgfqpoint{2.539965in}{2.409588in}}%
\pgfpathlineto{\pgfqpoint{2.542334in}{2.454471in}}%
\pgfpathlineto{\pgfqpoint{2.542531in}{2.449692in}}%
\pgfpathlineto{\pgfqpoint{2.543617in}{2.437989in}}%
\pgfpathlineto{\pgfqpoint{2.543025in}{2.454595in}}%
\pgfpathlineto{\pgfqpoint{2.543814in}{2.441064in}}%
\pgfpathlineto{\pgfqpoint{2.544110in}{2.452589in}}%
\pgfpathlineto{\pgfqpoint{2.544703in}{2.431724in}}%
\pgfpathlineto{\pgfqpoint{2.546578in}{2.393377in}}%
\pgfpathlineto{\pgfqpoint{2.548058in}{2.371845in}}%
\pgfpathlineto{\pgfqpoint{2.548256in}{2.374169in}}%
\pgfpathlineto{\pgfqpoint{2.548947in}{2.407788in}}%
\pgfpathlineto{\pgfqpoint{2.549341in}{2.434077in}}%
\pgfpathlineto{\pgfqpoint{2.549934in}{2.408769in}}%
\pgfpathlineto{\pgfqpoint{2.550427in}{2.387356in}}%
\pgfpathlineto{\pgfqpoint{2.551217in}{2.394663in}}%
\pgfpathlineto{\pgfqpoint{2.551611in}{2.386310in}}%
\pgfpathlineto{\pgfqpoint{2.551809in}{2.390691in}}%
\pgfpathlineto{\pgfqpoint{2.553191in}{2.570073in}}%
\pgfpathlineto{\pgfqpoint{2.553585in}{2.624060in}}%
\pgfpathlineto{\pgfqpoint{2.554375in}{2.610665in}}%
\pgfpathlineto{\pgfqpoint{2.554869in}{2.615662in}}%
\pgfpathlineto{\pgfqpoint{2.555461in}{2.480109in}}%
\pgfpathlineto{\pgfqpoint{2.555954in}{2.345489in}}%
\pgfpathlineto{\pgfqpoint{2.556842in}{2.383646in}}%
\pgfpathlineto{\pgfqpoint{2.556941in}{2.382944in}}%
\pgfpathlineto{\pgfqpoint{2.557040in}{2.385880in}}%
\pgfpathlineto{\pgfqpoint{2.559014in}{2.463457in}}%
\pgfpathlineto{\pgfqpoint{2.559310in}{2.455962in}}%
\pgfpathlineto{\pgfqpoint{2.560198in}{2.408169in}}%
\pgfpathlineto{\pgfqpoint{2.560692in}{2.435208in}}%
\pgfpathlineto{\pgfqpoint{2.562370in}{2.475973in}}%
\pgfpathlineto{\pgfqpoint{2.561185in}{2.432752in}}%
\pgfpathlineto{\pgfqpoint{2.562666in}{2.460941in}}%
\pgfpathlineto{\pgfqpoint{2.563455in}{2.439811in}}%
\pgfpathlineto{\pgfqpoint{2.563751in}{2.449967in}}%
\pgfpathlineto{\pgfqpoint{2.564738in}{2.469103in}}%
\pgfpathlineto{\pgfqpoint{2.565133in}{2.466856in}}%
\pgfpathlineto{\pgfqpoint{2.565528in}{2.461222in}}%
\pgfpathlineto{\pgfqpoint{2.566021in}{2.467980in}}%
\pgfpathlineto{\pgfqpoint{2.566416in}{2.479698in}}%
\pgfpathlineto{\pgfqpoint{2.566811in}{2.460395in}}%
\pgfpathlineto{\pgfqpoint{2.567008in}{2.455733in}}%
\pgfpathlineto{\pgfqpoint{2.567502in}{2.476893in}}%
\pgfpathlineto{\pgfqpoint{2.567995in}{2.454762in}}%
\pgfpathlineto{\pgfqpoint{2.568785in}{2.469614in}}%
\pgfpathlineto{\pgfqpoint{2.568982in}{2.472337in}}%
\pgfpathlineto{\pgfqpoint{2.569377in}{2.465012in}}%
\pgfpathlineto{\pgfqpoint{2.569871in}{2.469897in}}%
\pgfpathlineto{\pgfqpoint{2.571055in}{2.464471in}}%
\pgfpathlineto{\pgfqpoint{2.571154in}{2.465084in}}%
\pgfpathlineto{\pgfqpoint{2.571549in}{2.472355in}}%
\pgfpathlineto{\pgfqpoint{2.572042in}{2.461765in}}%
\pgfpathlineto{\pgfqpoint{2.572930in}{2.451625in}}%
\pgfpathlineto{\pgfqpoint{2.573227in}{2.457295in}}%
\pgfpathlineto{\pgfqpoint{2.573424in}{2.459719in}}%
\pgfpathlineto{\pgfqpoint{2.574016in}{2.452538in}}%
\pgfpathlineto{\pgfqpoint{2.574312in}{2.438225in}}%
\pgfpathlineto{\pgfqpoint{2.575200in}{2.445710in}}%
\pgfpathlineto{\pgfqpoint{2.575990in}{2.452609in}}%
\pgfpathlineto{\pgfqpoint{2.576187in}{2.446837in}}%
\pgfpathlineto{\pgfqpoint{2.577174in}{2.424163in}}%
\pgfpathlineto{\pgfqpoint{2.577471in}{2.436947in}}%
\pgfpathlineto{\pgfqpoint{2.577569in}{2.438605in}}%
\pgfpathlineto{\pgfqpoint{2.577865in}{2.427035in}}%
\pgfpathlineto{\pgfqpoint{2.579839in}{2.403131in}}%
\pgfpathlineto{\pgfqpoint{2.578161in}{2.427075in}}%
\pgfpathlineto{\pgfqpoint{2.580037in}{2.404041in}}%
\pgfpathlineto{\pgfqpoint{2.580432in}{2.397164in}}%
\pgfpathlineto{\pgfqpoint{2.580629in}{2.404752in}}%
\pgfpathlineto{\pgfqpoint{2.580826in}{2.414851in}}%
\pgfpathlineto{\pgfqpoint{2.581320in}{2.388496in}}%
\pgfpathlineto{\pgfqpoint{2.581616in}{2.398076in}}%
\pgfpathlineto{\pgfqpoint{2.581912in}{2.396860in}}%
\pgfpathlineto{\pgfqpoint{2.582109in}{2.404275in}}%
\pgfpathlineto{\pgfqpoint{2.582307in}{2.410189in}}%
\pgfpathlineto{\pgfqpoint{2.582800in}{2.391894in}}%
\pgfpathlineto{\pgfqpoint{2.583096in}{2.399511in}}%
\pgfpathlineto{\pgfqpoint{2.583491in}{2.401436in}}%
\pgfpathlineto{\pgfqpoint{2.583689in}{2.398397in}}%
\pgfpathlineto{\pgfqpoint{2.584083in}{2.388658in}}%
\pgfpathlineto{\pgfqpoint{2.584676in}{2.401493in}}%
\pgfpathlineto{\pgfqpoint{2.585860in}{2.391869in}}%
\pgfpathlineto{\pgfqpoint{2.585959in}{2.395124in}}%
\pgfpathlineto{\pgfqpoint{2.587735in}{2.429980in}}%
\pgfpathlineto{\pgfqpoint{2.588229in}{2.442120in}}%
\pgfpathlineto{\pgfqpoint{2.589216in}{2.457783in}}%
\pgfpathlineto{\pgfqpoint{2.589413in}{2.454029in}}%
\pgfpathlineto{\pgfqpoint{2.591782in}{2.393809in}}%
\pgfpathlineto{\pgfqpoint{2.592078in}{2.401195in}}%
\pgfpathlineto{\pgfqpoint{2.592177in}{2.403179in}}%
\pgfpathlineto{\pgfqpoint{2.592572in}{2.391503in}}%
\pgfpathlineto{\pgfqpoint{2.594743in}{2.356353in}}%
\pgfpathlineto{\pgfqpoint{2.594842in}{2.358884in}}%
\pgfpathlineto{\pgfqpoint{2.595927in}{2.417013in}}%
\pgfpathlineto{\pgfqpoint{2.596322in}{2.388909in}}%
\pgfpathlineto{\pgfqpoint{2.596717in}{2.354010in}}%
\pgfpathlineto{\pgfqpoint{2.597605in}{2.360711in}}%
\pgfpathlineto{\pgfqpoint{2.598000in}{2.358568in}}%
\pgfpathlineto{\pgfqpoint{2.598197in}{2.362919in}}%
\pgfpathlineto{\pgfqpoint{2.599283in}{2.474909in}}%
\pgfpathlineto{\pgfqpoint{2.600073in}{2.592532in}}%
\pgfpathlineto{\pgfqpoint{2.600764in}{2.571329in}}%
\pgfpathlineto{\pgfqpoint{2.601060in}{2.577965in}}%
\pgfpathlineto{\pgfqpoint{2.601257in}{2.571567in}}%
\pgfpathlineto{\pgfqpoint{2.602047in}{2.372069in}}%
\pgfpathlineto{\pgfqpoint{2.602343in}{2.325824in}}%
\pgfpathlineto{\pgfqpoint{2.602836in}{2.379869in}}%
\pgfpathlineto{\pgfqpoint{2.603132in}{2.372752in}}%
\pgfpathlineto{\pgfqpoint{2.603231in}{2.372144in}}%
\pgfpathlineto{\pgfqpoint{2.603428in}{2.377209in}}%
\pgfpathlineto{\pgfqpoint{2.605402in}{2.437591in}}%
\pgfpathlineto{\pgfqpoint{2.605797in}{2.418537in}}%
\pgfpathlineto{\pgfqpoint{2.606488in}{2.376584in}}%
\pgfpathlineto{\pgfqpoint{2.607376in}{2.384264in}}%
\pgfpathlineto{\pgfqpoint{2.609054in}{2.360667in}}%
\pgfpathlineto{\pgfqpoint{2.609548in}{2.327992in}}%
\pgfpathlineto{\pgfqpoint{2.610239in}{2.353899in}}%
\pgfpathlineto{\pgfqpoint{2.612805in}{2.448350in}}%
\pgfpathlineto{\pgfqpoint{2.613101in}{2.444956in}}%
\pgfpathlineto{\pgfqpoint{2.613693in}{2.441551in}}%
\pgfpathlineto{\pgfqpoint{2.613890in}{2.446186in}}%
\pgfpathlineto{\pgfqpoint{2.614483in}{2.443623in}}%
\pgfpathlineto{\pgfqpoint{2.615371in}{2.462447in}}%
\pgfpathlineto{\pgfqpoint{2.615568in}{2.463015in}}%
\pgfpathlineto{\pgfqpoint{2.617246in}{2.483958in}}%
\pgfpathlineto{\pgfqpoint{2.617444in}{2.479155in}}%
\pgfpathlineto{\pgfqpoint{2.618628in}{2.459126in}}%
\pgfpathlineto{\pgfqpoint{2.618825in}{2.461903in}}%
\pgfpathlineto{\pgfqpoint{2.619714in}{2.456295in}}%
\pgfpathlineto{\pgfqpoint{2.620207in}{2.474603in}}%
\pgfpathlineto{\pgfqpoint{2.620306in}{2.475379in}}%
\pgfpathlineto{\pgfqpoint{2.620602in}{2.470977in}}%
\pgfpathlineto{\pgfqpoint{2.620799in}{2.468202in}}%
\pgfpathlineto{\pgfqpoint{2.621096in}{2.476640in}}%
\pgfpathlineto{\pgfqpoint{2.622773in}{2.510728in}}%
\pgfpathlineto{\pgfqpoint{2.623760in}{2.494949in}}%
\pgfpathlineto{\pgfqpoint{2.624056in}{2.503549in}}%
\pgfpathlineto{\pgfqpoint{2.624155in}{2.505257in}}%
\pgfpathlineto{\pgfqpoint{2.624353in}{2.495849in}}%
\pgfpathlineto{\pgfqpoint{2.625636in}{2.466055in}}%
\pgfpathlineto{\pgfqpoint{2.625734in}{2.467787in}}%
\pgfpathlineto{\pgfqpoint{2.625833in}{2.469512in}}%
\pgfpathlineto{\pgfqpoint{2.626129in}{2.459108in}}%
\pgfpathlineto{\pgfqpoint{2.627610in}{2.432973in}}%
\pgfpathlineto{\pgfqpoint{2.627708in}{2.431418in}}%
\pgfpathlineto{\pgfqpoint{2.627906in}{2.440176in}}%
\pgfpathlineto{\pgfqpoint{2.628103in}{2.447343in}}%
\pgfpathlineto{\pgfqpoint{2.628597in}{2.425915in}}%
\pgfpathlineto{\pgfqpoint{2.628794in}{2.431800in}}%
\pgfpathlineto{\pgfqpoint{2.628893in}{2.432566in}}%
\pgfpathlineto{\pgfqpoint{2.628991in}{2.430074in}}%
\pgfpathlineto{\pgfqpoint{2.630176in}{2.413824in}}%
\pgfpathlineto{\pgfqpoint{2.630275in}{2.416674in}}%
\pgfpathlineto{\pgfqpoint{2.631360in}{2.429365in}}%
\pgfpathlineto{\pgfqpoint{2.630867in}{2.412649in}}%
\pgfpathlineto{\pgfqpoint{2.631459in}{2.425712in}}%
\pgfpathlineto{\pgfqpoint{2.631755in}{2.417914in}}%
\pgfpathlineto{\pgfqpoint{2.632545in}{2.422584in}}%
\pgfpathlineto{\pgfqpoint{2.632841in}{2.421101in}}%
\pgfpathlineto{\pgfqpoint{2.633038in}{2.423952in}}%
\pgfpathlineto{\pgfqpoint{2.634420in}{2.434880in}}%
\pgfpathlineto{\pgfqpoint{2.634519in}{2.434417in}}%
\pgfpathlineto{\pgfqpoint{2.634815in}{2.429438in}}%
\pgfpathlineto{\pgfqpoint{2.635111in}{2.446049in}}%
\pgfpathlineto{\pgfqpoint{2.635999in}{2.458217in}}%
\pgfpathlineto{\pgfqpoint{2.636196in}{2.450379in}}%
\pgfpathlineto{\pgfqpoint{2.638861in}{2.354530in}}%
\pgfpathlineto{\pgfqpoint{2.639059in}{2.356145in}}%
\pgfpathlineto{\pgfqpoint{2.639256in}{2.348564in}}%
\pgfpathlineto{\pgfqpoint{2.640440in}{2.300823in}}%
\pgfpathlineto{\pgfqpoint{2.641033in}{2.302871in}}%
\pgfpathlineto{\pgfqpoint{2.641724in}{2.292375in}}%
\pgfpathlineto{\pgfqpoint{2.641921in}{2.301389in}}%
\pgfpathlineto{\pgfqpoint{2.642414in}{2.350505in}}%
\pgfpathlineto{\pgfqpoint{2.643204in}{2.327431in}}%
\pgfpathlineto{\pgfqpoint{2.643500in}{2.303355in}}%
\pgfpathlineto{\pgfqpoint{2.644388in}{2.311460in}}%
\pgfpathlineto{\pgfqpoint{2.644981in}{2.317869in}}%
\pgfpathlineto{\pgfqpoint{2.645672in}{2.366165in}}%
\pgfpathlineto{\pgfqpoint{2.646955in}{2.566673in}}%
\pgfpathlineto{\pgfqpoint{2.648336in}{2.526761in}}%
\pgfpathlineto{\pgfqpoint{2.649225in}{2.312294in}}%
\pgfpathlineto{\pgfqpoint{2.650409in}{2.367559in}}%
\pgfpathlineto{\pgfqpoint{2.652482in}{2.435634in}}%
\pgfpathlineto{\pgfqpoint{2.652679in}{2.422745in}}%
\pgfpathlineto{\pgfqpoint{2.653173in}{2.372893in}}%
\pgfpathlineto{\pgfqpoint{2.653962in}{2.391662in}}%
\pgfpathlineto{\pgfqpoint{2.655443in}{2.421016in}}%
\pgfpathlineto{\pgfqpoint{2.655640in}{2.417052in}}%
\pgfpathlineto{\pgfqpoint{2.656825in}{2.391953in}}%
\pgfpathlineto{\pgfqpoint{2.657121in}{2.394816in}}%
\pgfpathlineto{\pgfqpoint{2.658897in}{2.428206in}}%
\pgfpathlineto{\pgfqpoint{2.659095in}{2.422040in}}%
\pgfpathlineto{\pgfqpoint{2.659292in}{2.415649in}}%
\pgfpathlineto{\pgfqpoint{2.659983in}{2.425013in}}%
\pgfpathlineto{\pgfqpoint{2.660970in}{2.437237in}}%
\pgfpathlineto{\pgfqpoint{2.660575in}{2.423385in}}%
\pgfpathlineto{\pgfqpoint{2.661167in}{2.432560in}}%
\pgfpathlineto{\pgfqpoint{2.662056in}{2.426480in}}%
\pgfpathlineto{\pgfqpoint{2.661661in}{2.433724in}}%
\pgfpathlineto{\pgfqpoint{2.662154in}{2.429575in}}%
\pgfpathlineto{\pgfqpoint{2.663141in}{2.453661in}}%
\pgfpathlineto{\pgfqpoint{2.663339in}{2.441727in}}%
\pgfpathlineto{\pgfqpoint{2.663437in}{2.436143in}}%
\pgfpathlineto{\pgfqpoint{2.663832in}{2.453361in}}%
\pgfpathlineto{\pgfqpoint{2.664326in}{2.441976in}}%
\pgfpathlineto{\pgfqpoint{2.664523in}{2.452725in}}%
\pgfpathlineto{\pgfqpoint{2.664918in}{2.436202in}}%
\pgfpathlineto{\pgfqpoint{2.665411in}{2.443604in}}%
\pgfpathlineto{\pgfqpoint{2.666793in}{2.422026in}}%
\pgfpathlineto{\pgfqpoint{2.666991in}{2.425081in}}%
\pgfpathlineto{\pgfqpoint{2.667089in}{2.426204in}}%
\pgfpathlineto{\pgfqpoint{2.667583in}{2.419740in}}%
\pgfpathlineto{\pgfqpoint{2.668076in}{2.424903in}}%
\pgfpathlineto{\pgfqpoint{2.668570in}{2.421620in}}%
\pgfpathlineto{\pgfqpoint{2.668866in}{2.424826in}}%
\pgfpathlineto{\pgfqpoint{2.669063in}{2.426742in}}%
\pgfpathlineto{\pgfqpoint{2.669359in}{2.414675in}}%
\pgfpathlineto{\pgfqpoint{2.670544in}{2.403100in}}%
\pgfpathlineto{\pgfqpoint{2.669951in}{2.424524in}}%
\pgfpathlineto{\pgfqpoint{2.670642in}{2.404801in}}%
\pgfpathlineto{\pgfqpoint{2.670840in}{2.409728in}}%
\pgfpathlineto{\pgfqpoint{2.671333in}{2.397041in}}%
\pgfpathlineto{\pgfqpoint{2.671629in}{2.401827in}}%
\pgfpathlineto{\pgfqpoint{2.674985in}{2.335469in}}%
\pgfpathlineto{\pgfqpoint{2.675281in}{2.343343in}}%
\pgfpathlineto{\pgfqpoint{2.675380in}{2.345343in}}%
\pgfpathlineto{\pgfqpoint{2.675775in}{2.331780in}}%
\pgfpathlineto{\pgfqpoint{2.675972in}{2.328878in}}%
\pgfpathlineto{\pgfqpoint{2.676367in}{2.343153in}}%
\pgfpathlineto{\pgfqpoint{2.676860in}{2.330555in}}%
\pgfpathlineto{\pgfqpoint{2.676959in}{2.330283in}}%
\pgfpathlineto{\pgfqpoint{2.677058in}{2.331627in}}%
\pgfpathlineto{\pgfqpoint{2.677749in}{2.324997in}}%
\pgfpathlineto{\pgfqpoint{2.678143in}{2.335147in}}%
\pgfpathlineto{\pgfqpoint{2.678440in}{2.324451in}}%
\pgfpathlineto{\pgfqpoint{2.678834in}{2.344866in}}%
\pgfpathlineto{\pgfqpoint{2.678933in}{2.346535in}}%
\pgfpathlineto{\pgfqpoint{2.679229in}{2.333874in}}%
\pgfpathlineto{\pgfqpoint{2.680019in}{2.324846in}}%
\pgfpathlineto{\pgfqpoint{2.680216in}{2.334447in}}%
\pgfpathlineto{\pgfqpoint{2.680611in}{2.365362in}}%
\pgfpathlineto{\pgfqpoint{2.681499in}{2.355564in}}%
\pgfpathlineto{\pgfqpoint{2.681598in}{2.356072in}}%
\pgfpathlineto{\pgfqpoint{2.681795in}{2.353568in}}%
\pgfpathlineto{\pgfqpoint{2.683769in}{2.310481in}}%
\pgfpathlineto{\pgfqpoint{2.684658in}{2.327534in}}%
\pgfpathlineto{\pgfqpoint{2.685447in}{2.322558in}}%
\pgfpathlineto{\pgfqpoint{2.686632in}{2.359816in}}%
\pgfpathlineto{\pgfqpoint{2.688704in}{2.304952in}}%
\pgfpathlineto{\pgfqpoint{2.689296in}{2.320601in}}%
\pgfpathlineto{\pgfqpoint{2.689889in}{2.359651in}}%
\pgfpathlineto{\pgfqpoint{2.690481in}{2.331121in}}%
\pgfpathlineto{\pgfqpoint{2.691863in}{2.298349in}}%
\pgfpathlineto{\pgfqpoint{2.691961in}{2.299635in}}%
\pgfpathlineto{\pgfqpoint{2.693244in}{2.399649in}}%
\pgfpathlineto{\pgfqpoint{2.694231in}{2.546974in}}%
\pgfpathlineto{\pgfqpoint{2.695021in}{2.525298in}}%
\pgfpathlineto{\pgfqpoint{2.695416in}{2.514721in}}%
\pgfpathlineto{\pgfqpoint{2.696501in}{2.278456in}}%
\pgfpathlineto{\pgfqpoint{2.697883in}{2.341179in}}%
\pgfpathlineto{\pgfqpoint{2.699561in}{2.399832in}}%
\pgfpathlineto{\pgfqpoint{2.700055in}{2.367404in}}%
\pgfpathlineto{\pgfqpoint{2.700351in}{2.342735in}}%
\pgfpathlineto{\pgfqpoint{2.701239in}{2.351973in}}%
\pgfpathlineto{\pgfqpoint{2.702818in}{2.401522in}}%
\pgfpathlineto{\pgfqpoint{2.703016in}{2.398430in}}%
\pgfpathlineto{\pgfqpoint{2.703805in}{2.369928in}}%
\pgfpathlineto{\pgfqpoint{2.704397in}{2.383722in}}%
\pgfpathlineto{\pgfqpoint{2.705977in}{2.408934in}}%
\pgfpathlineto{\pgfqpoint{2.704792in}{2.380348in}}%
\pgfpathlineto{\pgfqpoint{2.706766in}{2.399904in}}%
\pgfpathlineto{\pgfqpoint{2.706964in}{2.398366in}}%
\pgfpathlineto{\pgfqpoint{2.707260in}{2.403385in}}%
\pgfpathlineto{\pgfqpoint{2.707753in}{2.416058in}}%
\pgfpathlineto{\pgfqpoint{2.708641in}{2.414768in}}%
\pgfpathlineto{\pgfqpoint{2.710122in}{2.426387in}}%
\pgfpathlineto{\pgfqpoint{2.709234in}{2.406103in}}%
\pgfpathlineto{\pgfqpoint{2.710221in}{2.425569in}}%
\pgfpathlineto{\pgfqpoint{2.711306in}{2.407285in}}%
\pgfpathlineto{\pgfqpoint{2.711800in}{2.415077in}}%
\pgfpathlineto{\pgfqpoint{2.712984in}{2.406316in}}%
\pgfpathlineto{\pgfqpoint{2.713280in}{2.408687in}}%
\pgfpathlineto{\pgfqpoint{2.713379in}{2.409352in}}%
\pgfpathlineto{\pgfqpoint{2.713576in}{2.406091in}}%
\pgfpathlineto{\pgfqpoint{2.714267in}{2.412284in}}%
\pgfpathlineto{\pgfqpoint{2.714662in}{2.402669in}}%
\pgfpathlineto{\pgfqpoint{2.715057in}{2.418535in}}%
\pgfpathlineto{\pgfqpoint{2.715846in}{2.410929in}}%
\pgfpathlineto{\pgfqpoint{2.716143in}{2.402908in}}%
\pgfpathlineto{\pgfqpoint{2.716932in}{2.411075in}}%
\pgfpathlineto{\pgfqpoint{2.717130in}{2.414142in}}%
\pgfpathlineto{\pgfqpoint{2.717623in}{2.402102in}}%
\pgfpathlineto{\pgfqpoint{2.717820in}{2.405825in}}%
\pgfpathlineto{\pgfqpoint{2.717919in}{2.406150in}}%
\pgfpathlineto{\pgfqpoint{2.718117in}{2.403400in}}%
\pgfpathlineto{\pgfqpoint{2.721275in}{2.374603in}}%
\pgfpathlineto{\pgfqpoint{2.721472in}{2.376684in}}%
\pgfpathlineto{\pgfqpoint{2.721571in}{2.376621in}}%
\pgfpathlineto{\pgfqpoint{2.721966in}{2.366826in}}%
\pgfpathlineto{\pgfqpoint{2.722361in}{2.380144in}}%
\pgfpathlineto{\pgfqpoint{2.722657in}{2.375544in}}%
\pgfpathlineto{\pgfqpoint{2.722854in}{2.376202in}}%
\pgfpathlineto{\pgfqpoint{2.722953in}{2.376110in}}%
\pgfpathlineto{\pgfqpoint{2.723348in}{2.366408in}}%
\pgfpathlineto{\pgfqpoint{2.724039in}{2.373432in}}%
\pgfpathlineto{\pgfqpoint{2.724433in}{2.379317in}}%
\pgfpathlineto{\pgfqpoint{2.725223in}{2.375435in}}%
\pgfpathlineto{\pgfqpoint{2.725420in}{2.372804in}}%
\pgfpathlineto{\pgfqpoint{2.725716in}{2.380829in}}%
\pgfpathlineto{\pgfqpoint{2.726901in}{2.390816in}}%
\pgfpathlineto{\pgfqpoint{2.726407in}{2.372959in}}%
\pgfpathlineto{\pgfqpoint{2.726999in}{2.389796in}}%
\pgfpathlineto{\pgfqpoint{2.727296in}{2.381125in}}%
\pgfpathlineto{\pgfqpoint{2.727690in}{2.403636in}}%
\pgfpathlineto{\pgfqpoint{2.729368in}{2.440428in}}%
\pgfpathlineto{\pgfqpoint{2.730158in}{2.460489in}}%
\pgfpathlineto{\pgfqpoint{2.731046in}{2.449335in}}%
\pgfpathlineto{\pgfqpoint{2.731244in}{2.450289in}}%
\pgfpathlineto{\pgfqpoint{2.731342in}{2.448789in}}%
\pgfpathlineto{\pgfqpoint{2.732724in}{2.401146in}}%
\pgfpathlineto{\pgfqpoint{2.733218in}{2.412389in}}%
\pgfpathlineto{\pgfqpoint{2.733415in}{2.415712in}}%
\pgfpathlineto{\pgfqpoint{2.733711in}{2.427416in}}%
\pgfpathlineto{\pgfqpoint{2.734204in}{2.406546in}}%
\pgfpathlineto{\pgfqpoint{2.734402in}{2.412427in}}%
\pgfpathlineto{\pgfqpoint{2.734501in}{2.413378in}}%
\pgfpathlineto{\pgfqpoint{2.734797in}{2.405519in}}%
\pgfpathlineto{\pgfqpoint{2.735784in}{2.392573in}}%
\pgfpathlineto{\pgfqpoint{2.736080in}{2.403472in}}%
\pgfpathlineto{\pgfqpoint{2.736771in}{2.446685in}}%
\pgfpathlineto{\pgfqpoint{2.737264in}{2.422916in}}%
\pgfpathlineto{\pgfqpoint{2.738449in}{2.392842in}}%
\pgfpathlineto{\pgfqpoint{2.738547in}{2.393198in}}%
\pgfpathlineto{\pgfqpoint{2.740126in}{2.497710in}}%
\pgfpathlineto{\pgfqpoint{2.741706in}{2.671159in}}%
\pgfpathlineto{\pgfqpoint{2.742002in}{2.658095in}}%
\pgfpathlineto{\pgfqpoint{2.742693in}{2.577126in}}%
\pgfpathlineto{\pgfqpoint{2.743383in}{2.403803in}}%
\pgfpathlineto{\pgfqpoint{2.744173in}{2.447447in}}%
\pgfpathlineto{\pgfqpoint{2.744272in}{2.447668in}}%
\pgfpathlineto{\pgfqpoint{2.746443in}{2.538400in}}%
\pgfpathlineto{\pgfqpoint{2.747035in}{2.506800in}}%
\pgfpathlineto{\pgfqpoint{2.747628in}{2.486985in}}%
\pgfpathlineto{\pgfqpoint{2.748022in}{2.511208in}}%
\pgfpathlineto{\pgfqpoint{2.749700in}{2.552011in}}%
\pgfpathlineto{\pgfqpoint{2.748516in}{2.504497in}}%
\pgfpathlineto{\pgfqpoint{2.749799in}{2.550702in}}%
\pgfpathlineto{\pgfqpoint{2.750786in}{2.513258in}}%
\pgfpathlineto{\pgfqpoint{2.751477in}{2.528614in}}%
\pgfpathlineto{\pgfqpoint{2.752661in}{2.557259in}}%
\pgfpathlineto{\pgfqpoint{2.752859in}{2.554527in}}%
\pgfpathlineto{\pgfqpoint{2.754438in}{2.511327in}}%
\pgfpathlineto{\pgfqpoint{2.754734in}{2.501039in}}%
\pgfpathlineto{\pgfqpoint{2.755227in}{2.512939in}}%
\pgfpathlineto{\pgfqpoint{2.755523in}{2.510139in}}%
\pgfpathlineto{\pgfqpoint{2.756116in}{2.549810in}}%
\pgfpathlineto{\pgfqpoint{2.757794in}{2.634533in}}%
\pgfpathlineto{\pgfqpoint{2.758386in}{2.619486in}}%
\pgfpathlineto{\pgfqpoint{2.758484in}{2.620905in}}%
\pgfpathlineto{\pgfqpoint{2.758879in}{2.612522in}}%
\pgfpathlineto{\pgfqpoint{2.759768in}{2.605352in}}%
\pgfpathlineto{\pgfqpoint{2.759274in}{2.613180in}}%
\pgfpathlineto{\pgfqpoint{2.760162in}{2.610183in}}%
\pgfpathlineto{\pgfqpoint{2.761051in}{2.616813in}}%
\pgfpathlineto{\pgfqpoint{2.762235in}{2.632408in}}%
\pgfpathlineto{\pgfqpoint{2.762531in}{2.625134in}}%
\pgfpathlineto{\pgfqpoint{2.762926in}{2.619366in}}%
\pgfpathlineto{\pgfqpoint{2.763419in}{2.628928in}}%
\pgfpathlineto{\pgfqpoint{2.763518in}{2.629848in}}%
\pgfpathlineto{\pgfqpoint{2.763913in}{2.623125in}}%
\pgfpathlineto{\pgfqpoint{2.764012in}{2.622926in}}%
\pgfpathlineto{\pgfqpoint{2.764110in}{2.624494in}}%
\pgfpathlineto{\pgfqpoint{2.764900in}{2.631949in}}%
\pgfpathlineto{\pgfqpoint{2.765295in}{2.626607in}}%
\pgfpathlineto{\pgfqpoint{2.765591in}{2.626962in}}%
\pgfpathlineto{\pgfqpoint{2.768453in}{2.584636in}}%
\pgfpathlineto{\pgfqpoint{2.768552in}{2.586863in}}%
\pgfpathlineto{\pgfqpoint{2.768650in}{2.588546in}}%
\pgfpathlineto{\pgfqpoint{2.769045in}{2.583176in}}%
\pgfpathlineto{\pgfqpoint{2.769440in}{2.584369in}}%
\pgfpathlineto{\pgfqpoint{2.770328in}{2.574459in}}%
\pgfpathlineto{\pgfqpoint{2.770624in}{2.580765in}}%
\pgfpathlineto{\pgfqpoint{2.770822in}{2.584606in}}%
\pgfpathlineto{\pgfqpoint{2.771217in}{2.579262in}}%
\pgfpathlineto{\pgfqpoint{2.771809in}{2.583886in}}%
\pgfpathlineto{\pgfqpoint{2.772105in}{2.585860in}}%
\pgfpathlineto{\pgfqpoint{2.772598in}{2.580694in}}%
\pgfpathlineto{\pgfqpoint{2.773092in}{2.571975in}}%
\pgfpathlineto{\pgfqpoint{2.773388in}{2.582367in}}%
\pgfpathlineto{\pgfqpoint{2.773487in}{2.584509in}}%
\pgfpathlineto{\pgfqpoint{2.773881in}{2.569244in}}%
\pgfpathlineto{\pgfqpoint{2.773980in}{2.569563in}}%
\pgfpathlineto{\pgfqpoint{2.774375in}{2.589159in}}%
\pgfpathlineto{\pgfqpoint{2.775263in}{2.580419in}}%
\pgfpathlineto{\pgfqpoint{2.775362in}{2.576461in}}%
\pgfpathlineto{\pgfqpoint{2.775757in}{2.599022in}}%
\pgfpathlineto{\pgfqpoint{2.776250in}{2.596754in}}%
\pgfpathlineto{\pgfqpoint{2.777237in}{2.615270in}}%
\pgfpathlineto{\pgfqpoint{2.777435in}{2.619174in}}%
\pgfpathlineto{\pgfqpoint{2.777928in}{2.605642in}}%
\pgfpathlineto{\pgfqpoint{2.778027in}{2.606218in}}%
\pgfpathlineto{\pgfqpoint{2.778422in}{2.614089in}}%
\pgfpathlineto{\pgfqpoint{2.778718in}{2.603569in}}%
\pgfpathlineto{\pgfqpoint{2.780494in}{2.552825in}}%
\pgfpathlineto{\pgfqpoint{2.780593in}{2.555198in}}%
\pgfpathlineto{\pgfqpoint{2.781679in}{2.593250in}}%
\pgfpathlineto{\pgfqpoint{2.782073in}{2.574965in}}%
\pgfpathlineto{\pgfqpoint{2.783357in}{2.543829in}}%
\pgfpathlineto{\pgfqpoint{2.783653in}{2.561288in}}%
\pgfpathlineto{\pgfqpoint{2.784738in}{2.609845in}}%
\pgfpathlineto{\pgfqpoint{2.784047in}{2.561049in}}%
\pgfpathlineto{\pgfqpoint{2.785429in}{2.584034in}}%
\pgfpathlineto{\pgfqpoint{2.785824in}{2.560728in}}%
\pgfpathlineto{\pgfqpoint{2.786811in}{2.561481in}}%
\pgfpathlineto{\pgfqpoint{2.787403in}{2.571921in}}%
\pgfpathlineto{\pgfqpoint{2.788686in}{2.734313in}}%
\pgfpathlineto{\pgfqpoint{2.789377in}{2.850498in}}%
\pgfpathlineto{\pgfqpoint{2.790167in}{2.817346in}}%
\pgfpathlineto{\pgfqpoint{2.790562in}{2.795018in}}%
\pgfpathlineto{\pgfqpoint{2.791351in}{2.575508in}}%
\pgfpathlineto{\pgfqpoint{2.792437in}{2.608001in}}%
\pgfpathlineto{\pgfqpoint{2.792733in}{2.601670in}}%
\pgfpathlineto{\pgfqpoint{2.793029in}{2.618421in}}%
\pgfpathlineto{\pgfqpoint{2.794510in}{2.686215in}}%
\pgfpathlineto{\pgfqpoint{2.794904in}{2.658445in}}%
\pgfpathlineto{\pgfqpoint{2.795595in}{2.613612in}}%
\pgfpathlineto{\pgfqpoint{2.796187in}{2.639029in}}%
\pgfpathlineto{\pgfqpoint{2.797964in}{2.689065in}}%
\pgfpathlineto{\pgfqpoint{2.798852in}{2.668504in}}%
\pgfpathlineto{\pgfqpoint{2.799050in}{2.664654in}}%
\pgfpathlineto{\pgfqpoint{2.799346in}{2.676863in}}%
\pgfpathlineto{\pgfqpoint{2.800826in}{2.699704in}}%
\pgfpathlineto{\pgfqpoint{2.801122in}{2.697574in}}%
\pgfpathlineto{\pgfqpoint{2.801320in}{2.702272in}}%
\pgfpathlineto{\pgfqpoint{2.802208in}{2.707893in}}%
\pgfpathlineto{\pgfqpoint{2.801813in}{2.700164in}}%
\pgfpathlineto{\pgfqpoint{2.802307in}{2.704976in}}%
\pgfpathlineto{\pgfqpoint{2.802504in}{2.697977in}}%
\pgfpathlineto{\pgfqpoint{2.802899in}{2.706034in}}%
\pgfpathlineto{\pgfqpoint{2.803392in}{2.703617in}}%
\pgfpathlineto{\pgfqpoint{2.804774in}{2.713483in}}%
\pgfpathlineto{\pgfqpoint{2.804281in}{2.702133in}}%
\pgfpathlineto{\pgfqpoint{2.804972in}{2.712501in}}%
\pgfpathlineto{\pgfqpoint{2.806156in}{2.730228in}}%
\pgfpathlineto{\pgfqpoint{2.805663in}{2.711722in}}%
\pgfpathlineto{\pgfqpoint{2.806551in}{2.717321in}}%
\pgfpathlineto{\pgfqpoint{2.806650in}{2.716152in}}%
\pgfpathlineto{\pgfqpoint{2.806946in}{2.723772in}}%
\pgfpathlineto{\pgfqpoint{2.808031in}{2.734911in}}%
\pgfpathlineto{\pgfqpoint{2.807637in}{2.723595in}}%
\pgfpathlineto{\pgfqpoint{2.808229in}{2.733393in}}%
\pgfpathlineto{\pgfqpoint{2.809216in}{2.727203in}}%
\pgfpathlineto{\pgfqpoint{2.808722in}{2.739814in}}%
\pgfpathlineto{\pgfqpoint{2.809314in}{2.731376in}}%
\pgfpathlineto{\pgfqpoint{2.809709in}{2.752017in}}%
\pgfpathlineto{\pgfqpoint{2.810597in}{2.744840in}}%
\pgfpathlineto{\pgfqpoint{2.811782in}{2.730942in}}%
\pgfpathlineto{\pgfqpoint{2.812078in}{2.736006in}}%
\pgfpathlineto{\pgfqpoint{2.812177in}{2.737260in}}%
\pgfpathlineto{\pgfqpoint{2.812473in}{2.729780in}}%
\pgfpathlineto{\pgfqpoint{2.813262in}{2.715493in}}%
\pgfpathlineto{\pgfqpoint{2.813558in}{2.726845in}}%
\pgfpathlineto{\pgfqpoint{2.813657in}{2.728020in}}%
\pgfpathlineto{\pgfqpoint{2.813756in}{2.723401in}}%
\pgfpathlineto{\pgfqpoint{2.815039in}{2.688921in}}%
\pgfpathlineto{\pgfqpoint{2.815236in}{2.691124in}}%
\pgfpathlineto{\pgfqpoint{2.815335in}{2.691359in}}%
\pgfpathlineto{\pgfqpoint{2.815434in}{2.689909in}}%
\pgfpathlineto{\pgfqpoint{2.816717in}{2.659027in}}%
\pgfpathlineto{\pgfqpoint{2.817309in}{2.660268in}}%
\pgfpathlineto{\pgfqpoint{2.818987in}{2.641043in}}%
\pgfpathlineto{\pgfqpoint{2.819086in}{2.641869in}}%
\pgfpathlineto{\pgfqpoint{2.819382in}{2.655720in}}%
\pgfpathlineto{\pgfqpoint{2.819776in}{2.638515in}}%
\pgfpathlineto{\pgfqpoint{2.820270in}{2.650574in}}%
\pgfpathlineto{\pgfqpoint{2.823330in}{2.537320in}}%
\pgfpathlineto{\pgfqpoint{2.823428in}{2.537488in}}%
\pgfpathlineto{\pgfqpoint{2.823922in}{2.548641in}}%
\pgfpathlineto{\pgfqpoint{2.826192in}{2.665508in}}%
\pgfpathlineto{\pgfqpoint{2.827080in}{2.645652in}}%
\pgfpathlineto{\pgfqpoint{2.828561in}{2.554148in}}%
\pgfpathlineto{\pgfqpoint{2.829646in}{2.565217in}}%
\pgfpathlineto{\pgfqpoint{2.829745in}{2.567008in}}%
\pgfpathlineto{\pgfqpoint{2.830041in}{2.556911in}}%
\pgfpathlineto{\pgfqpoint{2.831916in}{2.504531in}}%
\pgfpathlineto{\pgfqpoint{2.832015in}{2.504833in}}%
\pgfpathlineto{\pgfqpoint{2.832805in}{2.560010in}}%
\pgfpathlineto{\pgfqpoint{2.833496in}{2.525669in}}%
\pgfpathlineto{\pgfqpoint{2.834976in}{2.480208in}}%
\pgfpathlineto{\pgfqpoint{2.835272in}{2.492016in}}%
\pgfpathlineto{\pgfqpoint{2.836654in}{2.629975in}}%
\pgfpathlineto{\pgfqpoint{2.837444in}{2.753548in}}%
\pgfpathlineto{\pgfqpoint{2.838134in}{2.720722in}}%
\pgfpathlineto{\pgfqpoint{2.838628in}{2.681962in}}%
\pgfpathlineto{\pgfqpoint{2.839418in}{2.478368in}}%
\pgfpathlineto{\pgfqpoint{2.840405in}{2.488524in}}%
\pgfpathlineto{\pgfqpoint{2.840503in}{2.486115in}}%
\pgfpathlineto{\pgfqpoint{2.840898in}{2.499583in}}%
\pgfpathlineto{\pgfqpoint{2.842675in}{2.567839in}}%
\pgfpathlineto{\pgfqpoint{2.842971in}{2.544943in}}%
\pgfpathlineto{\pgfqpoint{2.843859in}{2.500349in}}%
\pgfpathlineto{\pgfqpoint{2.844254in}{2.514843in}}%
\pgfpathlineto{\pgfqpoint{2.844550in}{2.511472in}}%
\pgfpathlineto{\pgfqpoint{2.844846in}{2.521771in}}%
\pgfpathlineto{\pgfqpoint{2.845833in}{2.534175in}}%
\pgfpathlineto{\pgfqpoint{2.846030in}{2.530669in}}%
\pgfpathlineto{\pgfqpoint{2.847906in}{2.492527in}}%
\pgfpathlineto{\pgfqpoint{2.848004in}{2.490721in}}%
\pgfpathlineto{\pgfqpoint{2.848399in}{2.503327in}}%
\pgfpathlineto{\pgfqpoint{2.848695in}{2.496680in}}%
\pgfpathlineto{\pgfqpoint{2.848893in}{2.500638in}}%
\pgfpathlineto{\pgfqpoint{2.849189in}{2.512737in}}%
\pgfpathlineto{\pgfqpoint{2.849781in}{2.490511in}}%
\pgfpathlineto{\pgfqpoint{2.851064in}{2.522075in}}%
\pgfpathlineto{\pgfqpoint{2.852150in}{2.516088in}}%
\pgfpathlineto{\pgfqpoint{2.852347in}{2.510450in}}%
\pgfpathlineto{\pgfqpoint{2.852841in}{2.527216in}}%
\pgfpathlineto{\pgfqpoint{2.853137in}{2.531809in}}%
\pgfpathlineto{\pgfqpoint{2.853729in}{2.521053in}}%
\pgfpathlineto{\pgfqpoint{2.855604in}{2.493643in}}%
\pgfpathlineto{\pgfqpoint{2.855802in}{2.497520in}}%
\pgfpathlineto{\pgfqpoint{2.855999in}{2.505003in}}%
\pgfpathlineto{\pgfqpoint{2.856789in}{2.492411in}}%
\pgfpathlineto{\pgfqpoint{2.857183in}{2.495417in}}%
\pgfpathlineto{\pgfqpoint{2.857479in}{2.484836in}}%
\pgfpathlineto{\pgfqpoint{2.857578in}{2.481993in}}%
\pgfpathlineto{\pgfqpoint{2.857973in}{2.491551in}}%
\pgfpathlineto{\pgfqpoint{2.858368in}{2.486349in}}%
\pgfpathlineto{\pgfqpoint{2.858664in}{2.500713in}}%
\pgfpathlineto{\pgfqpoint{2.859157in}{2.480147in}}%
\pgfpathlineto{\pgfqpoint{2.859355in}{2.481784in}}%
\pgfpathlineto{\pgfqpoint{2.859453in}{2.482034in}}%
\pgfpathlineto{\pgfqpoint{2.859552in}{2.480864in}}%
\pgfpathlineto{\pgfqpoint{2.860737in}{2.469955in}}%
\pgfpathlineto{\pgfqpoint{2.860243in}{2.488215in}}%
\pgfpathlineto{\pgfqpoint{2.860934in}{2.471968in}}%
\pgfpathlineto{\pgfqpoint{2.861033in}{2.473756in}}%
\pgfpathlineto{\pgfqpoint{2.861526in}{2.463555in}}%
\pgfpathlineto{\pgfqpoint{2.865474in}{2.385487in}}%
\pgfpathlineto{\pgfqpoint{2.862414in}{2.467083in}}%
\pgfpathlineto{\pgfqpoint{2.865770in}{2.389763in}}%
\pgfpathlineto{\pgfqpoint{2.865869in}{2.389793in}}%
\pgfpathlineto{\pgfqpoint{2.867744in}{2.360207in}}%
\pgfpathlineto{\pgfqpoint{2.867843in}{2.362070in}}%
\pgfpathlineto{\pgfqpoint{2.868238in}{2.379636in}}%
\pgfpathlineto{\pgfqpoint{2.869027in}{2.365792in}}%
\pgfpathlineto{\pgfqpoint{2.869916in}{2.353569in}}%
\pgfpathlineto{\pgfqpoint{2.869422in}{2.365870in}}%
\pgfpathlineto{\pgfqpoint{2.870113in}{2.362308in}}%
\pgfpathlineto{\pgfqpoint{2.871001in}{2.377436in}}%
\pgfpathlineto{\pgfqpoint{2.870606in}{2.359460in}}%
\pgfpathlineto{\pgfqpoint{2.871297in}{2.365198in}}%
\pgfpathlineto{\pgfqpoint{2.871396in}{2.362665in}}%
\pgfpathlineto{\pgfqpoint{2.871791in}{2.377801in}}%
\pgfpathlineto{\pgfqpoint{2.873864in}{2.419874in}}%
\pgfpathlineto{\pgfqpoint{2.873962in}{2.422018in}}%
\pgfpathlineto{\pgfqpoint{2.874258in}{2.404414in}}%
\pgfpathlineto{\pgfqpoint{2.874752in}{2.418042in}}%
\pgfpathlineto{\pgfqpoint{2.875837in}{2.379532in}}%
\pgfpathlineto{\pgfqpoint{2.876923in}{2.395600in}}%
\pgfpathlineto{\pgfqpoint{2.877515in}{2.390469in}}%
\pgfpathlineto{\pgfqpoint{2.877713in}{2.388262in}}%
\pgfpathlineto{\pgfqpoint{2.878305in}{2.371841in}}%
\pgfpathlineto{\pgfqpoint{2.878996in}{2.376647in}}%
\pgfpathlineto{\pgfqpoint{2.880378in}{2.431647in}}%
\pgfpathlineto{\pgfqpoint{2.880772in}{2.401375in}}%
\pgfpathlineto{\pgfqpoint{2.882253in}{2.356889in}}%
\pgfpathlineto{\pgfqpoint{2.882648in}{2.355670in}}%
\pgfpathlineto{\pgfqpoint{2.882746in}{2.358054in}}%
\pgfpathlineto{\pgfqpoint{2.883832in}{2.469737in}}%
\pgfpathlineto{\pgfqpoint{2.884720in}{2.596588in}}%
\pgfpathlineto{\pgfqpoint{2.885411in}{2.585540in}}%
\pgfpathlineto{\pgfqpoint{2.886102in}{2.508202in}}%
\pgfpathlineto{\pgfqpoint{2.886793in}{2.321847in}}%
\pgfpathlineto{\pgfqpoint{2.887681in}{2.350865in}}%
\pgfpathlineto{\pgfqpoint{2.887879in}{2.349629in}}%
\pgfpathlineto{\pgfqpoint{2.888076in}{2.353774in}}%
\pgfpathlineto{\pgfqpoint{2.889162in}{2.360800in}}%
\pgfpathlineto{\pgfqpoint{2.888668in}{2.350383in}}%
\pgfpathlineto{\pgfqpoint{2.889359in}{2.360653in}}%
\pgfpathlineto{\pgfqpoint{2.889754in}{2.366752in}}%
\pgfpathlineto{\pgfqpoint{2.890050in}{2.359276in}}%
\pgfpathlineto{\pgfqpoint{2.891432in}{2.273598in}}%
\pgfpathlineto{\pgfqpoint{2.892024in}{2.301712in}}%
\pgfpathlineto{\pgfqpoint{2.893307in}{2.395427in}}%
\pgfpathlineto{\pgfqpoint{2.893998in}{2.381312in}}%
\pgfpathlineto{\pgfqpoint{2.894195in}{2.378593in}}%
\pgfpathlineto{\pgfqpoint{2.894590in}{2.388500in}}%
\pgfpathlineto{\pgfqpoint{2.894788in}{2.392102in}}%
\pgfpathlineto{\pgfqpoint{2.895380in}{2.380100in}}%
\pgfpathlineto{\pgfqpoint{2.896071in}{2.378457in}}%
\pgfpathlineto{\pgfqpoint{2.895775in}{2.380765in}}%
\pgfpathlineto{\pgfqpoint{2.896268in}{2.380368in}}%
\pgfpathlineto{\pgfqpoint{2.897255in}{2.397916in}}%
\pgfpathlineto{\pgfqpoint{2.897551in}{2.388682in}}%
\pgfpathlineto{\pgfqpoint{2.897650in}{2.387073in}}%
\pgfpathlineto{\pgfqpoint{2.897946in}{2.399820in}}%
\pgfpathlineto{\pgfqpoint{2.898242in}{2.409074in}}%
\pgfpathlineto{\pgfqpoint{2.898933in}{2.398384in}}%
\pgfpathlineto{\pgfqpoint{2.899328in}{2.393374in}}%
\pgfpathlineto{\pgfqpoint{2.899624in}{2.402555in}}%
\pgfpathlineto{\pgfqpoint{2.900710in}{2.412093in}}%
\pgfpathlineto{\pgfqpoint{2.900216in}{2.395067in}}%
\pgfpathlineto{\pgfqpoint{2.900808in}{2.410274in}}%
\pgfpathlineto{\pgfqpoint{2.901795in}{2.400784in}}%
\pgfpathlineto{\pgfqpoint{2.902091in}{2.402590in}}%
\pgfpathlineto{\pgfqpoint{2.902190in}{2.403168in}}%
\pgfpathlineto{\pgfqpoint{2.902585in}{2.399404in}}%
\pgfpathlineto{\pgfqpoint{2.902881in}{2.402078in}}%
\pgfpathlineto{\pgfqpoint{2.903374in}{2.382310in}}%
\pgfpathlineto{\pgfqpoint{2.904263in}{2.395046in}}%
\pgfpathlineto{\pgfqpoint{2.904756in}{2.404749in}}%
\pgfpathlineto{\pgfqpoint{2.905052in}{2.394206in}}%
\pgfpathlineto{\pgfqpoint{2.905250in}{2.390679in}}%
\pgfpathlineto{\pgfqpoint{2.905743in}{2.405859in}}%
\pgfpathlineto{\pgfqpoint{2.905842in}{2.406626in}}%
\pgfpathlineto{\pgfqpoint{2.906039in}{2.403166in}}%
\pgfpathlineto{\pgfqpoint{2.906335in}{2.393826in}}%
\pgfpathlineto{\pgfqpoint{2.906928in}{2.406210in}}%
\pgfpathlineto{\pgfqpoint{2.907026in}{2.406150in}}%
\pgfpathlineto{\pgfqpoint{2.911073in}{2.359416in}}%
\pgfpathlineto{\pgfqpoint{2.911369in}{2.371222in}}%
\pgfpathlineto{\pgfqpoint{2.911567in}{2.377863in}}%
\pgfpathlineto{\pgfqpoint{2.912159in}{2.356914in}}%
\pgfpathlineto{\pgfqpoint{2.912455in}{2.360855in}}%
\pgfpathlineto{\pgfqpoint{2.912948in}{2.352764in}}%
\pgfpathlineto{\pgfqpoint{2.913935in}{2.339425in}}%
\pgfpathlineto{\pgfqpoint{2.914231in}{2.349448in}}%
\pgfpathlineto{\pgfqpoint{2.914330in}{2.351842in}}%
\pgfpathlineto{\pgfqpoint{2.914725in}{2.337518in}}%
\pgfpathlineto{\pgfqpoint{2.914922in}{2.330476in}}%
\pgfpathlineto{\pgfqpoint{2.915811in}{2.337049in}}%
\pgfpathlineto{\pgfqpoint{2.916896in}{2.352216in}}%
\pgfpathlineto{\pgfqpoint{2.917291in}{2.344153in}}%
\pgfpathlineto{\pgfqpoint{2.917587in}{2.354316in}}%
\pgfpathlineto{\pgfqpoint{2.919166in}{2.384691in}}%
\pgfpathlineto{\pgfqpoint{2.919265in}{2.384476in}}%
\pgfpathlineto{\pgfqpoint{2.919364in}{2.384320in}}%
\pgfpathlineto{\pgfqpoint{2.919462in}{2.385524in}}%
\pgfpathlineto{\pgfqpoint{2.920943in}{2.398290in}}%
\pgfpathlineto{\pgfqpoint{2.921042in}{2.396976in}}%
\pgfpathlineto{\pgfqpoint{2.923114in}{2.331517in}}%
\pgfpathlineto{\pgfqpoint{2.924101in}{2.343068in}}%
\pgfpathlineto{\pgfqpoint{2.924299in}{2.340654in}}%
\pgfpathlineto{\pgfqpoint{2.925977in}{2.301842in}}%
\pgfpathlineto{\pgfqpoint{2.926174in}{2.304088in}}%
\pgfpathlineto{\pgfqpoint{2.926667in}{2.320264in}}%
\pgfpathlineto{\pgfqpoint{2.927260in}{2.354359in}}%
\pgfpathlineto{\pgfqpoint{2.927852in}{2.327340in}}%
\pgfpathlineto{\pgfqpoint{2.929135in}{2.295741in}}%
\pgfpathlineto{\pgfqpoint{2.929332in}{2.298253in}}%
\pgfpathlineto{\pgfqpoint{2.930418in}{2.346694in}}%
\pgfpathlineto{\pgfqpoint{2.931997in}{2.529197in}}%
\pgfpathlineto{\pgfqpoint{2.932589in}{2.509181in}}%
\pgfpathlineto{\pgfqpoint{2.932688in}{2.511082in}}%
\pgfpathlineto{\pgfqpoint{2.932885in}{2.503475in}}%
\pgfpathlineto{\pgfqpoint{2.933872in}{2.269892in}}%
\pgfpathlineto{\pgfqpoint{2.935254in}{2.318574in}}%
\pgfpathlineto{\pgfqpoint{2.937130in}{2.392327in}}%
\pgfpathlineto{\pgfqpoint{2.937623in}{2.357702in}}%
\pgfpathlineto{\pgfqpoint{2.938314in}{2.335238in}}%
\pgfpathlineto{\pgfqpoint{2.938709in}{2.354147in}}%
\pgfpathlineto{\pgfqpoint{2.940189in}{2.382280in}}%
\pgfpathlineto{\pgfqpoint{2.940288in}{2.381955in}}%
\pgfpathlineto{\pgfqpoint{2.940979in}{2.338212in}}%
\pgfpathlineto{\pgfqpoint{2.941867in}{2.363872in}}%
\pgfpathlineto{\pgfqpoint{2.943348in}{2.386597in}}%
\pgfpathlineto{\pgfqpoint{2.942755in}{2.363820in}}%
\pgfpathlineto{\pgfqpoint{2.943545in}{2.386303in}}%
\pgfpathlineto{\pgfqpoint{2.943841in}{2.389061in}}%
\pgfpathlineto{\pgfqpoint{2.944137in}{2.379619in}}%
\pgfpathlineto{\pgfqpoint{2.944433in}{2.375050in}}%
\pgfpathlineto{\pgfqpoint{2.944927in}{2.387815in}}%
\pgfpathlineto{\pgfqpoint{2.946012in}{2.409143in}}%
\pgfpathlineto{\pgfqpoint{2.945519in}{2.382795in}}%
\pgfpathlineto{\pgfqpoint{2.946210in}{2.403653in}}%
\pgfpathlineto{\pgfqpoint{2.946407in}{2.392213in}}%
\pgfpathlineto{\pgfqpoint{2.946999in}{2.413180in}}%
\pgfpathlineto{\pgfqpoint{2.947197in}{2.411550in}}%
\pgfpathlineto{\pgfqpoint{2.948381in}{2.391473in}}%
\pgfpathlineto{\pgfqpoint{2.948579in}{2.401027in}}%
\pgfpathlineto{\pgfqpoint{2.948776in}{2.406267in}}%
\pgfpathlineto{\pgfqpoint{2.949467in}{2.397784in}}%
\pgfpathlineto{\pgfqpoint{2.950454in}{2.385477in}}%
\pgfpathlineto{\pgfqpoint{2.950059in}{2.401623in}}%
\pgfpathlineto{\pgfqpoint{2.950651in}{2.392897in}}%
\pgfpathlineto{\pgfqpoint{2.950947in}{2.407102in}}%
\pgfpathlineto{\pgfqpoint{2.951836in}{2.399849in}}%
\pgfpathlineto{\pgfqpoint{2.952329in}{2.396758in}}%
\pgfpathlineto{\pgfqpoint{2.953020in}{2.406125in}}%
\pgfpathlineto{\pgfqpoint{2.953514in}{2.390730in}}%
\pgfpathlineto{\pgfqpoint{2.953908in}{2.406858in}}%
\pgfpathlineto{\pgfqpoint{2.954402in}{2.386959in}}%
\pgfpathlineto{\pgfqpoint{2.954501in}{2.387710in}}%
\pgfpathlineto{\pgfqpoint{2.954797in}{2.389417in}}%
\pgfpathlineto{\pgfqpoint{2.955093in}{2.397287in}}%
\pgfpathlineto{\pgfqpoint{2.955488in}{2.384723in}}%
\pgfpathlineto{\pgfqpoint{2.955882in}{2.390714in}}%
\pgfpathlineto{\pgfqpoint{2.957067in}{2.364256in}}%
\pgfpathlineto{\pgfqpoint{2.957363in}{2.365002in}}%
\pgfpathlineto{\pgfqpoint{2.959041in}{2.350138in}}%
\pgfpathlineto{\pgfqpoint{2.959534in}{2.368388in}}%
\pgfpathlineto{\pgfqpoint{2.960324in}{2.358383in}}%
\pgfpathlineto{\pgfqpoint{2.961212in}{2.356694in}}%
\pgfpathlineto{\pgfqpoint{2.960817in}{2.362058in}}%
\pgfpathlineto{\pgfqpoint{2.961311in}{2.357948in}}%
\pgfpathlineto{\pgfqpoint{2.961804in}{2.353746in}}%
\pgfpathlineto{\pgfqpoint{2.962791in}{2.364434in}}%
\pgfpathlineto{\pgfqpoint{2.963482in}{2.371490in}}%
\pgfpathlineto{\pgfqpoint{2.963087in}{2.362911in}}%
\pgfpathlineto{\pgfqpoint{2.963680in}{2.367879in}}%
\pgfpathlineto{\pgfqpoint{2.965555in}{2.322720in}}%
\pgfpathlineto{\pgfqpoint{2.966147in}{2.324630in}}%
\pgfpathlineto{\pgfqpoint{2.967134in}{2.385309in}}%
\pgfpathlineto{\pgfqpoint{2.967924in}{2.431727in}}%
\pgfpathlineto{\pgfqpoint{2.968713in}{2.425425in}}%
\pgfpathlineto{\pgfqpoint{2.972957in}{2.342318in}}%
\pgfpathlineto{\pgfqpoint{2.973253in}{2.344864in}}%
\pgfpathlineto{\pgfqpoint{2.973549in}{2.337610in}}%
\pgfpathlineto{\pgfqpoint{2.973648in}{2.337188in}}%
\pgfpathlineto{\pgfqpoint{2.973747in}{2.340275in}}%
\pgfpathlineto{\pgfqpoint{2.974339in}{2.372859in}}%
\pgfpathlineto{\pgfqpoint{2.975030in}{2.355750in}}%
\pgfpathlineto{\pgfqpoint{2.976510in}{2.318627in}}%
\pgfpathlineto{\pgfqpoint{2.976708in}{2.323036in}}%
\pgfpathlineto{\pgfqpoint{2.978188in}{2.495788in}}%
\pgfpathlineto{\pgfqpoint{2.978780in}{2.560006in}}%
\pgfpathlineto{\pgfqpoint{2.979471in}{2.535104in}}%
\pgfpathlineto{\pgfqpoint{2.979570in}{2.534592in}}%
\pgfpathlineto{\pgfqpoint{2.979669in}{2.537058in}}%
\pgfpathlineto{\pgfqpoint{2.979866in}{2.542338in}}%
\pgfpathlineto{\pgfqpoint{2.980162in}{2.514638in}}%
\pgfpathlineto{\pgfqpoint{2.981051in}{2.303014in}}%
\pgfpathlineto{\pgfqpoint{2.982136in}{2.338078in}}%
\pgfpathlineto{\pgfqpoint{2.984209in}{2.427769in}}%
\pgfpathlineto{\pgfqpoint{2.984505in}{2.406494in}}%
\pgfpathlineto{\pgfqpoint{2.985393in}{2.374744in}}%
\pgfpathlineto{\pgfqpoint{2.985689in}{2.385051in}}%
\pgfpathlineto{\pgfqpoint{2.986380in}{2.374583in}}%
\pgfpathlineto{\pgfqpoint{2.987071in}{2.408384in}}%
\pgfpathlineto{\pgfqpoint{2.987170in}{2.408718in}}%
\pgfpathlineto{\pgfqpoint{2.987367in}{2.406120in}}%
\pgfpathlineto{\pgfqpoint{2.988453in}{2.371633in}}%
\pgfpathlineto{\pgfqpoint{2.989144in}{2.391761in}}%
\pgfpathlineto{\pgfqpoint{2.990822in}{2.414036in}}%
\pgfpathlineto{\pgfqpoint{2.991019in}{2.411254in}}%
\pgfpathlineto{\pgfqpoint{2.991118in}{2.410568in}}%
\pgfpathlineto{\pgfqpoint{2.991315in}{2.414529in}}%
\pgfpathlineto{\pgfqpoint{2.991611in}{2.419196in}}%
\pgfpathlineto{\pgfqpoint{2.991907in}{2.414311in}}%
\pgfpathlineto{\pgfqpoint{2.992500in}{2.416550in}}%
\pgfpathlineto{\pgfqpoint{2.992796in}{2.421239in}}%
\pgfpathlineto{\pgfqpoint{2.992993in}{2.425877in}}%
\pgfpathlineto{\pgfqpoint{2.993487in}{2.417185in}}%
\pgfpathlineto{\pgfqpoint{2.993783in}{2.418892in}}%
\pgfpathlineto{\pgfqpoint{2.994079in}{2.419516in}}%
\pgfpathlineto{\pgfqpoint{2.995658in}{2.435078in}}%
\pgfpathlineto{\pgfqpoint{2.995757in}{2.434271in}}%
\pgfpathlineto{\pgfqpoint{2.996053in}{2.427836in}}%
\pgfpathlineto{\pgfqpoint{2.996349in}{2.443813in}}%
\pgfpathlineto{\pgfqpoint{2.996448in}{2.447163in}}%
\pgfpathlineto{\pgfqpoint{2.996842in}{2.438020in}}%
\pgfpathlineto{\pgfqpoint{2.997237in}{2.442359in}}%
\pgfpathlineto{\pgfqpoint{2.997632in}{2.426468in}}%
\pgfpathlineto{\pgfqpoint{2.998520in}{2.430627in}}%
\pgfpathlineto{\pgfqpoint{2.999310in}{2.432364in}}%
\pgfpathlineto{\pgfqpoint{3.000099in}{2.415971in}}%
\pgfpathlineto{\pgfqpoint{3.000396in}{2.421750in}}%
\pgfpathlineto{\pgfqpoint{3.000790in}{2.436752in}}%
\pgfpathlineto{\pgfqpoint{3.001679in}{2.431699in}}%
\pgfpathlineto{\pgfqpoint{3.002962in}{2.416893in}}%
\pgfpathlineto{\pgfqpoint{3.002468in}{2.434798in}}%
\pgfpathlineto{\pgfqpoint{3.003159in}{2.419975in}}%
\pgfpathlineto{\pgfqpoint{3.003258in}{2.420917in}}%
\pgfpathlineto{\pgfqpoint{3.003653in}{2.415486in}}%
\pgfpathlineto{\pgfqpoint{3.003850in}{2.416955in}}%
\pgfpathlineto{\pgfqpoint{3.004047in}{2.414157in}}%
\pgfpathlineto{\pgfqpoint{3.006120in}{2.371356in}}%
\pgfpathlineto{\pgfqpoint{3.006219in}{2.373966in}}%
\pgfpathlineto{\pgfqpoint{3.006515in}{2.386370in}}%
\pgfpathlineto{\pgfqpoint{3.007008in}{2.361760in}}%
\pgfpathlineto{\pgfqpoint{3.007206in}{2.363676in}}%
\pgfpathlineto{\pgfqpoint{3.007502in}{2.368744in}}%
\pgfpathlineto{\pgfqpoint{3.007995in}{2.356621in}}%
\pgfpathlineto{\pgfqpoint{3.008094in}{2.355628in}}%
\pgfpathlineto{\pgfqpoint{3.008390in}{2.362132in}}%
\pgfpathlineto{\pgfqpoint{3.009377in}{2.368900in}}%
\pgfpathlineto{\pgfqpoint{3.009476in}{2.367143in}}%
\pgfpathlineto{\pgfqpoint{3.010364in}{2.346389in}}%
\pgfpathlineto{\pgfqpoint{3.010562in}{2.356630in}}%
\pgfpathlineto{\pgfqpoint{3.010858in}{2.372249in}}%
\pgfpathlineto{\pgfqpoint{3.011252in}{2.349277in}}%
\pgfpathlineto{\pgfqpoint{3.011549in}{2.353914in}}%
\pgfpathlineto{\pgfqpoint{3.011845in}{2.351420in}}%
\pgfpathlineto{\pgfqpoint{3.012042in}{2.354770in}}%
\pgfpathlineto{\pgfqpoint{3.013128in}{2.373815in}}%
\pgfpathlineto{\pgfqpoint{3.013424in}{2.364560in}}%
\pgfpathlineto{\pgfqpoint{3.013621in}{2.357956in}}%
\pgfpathlineto{\pgfqpoint{3.014115in}{2.381421in}}%
\pgfpathlineto{\pgfqpoint{3.014806in}{2.397147in}}%
\pgfpathlineto{\pgfqpoint{3.015595in}{2.387152in}}%
\pgfpathlineto{\pgfqpoint{3.015891in}{2.377751in}}%
\pgfpathlineto{\pgfqpoint{3.016582in}{2.387962in}}%
\pgfpathlineto{\pgfqpoint{3.016681in}{2.387810in}}%
\pgfpathlineto{\pgfqpoint{3.019839in}{2.309878in}}%
\pgfpathlineto{\pgfqpoint{3.020135in}{2.316442in}}%
\pgfpathlineto{\pgfqpoint{3.021813in}{2.353448in}}%
\pgfpathlineto{\pgfqpoint{3.020826in}{2.310191in}}%
\pgfpathlineto{\pgfqpoint{3.022109in}{2.348436in}}%
\pgfpathlineto{\pgfqpoint{3.023787in}{2.311977in}}%
\pgfpathlineto{\pgfqpoint{3.023886in}{2.310387in}}%
\pgfpathlineto{\pgfqpoint{3.024182in}{2.319515in}}%
\pgfpathlineto{\pgfqpoint{3.025366in}{2.440275in}}%
\pgfpathlineto{\pgfqpoint{3.026057in}{2.551549in}}%
\pgfpathlineto{\pgfqpoint{3.026847in}{2.532593in}}%
\pgfpathlineto{\pgfqpoint{3.027636in}{2.481053in}}%
\pgfpathlineto{\pgfqpoint{3.028327in}{2.302200in}}%
\pgfpathlineto{\pgfqpoint{3.029117in}{2.337530in}}%
\pgfpathlineto{\pgfqpoint{3.029413in}{2.343542in}}%
\pgfpathlineto{\pgfqpoint{3.031486in}{2.423377in}}%
\pgfpathlineto{\pgfqpoint{3.031881in}{2.405197in}}%
\pgfpathlineto{\pgfqpoint{3.032374in}{2.383728in}}%
\pgfpathlineto{\pgfqpoint{3.033065in}{2.394607in}}%
\pgfpathlineto{\pgfqpoint{3.034841in}{2.447939in}}%
\pgfpathlineto{\pgfqpoint{3.035631in}{2.419974in}}%
\pgfpathlineto{\pgfqpoint{3.036322in}{2.431207in}}%
\pgfpathlineto{\pgfqpoint{3.037506in}{2.445514in}}%
\pgfpathlineto{\pgfqpoint{3.037013in}{2.429685in}}%
\pgfpathlineto{\pgfqpoint{3.037605in}{2.444875in}}%
\pgfpathlineto{\pgfqpoint{3.038493in}{2.424690in}}%
\pgfpathlineto{\pgfqpoint{3.039875in}{2.394024in}}%
\pgfpathlineto{\pgfqpoint{3.040270in}{2.400255in}}%
\pgfpathlineto{\pgfqpoint{3.040369in}{2.400030in}}%
\pgfpathlineto{\pgfqpoint{3.040467in}{2.400571in}}%
\pgfpathlineto{\pgfqpoint{3.041652in}{2.451606in}}%
\pgfpathlineto{\pgfqpoint{3.043428in}{2.501061in}}%
\pgfpathlineto{\pgfqpoint{3.044317in}{2.512898in}}%
\pgfpathlineto{\pgfqpoint{3.043823in}{2.498850in}}%
\pgfpathlineto{\pgfqpoint{3.044613in}{2.503715in}}%
\pgfpathlineto{\pgfqpoint{3.044909in}{2.497893in}}%
\pgfpathlineto{\pgfqpoint{3.045600in}{2.503661in}}%
\pgfpathlineto{\pgfqpoint{3.046685in}{2.518390in}}%
\pgfpathlineto{\pgfqpoint{3.047475in}{2.514422in}}%
\pgfpathlineto{\pgfqpoint{3.047672in}{2.513230in}}%
\pgfpathlineto{\pgfqpoint{3.047968in}{2.508176in}}%
\pgfpathlineto{\pgfqpoint{3.048363in}{2.521637in}}%
\pgfpathlineto{\pgfqpoint{3.048462in}{2.522306in}}%
\pgfpathlineto{\pgfqpoint{3.048561in}{2.518424in}}%
\pgfpathlineto{\pgfqpoint{3.048955in}{2.495096in}}%
\pgfpathlineto{\pgfqpoint{3.049745in}{2.510265in}}%
\pgfpathlineto{\pgfqpoint{3.049844in}{2.510506in}}%
\pgfpathlineto{\pgfqpoint{3.049942in}{2.509297in}}%
\pgfpathlineto{\pgfqpoint{3.053199in}{2.437966in}}%
\pgfpathlineto{\pgfqpoint{3.053890in}{2.451145in}}%
\pgfpathlineto{\pgfqpoint{3.054088in}{2.451931in}}%
\pgfpathlineto{\pgfqpoint{3.054186in}{2.451678in}}%
\pgfpathlineto{\pgfqpoint{3.054581in}{2.439631in}}%
\pgfpathlineto{\pgfqpoint{3.055075in}{2.452988in}}%
\pgfpathlineto{\pgfqpoint{3.055568in}{2.445353in}}%
\pgfpathlineto{\pgfqpoint{3.055864in}{2.450181in}}%
\pgfpathlineto{\pgfqpoint{3.056160in}{2.445211in}}%
\pgfpathlineto{\pgfqpoint{3.056654in}{2.448193in}}%
\pgfpathlineto{\pgfqpoint{3.057049in}{2.435827in}}%
\pgfpathlineto{\pgfqpoint{3.057937in}{2.440627in}}%
\pgfpathlineto{\pgfqpoint{3.058233in}{2.441162in}}%
\pgfpathlineto{\pgfqpoint{3.058332in}{2.440005in}}%
\pgfpathlineto{\pgfqpoint{3.059319in}{2.432296in}}%
\pgfpathlineto{\pgfqpoint{3.059516in}{2.435747in}}%
\pgfpathlineto{\pgfqpoint{3.060997in}{2.459234in}}%
\pgfpathlineto{\pgfqpoint{3.061293in}{2.453192in}}%
\pgfpathlineto{\pgfqpoint{3.061392in}{2.452040in}}%
\pgfpathlineto{\pgfqpoint{3.061589in}{2.458646in}}%
\pgfpathlineto{\pgfqpoint{3.062280in}{2.476290in}}%
\pgfpathlineto{\pgfqpoint{3.062773in}{2.464945in}}%
\pgfpathlineto{\pgfqpoint{3.064352in}{2.429932in}}%
\pgfpathlineto{\pgfqpoint{3.064747in}{2.403971in}}%
\pgfpathlineto{\pgfqpoint{3.065636in}{2.412403in}}%
\pgfpathlineto{\pgfqpoint{3.065833in}{2.415261in}}%
\pgfpathlineto{\pgfqpoint{3.066425in}{2.407908in}}%
\pgfpathlineto{\pgfqpoint{3.066919in}{2.393701in}}%
\pgfpathlineto{\pgfqpoint{3.067906in}{2.398789in}}%
\pgfpathlineto{\pgfqpoint{3.068597in}{2.420023in}}%
\pgfpathlineto{\pgfqpoint{3.069090in}{2.450142in}}%
\pgfpathlineto{\pgfqpoint{3.069682in}{2.417822in}}%
\pgfpathlineto{\pgfqpoint{3.071064in}{2.391308in}}%
\pgfpathlineto{\pgfqpoint{3.071163in}{2.395175in}}%
\pgfpathlineto{\pgfqpoint{3.073038in}{2.605857in}}%
\pgfpathlineto{\pgfqpoint{3.073433in}{2.644714in}}%
\pgfpathlineto{\pgfqpoint{3.074222in}{2.623302in}}%
\pgfpathlineto{\pgfqpoint{3.074518in}{2.618746in}}%
\pgfpathlineto{\pgfqpoint{3.075111in}{2.512138in}}%
\pgfpathlineto{\pgfqpoint{3.075604in}{2.385457in}}%
\pgfpathlineto{\pgfqpoint{3.076394in}{2.435163in}}%
\pgfpathlineto{\pgfqpoint{3.076887in}{2.443464in}}%
\pgfpathlineto{\pgfqpoint{3.078861in}{2.512852in}}%
\pgfpathlineto{\pgfqpoint{3.078960in}{2.511501in}}%
\pgfpathlineto{\pgfqpoint{3.079750in}{2.462097in}}%
\pgfpathlineto{\pgfqpoint{3.080638in}{2.485711in}}%
\pgfpathlineto{\pgfqpoint{3.080736in}{2.485431in}}%
\pgfpathlineto{\pgfqpoint{3.080835in}{2.486890in}}%
\pgfpathlineto{\pgfqpoint{3.082020in}{2.529694in}}%
\pgfpathlineto{\pgfqpoint{3.082316in}{2.508020in}}%
\pgfpathlineto{\pgfqpoint{3.083105in}{2.494366in}}%
\pgfpathlineto{\pgfqpoint{3.083401in}{2.507256in}}%
\pgfpathlineto{\pgfqpoint{3.084882in}{2.529211in}}%
\pgfpathlineto{\pgfqpoint{3.085178in}{2.526437in}}%
\pgfpathlineto{\pgfqpoint{3.085375in}{2.533419in}}%
\pgfpathlineto{\pgfqpoint{3.086461in}{2.545980in}}%
\pgfpathlineto{\pgfqpoint{3.085968in}{2.531171in}}%
\pgfpathlineto{\pgfqpoint{3.086560in}{2.545206in}}%
\pgfpathlineto{\pgfqpoint{3.086955in}{2.530134in}}%
\pgfpathlineto{\pgfqpoint{3.087349in}{2.552747in}}%
\pgfpathlineto{\pgfqpoint{3.087744in}{2.539345in}}%
\pgfpathlineto{\pgfqpoint{3.087942in}{2.541951in}}%
\pgfpathlineto{\pgfqpoint{3.088534in}{2.547147in}}%
\pgfpathlineto{\pgfqpoint{3.088929in}{2.539034in}}%
\pgfpathlineto{\pgfqpoint{3.089027in}{2.539011in}}%
\pgfpathlineto{\pgfqpoint{3.089422in}{2.563746in}}%
\pgfpathlineto{\pgfqpoint{3.090409in}{2.549641in}}%
\pgfpathlineto{\pgfqpoint{3.090705in}{2.543535in}}%
\pgfpathlineto{\pgfqpoint{3.091297in}{2.556583in}}%
\pgfpathlineto{\pgfqpoint{3.091396in}{2.556721in}}%
\pgfpathlineto{\pgfqpoint{3.091495in}{2.555699in}}%
\pgfpathlineto{\pgfqpoint{3.092383in}{2.537718in}}%
\pgfpathlineto{\pgfqpoint{3.092778in}{2.550273in}}%
\pgfpathlineto{\pgfqpoint{3.092876in}{2.551060in}}%
\pgfpathlineto{\pgfqpoint{3.093074in}{2.544374in}}%
\pgfpathlineto{\pgfqpoint{3.093173in}{2.542613in}}%
\pgfpathlineto{\pgfqpoint{3.093469in}{2.555832in}}%
\pgfpathlineto{\pgfqpoint{3.093567in}{2.559841in}}%
\pgfpathlineto{\pgfqpoint{3.094160in}{2.543275in}}%
\pgfpathlineto{\pgfqpoint{3.094258in}{2.543365in}}%
\pgfpathlineto{\pgfqpoint{3.094357in}{2.543303in}}%
\pgfpathlineto{\pgfqpoint{3.094456in}{2.544121in}}%
\pgfpathlineto{\pgfqpoint{3.095541in}{2.556492in}}%
\pgfpathlineto{\pgfqpoint{3.095837in}{2.549106in}}%
\pgfpathlineto{\pgfqpoint{3.096035in}{2.542629in}}%
\pgfpathlineto{\pgfqpoint{3.096923in}{2.550162in}}%
\pgfpathlineto{\pgfqpoint{3.100674in}{2.487834in}}%
\pgfpathlineto{\pgfqpoint{3.100871in}{2.494719in}}%
\pgfpathlineto{\pgfqpoint{3.101167in}{2.510211in}}%
\pgfpathlineto{\pgfqpoint{3.101562in}{2.493975in}}%
\pgfpathlineto{\pgfqpoint{3.102055in}{2.502741in}}%
\pgfpathlineto{\pgfqpoint{3.103042in}{2.490700in}}%
\pgfpathlineto{\pgfqpoint{3.102648in}{2.503123in}}%
\pgfpathlineto{\pgfqpoint{3.103240in}{2.499111in}}%
\pgfpathlineto{\pgfqpoint{3.103437in}{2.504957in}}%
\pgfpathlineto{\pgfqpoint{3.103832in}{2.487284in}}%
\pgfpathlineto{\pgfqpoint{3.104227in}{2.493944in}}%
\pgfpathlineto{\pgfqpoint{3.104622in}{2.478026in}}%
\pgfpathlineto{\pgfqpoint{3.105411in}{2.483699in}}%
\pgfpathlineto{\pgfqpoint{3.105707in}{2.491002in}}%
\pgfpathlineto{\pgfqpoint{3.106102in}{2.478277in}}%
\pgfpathlineto{\pgfqpoint{3.106497in}{2.487242in}}%
\pgfpathlineto{\pgfqpoint{3.106892in}{2.475577in}}%
\pgfpathlineto{\pgfqpoint{3.107484in}{2.487497in}}%
\pgfpathlineto{\pgfqpoint{3.109359in}{2.530073in}}%
\pgfpathlineto{\pgfqpoint{3.108274in}{2.482944in}}%
\pgfpathlineto{\pgfqpoint{3.110050in}{2.512328in}}%
\pgfpathlineto{\pgfqpoint{3.112419in}{2.384016in}}%
\pgfpathlineto{\pgfqpoint{3.112912in}{2.402478in}}%
\pgfpathlineto{\pgfqpoint{3.114590in}{2.456748in}}%
\pgfpathlineto{\pgfqpoint{3.114886in}{2.453210in}}%
\pgfpathlineto{\pgfqpoint{3.115577in}{2.456548in}}%
\pgfpathlineto{\pgfqpoint{3.116367in}{2.497070in}}%
\pgfpathlineto{\pgfqpoint{3.116762in}{2.469366in}}%
\pgfpathlineto{\pgfqpoint{3.117551in}{2.435312in}}%
\pgfpathlineto{\pgfqpoint{3.118045in}{2.453245in}}%
\pgfpathlineto{\pgfqpoint{3.118341in}{2.435526in}}%
\pgfpathlineto{\pgfqpoint{3.118834in}{2.465691in}}%
\pgfpathlineto{\pgfqpoint{3.119821in}{2.558264in}}%
\pgfpathlineto{\pgfqpoint{3.120710in}{2.683593in}}%
\pgfpathlineto{\pgfqpoint{3.121400in}{2.657234in}}%
\pgfpathlineto{\pgfqpoint{3.121499in}{2.657154in}}%
\pgfpathlineto{\pgfqpoint{3.121795in}{2.662174in}}%
\pgfpathlineto{\pgfqpoint{3.121993in}{2.645699in}}%
\pgfpathlineto{\pgfqpoint{3.122881in}{2.414456in}}%
\pgfpathlineto{\pgfqpoint{3.123967in}{2.461876in}}%
\pgfpathlineto{\pgfqpoint{3.126039in}{2.536992in}}%
\pgfpathlineto{\pgfqpoint{3.126335in}{2.517734in}}%
\pgfpathlineto{\pgfqpoint{3.127224in}{2.475429in}}%
\pgfpathlineto{\pgfqpoint{3.127619in}{2.487955in}}%
\pgfpathlineto{\pgfqpoint{3.128902in}{2.512744in}}%
\pgfpathlineto{\pgfqpoint{3.128211in}{2.482131in}}%
\pgfpathlineto{\pgfqpoint{3.129494in}{2.509510in}}%
\pgfpathlineto{\pgfqpoint{3.130481in}{2.480812in}}%
\pgfpathlineto{\pgfqpoint{3.130876in}{2.497396in}}%
\pgfpathlineto{\pgfqpoint{3.131764in}{2.503659in}}%
\pgfpathlineto{\pgfqpoint{3.131369in}{2.494581in}}%
\pgfpathlineto{\pgfqpoint{3.131961in}{2.497820in}}%
\pgfpathlineto{\pgfqpoint{3.132060in}{2.495458in}}%
\pgfpathlineto{\pgfqpoint{3.132553in}{2.507243in}}%
\pgfpathlineto{\pgfqpoint{3.132850in}{2.500565in}}%
\pgfpathlineto{\pgfqpoint{3.133047in}{2.501067in}}%
\pgfpathlineto{\pgfqpoint{3.133244in}{2.498213in}}%
\pgfpathlineto{\pgfqpoint{3.133442in}{2.494795in}}%
\pgfpathlineto{\pgfqpoint{3.133837in}{2.502234in}}%
\pgfpathlineto{\pgfqpoint{3.134330in}{2.495979in}}%
\pgfpathlineto{\pgfqpoint{3.135120in}{2.500111in}}%
\pgfpathlineto{\pgfqpoint{3.135416in}{2.495636in}}%
\pgfpathlineto{\pgfqpoint{3.135613in}{2.492523in}}%
\pgfpathlineto{\pgfqpoint{3.136107in}{2.503963in}}%
\pgfpathlineto{\pgfqpoint{3.136501in}{2.507542in}}%
\pgfpathlineto{\pgfqpoint{3.136896in}{2.498419in}}%
\pgfpathlineto{\pgfqpoint{3.137094in}{2.503497in}}%
\pgfpathlineto{\pgfqpoint{3.137390in}{2.515009in}}%
\pgfpathlineto{\pgfqpoint{3.137784in}{2.496461in}}%
\pgfpathlineto{\pgfqpoint{3.138278in}{2.510844in}}%
\pgfpathlineto{\pgfqpoint{3.140055in}{2.492338in}}%
\pgfpathlineto{\pgfqpoint{3.138870in}{2.512096in}}%
\pgfpathlineto{\pgfqpoint{3.140252in}{2.496881in}}%
\pgfpathlineto{\pgfqpoint{3.140449in}{2.499179in}}%
\pgfpathlineto{\pgfqpoint{3.140943in}{2.489574in}}%
\pgfpathlineto{\pgfqpoint{3.141042in}{2.490160in}}%
\pgfpathlineto{\pgfqpoint{3.141634in}{2.498348in}}%
\pgfpathlineto{\pgfqpoint{3.142325in}{2.492050in}}%
\pgfpathlineto{\pgfqpoint{3.143410in}{2.480203in}}%
\pgfpathlineto{\pgfqpoint{3.142917in}{2.508469in}}%
\pgfpathlineto{\pgfqpoint{3.143509in}{2.482704in}}%
\pgfpathlineto{\pgfqpoint{3.143805in}{2.494562in}}%
\pgfpathlineto{\pgfqpoint{3.144496in}{2.476936in}}%
\pgfpathlineto{\pgfqpoint{3.146075in}{2.452814in}}%
\pgfpathlineto{\pgfqpoint{3.146174in}{2.454573in}}%
\pgfpathlineto{\pgfqpoint{3.146470in}{2.460609in}}%
\pgfpathlineto{\pgfqpoint{3.146963in}{2.450098in}}%
\pgfpathlineto{\pgfqpoint{3.148049in}{2.432322in}}%
\pgfpathlineto{\pgfqpoint{3.148148in}{2.435564in}}%
\pgfpathlineto{\pgfqpoint{3.149135in}{2.445685in}}%
\pgfpathlineto{\pgfqpoint{3.148740in}{2.429653in}}%
\pgfpathlineto{\pgfqpoint{3.149234in}{2.444465in}}%
\pgfpathlineto{\pgfqpoint{3.149530in}{2.428547in}}%
\pgfpathlineto{\pgfqpoint{3.149924in}{2.445144in}}%
\pgfpathlineto{\pgfqpoint{3.150418in}{2.434565in}}%
\pgfpathlineto{\pgfqpoint{3.150813in}{2.447649in}}%
\pgfpathlineto{\pgfqpoint{3.151405in}{2.433349in}}%
\pgfpathlineto{\pgfqpoint{3.151701in}{2.431202in}}%
\pgfpathlineto{\pgfqpoint{3.151898in}{2.436268in}}%
\pgfpathlineto{\pgfqpoint{3.152984in}{2.450986in}}%
\pgfpathlineto{\pgfqpoint{3.152589in}{2.431333in}}%
\pgfpathlineto{\pgfqpoint{3.153182in}{2.446590in}}%
\pgfpathlineto{\pgfqpoint{3.153379in}{2.438894in}}%
\pgfpathlineto{\pgfqpoint{3.153872in}{2.448056in}}%
\pgfpathlineto{\pgfqpoint{3.154169in}{2.445620in}}%
\pgfpathlineto{\pgfqpoint{3.156142in}{2.487988in}}%
\pgfpathlineto{\pgfqpoint{3.156340in}{2.485175in}}%
\pgfpathlineto{\pgfqpoint{3.156439in}{2.483612in}}%
\pgfpathlineto{\pgfqpoint{3.156833in}{2.493310in}}%
\pgfpathlineto{\pgfqpoint{3.156932in}{2.495177in}}%
\pgfpathlineto{\pgfqpoint{3.157327in}{2.482871in}}%
\pgfpathlineto{\pgfqpoint{3.159202in}{2.416889in}}%
\pgfpathlineto{\pgfqpoint{3.159498in}{2.422545in}}%
\pgfpathlineto{\pgfqpoint{3.160485in}{2.408316in}}%
\pgfpathlineto{\pgfqpoint{3.161374in}{2.379637in}}%
\pgfpathlineto{\pgfqpoint{3.161867in}{2.385127in}}%
\pgfpathlineto{\pgfqpoint{3.162064in}{2.378328in}}%
\pgfpathlineto{\pgfqpoint{3.162854in}{2.383095in}}%
\pgfpathlineto{\pgfqpoint{3.163348in}{2.427478in}}%
\pgfpathlineto{\pgfqpoint{3.164137in}{2.403091in}}%
\pgfpathlineto{\pgfqpoint{3.165519in}{2.367161in}}%
\pgfpathlineto{\pgfqpoint{3.165716in}{2.364932in}}%
\pgfpathlineto{\pgfqpoint{3.165914in}{2.375781in}}%
\pgfpathlineto{\pgfqpoint{3.167888in}{2.617114in}}%
\pgfpathlineto{\pgfqpoint{3.168973in}{2.597526in}}%
\pgfpathlineto{\pgfqpoint{3.169072in}{2.597581in}}%
\pgfpathlineto{\pgfqpoint{3.169566in}{2.483186in}}%
\pgfpathlineto{\pgfqpoint{3.170059in}{2.364724in}}%
\pgfpathlineto{\pgfqpoint{3.170849in}{2.405940in}}%
\pgfpathlineto{\pgfqpoint{3.171046in}{2.402381in}}%
\pgfpathlineto{\pgfqpoint{3.171441in}{2.417171in}}%
\pgfpathlineto{\pgfqpoint{3.173217in}{2.472212in}}%
\pgfpathlineto{\pgfqpoint{3.172230in}{2.416155in}}%
\pgfpathlineto{\pgfqpoint{3.173316in}{2.471310in}}%
\pgfpathlineto{\pgfqpoint{3.174402in}{2.412750in}}%
\pgfpathlineto{\pgfqpoint{3.174895in}{2.442378in}}%
\pgfpathlineto{\pgfqpoint{3.175093in}{2.438602in}}%
\pgfpathlineto{\pgfqpoint{3.175290in}{2.432283in}}%
\pgfpathlineto{\pgfqpoint{3.175882in}{2.451670in}}%
\pgfpathlineto{\pgfqpoint{3.176080in}{2.452865in}}%
\pgfpathlineto{\pgfqpoint{3.176474in}{2.467023in}}%
\pgfpathlineto{\pgfqpoint{3.177067in}{2.448749in}}%
\pgfpathlineto{\pgfqpoint{3.177461in}{2.441482in}}%
\pgfpathlineto{\pgfqpoint{3.177758in}{2.451038in}}%
\pgfpathlineto{\pgfqpoint{3.177955in}{2.459467in}}%
\pgfpathlineto{\pgfqpoint{3.178745in}{2.444799in}}%
\pgfpathlineto{\pgfqpoint{3.180620in}{2.398684in}}%
\pgfpathlineto{\pgfqpoint{3.180817in}{2.399379in}}%
\pgfpathlineto{\pgfqpoint{3.181015in}{2.401086in}}%
\pgfpathlineto{\pgfqpoint{3.183976in}{2.524899in}}%
\pgfpathlineto{\pgfqpoint{3.184666in}{2.514460in}}%
\pgfpathlineto{\pgfqpoint{3.184765in}{2.514671in}}%
\pgfpathlineto{\pgfqpoint{3.184864in}{2.512885in}}%
\pgfpathlineto{\pgfqpoint{3.185160in}{2.504614in}}%
\pgfpathlineto{\pgfqpoint{3.185653in}{2.516901in}}%
\pgfpathlineto{\pgfqpoint{3.185851in}{2.514520in}}%
\pgfpathlineto{\pgfqpoint{3.186048in}{2.515425in}}%
\pgfpathlineto{\pgfqpoint{3.186147in}{2.516497in}}%
\pgfpathlineto{\pgfqpoint{3.186443in}{2.509954in}}%
\pgfpathlineto{\pgfqpoint{3.186640in}{2.504399in}}%
\pgfpathlineto{\pgfqpoint{3.187430in}{2.513546in}}%
\pgfpathlineto{\pgfqpoint{3.187627in}{2.515775in}}%
\pgfpathlineto{\pgfqpoint{3.188121in}{2.508251in}}%
\pgfpathlineto{\pgfqpoint{3.188614in}{2.515248in}}%
\pgfpathlineto{\pgfqpoint{3.189009in}{2.512381in}}%
\pgfpathlineto{\pgfqpoint{3.189305in}{2.517073in}}%
\pgfpathlineto{\pgfqpoint{3.189404in}{2.517388in}}%
\pgfpathlineto{\pgfqpoint{3.189601in}{2.514406in}}%
\pgfpathlineto{\pgfqpoint{3.189700in}{2.514179in}}%
\pgfpathlineto{\pgfqpoint{3.189799in}{2.516191in}}%
\pgfpathlineto{\pgfqpoint{3.190095in}{2.524436in}}%
\pgfpathlineto{\pgfqpoint{3.190490in}{2.512450in}}%
\pgfpathlineto{\pgfqpoint{3.190885in}{2.517372in}}%
\pgfpathlineto{\pgfqpoint{3.192760in}{2.474870in}}%
\pgfpathlineto{\pgfqpoint{3.195129in}{2.437874in}}%
\pgfpathlineto{\pgfqpoint{3.195819in}{2.449662in}}%
\pgfpathlineto{\pgfqpoint{3.196116in}{2.454076in}}%
\pgfpathlineto{\pgfqpoint{3.196412in}{2.443582in}}%
\pgfpathlineto{\pgfqpoint{3.197300in}{2.429545in}}%
\pgfpathlineto{\pgfqpoint{3.197596in}{2.436626in}}%
\pgfpathlineto{\pgfqpoint{3.197695in}{2.437551in}}%
\pgfpathlineto{\pgfqpoint{3.197892in}{2.433070in}}%
\pgfpathlineto{\pgfqpoint{3.198682in}{2.421421in}}%
\pgfpathlineto{\pgfqpoint{3.199077in}{2.427066in}}%
\pgfpathlineto{\pgfqpoint{3.199274in}{2.428582in}}%
\pgfpathlineto{\pgfqpoint{3.199767in}{2.421169in}}%
\pgfpathlineto{\pgfqpoint{3.201051in}{2.424681in}}%
\pgfpathlineto{\pgfqpoint{3.201840in}{2.439593in}}%
\pgfpathlineto{\pgfqpoint{3.202235in}{2.427052in}}%
\pgfpathlineto{\pgfqpoint{3.202334in}{2.426359in}}%
\pgfpathlineto{\pgfqpoint{3.202531in}{2.432334in}}%
\pgfpathlineto{\pgfqpoint{3.203419in}{2.462567in}}%
\pgfpathlineto{\pgfqpoint{3.203814in}{2.438409in}}%
\pgfpathlineto{\pgfqpoint{3.204308in}{2.444526in}}%
\pgfpathlineto{\pgfqpoint{3.206183in}{2.388058in}}%
\pgfpathlineto{\pgfqpoint{3.206479in}{2.380978in}}%
\pgfpathlineto{\pgfqpoint{3.207170in}{2.385567in}}%
\pgfpathlineto{\pgfqpoint{3.207466in}{2.402376in}}%
\pgfpathlineto{\pgfqpoint{3.207861in}{2.371307in}}%
\pgfpathlineto{\pgfqpoint{3.208354in}{2.374305in}}%
\pgfpathlineto{\pgfqpoint{3.209144in}{2.359284in}}%
\pgfpathlineto{\pgfqpoint{3.209341in}{2.353832in}}%
\pgfpathlineto{\pgfqpoint{3.210032in}{2.366390in}}%
\pgfpathlineto{\pgfqpoint{3.210427in}{2.412001in}}%
\pgfpathlineto{\pgfqpoint{3.211118in}{2.373193in}}%
\pgfpathlineto{\pgfqpoint{3.212203in}{2.343366in}}%
\pgfpathlineto{\pgfqpoint{3.212500in}{2.346740in}}%
\pgfpathlineto{\pgfqpoint{3.212796in}{2.332947in}}%
\pgfpathlineto{\pgfqpoint{3.213190in}{2.358328in}}%
\pgfpathlineto{\pgfqpoint{3.214375in}{2.508157in}}%
\pgfpathlineto{\pgfqpoint{3.215362in}{2.570982in}}%
\pgfpathlineto{\pgfqpoint{3.215658in}{2.564138in}}%
\pgfpathlineto{\pgfqpoint{3.216349in}{2.510065in}}%
\pgfpathlineto{\pgfqpoint{3.217138in}{2.303030in}}%
\pgfpathlineto{\pgfqpoint{3.218027in}{2.350150in}}%
\pgfpathlineto{\pgfqpoint{3.218520in}{2.353422in}}%
\pgfpathlineto{\pgfqpoint{3.220297in}{2.422692in}}%
\pgfpathlineto{\pgfqpoint{3.220494in}{2.415221in}}%
\pgfpathlineto{\pgfqpoint{3.221481in}{2.361977in}}%
\pgfpathlineto{\pgfqpoint{3.221975in}{2.379225in}}%
\pgfpathlineto{\pgfqpoint{3.222271in}{2.375431in}}%
\pgfpathlineto{\pgfqpoint{3.222567in}{2.380022in}}%
\pgfpathlineto{\pgfqpoint{3.223455in}{2.416646in}}%
\pgfpathlineto{\pgfqpoint{3.223949in}{2.392697in}}%
\pgfpathlineto{\pgfqpoint{3.224442in}{2.378593in}}%
\pgfpathlineto{\pgfqpoint{3.224936in}{2.393649in}}%
\pgfpathlineto{\pgfqpoint{3.226614in}{2.414999in}}%
\pgfpathlineto{\pgfqpoint{3.227206in}{2.424744in}}%
\pgfpathlineto{\pgfqpoint{3.227897in}{2.417709in}}%
\pgfpathlineto{\pgfqpoint{3.228193in}{2.415246in}}%
\pgfpathlineto{\pgfqpoint{3.228390in}{2.420580in}}%
\pgfpathlineto{\pgfqpoint{3.228686in}{2.430971in}}%
\pgfpathlineto{\pgfqpoint{3.229575in}{2.428784in}}%
\pgfpathlineto{\pgfqpoint{3.229871in}{2.416616in}}%
\pgfpathlineto{\pgfqpoint{3.230364in}{2.439854in}}%
\pgfpathlineto{\pgfqpoint{3.230463in}{2.441145in}}%
\pgfpathlineto{\pgfqpoint{3.230759in}{2.432322in}}%
\pgfpathlineto{\pgfqpoint{3.231647in}{2.432637in}}%
\pgfpathlineto{\pgfqpoint{3.232042in}{2.428106in}}%
\pgfpathlineto{\pgfqpoint{3.233226in}{2.442971in}}%
\pgfpathlineto{\pgfqpoint{3.233424in}{2.437135in}}%
\pgfpathlineto{\pgfqpoint{3.233621in}{2.429843in}}%
\pgfpathlineto{\pgfqpoint{3.234411in}{2.440177in}}%
\pgfpathlineto{\pgfqpoint{3.235299in}{2.446262in}}%
\pgfpathlineto{\pgfqpoint{3.234904in}{2.438950in}}%
\pgfpathlineto{\pgfqpoint{3.235398in}{2.443864in}}%
\pgfpathlineto{\pgfqpoint{3.235694in}{2.434076in}}%
\pgfpathlineto{\pgfqpoint{3.236089in}{2.444002in}}%
\pgfpathlineto{\pgfqpoint{3.236483in}{2.441496in}}%
\pgfpathlineto{\pgfqpoint{3.236582in}{2.441373in}}%
\pgfpathlineto{\pgfqpoint{3.237372in}{2.448454in}}%
\pgfpathlineto{\pgfqpoint{3.238161in}{2.428851in}}%
\pgfpathlineto{\pgfqpoint{3.241122in}{2.393540in}}%
\pgfpathlineto{\pgfqpoint{3.238556in}{2.429366in}}%
\pgfpathlineto{\pgfqpoint{3.241418in}{2.401141in}}%
\pgfpathlineto{\pgfqpoint{3.241517in}{2.402378in}}%
\pgfpathlineto{\pgfqpoint{3.241912in}{2.394223in}}%
\pgfpathlineto{\pgfqpoint{3.242504in}{2.377420in}}%
\pgfpathlineto{\pgfqpoint{3.243195in}{2.387785in}}%
\pgfpathlineto{\pgfqpoint{3.244182in}{2.369450in}}%
\pgfpathlineto{\pgfqpoint{3.244478in}{2.378222in}}%
\pgfpathlineto{\pgfqpoint{3.244675in}{2.385439in}}%
\pgfpathlineto{\pgfqpoint{3.245070in}{2.377479in}}%
\pgfpathlineto{\pgfqpoint{3.245564in}{2.383584in}}%
\pgfpathlineto{\pgfqpoint{3.246452in}{2.367069in}}%
\pgfpathlineto{\pgfqpoint{3.247044in}{2.374448in}}%
\pgfpathlineto{\pgfqpoint{3.248031in}{2.359969in}}%
\pgfpathlineto{\pgfqpoint{3.247538in}{2.380410in}}%
\pgfpathlineto{\pgfqpoint{3.248229in}{2.373350in}}%
\pgfpathlineto{\pgfqpoint{3.248722in}{2.393536in}}%
\pgfpathlineto{\pgfqpoint{3.249314in}{2.371731in}}%
\pgfpathlineto{\pgfqpoint{3.249709in}{2.390121in}}%
\pgfpathlineto{\pgfqpoint{3.250203in}{2.367758in}}%
\pgfpathlineto{\pgfqpoint{3.250597in}{2.378908in}}%
\pgfpathlineto{\pgfqpoint{3.251979in}{2.356622in}}%
\pgfpathlineto{\pgfqpoint{3.252177in}{2.362744in}}%
\pgfpathlineto{\pgfqpoint{3.252867in}{2.401091in}}%
\pgfpathlineto{\pgfqpoint{3.253657in}{2.392629in}}%
\pgfpathlineto{\pgfqpoint{3.253756in}{2.388929in}}%
\pgfpathlineto{\pgfqpoint{3.254249in}{2.406915in}}%
\pgfpathlineto{\pgfqpoint{3.254447in}{2.404313in}}%
\pgfpathlineto{\pgfqpoint{3.254545in}{2.404137in}}%
\pgfpathlineto{\pgfqpoint{3.254644in}{2.406127in}}%
\pgfpathlineto{\pgfqpoint{3.254841in}{2.409725in}}%
\pgfpathlineto{\pgfqpoint{3.255138in}{2.396383in}}%
\pgfpathlineto{\pgfqpoint{3.256618in}{2.365274in}}%
\pgfpathlineto{\pgfqpoint{3.256717in}{2.366784in}}%
\pgfpathlineto{\pgfqpoint{3.257901in}{2.432308in}}%
\pgfpathlineto{\pgfqpoint{3.258592in}{2.401176in}}%
\pgfpathlineto{\pgfqpoint{3.259974in}{2.367205in}}%
\pgfpathlineto{\pgfqpoint{3.260171in}{2.373719in}}%
\pgfpathlineto{\pgfqpoint{3.262046in}{2.605952in}}%
\pgfpathlineto{\pgfqpoint{3.262343in}{2.630708in}}%
\pgfpathlineto{\pgfqpoint{3.263231in}{2.628821in}}%
\pgfpathlineto{\pgfqpoint{3.263626in}{2.580346in}}%
\pgfpathlineto{\pgfqpoint{3.264514in}{2.368712in}}%
\pgfpathlineto{\pgfqpoint{3.265501in}{2.402104in}}%
\pgfpathlineto{\pgfqpoint{3.266981in}{2.446442in}}%
\pgfpathlineto{\pgfqpoint{3.267771in}{2.482244in}}%
\pgfpathlineto{\pgfqpoint{3.268166in}{2.462466in}}%
\pgfpathlineto{\pgfqpoint{3.268659in}{2.428177in}}%
\pgfpathlineto{\pgfqpoint{3.269449in}{2.443348in}}%
\pgfpathlineto{\pgfqpoint{3.269548in}{2.441514in}}%
\pgfpathlineto{\pgfqpoint{3.270041in}{2.450446in}}%
\pgfpathlineto{\pgfqpoint{3.270140in}{2.450450in}}%
\pgfpathlineto{\pgfqpoint{3.270436in}{2.460098in}}%
\pgfpathlineto{\pgfqpoint{3.270831in}{2.484262in}}%
\pgfpathlineto{\pgfqpoint{3.271423in}{2.452364in}}%
\pgfpathlineto{\pgfqpoint{3.272410in}{2.437153in}}%
\pgfpathlineto{\pgfqpoint{3.272607in}{2.445262in}}%
\pgfpathlineto{\pgfqpoint{3.272903in}{2.464233in}}%
\pgfpathlineto{\pgfqpoint{3.273792in}{2.456010in}}%
\pgfpathlineto{\pgfqpoint{3.274877in}{2.463399in}}%
\pgfpathlineto{\pgfqpoint{3.274483in}{2.452445in}}%
\pgfpathlineto{\pgfqpoint{3.275075in}{2.460416in}}%
\pgfpathlineto{\pgfqpoint{3.276160in}{2.455490in}}%
\pgfpathlineto{\pgfqpoint{3.276259in}{2.456607in}}%
\pgfpathlineto{\pgfqpoint{3.277542in}{2.475691in}}%
\pgfpathlineto{\pgfqpoint{3.277838in}{2.470115in}}%
\pgfpathlineto{\pgfqpoint{3.278036in}{2.466360in}}%
\pgfpathlineto{\pgfqpoint{3.278529in}{2.475818in}}%
\pgfpathlineto{\pgfqpoint{3.278924in}{2.468166in}}%
\pgfpathlineto{\pgfqpoint{3.279220in}{2.472575in}}%
\pgfpathlineto{\pgfqpoint{3.279516in}{2.465425in}}%
\pgfpathlineto{\pgfqpoint{3.280602in}{2.455522in}}%
\pgfpathlineto{\pgfqpoint{3.280108in}{2.472606in}}%
\pgfpathlineto{\pgfqpoint{3.280701in}{2.457976in}}%
\pgfpathlineto{\pgfqpoint{3.281095in}{2.466892in}}%
\pgfpathlineto{\pgfqpoint{3.281885in}{2.460928in}}%
\pgfpathlineto{\pgfqpoint{3.281984in}{2.460787in}}%
\pgfpathlineto{\pgfqpoint{3.282082in}{2.462379in}}%
\pgfpathlineto{\pgfqpoint{3.282378in}{2.466219in}}%
\pgfpathlineto{\pgfqpoint{3.282675in}{2.461301in}}%
\pgfpathlineto{\pgfqpoint{3.283168in}{2.462029in}}%
\pgfpathlineto{\pgfqpoint{3.284155in}{2.449106in}}%
\pgfpathlineto{\pgfqpoint{3.284254in}{2.452741in}}%
\pgfpathlineto{\pgfqpoint{3.284550in}{2.472210in}}%
\pgfpathlineto{\pgfqpoint{3.285043in}{2.450915in}}%
\pgfpathlineto{\pgfqpoint{3.285438in}{2.461144in}}%
\pgfpathlineto{\pgfqpoint{3.289386in}{2.393817in}}%
\pgfpathlineto{\pgfqpoint{3.290274in}{2.388485in}}%
\pgfpathlineto{\pgfqpoint{3.289880in}{2.407227in}}%
\pgfpathlineto{\pgfqpoint{3.290373in}{2.390094in}}%
\pgfpathlineto{\pgfqpoint{3.290669in}{2.408582in}}%
\pgfpathlineto{\pgfqpoint{3.291459in}{2.395188in}}%
\pgfpathlineto{\pgfqpoint{3.293729in}{2.363197in}}%
\pgfpathlineto{\pgfqpoint{3.293926in}{2.367070in}}%
\pgfpathlineto{\pgfqpoint{3.294222in}{2.380143in}}%
\pgfpathlineto{\pgfqpoint{3.295111in}{2.377408in}}%
\pgfpathlineto{\pgfqpoint{3.295900in}{2.364119in}}%
\pgfpathlineto{\pgfqpoint{3.296196in}{2.376056in}}%
\pgfpathlineto{\pgfqpoint{3.297183in}{2.388982in}}%
\pgfpathlineto{\pgfqpoint{3.296690in}{2.372206in}}%
\pgfpathlineto{\pgfqpoint{3.297381in}{2.383567in}}%
\pgfpathlineto{\pgfqpoint{3.297479in}{2.382259in}}%
\pgfpathlineto{\pgfqpoint{3.297677in}{2.387975in}}%
\pgfpathlineto{\pgfqpoint{3.299256in}{2.418253in}}%
\pgfpathlineto{\pgfqpoint{3.299651in}{2.408453in}}%
\pgfpathlineto{\pgfqpoint{3.301822in}{2.374008in}}%
\pgfpathlineto{\pgfqpoint{3.300144in}{2.413895in}}%
\pgfpathlineto{\pgfqpoint{3.302020in}{2.378172in}}%
\pgfpathlineto{\pgfqpoint{3.302316in}{2.385795in}}%
\pgfpathlineto{\pgfqpoint{3.302809in}{2.377682in}}%
\pgfpathlineto{\pgfqpoint{3.303105in}{2.377982in}}%
\pgfpathlineto{\pgfqpoint{3.304487in}{2.347541in}}%
\pgfpathlineto{\pgfqpoint{3.304981in}{2.354740in}}%
\pgfpathlineto{\pgfqpoint{3.305375in}{2.382832in}}%
\pgfpathlineto{\pgfqpoint{3.305770in}{2.415951in}}%
\pgfpathlineto{\pgfqpoint{3.306461in}{2.384585in}}%
\pgfpathlineto{\pgfqpoint{3.307349in}{2.355125in}}%
\pgfpathlineto{\pgfqpoint{3.307744in}{2.363540in}}%
\pgfpathlineto{\pgfqpoint{3.308830in}{2.396490in}}%
\pgfpathlineto{\pgfqpoint{3.310409in}{2.643701in}}%
\pgfpathlineto{\pgfqpoint{3.311495in}{2.577746in}}%
\pgfpathlineto{\pgfqpoint{3.312383in}{2.356605in}}%
\pgfpathlineto{\pgfqpoint{3.313567in}{2.381117in}}%
\pgfpathlineto{\pgfqpoint{3.315640in}{2.472653in}}%
\pgfpathlineto{\pgfqpoint{3.315936in}{2.447349in}}%
\pgfpathlineto{\pgfqpoint{3.316528in}{2.402047in}}%
\pgfpathlineto{\pgfqpoint{3.317120in}{2.422565in}}%
\pgfpathlineto{\pgfqpoint{3.318601in}{2.472259in}}%
\pgfpathlineto{\pgfqpoint{3.318996in}{2.459122in}}%
\pgfpathlineto{\pgfqpoint{3.319884in}{2.426743in}}%
\pgfpathlineto{\pgfqpoint{3.320772in}{2.435652in}}%
\pgfpathlineto{\pgfqpoint{3.321759in}{2.467073in}}%
\pgfpathlineto{\pgfqpoint{3.322154in}{2.454313in}}%
\pgfpathlineto{\pgfqpoint{3.322450in}{2.456603in}}%
\pgfpathlineto{\pgfqpoint{3.322549in}{2.454164in}}%
\pgfpathlineto{\pgfqpoint{3.324326in}{2.413043in}}%
\pgfpathlineto{\pgfqpoint{3.324424in}{2.415813in}}%
\pgfpathlineto{\pgfqpoint{3.327583in}{2.519837in}}%
\pgfpathlineto{\pgfqpoint{3.325016in}{2.406885in}}%
\pgfpathlineto{\pgfqpoint{3.327879in}{2.510412in}}%
\pgfpathlineto{\pgfqpoint{3.328471in}{2.518066in}}%
\pgfpathlineto{\pgfqpoint{3.329359in}{2.495039in}}%
\pgfpathlineto{\pgfqpoint{3.330445in}{2.513773in}}%
\pgfpathlineto{\pgfqpoint{3.331037in}{2.508200in}}%
\pgfpathlineto{\pgfqpoint{3.331333in}{2.514519in}}%
\pgfpathlineto{\pgfqpoint{3.331531in}{2.519660in}}%
\pgfpathlineto{\pgfqpoint{3.332024in}{2.504090in}}%
\pgfpathlineto{\pgfqpoint{3.332221in}{2.508053in}}%
\pgfpathlineto{\pgfqpoint{3.332912in}{2.513350in}}%
\pgfpathlineto{\pgfqpoint{3.332518in}{2.507737in}}%
\pgfpathlineto{\pgfqpoint{3.333110in}{2.508370in}}%
\pgfpathlineto{\pgfqpoint{3.334294in}{2.494730in}}%
\pgfpathlineto{\pgfqpoint{3.334393in}{2.494770in}}%
\pgfpathlineto{\pgfqpoint{3.334590in}{2.495901in}}%
\pgfpathlineto{\pgfqpoint{3.334985in}{2.494142in}}%
\pgfpathlineto{\pgfqpoint{3.335281in}{2.494653in}}%
\pgfpathlineto{\pgfqpoint{3.335676in}{2.482580in}}%
\pgfpathlineto{\pgfqpoint{3.336564in}{2.467306in}}%
\pgfpathlineto{\pgfqpoint{3.336959in}{2.475412in}}%
\pgfpathlineto{\pgfqpoint{3.337354in}{2.461998in}}%
\pgfpathlineto{\pgfqpoint{3.338439in}{2.463587in}}%
\pgfpathlineto{\pgfqpoint{3.338637in}{2.464943in}}%
\pgfpathlineto{\pgfqpoint{3.339624in}{2.470252in}}%
\pgfpathlineto{\pgfqpoint{3.339229in}{2.457254in}}%
\pgfpathlineto{\pgfqpoint{3.339821in}{2.466481in}}%
\pgfpathlineto{\pgfqpoint{3.340808in}{2.458285in}}%
\pgfpathlineto{\pgfqpoint{3.341104in}{2.459691in}}%
\pgfpathlineto{\pgfqpoint{3.342782in}{2.445917in}}%
\pgfpathlineto{\pgfqpoint{3.342881in}{2.446997in}}%
\pgfpathlineto{\pgfqpoint{3.343374in}{2.455220in}}%
\pgfpathlineto{\pgfqpoint{3.344065in}{2.449644in}}%
\pgfpathlineto{\pgfqpoint{3.344164in}{2.448612in}}%
\pgfpathlineto{\pgfqpoint{3.344559in}{2.455738in}}%
\pgfpathlineto{\pgfqpoint{3.346335in}{2.490634in}}%
\pgfpathlineto{\pgfqpoint{3.346730in}{2.479579in}}%
\pgfpathlineto{\pgfqpoint{3.346829in}{2.479098in}}%
\pgfpathlineto{\pgfqpoint{3.346928in}{2.482413in}}%
\pgfpathlineto{\pgfqpoint{3.347224in}{2.495068in}}%
\pgfpathlineto{\pgfqpoint{3.347915in}{2.482811in}}%
\pgfpathlineto{\pgfqpoint{3.352455in}{2.367051in}}%
\pgfpathlineto{\pgfqpoint{3.353047in}{2.382992in}}%
\pgfpathlineto{\pgfqpoint{3.353639in}{2.432932in}}%
\pgfpathlineto{\pgfqpoint{3.354231in}{2.400598in}}%
\pgfpathlineto{\pgfqpoint{3.354725in}{2.364574in}}%
\pgfpathlineto{\pgfqpoint{3.355810in}{2.364881in}}%
\pgfpathlineto{\pgfqpoint{3.356699in}{2.406613in}}%
\pgfpathlineto{\pgfqpoint{3.358377in}{2.665269in}}%
\pgfpathlineto{\pgfqpoint{3.359364in}{2.603308in}}%
\pgfpathlineto{\pgfqpoint{3.360252in}{2.379120in}}%
\pgfpathlineto{\pgfqpoint{3.361436in}{2.416413in}}%
\pgfpathlineto{\pgfqpoint{3.363312in}{2.517232in}}%
\pgfpathlineto{\pgfqpoint{3.363805in}{2.488035in}}%
\pgfpathlineto{\pgfqpoint{3.364397in}{2.448309in}}%
\pgfpathlineto{\pgfqpoint{3.365187in}{2.469822in}}%
\pgfpathlineto{\pgfqpoint{3.365384in}{2.467371in}}%
\pgfpathlineto{\pgfqpoint{3.365680in}{2.483178in}}%
\pgfpathlineto{\pgfqpoint{3.366865in}{2.522130in}}%
\pgfpathlineto{\pgfqpoint{3.367161in}{2.514080in}}%
\pgfpathlineto{\pgfqpoint{3.367556in}{2.498121in}}%
\pgfpathlineto{\pgfqpoint{3.368247in}{2.511908in}}%
\pgfpathlineto{\pgfqpoint{3.370221in}{2.547987in}}%
\pgfpathlineto{\pgfqpoint{3.370615in}{2.546544in}}%
\pgfpathlineto{\pgfqpoint{3.373280in}{2.577197in}}%
\pgfpathlineto{\pgfqpoint{3.374366in}{2.596294in}}%
\pgfpathlineto{\pgfqpoint{3.374563in}{2.593854in}}%
\pgfpathlineto{\pgfqpoint{3.374958in}{2.601934in}}%
\pgfpathlineto{\pgfqpoint{3.375254in}{2.593914in}}%
\pgfpathlineto{\pgfqpoint{3.375945in}{2.601764in}}%
\pgfpathlineto{\pgfqpoint{3.376439in}{2.588373in}}%
\pgfpathlineto{\pgfqpoint{3.377524in}{2.615583in}}%
\pgfpathlineto{\pgfqpoint{3.377820in}{2.596562in}}%
\pgfpathlineto{\pgfqpoint{3.377919in}{2.591433in}}%
\pgfpathlineto{\pgfqpoint{3.378413in}{2.608851in}}%
\pgfpathlineto{\pgfqpoint{3.378807in}{2.602959in}}%
\pgfpathlineto{\pgfqpoint{3.379301in}{2.603173in}}%
\pgfpathlineto{\pgfqpoint{3.379794in}{2.607162in}}%
\pgfpathlineto{\pgfqpoint{3.380189in}{2.601134in}}%
\pgfpathlineto{\pgfqpoint{3.381077in}{2.586018in}}%
\pgfpathlineto{\pgfqpoint{3.381373in}{2.595744in}}%
\pgfpathlineto{\pgfqpoint{3.382163in}{2.598168in}}%
\pgfpathlineto{\pgfqpoint{3.381867in}{2.584418in}}%
\pgfpathlineto{\pgfqpoint{3.382262in}{2.596799in}}%
\pgfpathlineto{\pgfqpoint{3.383644in}{2.564193in}}%
\pgfpathlineto{\pgfqpoint{3.383742in}{2.565089in}}%
\pgfpathlineto{\pgfqpoint{3.384137in}{2.573674in}}%
\pgfpathlineto{\pgfqpoint{3.384532in}{2.559486in}}%
\pgfpathlineto{\pgfqpoint{3.385914in}{2.540720in}}%
\pgfpathlineto{\pgfqpoint{3.386012in}{2.539991in}}%
\pgfpathlineto{\pgfqpoint{3.386210in}{2.544150in}}%
\pgfpathlineto{\pgfqpoint{3.386506in}{2.553206in}}%
\pgfpathlineto{\pgfqpoint{3.386999in}{2.536055in}}%
\pgfpathlineto{\pgfqpoint{3.387098in}{2.534815in}}%
\pgfpathlineto{\pgfqpoint{3.387295in}{2.541608in}}%
\pgfpathlineto{\pgfqpoint{3.387592in}{2.555183in}}%
\pgfpathlineto{\pgfqpoint{3.388085in}{2.536021in}}%
\pgfpathlineto{\pgfqpoint{3.388381in}{2.540010in}}%
\pgfpathlineto{\pgfqpoint{3.390059in}{2.530494in}}%
\pgfpathlineto{\pgfqpoint{3.389072in}{2.541788in}}%
\pgfpathlineto{\pgfqpoint{3.390256in}{2.532904in}}%
\pgfpathlineto{\pgfqpoint{3.391145in}{2.542738in}}%
\pgfpathlineto{\pgfqpoint{3.391441in}{2.535201in}}%
\pgfpathlineto{\pgfqpoint{3.391539in}{2.533258in}}%
\pgfpathlineto{\pgfqpoint{3.391934in}{2.543912in}}%
\pgfpathlineto{\pgfqpoint{3.392230in}{2.539797in}}%
\pgfpathlineto{\pgfqpoint{3.392428in}{2.541375in}}%
\pgfpathlineto{\pgfqpoint{3.394599in}{2.575021in}}%
\pgfpathlineto{\pgfqpoint{3.392823in}{2.539802in}}%
\pgfpathlineto{\pgfqpoint{3.394797in}{2.570603in}}%
\pgfpathlineto{\pgfqpoint{3.397856in}{2.430227in}}%
\pgfpathlineto{\pgfqpoint{3.398350in}{2.450269in}}%
\pgfpathlineto{\pgfqpoint{3.399238in}{2.493245in}}%
\pgfpathlineto{\pgfqpoint{3.400028in}{2.485468in}}%
\pgfpathlineto{\pgfqpoint{3.400126in}{2.485524in}}%
\pgfpathlineto{\pgfqpoint{3.400225in}{2.484582in}}%
\pgfpathlineto{\pgfqpoint{3.401015in}{2.470533in}}%
\pgfpathlineto{\pgfqpoint{3.401212in}{2.480500in}}%
\pgfpathlineto{\pgfqpoint{3.402002in}{2.522744in}}%
\pgfpathlineto{\pgfqpoint{3.402396in}{2.496721in}}%
\pgfpathlineto{\pgfqpoint{3.403679in}{2.437227in}}%
\pgfpathlineto{\pgfqpoint{3.404074in}{2.441695in}}%
\pgfpathlineto{\pgfqpoint{3.404173in}{2.440351in}}%
\pgfpathlineto{\pgfqpoint{3.404469in}{2.449640in}}%
\pgfpathlineto{\pgfqpoint{3.405653in}{2.570545in}}%
\pgfpathlineto{\pgfqpoint{3.406739in}{2.715965in}}%
\pgfpathlineto{\pgfqpoint{3.407233in}{2.674478in}}%
\pgfpathlineto{\pgfqpoint{3.407825in}{2.598717in}}%
\pgfpathlineto{\pgfqpoint{3.408516in}{2.421744in}}%
\pgfpathlineto{\pgfqpoint{3.409404in}{2.449394in}}%
\pgfpathlineto{\pgfqpoint{3.409601in}{2.441917in}}%
\pgfpathlineto{\pgfqpoint{3.410095in}{2.470051in}}%
\pgfpathlineto{\pgfqpoint{3.410490in}{2.468274in}}%
\pgfpathlineto{\pgfqpoint{3.410983in}{2.496997in}}%
\pgfpathlineto{\pgfqpoint{3.411477in}{2.534664in}}%
\pgfpathlineto{\pgfqpoint{3.412168in}{2.504265in}}%
\pgfpathlineto{\pgfqpoint{3.412858in}{2.472243in}}%
\pgfpathlineto{\pgfqpoint{3.413352in}{2.495541in}}%
\pgfpathlineto{\pgfqpoint{3.415030in}{2.521022in}}%
\pgfpathlineto{\pgfqpoint{3.415326in}{2.511860in}}%
\pgfpathlineto{\pgfqpoint{3.416116in}{2.482115in}}%
\pgfpathlineto{\pgfqpoint{3.416708in}{2.497924in}}%
\pgfpathlineto{\pgfqpoint{3.416905in}{2.499804in}}%
\pgfpathlineto{\pgfqpoint{3.418090in}{2.522627in}}%
\pgfpathlineto{\pgfqpoint{3.418386in}{2.512431in}}%
\pgfpathlineto{\pgfqpoint{3.418682in}{2.504174in}}%
\pgfpathlineto{\pgfqpoint{3.419175in}{2.516590in}}%
\pgfpathlineto{\pgfqpoint{3.419570in}{2.504778in}}%
\pgfpathlineto{\pgfqpoint{3.419866in}{2.514881in}}%
\pgfpathlineto{\pgfqpoint{3.420360in}{2.496917in}}%
\pgfpathlineto{\pgfqpoint{3.420656in}{2.506058in}}%
\pgfpathlineto{\pgfqpoint{3.420754in}{2.505944in}}%
\pgfpathlineto{\pgfqpoint{3.421050in}{2.500213in}}%
\pgfpathlineto{\pgfqpoint{3.421741in}{2.504940in}}%
\pgfpathlineto{\pgfqpoint{3.421939in}{2.510935in}}%
\pgfpathlineto{\pgfqpoint{3.422432in}{2.496864in}}%
\pgfpathlineto{\pgfqpoint{3.422827in}{2.507649in}}%
\pgfpathlineto{\pgfqpoint{3.423814in}{2.498557in}}%
\pgfpathlineto{\pgfqpoint{3.423518in}{2.508820in}}%
\pgfpathlineto{\pgfqpoint{3.423913in}{2.500496in}}%
\pgfpathlineto{\pgfqpoint{3.424209in}{2.512774in}}%
\pgfpathlineto{\pgfqpoint{3.424998in}{2.500292in}}%
\pgfpathlineto{\pgfqpoint{3.425985in}{2.495982in}}%
\pgfpathlineto{\pgfqpoint{3.425591in}{2.513604in}}%
\pgfpathlineto{\pgfqpoint{3.426084in}{2.499593in}}%
\pgfpathlineto{\pgfqpoint{3.426380in}{2.509023in}}%
\pgfpathlineto{\pgfqpoint{3.427071in}{2.499325in}}%
\pgfpathlineto{\pgfqpoint{3.428058in}{2.494714in}}%
\pgfpathlineto{\pgfqpoint{3.427663in}{2.505046in}}%
\pgfpathlineto{\pgfqpoint{3.428157in}{2.497275in}}%
\pgfpathlineto{\pgfqpoint{3.428453in}{2.507006in}}%
\pgfpathlineto{\pgfqpoint{3.429242in}{2.497942in}}%
\pgfpathlineto{\pgfqpoint{3.429835in}{2.502969in}}%
\pgfpathlineto{\pgfqpoint{3.430920in}{2.469429in}}%
\pgfpathlineto{\pgfqpoint{3.431118in}{2.471764in}}%
\pgfpathlineto{\pgfqpoint{3.431414in}{2.464710in}}%
\pgfpathlineto{\pgfqpoint{3.433487in}{2.421329in}}%
\pgfpathlineto{\pgfqpoint{3.433881in}{2.429675in}}%
\pgfpathlineto{\pgfqpoint{3.433980in}{2.431472in}}%
\pgfpathlineto{\pgfqpoint{3.434276in}{2.417498in}}%
\pgfpathlineto{\pgfqpoint{3.435263in}{2.400105in}}%
\pgfpathlineto{\pgfqpoint{3.435559in}{2.405437in}}%
\pgfpathlineto{\pgfqpoint{3.435757in}{2.401739in}}%
\pgfpathlineto{\pgfqpoint{3.436151in}{2.387031in}}%
\pgfpathlineto{\pgfqpoint{3.436645in}{2.404715in}}%
\pgfpathlineto{\pgfqpoint{3.436941in}{2.396005in}}%
\pgfpathlineto{\pgfqpoint{3.438027in}{2.387795in}}%
\pgfpathlineto{\pgfqpoint{3.437632in}{2.402054in}}%
\pgfpathlineto{\pgfqpoint{3.438125in}{2.389954in}}%
\pgfpathlineto{\pgfqpoint{3.438421in}{2.405355in}}%
\pgfpathlineto{\pgfqpoint{3.438915in}{2.385241in}}%
\pgfpathlineto{\pgfqpoint{3.439408in}{2.404740in}}%
\pgfpathlineto{\pgfqpoint{3.439507in}{2.405195in}}%
\pgfpathlineto{\pgfqpoint{3.439606in}{2.403127in}}%
\pgfpathlineto{\pgfqpoint{3.440790in}{2.391005in}}%
\pgfpathlineto{\pgfqpoint{3.440889in}{2.392248in}}%
\pgfpathlineto{\pgfqpoint{3.444047in}{2.436323in}}%
\pgfpathlineto{\pgfqpoint{3.444343in}{2.431786in}}%
\pgfpathlineto{\pgfqpoint{3.445923in}{2.385487in}}%
\pgfpathlineto{\pgfqpoint{3.446120in}{2.380559in}}%
\pgfpathlineto{\pgfqpoint{3.446515in}{2.388173in}}%
\pgfpathlineto{\pgfqpoint{3.446910in}{2.386673in}}%
\pgfpathlineto{\pgfqpoint{3.447107in}{2.394084in}}%
\pgfpathlineto{\pgfqpoint{3.447600in}{2.369849in}}%
\pgfpathlineto{\pgfqpoint{3.448587in}{2.345070in}}%
\pgfpathlineto{\pgfqpoint{3.449476in}{2.352039in}}%
\pgfpathlineto{\pgfqpoint{3.450265in}{2.405188in}}%
\pgfpathlineto{\pgfqpoint{3.450956in}{2.381893in}}%
\pgfpathlineto{\pgfqpoint{3.452437in}{2.352901in}}%
\pgfpathlineto{\pgfqpoint{3.452535in}{2.354504in}}%
\pgfpathlineto{\pgfqpoint{3.453917in}{2.487869in}}%
\pgfpathlineto{\pgfqpoint{3.454904in}{2.643661in}}%
\pgfpathlineto{\pgfqpoint{3.455595in}{2.614308in}}%
\pgfpathlineto{\pgfqpoint{3.456187in}{2.539754in}}%
\pgfpathlineto{\pgfqpoint{3.456779in}{2.367902in}}%
\pgfpathlineto{\pgfqpoint{3.457668in}{2.390889in}}%
\pgfpathlineto{\pgfqpoint{3.458655in}{2.420006in}}%
\pgfpathlineto{\pgfqpoint{3.460135in}{2.493432in}}%
\pgfpathlineto{\pgfqpoint{3.460333in}{2.481810in}}%
\pgfpathlineto{\pgfqpoint{3.461024in}{2.432284in}}%
\pgfpathlineto{\pgfqpoint{3.461714in}{2.447178in}}%
\pgfpathlineto{\pgfqpoint{3.462011in}{2.453533in}}%
\pgfpathlineto{\pgfqpoint{3.463294in}{2.484895in}}%
\pgfpathlineto{\pgfqpoint{3.463590in}{2.475380in}}%
\pgfpathlineto{\pgfqpoint{3.464478in}{2.444510in}}%
\pgfpathlineto{\pgfqpoint{3.464873in}{2.458989in}}%
\pgfpathlineto{\pgfqpoint{3.466452in}{2.476177in}}%
\pgfpathlineto{\pgfqpoint{3.465268in}{2.457105in}}%
\pgfpathlineto{\pgfqpoint{3.466551in}{2.475321in}}%
\pgfpathlineto{\pgfqpoint{3.469117in}{2.412922in}}%
\pgfpathlineto{\pgfqpoint{3.469314in}{2.417220in}}%
\pgfpathlineto{\pgfqpoint{3.471979in}{2.513955in}}%
\pgfpathlineto{\pgfqpoint{3.472078in}{2.513560in}}%
\pgfpathlineto{\pgfqpoint{3.473756in}{2.497928in}}%
\pgfpathlineto{\pgfqpoint{3.474151in}{2.499783in}}%
\pgfpathlineto{\pgfqpoint{3.475039in}{2.510880in}}%
\pgfpathlineto{\pgfqpoint{3.475236in}{2.505023in}}%
\pgfpathlineto{\pgfqpoint{3.475434in}{2.495883in}}%
\pgfpathlineto{\pgfqpoint{3.475828in}{2.506916in}}%
\pgfpathlineto{\pgfqpoint{3.476322in}{2.504263in}}%
\pgfpathlineto{\pgfqpoint{3.476421in}{2.505676in}}%
\pgfpathlineto{\pgfqpoint{3.476717in}{2.497158in}}%
\pgfpathlineto{\pgfqpoint{3.477901in}{2.476944in}}%
\pgfpathlineto{\pgfqpoint{3.478098in}{2.480916in}}%
\pgfpathlineto{\pgfqpoint{3.478987in}{2.488257in}}%
\pgfpathlineto{\pgfqpoint{3.478691in}{2.471567in}}%
\pgfpathlineto{\pgfqpoint{3.479085in}{2.486492in}}%
\pgfpathlineto{\pgfqpoint{3.480763in}{2.440728in}}%
\pgfpathlineto{\pgfqpoint{3.483823in}{2.404454in}}%
\pgfpathlineto{\pgfqpoint{3.481158in}{2.443446in}}%
\pgfpathlineto{\pgfqpoint{3.484218in}{2.406989in}}%
\pgfpathlineto{\pgfqpoint{3.485106in}{2.416729in}}%
\pgfpathlineto{\pgfqpoint{3.485402in}{2.412318in}}%
\pgfpathlineto{\pgfqpoint{3.485600in}{2.408241in}}%
\pgfpathlineto{\pgfqpoint{3.485994in}{2.417854in}}%
\pgfpathlineto{\pgfqpoint{3.486389in}{2.411723in}}%
\pgfpathlineto{\pgfqpoint{3.486784in}{2.422899in}}%
\pgfpathlineto{\pgfqpoint{3.487968in}{2.420361in}}%
\pgfpathlineto{\pgfqpoint{3.488363in}{2.403252in}}%
\pgfpathlineto{\pgfqpoint{3.488758in}{2.423905in}}%
\pgfpathlineto{\pgfqpoint{3.489054in}{2.428195in}}%
\pgfpathlineto{\pgfqpoint{3.489449in}{2.416878in}}%
\pgfpathlineto{\pgfqpoint{3.489646in}{2.411977in}}%
\pgfpathlineto{\pgfqpoint{3.489942in}{2.432823in}}%
\pgfpathlineto{\pgfqpoint{3.490140in}{2.443063in}}%
\pgfpathlineto{\pgfqpoint{3.491127in}{2.439214in}}%
\pgfpathlineto{\pgfqpoint{3.491423in}{2.445564in}}%
\pgfpathlineto{\pgfqpoint{3.491818in}{2.434626in}}%
\pgfpathlineto{\pgfqpoint{3.492311in}{2.444422in}}%
\pgfpathlineto{\pgfqpoint{3.493397in}{2.398789in}}%
\pgfpathlineto{\pgfqpoint{3.494482in}{2.374939in}}%
\pgfpathlineto{\pgfqpoint{3.494779in}{2.380395in}}%
\pgfpathlineto{\pgfqpoint{3.495173in}{2.389087in}}%
\pgfpathlineto{\pgfqpoint{3.495667in}{2.373269in}}%
\pgfpathlineto{\pgfqpoint{3.497246in}{2.355224in}}%
\pgfpathlineto{\pgfqpoint{3.497443in}{2.352751in}}%
\pgfpathlineto{\pgfqpoint{3.497740in}{2.359417in}}%
\pgfpathlineto{\pgfqpoint{3.498628in}{2.413338in}}%
\pgfpathlineto{\pgfqpoint{3.499121in}{2.387474in}}%
\pgfpathlineto{\pgfqpoint{3.500306in}{2.353554in}}%
\pgfpathlineto{\pgfqpoint{3.500503in}{2.358553in}}%
\pgfpathlineto{\pgfqpoint{3.502181in}{2.498614in}}%
\pgfpathlineto{\pgfqpoint{3.503267in}{2.642161in}}%
\pgfpathlineto{\pgfqpoint{3.503760in}{2.612342in}}%
\pgfpathlineto{\pgfqpoint{3.504550in}{2.490453in}}%
\pgfpathlineto{\pgfqpoint{3.505043in}{2.354753in}}%
\pgfpathlineto{\pgfqpoint{3.505932in}{2.386135in}}%
\pgfpathlineto{\pgfqpoint{3.506129in}{2.379889in}}%
\pgfpathlineto{\pgfqpoint{3.506820in}{2.393653in}}%
\pgfpathlineto{\pgfqpoint{3.508300in}{2.465908in}}%
\pgfpathlineto{\pgfqpoint{3.508794in}{2.434364in}}%
\pgfpathlineto{\pgfqpoint{3.509485in}{2.402117in}}%
\pgfpathlineto{\pgfqpoint{3.509978in}{2.425112in}}%
\pgfpathlineto{\pgfqpoint{3.511459in}{2.461247in}}%
\pgfpathlineto{\pgfqpoint{3.511656in}{2.457486in}}%
\pgfpathlineto{\pgfqpoint{3.512643in}{2.432730in}}%
\pgfpathlineto{\pgfqpoint{3.512939in}{2.442369in}}%
\pgfpathlineto{\pgfqpoint{3.513827in}{2.453517in}}%
\pgfpathlineto{\pgfqpoint{3.514321in}{2.451213in}}%
\pgfpathlineto{\pgfqpoint{3.516196in}{2.467672in}}%
\pgfpathlineto{\pgfqpoint{3.516492in}{2.459646in}}%
\pgfpathlineto{\pgfqpoint{3.516591in}{2.458461in}}%
\pgfpathlineto{\pgfqpoint{3.516788in}{2.465624in}}%
\pgfpathlineto{\pgfqpoint{3.517677in}{2.476011in}}%
\pgfpathlineto{\pgfqpoint{3.517282in}{2.459376in}}%
\pgfpathlineto{\pgfqpoint{3.517874in}{2.471298in}}%
\pgfpathlineto{\pgfqpoint{3.518072in}{2.466122in}}%
\pgfpathlineto{\pgfqpoint{3.518762in}{2.478364in}}%
\pgfpathlineto{\pgfqpoint{3.519552in}{2.467801in}}%
\pgfpathlineto{\pgfqpoint{3.519947in}{2.475734in}}%
\pgfpathlineto{\pgfqpoint{3.520144in}{2.478395in}}%
\pgfpathlineto{\pgfqpoint{3.520539in}{2.465600in}}%
\pgfpathlineto{\pgfqpoint{3.521625in}{2.456864in}}%
\pgfpathlineto{\pgfqpoint{3.521131in}{2.474555in}}%
\pgfpathlineto{\pgfqpoint{3.521723in}{2.460391in}}%
\pgfpathlineto{\pgfqpoint{3.522019in}{2.472865in}}%
\pgfpathlineto{\pgfqpoint{3.523006in}{2.470886in}}%
\pgfpathlineto{\pgfqpoint{3.524290in}{2.475361in}}%
\pgfpathlineto{\pgfqpoint{3.523796in}{2.470321in}}%
\pgfpathlineto{\pgfqpoint{3.524487in}{2.472439in}}%
\pgfpathlineto{\pgfqpoint{3.525770in}{2.459676in}}%
\pgfpathlineto{\pgfqpoint{3.525178in}{2.474490in}}%
\pgfpathlineto{\pgfqpoint{3.525869in}{2.461127in}}%
\pgfpathlineto{\pgfqpoint{3.526066in}{2.463887in}}%
\pgfpathlineto{\pgfqpoint{3.526362in}{2.453366in}}%
\pgfpathlineto{\pgfqpoint{3.527843in}{2.432158in}}%
\pgfpathlineto{\pgfqpoint{3.527941in}{2.432428in}}%
\pgfpathlineto{\pgfqpoint{3.528238in}{2.443834in}}%
\pgfpathlineto{\pgfqpoint{3.528632in}{2.425281in}}%
\pgfpathlineto{\pgfqpoint{3.529027in}{2.434839in}}%
\pgfpathlineto{\pgfqpoint{3.529422in}{2.408800in}}%
\pgfpathlineto{\pgfqpoint{3.530409in}{2.417581in}}%
\pgfpathlineto{\pgfqpoint{3.530606in}{2.421217in}}%
\pgfpathlineto{\pgfqpoint{3.531100in}{2.409561in}}%
\pgfpathlineto{\pgfqpoint{3.531198in}{2.409780in}}%
\pgfpathlineto{\pgfqpoint{3.531297in}{2.410042in}}%
\pgfpathlineto{\pgfqpoint{3.531495in}{2.408507in}}%
\pgfpathlineto{\pgfqpoint{3.531692in}{2.405162in}}%
\pgfpathlineto{\pgfqpoint{3.532087in}{2.416196in}}%
\pgfpathlineto{\pgfqpoint{3.532975in}{2.423911in}}%
\pgfpathlineto{\pgfqpoint{3.532580in}{2.409302in}}%
\pgfpathlineto{\pgfqpoint{3.533172in}{2.417631in}}%
\pgfpathlineto{\pgfqpoint{3.534061in}{2.406446in}}%
\pgfpathlineto{\pgfqpoint{3.533666in}{2.418193in}}%
\pgfpathlineto{\pgfqpoint{3.534357in}{2.414449in}}%
\pgfpathlineto{\pgfqpoint{3.534456in}{2.415140in}}%
\pgfpathlineto{\pgfqpoint{3.534653in}{2.408724in}}%
\pgfpathlineto{\pgfqpoint{3.534850in}{2.403591in}}%
\pgfpathlineto{\pgfqpoint{3.535443in}{2.416709in}}%
\pgfpathlineto{\pgfqpoint{3.535541in}{2.416599in}}%
\pgfpathlineto{\pgfqpoint{3.536232in}{2.417588in}}%
\pgfpathlineto{\pgfqpoint{3.538502in}{2.452824in}}%
\pgfpathlineto{\pgfqpoint{3.537120in}{2.414750in}}%
\pgfpathlineto{\pgfqpoint{3.538601in}{2.450579in}}%
\pgfpathlineto{\pgfqpoint{3.538897in}{2.441912in}}%
\pgfpathlineto{\pgfqpoint{3.539391in}{2.464847in}}%
\pgfpathlineto{\pgfqpoint{3.541661in}{2.427288in}}%
\pgfpathlineto{\pgfqpoint{3.543437in}{2.374414in}}%
\pgfpathlineto{\pgfqpoint{3.544424in}{2.322040in}}%
\pgfpathlineto{\pgfqpoint{3.545707in}{2.279671in}}%
\pgfpathlineto{\pgfqpoint{3.546003in}{2.298190in}}%
\pgfpathlineto{\pgfqpoint{3.546990in}{2.386907in}}%
\pgfpathlineto{\pgfqpoint{3.547484in}{2.362051in}}%
\pgfpathlineto{\pgfqpoint{3.547681in}{2.364919in}}%
\pgfpathlineto{\pgfqpoint{3.547977in}{2.353535in}}%
\pgfpathlineto{\pgfqpoint{3.548767in}{2.342579in}}%
\pgfpathlineto{\pgfqpoint{3.548372in}{2.358034in}}%
\pgfpathlineto{\pgfqpoint{3.548964in}{2.352893in}}%
\pgfpathlineto{\pgfqpoint{3.550741in}{2.544992in}}%
\pgfpathlineto{\pgfqpoint{3.551530in}{2.625366in}}%
\pgfpathlineto{\pgfqpoint{3.552024in}{2.579811in}}%
\pgfpathlineto{\pgfqpoint{3.552616in}{2.533925in}}%
\pgfpathlineto{\pgfqpoint{3.553307in}{2.332414in}}%
\pgfpathlineto{\pgfqpoint{3.554294in}{2.360423in}}%
\pgfpathlineto{\pgfqpoint{3.554491in}{2.356622in}}%
\pgfpathlineto{\pgfqpoint{3.554886in}{2.378866in}}%
\pgfpathlineto{\pgfqpoint{3.556071in}{2.419691in}}%
\pgfpathlineto{\pgfqpoint{3.556465in}{2.445323in}}%
\pgfpathlineto{\pgfqpoint{3.556959in}{2.406822in}}%
\pgfpathlineto{\pgfqpoint{3.557749in}{2.373040in}}%
\pgfpathlineto{\pgfqpoint{3.558143in}{2.395290in}}%
\pgfpathlineto{\pgfqpoint{3.559722in}{2.432668in}}%
\pgfpathlineto{\pgfqpoint{3.559821in}{2.431709in}}%
\pgfpathlineto{\pgfqpoint{3.560413in}{2.404057in}}%
\pgfpathlineto{\pgfqpoint{3.561302in}{2.416510in}}%
\pgfpathlineto{\pgfqpoint{3.563473in}{2.451001in}}%
\pgfpathlineto{\pgfqpoint{3.563670in}{2.446973in}}%
\pgfpathlineto{\pgfqpoint{3.563967in}{2.440556in}}%
\pgfpathlineto{\pgfqpoint{3.564361in}{2.452585in}}%
\pgfpathlineto{\pgfqpoint{3.564657in}{2.448082in}}%
\pgfpathlineto{\pgfqpoint{3.566829in}{2.459334in}}%
\pgfpathlineto{\pgfqpoint{3.567026in}{2.458673in}}%
\pgfpathlineto{\pgfqpoint{3.568211in}{2.444578in}}%
\pgfpathlineto{\pgfqpoint{3.568507in}{2.451989in}}%
\pgfpathlineto{\pgfqpoint{3.568704in}{2.454496in}}%
\pgfpathlineto{\pgfqpoint{3.569099in}{2.445766in}}%
\pgfpathlineto{\pgfqpoint{3.569395in}{2.448257in}}%
\pgfpathlineto{\pgfqpoint{3.569494in}{2.448716in}}%
\pgfpathlineto{\pgfqpoint{3.569691in}{2.446093in}}%
\pgfpathlineto{\pgfqpoint{3.570185in}{2.438057in}}%
\pgfpathlineto{\pgfqpoint{3.570481in}{2.448896in}}%
\pgfpathlineto{\pgfqpoint{3.570678in}{2.453661in}}%
\pgfpathlineto{\pgfqpoint{3.571073in}{2.445158in}}%
\pgfpathlineto{\pgfqpoint{3.571566in}{2.449542in}}%
\pgfpathlineto{\pgfqpoint{3.571862in}{2.454502in}}%
\pgfpathlineto{\pgfqpoint{3.572159in}{2.444521in}}%
\pgfpathlineto{\pgfqpoint{3.572257in}{2.441638in}}%
\pgfpathlineto{\pgfqpoint{3.572751in}{2.458108in}}%
\pgfpathlineto{\pgfqpoint{3.574922in}{2.441682in}}%
\pgfpathlineto{\pgfqpoint{3.575120in}{2.444333in}}%
\pgfpathlineto{\pgfqpoint{3.575317in}{2.440799in}}%
\pgfpathlineto{\pgfqpoint{3.576995in}{2.407991in}}%
\pgfpathlineto{\pgfqpoint{3.577291in}{2.413029in}}%
\pgfpathlineto{\pgfqpoint{3.577488in}{2.404880in}}%
\pgfpathlineto{\pgfqpoint{3.578771in}{2.390832in}}%
\pgfpathlineto{\pgfqpoint{3.579166in}{2.386280in}}%
\pgfpathlineto{\pgfqpoint{3.579758in}{2.391075in}}%
\pgfpathlineto{\pgfqpoint{3.579857in}{2.391075in}}%
\pgfpathlineto{\pgfqpoint{3.581732in}{2.372213in}}%
\pgfpathlineto{\pgfqpoint{3.581831in}{2.372935in}}%
\pgfpathlineto{\pgfqpoint{3.582226in}{2.388757in}}%
\pgfpathlineto{\pgfqpoint{3.582719in}{2.367883in}}%
\pgfpathlineto{\pgfqpoint{3.582818in}{2.367331in}}%
\pgfpathlineto{\pgfqpoint{3.583015in}{2.371197in}}%
\pgfpathlineto{\pgfqpoint{3.584397in}{2.395216in}}%
\pgfpathlineto{\pgfqpoint{3.584595in}{2.389505in}}%
\pgfpathlineto{\pgfqpoint{3.584792in}{2.385002in}}%
\pgfpathlineto{\pgfqpoint{3.585483in}{2.392337in}}%
\pgfpathlineto{\pgfqpoint{3.587259in}{2.443143in}}%
\pgfpathlineto{\pgfqpoint{3.588148in}{2.440677in}}%
\pgfpathlineto{\pgfqpoint{3.588444in}{2.441778in}}%
\pgfpathlineto{\pgfqpoint{3.588641in}{2.436861in}}%
\pgfpathlineto{\pgfqpoint{3.590418in}{2.377265in}}%
\pgfpathlineto{\pgfqpoint{3.590517in}{2.379397in}}%
\pgfpathlineto{\pgfqpoint{3.591405in}{2.395315in}}%
\pgfpathlineto{\pgfqpoint{3.590911in}{2.378106in}}%
\pgfpathlineto{\pgfqpoint{3.591701in}{2.382596in}}%
\pgfpathlineto{\pgfqpoint{3.593872in}{2.350432in}}%
\pgfpathlineto{\pgfqpoint{3.594168in}{2.368456in}}%
\pgfpathlineto{\pgfqpoint{3.594761in}{2.411119in}}%
\pgfpathlineto{\pgfqpoint{3.595353in}{2.384620in}}%
\pgfpathlineto{\pgfqpoint{3.596833in}{2.361940in}}%
\pgfpathlineto{\pgfqpoint{3.597722in}{2.421465in}}%
\pgfpathlineto{\pgfqpoint{3.598610in}{2.570510in}}%
\pgfpathlineto{\pgfqpoint{3.599498in}{2.653089in}}%
\pgfpathlineto{\pgfqpoint{3.599893in}{2.633050in}}%
\pgfpathlineto{\pgfqpoint{3.600683in}{2.534899in}}%
\pgfpathlineto{\pgfqpoint{3.601275in}{2.373981in}}%
\pgfpathlineto{\pgfqpoint{3.602064in}{2.410325in}}%
\pgfpathlineto{\pgfqpoint{3.604433in}{2.506303in}}%
\pgfpathlineto{\pgfqpoint{3.602657in}{2.408643in}}%
\pgfpathlineto{\pgfqpoint{3.604828in}{2.472369in}}%
\pgfpathlineto{\pgfqpoint{3.605519in}{2.431829in}}%
\pgfpathlineto{\pgfqpoint{3.606111in}{2.459789in}}%
\pgfpathlineto{\pgfqpoint{3.607789in}{2.501933in}}%
\pgfpathlineto{\pgfqpoint{3.607888in}{2.497304in}}%
\pgfpathlineto{\pgfqpoint{3.608875in}{2.466460in}}%
\pgfpathlineto{\pgfqpoint{3.609171in}{2.475590in}}%
\pgfpathlineto{\pgfqpoint{3.611243in}{2.512333in}}%
\pgfpathlineto{\pgfqpoint{3.611342in}{2.511299in}}%
\pgfpathlineto{\pgfqpoint{3.611539in}{2.509394in}}%
\pgfpathlineto{\pgfqpoint{3.611737in}{2.516904in}}%
\pgfpathlineto{\pgfqpoint{3.612921in}{2.529705in}}%
\pgfpathlineto{\pgfqpoint{3.612428in}{2.515110in}}%
\pgfpathlineto{\pgfqpoint{3.613020in}{2.529061in}}%
\pgfpathlineto{\pgfqpoint{3.613119in}{2.528617in}}%
\pgfpathlineto{\pgfqpoint{3.613415in}{2.530785in}}%
\pgfpathlineto{\pgfqpoint{3.616178in}{2.566968in}}%
\pgfpathlineto{\pgfqpoint{3.616573in}{2.555849in}}%
\pgfpathlineto{\pgfqpoint{3.616672in}{2.554467in}}%
\pgfpathlineto{\pgfqpoint{3.617067in}{2.563542in}}%
\pgfpathlineto{\pgfqpoint{3.617165in}{2.563762in}}%
\pgfpathlineto{\pgfqpoint{3.617264in}{2.562804in}}%
\pgfpathlineto{\pgfqpoint{3.618152in}{2.557058in}}%
\pgfpathlineto{\pgfqpoint{3.617757in}{2.567119in}}%
\pgfpathlineto{\pgfqpoint{3.618251in}{2.559920in}}%
\pgfpathlineto{\pgfqpoint{3.618547in}{2.569875in}}%
\pgfpathlineto{\pgfqpoint{3.618942in}{2.554792in}}%
\pgfpathlineto{\pgfqpoint{3.619337in}{2.558818in}}%
\pgfpathlineto{\pgfqpoint{3.621903in}{2.472349in}}%
\pgfpathlineto{\pgfqpoint{3.622199in}{2.478139in}}%
\pgfpathlineto{\pgfqpoint{3.622298in}{2.478641in}}%
\pgfpathlineto{\pgfqpoint{3.622495in}{2.475053in}}%
\pgfpathlineto{\pgfqpoint{3.622692in}{2.472968in}}%
\pgfpathlineto{\pgfqpoint{3.622890in}{2.481752in}}%
\pgfpathlineto{\pgfqpoint{3.624074in}{2.530217in}}%
\pgfpathlineto{\pgfqpoint{3.624370in}{2.519096in}}%
\pgfpathlineto{\pgfqpoint{3.624469in}{2.518647in}}%
\pgfpathlineto{\pgfqpoint{3.624568in}{2.521148in}}%
\pgfpathlineto{\pgfqpoint{3.624864in}{2.528853in}}%
\pgfpathlineto{\pgfqpoint{3.625456in}{2.520929in}}%
\pgfpathlineto{\pgfqpoint{3.626443in}{2.501761in}}%
\pgfpathlineto{\pgfqpoint{3.626739in}{2.507567in}}%
\pgfpathlineto{\pgfqpoint{3.626838in}{2.507894in}}%
\pgfpathlineto{\pgfqpoint{3.626936in}{2.505543in}}%
\pgfpathlineto{\pgfqpoint{3.627134in}{2.500320in}}%
\pgfpathlineto{\pgfqpoint{3.627430in}{2.507767in}}%
\pgfpathlineto{\pgfqpoint{3.627923in}{2.505444in}}%
\pgfpathlineto{\pgfqpoint{3.628220in}{2.522434in}}%
\pgfpathlineto{\pgfqpoint{3.629009in}{2.512483in}}%
\pgfpathlineto{\pgfqpoint{3.630687in}{2.497634in}}%
\pgfpathlineto{\pgfqpoint{3.631181in}{2.488102in}}%
\pgfpathlineto{\pgfqpoint{3.632168in}{2.492667in}}%
\pgfpathlineto{\pgfqpoint{3.632760in}{2.500750in}}%
\pgfpathlineto{\pgfqpoint{3.632957in}{2.492743in}}%
\pgfpathlineto{\pgfqpoint{3.633253in}{2.471518in}}%
\pgfpathlineto{\pgfqpoint{3.633747in}{2.500840in}}%
\pgfpathlineto{\pgfqpoint{3.634043in}{2.491779in}}%
\pgfpathlineto{\pgfqpoint{3.636510in}{2.527270in}}%
\pgfpathlineto{\pgfqpoint{3.637596in}{2.497617in}}%
\pgfpathlineto{\pgfqpoint{3.639175in}{2.445923in}}%
\pgfpathlineto{\pgfqpoint{3.639471in}{2.454958in}}%
\pgfpathlineto{\pgfqpoint{3.640063in}{2.462601in}}%
\pgfpathlineto{\pgfqpoint{3.640360in}{2.454348in}}%
\pgfpathlineto{\pgfqpoint{3.641445in}{2.422250in}}%
\pgfpathlineto{\pgfqpoint{3.642235in}{2.424962in}}%
\pgfpathlineto{\pgfqpoint{3.642531in}{2.431568in}}%
\pgfpathlineto{\pgfqpoint{3.643123in}{2.478756in}}%
\pgfpathlineto{\pgfqpoint{3.643715in}{2.442500in}}%
\pgfpathlineto{\pgfqpoint{3.645393in}{2.396129in}}%
\pgfpathlineto{\pgfqpoint{3.645788in}{2.416349in}}%
\pgfpathlineto{\pgfqpoint{3.647071in}{2.577696in}}%
\pgfpathlineto{\pgfqpoint{3.647959in}{2.685139in}}%
\pgfpathlineto{\pgfqpoint{3.648453in}{2.655888in}}%
\pgfpathlineto{\pgfqpoint{3.649144in}{2.561816in}}%
\pgfpathlineto{\pgfqpoint{3.649736in}{2.392304in}}%
\pgfpathlineto{\pgfqpoint{3.650624in}{2.416607in}}%
\pgfpathlineto{\pgfqpoint{3.650920in}{2.410329in}}%
\pgfpathlineto{\pgfqpoint{3.651315in}{2.432594in}}%
\pgfpathlineto{\pgfqpoint{3.652993in}{2.490926in}}%
\pgfpathlineto{\pgfqpoint{3.653092in}{2.486889in}}%
\pgfpathlineto{\pgfqpoint{3.654671in}{2.424246in}}%
\pgfpathlineto{\pgfqpoint{3.654868in}{2.431663in}}%
\pgfpathlineto{\pgfqpoint{3.656053in}{2.480660in}}%
\pgfpathlineto{\pgfqpoint{3.656447in}{2.469899in}}%
\pgfpathlineto{\pgfqpoint{3.657138in}{2.443552in}}%
\pgfpathlineto{\pgfqpoint{3.658224in}{2.454785in}}%
\pgfpathlineto{\pgfqpoint{3.660099in}{2.471299in}}%
\pgfpathlineto{\pgfqpoint{3.663060in}{2.521198in}}%
\pgfpathlineto{\pgfqpoint{3.660494in}{2.469398in}}%
\pgfpathlineto{\pgfqpoint{3.663356in}{2.516204in}}%
\pgfpathlineto{\pgfqpoint{3.663652in}{2.519533in}}%
\pgfpathlineto{\pgfqpoint{3.663850in}{2.516209in}}%
\pgfpathlineto{\pgfqpoint{3.664146in}{2.507494in}}%
\pgfpathlineto{\pgfqpoint{3.664541in}{2.518751in}}%
\pgfpathlineto{\pgfqpoint{3.665034in}{2.512588in}}%
\pgfpathlineto{\pgfqpoint{3.665133in}{2.514068in}}%
\pgfpathlineto{\pgfqpoint{3.665528in}{2.504592in}}%
\pgfpathlineto{\pgfqpoint{3.666021in}{2.510787in}}%
\pgfpathlineto{\pgfqpoint{3.666317in}{2.501953in}}%
\pgfpathlineto{\pgfqpoint{3.667206in}{2.506335in}}%
\pgfpathlineto{\pgfqpoint{3.668193in}{2.514728in}}%
\pgfpathlineto{\pgfqpoint{3.667699in}{2.495945in}}%
\pgfpathlineto{\pgfqpoint{3.668291in}{2.511478in}}%
\pgfpathlineto{\pgfqpoint{3.669081in}{2.492137in}}%
\pgfpathlineto{\pgfqpoint{3.669574in}{2.500297in}}%
\pgfpathlineto{\pgfqpoint{3.671844in}{2.464705in}}%
\pgfpathlineto{\pgfqpoint{3.671943in}{2.464903in}}%
\pgfpathlineto{\pgfqpoint{3.672239in}{2.466660in}}%
\pgfpathlineto{\pgfqpoint{3.672437in}{2.464945in}}%
\pgfpathlineto{\pgfqpoint{3.674115in}{2.428114in}}%
\pgfpathlineto{\pgfqpoint{3.674608in}{2.428386in}}%
\pgfpathlineto{\pgfqpoint{3.674904in}{2.426775in}}%
\pgfpathlineto{\pgfqpoint{3.676779in}{2.384948in}}%
\pgfpathlineto{\pgfqpoint{3.677766in}{2.386095in}}%
\pgfpathlineto{\pgfqpoint{3.678063in}{2.387542in}}%
\pgfpathlineto{\pgfqpoint{3.678260in}{2.386373in}}%
\pgfpathlineto{\pgfqpoint{3.679543in}{2.373838in}}%
\pgfpathlineto{\pgfqpoint{3.679642in}{2.375295in}}%
\pgfpathlineto{\pgfqpoint{3.680826in}{2.389346in}}%
\pgfpathlineto{\pgfqpoint{3.680333in}{2.371484in}}%
\pgfpathlineto{\pgfqpoint{3.680925in}{2.385440in}}%
\pgfpathlineto{\pgfqpoint{3.681320in}{2.371127in}}%
\pgfpathlineto{\pgfqpoint{3.682109in}{2.380528in}}%
\pgfpathlineto{\pgfqpoint{3.682208in}{2.380217in}}%
\pgfpathlineto{\pgfqpoint{3.682307in}{2.381231in}}%
\pgfpathlineto{\pgfqpoint{3.684774in}{2.431439in}}%
\pgfpathlineto{\pgfqpoint{3.685070in}{2.427193in}}%
\pgfpathlineto{\pgfqpoint{3.687143in}{2.374400in}}%
\pgfpathlineto{\pgfqpoint{3.688031in}{2.389705in}}%
\pgfpathlineto{\pgfqpoint{3.688327in}{2.399166in}}%
\pgfpathlineto{\pgfqpoint{3.688821in}{2.376673in}}%
\pgfpathlineto{\pgfqpoint{3.690203in}{2.350093in}}%
\pgfpathlineto{\pgfqpoint{3.690499in}{2.355537in}}%
\pgfpathlineto{\pgfqpoint{3.691486in}{2.415662in}}%
\pgfpathlineto{\pgfqpoint{3.692078in}{2.379713in}}%
\pgfpathlineto{\pgfqpoint{3.693657in}{2.294295in}}%
\pgfpathlineto{\pgfqpoint{3.694545in}{2.315095in}}%
\pgfpathlineto{\pgfqpoint{3.695631in}{2.529350in}}%
\pgfpathlineto{\pgfqpoint{3.696322in}{2.594477in}}%
\pgfpathlineto{\pgfqpoint{3.696914in}{2.576526in}}%
\pgfpathlineto{\pgfqpoint{3.697506in}{2.470309in}}%
\pgfpathlineto{\pgfqpoint{3.698000in}{2.335142in}}%
\pgfpathlineto{\pgfqpoint{3.698888in}{2.364347in}}%
\pgfpathlineto{\pgfqpoint{3.699085in}{2.362285in}}%
\pgfpathlineto{\pgfqpoint{3.699382in}{2.371687in}}%
\pgfpathlineto{\pgfqpoint{3.701158in}{2.450855in}}%
\pgfpathlineto{\pgfqpoint{3.701652in}{2.415525in}}%
\pgfpathlineto{\pgfqpoint{3.702540in}{2.380704in}}%
\pgfpathlineto{\pgfqpoint{3.702836in}{2.395575in}}%
\pgfpathlineto{\pgfqpoint{3.704514in}{2.437369in}}%
\pgfpathlineto{\pgfqpoint{3.704613in}{2.436108in}}%
\pgfpathlineto{\pgfqpoint{3.705303in}{2.404572in}}%
\pgfpathlineto{\pgfqpoint{3.706093in}{2.415420in}}%
\pgfpathlineto{\pgfqpoint{3.707179in}{2.437534in}}%
\pgfpathlineto{\pgfqpoint{3.706685in}{2.413387in}}%
\pgfpathlineto{\pgfqpoint{3.707574in}{2.430569in}}%
\pgfpathlineto{\pgfqpoint{3.708166in}{2.438600in}}%
\pgfpathlineto{\pgfqpoint{3.709251in}{2.452328in}}%
\pgfpathlineto{\pgfqpoint{3.708758in}{2.434670in}}%
\pgfpathlineto{\pgfqpoint{3.709449in}{2.446551in}}%
\pgfpathlineto{\pgfqpoint{3.709547in}{2.443907in}}%
\pgfpathlineto{\pgfqpoint{3.710041in}{2.450685in}}%
\pgfpathlineto{\pgfqpoint{3.710534in}{2.446179in}}%
\pgfpathlineto{\pgfqpoint{3.711620in}{2.469612in}}%
\pgfpathlineto{\pgfqpoint{3.712410in}{2.469466in}}%
\pgfpathlineto{\pgfqpoint{3.712903in}{2.457650in}}%
\pgfpathlineto{\pgfqpoint{3.713397in}{2.470272in}}%
\pgfpathlineto{\pgfqpoint{3.714186in}{2.458890in}}%
\pgfpathlineto{\pgfqpoint{3.714482in}{2.464370in}}%
\pgfpathlineto{\pgfqpoint{3.714877in}{2.454542in}}%
\pgfpathlineto{\pgfqpoint{3.714976in}{2.453101in}}%
\pgfpathlineto{\pgfqpoint{3.715469in}{2.459885in}}%
\pgfpathlineto{\pgfqpoint{3.716259in}{2.465928in}}%
\pgfpathlineto{\pgfqpoint{3.716555in}{2.462084in}}%
\pgfpathlineto{\pgfqpoint{3.717641in}{2.451093in}}%
\pgfpathlineto{\pgfqpoint{3.717838in}{2.453502in}}%
\pgfpathlineto{\pgfqpoint{3.718628in}{2.457397in}}%
\pgfpathlineto{\pgfqpoint{3.718233in}{2.449970in}}%
\pgfpathlineto{\pgfqpoint{3.718825in}{2.452624in}}%
\pgfpathlineto{\pgfqpoint{3.720207in}{2.429869in}}%
\pgfpathlineto{\pgfqpoint{3.723168in}{2.406417in}}%
\pgfpathlineto{\pgfqpoint{3.723267in}{2.408257in}}%
\pgfpathlineto{\pgfqpoint{3.723563in}{2.414160in}}%
\pgfpathlineto{\pgfqpoint{3.724155in}{2.405570in}}%
\pgfpathlineto{\pgfqpoint{3.724451in}{2.402020in}}%
\pgfpathlineto{\pgfqpoint{3.724846in}{2.408346in}}%
\pgfpathlineto{\pgfqpoint{3.725734in}{2.413968in}}%
\pgfpathlineto{\pgfqpoint{3.725932in}{2.411934in}}%
\pgfpathlineto{\pgfqpoint{3.726228in}{2.403736in}}%
\pgfpathlineto{\pgfqpoint{3.726721in}{2.414845in}}%
\pgfpathlineto{\pgfqpoint{3.727116in}{2.405124in}}%
\pgfpathlineto{\pgfqpoint{3.727511in}{2.415399in}}%
\pgfpathlineto{\pgfqpoint{3.728300in}{2.408512in}}%
\pgfpathlineto{\pgfqpoint{3.728695in}{2.399831in}}%
\pgfpathlineto{\pgfqpoint{3.729090in}{2.411979in}}%
\pgfpathlineto{\pgfqpoint{3.729682in}{2.424418in}}%
\pgfpathlineto{\pgfqpoint{3.730472in}{2.419434in}}%
\pgfpathlineto{\pgfqpoint{3.730669in}{2.413760in}}%
\pgfpathlineto{\pgfqpoint{3.731163in}{2.429752in}}%
\pgfpathlineto{\pgfqpoint{3.731557in}{2.416412in}}%
\pgfpathlineto{\pgfqpoint{3.731853in}{2.428986in}}%
\pgfpathlineto{\pgfqpoint{3.733531in}{2.472746in}}%
\pgfpathlineto{\pgfqpoint{3.736492in}{2.390941in}}%
\pgfpathlineto{\pgfqpoint{3.736986in}{2.406135in}}%
\pgfpathlineto{\pgfqpoint{3.737775in}{2.393109in}}%
\pgfpathlineto{\pgfqpoint{3.739256in}{2.356280in}}%
\pgfpathlineto{\pgfqpoint{3.739453in}{2.353517in}}%
\pgfpathlineto{\pgfqpoint{3.739749in}{2.368906in}}%
\pgfpathlineto{\pgfqpoint{3.740440in}{2.420908in}}%
\pgfpathlineto{\pgfqpoint{3.741032in}{2.391486in}}%
\pgfpathlineto{\pgfqpoint{3.741921in}{2.366632in}}%
\pgfpathlineto{\pgfqpoint{3.742513in}{2.370255in}}%
\pgfpathlineto{\pgfqpoint{3.743895in}{2.455293in}}%
\pgfpathlineto{\pgfqpoint{3.745178in}{2.655580in}}%
\pgfpathlineto{\pgfqpoint{3.745770in}{2.617725in}}%
\pgfpathlineto{\pgfqpoint{3.746264in}{2.572372in}}%
\pgfpathlineto{\pgfqpoint{3.747053in}{2.375520in}}%
\pgfpathlineto{\pgfqpoint{3.747941in}{2.396301in}}%
\pgfpathlineto{\pgfqpoint{3.748040in}{2.394778in}}%
\pgfpathlineto{\pgfqpoint{3.748336in}{2.406813in}}%
\pgfpathlineto{\pgfqpoint{3.750409in}{2.476507in}}%
\pgfpathlineto{\pgfqpoint{3.750508in}{2.471480in}}%
\pgfpathlineto{\pgfqpoint{3.751495in}{2.413078in}}%
\pgfpathlineto{\pgfqpoint{3.751889in}{2.428969in}}%
\pgfpathlineto{\pgfqpoint{3.753567in}{2.456572in}}%
\pgfpathlineto{\pgfqpoint{3.753666in}{2.458189in}}%
\pgfpathlineto{\pgfqpoint{3.753863in}{2.450767in}}%
\pgfpathlineto{\pgfqpoint{3.754258in}{2.419848in}}%
\pgfpathlineto{\pgfqpoint{3.755048in}{2.434442in}}%
\pgfpathlineto{\pgfqpoint{3.756133in}{2.453621in}}%
\pgfpathlineto{\pgfqpoint{3.755640in}{2.434088in}}%
\pgfpathlineto{\pgfqpoint{3.756528in}{2.450838in}}%
\pgfpathlineto{\pgfqpoint{3.757515in}{2.463470in}}%
\pgfpathlineto{\pgfqpoint{3.757120in}{2.450084in}}%
\pgfpathlineto{\pgfqpoint{3.757811in}{2.459127in}}%
\pgfpathlineto{\pgfqpoint{3.757910in}{2.458107in}}%
\pgfpathlineto{\pgfqpoint{3.758403in}{2.463676in}}%
\pgfpathlineto{\pgfqpoint{3.759983in}{2.481342in}}%
\pgfpathlineto{\pgfqpoint{3.760180in}{2.478491in}}%
\pgfpathlineto{\pgfqpoint{3.760871in}{2.473431in}}%
\pgfpathlineto{\pgfqpoint{3.761068in}{2.478677in}}%
\pgfpathlineto{\pgfqpoint{3.761364in}{2.487662in}}%
\pgfpathlineto{\pgfqpoint{3.762253in}{2.486340in}}%
\pgfpathlineto{\pgfqpoint{3.763832in}{2.498667in}}%
\pgfpathlineto{\pgfqpoint{3.764128in}{2.505635in}}%
\pgfpathlineto{\pgfqpoint{3.764424in}{2.490736in}}%
\pgfpathlineto{\pgfqpoint{3.764622in}{2.483378in}}%
\pgfpathlineto{\pgfqpoint{3.765016in}{2.497475in}}%
\pgfpathlineto{\pgfqpoint{3.765510in}{2.490605in}}%
\pgfpathlineto{\pgfqpoint{3.766892in}{2.502053in}}%
\pgfpathlineto{\pgfqpoint{3.766990in}{2.500126in}}%
\pgfpathlineto{\pgfqpoint{3.769655in}{2.416502in}}%
\pgfpathlineto{\pgfqpoint{3.769853in}{2.414313in}}%
\pgfpathlineto{\pgfqpoint{3.770247in}{2.425510in}}%
\pgfpathlineto{\pgfqpoint{3.772616in}{2.474575in}}%
\pgfpathlineto{\pgfqpoint{3.772715in}{2.474897in}}%
\pgfpathlineto{\pgfqpoint{3.772814in}{2.472776in}}%
\pgfpathlineto{\pgfqpoint{3.773801in}{2.447874in}}%
\pgfpathlineto{\pgfqpoint{3.774097in}{2.465829in}}%
\pgfpathlineto{\pgfqpoint{3.774195in}{2.465999in}}%
\pgfpathlineto{\pgfqpoint{3.775478in}{2.447589in}}%
\pgfpathlineto{\pgfqpoint{3.774985in}{2.474340in}}%
\pgfpathlineto{\pgfqpoint{3.775577in}{2.450447in}}%
\pgfpathlineto{\pgfqpoint{3.775873in}{2.461069in}}%
\pgfpathlineto{\pgfqpoint{3.776564in}{2.448244in}}%
\pgfpathlineto{\pgfqpoint{3.777650in}{2.435923in}}%
\pgfpathlineto{\pgfqpoint{3.777058in}{2.454124in}}%
\pgfpathlineto{\pgfqpoint{3.778045in}{2.440620in}}%
\pgfpathlineto{\pgfqpoint{3.778538in}{2.446509in}}%
\pgfpathlineto{\pgfqpoint{3.779130in}{2.440656in}}%
\pgfpathlineto{\pgfqpoint{3.779525in}{2.436544in}}%
\pgfpathlineto{\pgfqpoint{3.779722in}{2.440253in}}%
\pgfpathlineto{\pgfqpoint{3.781203in}{2.476940in}}%
\pgfpathlineto{\pgfqpoint{3.781400in}{2.473847in}}%
\pgfpathlineto{\pgfqpoint{3.781696in}{2.467774in}}%
\pgfpathlineto{\pgfqpoint{3.782091in}{2.478514in}}%
\pgfpathlineto{\pgfqpoint{3.782387in}{2.490229in}}%
\pgfpathlineto{\pgfqpoint{3.782881in}{2.467795in}}%
\pgfpathlineto{\pgfqpoint{3.783177in}{2.468902in}}%
\pgfpathlineto{\pgfqpoint{3.783670in}{2.447729in}}%
\pgfpathlineto{\pgfqpoint{3.785052in}{2.410014in}}%
\pgfpathlineto{\pgfqpoint{3.785250in}{2.418133in}}%
\pgfpathlineto{\pgfqpoint{3.785546in}{2.434111in}}%
\pgfpathlineto{\pgfqpoint{3.786237in}{2.416679in}}%
\pgfpathlineto{\pgfqpoint{3.787520in}{2.378217in}}%
\pgfpathlineto{\pgfqpoint{3.787914in}{2.393385in}}%
\pgfpathlineto{\pgfqpoint{3.788507in}{2.413223in}}%
\pgfpathlineto{\pgfqpoint{3.788901in}{2.440384in}}%
\pgfpathlineto{\pgfqpoint{3.789494in}{2.403996in}}%
\pgfpathlineto{\pgfqpoint{3.790875in}{2.372683in}}%
\pgfpathlineto{\pgfqpoint{3.791172in}{2.382914in}}%
\pgfpathlineto{\pgfqpoint{3.792257in}{2.456286in}}%
\pgfpathlineto{\pgfqpoint{3.793343in}{2.636452in}}%
\pgfpathlineto{\pgfqpoint{3.794231in}{2.604213in}}%
\pgfpathlineto{\pgfqpoint{3.794725in}{2.535984in}}%
\pgfpathlineto{\pgfqpoint{3.795416in}{2.359651in}}%
\pgfpathlineto{\pgfqpoint{3.796304in}{2.392470in}}%
\pgfpathlineto{\pgfqpoint{3.796699in}{2.391969in}}%
\pgfpathlineto{\pgfqpoint{3.796995in}{2.395563in}}%
\pgfpathlineto{\pgfqpoint{3.798080in}{2.441555in}}%
\pgfpathlineto{\pgfqpoint{3.798574in}{2.474898in}}%
\pgfpathlineto{\pgfqpoint{3.799067in}{2.447028in}}%
\pgfpathlineto{\pgfqpoint{3.799660in}{2.413954in}}%
\pgfpathlineto{\pgfqpoint{3.800252in}{2.437260in}}%
\pgfpathlineto{\pgfqpoint{3.800548in}{2.437897in}}%
\pgfpathlineto{\pgfqpoint{3.802226in}{2.463225in}}%
\pgfpathlineto{\pgfqpoint{3.802423in}{2.456961in}}%
\pgfpathlineto{\pgfqpoint{3.802917in}{2.429742in}}%
\pgfpathlineto{\pgfqpoint{3.803509in}{2.453123in}}%
\pgfpathlineto{\pgfqpoint{3.806470in}{2.490835in}}%
\pgfpathlineto{\pgfqpoint{3.806667in}{2.488936in}}%
\pgfpathlineto{\pgfqpoint{3.806766in}{2.488445in}}%
\pgfpathlineto{\pgfqpoint{3.806963in}{2.492938in}}%
\pgfpathlineto{\pgfqpoint{3.807161in}{2.498807in}}%
\pgfpathlineto{\pgfqpoint{3.807556in}{2.490295in}}%
\pgfpathlineto{\pgfqpoint{3.808148in}{2.497211in}}%
\pgfpathlineto{\pgfqpoint{3.808444in}{2.495176in}}%
\pgfpathlineto{\pgfqpoint{3.808839in}{2.501425in}}%
\pgfpathlineto{\pgfqpoint{3.809924in}{2.508414in}}%
\pgfpathlineto{\pgfqpoint{3.810122in}{2.505421in}}%
\pgfpathlineto{\pgfqpoint{3.810517in}{2.494838in}}%
\pgfpathlineto{\pgfqpoint{3.811306in}{2.501092in}}%
\pgfpathlineto{\pgfqpoint{3.811602in}{2.505728in}}%
\pgfpathlineto{\pgfqpoint{3.812096in}{2.497048in}}%
\pgfpathlineto{\pgfqpoint{3.813083in}{2.489729in}}%
\pgfpathlineto{\pgfqpoint{3.812688in}{2.498856in}}%
\pgfpathlineto{\pgfqpoint{3.813280in}{2.493262in}}%
\pgfpathlineto{\pgfqpoint{3.814366in}{2.501937in}}%
\pgfpathlineto{\pgfqpoint{3.814464in}{2.500896in}}%
\pgfpathlineto{\pgfqpoint{3.815451in}{2.490856in}}%
\pgfpathlineto{\pgfqpoint{3.815649in}{2.496540in}}%
\pgfpathlineto{\pgfqpoint{3.815846in}{2.500465in}}%
\pgfpathlineto{\pgfqpoint{3.816142in}{2.495816in}}%
\pgfpathlineto{\pgfqpoint{3.816636in}{2.497494in}}%
\pgfpathlineto{\pgfqpoint{3.818018in}{2.482000in}}%
\pgfpathlineto{\pgfqpoint{3.818116in}{2.482352in}}%
\pgfpathlineto{\pgfqpoint{3.818610in}{2.483338in}}%
\pgfpathlineto{\pgfqpoint{3.818807in}{2.481639in}}%
\pgfpathlineto{\pgfqpoint{3.820683in}{2.453805in}}%
\pgfpathlineto{\pgfqpoint{3.821176in}{2.460909in}}%
\pgfpathlineto{\pgfqpoint{3.821472in}{2.452947in}}%
\pgfpathlineto{\pgfqpoint{3.821670in}{2.449671in}}%
\pgfpathlineto{\pgfqpoint{3.822064in}{2.463335in}}%
\pgfpathlineto{\pgfqpoint{3.822163in}{2.464833in}}%
\pgfpathlineto{\pgfqpoint{3.822558in}{2.454765in}}%
\pgfpathlineto{\pgfqpoint{3.822656in}{2.454033in}}%
\pgfpathlineto{\pgfqpoint{3.822854in}{2.457892in}}%
\pgfpathlineto{\pgfqpoint{3.824038in}{2.470083in}}%
\pgfpathlineto{\pgfqpoint{3.823545in}{2.457043in}}%
\pgfpathlineto{\pgfqpoint{3.824137in}{2.468486in}}%
\pgfpathlineto{\pgfqpoint{3.825124in}{2.453387in}}%
\pgfpathlineto{\pgfqpoint{3.825420in}{2.459681in}}%
\pgfpathlineto{\pgfqpoint{3.825519in}{2.461274in}}%
\pgfpathlineto{\pgfqpoint{3.825914in}{2.449977in}}%
\pgfpathlineto{\pgfqpoint{3.826111in}{2.454056in}}%
\pgfpathlineto{\pgfqpoint{3.826506in}{2.472591in}}%
\pgfpathlineto{\pgfqpoint{3.827493in}{2.472503in}}%
\pgfpathlineto{\pgfqpoint{3.829664in}{2.517507in}}%
\pgfpathlineto{\pgfqpoint{3.830651in}{2.508792in}}%
\pgfpathlineto{\pgfqpoint{3.832428in}{2.453153in}}%
\pgfpathlineto{\pgfqpoint{3.832724in}{2.456720in}}%
\pgfpathlineto{\pgfqpoint{3.833020in}{2.449104in}}%
\pgfpathlineto{\pgfqpoint{3.833415in}{2.466588in}}%
\pgfpathlineto{\pgfqpoint{3.833513in}{2.469160in}}%
\pgfpathlineto{\pgfqpoint{3.834303in}{2.462112in}}%
\pgfpathlineto{\pgfqpoint{3.834796in}{2.436447in}}%
\pgfpathlineto{\pgfqpoint{3.835783in}{2.438179in}}%
\pgfpathlineto{\pgfqpoint{3.835882in}{2.438284in}}%
\pgfpathlineto{\pgfqpoint{3.836770in}{2.484863in}}%
\pgfpathlineto{\pgfqpoint{3.837560in}{2.453644in}}%
\pgfpathlineto{\pgfqpoint{3.838547in}{2.427068in}}%
\pgfpathlineto{\pgfqpoint{3.838843in}{2.432464in}}%
\pgfpathlineto{\pgfqpoint{3.839238in}{2.429891in}}%
\pgfpathlineto{\pgfqpoint{3.839435in}{2.437527in}}%
\pgfpathlineto{\pgfqpoint{3.840916in}{2.626693in}}%
\pgfpathlineto{\pgfqpoint{3.841607in}{2.664980in}}%
\pgfpathlineto{\pgfqpoint{3.842001in}{2.637232in}}%
\pgfpathlineto{\pgfqpoint{3.843383in}{2.386205in}}%
\pgfpathlineto{\pgfqpoint{3.845061in}{2.426963in}}%
\pgfpathlineto{\pgfqpoint{3.846443in}{2.449335in}}%
\pgfpathlineto{\pgfqpoint{3.845555in}{2.424156in}}%
\pgfpathlineto{\pgfqpoint{3.846640in}{2.443257in}}%
\pgfpathlineto{\pgfqpoint{3.847627in}{2.373684in}}%
\pgfpathlineto{\pgfqpoint{3.848220in}{2.401395in}}%
\pgfpathlineto{\pgfqpoint{3.850095in}{2.517409in}}%
\pgfpathlineto{\pgfqpoint{3.851082in}{2.497391in}}%
\pgfpathlineto{\pgfqpoint{3.851279in}{2.494204in}}%
\pgfpathlineto{\pgfqpoint{3.851674in}{2.498765in}}%
\pgfpathlineto{\pgfqpoint{3.851970in}{2.498245in}}%
\pgfpathlineto{\pgfqpoint{3.853154in}{2.508067in}}%
\pgfpathlineto{\pgfqpoint{3.853352in}{2.505614in}}%
\pgfpathlineto{\pgfqpoint{3.853747in}{2.501088in}}%
\pgfpathlineto{\pgfqpoint{3.854339in}{2.506452in}}%
\pgfpathlineto{\pgfqpoint{3.855326in}{2.512431in}}%
\pgfpathlineto{\pgfqpoint{3.855523in}{2.507539in}}%
\pgfpathlineto{\pgfqpoint{3.855918in}{2.493109in}}%
\pgfpathlineto{\pgfqpoint{3.856412in}{2.510709in}}%
\pgfpathlineto{\pgfqpoint{3.856609in}{2.506679in}}%
\pgfpathlineto{\pgfqpoint{3.856806in}{2.503595in}}%
\pgfpathlineto{\pgfqpoint{3.857201in}{2.509995in}}%
\pgfpathlineto{\pgfqpoint{3.857596in}{2.507459in}}%
\pgfpathlineto{\pgfqpoint{3.857892in}{2.512284in}}%
\pgfpathlineto{\pgfqpoint{3.858386in}{2.501512in}}%
\pgfpathlineto{\pgfqpoint{3.858978in}{2.514212in}}%
\pgfpathlineto{\pgfqpoint{3.859274in}{2.522475in}}%
\pgfpathlineto{\pgfqpoint{3.859866in}{2.507172in}}%
\pgfpathlineto{\pgfqpoint{3.859965in}{2.506989in}}%
\pgfpathlineto{\pgfqpoint{3.860063in}{2.508163in}}%
\pgfpathlineto{\pgfqpoint{3.860458in}{2.519331in}}%
\pgfpathlineto{\pgfqpoint{3.860853in}{2.507270in}}%
\pgfpathlineto{\pgfqpoint{3.861346in}{2.517632in}}%
\pgfpathlineto{\pgfqpoint{3.862432in}{2.487245in}}%
\pgfpathlineto{\pgfqpoint{3.863419in}{2.497021in}}%
\pgfpathlineto{\pgfqpoint{3.863617in}{2.502779in}}%
\pgfpathlineto{\pgfqpoint{3.864110in}{2.482704in}}%
\pgfpathlineto{\pgfqpoint{3.864307in}{2.483947in}}%
\pgfpathlineto{\pgfqpoint{3.864505in}{2.485641in}}%
\pgfpathlineto{\pgfqpoint{3.864801in}{2.479260in}}%
\pgfpathlineto{\pgfqpoint{3.866874in}{2.447688in}}%
\pgfpathlineto{\pgfqpoint{3.865393in}{2.487177in}}%
\pgfpathlineto{\pgfqpoint{3.866972in}{2.448510in}}%
\pgfpathlineto{\pgfqpoint{3.867170in}{2.451019in}}%
\pgfpathlineto{\pgfqpoint{3.867762in}{2.443442in}}%
\pgfpathlineto{\pgfqpoint{3.868157in}{2.437895in}}%
\pgfpathlineto{\pgfqpoint{3.869045in}{2.440988in}}%
\pgfpathlineto{\pgfqpoint{3.869341in}{2.440213in}}%
\pgfpathlineto{\pgfqpoint{3.869538in}{2.441565in}}%
\pgfpathlineto{\pgfqpoint{3.869736in}{2.443314in}}%
\pgfpathlineto{\pgfqpoint{3.869933in}{2.437460in}}%
\pgfpathlineto{\pgfqpoint{3.870131in}{2.430583in}}%
\pgfpathlineto{\pgfqpoint{3.870624in}{2.447138in}}%
\pgfpathlineto{\pgfqpoint{3.870920in}{2.442593in}}%
\pgfpathlineto{\pgfqpoint{3.872105in}{2.428643in}}%
\pgfpathlineto{\pgfqpoint{3.872302in}{2.434942in}}%
\pgfpathlineto{\pgfqpoint{3.872499in}{2.439895in}}%
\pgfpathlineto{\pgfqpoint{3.872796in}{2.427442in}}%
\pgfpathlineto{\pgfqpoint{3.873388in}{2.435896in}}%
\pgfpathlineto{\pgfqpoint{3.874276in}{2.420758in}}%
\pgfpathlineto{\pgfqpoint{3.874671in}{2.428399in}}%
\pgfpathlineto{\pgfqpoint{3.875658in}{2.443667in}}%
\pgfpathlineto{\pgfqpoint{3.877138in}{2.468501in}}%
\pgfpathlineto{\pgfqpoint{3.876151in}{2.442523in}}%
\pgfpathlineto{\pgfqpoint{3.877336in}{2.464054in}}%
\pgfpathlineto{\pgfqpoint{3.878027in}{2.466937in}}%
\pgfpathlineto{\pgfqpoint{3.878816in}{2.447992in}}%
\pgfpathlineto{\pgfqpoint{3.880099in}{2.406587in}}%
\pgfpathlineto{\pgfqpoint{3.881185in}{2.421012in}}%
\pgfpathlineto{\pgfqpoint{3.881382in}{2.427779in}}%
\pgfpathlineto{\pgfqpoint{3.881876in}{2.406302in}}%
\pgfpathlineto{\pgfqpoint{3.882271in}{2.389935in}}%
\pgfpathlineto{\pgfqpoint{3.882764in}{2.410276in}}%
\pgfpathlineto{\pgfqpoint{3.883060in}{2.401475in}}%
\pgfpathlineto{\pgfqpoint{3.883159in}{2.399165in}}%
\pgfpathlineto{\pgfqpoint{3.883554in}{2.409184in}}%
\pgfpathlineto{\pgfqpoint{3.884541in}{2.454222in}}%
\pgfpathlineto{\pgfqpoint{3.885034in}{2.431065in}}%
\pgfpathlineto{\pgfqpoint{3.886317in}{2.397700in}}%
\pgfpathlineto{\pgfqpoint{3.886416in}{2.398040in}}%
\pgfpathlineto{\pgfqpoint{3.887798in}{2.479185in}}%
\pgfpathlineto{\pgfqpoint{3.889081in}{2.651134in}}%
\pgfpathlineto{\pgfqpoint{3.889772in}{2.632643in}}%
\pgfpathlineto{\pgfqpoint{3.890167in}{2.587503in}}%
\pgfpathlineto{\pgfqpoint{3.891055in}{2.383870in}}%
\pgfpathlineto{\pgfqpoint{3.891943in}{2.410226in}}%
\pgfpathlineto{\pgfqpoint{3.892535in}{2.430713in}}%
\pgfpathlineto{\pgfqpoint{3.894115in}{2.510437in}}%
\pgfpathlineto{\pgfqpoint{3.894411in}{2.502759in}}%
\pgfpathlineto{\pgfqpoint{3.895102in}{2.460165in}}%
\pgfpathlineto{\pgfqpoint{3.895990in}{2.481830in}}%
\pgfpathlineto{\pgfqpoint{3.897273in}{2.529578in}}%
\pgfpathlineto{\pgfqpoint{3.897766in}{2.507497in}}%
\pgfpathlineto{\pgfqpoint{3.898062in}{2.503090in}}%
\pgfpathlineto{\pgfqpoint{3.898655in}{2.487058in}}%
\pgfpathlineto{\pgfqpoint{3.899049in}{2.501329in}}%
\pgfpathlineto{\pgfqpoint{3.900036in}{2.510726in}}%
\pgfpathlineto{\pgfqpoint{3.899543in}{2.498183in}}%
\pgfpathlineto{\pgfqpoint{3.900234in}{2.504843in}}%
\pgfpathlineto{\pgfqpoint{3.900333in}{2.502016in}}%
\pgfpathlineto{\pgfqpoint{3.900727in}{2.517488in}}%
\pgfpathlineto{\pgfqpoint{3.902899in}{2.548425in}}%
\pgfpathlineto{\pgfqpoint{3.901221in}{2.512681in}}%
\pgfpathlineto{\pgfqpoint{3.903096in}{2.548126in}}%
\pgfpathlineto{\pgfqpoint{3.905268in}{2.575542in}}%
\pgfpathlineto{\pgfqpoint{3.905465in}{2.572558in}}%
\pgfpathlineto{\pgfqpoint{3.905761in}{2.561597in}}%
\pgfpathlineto{\pgfqpoint{3.906649in}{2.568225in}}%
\pgfpathlineto{\pgfqpoint{3.907735in}{2.574518in}}%
\pgfpathlineto{\pgfqpoint{3.907932in}{2.572163in}}%
\pgfpathlineto{\pgfqpoint{3.909215in}{2.551022in}}%
\pgfpathlineto{\pgfqpoint{3.909413in}{2.557303in}}%
\pgfpathlineto{\pgfqpoint{3.909512in}{2.558932in}}%
\pgfpathlineto{\pgfqpoint{3.909906in}{2.550137in}}%
\pgfpathlineto{\pgfqpoint{3.910301in}{2.554814in}}%
\pgfpathlineto{\pgfqpoint{3.910597in}{2.558597in}}%
\pgfpathlineto{\pgfqpoint{3.911288in}{2.551392in}}%
\pgfpathlineto{\pgfqpoint{3.911782in}{2.563674in}}%
\pgfpathlineto{\pgfqpoint{3.912176in}{2.554520in}}%
\pgfpathlineto{\pgfqpoint{3.912473in}{2.569214in}}%
\pgfpathlineto{\pgfqpoint{3.912670in}{2.574948in}}%
\pgfpathlineto{\pgfqpoint{3.913262in}{2.554552in}}%
\pgfpathlineto{\pgfqpoint{3.913657in}{2.547638in}}%
\pgfpathlineto{\pgfqpoint{3.914249in}{2.556758in}}%
\pgfpathlineto{\pgfqpoint{3.914545in}{2.558728in}}%
\pgfpathlineto{\pgfqpoint{3.915039in}{2.554545in}}%
\pgfpathlineto{\pgfqpoint{3.915335in}{2.557458in}}%
\pgfpathlineto{\pgfqpoint{3.915828in}{2.551942in}}%
\pgfpathlineto{\pgfqpoint{3.916519in}{2.555904in}}%
\pgfpathlineto{\pgfqpoint{3.916815in}{2.559243in}}%
\pgfpathlineto{\pgfqpoint{3.917111in}{2.550390in}}%
\pgfpathlineto{\pgfqpoint{3.917309in}{2.544783in}}%
\pgfpathlineto{\pgfqpoint{3.917802in}{2.555795in}}%
\pgfpathlineto{\pgfqpoint{3.918098in}{2.553712in}}%
\pgfpathlineto{\pgfqpoint{3.919381in}{2.539541in}}%
\pgfpathlineto{\pgfqpoint{3.919974in}{2.545913in}}%
\pgfpathlineto{\pgfqpoint{3.920072in}{2.545864in}}%
\pgfpathlineto{\pgfqpoint{3.920171in}{2.546724in}}%
\pgfpathlineto{\pgfqpoint{3.920368in}{2.549484in}}%
\pgfpathlineto{\pgfqpoint{3.920763in}{2.540145in}}%
\pgfpathlineto{\pgfqpoint{3.921355in}{2.531781in}}%
\pgfpathlineto{\pgfqpoint{3.921849in}{2.536440in}}%
\pgfpathlineto{\pgfqpoint{3.922836in}{2.548550in}}%
\pgfpathlineto{\pgfqpoint{3.923527in}{2.547715in}}%
\pgfpathlineto{\pgfqpoint{3.926290in}{2.500758in}}%
\pgfpathlineto{\pgfqpoint{3.926586in}{2.511053in}}%
\pgfpathlineto{\pgfqpoint{3.928363in}{2.554975in}}%
\pgfpathlineto{\pgfqpoint{3.928659in}{2.560394in}}%
\pgfpathlineto{\pgfqpoint{3.929153in}{2.547791in}}%
\pgfpathlineto{\pgfqpoint{3.930633in}{2.478201in}}%
\pgfpathlineto{\pgfqpoint{3.931225in}{2.493587in}}%
\pgfpathlineto{\pgfqpoint{3.931916in}{2.527161in}}%
\pgfpathlineto{\pgfqpoint{3.932410in}{2.501190in}}%
\pgfpathlineto{\pgfqpoint{3.933693in}{2.448453in}}%
\pgfpathlineto{\pgfqpoint{3.933890in}{2.457150in}}%
\pgfpathlineto{\pgfqpoint{3.935469in}{2.592958in}}%
\pgfpathlineto{\pgfqpoint{3.936555in}{2.733452in}}%
\pgfpathlineto{\pgfqpoint{3.937049in}{2.702727in}}%
\pgfpathlineto{\pgfqpoint{3.937838in}{2.614949in}}%
\pgfpathlineto{\pgfqpoint{3.938430in}{2.453882in}}%
\pgfpathlineto{\pgfqpoint{3.939319in}{2.484936in}}%
\pgfpathlineto{\pgfqpoint{3.939516in}{2.483352in}}%
\pgfpathlineto{\pgfqpoint{3.939911in}{2.486917in}}%
\pgfpathlineto{\pgfqpoint{3.941687in}{2.528036in}}%
\pgfpathlineto{\pgfqpoint{3.941885in}{2.518387in}}%
\pgfpathlineto{\pgfqpoint{3.942872in}{2.461727in}}%
\pgfpathlineto{\pgfqpoint{3.943365in}{2.477798in}}%
\pgfpathlineto{\pgfqpoint{3.944944in}{2.520889in}}%
\pgfpathlineto{\pgfqpoint{3.945438in}{2.499898in}}%
\pgfpathlineto{\pgfqpoint{3.945635in}{2.496375in}}%
\pgfpathlineto{\pgfqpoint{3.946228in}{2.506004in}}%
\pgfpathlineto{\pgfqpoint{3.946622in}{2.519796in}}%
\pgfpathlineto{\pgfqpoint{3.947511in}{2.516246in}}%
\pgfpathlineto{\pgfqpoint{3.947905in}{2.533717in}}%
\pgfpathlineto{\pgfqpoint{3.948004in}{2.537426in}}%
\pgfpathlineto{\pgfqpoint{3.948399in}{2.513950in}}%
\pgfpathlineto{\pgfqpoint{3.948991in}{2.522041in}}%
\pgfpathlineto{\pgfqpoint{3.949583in}{2.509698in}}%
\pgfpathlineto{\pgfqpoint{3.949978in}{2.529499in}}%
\pgfpathlineto{\pgfqpoint{3.950866in}{2.515018in}}%
\pgfpathlineto{\pgfqpoint{3.951163in}{2.516200in}}%
\pgfpathlineto{\pgfqpoint{3.951360in}{2.512703in}}%
\pgfpathlineto{\pgfqpoint{3.951557in}{2.510143in}}%
\pgfpathlineto{\pgfqpoint{3.951755in}{2.521067in}}%
\pgfpathlineto{\pgfqpoint{3.951952in}{2.532896in}}%
\pgfpathlineto{\pgfqpoint{3.952544in}{2.498174in}}%
\pgfpathlineto{\pgfqpoint{3.953038in}{2.510746in}}%
\pgfpathlineto{\pgfqpoint{3.954025in}{2.508202in}}%
\pgfpathlineto{\pgfqpoint{3.954321in}{2.506306in}}%
\pgfpathlineto{\pgfqpoint{3.954814in}{2.496180in}}%
\pgfpathlineto{\pgfqpoint{3.955308in}{2.508067in}}%
\pgfpathlineto{\pgfqpoint{3.955604in}{2.504091in}}%
\pgfpathlineto{\pgfqpoint{3.955900in}{2.510966in}}%
\pgfpathlineto{\pgfqpoint{3.956097in}{2.515035in}}%
\pgfpathlineto{\pgfqpoint{3.956591in}{2.500449in}}%
\pgfpathlineto{\pgfqpoint{3.956788in}{2.498674in}}%
\pgfpathlineto{\pgfqpoint{3.957381in}{2.502162in}}%
\pgfpathlineto{\pgfqpoint{3.957677in}{2.508509in}}%
\pgfpathlineto{\pgfqpoint{3.958565in}{2.504650in}}%
\pgfpathlineto{\pgfqpoint{3.958960in}{2.505726in}}%
\pgfpathlineto{\pgfqpoint{3.959355in}{2.496185in}}%
\pgfpathlineto{\pgfqpoint{3.959453in}{2.494077in}}%
\pgfpathlineto{\pgfqpoint{3.959848in}{2.505992in}}%
\pgfpathlineto{\pgfqpoint{3.959947in}{2.507476in}}%
\pgfpathlineto{\pgfqpoint{3.960243in}{2.495892in}}%
\pgfpathlineto{\pgfqpoint{3.961329in}{2.478046in}}%
\pgfpathlineto{\pgfqpoint{3.960835in}{2.496784in}}%
\pgfpathlineto{\pgfqpoint{3.961526in}{2.486170in}}%
\pgfpathlineto{\pgfqpoint{3.961625in}{2.488701in}}%
\pgfpathlineto{\pgfqpoint{3.961921in}{2.474023in}}%
\pgfpathlineto{\pgfqpoint{3.963302in}{2.457848in}}%
\pgfpathlineto{\pgfqpoint{3.964191in}{2.451576in}}%
\pgfpathlineto{\pgfqpoint{3.963796in}{2.466943in}}%
\pgfpathlineto{\pgfqpoint{3.964388in}{2.455940in}}%
\pgfpathlineto{\pgfqpoint{3.964684in}{2.465179in}}%
\pgfpathlineto{\pgfqpoint{3.965079in}{2.452798in}}%
\pgfpathlineto{\pgfqpoint{3.965573in}{2.462198in}}%
\pgfpathlineto{\pgfqpoint{3.967152in}{2.440999in}}%
\pgfpathlineto{\pgfqpoint{3.965967in}{2.463213in}}%
\pgfpathlineto{\pgfqpoint{3.967349in}{2.448567in}}%
\pgfpathlineto{\pgfqpoint{3.968534in}{2.464439in}}%
\pgfpathlineto{\pgfqpoint{3.968632in}{2.463192in}}%
\pgfpathlineto{\pgfqpoint{3.968830in}{2.459726in}}%
\pgfpathlineto{\pgfqpoint{3.969422in}{2.469418in}}%
\pgfpathlineto{\pgfqpoint{3.971988in}{2.519985in}}%
\pgfpathlineto{\pgfqpoint{3.972284in}{2.511436in}}%
\pgfpathlineto{\pgfqpoint{3.975245in}{2.451731in}}%
\pgfpathlineto{\pgfqpoint{3.975344in}{2.452229in}}%
\pgfpathlineto{\pgfqpoint{3.975739in}{2.471978in}}%
\pgfpathlineto{\pgfqpoint{3.976528in}{2.460305in}}%
\pgfpathlineto{\pgfqpoint{3.977614in}{2.448955in}}%
\pgfpathlineto{\pgfqpoint{3.977811in}{2.451271in}}%
\pgfpathlineto{\pgfqpoint{3.978996in}{2.499654in}}%
\pgfpathlineto{\pgfqpoint{3.979193in}{2.507850in}}%
\pgfpathlineto{\pgfqpoint{3.979588in}{2.483137in}}%
\pgfpathlineto{\pgfqpoint{3.980970in}{2.436569in}}%
\pgfpathlineto{\pgfqpoint{3.981068in}{2.437894in}}%
\pgfpathlineto{\pgfqpoint{3.982154in}{2.481371in}}%
\pgfpathlineto{\pgfqpoint{3.983733in}{2.683886in}}%
\pgfpathlineto{\pgfqpoint{3.984621in}{2.651667in}}%
\pgfpathlineto{\pgfqpoint{3.985115in}{2.564077in}}%
\pgfpathlineto{\pgfqpoint{3.985608in}{2.423433in}}%
\pgfpathlineto{\pgfqpoint{3.986497in}{2.454923in}}%
\pgfpathlineto{\pgfqpoint{3.988866in}{2.502415in}}%
\pgfpathlineto{\pgfqpoint{3.986892in}{2.449999in}}%
\pgfpathlineto{\pgfqpoint{3.989063in}{2.496927in}}%
\pgfpathlineto{\pgfqpoint{3.990247in}{2.422460in}}%
\pgfpathlineto{\pgfqpoint{3.990741in}{2.439025in}}%
\pgfpathlineto{\pgfqpoint{3.991037in}{2.435208in}}%
\pgfpathlineto{\pgfqpoint{3.991333in}{2.442502in}}%
\pgfpathlineto{\pgfqpoint{3.992123in}{2.461673in}}%
\pgfpathlineto{\pgfqpoint{3.992419in}{2.447620in}}%
\pgfpathlineto{\pgfqpoint{3.993110in}{2.422471in}}%
\pgfpathlineto{\pgfqpoint{3.993603in}{2.440142in}}%
\pgfpathlineto{\pgfqpoint{3.993899in}{2.435063in}}%
\pgfpathlineto{\pgfqpoint{3.994294in}{2.444642in}}%
\pgfpathlineto{\pgfqpoint{3.995281in}{2.451286in}}%
\pgfpathlineto{\pgfqpoint{3.995478in}{2.448510in}}%
\pgfpathlineto{\pgfqpoint{3.995676in}{2.442372in}}%
\pgfpathlineto{\pgfqpoint{3.996071in}{2.462160in}}%
\pgfpathlineto{\pgfqpoint{3.996169in}{2.465900in}}%
\pgfpathlineto{\pgfqpoint{3.996663in}{2.457277in}}%
\pgfpathlineto{\pgfqpoint{3.997058in}{2.460549in}}%
\pgfpathlineto{\pgfqpoint{3.997452in}{2.462147in}}%
\pgfpathlineto{\pgfqpoint{3.997650in}{2.459076in}}%
\pgfpathlineto{\pgfqpoint{3.997946in}{2.449894in}}%
\pgfpathlineto{\pgfqpoint{3.998439in}{2.463283in}}%
\pgfpathlineto{\pgfqpoint{3.998735in}{2.455910in}}%
\pgfpathlineto{\pgfqpoint{3.999130in}{2.467860in}}%
\pgfpathlineto{\pgfqpoint{4.000117in}{2.462251in}}%
\pgfpathlineto{\pgfqpoint{4.000315in}{2.456871in}}%
\pgfpathlineto{\pgfqpoint{4.000808in}{2.477510in}}%
\pgfpathlineto{\pgfqpoint{4.001499in}{2.461685in}}%
\pgfpathlineto{\pgfqpoint{4.002486in}{2.463406in}}%
\pgfpathlineto{\pgfqpoint{4.002881in}{2.471372in}}%
\pgfpathlineto{\pgfqpoint{4.003473in}{2.463260in}}%
\pgfpathlineto{\pgfqpoint{4.003769in}{2.458791in}}%
\pgfpathlineto{\pgfqpoint{4.006138in}{2.389022in}}%
\pgfpathlineto{\pgfqpoint{4.006237in}{2.391491in}}%
\pgfpathlineto{\pgfqpoint{4.008112in}{2.459389in}}%
\pgfpathlineto{\pgfqpoint{4.008605in}{2.440820in}}%
\pgfpathlineto{\pgfqpoint{4.008901in}{2.447401in}}%
\pgfpathlineto{\pgfqpoint{4.009691in}{2.440421in}}%
\pgfpathlineto{\pgfqpoint{4.011270in}{2.404638in}}%
\pgfpathlineto{\pgfqpoint{4.011566in}{2.412454in}}%
\pgfpathlineto{\pgfqpoint{4.012553in}{2.419877in}}%
\pgfpathlineto{\pgfqpoint{4.012158in}{2.411654in}}%
\pgfpathlineto{\pgfqpoint{4.012652in}{2.417545in}}%
\pgfpathlineto{\pgfqpoint{4.013639in}{2.408120in}}%
\pgfpathlineto{\pgfqpoint{4.013244in}{2.425253in}}%
\pgfpathlineto{\pgfqpoint{4.013738in}{2.411873in}}%
\pgfpathlineto{\pgfqpoint{4.013935in}{2.419024in}}%
\pgfpathlineto{\pgfqpoint{4.014330in}{2.403766in}}%
\pgfpathlineto{\pgfqpoint{4.014922in}{2.415776in}}%
\pgfpathlineto{\pgfqpoint{4.016797in}{2.440630in}}%
\pgfpathlineto{\pgfqpoint{4.018377in}{2.472080in}}%
\pgfpathlineto{\pgfqpoint{4.019758in}{2.506824in}}%
\pgfpathlineto{\pgfqpoint{4.020943in}{2.503094in}}%
\pgfpathlineto{\pgfqpoint{4.021634in}{2.473967in}}%
\pgfpathlineto{\pgfqpoint{4.022226in}{2.494623in}}%
\pgfpathlineto{\pgfqpoint{4.022522in}{2.485269in}}%
\pgfpathlineto{\pgfqpoint{4.023114in}{2.502326in}}%
\pgfpathlineto{\pgfqpoint{4.023509in}{2.492760in}}%
\pgfpathlineto{\pgfqpoint{4.025088in}{2.464858in}}%
\pgfpathlineto{\pgfqpoint{4.026075in}{2.513956in}}%
\pgfpathlineto{\pgfqpoint{4.026569in}{2.479725in}}%
\pgfpathlineto{\pgfqpoint{4.028542in}{2.396955in}}%
\pgfpathlineto{\pgfqpoint{4.028740in}{2.392007in}}%
\pgfpathlineto{\pgfqpoint{4.029036in}{2.406676in}}%
\pgfpathlineto{\pgfqpoint{4.030418in}{2.579312in}}%
\pgfpathlineto{\pgfqpoint{4.031405in}{2.545432in}}%
\pgfpathlineto{\pgfqpoint{4.032194in}{2.402418in}}%
\pgfpathlineto{\pgfqpoint{4.032688in}{2.277582in}}%
\pgfpathlineto{\pgfqpoint{4.033477in}{2.302175in}}%
\pgfpathlineto{\pgfqpoint{4.035846in}{2.365899in}}%
\pgfpathlineto{\pgfqpoint{4.036142in}{2.347560in}}%
\pgfpathlineto{\pgfqpoint{4.036932in}{2.296655in}}%
\pgfpathlineto{\pgfqpoint{4.037524in}{2.320754in}}%
\pgfpathlineto{\pgfqpoint{4.037721in}{2.316302in}}%
\pgfpathlineto{\pgfqpoint{4.038116in}{2.332701in}}%
\pgfpathlineto{\pgfqpoint{4.038906in}{2.352304in}}%
\pgfpathlineto{\pgfqpoint{4.039794in}{2.343699in}}%
\pgfpathlineto{\pgfqpoint{4.039893in}{2.341542in}}%
\pgfpathlineto{\pgfqpoint{4.040288in}{2.353168in}}%
\pgfpathlineto{\pgfqpoint{4.042459in}{2.402761in}}%
\pgfpathlineto{\pgfqpoint{4.043150in}{2.396083in}}%
\pgfpathlineto{\pgfqpoint{4.043347in}{2.391060in}}%
\pgfpathlineto{\pgfqpoint{4.043841in}{2.406071in}}%
\pgfpathlineto{\pgfqpoint{4.044038in}{2.402794in}}%
\pgfpathlineto{\pgfqpoint{4.044137in}{2.401235in}}%
\pgfpathlineto{\pgfqpoint{4.044532in}{2.410167in}}%
\pgfpathlineto{\pgfqpoint{4.045124in}{2.425122in}}%
\pgfpathlineto{\pgfqpoint{4.046012in}{2.421987in}}%
\pgfpathlineto{\pgfqpoint{4.046210in}{2.420871in}}%
\pgfpathlineto{\pgfqpoint{4.046506in}{2.424850in}}%
\pgfpathlineto{\pgfqpoint{4.048578in}{2.444038in}}%
\pgfpathlineto{\pgfqpoint{4.048874in}{2.442098in}}%
\pgfpathlineto{\pgfqpoint{4.049072in}{2.446851in}}%
\pgfpathlineto{\pgfqpoint{4.050750in}{2.483477in}}%
\pgfpathlineto{\pgfqpoint{4.050947in}{2.479700in}}%
\pgfpathlineto{\pgfqpoint{4.051145in}{2.475679in}}%
\pgfpathlineto{\pgfqpoint{4.051737in}{2.489131in}}%
\pgfpathlineto{\pgfqpoint{4.052033in}{2.497411in}}%
\pgfpathlineto{\pgfqpoint{4.053809in}{2.530234in}}%
\pgfpathlineto{\pgfqpoint{4.054106in}{2.527186in}}%
\pgfpathlineto{\pgfqpoint{4.054500in}{2.533453in}}%
\pgfpathlineto{\pgfqpoint{4.055981in}{2.545038in}}%
\pgfpathlineto{\pgfqpoint{4.056178in}{2.542540in}}%
\pgfpathlineto{\pgfqpoint{4.057066in}{2.544869in}}%
\pgfpathlineto{\pgfqpoint{4.058053in}{2.520664in}}%
\pgfpathlineto{\pgfqpoint{4.063383in}{2.426665in}}%
\pgfpathlineto{\pgfqpoint{4.064272in}{2.416947in}}%
\pgfpathlineto{\pgfqpoint{4.063877in}{2.430315in}}%
\pgfpathlineto{\pgfqpoint{4.064469in}{2.426497in}}%
\pgfpathlineto{\pgfqpoint{4.064568in}{2.430035in}}%
\pgfpathlineto{\pgfqpoint{4.065061in}{2.408585in}}%
\pgfpathlineto{\pgfqpoint{4.065259in}{2.415841in}}%
\pgfpathlineto{\pgfqpoint{4.065357in}{2.416580in}}%
\pgfpathlineto{\pgfqpoint{4.065456in}{2.412548in}}%
\pgfpathlineto{\pgfqpoint{4.066542in}{2.385291in}}%
\pgfpathlineto{\pgfqpoint{4.066739in}{2.393260in}}%
\pgfpathlineto{\pgfqpoint{4.066838in}{2.395910in}}%
\pgfpathlineto{\pgfqpoint{4.067134in}{2.374797in}}%
\pgfpathlineto{\pgfqpoint{4.068910in}{2.295565in}}%
\pgfpathlineto{\pgfqpoint{4.069009in}{2.295802in}}%
\pgfpathlineto{\pgfqpoint{4.069206in}{2.294257in}}%
\pgfpathlineto{\pgfqpoint{4.072069in}{2.242073in}}%
\pgfpathlineto{\pgfqpoint{4.072365in}{2.256754in}}%
\pgfpathlineto{\pgfqpoint{4.073253in}{2.305077in}}%
\pgfpathlineto{\pgfqpoint{4.073648in}{2.282128in}}%
\pgfpathlineto{\pgfqpoint{4.075030in}{2.243526in}}%
\pgfpathlineto{\pgfqpoint{4.075326in}{2.236278in}}%
\pgfpathlineto{\pgfqpoint{4.075622in}{2.251096in}}%
\pgfpathlineto{\pgfqpoint{4.077793in}{2.519373in}}%
\pgfpathlineto{\pgfqpoint{4.078978in}{2.450394in}}%
\pgfpathlineto{\pgfqpoint{4.079669in}{2.286583in}}%
\pgfpathlineto{\pgfqpoint{4.080557in}{2.306264in}}%
\pgfpathlineto{\pgfqpoint{4.080754in}{2.303943in}}%
\pgfpathlineto{\pgfqpoint{4.080952in}{2.312357in}}%
\pgfpathlineto{\pgfqpoint{4.082827in}{2.407999in}}%
\pgfpathlineto{\pgfqpoint{4.083024in}{2.397116in}}%
\pgfpathlineto{\pgfqpoint{4.083913in}{2.361323in}}%
\pgfpathlineto{\pgfqpoint{4.084406in}{2.377945in}}%
\pgfpathlineto{\pgfqpoint{4.085985in}{2.418933in}}%
\pgfpathlineto{\pgfqpoint{4.086084in}{2.417890in}}%
\pgfpathlineto{\pgfqpoint{4.087071in}{2.364861in}}%
\pgfpathlineto{\pgfqpoint{4.088255in}{2.375262in}}%
\pgfpathlineto{\pgfqpoint{4.089242in}{2.435743in}}%
\pgfpathlineto{\pgfqpoint{4.091216in}{2.520435in}}%
\pgfpathlineto{\pgfqpoint{4.091414in}{2.518177in}}%
\pgfpathlineto{\pgfqpoint{4.091611in}{2.521679in}}%
\pgfpathlineto{\pgfqpoint{4.093190in}{2.548326in}}%
\pgfpathlineto{\pgfqpoint{4.094572in}{2.572602in}}%
\pgfpathlineto{\pgfqpoint{4.094769in}{2.563016in}}%
\pgfpathlineto{\pgfqpoint{4.095756in}{2.541027in}}%
\pgfpathlineto{\pgfqpoint{4.096053in}{2.548896in}}%
\pgfpathlineto{\pgfqpoint{4.096151in}{2.548753in}}%
\pgfpathlineto{\pgfqpoint{4.096546in}{2.535762in}}%
\pgfpathlineto{\pgfqpoint{4.097336in}{2.542419in}}%
\pgfpathlineto{\pgfqpoint{4.097632in}{2.539636in}}%
\pgfpathlineto{\pgfqpoint{4.097829in}{2.537533in}}%
\pgfpathlineto{\pgfqpoint{4.098323in}{2.544397in}}%
\pgfpathlineto{\pgfqpoint{4.098619in}{2.550378in}}%
\pgfpathlineto{\pgfqpoint{4.098915in}{2.540177in}}%
\pgfpathlineto{\pgfqpoint{4.100099in}{2.515438in}}%
\pgfpathlineto{\pgfqpoint{4.099606in}{2.544011in}}%
\pgfpathlineto{\pgfqpoint{4.100297in}{2.519479in}}%
\pgfpathlineto{\pgfqpoint{4.100494in}{2.524544in}}%
\pgfpathlineto{\pgfqpoint{4.100988in}{2.504577in}}%
\pgfpathlineto{\pgfqpoint{4.101481in}{2.489181in}}%
\pgfpathlineto{\pgfqpoint{4.103751in}{2.404178in}}%
\pgfpathlineto{\pgfqpoint{4.106613in}{2.325762in}}%
\pgfpathlineto{\pgfqpoint{4.106712in}{2.325805in}}%
\pgfpathlineto{\pgfqpoint{4.107304in}{2.316180in}}%
\pgfpathlineto{\pgfqpoint{4.108587in}{2.288946in}}%
\pgfpathlineto{\pgfqpoint{4.109278in}{2.292527in}}%
\pgfpathlineto{\pgfqpoint{4.109377in}{2.293373in}}%
\pgfpathlineto{\pgfqpoint{4.109673in}{2.288769in}}%
\pgfpathlineto{\pgfqpoint{4.110561in}{2.278923in}}%
\pgfpathlineto{\pgfqpoint{4.111055in}{2.282293in}}%
\pgfpathlineto{\pgfqpoint{4.111844in}{2.294023in}}%
\pgfpathlineto{\pgfqpoint{4.113029in}{2.312100in}}%
\pgfpathlineto{\pgfqpoint{4.112437in}{2.283330in}}%
\pgfpathlineto{\pgfqpoint{4.113226in}{2.307112in}}%
\pgfpathlineto{\pgfqpoint{4.113424in}{2.304158in}}%
\pgfpathlineto{\pgfqpoint{4.113818in}{2.318887in}}%
\pgfpathlineto{\pgfqpoint{4.113917in}{2.321172in}}%
\pgfpathlineto{\pgfqpoint{4.114312in}{2.308233in}}%
\pgfpathlineto{\pgfqpoint{4.116286in}{2.266687in}}%
\pgfpathlineto{\pgfqpoint{4.116483in}{2.275324in}}%
\pgfpathlineto{\pgfqpoint{4.116681in}{2.282588in}}%
\pgfpathlineto{\pgfqpoint{4.117075in}{2.267179in}}%
\pgfpathlineto{\pgfqpoint{4.117470in}{2.271727in}}%
\pgfpathlineto{\pgfqpoint{4.118359in}{2.250245in}}%
\pgfpathlineto{\pgfqpoint{4.118852in}{2.261462in}}%
\pgfpathlineto{\pgfqpoint{4.119148in}{2.253284in}}%
\pgfpathlineto{\pgfqpoint{4.119444in}{2.268952in}}%
\pgfpathlineto{\pgfqpoint{4.120431in}{2.337445in}}%
\pgfpathlineto{\pgfqpoint{4.120826in}{2.311094in}}%
\pgfpathlineto{\pgfqpoint{4.121616in}{2.284823in}}%
\pgfpathlineto{\pgfqpoint{4.122010in}{2.295798in}}%
\pgfpathlineto{\pgfqpoint{4.122405in}{2.283998in}}%
\pgfpathlineto{\pgfqpoint{4.122701in}{2.299268in}}%
\pgfpathlineto{\pgfqpoint{4.123590in}{2.378116in}}%
\pgfpathlineto{\pgfqpoint{4.125070in}{2.585928in}}%
\pgfpathlineto{\pgfqpoint{4.125958in}{2.537198in}}%
\pgfpathlineto{\pgfqpoint{4.126748in}{2.333454in}}%
\pgfpathlineto{\pgfqpoint{4.127932in}{2.385350in}}%
\pgfpathlineto{\pgfqpoint{4.130005in}{2.517004in}}%
\pgfpathlineto{\pgfqpoint{4.130499in}{2.477403in}}%
\pgfpathlineto{\pgfqpoint{4.131091in}{2.464447in}}%
\pgfpathlineto{\pgfqpoint{4.131387in}{2.480858in}}%
\pgfpathlineto{\pgfqpoint{4.132966in}{2.546992in}}%
\pgfpathlineto{\pgfqpoint{4.133163in}{2.545094in}}%
\pgfpathlineto{\pgfqpoint{4.133459in}{2.555449in}}%
\pgfpathlineto{\pgfqpoint{4.133756in}{2.541110in}}%
\pgfpathlineto{\pgfqpoint{4.134052in}{2.524983in}}%
\pgfpathlineto{\pgfqpoint{4.134841in}{2.540292in}}%
\pgfpathlineto{\pgfqpoint{4.135137in}{2.556728in}}%
\pgfpathlineto{\pgfqpoint{4.136124in}{2.556237in}}%
\pgfpathlineto{\pgfqpoint{4.136420in}{2.569033in}}%
\pgfpathlineto{\pgfqpoint{4.137309in}{2.563917in}}%
\pgfpathlineto{\pgfqpoint{4.138394in}{2.552957in}}%
\pgfpathlineto{\pgfqpoint{4.137901in}{2.572506in}}%
\pgfpathlineto{\pgfqpoint{4.138493in}{2.555593in}}%
\pgfpathlineto{\pgfqpoint{4.138691in}{2.561952in}}%
\pgfpathlineto{\pgfqpoint{4.139184in}{2.549086in}}%
\pgfpathlineto{\pgfqpoint{4.139480in}{2.552612in}}%
\pgfpathlineto{\pgfqpoint{4.142441in}{2.480749in}}%
\pgfpathlineto{\pgfqpoint{4.142638in}{2.482771in}}%
\pgfpathlineto{\pgfqpoint{4.142836in}{2.476337in}}%
\pgfpathlineto{\pgfqpoint{4.144514in}{2.428772in}}%
\pgfpathlineto{\pgfqpoint{4.145698in}{2.395263in}}%
\pgfpathlineto{\pgfqpoint{4.146685in}{2.397514in}}%
\pgfpathlineto{\pgfqpoint{4.146883in}{2.399825in}}%
\pgfpathlineto{\pgfqpoint{4.147277in}{2.388321in}}%
\pgfpathlineto{\pgfqpoint{4.152311in}{2.304856in}}%
\pgfpathlineto{\pgfqpoint{4.147672in}{2.390310in}}%
\pgfpathlineto{\pgfqpoint{4.152607in}{2.315671in}}%
\pgfpathlineto{\pgfqpoint{4.152706in}{2.317051in}}%
\pgfpathlineto{\pgfqpoint{4.153002in}{2.309217in}}%
\pgfpathlineto{\pgfqpoint{4.154186in}{2.297384in}}%
\pgfpathlineto{\pgfqpoint{4.154285in}{2.297602in}}%
\pgfpathlineto{\pgfqpoint{4.154680in}{2.311909in}}%
\pgfpathlineto{\pgfqpoint{4.155667in}{2.307054in}}%
\pgfpathlineto{\pgfqpoint{4.156752in}{2.297972in}}%
\pgfpathlineto{\pgfqpoint{4.156358in}{2.310566in}}%
\pgfpathlineto{\pgfqpoint{4.156851in}{2.300774in}}%
\pgfpathlineto{\pgfqpoint{4.158628in}{2.343533in}}%
\pgfpathlineto{\pgfqpoint{4.158726in}{2.342140in}}%
\pgfpathlineto{\pgfqpoint{4.158924in}{2.339128in}}%
\pgfpathlineto{\pgfqpoint{4.159319in}{2.351637in}}%
\pgfpathlineto{\pgfqpoint{4.160503in}{2.376860in}}%
\pgfpathlineto{\pgfqpoint{4.161490in}{2.366173in}}%
\pgfpathlineto{\pgfqpoint{4.162970in}{2.329151in}}%
\pgfpathlineto{\pgfqpoint{4.163168in}{2.330747in}}%
\pgfpathlineto{\pgfqpoint{4.163464in}{2.333699in}}%
\pgfpathlineto{\pgfqpoint{4.163859in}{2.325954in}}%
\pgfpathlineto{\pgfqpoint{4.164254in}{2.332519in}}%
\pgfpathlineto{\pgfqpoint{4.164550in}{2.325510in}}%
\pgfpathlineto{\pgfqpoint{4.164944in}{2.315666in}}%
\pgfpathlineto{\pgfqpoint{4.165734in}{2.320717in}}%
\pgfpathlineto{\pgfqpoint{4.165833in}{2.320712in}}%
\pgfpathlineto{\pgfqpoint{4.166228in}{2.338870in}}%
\pgfpathlineto{\pgfqpoint{4.167017in}{2.384350in}}%
\pgfpathlineto{\pgfqpoint{4.167511in}{2.360663in}}%
\pgfpathlineto{\pgfqpoint{4.168596in}{2.336425in}}%
\pgfpathlineto{\pgfqpoint{4.168794in}{2.338342in}}%
\pgfpathlineto{\pgfqpoint{4.169287in}{2.347076in}}%
\pgfpathlineto{\pgfqpoint{4.171162in}{2.556904in}}%
\pgfpathlineto{\pgfqpoint{4.171656in}{2.586811in}}%
\pgfpathlineto{\pgfqpoint{4.172347in}{2.574453in}}%
\pgfpathlineto{\pgfqpoint{4.172446in}{2.575741in}}%
\pgfpathlineto{\pgfqpoint{4.172544in}{2.571469in}}%
\pgfpathlineto{\pgfqpoint{4.173334in}{2.393036in}}%
\pgfpathlineto{\pgfqpoint{4.174716in}{2.439509in}}%
\pgfpathlineto{\pgfqpoint{4.174913in}{2.442329in}}%
\pgfpathlineto{\pgfqpoint{4.176591in}{2.485584in}}%
\pgfpathlineto{\pgfqpoint{4.176690in}{2.483489in}}%
\pgfpathlineto{\pgfqpoint{4.178861in}{2.374509in}}%
\pgfpathlineto{\pgfqpoint{4.179157in}{2.375352in}}%
\pgfpathlineto{\pgfqpoint{4.179749in}{2.369373in}}%
\pgfpathlineto{\pgfqpoint{4.181526in}{2.305415in}}%
\pgfpathlineto{\pgfqpoint{4.181723in}{2.310958in}}%
\pgfpathlineto{\pgfqpoint{4.182908in}{2.321199in}}%
\pgfpathlineto{\pgfqpoint{4.182315in}{2.304774in}}%
\pgfpathlineto{\pgfqpoint{4.183105in}{2.320388in}}%
\pgfpathlineto{\pgfqpoint{4.183599in}{2.309876in}}%
\pgfpathlineto{\pgfqpoint{4.183993in}{2.319816in}}%
\pgfpathlineto{\pgfqpoint{4.184882in}{2.328696in}}%
\pgfpathlineto{\pgfqpoint{4.184487in}{2.316570in}}%
\pgfpathlineto{\pgfqpoint{4.185079in}{2.320662in}}%
\pgfpathlineto{\pgfqpoint{4.185276in}{2.315224in}}%
\pgfpathlineto{\pgfqpoint{4.186066in}{2.322432in}}%
\pgfpathlineto{\pgfqpoint{4.187053in}{2.330420in}}%
\pgfpathlineto{\pgfqpoint{4.187250in}{2.325117in}}%
\pgfpathlineto{\pgfqpoint{4.187349in}{2.323259in}}%
\pgfpathlineto{\pgfqpoint{4.187645in}{2.333209in}}%
\pgfpathlineto{\pgfqpoint{4.188533in}{2.344862in}}%
\pgfpathlineto{\pgfqpoint{4.188830in}{2.337623in}}%
\pgfpathlineto{\pgfqpoint{4.188928in}{2.335611in}}%
\pgfpathlineto{\pgfqpoint{4.189323in}{2.348171in}}%
\pgfpathlineto{\pgfqpoint{4.190804in}{2.360459in}}%
\pgfpathlineto{\pgfqpoint{4.189718in}{2.347083in}}%
\pgfpathlineto{\pgfqpoint{4.190902in}{2.358302in}}%
\pgfpathlineto{\pgfqpoint{4.191198in}{2.349465in}}%
\pgfpathlineto{\pgfqpoint{4.191593in}{2.366040in}}%
\pgfpathlineto{\pgfqpoint{4.191988in}{2.358817in}}%
\pgfpathlineto{\pgfqpoint{4.192383in}{2.363645in}}%
\pgfpathlineto{\pgfqpoint{4.192778in}{2.358856in}}%
\pgfpathlineto{\pgfqpoint{4.193962in}{2.343251in}}%
\pgfpathlineto{\pgfqpoint{4.194159in}{2.349537in}}%
\pgfpathlineto{\pgfqpoint{4.194357in}{2.353012in}}%
\pgfpathlineto{\pgfqpoint{4.194653in}{2.337055in}}%
\pgfpathlineto{\pgfqpoint{4.196035in}{2.317706in}}%
\pgfpathlineto{\pgfqpoint{4.198009in}{2.296535in}}%
\pgfpathlineto{\pgfqpoint{4.198502in}{2.305042in}}%
\pgfpathlineto{\pgfqpoint{4.199489in}{2.313689in}}%
\pgfpathlineto{\pgfqpoint{4.198996in}{2.304711in}}%
\pgfpathlineto{\pgfqpoint{4.200081in}{2.311933in}}%
\pgfpathlineto{\pgfqpoint{4.200279in}{2.311180in}}%
\pgfpathlineto{\pgfqpoint{4.200377in}{2.310071in}}%
\pgfpathlineto{\pgfqpoint{4.200673in}{2.317397in}}%
\pgfpathlineto{\pgfqpoint{4.200772in}{2.319981in}}%
\pgfpathlineto{\pgfqpoint{4.201266in}{2.307824in}}%
\pgfpathlineto{\pgfqpoint{4.201660in}{2.315025in}}%
\pgfpathlineto{\pgfqpoint{4.202845in}{2.335936in}}%
\pgfpathlineto{\pgfqpoint{4.203338in}{2.335238in}}%
\pgfpathlineto{\pgfqpoint{4.204720in}{2.367424in}}%
\pgfpathlineto{\pgfqpoint{4.205016in}{2.364710in}}%
\pgfpathlineto{\pgfqpoint{4.205214in}{2.370202in}}%
\pgfpathlineto{\pgfqpoint{4.206497in}{2.413566in}}%
\pgfpathlineto{\pgfqpoint{4.206990in}{2.404229in}}%
\pgfpathlineto{\pgfqpoint{4.207385in}{2.410733in}}%
\pgfpathlineto{\pgfqpoint{4.207780in}{2.399141in}}%
\pgfpathlineto{\pgfqpoint{4.207878in}{2.398508in}}%
\pgfpathlineto{\pgfqpoint{4.208076in}{2.403506in}}%
\pgfpathlineto{\pgfqpoint{4.208273in}{2.406528in}}%
\pgfpathlineto{\pgfqpoint{4.208668in}{2.390232in}}%
\pgfpathlineto{\pgfqpoint{4.209556in}{2.362083in}}%
\pgfpathlineto{\pgfqpoint{4.210149in}{2.374764in}}%
\pgfpathlineto{\pgfqpoint{4.210445in}{2.390312in}}%
\pgfpathlineto{\pgfqpoint{4.211037in}{2.362200in}}%
\pgfpathlineto{\pgfqpoint{4.212024in}{2.348428in}}%
\pgfpathlineto{\pgfqpoint{4.212715in}{2.352454in}}%
\pgfpathlineto{\pgfqpoint{4.213800in}{2.402512in}}%
\pgfpathlineto{\pgfqpoint{4.214195in}{2.375225in}}%
\pgfpathlineto{\pgfqpoint{4.215577in}{2.351203in}}%
\pgfpathlineto{\pgfqpoint{4.215873in}{2.361710in}}%
\pgfpathlineto{\pgfqpoint{4.217156in}{2.475333in}}%
\pgfpathlineto{\pgfqpoint{4.218143in}{2.623127in}}%
\pgfpathlineto{\pgfqpoint{4.218834in}{2.596652in}}%
\pgfpathlineto{\pgfqpoint{4.219130in}{2.580153in}}%
\pgfpathlineto{\pgfqpoint{4.220117in}{2.354459in}}%
\pgfpathlineto{\pgfqpoint{4.221400in}{2.379425in}}%
\pgfpathlineto{\pgfqpoint{4.221499in}{2.376844in}}%
\pgfpathlineto{\pgfqpoint{4.221795in}{2.391097in}}%
\pgfpathlineto{\pgfqpoint{4.223177in}{2.462861in}}%
\pgfpathlineto{\pgfqpoint{4.223670in}{2.446092in}}%
\pgfpathlineto{\pgfqpoint{4.224263in}{2.404387in}}%
\pgfpathlineto{\pgfqpoint{4.224953in}{2.424237in}}%
\pgfpathlineto{\pgfqpoint{4.226631in}{2.452444in}}%
\pgfpathlineto{\pgfqpoint{4.226730in}{2.452599in}}%
\pgfpathlineto{\pgfqpoint{4.227816in}{2.424849in}}%
\pgfpathlineto{\pgfqpoint{4.228112in}{2.438328in}}%
\pgfpathlineto{\pgfqpoint{4.229296in}{2.452389in}}%
\pgfpathlineto{\pgfqpoint{4.228803in}{2.437651in}}%
\pgfpathlineto{\pgfqpoint{4.229395in}{2.452279in}}%
\pgfpathlineto{\pgfqpoint{4.230678in}{2.434813in}}%
\pgfpathlineto{\pgfqpoint{4.230875in}{2.438289in}}%
\pgfpathlineto{\pgfqpoint{4.232455in}{2.457580in}}%
\pgfpathlineto{\pgfqpoint{4.232849in}{2.451133in}}%
\pgfpathlineto{\pgfqpoint{4.233442in}{2.456944in}}%
\pgfpathlineto{\pgfqpoint{4.233738in}{2.460161in}}%
\pgfpathlineto{\pgfqpoint{4.234034in}{2.451552in}}%
\pgfpathlineto{\pgfqpoint{4.234132in}{2.449966in}}%
\pgfpathlineto{\pgfqpoint{4.234527in}{2.458977in}}%
\pgfpathlineto{\pgfqpoint{4.234922in}{2.455007in}}%
\pgfpathlineto{\pgfqpoint{4.236205in}{2.476538in}}%
\pgfpathlineto{\pgfqpoint{4.236402in}{2.469601in}}%
\pgfpathlineto{\pgfqpoint{4.236600in}{2.462888in}}%
\pgfpathlineto{\pgfqpoint{4.237093in}{2.485319in}}%
\pgfpathlineto{\pgfqpoint{4.237291in}{2.478003in}}%
\pgfpathlineto{\pgfqpoint{4.237389in}{2.476615in}}%
\pgfpathlineto{\pgfqpoint{4.237686in}{2.487021in}}%
\pgfpathlineto{\pgfqpoint{4.238475in}{2.489333in}}%
\pgfpathlineto{\pgfqpoint{4.238080in}{2.481723in}}%
\pgfpathlineto{\pgfqpoint{4.238574in}{2.487124in}}%
\pgfpathlineto{\pgfqpoint{4.238870in}{2.481326in}}%
\pgfpathlineto{\pgfqpoint{4.239363in}{2.490874in}}%
\pgfpathlineto{\pgfqpoint{4.239561in}{2.490577in}}%
\pgfpathlineto{\pgfqpoint{4.239660in}{2.490401in}}%
\pgfpathlineto{\pgfqpoint{4.239758in}{2.491220in}}%
\pgfpathlineto{\pgfqpoint{4.240153in}{2.502655in}}%
\pgfpathlineto{\pgfqpoint{4.240449in}{2.490072in}}%
\pgfpathlineto{\pgfqpoint{4.241634in}{2.471734in}}%
\pgfpathlineto{\pgfqpoint{4.241831in}{2.471995in}}%
\pgfpathlineto{\pgfqpoint{4.242818in}{2.460484in}}%
\pgfpathlineto{\pgfqpoint{4.244002in}{2.439791in}}%
\pgfpathlineto{\pgfqpoint{4.243311in}{2.464923in}}%
\pgfpathlineto{\pgfqpoint{4.244200in}{2.443667in}}%
\pgfpathlineto{\pgfqpoint{4.244496in}{2.449031in}}%
\pgfpathlineto{\pgfqpoint{4.244891in}{2.436918in}}%
\pgfpathlineto{\pgfqpoint{4.245285in}{2.443946in}}%
\pgfpathlineto{\pgfqpoint{4.246470in}{2.421460in}}%
\pgfpathlineto{\pgfqpoint{4.246568in}{2.418489in}}%
\pgfpathlineto{\pgfqpoint{4.246865in}{2.434387in}}%
\pgfpathlineto{\pgfqpoint{4.247753in}{2.445638in}}%
\pgfpathlineto{\pgfqpoint{4.247358in}{2.433709in}}%
\pgfpathlineto{\pgfqpoint{4.247950in}{2.437056in}}%
\pgfpathlineto{\pgfqpoint{4.248740in}{2.422694in}}%
\pgfpathlineto{\pgfqpoint{4.249135in}{2.433880in}}%
\pgfpathlineto{\pgfqpoint{4.249924in}{2.443371in}}%
\pgfpathlineto{\pgfqpoint{4.250714in}{2.438807in}}%
\pgfpathlineto{\pgfqpoint{4.250813in}{2.437323in}}%
\pgfpathlineto{\pgfqpoint{4.251306in}{2.444052in}}%
\pgfpathlineto{\pgfqpoint{4.251701in}{2.440021in}}%
\pgfpathlineto{\pgfqpoint{4.251997in}{2.437208in}}%
\pgfpathlineto{\pgfqpoint{4.253379in}{2.416675in}}%
\pgfpathlineto{\pgfqpoint{4.253675in}{2.419857in}}%
\pgfpathlineto{\pgfqpoint{4.254760in}{2.448439in}}%
\pgfpathlineto{\pgfqpoint{4.255155in}{2.462615in}}%
\pgfpathlineto{\pgfqpoint{4.255945in}{2.455046in}}%
\pgfpathlineto{\pgfqpoint{4.256142in}{2.461189in}}%
\pgfpathlineto{\pgfqpoint{4.256932in}{2.451889in}}%
\pgfpathlineto{\pgfqpoint{4.257425in}{2.444465in}}%
\pgfpathlineto{\pgfqpoint{4.259005in}{2.409119in}}%
\pgfpathlineto{\pgfqpoint{4.259399in}{2.417902in}}%
\pgfpathlineto{\pgfqpoint{4.260288in}{2.471265in}}%
\pgfpathlineto{\pgfqpoint{4.260979in}{2.439008in}}%
\pgfpathlineto{\pgfqpoint{4.262360in}{2.406306in}}%
\pgfpathlineto{\pgfqpoint{4.262459in}{2.404542in}}%
\pgfpathlineto{\pgfqpoint{4.262755in}{2.417500in}}%
\pgfpathlineto{\pgfqpoint{4.264236in}{2.609610in}}%
\pgfpathlineto{\pgfqpoint{4.265025in}{2.682729in}}%
\pgfpathlineto{\pgfqpoint{4.265519in}{2.663814in}}%
\pgfpathlineto{\pgfqpoint{4.265617in}{2.665069in}}%
\pgfpathlineto{\pgfqpoint{4.265815in}{2.656700in}}%
\pgfpathlineto{\pgfqpoint{4.266900in}{2.417438in}}%
\pgfpathlineto{\pgfqpoint{4.268184in}{2.463106in}}%
\pgfpathlineto{\pgfqpoint{4.268381in}{2.457908in}}%
\pgfpathlineto{\pgfqpoint{4.268677in}{2.471696in}}%
\pgfpathlineto{\pgfqpoint{4.269664in}{2.526594in}}%
\pgfpathlineto{\pgfqpoint{4.270355in}{2.517341in}}%
\pgfpathlineto{\pgfqpoint{4.270848in}{2.469596in}}%
\pgfpathlineto{\pgfqpoint{4.271835in}{2.478063in}}%
\pgfpathlineto{\pgfqpoint{4.273020in}{2.514504in}}%
\pgfpathlineto{\pgfqpoint{4.273612in}{2.500322in}}%
\pgfpathlineto{\pgfqpoint{4.274402in}{2.484615in}}%
\pgfpathlineto{\pgfqpoint{4.274796in}{2.492420in}}%
\pgfpathlineto{\pgfqpoint{4.275092in}{2.490287in}}%
\pgfpathlineto{\pgfqpoint{4.275290in}{2.495452in}}%
\pgfpathlineto{\pgfqpoint{4.275389in}{2.497042in}}%
\pgfpathlineto{\pgfqpoint{4.275685in}{2.485738in}}%
\pgfpathlineto{\pgfqpoint{4.275783in}{2.483788in}}%
\pgfpathlineto{\pgfqpoint{4.275981in}{2.493275in}}%
\pgfpathlineto{\pgfqpoint{4.276178in}{2.503130in}}%
\pgfpathlineto{\pgfqpoint{4.276573in}{2.493234in}}%
\pgfpathlineto{\pgfqpoint{4.277066in}{2.494527in}}%
\pgfpathlineto{\pgfqpoint{4.278843in}{2.509722in}}%
\pgfpathlineto{\pgfqpoint{4.277757in}{2.493121in}}%
\pgfpathlineto{\pgfqpoint{4.279238in}{2.505333in}}%
\pgfpathlineto{\pgfqpoint{4.279337in}{2.505300in}}%
\pgfpathlineto{\pgfqpoint{4.279731in}{2.500084in}}%
\pgfpathlineto{\pgfqpoint{4.280027in}{2.508277in}}%
\pgfpathlineto{\pgfqpoint{4.281311in}{2.516537in}}%
\pgfpathlineto{\pgfqpoint{4.280817in}{2.505185in}}%
\pgfpathlineto{\pgfqpoint{4.281409in}{2.515550in}}%
\pgfpathlineto{\pgfqpoint{4.281705in}{2.509532in}}%
\pgfpathlineto{\pgfqpoint{4.282001in}{2.518809in}}%
\pgfpathlineto{\pgfqpoint{4.282495in}{2.511124in}}%
\pgfpathlineto{\pgfqpoint{4.283383in}{2.520807in}}%
\pgfpathlineto{\pgfqpoint{4.283679in}{2.516029in}}%
\pgfpathlineto{\pgfqpoint{4.284765in}{2.525995in}}%
\pgfpathlineto{\pgfqpoint{4.285258in}{2.519416in}}%
\pgfpathlineto{\pgfqpoint{4.286245in}{2.525492in}}%
\pgfpathlineto{\pgfqpoint{4.286640in}{2.520737in}}%
\pgfpathlineto{\pgfqpoint{4.290588in}{2.472111in}}%
\pgfpathlineto{\pgfqpoint{4.291575in}{2.447399in}}%
\pgfpathlineto{\pgfqpoint{4.291180in}{2.478584in}}%
\pgfpathlineto{\pgfqpoint{4.291871in}{2.461865in}}%
\pgfpathlineto{\pgfqpoint{4.292069in}{2.470073in}}%
\pgfpathlineto{\pgfqpoint{4.292760in}{2.449830in}}%
\pgfpathlineto{\pgfqpoint{4.292957in}{2.446751in}}%
\pgfpathlineto{\pgfqpoint{4.293352in}{2.459714in}}%
\pgfpathlineto{\pgfqpoint{4.293450in}{2.460268in}}%
\pgfpathlineto{\pgfqpoint{4.293549in}{2.457861in}}%
\pgfpathlineto{\pgfqpoint{4.294635in}{2.445405in}}%
\pgfpathlineto{\pgfqpoint{4.294240in}{2.457939in}}%
\pgfpathlineto{\pgfqpoint{4.294832in}{2.447846in}}%
\pgfpathlineto{\pgfqpoint{4.294931in}{2.449579in}}%
\pgfpathlineto{\pgfqpoint{4.295326in}{2.443424in}}%
\pgfpathlineto{\pgfqpoint{4.295819in}{2.446147in}}%
\pgfpathlineto{\pgfqpoint{4.296115in}{2.437693in}}%
\pgfpathlineto{\pgfqpoint{4.296708in}{2.452446in}}%
\pgfpathlineto{\pgfqpoint{4.296806in}{2.452617in}}%
\pgfpathlineto{\pgfqpoint{4.296905in}{2.451680in}}%
\pgfpathlineto{\pgfqpoint{4.297793in}{2.444404in}}%
\pgfpathlineto{\pgfqpoint{4.297991in}{2.448350in}}%
\pgfpathlineto{\pgfqpoint{4.299570in}{2.467657in}}%
\pgfpathlineto{\pgfqpoint{4.298583in}{2.443987in}}%
\pgfpathlineto{\pgfqpoint{4.299669in}{2.466861in}}%
\pgfpathlineto{\pgfqpoint{4.299866in}{2.464971in}}%
\pgfpathlineto{\pgfqpoint{4.300261in}{2.471941in}}%
\pgfpathlineto{\pgfqpoint{4.301544in}{2.492454in}}%
\pgfpathlineto{\pgfqpoint{4.301840in}{2.482935in}}%
\pgfpathlineto{\pgfqpoint{4.304209in}{2.419706in}}%
\pgfpathlineto{\pgfqpoint{4.302333in}{2.488386in}}%
\pgfpathlineto{\pgfqpoint{4.304505in}{2.431628in}}%
\pgfpathlineto{\pgfqpoint{4.305196in}{2.440203in}}%
\pgfpathlineto{\pgfqpoint{4.305492in}{2.430891in}}%
\pgfpathlineto{\pgfqpoint{4.306775in}{2.399114in}}%
\pgfpathlineto{\pgfqpoint{4.307071in}{2.406985in}}%
\pgfpathlineto{\pgfqpoint{4.308354in}{2.471215in}}%
\pgfpathlineto{\pgfqpoint{4.308848in}{2.430798in}}%
\pgfpathlineto{\pgfqpoint{4.310131in}{2.411305in}}%
\pgfpathlineto{\pgfqpoint{4.310427in}{2.417775in}}%
\pgfpathlineto{\pgfqpoint{4.311710in}{2.508606in}}%
\pgfpathlineto{\pgfqpoint{4.312993in}{2.681938in}}%
\pgfpathlineto{\pgfqpoint{4.313486in}{2.631080in}}%
\pgfpathlineto{\pgfqpoint{4.313782in}{2.641298in}}%
\pgfpathlineto{\pgfqpoint{4.313881in}{2.633901in}}%
\pgfpathlineto{\pgfqpoint{4.314769in}{2.408021in}}%
\pgfpathlineto{\pgfqpoint{4.316053in}{2.426474in}}%
\pgfpathlineto{\pgfqpoint{4.316349in}{2.445299in}}%
\pgfpathlineto{\pgfqpoint{4.317928in}{2.511845in}}%
\pgfpathlineto{\pgfqpoint{4.318027in}{2.511449in}}%
\pgfpathlineto{\pgfqpoint{4.318717in}{2.456853in}}%
\pgfpathlineto{\pgfqpoint{4.318915in}{2.442547in}}%
\pgfpathlineto{\pgfqpoint{4.319606in}{2.473527in}}%
\pgfpathlineto{\pgfqpoint{4.321086in}{2.514757in}}%
\pgfpathlineto{\pgfqpoint{4.320198in}{2.470993in}}%
\pgfpathlineto{\pgfqpoint{4.321580in}{2.496277in}}%
\pgfpathlineto{\pgfqpoint{4.322567in}{2.489449in}}%
\pgfpathlineto{\pgfqpoint{4.322764in}{2.492162in}}%
\pgfpathlineto{\pgfqpoint{4.323948in}{2.507075in}}%
\pgfpathlineto{\pgfqpoint{4.324146in}{2.505400in}}%
\pgfpathlineto{\pgfqpoint{4.324442in}{2.501884in}}%
\pgfpathlineto{\pgfqpoint{4.324837in}{2.511105in}}%
\pgfpathlineto{\pgfqpoint{4.324935in}{2.511857in}}%
\pgfpathlineto{\pgfqpoint{4.325232in}{2.507120in}}%
\pgfpathlineto{\pgfqpoint{4.325429in}{2.505245in}}%
\pgfpathlineto{\pgfqpoint{4.325922in}{2.512138in}}%
\pgfpathlineto{\pgfqpoint{4.326219in}{2.511242in}}%
\pgfpathlineto{\pgfqpoint{4.326515in}{2.521068in}}%
\pgfpathlineto{\pgfqpoint{4.326712in}{2.528363in}}%
\pgfpathlineto{\pgfqpoint{4.327107in}{2.520470in}}%
\pgfpathlineto{\pgfqpoint{4.327600in}{2.522104in}}%
\pgfpathlineto{\pgfqpoint{4.327896in}{2.518317in}}%
\pgfpathlineto{\pgfqpoint{4.328291in}{2.527467in}}%
\pgfpathlineto{\pgfqpoint{4.328489in}{2.531749in}}%
\pgfpathlineto{\pgfqpoint{4.328982in}{2.517476in}}%
\pgfpathlineto{\pgfqpoint{4.329377in}{2.529704in}}%
\pgfpathlineto{\pgfqpoint{4.329870in}{2.515851in}}%
\pgfpathlineto{\pgfqpoint{4.330759in}{2.524379in}}%
\pgfpathlineto{\pgfqpoint{4.330857in}{2.525670in}}%
\pgfpathlineto{\pgfqpoint{4.331153in}{2.517156in}}%
\pgfpathlineto{\pgfqpoint{4.331252in}{2.515939in}}%
\pgfpathlineto{\pgfqpoint{4.331548in}{2.525221in}}%
\pgfpathlineto{\pgfqpoint{4.332239in}{2.529935in}}%
\pgfpathlineto{\pgfqpoint{4.332437in}{2.523820in}}%
\pgfpathlineto{\pgfqpoint{4.334608in}{2.466655in}}%
\pgfpathlineto{\pgfqpoint{4.334904in}{2.467947in}}%
\pgfpathlineto{\pgfqpoint{4.335003in}{2.467612in}}%
\pgfpathlineto{\pgfqpoint{4.335496in}{2.438380in}}%
\pgfpathlineto{\pgfqpoint{4.335792in}{2.467282in}}%
\pgfpathlineto{\pgfqpoint{4.336878in}{2.509220in}}%
\pgfpathlineto{\pgfqpoint{4.337075in}{2.506406in}}%
\pgfpathlineto{\pgfqpoint{4.339938in}{2.469902in}}%
\pgfpathlineto{\pgfqpoint{4.337569in}{2.508434in}}%
\pgfpathlineto{\pgfqpoint{4.340332in}{2.472542in}}%
\pgfpathlineto{\pgfqpoint{4.340530in}{2.471381in}}%
\pgfpathlineto{\pgfqpoint{4.340925in}{2.460619in}}%
\pgfpathlineto{\pgfqpoint{4.341714in}{2.466135in}}%
\pgfpathlineto{\pgfqpoint{4.342010in}{2.472892in}}%
\pgfpathlineto{\pgfqpoint{4.342405in}{2.453330in}}%
\pgfpathlineto{\pgfqpoint{4.342701in}{2.461329in}}%
\pgfpathlineto{\pgfqpoint{4.342997in}{2.450489in}}%
\pgfpathlineto{\pgfqpoint{4.343886in}{2.443760in}}%
\pgfpathlineto{\pgfqpoint{4.343491in}{2.450792in}}%
\pgfpathlineto{\pgfqpoint{4.344083in}{2.446893in}}%
\pgfpathlineto{\pgfqpoint{4.345070in}{2.462350in}}%
\pgfpathlineto{\pgfqpoint{4.345366in}{2.450170in}}%
\pgfpathlineto{\pgfqpoint{4.345564in}{2.444666in}}%
\pgfpathlineto{\pgfqpoint{4.346254in}{2.457001in}}%
\pgfpathlineto{\pgfqpoint{4.347044in}{2.456396in}}%
\pgfpathlineto{\pgfqpoint{4.347143in}{2.457082in}}%
\pgfpathlineto{\pgfqpoint{4.349511in}{2.497916in}}%
\pgfpathlineto{\pgfqpoint{4.349709in}{2.493682in}}%
\pgfpathlineto{\pgfqpoint{4.351683in}{2.430637in}}%
\pgfpathlineto{\pgfqpoint{4.352374in}{2.445726in}}%
\pgfpathlineto{\pgfqpoint{4.352670in}{2.449445in}}%
\pgfpathlineto{\pgfqpoint{4.352867in}{2.451905in}}%
\pgfpathlineto{\pgfqpoint{4.353163in}{2.437501in}}%
\pgfpathlineto{\pgfqpoint{4.354249in}{2.416526in}}%
\pgfpathlineto{\pgfqpoint{4.354545in}{2.419654in}}%
\pgfpathlineto{\pgfqpoint{4.354841in}{2.418557in}}%
\pgfpathlineto{\pgfqpoint{4.355137in}{2.426648in}}%
\pgfpathlineto{\pgfqpoint{4.356026in}{2.470442in}}%
\pgfpathlineto{\pgfqpoint{4.356519in}{2.447180in}}%
\pgfpathlineto{\pgfqpoint{4.358000in}{2.388585in}}%
\pgfpathlineto{\pgfqpoint{4.358098in}{2.390083in}}%
\pgfpathlineto{\pgfqpoint{4.359579in}{2.522488in}}%
\pgfpathlineto{\pgfqpoint{4.360664in}{2.670080in}}%
\pgfpathlineto{\pgfqpoint{4.361257in}{2.632540in}}%
\pgfpathlineto{\pgfqpoint{4.361651in}{2.593415in}}%
\pgfpathlineto{\pgfqpoint{4.362540in}{2.386515in}}%
\pgfpathlineto{\pgfqpoint{4.363428in}{2.426221in}}%
\pgfpathlineto{\pgfqpoint{4.364612in}{2.457357in}}%
\pgfpathlineto{\pgfqpoint{4.365698in}{2.506991in}}%
\pgfpathlineto{\pgfqpoint{4.366093in}{2.487801in}}%
\pgfpathlineto{\pgfqpoint{4.366586in}{2.442017in}}%
\pgfpathlineto{\pgfqpoint{4.367376in}{2.469187in}}%
\pgfpathlineto{\pgfqpoint{4.367573in}{2.469292in}}%
\pgfpathlineto{\pgfqpoint{4.368856in}{2.503932in}}%
\pgfpathlineto{\pgfqpoint{4.369449in}{2.481069in}}%
\pgfpathlineto{\pgfqpoint{4.370238in}{2.469868in}}%
\pgfpathlineto{\pgfqpoint{4.369745in}{2.481592in}}%
\pgfpathlineto{\pgfqpoint{4.370732in}{2.474624in}}%
\pgfpathlineto{\pgfqpoint{4.372607in}{2.516655in}}%
\pgfpathlineto{\pgfqpoint{4.373002in}{2.510267in}}%
\pgfpathlineto{\pgfqpoint{4.373199in}{2.508887in}}%
\pgfpathlineto{\pgfqpoint{4.373495in}{2.514445in}}%
\pgfpathlineto{\pgfqpoint{4.374088in}{2.535104in}}%
\pgfpathlineto{\pgfqpoint{4.374976in}{2.526694in}}%
\pgfpathlineto{\pgfqpoint{4.375371in}{2.531087in}}%
\pgfpathlineto{\pgfqpoint{4.376160in}{2.541946in}}%
\pgfpathlineto{\pgfqpoint{4.376950in}{2.538113in}}%
\pgfpathlineto{\pgfqpoint{4.377246in}{2.530299in}}%
\pgfpathlineto{\pgfqpoint{4.377739in}{2.539481in}}%
\pgfpathlineto{\pgfqpoint{4.378233in}{2.532434in}}%
\pgfpathlineto{\pgfqpoint{4.378332in}{2.532317in}}%
\pgfpathlineto{\pgfqpoint{4.378529in}{2.533730in}}%
\pgfpathlineto{\pgfqpoint{4.379615in}{2.541059in}}%
\pgfpathlineto{\pgfqpoint{4.379220in}{2.531410in}}%
\pgfpathlineto{\pgfqpoint{4.379812in}{2.538430in}}%
\pgfpathlineto{\pgfqpoint{4.380799in}{2.532129in}}%
\pgfpathlineto{\pgfqpoint{4.380404in}{2.540714in}}%
\pgfpathlineto{\pgfqpoint{4.380996in}{2.535953in}}%
\pgfpathlineto{\pgfqpoint{4.381885in}{2.542408in}}%
\pgfpathlineto{\pgfqpoint{4.381490in}{2.530047in}}%
\pgfpathlineto{\pgfqpoint{4.382181in}{2.536678in}}%
\pgfpathlineto{\pgfqpoint{4.383267in}{2.524504in}}%
\pgfpathlineto{\pgfqpoint{4.382674in}{2.541122in}}%
\pgfpathlineto{\pgfqpoint{4.383563in}{2.530567in}}%
\pgfpathlineto{\pgfqpoint{4.383661in}{2.532750in}}%
\pgfpathlineto{\pgfqpoint{4.384056in}{2.522204in}}%
\pgfpathlineto{\pgfqpoint{4.385734in}{2.498363in}}%
\pgfpathlineto{\pgfqpoint{4.385833in}{2.498530in}}%
\pgfpathlineto{\pgfqpoint{4.385931in}{2.499041in}}%
\pgfpathlineto{\pgfqpoint{4.386129in}{2.495627in}}%
\pgfpathlineto{\pgfqpoint{4.387511in}{2.474467in}}%
\pgfpathlineto{\pgfqpoint{4.387609in}{2.476005in}}%
\pgfpathlineto{\pgfqpoint{4.388201in}{2.468177in}}%
\pgfpathlineto{\pgfqpoint{4.388695in}{2.480901in}}%
\pgfpathlineto{\pgfqpoint{4.389090in}{2.456703in}}%
\pgfpathlineto{\pgfqpoint{4.390077in}{2.464357in}}%
\pgfpathlineto{\pgfqpoint{4.390274in}{2.461038in}}%
\pgfpathlineto{\pgfqpoint{4.391360in}{2.454211in}}%
\pgfpathlineto{\pgfqpoint{4.390866in}{2.472107in}}%
\pgfpathlineto{\pgfqpoint{4.391459in}{2.455557in}}%
\pgfpathlineto{\pgfqpoint{4.391557in}{2.455559in}}%
\pgfpathlineto{\pgfqpoint{4.391952in}{2.444122in}}%
\pgfpathlineto{\pgfqpoint{4.392347in}{2.460761in}}%
\pgfpathlineto{\pgfqpoint{4.392840in}{2.448029in}}%
\pgfpathlineto{\pgfqpoint{4.393235in}{2.461906in}}%
\pgfpathlineto{\pgfqpoint{4.393827in}{2.446208in}}%
\pgfpathlineto{\pgfqpoint{4.394716in}{2.444001in}}%
\pgfpathlineto{\pgfqpoint{4.394222in}{2.447280in}}%
\pgfpathlineto{\pgfqpoint{4.394814in}{2.445363in}}%
\pgfpathlineto{\pgfqpoint{4.396788in}{2.483555in}}%
\pgfpathlineto{\pgfqpoint{4.396887in}{2.482750in}}%
\pgfpathlineto{\pgfqpoint{4.397084in}{2.487101in}}%
\pgfpathlineto{\pgfqpoint{4.397380in}{2.503669in}}%
\pgfpathlineto{\pgfqpoint{4.397775in}{2.479715in}}%
\pgfpathlineto{\pgfqpoint{4.398269in}{2.496153in}}%
\pgfpathlineto{\pgfqpoint{4.399848in}{2.437035in}}%
\pgfpathlineto{\pgfqpoint{4.400243in}{2.453688in}}%
\pgfpathlineto{\pgfqpoint{4.400440in}{2.460252in}}%
\pgfpathlineto{\pgfqpoint{4.401131in}{2.447892in}}%
\pgfpathlineto{\pgfqpoint{4.402809in}{2.412293in}}%
\pgfpathlineto{\pgfqpoint{4.403006in}{2.421781in}}%
\pgfpathlineto{\pgfqpoint{4.403993in}{2.476813in}}%
\pgfpathlineto{\pgfqpoint{4.404388in}{2.457456in}}%
\pgfpathlineto{\pgfqpoint{4.405967in}{2.418481in}}%
\pgfpathlineto{\pgfqpoint{4.407250in}{2.493642in}}%
\pgfpathlineto{\pgfqpoint{4.408731in}{2.693233in}}%
\pgfpathlineto{\pgfqpoint{4.409224in}{2.653773in}}%
\pgfpathlineto{\pgfqpoint{4.409915in}{2.539634in}}%
\pgfpathlineto{\pgfqpoint{4.410507in}{2.411539in}}%
\pgfpathlineto{\pgfqpoint{4.411297in}{2.436645in}}%
\pgfpathlineto{\pgfqpoint{4.411692in}{2.441325in}}%
\pgfpathlineto{\pgfqpoint{4.412481in}{2.452696in}}%
\pgfpathlineto{\pgfqpoint{4.413567in}{2.518470in}}%
\pgfpathlineto{\pgfqpoint{4.414061in}{2.486685in}}%
\pgfpathlineto{\pgfqpoint{4.414751in}{2.442805in}}%
\pgfpathlineto{\pgfqpoint{4.415541in}{2.453653in}}%
\pgfpathlineto{\pgfqpoint{4.415640in}{2.454299in}}%
\pgfpathlineto{\pgfqpoint{4.415837in}{2.450554in}}%
\pgfpathlineto{\pgfqpoint{4.417910in}{2.403257in}}%
\pgfpathlineto{\pgfqpoint{4.418206in}{2.420871in}}%
\pgfpathlineto{\pgfqpoint{4.419785in}{2.509858in}}%
\pgfpathlineto{\pgfqpoint{4.419983in}{2.504488in}}%
\pgfpathlineto{\pgfqpoint{4.420180in}{2.498047in}}%
\pgfpathlineto{\pgfqpoint{4.420673in}{2.518235in}}%
\pgfpathlineto{\pgfqpoint{4.421167in}{2.500946in}}%
\pgfpathlineto{\pgfqpoint{4.421364in}{2.501056in}}%
\pgfpathlineto{\pgfqpoint{4.421463in}{2.499985in}}%
\pgfpathlineto{\pgfqpoint{4.421759in}{2.496857in}}%
\pgfpathlineto{\pgfqpoint{4.422351in}{2.499376in}}%
\pgfpathlineto{\pgfqpoint{4.423437in}{2.513346in}}%
\pgfpathlineto{\pgfqpoint{4.423634in}{2.507372in}}%
\pgfpathlineto{\pgfqpoint{4.423832in}{2.504516in}}%
\pgfpathlineto{\pgfqpoint{4.424227in}{2.518038in}}%
\pgfpathlineto{\pgfqpoint{4.425312in}{2.520416in}}%
\pgfpathlineto{\pgfqpoint{4.424917in}{2.513218in}}%
\pgfpathlineto{\pgfqpoint{4.425411in}{2.519732in}}%
\pgfpathlineto{\pgfqpoint{4.425806in}{2.506921in}}%
\pgfpathlineto{\pgfqpoint{4.426694in}{2.513976in}}%
\pgfpathlineto{\pgfqpoint{4.426990in}{2.513454in}}%
\pgfpathlineto{\pgfqpoint{4.427286in}{2.516020in}}%
\pgfpathlineto{\pgfqpoint{4.427681in}{2.521958in}}%
\pgfpathlineto{\pgfqpoint{4.428569in}{2.521286in}}%
\pgfpathlineto{\pgfqpoint{4.428668in}{2.520733in}}%
\pgfpathlineto{\pgfqpoint{4.429359in}{2.523116in}}%
\pgfpathlineto{\pgfqpoint{4.430149in}{2.537429in}}%
\pgfpathlineto{\pgfqpoint{4.430346in}{2.530451in}}%
\pgfpathlineto{\pgfqpoint{4.431333in}{2.511984in}}%
\pgfpathlineto{\pgfqpoint{4.431530in}{2.522405in}}%
\pgfpathlineto{\pgfqpoint{4.431629in}{2.525178in}}%
\pgfpathlineto{\pgfqpoint{4.431925in}{2.507989in}}%
\pgfpathlineto{\pgfqpoint{4.433307in}{2.489082in}}%
\pgfpathlineto{\pgfqpoint{4.433702in}{2.500269in}}%
\pgfpathlineto{\pgfqpoint{4.434195in}{2.487843in}}%
\pgfpathlineto{\pgfqpoint{4.434590in}{2.480025in}}%
\pgfpathlineto{\pgfqpoint{4.436169in}{2.452038in}}%
\pgfpathlineto{\pgfqpoint{4.436268in}{2.454681in}}%
\pgfpathlineto{\pgfqpoint{4.436465in}{2.460876in}}%
\pgfpathlineto{\pgfqpoint{4.437057in}{2.447114in}}%
\pgfpathlineto{\pgfqpoint{4.437156in}{2.448220in}}%
\pgfpathlineto{\pgfqpoint{4.438242in}{2.452893in}}%
\pgfpathlineto{\pgfqpoint{4.438341in}{2.450819in}}%
\pgfpathlineto{\pgfqpoint{4.439426in}{2.431407in}}%
\pgfpathlineto{\pgfqpoint{4.439624in}{2.435636in}}%
\pgfpathlineto{\pgfqpoint{4.439821in}{2.441006in}}%
\pgfpathlineto{\pgfqpoint{4.440315in}{2.424143in}}%
\pgfpathlineto{\pgfqpoint{4.440709in}{2.420601in}}%
\pgfpathlineto{\pgfqpoint{4.441104in}{2.428707in}}%
\pgfpathlineto{\pgfqpoint{4.441203in}{2.428906in}}%
\pgfpathlineto{\pgfqpoint{4.441301in}{2.427592in}}%
\pgfpathlineto{\pgfqpoint{4.442585in}{2.419137in}}%
\pgfpathlineto{\pgfqpoint{4.442683in}{2.420068in}}%
\pgfpathlineto{\pgfqpoint{4.444361in}{2.450623in}}%
\pgfpathlineto{\pgfqpoint{4.444559in}{2.449248in}}%
\pgfpathlineto{\pgfqpoint{4.444657in}{2.447956in}}%
\pgfpathlineto{\pgfqpoint{4.445052in}{2.454502in}}%
\pgfpathlineto{\pgfqpoint{4.445546in}{2.450246in}}%
\pgfpathlineto{\pgfqpoint{4.445743in}{2.452769in}}%
\pgfpathlineto{\pgfqpoint{4.446138in}{2.440900in}}%
\pgfpathlineto{\pgfqpoint{4.448013in}{2.389460in}}%
\pgfpathlineto{\pgfqpoint{4.448704in}{2.398810in}}%
\pgfpathlineto{\pgfqpoint{4.450777in}{2.360530in}}%
\pgfpathlineto{\pgfqpoint{4.451171in}{2.370016in}}%
\pgfpathlineto{\pgfqpoint{4.451862in}{2.415137in}}%
\pgfpathlineto{\pgfqpoint{4.452454in}{2.393242in}}%
\pgfpathlineto{\pgfqpoint{4.453836in}{2.360481in}}%
\pgfpathlineto{\pgfqpoint{4.453935in}{2.361191in}}%
\pgfpathlineto{\pgfqpoint{4.454823in}{2.402988in}}%
\pgfpathlineto{\pgfqpoint{4.456600in}{2.658135in}}%
\pgfpathlineto{\pgfqpoint{4.457587in}{2.588808in}}%
\pgfpathlineto{\pgfqpoint{4.458475in}{2.375126in}}%
\pgfpathlineto{\pgfqpoint{4.459462in}{2.391173in}}%
\pgfpathlineto{\pgfqpoint{4.461732in}{2.463783in}}%
\pgfpathlineto{\pgfqpoint{4.462127in}{2.444302in}}%
\pgfpathlineto{\pgfqpoint{4.462620in}{2.406189in}}%
\pgfpathlineto{\pgfqpoint{4.463213in}{2.436093in}}%
\pgfpathlineto{\pgfqpoint{4.464792in}{2.458859in}}%
\pgfpathlineto{\pgfqpoint{4.463805in}{2.435153in}}%
\pgfpathlineto{\pgfqpoint{4.464989in}{2.454110in}}%
\pgfpathlineto{\pgfqpoint{4.466075in}{2.427358in}}%
\pgfpathlineto{\pgfqpoint{4.466371in}{2.432698in}}%
\pgfpathlineto{\pgfqpoint{4.468641in}{2.466532in}}%
\pgfpathlineto{\pgfqpoint{4.469727in}{2.454973in}}%
\pgfpathlineto{\pgfqpoint{4.469924in}{2.463029in}}%
\pgfpathlineto{\pgfqpoint{4.470122in}{2.471194in}}%
\pgfpathlineto{\pgfqpoint{4.470911in}{2.463547in}}%
\pgfpathlineto{\pgfqpoint{4.471207in}{2.459512in}}%
\pgfpathlineto{\pgfqpoint{4.471799in}{2.464888in}}%
\pgfpathlineto{\pgfqpoint{4.472688in}{2.475319in}}%
\pgfpathlineto{\pgfqpoint{4.472194in}{2.464463in}}%
\pgfpathlineto{\pgfqpoint{4.472984in}{2.467419in}}%
\pgfpathlineto{\pgfqpoint{4.473675in}{2.458626in}}%
\pgfpathlineto{\pgfqpoint{4.473280in}{2.467926in}}%
\pgfpathlineto{\pgfqpoint{4.473872in}{2.463541in}}%
\pgfpathlineto{\pgfqpoint{4.474464in}{2.480800in}}%
\pgfpathlineto{\pgfqpoint{4.475057in}{2.467761in}}%
\pgfpathlineto{\pgfqpoint{4.475155in}{2.467517in}}%
\pgfpathlineto{\pgfqpoint{4.475353in}{2.469059in}}%
\pgfpathlineto{\pgfqpoint{4.475846in}{2.479292in}}%
\pgfpathlineto{\pgfqpoint{4.476142in}{2.469673in}}%
\pgfpathlineto{\pgfqpoint{4.476340in}{2.463583in}}%
\pgfpathlineto{\pgfqpoint{4.477228in}{2.470635in}}%
\pgfpathlineto{\pgfqpoint{4.477524in}{2.459885in}}%
\pgfpathlineto{\pgfqpoint{4.477721in}{2.453542in}}%
\pgfpathlineto{\pgfqpoint{4.478511in}{2.462259in}}%
\pgfpathlineto{\pgfqpoint{4.478610in}{2.463864in}}%
\pgfpathlineto{\pgfqpoint{4.478807in}{2.453227in}}%
\pgfpathlineto{\pgfqpoint{4.479004in}{2.439924in}}%
\pgfpathlineto{\pgfqpoint{4.479498in}{2.456123in}}%
\pgfpathlineto{\pgfqpoint{4.479991in}{2.445667in}}%
\pgfpathlineto{\pgfqpoint{4.481571in}{2.418969in}}%
\pgfpathlineto{\pgfqpoint{4.481867in}{2.428510in}}%
\pgfpathlineto{\pgfqpoint{4.481965in}{2.429615in}}%
\pgfpathlineto{\pgfqpoint{4.482163in}{2.423309in}}%
\pgfpathlineto{\pgfqpoint{4.482755in}{2.406944in}}%
\pgfpathlineto{\pgfqpoint{4.483545in}{2.409051in}}%
\pgfpathlineto{\pgfqpoint{4.484630in}{2.397181in}}%
\pgfpathlineto{\pgfqpoint{4.484236in}{2.416469in}}%
\pgfpathlineto{\pgfqpoint{4.484828in}{2.399754in}}%
\pgfpathlineto{\pgfqpoint{4.485025in}{2.403806in}}%
\pgfpathlineto{\pgfqpoint{4.485420in}{2.391190in}}%
\pgfpathlineto{\pgfqpoint{4.485716in}{2.395271in}}%
\pgfpathlineto{\pgfqpoint{4.485815in}{2.395333in}}%
\pgfpathlineto{\pgfqpoint{4.485913in}{2.394271in}}%
\pgfpathlineto{\pgfqpoint{4.486802in}{2.390979in}}%
\pgfpathlineto{\pgfqpoint{4.486407in}{2.400506in}}%
\pgfpathlineto{\pgfqpoint{4.486999in}{2.393094in}}%
\pgfpathlineto{\pgfqpoint{4.487690in}{2.402754in}}%
\pgfpathlineto{\pgfqpoint{4.488085in}{2.393997in}}%
\pgfpathlineto{\pgfqpoint{4.488282in}{2.390783in}}%
\pgfpathlineto{\pgfqpoint{4.488578in}{2.404217in}}%
\pgfpathlineto{\pgfqpoint{4.488776in}{2.408558in}}%
\pgfpathlineto{\pgfqpoint{4.489368in}{2.393015in}}%
\pgfpathlineto{\pgfqpoint{4.489664in}{2.390381in}}%
\pgfpathlineto{\pgfqpoint{4.489960in}{2.395926in}}%
\pgfpathlineto{\pgfqpoint{4.491539in}{2.435986in}}%
\pgfpathlineto{\pgfqpoint{4.492724in}{2.464947in}}%
\pgfpathlineto{\pgfqpoint{4.492921in}{2.455268in}}%
\pgfpathlineto{\pgfqpoint{4.493118in}{2.449512in}}%
\pgfpathlineto{\pgfqpoint{4.493513in}{2.458438in}}%
\pgfpathlineto{\pgfqpoint{4.494007in}{2.453035in}}%
\pgfpathlineto{\pgfqpoint{4.494204in}{2.457344in}}%
\pgfpathlineto{\pgfqpoint{4.494599in}{2.439242in}}%
\pgfpathlineto{\pgfqpoint{4.496474in}{2.385845in}}%
\pgfpathlineto{\pgfqpoint{4.496672in}{2.387608in}}%
\pgfpathlineto{\pgfqpoint{4.497757in}{2.344429in}}%
\pgfpathlineto{\pgfqpoint{4.497856in}{2.341143in}}%
\pgfpathlineto{\pgfqpoint{4.498152in}{2.360550in}}%
\pgfpathlineto{\pgfqpoint{4.499929in}{2.492542in}}%
\pgfpathlineto{\pgfqpoint{4.500323in}{2.461661in}}%
\pgfpathlineto{\pgfqpoint{4.501508in}{2.416259in}}%
\pgfpathlineto{\pgfqpoint{4.501804in}{2.418680in}}%
\pgfpathlineto{\pgfqpoint{4.501903in}{2.418585in}}%
\pgfpathlineto{\pgfqpoint{4.502001in}{2.418833in}}%
\pgfpathlineto{\pgfqpoint{4.502890in}{2.465195in}}%
\pgfpathlineto{\pgfqpoint{4.504568in}{2.708915in}}%
\pgfpathlineto{\pgfqpoint{4.505258in}{2.661759in}}%
\pgfpathlineto{\pgfqpoint{4.505555in}{2.620796in}}%
\pgfpathlineto{\pgfqpoint{4.506344in}{2.422098in}}%
\pgfpathlineto{\pgfqpoint{4.507232in}{2.449110in}}%
\pgfpathlineto{\pgfqpoint{4.507430in}{2.442074in}}%
\pgfpathlineto{\pgfqpoint{4.508022in}{2.462290in}}%
\pgfpathlineto{\pgfqpoint{4.509206in}{2.519557in}}%
\pgfpathlineto{\pgfqpoint{4.509404in}{2.527845in}}%
\pgfpathlineto{\pgfqpoint{4.509897in}{2.507761in}}%
\pgfpathlineto{\pgfqpoint{4.510588in}{2.457793in}}%
\pgfpathlineto{\pgfqpoint{4.511279in}{2.486337in}}%
\pgfpathlineto{\pgfqpoint{4.511476in}{2.481961in}}%
\pgfpathlineto{\pgfqpoint{4.512069in}{2.491211in}}%
\pgfpathlineto{\pgfqpoint{4.512562in}{2.508798in}}%
\pgfpathlineto{\pgfqpoint{4.513253in}{2.496585in}}%
\pgfpathlineto{\pgfqpoint{4.513747in}{2.470442in}}%
\pgfpathlineto{\pgfqpoint{4.514437in}{2.489500in}}%
\pgfpathlineto{\pgfqpoint{4.514734in}{2.487090in}}%
\pgfpathlineto{\pgfqpoint{4.514931in}{2.490804in}}%
\pgfpathlineto{\pgfqpoint{4.515918in}{2.508357in}}%
\pgfpathlineto{\pgfqpoint{4.516313in}{2.503840in}}%
\pgfpathlineto{\pgfqpoint{4.516510in}{2.507931in}}%
\pgfpathlineto{\pgfqpoint{4.516905in}{2.494167in}}%
\pgfpathlineto{\pgfqpoint{4.517004in}{2.491661in}}%
\pgfpathlineto{\pgfqpoint{4.517793in}{2.498598in}}%
\pgfpathlineto{\pgfqpoint{4.518089in}{2.501700in}}%
\pgfpathlineto{\pgfqpoint{4.518583in}{2.496852in}}%
\pgfpathlineto{\pgfqpoint{4.518780in}{2.497202in}}%
\pgfpathlineto{\pgfqpoint{4.518879in}{2.497242in}}%
\pgfpathlineto{\pgfqpoint{4.520261in}{2.484988in}}%
\pgfpathlineto{\pgfqpoint{4.519866in}{2.499070in}}%
\pgfpathlineto{\pgfqpoint{4.520458in}{2.488648in}}%
\pgfpathlineto{\pgfqpoint{4.520557in}{2.490995in}}%
\pgfpathlineto{\pgfqpoint{4.520853in}{2.479340in}}%
\pgfpathlineto{\pgfqpoint{4.521149in}{2.466386in}}%
\pgfpathlineto{\pgfqpoint{4.521544in}{2.484720in}}%
\pgfpathlineto{\pgfqpoint{4.521939in}{2.474715in}}%
\pgfpathlineto{\pgfqpoint{4.522432in}{2.483128in}}%
\pgfpathlineto{\pgfqpoint{4.523518in}{2.480873in}}%
\pgfpathlineto{\pgfqpoint{4.524702in}{2.459617in}}%
\pgfpathlineto{\pgfqpoint{4.524998in}{2.470938in}}%
\pgfpathlineto{\pgfqpoint{4.525196in}{2.475717in}}%
\pgfpathlineto{\pgfqpoint{4.525590in}{2.459323in}}%
\pgfpathlineto{\pgfqpoint{4.525886in}{2.464601in}}%
\pgfpathlineto{\pgfqpoint{4.526281in}{2.455278in}}%
\pgfpathlineto{\pgfqpoint{4.526775in}{2.466628in}}%
\pgfpathlineto{\pgfqpoint{4.527071in}{2.462990in}}%
\pgfpathlineto{\pgfqpoint{4.527170in}{2.462722in}}%
\pgfpathlineto{\pgfqpoint{4.527466in}{2.464993in}}%
\pgfpathlineto{\pgfqpoint{4.527762in}{2.468303in}}%
\pgfpathlineto{\pgfqpoint{4.528058in}{2.460741in}}%
\pgfpathlineto{\pgfqpoint{4.529341in}{2.433654in}}%
\pgfpathlineto{\pgfqpoint{4.529538in}{2.437501in}}%
\pgfpathlineto{\pgfqpoint{4.529736in}{2.441603in}}%
\pgfpathlineto{\pgfqpoint{4.530131in}{2.428459in}}%
\pgfpathlineto{\pgfqpoint{4.531315in}{2.411193in}}%
\pgfpathlineto{\pgfqpoint{4.531512in}{2.417370in}}%
\pgfpathlineto{\pgfqpoint{4.531710in}{2.425584in}}%
\pgfpathlineto{\pgfqpoint{4.532302in}{2.409814in}}%
\pgfpathlineto{\pgfqpoint{4.532598in}{2.416767in}}%
\pgfpathlineto{\pgfqpoint{4.532795in}{2.419790in}}%
\pgfpathlineto{\pgfqpoint{4.533092in}{2.405678in}}%
\pgfpathlineto{\pgfqpoint{4.533289in}{2.401062in}}%
\pgfpathlineto{\pgfqpoint{4.534079in}{2.407505in}}%
\pgfpathlineto{\pgfqpoint{4.535065in}{2.419376in}}%
\pgfpathlineto{\pgfqpoint{4.534572in}{2.401974in}}%
\pgfpathlineto{\pgfqpoint{4.535460in}{2.414127in}}%
\pgfpathlineto{\pgfqpoint{4.535756in}{2.411018in}}%
\pgfpathlineto{\pgfqpoint{4.536151in}{2.418691in}}%
\pgfpathlineto{\pgfqpoint{4.536941in}{2.430655in}}%
\pgfpathlineto{\pgfqpoint{4.536546in}{2.417643in}}%
\pgfpathlineto{\pgfqpoint{4.537928in}{2.425562in}}%
\pgfpathlineto{\pgfqpoint{4.538816in}{2.417587in}}%
\pgfpathlineto{\pgfqpoint{4.538520in}{2.426334in}}%
\pgfpathlineto{\pgfqpoint{4.539013in}{2.422659in}}%
\pgfpathlineto{\pgfqpoint{4.540494in}{2.450334in}}%
\pgfpathlineto{\pgfqpoint{4.540987in}{2.468438in}}%
\pgfpathlineto{\pgfqpoint{4.541974in}{2.464807in}}%
\pgfpathlineto{\pgfqpoint{4.543948in}{2.401297in}}%
\pgfpathlineto{\pgfqpoint{4.544244in}{2.411051in}}%
\pgfpathlineto{\pgfqpoint{4.545034in}{2.421789in}}%
\pgfpathlineto{\pgfqpoint{4.545330in}{2.410060in}}%
\pgfpathlineto{\pgfqpoint{4.547008in}{2.380226in}}%
\pgfpathlineto{\pgfqpoint{4.547107in}{2.380304in}}%
\pgfpathlineto{\pgfqpoint{4.547502in}{2.405386in}}%
\pgfpathlineto{\pgfqpoint{4.548192in}{2.436530in}}%
\pgfpathlineto{\pgfqpoint{4.548587in}{2.411184in}}%
\pgfpathlineto{\pgfqpoint{4.550068in}{2.379030in}}%
\pgfpathlineto{\pgfqpoint{4.550166in}{2.377523in}}%
\pgfpathlineto{\pgfqpoint{4.550364in}{2.384562in}}%
\pgfpathlineto{\pgfqpoint{4.552140in}{2.612048in}}%
\pgfpathlineto{\pgfqpoint{4.552831in}{2.667959in}}%
\pgfpathlineto{\pgfqpoint{4.553325in}{2.639517in}}%
\pgfpathlineto{\pgfqpoint{4.554016in}{2.552790in}}%
\pgfpathlineto{\pgfqpoint{4.554707in}{2.405442in}}%
\pgfpathlineto{\pgfqpoint{4.555397in}{2.432198in}}%
\pgfpathlineto{\pgfqpoint{4.555595in}{2.424351in}}%
\pgfpathlineto{\pgfqpoint{4.556088in}{2.441565in}}%
\pgfpathlineto{\pgfqpoint{4.556286in}{2.441518in}}%
\pgfpathlineto{\pgfqpoint{4.557470in}{2.503445in}}%
\pgfpathlineto{\pgfqpoint{4.557668in}{2.512004in}}%
\pgfpathlineto{\pgfqpoint{4.558260in}{2.485870in}}%
\pgfpathlineto{\pgfqpoint{4.559049in}{2.433314in}}%
\pgfpathlineto{\pgfqpoint{4.559839in}{2.453820in}}%
\pgfpathlineto{\pgfqpoint{4.560036in}{2.456923in}}%
\pgfpathlineto{\pgfqpoint{4.561023in}{2.475781in}}%
\pgfpathlineto{\pgfqpoint{4.561319in}{2.469160in}}%
\pgfpathlineto{\pgfqpoint{4.562603in}{2.440434in}}%
\pgfpathlineto{\pgfqpoint{4.562800in}{2.444155in}}%
\pgfpathlineto{\pgfqpoint{4.564182in}{2.467197in}}%
\pgfpathlineto{\pgfqpoint{4.564280in}{2.464428in}}%
\pgfpathlineto{\pgfqpoint{4.564675in}{2.445549in}}%
\pgfpathlineto{\pgfqpoint{4.565465in}{2.457810in}}%
\pgfpathlineto{\pgfqpoint{4.566649in}{2.472733in}}%
\pgfpathlineto{\pgfqpoint{4.566748in}{2.472409in}}%
\pgfpathlineto{\pgfqpoint{4.567044in}{2.471062in}}%
\pgfpathlineto{\pgfqpoint{4.567340in}{2.476177in}}%
\pgfpathlineto{\pgfqpoint{4.567537in}{2.478729in}}%
\pgfpathlineto{\pgfqpoint{4.567932in}{2.468762in}}%
\pgfpathlineto{\pgfqpoint{4.568327in}{2.474698in}}%
\pgfpathlineto{\pgfqpoint{4.570597in}{2.460808in}}%
\pgfpathlineto{\pgfqpoint{4.570696in}{2.462323in}}%
\pgfpathlineto{\pgfqpoint{4.571091in}{2.474101in}}%
\pgfpathlineto{\pgfqpoint{4.571485in}{2.455763in}}%
\pgfpathlineto{\pgfqpoint{4.574150in}{2.383725in}}%
\pgfpathlineto{\pgfqpoint{4.574446in}{2.399615in}}%
\pgfpathlineto{\pgfqpoint{4.576322in}{2.475610in}}%
\pgfpathlineto{\pgfqpoint{4.576618in}{2.470177in}}%
\pgfpathlineto{\pgfqpoint{4.579085in}{2.427776in}}%
\pgfpathlineto{\pgfqpoint{4.577111in}{2.475722in}}%
\pgfpathlineto{\pgfqpoint{4.579283in}{2.432425in}}%
\pgfpathlineto{\pgfqpoint{4.580467in}{2.443509in}}%
\pgfpathlineto{\pgfqpoint{4.580566in}{2.443351in}}%
\pgfpathlineto{\pgfqpoint{4.581750in}{2.426223in}}%
\pgfpathlineto{\pgfqpoint{4.582046in}{2.438164in}}%
\pgfpathlineto{\pgfqpoint{4.582145in}{2.440453in}}%
\pgfpathlineto{\pgfqpoint{4.582934in}{2.435657in}}%
\pgfpathlineto{\pgfqpoint{4.583231in}{2.428300in}}%
\pgfpathlineto{\pgfqpoint{4.583625in}{2.445836in}}%
\pgfpathlineto{\pgfqpoint{4.583823in}{2.449280in}}%
\pgfpathlineto{\pgfqpoint{4.584316in}{2.432355in}}%
\pgfpathlineto{\pgfqpoint{4.584612in}{2.440334in}}%
\pgfpathlineto{\pgfqpoint{4.584908in}{2.450419in}}%
\pgfpathlineto{\pgfqpoint{4.585402in}{2.429045in}}%
\pgfpathlineto{\pgfqpoint{4.585501in}{2.427864in}}%
\pgfpathlineto{\pgfqpoint{4.585698in}{2.435489in}}%
\pgfpathlineto{\pgfqpoint{4.585994in}{2.446762in}}%
\pgfpathlineto{\pgfqpoint{4.586685in}{2.430939in}}%
\pgfpathlineto{\pgfqpoint{4.586882in}{2.433374in}}%
\pgfpathlineto{\pgfqpoint{4.588462in}{2.461045in}}%
\pgfpathlineto{\pgfqpoint{4.588758in}{2.455796in}}%
\pgfpathlineto{\pgfqpoint{4.589843in}{2.470895in}}%
\pgfpathlineto{\pgfqpoint{4.590140in}{2.461129in}}%
\pgfpathlineto{\pgfqpoint{4.590534in}{2.465573in}}%
\pgfpathlineto{\pgfqpoint{4.591225in}{2.436955in}}%
\pgfpathlineto{\pgfqpoint{4.592212in}{2.400927in}}%
\pgfpathlineto{\pgfqpoint{4.592804in}{2.407497in}}%
\pgfpathlineto{\pgfqpoint{4.593199in}{2.427359in}}%
\pgfpathlineto{\pgfqpoint{4.593791in}{2.403557in}}%
\pgfpathlineto{\pgfqpoint{4.594186in}{2.399755in}}%
\pgfpathlineto{\pgfqpoint{4.595371in}{2.383057in}}%
\pgfpathlineto{\pgfqpoint{4.595568in}{2.387955in}}%
\pgfpathlineto{\pgfqpoint{4.596456in}{2.428801in}}%
\pgfpathlineto{\pgfqpoint{4.596950in}{2.405408in}}%
\pgfpathlineto{\pgfqpoint{4.598134in}{2.362461in}}%
\pgfpathlineto{\pgfqpoint{4.598332in}{2.366099in}}%
\pgfpathlineto{\pgfqpoint{4.599516in}{2.406257in}}%
\pgfpathlineto{\pgfqpoint{4.601194in}{2.636440in}}%
\pgfpathlineto{\pgfqpoint{4.601687in}{2.595227in}}%
\pgfpathlineto{\pgfqpoint{4.602576in}{2.449876in}}%
\pgfpathlineto{\pgfqpoint{4.602970in}{2.372252in}}%
\pgfpathlineto{\pgfqpoint{4.603760in}{2.409910in}}%
\pgfpathlineto{\pgfqpoint{4.604155in}{2.401663in}}%
\pgfpathlineto{\pgfqpoint{4.604451in}{2.411838in}}%
\pgfpathlineto{\pgfqpoint{4.606030in}{2.476717in}}%
\pgfpathlineto{\pgfqpoint{4.606227in}{2.467196in}}%
\pgfpathlineto{\pgfqpoint{4.607412in}{2.401802in}}%
\pgfpathlineto{\pgfqpoint{4.607807in}{2.416907in}}%
\pgfpathlineto{\pgfqpoint{4.609188in}{2.449119in}}%
\pgfpathlineto{\pgfqpoint{4.609287in}{2.449067in}}%
\pgfpathlineto{\pgfqpoint{4.609386in}{2.449499in}}%
\pgfpathlineto{\pgfqpoint{4.609583in}{2.446930in}}%
\pgfpathlineto{\pgfqpoint{4.610669in}{2.424182in}}%
\pgfpathlineto{\pgfqpoint{4.610866in}{2.431856in}}%
\pgfpathlineto{\pgfqpoint{4.611952in}{2.448622in}}%
\pgfpathlineto{\pgfqpoint{4.612149in}{2.448411in}}%
\pgfpathlineto{\pgfqpoint{4.612445in}{2.454793in}}%
\pgfpathlineto{\pgfqpoint{4.612840in}{2.466076in}}%
\pgfpathlineto{\pgfqpoint{4.613729in}{2.464640in}}%
\pgfpathlineto{\pgfqpoint{4.614025in}{2.463044in}}%
\pgfpathlineto{\pgfqpoint{4.614321in}{2.465865in}}%
\pgfpathlineto{\pgfqpoint{4.614814in}{2.482601in}}%
\pgfpathlineto{\pgfqpoint{4.615703in}{2.474750in}}%
\pgfpathlineto{\pgfqpoint{4.615801in}{2.474701in}}%
\pgfpathlineto{\pgfqpoint{4.617380in}{2.498943in}}%
\pgfpathlineto{\pgfqpoint{4.617578in}{2.491200in}}%
\pgfpathlineto{\pgfqpoint{4.617874in}{2.473911in}}%
\pgfpathlineto{\pgfqpoint{4.618565in}{2.493169in}}%
\pgfpathlineto{\pgfqpoint{4.618861in}{2.494739in}}%
\pgfpathlineto{\pgfqpoint{4.619256in}{2.489767in}}%
\pgfpathlineto{\pgfqpoint{4.620539in}{2.480267in}}%
\pgfpathlineto{\pgfqpoint{4.620045in}{2.491220in}}%
\pgfpathlineto{\pgfqpoint{4.620637in}{2.481710in}}%
\pgfpathlineto{\pgfqpoint{4.621131in}{2.488818in}}%
\pgfpathlineto{\pgfqpoint{4.621822in}{2.483751in}}%
\pgfpathlineto{\pgfqpoint{4.624585in}{2.457516in}}%
\pgfpathlineto{\pgfqpoint{4.624684in}{2.458691in}}%
\pgfpathlineto{\pgfqpoint{4.624882in}{2.463297in}}%
\pgfpathlineto{\pgfqpoint{4.625375in}{2.451461in}}%
\pgfpathlineto{\pgfqpoint{4.625572in}{2.452541in}}%
\pgfpathlineto{\pgfqpoint{4.625770in}{2.450297in}}%
\pgfpathlineto{\pgfqpoint{4.627546in}{2.426605in}}%
\pgfpathlineto{\pgfqpoint{4.627645in}{2.427311in}}%
\pgfpathlineto{\pgfqpoint{4.628040in}{2.447187in}}%
\pgfpathlineto{\pgfqpoint{4.628533in}{2.420248in}}%
\pgfpathlineto{\pgfqpoint{4.628928in}{2.434778in}}%
\pgfpathlineto{\pgfqpoint{4.630606in}{2.422071in}}%
\pgfpathlineto{\pgfqpoint{4.630803in}{2.426685in}}%
\pgfpathlineto{\pgfqpoint{4.631001in}{2.432098in}}%
\pgfpathlineto{\pgfqpoint{4.631396in}{2.419667in}}%
\pgfpathlineto{\pgfqpoint{4.631790in}{2.422223in}}%
\pgfpathlineto{\pgfqpoint{4.632876in}{2.416013in}}%
\pgfpathlineto{\pgfqpoint{4.632383in}{2.426450in}}%
\pgfpathlineto{\pgfqpoint{4.633074in}{2.419271in}}%
\pgfpathlineto{\pgfqpoint{4.634357in}{2.429245in}}%
\pgfpathlineto{\pgfqpoint{4.634455in}{2.429133in}}%
\pgfpathlineto{\pgfqpoint{4.634949in}{2.424050in}}%
\pgfpathlineto{\pgfqpoint{4.635146in}{2.427486in}}%
\pgfpathlineto{\pgfqpoint{4.635738in}{2.426370in}}%
\pgfpathlineto{\pgfqpoint{4.637120in}{2.458410in}}%
\pgfpathlineto{\pgfqpoint{4.637712in}{2.461234in}}%
\pgfpathlineto{\pgfqpoint{4.638502in}{2.470894in}}%
\pgfpathlineto{\pgfqpoint{4.638897in}{2.462247in}}%
\pgfpathlineto{\pgfqpoint{4.638995in}{2.462069in}}%
\pgfpathlineto{\pgfqpoint{4.639094in}{2.463423in}}%
\pgfpathlineto{\pgfqpoint{4.639193in}{2.464492in}}%
\pgfpathlineto{\pgfqpoint{4.639390in}{2.458797in}}%
\pgfpathlineto{\pgfqpoint{4.640871in}{2.415479in}}%
\pgfpathlineto{\pgfqpoint{4.640969in}{2.416751in}}%
\pgfpathlineto{\pgfqpoint{4.641463in}{2.437645in}}%
\pgfpathlineto{\pgfqpoint{4.642055in}{2.420022in}}%
\pgfpathlineto{\pgfqpoint{4.643634in}{2.377298in}}%
\pgfpathlineto{\pgfqpoint{4.643832in}{2.384402in}}%
\pgfpathlineto{\pgfqpoint{4.644720in}{2.433747in}}%
\pgfpathlineto{\pgfqpoint{4.645312in}{2.407133in}}%
\pgfpathlineto{\pgfqpoint{4.646497in}{2.375192in}}%
\pgfpathlineto{\pgfqpoint{4.646694in}{2.382251in}}%
\pgfpathlineto{\pgfqpoint{4.647385in}{2.378695in}}%
\pgfpathlineto{\pgfqpoint{4.648668in}{2.507193in}}%
\pgfpathlineto{\pgfqpoint{4.649556in}{2.574195in}}%
\pgfpathlineto{\pgfqpoint{4.649951in}{2.545287in}}%
\pgfpathlineto{\pgfqpoint{4.651234in}{2.345481in}}%
\pgfpathlineto{\pgfqpoint{4.652616in}{2.398361in}}%
\pgfpathlineto{\pgfqpoint{4.652912in}{2.391949in}}%
\pgfpathlineto{\pgfqpoint{4.653208in}{2.410951in}}%
\pgfpathlineto{\pgfqpoint{4.654393in}{2.451062in}}%
\pgfpathlineto{\pgfqpoint{4.653702in}{2.400643in}}%
\pgfpathlineto{\pgfqpoint{4.654689in}{2.431915in}}%
\pgfpathlineto{\pgfqpoint{4.655676in}{2.380500in}}%
\pgfpathlineto{\pgfqpoint{4.656070in}{2.393941in}}%
\pgfpathlineto{\pgfqpoint{4.657353in}{2.433831in}}%
\pgfpathlineto{\pgfqpoint{4.658242in}{2.424635in}}%
\pgfpathlineto{\pgfqpoint{4.658637in}{2.394937in}}%
\pgfpathlineto{\pgfqpoint{4.659426in}{2.416061in}}%
\pgfpathlineto{\pgfqpoint{4.661400in}{2.441910in}}%
\pgfpathlineto{\pgfqpoint{4.661499in}{2.439775in}}%
\pgfpathlineto{\pgfqpoint{4.661795in}{2.431764in}}%
\pgfpathlineto{\pgfqpoint{4.662091in}{2.453410in}}%
\pgfpathlineto{\pgfqpoint{4.662979in}{2.463745in}}%
\pgfpathlineto{\pgfqpoint{4.662585in}{2.448944in}}%
\pgfpathlineto{\pgfqpoint{4.663177in}{2.456005in}}%
\pgfpathlineto{\pgfqpoint{4.663275in}{2.452879in}}%
\pgfpathlineto{\pgfqpoint{4.663572in}{2.467657in}}%
\pgfpathlineto{\pgfqpoint{4.663868in}{2.481003in}}%
\pgfpathlineto{\pgfqpoint{4.664657in}{2.472446in}}%
\pgfpathlineto{\pgfqpoint{4.664756in}{2.471355in}}%
\pgfpathlineto{\pgfqpoint{4.664953in}{2.479922in}}%
\pgfpathlineto{\pgfqpoint{4.666039in}{2.498418in}}%
\pgfpathlineto{\pgfqpoint{4.666236in}{2.494841in}}%
\pgfpathlineto{\pgfqpoint{4.666434in}{2.489854in}}%
\pgfpathlineto{\pgfqpoint{4.666927in}{2.504460in}}%
\pgfpathlineto{\pgfqpoint{4.667421in}{2.490945in}}%
\pgfpathlineto{\pgfqpoint{4.668901in}{2.502659in}}%
\pgfpathlineto{\pgfqpoint{4.669099in}{2.501047in}}%
\pgfpathlineto{\pgfqpoint{4.670184in}{2.494902in}}%
\pgfpathlineto{\pgfqpoint{4.669691in}{2.506054in}}%
\pgfpathlineto{\pgfqpoint{4.670382in}{2.496666in}}%
\pgfpathlineto{\pgfqpoint{4.670875in}{2.504298in}}%
\pgfpathlineto{\pgfqpoint{4.670974in}{2.505951in}}%
\pgfpathlineto{\pgfqpoint{4.671467in}{2.496005in}}%
\pgfpathlineto{\pgfqpoint{4.671862in}{2.484807in}}%
\pgfpathlineto{\pgfqpoint{4.672948in}{2.470696in}}%
\pgfpathlineto{\pgfqpoint{4.672454in}{2.485613in}}%
\pgfpathlineto{\pgfqpoint{4.673047in}{2.473373in}}%
\pgfpathlineto{\pgfqpoint{4.673244in}{2.479372in}}%
\pgfpathlineto{\pgfqpoint{4.673639in}{2.466418in}}%
\pgfpathlineto{\pgfqpoint{4.673935in}{2.468635in}}%
\pgfpathlineto{\pgfqpoint{4.674330in}{2.459190in}}%
\pgfpathlineto{\pgfqpoint{4.674725in}{2.469081in}}%
\pgfpathlineto{\pgfqpoint{4.675119in}{2.463719in}}%
\pgfpathlineto{\pgfqpoint{4.676008in}{2.467197in}}%
\pgfpathlineto{\pgfqpoint{4.675613in}{2.461738in}}%
\pgfpathlineto{\pgfqpoint{4.676106in}{2.464310in}}%
\pgfpathlineto{\pgfqpoint{4.676402in}{2.455187in}}%
\pgfpathlineto{\pgfqpoint{4.676896in}{2.472832in}}%
\pgfpathlineto{\pgfqpoint{4.677192in}{2.482671in}}%
\pgfpathlineto{\pgfqpoint{4.677685in}{2.467763in}}%
\pgfpathlineto{\pgfqpoint{4.678080in}{2.477478in}}%
\pgfpathlineto{\pgfqpoint{4.678376in}{2.477280in}}%
\pgfpathlineto{\pgfqpoint{4.678475in}{2.478023in}}%
\pgfpathlineto{\pgfqpoint{4.679067in}{2.491569in}}%
\pgfpathlineto{\pgfqpoint{4.679758in}{2.483281in}}%
\pgfpathlineto{\pgfqpoint{4.680844in}{2.469445in}}%
\pgfpathlineto{\pgfqpoint{4.680449in}{2.489128in}}%
\pgfpathlineto{\pgfqpoint{4.681041in}{2.474676in}}%
\pgfpathlineto{\pgfqpoint{4.682423in}{2.492979in}}%
\pgfpathlineto{\pgfqpoint{4.686470in}{2.565359in}}%
\pgfpathlineto{\pgfqpoint{4.682917in}{2.488816in}}%
\pgfpathlineto{\pgfqpoint{4.686667in}{2.563319in}}%
\pgfpathlineto{\pgfqpoint{4.687555in}{2.542943in}}%
\pgfpathlineto{\pgfqpoint{4.688838in}{2.513822in}}%
\pgfpathlineto{\pgfqpoint{4.689135in}{2.521734in}}%
\pgfpathlineto{\pgfqpoint{4.689233in}{2.523916in}}%
\pgfpathlineto{\pgfqpoint{4.689727in}{2.513633in}}%
\pgfpathlineto{\pgfqpoint{4.689924in}{2.516178in}}%
\pgfpathlineto{\pgfqpoint{4.690220in}{2.505492in}}%
\pgfpathlineto{\pgfqpoint{4.691898in}{2.453146in}}%
\pgfpathlineto{\pgfqpoint{4.693083in}{2.534078in}}%
\pgfpathlineto{\pgfqpoint{4.693773in}{2.495359in}}%
\pgfpathlineto{\pgfqpoint{4.694168in}{2.485172in}}%
\pgfpathlineto{\pgfqpoint{4.694958in}{2.490622in}}%
\pgfpathlineto{\pgfqpoint{4.696734in}{2.658468in}}%
\pgfpathlineto{\pgfqpoint{4.697623in}{2.764005in}}%
\pgfpathlineto{\pgfqpoint{4.698116in}{2.734830in}}%
\pgfpathlineto{\pgfqpoint{4.698807in}{2.659159in}}%
\pgfpathlineto{\pgfqpoint{4.699399in}{2.491039in}}%
\pgfpathlineto{\pgfqpoint{4.700288in}{2.530424in}}%
\pgfpathlineto{\pgfqpoint{4.700682in}{2.522559in}}%
\pgfpathlineto{\pgfqpoint{4.700978in}{2.531927in}}%
\pgfpathlineto{\pgfqpoint{4.702656in}{2.606242in}}%
\pgfpathlineto{\pgfqpoint{4.702952in}{2.592558in}}%
\pgfpathlineto{\pgfqpoint{4.703939in}{2.545928in}}%
\pgfpathlineto{\pgfqpoint{4.704334in}{2.559133in}}%
\pgfpathlineto{\pgfqpoint{4.706012in}{2.604978in}}%
\pgfpathlineto{\pgfqpoint{4.706111in}{2.601179in}}%
\pgfpathlineto{\pgfqpoint{4.707196in}{2.567450in}}%
\pgfpathlineto{\pgfqpoint{4.707394in}{2.576504in}}%
\pgfpathlineto{\pgfqpoint{4.708183in}{2.585933in}}%
\pgfpathlineto{\pgfqpoint{4.708480in}{2.576426in}}%
\pgfpathlineto{\pgfqpoint{4.708578in}{2.575152in}}%
\pgfpathlineto{\pgfqpoint{4.708776in}{2.582411in}}%
\pgfpathlineto{\pgfqpoint{4.709072in}{2.590556in}}%
\pgfpathlineto{\pgfqpoint{4.709763in}{2.578537in}}%
\pgfpathlineto{\pgfqpoint{4.709861in}{2.578579in}}%
\pgfpathlineto{\pgfqpoint{4.711144in}{2.593057in}}%
\pgfpathlineto{\pgfqpoint{4.710552in}{2.574589in}}%
\pgfpathlineto{\pgfqpoint{4.711441in}{2.588309in}}%
\pgfpathlineto{\pgfqpoint{4.711835in}{2.584836in}}%
\pgfpathlineto{\pgfqpoint{4.712131in}{2.592534in}}%
\pgfpathlineto{\pgfqpoint{4.713217in}{2.610687in}}%
\pgfpathlineto{\pgfqpoint{4.712724in}{2.586388in}}%
\pgfpathlineto{\pgfqpoint{4.713414in}{2.606535in}}%
\pgfpathlineto{\pgfqpoint{4.714204in}{2.593129in}}%
\pgfpathlineto{\pgfqpoint{4.714698in}{2.596570in}}%
\pgfpathlineto{\pgfqpoint{4.716277in}{2.578216in}}%
\pgfpathlineto{\pgfqpoint{4.716375in}{2.580852in}}%
\pgfpathlineto{\pgfqpoint{4.716770in}{2.599055in}}%
\pgfpathlineto{\pgfqpoint{4.717165in}{2.575056in}}%
\pgfpathlineto{\pgfqpoint{4.717560in}{2.590258in}}%
\pgfpathlineto{\pgfqpoint{4.717757in}{2.585052in}}%
\pgfpathlineto{\pgfqpoint{4.718053in}{2.571689in}}%
\pgfpathlineto{\pgfqpoint{4.719040in}{2.573791in}}%
\pgfpathlineto{\pgfqpoint{4.720718in}{2.540590in}}%
\pgfpathlineto{\pgfqpoint{4.721014in}{2.546011in}}%
\pgfpathlineto{\pgfqpoint{4.721212in}{2.541066in}}%
\pgfpathlineto{\pgfqpoint{4.724271in}{2.395672in}}%
\pgfpathlineto{\pgfqpoint{4.724370in}{2.396060in}}%
\pgfpathlineto{\pgfqpoint{4.724765in}{2.403476in}}%
\pgfpathlineto{\pgfqpoint{4.726443in}{2.456060in}}%
\pgfpathlineto{\pgfqpoint{4.726541in}{2.455387in}}%
\pgfpathlineto{\pgfqpoint{4.727331in}{2.457096in}}%
\pgfpathlineto{\pgfqpoint{4.728022in}{2.446903in}}%
\pgfpathlineto{\pgfqpoint{4.728713in}{2.423654in}}%
\pgfpathlineto{\pgfqpoint{4.730687in}{2.424642in}}%
\pgfpathlineto{\pgfqpoint{4.730884in}{2.426920in}}%
\pgfpathlineto{\pgfqpoint{4.731180in}{2.433212in}}%
\pgfpathlineto{\pgfqpoint{4.731575in}{2.417340in}}%
\pgfpathlineto{\pgfqpoint{4.731674in}{2.417132in}}%
\pgfpathlineto{\pgfqpoint{4.733549in}{2.458096in}}%
\pgfpathlineto{\pgfqpoint{4.733648in}{2.457033in}}%
\pgfpathlineto{\pgfqpoint{4.734536in}{2.462675in}}%
\pgfpathlineto{\pgfqpoint{4.735030in}{2.445469in}}%
\pgfpathlineto{\pgfqpoint{4.735819in}{2.418044in}}%
\pgfpathlineto{\pgfqpoint{4.736806in}{2.392384in}}%
\pgfpathlineto{\pgfqpoint{4.737300in}{2.398232in}}%
\pgfpathlineto{\pgfqpoint{4.737497in}{2.401476in}}%
\pgfpathlineto{\pgfqpoint{4.737892in}{2.391486in}}%
\pgfpathlineto{\pgfqpoint{4.739767in}{2.354313in}}%
\pgfpathlineto{\pgfqpoint{4.739866in}{2.355191in}}%
\pgfpathlineto{\pgfqpoint{4.740359in}{2.379813in}}%
\pgfpathlineto{\pgfqpoint{4.740754in}{2.420466in}}%
\pgfpathlineto{\pgfqpoint{4.741445in}{2.377215in}}%
\pgfpathlineto{\pgfqpoint{4.743024in}{2.344976in}}%
\pgfpathlineto{\pgfqpoint{4.743123in}{2.344727in}}%
\pgfpathlineto{\pgfqpoint{4.743222in}{2.346995in}}%
\pgfpathlineto{\pgfqpoint{4.744307in}{2.433950in}}%
\pgfpathlineto{\pgfqpoint{4.745294in}{2.592659in}}%
\pgfpathlineto{\pgfqpoint{4.746084in}{2.566971in}}%
\pgfpathlineto{\pgfqpoint{4.746873in}{2.436452in}}%
\pgfpathlineto{\pgfqpoint{4.747367in}{2.302339in}}%
\pgfpathlineto{\pgfqpoint{4.748255in}{2.332519in}}%
\pgfpathlineto{\pgfqpoint{4.748453in}{2.326297in}}%
\pgfpathlineto{\pgfqpoint{4.749045in}{2.342094in}}%
\pgfpathlineto{\pgfqpoint{4.750032in}{2.386355in}}%
\pgfpathlineto{\pgfqpoint{4.750328in}{2.415112in}}%
\pgfpathlineto{\pgfqpoint{4.751117in}{2.387940in}}%
\pgfpathlineto{\pgfqpoint{4.751710in}{2.356058in}}%
\pgfpathlineto{\pgfqpoint{4.752203in}{2.380727in}}%
\pgfpathlineto{\pgfqpoint{4.753782in}{2.417223in}}%
\pgfpathlineto{\pgfqpoint{4.753881in}{2.418504in}}%
\pgfpathlineto{\pgfqpoint{4.754177in}{2.411302in}}%
\pgfpathlineto{\pgfqpoint{4.754572in}{2.387707in}}%
\pgfpathlineto{\pgfqpoint{4.755362in}{2.400112in}}%
\pgfpathlineto{\pgfqpoint{4.757039in}{2.428089in}}%
\pgfpathlineto{\pgfqpoint{4.757138in}{2.427931in}}%
\pgfpathlineto{\pgfqpoint{4.757632in}{2.415247in}}%
\pgfpathlineto{\pgfqpoint{4.758520in}{2.419843in}}%
\pgfpathlineto{\pgfqpoint{4.759310in}{2.433244in}}%
\pgfpathlineto{\pgfqpoint{4.760000in}{2.428761in}}%
\pgfpathlineto{\pgfqpoint{4.761185in}{2.444033in}}%
\pgfpathlineto{\pgfqpoint{4.761481in}{2.437258in}}%
\pgfpathlineto{\pgfqpoint{4.761777in}{2.429961in}}%
\pgfpathlineto{\pgfqpoint{4.762567in}{2.437954in}}%
\pgfpathlineto{\pgfqpoint{4.762764in}{2.441766in}}%
\pgfpathlineto{\pgfqpoint{4.763159in}{2.425016in}}%
\pgfpathlineto{\pgfqpoint{4.763652in}{2.438362in}}%
\pgfpathlineto{\pgfqpoint{4.764047in}{2.427184in}}%
\pgfpathlineto{\pgfqpoint{4.764935in}{2.430406in}}%
\pgfpathlineto{\pgfqpoint{4.765330in}{2.428326in}}%
\pgfpathlineto{\pgfqpoint{4.765725in}{2.432634in}}%
\pgfpathlineto{\pgfqpoint{4.766811in}{2.439352in}}%
\pgfpathlineto{\pgfqpoint{4.767205in}{2.434996in}}%
\pgfpathlineto{\pgfqpoint{4.767896in}{2.430832in}}%
\pgfpathlineto{\pgfqpoint{4.768192in}{2.434642in}}%
\pgfpathlineto{\pgfqpoint{4.768291in}{2.435702in}}%
\pgfpathlineto{\pgfqpoint{4.768686in}{2.430128in}}%
\pgfpathlineto{\pgfqpoint{4.770561in}{2.405460in}}%
\pgfpathlineto{\pgfqpoint{4.769179in}{2.431017in}}%
\pgfpathlineto{\pgfqpoint{4.770759in}{2.408267in}}%
\pgfpathlineto{\pgfqpoint{4.770857in}{2.410068in}}%
\pgfpathlineto{\pgfqpoint{4.771153in}{2.399074in}}%
\pgfpathlineto{\pgfqpoint{4.772239in}{2.372847in}}%
\pgfpathlineto{\pgfqpoint{4.771746in}{2.399547in}}%
\pgfpathlineto{\pgfqpoint{4.772436in}{2.380012in}}%
\pgfpathlineto{\pgfqpoint{4.772535in}{2.381736in}}%
\pgfpathlineto{\pgfqpoint{4.773226in}{2.374902in}}%
\pgfpathlineto{\pgfqpoint{4.774509in}{2.365940in}}%
\pgfpathlineto{\pgfqpoint{4.773917in}{2.380764in}}%
\pgfpathlineto{\pgfqpoint{4.774608in}{2.366641in}}%
\pgfpathlineto{\pgfqpoint{4.774904in}{2.374082in}}%
\pgfpathlineto{\pgfqpoint{4.775299in}{2.365152in}}%
\pgfpathlineto{\pgfqpoint{4.775694in}{2.368752in}}%
\pgfpathlineto{\pgfqpoint{4.776681in}{2.355526in}}%
\pgfpathlineto{\pgfqpoint{4.776187in}{2.372112in}}%
\pgfpathlineto{\pgfqpoint{4.777075in}{2.365461in}}%
\pgfpathlineto{\pgfqpoint{4.777174in}{2.365599in}}%
\pgfpathlineto{\pgfqpoint{4.777273in}{2.364228in}}%
\pgfpathlineto{\pgfqpoint{4.777470in}{2.362195in}}%
\pgfpathlineto{\pgfqpoint{4.777865in}{2.367054in}}%
\pgfpathlineto{\pgfqpoint{4.778260in}{2.365984in}}%
\pgfpathlineto{\pgfqpoint{4.778654in}{2.381647in}}%
\pgfpathlineto{\pgfqpoint{4.779444in}{2.371921in}}%
\pgfpathlineto{\pgfqpoint{4.779641in}{2.371181in}}%
\pgfpathlineto{\pgfqpoint{4.779740in}{2.373983in}}%
\pgfpathlineto{\pgfqpoint{4.781319in}{2.410064in}}%
\pgfpathlineto{\pgfqpoint{4.781615in}{2.409700in}}%
\pgfpathlineto{\pgfqpoint{4.781714in}{2.410633in}}%
\pgfpathlineto{\pgfqpoint{4.782306in}{2.423922in}}%
\pgfpathlineto{\pgfqpoint{4.782997in}{2.416288in}}%
\pgfpathlineto{\pgfqpoint{4.784873in}{2.365124in}}%
\pgfpathlineto{\pgfqpoint{4.784971in}{2.363461in}}%
\pgfpathlineto{\pgfqpoint{4.785662in}{2.369325in}}%
\pgfpathlineto{\pgfqpoint{4.785860in}{2.370624in}}%
\pgfpathlineto{\pgfqpoint{4.786057in}{2.365548in}}%
\pgfpathlineto{\pgfqpoint{4.787340in}{2.334633in}}%
\pgfpathlineto{\pgfqpoint{4.787636in}{2.337408in}}%
\pgfpathlineto{\pgfqpoint{4.787735in}{2.337806in}}%
\pgfpathlineto{\pgfqpoint{4.787833in}{2.336400in}}%
\pgfpathlineto{\pgfqpoint{4.788130in}{2.329762in}}%
\pgfpathlineto{\pgfqpoint{4.788327in}{2.345273in}}%
\pgfpathlineto{\pgfqpoint{4.788820in}{2.397058in}}%
\pgfpathlineto{\pgfqpoint{4.789511in}{2.372283in}}%
\pgfpathlineto{\pgfqpoint{4.791091in}{2.338534in}}%
\pgfpathlineto{\pgfqpoint{4.792275in}{2.382569in}}%
\pgfpathlineto{\pgfqpoint{4.793657in}{2.554864in}}%
\pgfpathlineto{\pgfqpoint{4.794348in}{2.518454in}}%
\pgfpathlineto{\pgfqpoint{4.794545in}{2.517200in}}%
\pgfpathlineto{\pgfqpoint{4.794940in}{2.464897in}}%
\pgfpathlineto{\pgfqpoint{4.795433in}{2.342828in}}%
\pgfpathlineto{\pgfqpoint{4.796223in}{2.399527in}}%
\pgfpathlineto{\pgfqpoint{4.796716in}{2.394922in}}%
\pgfpathlineto{\pgfqpoint{4.797111in}{2.419015in}}%
\pgfpathlineto{\pgfqpoint{4.798592in}{2.477956in}}%
\pgfpathlineto{\pgfqpoint{4.798789in}{2.469221in}}%
\pgfpathlineto{\pgfqpoint{4.799776in}{2.411537in}}%
\pgfpathlineto{\pgfqpoint{4.800270in}{2.433777in}}%
\pgfpathlineto{\pgfqpoint{4.801750in}{2.471323in}}%
\pgfpathlineto{\pgfqpoint{4.801947in}{2.468834in}}%
\pgfpathlineto{\pgfqpoint{4.802934in}{2.448630in}}%
\pgfpathlineto{\pgfqpoint{4.803527in}{2.456973in}}%
\pgfpathlineto{\pgfqpoint{4.803823in}{2.458909in}}%
\pgfpathlineto{\pgfqpoint{4.804119in}{2.453853in}}%
\pgfpathlineto{\pgfqpoint{4.804218in}{2.452771in}}%
\pgfpathlineto{\pgfqpoint{4.804415in}{2.458635in}}%
\pgfpathlineto{\pgfqpoint{4.805402in}{2.467545in}}%
\pgfpathlineto{\pgfqpoint{4.805599in}{2.463616in}}%
\pgfpathlineto{\pgfqpoint{4.805698in}{2.462623in}}%
\pgfpathlineto{\pgfqpoint{4.805895in}{2.468012in}}%
\pgfpathlineto{\pgfqpoint{4.806192in}{2.477102in}}%
\pgfpathlineto{\pgfqpoint{4.806586in}{2.464456in}}%
\pgfpathlineto{\pgfqpoint{4.807178in}{2.476428in}}%
\pgfpathlineto{\pgfqpoint{4.807475in}{2.475766in}}%
\pgfpathlineto{\pgfqpoint{4.807573in}{2.476493in}}%
\pgfpathlineto{\pgfqpoint{4.808659in}{2.484573in}}%
\pgfpathlineto{\pgfqpoint{4.808955in}{2.482097in}}%
\pgfpathlineto{\pgfqpoint{4.809350in}{2.478376in}}%
\pgfpathlineto{\pgfqpoint{4.809547in}{2.473992in}}%
\pgfpathlineto{\pgfqpoint{4.809942in}{2.489060in}}%
\pgfpathlineto{\pgfqpoint{4.810041in}{2.490912in}}%
\pgfpathlineto{\pgfqpoint{4.810436in}{2.482060in}}%
\pgfpathlineto{\pgfqpoint{4.810830in}{2.484914in}}%
\pgfpathlineto{\pgfqpoint{4.811126in}{2.478851in}}%
\pgfpathlineto{\pgfqpoint{4.811521in}{2.492934in}}%
\pgfpathlineto{\pgfqpoint{4.812804in}{2.500016in}}%
\pgfpathlineto{\pgfqpoint{4.812410in}{2.489651in}}%
\pgfpathlineto{\pgfqpoint{4.812903in}{2.499704in}}%
\pgfpathlineto{\pgfqpoint{4.813298in}{2.486509in}}%
\pgfpathlineto{\pgfqpoint{4.813693in}{2.506446in}}%
\pgfpathlineto{\pgfqpoint{4.813791in}{2.506544in}}%
\pgfpathlineto{\pgfqpoint{4.815074in}{2.494856in}}%
\pgfpathlineto{\pgfqpoint{4.815173in}{2.495289in}}%
\pgfpathlineto{\pgfqpoint{4.816555in}{2.505449in}}%
\pgfpathlineto{\pgfqpoint{4.816654in}{2.503624in}}%
\pgfpathlineto{\pgfqpoint{4.817147in}{2.475211in}}%
\pgfpathlineto{\pgfqpoint{4.818233in}{2.475896in}}%
\pgfpathlineto{\pgfqpoint{4.819713in}{2.463909in}}%
\pgfpathlineto{\pgfqpoint{4.818726in}{2.477769in}}%
\pgfpathlineto{\pgfqpoint{4.819812in}{2.464252in}}%
\pgfpathlineto{\pgfqpoint{4.819911in}{2.464750in}}%
\pgfpathlineto{\pgfqpoint{4.820108in}{2.461944in}}%
\pgfpathlineto{\pgfqpoint{4.821194in}{2.443822in}}%
\pgfpathlineto{\pgfqpoint{4.821391in}{2.449846in}}%
\pgfpathlineto{\pgfqpoint{4.821490in}{2.452212in}}%
\pgfpathlineto{\pgfqpoint{4.821885in}{2.437472in}}%
\pgfpathlineto{\pgfqpoint{4.822279in}{2.449409in}}%
\pgfpathlineto{\pgfqpoint{4.822576in}{2.433979in}}%
\pgfpathlineto{\pgfqpoint{4.823365in}{2.447335in}}%
\pgfpathlineto{\pgfqpoint{4.823563in}{2.449769in}}%
\pgfpathlineto{\pgfqpoint{4.823957in}{2.446488in}}%
\pgfpathlineto{\pgfqpoint{4.824352in}{2.446973in}}%
\pgfpathlineto{\pgfqpoint{4.825339in}{2.441979in}}%
\pgfpathlineto{\pgfqpoint{4.825635in}{2.443789in}}%
\pgfpathlineto{\pgfqpoint{4.826129in}{2.449731in}}%
\pgfpathlineto{\pgfqpoint{4.826622in}{2.443115in}}%
\pgfpathlineto{\pgfqpoint{4.827313in}{2.441081in}}%
\pgfpathlineto{\pgfqpoint{4.827510in}{2.442557in}}%
\pgfpathlineto{\pgfqpoint{4.830175in}{2.498036in}}%
\pgfpathlineto{\pgfqpoint{4.830373in}{2.495098in}}%
\pgfpathlineto{\pgfqpoint{4.832544in}{2.439289in}}%
\pgfpathlineto{\pgfqpoint{4.833432in}{2.455712in}}%
\pgfpathlineto{\pgfqpoint{4.833531in}{2.457896in}}%
\pgfpathlineto{\pgfqpoint{4.833926in}{2.444344in}}%
\pgfpathlineto{\pgfqpoint{4.835702in}{2.406260in}}%
\pgfpathlineto{\pgfqpoint{4.835900in}{2.410190in}}%
\pgfpathlineto{\pgfqpoint{4.836887in}{2.471921in}}%
\pgfpathlineto{\pgfqpoint{4.837380in}{2.438064in}}%
\pgfpathlineto{\pgfqpoint{4.838960in}{2.406548in}}%
\pgfpathlineto{\pgfqpoint{4.840045in}{2.467200in}}%
\pgfpathlineto{\pgfqpoint{4.841427in}{2.680000in}}%
\pgfpathlineto{\pgfqpoint{4.842217in}{2.645802in}}%
\pgfpathlineto{\pgfqpoint{4.842710in}{2.570560in}}%
\pgfpathlineto{\pgfqpoint{4.843302in}{2.407706in}}%
\pgfpathlineto{\pgfqpoint{4.844191in}{2.429127in}}%
\pgfpathlineto{\pgfqpoint{4.844289in}{2.427168in}}%
\pgfpathlineto{\pgfqpoint{4.844585in}{2.440377in}}%
\pgfpathlineto{\pgfqpoint{4.846658in}{2.517530in}}%
\pgfpathlineto{\pgfqpoint{4.846855in}{2.502211in}}%
\pgfpathlineto{\pgfqpoint{4.847645in}{2.462148in}}%
\pgfpathlineto{\pgfqpoint{4.848237in}{2.464649in}}%
\pgfpathlineto{\pgfqpoint{4.848336in}{2.464530in}}%
\pgfpathlineto{\pgfqpoint{4.849520in}{2.507852in}}%
\pgfpathlineto{\pgfqpoint{4.850211in}{2.500702in}}%
\pgfpathlineto{\pgfqpoint{4.850606in}{2.471365in}}%
\pgfpathlineto{\pgfqpoint{4.851494in}{2.479883in}}%
\pgfpathlineto{\pgfqpoint{4.852580in}{2.484022in}}%
\pgfpathlineto{\pgfqpoint{4.852087in}{2.476504in}}%
\pgfpathlineto{\pgfqpoint{4.852679in}{2.483714in}}%
\pgfpathlineto{\pgfqpoint{4.854357in}{2.467575in}}%
\pgfpathlineto{\pgfqpoint{4.854653in}{2.471149in}}%
\pgfpathlineto{\pgfqpoint{4.854949in}{2.476785in}}%
\pgfpathlineto{\pgfqpoint{4.855837in}{2.473562in}}%
\pgfpathlineto{\pgfqpoint{4.856034in}{2.471323in}}%
\pgfpathlineto{\pgfqpoint{4.856429in}{2.479705in}}%
\pgfpathlineto{\pgfqpoint{4.856627in}{2.476551in}}%
\pgfpathlineto{\pgfqpoint{4.858008in}{2.460400in}}%
\pgfpathlineto{\pgfqpoint{4.859390in}{2.443718in}}%
\pgfpathlineto{\pgfqpoint{4.859489in}{2.445715in}}%
\pgfpathlineto{\pgfqpoint{4.859785in}{2.456379in}}%
\pgfpathlineto{\pgfqpoint{4.860180in}{2.434146in}}%
\pgfpathlineto{\pgfqpoint{4.861562in}{2.423998in}}%
\pgfpathlineto{\pgfqpoint{4.863042in}{2.406726in}}%
\pgfpathlineto{\pgfqpoint{4.863141in}{2.407464in}}%
\pgfpathlineto{\pgfqpoint{4.863536in}{2.416563in}}%
\pgfpathlineto{\pgfqpoint{4.863930in}{2.397448in}}%
\pgfpathlineto{\pgfqpoint{4.866398in}{2.312618in}}%
\pgfpathlineto{\pgfqpoint{4.866497in}{2.312507in}}%
\pgfpathlineto{\pgfqpoint{4.867582in}{2.361786in}}%
\pgfpathlineto{\pgfqpoint{4.868865in}{2.387948in}}%
\pgfpathlineto{\pgfqpoint{4.869063in}{2.380466in}}%
\pgfpathlineto{\pgfqpoint{4.870148in}{2.358347in}}%
\pgfpathlineto{\pgfqpoint{4.870346in}{2.362785in}}%
\pgfpathlineto{\pgfqpoint{4.870445in}{2.362668in}}%
\pgfpathlineto{\pgfqpoint{4.870839in}{2.343178in}}%
\pgfpathlineto{\pgfqpoint{4.871234in}{2.362961in}}%
\pgfpathlineto{\pgfqpoint{4.871728in}{2.354834in}}%
\pgfpathlineto{\pgfqpoint{4.871925in}{2.362650in}}%
\pgfpathlineto{\pgfqpoint{4.872320in}{2.349186in}}%
\pgfpathlineto{\pgfqpoint{4.872813in}{2.355176in}}%
\pgfpathlineto{\pgfqpoint{4.872912in}{2.353421in}}%
\pgfpathlineto{\pgfqpoint{4.873208in}{2.364945in}}%
\pgfpathlineto{\pgfqpoint{4.874195in}{2.373371in}}%
\pgfpathlineto{\pgfqpoint{4.873702in}{2.352885in}}%
\pgfpathlineto{\pgfqpoint{4.874392in}{2.369911in}}%
\pgfpathlineto{\pgfqpoint{4.874689in}{2.360189in}}%
\pgfpathlineto{\pgfqpoint{4.875281in}{2.378757in}}%
\pgfpathlineto{\pgfqpoint{4.876761in}{2.414735in}}%
\pgfpathlineto{\pgfqpoint{4.878044in}{2.447566in}}%
\pgfpathlineto{\pgfqpoint{4.878834in}{2.445062in}}%
\pgfpathlineto{\pgfqpoint{4.879624in}{2.440723in}}%
\pgfpathlineto{\pgfqpoint{4.880709in}{2.417075in}}%
\pgfpathlineto{\pgfqpoint{4.881005in}{2.427260in}}%
\pgfpathlineto{\pgfqpoint{4.881499in}{2.454296in}}%
\pgfpathlineto{\pgfqpoint{4.882190in}{2.438120in}}%
\pgfpathlineto{\pgfqpoint{4.882584in}{2.422795in}}%
\pgfpathlineto{\pgfqpoint{4.883473in}{2.427561in}}%
\pgfpathlineto{\pgfqpoint{4.884756in}{2.492494in}}%
\pgfpathlineto{\pgfqpoint{4.885348in}{2.477456in}}%
\pgfpathlineto{\pgfqpoint{4.886532in}{2.442433in}}%
\pgfpathlineto{\pgfqpoint{4.886730in}{2.450601in}}%
\pgfpathlineto{\pgfqpoint{4.888112in}{2.536420in}}%
\pgfpathlineto{\pgfqpoint{4.889493in}{2.734921in}}%
\pgfpathlineto{\pgfqpoint{4.890184in}{2.682933in}}%
\pgfpathlineto{\pgfqpoint{4.890678in}{2.593828in}}%
\pgfpathlineto{\pgfqpoint{4.891369in}{2.439692in}}%
\pgfpathlineto{\pgfqpoint{4.892158in}{2.475088in}}%
\pgfpathlineto{\pgfqpoint{4.892553in}{2.468059in}}%
\pgfpathlineto{\pgfqpoint{4.892849in}{2.479951in}}%
\pgfpathlineto{\pgfqpoint{4.894231in}{2.540651in}}%
\pgfpathlineto{\pgfqpoint{4.894527in}{2.531556in}}%
\pgfpathlineto{\pgfqpoint{4.895711in}{2.463864in}}%
\pgfpathlineto{\pgfqpoint{4.896205in}{2.492220in}}%
\pgfpathlineto{\pgfqpoint{4.897488in}{2.524251in}}%
\pgfpathlineto{\pgfqpoint{4.897883in}{2.509362in}}%
\pgfpathlineto{\pgfqpoint{4.898969in}{2.479515in}}%
\pgfpathlineto{\pgfqpoint{4.899462in}{2.488881in}}%
\pgfpathlineto{\pgfqpoint{4.900548in}{2.499551in}}%
\pgfpathlineto{\pgfqpoint{4.900844in}{2.495165in}}%
\pgfpathlineto{\pgfqpoint{4.900942in}{2.493848in}}%
\pgfpathlineto{\pgfqpoint{4.901436in}{2.498074in}}%
\pgfpathlineto{\pgfqpoint{4.901732in}{2.497059in}}%
\pgfpathlineto{\pgfqpoint{4.903114in}{2.516676in}}%
\pgfpathlineto{\pgfqpoint{4.903311in}{2.511506in}}%
\pgfpathlineto{\pgfqpoint{4.903607in}{2.502238in}}%
\pgfpathlineto{\pgfqpoint{4.904101in}{2.517998in}}%
\pgfpathlineto{\pgfqpoint{4.904397in}{2.510590in}}%
\pgfpathlineto{\pgfqpoint{4.904594in}{2.514423in}}%
\pgfpathlineto{\pgfqpoint{4.905483in}{2.524739in}}%
\pgfpathlineto{\pgfqpoint{4.905680in}{2.520161in}}%
\pgfpathlineto{\pgfqpoint{4.906371in}{2.509309in}}%
\pgfpathlineto{\pgfqpoint{4.906963in}{2.512624in}}%
\pgfpathlineto{\pgfqpoint{4.907062in}{2.512484in}}%
\pgfpathlineto{\pgfqpoint{4.907555in}{2.528817in}}%
\pgfpathlineto{\pgfqpoint{4.907950in}{2.510073in}}%
\pgfpathlineto{\pgfqpoint{4.908345in}{2.517094in}}%
\pgfpathlineto{\pgfqpoint{4.908444in}{2.517831in}}%
\pgfpathlineto{\pgfqpoint{4.908740in}{2.513471in}}%
\pgfpathlineto{\pgfqpoint{4.909825in}{2.507585in}}%
\pgfpathlineto{\pgfqpoint{4.909332in}{2.521163in}}%
\pgfpathlineto{\pgfqpoint{4.910023in}{2.509382in}}%
\pgfpathlineto{\pgfqpoint{4.910714in}{2.505462in}}%
\pgfpathlineto{\pgfqpoint{4.911997in}{2.488323in}}%
\pgfpathlineto{\pgfqpoint{4.911405in}{2.507656in}}%
\pgfpathlineto{\pgfqpoint{4.912293in}{2.489225in}}%
\pgfpathlineto{\pgfqpoint{4.912688in}{2.489347in}}%
\pgfpathlineto{\pgfqpoint{4.912885in}{2.485197in}}%
\pgfpathlineto{\pgfqpoint{4.915155in}{2.431201in}}%
\pgfpathlineto{\pgfqpoint{4.915945in}{2.425746in}}%
\pgfpathlineto{\pgfqpoint{4.915550in}{2.436742in}}%
\pgfpathlineto{\pgfqpoint{4.916142in}{2.429610in}}%
\pgfpathlineto{\pgfqpoint{4.916241in}{2.432068in}}%
\pgfpathlineto{\pgfqpoint{4.916734in}{2.417552in}}%
\pgfpathlineto{\pgfqpoint{4.917030in}{2.424350in}}%
\pgfpathlineto{\pgfqpoint{4.917425in}{2.416954in}}%
\pgfpathlineto{\pgfqpoint{4.917919in}{2.425478in}}%
\pgfpathlineto{\pgfqpoint{4.918215in}{2.434275in}}%
\pgfpathlineto{\pgfqpoint{4.918708in}{2.419042in}}%
\pgfpathlineto{\pgfqpoint{4.918807in}{2.418509in}}%
\pgfpathlineto{\pgfqpoint{4.919004in}{2.422147in}}%
\pgfpathlineto{\pgfqpoint{4.919991in}{2.428644in}}%
\pgfpathlineto{\pgfqpoint{4.919498in}{2.421108in}}%
\pgfpathlineto{\pgfqpoint{4.920189in}{2.425839in}}%
\pgfpathlineto{\pgfqpoint{4.920584in}{2.412972in}}%
\pgfpathlineto{\pgfqpoint{4.921472in}{2.414352in}}%
\pgfpathlineto{\pgfqpoint{4.921571in}{2.414101in}}%
\pgfpathlineto{\pgfqpoint{4.921669in}{2.415397in}}%
\pgfpathlineto{\pgfqpoint{4.921965in}{2.419451in}}%
\pgfpathlineto{\pgfqpoint{4.922755in}{2.416348in}}%
\pgfpathlineto{\pgfqpoint{4.922854in}{2.415257in}}%
\pgfpathlineto{\pgfqpoint{4.923150in}{2.423289in}}%
\pgfpathlineto{\pgfqpoint{4.924334in}{2.438584in}}%
\pgfpathlineto{\pgfqpoint{4.923643in}{2.421579in}}%
\pgfpathlineto{\pgfqpoint{4.924532in}{2.435226in}}%
\pgfpathlineto{\pgfqpoint{4.924729in}{2.431825in}}%
\pgfpathlineto{\pgfqpoint{4.925025in}{2.447556in}}%
\pgfpathlineto{\pgfqpoint{4.925321in}{2.462809in}}%
\pgfpathlineto{\pgfqpoint{4.926111in}{2.457843in}}%
\pgfpathlineto{\pgfqpoint{4.928282in}{2.405440in}}%
\pgfpathlineto{\pgfqpoint{4.928677in}{2.395897in}}%
\pgfpathlineto{\pgfqpoint{4.928973in}{2.408586in}}%
\pgfpathlineto{\pgfqpoint{4.929170in}{2.413917in}}%
\pgfpathlineto{\pgfqpoint{4.929960in}{2.407319in}}%
\pgfpathlineto{\pgfqpoint{4.930947in}{2.394888in}}%
\pgfpathlineto{\pgfqpoint{4.931243in}{2.400693in}}%
\pgfpathlineto{\pgfqpoint{4.932033in}{2.428059in}}%
\pgfpathlineto{\pgfqpoint{4.932724in}{2.469506in}}%
\pgfpathlineto{\pgfqpoint{4.933118in}{2.438454in}}%
\pgfpathlineto{\pgfqpoint{4.934303in}{2.409341in}}%
\pgfpathlineto{\pgfqpoint{4.934500in}{2.409572in}}%
\pgfpathlineto{\pgfqpoint{4.935092in}{2.422302in}}%
\pgfpathlineto{\pgfqpoint{4.936474in}{2.603177in}}%
\pgfpathlineto{\pgfqpoint{4.937362in}{2.684019in}}%
\pgfpathlineto{\pgfqpoint{4.937757in}{2.661181in}}%
\pgfpathlineto{\pgfqpoint{4.938645in}{2.516521in}}%
\pgfpathlineto{\pgfqpoint{4.939139in}{2.408831in}}%
\pgfpathlineto{\pgfqpoint{4.939929in}{2.445621in}}%
\pgfpathlineto{\pgfqpoint{4.941014in}{2.431198in}}%
\pgfpathlineto{\pgfqpoint{4.941113in}{2.432645in}}%
\pgfpathlineto{\pgfqpoint{4.942199in}{2.468230in}}%
\pgfpathlineto{\pgfqpoint{4.942593in}{2.447033in}}%
\pgfpathlineto{\pgfqpoint{4.943087in}{2.424168in}}%
\pgfpathlineto{\pgfqpoint{4.943580in}{2.448565in}}%
\pgfpathlineto{\pgfqpoint{4.945456in}{2.540536in}}%
\pgfpathlineto{\pgfqpoint{4.945554in}{2.541917in}}%
\pgfpathlineto{\pgfqpoint{4.945851in}{2.533752in}}%
\pgfpathlineto{\pgfqpoint{4.946739in}{2.489864in}}%
\pgfpathlineto{\pgfqpoint{4.947528in}{2.505967in}}%
\pgfpathlineto{\pgfqpoint{4.947824in}{2.504003in}}%
\pgfpathlineto{\pgfqpoint{4.948022in}{2.509253in}}%
\pgfpathlineto{\pgfqpoint{4.948515in}{2.524192in}}%
\pgfpathlineto{\pgfqpoint{4.949206in}{2.513695in}}%
\pgfpathlineto{\pgfqpoint{4.950687in}{2.523046in}}%
\pgfpathlineto{\pgfqpoint{4.950785in}{2.522194in}}%
\pgfpathlineto{\pgfqpoint{4.950983in}{2.520887in}}%
\pgfpathlineto{\pgfqpoint{4.951279in}{2.525902in}}%
\pgfpathlineto{\pgfqpoint{4.952365in}{2.538237in}}%
\pgfpathlineto{\pgfqpoint{4.951871in}{2.521220in}}%
\pgfpathlineto{\pgfqpoint{4.952957in}{2.537930in}}%
\pgfpathlineto{\pgfqpoint{4.953056in}{2.538408in}}%
\pgfpathlineto{\pgfqpoint{4.953253in}{2.535186in}}%
\pgfpathlineto{\pgfqpoint{4.954141in}{2.528963in}}%
\pgfpathlineto{\pgfqpoint{4.953648in}{2.539377in}}%
\pgfpathlineto{\pgfqpoint{4.954339in}{2.532682in}}%
\pgfpathlineto{\pgfqpoint{4.955424in}{2.540236in}}%
\pgfpathlineto{\pgfqpoint{4.955622in}{2.537723in}}%
\pgfpathlineto{\pgfqpoint{4.956609in}{2.523327in}}%
\pgfpathlineto{\pgfqpoint{4.956905in}{2.532989in}}%
\pgfpathlineto{\pgfqpoint{4.957003in}{2.535693in}}%
\pgfpathlineto{\pgfqpoint{4.957398in}{2.523276in}}%
\pgfpathlineto{\pgfqpoint{4.957892in}{2.533465in}}%
\pgfpathlineto{\pgfqpoint{4.958287in}{2.514646in}}%
\pgfpathlineto{\pgfqpoint{4.959076in}{2.525709in}}%
\pgfpathlineto{\pgfqpoint{4.959175in}{2.525872in}}%
\pgfpathlineto{\pgfqpoint{4.959372in}{2.524876in}}%
\pgfpathlineto{\pgfqpoint{4.960853in}{2.501019in}}%
\pgfpathlineto{\pgfqpoint{4.962629in}{2.467075in}}%
\pgfpathlineto{\pgfqpoint{4.961346in}{2.502277in}}%
\pgfpathlineto{\pgfqpoint{4.962827in}{2.468039in}}%
\pgfpathlineto{\pgfqpoint{4.962925in}{2.468426in}}%
\pgfpathlineto{\pgfqpoint{4.963024in}{2.467304in}}%
\pgfpathlineto{\pgfqpoint{4.965097in}{2.436421in}}%
\pgfpathlineto{\pgfqpoint{4.963912in}{2.470782in}}%
\pgfpathlineto{\pgfqpoint{4.965196in}{2.437270in}}%
\pgfpathlineto{\pgfqpoint{4.966281in}{2.445413in}}%
\pgfpathlineto{\pgfqpoint{4.965788in}{2.434216in}}%
\pgfpathlineto{\pgfqpoint{4.966479in}{2.442470in}}%
\pgfpathlineto{\pgfqpoint{4.967564in}{2.428854in}}%
\pgfpathlineto{\pgfqpoint{4.967762in}{2.432075in}}%
\pgfpathlineto{\pgfqpoint{4.968551in}{2.439724in}}%
\pgfpathlineto{\pgfqpoint{4.968749in}{2.435209in}}%
\pgfpathlineto{\pgfqpoint{4.969834in}{2.417856in}}%
\pgfpathlineto{\pgfqpoint{4.970032in}{2.422781in}}%
\pgfpathlineto{\pgfqpoint{4.973289in}{2.476301in}}%
\pgfpathlineto{\pgfqpoint{4.970427in}{2.422126in}}%
\pgfpathlineto{\pgfqpoint{4.973684in}{2.466027in}}%
\pgfpathlineto{\pgfqpoint{4.974473in}{2.466395in}}%
\pgfpathlineto{\pgfqpoint{4.975756in}{2.436381in}}%
\pgfpathlineto{\pgfqpoint{4.976151in}{2.407608in}}%
\pgfpathlineto{\pgfqpoint{4.977039in}{2.415629in}}%
\pgfpathlineto{\pgfqpoint{4.977335in}{2.414568in}}%
\pgfpathlineto{\pgfqpoint{4.977533in}{2.418086in}}%
\pgfpathlineto{\pgfqpoint{4.977829in}{2.424582in}}%
\pgfpathlineto{\pgfqpoint{4.978322in}{2.406572in}}%
\pgfpathlineto{\pgfqpoint{4.979704in}{2.386305in}}%
\pgfpathlineto{\pgfqpoint{4.979902in}{2.392321in}}%
\pgfpathlineto{\pgfqpoint{4.980889in}{2.446084in}}%
\pgfpathlineto{\pgfqpoint{4.981283in}{2.419611in}}%
\pgfpathlineto{\pgfqpoint{4.982073in}{2.385465in}}%
\pgfpathlineto{\pgfqpoint{4.982468in}{2.399424in}}%
\pgfpathlineto{\pgfqpoint{4.982567in}{2.401301in}}%
\pgfpathlineto{\pgfqpoint{4.982764in}{2.391230in}}%
\pgfpathlineto{\pgfqpoint{4.982961in}{2.381063in}}%
\pgfpathlineto{\pgfqpoint{4.983554in}{2.411986in}}%
\pgfpathlineto{\pgfqpoint{4.984047in}{2.454099in}}%
\pgfpathlineto{\pgfqpoint{4.985429in}{2.653910in}}%
\pgfpathlineto{\pgfqpoint{4.986317in}{2.610354in}}%
\pgfpathlineto{\pgfqpoint{4.987304in}{2.375252in}}%
\pgfpathlineto{\pgfqpoint{4.988785in}{2.423662in}}%
\pgfpathlineto{\pgfqpoint{4.990462in}{2.501047in}}%
\pgfpathlineto{\pgfqpoint{4.990857in}{2.478397in}}%
\pgfpathlineto{\pgfqpoint{4.991351in}{2.437347in}}%
\pgfpathlineto{\pgfqpoint{4.992140in}{2.454237in}}%
\pgfpathlineto{\pgfqpoint{4.993720in}{2.489900in}}%
\pgfpathlineto{\pgfqpoint{4.993917in}{2.475692in}}%
\pgfpathlineto{\pgfqpoint{4.994904in}{2.437863in}}%
\pgfpathlineto{\pgfqpoint{4.995200in}{2.443765in}}%
\pgfpathlineto{\pgfqpoint{4.996088in}{2.459575in}}%
\pgfpathlineto{\pgfqpoint{4.996779in}{2.453917in}}%
\pgfpathlineto{\pgfqpoint{4.999345in}{2.483588in}}%
\pgfpathlineto{\pgfqpoint{5.000431in}{2.499617in}}%
\pgfpathlineto{\pgfqpoint{4.999938in}{2.481566in}}%
\pgfpathlineto{\pgfqpoint{5.000628in}{2.490536in}}%
\pgfpathlineto{\pgfqpoint{5.000826in}{2.484527in}}%
\pgfpathlineto{\pgfqpoint{5.001221in}{2.500294in}}%
\pgfpathlineto{\pgfqpoint{5.001615in}{2.491598in}}%
\pgfpathlineto{\pgfqpoint{5.001912in}{2.494605in}}%
\pgfpathlineto{\pgfqpoint{5.002208in}{2.488423in}}%
\pgfpathlineto{\pgfqpoint{5.003984in}{2.461333in}}%
\pgfpathlineto{\pgfqpoint{5.004379in}{2.468097in}}%
\pgfpathlineto{\pgfqpoint{5.004872in}{2.470967in}}%
\pgfpathlineto{\pgfqpoint{5.005070in}{2.463761in}}%
\pgfpathlineto{\pgfqpoint{5.005267in}{2.455641in}}%
\pgfpathlineto{\pgfqpoint{5.005662in}{2.468550in}}%
\pgfpathlineto{\pgfqpoint{5.006156in}{2.460124in}}%
\pgfpathlineto{\pgfqpoint{5.006452in}{2.469345in}}%
\pgfpathlineto{\pgfqpoint{5.007143in}{2.458478in}}%
\pgfpathlineto{\pgfqpoint{5.007241in}{2.458107in}}%
\pgfpathlineto{\pgfqpoint{5.007340in}{2.459774in}}%
\pgfpathlineto{\pgfqpoint{5.007537in}{2.463391in}}%
\pgfpathlineto{\pgfqpoint{5.007932in}{2.450623in}}%
\pgfpathlineto{\pgfqpoint{5.008327in}{2.456698in}}%
\pgfpathlineto{\pgfqpoint{5.008722in}{2.443345in}}%
\pgfpathlineto{\pgfqpoint{5.009807in}{2.445445in}}%
\pgfpathlineto{\pgfqpoint{5.010104in}{2.449337in}}%
\pgfpathlineto{\pgfqpoint{5.010400in}{2.442073in}}%
\pgfpathlineto{\pgfqpoint{5.011584in}{2.430494in}}%
\pgfpathlineto{\pgfqpoint{5.010992in}{2.442742in}}%
\pgfpathlineto{\pgfqpoint{5.011683in}{2.430634in}}%
\pgfpathlineto{\pgfqpoint{5.012078in}{2.434259in}}%
\pgfpathlineto{\pgfqpoint{5.012275in}{2.431579in}}%
\pgfpathlineto{\pgfqpoint{5.013854in}{2.408237in}}%
\pgfpathlineto{\pgfqpoint{5.014742in}{2.372412in}}%
\pgfpathlineto{\pgfqpoint{5.016322in}{2.321429in}}%
\pgfpathlineto{\pgfqpoint{5.016618in}{2.331159in}}%
\pgfpathlineto{\pgfqpoint{5.020467in}{2.494695in}}%
\pgfpathlineto{\pgfqpoint{5.020566in}{2.494442in}}%
\pgfpathlineto{\pgfqpoint{5.020763in}{2.488291in}}%
\pgfpathlineto{\pgfqpoint{5.021059in}{2.512168in}}%
\pgfpathlineto{\pgfqpoint{5.022342in}{2.543733in}}%
\pgfpathlineto{\pgfqpoint{5.022540in}{2.546412in}}%
\pgfpathlineto{\pgfqpoint{5.022836in}{2.533077in}}%
\pgfpathlineto{\pgfqpoint{5.024810in}{2.466790in}}%
\pgfpathlineto{\pgfqpoint{5.023428in}{2.533239in}}%
\pgfpathlineto{\pgfqpoint{5.024908in}{2.468516in}}%
\pgfpathlineto{\pgfqpoint{5.025402in}{2.483978in}}%
\pgfpathlineto{\pgfqpoint{5.025994in}{2.469615in}}%
\pgfpathlineto{\pgfqpoint{5.027573in}{2.436485in}}%
\pgfpathlineto{\pgfqpoint{5.027771in}{2.444073in}}%
\pgfpathlineto{\pgfqpoint{5.028758in}{2.500728in}}%
\pgfpathlineto{\pgfqpoint{5.029152in}{2.482290in}}%
\pgfpathlineto{\pgfqpoint{5.030337in}{2.421052in}}%
\pgfpathlineto{\pgfqpoint{5.030633in}{2.427517in}}%
\pgfpathlineto{\pgfqpoint{5.032113in}{2.497225in}}%
\pgfpathlineto{\pgfqpoint{5.033396in}{2.670244in}}%
\pgfpathlineto{\pgfqpoint{5.033890in}{2.630685in}}%
\pgfpathlineto{\pgfqpoint{5.034680in}{2.517744in}}%
\pgfpathlineto{\pgfqpoint{5.035272in}{2.386750in}}%
\pgfpathlineto{\pgfqpoint{5.036061in}{2.413332in}}%
\pgfpathlineto{\pgfqpoint{5.036259in}{2.405073in}}%
\pgfpathlineto{\pgfqpoint{5.036752in}{2.415812in}}%
\pgfpathlineto{\pgfqpoint{5.037048in}{2.414342in}}%
\pgfpathlineto{\pgfqpoint{5.037838in}{2.447749in}}%
\pgfpathlineto{\pgfqpoint{5.038134in}{2.463334in}}%
\pgfpathlineto{\pgfqpoint{5.038825in}{2.440773in}}%
\pgfpathlineto{\pgfqpoint{5.039318in}{2.393893in}}%
\pgfpathlineto{\pgfqpoint{5.040108in}{2.405488in}}%
\pgfpathlineto{\pgfqpoint{5.041490in}{2.436853in}}%
\pgfpathlineto{\pgfqpoint{5.041983in}{2.429001in}}%
\pgfpathlineto{\pgfqpoint{5.042575in}{2.395938in}}%
\pgfpathlineto{\pgfqpoint{5.043464in}{2.410637in}}%
\pgfpathlineto{\pgfqpoint{5.043661in}{2.407503in}}%
\pgfpathlineto{\pgfqpoint{5.043859in}{2.403048in}}%
\pgfpathlineto{\pgfqpoint{5.044155in}{2.422084in}}%
\pgfpathlineto{\pgfqpoint{5.044352in}{2.428162in}}%
\pgfpathlineto{\pgfqpoint{5.044648in}{2.418693in}}%
\pgfpathlineto{\pgfqpoint{5.045142in}{2.420417in}}%
\pgfpathlineto{\pgfqpoint{5.045339in}{2.417400in}}%
\pgfpathlineto{\pgfqpoint{5.045734in}{2.432228in}}%
\pgfpathlineto{\pgfqpoint{5.046129in}{2.423502in}}%
\pgfpathlineto{\pgfqpoint{5.046523in}{2.442743in}}%
\pgfpathlineto{\pgfqpoint{5.047510in}{2.437922in}}%
\pgfpathlineto{\pgfqpoint{5.048497in}{2.458393in}}%
\pgfpathlineto{\pgfqpoint{5.048794in}{2.467030in}}%
\pgfpathlineto{\pgfqpoint{5.049188in}{2.455839in}}%
\pgfpathlineto{\pgfqpoint{5.049682in}{2.464651in}}%
\pgfpathlineto{\pgfqpoint{5.049978in}{2.460010in}}%
\pgfpathlineto{\pgfqpoint{5.050373in}{2.471949in}}%
\pgfpathlineto{\pgfqpoint{5.050866in}{2.461955in}}%
\pgfpathlineto{\pgfqpoint{5.052741in}{2.486125in}}%
\pgfpathlineto{\pgfqpoint{5.053038in}{2.479523in}}%
\pgfpathlineto{\pgfqpoint{5.053334in}{2.474982in}}%
\pgfpathlineto{\pgfqpoint{5.053728in}{2.484236in}}%
\pgfpathlineto{\pgfqpoint{5.054321in}{2.491524in}}%
\pgfpathlineto{\pgfqpoint{5.054617in}{2.483589in}}%
\pgfpathlineto{\pgfqpoint{5.054913in}{2.470861in}}%
\pgfpathlineto{\pgfqpoint{5.055406in}{2.486500in}}%
\pgfpathlineto{\pgfqpoint{5.055801in}{2.478281in}}%
\pgfpathlineto{\pgfqpoint{5.055999in}{2.473815in}}%
\pgfpathlineto{\pgfqpoint{5.056591in}{2.481490in}}%
\pgfpathlineto{\pgfqpoint{5.056689in}{2.481335in}}%
\pgfpathlineto{\pgfqpoint{5.056986in}{2.483509in}}%
\pgfpathlineto{\pgfqpoint{5.057183in}{2.480051in}}%
\pgfpathlineto{\pgfqpoint{5.057479in}{2.471025in}}%
\pgfpathlineto{\pgfqpoint{5.058269in}{2.477774in}}%
\pgfpathlineto{\pgfqpoint{5.058367in}{2.477808in}}%
\pgfpathlineto{\pgfqpoint{5.059354in}{2.471661in}}%
\pgfpathlineto{\pgfqpoint{5.059453in}{2.474100in}}%
\pgfpathlineto{\pgfqpoint{5.059749in}{2.483382in}}%
\pgfpathlineto{\pgfqpoint{5.060637in}{2.478595in}}%
\pgfpathlineto{\pgfqpoint{5.060736in}{2.478296in}}%
\pgfpathlineto{\pgfqpoint{5.060933in}{2.479574in}}%
\pgfpathlineto{\pgfqpoint{5.061230in}{2.482875in}}%
\pgfpathlineto{\pgfqpoint{5.061526in}{2.472532in}}%
\pgfpathlineto{\pgfqpoint{5.061723in}{2.468199in}}%
\pgfpathlineto{\pgfqpoint{5.062217in}{2.483607in}}%
\pgfpathlineto{\pgfqpoint{5.062414in}{2.480491in}}%
\pgfpathlineto{\pgfqpoint{5.066165in}{2.421665in}}%
\pgfpathlineto{\pgfqpoint{5.066362in}{2.427535in}}%
\pgfpathlineto{\pgfqpoint{5.067546in}{2.444844in}}%
\pgfpathlineto{\pgfqpoint{5.067645in}{2.443431in}}%
\pgfpathlineto{\pgfqpoint{5.067744in}{2.442302in}}%
\pgfpathlineto{\pgfqpoint{5.067941in}{2.447416in}}%
\pgfpathlineto{\pgfqpoint{5.069619in}{2.480475in}}%
\pgfpathlineto{\pgfqpoint{5.069718in}{2.482179in}}%
\pgfpathlineto{\pgfqpoint{5.070112in}{2.478143in}}%
\pgfpathlineto{\pgfqpoint{5.070409in}{2.478471in}}%
\pgfpathlineto{\pgfqpoint{5.073271in}{2.398998in}}%
\pgfpathlineto{\pgfqpoint{5.074554in}{2.375050in}}%
\pgfpathlineto{\pgfqpoint{5.075640in}{2.363107in}}%
\pgfpathlineto{\pgfqpoint{5.075936in}{2.367439in}}%
\pgfpathlineto{\pgfqpoint{5.076627in}{2.419514in}}%
\pgfpathlineto{\pgfqpoint{5.077219in}{2.391266in}}%
\pgfpathlineto{\pgfqpoint{5.078699in}{2.346723in}}%
\pgfpathlineto{\pgfqpoint{5.079390in}{2.362361in}}%
\pgfpathlineto{\pgfqpoint{5.080278in}{2.463835in}}%
\pgfpathlineto{\pgfqpoint{5.081562in}{2.609790in}}%
\pgfpathlineto{\pgfqpoint{5.081858in}{2.584373in}}%
\pgfpathlineto{\pgfqpoint{5.082943in}{2.385732in}}%
\pgfpathlineto{\pgfqpoint{5.083239in}{2.335124in}}%
\pgfpathlineto{\pgfqpoint{5.084029in}{2.370966in}}%
\pgfpathlineto{\pgfqpoint{5.084424in}{2.368283in}}%
\pgfpathlineto{\pgfqpoint{5.084621in}{2.371399in}}%
\pgfpathlineto{\pgfqpoint{5.086398in}{2.459901in}}%
\pgfpathlineto{\pgfqpoint{5.086891in}{2.430784in}}%
\pgfpathlineto{\pgfqpoint{5.087780in}{2.384229in}}%
\pgfpathlineto{\pgfqpoint{5.088273in}{2.397060in}}%
\pgfpathlineto{\pgfqpoint{5.088767in}{2.371628in}}%
\pgfpathlineto{\pgfqpoint{5.089457in}{2.392207in}}%
\pgfpathlineto{\pgfqpoint{5.089951in}{2.378620in}}%
\pgfpathlineto{\pgfqpoint{5.090247in}{2.365555in}}%
\pgfpathlineto{\pgfqpoint{5.090741in}{2.388773in}}%
\pgfpathlineto{\pgfqpoint{5.093011in}{2.465128in}}%
\pgfpathlineto{\pgfqpoint{5.093702in}{2.458068in}}%
\pgfpathlineto{\pgfqpoint{5.093998in}{2.450005in}}%
\pgfpathlineto{\pgfqpoint{5.094392in}{2.460609in}}%
\pgfpathlineto{\pgfqpoint{5.094985in}{2.452898in}}%
\pgfpathlineto{\pgfqpoint{5.096959in}{2.472736in}}%
\pgfpathlineto{\pgfqpoint{5.097057in}{2.470078in}}%
\pgfpathlineto{\pgfqpoint{5.097353in}{2.461159in}}%
\pgfpathlineto{\pgfqpoint{5.097748in}{2.472293in}}%
\pgfpathlineto{\pgfqpoint{5.098143in}{2.468660in}}%
\pgfpathlineto{\pgfqpoint{5.099130in}{2.479820in}}%
\pgfpathlineto{\pgfqpoint{5.098735in}{2.467922in}}%
\pgfpathlineto{\pgfqpoint{5.099327in}{2.476294in}}%
\pgfpathlineto{\pgfqpoint{5.099525in}{2.470195in}}%
\pgfpathlineto{\pgfqpoint{5.100018in}{2.478107in}}%
\pgfpathlineto{\pgfqpoint{5.100413in}{2.475304in}}%
\pgfpathlineto{\pgfqpoint{5.100512in}{2.475457in}}%
\pgfpathlineto{\pgfqpoint{5.100610in}{2.474719in}}%
\pgfpathlineto{\pgfqpoint{5.101005in}{2.467272in}}%
\pgfpathlineto{\pgfqpoint{5.101301in}{2.480524in}}%
\pgfpathlineto{\pgfqpoint{5.101400in}{2.483967in}}%
\pgfpathlineto{\pgfqpoint{5.101894in}{2.467666in}}%
\pgfpathlineto{\pgfqpoint{5.102288in}{2.476816in}}%
\pgfpathlineto{\pgfqpoint{5.102387in}{2.476769in}}%
\pgfpathlineto{\pgfqpoint{5.102486in}{2.477392in}}%
\pgfpathlineto{\pgfqpoint{5.102881in}{2.484802in}}%
\pgfpathlineto{\pgfqpoint{5.103571in}{2.479499in}}%
\pgfpathlineto{\pgfqpoint{5.104262in}{2.475465in}}%
\pgfpathlineto{\pgfqpoint{5.104558in}{2.478372in}}%
\pgfpathlineto{\pgfqpoint{5.104756in}{2.480166in}}%
\pgfpathlineto{\pgfqpoint{5.105151in}{2.471981in}}%
\pgfpathlineto{\pgfqpoint{5.106631in}{2.450316in}}%
\pgfpathlineto{\pgfqpoint{5.106828in}{2.455673in}}%
\pgfpathlineto{\pgfqpoint{5.107026in}{2.463255in}}%
\pgfpathlineto{\pgfqpoint{5.107618in}{2.440818in}}%
\pgfpathlineto{\pgfqpoint{5.108901in}{2.427535in}}%
\pgfpathlineto{\pgfqpoint{5.108408in}{2.441690in}}%
\pgfpathlineto{\pgfqpoint{5.109099in}{2.429309in}}%
\pgfpathlineto{\pgfqpoint{5.111369in}{2.409927in}}%
\pgfpathlineto{\pgfqpoint{5.112356in}{2.400221in}}%
\pgfpathlineto{\pgfqpoint{5.111862in}{2.410310in}}%
\pgfpathlineto{\pgfqpoint{5.112553in}{2.408113in}}%
\pgfpathlineto{\pgfqpoint{5.112750in}{2.412795in}}%
\pgfpathlineto{\pgfqpoint{5.113145in}{2.398446in}}%
\pgfpathlineto{\pgfqpoint{5.113540in}{2.405129in}}%
\pgfpathlineto{\pgfqpoint{5.114034in}{2.391846in}}%
\pgfpathlineto{\pgfqpoint{5.114527in}{2.406117in}}%
\pgfpathlineto{\pgfqpoint{5.115021in}{2.396444in}}%
\pgfpathlineto{\pgfqpoint{5.116600in}{2.428814in}}%
\pgfpathlineto{\pgfqpoint{5.116797in}{2.428429in}}%
\pgfpathlineto{\pgfqpoint{5.117883in}{2.449487in}}%
\pgfpathlineto{\pgfqpoint{5.118080in}{2.441810in}}%
\pgfpathlineto{\pgfqpoint{5.119659in}{2.406234in}}%
\pgfpathlineto{\pgfqpoint{5.120745in}{2.390229in}}%
\pgfpathlineto{\pgfqpoint{5.120942in}{2.396725in}}%
\pgfpathlineto{\pgfqpoint{5.121239in}{2.409229in}}%
\pgfpathlineto{\pgfqpoint{5.121732in}{2.387998in}}%
\pgfpathlineto{\pgfqpoint{5.123015in}{2.364397in}}%
\pgfpathlineto{\pgfqpoint{5.123213in}{2.366063in}}%
\pgfpathlineto{\pgfqpoint{5.123706in}{2.389300in}}%
\pgfpathlineto{\pgfqpoint{5.124496in}{2.419738in}}%
\pgfpathlineto{\pgfqpoint{5.124792in}{2.398278in}}%
\pgfpathlineto{\pgfqpoint{5.126272in}{2.364326in}}%
\pgfpathlineto{\pgfqpoint{5.126568in}{2.355866in}}%
\pgfpathlineto{\pgfqpoint{5.126963in}{2.370872in}}%
\pgfpathlineto{\pgfqpoint{5.128345in}{2.564646in}}%
\pgfpathlineto{\pgfqpoint{5.128937in}{2.638510in}}%
\pgfpathlineto{\pgfqpoint{5.129628in}{2.596304in}}%
\pgfpathlineto{\pgfqpoint{5.130220in}{2.533870in}}%
\pgfpathlineto{\pgfqpoint{5.130911in}{2.346817in}}%
\pgfpathlineto{\pgfqpoint{5.131799in}{2.377835in}}%
\pgfpathlineto{\pgfqpoint{5.131997in}{2.370471in}}%
\pgfpathlineto{\pgfqpoint{5.132490in}{2.400949in}}%
\pgfpathlineto{\pgfqpoint{5.132786in}{2.396985in}}%
\pgfpathlineto{\pgfqpoint{5.132885in}{2.399061in}}%
\pgfpathlineto{\pgfqpoint{5.134267in}{2.455715in}}%
\pgfpathlineto{\pgfqpoint{5.134563in}{2.432556in}}%
\pgfpathlineto{\pgfqpoint{5.135254in}{2.397327in}}%
\pgfpathlineto{\pgfqpoint{5.135747in}{2.419749in}}%
\pgfpathlineto{\pgfqpoint{5.137129in}{2.453300in}}%
\pgfpathlineto{\pgfqpoint{5.136438in}{2.419646in}}%
\pgfpathlineto{\pgfqpoint{5.137326in}{2.448929in}}%
\pgfpathlineto{\pgfqpoint{5.138313in}{2.415420in}}%
\pgfpathlineto{\pgfqpoint{5.138708in}{2.429404in}}%
\pgfpathlineto{\pgfqpoint{5.138906in}{2.427403in}}%
\pgfpathlineto{\pgfqpoint{5.139202in}{2.433256in}}%
\pgfpathlineto{\pgfqpoint{5.141077in}{2.454735in}}%
\pgfpathlineto{\pgfqpoint{5.139695in}{2.431317in}}%
\pgfpathlineto{\pgfqpoint{5.141176in}{2.454498in}}%
\pgfpathlineto{\pgfqpoint{5.141571in}{2.436186in}}%
\pgfpathlineto{\pgfqpoint{5.142064in}{2.457048in}}%
\pgfpathlineto{\pgfqpoint{5.142360in}{2.449909in}}%
\pgfpathlineto{\pgfqpoint{5.143150in}{2.464555in}}%
\pgfpathlineto{\pgfqpoint{5.144137in}{2.471253in}}%
\pgfpathlineto{\pgfqpoint{5.143643in}{2.457638in}}%
\pgfpathlineto{\pgfqpoint{5.144334in}{2.465865in}}%
\pgfpathlineto{\pgfqpoint{5.144531in}{2.463021in}}%
\pgfpathlineto{\pgfqpoint{5.145124in}{2.470546in}}%
\pgfpathlineto{\pgfqpoint{5.146505in}{2.479205in}}%
\pgfpathlineto{\pgfqpoint{5.146703in}{2.476612in}}%
\pgfpathlineto{\pgfqpoint{5.146999in}{2.470333in}}%
\pgfpathlineto{\pgfqpoint{5.147690in}{2.479983in}}%
\pgfpathlineto{\pgfqpoint{5.147789in}{2.480096in}}%
\pgfpathlineto{\pgfqpoint{5.147887in}{2.478845in}}%
\pgfpathlineto{\pgfqpoint{5.148874in}{2.466670in}}%
\pgfpathlineto{\pgfqpoint{5.149072in}{2.472306in}}%
\pgfpathlineto{\pgfqpoint{5.150157in}{2.484049in}}%
\pgfpathlineto{\pgfqpoint{5.149664in}{2.465069in}}%
\pgfpathlineto{\pgfqpoint{5.150256in}{2.482563in}}%
\pgfpathlineto{\pgfqpoint{5.151440in}{2.470537in}}%
\pgfpathlineto{\pgfqpoint{5.150947in}{2.482662in}}%
\pgfpathlineto{\pgfqpoint{5.151539in}{2.471233in}}%
\pgfpathlineto{\pgfqpoint{5.151835in}{2.476118in}}%
\pgfpathlineto{\pgfqpoint{5.152131in}{2.463253in}}%
\pgfpathlineto{\pgfqpoint{5.153118in}{2.454115in}}%
\pgfpathlineto{\pgfqpoint{5.152625in}{2.464170in}}%
\pgfpathlineto{\pgfqpoint{5.153316in}{2.457392in}}%
\pgfpathlineto{\pgfqpoint{5.153414in}{2.458960in}}%
\pgfpathlineto{\pgfqpoint{5.153809in}{2.448161in}}%
\pgfpathlineto{\pgfqpoint{5.155388in}{2.431978in}}%
\pgfpathlineto{\pgfqpoint{5.155586in}{2.434408in}}%
\pgfpathlineto{\pgfqpoint{5.155882in}{2.441897in}}%
\pgfpathlineto{\pgfqpoint{5.156277in}{2.423943in}}%
\pgfpathlineto{\pgfqpoint{5.156474in}{2.419228in}}%
\pgfpathlineto{\pgfqpoint{5.156869in}{2.442442in}}%
\pgfpathlineto{\pgfqpoint{5.157362in}{2.423396in}}%
\pgfpathlineto{\pgfqpoint{5.157658in}{2.432509in}}%
\pgfpathlineto{\pgfqpoint{5.158053in}{2.416024in}}%
\pgfpathlineto{\pgfqpoint{5.158448in}{2.429120in}}%
\pgfpathlineto{\pgfqpoint{5.160027in}{2.410782in}}%
\pgfpathlineto{\pgfqpoint{5.160126in}{2.411946in}}%
\pgfpathlineto{\pgfqpoint{5.160422in}{2.421560in}}%
\pgfpathlineto{\pgfqpoint{5.161310in}{2.418689in}}%
\pgfpathlineto{\pgfqpoint{5.164271in}{2.370541in}}%
\pgfpathlineto{\pgfqpoint{5.161804in}{2.423214in}}%
\pgfpathlineto{\pgfqpoint{5.164666in}{2.378247in}}%
\pgfpathlineto{\pgfqpoint{5.167134in}{2.434854in}}%
\pgfpathlineto{\pgfqpoint{5.167627in}{2.417233in}}%
\pgfpathlineto{\pgfqpoint{5.169108in}{2.394817in}}%
\pgfpathlineto{\pgfqpoint{5.169206in}{2.395095in}}%
\pgfpathlineto{\pgfqpoint{5.169601in}{2.380060in}}%
\pgfpathlineto{\pgfqpoint{5.170884in}{2.367079in}}%
\pgfpathlineto{\pgfqpoint{5.171476in}{2.393602in}}%
\pgfpathlineto{\pgfqpoint{5.172069in}{2.437549in}}%
\pgfpathlineto{\pgfqpoint{5.172661in}{2.410193in}}%
\pgfpathlineto{\pgfqpoint{5.173549in}{2.369714in}}%
\pgfpathlineto{\pgfqpoint{5.174240in}{2.369993in}}%
\pgfpathlineto{\pgfqpoint{5.174733in}{2.406501in}}%
\pgfpathlineto{\pgfqpoint{5.175819in}{2.532879in}}%
\pgfpathlineto{\pgfqpoint{5.176707in}{2.669041in}}%
\pgfpathlineto{\pgfqpoint{5.177398in}{2.640345in}}%
\pgfpathlineto{\pgfqpoint{5.178188in}{2.503176in}}%
\pgfpathlineto{\pgfqpoint{5.178583in}{2.397770in}}%
\pgfpathlineto{\pgfqpoint{5.179471in}{2.441276in}}%
\pgfpathlineto{\pgfqpoint{5.179767in}{2.431264in}}%
\pgfpathlineto{\pgfqpoint{5.180261in}{2.451978in}}%
\pgfpathlineto{\pgfqpoint{5.181544in}{2.506575in}}%
\pgfpathlineto{\pgfqpoint{5.181840in}{2.522404in}}%
\pgfpathlineto{\pgfqpoint{5.182333in}{2.490240in}}%
\pgfpathlineto{\pgfqpoint{5.182728in}{2.459334in}}%
\pgfpathlineto{\pgfqpoint{5.183419in}{2.485059in}}%
\pgfpathlineto{\pgfqpoint{5.184702in}{2.526262in}}%
\pgfpathlineto{\pgfqpoint{5.185590in}{2.511918in}}%
\pgfpathlineto{\pgfqpoint{5.186084in}{2.498330in}}%
\pgfpathlineto{\pgfqpoint{5.186577in}{2.514269in}}%
\pgfpathlineto{\pgfqpoint{5.186873in}{2.514207in}}%
\pgfpathlineto{\pgfqpoint{5.187762in}{2.531459in}}%
\pgfpathlineto{\pgfqpoint{5.188058in}{2.544304in}}%
\pgfpathlineto{\pgfqpoint{5.188551in}{2.530900in}}%
\pgfpathlineto{\pgfqpoint{5.188946in}{2.543770in}}%
\pgfpathlineto{\pgfqpoint{5.189440in}{2.526129in}}%
\pgfpathlineto{\pgfqpoint{5.189834in}{2.548771in}}%
\pgfpathlineto{\pgfqpoint{5.189933in}{2.551074in}}%
\pgfpathlineto{\pgfqpoint{5.190821in}{2.545914in}}%
\pgfpathlineto{\pgfqpoint{5.190920in}{2.544610in}}%
\pgfpathlineto{\pgfqpoint{5.191117in}{2.550572in}}%
\pgfpathlineto{\pgfqpoint{5.192203in}{2.571151in}}%
\pgfpathlineto{\pgfqpoint{5.192499in}{2.568211in}}%
\pgfpathlineto{\pgfqpoint{5.192795in}{2.568618in}}%
\pgfpathlineto{\pgfqpoint{5.193289in}{2.553447in}}%
\pgfpathlineto{\pgfqpoint{5.193980in}{2.565988in}}%
\pgfpathlineto{\pgfqpoint{5.194769in}{2.576575in}}%
\pgfpathlineto{\pgfqpoint{5.195065in}{2.567645in}}%
\pgfpathlineto{\pgfqpoint{5.195263in}{2.561347in}}%
\pgfpathlineto{\pgfqpoint{5.195756in}{2.580875in}}%
\pgfpathlineto{\pgfqpoint{5.196052in}{2.571127in}}%
\pgfpathlineto{\pgfqpoint{5.196250in}{2.568763in}}%
\pgfpathlineto{\pgfqpoint{5.196743in}{2.580104in}}%
\pgfpathlineto{\pgfqpoint{5.197730in}{2.566438in}}%
\pgfpathlineto{\pgfqpoint{5.198717in}{2.555248in}}%
\pgfpathlineto{\pgfqpoint{5.198224in}{2.566820in}}%
\pgfpathlineto{\pgfqpoint{5.199507in}{2.558207in}}%
\pgfpathlineto{\pgfqpoint{5.199704in}{2.561600in}}%
\pgfpathlineto{\pgfqpoint{5.200198in}{2.550273in}}%
\pgfpathlineto{\pgfqpoint{5.200494in}{2.546862in}}%
\pgfpathlineto{\pgfqpoint{5.203356in}{2.494226in}}%
\pgfpathlineto{\pgfqpoint{5.203455in}{2.494099in}}%
\pgfpathlineto{\pgfqpoint{5.203652in}{2.496973in}}%
\pgfpathlineto{\pgfqpoint{5.203948in}{2.486086in}}%
\pgfpathlineto{\pgfqpoint{5.204047in}{2.482965in}}%
\pgfpathlineto{\pgfqpoint{5.204442in}{2.494909in}}%
\pgfpathlineto{\pgfqpoint{5.204935in}{2.488619in}}%
\pgfpathlineto{\pgfqpoint{5.205034in}{2.490048in}}%
\pgfpathlineto{\pgfqpoint{5.205330in}{2.482438in}}%
\pgfpathlineto{\pgfqpoint{5.206120in}{2.487493in}}%
\pgfpathlineto{\pgfqpoint{5.206613in}{2.471614in}}%
\pgfpathlineto{\pgfqpoint{5.208192in}{2.493098in}}%
\pgfpathlineto{\pgfqpoint{5.208291in}{2.491536in}}%
\pgfpathlineto{\pgfqpoint{5.208686in}{2.474136in}}%
\pgfpathlineto{\pgfqpoint{5.209475in}{2.485768in}}%
\pgfpathlineto{\pgfqpoint{5.209574in}{2.486423in}}%
\pgfpathlineto{\pgfqpoint{5.209772in}{2.482591in}}%
\pgfpathlineto{\pgfqpoint{5.209969in}{2.477555in}}%
\pgfpathlineto{\pgfqpoint{5.210462in}{2.495523in}}%
\pgfpathlineto{\pgfqpoint{5.211055in}{2.490002in}}%
\pgfpathlineto{\pgfqpoint{5.211351in}{2.496919in}}%
\pgfpathlineto{\pgfqpoint{5.213127in}{2.534600in}}%
\pgfpathlineto{\pgfqpoint{5.215693in}{2.461230in}}%
\pgfpathlineto{\pgfqpoint{5.215990in}{2.467275in}}%
\pgfpathlineto{\pgfqpoint{5.216384in}{2.477074in}}%
\pgfpathlineto{\pgfqpoint{5.217075in}{2.469605in}}%
\pgfpathlineto{\pgfqpoint{5.218457in}{2.433228in}}%
\pgfpathlineto{\pgfqpoint{5.218753in}{2.445664in}}%
\pgfpathlineto{\pgfqpoint{5.219740in}{2.502576in}}%
\pgfpathlineto{\pgfqpoint{5.220135in}{2.467253in}}%
\pgfpathlineto{\pgfqpoint{5.221615in}{2.438175in}}%
\pgfpathlineto{\pgfqpoint{5.221813in}{2.433108in}}%
\pgfpathlineto{\pgfqpoint{5.222208in}{2.454719in}}%
\pgfpathlineto{\pgfqpoint{5.223491in}{2.583227in}}%
\pgfpathlineto{\pgfqpoint{5.224379in}{2.697838in}}%
\pgfpathlineto{\pgfqpoint{5.224872in}{2.669292in}}%
\pgfpathlineto{\pgfqpoint{5.225662in}{2.567445in}}%
\pgfpathlineto{\pgfqpoint{5.226254in}{2.413985in}}%
\pgfpathlineto{\pgfqpoint{5.227143in}{2.443109in}}%
\pgfpathlineto{\pgfqpoint{5.227340in}{2.433082in}}%
\pgfpathlineto{\pgfqpoint{5.227833in}{2.453599in}}%
\pgfpathlineto{\pgfqpoint{5.228130in}{2.450338in}}%
\pgfpathlineto{\pgfqpoint{5.228919in}{2.489928in}}%
\pgfpathlineto{\pgfqpoint{5.229413in}{2.515310in}}%
\pgfpathlineto{\pgfqpoint{5.229906in}{2.491515in}}%
\pgfpathlineto{\pgfqpoint{5.230400in}{2.440947in}}%
\pgfpathlineto{\pgfqpoint{5.231288in}{2.455197in}}%
\pgfpathlineto{\pgfqpoint{5.232670in}{2.498014in}}%
\pgfpathlineto{\pgfqpoint{5.233361in}{2.483293in}}%
\pgfpathlineto{\pgfqpoint{5.233657in}{2.468367in}}%
\pgfpathlineto{\pgfqpoint{5.234150in}{2.497604in}}%
\pgfpathlineto{\pgfqpoint{5.234249in}{2.497369in}}%
\pgfpathlineto{\pgfqpoint{5.234446in}{2.496216in}}%
\pgfpathlineto{\pgfqpoint{5.234644in}{2.501656in}}%
\pgfpathlineto{\pgfqpoint{5.235729in}{2.524102in}}%
\pgfpathlineto{\pgfqpoint{5.236124in}{2.516424in}}%
\pgfpathlineto{\pgfqpoint{5.238592in}{2.454909in}}%
\pgfpathlineto{\pgfqpoint{5.239184in}{2.469042in}}%
\pgfpathlineto{\pgfqpoint{5.240171in}{2.528692in}}%
\pgfpathlineto{\pgfqpoint{5.241059in}{2.550921in}}%
\pgfpathlineto{\pgfqpoint{5.241454in}{2.542720in}}%
\pgfpathlineto{\pgfqpoint{5.241553in}{2.541430in}}%
\pgfpathlineto{\pgfqpoint{5.241849in}{2.550698in}}%
\pgfpathlineto{\pgfqpoint{5.242046in}{2.553507in}}%
\pgfpathlineto{\pgfqpoint{5.242342in}{2.535949in}}%
\pgfpathlineto{\pgfqpoint{5.242540in}{2.528404in}}%
\pgfpathlineto{\pgfqpoint{5.242934in}{2.544249in}}%
\pgfpathlineto{\pgfqpoint{5.243428in}{2.534345in}}%
\pgfpathlineto{\pgfqpoint{5.244316in}{2.537159in}}%
\pgfpathlineto{\pgfqpoint{5.243921in}{2.533147in}}%
\pgfpathlineto{\pgfqpoint{5.244415in}{2.534306in}}%
\pgfpathlineto{\pgfqpoint{5.244711in}{2.524343in}}%
\pgfpathlineto{\pgfqpoint{5.245106in}{2.538285in}}%
\pgfpathlineto{\pgfqpoint{5.245501in}{2.529442in}}%
\pgfpathlineto{\pgfqpoint{5.245797in}{2.536000in}}%
\pgfpathlineto{\pgfqpoint{5.246191in}{2.526239in}}%
\pgfpathlineto{\pgfqpoint{5.247474in}{2.507551in}}%
\pgfpathlineto{\pgfqpoint{5.247672in}{2.514477in}}%
\pgfpathlineto{\pgfqpoint{5.247869in}{2.518617in}}%
\pgfpathlineto{\pgfqpoint{5.248264in}{2.499427in}}%
\pgfpathlineto{\pgfqpoint{5.248461in}{2.501485in}}%
\pgfpathlineto{\pgfqpoint{5.248659in}{2.498188in}}%
\pgfpathlineto{\pgfqpoint{5.250238in}{2.472408in}}%
\pgfpathlineto{\pgfqpoint{5.252607in}{2.428531in}}%
\pgfpathlineto{\pgfqpoint{5.252804in}{2.434757in}}%
\pgfpathlineto{\pgfqpoint{5.253002in}{2.437427in}}%
\pgfpathlineto{\pgfqpoint{5.253396in}{2.428721in}}%
\pgfpathlineto{\pgfqpoint{5.253890in}{2.434999in}}%
\pgfpathlineto{\pgfqpoint{5.255667in}{2.414553in}}%
\pgfpathlineto{\pgfqpoint{5.255765in}{2.416464in}}%
\pgfpathlineto{\pgfqpoint{5.255963in}{2.421213in}}%
\pgfpathlineto{\pgfqpoint{5.256456in}{2.405736in}}%
\pgfpathlineto{\pgfqpoint{5.256555in}{2.406354in}}%
\pgfpathlineto{\pgfqpoint{5.256752in}{2.405560in}}%
\pgfpathlineto{\pgfqpoint{5.256950in}{2.402800in}}%
\pgfpathlineto{\pgfqpoint{5.257443in}{2.410137in}}%
\pgfpathlineto{\pgfqpoint{5.257640in}{2.407769in}}%
\pgfpathlineto{\pgfqpoint{5.259417in}{2.433107in}}%
\pgfpathlineto{\pgfqpoint{5.259713in}{2.448738in}}%
\pgfpathlineto{\pgfqpoint{5.260305in}{2.419079in}}%
\pgfpathlineto{\pgfqpoint{5.260700in}{2.421691in}}%
\pgfpathlineto{\pgfqpoint{5.261885in}{2.397675in}}%
\pgfpathlineto{\pgfqpoint{5.263266in}{2.340792in}}%
\pgfpathlineto{\pgfqpoint{5.263562in}{2.356560in}}%
\pgfpathlineto{\pgfqpoint{5.263760in}{2.366801in}}%
\pgfpathlineto{\pgfqpoint{5.264352in}{2.340402in}}%
\pgfpathlineto{\pgfqpoint{5.265635in}{2.308130in}}%
\pgfpathlineto{\pgfqpoint{5.265931in}{2.318894in}}%
\pgfpathlineto{\pgfqpoint{5.266523in}{2.343983in}}%
\pgfpathlineto{\pgfqpoint{5.267116in}{2.377296in}}%
\pgfpathlineto{\pgfqpoint{5.267609in}{2.342224in}}%
\pgfpathlineto{\pgfqpoint{5.269090in}{2.317319in}}%
\pgfpathlineto{\pgfqpoint{5.269188in}{2.317806in}}%
\pgfpathlineto{\pgfqpoint{5.270274in}{2.398153in}}%
\pgfpathlineto{\pgfqpoint{5.271853in}{2.608135in}}%
\pgfpathlineto{\pgfqpoint{5.272347in}{2.580139in}}%
\pgfpathlineto{\pgfqpoint{5.272840in}{2.513634in}}%
\pgfpathlineto{\pgfqpoint{5.273630in}{2.333625in}}%
\pgfpathlineto{\pgfqpoint{5.274518in}{2.353550in}}%
\pgfpathlineto{\pgfqpoint{5.274617in}{2.353919in}}%
\pgfpathlineto{\pgfqpoint{5.276887in}{2.429284in}}%
\pgfpathlineto{\pgfqpoint{5.276985in}{2.425220in}}%
\pgfpathlineto{\pgfqpoint{5.277479in}{2.385927in}}%
\pgfpathlineto{\pgfqpoint{5.278367in}{2.397624in}}%
\pgfpathlineto{\pgfqpoint{5.278663in}{2.390954in}}%
\pgfpathlineto{\pgfqpoint{5.278959in}{2.402594in}}%
\pgfpathlineto{\pgfqpoint{5.279749in}{2.436161in}}%
\pgfpathlineto{\pgfqpoint{5.280341in}{2.431661in}}%
\pgfpathlineto{\pgfqpoint{5.280933in}{2.390569in}}%
\pgfpathlineto{\pgfqpoint{5.281723in}{2.412666in}}%
\pgfpathlineto{\pgfqpoint{5.282710in}{2.403706in}}%
\pgfpathlineto{\pgfqpoint{5.282907in}{2.407436in}}%
\pgfpathlineto{\pgfqpoint{5.283401in}{2.428957in}}%
\pgfpathlineto{\pgfqpoint{5.284092in}{2.417377in}}%
\pgfpathlineto{\pgfqpoint{5.284289in}{2.416156in}}%
\pgfpathlineto{\pgfqpoint{5.284684in}{2.421619in}}%
\pgfpathlineto{\pgfqpoint{5.284881in}{2.421065in}}%
\pgfpathlineto{\pgfqpoint{5.286164in}{2.433417in}}%
\pgfpathlineto{\pgfqpoint{5.286362in}{2.428454in}}%
\pgfpathlineto{\pgfqpoint{5.286658in}{2.418662in}}%
\pgfpathlineto{\pgfqpoint{5.287151in}{2.438898in}}%
\pgfpathlineto{\pgfqpoint{5.287349in}{2.440877in}}%
\pgfpathlineto{\pgfqpoint{5.288138in}{2.439077in}}%
\pgfpathlineto{\pgfqpoint{5.288731in}{2.435204in}}%
\pgfpathlineto{\pgfqpoint{5.289027in}{2.440748in}}%
\pgfpathlineto{\pgfqpoint{5.289915in}{2.452341in}}%
\pgfpathlineto{\pgfqpoint{5.290112in}{2.443314in}}%
\pgfpathlineto{\pgfqpoint{5.290310in}{2.435602in}}%
\pgfpathlineto{\pgfqpoint{5.290803in}{2.452652in}}%
\pgfpathlineto{\pgfqpoint{5.291001in}{2.451305in}}%
\pgfpathlineto{\pgfqpoint{5.291494in}{2.447509in}}%
\pgfpathlineto{\pgfqpoint{5.291790in}{2.452633in}}%
\pgfpathlineto{\pgfqpoint{5.291988in}{2.455934in}}%
\pgfpathlineto{\pgfqpoint{5.292481in}{2.443217in}}%
\pgfpathlineto{\pgfqpoint{5.292975in}{2.455093in}}%
\pgfpathlineto{\pgfqpoint{5.298897in}{2.393806in}}%
\pgfpathlineto{\pgfqpoint{5.299094in}{2.398272in}}%
\pgfpathlineto{\pgfqpoint{5.299291in}{2.404733in}}%
\pgfpathlineto{\pgfqpoint{5.299884in}{2.386660in}}%
\pgfpathlineto{\pgfqpoint{5.300180in}{2.391424in}}%
\pgfpathlineto{\pgfqpoint{5.300969in}{2.387847in}}%
\pgfpathlineto{\pgfqpoint{5.301167in}{2.382698in}}%
\pgfpathlineto{\pgfqpoint{5.301562in}{2.399197in}}%
\pgfpathlineto{\pgfqpoint{5.301660in}{2.400094in}}%
\pgfpathlineto{\pgfqpoint{5.301759in}{2.396326in}}%
\pgfpathlineto{\pgfqpoint{5.302055in}{2.380949in}}%
\pgfpathlineto{\pgfqpoint{5.302845in}{2.391756in}}%
\pgfpathlineto{\pgfqpoint{5.303042in}{2.396571in}}%
\pgfpathlineto{\pgfqpoint{5.303437in}{2.386445in}}%
\pgfpathlineto{\pgfqpoint{5.303832in}{2.390962in}}%
\pgfpathlineto{\pgfqpoint{5.304029in}{2.386213in}}%
\pgfpathlineto{\pgfqpoint{5.304522in}{2.400856in}}%
\pgfpathlineto{\pgfqpoint{5.304917in}{2.391950in}}%
\pgfpathlineto{\pgfqpoint{5.305115in}{2.400804in}}%
\pgfpathlineto{\pgfqpoint{5.306990in}{2.454882in}}%
\pgfpathlineto{\pgfqpoint{5.308668in}{2.440019in}}%
\pgfpathlineto{\pgfqpoint{5.310247in}{2.401261in}}%
\pgfpathlineto{\pgfqpoint{5.310444in}{2.405121in}}%
\pgfpathlineto{\pgfqpoint{5.310839in}{2.427743in}}%
\pgfpathlineto{\pgfqpoint{5.311431in}{2.407052in}}%
\pgfpathlineto{\pgfqpoint{5.313307in}{2.322718in}}%
\pgfpathlineto{\pgfqpoint{5.313800in}{2.352532in}}%
\pgfpathlineto{\pgfqpoint{5.314590in}{2.378664in}}%
\pgfpathlineto{\pgfqpoint{5.314985in}{2.363011in}}%
\pgfpathlineto{\pgfqpoint{5.315182in}{2.358714in}}%
\pgfpathlineto{\pgfqpoint{5.315675in}{2.376269in}}%
\pgfpathlineto{\pgfqpoint{5.317748in}{2.485413in}}%
\pgfpathlineto{\pgfqpoint{5.319031in}{2.651269in}}%
\pgfpathlineto{\pgfqpoint{5.319426in}{2.610024in}}%
\pgfpathlineto{\pgfqpoint{5.320216in}{2.510199in}}%
\pgfpathlineto{\pgfqpoint{5.320808in}{2.376008in}}%
\pgfpathlineto{\pgfqpoint{5.321597in}{2.408454in}}%
\pgfpathlineto{\pgfqpoint{5.321894in}{2.396827in}}%
\pgfpathlineto{\pgfqpoint{5.322486in}{2.418925in}}%
\pgfpathlineto{\pgfqpoint{5.324065in}{2.490385in}}%
\pgfpathlineto{\pgfqpoint{5.324361in}{2.470869in}}%
\pgfpathlineto{\pgfqpoint{5.325052in}{2.417936in}}%
\pgfpathlineto{\pgfqpoint{5.325743in}{2.435675in}}%
\pgfpathlineto{\pgfqpoint{5.326335in}{2.461138in}}%
\pgfpathlineto{\pgfqpoint{5.327125in}{2.477041in}}%
\pgfpathlineto{\pgfqpoint{5.327421in}{2.464094in}}%
\pgfpathlineto{\pgfqpoint{5.328309in}{2.431039in}}%
\pgfpathlineto{\pgfqpoint{5.328704in}{2.455128in}}%
\pgfpathlineto{\pgfqpoint{5.328802in}{2.456364in}}%
\pgfpathlineto{\pgfqpoint{5.329099in}{2.453278in}}%
\pgfpathlineto{\pgfqpoint{5.329493in}{2.454124in}}%
\pgfpathlineto{\pgfqpoint{5.329691in}{2.450024in}}%
\pgfpathlineto{\pgfqpoint{5.330086in}{2.461757in}}%
\pgfpathlineto{\pgfqpoint{5.330283in}{2.466597in}}%
\pgfpathlineto{\pgfqpoint{5.330776in}{2.460248in}}%
\pgfpathlineto{\pgfqpoint{5.331270in}{2.465562in}}%
\pgfpathlineto{\pgfqpoint{5.331763in}{2.471150in}}%
\pgfpathlineto{\pgfqpoint{5.332454in}{2.467793in}}%
\pgfpathlineto{\pgfqpoint{5.332849in}{2.461499in}}%
\pgfpathlineto{\pgfqpoint{5.333145in}{2.471711in}}%
\pgfpathlineto{\pgfqpoint{5.333244in}{2.473939in}}%
\pgfpathlineto{\pgfqpoint{5.333737in}{2.465089in}}%
\pgfpathlineto{\pgfqpoint{5.334033in}{2.466962in}}%
\pgfpathlineto{\pgfqpoint{5.334527in}{2.473680in}}%
\pgfpathlineto{\pgfqpoint{5.334724in}{2.464838in}}%
\pgfpathlineto{\pgfqpoint{5.335020in}{2.450485in}}%
\pgfpathlineto{\pgfqpoint{5.335415in}{2.466768in}}%
\pgfpathlineto{\pgfqpoint{5.335810in}{2.459604in}}%
\pgfpathlineto{\pgfqpoint{5.336304in}{2.471155in}}%
\pgfpathlineto{\pgfqpoint{5.336402in}{2.472580in}}%
\pgfpathlineto{\pgfqpoint{5.336698in}{2.463144in}}%
\pgfpathlineto{\pgfqpoint{5.337784in}{2.447456in}}%
\pgfpathlineto{\pgfqpoint{5.337389in}{2.470464in}}%
\pgfpathlineto{\pgfqpoint{5.337883in}{2.448345in}}%
\pgfpathlineto{\pgfqpoint{5.338179in}{2.468666in}}%
\pgfpathlineto{\pgfqpoint{5.339067in}{2.457486in}}%
\pgfpathlineto{\pgfqpoint{5.339363in}{2.442259in}}%
\pgfpathlineto{\pgfqpoint{5.340252in}{2.444937in}}%
\pgfpathlineto{\pgfqpoint{5.340449in}{2.446097in}}%
\pgfpathlineto{\pgfqpoint{5.340745in}{2.442710in}}%
\pgfpathlineto{\pgfqpoint{5.341140in}{2.442772in}}%
\pgfpathlineto{\pgfqpoint{5.343212in}{2.417168in}}%
\pgfpathlineto{\pgfqpoint{5.343509in}{2.430484in}}%
\pgfpathlineto{\pgfqpoint{5.343706in}{2.437613in}}%
\pgfpathlineto{\pgfqpoint{5.344199in}{2.409563in}}%
\pgfpathlineto{\pgfqpoint{5.344298in}{2.408545in}}%
\pgfpathlineto{\pgfqpoint{5.344989in}{2.412674in}}%
\pgfpathlineto{\pgfqpoint{5.345088in}{2.412983in}}%
\pgfpathlineto{\pgfqpoint{5.345285in}{2.410474in}}%
\pgfpathlineto{\pgfqpoint{5.346667in}{2.400351in}}%
\pgfpathlineto{\pgfqpoint{5.345877in}{2.412545in}}%
\pgfpathlineto{\pgfqpoint{5.346766in}{2.401100in}}%
\pgfpathlineto{\pgfqpoint{5.346864in}{2.402167in}}%
\pgfpathlineto{\pgfqpoint{5.347160in}{2.396025in}}%
\pgfpathlineto{\pgfqpoint{5.347358in}{2.389822in}}%
\pgfpathlineto{\pgfqpoint{5.348049in}{2.399222in}}%
\pgfpathlineto{\pgfqpoint{5.348246in}{2.405935in}}%
\pgfpathlineto{\pgfqpoint{5.348740in}{2.387475in}}%
\pgfpathlineto{\pgfqpoint{5.349233in}{2.402200in}}%
\pgfpathlineto{\pgfqpoint{5.349727in}{2.388167in}}%
\pgfpathlineto{\pgfqpoint{5.350220in}{2.404143in}}%
\pgfpathlineto{\pgfqpoint{5.351108in}{2.408242in}}%
\pgfpathlineto{\pgfqpoint{5.350714in}{2.400369in}}%
\pgfpathlineto{\pgfqpoint{5.351306in}{2.404990in}}%
\pgfpathlineto{\pgfqpoint{5.351602in}{2.397370in}}%
\pgfpathlineto{\pgfqpoint{5.352194in}{2.411390in}}%
\pgfpathlineto{\pgfqpoint{5.354859in}{2.460825in}}%
\pgfpathlineto{\pgfqpoint{5.355254in}{2.446482in}}%
\pgfpathlineto{\pgfqpoint{5.357623in}{2.378396in}}%
\pgfpathlineto{\pgfqpoint{5.358017in}{2.389371in}}%
\pgfpathlineto{\pgfqpoint{5.358412in}{2.391712in}}%
\pgfpathlineto{\pgfqpoint{5.358807in}{2.384247in}}%
\pgfpathlineto{\pgfqpoint{5.359794in}{2.354748in}}%
\pgfpathlineto{\pgfqpoint{5.360386in}{2.357872in}}%
\pgfpathlineto{\pgfqpoint{5.360485in}{2.357810in}}%
\pgfpathlineto{\pgfqpoint{5.361373in}{2.404273in}}%
\pgfpathlineto{\pgfqpoint{5.361965in}{2.417360in}}%
\pgfpathlineto{\pgfqpoint{5.362163in}{2.404918in}}%
\pgfpathlineto{\pgfqpoint{5.363643in}{2.347869in}}%
\pgfpathlineto{\pgfqpoint{5.363742in}{2.346787in}}%
\pgfpathlineto{\pgfqpoint{5.363939in}{2.350065in}}%
\pgfpathlineto{\pgfqpoint{5.365420in}{2.498362in}}%
\pgfpathlineto{\pgfqpoint{5.366505in}{2.632833in}}%
\pgfpathlineto{\pgfqpoint{5.366802in}{2.603087in}}%
\pgfpathlineto{\pgfqpoint{5.368183in}{2.353755in}}%
\pgfpathlineto{\pgfqpoint{5.369565in}{2.383509in}}%
\pgfpathlineto{\pgfqpoint{5.371243in}{2.454506in}}%
\pgfpathlineto{\pgfqpoint{5.371440in}{2.453058in}}%
\pgfpathlineto{\pgfqpoint{5.371835in}{2.433498in}}%
\pgfpathlineto{\pgfqpoint{5.372329in}{2.395203in}}%
\pgfpathlineto{\pgfqpoint{5.373020in}{2.420577in}}%
\pgfpathlineto{\pgfqpoint{5.374105in}{2.452143in}}%
\pgfpathlineto{\pgfqpoint{5.374895in}{2.435275in}}%
\pgfpathlineto{\pgfqpoint{5.375783in}{2.405817in}}%
\pgfpathlineto{\pgfqpoint{5.376079in}{2.430707in}}%
\pgfpathlineto{\pgfqpoint{5.377560in}{2.456990in}}%
\pgfpathlineto{\pgfqpoint{5.376573in}{2.417629in}}%
\pgfpathlineto{\pgfqpoint{5.377757in}{2.448162in}}%
\pgfpathlineto{\pgfqpoint{5.377856in}{2.443476in}}%
\pgfpathlineto{\pgfqpoint{5.378349in}{2.468978in}}%
\pgfpathlineto{\pgfqpoint{5.378645in}{2.456875in}}%
\pgfpathlineto{\pgfqpoint{5.379238in}{2.454663in}}%
\pgfpathlineto{\pgfqpoint{5.378941in}{2.459606in}}%
\pgfpathlineto{\pgfqpoint{5.379435in}{2.457570in}}%
\pgfpathlineto{\pgfqpoint{5.380126in}{2.469420in}}%
\pgfpathlineto{\pgfqpoint{5.380817in}{2.469002in}}%
\pgfpathlineto{\pgfqpoint{5.381705in}{2.475885in}}%
\pgfpathlineto{\pgfqpoint{5.381212in}{2.464163in}}%
\pgfpathlineto{\pgfqpoint{5.382001in}{2.470997in}}%
\pgfpathlineto{\pgfqpoint{5.382100in}{2.470492in}}%
\pgfpathlineto{\pgfqpoint{5.382297in}{2.473350in}}%
\pgfpathlineto{\pgfqpoint{5.382593in}{2.477944in}}%
\pgfpathlineto{\pgfqpoint{5.382988in}{2.463390in}}%
\pgfpathlineto{\pgfqpoint{5.383087in}{2.461858in}}%
\pgfpathlineto{\pgfqpoint{5.383383in}{2.471635in}}%
\pgfpathlineto{\pgfqpoint{5.384370in}{2.475680in}}%
\pgfpathlineto{\pgfqpoint{5.383876in}{2.460261in}}%
\pgfpathlineto{\pgfqpoint{5.384469in}{2.472737in}}%
\pgfpathlineto{\pgfqpoint{5.384666in}{2.466285in}}%
\pgfpathlineto{\pgfqpoint{5.385160in}{2.479036in}}%
\pgfpathlineto{\pgfqpoint{5.385357in}{2.478526in}}%
\pgfpathlineto{\pgfqpoint{5.385456in}{2.477902in}}%
\pgfpathlineto{\pgfqpoint{5.385653in}{2.480529in}}%
\pgfpathlineto{\pgfqpoint{5.386147in}{2.489181in}}%
\pgfpathlineto{\pgfqpoint{5.386739in}{2.481587in}}%
\pgfpathlineto{\pgfqpoint{5.387923in}{2.473325in}}%
\pgfpathlineto{\pgfqpoint{5.387430in}{2.488408in}}%
\pgfpathlineto{\pgfqpoint{5.388022in}{2.475497in}}%
\pgfpathlineto{\pgfqpoint{5.388219in}{2.480589in}}%
\pgfpathlineto{\pgfqpoint{5.388713in}{2.463315in}}%
\pgfpathlineto{\pgfqpoint{5.389107in}{2.452479in}}%
\pgfpathlineto{\pgfqpoint{5.390687in}{2.395436in}}%
\pgfpathlineto{\pgfqpoint{5.390884in}{2.398316in}}%
\pgfpathlineto{\pgfqpoint{5.393746in}{2.459650in}}%
\pgfpathlineto{\pgfqpoint{5.394141in}{2.450093in}}%
\pgfpathlineto{\pgfqpoint{5.394339in}{2.451797in}}%
\pgfpathlineto{\pgfqpoint{5.394635in}{2.445573in}}%
\pgfpathlineto{\pgfqpoint{5.395128in}{2.433373in}}%
\pgfpathlineto{\pgfqpoint{5.395622in}{2.446598in}}%
\pgfpathlineto{\pgfqpoint{5.395720in}{2.446871in}}%
\pgfpathlineto{\pgfqpoint{5.395819in}{2.444289in}}%
\pgfpathlineto{\pgfqpoint{5.396905in}{2.430224in}}%
\pgfpathlineto{\pgfqpoint{5.397102in}{2.434674in}}%
\pgfpathlineto{\pgfqpoint{5.397398in}{2.443512in}}%
\pgfpathlineto{\pgfqpoint{5.398188in}{2.437191in}}%
\pgfpathlineto{\pgfqpoint{5.398385in}{2.437458in}}%
\pgfpathlineto{\pgfqpoint{5.398583in}{2.435343in}}%
\pgfpathlineto{\pgfqpoint{5.398780in}{2.433298in}}%
\pgfpathlineto{\pgfqpoint{5.399175in}{2.439135in}}%
\pgfpathlineto{\pgfqpoint{5.399471in}{2.436720in}}%
\pgfpathlineto{\pgfqpoint{5.401741in}{2.484006in}}%
\pgfpathlineto{\pgfqpoint{5.402925in}{2.457381in}}%
\pgfpathlineto{\pgfqpoint{5.403320in}{2.467890in}}%
\pgfpathlineto{\pgfqpoint{5.403419in}{2.468836in}}%
\pgfpathlineto{\pgfqpoint{5.403616in}{2.462022in}}%
\pgfpathlineto{\pgfqpoint{5.405195in}{2.423739in}}%
\pgfpathlineto{\pgfqpoint{5.405294in}{2.424349in}}%
\pgfpathlineto{\pgfqpoint{5.405689in}{2.444038in}}%
\pgfpathlineto{\pgfqpoint{5.406182in}{2.418258in}}%
\pgfpathlineto{\pgfqpoint{5.407663in}{2.390564in}}%
\pgfpathlineto{\pgfqpoint{5.407959in}{2.401005in}}%
\pgfpathlineto{\pgfqpoint{5.408946in}{2.455168in}}%
\pgfpathlineto{\pgfqpoint{5.409341in}{2.433608in}}%
\pgfpathlineto{\pgfqpoint{5.410821in}{2.395646in}}%
\pgfpathlineto{\pgfqpoint{5.411019in}{2.394519in}}%
\pgfpathlineto{\pgfqpoint{5.411216in}{2.399190in}}%
\pgfpathlineto{\pgfqpoint{5.412795in}{2.588867in}}%
\pgfpathlineto{\pgfqpoint{5.413585in}{2.668650in}}%
\pgfpathlineto{\pgfqpoint{5.414078in}{2.625513in}}%
\pgfpathlineto{\pgfqpoint{5.414572in}{2.588136in}}%
\pgfpathlineto{\pgfqpoint{5.415361in}{2.379125in}}%
\pgfpathlineto{\pgfqpoint{5.416348in}{2.404575in}}%
\pgfpathlineto{\pgfqpoint{5.416546in}{2.401667in}}%
\pgfpathlineto{\pgfqpoint{5.416842in}{2.413612in}}%
\pgfpathlineto{\pgfqpoint{5.418520in}{2.482787in}}%
\pgfpathlineto{\pgfqpoint{5.418717in}{2.477212in}}%
\pgfpathlineto{\pgfqpoint{5.419408in}{2.428888in}}%
\pgfpathlineto{\pgfqpoint{5.420198in}{2.439576in}}%
\pgfpathlineto{\pgfqpoint{5.421678in}{2.487748in}}%
\pgfpathlineto{\pgfqpoint{5.422270in}{2.476395in}}%
\pgfpathlineto{\pgfqpoint{5.422764in}{2.441066in}}%
\pgfpathlineto{\pgfqpoint{5.423652in}{2.463740in}}%
\pgfpathlineto{\pgfqpoint{5.424837in}{2.477386in}}%
\pgfpathlineto{\pgfqpoint{5.424343in}{2.461236in}}%
\pgfpathlineto{\pgfqpoint{5.425034in}{2.474743in}}%
\pgfpathlineto{\pgfqpoint{5.425330in}{2.485717in}}%
\pgfpathlineto{\pgfqpoint{5.425527in}{2.492886in}}%
\pgfpathlineto{\pgfqpoint{5.426416in}{2.486913in}}%
\pgfpathlineto{\pgfqpoint{5.426514in}{2.485755in}}%
\pgfpathlineto{\pgfqpoint{5.426810in}{2.493616in}}%
\pgfpathlineto{\pgfqpoint{5.427896in}{2.516332in}}%
\pgfpathlineto{\pgfqpoint{5.428094in}{2.506191in}}%
\pgfpathlineto{\pgfqpoint{5.428291in}{2.496907in}}%
\pgfpathlineto{\pgfqpoint{5.428784in}{2.521607in}}%
\pgfpathlineto{\pgfqpoint{5.429081in}{2.511356in}}%
\pgfpathlineto{\pgfqpoint{5.429179in}{2.508430in}}%
\pgfpathlineto{\pgfqpoint{5.429673in}{2.524006in}}%
\pgfpathlineto{\pgfqpoint{5.429771in}{2.524637in}}%
\pgfpathlineto{\pgfqpoint{5.429870in}{2.522315in}}%
\pgfpathlineto{\pgfqpoint{5.430265in}{2.511809in}}%
\pgfpathlineto{\pgfqpoint{5.430758in}{2.525865in}}%
\pgfpathlineto{\pgfqpoint{5.430857in}{2.525816in}}%
\pgfpathlineto{\pgfqpoint{5.431055in}{2.524392in}}%
\pgfpathlineto{\pgfqpoint{5.431351in}{2.529713in}}%
\pgfpathlineto{\pgfqpoint{5.431548in}{2.533303in}}%
\pgfpathlineto{\pgfqpoint{5.431943in}{2.519539in}}%
\pgfpathlineto{\pgfqpoint{5.432042in}{2.517462in}}%
\pgfpathlineto{\pgfqpoint{5.432436in}{2.528410in}}%
\pgfpathlineto{\pgfqpoint{5.432732in}{2.524186in}}%
\pgfpathlineto{\pgfqpoint{5.432930in}{2.526814in}}%
\pgfpathlineto{\pgfqpoint{5.433226in}{2.534013in}}%
\pgfpathlineto{\pgfqpoint{5.433818in}{2.519457in}}%
\pgfpathlineto{\pgfqpoint{5.434904in}{2.512661in}}%
\pgfpathlineto{\pgfqpoint{5.434312in}{2.521796in}}%
\pgfpathlineto{\pgfqpoint{5.435101in}{2.516239in}}%
\pgfpathlineto{\pgfqpoint{5.435397in}{2.524824in}}%
\pgfpathlineto{\pgfqpoint{5.435792in}{2.505925in}}%
\pgfpathlineto{\pgfqpoint{5.436779in}{2.493050in}}%
\pgfpathlineto{\pgfqpoint{5.436976in}{2.495353in}}%
\pgfpathlineto{\pgfqpoint{5.437075in}{2.496475in}}%
\pgfpathlineto{\pgfqpoint{5.437470in}{2.489289in}}%
\pgfpathlineto{\pgfqpoint{5.440628in}{2.427398in}}%
\pgfpathlineto{\pgfqpoint{5.441023in}{2.441869in}}%
\pgfpathlineto{\pgfqpoint{5.441122in}{2.444818in}}%
\pgfpathlineto{\pgfqpoint{5.441615in}{2.433617in}}%
\pgfpathlineto{\pgfqpoint{5.441911in}{2.438853in}}%
\pgfpathlineto{\pgfqpoint{5.443293in}{2.428346in}}%
\pgfpathlineto{\pgfqpoint{5.448820in}{2.518165in}}%
\pgfpathlineto{\pgfqpoint{5.449709in}{2.511647in}}%
\pgfpathlineto{\pgfqpoint{5.451979in}{2.444530in}}%
\pgfpathlineto{\pgfqpoint{5.452275in}{2.450375in}}%
\pgfpathlineto{\pgfqpoint{5.452472in}{2.453471in}}%
\pgfpathlineto{\pgfqpoint{5.452768in}{2.447533in}}%
\pgfpathlineto{\pgfqpoint{5.453163in}{2.448999in}}%
\pgfpathlineto{\pgfqpoint{5.454545in}{2.405795in}}%
\pgfpathlineto{\pgfqpoint{5.454940in}{2.407230in}}%
\pgfpathlineto{\pgfqpoint{5.455236in}{2.417124in}}%
\pgfpathlineto{\pgfqpoint{5.456025in}{2.461316in}}%
\pgfpathlineto{\pgfqpoint{5.456420in}{2.425261in}}%
\pgfpathlineto{\pgfqpoint{5.457999in}{2.375828in}}%
\pgfpathlineto{\pgfqpoint{5.458098in}{2.376017in}}%
\pgfpathlineto{\pgfqpoint{5.459480in}{2.491025in}}%
\pgfpathlineto{\pgfqpoint{5.460862in}{2.646880in}}%
\pgfpathlineto{\pgfqpoint{5.461158in}{2.616120in}}%
\pgfpathlineto{\pgfqpoint{5.462540in}{2.375233in}}%
\pgfpathlineto{\pgfqpoint{5.464119in}{2.393266in}}%
\pgfpathlineto{\pgfqpoint{5.464908in}{2.395768in}}%
\pgfpathlineto{\pgfqpoint{5.464612in}{2.392043in}}%
\pgfpathlineto{\pgfqpoint{5.465106in}{2.393396in}}%
\pgfpathlineto{\pgfqpoint{5.465303in}{2.390827in}}%
\pgfpathlineto{\pgfqpoint{5.465599in}{2.402709in}}%
\pgfpathlineto{\pgfqpoint{5.465698in}{2.404510in}}%
\pgfpathlineto{\pgfqpoint{5.465895in}{2.394050in}}%
\pgfpathlineto{\pgfqpoint{5.466586in}{2.340463in}}%
\pgfpathlineto{\pgfqpoint{5.467080in}{2.369497in}}%
\pgfpathlineto{\pgfqpoint{5.468856in}{2.478121in}}%
\pgfpathlineto{\pgfqpoint{5.469350in}{2.457739in}}%
\pgfpathlineto{\pgfqpoint{5.470633in}{2.434385in}}%
\pgfpathlineto{\pgfqpoint{5.471422in}{2.439383in}}%
\pgfpathlineto{\pgfqpoint{5.472212in}{2.454961in}}%
\pgfpathlineto{\pgfqpoint{5.473100in}{2.461537in}}%
\pgfpathlineto{\pgfqpoint{5.472705in}{2.453826in}}%
\pgfpathlineto{\pgfqpoint{5.473298in}{2.456729in}}%
\pgfpathlineto{\pgfqpoint{5.473594in}{2.445469in}}%
\pgfpathlineto{\pgfqpoint{5.474087in}{2.465291in}}%
\pgfpathlineto{\pgfqpoint{5.474383in}{2.456699in}}%
\pgfpathlineto{\pgfqpoint{5.474482in}{2.455979in}}%
\pgfpathlineto{\pgfqpoint{5.474679in}{2.460204in}}%
\pgfpathlineto{\pgfqpoint{5.475074in}{2.467584in}}%
\pgfpathlineto{\pgfqpoint{5.475568in}{2.455102in}}%
\pgfpathlineto{\pgfqpoint{5.475666in}{2.453211in}}%
\pgfpathlineto{\pgfqpoint{5.475963in}{2.462182in}}%
\pgfpathlineto{\pgfqpoint{5.477048in}{2.473920in}}%
\pgfpathlineto{\pgfqpoint{5.476555in}{2.454307in}}%
\pgfpathlineto{\pgfqpoint{5.477147in}{2.471015in}}%
\pgfpathlineto{\pgfqpoint{5.477542in}{2.453161in}}%
\pgfpathlineto{\pgfqpoint{5.478233in}{2.465134in}}%
\pgfpathlineto{\pgfqpoint{5.479022in}{2.471182in}}%
\pgfpathlineto{\pgfqpoint{5.478726in}{2.462704in}}%
\pgfpathlineto{\pgfqpoint{5.479220in}{2.468035in}}%
\pgfpathlineto{\pgfqpoint{5.479417in}{2.462064in}}%
\pgfpathlineto{\pgfqpoint{5.479812in}{2.469283in}}%
\pgfpathlineto{\pgfqpoint{5.480207in}{2.468187in}}%
\pgfpathlineto{\pgfqpoint{5.481095in}{2.473294in}}%
\pgfpathlineto{\pgfqpoint{5.480700in}{2.459321in}}%
\pgfpathlineto{\pgfqpoint{5.481194in}{2.470374in}}%
\pgfpathlineto{\pgfqpoint{5.481490in}{2.457146in}}%
\pgfpathlineto{\pgfqpoint{5.482279in}{2.466808in}}%
\pgfpathlineto{\pgfqpoint{5.482477in}{2.471408in}}%
\pgfpathlineto{\pgfqpoint{5.483168in}{2.462935in}}%
\pgfpathlineto{\pgfqpoint{5.483562in}{2.449792in}}%
\pgfpathlineto{\pgfqpoint{5.484155in}{2.464239in}}%
\pgfpathlineto{\pgfqpoint{5.484253in}{2.465167in}}%
\pgfpathlineto{\pgfqpoint{5.484549in}{2.458906in}}%
\pgfpathlineto{\pgfqpoint{5.485931in}{2.422802in}}%
\pgfpathlineto{\pgfqpoint{5.486326in}{2.437521in}}%
\pgfpathlineto{\pgfqpoint{5.486425in}{2.438519in}}%
\pgfpathlineto{\pgfqpoint{5.486622in}{2.431833in}}%
\pgfpathlineto{\pgfqpoint{5.487905in}{2.405013in}}%
\pgfpathlineto{\pgfqpoint{5.488103in}{2.407331in}}%
\pgfpathlineto{\pgfqpoint{5.488497in}{2.419625in}}%
\pgfpathlineto{\pgfqpoint{5.488991in}{2.405128in}}%
\pgfpathlineto{\pgfqpoint{5.490077in}{2.399503in}}%
\pgfpathlineto{\pgfqpoint{5.490175in}{2.399856in}}%
\pgfpathlineto{\pgfqpoint{5.490767in}{2.406054in}}%
\pgfpathlineto{\pgfqpoint{5.491064in}{2.400063in}}%
\pgfpathlineto{\pgfqpoint{5.492050in}{2.393851in}}%
\pgfpathlineto{\pgfqpoint{5.491656in}{2.404592in}}%
\pgfpathlineto{\pgfqpoint{5.492149in}{2.397093in}}%
\pgfpathlineto{\pgfqpoint{5.492445in}{2.414665in}}%
\pgfpathlineto{\pgfqpoint{5.492939in}{2.393182in}}%
\pgfpathlineto{\pgfqpoint{5.493334in}{2.405278in}}%
\pgfpathlineto{\pgfqpoint{5.493531in}{2.403020in}}%
\pgfpathlineto{\pgfqpoint{5.494024in}{2.390719in}}%
\pgfpathlineto{\pgfqpoint{5.494617in}{2.401845in}}%
\pgfpathlineto{\pgfqpoint{5.496788in}{2.442956in}}%
\pgfpathlineto{\pgfqpoint{5.497084in}{2.437468in}}%
\pgfpathlineto{\pgfqpoint{5.497183in}{2.436658in}}%
\pgfpathlineto{\pgfqpoint{5.497282in}{2.439400in}}%
\pgfpathlineto{\pgfqpoint{5.497578in}{2.452626in}}%
\pgfpathlineto{\pgfqpoint{5.497972in}{2.421465in}}%
\pgfpathlineto{\pgfqpoint{5.499650in}{2.379712in}}%
\pgfpathlineto{\pgfqpoint{5.498367in}{2.425344in}}%
\pgfpathlineto{\pgfqpoint{5.500144in}{2.391218in}}%
\pgfpathlineto{\pgfqpoint{5.500440in}{2.399131in}}%
\pgfpathlineto{\pgfqpoint{5.501032in}{2.386915in}}%
\pgfpathlineto{\pgfqpoint{5.501427in}{2.374167in}}%
\pgfpathlineto{\pgfqpoint{5.501920in}{2.358619in}}%
\pgfpathlineto{\pgfqpoint{5.502710in}{2.362847in}}%
\pgfpathlineto{\pgfqpoint{5.502907in}{2.367831in}}%
\pgfpathlineto{\pgfqpoint{5.503796in}{2.436516in}}%
\pgfpathlineto{\pgfqpoint{5.504289in}{2.403786in}}%
\pgfpathlineto{\pgfqpoint{5.505770in}{2.366007in}}%
\pgfpathlineto{\pgfqpoint{5.506855in}{2.415158in}}%
\pgfpathlineto{\pgfqpoint{5.508533in}{2.650948in}}%
\pgfpathlineto{\pgfqpoint{5.509422in}{2.585740in}}%
\pgfpathlineto{\pgfqpoint{5.510211in}{2.373841in}}%
\pgfpathlineto{\pgfqpoint{5.511692in}{2.402619in}}%
\pgfpathlineto{\pgfqpoint{5.513369in}{2.470225in}}%
\pgfpathlineto{\pgfqpoint{5.513567in}{2.465527in}}%
\pgfpathlineto{\pgfqpoint{5.514751in}{2.407086in}}%
\pgfpathlineto{\pgfqpoint{5.515146in}{2.429799in}}%
\pgfpathlineto{\pgfqpoint{5.516725in}{2.468869in}}%
\pgfpathlineto{\pgfqpoint{5.516824in}{2.467656in}}%
\pgfpathlineto{\pgfqpoint{5.517811in}{2.439189in}}%
\pgfpathlineto{\pgfqpoint{5.518206in}{2.455286in}}%
\pgfpathlineto{\pgfqpoint{5.518995in}{2.469728in}}%
\pgfpathlineto{\pgfqpoint{5.519291in}{2.459722in}}%
\pgfpathlineto{\pgfqpoint{5.519390in}{2.456414in}}%
\pgfpathlineto{\pgfqpoint{5.519686in}{2.471624in}}%
\pgfpathlineto{\pgfqpoint{5.519982in}{2.488964in}}%
\pgfpathlineto{\pgfqpoint{5.520377in}{2.470591in}}%
\pgfpathlineto{\pgfqpoint{5.520772in}{2.479352in}}%
\pgfpathlineto{\pgfqpoint{5.521759in}{2.471919in}}%
\pgfpathlineto{\pgfqpoint{5.521463in}{2.480455in}}%
\pgfpathlineto{\pgfqpoint{5.521956in}{2.475687in}}%
\pgfpathlineto{\pgfqpoint{5.522351in}{2.488833in}}%
\pgfpathlineto{\pgfqpoint{5.523141in}{2.481270in}}%
\pgfpathlineto{\pgfqpoint{5.523338in}{2.479026in}}%
\pgfpathlineto{\pgfqpoint{5.523733in}{2.487732in}}%
\pgfpathlineto{\pgfqpoint{5.524720in}{2.496893in}}%
\pgfpathlineto{\pgfqpoint{5.524226in}{2.483331in}}%
\pgfpathlineto{\pgfqpoint{5.524917in}{2.490714in}}%
\pgfpathlineto{\pgfqpoint{5.525213in}{2.482592in}}%
\pgfpathlineto{\pgfqpoint{5.526102in}{2.486790in}}%
\pgfpathlineto{\pgfqpoint{5.526496in}{2.482007in}}%
\pgfpathlineto{\pgfqpoint{5.526891in}{2.487362in}}%
\pgfpathlineto{\pgfqpoint{5.527187in}{2.485570in}}%
\pgfpathlineto{\pgfqpoint{5.527483in}{2.487869in}}%
\pgfpathlineto{\pgfqpoint{5.527681in}{2.484538in}}%
\pgfpathlineto{\pgfqpoint{5.527977in}{2.476894in}}%
\pgfpathlineto{\pgfqpoint{5.528470in}{2.490423in}}%
\pgfpathlineto{\pgfqpoint{5.528668in}{2.487273in}}%
\pgfpathlineto{\pgfqpoint{5.529161in}{2.481109in}}%
\pgfpathlineto{\pgfqpoint{5.529359in}{2.475707in}}%
\pgfpathlineto{\pgfqpoint{5.530148in}{2.481880in}}%
\pgfpathlineto{\pgfqpoint{5.530346in}{2.483705in}}%
\pgfpathlineto{\pgfqpoint{5.530642in}{2.475481in}}%
\pgfpathlineto{\pgfqpoint{5.531826in}{2.458561in}}%
\pgfpathlineto{\pgfqpoint{5.531333in}{2.477571in}}%
\pgfpathlineto{\pgfqpoint{5.532024in}{2.463987in}}%
\pgfpathlineto{\pgfqpoint{5.532320in}{2.477611in}}%
\pgfpathlineto{\pgfqpoint{5.533109in}{2.465549in}}%
\pgfpathlineto{\pgfqpoint{5.533998in}{2.455389in}}%
\pgfpathlineto{\pgfqpoint{5.534294in}{2.459603in}}%
\pgfpathlineto{\pgfqpoint{5.534491in}{2.464569in}}%
\pgfpathlineto{\pgfqpoint{5.534985in}{2.451702in}}%
\pgfpathlineto{\pgfqpoint{5.535379in}{2.464100in}}%
\pgfpathlineto{\pgfqpoint{5.536761in}{2.389866in}}%
\pgfpathlineto{\pgfqpoint{5.538242in}{2.359650in}}%
\pgfpathlineto{\pgfqpoint{5.538439in}{2.349839in}}%
\pgfpathlineto{\pgfqpoint{5.539031in}{2.378341in}}%
\pgfpathlineto{\pgfqpoint{5.540216in}{2.447713in}}%
\pgfpathlineto{\pgfqpoint{5.541005in}{2.434844in}}%
\pgfpathlineto{\pgfqpoint{5.541400in}{2.441728in}}%
\pgfpathlineto{\pgfqpoint{5.541696in}{2.434656in}}%
\pgfpathlineto{\pgfqpoint{5.541893in}{2.431508in}}%
\pgfpathlineto{\pgfqpoint{5.542190in}{2.449673in}}%
\pgfpathlineto{\pgfqpoint{5.542782in}{2.438962in}}%
\pgfpathlineto{\pgfqpoint{5.543374in}{2.455454in}}%
\pgfpathlineto{\pgfqpoint{5.543670in}{2.449718in}}%
\pgfpathlineto{\pgfqpoint{5.543966in}{2.462015in}}%
\pgfpathlineto{\pgfqpoint{5.544164in}{2.470538in}}%
\pgfpathlineto{\pgfqpoint{5.544558in}{2.454371in}}%
\pgfpathlineto{\pgfqpoint{5.544953in}{2.464719in}}%
\pgfpathlineto{\pgfqpoint{5.546828in}{2.404368in}}%
\pgfpathlineto{\pgfqpoint{5.546927in}{2.405646in}}%
\pgfpathlineto{\pgfqpoint{5.547914in}{2.427644in}}%
\pgfpathlineto{\pgfqpoint{5.548408in}{2.414464in}}%
\pgfpathlineto{\pgfqpoint{5.549691in}{2.390066in}}%
\pgfpathlineto{\pgfqpoint{5.550184in}{2.401425in}}%
\pgfpathlineto{\pgfqpoint{5.550776in}{2.435445in}}%
\pgfpathlineto{\pgfqpoint{5.551270in}{2.456046in}}%
\pgfpathlineto{\pgfqpoint{5.551566in}{2.430527in}}%
\pgfpathlineto{\pgfqpoint{5.552454in}{2.393641in}}%
\pgfpathlineto{\pgfqpoint{5.552948in}{2.395054in}}%
\pgfpathlineto{\pgfqpoint{5.553046in}{2.394516in}}%
\pgfpathlineto{\pgfqpoint{5.553145in}{2.396737in}}%
\pgfpathlineto{\pgfqpoint{5.554823in}{2.545826in}}%
\pgfpathlineto{\pgfqpoint{5.555514in}{2.668576in}}%
\pgfpathlineto{\pgfqpoint{5.556304in}{2.645226in}}%
\pgfpathlineto{\pgfqpoint{5.556994in}{2.548728in}}%
\pgfpathlineto{\pgfqpoint{5.557685in}{2.383435in}}%
\pgfpathlineto{\pgfqpoint{5.558475in}{2.411434in}}%
\pgfpathlineto{\pgfqpoint{5.558771in}{2.407869in}}%
\pgfpathlineto{\pgfqpoint{5.559067in}{2.415896in}}%
\pgfpathlineto{\pgfqpoint{5.560745in}{2.476685in}}%
\pgfpathlineto{\pgfqpoint{5.560942in}{2.470923in}}%
\pgfpathlineto{\pgfqpoint{5.561929in}{2.403217in}}%
\pgfpathlineto{\pgfqpoint{5.562620in}{2.425200in}}%
\pgfpathlineto{\pgfqpoint{5.563410in}{2.438473in}}%
\pgfpathlineto{\pgfqpoint{5.563805in}{2.456033in}}%
\pgfpathlineto{\pgfqpoint{5.564496in}{2.438686in}}%
\pgfpathlineto{\pgfqpoint{5.564890in}{2.423384in}}%
\pgfpathlineto{\pgfqpoint{5.565088in}{2.415158in}}%
\pgfpathlineto{\pgfqpoint{5.565680in}{2.436146in}}%
\pgfpathlineto{\pgfqpoint{5.565779in}{2.436149in}}%
\pgfpathlineto{\pgfqpoint{5.567456in}{2.463190in}}%
\pgfpathlineto{\pgfqpoint{5.568246in}{2.457913in}}%
\pgfpathlineto{\pgfqpoint{5.568443in}{2.453295in}}%
\pgfpathlineto{\pgfqpoint{5.568838in}{2.472731in}}%
\pgfpathlineto{\pgfqpoint{5.569233in}{2.459382in}}%
\pgfpathlineto{\pgfqpoint{5.571207in}{2.485637in}}%
\pgfpathlineto{\pgfqpoint{5.571503in}{2.475628in}}%
\pgfpathlineto{\pgfqpoint{5.571701in}{2.471230in}}%
\pgfpathlineto{\pgfqpoint{5.572194in}{2.485258in}}%
\pgfpathlineto{\pgfqpoint{5.572293in}{2.485246in}}%
\pgfpathlineto{\pgfqpoint{5.572688in}{2.481368in}}%
\pgfpathlineto{\pgfqpoint{5.572885in}{2.478903in}}%
\pgfpathlineto{\pgfqpoint{5.573378in}{2.485933in}}%
\pgfpathlineto{\pgfqpoint{5.575254in}{2.502296in}}%
\pgfpathlineto{\pgfqpoint{5.574069in}{2.481293in}}%
\pgfpathlineto{\pgfqpoint{5.575352in}{2.500540in}}%
\pgfpathlineto{\pgfqpoint{5.576339in}{2.492376in}}%
\pgfpathlineto{\pgfqpoint{5.575945in}{2.502471in}}%
\pgfpathlineto{\pgfqpoint{5.576537in}{2.496882in}}%
\pgfpathlineto{\pgfqpoint{5.578017in}{2.517126in}}%
\pgfpathlineto{\pgfqpoint{5.578215in}{2.512975in}}%
\pgfpathlineto{\pgfqpoint{5.578511in}{2.503642in}}%
\pgfpathlineto{\pgfqpoint{5.579202in}{2.514217in}}%
\pgfpathlineto{\pgfqpoint{5.579300in}{2.513713in}}%
\pgfpathlineto{\pgfqpoint{5.579991in}{2.513753in}}%
\pgfpathlineto{\pgfqpoint{5.581472in}{2.492681in}}%
\pgfpathlineto{\pgfqpoint{5.581669in}{2.493490in}}%
\pgfpathlineto{\pgfqpoint{5.581768in}{2.494019in}}%
\pgfpathlineto{\pgfqpoint{5.581965in}{2.491842in}}%
\pgfpathlineto{\pgfqpoint{5.583248in}{2.470530in}}%
\pgfpathlineto{\pgfqpoint{5.583446in}{2.471433in}}%
\pgfpathlineto{\pgfqpoint{5.583643in}{2.473423in}}%
\pgfpathlineto{\pgfqpoint{5.583939in}{2.485961in}}%
\pgfpathlineto{\pgfqpoint{5.584334in}{2.463847in}}%
\pgfpathlineto{\pgfqpoint{5.584433in}{2.458765in}}%
\pgfpathlineto{\pgfqpoint{5.584926in}{2.470781in}}%
\pgfpathlineto{\pgfqpoint{5.585321in}{2.468398in}}%
\pgfpathlineto{\pgfqpoint{5.585814in}{2.460742in}}%
\pgfpathlineto{\pgfqpoint{5.586012in}{2.469199in}}%
\pgfpathlineto{\pgfqpoint{5.586209in}{2.477893in}}%
\pgfpathlineto{\pgfqpoint{5.586703in}{2.456047in}}%
\pgfpathlineto{\pgfqpoint{5.587098in}{2.468303in}}%
\pgfpathlineto{\pgfqpoint{5.587295in}{2.469308in}}%
\pgfpathlineto{\pgfqpoint{5.587788in}{2.464155in}}%
\pgfpathlineto{\pgfqpoint{5.588677in}{2.455728in}}%
\pgfpathlineto{\pgfqpoint{5.588973in}{2.462482in}}%
\pgfpathlineto{\pgfqpoint{5.592033in}{2.511022in}}%
\pgfpathlineto{\pgfqpoint{5.589466in}{2.457022in}}%
\pgfpathlineto{\pgfqpoint{5.592131in}{2.508450in}}%
\pgfpathlineto{\pgfqpoint{5.594204in}{2.444058in}}%
\pgfpathlineto{\pgfqpoint{5.594697in}{2.434211in}}%
\pgfpathlineto{\pgfqpoint{5.595388in}{2.443081in}}%
\pgfpathlineto{\pgfqpoint{5.595487in}{2.444523in}}%
\pgfpathlineto{\pgfqpoint{5.595783in}{2.436313in}}%
\pgfpathlineto{\pgfqpoint{5.596573in}{2.391653in}}%
\pgfpathlineto{\pgfqpoint{5.597560in}{2.396553in}}%
\pgfpathlineto{\pgfqpoint{5.598547in}{2.453953in}}%
\pgfpathlineto{\pgfqpoint{5.598941in}{2.424653in}}%
\pgfpathlineto{\pgfqpoint{5.600225in}{2.393617in}}%
\pgfpathlineto{\pgfqpoint{5.600422in}{2.393866in}}%
\pgfpathlineto{\pgfqpoint{5.600915in}{2.400844in}}%
\pgfpathlineto{\pgfqpoint{5.602199in}{2.526708in}}%
\pgfpathlineto{\pgfqpoint{5.603284in}{2.669439in}}%
\pgfpathlineto{\pgfqpoint{5.603876in}{2.633202in}}%
\pgfpathlineto{\pgfqpoint{5.604271in}{2.598714in}}%
\pgfpathlineto{\pgfqpoint{5.606245in}{2.368286in}}%
\pgfpathlineto{\pgfqpoint{5.606344in}{2.367357in}}%
\pgfpathlineto{\pgfqpoint{5.606541in}{2.375124in}}%
\pgfpathlineto{\pgfqpoint{5.608417in}{2.454950in}}%
\pgfpathlineto{\pgfqpoint{5.607232in}{2.371904in}}%
\pgfpathlineto{\pgfqpoint{5.608811in}{2.437936in}}%
\pgfpathlineto{\pgfqpoint{5.609305in}{2.419449in}}%
\pgfpathlineto{\pgfqpoint{5.609700in}{2.449590in}}%
\pgfpathlineto{\pgfqpoint{5.611180in}{2.483426in}}%
\pgfpathlineto{\pgfqpoint{5.610094in}{2.446610in}}%
\pgfpathlineto{\pgfqpoint{5.611378in}{2.478826in}}%
\pgfpathlineto{\pgfqpoint{5.612463in}{2.438684in}}%
\pgfpathlineto{\pgfqpoint{5.612759in}{2.454964in}}%
\pgfpathlineto{\pgfqpoint{5.614141in}{2.489695in}}%
\pgfpathlineto{\pgfqpoint{5.614240in}{2.486205in}}%
\pgfpathlineto{\pgfqpoint{5.614536in}{2.469462in}}%
\pgfpathlineto{\pgfqpoint{5.614931in}{2.494817in}}%
\pgfpathlineto{\pgfqpoint{5.615325in}{2.479853in}}%
\pgfpathlineto{\pgfqpoint{5.618286in}{2.525544in}}%
\pgfpathlineto{\pgfqpoint{5.618681in}{2.516140in}}%
\pgfpathlineto{\pgfqpoint{5.618977in}{2.526666in}}%
\pgfpathlineto{\pgfqpoint{5.619668in}{2.519231in}}%
\pgfpathlineto{\pgfqpoint{5.620162in}{2.535413in}}%
\pgfpathlineto{\pgfqpoint{5.622234in}{2.522457in}}%
\pgfpathlineto{\pgfqpoint{5.622925in}{2.523206in}}%
\pgfpathlineto{\pgfqpoint{5.623123in}{2.523465in}}%
\pgfpathlineto{\pgfqpoint{5.623221in}{2.522747in}}%
\pgfpathlineto{\pgfqpoint{5.623517in}{2.518941in}}%
\pgfpathlineto{\pgfqpoint{5.624208in}{2.523721in}}%
\pgfpathlineto{\pgfqpoint{5.625097in}{2.531964in}}%
\pgfpathlineto{\pgfqpoint{5.625294in}{2.526105in}}%
\pgfpathlineto{\pgfqpoint{5.626380in}{2.512743in}}%
\pgfpathlineto{\pgfqpoint{5.625886in}{2.534852in}}%
\pgfpathlineto{\pgfqpoint{5.626577in}{2.514252in}}%
\pgfpathlineto{\pgfqpoint{5.627169in}{2.515777in}}%
\pgfpathlineto{\pgfqpoint{5.626873in}{2.513443in}}%
\pgfpathlineto{\pgfqpoint{5.627268in}{2.515398in}}%
\pgfpathlineto{\pgfqpoint{5.629341in}{2.468858in}}%
\pgfpathlineto{\pgfqpoint{5.629637in}{2.477410in}}%
\pgfpathlineto{\pgfqpoint{5.629736in}{2.479126in}}%
\pgfpathlineto{\pgfqpoint{5.630032in}{2.467554in}}%
\pgfpathlineto{\pgfqpoint{5.630920in}{2.458018in}}%
\pgfpathlineto{\pgfqpoint{5.631216in}{2.460441in}}%
\pgfpathlineto{\pgfqpoint{5.631413in}{2.458065in}}%
\pgfpathlineto{\pgfqpoint{5.632302in}{2.451542in}}%
\pgfpathlineto{\pgfqpoint{5.632598in}{2.454615in}}%
\pgfpathlineto{\pgfqpoint{5.633486in}{2.457300in}}%
\pgfpathlineto{\pgfqpoint{5.633091in}{2.451969in}}%
\pgfpathlineto{\pgfqpoint{5.633585in}{2.456240in}}%
\pgfpathlineto{\pgfqpoint{5.634572in}{2.447779in}}%
\pgfpathlineto{\pgfqpoint{5.634177in}{2.460837in}}%
\pgfpathlineto{\pgfqpoint{5.634670in}{2.449574in}}%
\pgfpathlineto{\pgfqpoint{5.634967in}{2.465445in}}%
\pgfpathlineto{\pgfqpoint{5.635954in}{2.462612in}}%
\pgfpathlineto{\pgfqpoint{5.636743in}{2.459275in}}%
\pgfpathlineto{\pgfqpoint{5.636348in}{2.464310in}}%
\pgfpathlineto{\pgfqpoint{5.636842in}{2.460626in}}%
\pgfpathlineto{\pgfqpoint{5.639408in}{2.517634in}}%
\pgfpathlineto{\pgfqpoint{5.639507in}{2.516296in}}%
\pgfpathlineto{\pgfqpoint{5.642172in}{2.437734in}}%
\pgfpathlineto{\pgfqpoint{5.640099in}{2.518176in}}%
\pgfpathlineto{\pgfqpoint{5.642764in}{2.450896in}}%
\pgfpathlineto{\pgfqpoint{5.643257in}{2.455765in}}%
\pgfpathlineto{\pgfqpoint{5.643356in}{2.456360in}}%
\pgfpathlineto{\pgfqpoint{5.643553in}{2.453868in}}%
\pgfpathlineto{\pgfqpoint{5.645133in}{2.411083in}}%
\pgfpathlineto{\pgfqpoint{5.645527in}{2.422171in}}%
\pgfpathlineto{\pgfqpoint{5.646613in}{2.475734in}}%
\pgfpathlineto{\pgfqpoint{5.647008in}{2.452003in}}%
\pgfpathlineto{\pgfqpoint{5.647797in}{2.427351in}}%
\pgfpathlineto{\pgfqpoint{5.648291in}{2.428530in}}%
\pgfpathlineto{\pgfqpoint{5.648488in}{2.423739in}}%
\pgfpathlineto{\pgfqpoint{5.648883in}{2.445598in}}%
\pgfpathlineto{\pgfqpoint{5.649673in}{2.481245in}}%
\pgfpathlineto{\pgfqpoint{5.651153in}{2.688146in}}%
\pgfpathlineto{\pgfqpoint{5.651647in}{2.648958in}}%
\pgfpathlineto{\pgfqpoint{5.652239in}{2.590840in}}%
\pgfpathlineto{\pgfqpoint{5.652930in}{2.405759in}}%
\pgfpathlineto{\pgfqpoint{5.653818in}{2.437077in}}%
\pgfpathlineto{\pgfqpoint{5.653917in}{2.435746in}}%
\pgfpathlineto{\pgfqpoint{5.654213in}{2.444486in}}%
\pgfpathlineto{\pgfqpoint{5.656088in}{2.510852in}}%
\pgfpathlineto{\pgfqpoint{5.656187in}{2.510087in}}%
\pgfpathlineto{\pgfqpoint{5.656680in}{2.461831in}}%
\pgfpathlineto{\pgfqpoint{5.657766in}{2.425806in}}%
\pgfpathlineto{\pgfqpoint{5.657963in}{2.433086in}}%
\pgfpathlineto{\pgfqpoint{5.659049in}{2.482793in}}%
\pgfpathlineto{\pgfqpoint{5.659444in}{2.466463in}}%
\pgfpathlineto{\pgfqpoint{5.660431in}{2.451320in}}%
\pgfpathlineto{\pgfqpoint{5.660628in}{2.457939in}}%
\pgfpathlineto{\pgfqpoint{5.661615in}{2.481200in}}%
\pgfpathlineto{\pgfqpoint{5.662010in}{2.470130in}}%
\pgfpathlineto{\pgfqpoint{5.662898in}{2.467616in}}%
\pgfpathlineto{\pgfqpoint{5.662504in}{2.472594in}}%
\pgfpathlineto{\pgfqpoint{5.663194in}{2.469473in}}%
\pgfpathlineto{\pgfqpoint{5.663293in}{2.469588in}}%
\pgfpathlineto{\pgfqpoint{5.663392in}{2.468932in}}%
\pgfpathlineto{\pgfqpoint{5.664083in}{2.456409in}}%
\pgfpathlineto{\pgfqpoint{5.664379in}{2.466582in}}%
\pgfpathlineto{\pgfqpoint{5.665168in}{2.462322in}}%
\pgfpathlineto{\pgfqpoint{5.665761in}{2.477462in}}%
\pgfpathlineto{\pgfqpoint{5.668524in}{2.444882in}}%
\pgfpathlineto{\pgfqpoint{5.668722in}{2.450460in}}%
\pgfpathlineto{\pgfqpoint{5.669018in}{2.466502in}}%
\pgfpathlineto{\pgfqpoint{5.669413in}{2.439243in}}%
\pgfpathlineto{\pgfqpoint{5.670005in}{2.465222in}}%
\pgfpathlineto{\pgfqpoint{5.670498in}{2.454751in}}%
\pgfpathlineto{\pgfqpoint{5.671485in}{2.459638in}}%
\pgfpathlineto{\pgfqpoint{5.671584in}{2.460155in}}%
\pgfpathlineto{\pgfqpoint{5.671781in}{2.456382in}}%
\pgfpathlineto{\pgfqpoint{5.672768in}{2.451717in}}%
\pgfpathlineto{\pgfqpoint{5.672373in}{2.463808in}}%
\pgfpathlineto{\pgfqpoint{5.672867in}{2.453043in}}%
\pgfpathlineto{\pgfqpoint{5.673163in}{2.460619in}}%
\pgfpathlineto{\pgfqpoint{5.673755in}{2.446565in}}%
\pgfpathlineto{\pgfqpoint{5.673953in}{2.447751in}}%
\pgfpathlineto{\pgfqpoint{5.674150in}{2.444988in}}%
\pgfpathlineto{\pgfqpoint{5.676618in}{2.364354in}}%
\pgfpathlineto{\pgfqpoint{5.676914in}{2.368583in}}%
\pgfpathlineto{\pgfqpoint{5.677703in}{2.391768in}}%
\pgfpathlineto{\pgfqpoint{5.678888in}{2.421620in}}%
\pgfpathlineto{\pgfqpoint{5.679282in}{2.414184in}}%
\pgfpathlineto{\pgfqpoint{5.680862in}{2.396807in}}%
\pgfpathlineto{\pgfqpoint{5.682046in}{2.384595in}}%
\pgfpathlineto{\pgfqpoint{5.681454in}{2.399167in}}%
\pgfpathlineto{\pgfqpoint{5.682243in}{2.389848in}}%
\pgfpathlineto{\pgfqpoint{5.682539in}{2.405056in}}%
\pgfpathlineto{\pgfqpoint{5.683033in}{2.379249in}}%
\pgfpathlineto{\pgfqpoint{5.683329in}{2.389964in}}%
\pgfpathlineto{\pgfqpoint{5.683526in}{2.394182in}}%
\pgfpathlineto{\pgfqpoint{5.684119in}{2.384369in}}%
\pgfpathlineto{\pgfqpoint{5.684908in}{2.373545in}}%
\pgfpathlineto{\pgfqpoint{5.685204in}{2.384598in}}%
\pgfpathlineto{\pgfqpoint{5.687080in}{2.433938in}}%
\pgfpathlineto{\pgfqpoint{5.687277in}{2.431909in}}%
\pgfpathlineto{\pgfqpoint{5.687573in}{2.441266in}}%
\pgfpathlineto{\pgfqpoint{5.688264in}{2.449039in}}%
\pgfpathlineto{\pgfqpoint{5.688659in}{2.441221in}}%
\pgfpathlineto{\pgfqpoint{5.688955in}{2.438109in}}%
\pgfpathlineto{\pgfqpoint{5.690041in}{2.398095in}}%
\pgfpathlineto{\pgfqpoint{5.690633in}{2.406272in}}%
\pgfpathlineto{\pgfqpoint{5.691126in}{2.426754in}}%
\pgfpathlineto{\pgfqpoint{5.691916in}{2.416210in}}%
\pgfpathlineto{\pgfqpoint{5.693298in}{2.383326in}}%
\pgfpathlineto{\pgfqpoint{5.693396in}{2.386516in}}%
\pgfpathlineto{\pgfqpoint{5.694482in}{2.442651in}}%
\pgfpathlineto{\pgfqpoint{5.695074in}{2.422776in}}%
\pgfpathlineto{\pgfqpoint{5.696357in}{2.381969in}}%
\pgfpathlineto{\pgfqpoint{5.696456in}{2.382834in}}%
\pgfpathlineto{\pgfqpoint{5.696851in}{2.389763in}}%
\pgfpathlineto{\pgfqpoint{5.698331in}{2.530781in}}%
\pgfpathlineto{\pgfqpoint{5.699417in}{2.649593in}}%
\pgfpathlineto{\pgfqpoint{5.699812in}{2.615127in}}%
\pgfpathlineto{\pgfqpoint{5.700601in}{2.471264in}}%
\pgfpathlineto{\pgfqpoint{5.701095in}{2.373810in}}%
\pgfpathlineto{\pgfqpoint{5.701884in}{2.409092in}}%
\pgfpathlineto{\pgfqpoint{5.702378in}{2.389603in}}%
\pgfpathlineto{\pgfqpoint{5.703069in}{2.404081in}}%
\pgfpathlineto{\pgfqpoint{5.704253in}{2.467308in}}%
\pgfpathlineto{\pgfqpoint{5.704845in}{2.430074in}}%
\pgfpathlineto{\pgfqpoint{5.705438in}{2.411563in}}%
\pgfpathlineto{\pgfqpoint{5.705832in}{2.430450in}}%
\pgfpathlineto{\pgfqpoint{5.707313in}{2.469025in}}%
\pgfpathlineto{\pgfqpoint{5.707510in}{2.467875in}}%
\pgfpathlineto{\pgfqpoint{5.708102in}{2.455349in}}%
\pgfpathlineto{\pgfqpoint{5.708497in}{2.437391in}}%
\pgfpathlineto{\pgfqpoint{5.709188in}{2.454803in}}%
\pgfpathlineto{\pgfqpoint{5.710669in}{2.466890in}}%
\pgfpathlineto{\pgfqpoint{5.709780in}{2.450248in}}%
\pgfpathlineto{\pgfqpoint{5.710767in}{2.466807in}}%
\pgfpathlineto{\pgfqpoint{5.711162in}{2.456574in}}%
\pgfpathlineto{\pgfqpoint{5.711458in}{2.472282in}}%
\pgfpathlineto{\pgfqpoint{5.712347in}{2.479348in}}%
\pgfpathlineto{\pgfqpoint{5.711952in}{2.463979in}}%
\pgfpathlineto{\pgfqpoint{5.712643in}{2.475142in}}%
\pgfpathlineto{\pgfqpoint{5.713926in}{2.494295in}}%
\pgfpathlineto{\pgfqpoint{5.714419in}{2.485810in}}%
\pgfpathlineto{\pgfqpoint{5.714518in}{2.484254in}}%
\pgfpathlineto{\pgfqpoint{5.714814in}{2.495479in}}%
\pgfpathlineto{\pgfqpoint{5.715900in}{2.505612in}}%
\pgfpathlineto{\pgfqpoint{5.715406in}{2.483981in}}%
\pgfpathlineto{\pgfqpoint{5.715998in}{2.501532in}}%
\pgfpathlineto{\pgfqpoint{5.717084in}{2.478659in}}%
\pgfpathlineto{\pgfqpoint{5.716689in}{2.508379in}}%
\pgfpathlineto{\pgfqpoint{5.717183in}{2.480410in}}%
\pgfpathlineto{\pgfqpoint{5.717578in}{2.507160in}}%
\pgfpathlineto{\pgfqpoint{5.718466in}{2.498209in}}%
\pgfpathlineto{\pgfqpoint{5.719749in}{2.483601in}}%
\pgfpathlineto{\pgfqpoint{5.719255in}{2.501224in}}%
\pgfpathlineto{\pgfqpoint{5.719848in}{2.486478in}}%
\pgfpathlineto{\pgfqpoint{5.720242in}{2.501466in}}%
\pgfpathlineto{\pgfqpoint{5.720736in}{2.485668in}}%
\pgfpathlineto{\pgfqpoint{5.720933in}{2.486483in}}%
\pgfpathlineto{\pgfqpoint{5.721131in}{2.484521in}}%
\pgfpathlineto{\pgfqpoint{5.722216in}{2.475294in}}%
\pgfpathlineto{\pgfqpoint{5.721723in}{2.490666in}}%
\pgfpathlineto{\pgfqpoint{5.722315in}{2.478478in}}%
\pgfpathlineto{\pgfqpoint{5.722513in}{2.487152in}}%
\pgfpathlineto{\pgfqpoint{5.723203in}{2.469676in}}%
\pgfpathlineto{\pgfqpoint{5.725770in}{2.425768in}}%
\pgfpathlineto{\pgfqpoint{5.725967in}{2.428718in}}%
\pgfpathlineto{\pgfqpoint{5.726757in}{2.431959in}}%
\pgfpathlineto{\pgfqpoint{5.726954in}{2.428774in}}%
\pgfpathlineto{\pgfqpoint{5.727645in}{2.419213in}}%
\pgfpathlineto{\pgfqpoint{5.727842in}{2.427326in}}%
\pgfpathlineto{\pgfqpoint{5.728040in}{2.432778in}}%
\pgfpathlineto{\pgfqpoint{5.728434in}{2.414564in}}%
\pgfpathlineto{\pgfqpoint{5.728829in}{2.428921in}}%
\pgfpathlineto{\pgfqpoint{5.729125in}{2.418759in}}%
\pgfpathlineto{\pgfqpoint{5.729619in}{2.432464in}}%
\pgfpathlineto{\pgfqpoint{5.729816in}{2.431231in}}%
\pgfpathlineto{\pgfqpoint{5.730112in}{2.427893in}}%
\pgfpathlineto{\pgfqpoint{5.730408in}{2.421155in}}%
\pgfpathlineto{\pgfqpoint{5.730902in}{2.431881in}}%
\pgfpathlineto{\pgfqpoint{5.731198in}{2.428089in}}%
\pgfpathlineto{\pgfqpoint{5.731297in}{2.426794in}}%
\pgfpathlineto{\pgfqpoint{5.731692in}{2.435394in}}%
\pgfpathlineto{\pgfqpoint{5.731790in}{2.435852in}}%
\pgfpathlineto{\pgfqpoint{5.731889in}{2.433225in}}%
\pgfpathlineto{\pgfqpoint{5.732185in}{2.421872in}}%
\pgfpathlineto{\pgfqpoint{5.733073in}{2.428958in}}%
\pgfpathlineto{\pgfqpoint{5.734455in}{2.463613in}}%
\pgfpathlineto{\pgfqpoint{5.734751in}{2.455154in}}%
\pgfpathlineto{\pgfqpoint{5.734850in}{2.453974in}}%
\pgfpathlineto{\pgfqpoint{5.735146in}{2.463102in}}%
\pgfpathlineto{\pgfqpoint{5.735541in}{2.479843in}}%
\pgfpathlineto{\pgfqpoint{5.736034in}{2.460046in}}%
\pgfpathlineto{\pgfqpoint{5.736429in}{2.475546in}}%
\pgfpathlineto{\pgfqpoint{5.736824in}{2.457933in}}%
\pgfpathlineto{\pgfqpoint{5.738304in}{2.411363in}}%
\pgfpathlineto{\pgfqpoint{5.738502in}{2.416740in}}%
\pgfpathlineto{\pgfqpoint{5.738897in}{2.443962in}}%
\pgfpathlineto{\pgfqpoint{5.739587in}{2.421981in}}%
\pgfpathlineto{\pgfqpoint{5.741364in}{2.376568in}}%
\pgfpathlineto{\pgfqpoint{5.741956in}{2.414381in}}%
\pgfpathlineto{\pgfqpoint{5.742450in}{2.439972in}}%
\pgfpathlineto{\pgfqpoint{5.742943in}{2.419042in}}%
\pgfpathlineto{\pgfqpoint{5.744522in}{2.327726in}}%
\pgfpathlineto{\pgfqpoint{5.744917in}{2.347445in}}%
\pgfpathlineto{\pgfqpoint{5.745312in}{2.338650in}}%
\pgfpathlineto{\pgfqpoint{5.745608in}{2.379521in}}%
\pgfpathlineto{\pgfqpoint{5.747187in}{2.671145in}}%
\pgfpathlineto{\pgfqpoint{5.747779in}{2.648036in}}%
\pgfpathlineto{\pgfqpoint{5.748372in}{2.567499in}}%
\pgfpathlineto{\pgfqpoint{5.748964in}{2.393178in}}%
\pgfpathlineto{\pgfqpoint{5.749852in}{2.412157in}}%
\pgfpathlineto{\pgfqpoint{5.749951in}{2.410453in}}%
\pgfpathlineto{\pgfqpoint{5.750346in}{2.419647in}}%
\pgfpathlineto{\pgfqpoint{5.752221in}{2.500181in}}%
\pgfpathlineto{\pgfqpoint{5.752616in}{2.465571in}}%
\pgfpathlineto{\pgfqpoint{5.753307in}{2.422296in}}%
\pgfpathlineto{\pgfqpoint{5.753800in}{2.455899in}}%
\pgfpathlineto{\pgfqpoint{5.755478in}{2.495492in}}%
\pgfpathlineto{\pgfqpoint{5.754294in}{2.448104in}}%
\pgfpathlineto{\pgfqpoint{5.755774in}{2.485437in}}%
\pgfpathlineto{\pgfqpoint{5.756169in}{2.465061in}}%
\pgfpathlineto{\pgfqpoint{5.757057in}{2.468163in}}%
\pgfpathlineto{\pgfqpoint{5.760018in}{2.494901in}}%
\pgfpathlineto{\pgfqpoint{5.760314in}{2.489653in}}%
\pgfpathlineto{\pgfqpoint{5.760610in}{2.499649in}}%
\pgfpathlineto{\pgfqpoint{5.760906in}{2.508156in}}%
\pgfpathlineto{\pgfqpoint{5.761795in}{2.507769in}}%
\pgfpathlineto{\pgfqpoint{5.762091in}{2.506666in}}%
\pgfpathlineto{\pgfqpoint{5.762288in}{2.508179in}}%
\pgfpathlineto{\pgfqpoint{5.762683in}{2.517110in}}%
\pgfpathlineto{\pgfqpoint{5.763473in}{2.513139in}}%
\pgfpathlineto{\pgfqpoint{5.763571in}{2.511489in}}%
\pgfpathlineto{\pgfqpoint{5.763867in}{2.521537in}}%
\pgfpathlineto{\pgfqpoint{5.764065in}{2.526861in}}%
\pgfpathlineto{\pgfqpoint{5.764558in}{2.503364in}}%
\pgfpathlineto{\pgfqpoint{5.764953in}{2.518120in}}%
\pgfpathlineto{\pgfqpoint{5.765841in}{2.515079in}}%
\pgfpathlineto{\pgfqpoint{5.767421in}{2.480742in}}%
\pgfpathlineto{\pgfqpoint{5.767618in}{2.485015in}}%
\pgfpathlineto{\pgfqpoint{5.767815in}{2.492280in}}%
\pgfpathlineto{\pgfqpoint{5.768506in}{2.476709in}}%
\pgfpathlineto{\pgfqpoint{5.768901in}{2.479801in}}%
\pgfpathlineto{\pgfqpoint{5.769197in}{2.475940in}}%
\pgfpathlineto{\pgfqpoint{5.770085in}{2.470518in}}%
\pgfpathlineto{\pgfqpoint{5.769691in}{2.478469in}}%
\pgfpathlineto{\pgfqpoint{5.770382in}{2.474227in}}%
\pgfpathlineto{\pgfqpoint{5.770480in}{2.474410in}}%
\pgfpathlineto{\pgfqpoint{5.770579in}{2.472455in}}%
\pgfpathlineto{\pgfqpoint{5.772257in}{2.445476in}}%
\pgfpathlineto{\pgfqpoint{5.771171in}{2.474176in}}%
\pgfpathlineto{\pgfqpoint{5.772454in}{2.451088in}}%
\pgfpathlineto{\pgfqpoint{5.772652in}{2.454562in}}%
\pgfpathlineto{\pgfqpoint{5.773046in}{2.439634in}}%
\pgfpathlineto{\pgfqpoint{5.774132in}{2.431415in}}%
\pgfpathlineto{\pgfqpoint{5.774329in}{2.432646in}}%
\pgfpathlineto{\pgfqpoint{5.774527in}{2.435109in}}%
\pgfpathlineto{\pgfqpoint{5.774922in}{2.424311in}}%
\pgfpathlineto{\pgfqpoint{5.775119in}{2.421348in}}%
\pgfpathlineto{\pgfqpoint{5.775514in}{2.431583in}}%
\pgfpathlineto{\pgfqpoint{5.776007in}{2.422517in}}%
\pgfpathlineto{\pgfqpoint{5.776205in}{2.425632in}}%
\pgfpathlineto{\pgfqpoint{5.776501in}{2.417907in}}%
\pgfpathlineto{\pgfqpoint{5.777784in}{2.406549in}}%
\pgfpathlineto{\pgfqpoint{5.777093in}{2.420995in}}%
\pgfpathlineto{\pgfqpoint{5.777883in}{2.407319in}}%
\pgfpathlineto{\pgfqpoint{5.778179in}{2.415624in}}%
\pgfpathlineto{\pgfqpoint{5.778870in}{2.406685in}}%
\pgfpathlineto{\pgfqpoint{5.779166in}{2.402740in}}%
\pgfpathlineto{\pgfqpoint{5.779758in}{2.408765in}}%
\pgfpathlineto{\pgfqpoint{5.780153in}{2.418239in}}%
\pgfpathlineto{\pgfqpoint{5.780548in}{2.403436in}}%
\pgfpathlineto{\pgfqpoint{5.781041in}{2.415255in}}%
\pgfpathlineto{\pgfqpoint{5.781238in}{2.412753in}}%
\pgfpathlineto{\pgfqpoint{5.781535in}{2.423035in}}%
\pgfpathlineto{\pgfqpoint{5.783212in}{2.460080in}}%
\pgfpathlineto{\pgfqpoint{5.783410in}{2.457272in}}%
\pgfpathlineto{\pgfqpoint{5.786173in}{2.407900in}}%
\pgfpathlineto{\pgfqpoint{5.786568in}{2.416403in}}%
\pgfpathlineto{\pgfqpoint{5.786766in}{2.415065in}}%
\pgfpathlineto{\pgfqpoint{5.787259in}{2.417574in}}%
\pgfpathlineto{\pgfqpoint{5.787654in}{2.404527in}}%
\pgfpathlineto{\pgfqpoint{5.789036in}{2.383252in}}%
\pgfpathlineto{\pgfqpoint{5.789233in}{2.384345in}}%
\pgfpathlineto{\pgfqpoint{5.790417in}{2.446157in}}%
\pgfpathlineto{\pgfqpoint{5.790911in}{2.402281in}}%
\pgfpathlineto{\pgfqpoint{5.791898in}{2.380393in}}%
\pgfpathlineto{\pgfqpoint{5.792095in}{2.382882in}}%
\pgfpathlineto{\pgfqpoint{5.793674in}{2.481802in}}%
\pgfpathlineto{\pgfqpoint{5.794859in}{2.668345in}}%
\pgfpathlineto{\pgfqpoint{5.795352in}{2.624223in}}%
\pgfpathlineto{\pgfqpoint{5.796142in}{2.518711in}}%
\pgfpathlineto{\pgfqpoint{5.796734in}{2.388871in}}%
\pgfpathlineto{\pgfqpoint{5.797524in}{2.413447in}}%
\pgfpathlineto{\pgfqpoint{5.797622in}{2.410643in}}%
\pgfpathlineto{\pgfqpoint{5.798017in}{2.417336in}}%
\pgfpathlineto{\pgfqpoint{5.798313in}{2.416833in}}%
\pgfpathlineto{\pgfqpoint{5.799893in}{2.494319in}}%
\pgfpathlineto{\pgfqpoint{5.800287in}{2.450162in}}%
\pgfpathlineto{\pgfqpoint{5.801077in}{2.415055in}}%
\pgfpathlineto{\pgfqpoint{5.801373in}{2.435954in}}%
\pgfpathlineto{\pgfqpoint{5.802952in}{2.479124in}}%
\pgfpathlineto{\pgfqpoint{5.804137in}{2.440927in}}%
\pgfpathlineto{\pgfqpoint{5.804630in}{2.459468in}}%
\pgfpathlineto{\pgfqpoint{5.804827in}{2.463818in}}%
\pgfpathlineto{\pgfqpoint{5.805222in}{2.452178in}}%
\pgfpathlineto{\pgfqpoint{5.805814in}{2.462749in}}%
\pgfpathlineto{\pgfqpoint{5.805913in}{2.462699in}}%
\pgfpathlineto{\pgfqpoint{5.807394in}{2.475855in}}%
\pgfpathlineto{\pgfqpoint{5.807492in}{2.474586in}}%
\pgfpathlineto{\pgfqpoint{5.807690in}{2.469644in}}%
\pgfpathlineto{\pgfqpoint{5.808085in}{2.478463in}}%
\pgfpathlineto{\pgfqpoint{5.808381in}{2.475861in}}%
\pgfpathlineto{\pgfqpoint{5.808775in}{2.489753in}}%
\pgfpathlineto{\pgfqpoint{5.809170in}{2.470069in}}%
\pgfpathlineto{\pgfqpoint{5.810848in}{2.429998in}}%
\pgfpathlineto{\pgfqpoint{5.811243in}{2.411936in}}%
\pgfpathlineto{\pgfqpoint{5.812131in}{2.419281in}}%
\pgfpathlineto{\pgfqpoint{5.814303in}{2.499344in}}%
\pgfpathlineto{\pgfqpoint{5.815290in}{2.491402in}}%
\pgfpathlineto{\pgfqpoint{5.815586in}{2.481071in}}%
\pgfpathlineto{\pgfqpoint{5.816573in}{2.481880in}}%
\pgfpathlineto{\pgfqpoint{5.817165in}{2.484524in}}%
\pgfpathlineto{\pgfqpoint{5.819336in}{2.462283in}}%
\pgfpathlineto{\pgfqpoint{5.820915in}{2.444363in}}%
\pgfpathlineto{\pgfqpoint{5.821014in}{2.445967in}}%
\pgfpathlineto{\pgfqpoint{5.821310in}{2.450492in}}%
\pgfpathlineto{\pgfqpoint{5.821705in}{2.439342in}}%
\pgfpathlineto{\pgfqpoint{5.821804in}{2.439454in}}%
\pgfpathlineto{\pgfqpoint{5.822001in}{2.440469in}}%
\pgfpathlineto{\pgfqpoint{5.822198in}{2.436974in}}%
\pgfpathlineto{\pgfqpoint{5.823679in}{2.413122in}}%
\pgfpathlineto{\pgfqpoint{5.823876in}{2.416719in}}%
\pgfpathlineto{\pgfqpoint{5.824172in}{2.424627in}}%
\pgfpathlineto{\pgfqpoint{5.824863in}{2.415289in}}%
\pgfpathlineto{\pgfqpoint{5.825159in}{2.409123in}}%
\pgfpathlineto{\pgfqpoint{5.825949in}{2.415057in}}%
\pgfpathlineto{\pgfqpoint{5.826245in}{2.423153in}}%
\pgfpathlineto{\pgfqpoint{5.826541in}{2.412199in}}%
\pgfpathlineto{\pgfqpoint{5.827035in}{2.420974in}}%
\pgfpathlineto{\pgfqpoint{5.827331in}{2.406326in}}%
\pgfpathlineto{\pgfqpoint{5.828219in}{2.416164in}}%
\pgfpathlineto{\pgfqpoint{5.829009in}{2.429276in}}%
\pgfpathlineto{\pgfqpoint{5.829403in}{2.417858in}}%
\pgfpathlineto{\pgfqpoint{5.829502in}{2.418087in}}%
\pgfpathlineto{\pgfqpoint{5.830094in}{2.448201in}}%
\pgfpathlineto{\pgfqpoint{5.831180in}{2.445947in}}%
\pgfpathlineto{\pgfqpoint{5.833549in}{2.374745in}}%
\pgfpathlineto{\pgfqpoint{5.834536in}{2.385387in}}%
\pgfpathlineto{\pgfqpoint{5.834931in}{2.373791in}}%
\pgfpathlineto{\pgfqpoint{5.836016in}{2.341417in}}%
\pgfpathlineto{\pgfqpoint{5.836411in}{2.346853in}}%
\pgfpathlineto{\pgfqpoint{5.836905in}{2.360671in}}%
\pgfpathlineto{\pgfqpoint{5.837596in}{2.409658in}}%
\pgfpathlineto{\pgfqpoint{5.838286in}{2.370487in}}%
\pgfpathlineto{\pgfqpoint{5.839569in}{2.341995in}}%
\pgfpathlineto{\pgfqpoint{5.839767in}{2.347001in}}%
\pgfpathlineto{\pgfqpoint{5.841543in}{2.550638in}}%
\pgfpathlineto{\pgfqpoint{5.842136in}{2.630939in}}%
\pgfpathlineto{\pgfqpoint{5.842827in}{2.596602in}}%
\pgfpathlineto{\pgfqpoint{5.843320in}{2.572903in}}%
\pgfpathlineto{\pgfqpoint{5.844110in}{2.354527in}}%
\pgfpathlineto{\pgfqpoint{5.845195in}{2.367694in}}%
\pgfpathlineto{\pgfqpoint{5.845294in}{2.366808in}}%
\pgfpathlineto{\pgfqpoint{5.845491in}{2.371302in}}%
\pgfpathlineto{\pgfqpoint{5.847169in}{2.452338in}}%
\pgfpathlineto{\pgfqpoint{5.847564in}{2.436669in}}%
\pgfpathlineto{\pgfqpoint{5.848452in}{2.395192in}}%
\pgfpathlineto{\pgfqpoint{5.848946in}{2.410124in}}%
\pgfpathlineto{\pgfqpoint{5.850525in}{2.456620in}}%
\pgfpathlineto{\pgfqpoint{5.850821in}{2.440563in}}%
\pgfpathlineto{\pgfqpoint{5.851512in}{2.415651in}}%
\pgfpathlineto{\pgfqpoint{5.851808in}{2.435858in}}%
\pgfpathlineto{\pgfqpoint{5.852400in}{2.431875in}}%
\pgfpathlineto{\pgfqpoint{5.853190in}{2.456758in}}%
\pgfpathlineto{\pgfqpoint{5.853980in}{2.462581in}}%
\pgfpathlineto{\pgfqpoint{5.854177in}{2.458795in}}%
\pgfpathlineto{\pgfqpoint{5.855263in}{2.444966in}}%
\pgfpathlineto{\pgfqpoint{5.855460in}{2.450812in}}%
\pgfpathlineto{\pgfqpoint{5.856447in}{2.467993in}}%
\pgfpathlineto{\pgfqpoint{5.856743in}{2.463359in}}%
\pgfpathlineto{\pgfqpoint{5.856842in}{2.463391in}}%
\pgfpathlineto{\pgfqpoint{5.857927in}{2.487488in}}%
\pgfpathlineto{\pgfqpoint{5.858125in}{2.474549in}}%
\pgfpathlineto{\pgfqpoint{5.858421in}{2.457877in}}%
\pgfpathlineto{\pgfqpoint{5.858816in}{2.481919in}}%
\pgfpathlineto{\pgfqpoint{5.859309in}{2.459320in}}%
\pgfpathlineto{\pgfqpoint{5.860296in}{2.474316in}}%
\pgfpathlineto{\pgfqpoint{5.860592in}{2.470295in}}%
\pgfpathlineto{\pgfqpoint{5.860987in}{2.465940in}}%
\pgfpathlineto{\pgfqpoint{5.861185in}{2.471169in}}%
\pgfpathlineto{\pgfqpoint{5.862270in}{2.485346in}}%
\pgfpathlineto{\pgfqpoint{5.862369in}{2.483615in}}%
\pgfpathlineto{\pgfqpoint{5.862862in}{2.466197in}}%
\pgfpathlineto{\pgfqpoint{5.863751in}{2.468279in}}%
\pgfpathlineto{\pgfqpoint{5.864146in}{2.471626in}}%
\pgfpathlineto{\pgfqpoint{5.864442in}{2.467161in}}%
\pgfpathlineto{\pgfqpoint{5.865527in}{2.452490in}}%
\pgfpathlineto{\pgfqpoint{5.865823in}{2.457960in}}%
\pgfpathlineto{\pgfqpoint{5.865922in}{2.460101in}}%
\pgfpathlineto{\pgfqpoint{5.866317in}{2.444975in}}%
\pgfpathlineto{\pgfqpoint{5.867205in}{2.428319in}}%
\pgfpathlineto{\pgfqpoint{5.866810in}{2.445646in}}%
\pgfpathlineto{\pgfqpoint{5.867501in}{2.435888in}}%
\pgfpathlineto{\pgfqpoint{5.867600in}{2.437241in}}%
\pgfpathlineto{\pgfqpoint{5.867995in}{2.428656in}}%
\pgfpathlineto{\pgfqpoint{5.868093in}{2.429261in}}%
\pgfpathlineto{\pgfqpoint{5.868192in}{2.429266in}}%
\pgfpathlineto{\pgfqpoint{5.870758in}{2.394198in}}%
\pgfpathlineto{\pgfqpoint{5.870857in}{2.392462in}}%
\pgfpathlineto{\pgfqpoint{5.871252in}{2.403228in}}%
\pgfpathlineto{\pgfqpoint{5.871449in}{2.405858in}}%
\pgfpathlineto{\pgfqpoint{5.871943in}{2.395695in}}%
\pgfpathlineto{\pgfqpoint{5.872239in}{2.400884in}}%
\pgfpathlineto{\pgfqpoint{5.872436in}{2.402456in}}%
\pgfpathlineto{\pgfqpoint{5.872732in}{2.393995in}}%
\pgfpathlineto{\pgfqpoint{5.873719in}{2.388582in}}%
\pgfpathlineto{\pgfqpoint{5.873818in}{2.390325in}}%
\pgfpathlineto{\pgfqpoint{5.874015in}{2.395280in}}%
\pgfpathlineto{\pgfqpoint{5.874410in}{2.379184in}}%
\pgfpathlineto{\pgfqpoint{5.874509in}{2.377205in}}%
\pgfpathlineto{\pgfqpoint{5.874706in}{2.388031in}}%
\pgfpathlineto{\pgfqpoint{5.875002in}{2.403534in}}%
\pgfpathlineto{\pgfqpoint{5.875792in}{2.393468in}}%
\pgfpathlineto{\pgfqpoint{5.877272in}{2.364690in}}%
\pgfpathlineto{\pgfqpoint{5.877667in}{2.378524in}}%
\pgfpathlineto{\pgfqpoint{5.878062in}{2.360029in}}%
\pgfpathlineto{\pgfqpoint{5.878161in}{2.357274in}}%
\pgfpathlineto{\pgfqpoint{5.878556in}{2.373383in}}%
\pgfpathlineto{\pgfqpoint{5.880036in}{2.418120in}}%
\pgfpathlineto{\pgfqpoint{5.880924in}{2.405648in}}%
\pgfpathlineto{\pgfqpoint{5.881122in}{2.397021in}}%
\pgfpathlineto{\pgfqpoint{5.881714in}{2.416363in}}%
\pgfpathlineto{\pgfqpoint{5.881813in}{2.415375in}}%
\pgfpathlineto{\pgfqpoint{5.883885in}{2.365689in}}%
\pgfpathlineto{\pgfqpoint{5.884181in}{2.377006in}}%
\pgfpathlineto{\pgfqpoint{5.885168in}{2.424695in}}%
\pgfpathlineto{\pgfqpoint{5.885662in}{2.407169in}}%
\pgfpathlineto{\pgfqpoint{5.887142in}{2.355183in}}%
\pgfpathlineto{\pgfqpoint{5.887241in}{2.356045in}}%
\pgfpathlineto{\pgfqpoint{5.888820in}{2.517257in}}%
\pgfpathlineto{\pgfqpoint{5.890005in}{2.649370in}}%
\pgfpathlineto{\pgfqpoint{5.890399in}{2.607675in}}%
\pgfpathlineto{\pgfqpoint{5.890893in}{2.570948in}}%
\pgfpathlineto{\pgfqpoint{5.891584in}{2.369925in}}%
\pgfpathlineto{\pgfqpoint{5.892472in}{2.397961in}}%
\pgfpathlineto{\pgfqpoint{5.892571in}{2.396404in}}%
\pgfpathlineto{\pgfqpoint{5.892867in}{2.407394in}}%
\pgfpathlineto{\pgfqpoint{5.894742in}{2.490158in}}%
\pgfpathlineto{\pgfqpoint{5.895137in}{2.462837in}}%
\pgfpathlineto{\pgfqpoint{5.895927in}{2.427406in}}%
\pgfpathlineto{\pgfqpoint{5.896420in}{2.439444in}}%
\pgfpathlineto{\pgfqpoint{5.897901in}{2.472367in}}%
\pgfpathlineto{\pgfqpoint{5.898197in}{2.464225in}}%
\pgfpathlineto{\pgfqpoint{5.899282in}{2.436003in}}%
\pgfpathlineto{\pgfqpoint{5.899480in}{2.442656in}}%
\pgfusepath{stroke}%
\end{pgfscope}%
\begin{pgfscope}%
\pgfsetrectcap%
\pgfsetmiterjoin%
\pgfsetlinewidth{0.803000pt}%
\definecolor{currentstroke}{rgb}{0.000000,0.000000,0.000000}%
\pgfsetstrokecolor{currentstroke}%
\pgfsetdash{}{0pt}%
\pgfpathmoveto{\pgfqpoint{0.717889in}{2.114143in}}%
\pgfpathlineto{\pgfqpoint{0.717889in}{2.901359in}}%
\pgfusepath{stroke}%
\end{pgfscope}%
\begin{pgfscope}%
\pgfsetrectcap%
\pgfsetmiterjoin%
\pgfsetlinewidth{0.803000pt}%
\definecolor{currentstroke}{rgb}{0.000000,0.000000,0.000000}%
\pgfsetstrokecolor{currentstroke}%
\pgfsetdash{}{0pt}%
\pgfpathmoveto{\pgfqpoint{6.146222in}{2.114143in}}%
\pgfpathlineto{\pgfqpoint{6.146222in}{2.901359in}}%
\pgfusepath{stroke}%
\end{pgfscope}%
\begin{pgfscope}%
\pgfsetrectcap%
\pgfsetmiterjoin%
\pgfsetlinewidth{0.803000pt}%
\definecolor{currentstroke}{rgb}{0.000000,0.000000,0.000000}%
\pgfsetstrokecolor{currentstroke}%
\pgfsetdash{}{0pt}%
\pgfpathmoveto{\pgfqpoint{0.717889in}{2.114143in}}%
\pgfpathlineto{\pgfqpoint{6.146222in}{2.114143in}}%
\pgfusepath{stroke}%
\end{pgfscope}%
\begin{pgfscope}%
\pgfsetrectcap%
\pgfsetmiterjoin%
\pgfsetlinewidth{0.803000pt}%
\definecolor{currentstroke}{rgb}{0.000000,0.000000,0.000000}%
\pgfsetstrokecolor{currentstroke}%
\pgfsetdash{}{0pt}%
\pgfpathmoveto{\pgfqpoint{0.717889in}{2.901359in}}%
\pgfpathlineto{\pgfqpoint{6.146222in}{2.901359in}}%
\pgfusepath{stroke}%
\end{pgfscope}%
\begin{pgfscope}%
\definecolor{textcolor}{rgb}{0.000000,0.000000,0.000000}%
\pgfsetstrokecolor{textcolor}%
\pgfsetfillcolor{textcolor}%
\pgftext[x=3.432055in,y=2.984692in,,base]{\color{textcolor}\rmfamily\fontsize{12.000000}{14.400000}\selectfont Filtered ECG Signal}%
\end{pgfscope}%
\begin{pgfscope}%
\pgfsetbuttcap%
\pgfsetmiterjoin%
\definecolor{currentfill}{rgb}{1.000000,1.000000,1.000000}%
\pgfsetfillcolor{currentfill}%
\pgfsetlinewidth{0.000000pt}%
\definecolor{currentstroke}{rgb}{0.000000,0.000000,0.000000}%
\pgfsetstrokecolor{currentstroke}%
\pgfsetstrokeopacity{0.000000}%
\pgfsetdash{}{0pt}%
\pgfpathmoveto{\pgfqpoint{0.717889in}{0.564143in}}%
\pgfpathlineto{\pgfqpoint{6.146222in}{0.564143in}}%
\pgfpathlineto{\pgfqpoint{6.146222in}{1.351359in}}%
\pgfpathlineto{\pgfqpoint{0.717889in}{1.351359in}}%
\pgfpathclose%
\pgfusepath{fill}%
\end{pgfscope}%
\begin{pgfscope}%
\pgfsetbuttcap%
\pgfsetroundjoin%
\definecolor{currentfill}{rgb}{0.000000,0.000000,0.000000}%
\pgfsetfillcolor{currentfill}%
\pgfsetlinewidth{0.803000pt}%
\definecolor{currentstroke}{rgb}{0.000000,0.000000,0.000000}%
\pgfsetstrokecolor{currentstroke}%
\pgfsetdash{}{0pt}%
\pgfsys@defobject{currentmarker}{\pgfqpoint{0.000000in}{-0.048611in}}{\pgfqpoint{0.000000in}{0.000000in}}{%
\pgfpathmoveto{\pgfqpoint{0.000000in}{0.000000in}}%
\pgfpathlineto{\pgfqpoint{0.000000in}{-0.048611in}}%
\pgfusepath{stroke,fill}%
}%
\begin{pgfscope}%
\pgfsys@transformshift{0.717889in}{0.564143in}%
\pgfsys@useobject{currentmarker}{}%
\end{pgfscope}%
\end{pgfscope}%
\begin{pgfscope}%
\definecolor{textcolor}{rgb}{0.000000,0.000000,0.000000}%
\pgfsetstrokecolor{textcolor}%
\pgfsetfillcolor{textcolor}%
\pgftext[x=0.717889in,y=0.466921in,,top]{\color{textcolor}\rmfamily\fontsize{10.000000}{12.000000}\selectfont \(\displaystyle {-100}\)}%
\end{pgfscope}%
\begin{pgfscope}%
\pgfsetbuttcap%
\pgfsetroundjoin%
\definecolor{currentfill}{rgb}{0.000000,0.000000,0.000000}%
\pgfsetfillcolor{currentfill}%
\pgfsetlinewidth{0.803000pt}%
\definecolor{currentstroke}{rgb}{0.000000,0.000000,0.000000}%
\pgfsetstrokecolor{currentstroke}%
\pgfsetdash{}{0pt}%
\pgfsys@defobject{currentmarker}{\pgfqpoint{0.000000in}{-0.048611in}}{\pgfqpoint{0.000000in}{0.000000in}}{%
\pgfpathmoveto{\pgfqpoint{0.000000in}{0.000000in}}%
\pgfpathlineto{\pgfqpoint{0.000000in}{-0.048611in}}%
\pgfusepath{stroke,fill}%
}%
\begin{pgfscope}%
\pgfsys@transformshift{1.396430in}{0.564143in}%
\pgfsys@useobject{currentmarker}{}%
\end{pgfscope}%
\end{pgfscope}%
\begin{pgfscope}%
\definecolor{textcolor}{rgb}{0.000000,0.000000,0.000000}%
\pgfsetstrokecolor{textcolor}%
\pgfsetfillcolor{textcolor}%
\pgftext[x=1.396430in,y=0.466921in,,top]{\color{textcolor}\rmfamily\fontsize{10.000000}{12.000000}\selectfont \(\displaystyle {-75}\)}%
\end{pgfscope}%
\begin{pgfscope}%
\pgfsetbuttcap%
\pgfsetroundjoin%
\definecolor{currentfill}{rgb}{0.000000,0.000000,0.000000}%
\pgfsetfillcolor{currentfill}%
\pgfsetlinewidth{0.803000pt}%
\definecolor{currentstroke}{rgb}{0.000000,0.000000,0.000000}%
\pgfsetstrokecolor{currentstroke}%
\pgfsetdash{}{0pt}%
\pgfsys@defobject{currentmarker}{\pgfqpoint{0.000000in}{-0.048611in}}{\pgfqpoint{0.000000in}{0.000000in}}{%
\pgfpathmoveto{\pgfqpoint{0.000000in}{0.000000in}}%
\pgfpathlineto{\pgfqpoint{0.000000in}{-0.048611in}}%
\pgfusepath{stroke,fill}%
}%
\begin{pgfscope}%
\pgfsys@transformshift{2.074972in}{0.564143in}%
\pgfsys@useobject{currentmarker}{}%
\end{pgfscope}%
\end{pgfscope}%
\begin{pgfscope}%
\definecolor{textcolor}{rgb}{0.000000,0.000000,0.000000}%
\pgfsetstrokecolor{textcolor}%
\pgfsetfillcolor{textcolor}%
\pgftext[x=2.074972in,y=0.466921in,,top]{\color{textcolor}\rmfamily\fontsize{10.000000}{12.000000}\selectfont \(\displaystyle {-50}\)}%
\end{pgfscope}%
\begin{pgfscope}%
\pgfsetbuttcap%
\pgfsetroundjoin%
\definecolor{currentfill}{rgb}{0.000000,0.000000,0.000000}%
\pgfsetfillcolor{currentfill}%
\pgfsetlinewidth{0.803000pt}%
\definecolor{currentstroke}{rgb}{0.000000,0.000000,0.000000}%
\pgfsetstrokecolor{currentstroke}%
\pgfsetdash{}{0pt}%
\pgfsys@defobject{currentmarker}{\pgfqpoint{0.000000in}{-0.048611in}}{\pgfqpoint{0.000000in}{0.000000in}}{%
\pgfpathmoveto{\pgfqpoint{0.000000in}{0.000000in}}%
\pgfpathlineto{\pgfqpoint{0.000000in}{-0.048611in}}%
\pgfusepath{stroke,fill}%
}%
\begin{pgfscope}%
\pgfsys@transformshift{2.753514in}{0.564143in}%
\pgfsys@useobject{currentmarker}{}%
\end{pgfscope}%
\end{pgfscope}%
\begin{pgfscope}%
\definecolor{textcolor}{rgb}{0.000000,0.000000,0.000000}%
\pgfsetstrokecolor{textcolor}%
\pgfsetfillcolor{textcolor}%
\pgftext[x=2.753514in,y=0.466921in,,top]{\color{textcolor}\rmfamily\fontsize{10.000000}{12.000000}\selectfont \(\displaystyle {-25}\)}%
\end{pgfscope}%
\begin{pgfscope}%
\pgfsetbuttcap%
\pgfsetroundjoin%
\definecolor{currentfill}{rgb}{0.000000,0.000000,0.000000}%
\pgfsetfillcolor{currentfill}%
\pgfsetlinewidth{0.803000pt}%
\definecolor{currentstroke}{rgb}{0.000000,0.000000,0.000000}%
\pgfsetstrokecolor{currentstroke}%
\pgfsetdash{}{0pt}%
\pgfsys@defobject{currentmarker}{\pgfqpoint{0.000000in}{-0.048611in}}{\pgfqpoint{0.000000in}{0.000000in}}{%
\pgfpathmoveto{\pgfqpoint{0.000000in}{0.000000in}}%
\pgfpathlineto{\pgfqpoint{0.000000in}{-0.048611in}}%
\pgfusepath{stroke,fill}%
}%
\begin{pgfscope}%
\pgfsys@transformshift{3.432055in}{0.564143in}%
\pgfsys@useobject{currentmarker}{}%
\end{pgfscope}%
\end{pgfscope}%
\begin{pgfscope}%
\definecolor{textcolor}{rgb}{0.000000,0.000000,0.000000}%
\pgfsetstrokecolor{textcolor}%
\pgfsetfillcolor{textcolor}%
\pgftext[x=3.432055in,y=0.466921in,,top]{\color{textcolor}\rmfamily\fontsize{10.000000}{12.000000}\selectfont \(\displaystyle {0}\)}%
\end{pgfscope}%
\begin{pgfscope}%
\pgfsetbuttcap%
\pgfsetroundjoin%
\definecolor{currentfill}{rgb}{0.000000,0.000000,0.000000}%
\pgfsetfillcolor{currentfill}%
\pgfsetlinewidth{0.803000pt}%
\definecolor{currentstroke}{rgb}{0.000000,0.000000,0.000000}%
\pgfsetstrokecolor{currentstroke}%
\pgfsetdash{}{0pt}%
\pgfsys@defobject{currentmarker}{\pgfqpoint{0.000000in}{-0.048611in}}{\pgfqpoint{0.000000in}{0.000000in}}{%
\pgfpathmoveto{\pgfqpoint{0.000000in}{0.000000in}}%
\pgfpathlineto{\pgfqpoint{0.000000in}{-0.048611in}}%
\pgfusepath{stroke,fill}%
}%
\begin{pgfscope}%
\pgfsys@transformshift{4.110597in}{0.564143in}%
\pgfsys@useobject{currentmarker}{}%
\end{pgfscope}%
\end{pgfscope}%
\begin{pgfscope}%
\definecolor{textcolor}{rgb}{0.000000,0.000000,0.000000}%
\pgfsetstrokecolor{textcolor}%
\pgfsetfillcolor{textcolor}%
\pgftext[x=4.110597in,y=0.466921in,,top]{\color{textcolor}\rmfamily\fontsize{10.000000}{12.000000}\selectfont \(\displaystyle {25}\)}%
\end{pgfscope}%
\begin{pgfscope}%
\pgfsetbuttcap%
\pgfsetroundjoin%
\definecolor{currentfill}{rgb}{0.000000,0.000000,0.000000}%
\pgfsetfillcolor{currentfill}%
\pgfsetlinewidth{0.803000pt}%
\definecolor{currentstroke}{rgb}{0.000000,0.000000,0.000000}%
\pgfsetstrokecolor{currentstroke}%
\pgfsetdash{}{0pt}%
\pgfsys@defobject{currentmarker}{\pgfqpoint{0.000000in}{-0.048611in}}{\pgfqpoint{0.000000in}{0.000000in}}{%
\pgfpathmoveto{\pgfqpoint{0.000000in}{0.000000in}}%
\pgfpathlineto{\pgfqpoint{0.000000in}{-0.048611in}}%
\pgfusepath{stroke,fill}%
}%
\begin{pgfscope}%
\pgfsys@transformshift{4.789139in}{0.564143in}%
\pgfsys@useobject{currentmarker}{}%
\end{pgfscope}%
\end{pgfscope}%
\begin{pgfscope}%
\definecolor{textcolor}{rgb}{0.000000,0.000000,0.000000}%
\pgfsetstrokecolor{textcolor}%
\pgfsetfillcolor{textcolor}%
\pgftext[x=4.789139in,y=0.466921in,,top]{\color{textcolor}\rmfamily\fontsize{10.000000}{12.000000}\selectfont \(\displaystyle {50}\)}%
\end{pgfscope}%
\begin{pgfscope}%
\pgfsetbuttcap%
\pgfsetroundjoin%
\definecolor{currentfill}{rgb}{0.000000,0.000000,0.000000}%
\pgfsetfillcolor{currentfill}%
\pgfsetlinewidth{0.803000pt}%
\definecolor{currentstroke}{rgb}{0.000000,0.000000,0.000000}%
\pgfsetstrokecolor{currentstroke}%
\pgfsetdash{}{0pt}%
\pgfsys@defobject{currentmarker}{\pgfqpoint{0.000000in}{-0.048611in}}{\pgfqpoint{0.000000in}{0.000000in}}{%
\pgfpathmoveto{\pgfqpoint{0.000000in}{0.000000in}}%
\pgfpathlineto{\pgfqpoint{0.000000in}{-0.048611in}}%
\pgfusepath{stroke,fill}%
}%
\begin{pgfscope}%
\pgfsys@transformshift{5.467680in}{0.564143in}%
\pgfsys@useobject{currentmarker}{}%
\end{pgfscope}%
\end{pgfscope}%
\begin{pgfscope}%
\definecolor{textcolor}{rgb}{0.000000,0.000000,0.000000}%
\pgfsetstrokecolor{textcolor}%
\pgfsetfillcolor{textcolor}%
\pgftext[x=5.467680in,y=0.466921in,,top]{\color{textcolor}\rmfamily\fontsize{10.000000}{12.000000}\selectfont \(\displaystyle {75}\)}%
\end{pgfscope}%
\begin{pgfscope}%
\pgfsetbuttcap%
\pgfsetroundjoin%
\definecolor{currentfill}{rgb}{0.000000,0.000000,0.000000}%
\pgfsetfillcolor{currentfill}%
\pgfsetlinewidth{0.803000pt}%
\definecolor{currentstroke}{rgb}{0.000000,0.000000,0.000000}%
\pgfsetstrokecolor{currentstroke}%
\pgfsetdash{}{0pt}%
\pgfsys@defobject{currentmarker}{\pgfqpoint{0.000000in}{-0.048611in}}{\pgfqpoint{0.000000in}{0.000000in}}{%
\pgfpathmoveto{\pgfqpoint{0.000000in}{0.000000in}}%
\pgfpathlineto{\pgfqpoint{0.000000in}{-0.048611in}}%
\pgfusepath{stroke,fill}%
}%
\begin{pgfscope}%
\pgfsys@transformshift{6.146222in}{0.564143in}%
\pgfsys@useobject{currentmarker}{}%
\end{pgfscope}%
\end{pgfscope}%
\begin{pgfscope}%
\definecolor{textcolor}{rgb}{0.000000,0.000000,0.000000}%
\pgfsetstrokecolor{textcolor}%
\pgfsetfillcolor{textcolor}%
\pgftext[x=6.146222in,y=0.466921in,,top]{\color{textcolor}\rmfamily\fontsize{10.000000}{12.000000}\selectfont \(\displaystyle {100}\)}%
\end{pgfscope}%
\begin{pgfscope}%
\definecolor{textcolor}{rgb}{0.000000,0.000000,0.000000}%
\pgfsetstrokecolor{textcolor}%
\pgfsetfillcolor{textcolor}%
\pgftext[x=3.432055in,y=0.287909in,,top]{\color{textcolor}\rmfamily\fontsize{10.000000}{12.000000}\selectfont Frequency (Hz)}%
\end{pgfscope}%
\begin{pgfscope}%
\pgfsetbuttcap%
\pgfsetroundjoin%
\definecolor{currentfill}{rgb}{0.000000,0.000000,0.000000}%
\pgfsetfillcolor{currentfill}%
\pgfsetlinewidth{0.803000pt}%
\definecolor{currentstroke}{rgb}{0.000000,0.000000,0.000000}%
\pgfsetstrokecolor{currentstroke}%
\pgfsetdash{}{0pt}%
\pgfsys@defobject{currentmarker}{\pgfqpoint{-0.048611in}{0.000000in}}{\pgfqpoint{0.000000in}{0.000000in}}{%
\pgfpathmoveto{\pgfqpoint{0.000000in}{0.000000in}}%
\pgfpathlineto{\pgfqpoint{-0.048611in}{0.000000in}}%
\pgfusepath{stroke,fill}%
}%
\begin{pgfscope}%
\pgfsys@transformshift{0.717889in}{0.599919in}%
\pgfsys@useobject{currentmarker}{}%
\end{pgfscope}%
\end{pgfscope}%
\begin{pgfscope}%
\definecolor{textcolor}{rgb}{0.000000,0.000000,0.000000}%
\pgfsetstrokecolor{textcolor}%
\pgfsetfillcolor{textcolor}%
\pgftext[x=0.551222in, y=0.551693in, left, base]{\color{textcolor}\rmfamily\fontsize{10.000000}{12.000000}\selectfont \(\displaystyle {0}\)}%
\end{pgfscope}%
\begin{pgfscope}%
\pgfsetbuttcap%
\pgfsetroundjoin%
\definecolor{currentfill}{rgb}{0.000000,0.000000,0.000000}%
\pgfsetfillcolor{currentfill}%
\pgfsetlinewidth{0.803000pt}%
\definecolor{currentstroke}{rgb}{0.000000,0.000000,0.000000}%
\pgfsetstrokecolor{currentstroke}%
\pgfsetdash{}{0pt}%
\pgfsys@defobject{currentmarker}{\pgfqpoint{-0.048611in}{0.000000in}}{\pgfqpoint{0.000000in}{0.000000in}}{%
\pgfpathmoveto{\pgfqpoint{0.000000in}{0.000000in}}%
\pgfpathlineto{\pgfqpoint{-0.048611in}{0.000000in}}%
\pgfusepath{stroke,fill}%
}%
\begin{pgfscope}%
\pgfsys@transformshift{0.717889in}{1.216536in}%
\pgfsys@useobject{currentmarker}{}%
\end{pgfscope}%
\end{pgfscope}%
\begin{pgfscope}%
\definecolor{textcolor}{rgb}{0.000000,0.000000,0.000000}%
\pgfsetstrokecolor{textcolor}%
\pgfsetfillcolor{textcolor}%
\pgftext[x=0.551222in, y=1.168311in, left, base]{\color{textcolor}\rmfamily\fontsize{10.000000}{12.000000}\selectfont \(\displaystyle {2}\)}%
\end{pgfscope}%
\begin{pgfscope}%
\definecolor{textcolor}{rgb}{0.000000,0.000000,0.000000}%
\pgfsetstrokecolor{textcolor}%
\pgfsetfillcolor{textcolor}%
\pgftext[x=0.495666in,y=0.957751in,,bottom,rotate=90.000000]{\color{textcolor}\rmfamily\fontsize{10.000000}{12.000000}\selectfont abs(Y(f)) (\(\displaystyle \mu V^2\))}%
\end{pgfscope}%
\begin{pgfscope}%
\definecolor{textcolor}{rgb}{0.000000,0.000000,0.000000}%
\pgfsetstrokecolor{textcolor}%
\pgfsetfillcolor{textcolor}%
\pgftext[x=0.717889in,y=1.393025in,left,base]{\color{textcolor}\rmfamily\fontsize{10.000000}{12.000000}\selectfont \(\displaystyle \times{10^{6}}{}\)}%
\end{pgfscope}%
\begin{pgfscope}%
\pgfpathrectangle{\pgfqpoint{0.717889in}{0.564143in}}{\pgfqpoint{5.428334in}{0.787215in}}%
\pgfusepath{clip}%
\pgfsetrectcap%
\pgfsetroundjoin%
\pgfsetlinewidth{1.505625pt}%
\definecolor{currentstroke}{rgb}{0.121569,0.466667,0.705882}%
\pgfsetstrokecolor{currentstroke}%
\pgfsetdash{}{0pt}%
\pgfpathmoveto{\pgfqpoint{3.432055in}{0.629520in}}%
\pgfpathlineto{\pgfqpoint{3.432611in}{0.623656in}}%
\pgfpathlineto{\pgfqpoint{3.433167in}{0.661978in}}%
\pgfpathlineto{\pgfqpoint{3.433723in}{0.657398in}}%
\pgfpathlineto{\pgfqpoint{3.434279in}{0.629125in}}%
\pgfpathlineto{\pgfqpoint{3.434835in}{0.738625in}}%
\pgfpathlineto{\pgfqpoint{3.435391in}{0.679923in}}%
\pgfpathlineto{\pgfqpoint{3.435946in}{0.622166in}}%
\pgfpathlineto{\pgfqpoint{3.437058in}{0.895775in}}%
\pgfpathlineto{\pgfqpoint{3.437614in}{0.823773in}}%
\pgfpathlineto{\pgfqpoint{3.438170in}{0.612411in}}%
\pgfpathlineto{\pgfqpoint{3.440393in}{1.154218in}}%
\pgfpathlineto{\pgfqpoint{3.440949in}{1.315576in}}%
\pgfpathlineto{\pgfqpoint{3.441505in}{0.732049in}}%
\pgfpathlineto{\pgfqpoint{3.442061in}{0.781863in}}%
\pgfpathlineto{\pgfqpoint{3.442617in}{1.117537in}}%
\pgfpathlineto{\pgfqpoint{3.443173in}{0.864768in}}%
\pgfpathlineto{\pgfqpoint{3.444840in}{0.758445in}}%
\pgfpathlineto{\pgfqpoint{3.445396in}{0.638562in}}%
\pgfpathlineto{\pgfqpoint{3.445952in}{0.711149in}}%
\pgfpathlineto{\pgfqpoint{3.447064in}{0.892694in}}%
\pgfpathlineto{\pgfqpoint{3.447620in}{0.873028in}}%
\pgfpathlineto{\pgfqpoint{3.448175in}{0.633691in}}%
\pgfpathlineto{\pgfqpoint{3.448731in}{0.701383in}}%
\pgfpathlineto{\pgfqpoint{3.449287in}{0.868118in}}%
\pgfpathlineto{\pgfqpoint{3.449843in}{0.695839in}}%
\pgfpathlineto{\pgfqpoint{3.450399in}{0.867531in}}%
\pgfpathlineto{\pgfqpoint{3.451511in}{0.697686in}}%
\pgfpathlineto{\pgfqpoint{3.452066in}{0.702838in}}%
\pgfpathlineto{\pgfqpoint{3.452622in}{0.871016in}}%
\pgfpathlineto{\pgfqpoint{3.453178in}{0.748850in}}%
\pgfpathlineto{\pgfqpoint{3.454290in}{0.699604in}}%
\pgfpathlineto{\pgfqpoint{3.454846in}{0.822450in}}%
\pgfpathlineto{\pgfqpoint{3.455402in}{0.617108in}}%
\pgfpathlineto{\pgfqpoint{3.455957in}{0.651860in}}%
\pgfpathlineto{\pgfqpoint{3.458737in}{0.924833in}}%
\pgfpathlineto{\pgfqpoint{3.460404in}{0.736076in}}%
\pgfpathlineto{\pgfqpoint{3.460960in}{0.786469in}}%
\pgfpathlineto{\pgfqpoint{3.461516in}{0.654834in}}%
\pgfpathlineto{\pgfqpoint{3.462072in}{0.673783in}}%
\pgfpathlineto{\pgfqpoint{3.462628in}{0.721071in}}%
\pgfpathlineto{\pgfqpoint{3.463184in}{0.712456in}}%
\pgfpathlineto{\pgfqpoint{3.463740in}{0.679929in}}%
\pgfpathlineto{\pgfqpoint{3.464295in}{0.715293in}}%
\pgfpathlineto{\pgfqpoint{3.464851in}{0.713447in}}%
\pgfpathlineto{\pgfqpoint{3.465963in}{0.638247in}}%
\pgfpathlineto{\pgfqpoint{3.466519in}{0.647732in}}%
\pgfpathlineto{\pgfqpoint{3.467075in}{0.638621in}}%
\pgfpathlineto{\pgfqpoint{3.467631in}{0.645913in}}%
\pgfpathlineto{\pgfqpoint{3.468186in}{0.651100in}}%
\pgfpathlineto{\pgfqpoint{3.468742in}{0.685964in}}%
\pgfpathlineto{\pgfqpoint{3.469298in}{0.615192in}}%
\pgfpathlineto{\pgfqpoint{3.469854in}{0.643030in}}%
\pgfpathlineto{\pgfqpoint{3.470410in}{0.766370in}}%
\pgfpathlineto{\pgfqpoint{3.470966in}{0.716574in}}%
\pgfpathlineto{\pgfqpoint{3.471522in}{0.753012in}}%
\pgfpathlineto{\pgfqpoint{3.472077in}{0.723355in}}%
\pgfpathlineto{\pgfqpoint{3.472633in}{0.710637in}}%
\pgfpathlineto{\pgfqpoint{3.473189in}{0.726963in}}%
\pgfpathlineto{\pgfqpoint{3.473745in}{0.633543in}}%
\pgfpathlineto{\pgfqpoint{3.474301in}{0.751041in}}%
\pgfpathlineto{\pgfqpoint{3.474857in}{0.711782in}}%
\pgfpathlineto{\pgfqpoint{3.475968in}{0.745569in}}%
\pgfpathlineto{\pgfqpoint{3.476524in}{0.619402in}}%
\pgfpathlineto{\pgfqpoint{3.477080in}{0.685241in}}%
\pgfpathlineto{\pgfqpoint{3.477636in}{0.620587in}}%
\pgfpathlineto{\pgfqpoint{3.478192in}{0.658671in}}%
\pgfpathlineto{\pgfqpoint{3.478748in}{0.673780in}}%
\pgfpathlineto{\pgfqpoint{3.480415in}{0.619577in}}%
\pgfpathlineto{\pgfqpoint{3.480971in}{0.689819in}}%
\pgfpathlineto{\pgfqpoint{3.481527in}{0.675876in}}%
\pgfpathlineto{\pgfqpoint{3.483195in}{0.630726in}}%
\pgfpathlineto{\pgfqpoint{3.483751in}{0.668155in}}%
\pgfpathlineto{\pgfqpoint{3.484306in}{0.645720in}}%
\pgfpathlineto{\pgfqpoint{3.485418in}{0.608801in}}%
\pgfpathlineto{\pgfqpoint{3.485974in}{0.719233in}}%
\pgfpathlineto{\pgfqpoint{3.486530in}{0.664181in}}%
\pgfpathlineto{\pgfqpoint{3.487086in}{0.713098in}}%
\pgfpathlineto{\pgfqpoint{3.487642in}{0.620593in}}%
\pgfpathlineto{\pgfqpoint{3.488197in}{0.703727in}}%
\pgfpathlineto{\pgfqpoint{3.488753in}{0.662858in}}%
\pgfpathlineto{\pgfqpoint{3.489309in}{0.959602in}}%
\pgfpathlineto{\pgfqpoint{3.489865in}{0.887217in}}%
\pgfpathlineto{\pgfqpoint{3.490977in}{0.755675in}}%
\pgfpathlineto{\pgfqpoint{3.491533in}{0.803874in}}%
\pgfpathlineto{\pgfqpoint{3.492644in}{0.635312in}}%
\pgfpathlineto{\pgfqpoint{3.493200in}{0.694847in}}%
\pgfpathlineto{\pgfqpoint{3.493756in}{0.621865in}}%
\pgfpathlineto{\pgfqpoint{3.494312in}{0.690273in}}%
\pgfpathlineto{\pgfqpoint{3.494868in}{0.698770in}}%
\pgfpathlineto{\pgfqpoint{3.495424in}{0.675523in}}%
\pgfpathlineto{\pgfqpoint{3.495979in}{0.698032in}}%
\pgfpathlineto{\pgfqpoint{3.496535in}{0.694102in}}%
\pgfpathlineto{\pgfqpoint{3.497091in}{0.817804in}}%
\pgfpathlineto{\pgfqpoint{3.497647in}{0.619213in}}%
\pgfpathlineto{\pgfqpoint{3.498203in}{0.766098in}}%
\pgfpathlineto{\pgfqpoint{3.498759in}{0.671591in}}%
\pgfpathlineto{\pgfqpoint{3.499315in}{0.718337in}}%
\pgfpathlineto{\pgfqpoint{3.499870in}{0.788370in}}%
\pgfpathlineto{\pgfqpoint{3.500426in}{0.689152in}}%
\pgfpathlineto{\pgfqpoint{3.502094in}{0.882982in}}%
\pgfpathlineto{\pgfqpoint{3.502650in}{0.698591in}}%
\pgfpathlineto{\pgfqpoint{3.503206in}{0.872096in}}%
\pgfpathlineto{\pgfqpoint{3.503762in}{0.746891in}}%
\pgfpathlineto{\pgfqpoint{3.504317in}{0.774768in}}%
\pgfpathlineto{\pgfqpoint{3.504873in}{0.785012in}}%
\pgfpathlineto{\pgfqpoint{3.505429in}{0.718927in}}%
\pgfpathlineto{\pgfqpoint{3.507097in}{0.874743in}}%
\pgfpathlineto{\pgfqpoint{3.509320in}{0.730698in}}%
\pgfpathlineto{\pgfqpoint{3.509876in}{0.708979in}}%
\pgfpathlineto{\pgfqpoint{3.510432in}{0.628427in}}%
\pgfpathlineto{\pgfqpoint{3.510988in}{0.754861in}}%
\pgfpathlineto{\pgfqpoint{3.511544in}{0.652527in}}%
\pgfpathlineto{\pgfqpoint{3.512099in}{0.658963in}}%
\pgfpathlineto{\pgfqpoint{3.513767in}{0.635915in}}%
\pgfpathlineto{\pgfqpoint{3.514323in}{0.668994in}}%
\pgfpathlineto{\pgfqpoint{3.514879in}{0.655884in}}%
\pgfpathlineto{\pgfqpoint{3.515435in}{0.611188in}}%
\pgfpathlineto{\pgfqpoint{3.515990in}{0.654740in}}%
\pgfpathlineto{\pgfqpoint{3.516546in}{0.668982in}}%
\pgfpathlineto{\pgfqpoint{3.517658in}{0.629802in}}%
\pgfpathlineto{\pgfqpoint{3.518214in}{0.634517in}}%
\pgfpathlineto{\pgfqpoint{3.518770in}{0.657239in}}%
\pgfpathlineto{\pgfqpoint{3.520437in}{0.607239in}}%
\pgfpathlineto{\pgfqpoint{3.520993in}{0.620687in}}%
\pgfpathlineto{\pgfqpoint{3.521549in}{0.668321in}}%
\pgfpathlineto{\pgfqpoint{3.522105in}{0.656486in}}%
\pgfpathlineto{\pgfqpoint{3.522661in}{0.612585in}}%
\pgfpathlineto{\pgfqpoint{3.523217in}{0.617408in}}%
\pgfpathlineto{\pgfqpoint{3.524328in}{0.641681in}}%
\pgfpathlineto{\pgfqpoint{3.524884in}{0.631938in}}%
\pgfpathlineto{\pgfqpoint{3.525440in}{0.626766in}}%
\pgfpathlineto{\pgfqpoint{3.525996in}{0.605736in}}%
\pgfpathlineto{\pgfqpoint{3.526552in}{0.626483in}}%
\pgfpathlineto{\pgfqpoint{3.527108in}{0.629306in}}%
\pgfpathlineto{\pgfqpoint{3.527664in}{0.664772in}}%
\pgfpathlineto{\pgfqpoint{3.528219in}{0.614502in}}%
\pgfpathlineto{\pgfqpoint{3.528775in}{0.642243in}}%
\pgfpathlineto{\pgfqpoint{3.529887in}{0.633078in}}%
\pgfpathlineto{\pgfqpoint{3.530999in}{0.659237in}}%
\pgfpathlineto{\pgfqpoint{3.531555in}{0.641388in}}%
\pgfpathlineto{\pgfqpoint{3.532666in}{0.677530in}}%
\pgfpathlineto{\pgfqpoint{3.534334in}{0.623472in}}%
\pgfpathlineto{\pgfqpoint{3.534890in}{0.607293in}}%
\pgfpathlineto{\pgfqpoint{3.537113in}{0.681536in}}%
\pgfpathlineto{\pgfqpoint{3.538781in}{0.644674in}}%
\pgfpathlineto{\pgfqpoint{3.539337in}{0.653985in}}%
\pgfpathlineto{\pgfqpoint{3.539893in}{0.652100in}}%
\pgfpathlineto{\pgfqpoint{3.540448in}{0.656274in}}%
\pgfpathlineto{\pgfqpoint{3.541004in}{0.651390in}}%
\pgfpathlineto{\pgfqpoint{3.541560in}{0.636333in}}%
\pgfpathlineto{\pgfqpoint{3.542116in}{0.659786in}}%
\pgfpathlineto{\pgfqpoint{3.542672in}{0.623845in}}%
\pgfpathlineto{\pgfqpoint{3.543228in}{0.665806in}}%
\pgfpathlineto{\pgfqpoint{3.543784in}{0.664360in}}%
\pgfpathlineto{\pgfqpoint{3.544895in}{0.636547in}}%
\pgfpathlineto{\pgfqpoint{3.545451in}{0.656588in}}%
\pgfpathlineto{\pgfqpoint{3.546007in}{0.625693in}}%
\pgfpathlineto{\pgfqpoint{3.546563in}{0.661773in}}%
\pgfpathlineto{\pgfqpoint{3.547119in}{0.808851in}}%
\pgfpathlineto{\pgfqpoint{3.547675in}{0.651136in}}%
\pgfpathlineto{\pgfqpoint{3.548230in}{0.687581in}}%
\pgfpathlineto{\pgfqpoint{3.549342in}{0.658003in}}%
\pgfpathlineto{\pgfqpoint{3.549898in}{0.660340in}}%
\pgfpathlineto{\pgfqpoint{3.551010in}{0.679786in}}%
\pgfpathlineto{\pgfqpoint{3.551566in}{0.649659in}}%
\pgfpathlineto{\pgfqpoint{3.552121in}{0.661240in}}%
\pgfpathlineto{\pgfqpoint{3.555457in}{0.608741in}}%
\pgfpathlineto{\pgfqpoint{3.556012in}{0.621483in}}%
\pgfpathlineto{\pgfqpoint{3.557124in}{0.629482in}}%
\pgfpathlineto{\pgfqpoint{3.557680in}{0.635185in}}%
\pgfpathlineto{\pgfqpoint{3.558792in}{0.613360in}}%
\pgfpathlineto{\pgfqpoint{3.559348in}{0.632629in}}%
\pgfpathlineto{\pgfqpoint{3.559904in}{0.616594in}}%
\pgfpathlineto{\pgfqpoint{3.561015in}{0.644620in}}%
\pgfpathlineto{\pgfqpoint{3.561571in}{0.618482in}}%
\pgfpathlineto{\pgfqpoint{3.562127in}{0.632429in}}%
\pgfpathlineto{\pgfqpoint{3.564906in}{0.613911in}}%
\pgfpathlineto{\pgfqpoint{3.565462in}{0.635425in}}%
\pgfpathlineto{\pgfqpoint{3.566018in}{0.620794in}}%
\pgfpathlineto{\pgfqpoint{3.566574in}{0.626832in}}%
\pgfpathlineto{\pgfqpoint{3.567130in}{0.622467in}}%
\pgfpathlineto{\pgfqpoint{3.567686in}{0.654411in}}%
\pgfpathlineto{\pgfqpoint{3.568241in}{0.612405in}}%
\pgfpathlineto{\pgfqpoint{3.568797in}{0.638771in}}%
\pgfpathlineto{\pgfqpoint{3.570465in}{0.604179in}}%
\pgfpathlineto{\pgfqpoint{3.572688in}{0.656467in}}%
\pgfpathlineto{\pgfqpoint{3.573244in}{0.631842in}}%
\pgfpathlineto{\pgfqpoint{3.573800in}{0.673101in}}%
\pgfpathlineto{\pgfqpoint{3.574356in}{0.628970in}}%
\pgfpathlineto{\pgfqpoint{3.574912in}{0.644014in}}%
\pgfpathlineto{\pgfqpoint{3.576579in}{0.618072in}}%
\pgfpathlineto{\pgfqpoint{3.577135in}{0.620609in}}%
\pgfpathlineto{\pgfqpoint{3.578803in}{0.648079in}}%
\pgfpathlineto{\pgfqpoint{3.579359in}{0.610930in}}%
\pgfpathlineto{\pgfqpoint{3.579915in}{0.652166in}}%
\pgfpathlineto{\pgfqpoint{3.580470in}{0.632549in}}%
\pgfpathlineto{\pgfqpoint{3.581026in}{0.631631in}}%
\pgfpathlineto{\pgfqpoint{3.582138in}{0.601082in}}%
\pgfpathlineto{\pgfqpoint{3.583806in}{0.630557in}}%
\pgfpathlineto{\pgfqpoint{3.584361in}{0.604076in}}%
\pgfpathlineto{\pgfqpoint{3.584917in}{0.607598in}}%
\pgfpathlineto{\pgfqpoint{3.586029in}{0.622861in}}%
\pgfpathlineto{\pgfqpoint{3.586585in}{0.606759in}}%
\pgfpathlineto{\pgfqpoint{3.587141in}{0.610562in}}%
\pgfpathlineto{\pgfqpoint{3.588252in}{0.616759in}}%
\pgfpathlineto{\pgfqpoint{3.588808in}{0.625307in}}%
\pgfpathlineto{\pgfqpoint{3.589364in}{0.602375in}}%
\pgfpathlineto{\pgfqpoint{3.589920in}{0.617280in}}%
\pgfpathlineto{\pgfqpoint{3.590476in}{0.616864in}}%
\pgfpathlineto{\pgfqpoint{3.591588in}{0.635858in}}%
\pgfpathlineto{\pgfqpoint{3.592143in}{0.624471in}}%
\pgfpathlineto{\pgfqpoint{3.592699in}{0.601963in}}%
\pgfpathlineto{\pgfqpoint{3.593811in}{0.636417in}}%
\pgfpathlineto{\pgfqpoint{3.595479in}{0.609984in}}%
\pgfpathlineto{\pgfqpoint{3.596035in}{0.609871in}}%
\pgfpathlineto{\pgfqpoint{3.597702in}{0.624461in}}%
\pgfpathlineto{\pgfqpoint{3.598258in}{0.618876in}}%
\pgfpathlineto{\pgfqpoint{3.598814in}{0.670484in}}%
\pgfpathlineto{\pgfqpoint{3.599370in}{0.604233in}}%
\pgfpathlineto{\pgfqpoint{3.599926in}{0.647208in}}%
\pgfpathlineto{\pgfqpoint{3.600481in}{0.614832in}}%
\pgfpathlineto{\pgfqpoint{3.601037in}{0.623416in}}%
\pgfpathlineto{\pgfqpoint{3.601593in}{0.632076in}}%
\pgfpathlineto{\pgfqpoint{3.602149in}{0.745036in}}%
\pgfpathlineto{\pgfqpoint{3.602705in}{0.650027in}}%
\pgfpathlineto{\pgfqpoint{3.603817in}{0.690683in}}%
\pgfpathlineto{\pgfqpoint{3.604372in}{0.871824in}}%
\pgfpathlineto{\pgfqpoint{3.604928in}{0.715884in}}%
\pgfpathlineto{\pgfqpoint{3.606040in}{0.627592in}}%
\pgfpathlineto{\pgfqpoint{3.606596in}{0.735202in}}%
\pgfpathlineto{\pgfqpoint{3.607152in}{0.676335in}}%
\pgfpathlineto{\pgfqpoint{3.607708in}{0.698341in}}%
\pgfpathlineto{\pgfqpoint{3.608263in}{0.689556in}}%
\pgfpathlineto{\pgfqpoint{3.609931in}{0.636020in}}%
\pgfpathlineto{\pgfqpoint{3.611599in}{0.729344in}}%
\pgfpathlineto{\pgfqpoint{3.613822in}{0.634242in}}%
\pgfpathlineto{\pgfqpoint{3.614378in}{0.639793in}}%
\pgfpathlineto{\pgfqpoint{3.616046in}{0.621224in}}%
\pgfpathlineto{\pgfqpoint{3.616601in}{0.630031in}}%
\pgfpathlineto{\pgfqpoint{3.617157in}{0.623650in}}%
\pgfpathlineto{\pgfqpoint{3.618825in}{0.605037in}}%
\pgfpathlineto{\pgfqpoint{3.620492in}{0.620657in}}%
\pgfpathlineto{\pgfqpoint{3.621604in}{0.602595in}}%
\pgfpathlineto{\pgfqpoint{3.622716in}{0.620083in}}%
\pgfpathlineto{\pgfqpoint{3.623272in}{0.604364in}}%
\pgfpathlineto{\pgfqpoint{3.623828in}{0.617459in}}%
\pgfpathlineto{\pgfqpoint{3.624383in}{0.632730in}}%
\pgfpathlineto{\pgfqpoint{3.624939in}{0.604529in}}%
\pgfpathlineto{\pgfqpoint{3.625495in}{0.615701in}}%
\pgfpathlineto{\pgfqpoint{3.626607in}{0.606981in}}%
\pgfpathlineto{\pgfqpoint{3.627163in}{0.616901in}}%
\pgfpathlineto{\pgfqpoint{3.627719in}{0.606093in}}%
\pgfpathlineto{\pgfqpoint{3.628274in}{0.613198in}}%
\pgfpathlineto{\pgfqpoint{3.628830in}{0.612172in}}%
\pgfpathlineto{\pgfqpoint{3.629386in}{0.616647in}}%
\pgfpathlineto{\pgfqpoint{3.630498in}{0.607152in}}%
\pgfpathlineto{\pgfqpoint{3.631054in}{0.629665in}}%
\pgfpathlineto{\pgfqpoint{3.631610in}{0.609023in}}%
\pgfpathlineto{\pgfqpoint{3.632165in}{0.613307in}}%
\pgfpathlineto{\pgfqpoint{3.632721in}{0.609045in}}%
\pgfpathlineto{\pgfqpoint{3.633277in}{0.636433in}}%
\pgfpathlineto{\pgfqpoint{3.633833in}{0.629559in}}%
\pgfpathlineto{\pgfqpoint{3.636057in}{0.607780in}}%
\pgfpathlineto{\pgfqpoint{3.636612in}{0.620037in}}%
\pgfpathlineto{\pgfqpoint{3.637168in}{0.616266in}}%
\pgfpathlineto{\pgfqpoint{3.637724in}{0.616379in}}%
\pgfpathlineto{\pgfqpoint{3.638280in}{0.623577in}}%
\pgfpathlineto{\pgfqpoint{3.638836in}{0.604605in}}%
\pgfpathlineto{\pgfqpoint{3.639392in}{0.607899in}}%
\pgfpathlineto{\pgfqpoint{3.641615in}{0.620013in}}%
\pgfpathlineto{\pgfqpoint{3.643283in}{0.607839in}}%
\pgfpathlineto{\pgfqpoint{3.643839in}{0.612216in}}%
\pgfpathlineto{\pgfqpoint{3.644394in}{0.604645in}}%
\pgfpathlineto{\pgfqpoint{3.644950in}{0.605076in}}%
\pgfpathlineto{\pgfqpoint{3.646062in}{0.628015in}}%
\pgfpathlineto{\pgfqpoint{3.646618in}{0.617380in}}%
\pgfpathlineto{\pgfqpoint{3.649397in}{0.642359in}}%
\pgfpathlineto{\pgfqpoint{3.651621in}{0.607152in}}%
\pgfpathlineto{\pgfqpoint{3.652177in}{0.623557in}}%
\pgfpathlineto{\pgfqpoint{3.652732in}{0.620482in}}%
\pgfpathlineto{\pgfqpoint{3.653288in}{0.608329in}}%
\pgfpathlineto{\pgfqpoint{3.654956in}{0.635989in}}%
\pgfpathlineto{\pgfqpoint{3.655512in}{0.625318in}}%
\pgfpathlineto{\pgfqpoint{3.656068in}{0.662160in}}%
\pgfpathlineto{\pgfqpoint{3.656623in}{0.608087in}}%
\pgfpathlineto{\pgfqpoint{3.657179in}{0.656259in}}%
\pgfpathlineto{\pgfqpoint{3.657735in}{0.614342in}}%
\pgfpathlineto{\pgfqpoint{3.659403in}{0.796320in}}%
\pgfpathlineto{\pgfqpoint{3.659959in}{0.652463in}}%
\pgfpathlineto{\pgfqpoint{3.660514in}{0.677790in}}%
\pgfpathlineto{\pgfqpoint{3.661070in}{0.665045in}}%
\pgfpathlineto{\pgfqpoint{3.661626in}{0.759492in}}%
\pgfpathlineto{\pgfqpoint{3.662182in}{0.753940in}}%
\pgfpathlineto{\pgfqpoint{3.662738in}{0.633377in}}%
\pgfpathlineto{\pgfqpoint{3.663294in}{0.724325in}}%
\pgfpathlineto{\pgfqpoint{3.663850in}{0.781810in}}%
\pgfpathlineto{\pgfqpoint{3.664405in}{0.734020in}}%
\pgfpathlineto{\pgfqpoint{3.665517in}{0.660713in}}%
\pgfpathlineto{\pgfqpoint{3.666073in}{0.690628in}}%
\pgfpathlineto{\pgfqpoint{3.667185in}{0.635053in}}%
\pgfpathlineto{\pgfqpoint{3.667741in}{0.664819in}}%
\pgfpathlineto{\pgfqpoint{3.668296in}{0.693822in}}%
\pgfpathlineto{\pgfqpoint{3.668852in}{0.666442in}}%
\pgfpathlineto{\pgfqpoint{3.669408in}{0.613196in}}%
\pgfpathlineto{\pgfqpoint{3.671632in}{0.726113in}}%
\pgfpathlineto{\pgfqpoint{3.672743in}{0.690189in}}%
\pgfpathlineto{\pgfqpoint{3.673299in}{0.641731in}}%
\pgfpathlineto{\pgfqpoint{3.673855in}{0.647533in}}%
\pgfpathlineto{\pgfqpoint{3.674411in}{0.644360in}}%
\pgfpathlineto{\pgfqpoint{3.676079in}{0.606705in}}%
\pgfpathlineto{\pgfqpoint{3.676634in}{0.611625in}}%
\pgfpathlineto{\pgfqpoint{3.677190in}{0.623407in}}%
\pgfpathlineto{\pgfqpoint{3.677746in}{0.606271in}}%
\pgfpathlineto{\pgfqpoint{3.678302in}{0.616750in}}%
\pgfpathlineto{\pgfqpoint{3.678858in}{0.613485in}}%
\pgfpathlineto{\pgfqpoint{3.679414in}{0.617764in}}%
\pgfpathlineto{\pgfqpoint{3.679970in}{0.601813in}}%
\pgfpathlineto{\pgfqpoint{3.680525in}{0.618950in}}%
\pgfpathlineto{\pgfqpoint{3.681081in}{0.605189in}}%
\pgfpathlineto{\pgfqpoint{3.681637in}{0.615231in}}%
\pgfpathlineto{\pgfqpoint{3.682193in}{0.603576in}}%
\pgfpathlineto{\pgfqpoint{3.682749in}{0.611870in}}%
\pgfpathlineto{\pgfqpoint{3.683305in}{0.621776in}}%
\pgfpathlineto{\pgfqpoint{3.683861in}{0.611921in}}%
\pgfpathlineto{\pgfqpoint{3.684972in}{0.618521in}}%
\pgfpathlineto{\pgfqpoint{3.685528in}{0.604832in}}%
\pgfpathlineto{\pgfqpoint{3.686084in}{0.616927in}}%
\pgfpathlineto{\pgfqpoint{3.687196in}{0.613427in}}%
\pgfpathlineto{\pgfqpoint{3.687752in}{0.624238in}}%
\pgfpathlineto{\pgfqpoint{3.688307in}{0.614784in}}%
\pgfpathlineto{\pgfqpoint{3.688863in}{0.617500in}}%
\pgfpathlineto{\pgfqpoint{3.691087in}{0.604324in}}%
\pgfpathlineto{\pgfqpoint{3.691643in}{0.628681in}}%
\pgfpathlineto{\pgfqpoint{3.692199in}{0.605175in}}%
\pgfpathlineto{\pgfqpoint{3.693310in}{0.618542in}}%
\pgfpathlineto{\pgfqpoint{3.693866in}{0.603449in}}%
\pgfpathlineto{\pgfqpoint{3.694422in}{0.624074in}}%
\pgfpathlineto{\pgfqpoint{3.694978in}{0.614043in}}%
\pgfpathlineto{\pgfqpoint{3.695534in}{0.602651in}}%
\pgfpathlineto{\pgfqpoint{3.696090in}{0.605770in}}%
\pgfpathlineto{\pgfqpoint{3.697757in}{0.626239in}}%
\pgfpathlineto{\pgfqpoint{3.698313in}{0.601769in}}%
\pgfpathlineto{\pgfqpoint{3.698869in}{0.619047in}}%
\pgfpathlineto{\pgfqpoint{3.699425in}{0.621808in}}%
\pgfpathlineto{\pgfqpoint{3.699981in}{0.631474in}}%
\pgfpathlineto{\pgfqpoint{3.700536in}{0.629154in}}%
\pgfpathlineto{\pgfqpoint{3.701648in}{0.606406in}}%
\pgfpathlineto{\pgfqpoint{3.702204in}{0.607709in}}%
\pgfpathlineto{\pgfqpoint{3.702760in}{0.615635in}}%
\pgfpathlineto{\pgfqpoint{3.704427in}{0.602272in}}%
\pgfpathlineto{\pgfqpoint{3.704983in}{0.602996in}}%
\pgfpathlineto{\pgfqpoint{3.706095in}{0.635130in}}%
\pgfpathlineto{\pgfqpoint{3.706651in}{0.615236in}}%
\pgfpathlineto{\pgfqpoint{3.707207in}{0.618462in}}%
\pgfpathlineto{\pgfqpoint{3.707763in}{0.626018in}}%
\pgfpathlineto{\pgfqpoint{3.708318in}{0.611919in}}%
\pgfpathlineto{\pgfqpoint{3.708874in}{0.615916in}}%
\pgfpathlineto{\pgfqpoint{3.709430in}{0.624737in}}%
\pgfpathlineto{\pgfqpoint{3.709986in}{0.607879in}}%
\pgfpathlineto{\pgfqpoint{3.710542in}{0.613675in}}%
\pgfpathlineto{\pgfqpoint{3.711654in}{0.621081in}}%
\pgfpathlineto{\pgfqpoint{3.712210in}{0.613141in}}%
\pgfpathlineto{\pgfqpoint{3.712765in}{0.617482in}}%
\pgfpathlineto{\pgfqpoint{3.714433in}{0.654607in}}%
\pgfpathlineto{\pgfqpoint{3.714989in}{0.609004in}}%
\pgfpathlineto{\pgfqpoint{3.716656in}{0.719108in}}%
\pgfpathlineto{\pgfqpoint{3.717212in}{0.711499in}}%
\pgfpathlineto{\pgfqpoint{3.718324in}{0.608318in}}%
\pgfpathlineto{\pgfqpoint{3.718880in}{0.653611in}}%
\pgfpathlineto{\pgfqpoint{3.719436in}{0.766332in}}%
\pgfpathlineto{\pgfqpoint{3.719992in}{0.668628in}}%
\pgfpathlineto{\pgfqpoint{3.720547in}{0.710535in}}%
\pgfpathlineto{\pgfqpoint{3.721103in}{0.661893in}}%
\pgfpathlineto{\pgfqpoint{3.721659in}{0.704388in}}%
\pgfpathlineto{\pgfqpoint{3.723327in}{0.662367in}}%
\pgfpathlineto{\pgfqpoint{3.723883in}{0.669749in}}%
\pgfpathlineto{\pgfqpoint{3.724438in}{0.665606in}}%
\pgfpathlineto{\pgfqpoint{3.726106in}{0.627492in}}%
\pgfpathlineto{\pgfqpoint{3.726662in}{0.618774in}}%
\pgfpathlineto{\pgfqpoint{3.727774in}{0.655671in}}%
\pgfpathlineto{\pgfqpoint{3.728885in}{0.649128in}}%
\pgfpathlineto{\pgfqpoint{3.729441in}{0.619862in}}%
\pgfpathlineto{\pgfqpoint{3.729997in}{0.644526in}}%
\pgfpathlineto{\pgfqpoint{3.731665in}{0.661134in}}%
\pgfpathlineto{\pgfqpoint{3.732221in}{0.656867in}}%
\pgfpathlineto{\pgfqpoint{3.732776in}{0.671098in}}%
\pgfpathlineto{\pgfqpoint{3.734444in}{0.631907in}}%
\pgfpathlineto{\pgfqpoint{3.735556in}{0.607196in}}%
\pgfpathlineto{\pgfqpoint{3.736112in}{0.617018in}}%
\pgfpathlineto{\pgfqpoint{3.736667in}{0.610020in}}%
\pgfpathlineto{\pgfqpoint{3.737223in}{0.606087in}}%
\pgfpathlineto{\pgfqpoint{3.737779in}{0.609622in}}%
\pgfpathlineto{\pgfqpoint{3.738891in}{0.616203in}}%
\pgfpathlineto{\pgfqpoint{3.739447in}{0.600595in}}%
\pgfpathlineto{\pgfqpoint{3.740003in}{0.607205in}}%
\pgfpathlineto{\pgfqpoint{3.740558in}{0.616409in}}%
\pgfpathlineto{\pgfqpoint{3.741114in}{0.607901in}}%
\pgfpathlineto{\pgfqpoint{3.741670in}{0.607764in}}%
\pgfpathlineto{\pgfqpoint{3.742226in}{0.602500in}}%
\pgfpathlineto{\pgfqpoint{3.742782in}{0.615422in}}%
\pgfpathlineto{\pgfqpoint{3.743338in}{0.604686in}}%
\pgfpathlineto{\pgfqpoint{3.745005in}{0.618821in}}%
\pgfpathlineto{\pgfqpoint{3.746117in}{0.605949in}}%
\pgfpathlineto{\pgfqpoint{3.746673in}{0.611888in}}%
\pgfpathlineto{\pgfqpoint{3.747229in}{0.616854in}}%
\pgfpathlineto{\pgfqpoint{3.747785in}{0.604752in}}%
\pgfpathlineto{\pgfqpoint{3.748341in}{0.625364in}}%
\pgfpathlineto{\pgfqpoint{3.748896in}{0.614163in}}%
\pgfpathlineto{\pgfqpoint{3.751120in}{0.606819in}}%
\pgfpathlineto{\pgfqpoint{3.751676in}{0.622314in}}%
\pgfpathlineto{\pgfqpoint{3.752232in}{0.622276in}}%
\pgfpathlineto{\pgfqpoint{3.752787in}{0.618528in}}%
\pgfpathlineto{\pgfqpoint{3.753343in}{0.623100in}}%
\pgfpathlineto{\pgfqpoint{3.753899in}{0.606437in}}%
\pgfpathlineto{\pgfqpoint{3.754455in}{0.615027in}}%
\pgfpathlineto{\pgfqpoint{3.755011in}{0.622557in}}%
\pgfpathlineto{\pgfqpoint{3.755567in}{0.608784in}}%
\pgfpathlineto{\pgfqpoint{3.756123in}{0.619006in}}%
\pgfpathlineto{\pgfqpoint{3.756678in}{0.612637in}}%
\pgfpathlineto{\pgfqpoint{3.757234in}{0.618479in}}%
\pgfpathlineto{\pgfqpoint{3.757790in}{0.625575in}}%
\pgfpathlineto{\pgfqpoint{3.758902in}{0.603830in}}%
\pgfpathlineto{\pgfqpoint{3.759458in}{0.608751in}}%
\pgfpathlineto{\pgfqpoint{3.760569in}{0.610132in}}%
\pgfpathlineto{\pgfqpoint{3.761125in}{0.625047in}}%
\pgfpathlineto{\pgfqpoint{3.762237in}{0.607806in}}%
\pgfpathlineto{\pgfqpoint{3.763905in}{0.623165in}}%
\pgfpathlineto{\pgfqpoint{3.764460in}{0.602481in}}%
\pgfpathlineto{\pgfqpoint{3.765016in}{0.616926in}}%
\pgfpathlineto{\pgfqpoint{3.766128in}{0.606270in}}%
\pgfpathlineto{\pgfqpoint{3.766684in}{0.636029in}}%
\pgfpathlineto{\pgfqpoint{3.767240in}{0.616483in}}%
\pgfpathlineto{\pgfqpoint{3.767796in}{0.610492in}}%
\pgfpathlineto{\pgfqpoint{3.768352in}{0.616160in}}%
\pgfpathlineto{\pgfqpoint{3.768907in}{0.622250in}}%
\pgfpathlineto{\pgfqpoint{3.770019in}{0.609804in}}%
\pgfpathlineto{\pgfqpoint{3.770575in}{0.611800in}}%
\pgfpathlineto{\pgfqpoint{3.771131in}{0.652083in}}%
\pgfpathlineto{\pgfqpoint{3.771687in}{0.623075in}}%
\pgfpathlineto{\pgfqpoint{3.772243in}{0.650710in}}%
\pgfpathlineto{\pgfqpoint{3.772798in}{0.640564in}}%
\pgfpathlineto{\pgfqpoint{3.773910in}{0.637786in}}%
\pgfpathlineto{\pgfqpoint{3.774466in}{0.734084in}}%
\pgfpathlineto{\pgfqpoint{3.775022in}{0.653262in}}%
\pgfpathlineto{\pgfqpoint{3.775578in}{0.620903in}}%
\pgfpathlineto{\pgfqpoint{3.776689in}{0.683398in}}%
\pgfpathlineto{\pgfqpoint{3.777245in}{0.627719in}}%
\pgfpathlineto{\pgfqpoint{3.777801in}{0.650086in}}%
\pgfpathlineto{\pgfqpoint{3.778357in}{0.644381in}}%
\pgfpathlineto{\pgfqpoint{3.778913in}{0.707400in}}%
\pgfpathlineto{\pgfqpoint{3.779469in}{0.657724in}}%
\pgfpathlineto{\pgfqpoint{3.780025in}{0.666765in}}%
\pgfpathlineto{\pgfqpoint{3.780580in}{0.619859in}}%
\pgfpathlineto{\pgfqpoint{3.781136in}{0.657659in}}%
\pgfpathlineto{\pgfqpoint{3.781692in}{0.677336in}}%
\pgfpathlineto{\pgfqpoint{3.782248in}{0.617874in}}%
\pgfpathlineto{\pgfqpoint{3.782804in}{0.630927in}}%
\pgfpathlineto{\pgfqpoint{3.783360in}{0.628765in}}%
\pgfpathlineto{\pgfqpoint{3.784472in}{0.620231in}}%
\pgfpathlineto{\pgfqpoint{3.785027in}{0.660396in}}%
\pgfpathlineto{\pgfqpoint{3.785583in}{0.625805in}}%
\pgfpathlineto{\pgfqpoint{3.786139in}{0.625255in}}%
\pgfpathlineto{\pgfqpoint{3.786695in}{0.621631in}}%
\pgfpathlineto{\pgfqpoint{3.788363in}{0.661101in}}%
\pgfpathlineto{\pgfqpoint{3.789474in}{0.617310in}}%
\pgfpathlineto{\pgfqpoint{3.791142in}{0.658866in}}%
\pgfpathlineto{\pgfqpoint{3.791698in}{0.657297in}}%
\pgfpathlineto{\pgfqpoint{3.793921in}{0.627701in}}%
\pgfpathlineto{\pgfqpoint{3.794477in}{0.630364in}}%
\pgfpathlineto{\pgfqpoint{3.795033in}{0.613508in}}%
\pgfpathlineto{\pgfqpoint{3.795589in}{0.625573in}}%
\pgfpathlineto{\pgfqpoint{3.796700in}{0.610577in}}%
\pgfpathlineto{\pgfqpoint{3.797256in}{0.617176in}}%
\pgfpathlineto{\pgfqpoint{3.797812in}{0.616057in}}%
\pgfpathlineto{\pgfqpoint{3.798368in}{0.611137in}}%
\pgfpathlineto{\pgfqpoint{3.798924in}{0.622899in}}%
\pgfpathlineto{\pgfqpoint{3.799480in}{0.600751in}}%
\pgfpathlineto{\pgfqpoint{3.800036in}{0.618608in}}%
\pgfpathlineto{\pgfqpoint{3.800591in}{0.607665in}}%
\pgfpathlineto{\pgfqpoint{3.801147in}{0.613436in}}%
\pgfpathlineto{\pgfqpoint{3.802815in}{0.621593in}}%
\pgfpathlineto{\pgfqpoint{3.803371in}{0.602877in}}%
\pgfpathlineto{\pgfqpoint{3.803927in}{0.617993in}}%
\pgfpathlineto{\pgfqpoint{3.806150in}{0.604520in}}%
\pgfpathlineto{\pgfqpoint{3.807262in}{0.614194in}}%
\pgfpathlineto{\pgfqpoint{3.807818in}{0.603705in}}%
\pgfpathlineto{\pgfqpoint{3.808374in}{0.608163in}}%
\pgfpathlineto{\pgfqpoint{3.808929in}{0.613424in}}%
\pgfpathlineto{\pgfqpoint{3.809485in}{0.608379in}}%
\pgfpathlineto{\pgfqpoint{3.811153in}{0.615080in}}%
\pgfpathlineto{\pgfqpoint{3.811709in}{0.613424in}}%
\pgfpathlineto{\pgfqpoint{3.812820in}{0.604191in}}%
\pgfpathlineto{\pgfqpoint{3.814488in}{0.609131in}}%
\pgfpathlineto{\pgfqpoint{3.815044in}{0.604787in}}%
\pgfpathlineto{\pgfqpoint{3.815600in}{0.606580in}}%
\pgfpathlineto{\pgfqpoint{3.816156in}{0.609873in}}%
\pgfpathlineto{\pgfqpoint{3.816711in}{0.605767in}}%
\pgfpathlineto{\pgfqpoint{3.817267in}{0.607304in}}%
\pgfpathlineto{\pgfqpoint{3.817823in}{0.615288in}}%
\pgfpathlineto{\pgfqpoint{3.818379in}{0.614064in}}%
\pgfpathlineto{\pgfqpoint{3.820047in}{0.612309in}}%
\pgfpathlineto{\pgfqpoint{3.820602in}{0.614137in}}%
\pgfpathlineto{\pgfqpoint{3.821158in}{0.621713in}}%
\pgfpathlineto{\pgfqpoint{3.821714in}{0.601650in}}%
\pgfpathlineto{\pgfqpoint{3.822270in}{0.619902in}}%
\pgfpathlineto{\pgfqpoint{3.822826in}{0.614557in}}%
\pgfpathlineto{\pgfqpoint{3.823382in}{0.621325in}}%
\pgfpathlineto{\pgfqpoint{3.823938in}{0.607209in}}%
\pgfpathlineto{\pgfqpoint{3.824494in}{0.617665in}}%
\pgfpathlineto{\pgfqpoint{3.826161in}{0.627823in}}%
\pgfpathlineto{\pgfqpoint{3.826717in}{0.604770in}}%
\pgfpathlineto{\pgfqpoint{3.828385in}{0.658490in}}%
\pgfpathlineto{\pgfqpoint{3.828940in}{0.649131in}}%
\pgfpathlineto{\pgfqpoint{3.829496in}{0.679887in}}%
\pgfpathlineto{\pgfqpoint{3.830052in}{0.614452in}}%
\pgfpathlineto{\pgfqpoint{3.830608in}{0.643898in}}%
\pgfpathlineto{\pgfqpoint{3.831164in}{0.653427in}}%
\pgfpathlineto{\pgfqpoint{3.831720in}{0.705008in}}%
\pgfpathlineto{\pgfqpoint{3.832276in}{0.673377in}}%
\pgfpathlineto{\pgfqpoint{3.833943in}{0.610247in}}%
\pgfpathlineto{\pgfqpoint{3.835611in}{0.694493in}}%
\pgfpathlineto{\pgfqpoint{3.837834in}{0.630282in}}%
\pgfpathlineto{\pgfqpoint{3.838946in}{0.667823in}}%
\pgfpathlineto{\pgfqpoint{3.840058in}{0.618677in}}%
\pgfpathlineto{\pgfqpoint{3.840614in}{0.633464in}}%
\pgfpathlineto{\pgfqpoint{3.841169in}{0.655446in}}%
\pgfpathlineto{\pgfqpoint{3.841725in}{0.613286in}}%
\pgfpathlineto{\pgfqpoint{3.842281in}{0.643342in}}%
\pgfpathlineto{\pgfqpoint{3.843393in}{0.609767in}}%
\pgfpathlineto{\pgfqpoint{3.843949in}{0.613072in}}%
\pgfpathlineto{\pgfqpoint{3.844505in}{0.658450in}}%
\pgfpathlineto{\pgfqpoint{3.845060in}{0.629476in}}%
\pgfpathlineto{\pgfqpoint{3.846172in}{0.613169in}}%
\pgfpathlineto{\pgfqpoint{3.847840in}{0.653091in}}%
\pgfpathlineto{\pgfqpoint{3.848951in}{0.615041in}}%
\pgfpathlineto{\pgfqpoint{3.850063in}{0.618941in}}%
\pgfpathlineto{\pgfqpoint{3.851731in}{0.645210in}}%
\pgfpathlineto{\pgfqpoint{3.852287in}{0.663310in}}%
\pgfpathlineto{\pgfqpoint{3.853954in}{0.627703in}}%
\pgfpathlineto{\pgfqpoint{3.854510in}{0.635537in}}%
\pgfpathlineto{\pgfqpoint{3.855066in}{0.628530in}}%
\pgfpathlineto{\pgfqpoint{3.856178in}{0.615253in}}%
\pgfpathlineto{\pgfqpoint{3.856733in}{0.617637in}}%
\pgfpathlineto{\pgfqpoint{3.857289in}{0.619085in}}%
\pgfpathlineto{\pgfqpoint{3.857845in}{0.605329in}}%
\pgfpathlineto{\pgfqpoint{3.858401in}{0.611696in}}%
\pgfpathlineto{\pgfqpoint{3.858957in}{0.616124in}}%
\pgfpathlineto{\pgfqpoint{3.859513in}{0.610752in}}%
\pgfpathlineto{\pgfqpoint{3.860069in}{0.618723in}}%
\pgfpathlineto{\pgfqpoint{3.860625in}{0.604374in}}%
\pgfpathlineto{\pgfqpoint{3.861180in}{0.608290in}}%
\pgfpathlineto{\pgfqpoint{3.862848in}{0.612634in}}%
\pgfpathlineto{\pgfqpoint{3.863404in}{0.606090in}}%
\pgfpathlineto{\pgfqpoint{3.864516in}{0.618627in}}%
\pgfpathlineto{\pgfqpoint{3.865071in}{0.618174in}}%
\pgfpathlineto{\pgfqpoint{3.865627in}{0.619876in}}%
\pgfpathlineto{\pgfqpoint{3.866183in}{0.606010in}}%
\pgfpathlineto{\pgfqpoint{3.866739in}{0.608556in}}%
\pgfpathlineto{\pgfqpoint{3.867295in}{0.617404in}}%
\pgfpathlineto{\pgfqpoint{3.867851in}{0.609949in}}%
\pgfpathlineto{\pgfqpoint{3.868407in}{0.611360in}}%
\pgfpathlineto{\pgfqpoint{3.868962in}{0.617360in}}%
\pgfpathlineto{\pgfqpoint{3.869518in}{0.604204in}}%
\pgfpathlineto{\pgfqpoint{3.870074in}{0.610458in}}%
\pgfpathlineto{\pgfqpoint{3.870630in}{0.613760in}}%
\pgfpathlineto{\pgfqpoint{3.871186in}{0.604110in}}%
\pgfpathlineto{\pgfqpoint{3.871742in}{0.612306in}}%
\pgfpathlineto{\pgfqpoint{3.872298in}{0.627963in}}%
\pgfpathlineto{\pgfqpoint{3.872853in}{0.613143in}}%
\pgfpathlineto{\pgfqpoint{3.873409in}{0.615406in}}%
\pgfpathlineto{\pgfqpoint{3.873965in}{0.613570in}}%
\pgfpathlineto{\pgfqpoint{3.874521in}{0.609965in}}%
\pgfpathlineto{\pgfqpoint{3.875077in}{0.619477in}}%
\pgfpathlineto{\pgfqpoint{3.875633in}{0.604360in}}%
\pgfpathlineto{\pgfqpoint{3.876189in}{0.620138in}}%
\pgfpathlineto{\pgfqpoint{3.876744in}{0.618961in}}%
\pgfpathlineto{\pgfqpoint{3.877300in}{0.617928in}}%
\pgfpathlineto{\pgfqpoint{3.877856in}{0.620888in}}%
\pgfpathlineto{\pgfqpoint{3.878968in}{0.604524in}}%
\pgfpathlineto{\pgfqpoint{3.879524in}{0.627345in}}%
\pgfpathlineto{\pgfqpoint{3.880080in}{0.611507in}}%
\pgfpathlineto{\pgfqpoint{3.880636in}{0.617086in}}%
\pgfpathlineto{\pgfqpoint{3.881191in}{0.608009in}}%
\pgfpathlineto{\pgfqpoint{3.881747in}{0.616434in}}%
\pgfpathlineto{\pgfqpoint{3.883971in}{0.627781in}}%
\pgfpathlineto{\pgfqpoint{3.884527in}{0.664976in}}%
\pgfpathlineto{\pgfqpoint{3.885082in}{0.627276in}}%
\pgfpathlineto{\pgfqpoint{3.885638in}{0.660822in}}%
\pgfpathlineto{\pgfqpoint{3.886750in}{0.679114in}}%
\pgfpathlineto{\pgfqpoint{3.888418in}{0.629702in}}%
\pgfpathlineto{\pgfqpoint{3.889529in}{0.740892in}}%
\pgfpathlineto{\pgfqpoint{3.890641in}{0.614330in}}%
\pgfpathlineto{\pgfqpoint{3.891197in}{0.667384in}}%
\pgfpathlineto{\pgfqpoint{3.891753in}{0.636748in}}%
\pgfpathlineto{\pgfqpoint{3.892864in}{0.683170in}}%
\pgfpathlineto{\pgfqpoint{3.893420in}{0.609807in}}%
\pgfpathlineto{\pgfqpoint{3.893976in}{0.706129in}}%
\pgfpathlineto{\pgfqpoint{3.894532in}{0.645511in}}%
\pgfpathlineto{\pgfqpoint{3.895088in}{0.694537in}}%
\pgfpathlineto{\pgfqpoint{3.895644in}{0.658078in}}%
\pgfpathlineto{\pgfqpoint{3.896755in}{0.706353in}}%
\pgfpathlineto{\pgfqpoint{3.897311in}{0.630832in}}%
\pgfpathlineto{\pgfqpoint{3.897867in}{0.641997in}}%
\pgfpathlineto{\pgfqpoint{3.898423in}{0.663832in}}%
\pgfpathlineto{\pgfqpoint{3.898979in}{0.644199in}}%
\pgfpathlineto{\pgfqpoint{3.899535in}{0.618950in}}%
\pgfpathlineto{\pgfqpoint{3.900091in}{0.659531in}}%
\pgfpathlineto{\pgfqpoint{3.900647in}{0.620989in}}%
\pgfpathlineto{\pgfqpoint{3.901202in}{0.608101in}}%
\pgfpathlineto{\pgfqpoint{3.901758in}{0.671951in}}%
\pgfpathlineto{\pgfqpoint{3.902314in}{0.632143in}}%
\pgfpathlineto{\pgfqpoint{3.902870in}{0.611178in}}%
\pgfpathlineto{\pgfqpoint{3.903426in}{0.626521in}}%
\pgfpathlineto{\pgfqpoint{3.905093in}{0.648057in}}%
\pgfpathlineto{\pgfqpoint{3.905649in}{0.619979in}}%
\pgfpathlineto{\pgfqpoint{3.906205in}{0.624606in}}%
\pgfpathlineto{\pgfqpoint{3.907317in}{0.673122in}}%
\pgfpathlineto{\pgfqpoint{3.909540in}{0.608246in}}%
\pgfpathlineto{\pgfqpoint{3.910096in}{0.619350in}}%
\pgfpathlineto{\pgfqpoint{3.911764in}{0.661829in}}%
\pgfpathlineto{\pgfqpoint{3.912320in}{0.661720in}}%
\pgfpathlineto{\pgfqpoint{3.913987in}{0.640406in}}%
\pgfpathlineto{\pgfqpoint{3.914543in}{0.631397in}}%
\pgfpathlineto{\pgfqpoint{3.915099in}{0.652510in}}%
\pgfpathlineto{\pgfqpoint{3.915655in}{0.642549in}}%
\pgfpathlineto{\pgfqpoint{3.917322in}{0.605627in}}%
\pgfpathlineto{\pgfqpoint{3.917878in}{0.608188in}}%
\pgfpathlineto{\pgfqpoint{3.918434in}{0.605365in}}%
\pgfpathlineto{\pgfqpoint{3.920102in}{0.613486in}}%
\pgfpathlineto{\pgfqpoint{3.920658in}{0.605072in}}%
\pgfpathlineto{\pgfqpoint{3.921213in}{0.609082in}}%
\pgfpathlineto{\pgfqpoint{3.922325in}{0.609919in}}%
\pgfpathlineto{\pgfqpoint{3.922881in}{0.611710in}}%
\pgfpathlineto{\pgfqpoint{3.923437in}{0.607391in}}%
\pgfpathlineto{\pgfqpoint{3.923993in}{0.617863in}}%
\pgfpathlineto{\pgfqpoint{3.924549in}{0.604426in}}%
\pgfpathlineto{\pgfqpoint{3.925104in}{0.607444in}}%
\pgfpathlineto{\pgfqpoint{3.925660in}{0.608408in}}%
\pgfpathlineto{\pgfqpoint{3.927328in}{0.602684in}}%
\pgfpathlineto{\pgfqpoint{3.928440in}{0.617261in}}%
\pgfpathlineto{\pgfqpoint{3.930107in}{0.606887in}}%
\pgfpathlineto{\pgfqpoint{3.931775in}{0.621499in}}%
\pgfpathlineto{\pgfqpoint{3.932331in}{0.602829in}}%
\pgfpathlineto{\pgfqpoint{3.932886in}{0.617897in}}%
\pgfpathlineto{\pgfqpoint{3.933442in}{0.621632in}}%
\pgfpathlineto{\pgfqpoint{3.933998in}{0.618851in}}%
\pgfpathlineto{\pgfqpoint{3.934554in}{0.612324in}}%
\pgfpathlineto{\pgfqpoint{3.935110in}{0.619499in}}%
\pgfpathlineto{\pgfqpoint{3.935666in}{0.602859in}}%
\pgfpathlineto{\pgfqpoint{3.936222in}{0.614246in}}%
\pgfpathlineto{\pgfqpoint{3.936778in}{0.614457in}}%
\pgfpathlineto{\pgfqpoint{3.937333in}{0.620876in}}%
\pgfpathlineto{\pgfqpoint{3.937889in}{0.613363in}}%
\pgfpathlineto{\pgfqpoint{3.939001in}{0.628909in}}%
\pgfpathlineto{\pgfqpoint{3.939557in}{0.606454in}}%
\pgfpathlineto{\pgfqpoint{3.941224in}{0.661001in}}%
\pgfpathlineto{\pgfqpoint{3.942892in}{0.628754in}}%
\pgfpathlineto{\pgfqpoint{3.944560in}{0.701966in}}%
\pgfpathlineto{\pgfqpoint{3.945671in}{0.632842in}}%
\pgfpathlineto{\pgfqpoint{3.946783in}{0.734600in}}%
\pgfpathlineto{\pgfqpoint{3.947895in}{0.618081in}}%
\pgfpathlineto{\pgfqpoint{3.948451in}{0.622045in}}%
\pgfpathlineto{\pgfqpoint{3.949006in}{0.623796in}}%
\pgfpathlineto{\pgfqpoint{3.949562in}{0.621222in}}%
\pgfpathlineto{\pgfqpoint{3.951230in}{0.690477in}}%
\pgfpathlineto{\pgfqpoint{3.952897in}{0.626898in}}%
\pgfpathlineto{\pgfqpoint{3.954009in}{0.701068in}}%
\pgfpathlineto{\pgfqpoint{3.955121in}{0.623285in}}%
\pgfpathlineto{\pgfqpoint{3.955677in}{0.625477in}}%
\pgfpathlineto{\pgfqpoint{3.956233in}{0.676793in}}%
\pgfpathlineto{\pgfqpoint{3.956789in}{0.609626in}}%
\pgfpathlineto{\pgfqpoint{3.957344in}{0.654382in}}%
\pgfpathlineto{\pgfqpoint{3.957900in}{0.619914in}}%
\pgfpathlineto{\pgfqpoint{3.958456in}{0.629056in}}%
\pgfpathlineto{\pgfqpoint{3.959012in}{0.628590in}}%
\pgfpathlineto{\pgfqpoint{3.959568in}{0.648772in}}%
\pgfpathlineto{\pgfqpoint{3.960680in}{0.607239in}}%
\pgfpathlineto{\pgfqpoint{3.961235in}{0.649141in}}%
\pgfpathlineto{\pgfqpoint{3.961791in}{0.639926in}}%
\pgfpathlineto{\pgfqpoint{3.962903in}{0.607921in}}%
\pgfpathlineto{\pgfqpoint{3.964571in}{0.650887in}}%
\pgfpathlineto{\pgfqpoint{3.965682in}{0.605040in}}%
\pgfpathlineto{\pgfqpoint{3.966238in}{0.618569in}}%
\pgfpathlineto{\pgfqpoint{3.967350in}{0.653562in}}%
\pgfpathlineto{\pgfqpoint{3.967906in}{0.647952in}}%
\pgfpathlineto{\pgfqpoint{3.969573in}{0.602216in}}%
\pgfpathlineto{\pgfqpoint{3.971797in}{0.645248in}}%
\pgfpathlineto{\pgfqpoint{3.972353in}{0.637035in}}%
\pgfpathlineto{\pgfqpoint{3.972909in}{0.638911in}}%
\pgfpathlineto{\pgfqpoint{3.973464in}{0.646305in}}%
\pgfpathlineto{\pgfqpoint{3.975688in}{0.625812in}}%
\pgfpathlineto{\pgfqpoint{3.977355in}{0.615526in}}%
\pgfpathlineto{\pgfqpoint{3.979023in}{0.607458in}}%
\pgfpathlineto{\pgfqpoint{3.979579in}{0.610538in}}%
\pgfpathlineto{\pgfqpoint{3.980135in}{0.601485in}}%
\pgfpathlineto{\pgfqpoint{3.980691in}{0.602498in}}%
\pgfpathlineto{\pgfqpoint{3.981246in}{0.601556in}}%
\pgfpathlineto{\pgfqpoint{3.981802in}{0.602226in}}%
\pgfpathlineto{\pgfqpoint{3.982914in}{0.609360in}}%
\pgfpathlineto{\pgfqpoint{3.983470in}{0.604008in}}%
\pgfpathlineto{\pgfqpoint{3.984582in}{0.614376in}}%
\pgfpathlineto{\pgfqpoint{3.985137in}{0.605215in}}%
\pgfpathlineto{\pgfqpoint{3.985693in}{0.608048in}}%
\pgfpathlineto{\pgfqpoint{3.986249in}{0.614809in}}%
\pgfpathlineto{\pgfqpoint{3.986805in}{0.608322in}}%
\pgfpathlineto{\pgfqpoint{3.987361in}{0.605135in}}%
\pgfpathlineto{\pgfqpoint{3.989028in}{0.614877in}}%
\pgfpathlineto{\pgfqpoint{3.989584in}{0.613415in}}%
\pgfpathlineto{\pgfqpoint{3.990140in}{0.614196in}}%
\pgfpathlineto{\pgfqpoint{3.991808in}{0.605738in}}%
\pgfpathlineto{\pgfqpoint{3.993475in}{0.628686in}}%
\pgfpathlineto{\pgfqpoint{3.995143in}{0.609387in}}%
\pgfpathlineto{\pgfqpoint{3.996255in}{0.635145in}}%
\pgfpathlineto{\pgfqpoint{3.996811in}{0.608468in}}%
\pgfpathlineto{\pgfqpoint{3.998478in}{0.646148in}}%
\pgfpathlineto{\pgfqpoint{3.999034in}{0.622525in}}%
\pgfpathlineto{\pgfqpoint{3.999590in}{0.659162in}}%
\pgfpathlineto{\pgfqpoint{4.000146in}{0.610675in}}%
\pgfpathlineto{\pgfqpoint{4.000702in}{0.649023in}}%
\pgfpathlineto{\pgfqpoint{4.001257in}{0.643633in}}%
\pgfpathlineto{\pgfqpoint{4.001813in}{0.696812in}}%
\pgfpathlineto{\pgfqpoint{4.002369in}{0.653513in}}%
\pgfpathlineto{\pgfqpoint{4.002925in}{0.651756in}}%
\pgfpathlineto{\pgfqpoint{4.003481in}{0.644895in}}%
\pgfpathlineto{\pgfqpoint{4.004593in}{0.673332in}}%
\pgfpathlineto{\pgfqpoint{4.005704in}{0.622739in}}%
\pgfpathlineto{\pgfqpoint{4.006260in}{0.653238in}}%
\pgfpathlineto{\pgfqpoint{4.006816in}{0.608096in}}%
\pgfpathlineto{\pgfqpoint{4.007928in}{0.677404in}}%
\pgfpathlineto{\pgfqpoint{4.008484in}{0.619243in}}%
\pgfpathlineto{\pgfqpoint{4.009039in}{0.626292in}}%
\pgfpathlineto{\pgfqpoint{4.009595in}{0.623024in}}%
\pgfpathlineto{\pgfqpoint{4.011263in}{0.672789in}}%
\pgfpathlineto{\pgfqpoint{4.012931in}{0.625071in}}%
\pgfpathlineto{\pgfqpoint{4.013486in}{0.675999in}}%
\pgfpathlineto{\pgfqpoint{4.014042in}{0.628687in}}%
\pgfpathlineto{\pgfqpoint{4.015710in}{0.640587in}}%
\pgfpathlineto{\pgfqpoint{4.016266in}{0.606996in}}%
\pgfpathlineto{\pgfqpoint{4.016822in}{0.652304in}}%
\pgfpathlineto{\pgfqpoint{4.017377in}{0.610610in}}%
\pgfpathlineto{\pgfqpoint{4.019045in}{0.630771in}}%
\pgfpathlineto{\pgfqpoint{4.019601in}{0.612712in}}%
\pgfpathlineto{\pgfqpoint{4.020157in}{0.616284in}}%
\pgfpathlineto{\pgfqpoint{4.021268in}{0.650159in}}%
\pgfpathlineto{\pgfqpoint{4.021824in}{0.632557in}}%
\pgfpathlineto{\pgfqpoint{4.022380in}{0.606560in}}%
\pgfpathlineto{\pgfqpoint{4.022936in}{0.625783in}}%
\pgfpathlineto{\pgfqpoint{4.024048in}{0.639710in}}%
\pgfpathlineto{\pgfqpoint{4.025715in}{0.607347in}}%
\pgfpathlineto{\pgfqpoint{4.027383in}{0.640540in}}%
\pgfpathlineto{\pgfqpoint{4.029051in}{0.604586in}}%
\pgfpathlineto{\pgfqpoint{4.029606in}{0.610020in}}%
\pgfpathlineto{\pgfqpoint{4.030162in}{0.616582in}}%
\pgfpathlineto{\pgfqpoint{4.030718in}{0.639038in}}%
\pgfpathlineto{\pgfqpoint{4.031274in}{0.633523in}}%
\pgfpathlineto{\pgfqpoint{4.031830in}{0.632294in}}%
\pgfpathlineto{\pgfqpoint{4.032942in}{0.649526in}}%
\pgfpathlineto{\pgfqpoint{4.034609in}{0.633163in}}%
\pgfpathlineto{\pgfqpoint{4.035165in}{0.635136in}}%
\pgfpathlineto{\pgfqpoint{4.036277in}{0.635024in}}%
\pgfpathlineto{\pgfqpoint{4.037388in}{0.612599in}}%
\pgfpathlineto{\pgfqpoint{4.037944in}{0.613565in}}%
\pgfpathlineto{\pgfqpoint{4.038500in}{0.607891in}}%
\pgfpathlineto{\pgfqpoint{4.040168in}{0.618623in}}%
\pgfpathlineto{\pgfqpoint{4.040724in}{0.614798in}}%
\pgfpathlineto{\pgfqpoint{4.041279in}{0.602289in}}%
\pgfpathlineto{\pgfqpoint{4.041835in}{0.606864in}}%
\pgfpathlineto{\pgfqpoint{4.042947in}{0.607742in}}%
\pgfpathlineto{\pgfqpoint{4.043503in}{0.619293in}}%
\pgfpathlineto{\pgfqpoint{4.044059in}{0.610115in}}%
\pgfpathlineto{\pgfqpoint{4.044615in}{0.617752in}}%
\pgfpathlineto{\pgfqpoint{4.045170in}{0.612832in}}%
\pgfpathlineto{\pgfqpoint{4.045726in}{0.611904in}}%
\pgfpathlineto{\pgfqpoint{4.046838in}{0.618497in}}%
\pgfpathlineto{\pgfqpoint{4.047950in}{0.609942in}}%
\pgfpathlineto{\pgfqpoint{4.049617in}{0.632159in}}%
\pgfpathlineto{\pgfqpoint{4.050173in}{0.606486in}}%
\pgfpathlineto{\pgfqpoint{4.050729in}{0.615415in}}%
\pgfpathlineto{\pgfqpoint{4.051841in}{0.617284in}}%
\pgfpathlineto{\pgfqpoint{4.052397in}{0.601350in}}%
\pgfpathlineto{\pgfqpoint{4.052953in}{0.613878in}}%
\pgfpathlineto{\pgfqpoint{4.053508in}{0.618607in}}%
\pgfpathlineto{\pgfqpoint{4.054064in}{0.635745in}}%
\pgfpathlineto{\pgfqpoint{4.054620in}{0.628595in}}%
\pgfpathlineto{\pgfqpoint{4.055176in}{0.628760in}}%
\pgfpathlineto{\pgfqpoint{4.055732in}{0.627016in}}%
\pgfpathlineto{\pgfqpoint{4.056844in}{0.655463in}}%
\pgfpathlineto{\pgfqpoint{4.057955in}{0.639581in}}%
\pgfpathlineto{\pgfqpoint{4.059623in}{0.670540in}}%
\pgfpathlineto{\pgfqpoint{4.060735in}{0.631432in}}%
\pgfpathlineto{\pgfqpoint{4.061846in}{0.670466in}}%
\pgfpathlineto{\pgfqpoint{4.063514in}{0.622552in}}%
\pgfpathlineto{\pgfqpoint{4.064070in}{0.624841in}}%
\pgfpathlineto{\pgfqpoint{4.065737in}{0.662217in}}%
\pgfpathlineto{\pgfqpoint{4.066293in}{0.635245in}}%
\pgfpathlineto{\pgfqpoint{4.066849in}{0.653497in}}%
\pgfpathlineto{\pgfqpoint{4.067405in}{0.668828in}}%
\pgfpathlineto{\pgfqpoint{4.067961in}{0.619683in}}%
\pgfpathlineto{\pgfqpoint{4.068517in}{0.657228in}}%
\pgfpathlineto{\pgfqpoint{4.069628in}{0.647321in}}%
\pgfpathlineto{\pgfqpoint{4.070184in}{0.611579in}}%
\pgfpathlineto{\pgfqpoint{4.071296in}{0.665303in}}%
\pgfpathlineto{\pgfqpoint{4.071852in}{0.611617in}}%
\pgfpathlineto{\pgfqpoint{4.072408in}{0.657708in}}%
\pgfpathlineto{\pgfqpoint{4.073519in}{0.607821in}}%
\pgfpathlineto{\pgfqpoint{4.074075in}{0.649254in}}%
\pgfpathlineto{\pgfqpoint{4.074631in}{0.626764in}}%
\pgfpathlineto{\pgfqpoint{4.075187in}{0.628094in}}%
\pgfpathlineto{\pgfqpoint{4.075743in}{0.605966in}}%
\pgfpathlineto{\pgfqpoint{4.076299in}{0.633190in}}%
\pgfpathlineto{\pgfqpoint{4.076855in}{0.623896in}}%
\pgfpathlineto{\pgfqpoint{4.077410in}{0.622899in}}%
\pgfpathlineto{\pgfqpoint{4.077966in}{0.627720in}}%
\pgfpathlineto{\pgfqpoint{4.078522in}{0.650796in}}%
\pgfpathlineto{\pgfqpoint{4.079634in}{0.607494in}}%
\pgfpathlineto{\pgfqpoint{4.080746in}{0.637737in}}%
\pgfpathlineto{\pgfqpoint{4.081301in}{0.627119in}}%
\pgfpathlineto{\pgfqpoint{4.081857in}{0.626491in}}%
\pgfpathlineto{\pgfqpoint{4.082969in}{0.611646in}}%
\pgfpathlineto{\pgfqpoint{4.083525in}{0.649717in}}%
\pgfpathlineto{\pgfqpoint{4.084081in}{0.632481in}}%
\pgfpathlineto{\pgfqpoint{4.085748in}{0.612670in}}%
\pgfpathlineto{\pgfqpoint{4.086304in}{0.618689in}}%
\pgfpathlineto{\pgfqpoint{4.087416in}{0.649311in}}%
\pgfpathlineto{\pgfqpoint{4.087972in}{0.646036in}}%
\pgfpathlineto{\pgfqpoint{4.089639in}{0.608982in}}%
\pgfpathlineto{\pgfqpoint{4.090195in}{0.614145in}}%
\pgfpathlineto{\pgfqpoint{4.091863in}{0.640370in}}%
\pgfpathlineto{\pgfqpoint{4.092419in}{0.641406in}}%
\pgfpathlineto{\pgfqpoint{4.092975in}{0.628140in}}%
\pgfpathlineto{\pgfqpoint{4.093530in}{0.639602in}}%
\pgfpathlineto{\pgfqpoint{4.094086in}{0.637606in}}%
\pgfpathlineto{\pgfqpoint{4.094642in}{0.647901in}}%
\pgfpathlineto{\pgfqpoint{4.095754in}{0.628650in}}%
\pgfpathlineto{\pgfqpoint{4.096310in}{0.630702in}}%
\pgfpathlineto{\pgfqpoint{4.098533in}{0.606958in}}%
\pgfpathlineto{\pgfqpoint{4.099089in}{0.607341in}}%
\pgfpathlineto{\pgfqpoint{4.099645in}{0.618014in}}%
\pgfpathlineto{\pgfqpoint{4.100201in}{0.601252in}}%
\pgfpathlineto{\pgfqpoint{4.100757in}{0.609039in}}%
\pgfpathlineto{\pgfqpoint{4.101312in}{0.604627in}}%
\pgfpathlineto{\pgfqpoint{4.101868in}{0.615567in}}%
\pgfpathlineto{\pgfqpoint{4.102424in}{0.602932in}}%
\pgfpathlineto{\pgfqpoint{4.102980in}{0.609583in}}%
\pgfpathlineto{\pgfqpoint{4.103536in}{0.605722in}}%
\pgfpathlineto{\pgfqpoint{4.104092in}{0.616789in}}%
\pgfpathlineto{\pgfqpoint{4.104648in}{0.607107in}}%
\pgfpathlineto{\pgfqpoint{4.105204in}{0.601741in}}%
\pgfpathlineto{\pgfqpoint{4.105759in}{0.618299in}}%
\pgfpathlineto{\pgfqpoint{4.106315in}{0.609658in}}%
\pgfpathlineto{\pgfqpoint{4.107427in}{0.617603in}}%
\pgfpathlineto{\pgfqpoint{4.108539in}{0.609289in}}%
\pgfpathlineto{\pgfqpoint{4.109095in}{0.643571in}}%
\pgfpathlineto{\pgfqpoint{4.109650in}{0.613966in}}%
\pgfpathlineto{\pgfqpoint{4.110206in}{0.624589in}}%
\pgfpathlineto{\pgfqpoint{4.110762in}{0.609088in}}%
\pgfpathlineto{\pgfqpoint{4.111318in}{0.641495in}}%
\pgfpathlineto{\pgfqpoint{4.111874in}{0.637197in}}%
\pgfpathlineto{\pgfqpoint{4.112430in}{0.633140in}}%
\pgfpathlineto{\pgfqpoint{4.112986in}{0.620741in}}%
\pgfpathlineto{\pgfqpoint{4.114653in}{0.666661in}}%
\pgfpathlineto{\pgfqpoint{4.115765in}{0.627742in}}%
\pgfpathlineto{\pgfqpoint{4.116877in}{0.689568in}}%
\pgfpathlineto{\pgfqpoint{4.117432in}{0.621110in}}%
\pgfpathlineto{\pgfqpoint{4.117988in}{0.634996in}}%
\pgfpathlineto{\pgfqpoint{4.118544in}{0.657640in}}%
\pgfpathlineto{\pgfqpoint{4.119100in}{0.646249in}}%
\pgfpathlineto{\pgfqpoint{4.119656in}{0.611893in}}%
\pgfpathlineto{\pgfqpoint{4.120212in}{0.649408in}}%
\pgfpathlineto{\pgfqpoint{4.120768in}{0.635936in}}%
\pgfpathlineto{\pgfqpoint{4.121323in}{0.645410in}}%
\pgfpathlineto{\pgfqpoint{4.121879in}{0.608778in}}%
\pgfpathlineto{\pgfqpoint{4.122435in}{0.643422in}}%
\pgfpathlineto{\pgfqpoint{4.122991in}{0.668849in}}%
\pgfpathlineto{\pgfqpoint{4.123547in}{0.618047in}}%
\pgfpathlineto{\pgfqpoint{4.124103in}{0.640495in}}%
\pgfpathlineto{\pgfqpoint{4.124659in}{0.641714in}}%
\pgfpathlineto{\pgfqpoint{4.126326in}{0.653898in}}%
\pgfpathlineto{\pgfqpoint{4.127994in}{0.617549in}}%
\pgfpathlineto{\pgfqpoint{4.128550in}{0.688442in}}%
\pgfpathlineto{\pgfqpoint{4.129106in}{0.610717in}}%
\pgfpathlineto{\pgfqpoint{4.129661in}{0.645844in}}%
\pgfpathlineto{\pgfqpoint{4.130217in}{0.619369in}}%
\pgfpathlineto{\pgfqpoint{4.130773in}{0.626554in}}%
\pgfpathlineto{\pgfqpoint{4.131329in}{0.619788in}}%
\pgfpathlineto{\pgfqpoint{4.131885in}{0.647537in}}%
\pgfpathlineto{\pgfqpoint{4.132997in}{0.607045in}}%
\pgfpathlineto{\pgfqpoint{4.133552in}{0.630970in}}%
\pgfpathlineto{\pgfqpoint{4.134108in}{0.627078in}}%
\pgfpathlineto{\pgfqpoint{4.134664in}{0.626288in}}%
\pgfpathlineto{\pgfqpoint{4.135220in}{0.605144in}}%
\pgfpathlineto{\pgfqpoint{4.135776in}{0.629530in}}%
\pgfpathlineto{\pgfqpoint{4.136332in}{0.615480in}}%
\pgfpathlineto{\pgfqpoint{4.136888in}{0.608357in}}%
\pgfpathlineto{\pgfqpoint{4.137443in}{0.612642in}}%
\pgfpathlineto{\pgfqpoint{4.137999in}{0.634081in}}%
\pgfpathlineto{\pgfqpoint{4.138555in}{0.619710in}}%
\pgfpathlineto{\pgfqpoint{4.139111in}{0.609662in}}%
\pgfpathlineto{\pgfqpoint{4.140223in}{0.629412in}}%
\pgfpathlineto{\pgfqpoint{4.140779in}{0.623834in}}%
\pgfpathlineto{\pgfqpoint{4.141334in}{0.633273in}}%
\pgfpathlineto{\pgfqpoint{4.142446in}{0.603374in}}%
\pgfpathlineto{\pgfqpoint{4.144114in}{0.635035in}}%
\pgfpathlineto{\pgfqpoint{4.145781in}{0.612586in}}%
\pgfpathlineto{\pgfqpoint{4.146337in}{0.617031in}}%
\pgfpathlineto{\pgfqpoint{4.146893in}{0.626861in}}%
\pgfpathlineto{\pgfqpoint{4.147449in}{0.626557in}}%
\pgfpathlineto{\pgfqpoint{4.149117in}{0.616709in}}%
\pgfpathlineto{\pgfqpoint{4.149672in}{0.609220in}}%
\pgfpathlineto{\pgfqpoint{4.150228in}{0.614120in}}%
\pgfpathlineto{\pgfqpoint{4.151340in}{0.615562in}}%
\pgfpathlineto{\pgfqpoint{4.153008in}{0.635831in}}%
\pgfpathlineto{\pgfqpoint{4.153563in}{0.635810in}}%
\pgfpathlineto{\pgfqpoint{4.154675in}{0.624525in}}%
\pgfpathlineto{\pgfqpoint{4.155231in}{0.634888in}}%
\pgfpathlineto{\pgfqpoint{4.156899in}{0.611838in}}%
\pgfpathlineto{\pgfqpoint{4.157454in}{0.613610in}}%
\pgfpathlineto{\pgfqpoint{4.158010in}{0.623124in}}%
\pgfpathlineto{\pgfqpoint{4.158566in}{0.621279in}}%
\pgfpathlineto{\pgfqpoint{4.159678in}{0.604172in}}%
\pgfpathlineto{\pgfqpoint{4.160234in}{0.618702in}}%
\pgfpathlineto{\pgfqpoint{4.160790in}{0.609745in}}%
\pgfpathlineto{\pgfqpoint{4.161346in}{0.616698in}}%
\pgfpathlineto{\pgfqpoint{4.161901in}{0.606824in}}%
\pgfpathlineto{\pgfqpoint{4.162457in}{0.621786in}}%
\pgfpathlineto{\pgfqpoint{4.164125in}{0.606024in}}%
\pgfpathlineto{\pgfqpoint{4.166348in}{0.632803in}}%
\pgfpathlineto{\pgfqpoint{4.166904in}{0.603544in}}%
\pgfpathlineto{\pgfqpoint{4.167460in}{0.621872in}}%
\pgfpathlineto{\pgfqpoint{4.168016in}{0.605694in}}%
\pgfpathlineto{\pgfqpoint{4.169683in}{0.636481in}}%
\pgfpathlineto{\pgfqpoint{4.170239in}{0.614961in}}%
\pgfpathlineto{\pgfqpoint{4.170795in}{0.628891in}}%
\pgfpathlineto{\pgfqpoint{4.171351in}{0.625134in}}%
\pgfpathlineto{\pgfqpoint{4.171907in}{0.643124in}}%
\pgfpathlineto{\pgfqpoint{4.172463in}{0.623990in}}%
\pgfpathlineto{\pgfqpoint{4.173019in}{0.624938in}}%
\pgfpathlineto{\pgfqpoint{4.173574in}{0.631603in}}%
\pgfpathlineto{\pgfqpoint{4.174130in}{0.654310in}}%
\pgfpathlineto{\pgfqpoint{4.175798in}{0.625605in}}%
\pgfpathlineto{\pgfqpoint{4.176354in}{0.630841in}}%
\pgfpathlineto{\pgfqpoint{4.176910in}{0.625359in}}%
\pgfpathlineto{\pgfqpoint{4.178021in}{0.647062in}}%
\pgfpathlineto{\pgfqpoint{4.179133in}{0.608419in}}%
\pgfpathlineto{\pgfqpoint{4.179689in}{0.620313in}}%
\pgfpathlineto{\pgfqpoint{4.180245in}{0.616202in}}%
\pgfpathlineto{\pgfqpoint{4.180801in}{0.624415in}}%
\pgfpathlineto{\pgfqpoint{4.181357in}{0.607209in}}%
\pgfpathlineto{\pgfqpoint{4.182468in}{0.639192in}}%
\pgfpathlineto{\pgfqpoint{4.184136in}{0.611945in}}%
\pgfpathlineto{\pgfqpoint{4.184692in}{0.639057in}}%
\pgfpathlineto{\pgfqpoint{4.185248in}{0.604308in}}%
\pgfpathlineto{\pgfqpoint{4.185803in}{0.651549in}}%
\pgfpathlineto{\pgfqpoint{4.186359in}{0.615085in}}%
\pgfpathlineto{\pgfqpoint{4.186915in}{0.614563in}}%
\pgfpathlineto{\pgfqpoint{4.187471in}{0.616147in}}%
\pgfpathlineto{\pgfqpoint{4.188027in}{0.632031in}}%
\pgfpathlineto{\pgfqpoint{4.188583in}{0.601970in}}%
\pgfpathlineto{\pgfqpoint{4.189139in}{0.631067in}}%
\pgfpathlineto{\pgfqpoint{4.189694in}{0.608344in}}%
\pgfpathlineto{\pgfqpoint{4.190250in}{0.609275in}}%
\pgfpathlineto{\pgfqpoint{4.190806in}{0.610101in}}%
\pgfpathlineto{\pgfqpoint{4.191918in}{0.619426in}}%
\pgfpathlineto{\pgfqpoint{4.193030in}{0.616002in}}%
\pgfpathlineto{\pgfqpoint{4.193585in}{0.620727in}}%
\pgfpathlineto{\pgfqpoint{4.194697in}{0.603554in}}%
\pgfpathlineto{\pgfqpoint{4.195253in}{0.617396in}}%
\pgfpathlineto{\pgfqpoint{4.195809in}{0.612629in}}%
\pgfpathlineto{\pgfqpoint{4.196365in}{0.602219in}}%
\pgfpathlineto{\pgfqpoint{4.196921in}{0.608928in}}%
\pgfpathlineto{\pgfqpoint{4.197476in}{0.618171in}}%
\pgfpathlineto{\pgfqpoint{4.199144in}{0.604152in}}%
\pgfpathlineto{\pgfqpoint{4.201368in}{0.615290in}}%
\pgfpathlineto{\pgfqpoint{4.201923in}{0.607556in}}%
\pgfpathlineto{\pgfqpoint{4.202479in}{0.608330in}}%
\pgfpathlineto{\pgfqpoint{4.203035in}{0.609038in}}%
\pgfpathlineto{\pgfqpoint{4.203591in}{0.623399in}}%
\pgfpathlineto{\pgfqpoint{4.204147in}{0.615358in}}%
\pgfpathlineto{\pgfqpoint{4.205814in}{0.600744in}}%
\pgfpathlineto{\pgfqpoint{4.206370in}{0.604552in}}%
\pgfpathlineto{\pgfqpoint{4.206926in}{0.620390in}}%
\pgfpathlineto{\pgfqpoint{4.207482in}{0.616789in}}%
\pgfpathlineto{\pgfqpoint{4.208038in}{0.618619in}}%
\pgfpathlineto{\pgfqpoint{4.209150in}{0.604523in}}%
\pgfpathlineto{\pgfqpoint{4.209705in}{0.604797in}}%
\pgfpathlineto{\pgfqpoint{4.210817in}{0.607043in}}%
\pgfpathlineto{\pgfqpoint{4.211929in}{0.613934in}}%
\pgfpathlineto{\pgfqpoint{4.212485in}{0.611530in}}%
\pgfpathlineto{\pgfqpoint{4.213041in}{0.612152in}}%
\pgfpathlineto{\pgfqpoint{4.213596in}{0.619350in}}%
\pgfpathlineto{\pgfqpoint{4.214152in}{0.615883in}}%
\pgfpathlineto{\pgfqpoint{4.214708in}{0.617921in}}%
\pgfpathlineto{\pgfqpoint{4.215264in}{0.607423in}}%
\pgfpathlineto{\pgfqpoint{4.215820in}{0.608374in}}%
\pgfpathlineto{\pgfqpoint{4.216932in}{0.615223in}}%
\pgfpathlineto{\pgfqpoint{4.218043in}{0.606857in}}%
\pgfpathlineto{\pgfqpoint{4.218599in}{0.608252in}}%
\pgfpathlineto{\pgfqpoint{4.219155in}{0.605058in}}%
\pgfpathlineto{\pgfqpoint{4.219711in}{0.609683in}}%
\pgfpathlineto{\pgfqpoint{4.220267in}{0.603998in}}%
\pgfpathlineto{\pgfqpoint{4.220823in}{0.607166in}}%
\pgfpathlineto{\pgfqpoint{4.221379in}{0.606834in}}%
\pgfpathlineto{\pgfqpoint{4.221934in}{0.607918in}}%
\pgfpathlineto{\pgfqpoint{4.222490in}{0.603678in}}%
\pgfpathlineto{\pgfqpoint{4.223046in}{0.604482in}}%
\pgfpathlineto{\pgfqpoint{4.224158in}{0.610816in}}%
\pgfpathlineto{\pgfqpoint{4.224714in}{0.605990in}}%
\pgfpathlineto{\pgfqpoint{4.225270in}{0.607141in}}%
\pgfpathlineto{\pgfqpoint{4.225825in}{0.606728in}}%
\pgfpathlineto{\pgfqpoint{4.226381in}{0.616217in}}%
\pgfpathlineto{\pgfqpoint{4.226937in}{0.615415in}}%
\pgfpathlineto{\pgfqpoint{4.228049in}{0.608551in}}%
\pgfpathlineto{\pgfqpoint{4.229716in}{0.615075in}}%
\pgfpathlineto{\pgfqpoint{4.230828in}{0.604816in}}%
\pgfpathlineto{\pgfqpoint{4.231940in}{0.620225in}}%
\pgfpathlineto{\pgfqpoint{4.233052in}{0.606412in}}%
\pgfpathlineto{\pgfqpoint{4.233607in}{0.612127in}}%
\pgfpathlineto{\pgfqpoint{4.234163in}{0.607325in}}%
\pgfpathlineto{\pgfqpoint{4.235275in}{0.614983in}}%
\pgfpathlineto{\pgfqpoint{4.236943in}{0.602282in}}%
\pgfpathlineto{\pgfqpoint{4.238054in}{0.612621in}}%
\pgfpathlineto{\pgfqpoint{4.238610in}{0.602742in}}%
\pgfpathlineto{\pgfqpoint{4.239166in}{0.613251in}}%
\pgfpathlineto{\pgfqpoint{4.239722in}{0.610683in}}%
\pgfpathlineto{\pgfqpoint{4.240278in}{0.604931in}}%
\pgfpathlineto{\pgfqpoint{4.240834in}{0.607648in}}%
\pgfpathlineto{\pgfqpoint{4.241390in}{0.608345in}}%
\pgfpathlineto{\pgfqpoint{4.241945in}{0.613962in}}%
\pgfpathlineto{\pgfqpoint{4.242501in}{0.606329in}}%
\pgfpathlineto{\pgfqpoint{4.243057in}{0.609418in}}%
\pgfpathlineto{\pgfqpoint{4.243613in}{0.615039in}}%
\pgfpathlineto{\pgfqpoint{4.244169in}{0.606424in}}%
\pgfpathlineto{\pgfqpoint{4.244725in}{0.610098in}}%
\pgfpathlineto{\pgfqpoint{4.245281in}{0.609586in}}%
\pgfpathlineto{\pgfqpoint{4.245836in}{0.603774in}}%
\pgfpathlineto{\pgfqpoint{4.246392in}{0.606040in}}%
\pgfpathlineto{\pgfqpoint{4.246948in}{0.607327in}}%
\pgfpathlineto{\pgfqpoint{4.248060in}{0.600755in}}%
\pgfpathlineto{\pgfqpoint{4.249727in}{0.607955in}}%
\pgfpathlineto{\pgfqpoint{4.250283in}{0.601904in}}%
\pgfpathlineto{\pgfqpoint{4.250839in}{0.606036in}}%
\pgfpathlineto{\pgfqpoint{4.252507in}{0.602781in}}%
\pgfpathlineto{\pgfqpoint{4.253063in}{0.604789in}}%
\pgfpathlineto{\pgfqpoint{4.254174in}{0.600767in}}%
\pgfpathlineto{\pgfqpoint{4.254730in}{0.606478in}}%
\pgfpathlineto{\pgfqpoint{4.255286in}{0.603427in}}%
\pgfpathlineto{\pgfqpoint{4.255842in}{0.603174in}}%
\pgfpathlineto{\pgfqpoint{4.257510in}{0.600804in}}%
\pgfpathlineto{\pgfqpoint{4.258065in}{0.603178in}}%
\pgfpathlineto{\pgfqpoint{4.258621in}{0.601513in}}%
\pgfpathlineto{\pgfqpoint{4.259177in}{0.601014in}}%
\pgfpathlineto{\pgfqpoint{4.260289in}{0.605248in}}%
\pgfpathlineto{\pgfqpoint{4.260845in}{0.604695in}}%
\pgfpathlineto{\pgfqpoint{4.262512in}{0.601158in}}%
\pgfpathlineto{\pgfqpoint{4.264180in}{0.604246in}}%
\pgfpathlineto{\pgfqpoint{4.265847in}{0.601124in}}%
\pgfpathlineto{\pgfqpoint{4.267515in}{0.603909in}}%
\pgfpathlineto{\pgfqpoint{4.270850in}{0.600425in}}%
\pgfpathlineto{\pgfqpoint{4.273074in}{0.603444in}}%
\pgfpathlineto{\pgfqpoint{4.274185in}{0.601329in}}%
\pgfpathlineto{\pgfqpoint{4.274741in}{0.602894in}}%
\pgfpathlineto{\pgfqpoint{4.275297in}{0.602560in}}%
\pgfpathlineto{\pgfqpoint{4.277521in}{0.600838in}}%
\pgfpathlineto{\pgfqpoint{4.278076in}{0.602228in}}%
\pgfpathlineto{\pgfqpoint{4.279744in}{0.600049in}}%
\pgfpathlineto{\pgfqpoint{4.280856in}{0.600137in}}%
\pgfpathlineto{\pgfqpoint{4.281412in}{0.603194in}}%
\pgfpathlineto{\pgfqpoint{4.281967in}{0.600697in}}%
\pgfpathlineto{\pgfqpoint{4.282523in}{0.601712in}}%
\pgfpathlineto{\pgfqpoint{4.283079in}{0.600322in}}%
\pgfpathlineto{\pgfqpoint{4.284747in}{0.601948in}}%
\pgfpathlineto{\pgfqpoint{4.285303in}{0.600577in}}%
\pgfpathlineto{\pgfqpoint{4.285858in}{0.601433in}}%
\pgfpathlineto{\pgfqpoint{4.287526in}{0.601250in}}%
\pgfpathlineto{\pgfqpoint{4.288082in}{0.600954in}}%
\pgfpathlineto{\pgfqpoint{4.288638in}{0.602499in}}%
\pgfpathlineto{\pgfqpoint{4.289194in}{0.602136in}}%
\pgfpathlineto{\pgfqpoint{4.289749in}{0.600150in}}%
\pgfpathlineto{\pgfqpoint{4.290305in}{0.601328in}}%
\pgfpathlineto{\pgfqpoint{4.291973in}{0.600291in}}%
\pgfpathlineto{\pgfqpoint{4.293085in}{0.602102in}}%
\pgfpathlineto{\pgfqpoint{4.294196in}{0.600325in}}%
\pgfpathlineto{\pgfqpoint{4.294752in}{0.600825in}}%
\pgfpathlineto{\pgfqpoint{4.296976in}{0.600665in}}%
\pgfpathlineto{\pgfqpoint{4.299755in}{0.600875in}}%
\pgfpathlineto{\pgfqpoint{4.300311in}{0.600054in}}%
\pgfpathlineto{\pgfqpoint{4.300867in}{0.601512in}}%
\pgfpathlineto{\pgfqpoint{4.301423in}{0.600238in}}%
\pgfpathlineto{\pgfqpoint{4.303090in}{0.600688in}}%
\pgfpathlineto{\pgfqpoint{4.304758in}{0.600019in}}%
\pgfpathlineto{\pgfqpoint{4.315319in}{0.599974in}}%
\pgfpathlineto{\pgfqpoint{4.317543in}{0.605534in}}%
\pgfpathlineto{\pgfqpoint{4.318098in}{0.600581in}}%
\pgfpathlineto{\pgfqpoint{4.318654in}{0.602594in}}%
\pgfpathlineto{\pgfqpoint{4.320322in}{0.600700in}}%
\pgfpathlineto{\pgfqpoint{4.324213in}{0.600378in}}%
\pgfpathlineto{\pgfqpoint{4.333107in}{0.600353in}}%
\pgfpathlineto{\pgfqpoint{4.334774in}{0.600879in}}%
\pgfpathlineto{\pgfqpoint{4.335886in}{0.600231in}}%
\pgfpathlineto{\pgfqpoint{4.337554in}{0.600758in}}%
\pgfpathlineto{\pgfqpoint{4.339221in}{0.600583in}}%
\pgfpathlineto{\pgfqpoint{4.340889in}{0.600600in}}%
\pgfpathlineto{\pgfqpoint{4.342556in}{0.600474in}}%
\pgfpathlineto{\pgfqpoint{4.343112in}{0.600199in}}%
\pgfpathlineto{\pgfqpoint{4.344224in}{0.601317in}}%
\pgfpathlineto{\pgfqpoint{4.345891in}{0.601395in}}%
\pgfpathlineto{\pgfqpoint{4.347559in}{0.601602in}}%
\pgfpathlineto{\pgfqpoint{4.349227in}{0.600861in}}%
\pgfpathlineto{\pgfqpoint{4.350338in}{0.602828in}}%
\pgfpathlineto{\pgfqpoint{4.350894in}{0.601318in}}%
\pgfpathlineto{\pgfqpoint{4.351450in}{0.600281in}}%
\pgfpathlineto{\pgfqpoint{4.352006in}{0.601420in}}%
\pgfpathlineto{\pgfqpoint{4.353674in}{0.600052in}}%
\pgfpathlineto{\pgfqpoint{4.354229in}{0.602170in}}%
\pgfpathlineto{\pgfqpoint{4.354785in}{0.601806in}}%
\pgfpathlineto{\pgfqpoint{4.355341in}{0.600304in}}%
\pgfpathlineto{\pgfqpoint{4.355897in}{0.600847in}}%
\pgfpathlineto{\pgfqpoint{4.357009in}{0.603013in}}%
\pgfpathlineto{\pgfqpoint{4.357565in}{0.600655in}}%
\pgfpathlineto{\pgfqpoint{4.358120in}{0.602902in}}%
\pgfpathlineto{\pgfqpoint{4.359788in}{0.601748in}}%
\pgfpathlineto{\pgfqpoint{4.360344in}{0.603987in}}%
\pgfpathlineto{\pgfqpoint{4.360900in}{0.600534in}}%
\pgfpathlineto{\pgfqpoint{4.361456in}{0.601727in}}%
\pgfpathlineto{\pgfqpoint{4.362567in}{0.600698in}}%
\pgfpathlineto{\pgfqpoint{4.364791in}{0.602411in}}%
\pgfpathlineto{\pgfqpoint{4.367014in}{0.601554in}}%
\pgfpathlineto{\pgfqpoint{4.369238in}{0.600248in}}%
\pgfpathlineto{\pgfqpoint{4.369794in}{0.603436in}}%
\pgfpathlineto{\pgfqpoint{4.370349in}{0.601122in}}%
\pgfpathlineto{\pgfqpoint{4.372017in}{0.602433in}}%
\pgfpathlineto{\pgfqpoint{4.373685in}{0.600912in}}%
\pgfpathlineto{\pgfqpoint{4.374796in}{0.602309in}}%
\pgfpathlineto{\pgfqpoint{4.375352in}{0.601104in}}%
\pgfpathlineto{\pgfqpoint{4.375908in}{0.601834in}}%
\pgfpathlineto{\pgfqpoint{4.377576in}{0.600516in}}%
\pgfpathlineto{\pgfqpoint{4.379243in}{0.602243in}}%
\pgfpathlineto{\pgfqpoint{4.380355in}{0.603894in}}%
\pgfpathlineto{\pgfqpoint{4.380911in}{0.602654in}}%
\pgfpathlineto{\pgfqpoint{4.382578in}{0.600805in}}%
\pgfpathlineto{\pgfqpoint{4.384246in}{0.605274in}}%
\pgfpathlineto{\pgfqpoint{4.384802in}{0.604086in}}%
\pgfpathlineto{\pgfqpoint{4.385358in}{0.600266in}}%
\pgfpathlineto{\pgfqpoint{4.385913in}{0.602645in}}%
\pgfpathlineto{\pgfqpoint{4.386469in}{0.601684in}}%
\pgfpathlineto{\pgfqpoint{4.387025in}{0.602560in}}%
\pgfpathlineto{\pgfqpoint{4.388693in}{0.605536in}}%
\pgfpathlineto{\pgfqpoint{4.390360in}{0.600410in}}%
\pgfpathlineto{\pgfqpoint{4.391472in}{0.601097in}}%
\pgfpathlineto{\pgfqpoint{4.393140in}{0.608983in}}%
\pgfpathlineto{\pgfqpoint{4.393696in}{0.601937in}}%
\pgfpathlineto{\pgfqpoint{4.394251in}{0.605437in}}%
\pgfpathlineto{\pgfqpoint{4.395919in}{0.606432in}}%
\pgfpathlineto{\pgfqpoint{4.396475in}{0.611093in}}%
\pgfpathlineto{\pgfqpoint{4.397031in}{0.601588in}}%
\pgfpathlineto{\pgfqpoint{4.397587in}{0.604522in}}%
\pgfpathlineto{\pgfqpoint{4.398142in}{0.604559in}}%
\pgfpathlineto{\pgfqpoint{4.399254in}{0.609207in}}%
\pgfpathlineto{\pgfqpoint{4.399810in}{0.608480in}}%
\pgfpathlineto{\pgfqpoint{4.400366in}{0.600524in}}%
\pgfpathlineto{\pgfqpoint{4.400922in}{0.606342in}}%
\pgfpathlineto{\pgfqpoint{4.402033in}{0.609072in}}%
\pgfpathlineto{\pgfqpoint{4.402589in}{0.606118in}}%
\pgfpathlineto{\pgfqpoint{4.403145in}{0.606297in}}%
\pgfpathlineto{\pgfqpoint{4.403701in}{0.613200in}}%
\pgfpathlineto{\pgfqpoint{4.404257in}{0.608856in}}%
\pgfpathlineto{\pgfqpoint{4.405925in}{0.600165in}}%
\pgfpathlineto{\pgfqpoint{4.406480in}{0.604547in}}%
\pgfpathlineto{\pgfqpoint{4.407036in}{0.603979in}}%
\pgfpathlineto{\pgfqpoint{4.408148in}{0.613313in}}%
\pgfpathlineto{\pgfqpoint{4.408704in}{0.605359in}}%
\pgfpathlineto{\pgfqpoint{4.409260in}{0.606996in}}%
\pgfpathlineto{\pgfqpoint{4.409816in}{0.609805in}}%
\pgfpathlineto{\pgfqpoint{4.410927in}{0.601151in}}%
\pgfpathlineto{\pgfqpoint{4.412595in}{0.608463in}}%
\pgfpathlineto{\pgfqpoint{4.413151in}{0.602703in}}%
\pgfpathlineto{\pgfqpoint{4.413707in}{0.611533in}}%
\pgfpathlineto{\pgfqpoint{4.414262in}{0.610536in}}%
\pgfpathlineto{\pgfqpoint{4.415374in}{0.603810in}}%
\pgfpathlineto{\pgfqpoint{4.415930in}{0.613687in}}%
\pgfpathlineto{\pgfqpoint{4.416486in}{0.607649in}}%
\pgfpathlineto{\pgfqpoint{4.417042in}{0.611839in}}%
\pgfpathlineto{\pgfqpoint{4.417598in}{0.610163in}}%
\pgfpathlineto{\pgfqpoint{4.418153in}{0.611090in}}%
\pgfpathlineto{\pgfqpoint{4.418709in}{0.602484in}}%
\pgfpathlineto{\pgfqpoint{4.419265in}{0.609642in}}%
\pgfpathlineto{\pgfqpoint{4.420377in}{0.604176in}}%
\pgfpathlineto{\pgfqpoint{4.422044in}{0.610367in}}%
\pgfpathlineto{\pgfqpoint{4.422600in}{0.602397in}}%
\pgfpathlineto{\pgfqpoint{4.423156in}{0.607249in}}%
\pgfpathlineto{\pgfqpoint{4.423712in}{0.607458in}}%
\pgfpathlineto{\pgfqpoint{4.424268in}{0.609250in}}%
\pgfpathlineto{\pgfqpoint{4.425936in}{0.601374in}}%
\pgfpathlineto{\pgfqpoint{4.427047in}{0.610640in}}%
\pgfpathlineto{\pgfqpoint{4.428715in}{0.600132in}}%
\pgfpathlineto{\pgfqpoint{4.429827in}{0.606290in}}%
\pgfpathlineto{\pgfqpoint{4.430938in}{0.601669in}}%
\pgfpathlineto{\pgfqpoint{4.432606in}{0.607234in}}%
\pgfpathlineto{\pgfqpoint{4.433162in}{0.607442in}}%
\pgfpathlineto{\pgfqpoint{4.435941in}{0.600362in}}%
\pgfpathlineto{\pgfqpoint{4.436497in}{0.610806in}}%
\pgfpathlineto{\pgfqpoint{4.437053in}{0.604550in}}%
\pgfpathlineto{\pgfqpoint{4.437609in}{0.607731in}}%
\pgfpathlineto{\pgfqpoint{4.438164in}{0.602442in}}%
\pgfpathlineto{\pgfqpoint{4.438720in}{0.604396in}}%
\pgfpathlineto{\pgfqpoint{4.440944in}{0.611159in}}%
\pgfpathlineto{\pgfqpoint{4.441500in}{0.601092in}}%
\pgfpathlineto{\pgfqpoint{4.442055in}{0.603529in}}%
\pgfpathlineto{\pgfqpoint{4.443167in}{0.614535in}}%
\pgfpathlineto{\pgfqpoint{4.443723in}{0.609351in}}%
\pgfpathlineto{\pgfqpoint{4.444279in}{0.604041in}}%
\pgfpathlineto{\pgfqpoint{4.444835in}{0.609893in}}%
\pgfpathlineto{\pgfqpoint{4.445391in}{0.604383in}}%
\pgfpathlineto{\pgfqpoint{4.446502in}{0.607558in}}%
\pgfpathlineto{\pgfqpoint{4.447058in}{0.612266in}}%
\pgfpathlineto{\pgfqpoint{4.447614in}{0.611283in}}%
\pgfpathlineto{\pgfqpoint{4.449282in}{0.605068in}}%
\pgfpathlineto{\pgfqpoint{4.450393in}{0.608258in}}%
\pgfpathlineto{\pgfqpoint{4.450949in}{0.605669in}}%
\pgfpathlineto{\pgfqpoint{4.451505in}{0.616061in}}%
\pgfpathlineto{\pgfqpoint{4.452061in}{0.604959in}}%
\pgfpathlineto{\pgfqpoint{4.452617in}{0.614924in}}%
\pgfpathlineto{\pgfqpoint{4.453173in}{0.609661in}}%
\pgfpathlineto{\pgfqpoint{4.453729in}{0.612729in}}%
\pgfpathlineto{\pgfqpoint{4.454840in}{0.623731in}}%
\pgfpathlineto{\pgfqpoint{4.455952in}{0.609485in}}%
\pgfpathlineto{\pgfqpoint{4.456508in}{0.622676in}}%
\pgfpathlineto{\pgfqpoint{4.458175in}{0.605761in}}%
\pgfpathlineto{\pgfqpoint{4.458731in}{0.631881in}}%
\pgfpathlineto{\pgfqpoint{4.459287in}{0.611343in}}%
\pgfpathlineto{\pgfqpoint{4.459843in}{0.609631in}}%
\pgfpathlineto{\pgfqpoint{4.460399in}{0.611128in}}%
\pgfpathlineto{\pgfqpoint{4.460955in}{0.623361in}}%
\pgfpathlineto{\pgfqpoint{4.461511in}{0.614343in}}%
\pgfpathlineto{\pgfqpoint{4.462066in}{0.602041in}}%
\pgfpathlineto{\pgfqpoint{4.462622in}{0.616335in}}%
\pgfpathlineto{\pgfqpoint{4.463178in}{0.609663in}}%
\pgfpathlineto{\pgfqpoint{4.464846in}{0.613665in}}%
\pgfpathlineto{\pgfqpoint{4.465402in}{0.626373in}}%
\pgfpathlineto{\pgfqpoint{4.466513in}{0.605508in}}%
\pgfpathlineto{\pgfqpoint{4.467069in}{0.616841in}}%
\pgfpathlineto{\pgfqpoint{4.467625in}{0.613760in}}%
\pgfpathlineto{\pgfqpoint{4.468737in}{0.606073in}}%
\pgfpathlineto{\pgfqpoint{4.469849in}{0.613348in}}%
\pgfpathlineto{\pgfqpoint{4.470404in}{0.607772in}}%
\pgfpathlineto{\pgfqpoint{4.470960in}{0.616158in}}%
\pgfpathlineto{\pgfqpoint{4.472628in}{0.602193in}}%
\pgfpathlineto{\pgfqpoint{4.473184in}{0.618325in}}%
\pgfpathlineto{\pgfqpoint{4.473740in}{0.609279in}}%
\pgfpathlineto{\pgfqpoint{4.474295in}{0.618112in}}%
\pgfpathlineto{\pgfqpoint{4.474851in}{0.604300in}}%
\pgfpathlineto{\pgfqpoint{4.475407in}{0.624462in}}%
\pgfpathlineto{\pgfqpoint{4.475963in}{0.604947in}}%
\pgfpathlineto{\pgfqpoint{4.476519in}{0.615930in}}%
\pgfpathlineto{\pgfqpoint{4.477075in}{0.604243in}}%
\pgfpathlineto{\pgfqpoint{4.477631in}{0.605791in}}%
\pgfpathlineto{\pgfqpoint{4.479298in}{0.612040in}}%
\pgfpathlineto{\pgfqpoint{4.479854in}{0.604858in}}%
\pgfpathlineto{\pgfqpoint{4.480410in}{0.610390in}}%
\pgfpathlineto{\pgfqpoint{4.480966in}{0.611373in}}%
\pgfpathlineto{\pgfqpoint{4.481522in}{0.616049in}}%
\pgfpathlineto{\pgfqpoint{4.483189in}{0.600347in}}%
\pgfpathlineto{\pgfqpoint{4.483745in}{0.613983in}}%
\pgfpathlineto{\pgfqpoint{4.484301in}{0.608385in}}%
\pgfpathlineto{\pgfqpoint{4.485413in}{0.605735in}}%
\pgfpathlineto{\pgfqpoint{4.486524in}{0.610784in}}%
\pgfpathlineto{\pgfqpoint{4.487080in}{0.608752in}}%
\pgfpathlineto{\pgfqpoint{4.488192in}{0.605474in}}%
\pgfpathlineto{\pgfqpoint{4.488748in}{0.607560in}}%
\pgfpathlineto{\pgfqpoint{4.489304in}{0.603328in}}%
\pgfpathlineto{\pgfqpoint{4.489860in}{0.608282in}}%
\pgfpathlineto{\pgfqpoint{4.490415in}{0.604683in}}%
\pgfpathlineto{\pgfqpoint{4.490971in}{0.607319in}}%
\pgfpathlineto{\pgfqpoint{4.491527in}{0.604803in}}%
\pgfpathlineto{\pgfqpoint{4.492083in}{0.605880in}}%
\pgfpathlineto{\pgfqpoint{4.493195in}{0.600790in}}%
\pgfpathlineto{\pgfqpoint{4.494306in}{0.614428in}}%
\pgfpathlineto{\pgfqpoint{4.495418in}{0.602184in}}%
\pgfpathlineto{\pgfqpoint{4.497086in}{0.609051in}}%
\pgfpathlineto{\pgfqpoint{4.497642in}{0.602099in}}%
\pgfpathlineto{\pgfqpoint{4.498197in}{0.604403in}}%
\pgfpathlineto{\pgfqpoint{4.498753in}{0.605273in}}%
\pgfpathlineto{\pgfqpoint{4.500421in}{0.618737in}}%
\pgfpathlineto{\pgfqpoint{4.502089in}{0.601425in}}%
\pgfpathlineto{\pgfqpoint{4.503756in}{0.612176in}}%
\pgfpathlineto{\pgfqpoint{4.504312in}{0.609999in}}%
\pgfpathlineto{\pgfqpoint{4.504868in}{0.612644in}}%
\pgfpathlineto{\pgfqpoint{4.506535in}{0.606448in}}%
\pgfpathlineto{\pgfqpoint{4.508203in}{0.602648in}}%
\pgfpathlineto{\pgfqpoint{4.508759in}{0.628430in}}%
\pgfpathlineto{\pgfqpoint{4.509315in}{0.602216in}}%
\pgfpathlineto{\pgfqpoint{4.509871in}{0.611848in}}%
\pgfpathlineto{\pgfqpoint{4.510426in}{0.604729in}}%
\pgfpathlineto{\pgfqpoint{4.511538in}{0.618035in}}%
\pgfpathlineto{\pgfqpoint{4.513206in}{0.606843in}}%
\pgfpathlineto{\pgfqpoint{4.514873in}{0.621461in}}%
\pgfpathlineto{\pgfqpoint{4.515429in}{0.610044in}}%
\pgfpathlineto{\pgfqpoint{4.515985in}{0.621412in}}%
\pgfpathlineto{\pgfqpoint{4.516541in}{0.622568in}}%
\pgfpathlineto{\pgfqpoint{4.518208in}{0.604756in}}%
\pgfpathlineto{\pgfqpoint{4.518764in}{0.619495in}}%
\pgfpathlineto{\pgfqpoint{4.519320in}{0.603294in}}%
\pgfpathlineto{\pgfqpoint{4.519876in}{0.616203in}}%
\pgfpathlineto{\pgfqpoint{4.520432in}{0.621641in}}%
\pgfpathlineto{\pgfqpoint{4.522100in}{0.606374in}}%
\pgfpathlineto{\pgfqpoint{4.523211in}{0.615458in}}%
\pgfpathlineto{\pgfqpoint{4.523767in}{0.606414in}}%
\pgfpathlineto{\pgfqpoint{4.524323in}{0.615389in}}%
\pgfpathlineto{\pgfqpoint{4.524879in}{0.618324in}}%
\pgfpathlineto{\pgfqpoint{4.525435in}{0.604883in}}%
\pgfpathlineto{\pgfqpoint{4.525991in}{0.605650in}}%
\pgfpathlineto{\pgfqpoint{4.527102in}{0.611361in}}%
\pgfpathlineto{\pgfqpoint{4.528214in}{0.600675in}}%
\pgfpathlineto{\pgfqpoint{4.529326in}{0.612173in}}%
\pgfpathlineto{\pgfqpoint{4.529882in}{0.602719in}}%
\pgfpathlineto{\pgfqpoint{4.530437in}{0.604022in}}%
\pgfpathlineto{\pgfqpoint{4.532105in}{0.612455in}}%
\pgfpathlineto{\pgfqpoint{4.532661in}{0.617953in}}%
\pgfpathlineto{\pgfqpoint{4.533773in}{0.607730in}}%
\pgfpathlineto{\pgfqpoint{4.534328in}{0.611363in}}%
\pgfpathlineto{\pgfqpoint{4.534884in}{0.612309in}}%
\pgfpathlineto{\pgfqpoint{4.535440in}{0.610532in}}%
\pgfpathlineto{\pgfqpoint{4.535996in}{0.604844in}}%
\pgfpathlineto{\pgfqpoint{4.536552in}{0.606766in}}%
\pgfpathlineto{\pgfqpoint{4.537108in}{0.610451in}}%
\pgfpathlineto{\pgfqpoint{4.537664in}{0.608303in}}%
\pgfpathlineto{\pgfqpoint{4.538220in}{0.603271in}}%
\pgfpathlineto{\pgfqpoint{4.538775in}{0.612608in}}%
\pgfpathlineto{\pgfqpoint{4.539331in}{0.611267in}}%
\pgfpathlineto{\pgfqpoint{4.539887in}{0.603377in}}%
\pgfpathlineto{\pgfqpoint{4.540443in}{0.605385in}}%
\pgfpathlineto{\pgfqpoint{4.541555in}{0.605552in}}%
\pgfpathlineto{\pgfqpoint{4.542111in}{0.608836in}}%
\pgfpathlineto{\pgfqpoint{4.542666in}{0.601132in}}%
\pgfpathlineto{\pgfqpoint{4.543222in}{0.610882in}}%
\pgfpathlineto{\pgfqpoint{4.543778in}{0.603549in}}%
\pgfpathlineto{\pgfqpoint{4.545446in}{0.610982in}}%
\pgfpathlineto{\pgfqpoint{4.546002in}{0.600806in}}%
\pgfpathlineto{\pgfqpoint{4.546557in}{0.609267in}}%
\pgfpathlineto{\pgfqpoint{4.547669in}{0.604292in}}%
\pgfpathlineto{\pgfqpoint{4.548225in}{0.607989in}}%
\pgfpathlineto{\pgfqpoint{4.548781in}{0.604913in}}%
\pgfpathlineto{\pgfqpoint{4.549893in}{0.600181in}}%
\pgfpathlineto{\pgfqpoint{4.551560in}{0.610927in}}%
\pgfpathlineto{\pgfqpoint{4.552116in}{0.601772in}}%
\pgfpathlineto{\pgfqpoint{4.552672in}{0.606662in}}%
\pgfpathlineto{\pgfqpoint{4.553228in}{0.609011in}}%
\pgfpathlineto{\pgfqpoint{4.554895in}{0.601281in}}%
\pgfpathlineto{\pgfqpoint{4.555451in}{0.603566in}}%
\pgfpathlineto{\pgfqpoint{4.556007in}{0.602530in}}%
\pgfpathlineto{\pgfqpoint{4.557675in}{0.612671in}}%
\pgfpathlineto{\pgfqpoint{4.558231in}{0.602519in}}%
\pgfpathlineto{\pgfqpoint{4.558786in}{0.608150in}}%
\pgfpathlineto{\pgfqpoint{4.559342in}{0.609832in}}%
\pgfpathlineto{\pgfqpoint{4.559898in}{0.601401in}}%
\pgfpathlineto{\pgfqpoint{4.560454in}{0.607456in}}%
\pgfpathlineto{\pgfqpoint{4.561010in}{0.607834in}}%
\pgfpathlineto{\pgfqpoint{4.561566in}{0.601164in}}%
\pgfpathlineto{\pgfqpoint{4.562122in}{0.606061in}}%
\pgfpathlineto{\pgfqpoint{4.563233in}{0.612174in}}%
\pgfpathlineto{\pgfqpoint{4.563789in}{0.607512in}}%
\pgfpathlineto{\pgfqpoint{4.564345in}{0.608636in}}%
\pgfpathlineto{\pgfqpoint{4.565457in}{0.603160in}}%
\pgfpathlineto{\pgfqpoint{4.566013in}{0.610085in}}%
\pgfpathlineto{\pgfqpoint{4.566568in}{0.608050in}}%
\pgfpathlineto{\pgfqpoint{4.567124in}{0.607843in}}%
\pgfpathlineto{\pgfqpoint{4.567680in}{0.602850in}}%
\pgfpathlineto{\pgfqpoint{4.568236in}{0.604392in}}%
\pgfpathlineto{\pgfqpoint{4.568792in}{0.619171in}}%
\pgfpathlineto{\pgfqpoint{4.570459in}{0.603008in}}%
\pgfpathlineto{\pgfqpoint{4.571015in}{0.608603in}}%
\pgfpathlineto{\pgfqpoint{4.571571in}{0.605228in}}%
\pgfpathlineto{\pgfqpoint{4.572683in}{0.609210in}}%
\pgfpathlineto{\pgfqpoint{4.573239in}{0.602960in}}%
\pgfpathlineto{\pgfqpoint{4.573795in}{0.625159in}}%
\pgfpathlineto{\pgfqpoint{4.574350in}{0.604154in}}%
\pgfpathlineto{\pgfqpoint{4.574906in}{0.609942in}}%
\pgfpathlineto{\pgfqpoint{4.575462in}{0.606056in}}%
\pgfpathlineto{\pgfqpoint{4.576574in}{0.612306in}}%
\pgfpathlineto{\pgfqpoint{4.577130in}{0.605972in}}%
\pgfpathlineto{\pgfqpoint{4.577686in}{0.613712in}}%
\pgfpathlineto{\pgfqpoint{4.578242in}{0.612969in}}%
\pgfpathlineto{\pgfqpoint{4.579353in}{0.603976in}}%
\pgfpathlineto{\pgfqpoint{4.579909in}{0.604055in}}%
\pgfpathlineto{\pgfqpoint{4.580465in}{0.614973in}}%
\pgfpathlineto{\pgfqpoint{4.581021in}{0.605659in}}%
\pgfpathlineto{\pgfqpoint{4.581577in}{0.608904in}}%
\pgfpathlineto{\pgfqpoint{4.582133in}{0.607347in}}%
\pgfpathlineto{\pgfqpoint{4.583244in}{0.602739in}}%
\pgfpathlineto{\pgfqpoint{4.583800in}{0.602955in}}%
\pgfpathlineto{\pgfqpoint{4.584356in}{0.609773in}}%
\pgfpathlineto{\pgfqpoint{4.584912in}{0.608255in}}%
\pgfpathlineto{\pgfqpoint{4.586579in}{0.602540in}}%
\pgfpathlineto{\pgfqpoint{4.587135in}{0.603344in}}%
\pgfpathlineto{\pgfqpoint{4.587691in}{0.601962in}}%
\pgfpathlineto{\pgfqpoint{4.589359in}{0.608045in}}%
\pgfpathlineto{\pgfqpoint{4.589915in}{0.602090in}}%
\pgfpathlineto{\pgfqpoint{4.590470in}{0.611759in}}%
\pgfpathlineto{\pgfqpoint{4.591026in}{0.606774in}}%
\pgfpathlineto{\pgfqpoint{4.591582in}{0.609698in}}%
\pgfpathlineto{\pgfqpoint{4.592138in}{0.604390in}}%
\pgfpathlineto{\pgfqpoint{4.592694in}{0.612026in}}%
\pgfpathlineto{\pgfqpoint{4.593806in}{0.600979in}}%
\pgfpathlineto{\pgfqpoint{4.594361in}{0.602210in}}%
\pgfpathlineto{\pgfqpoint{4.594917in}{0.601981in}}%
\pgfpathlineto{\pgfqpoint{4.596585in}{0.612749in}}%
\pgfpathlineto{\pgfqpoint{4.597141in}{0.602430in}}%
\pgfpathlineto{\pgfqpoint{4.597697in}{0.604649in}}%
\pgfpathlineto{\pgfqpoint{4.599364in}{0.605168in}}%
\pgfpathlineto{\pgfqpoint{4.599920in}{0.602129in}}%
\pgfpathlineto{\pgfqpoint{4.600476in}{0.604139in}}%
\pgfpathlineto{\pgfqpoint{4.601588in}{0.604350in}}%
\pgfpathlineto{\pgfqpoint{4.602144in}{0.604332in}}%
\pgfpathlineto{\pgfqpoint{4.602699in}{0.606088in}}%
\pgfpathlineto{\pgfqpoint{4.603255in}{0.604977in}}%
\pgfpathlineto{\pgfqpoint{4.603811in}{0.602722in}}%
\pgfpathlineto{\pgfqpoint{4.604367in}{0.603308in}}%
\pgfpathlineto{\pgfqpoint{4.604923in}{0.605969in}}%
\pgfpathlineto{\pgfqpoint{4.605479in}{0.605128in}}%
\pgfpathlineto{\pgfqpoint{4.607146in}{0.601651in}}%
\pgfpathlineto{\pgfqpoint{4.607702in}{0.602968in}}%
\pgfpathlineto{\pgfqpoint{4.608258in}{0.601913in}}%
\pgfpathlineto{\pgfqpoint{4.608814in}{0.600999in}}%
\pgfpathlineto{\pgfqpoint{4.609370in}{0.601429in}}%
\pgfpathlineto{\pgfqpoint{4.609926in}{0.607123in}}%
\pgfpathlineto{\pgfqpoint{4.610481in}{0.602700in}}%
\pgfpathlineto{\pgfqpoint{4.611037in}{0.604786in}}%
\pgfpathlineto{\pgfqpoint{4.612705in}{0.601192in}}%
\pgfpathlineto{\pgfqpoint{4.613817in}{0.605096in}}%
\pgfpathlineto{\pgfqpoint{4.614373in}{0.604856in}}%
\pgfpathlineto{\pgfqpoint{4.615484in}{0.600232in}}%
\pgfpathlineto{\pgfqpoint{4.616040in}{0.607420in}}%
\pgfpathlineto{\pgfqpoint{4.616596in}{0.606846in}}%
\pgfpathlineto{\pgfqpoint{4.617708in}{0.601696in}}%
\pgfpathlineto{\pgfqpoint{4.618264in}{0.602027in}}%
\pgfpathlineto{\pgfqpoint{4.618819in}{0.602438in}}%
\pgfpathlineto{\pgfqpoint{4.620487in}{0.607640in}}%
\pgfpathlineto{\pgfqpoint{4.621599in}{0.601859in}}%
\pgfpathlineto{\pgfqpoint{4.622710in}{0.606176in}}%
\pgfpathlineto{\pgfqpoint{4.623266in}{0.605287in}}%
\pgfpathlineto{\pgfqpoint{4.623822in}{0.604248in}}%
\pgfpathlineto{\pgfqpoint{4.624378in}{0.605730in}}%
\pgfpathlineto{\pgfqpoint{4.625490in}{0.603744in}}%
\pgfpathlineto{\pgfqpoint{4.626601in}{0.607274in}}%
\pgfpathlineto{\pgfqpoint{4.627713in}{0.603532in}}%
\pgfpathlineto{\pgfqpoint{4.628269in}{0.604540in}}%
\pgfpathlineto{\pgfqpoint{4.629381in}{0.605132in}}%
\pgfpathlineto{\pgfqpoint{4.630492in}{0.602659in}}%
\pgfpathlineto{\pgfqpoint{4.631048in}{0.609670in}}%
\pgfpathlineto{\pgfqpoint{4.631604in}{0.604069in}}%
\pgfpathlineto{\pgfqpoint{4.632160in}{0.605611in}}%
\pgfpathlineto{\pgfqpoint{4.632716in}{0.605151in}}%
\pgfpathlineto{\pgfqpoint{4.633272in}{0.601860in}}%
\pgfpathlineto{\pgfqpoint{4.633828in}{0.606626in}}%
\pgfpathlineto{\pgfqpoint{4.634384in}{0.603158in}}%
\pgfpathlineto{\pgfqpoint{4.634939in}{0.603832in}}%
\pgfpathlineto{\pgfqpoint{4.635495in}{0.612693in}}%
\pgfpathlineto{\pgfqpoint{4.636051in}{0.602074in}}%
\pgfpathlineto{\pgfqpoint{4.636607in}{0.611436in}}%
\pgfpathlineto{\pgfqpoint{4.638275in}{0.601596in}}%
\pgfpathlineto{\pgfqpoint{4.638830in}{0.602763in}}%
\pgfpathlineto{\pgfqpoint{4.639386in}{0.601246in}}%
\pgfpathlineto{\pgfqpoint{4.639942in}{0.610099in}}%
\pgfpathlineto{\pgfqpoint{4.640498in}{0.605195in}}%
\pgfpathlineto{\pgfqpoint{4.641054in}{0.602735in}}%
\pgfpathlineto{\pgfqpoint{4.641610in}{0.604929in}}%
\pgfpathlineto{\pgfqpoint{4.643277in}{0.602708in}}%
\pgfpathlineto{\pgfqpoint{4.643833in}{0.605094in}}%
\pgfpathlineto{\pgfqpoint{4.644389in}{0.601531in}}%
\pgfpathlineto{\pgfqpoint{4.644945in}{0.602477in}}%
\pgfpathlineto{\pgfqpoint{4.646612in}{0.600961in}}%
\pgfpathlineto{\pgfqpoint{4.647724in}{0.604985in}}%
\pgfpathlineto{\pgfqpoint{4.649392in}{0.603011in}}%
\pgfpathlineto{\pgfqpoint{4.649948in}{0.604935in}}%
\pgfpathlineto{\pgfqpoint{4.650503in}{0.603813in}}%
\pgfpathlineto{\pgfqpoint{4.651059in}{0.600312in}}%
\pgfpathlineto{\pgfqpoint{4.651615in}{0.601138in}}%
\pgfpathlineto{\pgfqpoint{4.652171in}{0.601388in}}%
\pgfpathlineto{\pgfqpoint{4.652727in}{0.606178in}}%
\pgfpathlineto{\pgfqpoint{4.653283in}{0.600689in}}%
\pgfpathlineto{\pgfqpoint{4.653839in}{0.604297in}}%
\pgfpathlineto{\pgfqpoint{4.654395in}{0.603128in}}%
\pgfpathlineto{\pgfqpoint{4.654950in}{0.603967in}}%
\pgfpathlineto{\pgfqpoint{4.657730in}{0.602411in}}%
\pgfpathlineto{\pgfqpoint{4.658286in}{0.600837in}}%
\pgfpathlineto{\pgfqpoint{4.658841in}{0.604518in}}%
\pgfpathlineto{\pgfqpoint{4.659397in}{0.602147in}}%
\pgfpathlineto{\pgfqpoint{4.661065in}{0.603741in}}%
\pgfpathlineto{\pgfqpoint{4.661621in}{0.602760in}}%
\pgfpathlineto{\pgfqpoint{4.663844in}{0.601704in}}%
\pgfpathlineto{\pgfqpoint{4.664400in}{0.602277in}}%
\pgfpathlineto{\pgfqpoint{4.664956in}{0.601553in}}%
\pgfpathlineto{\pgfqpoint{4.665512in}{0.606443in}}%
\pgfpathlineto{\pgfqpoint{4.666068in}{0.600895in}}%
\pgfpathlineto{\pgfqpoint{4.666623in}{0.601940in}}%
\pgfpathlineto{\pgfqpoint{4.667179in}{0.605302in}}%
\pgfpathlineto{\pgfqpoint{4.667735in}{0.604411in}}%
\pgfpathlineto{\pgfqpoint{4.668291in}{0.602429in}}%
\pgfpathlineto{\pgfqpoint{4.668847in}{0.603127in}}%
\pgfpathlineto{\pgfqpoint{4.669403in}{0.604249in}}%
\pgfpathlineto{\pgfqpoint{4.669959in}{0.603798in}}%
\pgfpathlineto{\pgfqpoint{4.671070in}{0.601595in}}%
\pgfpathlineto{\pgfqpoint{4.671626in}{0.604211in}}%
\pgfpathlineto{\pgfqpoint{4.672182in}{0.600204in}}%
\pgfpathlineto{\pgfqpoint{4.672738in}{0.600531in}}%
\pgfpathlineto{\pgfqpoint{4.674406in}{0.601854in}}%
\pgfpathlineto{\pgfqpoint{4.674961in}{0.605700in}}%
\pgfpathlineto{\pgfqpoint{4.675517in}{0.603669in}}%
\pgfpathlineto{\pgfqpoint{4.676073in}{0.601234in}}%
\pgfpathlineto{\pgfqpoint{4.676629in}{0.601609in}}%
\pgfpathlineto{\pgfqpoint{4.678852in}{0.604496in}}%
\pgfpathlineto{\pgfqpoint{4.679408in}{0.602136in}}%
\pgfpathlineto{\pgfqpoint{4.679964in}{0.604598in}}%
\pgfpathlineto{\pgfqpoint{4.680520in}{0.603922in}}%
\pgfpathlineto{\pgfqpoint{4.681076in}{0.605323in}}%
\pgfpathlineto{\pgfqpoint{4.682188in}{0.601833in}}%
\pgfpathlineto{\pgfqpoint{4.682743in}{0.602144in}}%
\pgfpathlineto{\pgfqpoint{4.683299in}{0.602926in}}%
\pgfpathlineto{\pgfqpoint{4.684411in}{0.601625in}}%
\pgfpathlineto{\pgfqpoint{4.686079in}{0.606526in}}%
\pgfpathlineto{\pgfqpoint{4.687746in}{0.600135in}}%
\pgfpathlineto{\pgfqpoint{4.689414in}{0.606654in}}%
\pgfpathlineto{\pgfqpoint{4.690526in}{0.600411in}}%
\pgfpathlineto{\pgfqpoint{4.691637in}{0.604982in}}%
\pgfpathlineto{\pgfqpoint{4.692193in}{0.600914in}}%
\pgfpathlineto{\pgfqpoint{4.692749in}{0.603940in}}%
\pgfpathlineto{\pgfqpoint{4.693305in}{0.605854in}}%
\pgfpathlineto{\pgfqpoint{4.694417in}{0.600873in}}%
\pgfpathlineto{\pgfqpoint{4.696084in}{0.606976in}}%
\pgfpathlineto{\pgfqpoint{4.697196in}{0.602344in}}%
\pgfpathlineto{\pgfqpoint{4.697752in}{0.606051in}}%
\pgfpathlineto{\pgfqpoint{4.698308in}{0.605154in}}%
\pgfpathlineto{\pgfqpoint{4.698863in}{0.600262in}}%
\pgfpathlineto{\pgfqpoint{4.699419in}{0.605189in}}%
\pgfpathlineto{\pgfqpoint{4.700531in}{0.600592in}}%
\pgfpathlineto{\pgfqpoint{4.702199in}{0.603638in}}%
\pgfpathlineto{\pgfqpoint{4.702754in}{0.602415in}}%
\pgfpathlineto{\pgfqpoint{4.704422in}{0.605371in}}%
\pgfpathlineto{\pgfqpoint{4.706090in}{0.602374in}}%
\pgfpathlineto{\pgfqpoint{4.707757in}{0.606884in}}%
\pgfpathlineto{\pgfqpoint{4.708313in}{0.601239in}}%
\pgfpathlineto{\pgfqpoint{4.708869in}{0.603348in}}%
\pgfpathlineto{\pgfqpoint{4.709425in}{0.602108in}}%
\pgfpathlineto{\pgfqpoint{4.711092in}{0.604617in}}%
\pgfpathlineto{\pgfqpoint{4.712204in}{0.605552in}}%
\pgfpathlineto{\pgfqpoint{4.712760in}{0.603872in}}%
\pgfpathlineto{\pgfqpoint{4.713316in}{0.607119in}}%
\pgfpathlineto{\pgfqpoint{4.714428in}{0.601543in}}%
\pgfpathlineto{\pgfqpoint{4.714983in}{0.602146in}}%
\pgfpathlineto{\pgfqpoint{4.715539in}{0.604962in}}%
\pgfpathlineto{\pgfqpoint{4.716095in}{0.601766in}}%
\pgfpathlineto{\pgfqpoint{4.716651in}{0.603580in}}%
\pgfpathlineto{\pgfqpoint{4.717207in}{0.602114in}}%
\pgfpathlineto{\pgfqpoint{4.717763in}{0.605320in}}%
\pgfpathlineto{\pgfqpoint{4.718319in}{0.601198in}}%
\pgfpathlineto{\pgfqpoint{4.718874in}{0.601811in}}%
\pgfpathlineto{\pgfqpoint{4.719986in}{0.602314in}}%
\pgfpathlineto{\pgfqpoint{4.720542in}{0.600389in}}%
\pgfpathlineto{\pgfqpoint{4.721098in}{0.604206in}}%
\pgfpathlineto{\pgfqpoint{4.721654in}{0.600933in}}%
\pgfpathlineto{\pgfqpoint{4.722765in}{0.605743in}}%
\pgfpathlineto{\pgfqpoint{4.723321in}{0.604320in}}%
\pgfpathlineto{\pgfqpoint{4.725545in}{0.600969in}}%
\pgfpathlineto{\pgfqpoint{4.726657in}{0.605439in}}%
\pgfpathlineto{\pgfqpoint{4.727212in}{0.602585in}}%
\pgfpathlineto{\pgfqpoint{4.727768in}{0.604916in}}%
\pgfpathlineto{\pgfqpoint{4.728880in}{0.602021in}}%
\pgfpathlineto{\pgfqpoint{4.729436in}{0.602335in}}%
\pgfpathlineto{\pgfqpoint{4.729992in}{0.603190in}}%
\pgfpathlineto{\pgfqpoint{4.730548in}{0.602792in}}%
\pgfpathlineto{\pgfqpoint{4.731659in}{0.600606in}}%
\pgfpathlineto{\pgfqpoint{4.732771in}{0.604135in}}%
\pgfpathlineto{\pgfqpoint{4.733327in}{0.600937in}}%
\pgfpathlineto{\pgfqpoint{4.733883in}{0.603051in}}%
\pgfpathlineto{\pgfqpoint{4.734439in}{0.603035in}}%
\pgfpathlineto{\pgfqpoint{4.735550in}{0.600952in}}%
\pgfpathlineto{\pgfqpoint{4.736106in}{0.609508in}}%
\pgfpathlineto{\pgfqpoint{4.736662in}{0.607298in}}%
\pgfpathlineto{\pgfqpoint{4.737774in}{0.602815in}}%
\pgfpathlineto{\pgfqpoint{4.738330in}{0.605176in}}%
\pgfpathlineto{\pgfqpoint{4.738885in}{0.606228in}}%
\pgfpathlineto{\pgfqpoint{4.740553in}{0.603557in}}%
\pgfpathlineto{\pgfqpoint{4.741109in}{0.606531in}}%
\pgfpathlineto{\pgfqpoint{4.741665in}{0.600327in}}%
\pgfpathlineto{\pgfqpoint{4.742221in}{0.602498in}}%
\pgfpathlineto{\pgfqpoint{4.743888in}{0.608316in}}%
\pgfpathlineto{\pgfqpoint{4.745000in}{0.602510in}}%
\pgfpathlineto{\pgfqpoint{4.745556in}{0.604168in}}%
\pgfpathlineto{\pgfqpoint{4.746112in}{0.605378in}}%
\pgfpathlineto{\pgfqpoint{4.747779in}{0.601004in}}%
\pgfpathlineto{\pgfqpoint{4.748335in}{0.602616in}}%
\pgfpathlineto{\pgfqpoint{4.748891in}{0.610586in}}%
\pgfpathlineto{\pgfqpoint{4.749447in}{0.603477in}}%
\pgfpathlineto{\pgfqpoint{4.750003in}{0.607907in}}%
\pgfpathlineto{\pgfqpoint{4.750559in}{0.606979in}}%
\pgfpathlineto{\pgfqpoint{4.751114in}{0.602561in}}%
\pgfpathlineto{\pgfqpoint{4.751670in}{0.604136in}}%
\pgfpathlineto{\pgfqpoint{4.752782in}{0.602086in}}%
\pgfpathlineto{\pgfqpoint{4.753338in}{0.603143in}}%
\pgfpathlineto{\pgfqpoint{4.755005in}{0.607834in}}%
\pgfpathlineto{\pgfqpoint{4.756117in}{0.602152in}}%
\pgfpathlineto{\pgfqpoint{4.757785in}{0.607375in}}%
\pgfpathlineto{\pgfqpoint{4.758341in}{0.604295in}}%
\pgfpathlineto{\pgfqpoint{4.758896in}{0.605751in}}%
\pgfpathlineto{\pgfqpoint{4.759452in}{0.606450in}}%
\pgfpathlineto{\pgfqpoint{4.760564in}{0.602946in}}%
\pgfpathlineto{\pgfqpoint{4.761120in}{0.603448in}}%
\pgfpathlineto{\pgfqpoint{4.761676in}{0.602010in}}%
\pgfpathlineto{\pgfqpoint{4.762787in}{0.611022in}}%
\pgfpathlineto{\pgfqpoint{4.763343in}{0.605707in}}%
\pgfpathlineto{\pgfqpoint{4.765011in}{0.612073in}}%
\pgfpathlineto{\pgfqpoint{4.765567in}{0.601824in}}%
\pgfpathlineto{\pgfqpoint{4.766123in}{0.604147in}}%
\pgfpathlineto{\pgfqpoint{4.766679in}{0.602850in}}%
\pgfpathlineto{\pgfqpoint{4.767790in}{0.612160in}}%
\pgfpathlineto{\pgfqpoint{4.768346in}{0.603453in}}%
\pgfpathlineto{\pgfqpoint{4.768902in}{0.607313in}}%
\pgfpathlineto{\pgfqpoint{4.770014in}{0.605708in}}%
\pgfpathlineto{\pgfqpoint{4.770570in}{0.610836in}}%
\pgfpathlineto{\pgfqpoint{4.772237in}{0.603428in}}%
\pgfpathlineto{\pgfqpoint{4.772793in}{0.601809in}}%
\pgfpathlineto{\pgfqpoint{4.773905in}{0.603996in}}%
\pgfpathlineto{\pgfqpoint{4.774461in}{0.601483in}}%
\pgfpathlineto{\pgfqpoint{4.775016in}{0.604439in}}%
\pgfpathlineto{\pgfqpoint{4.775572in}{0.602829in}}%
\pgfpathlineto{\pgfqpoint{4.777240in}{0.605176in}}%
\pgfpathlineto{\pgfqpoint{4.777796in}{0.603107in}}%
\pgfpathlineto{\pgfqpoint{4.778352in}{0.603739in}}%
\pgfpathlineto{\pgfqpoint{4.778907in}{0.604045in}}%
\pgfpathlineto{\pgfqpoint{4.779463in}{0.601904in}}%
\pgfpathlineto{\pgfqpoint{4.780019in}{0.606678in}}%
\pgfpathlineto{\pgfqpoint{4.780575in}{0.605951in}}%
\pgfpathlineto{\pgfqpoint{4.781131in}{0.600933in}}%
\pgfpathlineto{\pgfqpoint{4.781687in}{0.601763in}}%
\pgfpathlineto{\pgfqpoint{4.782243in}{0.606100in}}%
\pgfpathlineto{\pgfqpoint{4.782798in}{0.602830in}}%
\pgfpathlineto{\pgfqpoint{4.783354in}{0.602193in}}%
\pgfpathlineto{\pgfqpoint{4.784466in}{0.608213in}}%
\pgfpathlineto{\pgfqpoint{4.785022in}{0.600015in}}%
\pgfpathlineto{\pgfqpoint{4.785578in}{0.602557in}}%
\pgfpathlineto{\pgfqpoint{4.786134in}{0.600948in}}%
\pgfpathlineto{\pgfqpoint{4.787245in}{0.604023in}}%
\pgfpathlineto{\pgfqpoint{4.787801in}{0.602919in}}%
\pgfpathlineto{\pgfqpoint{4.788357in}{0.607126in}}%
\pgfpathlineto{\pgfqpoint{4.788913in}{0.605086in}}%
\pgfpathlineto{\pgfqpoint{4.790025in}{0.603305in}}%
\pgfpathlineto{\pgfqpoint{4.790581in}{0.605029in}}%
\pgfpathlineto{\pgfqpoint{4.791692in}{0.600862in}}%
\pgfpathlineto{\pgfqpoint{4.792248in}{0.601898in}}%
\pgfpathlineto{\pgfqpoint{4.792804in}{0.601196in}}%
\pgfpathlineto{\pgfqpoint{4.793916in}{0.608525in}}%
\pgfpathlineto{\pgfqpoint{4.794472in}{0.601523in}}%
\pgfpathlineto{\pgfqpoint{4.795027in}{0.602772in}}%
\pgfpathlineto{\pgfqpoint{4.795583in}{0.605849in}}%
\pgfpathlineto{\pgfqpoint{4.796139in}{0.615833in}}%
\pgfpathlineto{\pgfqpoint{4.796695in}{0.602567in}}%
\pgfpathlineto{\pgfqpoint{4.797251in}{0.605761in}}%
\pgfpathlineto{\pgfqpoint{4.797807in}{0.602765in}}%
\pgfpathlineto{\pgfqpoint{4.798363in}{0.605204in}}%
\pgfpathlineto{\pgfqpoint{4.798918in}{0.607554in}}%
\pgfpathlineto{\pgfqpoint{4.800030in}{0.601404in}}%
\pgfpathlineto{\pgfqpoint{4.800586in}{0.602201in}}%
\pgfpathlineto{\pgfqpoint{4.801142in}{0.612149in}}%
\pgfpathlineto{\pgfqpoint{4.802254in}{0.601465in}}%
\pgfpathlineto{\pgfqpoint{4.803921in}{0.607783in}}%
\pgfpathlineto{\pgfqpoint{4.805033in}{0.601777in}}%
\pgfpathlineto{\pgfqpoint{4.806145in}{0.608352in}}%
\pgfpathlineto{\pgfqpoint{4.806701in}{0.606107in}}%
\pgfpathlineto{\pgfqpoint{4.807256in}{0.606955in}}%
\pgfpathlineto{\pgfqpoint{4.807812in}{0.610326in}}%
\pgfpathlineto{\pgfqpoint{4.808368in}{0.600768in}}%
\pgfpathlineto{\pgfqpoint{4.808924in}{0.606703in}}%
\pgfpathlineto{\pgfqpoint{4.809480in}{0.608282in}}%
\pgfpathlineto{\pgfqpoint{4.810036in}{0.608059in}}%
\pgfpathlineto{\pgfqpoint{4.811147in}{0.603087in}}%
\pgfpathlineto{\pgfqpoint{4.811703in}{0.603259in}}%
\pgfpathlineto{\pgfqpoint{4.812815in}{0.608502in}}%
\pgfpathlineto{\pgfqpoint{4.813927in}{0.600555in}}%
\pgfpathlineto{\pgfqpoint{4.814483in}{0.605652in}}%
\pgfpathlineto{\pgfqpoint{4.815038in}{0.602485in}}%
\pgfpathlineto{\pgfqpoint{4.815594in}{0.601088in}}%
\pgfpathlineto{\pgfqpoint{4.816150in}{0.601889in}}%
\pgfpathlineto{\pgfqpoint{4.817262in}{0.601619in}}%
\pgfpathlineto{\pgfqpoint{4.817818in}{0.602510in}}%
\pgfpathlineto{\pgfqpoint{4.818374in}{0.601705in}}%
\pgfpathlineto{\pgfqpoint{4.819485in}{0.601026in}}%
\pgfpathlineto{\pgfqpoint{4.820041in}{0.601916in}}%
\pgfpathlineto{\pgfqpoint{4.820597in}{0.601491in}}%
\pgfpathlineto{\pgfqpoint{4.824488in}{0.601313in}}%
\pgfpathlineto{\pgfqpoint{4.827267in}{0.603738in}}%
\pgfpathlineto{\pgfqpoint{4.827823in}{0.603641in}}%
\pgfpathlineto{\pgfqpoint{4.828379in}{0.606588in}}%
\pgfpathlineto{\pgfqpoint{4.828935in}{0.603938in}}%
\pgfpathlineto{\pgfqpoint{4.829491in}{0.602546in}}%
\pgfpathlineto{\pgfqpoint{4.830603in}{0.607849in}}%
\pgfpathlineto{\pgfqpoint{4.831158in}{0.601646in}}%
\pgfpathlineto{\pgfqpoint{4.831714in}{0.601897in}}%
\pgfpathlineto{\pgfqpoint{4.832826in}{0.608236in}}%
\pgfpathlineto{\pgfqpoint{4.833382in}{0.606678in}}%
\pgfpathlineto{\pgfqpoint{4.833938in}{0.603499in}}%
\pgfpathlineto{\pgfqpoint{4.834494in}{0.607397in}}%
\pgfpathlineto{\pgfqpoint{4.835049in}{0.602346in}}%
\pgfpathlineto{\pgfqpoint{4.835605in}{0.605993in}}%
\pgfpathlineto{\pgfqpoint{4.836717in}{0.602575in}}%
\pgfpathlineto{\pgfqpoint{4.838385in}{0.605747in}}%
\pgfpathlineto{\pgfqpoint{4.838940in}{0.601778in}}%
\pgfpathlineto{\pgfqpoint{4.839496in}{0.606322in}}%
\pgfpathlineto{\pgfqpoint{4.840052in}{0.606008in}}%
\pgfpathlineto{\pgfqpoint{4.840608in}{0.602867in}}%
\pgfpathlineto{\pgfqpoint{4.841164in}{0.604924in}}%
\pgfpathlineto{\pgfqpoint{4.841720in}{0.605664in}}%
\pgfpathlineto{\pgfqpoint{4.842276in}{0.603283in}}%
\pgfpathlineto{\pgfqpoint{4.842832in}{0.603964in}}%
\pgfpathlineto{\pgfqpoint{4.843387in}{0.603693in}}%
\pgfpathlineto{\pgfqpoint{4.843943in}{0.601936in}}%
\pgfpathlineto{\pgfqpoint{4.844499in}{0.603496in}}%
\pgfpathlineto{\pgfqpoint{4.845055in}{0.606563in}}%
\pgfpathlineto{\pgfqpoint{4.846167in}{0.601497in}}%
\pgfpathlineto{\pgfqpoint{4.847834in}{0.605437in}}%
\pgfpathlineto{\pgfqpoint{4.849502in}{0.601160in}}%
\pgfpathlineto{\pgfqpoint{4.850614in}{0.604035in}}%
\pgfpathlineto{\pgfqpoint{4.851169in}{0.610316in}}%
\pgfpathlineto{\pgfqpoint{4.851725in}{0.606781in}}%
\pgfpathlineto{\pgfqpoint{4.852281in}{0.604305in}}%
\pgfpathlineto{\pgfqpoint{4.853949in}{0.612655in}}%
\pgfpathlineto{\pgfqpoint{4.854505in}{0.602913in}}%
\pgfpathlineto{\pgfqpoint{4.855060in}{0.607019in}}%
\pgfpathlineto{\pgfqpoint{4.856728in}{0.608590in}}%
\pgfpathlineto{\pgfqpoint{4.857840in}{0.604871in}}%
\pgfpathlineto{\pgfqpoint{4.858396in}{0.613258in}}%
\pgfpathlineto{\pgfqpoint{4.858952in}{0.608827in}}%
\pgfpathlineto{\pgfqpoint{4.860619in}{0.606476in}}%
\pgfpathlineto{\pgfqpoint{4.861175in}{0.614178in}}%
\pgfpathlineto{\pgfqpoint{4.862843in}{0.600607in}}%
\pgfpathlineto{\pgfqpoint{4.863398in}{0.615287in}}%
\pgfpathlineto{\pgfqpoint{4.863954in}{0.608518in}}%
\pgfpathlineto{\pgfqpoint{4.864510in}{0.607611in}}%
\pgfpathlineto{\pgfqpoint{4.865066in}{0.602913in}}%
\pgfpathlineto{\pgfqpoint{4.865622in}{0.605896in}}%
\pgfpathlineto{\pgfqpoint{4.867289in}{0.603178in}}%
\pgfpathlineto{\pgfqpoint{4.867845in}{0.603082in}}%
\pgfpathlineto{\pgfqpoint{4.868401in}{0.610685in}}%
\pgfpathlineto{\pgfqpoint{4.868957in}{0.609094in}}%
\pgfpathlineto{\pgfqpoint{4.870625in}{0.602444in}}%
\pgfpathlineto{\pgfqpoint{4.871180in}{0.612935in}}%
\pgfpathlineto{\pgfqpoint{4.871736in}{0.608312in}}%
\pgfpathlineto{\pgfqpoint{4.873404in}{0.600674in}}%
\pgfpathlineto{\pgfqpoint{4.873960in}{0.608841in}}%
\pgfpathlineto{\pgfqpoint{4.874516in}{0.603807in}}%
\pgfpathlineto{\pgfqpoint{4.875627in}{0.602192in}}%
\pgfpathlineto{\pgfqpoint{4.876739in}{0.610132in}}%
\pgfpathlineto{\pgfqpoint{4.877295in}{0.606032in}}%
\pgfpathlineto{\pgfqpoint{4.878407in}{0.615734in}}%
\pgfpathlineto{\pgfqpoint{4.878963in}{0.605372in}}%
\pgfpathlineto{\pgfqpoint{4.879518in}{0.606324in}}%
\pgfpathlineto{\pgfqpoint{4.880074in}{0.613056in}}%
\pgfpathlineto{\pgfqpoint{4.880630in}{0.603283in}}%
\pgfpathlineto{\pgfqpoint{4.881186in}{0.610592in}}%
\pgfpathlineto{\pgfqpoint{4.882854in}{0.611397in}}%
\pgfpathlineto{\pgfqpoint{4.883409in}{0.604495in}}%
\pgfpathlineto{\pgfqpoint{4.883965in}{0.605009in}}%
\pgfpathlineto{\pgfqpoint{4.884521in}{0.612829in}}%
\pgfpathlineto{\pgfqpoint{4.885077in}{0.603345in}}%
\pgfpathlineto{\pgfqpoint{4.885633in}{0.613106in}}%
\pgfpathlineto{\pgfqpoint{4.886189in}{0.606589in}}%
\pgfpathlineto{\pgfqpoint{4.886745in}{0.602049in}}%
\pgfpathlineto{\pgfqpoint{4.887300in}{0.612345in}}%
\pgfpathlineto{\pgfqpoint{4.887856in}{0.610285in}}%
\pgfpathlineto{\pgfqpoint{4.889524in}{0.601288in}}%
\pgfpathlineto{\pgfqpoint{4.890080in}{0.607698in}}%
\pgfpathlineto{\pgfqpoint{4.890636in}{0.605257in}}%
\pgfpathlineto{\pgfqpoint{4.891191in}{0.606231in}}%
\pgfpathlineto{\pgfqpoint{4.891747in}{0.610449in}}%
\pgfpathlineto{\pgfqpoint{4.892859in}{0.601522in}}%
\pgfpathlineto{\pgfqpoint{4.893415in}{0.603069in}}%
\pgfpathlineto{\pgfqpoint{4.895638in}{0.608840in}}%
\pgfpathlineto{\pgfqpoint{4.896194in}{0.603466in}}%
\pgfpathlineto{\pgfqpoint{4.896750in}{0.606086in}}%
\pgfpathlineto{\pgfqpoint{4.897306in}{0.606881in}}%
\pgfpathlineto{\pgfqpoint{4.897862in}{0.601613in}}%
\pgfpathlineto{\pgfqpoint{4.898418in}{0.605700in}}%
\pgfpathlineto{\pgfqpoint{4.898974in}{0.610966in}}%
\pgfpathlineto{\pgfqpoint{4.899529in}{0.606547in}}%
\pgfpathlineto{\pgfqpoint{4.901197in}{0.602933in}}%
\pgfpathlineto{\pgfqpoint{4.902309in}{0.606187in}}%
\pgfpathlineto{\pgfqpoint{4.902865in}{0.600959in}}%
\pgfpathlineto{\pgfqpoint{4.903420in}{0.602901in}}%
\pgfpathlineto{\pgfqpoint{4.903976in}{0.606587in}}%
\pgfpathlineto{\pgfqpoint{4.904532in}{0.605241in}}%
\pgfpathlineto{\pgfqpoint{4.905088in}{0.602434in}}%
\pgfpathlineto{\pgfqpoint{4.906756in}{0.606557in}}%
\pgfpathlineto{\pgfqpoint{4.907867in}{0.601159in}}%
\pgfpathlineto{\pgfqpoint{4.908979in}{0.613398in}}%
\pgfpathlineto{\pgfqpoint{4.910091in}{0.602628in}}%
\pgfpathlineto{\pgfqpoint{4.910647in}{0.603937in}}%
\pgfpathlineto{\pgfqpoint{4.911202in}{0.615580in}}%
\pgfpathlineto{\pgfqpoint{4.911758in}{0.602396in}}%
\pgfpathlineto{\pgfqpoint{4.912314in}{0.611614in}}%
\pgfpathlineto{\pgfqpoint{4.912870in}{0.615616in}}%
\pgfpathlineto{\pgfqpoint{4.913426in}{0.603703in}}%
\pgfpathlineto{\pgfqpoint{4.913982in}{0.609693in}}%
\pgfpathlineto{\pgfqpoint{4.914538in}{0.610865in}}%
\pgfpathlineto{\pgfqpoint{4.915094in}{0.604011in}}%
\pgfpathlineto{\pgfqpoint{4.915649in}{0.619682in}}%
\pgfpathlineto{\pgfqpoint{4.916205in}{0.605666in}}%
\pgfpathlineto{\pgfqpoint{4.916761in}{0.616006in}}%
\pgfpathlineto{\pgfqpoint{4.917317in}{0.601076in}}%
\pgfpathlineto{\pgfqpoint{4.917873in}{0.604457in}}%
\pgfpathlineto{\pgfqpoint{4.918429in}{0.605029in}}%
\pgfpathlineto{\pgfqpoint{4.918985in}{0.610469in}}%
\pgfpathlineto{\pgfqpoint{4.919540in}{0.601133in}}%
\pgfpathlineto{\pgfqpoint{4.920096in}{0.615460in}}%
\pgfpathlineto{\pgfqpoint{4.920652in}{0.614167in}}%
\pgfpathlineto{\pgfqpoint{4.921208in}{0.615312in}}%
\pgfpathlineto{\pgfqpoint{4.922876in}{0.606909in}}%
\pgfpathlineto{\pgfqpoint{4.923431in}{0.606012in}}%
\pgfpathlineto{\pgfqpoint{4.923987in}{0.612169in}}%
\pgfpathlineto{\pgfqpoint{4.924543in}{0.606833in}}%
\pgfpathlineto{\pgfqpoint{4.925655in}{0.605024in}}%
\pgfpathlineto{\pgfqpoint{4.926211in}{0.613483in}}%
\pgfpathlineto{\pgfqpoint{4.926767in}{0.602169in}}%
\pgfpathlineto{\pgfqpoint{4.927322in}{0.609148in}}%
\pgfpathlineto{\pgfqpoint{4.928990in}{0.603554in}}%
\pgfpathlineto{\pgfqpoint{4.929546in}{0.605188in}}%
\pgfpathlineto{\pgfqpoint{4.930102in}{0.613361in}}%
\pgfpathlineto{\pgfqpoint{4.930658in}{0.609726in}}%
\pgfpathlineto{\pgfqpoint{4.931769in}{0.602193in}}%
\pgfpathlineto{\pgfqpoint{4.932325in}{0.602343in}}%
\pgfpathlineto{\pgfqpoint{4.932881in}{0.607960in}}%
\pgfpathlineto{\pgfqpoint{4.933437in}{0.602431in}}%
\pgfpathlineto{\pgfqpoint{4.935105in}{0.607412in}}%
\pgfpathlineto{\pgfqpoint{4.935660in}{0.608189in}}%
\pgfpathlineto{\pgfqpoint{4.936216in}{0.613151in}}%
\pgfpathlineto{\pgfqpoint{4.936772in}{0.606536in}}%
\pgfpathlineto{\pgfqpoint{4.937328in}{0.619099in}}%
\pgfpathlineto{\pgfqpoint{4.937884in}{0.612886in}}%
\pgfpathlineto{\pgfqpoint{4.938996in}{0.601983in}}%
\pgfpathlineto{\pgfqpoint{4.939551in}{0.614022in}}%
\pgfpathlineto{\pgfqpoint{4.940107in}{0.613899in}}%
\pgfpathlineto{\pgfqpoint{4.940663in}{0.605966in}}%
\pgfpathlineto{\pgfqpoint{4.941219in}{0.607166in}}%
\pgfpathlineto{\pgfqpoint{4.941775in}{0.607852in}}%
\pgfpathlineto{\pgfqpoint{4.942331in}{0.615616in}}%
\pgfpathlineto{\pgfqpoint{4.942887in}{0.611557in}}%
\pgfpathlineto{\pgfqpoint{4.943442in}{0.612377in}}%
\pgfpathlineto{\pgfqpoint{4.944554in}{0.605894in}}%
\pgfpathlineto{\pgfqpoint{4.945110in}{0.613604in}}%
\pgfpathlineto{\pgfqpoint{4.945666in}{0.612343in}}%
\pgfpathlineto{\pgfqpoint{4.947333in}{0.603592in}}%
\pgfpathlineto{\pgfqpoint{4.947889in}{0.602078in}}%
\pgfpathlineto{\pgfqpoint{4.948445in}{0.609728in}}%
\pgfpathlineto{\pgfqpoint{4.949001in}{0.607988in}}%
\pgfpathlineto{\pgfqpoint{4.949557in}{0.609180in}}%
\pgfpathlineto{\pgfqpoint{4.950669in}{0.602330in}}%
\pgfpathlineto{\pgfqpoint{4.952892in}{0.607371in}}%
\pgfpathlineto{\pgfqpoint{4.953448in}{0.606091in}}%
\pgfpathlineto{\pgfqpoint{4.954004in}{0.603001in}}%
\pgfpathlineto{\pgfqpoint{4.954560in}{0.610824in}}%
\pgfpathlineto{\pgfqpoint{4.955116in}{0.605954in}}%
\pgfpathlineto{\pgfqpoint{4.955671in}{0.608983in}}%
\pgfpathlineto{\pgfqpoint{4.956227in}{0.605382in}}%
\pgfpathlineto{\pgfqpoint{4.956783in}{0.609894in}}%
\pgfpathlineto{\pgfqpoint{4.957339in}{0.605846in}}%
\pgfpathlineto{\pgfqpoint{4.959007in}{0.601570in}}%
\pgfpathlineto{\pgfqpoint{4.960118in}{0.606687in}}%
\pgfpathlineto{\pgfqpoint{4.960674in}{0.606130in}}%
\pgfpathlineto{\pgfqpoint{4.961230in}{0.605491in}}%
\pgfpathlineto{\pgfqpoint{4.961786in}{0.608290in}}%
\pgfpathlineto{\pgfqpoint{4.962342in}{0.601294in}}%
\pgfpathlineto{\pgfqpoint{4.962898in}{0.606986in}}%
\pgfpathlineto{\pgfqpoint{4.964565in}{0.602119in}}%
\pgfpathlineto{\pgfqpoint{4.965121in}{0.602641in}}%
\pgfpathlineto{\pgfqpoint{4.966233in}{0.608026in}}%
\pgfpathlineto{\pgfqpoint{4.966789in}{0.607615in}}%
\pgfpathlineto{\pgfqpoint{4.967344in}{0.602698in}}%
\pgfpathlineto{\pgfqpoint{4.967900in}{0.605801in}}%
\pgfpathlineto{\pgfqpoint{4.968456in}{0.607406in}}%
\pgfpathlineto{\pgfqpoint{4.969012in}{0.604017in}}%
\pgfpathlineto{\pgfqpoint{4.969568in}{0.604302in}}%
\pgfpathlineto{\pgfqpoint{4.970124in}{0.612047in}}%
\pgfpathlineto{\pgfqpoint{4.970680in}{0.608531in}}%
\pgfpathlineto{\pgfqpoint{4.971235in}{0.608860in}}%
\pgfpathlineto{\pgfqpoint{4.971791in}{0.605987in}}%
\pgfpathlineto{\pgfqpoint{4.973459in}{0.613805in}}%
\pgfpathlineto{\pgfqpoint{4.975127in}{0.602051in}}%
\pgfpathlineto{\pgfqpoint{4.976794in}{0.611328in}}%
\pgfpathlineto{\pgfqpoint{4.977350in}{0.609771in}}%
\pgfpathlineto{\pgfqpoint{4.977906in}{0.605454in}}%
\pgfpathlineto{\pgfqpoint{4.978462in}{0.607032in}}%
\pgfpathlineto{\pgfqpoint{4.979018in}{0.612612in}}%
\pgfpathlineto{\pgfqpoint{4.979573in}{0.603207in}}%
\pgfpathlineto{\pgfqpoint{4.980129in}{0.607551in}}%
\pgfpathlineto{\pgfqpoint{4.981797in}{0.603432in}}%
\pgfpathlineto{\pgfqpoint{4.983464in}{0.606936in}}%
\pgfpathlineto{\pgfqpoint{4.984020in}{0.607242in}}%
\pgfpathlineto{\pgfqpoint{4.984576in}{0.609187in}}%
\pgfpathlineto{\pgfqpoint{4.985132in}{0.602005in}}%
\pgfpathlineto{\pgfqpoint{4.985688in}{0.609191in}}%
\pgfpathlineto{\pgfqpoint{4.987355in}{0.604246in}}%
\pgfpathlineto{\pgfqpoint{4.988467in}{0.601561in}}%
\pgfpathlineto{\pgfqpoint{4.989023in}{0.602626in}}%
\pgfpathlineto{\pgfqpoint{4.989579in}{0.610401in}}%
\pgfpathlineto{\pgfqpoint{4.990135in}{0.607526in}}%
\pgfpathlineto{\pgfqpoint{4.991247in}{0.603247in}}%
\pgfpathlineto{\pgfqpoint{4.991802in}{0.604179in}}%
\pgfpathlineto{\pgfqpoint{4.992358in}{0.606110in}}%
\pgfpathlineto{\pgfqpoint{4.992914in}{0.603375in}}%
\pgfpathlineto{\pgfqpoint{4.993470in}{0.608322in}}%
\pgfpathlineto{\pgfqpoint{4.994026in}{0.604553in}}%
\pgfpathlineto{\pgfqpoint{4.995693in}{0.605726in}}%
\pgfpathlineto{\pgfqpoint{4.996249in}{0.609583in}}%
\pgfpathlineto{\pgfqpoint{4.996805in}{0.606681in}}%
\pgfpathlineto{\pgfqpoint{4.997361in}{0.607037in}}%
\pgfpathlineto{\pgfqpoint{4.997917in}{0.602618in}}%
\pgfpathlineto{\pgfqpoint{4.998473in}{0.606849in}}%
\pgfpathlineto{\pgfqpoint{4.999029in}{0.607179in}}%
\pgfpathlineto{\pgfqpoint{4.999584in}{0.606085in}}%
\pgfpathlineto{\pgfqpoint{5.000140in}{0.602025in}}%
\pgfpathlineto{\pgfqpoint{5.000696in}{0.608419in}}%
\pgfpathlineto{\pgfqpoint{5.001252in}{0.606975in}}%
\pgfpathlineto{\pgfqpoint{5.001808in}{0.605178in}}%
\pgfpathlineto{\pgfqpoint{5.002364in}{0.605514in}}%
\pgfpathlineto{\pgfqpoint{5.002920in}{0.607042in}}%
\pgfpathlineto{\pgfqpoint{5.003475in}{0.602990in}}%
\pgfpathlineto{\pgfqpoint{5.004031in}{0.604071in}}%
\pgfpathlineto{\pgfqpoint{5.004587in}{0.604575in}}%
\pgfpathlineto{\pgfqpoint{5.006255in}{0.602587in}}%
\pgfpathlineto{\pgfqpoint{5.009590in}{0.601264in}}%
\pgfpathlineto{\pgfqpoint{5.010146in}{0.603515in}}%
\pgfpathlineto{\pgfqpoint{5.010702in}{0.601477in}}%
\pgfpathlineto{\pgfqpoint{5.011813in}{0.604616in}}%
\pgfpathlineto{\pgfqpoint{5.012369in}{0.603163in}}%
\pgfpathlineto{\pgfqpoint{5.013481in}{0.601030in}}%
\pgfpathlineto{\pgfqpoint{5.014037in}{0.605094in}}%
\pgfpathlineto{\pgfqpoint{5.014593in}{0.604153in}}%
\pgfpathlineto{\pgfqpoint{5.015149in}{0.604090in}}%
\pgfpathlineto{\pgfqpoint{5.016260in}{0.601445in}}%
\pgfpathlineto{\pgfqpoint{5.016816in}{0.602356in}}%
\pgfpathlineto{\pgfqpoint{5.018484in}{0.599991in}}%
\pgfpathlineto{\pgfqpoint{5.020151in}{0.603893in}}%
\pgfpathlineto{\pgfqpoint{5.020707in}{0.601625in}}%
\pgfpathlineto{\pgfqpoint{5.021263in}{0.604107in}}%
\pgfpathlineto{\pgfqpoint{5.022375in}{0.601147in}}%
\pgfpathlineto{\pgfqpoint{5.022931in}{0.601537in}}%
\pgfpathlineto{\pgfqpoint{5.023486in}{0.604324in}}%
\pgfpathlineto{\pgfqpoint{5.024042in}{0.602324in}}%
\pgfpathlineto{\pgfqpoint{5.026822in}{0.600357in}}%
\pgfpathlineto{\pgfqpoint{5.028489in}{0.604158in}}%
\pgfpathlineto{\pgfqpoint{5.030157in}{0.601080in}}%
\pgfpathlineto{\pgfqpoint{5.030713in}{0.603891in}}%
\pgfpathlineto{\pgfqpoint{5.031269in}{0.603302in}}%
\pgfpathlineto{\pgfqpoint{5.033492in}{0.600434in}}%
\pgfpathlineto{\pgfqpoint{5.034048in}{0.603922in}}%
\pgfpathlineto{\pgfqpoint{5.034604in}{0.602607in}}%
\pgfpathlineto{\pgfqpoint{5.035160in}{0.602005in}}%
\pgfpathlineto{\pgfqpoint{5.035715in}{0.602875in}}%
\pgfpathlineto{\pgfqpoint{5.037383in}{0.602834in}}%
\pgfpathlineto{\pgfqpoint{5.037939in}{0.600445in}}%
\pgfpathlineto{\pgfqpoint{5.038495in}{0.603576in}}%
\pgfpathlineto{\pgfqpoint{5.039051in}{0.600387in}}%
\pgfpathlineto{\pgfqpoint{5.040718in}{0.601991in}}%
\pgfpathlineto{\pgfqpoint{5.042386in}{0.600817in}}%
\pgfpathlineto{\pgfqpoint{5.044053in}{0.602406in}}%
\pgfpathlineto{\pgfqpoint{5.045721in}{0.600765in}}%
\pgfpathlineto{\pgfqpoint{5.046833in}{0.601083in}}%
\pgfpathlineto{\pgfqpoint{5.048500in}{0.600344in}}%
\pgfpathlineto{\pgfqpoint{5.049056in}{0.601850in}}%
\pgfpathlineto{\pgfqpoint{5.049612in}{0.601605in}}%
\pgfpathlineto{\pgfqpoint{5.050724in}{0.600483in}}%
\pgfpathlineto{\pgfqpoint{5.051280in}{0.601218in}}%
\pgfpathlineto{\pgfqpoint{5.053503in}{0.601607in}}%
\pgfpathlineto{\pgfqpoint{5.056282in}{0.600971in}}%
\pgfpathlineto{\pgfqpoint{5.057950in}{0.601907in}}%
\pgfpathlineto{\pgfqpoint{5.059062in}{0.600372in}}%
\pgfpathlineto{\pgfqpoint{5.060173in}{0.602869in}}%
\pgfpathlineto{\pgfqpoint{5.060729in}{0.601905in}}%
\pgfpathlineto{\pgfqpoint{5.062397in}{0.600599in}}%
\pgfpathlineto{\pgfqpoint{5.062953in}{0.601373in}}%
\pgfpathlineto{\pgfqpoint{5.063508in}{0.600250in}}%
\pgfpathlineto{\pgfqpoint{5.065732in}{0.600834in}}%
\pgfpathlineto{\pgfqpoint{5.066288in}{0.599926in}}%
\pgfpathlineto{\pgfqpoint{5.066844in}{0.600845in}}%
\pgfpathlineto{\pgfqpoint{5.102419in}{0.600461in}}%
\pgfpathlineto{\pgfqpoint{5.104642in}{0.601220in}}%
\pgfpathlineto{\pgfqpoint{5.105198in}{0.602189in}}%
\pgfpathlineto{\pgfqpoint{5.105754in}{0.600668in}}%
\pgfpathlineto{\pgfqpoint{5.106310in}{0.607310in}}%
\pgfpathlineto{\pgfqpoint{5.106866in}{0.602458in}}%
\pgfpathlineto{\pgfqpoint{5.107422in}{0.603547in}}%
\pgfpathlineto{\pgfqpoint{5.107977in}{0.607502in}}%
\pgfpathlineto{\pgfqpoint{5.108533in}{0.601067in}}%
\pgfpathlineto{\pgfqpoint{5.109089in}{0.602557in}}%
\pgfpathlineto{\pgfqpoint{5.110757in}{0.600820in}}%
\pgfpathlineto{\pgfqpoint{5.116871in}{0.600024in}}%
\pgfpathlineto{\pgfqpoint{5.119095in}{0.600478in}}%
\pgfpathlineto{\pgfqpoint{5.120762in}{0.600279in}}%
\pgfpathlineto{\pgfqpoint{5.126877in}{0.600283in}}%
\pgfpathlineto{\pgfqpoint{5.139661in}{0.600169in}}%
\pgfpathlineto{\pgfqpoint{5.142441in}{0.600628in}}%
\pgfpathlineto{\pgfqpoint{5.143553in}{0.601656in}}%
\pgfpathlineto{\pgfqpoint{5.144108in}{0.600037in}}%
\pgfpathlineto{\pgfqpoint{5.144664in}{0.600960in}}%
\pgfpathlineto{\pgfqpoint{5.153002in}{0.600543in}}%
\pgfpathlineto{\pgfqpoint{5.153558in}{0.602279in}}%
\pgfpathlineto{\pgfqpoint{5.154114in}{0.600500in}}%
\pgfpathlineto{\pgfqpoint{5.154670in}{0.601412in}}%
\pgfpathlineto{\pgfqpoint{5.155226in}{0.600504in}}%
\pgfpathlineto{\pgfqpoint{5.158005in}{0.601044in}}%
\pgfpathlineto{\pgfqpoint{5.159117in}{0.600968in}}%
\pgfpathlineto{\pgfqpoint{5.160784in}{0.601970in}}%
\pgfpathlineto{\pgfqpoint{5.161340in}{0.600659in}}%
\pgfpathlineto{\pgfqpoint{5.161896in}{0.601596in}}%
\pgfpathlineto{\pgfqpoint{5.165787in}{0.600269in}}%
\pgfpathlineto{\pgfqpoint{5.166343in}{0.602978in}}%
\pgfpathlineto{\pgfqpoint{5.166899in}{0.600610in}}%
\pgfpathlineto{\pgfqpoint{5.168010in}{0.600675in}}%
\pgfpathlineto{\pgfqpoint{5.168566in}{0.602685in}}%
\pgfpathlineto{\pgfqpoint{5.169122in}{0.601884in}}%
\pgfpathlineto{\pgfqpoint{5.170234in}{0.601158in}}%
\pgfpathlineto{\pgfqpoint{5.171901in}{0.602681in}}%
\pgfpathlineto{\pgfqpoint{5.172457in}{0.600994in}}%
\pgfpathlineto{\pgfqpoint{5.173013in}{0.601480in}}%
\pgfpathlineto{\pgfqpoint{5.174125in}{0.602712in}}%
\pgfpathlineto{\pgfqpoint{5.174681in}{0.601908in}}%
\pgfpathlineto{\pgfqpoint{5.175237in}{0.605385in}}%
\pgfpathlineto{\pgfqpoint{5.175792in}{0.600774in}}%
\pgfpathlineto{\pgfqpoint{5.176348in}{0.601633in}}%
\pgfpathlineto{\pgfqpoint{5.177460in}{0.601854in}}%
\pgfpathlineto{\pgfqpoint{5.178016in}{0.605263in}}%
\pgfpathlineto{\pgfqpoint{5.178572in}{0.602085in}}%
\pgfpathlineto{\pgfqpoint{5.180795in}{0.603352in}}%
\pgfpathlineto{\pgfqpoint{5.181907in}{0.600701in}}%
\pgfpathlineto{\pgfqpoint{5.183575in}{0.602670in}}%
\pgfpathlineto{\pgfqpoint{5.186910in}{0.602265in}}%
\pgfpathlineto{\pgfqpoint{5.187466in}{0.603833in}}%
\pgfpathlineto{\pgfqpoint{5.188021in}{0.600662in}}%
\pgfpathlineto{\pgfqpoint{5.188577in}{0.602817in}}%
\pgfpathlineto{\pgfqpoint{5.190801in}{0.600067in}}%
\pgfpathlineto{\pgfqpoint{5.191357in}{0.603601in}}%
\pgfpathlineto{\pgfqpoint{5.191912in}{0.602529in}}%
\pgfpathlineto{\pgfqpoint{5.193024in}{0.602932in}}%
\pgfpathlineto{\pgfqpoint{5.193580in}{0.603887in}}%
\pgfpathlineto{\pgfqpoint{5.195248in}{0.602403in}}%
\pgfpathlineto{\pgfqpoint{5.195803in}{0.604650in}}%
\pgfpathlineto{\pgfqpoint{5.196359in}{0.602596in}}%
\pgfpathlineto{\pgfqpoint{5.196915in}{0.602602in}}%
\pgfpathlineto{\pgfqpoint{5.197471in}{0.600813in}}%
\pgfpathlineto{\pgfqpoint{5.198027in}{0.603297in}}%
\pgfpathlineto{\pgfqpoint{5.198583in}{0.602151in}}%
\pgfpathlineto{\pgfqpoint{5.200250in}{0.602606in}}%
\pgfpathlineto{\pgfqpoint{5.200806in}{0.605352in}}%
\pgfpathlineto{\pgfqpoint{5.201362in}{0.604457in}}%
\pgfpathlineto{\pgfqpoint{5.202474in}{0.601585in}}%
\pgfpathlineto{\pgfqpoint{5.203030in}{0.603116in}}%
\pgfpathlineto{\pgfqpoint{5.204141in}{0.605015in}}%
\pgfpathlineto{\pgfqpoint{5.205253in}{0.601845in}}%
\pgfpathlineto{\pgfqpoint{5.206921in}{0.606819in}}%
\pgfpathlineto{\pgfqpoint{5.208588in}{0.600780in}}%
\pgfpathlineto{\pgfqpoint{5.209144in}{0.603548in}}%
\pgfpathlineto{\pgfqpoint{5.209700in}{0.603066in}}%
\pgfpathlineto{\pgfqpoint{5.210256in}{0.602892in}}%
\pgfpathlineto{\pgfqpoint{5.210812in}{0.604304in}}%
\pgfpathlineto{\pgfqpoint{5.211368in}{0.601671in}}%
\pgfpathlineto{\pgfqpoint{5.211923in}{0.603976in}}%
\pgfpathlineto{\pgfqpoint{5.212479in}{0.604508in}}%
\pgfpathlineto{\pgfqpoint{5.213035in}{0.607161in}}%
\pgfpathlineto{\pgfqpoint{5.213591in}{0.601702in}}%
\pgfpathlineto{\pgfqpoint{5.214147in}{0.604309in}}%
\pgfpathlineto{\pgfqpoint{5.215259in}{0.601389in}}%
\pgfpathlineto{\pgfqpoint{5.216370in}{0.605909in}}%
\pgfpathlineto{\pgfqpoint{5.217482in}{0.601224in}}%
\pgfpathlineto{\pgfqpoint{5.218038in}{0.604329in}}%
\pgfpathlineto{\pgfqpoint{5.218594in}{0.600567in}}%
\pgfpathlineto{\pgfqpoint{5.219150in}{0.603996in}}%
\pgfpathlineto{\pgfqpoint{5.220817in}{0.603409in}}%
\pgfpathlineto{\pgfqpoint{5.221929in}{0.601689in}}%
\pgfpathlineto{\pgfqpoint{5.222485in}{0.604198in}}%
\pgfpathlineto{\pgfqpoint{5.223041in}{0.603609in}}%
\pgfpathlineto{\pgfqpoint{5.223597in}{0.602764in}}%
\pgfpathlineto{\pgfqpoint{5.224152in}{0.606718in}}%
\pgfpathlineto{\pgfqpoint{5.224708in}{0.601935in}}%
\pgfpathlineto{\pgfqpoint{5.225264in}{0.603083in}}%
\pgfpathlineto{\pgfqpoint{5.225820in}{0.609631in}}%
\pgfpathlineto{\pgfqpoint{5.226376in}{0.603638in}}%
\pgfpathlineto{\pgfqpoint{5.227488in}{0.600680in}}%
\pgfpathlineto{\pgfqpoint{5.228043in}{0.605839in}}%
\pgfpathlineto{\pgfqpoint{5.228599in}{0.601423in}}%
\pgfpathlineto{\pgfqpoint{5.230267in}{0.605241in}}%
\pgfpathlineto{\pgfqpoint{5.230823in}{0.604019in}}%
\pgfpathlineto{\pgfqpoint{5.231379in}{0.606435in}}%
\pgfpathlineto{\pgfqpoint{5.231934in}{0.601594in}}%
\pgfpathlineto{\pgfqpoint{5.232490in}{0.609765in}}%
\pgfpathlineto{\pgfqpoint{5.233046in}{0.608153in}}%
\pgfpathlineto{\pgfqpoint{5.233602in}{0.600604in}}%
\pgfpathlineto{\pgfqpoint{5.234158in}{0.611689in}}%
\pgfpathlineto{\pgfqpoint{5.234714in}{0.604513in}}%
\pgfpathlineto{\pgfqpoint{5.235270in}{0.605101in}}%
\pgfpathlineto{\pgfqpoint{5.235826in}{0.601910in}}%
\pgfpathlineto{\pgfqpoint{5.236381in}{0.602898in}}%
\pgfpathlineto{\pgfqpoint{5.236937in}{0.606443in}}%
\pgfpathlineto{\pgfqpoint{5.237493in}{0.606077in}}%
\pgfpathlineto{\pgfqpoint{5.238049in}{0.602008in}}%
\pgfpathlineto{\pgfqpoint{5.239161in}{0.613276in}}%
\pgfpathlineto{\pgfqpoint{5.240828in}{0.602006in}}%
\pgfpathlineto{\pgfqpoint{5.241384in}{0.605075in}}%
\pgfpathlineto{\pgfqpoint{5.241940in}{0.602822in}}%
\pgfpathlineto{\pgfqpoint{5.243052in}{0.605178in}}%
\pgfpathlineto{\pgfqpoint{5.243608in}{0.604353in}}%
\pgfpathlineto{\pgfqpoint{5.244163in}{0.605314in}}%
\pgfpathlineto{\pgfqpoint{5.244719in}{0.606736in}}%
\pgfpathlineto{\pgfqpoint{5.245275in}{0.603997in}}%
\pgfpathlineto{\pgfqpoint{5.245831in}{0.606301in}}%
\pgfpathlineto{\pgfqpoint{5.247499in}{0.600114in}}%
\pgfpathlineto{\pgfqpoint{5.248610in}{0.604064in}}%
\pgfpathlineto{\pgfqpoint{5.249722in}{0.601406in}}%
\pgfpathlineto{\pgfqpoint{5.251390in}{0.609602in}}%
\pgfpathlineto{\pgfqpoint{5.252501in}{0.603272in}}%
\pgfpathlineto{\pgfqpoint{5.253057in}{0.608975in}}%
\pgfpathlineto{\pgfqpoint{5.253613in}{0.607114in}}%
\pgfpathlineto{\pgfqpoint{5.254169in}{0.603767in}}%
\pgfpathlineto{\pgfqpoint{5.254725in}{0.604728in}}%
\pgfpathlineto{\pgfqpoint{5.255281in}{0.604464in}}%
\pgfpathlineto{\pgfqpoint{5.255837in}{0.608679in}}%
\pgfpathlineto{\pgfqpoint{5.256392in}{0.604488in}}%
\pgfpathlineto{\pgfqpoint{5.256948in}{0.604390in}}%
\pgfpathlineto{\pgfqpoint{5.257504in}{0.602206in}}%
\pgfpathlineto{\pgfqpoint{5.258060in}{0.609280in}}%
\pgfpathlineto{\pgfqpoint{5.258616in}{0.607220in}}%
\pgfpathlineto{\pgfqpoint{5.259172in}{0.603856in}}%
\pgfpathlineto{\pgfqpoint{5.259728in}{0.607029in}}%
\pgfpathlineto{\pgfqpoint{5.260839in}{0.607111in}}%
\pgfpathlineto{\pgfqpoint{5.261395in}{0.601958in}}%
\pgfpathlineto{\pgfqpoint{5.261951in}{0.607075in}}%
\pgfpathlineto{\pgfqpoint{5.262507in}{0.607241in}}%
\pgfpathlineto{\pgfqpoint{5.263063in}{0.610715in}}%
\pgfpathlineto{\pgfqpoint{5.263619in}{0.609023in}}%
\pgfpathlineto{\pgfqpoint{5.264174in}{0.603800in}}%
\pgfpathlineto{\pgfqpoint{5.264730in}{0.610565in}}%
\pgfpathlineto{\pgfqpoint{5.265286in}{0.604238in}}%
\pgfpathlineto{\pgfqpoint{5.266398in}{0.604882in}}%
\pgfpathlineto{\pgfqpoint{5.266954in}{0.603804in}}%
\pgfpathlineto{\pgfqpoint{5.267510in}{0.606497in}}%
\pgfpathlineto{\pgfqpoint{5.268065in}{0.604425in}}%
\pgfpathlineto{\pgfqpoint{5.268621in}{0.604067in}}%
\pgfpathlineto{\pgfqpoint{5.269733in}{0.611367in}}%
\pgfpathlineto{\pgfqpoint{5.270289in}{0.608350in}}%
\pgfpathlineto{\pgfqpoint{5.271401in}{0.605520in}}%
\pgfpathlineto{\pgfqpoint{5.271956in}{0.606517in}}%
\pgfpathlineto{\pgfqpoint{5.272512in}{0.603675in}}%
\pgfpathlineto{\pgfqpoint{5.273068in}{0.605053in}}%
\pgfpathlineto{\pgfqpoint{5.274736in}{0.601861in}}%
\pgfpathlineto{\pgfqpoint{5.275292in}{0.609363in}}%
\pgfpathlineto{\pgfqpoint{5.275848in}{0.603904in}}%
\pgfpathlineto{\pgfqpoint{5.276959in}{0.609424in}}%
\pgfpathlineto{\pgfqpoint{5.278071in}{0.600813in}}%
\pgfpathlineto{\pgfqpoint{5.278627in}{0.603648in}}%
\pgfpathlineto{\pgfqpoint{5.279183in}{0.606138in}}%
\pgfpathlineto{\pgfqpoint{5.279739in}{0.603209in}}%
\pgfpathlineto{\pgfqpoint{5.280294in}{0.607920in}}%
\pgfpathlineto{\pgfqpoint{5.281406in}{0.602092in}}%
\pgfpathlineto{\pgfqpoint{5.281962in}{0.607163in}}%
\pgfpathlineto{\pgfqpoint{5.282518in}{0.603814in}}%
\pgfpathlineto{\pgfqpoint{5.283630in}{0.607306in}}%
\pgfpathlineto{\pgfqpoint{5.284185in}{0.603830in}}%
\pgfpathlineto{\pgfqpoint{5.284741in}{0.604622in}}%
\pgfpathlineto{\pgfqpoint{5.285297in}{0.610653in}}%
\pgfpathlineto{\pgfqpoint{5.285853in}{0.604301in}}%
\pgfpathlineto{\pgfqpoint{5.286409in}{0.607035in}}%
\pgfpathlineto{\pgfqpoint{5.286965in}{0.611283in}}%
\pgfpathlineto{\pgfqpoint{5.288076in}{0.604957in}}%
\pgfpathlineto{\pgfqpoint{5.289744in}{0.608161in}}%
\pgfpathlineto{\pgfqpoint{5.290300in}{0.607541in}}%
\pgfpathlineto{\pgfqpoint{5.290856in}{0.601422in}}%
\pgfpathlineto{\pgfqpoint{5.291968in}{0.613315in}}%
\pgfpathlineto{\pgfqpoint{5.293635in}{0.605740in}}%
\pgfpathlineto{\pgfqpoint{5.294191in}{0.605398in}}%
\pgfpathlineto{\pgfqpoint{5.294747in}{0.608494in}}%
\pgfpathlineto{\pgfqpoint{5.295303in}{0.603795in}}%
\pgfpathlineto{\pgfqpoint{5.295859in}{0.608729in}}%
\pgfpathlineto{\pgfqpoint{5.296414in}{0.601753in}}%
\pgfpathlineto{\pgfqpoint{5.296970in}{0.606421in}}%
\pgfpathlineto{\pgfqpoint{5.297526in}{0.605604in}}%
\pgfpathlineto{\pgfqpoint{5.299194in}{0.613155in}}%
\pgfpathlineto{\pgfqpoint{5.299750in}{0.605613in}}%
\pgfpathlineto{\pgfqpoint{5.300305in}{0.606407in}}%
\pgfpathlineto{\pgfqpoint{5.300861in}{0.606519in}}%
\pgfpathlineto{\pgfqpoint{5.301417in}{0.602749in}}%
\pgfpathlineto{\pgfqpoint{5.301973in}{0.603742in}}%
\pgfpathlineto{\pgfqpoint{5.302529in}{0.609656in}}%
\pgfpathlineto{\pgfqpoint{5.303085in}{0.601719in}}%
\pgfpathlineto{\pgfqpoint{5.303641in}{0.610407in}}%
\pgfpathlineto{\pgfqpoint{5.304196in}{0.606750in}}%
\pgfpathlineto{\pgfqpoint{5.305308in}{0.601064in}}%
\pgfpathlineto{\pgfqpoint{5.305864in}{0.602814in}}%
\pgfpathlineto{\pgfqpoint{5.306420in}{0.601037in}}%
\pgfpathlineto{\pgfqpoint{5.306976in}{0.601134in}}%
\pgfpathlineto{\pgfqpoint{5.308087in}{0.611373in}}%
\pgfpathlineto{\pgfqpoint{5.308643in}{0.607344in}}%
\pgfpathlineto{\pgfqpoint{5.309755in}{0.602905in}}%
\pgfpathlineto{\pgfqpoint{5.310311in}{0.610641in}}%
\pgfpathlineto{\pgfqpoint{5.310867in}{0.607721in}}%
\pgfpathlineto{\pgfqpoint{5.311423in}{0.609779in}}%
\pgfpathlineto{\pgfqpoint{5.313646in}{0.603124in}}%
\pgfpathlineto{\pgfqpoint{5.314758in}{0.603566in}}%
\pgfpathlineto{\pgfqpoint{5.315314in}{0.605324in}}%
\pgfpathlineto{\pgfqpoint{5.315870in}{0.612615in}}%
\pgfpathlineto{\pgfqpoint{5.316425in}{0.602247in}}%
\pgfpathlineto{\pgfqpoint{5.316981in}{0.604640in}}%
\pgfpathlineto{\pgfqpoint{5.318649in}{0.606550in}}%
\pgfpathlineto{\pgfqpoint{5.319761in}{0.613138in}}%
\pgfpathlineto{\pgfqpoint{5.320316in}{0.603501in}}%
\pgfpathlineto{\pgfqpoint{5.320872in}{0.611705in}}%
\pgfpathlineto{\pgfqpoint{5.323096in}{0.602678in}}%
\pgfpathlineto{\pgfqpoint{5.323652in}{0.603334in}}%
\pgfpathlineto{\pgfqpoint{5.324207in}{0.606477in}}%
\pgfpathlineto{\pgfqpoint{5.324763in}{0.602139in}}%
\pgfpathlineto{\pgfqpoint{5.325319in}{0.603621in}}%
\pgfpathlineto{\pgfqpoint{5.325875in}{0.605191in}}%
\pgfpathlineto{\pgfqpoint{5.326431in}{0.602894in}}%
\pgfpathlineto{\pgfqpoint{5.326987in}{0.606120in}}%
\pgfpathlineto{\pgfqpoint{5.327543in}{0.600837in}}%
\pgfpathlineto{\pgfqpoint{5.329210in}{0.612033in}}%
\pgfpathlineto{\pgfqpoint{5.330322in}{0.603343in}}%
\pgfpathlineto{\pgfqpoint{5.330878in}{0.604455in}}%
\pgfpathlineto{\pgfqpoint{5.331434in}{0.608977in}}%
\pgfpathlineto{\pgfqpoint{5.331990in}{0.601528in}}%
\pgfpathlineto{\pgfqpoint{5.332545in}{0.611165in}}%
\pgfpathlineto{\pgfqpoint{5.333101in}{0.605496in}}%
\pgfpathlineto{\pgfqpoint{5.333657in}{0.603422in}}%
\pgfpathlineto{\pgfqpoint{5.334213in}{0.605366in}}%
\pgfpathlineto{\pgfqpoint{5.334769in}{0.609461in}}%
\pgfpathlineto{\pgfqpoint{5.335881in}{0.603719in}}%
\pgfpathlineto{\pgfqpoint{5.336436in}{0.617607in}}%
\pgfpathlineto{\pgfqpoint{5.336992in}{0.607111in}}%
\pgfpathlineto{\pgfqpoint{5.339216in}{0.603453in}}%
\pgfpathlineto{\pgfqpoint{5.339772in}{0.605842in}}%
\pgfpathlineto{\pgfqpoint{5.340327in}{0.607106in}}%
\pgfpathlineto{\pgfqpoint{5.340883in}{0.603798in}}%
\pgfpathlineto{\pgfqpoint{5.341439in}{0.607235in}}%
\pgfpathlineto{\pgfqpoint{5.341995in}{0.604898in}}%
\pgfpathlineto{\pgfqpoint{5.342551in}{0.603954in}}%
\pgfpathlineto{\pgfqpoint{5.343107in}{0.608849in}}%
\pgfpathlineto{\pgfqpoint{5.343663in}{0.605407in}}%
\pgfpathlineto{\pgfqpoint{5.344218in}{0.608149in}}%
\pgfpathlineto{\pgfqpoint{5.344774in}{0.605812in}}%
\pgfpathlineto{\pgfqpoint{5.345886in}{0.603766in}}%
\pgfpathlineto{\pgfqpoint{5.347554in}{0.606967in}}%
\pgfpathlineto{\pgfqpoint{5.348109in}{0.601024in}}%
\pgfpathlineto{\pgfqpoint{5.348665in}{0.606997in}}%
\pgfpathlineto{\pgfqpoint{5.349221in}{0.610199in}}%
\pgfpathlineto{\pgfqpoint{5.349777in}{0.602080in}}%
\pgfpathlineto{\pgfqpoint{5.350333in}{0.609925in}}%
\pgfpathlineto{\pgfqpoint{5.350889in}{0.606373in}}%
\pgfpathlineto{\pgfqpoint{5.351445in}{0.612809in}}%
\pgfpathlineto{\pgfqpoint{5.352001in}{0.604505in}}%
\pgfpathlineto{\pgfqpoint{5.352556in}{0.608069in}}%
\pgfpathlineto{\pgfqpoint{5.353112in}{0.608326in}}%
\pgfpathlineto{\pgfqpoint{5.354780in}{0.602472in}}%
\pgfpathlineto{\pgfqpoint{5.356447in}{0.605699in}}%
\pgfpathlineto{\pgfqpoint{5.357003in}{0.602960in}}%
\pgfpathlineto{\pgfqpoint{5.358671in}{0.609886in}}%
\pgfpathlineto{\pgfqpoint{5.359783in}{0.609725in}}%
\pgfpathlineto{\pgfqpoint{5.360894in}{0.603464in}}%
\pgfpathlineto{\pgfqpoint{5.362006in}{0.610773in}}%
\pgfpathlineto{\pgfqpoint{5.362562in}{0.605804in}}%
\pgfpathlineto{\pgfqpoint{5.363118in}{0.608090in}}%
\pgfpathlineto{\pgfqpoint{5.363674in}{0.606537in}}%
\pgfpathlineto{\pgfqpoint{5.364229in}{0.601355in}}%
\pgfpathlineto{\pgfqpoint{5.364785in}{0.602321in}}%
\pgfpathlineto{\pgfqpoint{5.365897in}{0.610143in}}%
\pgfpathlineto{\pgfqpoint{5.366453in}{0.608053in}}%
\pgfpathlineto{\pgfqpoint{5.367565in}{0.602018in}}%
\pgfpathlineto{\pgfqpoint{5.368121in}{0.609489in}}%
\pgfpathlineto{\pgfqpoint{5.368676in}{0.605591in}}%
\pgfpathlineto{\pgfqpoint{5.369232in}{0.606511in}}%
\pgfpathlineto{\pgfqpoint{5.369788in}{0.601758in}}%
\pgfpathlineto{\pgfqpoint{5.370344in}{0.603432in}}%
\pgfpathlineto{\pgfqpoint{5.371456in}{0.603203in}}%
\pgfpathlineto{\pgfqpoint{5.373123in}{0.607484in}}%
\pgfpathlineto{\pgfqpoint{5.373679in}{0.606945in}}%
\pgfpathlineto{\pgfqpoint{5.374791in}{0.604521in}}%
\pgfpathlineto{\pgfqpoint{5.377014in}{0.610532in}}%
\pgfpathlineto{\pgfqpoint{5.377570in}{0.609645in}}%
\pgfpathlineto{\pgfqpoint{5.378126in}{0.610133in}}%
\pgfpathlineto{\pgfqpoint{5.379238in}{0.601811in}}%
\pgfpathlineto{\pgfqpoint{5.379794in}{0.610803in}}%
\pgfpathlineto{\pgfqpoint{5.380349in}{0.606351in}}%
\pgfpathlineto{\pgfqpoint{5.380905in}{0.607288in}}%
\pgfpathlineto{\pgfqpoint{5.382017in}{0.603172in}}%
\pgfpathlineto{\pgfqpoint{5.382573in}{0.604467in}}%
\pgfpathlineto{\pgfqpoint{5.383129in}{0.606878in}}%
\pgfpathlineto{\pgfqpoint{5.383685in}{0.604854in}}%
\pgfpathlineto{\pgfqpoint{5.384240in}{0.600819in}}%
\pgfpathlineto{\pgfqpoint{5.384796in}{0.601384in}}%
\pgfpathlineto{\pgfqpoint{5.385908in}{0.609441in}}%
\pgfpathlineto{\pgfqpoint{5.387020in}{0.601216in}}%
\pgfpathlineto{\pgfqpoint{5.387576in}{0.602803in}}%
\pgfpathlineto{\pgfqpoint{5.388132in}{0.606862in}}%
\pgfpathlineto{\pgfqpoint{5.388687in}{0.602836in}}%
\pgfpathlineto{\pgfqpoint{5.389799in}{0.604998in}}%
\pgfpathlineto{\pgfqpoint{5.390355in}{0.606399in}}%
\pgfpathlineto{\pgfqpoint{5.390911in}{0.601538in}}%
\pgfpathlineto{\pgfqpoint{5.391467in}{0.604718in}}%
\pgfpathlineto{\pgfqpoint{5.392578in}{0.608350in}}%
\pgfpathlineto{\pgfqpoint{5.393134in}{0.604673in}}%
\pgfpathlineto{\pgfqpoint{5.393690in}{0.610276in}}%
\pgfpathlineto{\pgfqpoint{5.394246in}{0.605769in}}%
\pgfpathlineto{\pgfqpoint{5.394802in}{0.600995in}}%
\pgfpathlineto{\pgfqpoint{5.395358in}{0.602887in}}%
\pgfpathlineto{\pgfqpoint{5.396469in}{0.606383in}}%
\pgfpathlineto{\pgfqpoint{5.397581in}{0.600934in}}%
\pgfpathlineto{\pgfqpoint{5.398693in}{0.606819in}}%
\pgfpathlineto{\pgfqpoint{5.400916in}{0.603114in}}%
\pgfpathlineto{\pgfqpoint{5.402584in}{0.606997in}}%
\pgfpathlineto{\pgfqpoint{5.403140in}{0.604956in}}%
\pgfpathlineto{\pgfqpoint{5.403696in}{0.609771in}}%
\pgfpathlineto{\pgfqpoint{5.404251in}{0.604696in}}%
\pgfpathlineto{\pgfqpoint{5.404807in}{0.606939in}}%
\pgfpathlineto{\pgfqpoint{5.405363in}{0.605103in}}%
\pgfpathlineto{\pgfqpoint{5.405919in}{0.605349in}}%
\pgfpathlineto{\pgfqpoint{5.406475in}{0.611223in}}%
\pgfpathlineto{\pgfqpoint{5.407031in}{0.602068in}}%
\pgfpathlineto{\pgfqpoint{5.407587in}{0.608419in}}%
\pgfpathlineto{\pgfqpoint{5.408143in}{0.602775in}}%
\pgfpathlineto{\pgfqpoint{5.408698in}{0.608098in}}%
\pgfpathlineto{\pgfqpoint{5.409254in}{0.605835in}}%
\pgfpathlineto{\pgfqpoint{5.409810in}{0.610100in}}%
\pgfpathlineto{\pgfqpoint{5.410922in}{0.604171in}}%
\pgfpathlineto{\pgfqpoint{5.411478in}{0.607246in}}%
\pgfpathlineto{\pgfqpoint{5.413145in}{0.601998in}}%
\pgfpathlineto{\pgfqpoint{5.414257in}{0.603879in}}%
\pgfpathlineto{\pgfqpoint{5.415369in}{0.602869in}}%
\pgfpathlineto{\pgfqpoint{5.415925in}{0.607093in}}%
\pgfpathlineto{\pgfqpoint{5.416480in}{0.601604in}}%
\pgfpathlineto{\pgfqpoint{5.417036in}{0.605736in}}%
\pgfpathlineto{\pgfqpoint{5.417592in}{0.606029in}}%
\pgfpathlineto{\pgfqpoint{5.418148in}{0.603535in}}%
\pgfpathlineto{\pgfqpoint{5.418704in}{0.609085in}}%
\pgfpathlineto{\pgfqpoint{5.419260in}{0.606505in}}%
\pgfpathlineto{\pgfqpoint{5.419816in}{0.606644in}}%
\pgfpathlineto{\pgfqpoint{5.420371in}{0.602351in}}%
\pgfpathlineto{\pgfqpoint{5.420927in}{0.603178in}}%
\pgfpathlineto{\pgfqpoint{5.422595in}{0.610793in}}%
\pgfpathlineto{\pgfqpoint{5.424818in}{0.603531in}}%
\pgfpathlineto{\pgfqpoint{5.426486in}{0.606987in}}%
\pgfpathlineto{\pgfqpoint{5.427042in}{0.601753in}}%
\pgfpathlineto{\pgfqpoint{5.427598in}{0.603574in}}%
\pgfpathlineto{\pgfqpoint{5.428154in}{0.605738in}}%
\pgfpathlineto{\pgfqpoint{5.429821in}{0.600746in}}%
\pgfpathlineto{\pgfqpoint{5.430933in}{0.607615in}}%
\pgfpathlineto{\pgfqpoint{5.432045in}{0.602061in}}%
\pgfpathlineto{\pgfqpoint{5.433712in}{0.608502in}}%
\pgfpathlineto{\pgfqpoint{5.434268in}{0.606000in}}%
\pgfpathlineto{\pgfqpoint{5.434824in}{0.609580in}}%
\pgfpathlineto{\pgfqpoint{5.436491in}{0.601406in}}%
\pgfpathlineto{\pgfqpoint{5.437603in}{0.605328in}}%
\pgfpathlineto{\pgfqpoint{5.438159in}{0.603167in}}%
\pgfpathlineto{\pgfqpoint{5.439271in}{0.605884in}}%
\pgfpathlineto{\pgfqpoint{5.439827in}{0.600747in}}%
\pgfpathlineto{\pgfqpoint{5.440382in}{0.602810in}}%
\pgfpathlineto{\pgfqpoint{5.440938in}{0.605900in}}%
\pgfpathlineto{\pgfqpoint{5.441494in}{0.603421in}}%
\pgfpathlineto{\pgfqpoint{5.442050in}{0.600740in}}%
\pgfpathlineto{\pgfqpoint{5.442606in}{0.606252in}}%
\pgfpathlineto{\pgfqpoint{5.443162in}{0.605610in}}%
\pgfpathlineto{\pgfqpoint{5.444829in}{0.601631in}}%
\pgfpathlineto{\pgfqpoint{5.445385in}{0.611531in}}%
\pgfpathlineto{\pgfqpoint{5.445941in}{0.606356in}}%
\pgfpathlineto{\pgfqpoint{5.447053in}{0.601523in}}%
\pgfpathlineto{\pgfqpoint{5.447609in}{0.605608in}}%
\pgfpathlineto{\pgfqpoint{5.448165in}{0.603879in}}%
\pgfpathlineto{\pgfqpoint{5.448720in}{0.605249in}}%
\pgfpathlineto{\pgfqpoint{5.449276in}{0.603611in}}%
\pgfpathlineto{\pgfqpoint{5.449832in}{0.605433in}}%
\pgfpathlineto{\pgfqpoint{5.450388in}{0.600671in}}%
\pgfpathlineto{\pgfqpoint{5.450944in}{0.604482in}}%
\pgfpathlineto{\pgfqpoint{5.452056in}{0.603233in}}%
\pgfpathlineto{\pgfqpoint{5.452611in}{0.607961in}}%
\pgfpathlineto{\pgfqpoint{5.453167in}{0.606352in}}%
\pgfpathlineto{\pgfqpoint{5.454835in}{0.603341in}}%
\pgfpathlineto{\pgfqpoint{5.455391in}{0.602135in}}%
\pgfpathlineto{\pgfqpoint{5.455947in}{0.607111in}}%
\pgfpathlineto{\pgfqpoint{5.456502in}{0.602885in}}%
\pgfpathlineto{\pgfqpoint{5.458726in}{0.605744in}}%
\pgfpathlineto{\pgfqpoint{5.459282in}{0.607133in}}%
\pgfpathlineto{\pgfqpoint{5.460393in}{0.603076in}}%
\pgfpathlineto{\pgfqpoint{5.460949in}{0.607674in}}%
\pgfpathlineto{\pgfqpoint{5.462061in}{0.601763in}}%
\pgfpathlineto{\pgfqpoint{5.462617in}{0.604858in}}%
\pgfpathlineto{\pgfqpoint{5.463173in}{0.600723in}}%
\pgfpathlineto{\pgfqpoint{5.463729in}{0.609539in}}%
\pgfpathlineto{\pgfqpoint{5.464285in}{0.606583in}}%
\pgfpathlineto{\pgfqpoint{5.465396in}{0.602350in}}%
\pgfpathlineto{\pgfqpoint{5.465952in}{0.607588in}}%
\pgfpathlineto{\pgfqpoint{5.466508in}{0.607355in}}%
\pgfpathlineto{\pgfqpoint{5.468176in}{0.601205in}}%
\pgfpathlineto{\pgfqpoint{5.468731in}{0.605233in}}%
\pgfpathlineto{\pgfqpoint{5.469287in}{0.600206in}}%
\pgfpathlineto{\pgfqpoint{5.470955in}{0.606999in}}%
\pgfpathlineto{\pgfqpoint{5.472622in}{0.602366in}}%
\pgfpathlineto{\pgfqpoint{5.473178in}{0.604767in}}%
\pgfpathlineto{\pgfqpoint{5.473734in}{0.601855in}}%
\pgfpathlineto{\pgfqpoint{5.474290in}{0.604714in}}%
\pgfpathlineto{\pgfqpoint{5.474846in}{0.608436in}}%
\pgfpathlineto{\pgfqpoint{5.475402in}{0.602398in}}%
\pgfpathlineto{\pgfqpoint{5.475958in}{0.604514in}}%
\pgfpathlineto{\pgfqpoint{5.476513in}{0.604404in}}%
\pgfpathlineto{\pgfqpoint{5.477069in}{0.608129in}}%
\pgfpathlineto{\pgfqpoint{5.477625in}{0.605444in}}%
\pgfpathlineto{\pgfqpoint{5.479293in}{0.601896in}}%
\pgfpathlineto{\pgfqpoint{5.480960in}{0.611289in}}%
\pgfpathlineto{\pgfqpoint{5.482072in}{0.602287in}}%
\pgfpathlineto{\pgfqpoint{5.482628in}{0.602929in}}%
\pgfpathlineto{\pgfqpoint{5.483184in}{0.604655in}}%
\pgfpathlineto{\pgfqpoint{5.483740in}{0.600786in}}%
\pgfpathlineto{\pgfqpoint{5.484296in}{0.603362in}}%
\pgfpathlineto{\pgfqpoint{5.484851in}{0.601616in}}%
\pgfpathlineto{\pgfqpoint{5.485963in}{0.604760in}}%
\pgfpathlineto{\pgfqpoint{5.486519in}{0.601109in}}%
\pgfpathlineto{\pgfqpoint{5.487075in}{0.603359in}}%
\pgfpathlineto{\pgfqpoint{5.487631in}{0.605670in}}%
\pgfpathlineto{\pgfqpoint{5.488187in}{0.603377in}}%
\pgfpathlineto{\pgfqpoint{5.489854in}{0.601993in}}%
\pgfpathlineto{\pgfqpoint{5.492078in}{0.609016in}}%
\pgfpathlineto{\pgfqpoint{5.493745in}{0.600948in}}%
\pgfpathlineto{\pgfqpoint{5.494301in}{0.602157in}}%
\pgfpathlineto{\pgfqpoint{5.494857in}{0.604280in}}%
\pgfpathlineto{\pgfqpoint{5.495413in}{0.601997in}}%
\pgfpathlineto{\pgfqpoint{5.495969in}{0.607205in}}%
\pgfpathlineto{\pgfqpoint{5.496524in}{0.603216in}}%
\pgfpathlineto{\pgfqpoint{5.497080in}{0.603662in}}%
\pgfpathlineto{\pgfqpoint{5.497636in}{0.601310in}}%
\pgfpathlineto{\pgfqpoint{5.498192in}{0.602717in}}%
\pgfpathlineto{\pgfqpoint{5.498748in}{0.603277in}}%
\pgfpathlineto{\pgfqpoint{5.499304in}{0.600864in}}%
\pgfpathlineto{\pgfqpoint{5.499860in}{0.605379in}}%
\pgfpathlineto{\pgfqpoint{5.500416in}{0.602250in}}%
\pgfpathlineto{\pgfqpoint{5.502083in}{0.606771in}}%
\pgfpathlineto{\pgfqpoint{5.502639in}{0.605945in}}%
\pgfpathlineto{\pgfqpoint{5.503751in}{0.600516in}}%
\pgfpathlineto{\pgfqpoint{5.504862in}{0.602021in}}%
\pgfpathlineto{\pgfqpoint{5.505418in}{0.605371in}}%
\pgfpathlineto{\pgfqpoint{5.505974in}{0.604234in}}%
\pgfpathlineto{\pgfqpoint{5.506530in}{0.601178in}}%
\pgfpathlineto{\pgfqpoint{5.507086in}{0.602345in}}%
\pgfpathlineto{\pgfqpoint{5.507642in}{0.602365in}}%
\pgfpathlineto{\pgfqpoint{5.508198in}{0.600269in}}%
\pgfpathlineto{\pgfqpoint{5.509309in}{0.606189in}}%
\pgfpathlineto{\pgfqpoint{5.512089in}{0.601602in}}%
\pgfpathlineto{\pgfqpoint{5.512644in}{0.601798in}}%
\pgfpathlineto{\pgfqpoint{5.513200in}{0.606535in}}%
\pgfpathlineto{\pgfqpoint{5.513756in}{0.602739in}}%
\pgfpathlineto{\pgfqpoint{5.514312in}{0.600890in}}%
\pgfpathlineto{\pgfqpoint{5.515980in}{0.605365in}}%
\pgfpathlineto{\pgfqpoint{5.517091in}{0.601855in}}%
\pgfpathlineto{\pgfqpoint{5.518759in}{0.605119in}}%
\pgfpathlineto{\pgfqpoint{5.519315in}{0.603925in}}%
\pgfpathlineto{\pgfqpoint{5.519871in}{0.607764in}}%
\pgfpathlineto{\pgfqpoint{5.520427in}{0.604383in}}%
\pgfpathlineto{\pgfqpoint{5.520982in}{0.601105in}}%
\pgfpathlineto{\pgfqpoint{5.521538in}{0.606904in}}%
\pgfpathlineto{\pgfqpoint{5.522094in}{0.603649in}}%
\pgfpathlineto{\pgfqpoint{5.524318in}{0.608477in}}%
\pgfpathlineto{\pgfqpoint{5.524873in}{0.604276in}}%
\pgfpathlineto{\pgfqpoint{5.525429in}{0.604618in}}%
\pgfpathlineto{\pgfqpoint{5.525985in}{0.604993in}}%
\pgfpathlineto{\pgfqpoint{5.527653in}{0.602472in}}%
\pgfpathlineto{\pgfqpoint{5.528209in}{0.604090in}}%
\pgfpathlineto{\pgfqpoint{5.529320in}{0.601631in}}%
\pgfpathlineto{\pgfqpoint{5.529876in}{0.604805in}}%
\pgfpathlineto{\pgfqpoint{5.530432in}{0.600424in}}%
\pgfpathlineto{\pgfqpoint{5.530988in}{0.602341in}}%
\pgfpathlineto{\pgfqpoint{5.534323in}{0.603830in}}%
\pgfpathlineto{\pgfqpoint{5.536546in}{0.602170in}}%
\pgfpathlineto{\pgfqpoint{5.538214in}{0.606001in}}%
\pgfpathlineto{\pgfqpoint{5.538770in}{0.604693in}}%
\pgfpathlineto{\pgfqpoint{5.539882in}{0.600515in}}%
\pgfpathlineto{\pgfqpoint{5.541549in}{0.607707in}}%
\pgfpathlineto{\pgfqpoint{5.542105in}{0.602121in}}%
\pgfpathlineto{\pgfqpoint{5.542661in}{0.603233in}}%
\pgfpathlineto{\pgfqpoint{5.543217in}{0.603470in}}%
\pgfpathlineto{\pgfqpoint{5.543773in}{0.600683in}}%
\pgfpathlineto{\pgfqpoint{5.544329in}{0.603195in}}%
\pgfpathlineto{\pgfqpoint{5.545440in}{0.604215in}}%
\pgfpathlineto{\pgfqpoint{5.546552in}{0.602255in}}%
\pgfpathlineto{\pgfqpoint{5.547108in}{0.602978in}}%
\pgfpathlineto{\pgfqpoint{5.547664in}{0.604017in}}%
\pgfpathlineto{\pgfqpoint{5.548220in}{0.608167in}}%
\pgfpathlineto{\pgfqpoint{5.548775in}{0.605524in}}%
\pgfpathlineto{\pgfqpoint{5.549331in}{0.603796in}}%
\pgfpathlineto{\pgfqpoint{5.549887in}{0.607571in}}%
\pgfpathlineto{\pgfqpoint{5.551555in}{0.601235in}}%
\pgfpathlineto{\pgfqpoint{5.552111in}{0.603039in}}%
\pgfpathlineto{\pgfqpoint{5.552666in}{0.601885in}}%
\pgfpathlineto{\pgfqpoint{5.553222in}{0.601142in}}%
\pgfpathlineto{\pgfqpoint{5.553778in}{0.604955in}}%
\pgfpathlineto{\pgfqpoint{5.554334in}{0.600507in}}%
\pgfpathlineto{\pgfqpoint{5.554890in}{0.600984in}}%
\pgfpathlineto{\pgfqpoint{5.555446in}{0.604041in}}%
\pgfpathlineto{\pgfqpoint{5.556002in}{0.603305in}}%
\pgfpathlineto{\pgfqpoint{5.557113in}{0.601646in}}%
\pgfpathlineto{\pgfqpoint{5.557669in}{0.603950in}}%
\pgfpathlineto{\pgfqpoint{5.559337in}{0.600792in}}%
\pgfpathlineto{\pgfqpoint{5.560449in}{0.604910in}}%
\pgfpathlineto{\pgfqpoint{5.561004in}{0.601550in}}%
\pgfpathlineto{\pgfqpoint{5.561560in}{0.602142in}}%
\pgfpathlineto{\pgfqpoint{5.563228in}{0.602252in}}%
\pgfpathlineto{\pgfqpoint{5.564895in}{0.603516in}}%
\pgfpathlineto{\pgfqpoint{5.565451in}{0.600624in}}%
\pgfpathlineto{\pgfqpoint{5.566007in}{0.607005in}}%
\pgfpathlineto{\pgfqpoint{5.566563in}{0.605421in}}%
\pgfpathlineto{\pgfqpoint{5.567675in}{0.601362in}}%
\pgfpathlineto{\pgfqpoint{5.569342in}{0.604537in}}%
\pgfpathlineto{\pgfqpoint{5.570454in}{0.600445in}}%
\pgfpathlineto{\pgfqpoint{5.572677in}{0.608593in}}%
\pgfpathlineto{\pgfqpoint{5.573789in}{0.600915in}}%
\pgfpathlineto{\pgfqpoint{5.574901in}{0.603752in}}%
\pgfpathlineto{\pgfqpoint{5.575457in}{0.602054in}}%
\pgfpathlineto{\pgfqpoint{5.576013in}{0.604163in}}%
\pgfpathlineto{\pgfqpoint{5.577680in}{0.600906in}}%
\pgfpathlineto{\pgfqpoint{5.578236in}{0.601089in}}%
\pgfpathlineto{\pgfqpoint{5.578792in}{0.603909in}}%
\pgfpathlineto{\pgfqpoint{5.579348in}{0.601006in}}%
\pgfpathlineto{\pgfqpoint{5.579904in}{0.600078in}}%
\pgfpathlineto{\pgfqpoint{5.581015in}{0.605630in}}%
\pgfpathlineto{\pgfqpoint{5.581571in}{0.605317in}}%
\pgfpathlineto{\pgfqpoint{5.582127in}{0.604565in}}%
\pgfpathlineto{\pgfqpoint{5.582683in}{0.601669in}}%
\pgfpathlineto{\pgfqpoint{5.583239in}{0.603617in}}%
\pgfpathlineto{\pgfqpoint{5.584351in}{0.600804in}}%
\pgfpathlineto{\pgfqpoint{5.585462in}{0.603061in}}%
\pgfpathlineto{\pgfqpoint{5.586018in}{0.601914in}}%
\pgfpathlineto{\pgfqpoint{5.586574in}{0.604636in}}%
\pgfpathlineto{\pgfqpoint{5.587130in}{0.604512in}}%
\pgfpathlineto{\pgfqpoint{5.587686in}{0.601190in}}%
\pgfpathlineto{\pgfqpoint{5.588242in}{0.602041in}}%
\pgfpathlineto{\pgfqpoint{5.589909in}{0.602994in}}%
\pgfpathlineto{\pgfqpoint{5.590465in}{0.601102in}}%
\pgfpathlineto{\pgfqpoint{5.591021in}{0.603836in}}%
\pgfpathlineto{\pgfqpoint{5.591577in}{0.602954in}}%
\pgfpathlineto{\pgfqpoint{5.592133in}{0.600716in}}%
\pgfpathlineto{\pgfqpoint{5.592688in}{0.601919in}}%
\pgfpathlineto{\pgfqpoint{5.593244in}{0.604788in}}%
\pgfpathlineto{\pgfqpoint{5.593800in}{0.602967in}}%
\pgfpathlineto{\pgfqpoint{5.594356in}{0.602299in}}%
\pgfpathlineto{\pgfqpoint{5.596024in}{0.606524in}}%
\pgfpathlineto{\pgfqpoint{5.597691in}{0.602337in}}%
\pgfpathlineto{\pgfqpoint{5.598803in}{0.600623in}}%
\pgfpathlineto{\pgfqpoint{5.600471in}{0.604842in}}%
\pgfpathlineto{\pgfqpoint{5.601026in}{0.605025in}}%
\pgfpathlineto{\pgfqpoint{5.601582in}{0.602097in}}%
\pgfpathlineto{\pgfqpoint{5.602138in}{0.604730in}}%
\pgfpathlineto{\pgfqpoint{5.602694in}{0.605858in}}%
\pgfpathlineto{\pgfqpoint{5.603806in}{0.601744in}}%
\pgfpathlineto{\pgfqpoint{5.604917in}{0.602556in}}%
\pgfpathlineto{\pgfqpoint{5.606029in}{0.603169in}}%
\pgfpathlineto{\pgfqpoint{5.607141in}{0.608541in}}%
\pgfpathlineto{\pgfqpoint{5.607697in}{0.600605in}}%
\pgfpathlineto{\pgfqpoint{5.608253in}{0.601500in}}%
\pgfpathlineto{\pgfqpoint{5.609920in}{0.601070in}}%
\pgfpathlineto{\pgfqpoint{5.611032in}{0.605031in}}%
\pgfpathlineto{\pgfqpoint{5.611588in}{0.601139in}}%
\pgfpathlineto{\pgfqpoint{5.612144in}{0.602093in}}%
\pgfpathlineto{\pgfqpoint{5.612700in}{0.602318in}}%
\pgfpathlineto{\pgfqpoint{5.614367in}{0.600672in}}%
\pgfpathlineto{\pgfqpoint{5.616035in}{0.603715in}}%
\pgfpathlineto{\pgfqpoint{5.617146in}{0.602227in}}%
\pgfpathlineto{\pgfqpoint{5.617702in}{0.605840in}}%
\pgfpathlineto{\pgfqpoint{5.618258in}{0.602936in}}%
\pgfpathlineto{\pgfqpoint{5.619370in}{0.600220in}}%
\pgfpathlineto{\pgfqpoint{5.619926in}{0.602972in}}%
\pgfpathlineto{\pgfqpoint{5.620482in}{0.601878in}}%
\pgfpathlineto{\pgfqpoint{5.621593in}{0.600862in}}%
\pgfpathlineto{\pgfqpoint{5.622149in}{0.604503in}}%
\pgfpathlineto{\pgfqpoint{5.622705in}{0.602854in}}%
\pgfpathlineto{\pgfqpoint{5.623261in}{0.604271in}}%
\pgfpathlineto{\pgfqpoint{5.623817in}{0.602960in}}%
\pgfpathlineto{\pgfqpoint{5.624373in}{0.603496in}}%
\pgfpathlineto{\pgfqpoint{5.624928in}{0.601938in}}%
\pgfpathlineto{\pgfqpoint{5.626596in}{0.604117in}}%
\pgfpathlineto{\pgfqpoint{5.628819in}{0.601349in}}%
\pgfpathlineto{\pgfqpoint{5.629375in}{0.604372in}}%
\pgfpathlineto{\pgfqpoint{5.629931in}{0.600319in}}%
\pgfpathlineto{\pgfqpoint{5.630487in}{0.600443in}}%
\pgfpathlineto{\pgfqpoint{5.631043in}{0.601457in}}%
\pgfpathlineto{\pgfqpoint{5.631599in}{0.605710in}}%
\pgfpathlineto{\pgfqpoint{5.632155in}{0.603359in}}%
\pgfpathlineto{\pgfqpoint{5.632711in}{0.601234in}}%
\pgfpathlineto{\pgfqpoint{5.633266in}{0.602816in}}%
\pgfpathlineto{\pgfqpoint{5.633822in}{0.603598in}}%
\pgfpathlineto{\pgfqpoint{5.634378in}{0.600227in}}%
\pgfpathlineto{\pgfqpoint{5.634934in}{0.602474in}}%
\pgfpathlineto{\pgfqpoint{5.635490in}{0.601053in}}%
\pgfpathlineto{\pgfqpoint{5.636046in}{0.604588in}}%
\pgfpathlineto{\pgfqpoint{5.636602in}{0.600301in}}%
\pgfpathlineto{\pgfqpoint{5.637157in}{0.604237in}}%
\pgfpathlineto{\pgfqpoint{5.637713in}{0.602705in}}%
\pgfpathlineto{\pgfqpoint{5.638269in}{0.604195in}}%
\pgfpathlineto{\pgfqpoint{5.638825in}{0.603487in}}%
\pgfpathlineto{\pgfqpoint{5.639381in}{0.604197in}}%
\pgfpathlineto{\pgfqpoint{5.639937in}{0.604870in}}%
\pgfpathlineto{\pgfqpoint{5.642160in}{0.601432in}}%
\pgfpathlineto{\pgfqpoint{5.643828in}{0.603765in}}%
\pgfpathlineto{\pgfqpoint{5.646051in}{0.600362in}}%
\pgfpathlineto{\pgfqpoint{5.646607in}{0.603668in}}%
\pgfpathlineto{\pgfqpoint{5.647163in}{0.601714in}}%
\pgfpathlineto{\pgfqpoint{5.648275in}{0.604404in}}%
\pgfpathlineto{\pgfqpoint{5.649386in}{0.601347in}}%
\pgfpathlineto{\pgfqpoint{5.651054in}{0.604714in}}%
\pgfpathlineto{\pgfqpoint{5.652722in}{0.601152in}}%
\pgfpathlineto{\pgfqpoint{5.653833in}{0.607246in}}%
\pgfpathlineto{\pgfqpoint{5.654945in}{0.601183in}}%
\pgfpathlineto{\pgfqpoint{5.656057in}{0.602853in}}%
\pgfpathlineto{\pgfqpoint{5.656613in}{0.601879in}}%
\pgfpathlineto{\pgfqpoint{5.657168in}{0.602280in}}%
\pgfpathlineto{\pgfqpoint{5.658836in}{0.603566in}}%
\pgfpathlineto{\pgfqpoint{5.659948in}{0.601899in}}%
\pgfpathlineto{\pgfqpoint{5.661059in}{0.606635in}}%
\pgfpathlineto{\pgfqpoint{5.662727in}{0.601031in}}%
\pgfpathlineto{\pgfqpoint{5.663283in}{0.607184in}}%
\pgfpathlineto{\pgfqpoint{5.663839in}{0.600883in}}%
\pgfpathlineto{\pgfqpoint{5.664395in}{0.604223in}}%
\pgfpathlineto{\pgfqpoint{5.665506in}{0.602074in}}%
\pgfpathlineto{\pgfqpoint{5.666618in}{0.604529in}}%
\pgfpathlineto{\pgfqpoint{5.667730in}{0.601863in}}%
\pgfpathlineto{\pgfqpoint{5.668286in}{0.603206in}}%
\pgfpathlineto{\pgfqpoint{5.668842in}{0.601918in}}%
\pgfpathlineto{\pgfqpoint{5.669397in}{0.602002in}}%
\pgfpathlineto{\pgfqpoint{5.669953in}{0.600409in}}%
\pgfpathlineto{\pgfqpoint{5.670509in}{0.604080in}}%
\pgfpathlineto{\pgfqpoint{5.671065in}{0.601021in}}%
\pgfpathlineto{\pgfqpoint{5.671621in}{0.603299in}}%
\pgfpathlineto{\pgfqpoint{5.672177in}{0.601690in}}%
\pgfpathlineto{\pgfqpoint{5.673288in}{0.601831in}}%
\pgfpathlineto{\pgfqpoint{5.673844in}{0.603925in}}%
\pgfpathlineto{\pgfqpoint{5.674400in}{0.601818in}}%
\pgfpathlineto{\pgfqpoint{5.675512in}{0.601823in}}%
\pgfpathlineto{\pgfqpoint{5.676068in}{0.603780in}}%
\pgfpathlineto{\pgfqpoint{5.676624in}{0.602154in}}%
\pgfpathlineto{\pgfqpoint{5.678847in}{0.600798in}}%
\pgfpathlineto{\pgfqpoint{5.679403in}{0.602521in}}%
\pgfpathlineto{\pgfqpoint{5.679959in}{0.600959in}}%
\pgfpathlineto{\pgfqpoint{5.682182in}{0.603145in}}%
\pgfpathlineto{\pgfqpoint{5.682738in}{0.602138in}}%
\pgfpathlineto{\pgfqpoint{5.683294in}{0.605940in}}%
\pgfpathlineto{\pgfqpoint{5.683850in}{0.603565in}}%
\pgfpathlineto{\pgfqpoint{5.685517in}{0.601823in}}%
\pgfpathlineto{\pgfqpoint{5.686073in}{0.603404in}}%
\pgfpathlineto{\pgfqpoint{5.686629in}{0.601610in}}%
\pgfpathlineto{\pgfqpoint{5.687185in}{0.602490in}}%
\pgfpathlineto{\pgfqpoint{5.688297in}{0.605237in}}%
\pgfpathlineto{\pgfqpoint{5.689964in}{0.602572in}}%
\pgfpathlineto{\pgfqpoint{5.690520in}{0.601426in}}%
\pgfpathlineto{\pgfqpoint{5.691076in}{0.604973in}}%
\pgfpathlineto{\pgfqpoint{5.691632in}{0.602123in}}%
\pgfpathlineto{\pgfqpoint{5.692188in}{0.604570in}}%
\pgfpathlineto{\pgfqpoint{5.692744in}{0.603861in}}%
\pgfpathlineto{\pgfqpoint{5.693855in}{0.601038in}}%
\pgfpathlineto{\pgfqpoint{5.694411in}{0.601673in}}%
\pgfpathlineto{\pgfqpoint{5.694967in}{0.602265in}}%
\pgfpathlineto{\pgfqpoint{5.695523in}{0.601213in}}%
\pgfpathlineto{\pgfqpoint{5.696079in}{0.601912in}}%
\pgfpathlineto{\pgfqpoint{5.696635in}{0.601715in}}%
\pgfpathlineto{\pgfqpoint{5.697190in}{0.604535in}}%
\pgfpathlineto{\pgfqpoint{5.697746in}{0.603726in}}%
\pgfpathlineto{\pgfqpoint{5.698302in}{0.603718in}}%
\pgfpathlineto{\pgfqpoint{5.698858in}{0.600353in}}%
\pgfpathlineto{\pgfqpoint{5.699970in}{0.604514in}}%
\pgfpathlineto{\pgfqpoint{5.701637in}{0.601701in}}%
\pgfpathlineto{\pgfqpoint{5.702193in}{0.602356in}}%
\pgfpathlineto{\pgfqpoint{5.702749in}{0.600481in}}%
\pgfpathlineto{\pgfqpoint{5.703305in}{0.601907in}}%
\pgfpathlineto{\pgfqpoint{5.704417in}{0.603674in}}%
\pgfpathlineto{\pgfqpoint{5.704972in}{0.601165in}}%
\pgfpathlineto{\pgfqpoint{5.705528in}{0.602780in}}%
\pgfpathlineto{\pgfqpoint{5.706084in}{0.602865in}}%
\pgfpathlineto{\pgfqpoint{5.706640in}{0.601381in}}%
\pgfpathlineto{\pgfqpoint{5.707196in}{0.602369in}}%
\pgfpathlineto{\pgfqpoint{5.707752in}{0.605545in}}%
\pgfpathlineto{\pgfqpoint{5.708308in}{0.603839in}}%
\pgfpathlineto{\pgfqpoint{5.708864in}{0.605222in}}%
\pgfpathlineto{\pgfqpoint{5.709419in}{0.602117in}}%
\pgfpathlineto{\pgfqpoint{5.709975in}{0.602633in}}%
\pgfpathlineto{\pgfqpoint{5.710531in}{0.603996in}}%
\pgfpathlineto{\pgfqpoint{5.712199in}{0.600444in}}%
\pgfpathlineto{\pgfqpoint{5.713866in}{0.604754in}}%
\pgfpathlineto{\pgfqpoint{5.714978in}{0.601143in}}%
\pgfpathlineto{\pgfqpoint{5.715534in}{0.604453in}}%
\pgfpathlineto{\pgfqpoint{5.716090in}{0.601077in}}%
\pgfpathlineto{\pgfqpoint{5.716646in}{0.602482in}}%
\pgfpathlineto{\pgfqpoint{5.717201in}{0.601497in}}%
\pgfpathlineto{\pgfqpoint{5.718313in}{0.602164in}}%
\pgfpathlineto{\pgfqpoint{5.719425in}{0.607159in}}%
\pgfpathlineto{\pgfqpoint{5.719981in}{0.605638in}}%
\pgfpathlineto{\pgfqpoint{5.720537in}{0.602174in}}%
\pgfpathlineto{\pgfqpoint{5.721092in}{0.604363in}}%
\pgfpathlineto{\pgfqpoint{5.721648in}{0.602459in}}%
\pgfpathlineto{\pgfqpoint{5.722204in}{0.609046in}}%
\pgfpathlineto{\pgfqpoint{5.722760in}{0.603302in}}%
\pgfpathlineto{\pgfqpoint{5.723872in}{0.601053in}}%
\pgfpathlineto{\pgfqpoint{5.724428in}{0.602837in}}%
\pgfpathlineto{\pgfqpoint{5.725539in}{0.604230in}}%
\pgfpathlineto{\pgfqpoint{5.727207in}{0.603253in}}%
\pgfpathlineto{\pgfqpoint{5.730542in}{0.601067in}}%
\pgfpathlineto{\pgfqpoint{5.732210in}{0.603311in}}%
\pgfpathlineto{\pgfqpoint{5.733321in}{0.600492in}}%
\pgfpathlineto{\pgfqpoint{5.733877in}{0.602230in}}%
\pgfpathlineto{\pgfqpoint{5.734989in}{0.602095in}}%
\pgfpathlineto{\pgfqpoint{5.735545in}{0.603467in}}%
\pgfpathlineto{\pgfqpoint{5.736101in}{0.602659in}}%
\pgfpathlineto{\pgfqpoint{5.737212in}{0.600511in}}%
\pgfpathlineto{\pgfqpoint{5.737768in}{0.602143in}}%
\pgfpathlineto{\pgfqpoint{5.738880in}{0.602997in}}%
\pgfpathlineto{\pgfqpoint{5.739436in}{0.604213in}}%
\pgfpathlineto{\pgfqpoint{5.739992in}{0.601524in}}%
\pgfpathlineto{\pgfqpoint{5.740548in}{0.602852in}}%
\pgfpathlineto{\pgfqpoint{5.741103in}{0.604879in}}%
\pgfpathlineto{\pgfqpoint{5.742771in}{0.601228in}}%
\pgfpathlineto{\pgfqpoint{5.743327in}{0.600476in}}%
\pgfpathlineto{\pgfqpoint{5.743883in}{0.601849in}}%
\pgfpathlineto{\pgfqpoint{5.744439in}{0.601083in}}%
\pgfpathlineto{\pgfqpoint{5.745550in}{0.601570in}}%
\pgfpathlineto{\pgfqpoint{5.746106in}{0.604236in}}%
\pgfpathlineto{\pgfqpoint{5.746662in}{0.602134in}}%
\pgfpathlineto{\pgfqpoint{5.747218in}{0.600689in}}%
\pgfpathlineto{\pgfqpoint{5.747774in}{0.604420in}}%
\pgfpathlineto{\pgfqpoint{5.748330in}{0.604268in}}%
\pgfpathlineto{\pgfqpoint{5.749441in}{0.602370in}}%
\pgfpathlineto{\pgfqpoint{5.749997in}{0.603976in}}%
\pgfpathlineto{\pgfqpoint{5.750553in}{0.602511in}}%
\pgfpathlineto{\pgfqpoint{5.751109in}{0.603315in}}%
\pgfpathlineto{\pgfqpoint{5.751665in}{0.601329in}}%
\pgfpathlineto{\pgfqpoint{5.752221in}{0.603862in}}%
\pgfpathlineto{\pgfqpoint{5.752777in}{0.601485in}}%
\pgfpathlineto{\pgfqpoint{5.754444in}{0.602889in}}%
\pgfpathlineto{\pgfqpoint{5.755000in}{0.601786in}}%
\pgfpathlineto{\pgfqpoint{5.755556in}{0.604131in}}%
\pgfpathlineto{\pgfqpoint{5.756112in}{0.603683in}}%
\pgfpathlineto{\pgfqpoint{5.756668in}{0.602724in}}%
\pgfpathlineto{\pgfqpoint{5.757223in}{0.605196in}}%
\pgfpathlineto{\pgfqpoint{5.757779in}{0.602669in}}%
\pgfpathlineto{\pgfqpoint{5.759447in}{0.600652in}}%
\pgfpathlineto{\pgfqpoint{5.760003in}{0.601403in}}%
\pgfpathlineto{\pgfqpoint{5.761670in}{0.604020in}}%
\pgfpathlineto{\pgfqpoint{5.762226in}{0.601261in}}%
\pgfpathlineto{\pgfqpoint{5.762782in}{0.604174in}}%
\pgfpathlineto{\pgfqpoint{5.763338in}{0.600553in}}%
\pgfpathlineto{\pgfqpoint{5.763894in}{0.602780in}}%
\pgfpathlineto{\pgfqpoint{5.765006in}{0.601754in}}%
\pgfpathlineto{\pgfqpoint{5.766117in}{0.605918in}}%
\pgfpathlineto{\pgfqpoint{5.766673in}{0.605320in}}%
\pgfpathlineto{\pgfqpoint{5.768341in}{0.603268in}}%
\pgfpathlineto{\pgfqpoint{5.768897in}{0.603331in}}%
\pgfpathlineto{\pgfqpoint{5.770008in}{0.601026in}}%
\pgfpathlineto{\pgfqpoint{5.770564in}{0.603863in}}%
\pgfpathlineto{\pgfqpoint{5.771120in}{0.600853in}}%
\pgfpathlineto{\pgfqpoint{5.771676in}{0.602261in}}%
\pgfpathlineto{\pgfqpoint{5.772232in}{0.603906in}}%
\pgfpathlineto{\pgfqpoint{5.772788in}{0.602823in}}%
\pgfpathlineto{\pgfqpoint{5.775011in}{0.601137in}}%
\pgfpathlineto{\pgfqpoint{5.775567in}{0.605325in}}%
\pgfpathlineto{\pgfqpoint{5.776123in}{0.601594in}}%
\pgfpathlineto{\pgfqpoint{5.777234in}{0.603766in}}%
\pgfpathlineto{\pgfqpoint{5.777790in}{0.600982in}}%
\pgfpathlineto{\pgfqpoint{5.778346in}{0.602418in}}%
\pgfpathlineto{\pgfqpoint{5.779458in}{0.602297in}}%
\pgfpathlineto{\pgfqpoint{5.780014in}{0.600815in}}%
\pgfpathlineto{\pgfqpoint{5.781125in}{0.606467in}}%
\pgfpathlineto{\pgfqpoint{5.781681in}{0.604179in}}%
\pgfpathlineto{\pgfqpoint{5.782793in}{0.602575in}}%
\pgfpathlineto{\pgfqpoint{5.783349in}{0.604375in}}%
\pgfpathlineto{\pgfqpoint{5.783905in}{0.601048in}}%
\pgfpathlineto{\pgfqpoint{5.784461in}{0.604926in}}%
\pgfpathlineto{\pgfqpoint{5.785017in}{0.604320in}}%
\pgfpathlineto{\pgfqpoint{5.786684in}{0.602800in}}%
\pgfpathlineto{\pgfqpoint{5.787796in}{0.600038in}}%
\pgfpathlineto{\pgfqpoint{5.788352in}{0.601845in}}%
\pgfpathlineto{\pgfqpoint{5.788908in}{0.603607in}}%
\pgfpathlineto{\pgfqpoint{5.789463in}{0.603069in}}%
\pgfpathlineto{\pgfqpoint{5.790019in}{0.602554in}}%
\pgfpathlineto{\pgfqpoint{5.790575in}{0.603853in}}%
\pgfpathlineto{\pgfqpoint{5.791131in}{0.600835in}}%
\pgfpathlineto{\pgfqpoint{5.791687in}{0.603065in}}%
\pgfpathlineto{\pgfqpoint{5.792243in}{0.601396in}}%
\pgfpathlineto{\pgfqpoint{5.792799in}{0.602419in}}%
\pgfpathlineto{\pgfqpoint{5.793354in}{0.604153in}}%
\pgfpathlineto{\pgfqpoint{5.794466in}{0.600958in}}%
\pgfpathlineto{\pgfqpoint{5.795022in}{0.601699in}}%
\pgfpathlineto{\pgfqpoint{5.795578in}{0.600850in}}%
\pgfpathlineto{\pgfqpoint{5.801137in}{0.603970in}}%
\pgfpathlineto{\pgfqpoint{5.802804in}{0.601766in}}%
\pgfpathlineto{\pgfqpoint{5.803360in}{0.601764in}}%
\pgfpathlineto{\pgfqpoint{5.803916in}{0.604486in}}%
\pgfpathlineto{\pgfqpoint{5.804472in}{0.602949in}}%
\pgfpathlineto{\pgfqpoint{5.807807in}{0.602030in}}%
\pgfpathlineto{\pgfqpoint{5.808363in}{0.600888in}}%
\pgfpathlineto{\pgfqpoint{5.808919in}{0.601681in}}%
\pgfpathlineto{\pgfqpoint{5.809474in}{0.602648in}}%
\pgfpathlineto{\pgfqpoint{5.811142in}{0.601286in}}%
\pgfpathlineto{\pgfqpoint{5.811698in}{0.606771in}}%
\pgfpathlineto{\pgfqpoint{5.812254in}{0.602011in}}%
\pgfpathlineto{\pgfqpoint{5.812810in}{0.604544in}}%
\pgfpathlineto{\pgfqpoint{5.814477in}{0.601008in}}%
\pgfpathlineto{\pgfqpoint{5.815589in}{0.605052in}}%
\pgfpathlineto{\pgfqpoint{5.816145in}{0.600319in}}%
\pgfpathlineto{\pgfqpoint{5.816701in}{0.601263in}}%
\pgfpathlineto{\pgfqpoint{5.817256in}{0.602781in}}%
\pgfpathlineto{\pgfqpoint{5.817812in}{0.602400in}}%
\pgfpathlineto{\pgfqpoint{5.819480in}{0.600938in}}%
\pgfpathlineto{\pgfqpoint{5.820036in}{0.601981in}}%
\pgfpathlineto{\pgfqpoint{5.820592in}{0.602632in}}%
\pgfpathlineto{\pgfqpoint{5.821148in}{0.602022in}}%
\pgfpathlineto{\pgfqpoint{5.822259in}{0.600991in}}%
\pgfpathlineto{\pgfqpoint{5.823927in}{0.605405in}}%
\pgfpathlineto{\pgfqpoint{5.825594in}{0.601723in}}%
\pgfpathlineto{\pgfqpoint{5.826150in}{0.604371in}}%
\pgfpathlineto{\pgfqpoint{5.826706in}{0.600225in}}%
\pgfpathlineto{\pgfqpoint{5.827262in}{0.602094in}}%
\pgfpathlineto{\pgfqpoint{5.828374in}{0.600426in}}%
\pgfpathlineto{\pgfqpoint{5.828930in}{0.602362in}}%
\pgfpathlineto{\pgfqpoint{5.829485in}{0.600527in}}%
\pgfpathlineto{\pgfqpoint{5.831709in}{0.604613in}}%
\pgfpathlineto{\pgfqpoint{5.833376in}{0.602142in}}%
\pgfpathlineto{\pgfqpoint{5.834488in}{0.606615in}}%
\pgfpathlineto{\pgfqpoint{5.835600in}{0.600854in}}%
\pgfpathlineto{\pgfqpoint{5.837267in}{0.604826in}}%
\pgfpathlineto{\pgfqpoint{5.838935in}{0.601324in}}%
\pgfpathlineto{\pgfqpoint{5.839491in}{0.601887in}}%
\pgfpathlineto{\pgfqpoint{5.840047in}{0.605196in}}%
\pgfpathlineto{\pgfqpoint{5.840603in}{0.602134in}}%
\pgfpathlineto{\pgfqpoint{5.841159in}{0.604439in}}%
\pgfpathlineto{\pgfqpoint{5.841714in}{0.600697in}}%
\pgfpathlineto{\pgfqpoint{5.842270in}{0.603245in}}%
\pgfpathlineto{\pgfqpoint{5.843382in}{0.601824in}}%
\pgfpathlineto{\pgfqpoint{5.843938in}{0.602296in}}%
\pgfpathlineto{\pgfqpoint{5.844494in}{0.601777in}}%
\pgfpathlineto{\pgfqpoint{5.845050in}{0.602710in}}%
\pgfpathlineto{\pgfqpoint{5.845605in}{0.602332in}}%
\pgfpathlineto{\pgfqpoint{5.846161in}{0.603654in}}%
\pgfpathlineto{\pgfqpoint{5.846717in}{0.601409in}}%
\pgfpathlineto{\pgfqpoint{5.847273in}{0.603246in}}%
\pgfpathlineto{\pgfqpoint{5.848385in}{0.603732in}}%
\pgfpathlineto{\pgfqpoint{5.849496in}{0.601753in}}%
\pgfpathlineto{\pgfqpoint{5.850608in}{0.603507in}}%
\pgfpathlineto{\pgfqpoint{5.851164in}{0.600978in}}%
\pgfpathlineto{\pgfqpoint{5.851720in}{0.604699in}}%
\pgfpathlineto{\pgfqpoint{5.852276in}{0.603717in}}%
\pgfpathlineto{\pgfqpoint{5.852832in}{0.601358in}}%
\pgfpathlineto{\pgfqpoint{5.853387in}{0.601895in}}%
\pgfpathlineto{\pgfqpoint{5.853943in}{0.602425in}}%
\pgfpathlineto{\pgfqpoint{5.854499in}{0.600809in}}%
\pgfpathlineto{\pgfqpoint{5.855611in}{0.604518in}}%
\pgfpathlineto{\pgfqpoint{5.857279in}{0.601463in}}%
\pgfpathlineto{\pgfqpoint{5.857834in}{0.602626in}}%
\pgfpathlineto{\pgfqpoint{5.858390in}{0.601906in}}%
\pgfpathlineto{\pgfqpoint{5.858946in}{0.601002in}}%
\pgfpathlineto{\pgfqpoint{5.859502in}{0.603301in}}%
\pgfpathlineto{\pgfqpoint{5.860058in}{0.602770in}}%
\pgfpathlineto{\pgfqpoint{5.860614in}{0.602033in}}%
\pgfpathlineto{\pgfqpoint{5.861170in}{0.603086in}}%
\pgfpathlineto{\pgfqpoint{5.863393in}{0.603973in}}%
\pgfpathlineto{\pgfqpoint{5.863949in}{0.601748in}}%
\pgfpathlineto{\pgfqpoint{5.864505in}{0.603693in}}%
\pgfpathlineto{\pgfqpoint{5.866172in}{0.600735in}}%
\pgfpathlineto{\pgfqpoint{5.867284in}{0.604315in}}%
\pgfpathlineto{\pgfqpoint{5.867840in}{0.603139in}}%
\pgfpathlineto{\pgfqpoint{5.869507in}{0.601174in}}%
\pgfpathlineto{\pgfqpoint{5.870063in}{0.602348in}}%
\pgfpathlineto{\pgfqpoint{5.870619in}{0.605891in}}%
\pgfpathlineto{\pgfqpoint{5.871175in}{0.603761in}}%
\pgfpathlineto{\pgfqpoint{5.871731in}{0.604338in}}%
\pgfpathlineto{\pgfqpoint{5.872843in}{0.603348in}}%
\pgfpathlineto{\pgfqpoint{5.873398in}{0.604941in}}%
\pgfpathlineto{\pgfqpoint{5.874510in}{0.601832in}}%
\pgfpathlineto{\pgfqpoint{5.875066in}{0.601999in}}%
\pgfpathlineto{\pgfqpoint{5.875622in}{0.601006in}}%
\pgfpathlineto{\pgfqpoint{5.876734in}{0.605845in}}%
\pgfpathlineto{\pgfqpoint{5.877290in}{0.603757in}}%
\pgfpathlineto{\pgfqpoint{5.877845in}{0.603835in}}%
\pgfpathlineto{\pgfqpoint{5.878957in}{0.600135in}}%
\pgfpathlineto{\pgfqpoint{5.879513in}{0.600897in}}%
\pgfpathlineto{\pgfqpoint{5.880625in}{0.605974in}}%
\pgfpathlineto{\pgfqpoint{5.881736in}{0.601101in}}%
\pgfpathlineto{\pgfqpoint{5.883404in}{0.603360in}}%
\pgfpathlineto{\pgfqpoint{5.883960in}{0.600729in}}%
\pgfpathlineto{\pgfqpoint{5.884516in}{0.601757in}}%
\pgfpathlineto{\pgfqpoint{5.885072in}{0.601155in}}%
\pgfpathlineto{\pgfqpoint{5.885627in}{0.602082in}}%
\pgfpathlineto{\pgfqpoint{5.886183in}{0.602967in}}%
\pgfpathlineto{\pgfqpoint{5.886739in}{0.601247in}}%
\pgfpathlineto{\pgfqpoint{5.887295in}{0.601923in}}%
\pgfpathlineto{\pgfqpoint{5.888407in}{0.601419in}}%
\pgfpathlineto{\pgfqpoint{5.888963in}{0.602406in}}%
\pgfpathlineto{\pgfqpoint{5.889518in}{0.600936in}}%
\pgfpathlineto{\pgfqpoint{5.890630in}{0.603822in}}%
\pgfpathlineto{\pgfqpoint{5.891186in}{0.602178in}}%
\pgfpathlineto{\pgfqpoint{5.891742in}{0.605829in}}%
\pgfpathlineto{\pgfqpoint{5.892298in}{0.602145in}}%
\pgfpathlineto{\pgfqpoint{5.893965in}{0.600846in}}%
\pgfpathlineto{\pgfqpoint{5.895077in}{0.604928in}}%
\pgfpathlineto{\pgfqpoint{5.895633in}{0.604667in}}%
\pgfpathlineto{\pgfqpoint{5.897301in}{0.601259in}}%
\pgfpathlineto{\pgfqpoint{5.897856in}{0.601874in}}%
\pgfpathlineto{\pgfqpoint{5.898412in}{0.604818in}}%
\pgfpathlineto{\pgfqpoint{5.898968in}{0.600812in}}%
\pgfpathlineto{\pgfqpoint{5.899524in}{0.605001in}}%
\pgfpathlineto{\pgfqpoint{5.900080in}{0.603960in}}%
\pgfpathlineto{\pgfqpoint{5.901192in}{0.604250in}}%
\pgfpathlineto{\pgfqpoint{5.902859in}{0.601738in}}%
\pgfpathlineto{\pgfqpoint{5.903971in}{0.604834in}}%
\pgfpathlineto{\pgfqpoint{5.904527in}{0.602374in}}%
\pgfpathlineto{\pgfqpoint{5.905083in}{0.602902in}}%
\pgfpathlineto{\pgfqpoint{5.906194in}{0.602892in}}%
\pgfpathlineto{\pgfqpoint{5.906750in}{0.600806in}}%
\pgfpathlineto{\pgfqpoint{5.907306in}{0.604060in}}%
\pgfpathlineto{\pgfqpoint{5.907862in}{0.601919in}}%
\pgfpathlineto{\pgfqpoint{5.908418in}{0.600767in}}%
\pgfpathlineto{\pgfqpoint{5.908974in}{0.601959in}}%
\pgfpathlineto{\pgfqpoint{5.909529in}{0.602177in}}%
\pgfpathlineto{\pgfqpoint{5.910085in}{0.600454in}}%
\pgfpathlineto{\pgfqpoint{5.910641in}{0.601415in}}%
\pgfpathlineto{\pgfqpoint{5.911197in}{0.601701in}}%
\pgfpathlineto{\pgfqpoint{5.911753in}{0.600638in}}%
\pgfpathlineto{\pgfqpoint{5.913420in}{0.603642in}}%
\pgfpathlineto{\pgfqpoint{5.913976in}{0.603438in}}%
\pgfpathlineto{\pgfqpoint{5.914532in}{0.600594in}}%
\pgfpathlineto{\pgfqpoint{5.915088in}{0.603421in}}%
\pgfpathlineto{\pgfqpoint{5.916200in}{0.602522in}}%
\pgfpathlineto{\pgfqpoint{5.918423in}{0.605195in}}%
\pgfpathlineto{\pgfqpoint{5.918979in}{0.602415in}}%
\pgfpathlineto{\pgfqpoint{5.919535in}{0.606567in}}%
\pgfpathlineto{\pgfqpoint{5.920091in}{0.602424in}}%
\pgfpathlineto{\pgfqpoint{5.920647in}{0.603425in}}%
\pgfpathlineto{\pgfqpoint{5.921758in}{0.601184in}}%
\pgfpathlineto{\pgfqpoint{5.922314in}{0.603747in}}%
\pgfpathlineto{\pgfqpoint{5.922870in}{0.601035in}}%
\pgfpathlineto{\pgfqpoint{5.923426in}{0.601209in}}%
\pgfpathlineto{\pgfqpoint{5.924538in}{0.607212in}}%
\pgfpathlineto{\pgfqpoint{5.925094in}{0.602452in}}%
\pgfpathlineto{\pgfqpoint{5.925649in}{0.602645in}}%
\pgfpathlineto{\pgfqpoint{5.926761in}{0.603177in}}%
\pgfpathlineto{\pgfqpoint{5.927317in}{0.603845in}}%
\pgfpathlineto{\pgfqpoint{5.927873in}{0.608115in}}%
\pgfpathlineto{\pgfqpoint{5.928429in}{0.603006in}}%
\pgfpathlineto{\pgfqpoint{5.928985in}{0.604903in}}%
\pgfpathlineto{\pgfqpoint{5.930096in}{0.601682in}}%
\pgfpathlineto{\pgfqpoint{5.930652in}{0.606239in}}%
\pgfpathlineto{\pgfqpoint{5.931208in}{0.602523in}}%
\pgfpathlineto{\pgfqpoint{5.931764in}{0.605228in}}%
\pgfpathlineto{\pgfqpoint{5.932320in}{0.603405in}}%
\pgfpathlineto{\pgfqpoint{5.932876in}{0.604181in}}%
\pgfpathlineto{\pgfqpoint{5.933432in}{0.600886in}}%
\pgfpathlineto{\pgfqpoint{5.933987in}{0.601801in}}%
\pgfpathlineto{\pgfqpoint{5.935099in}{0.601985in}}%
\pgfpathlineto{\pgfqpoint{5.935655in}{0.603496in}}%
\pgfpathlineto{\pgfqpoint{5.936211in}{0.602444in}}%
\pgfpathlineto{\pgfqpoint{5.937323in}{0.602211in}}%
\pgfpathlineto{\pgfqpoint{5.937878in}{0.607423in}}%
\pgfpathlineto{\pgfqpoint{5.938990in}{0.601749in}}%
\pgfpathlineto{\pgfqpoint{5.939546in}{0.602462in}}%
\pgfpathlineto{\pgfqpoint{5.940102in}{0.601116in}}%
\pgfpathlineto{\pgfqpoint{5.940658in}{0.604208in}}%
\pgfpathlineto{\pgfqpoint{5.941214in}{0.600146in}}%
\pgfpathlineto{\pgfqpoint{5.941769in}{0.603354in}}%
\pgfpathlineto{\pgfqpoint{5.943993in}{0.600726in}}%
\pgfpathlineto{\pgfqpoint{5.945105in}{0.602335in}}%
\pgfpathlineto{\pgfqpoint{5.946216in}{0.601114in}}%
\pgfpathlineto{\pgfqpoint{5.947328in}{0.603722in}}%
\pgfpathlineto{\pgfqpoint{5.948440in}{0.601383in}}%
\pgfpathlineto{\pgfqpoint{5.948996in}{0.607005in}}%
\pgfpathlineto{\pgfqpoint{5.949551in}{0.603546in}}%
\pgfpathlineto{\pgfqpoint{5.950107in}{0.600080in}}%
\pgfpathlineto{\pgfqpoint{5.950663in}{0.600909in}}%
\pgfpathlineto{\pgfqpoint{5.951219in}{0.600616in}}%
\pgfpathlineto{\pgfqpoint{5.952331in}{0.603007in}}%
\pgfpathlineto{\pgfqpoint{5.952887in}{0.601492in}}%
\pgfpathlineto{\pgfqpoint{5.953443in}{0.604671in}}%
\pgfpathlineto{\pgfqpoint{5.953998in}{0.602559in}}%
\pgfpathlineto{\pgfqpoint{5.955110in}{0.601814in}}%
\pgfpathlineto{\pgfqpoint{5.955666in}{0.601128in}}%
\pgfpathlineto{\pgfqpoint{5.956222in}{0.603150in}}%
\pgfpathlineto{\pgfqpoint{5.956778in}{0.601919in}}%
\pgfpathlineto{\pgfqpoint{5.957334in}{0.602577in}}%
\pgfpathlineto{\pgfqpoint{5.957889in}{0.601495in}}%
\pgfpathlineto{\pgfqpoint{5.958445in}{0.602827in}}%
\pgfpathlineto{\pgfqpoint{5.959557in}{0.603502in}}%
\pgfpathlineto{\pgfqpoint{5.960669in}{0.601199in}}%
\pgfpathlineto{\pgfqpoint{5.962336in}{0.603709in}}%
\pgfpathlineto{\pgfqpoint{5.962892in}{0.602762in}}%
\pgfpathlineto{\pgfqpoint{5.963448in}{0.602256in}}%
\pgfpathlineto{\pgfqpoint{5.964004in}{0.604728in}}%
\pgfpathlineto{\pgfqpoint{5.964560in}{0.602314in}}%
\pgfpathlineto{\pgfqpoint{5.966227in}{0.602666in}}%
\pgfpathlineto{\pgfqpoint{5.966783in}{0.603676in}}%
\pgfpathlineto{\pgfqpoint{5.968451in}{0.601075in}}%
\pgfpathlineto{\pgfqpoint{5.970674in}{0.603276in}}%
\pgfpathlineto{\pgfqpoint{5.972342in}{0.600556in}}%
\pgfpathlineto{\pgfqpoint{5.974565in}{0.602955in}}%
\pgfpathlineto{\pgfqpoint{5.975121in}{0.601538in}}%
\pgfpathlineto{\pgfqpoint{5.975677in}{0.604283in}}%
\pgfpathlineto{\pgfqpoint{5.976233in}{0.603878in}}%
\pgfpathlineto{\pgfqpoint{5.977900in}{0.600853in}}%
\pgfpathlineto{\pgfqpoint{5.979012in}{0.603800in}}%
\pgfpathlineto{\pgfqpoint{5.979568in}{0.602126in}}%
\pgfpathlineto{\pgfqpoint{5.980680in}{0.601138in}}%
\pgfpathlineto{\pgfqpoint{5.981236in}{0.606510in}}%
\pgfpathlineto{\pgfqpoint{5.981791in}{0.604245in}}%
\pgfpathlineto{\pgfqpoint{5.982347in}{0.600429in}}%
\pgfpathlineto{\pgfqpoint{5.982903in}{0.602145in}}%
\pgfpathlineto{\pgfqpoint{5.985127in}{0.601465in}}%
\pgfpathlineto{\pgfqpoint{5.986238in}{0.603745in}}%
\pgfpathlineto{\pgfqpoint{5.986794in}{0.603146in}}%
\pgfpathlineto{\pgfqpoint{5.987350in}{0.602045in}}%
\pgfpathlineto{\pgfqpoint{5.987906in}{0.603588in}}%
\pgfpathlineto{\pgfqpoint{5.988462in}{0.600295in}}%
\pgfpathlineto{\pgfqpoint{5.989018in}{0.601218in}}%
\pgfpathlineto{\pgfqpoint{5.990685in}{0.602895in}}%
\pgfpathlineto{\pgfqpoint{5.991797in}{0.600990in}}%
\pgfpathlineto{\pgfqpoint{5.993465in}{0.602332in}}%
\pgfpathlineto{\pgfqpoint{5.994020in}{0.601249in}}%
\pgfpathlineto{\pgfqpoint{5.995132in}{0.604294in}}%
\pgfpathlineto{\pgfqpoint{5.995688in}{0.603768in}}%
\pgfpathlineto{\pgfqpoint{5.996244in}{0.603624in}}%
\pgfpathlineto{\pgfqpoint{5.996800in}{0.601453in}}%
\pgfpathlineto{\pgfqpoint{5.997356in}{0.602461in}}%
\pgfpathlineto{\pgfqpoint{5.997911in}{0.601965in}}%
\pgfpathlineto{\pgfqpoint{5.998467in}{0.603336in}}%
\pgfpathlineto{\pgfqpoint{5.999579in}{0.600610in}}%
\pgfpathlineto{\pgfqpoint{6.002914in}{0.602653in}}%
\pgfpathlineto{\pgfqpoint{6.004026in}{0.602558in}}%
\pgfpathlineto{\pgfqpoint{6.005138in}{0.602381in}}%
\pgfpathlineto{\pgfqpoint{6.005693in}{0.601271in}}%
\pgfpathlineto{\pgfqpoint{6.006249in}{0.601664in}}%
\pgfpathlineto{\pgfqpoint{6.008473in}{0.603247in}}%
\pgfpathlineto{\pgfqpoint{6.009029in}{0.601096in}}%
\pgfpathlineto{\pgfqpoint{6.009585in}{0.602892in}}%
\pgfpathlineto{\pgfqpoint{6.011808in}{0.601250in}}%
\pgfpathlineto{\pgfqpoint{6.012920in}{0.603651in}}%
\pgfpathlineto{\pgfqpoint{6.013476in}{0.602594in}}%
\pgfpathlineto{\pgfqpoint{6.014587in}{0.601233in}}%
\pgfpathlineto{\pgfqpoint{6.015699in}{0.603115in}}%
\pgfpathlineto{\pgfqpoint{6.016255in}{0.600712in}}%
\pgfpathlineto{\pgfqpoint{6.016811in}{0.600877in}}%
\pgfpathlineto{\pgfqpoint{6.017922in}{0.603404in}}%
\pgfpathlineto{\pgfqpoint{6.018478in}{0.600335in}}%
\pgfpathlineto{\pgfqpoint{6.019034in}{0.603486in}}%
\pgfpathlineto{\pgfqpoint{6.020702in}{0.600625in}}%
\pgfpathlineto{\pgfqpoint{6.021813in}{0.603732in}}%
\pgfpathlineto{\pgfqpoint{6.022925in}{0.600952in}}%
\pgfpathlineto{\pgfqpoint{6.024037in}{0.602979in}}%
\pgfpathlineto{\pgfqpoint{6.024593in}{0.603463in}}%
\pgfpathlineto{\pgfqpoint{6.025149in}{0.600584in}}%
\pgfpathlineto{\pgfqpoint{6.025704in}{0.602589in}}%
\pgfpathlineto{\pgfqpoint{6.026260in}{0.601157in}}%
\pgfpathlineto{\pgfqpoint{6.026816in}{0.604435in}}%
\pgfpathlineto{\pgfqpoint{6.027372in}{0.601952in}}%
\pgfpathlineto{\pgfqpoint{6.027928in}{0.601740in}}%
\pgfpathlineto{\pgfqpoint{6.028484in}{0.600283in}}%
\pgfpathlineto{\pgfqpoint{6.029040in}{0.600649in}}%
\pgfpathlineto{\pgfqpoint{6.029596in}{0.605082in}}%
\pgfpathlineto{\pgfqpoint{6.030151in}{0.603571in}}%
\pgfpathlineto{\pgfqpoint{6.031819in}{0.600640in}}%
\pgfpathlineto{\pgfqpoint{6.034598in}{0.601160in}}%
\pgfpathlineto{\pgfqpoint{6.035154in}{0.600421in}}%
\pgfpathlineto{\pgfqpoint{6.035710in}{0.603384in}}%
\pgfpathlineto{\pgfqpoint{6.036266in}{0.602329in}}%
\pgfpathlineto{\pgfqpoint{6.037933in}{0.600521in}}%
\pgfpathlineto{\pgfqpoint{6.038489in}{0.603081in}}%
\pgfpathlineto{\pgfqpoint{6.039045in}{0.602218in}}%
\pgfpathlineto{\pgfqpoint{6.039601in}{0.602951in}}%
\pgfpathlineto{\pgfqpoint{6.041269in}{0.601221in}}%
\pgfpathlineto{\pgfqpoint{6.043492in}{0.601334in}}%
\pgfpathlineto{\pgfqpoint{6.044048in}{0.603704in}}%
\pgfpathlineto{\pgfqpoint{6.044604in}{0.601930in}}%
\pgfpathlineto{\pgfqpoint{6.046271in}{0.602962in}}%
\pgfpathlineto{\pgfqpoint{6.047383in}{0.600271in}}%
\pgfpathlineto{\pgfqpoint{6.047939in}{0.602132in}}%
\pgfpathlineto{\pgfqpoint{6.048495in}{0.601301in}}%
\pgfpathlineto{\pgfqpoint{6.049051in}{0.605291in}}%
\pgfpathlineto{\pgfqpoint{6.049607in}{0.603440in}}%
\pgfpathlineto{\pgfqpoint{6.050162in}{0.601057in}}%
\pgfpathlineto{\pgfqpoint{6.050718in}{0.602515in}}%
\pgfpathlineto{\pgfqpoint{6.051830in}{0.601142in}}%
\pgfpathlineto{\pgfqpoint{6.053498in}{0.602041in}}%
\pgfpathlineto{\pgfqpoint{6.054609in}{0.600971in}}%
\pgfpathlineto{\pgfqpoint{6.055165in}{0.602591in}}%
\pgfpathlineto{\pgfqpoint{6.056277in}{0.600157in}}%
\pgfpathlineto{\pgfqpoint{6.057944in}{0.603444in}}%
\pgfpathlineto{\pgfqpoint{6.058500in}{0.600844in}}%
\pgfpathlineto{\pgfqpoint{6.059056in}{0.602113in}}%
\pgfpathlineto{\pgfqpoint{6.060724in}{0.603820in}}%
\pgfpathlineto{\pgfqpoint{6.061280in}{0.602177in}}%
\pgfpathlineto{\pgfqpoint{6.061835in}{0.604467in}}%
\pgfpathlineto{\pgfqpoint{6.062391in}{0.603954in}}%
\pgfpathlineto{\pgfqpoint{6.062947in}{0.601067in}}%
\pgfpathlineto{\pgfqpoint{6.063503in}{0.601565in}}%
\pgfpathlineto{\pgfqpoint{6.064059in}{0.604521in}}%
\pgfpathlineto{\pgfqpoint{6.064615in}{0.603160in}}%
\pgfpathlineto{\pgfqpoint{6.065171in}{0.603504in}}%
\pgfpathlineto{\pgfqpoint{6.066282in}{0.600441in}}%
\pgfpathlineto{\pgfqpoint{6.066838in}{0.601265in}}%
\pgfpathlineto{\pgfqpoint{6.068506in}{0.603057in}}%
\pgfpathlineto{\pgfqpoint{6.069618in}{0.601073in}}%
\pgfpathlineto{\pgfqpoint{6.070173in}{0.602406in}}%
\pgfpathlineto{\pgfqpoint{6.070729in}{0.601564in}}%
\pgfpathlineto{\pgfqpoint{6.074620in}{0.602774in}}%
\pgfpathlineto{\pgfqpoint{6.075176in}{0.600579in}}%
\pgfpathlineto{\pgfqpoint{6.075732in}{0.601396in}}%
\pgfpathlineto{\pgfqpoint{6.076288in}{0.601425in}}%
\pgfpathlineto{\pgfqpoint{6.076844in}{0.600302in}}%
\pgfpathlineto{\pgfqpoint{6.077400in}{0.601209in}}%
\pgfpathlineto{\pgfqpoint{6.077955in}{0.600566in}}%
\pgfpathlineto{\pgfqpoint{6.079067in}{0.602918in}}%
\pgfpathlineto{\pgfqpoint{6.079623in}{0.601280in}}%
\pgfpathlineto{\pgfqpoint{6.080179in}{0.602911in}}%
\pgfpathlineto{\pgfqpoint{6.080735in}{0.603264in}}%
\pgfpathlineto{\pgfqpoint{6.081291in}{0.601115in}}%
\pgfpathlineto{\pgfqpoint{6.081846in}{0.602784in}}%
\pgfpathlineto{\pgfqpoint{6.082402in}{0.601360in}}%
\pgfpathlineto{\pgfqpoint{6.083514in}{0.605066in}}%
\pgfpathlineto{\pgfqpoint{6.084626in}{0.600973in}}%
\pgfpathlineto{\pgfqpoint{6.085182in}{0.603439in}}%
\pgfpathlineto{\pgfqpoint{6.085738in}{0.601844in}}%
\pgfpathlineto{\pgfqpoint{6.086293in}{0.601455in}}%
\pgfpathlineto{\pgfqpoint{6.087961in}{0.602297in}}%
\pgfpathlineto{\pgfqpoint{6.088517in}{0.602853in}}%
\pgfpathlineto{\pgfqpoint{6.089629in}{0.600295in}}%
\pgfpathlineto{\pgfqpoint{6.090184in}{0.601275in}}%
\pgfpathlineto{\pgfqpoint{6.090740in}{0.600751in}}%
\pgfpathlineto{\pgfqpoint{6.091296in}{0.603532in}}%
\pgfpathlineto{\pgfqpoint{6.091852in}{0.600832in}}%
\pgfpathlineto{\pgfqpoint{6.092408in}{0.600264in}}%
\pgfpathlineto{\pgfqpoint{6.093520in}{0.604523in}}%
\pgfpathlineto{\pgfqpoint{6.094075in}{0.600783in}}%
\pgfpathlineto{\pgfqpoint{6.094631in}{0.601613in}}%
\pgfpathlineto{\pgfqpoint{6.096299in}{0.602586in}}%
\pgfpathlineto{\pgfqpoint{6.096855in}{0.605052in}}%
\pgfpathlineto{\pgfqpoint{6.098522in}{0.601566in}}%
\pgfpathlineto{\pgfqpoint{6.099078in}{0.604575in}}%
\pgfpathlineto{\pgfqpoint{6.100746in}{0.601055in}}%
\pgfpathlineto{\pgfqpoint{6.101302in}{0.600597in}}%
\pgfpathlineto{\pgfqpoint{6.101857in}{0.604010in}}%
\pgfpathlineto{\pgfqpoint{6.102413in}{0.602316in}}%
\pgfpathlineto{\pgfqpoint{6.104081in}{0.606048in}}%
\pgfpathlineto{\pgfqpoint{6.105193in}{0.601616in}}%
\pgfpathlineto{\pgfqpoint{6.105749in}{0.605559in}}%
\pgfpathlineto{\pgfqpoint{6.106304in}{0.602874in}}%
\pgfpathlineto{\pgfqpoint{6.107416in}{0.601074in}}%
\pgfpathlineto{\pgfqpoint{6.107972in}{0.605455in}}%
\pgfpathlineto{\pgfqpoint{6.108528in}{0.602995in}}%
\pgfpathlineto{\pgfqpoint{6.109084in}{0.600581in}}%
\pgfpathlineto{\pgfqpoint{6.109640in}{0.601908in}}%
\pgfpathlineto{\pgfqpoint{6.110751in}{0.603883in}}%
\pgfpathlineto{\pgfqpoint{6.111307in}{0.601034in}}%
\pgfpathlineto{\pgfqpoint{6.111863in}{0.602021in}}%
\pgfpathlineto{\pgfqpoint{6.114642in}{0.602662in}}%
\pgfpathlineto{\pgfqpoint{6.115198in}{0.600675in}}%
\pgfpathlineto{\pgfqpoint{6.115754in}{0.602069in}}%
\pgfpathlineto{\pgfqpoint{6.116310in}{0.602437in}}%
\pgfpathlineto{\pgfqpoint{6.116866in}{0.601355in}}%
\pgfpathlineto{\pgfqpoint{6.117422in}{0.603042in}}%
\pgfpathlineto{\pgfqpoint{6.117977in}{0.602687in}}%
\pgfpathlineto{\pgfqpoint{6.118533in}{0.600291in}}%
\pgfpathlineto{\pgfqpoint{6.119089in}{0.601486in}}%
\pgfpathlineto{\pgfqpoint{6.120757in}{0.603587in}}%
\pgfpathlineto{\pgfqpoint{6.121313in}{0.603192in}}%
\pgfpathlineto{\pgfqpoint{6.123536in}{0.600623in}}%
\pgfpathlineto{\pgfqpoint{6.124648in}{0.604873in}}%
\pgfpathlineto{\pgfqpoint{6.125204in}{0.602413in}}%
\pgfpathlineto{\pgfqpoint{6.125760in}{0.600157in}}%
\pgfpathlineto{\pgfqpoint{6.126315in}{0.601091in}}%
\pgfpathlineto{\pgfqpoint{6.126871in}{0.601003in}}%
\pgfpathlineto{\pgfqpoint{6.127983in}{0.602676in}}%
\pgfpathlineto{\pgfqpoint{6.128539in}{0.601178in}}%
\pgfpathlineto{\pgfqpoint{6.129095in}{0.602504in}}%
\pgfpathlineto{\pgfqpoint{6.130762in}{0.601274in}}%
\pgfpathlineto{\pgfqpoint{6.131318in}{0.602723in}}%
\pgfpathlineto{\pgfqpoint{6.131874in}{0.601282in}}%
\pgfpathlineto{\pgfqpoint{6.132430in}{0.602347in}}%
\pgfpathlineto{\pgfqpoint{6.132986in}{0.602016in}}%
\pgfpathlineto{\pgfqpoint{6.134653in}{0.601760in}}%
\pgfpathlineto{\pgfqpoint{6.135209in}{0.601869in}}%
\pgfpathlineto{\pgfqpoint{6.135765in}{0.603131in}}%
\pgfpathlineto{\pgfqpoint{6.136321in}{0.602047in}}%
\pgfpathlineto{\pgfqpoint{6.137433in}{0.603400in}}%
\pgfpathlineto{\pgfqpoint{6.137988in}{0.600731in}}%
\pgfpathlineto{\pgfqpoint{6.138544in}{0.602471in}}%
\pgfpathlineto{\pgfqpoint{6.140212in}{0.601012in}}%
\pgfpathlineto{\pgfqpoint{6.141324in}{0.602030in}}%
\pgfpathlineto{\pgfqpoint{6.141880in}{0.600440in}}%
\pgfpathlineto{\pgfqpoint{6.142435in}{0.602918in}}%
\pgfpathlineto{\pgfqpoint{6.142991in}{0.601093in}}%
\pgfpathlineto{\pgfqpoint{6.143547in}{0.600911in}}%
\pgfpathlineto{\pgfqpoint{6.144103in}{0.603114in}}%
\pgfpathlineto{\pgfqpoint{6.144659in}{0.600935in}}%
\pgfpathlineto{\pgfqpoint{6.147438in}{0.602811in}}%
\pgfpathlineto{\pgfqpoint{6.149106in}{0.600999in}}%
\pgfpathlineto{\pgfqpoint{6.150217in}{0.605233in}}%
\pgfpathlineto{\pgfqpoint{6.151885in}{0.601246in}}%
\pgfpathlineto{\pgfqpoint{6.152441in}{0.601138in}}%
\pgfpathlineto{\pgfqpoint{6.152997in}{0.602668in}}%
\pgfpathlineto{\pgfqpoint{6.153553in}{0.601959in}}%
\pgfpathlineto{\pgfqpoint{6.154664in}{0.601377in}}%
\pgfpathlineto{\pgfqpoint{6.156222in}{0.602167in}}%
\pgfpathmoveto{\pgfqpoint{6.156222in}{0.599978in}}%
\pgfpathlineto{\pgfqpoint{0.707889in}{0.599981in}}%
\pgfpathmoveto{\pgfqpoint{0.707889in}{0.602167in}}%
\pgfpathlineto{\pgfqpoint{0.709446in}{0.601377in}}%
\pgfpathlineto{\pgfqpoint{0.713338in}{0.602485in}}%
\pgfpathlineto{\pgfqpoint{0.713893in}{0.605233in}}%
\pgfpathlineto{\pgfqpoint{0.714449in}{0.603258in}}%
\pgfpathlineto{\pgfqpoint{0.715005in}{0.600999in}}%
\pgfpathlineto{\pgfqpoint{0.715561in}{0.601324in}}%
\pgfpathlineto{\pgfqpoint{0.717784in}{0.602207in}}%
\pgfpathlineto{\pgfqpoint{0.719452in}{0.600935in}}%
\pgfpathlineto{\pgfqpoint{0.720008in}{0.603114in}}%
\pgfpathlineto{\pgfqpoint{0.720564in}{0.600911in}}%
\pgfpathlineto{\pgfqpoint{0.721120in}{0.601093in}}%
\pgfpathlineto{\pgfqpoint{0.721675in}{0.602918in}}%
\pgfpathlineto{\pgfqpoint{0.722231in}{0.600440in}}%
\pgfpathlineto{\pgfqpoint{0.722787in}{0.602030in}}%
\pgfpathlineto{\pgfqpoint{0.725011in}{0.601665in}}%
\pgfpathlineto{\pgfqpoint{0.725566in}{0.602471in}}%
\pgfpathlineto{\pgfqpoint{0.726122in}{0.600731in}}%
\pgfpathlineto{\pgfqpoint{0.726678in}{0.603400in}}%
\pgfpathlineto{\pgfqpoint{0.727234in}{0.602690in}}%
\pgfpathlineto{\pgfqpoint{0.727790in}{0.602047in}}%
\pgfpathlineto{\pgfqpoint{0.728346in}{0.603131in}}%
\pgfpathlineto{\pgfqpoint{0.728902in}{0.601869in}}%
\pgfpathlineto{\pgfqpoint{0.732237in}{0.601282in}}%
\pgfpathlineto{\pgfqpoint{0.733904in}{0.602676in}}%
\pgfpathlineto{\pgfqpoint{0.735572in}{0.601178in}}%
\pgfpathlineto{\pgfqpoint{0.736684in}{0.602596in}}%
\pgfpathlineto{\pgfqpoint{0.738351in}{0.600157in}}%
\pgfpathlineto{\pgfqpoint{0.739463in}{0.604873in}}%
\pgfpathlineto{\pgfqpoint{0.740575in}{0.600623in}}%
\pgfpathlineto{\pgfqpoint{0.742798in}{0.603192in}}%
\pgfpathlineto{\pgfqpoint{0.743910in}{0.603259in}}%
\pgfpathlineto{\pgfqpoint{0.745577in}{0.600291in}}%
\pgfpathlineto{\pgfqpoint{0.746689in}{0.603042in}}%
\pgfpathlineto{\pgfqpoint{0.747245in}{0.601355in}}%
\pgfpathlineto{\pgfqpoint{0.747801in}{0.602437in}}%
\pgfpathlineto{\pgfqpoint{0.748913in}{0.600675in}}%
\pgfpathlineto{\pgfqpoint{0.749468in}{0.602662in}}%
\pgfpathlineto{\pgfqpoint{0.752248in}{0.602021in}}%
\pgfpathlineto{\pgfqpoint{0.752804in}{0.601034in}}%
\pgfpathlineto{\pgfqpoint{0.753360in}{0.603883in}}%
\pgfpathlineto{\pgfqpoint{0.753915in}{0.602522in}}%
\pgfpathlineto{\pgfqpoint{0.755027in}{0.600581in}}%
\pgfpathlineto{\pgfqpoint{0.756139in}{0.605455in}}%
\pgfpathlineto{\pgfqpoint{0.756695in}{0.601074in}}%
\pgfpathlineto{\pgfqpoint{0.757251in}{0.602460in}}%
\pgfpathlineto{\pgfqpoint{0.757806in}{0.602874in}}%
\pgfpathlineto{\pgfqpoint{0.758362in}{0.605559in}}%
\pgfpathlineto{\pgfqpoint{0.758918in}{0.601616in}}%
\pgfpathlineto{\pgfqpoint{0.759474in}{0.603138in}}%
\pgfpathlineto{\pgfqpoint{0.760030in}{0.606048in}}%
\pgfpathlineto{\pgfqpoint{0.760586in}{0.603640in}}%
\pgfpathlineto{\pgfqpoint{0.761697in}{0.602316in}}%
\pgfpathlineto{\pgfqpoint{0.762253in}{0.604010in}}%
\pgfpathlineto{\pgfqpoint{0.762809in}{0.600597in}}%
\pgfpathlineto{\pgfqpoint{0.763365in}{0.601055in}}%
\pgfpathlineto{\pgfqpoint{0.765033in}{0.604575in}}%
\pgfpathlineto{\pgfqpoint{0.765588in}{0.601566in}}%
\pgfpathlineto{\pgfqpoint{0.766144in}{0.602602in}}%
\pgfpathlineto{\pgfqpoint{0.766700in}{0.602331in}}%
\pgfpathlineto{\pgfqpoint{0.767256in}{0.605052in}}%
\pgfpathlineto{\pgfqpoint{0.767812in}{0.602586in}}%
\pgfpathlineto{\pgfqpoint{0.768368in}{0.601346in}}%
\pgfpathlineto{\pgfqpoint{0.768924in}{0.601742in}}%
\pgfpathlineto{\pgfqpoint{0.770035in}{0.600783in}}%
\pgfpathlineto{\pgfqpoint{0.770591in}{0.604523in}}%
\pgfpathlineto{\pgfqpoint{0.771147in}{0.601885in}}%
\pgfpathlineto{\pgfqpoint{0.771703in}{0.600264in}}%
\pgfpathlineto{\pgfqpoint{0.772259in}{0.600832in}}%
\pgfpathlineto{\pgfqpoint{0.772815in}{0.603532in}}%
\pgfpathlineto{\pgfqpoint{0.773371in}{0.600751in}}%
\pgfpathlineto{\pgfqpoint{0.776150in}{0.602297in}}%
\pgfpathlineto{\pgfqpoint{0.777817in}{0.601455in}}%
\pgfpathlineto{\pgfqpoint{0.778929in}{0.603439in}}%
\pgfpathlineto{\pgfqpoint{0.779485in}{0.600973in}}%
\pgfpathlineto{\pgfqpoint{0.780041in}{0.602016in}}%
\pgfpathlineto{\pgfqpoint{0.780597in}{0.605066in}}%
\pgfpathlineto{\pgfqpoint{0.781153in}{0.603054in}}%
\pgfpathlineto{\pgfqpoint{0.782820in}{0.601115in}}%
\pgfpathlineto{\pgfqpoint{0.783376in}{0.603264in}}%
\pgfpathlineto{\pgfqpoint{0.783932in}{0.602911in}}%
\pgfpathlineto{\pgfqpoint{0.784488in}{0.601280in}}%
\pgfpathlineto{\pgfqpoint{0.785044in}{0.602918in}}%
\pgfpathlineto{\pgfqpoint{0.785599in}{0.602624in}}%
\pgfpathlineto{\pgfqpoint{0.787267in}{0.600302in}}%
\pgfpathlineto{\pgfqpoint{0.788379in}{0.601396in}}%
\pgfpathlineto{\pgfqpoint{0.788935in}{0.600579in}}%
\pgfpathlineto{\pgfqpoint{0.790046in}{0.602653in}}%
\pgfpathlineto{\pgfqpoint{0.791158in}{0.602304in}}%
\pgfpathlineto{\pgfqpoint{0.791714in}{0.600756in}}%
\pgfpathlineto{\pgfqpoint{0.792270in}{0.601299in}}%
\pgfpathlineto{\pgfqpoint{0.793937in}{0.602406in}}%
\pgfpathlineto{\pgfqpoint{0.794493in}{0.601073in}}%
\pgfpathlineto{\pgfqpoint{0.795049in}{0.601917in}}%
\pgfpathlineto{\pgfqpoint{0.795605in}{0.603057in}}%
\pgfpathlineto{\pgfqpoint{0.796161in}{0.602707in}}%
\pgfpathlineto{\pgfqpoint{0.796717in}{0.602669in}}%
\pgfpathlineto{\pgfqpoint{0.797828in}{0.600441in}}%
\pgfpathlineto{\pgfqpoint{0.800052in}{0.604521in}}%
\pgfpathlineto{\pgfqpoint{0.801164in}{0.601067in}}%
\pgfpathlineto{\pgfqpoint{0.802275in}{0.604467in}}%
\pgfpathlineto{\pgfqpoint{0.802831in}{0.602177in}}%
\pgfpathlineto{\pgfqpoint{0.803387in}{0.603820in}}%
\pgfpathlineto{\pgfqpoint{0.805055in}{0.602113in}}%
\pgfpathlineto{\pgfqpoint{0.805610in}{0.600844in}}%
\pgfpathlineto{\pgfqpoint{0.806166in}{0.603444in}}%
\pgfpathlineto{\pgfqpoint{0.806722in}{0.602691in}}%
\pgfpathlineto{\pgfqpoint{0.808390in}{0.600211in}}%
\pgfpathlineto{\pgfqpoint{0.808946in}{0.602591in}}%
\pgfpathlineto{\pgfqpoint{0.809502in}{0.600971in}}%
\pgfpathlineto{\pgfqpoint{0.813393in}{0.602515in}}%
\pgfpathlineto{\pgfqpoint{0.813948in}{0.601057in}}%
\pgfpathlineto{\pgfqpoint{0.815060in}{0.605291in}}%
\pgfpathlineto{\pgfqpoint{0.816728in}{0.600271in}}%
\pgfpathlineto{\pgfqpoint{0.817839in}{0.602962in}}%
\pgfpathlineto{\pgfqpoint{0.818395in}{0.602493in}}%
\pgfpathlineto{\pgfqpoint{0.820063in}{0.603704in}}%
\pgfpathlineto{\pgfqpoint{0.821175in}{0.601167in}}%
\pgfpathlineto{\pgfqpoint{0.821730in}{0.602117in}}%
\pgfpathlineto{\pgfqpoint{0.822286in}{0.601118in}}%
\pgfpathlineto{\pgfqpoint{0.823954in}{0.601690in}}%
\pgfpathlineto{\pgfqpoint{0.825622in}{0.603081in}}%
\pgfpathlineto{\pgfqpoint{0.826733in}{0.600331in}}%
\pgfpathlineto{\pgfqpoint{0.828401in}{0.603384in}}%
\pgfpathlineto{\pgfqpoint{0.828957in}{0.600421in}}%
\pgfpathlineto{\pgfqpoint{0.829513in}{0.601160in}}%
\pgfpathlineto{\pgfqpoint{0.830068in}{0.601078in}}%
\pgfpathlineto{\pgfqpoint{0.830624in}{0.602151in}}%
\pgfpathlineto{\pgfqpoint{0.831180in}{0.601587in}}%
\pgfpathlineto{\pgfqpoint{0.832292in}{0.600640in}}%
\pgfpathlineto{\pgfqpoint{0.834515in}{0.605082in}}%
\pgfpathlineto{\pgfqpoint{0.835627in}{0.600283in}}%
\pgfpathlineto{\pgfqpoint{0.837295in}{0.604435in}}%
\pgfpathlineto{\pgfqpoint{0.838962in}{0.600584in}}%
\pgfpathlineto{\pgfqpoint{0.839518in}{0.603463in}}%
\pgfpathlineto{\pgfqpoint{0.840074in}{0.602979in}}%
\pgfpathlineto{\pgfqpoint{0.841741in}{0.601038in}}%
\pgfpathlineto{\pgfqpoint{0.842297in}{0.603732in}}%
\pgfpathlineto{\pgfqpoint{0.842853in}{0.601988in}}%
\pgfpathlineto{\pgfqpoint{0.843409in}{0.600625in}}%
\pgfpathlineto{\pgfqpoint{0.843965in}{0.601411in}}%
\pgfpathlineto{\pgfqpoint{0.845077in}{0.603486in}}%
\pgfpathlineto{\pgfqpoint{0.845633in}{0.600335in}}%
\pgfpathlineto{\pgfqpoint{0.846188in}{0.603404in}}%
\pgfpathlineto{\pgfqpoint{0.847856in}{0.600712in}}%
\pgfpathlineto{\pgfqpoint{0.848412in}{0.603115in}}%
\pgfpathlineto{\pgfqpoint{0.848968in}{0.602229in}}%
\pgfpathlineto{\pgfqpoint{0.850079in}{0.601147in}}%
\pgfpathlineto{\pgfqpoint{0.851191in}{0.603651in}}%
\pgfpathlineto{\pgfqpoint{0.852303in}{0.601250in}}%
\pgfpathlineto{\pgfqpoint{0.852859in}{0.601631in}}%
\pgfpathlineto{\pgfqpoint{0.854526in}{0.602892in}}%
\pgfpathlineto{\pgfqpoint{0.855082in}{0.601096in}}%
\pgfpathlineto{\pgfqpoint{0.856194in}{0.603259in}}%
\pgfpathlineto{\pgfqpoint{0.858417in}{0.601271in}}%
\pgfpathlineto{\pgfqpoint{0.860085in}{0.602558in}}%
\pgfpathlineto{\pgfqpoint{0.861752in}{0.602289in}}%
\pgfpathlineto{\pgfqpoint{0.863420in}{0.602340in}}%
\pgfpathlineto{\pgfqpoint{0.865088in}{0.600557in}}%
\pgfpathlineto{\pgfqpoint{0.865644in}{0.603336in}}%
\pgfpathlineto{\pgfqpoint{0.866199in}{0.601965in}}%
\pgfpathlineto{\pgfqpoint{0.866755in}{0.602461in}}%
\pgfpathlineto{\pgfqpoint{0.867311in}{0.601453in}}%
\pgfpathlineto{\pgfqpoint{0.868979in}{0.604294in}}%
\pgfpathlineto{\pgfqpoint{0.870090in}{0.601249in}}%
\pgfpathlineto{\pgfqpoint{0.870646in}{0.602332in}}%
\pgfpathlineto{\pgfqpoint{0.872314in}{0.600990in}}%
\pgfpathlineto{\pgfqpoint{0.874537in}{0.602810in}}%
\pgfpathlineto{\pgfqpoint{0.875649in}{0.600295in}}%
\pgfpathlineto{\pgfqpoint{0.876205in}{0.603588in}}%
\pgfpathlineto{\pgfqpoint{0.876761in}{0.602045in}}%
\pgfpathlineto{\pgfqpoint{0.877872in}{0.603745in}}%
\pgfpathlineto{\pgfqpoint{0.880096in}{0.601379in}}%
\pgfpathlineto{\pgfqpoint{0.881208in}{0.602145in}}%
\pgfpathlineto{\pgfqpoint{0.881764in}{0.600429in}}%
\pgfpathlineto{\pgfqpoint{0.882875in}{0.606510in}}%
\pgfpathlineto{\pgfqpoint{0.883431in}{0.601138in}}%
\pgfpathlineto{\pgfqpoint{0.883987in}{0.601400in}}%
\pgfpathlineto{\pgfqpoint{0.885099in}{0.603800in}}%
\pgfpathlineto{\pgfqpoint{0.885655in}{0.602153in}}%
\pgfpathlineto{\pgfqpoint{0.886210in}{0.600853in}}%
\pgfpathlineto{\pgfqpoint{0.886766in}{0.602072in}}%
\pgfpathlineto{\pgfqpoint{0.887322in}{0.601878in}}%
\pgfpathlineto{\pgfqpoint{0.888434in}{0.604283in}}%
\pgfpathlineto{\pgfqpoint{0.888990in}{0.601538in}}%
\pgfpathlineto{\pgfqpoint{0.889546in}{0.602955in}}%
\pgfpathlineto{\pgfqpoint{0.892325in}{0.600678in}}%
\pgfpathlineto{\pgfqpoint{0.892881in}{0.600892in}}%
\pgfpathlineto{\pgfqpoint{0.893437in}{0.603276in}}%
\pgfpathlineto{\pgfqpoint{0.893992in}{0.601769in}}%
\pgfpathlineto{\pgfqpoint{0.895660in}{0.601075in}}%
\pgfpathlineto{\pgfqpoint{0.897328in}{0.603676in}}%
\pgfpathlineto{\pgfqpoint{0.898439in}{0.601391in}}%
\pgfpathlineto{\pgfqpoint{0.898995in}{0.602199in}}%
\pgfpathlineto{\pgfqpoint{0.899551in}{0.602314in}}%
\pgfpathlineto{\pgfqpoint{0.900107in}{0.604728in}}%
\pgfpathlineto{\pgfqpoint{0.900663in}{0.602256in}}%
\pgfpathlineto{\pgfqpoint{0.902330in}{0.603866in}}%
\pgfpathlineto{\pgfqpoint{0.903442in}{0.601199in}}%
\pgfpathlineto{\pgfqpoint{0.903998in}{0.601632in}}%
\pgfpathlineto{\pgfqpoint{0.904554in}{0.603502in}}%
\pgfpathlineto{\pgfqpoint{0.905110in}{0.603227in}}%
\pgfpathlineto{\pgfqpoint{0.907333in}{0.601919in}}%
\pgfpathlineto{\pgfqpoint{0.907889in}{0.603150in}}%
\pgfpathlineto{\pgfqpoint{0.908445in}{0.601128in}}%
\pgfpathlineto{\pgfqpoint{0.909001in}{0.601814in}}%
\pgfpathlineto{\pgfqpoint{0.910668in}{0.604671in}}%
\pgfpathlineto{\pgfqpoint{0.911224in}{0.601492in}}%
\pgfpathlineto{\pgfqpoint{0.911780in}{0.603007in}}%
\pgfpathlineto{\pgfqpoint{0.912336in}{0.602635in}}%
\pgfpathlineto{\pgfqpoint{0.914003in}{0.600080in}}%
\pgfpathlineto{\pgfqpoint{0.915115in}{0.607005in}}%
\pgfpathlineto{\pgfqpoint{0.915671in}{0.601383in}}%
\pgfpathlineto{\pgfqpoint{0.916227in}{0.602679in}}%
\pgfpathlineto{\pgfqpoint{0.916783in}{0.603722in}}%
\pgfpathlineto{\pgfqpoint{0.917339in}{0.603154in}}%
\pgfpathlineto{\pgfqpoint{0.918450in}{0.601081in}}%
\pgfpathlineto{\pgfqpoint{0.919562in}{0.602446in}}%
\pgfpathlineto{\pgfqpoint{0.920118in}{0.600726in}}%
\pgfpathlineto{\pgfqpoint{0.920674in}{0.601685in}}%
\pgfpathlineto{\pgfqpoint{0.922341in}{0.603354in}}%
\pgfpathlineto{\pgfqpoint{0.922897in}{0.600146in}}%
\pgfpathlineto{\pgfqpoint{0.923453in}{0.604208in}}%
\pgfpathlineto{\pgfqpoint{0.924009in}{0.601116in}}%
\pgfpathlineto{\pgfqpoint{0.924565in}{0.602462in}}%
\pgfpathlineto{\pgfqpoint{0.925121in}{0.601749in}}%
\pgfpathlineto{\pgfqpoint{0.925677in}{0.601944in}}%
\pgfpathlineto{\pgfqpoint{0.926232in}{0.607423in}}%
\pgfpathlineto{\pgfqpoint{0.926788in}{0.602211in}}%
\pgfpathlineto{\pgfqpoint{0.927900in}{0.602444in}}%
\pgfpathlineto{\pgfqpoint{0.928456in}{0.603496in}}%
\pgfpathlineto{\pgfqpoint{0.930123in}{0.601801in}}%
\pgfpathlineto{\pgfqpoint{0.930679in}{0.600886in}}%
\pgfpathlineto{\pgfqpoint{0.932347in}{0.605228in}}%
\pgfpathlineto{\pgfqpoint{0.932903in}{0.602523in}}%
\pgfpathlineto{\pgfqpoint{0.933459in}{0.606239in}}%
\pgfpathlineto{\pgfqpoint{0.934014in}{0.601682in}}%
\pgfpathlineto{\pgfqpoint{0.934570in}{0.603989in}}%
\pgfpathlineto{\pgfqpoint{0.935126in}{0.604903in}}%
\pgfpathlineto{\pgfqpoint{0.935682in}{0.603006in}}%
\pgfpathlineto{\pgfqpoint{0.936238in}{0.608115in}}%
\pgfpathlineto{\pgfqpoint{0.936794in}{0.603845in}}%
\pgfpathlineto{\pgfqpoint{0.939017in}{0.602452in}}%
\pgfpathlineto{\pgfqpoint{0.939573in}{0.607212in}}%
\pgfpathlineto{\pgfqpoint{0.940129in}{0.602830in}}%
\pgfpathlineto{\pgfqpoint{0.941241in}{0.601035in}}%
\pgfpathlineto{\pgfqpoint{0.941797in}{0.603747in}}%
\pgfpathlineto{\pgfqpoint{0.942352in}{0.601184in}}%
\pgfpathlineto{\pgfqpoint{0.942908in}{0.601506in}}%
\pgfpathlineto{\pgfqpoint{0.944576in}{0.606567in}}%
\pgfpathlineto{\pgfqpoint{0.945132in}{0.602415in}}%
\pgfpathlineto{\pgfqpoint{0.945688in}{0.605195in}}%
\pgfpathlineto{\pgfqpoint{0.946799in}{0.602756in}}%
\pgfpathlineto{\pgfqpoint{0.947355in}{0.603315in}}%
\pgfpathlineto{\pgfqpoint{0.947911in}{0.602522in}}%
\pgfpathlineto{\pgfqpoint{0.949023in}{0.603421in}}%
\pgfpathlineto{\pgfqpoint{0.949579in}{0.600594in}}%
\pgfpathlineto{\pgfqpoint{0.950134in}{0.603438in}}%
\pgfpathlineto{\pgfqpoint{0.950690in}{0.603642in}}%
\pgfpathlineto{\pgfqpoint{0.952358in}{0.600638in}}%
\pgfpathlineto{\pgfqpoint{0.953470in}{0.601415in}}%
\pgfpathlineto{\pgfqpoint{0.954025in}{0.600454in}}%
\pgfpathlineto{\pgfqpoint{0.954581in}{0.602177in}}%
\pgfpathlineto{\pgfqpoint{0.955137in}{0.601959in}}%
\pgfpathlineto{\pgfqpoint{0.955693in}{0.600767in}}%
\pgfpathlineto{\pgfqpoint{0.956249in}{0.601919in}}%
\pgfpathlineto{\pgfqpoint{0.956805in}{0.604060in}}%
\pgfpathlineto{\pgfqpoint{0.957361in}{0.600806in}}%
\pgfpathlineto{\pgfqpoint{0.957917in}{0.602892in}}%
\pgfpathlineto{\pgfqpoint{0.959584in}{0.602374in}}%
\pgfpathlineto{\pgfqpoint{0.960140in}{0.604834in}}%
\pgfpathlineto{\pgfqpoint{0.960696in}{0.603124in}}%
\pgfpathlineto{\pgfqpoint{0.961252in}{0.601738in}}%
\pgfpathlineto{\pgfqpoint{0.961808in}{0.602557in}}%
\pgfpathlineto{\pgfqpoint{0.962363in}{0.602623in}}%
\pgfpathlineto{\pgfqpoint{0.963475in}{0.604597in}}%
\pgfpathlineto{\pgfqpoint{0.964031in}{0.603960in}}%
\pgfpathlineto{\pgfqpoint{0.964587in}{0.605001in}}%
\pgfpathlineto{\pgfqpoint{0.965143in}{0.600812in}}%
\pgfpathlineto{\pgfqpoint{0.965699in}{0.604818in}}%
\pgfpathlineto{\pgfqpoint{0.966810in}{0.601259in}}%
\pgfpathlineto{\pgfqpoint{0.967366in}{0.601685in}}%
\pgfpathlineto{\pgfqpoint{0.969034in}{0.604928in}}%
\pgfpathlineto{\pgfqpoint{0.970145in}{0.600846in}}%
\pgfpathlineto{\pgfqpoint{0.970701in}{0.602966in}}%
\pgfpathlineto{\pgfqpoint{0.971813in}{0.602145in}}%
\pgfpathlineto{\pgfqpoint{0.972369in}{0.605829in}}%
\pgfpathlineto{\pgfqpoint{0.972925in}{0.602178in}}%
\pgfpathlineto{\pgfqpoint{0.973481in}{0.603822in}}%
\pgfpathlineto{\pgfqpoint{0.974036in}{0.602997in}}%
\pgfpathlineto{\pgfqpoint{0.974592in}{0.600936in}}%
\pgfpathlineto{\pgfqpoint{0.975148in}{0.602406in}}%
\pgfpathlineto{\pgfqpoint{0.976260in}{0.601560in}}%
\pgfpathlineto{\pgfqpoint{0.977928in}{0.602967in}}%
\pgfpathlineto{\pgfqpoint{0.978483in}{0.602082in}}%
\pgfpathlineto{\pgfqpoint{0.980151in}{0.600729in}}%
\pgfpathlineto{\pgfqpoint{0.980707in}{0.603360in}}%
\pgfpathlineto{\pgfqpoint{0.981263in}{0.601780in}}%
\pgfpathlineto{\pgfqpoint{0.982930in}{0.601361in}}%
\pgfpathlineto{\pgfqpoint{0.983486in}{0.605974in}}%
\pgfpathlineto{\pgfqpoint{0.984042in}{0.602800in}}%
\pgfpathlineto{\pgfqpoint{0.985154in}{0.600135in}}%
\pgfpathlineto{\pgfqpoint{0.987377in}{0.605845in}}%
\pgfpathlineto{\pgfqpoint{0.988489in}{0.601006in}}%
\pgfpathlineto{\pgfqpoint{0.989045in}{0.601999in}}%
\pgfpathlineto{\pgfqpoint{0.990156in}{0.602219in}}%
\pgfpathlineto{\pgfqpoint{0.990712in}{0.604941in}}%
\pgfpathlineto{\pgfqpoint{0.991268in}{0.603348in}}%
\pgfpathlineto{\pgfqpoint{0.991824in}{0.603145in}}%
\pgfpathlineto{\pgfqpoint{0.993492in}{0.605891in}}%
\pgfpathlineto{\pgfqpoint{0.994603in}{0.601174in}}%
\pgfpathlineto{\pgfqpoint{0.995159in}{0.601782in}}%
\pgfpathlineto{\pgfqpoint{0.996827in}{0.604315in}}%
\pgfpathlineto{\pgfqpoint{0.997939in}{0.600735in}}%
\pgfpathlineto{\pgfqpoint{0.999606in}{0.603693in}}%
\pgfpathlineto{\pgfqpoint{1.000162in}{0.601748in}}%
\pgfpathlineto{\pgfqpoint{1.000718in}{0.603973in}}%
\pgfpathlineto{\pgfqpoint{1.001274in}{0.603645in}}%
\pgfpathlineto{\pgfqpoint{1.001830in}{0.602335in}}%
\pgfpathlineto{\pgfqpoint{1.002385in}{0.603256in}}%
\pgfpathlineto{\pgfqpoint{1.004609in}{0.603301in}}%
\pgfpathlineto{\pgfqpoint{1.005165in}{0.601002in}}%
\pgfpathlineto{\pgfqpoint{1.005721in}{0.601906in}}%
\pgfpathlineto{\pgfqpoint{1.006276in}{0.602626in}}%
\pgfpathlineto{\pgfqpoint{1.007944in}{0.601524in}}%
\pgfpathlineto{\pgfqpoint{1.008500in}{0.604518in}}%
\pgfpathlineto{\pgfqpoint{1.009056in}{0.602902in}}%
\pgfpathlineto{\pgfqpoint{1.009612in}{0.600809in}}%
\pgfpathlineto{\pgfqpoint{1.010167in}{0.602425in}}%
\pgfpathlineto{\pgfqpoint{1.011279in}{0.601358in}}%
\pgfpathlineto{\pgfqpoint{1.012391in}{0.604699in}}%
\pgfpathlineto{\pgfqpoint{1.012947in}{0.600978in}}%
\pgfpathlineto{\pgfqpoint{1.013503in}{0.603507in}}%
\pgfpathlineto{\pgfqpoint{1.014059in}{0.603361in}}%
\pgfpathlineto{\pgfqpoint{1.015170in}{0.601721in}}%
\pgfpathlineto{\pgfqpoint{1.015726in}{0.603732in}}%
\pgfpathlineto{\pgfqpoint{1.016282in}{0.603421in}}%
\pgfpathlineto{\pgfqpoint{1.016838in}{0.603246in}}%
\pgfpathlineto{\pgfqpoint{1.017394in}{0.601409in}}%
\pgfpathlineto{\pgfqpoint{1.017950in}{0.603654in}}%
\pgfpathlineto{\pgfqpoint{1.018505in}{0.602332in}}%
\pgfpathlineto{\pgfqpoint{1.020173in}{0.602296in}}%
\pgfpathlineto{\pgfqpoint{1.021285in}{0.602103in}}%
\pgfpathlineto{\pgfqpoint{1.021841in}{0.603245in}}%
\pgfpathlineto{\pgfqpoint{1.022396in}{0.600697in}}%
\pgfpathlineto{\pgfqpoint{1.024064in}{0.605196in}}%
\pgfpathlineto{\pgfqpoint{1.025176in}{0.601324in}}%
\pgfpathlineto{\pgfqpoint{1.026843in}{0.604826in}}%
\pgfpathlineto{\pgfqpoint{1.028511in}{0.600854in}}%
\pgfpathlineto{\pgfqpoint{1.029623in}{0.606615in}}%
\pgfpathlineto{\pgfqpoint{1.030178in}{0.603486in}}%
\pgfpathlineto{\pgfqpoint{1.030734in}{0.602142in}}%
\pgfpathlineto{\pgfqpoint{1.032402in}{0.604613in}}%
\pgfpathlineto{\pgfqpoint{1.034625in}{0.600527in}}%
\pgfpathlineto{\pgfqpoint{1.035181in}{0.602362in}}%
\pgfpathlineto{\pgfqpoint{1.035737in}{0.600426in}}%
\pgfpathlineto{\pgfqpoint{1.036849in}{0.602094in}}%
\pgfpathlineto{\pgfqpoint{1.037405in}{0.600225in}}%
\pgfpathlineto{\pgfqpoint{1.037961in}{0.604371in}}%
\pgfpathlineto{\pgfqpoint{1.038516in}{0.601723in}}%
\pgfpathlineto{\pgfqpoint{1.041296in}{0.604772in}}%
\pgfpathlineto{\pgfqpoint{1.041852in}{0.600991in}}%
\pgfpathlineto{\pgfqpoint{1.042407in}{0.602052in}}%
\pgfpathlineto{\pgfqpoint{1.047410in}{0.601263in}}%
\pgfpathlineto{\pgfqpoint{1.047966in}{0.600319in}}%
\pgfpathlineto{\pgfqpoint{1.048522in}{0.605052in}}%
\pgfpathlineto{\pgfqpoint{1.049078in}{0.602745in}}%
\pgfpathlineto{\pgfqpoint{1.050745in}{0.601111in}}%
\pgfpathlineto{\pgfqpoint{1.052413in}{0.606771in}}%
\pgfpathlineto{\pgfqpoint{1.054081in}{0.601251in}}%
\pgfpathlineto{\pgfqpoint{1.054636in}{0.602648in}}%
\pgfpathlineto{\pgfqpoint{1.055192in}{0.601681in}}%
\pgfpathlineto{\pgfqpoint{1.055748in}{0.600888in}}%
\pgfpathlineto{\pgfqpoint{1.056304in}{0.602030in}}%
\pgfpathlineto{\pgfqpoint{1.056860in}{0.603478in}}%
\pgfpathlineto{\pgfqpoint{1.057416in}{0.601967in}}%
\pgfpathlineto{\pgfqpoint{1.059083in}{0.602865in}}%
\pgfpathlineto{\pgfqpoint{1.059639in}{0.602949in}}%
\pgfpathlineto{\pgfqpoint{1.060195in}{0.604486in}}%
\pgfpathlineto{\pgfqpoint{1.061307in}{0.601766in}}%
\pgfpathlineto{\pgfqpoint{1.062974in}{0.603970in}}%
\pgfpathlineto{\pgfqpoint{1.063530in}{0.602423in}}%
\pgfpathlineto{\pgfqpoint{1.064086in}{0.603697in}}%
\pgfpathlineto{\pgfqpoint{1.065754in}{0.603190in}}%
\pgfpathlineto{\pgfqpoint{1.066309in}{0.600975in}}%
\pgfpathlineto{\pgfqpoint{1.066865in}{0.601875in}}%
\pgfpathlineto{\pgfqpoint{1.067977in}{0.601155in}}%
\pgfpathlineto{\pgfqpoint{1.069645in}{0.600958in}}%
\pgfpathlineto{\pgfqpoint{1.070201in}{0.601569in}}%
\pgfpathlineto{\pgfqpoint{1.070756in}{0.604153in}}%
\pgfpathlineto{\pgfqpoint{1.071312in}{0.602419in}}%
\pgfpathlineto{\pgfqpoint{1.071868in}{0.601396in}}%
\pgfpathlineto{\pgfqpoint{1.072424in}{0.603065in}}%
\pgfpathlineto{\pgfqpoint{1.072980in}{0.600835in}}%
\pgfpathlineto{\pgfqpoint{1.073536in}{0.603853in}}%
\pgfpathlineto{\pgfqpoint{1.074092in}{0.602554in}}%
\pgfpathlineto{\pgfqpoint{1.075203in}{0.603607in}}%
\pgfpathlineto{\pgfqpoint{1.076315in}{0.600038in}}%
\pgfpathlineto{\pgfqpoint{1.077983in}{0.602780in}}%
\pgfpathlineto{\pgfqpoint{1.078538in}{0.602762in}}%
\pgfpathlineto{\pgfqpoint{1.079650in}{0.604926in}}%
\pgfpathlineto{\pgfqpoint{1.080206in}{0.601048in}}%
\pgfpathlineto{\pgfqpoint{1.080762in}{0.604375in}}%
\pgfpathlineto{\pgfqpoint{1.081318in}{0.602575in}}%
\pgfpathlineto{\pgfqpoint{1.081874in}{0.603747in}}%
\pgfpathlineto{\pgfqpoint{1.082429in}{0.604179in}}%
\pgfpathlineto{\pgfqpoint{1.082985in}{0.606467in}}%
\pgfpathlineto{\pgfqpoint{1.084097in}{0.600815in}}%
\pgfpathlineto{\pgfqpoint{1.084653in}{0.602297in}}%
\pgfpathlineto{\pgfqpoint{1.085765in}{0.602418in}}%
\pgfpathlineto{\pgfqpoint{1.086320in}{0.600982in}}%
\pgfpathlineto{\pgfqpoint{1.086876in}{0.603766in}}%
\pgfpathlineto{\pgfqpoint{1.087432in}{0.603042in}}%
\pgfpathlineto{\pgfqpoint{1.087988in}{0.601594in}}%
\pgfpathlineto{\pgfqpoint{1.088544in}{0.605325in}}%
\pgfpathlineto{\pgfqpoint{1.089100in}{0.601137in}}%
\pgfpathlineto{\pgfqpoint{1.089656in}{0.601306in}}%
\pgfpathlineto{\pgfqpoint{1.090212in}{0.603582in}}%
\pgfpathlineto{\pgfqpoint{1.090767in}{0.602645in}}%
\pgfpathlineto{\pgfqpoint{1.092435in}{0.602261in}}%
\pgfpathlineto{\pgfqpoint{1.092991in}{0.600853in}}%
\pgfpathlineto{\pgfqpoint{1.093547in}{0.603863in}}%
\pgfpathlineto{\pgfqpoint{1.094103in}{0.601026in}}%
\pgfpathlineto{\pgfqpoint{1.096326in}{0.603523in}}%
\pgfpathlineto{\pgfqpoint{1.097438in}{0.605320in}}%
\pgfpathlineto{\pgfqpoint{1.097994in}{0.605918in}}%
\pgfpathlineto{\pgfqpoint{1.099105in}{0.601754in}}%
\pgfpathlineto{\pgfqpoint{1.099661in}{0.602066in}}%
\pgfpathlineto{\pgfqpoint{1.100217in}{0.602780in}}%
\pgfpathlineto{\pgfqpoint{1.100773in}{0.600553in}}%
\pgfpathlineto{\pgfqpoint{1.101329in}{0.604174in}}%
\pgfpathlineto{\pgfqpoint{1.101885in}{0.601261in}}%
\pgfpathlineto{\pgfqpoint{1.102440in}{0.604020in}}%
\pgfpathlineto{\pgfqpoint{1.102996in}{0.602600in}}%
\pgfpathlineto{\pgfqpoint{1.104664in}{0.600652in}}%
\pgfpathlineto{\pgfqpoint{1.106887in}{0.605196in}}%
\pgfpathlineto{\pgfqpoint{1.107443in}{0.602724in}}%
\pgfpathlineto{\pgfqpoint{1.107999in}{0.603683in}}%
\pgfpathlineto{\pgfqpoint{1.108555in}{0.604131in}}%
\pgfpathlineto{\pgfqpoint{1.110223in}{0.601786in}}%
\pgfpathlineto{\pgfqpoint{1.111890in}{0.603862in}}%
\pgfpathlineto{\pgfqpoint{1.112446in}{0.601329in}}%
\pgfpathlineto{\pgfqpoint{1.113002in}{0.603315in}}%
\pgfpathlineto{\pgfqpoint{1.113558in}{0.602511in}}%
\pgfpathlineto{\pgfqpoint{1.114114in}{0.603976in}}%
\pgfpathlineto{\pgfqpoint{1.114669in}{0.602370in}}%
\pgfpathlineto{\pgfqpoint{1.116337in}{0.604420in}}%
\pgfpathlineto{\pgfqpoint{1.116893in}{0.600689in}}%
\pgfpathlineto{\pgfqpoint{1.117449in}{0.602134in}}%
\pgfpathlineto{\pgfqpoint{1.118005in}{0.604236in}}%
\pgfpathlineto{\pgfqpoint{1.119672in}{0.601083in}}%
\pgfpathlineto{\pgfqpoint{1.120228in}{0.601849in}}%
\pgfpathlineto{\pgfqpoint{1.120784in}{0.600476in}}%
\pgfpathlineto{\pgfqpoint{1.121340in}{0.601228in}}%
\pgfpathlineto{\pgfqpoint{1.123007in}{0.604879in}}%
\pgfpathlineto{\pgfqpoint{1.124119in}{0.601524in}}%
\pgfpathlineto{\pgfqpoint{1.124675in}{0.604213in}}%
\pgfpathlineto{\pgfqpoint{1.125231in}{0.602997in}}%
\pgfpathlineto{\pgfqpoint{1.126898in}{0.600511in}}%
\pgfpathlineto{\pgfqpoint{1.128566in}{0.603467in}}%
\pgfpathlineto{\pgfqpoint{1.130789in}{0.600492in}}%
\pgfpathlineto{\pgfqpoint{1.131901in}{0.603311in}}%
\pgfpathlineto{\pgfqpoint{1.132457in}{0.602129in}}%
\pgfpathlineto{\pgfqpoint{1.133013in}{0.602355in}}%
\pgfpathlineto{\pgfqpoint{1.134680in}{0.600461in}}%
\pgfpathlineto{\pgfqpoint{1.136348in}{0.602843in}}%
\pgfpathlineto{\pgfqpoint{1.139127in}{0.603560in}}%
\pgfpathlineto{\pgfqpoint{1.140239in}{0.601053in}}%
\pgfpathlineto{\pgfqpoint{1.140795in}{0.602732in}}%
\pgfpathlineto{\pgfqpoint{1.141351in}{0.603302in}}%
\pgfpathlineto{\pgfqpoint{1.141907in}{0.609046in}}%
\pgfpathlineto{\pgfqpoint{1.143574in}{0.602174in}}%
\pgfpathlineto{\pgfqpoint{1.144686in}{0.607159in}}%
\pgfpathlineto{\pgfqpoint{1.146354in}{0.601409in}}%
\pgfpathlineto{\pgfqpoint{1.148021in}{0.601077in}}%
\pgfpathlineto{\pgfqpoint{1.148577in}{0.604453in}}%
\pgfpathlineto{\pgfqpoint{1.149133in}{0.601143in}}%
\pgfpathlineto{\pgfqpoint{1.150245in}{0.604754in}}%
\pgfpathlineto{\pgfqpoint{1.150800in}{0.603402in}}%
\pgfpathlineto{\pgfqpoint{1.151356in}{0.603584in}}%
\pgfpathlineto{\pgfqpoint{1.151912in}{0.600444in}}%
\pgfpathlineto{\pgfqpoint{1.152468in}{0.601131in}}%
\pgfpathlineto{\pgfqpoint{1.156359in}{0.605545in}}%
\pgfpathlineto{\pgfqpoint{1.157471in}{0.601381in}}%
\pgfpathlineto{\pgfqpoint{1.158582in}{0.602780in}}%
\pgfpathlineto{\pgfqpoint{1.159138in}{0.601165in}}%
\pgfpathlineto{\pgfqpoint{1.159694in}{0.603674in}}%
\pgfpathlineto{\pgfqpoint{1.160250in}{0.603249in}}%
\pgfpathlineto{\pgfqpoint{1.161362in}{0.600481in}}%
\pgfpathlineto{\pgfqpoint{1.163029in}{0.602587in}}%
\pgfpathlineto{\pgfqpoint{1.163585in}{0.602777in}}%
\pgfpathlineto{\pgfqpoint{1.164697in}{0.604379in}}%
\pgfpathlineto{\pgfqpoint{1.165253in}{0.600353in}}%
\pgfpathlineto{\pgfqpoint{1.165809in}{0.603718in}}%
\pgfpathlineto{\pgfqpoint{1.166920in}{0.604535in}}%
\pgfpathlineto{\pgfqpoint{1.168588in}{0.601213in}}%
\pgfpathlineto{\pgfqpoint{1.169144in}{0.602265in}}%
\pgfpathlineto{\pgfqpoint{1.169700in}{0.601673in}}%
\pgfpathlineto{\pgfqpoint{1.170256in}{0.601038in}}%
\pgfpathlineto{\pgfqpoint{1.171923in}{0.604570in}}%
\pgfpathlineto{\pgfqpoint{1.172479in}{0.602123in}}%
\pgfpathlineto{\pgfqpoint{1.173035in}{0.604973in}}%
\pgfpathlineto{\pgfqpoint{1.173591in}{0.601426in}}%
\pgfpathlineto{\pgfqpoint{1.174147in}{0.602572in}}%
\pgfpathlineto{\pgfqpoint{1.175258in}{0.602634in}}%
\pgfpathlineto{\pgfqpoint{1.175814in}{0.605237in}}%
\pgfpathlineto{\pgfqpoint{1.176370in}{0.603818in}}%
\pgfpathlineto{\pgfqpoint{1.177482in}{0.601610in}}%
\pgfpathlineto{\pgfqpoint{1.178038in}{0.603404in}}%
\pgfpathlineto{\pgfqpoint{1.178593in}{0.601823in}}%
\pgfpathlineto{\pgfqpoint{1.180817in}{0.605940in}}%
\pgfpathlineto{\pgfqpoint{1.182484in}{0.601722in}}%
\pgfpathlineto{\pgfqpoint{1.185820in}{0.602393in}}%
\pgfpathlineto{\pgfqpoint{1.186376in}{0.602790in}}%
\pgfpathlineto{\pgfqpoint{1.186931in}{0.601837in}}%
\pgfpathlineto{\pgfqpoint{1.187487in}{0.602154in}}%
\pgfpathlineto{\pgfqpoint{1.188043in}{0.603780in}}%
\pgfpathlineto{\pgfqpoint{1.189155in}{0.601282in}}%
\pgfpathlineto{\pgfqpoint{1.189711in}{0.601818in}}%
\pgfpathlineto{\pgfqpoint{1.190267in}{0.603925in}}%
\pgfpathlineto{\pgfqpoint{1.190822in}{0.601831in}}%
\pgfpathlineto{\pgfqpoint{1.191934in}{0.601690in}}%
\pgfpathlineto{\pgfqpoint{1.192490in}{0.603299in}}%
\pgfpathlineto{\pgfqpoint{1.193046in}{0.601021in}}%
\pgfpathlineto{\pgfqpoint{1.193602in}{0.604080in}}%
\pgfpathlineto{\pgfqpoint{1.194158in}{0.600409in}}%
\pgfpathlineto{\pgfqpoint{1.194713in}{0.602002in}}%
\pgfpathlineto{\pgfqpoint{1.195269in}{0.601918in}}%
\pgfpathlineto{\pgfqpoint{1.195825in}{0.603206in}}%
\pgfpathlineto{\pgfqpoint{1.196381in}{0.601863in}}%
\pgfpathlineto{\pgfqpoint{1.198049in}{0.604293in}}%
\pgfpathlineto{\pgfqpoint{1.198604in}{0.602074in}}%
\pgfpathlineto{\pgfqpoint{1.199160in}{0.603221in}}%
\pgfpathlineto{\pgfqpoint{1.199716in}{0.604223in}}%
\pgfpathlineto{\pgfqpoint{1.200272in}{0.600883in}}%
\pgfpathlineto{\pgfqpoint{1.200828in}{0.607184in}}%
\pgfpathlineto{\pgfqpoint{1.201384in}{0.601031in}}%
\pgfpathlineto{\pgfqpoint{1.203051in}{0.606635in}}%
\pgfpathlineto{\pgfqpoint{1.204163in}{0.601899in}}%
\pgfpathlineto{\pgfqpoint{1.204719in}{0.602310in}}%
\pgfpathlineto{\pgfqpoint{1.205275in}{0.603566in}}%
\pgfpathlineto{\pgfqpoint{1.205831in}{0.602023in}}%
\pgfpathlineto{\pgfqpoint{1.206387in}{0.602747in}}%
\pgfpathlineto{\pgfqpoint{1.207498in}{0.601879in}}%
\pgfpathlineto{\pgfqpoint{1.208610in}{0.602826in}}%
\pgfpathlineto{\pgfqpoint{1.209166in}{0.601183in}}%
\pgfpathlineto{\pgfqpoint{1.209722in}{0.601702in}}%
\pgfpathlineto{\pgfqpoint{1.210278in}{0.607246in}}%
\pgfpathlineto{\pgfqpoint{1.210833in}{0.606725in}}%
\pgfpathlineto{\pgfqpoint{1.211389in}{0.601152in}}%
\pgfpathlineto{\pgfqpoint{1.211945in}{0.602927in}}%
\pgfpathlineto{\pgfqpoint{1.212501in}{0.602491in}}%
\pgfpathlineto{\pgfqpoint{1.213613in}{0.604603in}}%
\pgfpathlineto{\pgfqpoint{1.214724in}{0.601347in}}%
\pgfpathlineto{\pgfqpoint{1.215280in}{0.602641in}}%
\pgfpathlineto{\pgfqpoint{1.215836in}{0.604404in}}%
\pgfpathlineto{\pgfqpoint{1.216392in}{0.603006in}}%
\pgfpathlineto{\pgfqpoint{1.216948in}{0.601714in}}%
\pgfpathlineto{\pgfqpoint{1.217504in}{0.603668in}}%
\pgfpathlineto{\pgfqpoint{1.218060in}{0.600362in}}%
\pgfpathlineto{\pgfqpoint{1.218615in}{0.602306in}}%
\pgfpathlineto{\pgfqpoint{1.219171in}{0.601732in}}%
\pgfpathlineto{\pgfqpoint{1.219727in}{0.602277in}}%
\pgfpathlineto{\pgfqpoint{1.220283in}{0.603765in}}%
\pgfpathlineto{\pgfqpoint{1.220839in}{0.603085in}}%
\pgfpathlineto{\pgfqpoint{1.223062in}{0.601390in}}%
\pgfpathlineto{\pgfqpoint{1.224174in}{0.604870in}}%
\pgfpathlineto{\pgfqpoint{1.224730in}{0.604197in}}%
\pgfpathlineto{\pgfqpoint{1.226398in}{0.602705in}}%
\pgfpathlineto{\pgfqpoint{1.226953in}{0.604237in}}%
\pgfpathlineto{\pgfqpoint{1.227509in}{0.600301in}}%
\pgfpathlineto{\pgfqpoint{1.228065in}{0.604588in}}%
\pgfpathlineto{\pgfqpoint{1.228621in}{0.601053in}}%
\pgfpathlineto{\pgfqpoint{1.229177in}{0.602474in}}%
\pgfpathlineto{\pgfqpoint{1.229733in}{0.600227in}}%
\pgfpathlineto{\pgfqpoint{1.230289in}{0.603598in}}%
\pgfpathlineto{\pgfqpoint{1.230844in}{0.602816in}}%
\pgfpathlineto{\pgfqpoint{1.231400in}{0.601234in}}%
\pgfpathlineto{\pgfqpoint{1.232512in}{0.605710in}}%
\pgfpathlineto{\pgfqpoint{1.234180in}{0.600319in}}%
\pgfpathlineto{\pgfqpoint{1.234735in}{0.604372in}}%
\pgfpathlineto{\pgfqpoint{1.235291in}{0.601349in}}%
\pgfpathlineto{\pgfqpoint{1.235847in}{0.603839in}}%
\pgfpathlineto{\pgfqpoint{1.236403in}{0.602843in}}%
\pgfpathlineto{\pgfqpoint{1.237515in}{0.604117in}}%
\pgfpathlineto{\pgfqpoint{1.239182in}{0.601938in}}%
\pgfpathlineto{\pgfqpoint{1.240850in}{0.604271in}}%
\pgfpathlineto{\pgfqpoint{1.241406in}{0.602854in}}%
\pgfpathlineto{\pgfqpoint{1.241962in}{0.604503in}}%
\pgfpathlineto{\pgfqpoint{1.242518in}{0.600862in}}%
\pgfpathlineto{\pgfqpoint{1.243073in}{0.601520in}}%
\pgfpathlineto{\pgfqpoint{1.244185in}{0.602972in}}%
\pgfpathlineto{\pgfqpoint{1.244741in}{0.600220in}}%
\pgfpathlineto{\pgfqpoint{1.245297in}{0.602285in}}%
\pgfpathlineto{\pgfqpoint{1.245853in}{0.602936in}}%
\pgfpathlineto{\pgfqpoint{1.246409in}{0.605840in}}%
\pgfpathlineto{\pgfqpoint{1.246964in}{0.602227in}}%
\pgfpathlineto{\pgfqpoint{1.247520in}{0.602358in}}%
\pgfpathlineto{\pgfqpoint{1.248076in}{0.603715in}}%
\pgfpathlineto{\pgfqpoint{1.248632in}{0.600550in}}%
\pgfpathlineto{\pgfqpoint{1.249188in}{0.601832in}}%
\pgfpathlineto{\pgfqpoint{1.250855in}{0.600814in}}%
\pgfpathlineto{\pgfqpoint{1.251411in}{0.602318in}}%
\pgfpathlineto{\pgfqpoint{1.251967in}{0.602093in}}%
\pgfpathlineto{\pgfqpoint{1.252523in}{0.601139in}}%
\pgfpathlineto{\pgfqpoint{1.253079in}{0.605031in}}%
\pgfpathlineto{\pgfqpoint{1.253635in}{0.603789in}}%
\pgfpathlineto{\pgfqpoint{1.255302in}{0.600828in}}%
\pgfpathlineto{\pgfqpoint{1.255858in}{0.601500in}}%
\pgfpathlineto{\pgfqpoint{1.256414in}{0.600605in}}%
\pgfpathlineto{\pgfqpoint{1.256970in}{0.608541in}}%
\pgfpathlineto{\pgfqpoint{1.257526in}{0.608224in}}%
\pgfpathlineto{\pgfqpoint{1.259193in}{0.602556in}}%
\pgfpathlineto{\pgfqpoint{1.259749in}{0.603907in}}%
\pgfpathlineto{\pgfqpoint{1.260305in}{0.601744in}}%
\pgfpathlineto{\pgfqpoint{1.261417in}{0.605858in}}%
\pgfpathlineto{\pgfqpoint{1.261973in}{0.604730in}}%
\pgfpathlineto{\pgfqpoint{1.262529in}{0.602097in}}%
\pgfpathlineto{\pgfqpoint{1.263084in}{0.605025in}}%
\pgfpathlineto{\pgfqpoint{1.263640in}{0.604842in}}%
\pgfpathlineto{\pgfqpoint{1.265308in}{0.600623in}}%
\pgfpathlineto{\pgfqpoint{1.266975in}{0.602653in}}%
\pgfpathlineto{\pgfqpoint{1.267531in}{0.602391in}}%
\pgfpathlineto{\pgfqpoint{1.268087in}{0.606524in}}%
\pgfpathlineto{\pgfqpoint{1.268643in}{0.603854in}}%
\pgfpathlineto{\pgfqpoint{1.269199in}{0.604290in}}%
\pgfpathlineto{\pgfqpoint{1.269755in}{0.602299in}}%
\pgfpathlineto{\pgfqpoint{1.270311in}{0.602967in}}%
\pgfpathlineto{\pgfqpoint{1.270866in}{0.604788in}}%
\pgfpathlineto{\pgfqpoint{1.271978in}{0.600716in}}%
\pgfpathlineto{\pgfqpoint{1.273090in}{0.603836in}}%
\pgfpathlineto{\pgfqpoint{1.273646in}{0.601102in}}%
\pgfpathlineto{\pgfqpoint{1.274202in}{0.602994in}}%
\pgfpathlineto{\pgfqpoint{1.275869in}{0.602041in}}%
\pgfpathlineto{\pgfqpoint{1.276425in}{0.601190in}}%
\pgfpathlineto{\pgfqpoint{1.277537in}{0.604636in}}%
\pgfpathlineto{\pgfqpoint{1.278093in}{0.601914in}}%
\pgfpathlineto{\pgfqpoint{1.278649in}{0.603061in}}%
\pgfpathlineto{\pgfqpoint{1.279204in}{0.603175in}}%
\pgfpathlineto{\pgfqpoint{1.279760in}{0.600804in}}%
\pgfpathlineto{\pgfqpoint{1.280316in}{0.601823in}}%
\pgfpathlineto{\pgfqpoint{1.282540in}{0.605317in}}%
\pgfpathlineto{\pgfqpoint{1.283095in}{0.605630in}}%
\pgfpathlineto{\pgfqpoint{1.284207in}{0.600078in}}%
\pgfpathlineto{\pgfqpoint{1.284763in}{0.601006in}}%
\pgfpathlineto{\pgfqpoint{1.285319in}{0.603909in}}%
\pgfpathlineto{\pgfqpoint{1.285875in}{0.601089in}}%
\pgfpathlineto{\pgfqpoint{1.286431in}{0.600906in}}%
\pgfpathlineto{\pgfqpoint{1.288098in}{0.604163in}}%
\pgfpathlineto{\pgfqpoint{1.288654in}{0.602054in}}%
\pgfpathlineto{\pgfqpoint{1.289210in}{0.603752in}}%
\pgfpathlineto{\pgfqpoint{1.289766in}{0.603228in}}%
\pgfpathlineto{\pgfqpoint{1.290322in}{0.600915in}}%
\pgfpathlineto{\pgfqpoint{1.290877in}{0.602509in}}%
\pgfpathlineto{\pgfqpoint{1.291433in}{0.608593in}}%
\pgfpathlineto{\pgfqpoint{1.291989in}{0.604243in}}%
\pgfpathlineto{\pgfqpoint{1.293657in}{0.600445in}}%
\pgfpathlineto{\pgfqpoint{1.294213in}{0.600923in}}%
\pgfpathlineto{\pgfqpoint{1.294768in}{0.604537in}}%
\pgfpathlineto{\pgfqpoint{1.295324in}{0.602221in}}%
\pgfpathlineto{\pgfqpoint{1.296992in}{0.601756in}}%
\pgfpathlineto{\pgfqpoint{1.298104in}{0.607005in}}%
\pgfpathlineto{\pgfqpoint{1.298660in}{0.600624in}}%
\pgfpathlineto{\pgfqpoint{1.299215in}{0.603516in}}%
\pgfpathlineto{\pgfqpoint{1.300883in}{0.602252in}}%
\pgfpathlineto{\pgfqpoint{1.303106in}{0.601550in}}%
\pgfpathlineto{\pgfqpoint{1.303662in}{0.604910in}}%
\pgfpathlineto{\pgfqpoint{1.304218in}{0.602212in}}%
\pgfpathlineto{\pgfqpoint{1.304774in}{0.600792in}}%
\pgfpathlineto{\pgfqpoint{1.305330in}{0.602397in}}%
\pgfpathlineto{\pgfqpoint{1.305886in}{0.601190in}}%
\pgfpathlineto{\pgfqpoint{1.306442in}{0.603950in}}%
\pgfpathlineto{\pgfqpoint{1.306997in}{0.601646in}}%
\pgfpathlineto{\pgfqpoint{1.307553in}{0.601721in}}%
\pgfpathlineto{\pgfqpoint{1.308665in}{0.604041in}}%
\pgfpathlineto{\pgfqpoint{1.309777in}{0.600507in}}%
\pgfpathlineto{\pgfqpoint{1.310333in}{0.604955in}}%
\pgfpathlineto{\pgfqpoint{1.310888in}{0.601142in}}%
\pgfpathlineto{\pgfqpoint{1.312000in}{0.603039in}}%
\pgfpathlineto{\pgfqpoint{1.312556in}{0.601235in}}%
\pgfpathlineto{\pgfqpoint{1.313112in}{0.601771in}}%
\pgfpathlineto{\pgfqpoint{1.313668in}{0.601502in}}%
\pgfpathlineto{\pgfqpoint{1.314224in}{0.607571in}}%
\pgfpathlineto{\pgfqpoint{1.314779in}{0.603796in}}%
\pgfpathlineto{\pgfqpoint{1.315891in}{0.608167in}}%
\pgfpathlineto{\pgfqpoint{1.317559in}{0.602255in}}%
\pgfpathlineto{\pgfqpoint{1.319226in}{0.604397in}}%
\pgfpathlineto{\pgfqpoint{1.320338in}{0.600683in}}%
\pgfpathlineto{\pgfqpoint{1.320894in}{0.603470in}}%
\pgfpathlineto{\pgfqpoint{1.321450in}{0.603233in}}%
\pgfpathlineto{\pgfqpoint{1.322006in}{0.602121in}}%
\pgfpathlineto{\pgfqpoint{1.322562in}{0.607707in}}%
\pgfpathlineto{\pgfqpoint{1.323117in}{0.603995in}}%
\pgfpathlineto{\pgfqpoint{1.324229in}{0.600515in}}%
\pgfpathlineto{\pgfqpoint{1.324785in}{0.602282in}}%
\pgfpathlineto{\pgfqpoint{1.325897in}{0.606001in}}%
\pgfpathlineto{\pgfqpoint{1.326453in}{0.605094in}}%
\pgfpathlineto{\pgfqpoint{1.327564in}{0.602170in}}%
\pgfpathlineto{\pgfqpoint{1.328120in}{0.602640in}}%
\pgfpathlineto{\pgfqpoint{1.330344in}{0.604209in}}%
\pgfpathlineto{\pgfqpoint{1.332567in}{0.602631in}}%
\pgfpathlineto{\pgfqpoint{1.333123in}{0.602341in}}%
\pgfpathlineto{\pgfqpoint{1.333679in}{0.600424in}}%
\pgfpathlineto{\pgfqpoint{1.334235in}{0.604805in}}%
\pgfpathlineto{\pgfqpoint{1.334791in}{0.601631in}}%
\pgfpathlineto{\pgfqpoint{1.335346in}{0.601583in}}%
\pgfpathlineto{\pgfqpoint{1.337014in}{0.604261in}}%
\pgfpathlineto{\pgfqpoint{1.337570in}{0.603590in}}%
\pgfpathlineto{\pgfqpoint{1.338126in}{0.604993in}}%
\pgfpathlineto{\pgfqpoint{1.338682in}{0.604618in}}%
\pgfpathlineto{\pgfqpoint{1.339237in}{0.604276in}}%
\pgfpathlineto{\pgfqpoint{1.339793in}{0.608477in}}%
\pgfpathlineto{\pgfqpoint{1.340349in}{0.605872in}}%
\pgfpathlineto{\pgfqpoint{1.341461in}{0.605957in}}%
\pgfpathlineto{\pgfqpoint{1.342017in}{0.603649in}}%
\pgfpathlineto{\pgfqpoint{1.342573in}{0.606904in}}%
\pgfpathlineto{\pgfqpoint{1.343128in}{0.601105in}}%
\pgfpathlineto{\pgfqpoint{1.343684in}{0.604383in}}%
\pgfpathlineto{\pgfqpoint{1.344240in}{0.607764in}}%
\pgfpathlineto{\pgfqpoint{1.345908in}{0.603931in}}%
\pgfpathlineto{\pgfqpoint{1.346464in}{0.604164in}}%
\pgfpathlineto{\pgfqpoint{1.347019in}{0.601855in}}%
\pgfpathlineto{\pgfqpoint{1.347575in}{0.602995in}}%
\pgfpathlineto{\pgfqpoint{1.348131in}{0.605365in}}%
\pgfpathlineto{\pgfqpoint{1.348687in}{0.605011in}}%
\pgfpathlineto{\pgfqpoint{1.349799in}{0.600890in}}%
\pgfpathlineto{\pgfqpoint{1.350355in}{0.602739in}}%
\pgfpathlineto{\pgfqpoint{1.350910in}{0.606535in}}%
\pgfpathlineto{\pgfqpoint{1.352578in}{0.601577in}}%
\pgfpathlineto{\pgfqpoint{1.354802in}{0.606189in}}%
\pgfpathlineto{\pgfqpoint{1.355357in}{0.605287in}}%
\pgfpathlineto{\pgfqpoint{1.355913in}{0.600269in}}%
\pgfpathlineto{\pgfqpoint{1.356469in}{0.602365in}}%
\pgfpathlineto{\pgfqpoint{1.357025in}{0.602345in}}%
\pgfpathlineto{\pgfqpoint{1.357581in}{0.601178in}}%
\pgfpathlineto{\pgfqpoint{1.358693in}{0.605371in}}%
\pgfpathlineto{\pgfqpoint{1.360360in}{0.600516in}}%
\pgfpathlineto{\pgfqpoint{1.362028in}{0.606771in}}%
\pgfpathlineto{\pgfqpoint{1.363695in}{0.602250in}}%
\pgfpathlineto{\pgfqpoint{1.364251in}{0.605379in}}%
\pgfpathlineto{\pgfqpoint{1.364807in}{0.600864in}}%
\pgfpathlineto{\pgfqpoint{1.365363in}{0.603277in}}%
\pgfpathlineto{\pgfqpoint{1.366475in}{0.601310in}}%
\pgfpathlineto{\pgfqpoint{1.368142in}{0.607205in}}%
\pgfpathlineto{\pgfqpoint{1.368698in}{0.601997in}}%
\pgfpathlineto{\pgfqpoint{1.369254in}{0.604280in}}%
\pgfpathlineto{\pgfqpoint{1.370366in}{0.600948in}}%
\pgfpathlineto{\pgfqpoint{1.372033in}{0.609016in}}%
\pgfpathlineto{\pgfqpoint{1.374257in}{0.601993in}}%
\pgfpathlineto{\pgfqpoint{1.374813in}{0.603588in}}%
\pgfpathlineto{\pgfqpoint{1.375368in}{0.602847in}}%
\pgfpathlineto{\pgfqpoint{1.375924in}{0.603377in}}%
\pgfpathlineto{\pgfqpoint{1.376480in}{0.605670in}}%
\pgfpathlineto{\pgfqpoint{1.377036in}{0.603359in}}%
\pgfpathlineto{\pgfqpoint{1.377592in}{0.601109in}}%
\pgfpathlineto{\pgfqpoint{1.378148in}{0.604760in}}%
\pgfpathlineto{\pgfqpoint{1.378704in}{0.603855in}}%
\pgfpathlineto{\pgfqpoint{1.380371in}{0.600786in}}%
\pgfpathlineto{\pgfqpoint{1.380927in}{0.604655in}}%
\pgfpathlineto{\pgfqpoint{1.381483in}{0.602929in}}%
\pgfpathlineto{\pgfqpoint{1.382039in}{0.602287in}}%
\pgfpathlineto{\pgfqpoint{1.383150in}{0.611289in}}%
\pgfpathlineto{\pgfqpoint{1.384818in}{0.601896in}}%
\pgfpathlineto{\pgfqpoint{1.387041in}{0.608129in}}%
\pgfpathlineto{\pgfqpoint{1.388709in}{0.602398in}}%
\pgfpathlineto{\pgfqpoint{1.389265in}{0.608436in}}%
\pgfpathlineto{\pgfqpoint{1.389821in}{0.604714in}}%
\pgfpathlineto{\pgfqpoint{1.390377in}{0.601855in}}%
\pgfpathlineto{\pgfqpoint{1.390933in}{0.604767in}}%
\pgfpathlineto{\pgfqpoint{1.391488in}{0.602366in}}%
\pgfpathlineto{\pgfqpoint{1.392044in}{0.604416in}}%
\pgfpathlineto{\pgfqpoint{1.392600in}{0.602701in}}%
\pgfpathlineto{\pgfqpoint{1.393156in}{0.606999in}}%
\pgfpathlineto{\pgfqpoint{1.393712in}{0.605873in}}%
\pgfpathlineto{\pgfqpoint{1.394268in}{0.605749in}}%
\pgfpathlineto{\pgfqpoint{1.394824in}{0.600206in}}%
\pgfpathlineto{\pgfqpoint{1.395379in}{0.605233in}}%
\pgfpathlineto{\pgfqpoint{1.395935in}{0.601205in}}%
\pgfpathlineto{\pgfqpoint{1.397603in}{0.607355in}}%
\pgfpathlineto{\pgfqpoint{1.398159in}{0.607588in}}%
\pgfpathlineto{\pgfqpoint{1.398715in}{0.602350in}}%
\pgfpathlineto{\pgfqpoint{1.399270in}{0.605436in}}%
\pgfpathlineto{\pgfqpoint{1.400382in}{0.609539in}}%
\pgfpathlineto{\pgfqpoint{1.400938in}{0.600723in}}%
\pgfpathlineto{\pgfqpoint{1.401494in}{0.604858in}}%
\pgfpathlineto{\pgfqpoint{1.402050in}{0.601763in}}%
\pgfpathlineto{\pgfqpoint{1.402606in}{0.602410in}}%
\pgfpathlineto{\pgfqpoint{1.403161in}{0.607674in}}%
\pgfpathlineto{\pgfqpoint{1.403717in}{0.603076in}}%
\pgfpathlineto{\pgfqpoint{1.404829in}{0.607133in}}%
\pgfpathlineto{\pgfqpoint{1.405385in}{0.605744in}}%
\pgfpathlineto{\pgfqpoint{1.406497in}{0.602574in}}%
\pgfpathlineto{\pgfqpoint{1.407052in}{0.603386in}}%
\pgfpathlineto{\pgfqpoint{1.407608in}{0.602885in}}%
\pgfpathlineto{\pgfqpoint{1.408164in}{0.607111in}}%
\pgfpathlineto{\pgfqpoint{1.408720in}{0.602135in}}%
\pgfpathlineto{\pgfqpoint{1.409276in}{0.603341in}}%
\pgfpathlineto{\pgfqpoint{1.410388in}{0.603856in}}%
\pgfpathlineto{\pgfqpoint{1.411499in}{0.607961in}}%
\pgfpathlineto{\pgfqpoint{1.412055in}{0.603233in}}%
\pgfpathlineto{\pgfqpoint{1.412611in}{0.604149in}}%
\pgfpathlineto{\pgfqpoint{1.413167in}{0.604482in}}%
\pgfpathlineto{\pgfqpoint{1.413723in}{0.600671in}}%
\pgfpathlineto{\pgfqpoint{1.414279in}{0.605433in}}%
\pgfpathlineto{\pgfqpoint{1.414835in}{0.603611in}}%
\pgfpathlineto{\pgfqpoint{1.416502in}{0.605608in}}%
\pgfpathlineto{\pgfqpoint{1.417058in}{0.601523in}}%
\pgfpathlineto{\pgfqpoint{1.417614in}{0.603377in}}%
\pgfpathlineto{\pgfqpoint{1.418726in}{0.611531in}}%
\pgfpathlineto{\pgfqpoint{1.419281in}{0.601631in}}%
\pgfpathlineto{\pgfqpoint{1.419837in}{0.603135in}}%
\pgfpathlineto{\pgfqpoint{1.420393in}{0.601716in}}%
\pgfpathlineto{\pgfqpoint{1.421505in}{0.606252in}}%
\pgfpathlineto{\pgfqpoint{1.422061in}{0.600740in}}%
\pgfpathlineto{\pgfqpoint{1.422617in}{0.603421in}}%
\pgfpathlineto{\pgfqpoint{1.423172in}{0.605900in}}%
\pgfpathlineto{\pgfqpoint{1.424284in}{0.600747in}}%
\pgfpathlineto{\pgfqpoint{1.424840in}{0.605884in}}%
\pgfpathlineto{\pgfqpoint{1.425396in}{0.605694in}}%
\pgfpathlineto{\pgfqpoint{1.425952in}{0.603167in}}%
\pgfpathlineto{\pgfqpoint{1.426508in}{0.605328in}}%
\pgfpathlineto{\pgfqpoint{1.427619in}{0.601406in}}%
\pgfpathlineto{\pgfqpoint{1.429287in}{0.609580in}}%
\pgfpathlineto{\pgfqpoint{1.430955in}{0.604070in}}%
\pgfpathlineto{\pgfqpoint{1.431510in}{0.606839in}}%
\pgfpathlineto{\pgfqpoint{1.432066in}{0.602061in}}%
\pgfpathlineto{\pgfqpoint{1.432622in}{0.604736in}}%
\pgfpathlineto{\pgfqpoint{1.433178in}{0.607615in}}%
\pgfpathlineto{\pgfqpoint{1.433734in}{0.604830in}}%
\pgfpathlineto{\pgfqpoint{1.434290in}{0.600746in}}%
\pgfpathlineto{\pgfqpoint{1.434846in}{0.601252in}}%
\pgfpathlineto{\pgfqpoint{1.435957in}{0.605738in}}%
\pgfpathlineto{\pgfqpoint{1.436513in}{0.603574in}}%
\pgfpathlineto{\pgfqpoint{1.437069in}{0.601753in}}%
\pgfpathlineto{\pgfqpoint{1.437625in}{0.606987in}}%
\pgfpathlineto{\pgfqpoint{1.438181in}{0.605167in}}%
\pgfpathlineto{\pgfqpoint{1.439292in}{0.603531in}}%
\pgfpathlineto{\pgfqpoint{1.441516in}{0.610793in}}%
\pgfpathlineto{\pgfqpoint{1.443183in}{0.603178in}}%
\pgfpathlineto{\pgfqpoint{1.443739in}{0.602351in}}%
\pgfpathlineto{\pgfqpoint{1.445407in}{0.609085in}}%
\pgfpathlineto{\pgfqpoint{1.445963in}{0.603535in}}%
\pgfpathlineto{\pgfqpoint{1.446519in}{0.606029in}}%
\pgfpathlineto{\pgfqpoint{1.447075in}{0.605736in}}%
\pgfpathlineto{\pgfqpoint{1.447630in}{0.601604in}}%
\pgfpathlineto{\pgfqpoint{1.448186in}{0.607093in}}%
\pgfpathlineto{\pgfqpoint{1.448742in}{0.602869in}}%
\pgfpathlineto{\pgfqpoint{1.450410in}{0.603640in}}%
\pgfpathlineto{\pgfqpoint{1.450966in}{0.601998in}}%
\pgfpathlineto{\pgfqpoint{1.452633in}{0.607246in}}%
\pgfpathlineto{\pgfqpoint{1.453189in}{0.604171in}}%
\pgfpathlineto{\pgfqpoint{1.453745in}{0.605567in}}%
\pgfpathlineto{\pgfqpoint{1.454301in}{0.610100in}}%
\pgfpathlineto{\pgfqpoint{1.454857in}{0.605835in}}%
\pgfpathlineto{\pgfqpoint{1.455412in}{0.608098in}}%
\pgfpathlineto{\pgfqpoint{1.455968in}{0.602775in}}%
\pgfpathlineto{\pgfqpoint{1.456524in}{0.608419in}}%
\pgfpathlineto{\pgfqpoint{1.457080in}{0.602068in}}%
\pgfpathlineto{\pgfqpoint{1.457636in}{0.611223in}}%
\pgfpathlineto{\pgfqpoint{1.458192in}{0.605349in}}%
\pgfpathlineto{\pgfqpoint{1.458748in}{0.605103in}}%
\pgfpathlineto{\pgfqpoint{1.459303in}{0.606939in}}%
\pgfpathlineto{\pgfqpoint{1.459859in}{0.604696in}}%
\pgfpathlineto{\pgfqpoint{1.460415in}{0.609771in}}%
\pgfpathlineto{\pgfqpoint{1.460971in}{0.604956in}}%
\pgfpathlineto{\pgfqpoint{1.461527in}{0.606997in}}%
\pgfpathlineto{\pgfqpoint{1.462083in}{0.605077in}}%
\pgfpathlineto{\pgfqpoint{1.462639in}{0.605409in}}%
\pgfpathlineto{\pgfqpoint{1.463194in}{0.603114in}}%
\pgfpathlineto{\pgfqpoint{1.463750in}{0.603630in}}%
\pgfpathlineto{\pgfqpoint{1.465418in}{0.606819in}}%
\pgfpathlineto{\pgfqpoint{1.465974in}{0.606455in}}%
\pgfpathlineto{\pgfqpoint{1.466530in}{0.600934in}}%
\pgfpathlineto{\pgfqpoint{1.467086in}{0.603058in}}%
\pgfpathlineto{\pgfqpoint{1.467641in}{0.606383in}}%
\pgfpathlineto{\pgfqpoint{1.468197in}{0.605602in}}%
\pgfpathlineto{\pgfqpoint{1.469309in}{0.600995in}}%
\pgfpathlineto{\pgfqpoint{1.470421in}{0.610276in}}%
\pgfpathlineto{\pgfqpoint{1.470977in}{0.604673in}}%
\pgfpathlineto{\pgfqpoint{1.471532in}{0.608350in}}%
\pgfpathlineto{\pgfqpoint{1.473200in}{0.601538in}}%
\pgfpathlineto{\pgfqpoint{1.473756in}{0.606399in}}%
\pgfpathlineto{\pgfqpoint{1.474312in}{0.604998in}}%
\pgfpathlineto{\pgfqpoint{1.475423in}{0.602836in}}%
\pgfpathlineto{\pgfqpoint{1.475979in}{0.606862in}}%
\pgfpathlineto{\pgfqpoint{1.476535in}{0.602803in}}%
\pgfpathlineto{\pgfqpoint{1.477091in}{0.601216in}}%
\pgfpathlineto{\pgfqpoint{1.478203in}{0.609441in}}%
\pgfpathlineto{\pgfqpoint{1.478759in}{0.608065in}}%
\pgfpathlineto{\pgfqpoint{1.479870in}{0.600819in}}%
\pgfpathlineto{\pgfqpoint{1.480982in}{0.606878in}}%
\pgfpathlineto{\pgfqpoint{1.482094in}{0.603172in}}%
\pgfpathlineto{\pgfqpoint{1.484317in}{0.610803in}}%
\pgfpathlineto{\pgfqpoint{1.484873in}{0.601811in}}%
\pgfpathlineto{\pgfqpoint{1.485429in}{0.605872in}}%
\pgfpathlineto{\pgfqpoint{1.487097in}{0.610532in}}%
\pgfpathlineto{\pgfqpoint{1.489320in}{0.604521in}}%
\pgfpathlineto{\pgfqpoint{1.490988in}{0.607484in}}%
\pgfpathlineto{\pgfqpoint{1.491543in}{0.603103in}}%
\pgfpathlineto{\pgfqpoint{1.492099in}{0.604482in}}%
\pgfpathlineto{\pgfqpoint{1.494323in}{0.601758in}}%
\pgfpathlineto{\pgfqpoint{1.495990in}{0.609489in}}%
\pgfpathlineto{\pgfqpoint{1.496546in}{0.602018in}}%
\pgfpathlineto{\pgfqpoint{1.497102in}{0.604153in}}%
\pgfpathlineto{\pgfqpoint{1.498214in}{0.610143in}}%
\pgfpathlineto{\pgfqpoint{1.499881in}{0.601355in}}%
\pgfpathlineto{\pgfqpoint{1.500993in}{0.608090in}}%
\pgfpathlineto{\pgfqpoint{1.501549in}{0.605804in}}%
\pgfpathlineto{\pgfqpoint{1.502105in}{0.610773in}}%
\pgfpathlineto{\pgfqpoint{1.502661in}{0.607492in}}%
\pgfpathlineto{\pgfqpoint{1.503216in}{0.603464in}}%
\pgfpathlineto{\pgfqpoint{1.503772in}{0.607299in}}%
\pgfpathlineto{\pgfqpoint{1.505440in}{0.609886in}}%
\pgfpathlineto{\pgfqpoint{1.507108in}{0.602960in}}%
\pgfpathlineto{\pgfqpoint{1.508219in}{0.604125in}}%
\pgfpathlineto{\pgfqpoint{1.508775in}{0.605281in}}%
\pgfpathlineto{\pgfqpoint{1.509331in}{0.602472in}}%
\pgfpathlineto{\pgfqpoint{1.510999in}{0.608326in}}%
\pgfpathlineto{\pgfqpoint{1.511554in}{0.608069in}}%
\pgfpathlineto{\pgfqpoint{1.512110in}{0.604505in}}%
\pgfpathlineto{\pgfqpoint{1.512666in}{0.612809in}}%
\pgfpathlineto{\pgfqpoint{1.513222in}{0.606373in}}%
\pgfpathlineto{\pgfqpoint{1.513778in}{0.609925in}}%
\pgfpathlineto{\pgfqpoint{1.514334in}{0.602080in}}%
\pgfpathlineto{\pgfqpoint{1.514890in}{0.610199in}}%
\pgfpathlineto{\pgfqpoint{1.515445in}{0.606997in}}%
\pgfpathlineto{\pgfqpoint{1.516001in}{0.601024in}}%
\pgfpathlineto{\pgfqpoint{1.516557in}{0.606967in}}%
\pgfpathlineto{\pgfqpoint{1.518225in}{0.603766in}}%
\pgfpathlineto{\pgfqpoint{1.518781in}{0.603846in}}%
\pgfpathlineto{\pgfqpoint{1.519892in}{0.608149in}}%
\pgfpathlineto{\pgfqpoint{1.520448in}{0.605407in}}%
\pgfpathlineto{\pgfqpoint{1.521004in}{0.608849in}}%
\pgfpathlineto{\pgfqpoint{1.521560in}{0.603954in}}%
\pgfpathlineto{\pgfqpoint{1.522116in}{0.604898in}}%
\pgfpathlineto{\pgfqpoint{1.522672in}{0.607235in}}%
\pgfpathlineto{\pgfqpoint{1.523228in}{0.603798in}}%
\pgfpathlineto{\pgfqpoint{1.523783in}{0.607106in}}%
\pgfpathlineto{\pgfqpoint{1.524895in}{0.603453in}}%
\pgfpathlineto{\pgfqpoint{1.526563in}{0.606575in}}%
\pgfpathlineto{\pgfqpoint{1.527119in}{0.607111in}}%
\pgfpathlineto{\pgfqpoint{1.527674in}{0.617607in}}%
\pgfpathlineto{\pgfqpoint{1.528230in}{0.603719in}}%
\pgfpathlineto{\pgfqpoint{1.528786in}{0.604159in}}%
\pgfpathlineto{\pgfqpoint{1.529342in}{0.609461in}}%
\pgfpathlineto{\pgfqpoint{1.529898in}{0.605366in}}%
\pgfpathlineto{\pgfqpoint{1.530454in}{0.603422in}}%
\pgfpathlineto{\pgfqpoint{1.531010in}{0.605496in}}%
\pgfpathlineto{\pgfqpoint{1.531565in}{0.611165in}}%
\pgfpathlineto{\pgfqpoint{1.532121in}{0.601528in}}%
\pgfpathlineto{\pgfqpoint{1.532677in}{0.608977in}}%
\pgfpathlineto{\pgfqpoint{1.533789in}{0.603343in}}%
\pgfpathlineto{\pgfqpoint{1.534345in}{0.604399in}}%
\pgfpathlineto{\pgfqpoint{1.534901in}{0.612033in}}%
\pgfpathlineto{\pgfqpoint{1.535456in}{0.608964in}}%
\pgfpathlineto{\pgfqpoint{1.536012in}{0.608316in}}%
\pgfpathlineto{\pgfqpoint{1.536568in}{0.600837in}}%
\pgfpathlineto{\pgfqpoint{1.537124in}{0.606120in}}%
\pgfpathlineto{\pgfqpoint{1.537680in}{0.602894in}}%
\pgfpathlineto{\pgfqpoint{1.538236in}{0.605191in}}%
\pgfpathlineto{\pgfqpoint{1.539347in}{0.602139in}}%
\pgfpathlineto{\pgfqpoint{1.539903in}{0.606477in}}%
\pgfpathlineto{\pgfqpoint{1.540459in}{0.603334in}}%
\pgfpathlineto{\pgfqpoint{1.541015in}{0.602678in}}%
\pgfpathlineto{\pgfqpoint{1.542683in}{0.608562in}}%
\pgfpathlineto{\pgfqpoint{1.543239in}{0.611705in}}%
\pgfpathlineto{\pgfqpoint{1.543794in}{0.603501in}}%
\pgfpathlineto{\pgfqpoint{1.544350in}{0.613138in}}%
\pgfpathlineto{\pgfqpoint{1.544906in}{0.612240in}}%
\pgfpathlineto{\pgfqpoint{1.546018in}{0.605427in}}%
\pgfpathlineto{\pgfqpoint{1.546574in}{0.605858in}}%
\pgfpathlineto{\pgfqpoint{1.547685in}{0.602247in}}%
\pgfpathlineto{\pgfqpoint{1.548241in}{0.612615in}}%
\pgfpathlineto{\pgfqpoint{1.548797in}{0.605324in}}%
\pgfpathlineto{\pgfqpoint{1.550465in}{0.603124in}}%
\pgfpathlineto{\pgfqpoint{1.552132in}{0.607298in}}%
\pgfpathlineto{\pgfqpoint{1.553800in}{0.610641in}}%
\pgfpathlineto{\pgfqpoint{1.554356in}{0.602905in}}%
\pgfpathlineto{\pgfqpoint{1.554912in}{0.604541in}}%
\pgfpathlineto{\pgfqpoint{1.556023in}{0.611373in}}%
\pgfpathlineto{\pgfqpoint{1.556579in}{0.607339in}}%
\pgfpathlineto{\pgfqpoint{1.557691in}{0.601037in}}%
\pgfpathlineto{\pgfqpoint{1.558247in}{0.602814in}}%
\pgfpathlineto{\pgfqpoint{1.558803in}{0.601064in}}%
\pgfpathlineto{\pgfqpoint{1.559358in}{0.601379in}}%
\pgfpathlineto{\pgfqpoint{1.560470in}{0.610407in}}%
\pgfpathlineto{\pgfqpoint{1.561026in}{0.601719in}}%
\pgfpathlineto{\pgfqpoint{1.561582in}{0.609656in}}%
\pgfpathlineto{\pgfqpoint{1.562694in}{0.602749in}}%
\pgfpathlineto{\pgfqpoint{1.563250in}{0.606519in}}%
\pgfpathlineto{\pgfqpoint{1.563805in}{0.606407in}}%
\pgfpathlineto{\pgfqpoint{1.564361in}{0.605613in}}%
\pgfpathlineto{\pgfqpoint{1.564917in}{0.613155in}}%
\pgfpathlineto{\pgfqpoint{1.565473in}{0.608706in}}%
\pgfpathlineto{\pgfqpoint{1.566029in}{0.611234in}}%
\pgfpathlineto{\pgfqpoint{1.567696in}{0.601753in}}%
\pgfpathlineto{\pgfqpoint{1.568252in}{0.608729in}}%
\pgfpathlineto{\pgfqpoint{1.568808in}{0.603795in}}%
\pgfpathlineto{\pgfqpoint{1.569364in}{0.608494in}}%
\pgfpathlineto{\pgfqpoint{1.569920in}{0.605398in}}%
\pgfpathlineto{\pgfqpoint{1.571587in}{0.605906in}}%
\pgfpathlineto{\pgfqpoint{1.572143in}{0.613315in}}%
\pgfpathlineto{\pgfqpoint{1.572699in}{0.611933in}}%
\pgfpathlineto{\pgfqpoint{1.573255in}{0.601422in}}%
\pgfpathlineto{\pgfqpoint{1.573811in}{0.607541in}}%
\pgfpathlineto{\pgfqpoint{1.574367in}{0.608161in}}%
\pgfpathlineto{\pgfqpoint{1.576034in}{0.604957in}}%
\pgfpathlineto{\pgfqpoint{1.576590in}{0.605346in}}%
\pgfpathlineto{\pgfqpoint{1.577146in}{0.611283in}}%
\pgfpathlineto{\pgfqpoint{1.577702in}{0.607035in}}%
\pgfpathlineto{\pgfqpoint{1.578258in}{0.604301in}}%
\pgfpathlineto{\pgfqpoint{1.578814in}{0.610653in}}%
\pgfpathlineto{\pgfqpoint{1.579370in}{0.604622in}}%
\pgfpathlineto{\pgfqpoint{1.579925in}{0.603830in}}%
\pgfpathlineto{\pgfqpoint{1.580481in}{0.607306in}}%
\pgfpathlineto{\pgfqpoint{1.581037in}{0.606676in}}%
\pgfpathlineto{\pgfqpoint{1.581593in}{0.603814in}}%
\pgfpathlineto{\pgfqpoint{1.582149in}{0.607163in}}%
\pgfpathlineto{\pgfqpoint{1.582705in}{0.602092in}}%
\pgfpathlineto{\pgfqpoint{1.583261in}{0.602454in}}%
\pgfpathlineto{\pgfqpoint{1.583816in}{0.607920in}}%
\pgfpathlineto{\pgfqpoint{1.584372in}{0.603209in}}%
\pgfpathlineto{\pgfqpoint{1.584928in}{0.606138in}}%
\pgfpathlineto{\pgfqpoint{1.585484in}{0.603648in}}%
\pgfpathlineto{\pgfqpoint{1.586040in}{0.600813in}}%
\pgfpathlineto{\pgfqpoint{1.586596in}{0.603631in}}%
\pgfpathlineto{\pgfqpoint{1.587152in}{0.609424in}}%
\pgfpathlineto{\pgfqpoint{1.587707in}{0.606961in}}%
\pgfpathlineto{\pgfqpoint{1.588263in}{0.603904in}}%
\pgfpathlineto{\pgfqpoint{1.588819in}{0.609363in}}%
\pgfpathlineto{\pgfqpoint{1.589375in}{0.601861in}}%
\pgfpathlineto{\pgfqpoint{1.589931in}{0.603566in}}%
\pgfpathlineto{\pgfqpoint{1.593266in}{0.605839in}}%
\pgfpathlineto{\pgfqpoint{1.594378in}{0.611367in}}%
\pgfpathlineto{\pgfqpoint{1.595489in}{0.604067in}}%
\pgfpathlineto{\pgfqpoint{1.596045in}{0.604425in}}%
\pgfpathlineto{\pgfqpoint{1.596601in}{0.606497in}}%
\pgfpathlineto{\pgfqpoint{1.597157in}{0.603804in}}%
\pgfpathlineto{\pgfqpoint{1.597713in}{0.604882in}}%
\pgfpathlineto{\pgfqpoint{1.598825in}{0.604238in}}%
\pgfpathlineto{\pgfqpoint{1.599381in}{0.610565in}}%
\pgfpathlineto{\pgfqpoint{1.599936in}{0.603800in}}%
\pgfpathlineto{\pgfqpoint{1.600492in}{0.609023in}}%
\pgfpathlineto{\pgfqpoint{1.601048in}{0.610715in}}%
\pgfpathlineto{\pgfqpoint{1.602716in}{0.601958in}}%
\pgfpathlineto{\pgfqpoint{1.603827in}{0.607229in}}%
\pgfpathlineto{\pgfqpoint{1.604383in}{0.607029in}}%
\pgfpathlineto{\pgfqpoint{1.604939in}{0.603856in}}%
\pgfpathlineto{\pgfqpoint{1.606051in}{0.609280in}}%
\pgfpathlineto{\pgfqpoint{1.606607in}{0.602206in}}%
\pgfpathlineto{\pgfqpoint{1.607163in}{0.604390in}}%
\pgfpathlineto{\pgfqpoint{1.607718in}{0.604488in}}%
\pgfpathlineto{\pgfqpoint{1.608274in}{0.608679in}}%
\pgfpathlineto{\pgfqpoint{1.608830in}{0.604464in}}%
\pgfpathlineto{\pgfqpoint{1.609942in}{0.603767in}}%
\pgfpathlineto{\pgfqpoint{1.611054in}{0.608975in}}%
\pgfpathlineto{\pgfqpoint{1.611609in}{0.603272in}}%
\pgfpathlineto{\pgfqpoint{1.612165in}{0.603380in}}%
\pgfpathlineto{\pgfqpoint{1.612721in}{0.609602in}}%
\pgfpathlineto{\pgfqpoint{1.614389in}{0.601406in}}%
\pgfpathlineto{\pgfqpoint{1.615500in}{0.604064in}}%
\pgfpathlineto{\pgfqpoint{1.616056in}{0.603303in}}%
\pgfpathlineto{\pgfqpoint{1.616612in}{0.600114in}}%
\pgfpathlineto{\pgfqpoint{1.617168in}{0.606377in}}%
\pgfpathlineto{\pgfqpoint{1.617724in}{0.604261in}}%
\pgfpathlineto{\pgfqpoint{1.618280in}{0.606301in}}%
\pgfpathlineto{\pgfqpoint{1.618836in}{0.603997in}}%
\pgfpathlineto{\pgfqpoint{1.619392in}{0.606736in}}%
\pgfpathlineto{\pgfqpoint{1.619947in}{0.605314in}}%
\pgfpathlineto{\pgfqpoint{1.620503in}{0.604353in}}%
\pgfpathlineto{\pgfqpoint{1.621059in}{0.605178in}}%
\pgfpathlineto{\pgfqpoint{1.623283in}{0.602006in}}%
\pgfpathlineto{\pgfqpoint{1.624950in}{0.613276in}}%
\pgfpathlineto{\pgfqpoint{1.626062in}{0.602008in}}%
\pgfpathlineto{\pgfqpoint{1.626618in}{0.606077in}}%
\pgfpathlineto{\pgfqpoint{1.627174in}{0.606443in}}%
\pgfpathlineto{\pgfqpoint{1.628285in}{0.601910in}}%
\pgfpathlineto{\pgfqpoint{1.629953in}{0.611689in}}%
\pgfpathlineto{\pgfqpoint{1.630509in}{0.600604in}}%
\pgfpathlineto{\pgfqpoint{1.631065in}{0.608153in}}%
\pgfpathlineto{\pgfqpoint{1.631620in}{0.609765in}}%
\pgfpathlineto{\pgfqpoint{1.632176in}{0.601594in}}%
\pgfpathlineto{\pgfqpoint{1.632732in}{0.606435in}}%
\pgfpathlineto{\pgfqpoint{1.633288in}{0.604019in}}%
\pgfpathlineto{\pgfqpoint{1.633844in}{0.605241in}}%
\pgfpathlineto{\pgfqpoint{1.634956in}{0.604245in}}%
\pgfpathlineto{\pgfqpoint{1.635512in}{0.601423in}}%
\pgfpathlineto{\pgfqpoint{1.636067in}{0.605839in}}%
\pgfpathlineto{\pgfqpoint{1.636623in}{0.600680in}}%
\pgfpathlineto{\pgfqpoint{1.637179in}{0.601634in}}%
\pgfpathlineto{\pgfqpoint{1.638291in}{0.609631in}}%
\pgfpathlineto{\pgfqpoint{1.639403in}{0.601935in}}%
\pgfpathlineto{\pgfqpoint{1.639958in}{0.606718in}}%
\pgfpathlineto{\pgfqpoint{1.640514in}{0.602764in}}%
\pgfpathlineto{\pgfqpoint{1.641626in}{0.604198in}}%
\pgfpathlineto{\pgfqpoint{1.642182in}{0.601689in}}%
\pgfpathlineto{\pgfqpoint{1.642738in}{0.602882in}}%
\pgfpathlineto{\pgfqpoint{1.644961in}{0.603996in}}%
\pgfpathlineto{\pgfqpoint{1.645517in}{0.600567in}}%
\pgfpathlineto{\pgfqpoint{1.646073in}{0.604329in}}%
\pgfpathlineto{\pgfqpoint{1.646629in}{0.601224in}}%
\pgfpathlineto{\pgfqpoint{1.647740in}{0.605909in}}%
\pgfpathlineto{\pgfqpoint{1.648296in}{0.604198in}}%
\pgfpathlineto{\pgfqpoint{1.648852in}{0.601389in}}%
\pgfpathlineto{\pgfqpoint{1.649408in}{0.602591in}}%
\pgfpathlineto{\pgfqpoint{1.649964in}{0.604309in}}%
\pgfpathlineto{\pgfqpoint{1.650520in}{0.601702in}}%
\pgfpathlineto{\pgfqpoint{1.651076in}{0.607161in}}%
\pgfpathlineto{\pgfqpoint{1.651631in}{0.604508in}}%
\pgfpathlineto{\pgfqpoint{1.652187in}{0.603976in}}%
\pgfpathlineto{\pgfqpoint{1.652743in}{0.601671in}}%
\pgfpathlineto{\pgfqpoint{1.653299in}{0.604304in}}%
\pgfpathlineto{\pgfqpoint{1.653855in}{0.602892in}}%
\pgfpathlineto{\pgfqpoint{1.654967in}{0.603548in}}%
\pgfpathlineto{\pgfqpoint{1.656078in}{0.600871in}}%
\pgfpathlineto{\pgfqpoint{1.657190in}{0.606819in}}%
\pgfpathlineto{\pgfqpoint{1.657746in}{0.605498in}}%
\pgfpathlineto{\pgfqpoint{1.658302in}{0.605425in}}%
\pgfpathlineto{\pgfqpoint{1.658858in}{0.601845in}}%
\pgfpathlineto{\pgfqpoint{1.659414in}{0.603948in}}%
\pgfpathlineto{\pgfqpoint{1.659969in}{0.605015in}}%
\pgfpathlineto{\pgfqpoint{1.660525in}{0.604376in}}%
\pgfpathlineto{\pgfqpoint{1.661637in}{0.601585in}}%
\pgfpathlineto{\pgfqpoint{1.663305in}{0.605352in}}%
\pgfpathlineto{\pgfqpoint{1.664972in}{0.601929in}}%
\pgfpathlineto{\pgfqpoint{1.666084in}{0.603297in}}%
\pgfpathlineto{\pgfqpoint{1.666640in}{0.600813in}}%
\pgfpathlineto{\pgfqpoint{1.667196in}{0.602602in}}%
\pgfpathlineto{\pgfqpoint{1.667751in}{0.602596in}}%
\pgfpathlineto{\pgfqpoint{1.668307in}{0.604650in}}%
\pgfpathlineto{\pgfqpoint{1.669975in}{0.602360in}}%
\pgfpathlineto{\pgfqpoint{1.670531in}{0.603887in}}%
\pgfpathlineto{\pgfqpoint{1.671087in}{0.602932in}}%
\pgfpathlineto{\pgfqpoint{1.672198in}{0.602529in}}%
\pgfpathlineto{\pgfqpoint{1.672754in}{0.603601in}}%
\pgfpathlineto{\pgfqpoint{1.673310in}{0.600067in}}%
\pgfpathlineto{\pgfqpoint{1.673866in}{0.600584in}}%
\pgfpathlineto{\pgfqpoint{1.675534in}{0.602817in}}%
\pgfpathlineto{\pgfqpoint{1.676089in}{0.600662in}}%
\pgfpathlineto{\pgfqpoint{1.676645in}{0.603833in}}%
\pgfpathlineto{\pgfqpoint{1.677201in}{0.602265in}}%
\pgfpathlineto{\pgfqpoint{1.677757in}{0.602754in}}%
\pgfpathlineto{\pgfqpoint{1.678313in}{0.601241in}}%
\pgfpathlineto{\pgfqpoint{1.678869in}{0.601683in}}%
\pgfpathlineto{\pgfqpoint{1.681092in}{0.602522in}}%
\pgfpathlineto{\pgfqpoint{1.681648in}{0.602755in}}%
\pgfpathlineto{\pgfqpoint{1.682204in}{0.600701in}}%
\pgfpathlineto{\pgfqpoint{1.682760in}{0.600873in}}%
\pgfpathlineto{\pgfqpoint{1.683316in}{0.603352in}}%
\pgfpathlineto{\pgfqpoint{1.683871in}{0.603129in}}%
\pgfpathlineto{\pgfqpoint{1.685539in}{0.602085in}}%
\pgfpathlineto{\pgfqpoint{1.686095in}{0.605263in}}%
\pgfpathlineto{\pgfqpoint{1.687762in}{0.601633in}}%
\pgfpathlineto{\pgfqpoint{1.688318in}{0.600774in}}%
\pgfpathlineto{\pgfqpoint{1.688874in}{0.605385in}}%
\pgfpathlineto{\pgfqpoint{1.689430in}{0.601908in}}%
\pgfpathlineto{\pgfqpoint{1.689986in}{0.602712in}}%
\pgfpathlineto{\pgfqpoint{1.690542in}{0.601873in}}%
\pgfpathlineto{\pgfqpoint{1.691653in}{0.600994in}}%
\pgfpathlineto{\pgfqpoint{1.692209in}{0.602681in}}%
\pgfpathlineto{\pgfqpoint{1.692765in}{0.601986in}}%
\pgfpathlineto{\pgfqpoint{1.694433in}{0.600919in}}%
\pgfpathlineto{\pgfqpoint{1.695545in}{0.602685in}}%
\pgfpathlineto{\pgfqpoint{1.696656in}{0.600436in}}%
\pgfpathlineto{\pgfqpoint{1.697212in}{0.600610in}}%
\pgfpathlineto{\pgfqpoint{1.697768in}{0.602978in}}%
\pgfpathlineto{\pgfqpoint{1.698880in}{0.600216in}}%
\pgfpathlineto{\pgfqpoint{1.699436in}{0.601740in}}%
\pgfpathlineto{\pgfqpoint{1.699991in}{0.600421in}}%
\pgfpathlineto{\pgfqpoint{1.702215in}{0.601596in}}%
\pgfpathlineto{\pgfqpoint{1.702771in}{0.600659in}}%
\pgfpathlineto{\pgfqpoint{1.703327in}{0.601970in}}%
\pgfpathlineto{\pgfqpoint{1.703882in}{0.601221in}}%
\pgfpathlineto{\pgfqpoint{1.704438in}{0.601725in}}%
\pgfpathlineto{\pgfqpoint{1.704994in}{0.600968in}}%
\pgfpathlineto{\pgfqpoint{1.707218in}{0.600408in}}%
\pgfpathlineto{\pgfqpoint{1.708329in}{0.600713in}}%
\pgfpathlineto{\pgfqpoint{1.709997in}{0.600500in}}%
\pgfpathlineto{\pgfqpoint{1.710553in}{0.602279in}}%
\pgfpathlineto{\pgfqpoint{1.711109in}{0.600543in}}%
\pgfpathlineto{\pgfqpoint{1.711665in}{0.601056in}}%
\pgfpathlineto{\pgfqpoint{1.712220in}{0.600339in}}%
\pgfpathlineto{\pgfqpoint{1.719447in}{0.600960in}}%
\pgfpathlineto{\pgfqpoint{1.720002in}{0.600037in}}%
\pgfpathlineto{\pgfqpoint{1.720558in}{0.601656in}}%
\pgfpathlineto{\pgfqpoint{1.721114in}{0.600837in}}%
\pgfpathlineto{\pgfqpoint{1.725005in}{0.600502in}}%
\pgfpathlineto{\pgfqpoint{1.726117in}{0.600218in}}%
\pgfpathlineto{\pgfqpoint{1.728340in}{0.600050in}}%
\pgfpathlineto{\pgfqpoint{1.728896in}{0.600840in}}%
\pgfpathlineto{\pgfqpoint{1.729452in}{0.600195in}}%
\pgfpathlineto{\pgfqpoint{1.732231in}{0.600346in}}%
\pgfpathlineto{\pgfqpoint{1.734455in}{0.600175in}}%
\pgfpathlineto{\pgfqpoint{1.736678in}{0.600325in}}%
\pgfpathlineto{\pgfqpoint{1.748907in}{0.600294in}}%
\pgfpathlineto{\pgfqpoint{1.752242in}{0.600572in}}%
\pgfpathlineto{\pgfqpoint{1.754466in}{0.601214in}}%
\pgfpathlineto{\pgfqpoint{1.756133in}{0.607502in}}%
\pgfpathlineto{\pgfqpoint{1.757245in}{0.602458in}}%
\pgfpathlineto{\pgfqpoint{1.757801in}{0.607310in}}%
\pgfpathlineto{\pgfqpoint{1.758357in}{0.600668in}}%
\pgfpathlineto{\pgfqpoint{1.758913in}{0.602189in}}%
\pgfpathlineto{\pgfqpoint{1.761136in}{0.600497in}}%
\pgfpathlineto{\pgfqpoint{1.777812in}{0.600407in}}%
\pgfpathlineto{\pgfqpoint{1.780591in}{0.600513in}}%
\pgfpathlineto{\pgfqpoint{1.782259in}{0.600396in}}%
\pgfpathlineto{\pgfqpoint{1.786150in}{0.600671in}}%
\pgfpathlineto{\pgfqpoint{1.787818in}{0.600276in}}%
\pgfpathlineto{\pgfqpoint{1.797267in}{0.600845in}}%
\pgfpathlineto{\pgfqpoint{1.797823in}{0.599926in}}%
\pgfpathlineto{\pgfqpoint{1.798379in}{0.600834in}}%
\pgfpathlineto{\pgfqpoint{1.800046in}{0.600693in}}%
\pgfpathlineto{\pgfqpoint{1.800602in}{0.600250in}}%
\pgfpathlineto{\pgfqpoint{1.802270in}{0.601639in}}%
\pgfpathlineto{\pgfqpoint{1.802826in}{0.600569in}}%
\pgfpathlineto{\pgfqpoint{1.803382in}{0.601905in}}%
\pgfpathlineto{\pgfqpoint{1.803937in}{0.602869in}}%
\pgfpathlineto{\pgfqpoint{1.805605in}{0.600448in}}%
\pgfpathlineto{\pgfqpoint{1.806717in}{0.601933in}}%
\pgfpathlineto{\pgfqpoint{1.808940in}{0.600805in}}%
\pgfpathlineto{\pgfqpoint{1.810608in}{0.601607in}}%
\pgfpathlineto{\pgfqpoint{1.814499in}{0.601605in}}%
\pgfpathlineto{\pgfqpoint{1.815055in}{0.601850in}}%
\pgfpathlineto{\pgfqpoint{1.816722in}{0.600266in}}%
\pgfpathlineto{\pgfqpoint{1.817834in}{0.601469in}}%
\pgfpathlineto{\pgfqpoint{1.818390in}{0.600765in}}%
\pgfpathlineto{\pgfqpoint{1.822281in}{0.601893in}}%
\pgfpathlineto{\pgfqpoint{1.822837in}{0.600772in}}%
\pgfpathlineto{\pgfqpoint{1.823393in}{0.601991in}}%
\pgfpathlineto{\pgfqpoint{1.825060in}{0.600387in}}%
\pgfpathlineto{\pgfqpoint{1.825616in}{0.603576in}}%
\pgfpathlineto{\pgfqpoint{1.826172in}{0.600445in}}%
\pgfpathlineto{\pgfqpoint{1.827840in}{0.603242in}}%
\pgfpathlineto{\pgfqpoint{1.829507in}{0.602607in}}%
\pgfpathlineto{\pgfqpoint{1.830063in}{0.603922in}}%
\pgfpathlineto{\pgfqpoint{1.830619in}{0.600434in}}%
\pgfpathlineto{\pgfqpoint{1.831175in}{0.602337in}}%
\pgfpathlineto{\pgfqpoint{1.831731in}{0.601652in}}%
\pgfpathlineto{\pgfqpoint{1.832286in}{0.602177in}}%
\pgfpathlineto{\pgfqpoint{1.833398in}{0.603891in}}%
\pgfpathlineto{\pgfqpoint{1.833954in}{0.601080in}}%
\pgfpathlineto{\pgfqpoint{1.834510in}{0.603492in}}%
\pgfpathlineto{\pgfqpoint{1.835066in}{0.601279in}}%
\pgfpathlineto{\pgfqpoint{1.836177in}{0.604165in}}%
\pgfpathlineto{\pgfqpoint{1.836733in}{0.603572in}}%
\pgfpathlineto{\pgfqpoint{1.837289in}{0.600357in}}%
\pgfpathlineto{\pgfqpoint{1.837845in}{0.601052in}}%
\pgfpathlineto{\pgfqpoint{1.838401in}{0.602802in}}%
\pgfpathlineto{\pgfqpoint{1.838957in}{0.601458in}}%
\pgfpathlineto{\pgfqpoint{1.840624in}{0.604324in}}%
\pgfpathlineto{\pgfqpoint{1.841736in}{0.601147in}}%
\pgfpathlineto{\pgfqpoint{1.842292in}{0.601318in}}%
\pgfpathlineto{\pgfqpoint{1.842848in}{0.604107in}}%
\pgfpathlineto{\pgfqpoint{1.843404in}{0.601625in}}%
\pgfpathlineto{\pgfqpoint{1.843960in}{0.603893in}}%
\pgfpathlineto{\pgfqpoint{1.844515in}{0.602780in}}%
\pgfpathlineto{\pgfqpoint{1.845071in}{0.602439in}}%
\pgfpathlineto{\pgfqpoint{1.845627in}{0.599991in}}%
\pgfpathlineto{\pgfqpoint{1.846183in}{0.601253in}}%
\pgfpathlineto{\pgfqpoint{1.846739in}{0.601102in}}%
\pgfpathlineto{\pgfqpoint{1.847295in}{0.602356in}}%
\pgfpathlineto{\pgfqpoint{1.847851in}{0.601445in}}%
\pgfpathlineto{\pgfqpoint{1.848406in}{0.601604in}}%
\pgfpathlineto{\pgfqpoint{1.850074in}{0.605094in}}%
\pgfpathlineto{\pgfqpoint{1.850630in}{0.601030in}}%
\pgfpathlineto{\pgfqpoint{1.851186in}{0.602182in}}%
\pgfpathlineto{\pgfqpoint{1.852297in}{0.604616in}}%
\pgfpathlineto{\pgfqpoint{1.853409in}{0.601477in}}%
\pgfpathlineto{\pgfqpoint{1.853965in}{0.603515in}}%
\pgfpathlineto{\pgfqpoint{1.854521in}{0.601264in}}%
\pgfpathlineto{\pgfqpoint{1.855077in}{0.603406in}}%
\pgfpathlineto{\pgfqpoint{1.856188in}{0.601530in}}%
\pgfpathlineto{\pgfqpoint{1.858412in}{0.603872in}}%
\pgfpathlineto{\pgfqpoint{1.858968in}{0.602753in}}%
\pgfpathlineto{\pgfqpoint{1.859524in}{0.604575in}}%
\pgfpathlineto{\pgfqpoint{1.860079in}{0.604071in}}%
\pgfpathlineto{\pgfqpoint{1.860635in}{0.602990in}}%
\pgfpathlineto{\pgfqpoint{1.861191in}{0.607042in}}%
\pgfpathlineto{\pgfqpoint{1.861747in}{0.605514in}}%
\pgfpathlineto{\pgfqpoint{1.862303in}{0.605178in}}%
\pgfpathlineto{\pgfqpoint{1.863415in}{0.608419in}}%
\pgfpathlineto{\pgfqpoint{1.863971in}{0.602025in}}%
\pgfpathlineto{\pgfqpoint{1.864526in}{0.606085in}}%
\pgfpathlineto{\pgfqpoint{1.865082in}{0.607179in}}%
\pgfpathlineto{\pgfqpoint{1.865638in}{0.606849in}}%
\pgfpathlineto{\pgfqpoint{1.866194in}{0.602618in}}%
\pgfpathlineto{\pgfqpoint{1.867862in}{0.609583in}}%
\pgfpathlineto{\pgfqpoint{1.868973in}{0.604869in}}%
\pgfpathlineto{\pgfqpoint{1.869529in}{0.605173in}}%
\pgfpathlineto{\pgfqpoint{1.870085in}{0.604553in}}%
\pgfpathlineto{\pgfqpoint{1.870641in}{0.608322in}}%
\pgfpathlineto{\pgfqpoint{1.871197in}{0.603375in}}%
\pgfpathlineto{\pgfqpoint{1.871753in}{0.606110in}}%
\pgfpathlineto{\pgfqpoint{1.872864in}{0.603247in}}%
\pgfpathlineto{\pgfqpoint{1.874532in}{0.610401in}}%
\pgfpathlineto{\pgfqpoint{1.875644in}{0.601561in}}%
\pgfpathlineto{\pgfqpoint{1.878423in}{0.609191in}}%
\pgfpathlineto{\pgfqpoint{1.878979in}{0.602005in}}%
\pgfpathlineto{\pgfqpoint{1.879535in}{0.609187in}}%
\pgfpathlineto{\pgfqpoint{1.881202in}{0.604305in}}%
\pgfpathlineto{\pgfqpoint{1.881758in}{0.605451in}}%
\pgfpathlineto{\pgfqpoint{1.882870in}{0.603514in}}%
\pgfpathlineto{\pgfqpoint{1.883982in}{0.607551in}}%
\pgfpathlineto{\pgfqpoint{1.884537in}{0.603207in}}%
\pgfpathlineto{\pgfqpoint{1.885093in}{0.612612in}}%
\pgfpathlineto{\pgfqpoint{1.885649in}{0.607032in}}%
\pgfpathlineto{\pgfqpoint{1.886205in}{0.605454in}}%
\pgfpathlineto{\pgfqpoint{1.887317in}{0.611328in}}%
\pgfpathlineto{\pgfqpoint{1.888984in}{0.602051in}}%
\pgfpathlineto{\pgfqpoint{1.890096in}{0.605905in}}%
\pgfpathlineto{\pgfqpoint{1.890652in}{0.613805in}}%
\pgfpathlineto{\pgfqpoint{1.891208in}{0.607016in}}%
\pgfpathlineto{\pgfqpoint{1.891764in}{0.610331in}}%
\pgfpathlineto{\pgfqpoint{1.892319in}{0.605987in}}%
\pgfpathlineto{\pgfqpoint{1.892875in}{0.608860in}}%
\pgfpathlineto{\pgfqpoint{1.893431in}{0.608531in}}%
\pgfpathlineto{\pgfqpoint{1.893987in}{0.612047in}}%
\pgfpathlineto{\pgfqpoint{1.895099in}{0.604017in}}%
\pgfpathlineto{\pgfqpoint{1.895655in}{0.607406in}}%
\pgfpathlineto{\pgfqpoint{1.896210in}{0.605801in}}%
\pgfpathlineto{\pgfqpoint{1.896766in}{0.602698in}}%
\pgfpathlineto{\pgfqpoint{1.897878in}{0.608026in}}%
\pgfpathlineto{\pgfqpoint{1.899546in}{0.602119in}}%
\pgfpathlineto{\pgfqpoint{1.901213in}{0.606986in}}%
\pgfpathlineto{\pgfqpoint{1.901769in}{0.601294in}}%
\pgfpathlineto{\pgfqpoint{1.902325in}{0.608290in}}%
\pgfpathlineto{\pgfqpoint{1.902881in}{0.605491in}}%
\pgfpathlineto{\pgfqpoint{1.903993in}{0.606687in}}%
\pgfpathlineto{\pgfqpoint{1.905104in}{0.601570in}}%
\pgfpathlineto{\pgfqpoint{1.907328in}{0.609894in}}%
\pgfpathlineto{\pgfqpoint{1.907884in}{0.605382in}}%
\pgfpathlineto{\pgfqpoint{1.908439in}{0.608983in}}%
\pgfpathlineto{\pgfqpoint{1.908995in}{0.605954in}}%
\pgfpathlineto{\pgfqpoint{1.909551in}{0.610824in}}%
\pgfpathlineto{\pgfqpoint{1.910107in}{0.603001in}}%
\pgfpathlineto{\pgfqpoint{1.910663in}{0.606091in}}%
\pgfpathlineto{\pgfqpoint{1.911219in}{0.607371in}}%
\pgfpathlineto{\pgfqpoint{1.912886in}{0.603149in}}%
\pgfpathlineto{\pgfqpoint{1.913998in}{0.602319in}}%
\pgfpathlineto{\pgfqpoint{1.915666in}{0.609728in}}%
\pgfpathlineto{\pgfqpoint{1.916221in}{0.602078in}}%
\pgfpathlineto{\pgfqpoint{1.916777in}{0.603592in}}%
\pgfpathlineto{\pgfqpoint{1.919001in}{0.613604in}}%
\pgfpathlineto{\pgfqpoint{1.919557in}{0.605894in}}%
\pgfpathlineto{\pgfqpoint{1.920113in}{0.607433in}}%
\pgfpathlineto{\pgfqpoint{1.921780in}{0.615616in}}%
\pgfpathlineto{\pgfqpoint{1.923448in}{0.605966in}}%
\pgfpathlineto{\pgfqpoint{1.924559in}{0.614022in}}%
\pgfpathlineto{\pgfqpoint{1.925115in}{0.601983in}}%
\pgfpathlineto{\pgfqpoint{1.925671in}{0.603372in}}%
\pgfpathlineto{\pgfqpoint{1.926783in}{0.619099in}}%
\pgfpathlineto{\pgfqpoint{1.927339in}{0.606536in}}%
\pgfpathlineto{\pgfqpoint{1.927895in}{0.613151in}}%
\pgfpathlineto{\pgfqpoint{1.929562in}{0.605441in}}%
\pgfpathlineto{\pgfqpoint{1.930118in}{0.605446in}}%
\pgfpathlineto{\pgfqpoint{1.930674in}{0.602431in}}%
\pgfpathlineto{\pgfqpoint{1.931230in}{0.607960in}}%
\pgfpathlineto{\pgfqpoint{1.932341in}{0.602193in}}%
\pgfpathlineto{\pgfqpoint{1.934009in}{0.613361in}}%
\pgfpathlineto{\pgfqpoint{1.935121in}{0.603554in}}%
\pgfpathlineto{\pgfqpoint{1.935677in}{0.603943in}}%
\pgfpathlineto{\pgfqpoint{1.936788in}{0.609148in}}%
\pgfpathlineto{\pgfqpoint{1.937344in}{0.602169in}}%
\pgfpathlineto{\pgfqpoint{1.937900in}{0.613483in}}%
\pgfpathlineto{\pgfqpoint{1.938456in}{0.605024in}}%
\pgfpathlineto{\pgfqpoint{1.939012in}{0.605255in}}%
\pgfpathlineto{\pgfqpoint{1.939568in}{0.606833in}}%
\pgfpathlineto{\pgfqpoint{1.940124in}{0.612169in}}%
\pgfpathlineto{\pgfqpoint{1.940679in}{0.606012in}}%
\pgfpathlineto{\pgfqpoint{1.941235in}{0.606909in}}%
\pgfpathlineto{\pgfqpoint{1.944015in}{0.615460in}}%
\pgfpathlineto{\pgfqpoint{1.944570in}{0.601133in}}%
\pgfpathlineto{\pgfqpoint{1.945126in}{0.610469in}}%
\pgfpathlineto{\pgfqpoint{1.946794in}{0.601076in}}%
\pgfpathlineto{\pgfqpoint{1.948461in}{0.619682in}}%
\pgfpathlineto{\pgfqpoint{1.949017in}{0.604011in}}%
\pgfpathlineto{\pgfqpoint{1.949573in}{0.610865in}}%
\pgfpathlineto{\pgfqpoint{1.950129in}{0.609693in}}%
\pgfpathlineto{\pgfqpoint{1.950685in}{0.603703in}}%
\pgfpathlineto{\pgfqpoint{1.951241in}{0.615616in}}%
\pgfpathlineto{\pgfqpoint{1.951797in}{0.611614in}}%
\pgfpathlineto{\pgfqpoint{1.952352in}{0.602396in}}%
\pgfpathlineto{\pgfqpoint{1.952908in}{0.615580in}}%
\pgfpathlineto{\pgfqpoint{1.953464in}{0.603937in}}%
\pgfpathlineto{\pgfqpoint{1.954020in}{0.602628in}}%
\pgfpathlineto{\pgfqpoint{1.955132in}{0.613398in}}%
\pgfpathlineto{\pgfqpoint{1.955688in}{0.609917in}}%
\pgfpathlineto{\pgfqpoint{1.956244in}{0.601159in}}%
\pgfpathlineto{\pgfqpoint{1.956799in}{0.604280in}}%
\pgfpathlineto{\pgfqpoint{1.957355in}{0.606557in}}%
\pgfpathlineto{\pgfqpoint{1.957911in}{0.604871in}}%
\pgfpathlineto{\pgfqpoint{1.958467in}{0.605813in}}%
\pgfpathlineto{\pgfqpoint{1.959023in}{0.602434in}}%
\pgfpathlineto{\pgfqpoint{1.959579in}{0.605241in}}%
\pgfpathlineto{\pgfqpoint{1.960135in}{0.606587in}}%
\pgfpathlineto{\pgfqpoint{1.961246in}{0.600959in}}%
\pgfpathlineto{\pgfqpoint{1.962358in}{0.606179in}}%
\pgfpathlineto{\pgfqpoint{1.962914in}{0.602933in}}%
\pgfpathlineto{\pgfqpoint{1.965137in}{0.610966in}}%
\pgfpathlineto{\pgfqpoint{1.966249in}{0.601613in}}%
\pgfpathlineto{\pgfqpoint{1.966805in}{0.606881in}}%
\pgfpathlineto{\pgfqpoint{1.967361in}{0.606086in}}%
\pgfpathlineto{\pgfqpoint{1.967917in}{0.603466in}}%
\pgfpathlineto{\pgfqpoint{1.968472in}{0.608840in}}%
\pgfpathlineto{\pgfqpoint{1.969028in}{0.606449in}}%
\pgfpathlineto{\pgfqpoint{1.970696in}{0.603069in}}%
\pgfpathlineto{\pgfqpoint{1.971252in}{0.601522in}}%
\pgfpathlineto{\pgfqpoint{1.972363in}{0.610449in}}%
\pgfpathlineto{\pgfqpoint{1.972919in}{0.606231in}}%
\pgfpathlineto{\pgfqpoint{1.973475in}{0.605257in}}%
\pgfpathlineto{\pgfqpoint{1.974031in}{0.607698in}}%
\pgfpathlineto{\pgfqpoint{1.974587in}{0.601288in}}%
\pgfpathlineto{\pgfqpoint{1.975143in}{0.601566in}}%
\pgfpathlineto{\pgfqpoint{1.976810in}{0.612345in}}%
\pgfpathlineto{\pgfqpoint{1.977366in}{0.602049in}}%
\pgfpathlineto{\pgfqpoint{1.977922in}{0.606589in}}%
\pgfpathlineto{\pgfqpoint{1.978478in}{0.613106in}}%
\pgfpathlineto{\pgfqpoint{1.979034in}{0.603345in}}%
\pgfpathlineto{\pgfqpoint{1.979590in}{0.612829in}}%
\pgfpathlineto{\pgfqpoint{1.980701in}{0.604495in}}%
\pgfpathlineto{\pgfqpoint{1.981257in}{0.611397in}}%
\pgfpathlineto{\pgfqpoint{1.981813in}{0.610401in}}%
\pgfpathlineto{\pgfqpoint{1.982369in}{0.611118in}}%
\pgfpathlineto{\pgfqpoint{1.982925in}{0.610592in}}%
\pgfpathlineto{\pgfqpoint{1.983481in}{0.603283in}}%
\pgfpathlineto{\pgfqpoint{1.984037in}{0.613056in}}%
\pgfpathlineto{\pgfqpoint{1.984592in}{0.606324in}}%
\pgfpathlineto{\pgfqpoint{1.985148in}{0.605372in}}%
\pgfpathlineto{\pgfqpoint{1.985704in}{0.615734in}}%
\pgfpathlineto{\pgfqpoint{1.986260in}{0.614722in}}%
\pgfpathlineto{\pgfqpoint{1.986816in}{0.606032in}}%
\pgfpathlineto{\pgfqpoint{1.987372in}{0.610132in}}%
\pgfpathlineto{\pgfqpoint{1.987928in}{0.609274in}}%
\pgfpathlineto{\pgfqpoint{1.988483in}{0.602192in}}%
\pgfpathlineto{\pgfqpoint{1.989039in}{0.602973in}}%
\pgfpathlineto{\pgfqpoint{1.989595in}{0.603807in}}%
\pgfpathlineto{\pgfqpoint{1.990151in}{0.608841in}}%
\pgfpathlineto{\pgfqpoint{1.990707in}{0.600674in}}%
\pgfpathlineto{\pgfqpoint{1.991263in}{0.603780in}}%
\pgfpathlineto{\pgfqpoint{1.991819in}{0.602473in}}%
\pgfpathlineto{\pgfqpoint{1.992930in}{0.612935in}}%
\pgfpathlineto{\pgfqpoint{1.993486in}{0.602444in}}%
\pgfpathlineto{\pgfqpoint{1.994042in}{0.607511in}}%
\pgfpathlineto{\pgfqpoint{1.994598in}{0.605400in}}%
\pgfpathlineto{\pgfqpoint{1.995710in}{0.610685in}}%
\pgfpathlineto{\pgfqpoint{1.996266in}{0.603082in}}%
\pgfpathlineto{\pgfqpoint{1.996821in}{0.603178in}}%
\pgfpathlineto{\pgfqpoint{1.998489in}{0.605896in}}%
\pgfpathlineto{\pgfqpoint{1.999045in}{0.602913in}}%
\pgfpathlineto{\pgfqpoint{2.000712in}{0.615287in}}%
\pgfpathlineto{\pgfqpoint{2.001268in}{0.600607in}}%
\pgfpathlineto{\pgfqpoint{2.001824in}{0.608700in}}%
\pgfpathlineto{\pgfqpoint{2.002380in}{0.603168in}}%
\pgfpathlineto{\pgfqpoint{2.002936in}{0.614178in}}%
\pgfpathlineto{\pgfqpoint{2.003492in}{0.606476in}}%
\pgfpathlineto{\pgfqpoint{2.005715in}{0.613258in}}%
\pgfpathlineto{\pgfqpoint{2.006271in}{0.604871in}}%
\pgfpathlineto{\pgfqpoint{2.006827in}{0.606522in}}%
\pgfpathlineto{\pgfqpoint{2.007383in}{0.608590in}}%
\pgfpathlineto{\pgfqpoint{2.007939in}{0.608103in}}%
\pgfpathlineto{\pgfqpoint{2.009050in}{0.607019in}}%
\pgfpathlineto{\pgfqpoint{2.009606in}{0.602913in}}%
\pgfpathlineto{\pgfqpoint{2.010162in}{0.612655in}}%
\pgfpathlineto{\pgfqpoint{2.010718in}{0.609456in}}%
\pgfpathlineto{\pgfqpoint{2.011274in}{0.608645in}}%
\pgfpathlineto{\pgfqpoint{2.011830in}{0.604305in}}%
\pgfpathlineto{\pgfqpoint{2.012386in}{0.606781in}}%
\pgfpathlineto{\pgfqpoint{2.012941in}{0.610316in}}%
\pgfpathlineto{\pgfqpoint{2.014609in}{0.601160in}}%
\pgfpathlineto{\pgfqpoint{2.016277in}{0.605437in}}%
\pgfpathlineto{\pgfqpoint{2.017944in}{0.601497in}}%
\pgfpathlineto{\pgfqpoint{2.018500in}{0.602245in}}%
\pgfpathlineto{\pgfqpoint{2.019056in}{0.606563in}}%
\pgfpathlineto{\pgfqpoint{2.019612in}{0.603496in}}%
\pgfpathlineto{\pgfqpoint{2.020168in}{0.601936in}}%
\pgfpathlineto{\pgfqpoint{2.021279in}{0.603964in}}%
\pgfpathlineto{\pgfqpoint{2.021835in}{0.603283in}}%
\pgfpathlineto{\pgfqpoint{2.022391in}{0.605664in}}%
\pgfpathlineto{\pgfqpoint{2.022947in}{0.604924in}}%
\pgfpathlineto{\pgfqpoint{2.023503in}{0.602867in}}%
\pgfpathlineto{\pgfqpoint{2.024614in}{0.606322in}}%
\pgfpathlineto{\pgfqpoint{2.025170in}{0.601778in}}%
\pgfpathlineto{\pgfqpoint{2.025726in}{0.605747in}}%
\pgfpathlineto{\pgfqpoint{2.026838in}{0.604807in}}%
\pgfpathlineto{\pgfqpoint{2.027394in}{0.602575in}}%
\pgfpathlineto{\pgfqpoint{2.027950in}{0.603976in}}%
\pgfpathlineto{\pgfqpoint{2.028505in}{0.605993in}}%
\pgfpathlineto{\pgfqpoint{2.029061in}{0.602346in}}%
\pgfpathlineto{\pgfqpoint{2.029617in}{0.607397in}}%
\pgfpathlineto{\pgfqpoint{2.030173in}{0.603499in}}%
\pgfpathlineto{\pgfqpoint{2.031285in}{0.608236in}}%
\pgfpathlineto{\pgfqpoint{2.032952in}{0.601646in}}%
\pgfpathlineto{\pgfqpoint{2.033508in}{0.607849in}}%
\pgfpathlineto{\pgfqpoint{2.034064in}{0.607034in}}%
\pgfpathlineto{\pgfqpoint{2.034620in}{0.602546in}}%
\pgfpathlineto{\pgfqpoint{2.035176in}{0.603938in}}%
\pgfpathlineto{\pgfqpoint{2.035732in}{0.606588in}}%
\pgfpathlineto{\pgfqpoint{2.037399in}{0.603493in}}%
\pgfpathlineto{\pgfqpoint{2.037955in}{0.601749in}}%
\pgfpathlineto{\pgfqpoint{2.038511in}{0.602058in}}%
\pgfpathlineto{\pgfqpoint{2.039623in}{0.601313in}}%
\pgfpathlineto{\pgfqpoint{2.041290in}{0.600601in}}%
\pgfpathlineto{\pgfqpoint{2.042958in}{0.601359in}}%
\pgfpathlineto{\pgfqpoint{2.047961in}{0.601889in}}%
\pgfpathlineto{\pgfqpoint{2.048516in}{0.601088in}}%
\pgfpathlineto{\pgfqpoint{2.049628in}{0.605652in}}%
\pgfpathlineto{\pgfqpoint{2.050184in}{0.600555in}}%
\pgfpathlineto{\pgfqpoint{2.050740in}{0.603150in}}%
\pgfpathlineto{\pgfqpoint{2.051296in}{0.608502in}}%
\pgfpathlineto{\pgfqpoint{2.051852in}{0.605806in}}%
\pgfpathlineto{\pgfqpoint{2.052963in}{0.603087in}}%
\pgfpathlineto{\pgfqpoint{2.053519in}{0.603218in}}%
\pgfpathlineto{\pgfqpoint{2.054631in}{0.608282in}}%
\pgfpathlineto{\pgfqpoint{2.055187in}{0.606703in}}%
\pgfpathlineto{\pgfqpoint{2.055743in}{0.600768in}}%
\pgfpathlineto{\pgfqpoint{2.056299in}{0.610326in}}%
\pgfpathlineto{\pgfqpoint{2.056854in}{0.606955in}}%
\pgfpathlineto{\pgfqpoint{2.057410in}{0.606107in}}%
\pgfpathlineto{\pgfqpoint{2.057966in}{0.608352in}}%
\pgfpathlineto{\pgfqpoint{2.059078in}{0.601777in}}%
\pgfpathlineto{\pgfqpoint{2.059634in}{0.604726in}}%
\pgfpathlineto{\pgfqpoint{2.060190in}{0.607783in}}%
\pgfpathlineto{\pgfqpoint{2.060745in}{0.605025in}}%
\pgfpathlineto{\pgfqpoint{2.061301in}{0.605717in}}%
\pgfpathlineto{\pgfqpoint{2.061857in}{0.601465in}}%
\pgfpathlineto{\pgfqpoint{2.062413in}{0.601764in}}%
\pgfpathlineto{\pgfqpoint{2.062969in}{0.612149in}}%
\pgfpathlineto{\pgfqpoint{2.063525in}{0.602201in}}%
\pgfpathlineto{\pgfqpoint{2.064081in}{0.601404in}}%
\pgfpathlineto{\pgfqpoint{2.065192in}{0.607554in}}%
\pgfpathlineto{\pgfqpoint{2.065748in}{0.605204in}}%
\pgfpathlineto{\pgfqpoint{2.066304in}{0.602765in}}%
\pgfpathlineto{\pgfqpoint{2.066860in}{0.605761in}}%
\pgfpathlineto{\pgfqpoint{2.067416in}{0.602567in}}%
\pgfpathlineto{\pgfqpoint{2.067972in}{0.615833in}}%
\pgfpathlineto{\pgfqpoint{2.068527in}{0.605849in}}%
\pgfpathlineto{\pgfqpoint{2.069639in}{0.601523in}}%
\pgfpathlineto{\pgfqpoint{2.070195in}{0.608525in}}%
\pgfpathlineto{\pgfqpoint{2.070751in}{0.604846in}}%
\pgfpathlineto{\pgfqpoint{2.072419in}{0.600862in}}%
\pgfpathlineto{\pgfqpoint{2.072974in}{0.601808in}}%
\pgfpathlineto{\pgfqpoint{2.073530in}{0.605029in}}%
\pgfpathlineto{\pgfqpoint{2.074086in}{0.603305in}}%
\pgfpathlineto{\pgfqpoint{2.075198in}{0.605086in}}%
\pgfpathlineto{\pgfqpoint{2.075754in}{0.607126in}}%
\pgfpathlineto{\pgfqpoint{2.076310in}{0.602919in}}%
\pgfpathlineto{\pgfqpoint{2.076865in}{0.604023in}}%
\pgfpathlineto{\pgfqpoint{2.079089in}{0.600015in}}%
\pgfpathlineto{\pgfqpoint{2.079645in}{0.608213in}}%
\pgfpathlineto{\pgfqpoint{2.080201in}{0.605277in}}%
\pgfpathlineto{\pgfqpoint{2.080756in}{0.602193in}}%
\pgfpathlineto{\pgfqpoint{2.081312in}{0.602830in}}%
\pgfpathlineto{\pgfqpoint{2.081868in}{0.606100in}}%
\pgfpathlineto{\pgfqpoint{2.082980in}{0.600933in}}%
\pgfpathlineto{\pgfqpoint{2.084092in}{0.606678in}}%
\pgfpathlineto{\pgfqpoint{2.084647in}{0.601904in}}%
\pgfpathlineto{\pgfqpoint{2.085203in}{0.604045in}}%
\pgfpathlineto{\pgfqpoint{2.086315in}{0.603107in}}%
\pgfpathlineto{\pgfqpoint{2.086871in}{0.605176in}}%
\pgfpathlineto{\pgfqpoint{2.087427in}{0.604498in}}%
\pgfpathlineto{\pgfqpoint{2.089650in}{0.601483in}}%
\pgfpathlineto{\pgfqpoint{2.090762in}{0.604023in}}%
\pgfpathlineto{\pgfqpoint{2.091318in}{0.601809in}}%
\pgfpathlineto{\pgfqpoint{2.091874in}{0.603428in}}%
\pgfpathlineto{\pgfqpoint{2.092430in}{0.603414in}}%
\pgfpathlineto{\pgfqpoint{2.093541in}{0.610836in}}%
\pgfpathlineto{\pgfqpoint{2.094097in}{0.605708in}}%
\pgfpathlineto{\pgfqpoint{2.095209in}{0.607313in}}%
\pgfpathlineto{\pgfqpoint{2.095765in}{0.603453in}}%
\pgfpathlineto{\pgfqpoint{2.096321in}{0.612160in}}%
\pgfpathlineto{\pgfqpoint{2.096876in}{0.606271in}}%
\pgfpathlineto{\pgfqpoint{2.098544in}{0.601824in}}%
\pgfpathlineto{\pgfqpoint{2.099100in}{0.612073in}}%
\pgfpathlineto{\pgfqpoint{2.099656in}{0.606675in}}%
\pgfpathlineto{\pgfqpoint{2.100212in}{0.607869in}}%
\pgfpathlineto{\pgfqpoint{2.100767in}{0.605707in}}%
\pgfpathlineto{\pgfqpoint{2.101323in}{0.611022in}}%
\pgfpathlineto{\pgfqpoint{2.102435in}{0.602010in}}%
\pgfpathlineto{\pgfqpoint{2.102991in}{0.603448in}}%
\pgfpathlineto{\pgfqpoint{2.103547in}{0.602946in}}%
\pgfpathlineto{\pgfqpoint{2.104103in}{0.603865in}}%
\pgfpathlineto{\pgfqpoint{2.104658in}{0.606450in}}%
\pgfpathlineto{\pgfqpoint{2.105214in}{0.605751in}}%
\pgfpathlineto{\pgfqpoint{2.105770in}{0.604295in}}%
\pgfpathlineto{\pgfqpoint{2.106326in}{0.607375in}}%
\pgfpathlineto{\pgfqpoint{2.107994in}{0.602152in}}%
\pgfpathlineto{\pgfqpoint{2.109105in}{0.607834in}}%
\pgfpathlineto{\pgfqpoint{2.109661in}{0.605191in}}%
\pgfpathlineto{\pgfqpoint{2.110217in}{0.605708in}}%
\pgfpathlineto{\pgfqpoint{2.111329in}{0.602086in}}%
\pgfpathlineto{\pgfqpoint{2.111885in}{0.603001in}}%
\pgfpathlineto{\pgfqpoint{2.112441in}{0.604136in}}%
\pgfpathlineto{\pgfqpoint{2.112996in}{0.602561in}}%
\pgfpathlineto{\pgfqpoint{2.114108in}{0.607907in}}%
\pgfpathlineto{\pgfqpoint{2.114664in}{0.603477in}}%
\pgfpathlineto{\pgfqpoint{2.115220in}{0.610586in}}%
\pgfpathlineto{\pgfqpoint{2.116332in}{0.601004in}}%
\pgfpathlineto{\pgfqpoint{2.116887in}{0.602190in}}%
\pgfpathlineto{\pgfqpoint{2.117443in}{0.602268in}}%
\pgfpathlineto{\pgfqpoint{2.117999in}{0.605378in}}%
\pgfpathlineto{\pgfqpoint{2.118555in}{0.604168in}}%
\pgfpathlineto{\pgfqpoint{2.119111in}{0.602510in}}%
\pgfpathlineto{\pgfqpoint{2.120223in}{0.608316in}}%
\pgfpathlineto{\pgfqpoint{2.120778in}{0.604751in}}%
\pgfpathlineto{\pgfqpoint{2.122446in}{0.600327in}}%
\pgfpathlineto{\pgfqpoint{2.123002in}{0.606531in}}%
\pgfpathlineto{\pgfqpoint{2.123558in}{0.603557in}}%
\pgfpathlineto{\pgfqpoint{2.125225in}{0.606228in}}%
\pgfpathlineto{\pgfqpoint{2.126337in}{0.602815in}}%
\pgfpathlineto{\pgfqpoint{2.126893in}{0.605068in}}%
\pgfpathlineto{\pgfqpoint{2.128005in}{0.609508in}}%
\pgfpathlineto{\pgfqpoint{2.128561in}{0.600952in}}%
\pgfpathlineto{\pgfqpoint{2.129116in}{0.601787in}}%
\pgfpathlineto{\pgfqpoint{2.130228in}{0.603051in}}%
\pgfpathlineto{\pgfqpoint{2.130784in}{0.600937in}}%
\pgfpathlineto{\pgfqpoint{2.131340in}{0.604135in}}%
\pgfpathlineto{\pgfqpoint{2.131896in}{0.602780in}}%
\pgfpathlineto{\pgfqpoint{2.132452in}{0.600606in}}%
\pgfpathlineto{\pgfqpoint{2.133007in}{0.601665in}}%
\pgfpathlineto{\pgfqpoint{2.134119in}{0.603190in}}%
\pgfpathlineto{\pgfqpoint{2.135231in}{0.602021in}}%
\pgfpathlineto{\pgfqpoint{2.138010in}{0.605022in}}%
\pgfpathlineto{\pgfqpoint{2.138566in}{0.600969in}}%
\pgfpathlineto{\pgfqpoint{2.139122in}{0.602217in}}%
\pgfpathlineto{\pgfqpoint{2.141345in}{0.605743in}}%
\pgfpathlineto{\pgfqpoint{2.142457in}{0.600933in}}%
\pgfpathlineto{\pgfqpoint{2.143013in}{0.604206in}}%
\pgfpathlineto{\pgfqpoint{2.143569in}{0.600389in}}%
\pgfpathlineto{\pgfqpoint{2.144125in}{0.602314in}}%
\pgfpathlineto{\pgfqpoint{2.145792in}{0.601198in}}%
\pgfpathlineto{\pgfqpoint{2.146348in}{0.605320in}}%
\pgfpathlineto{\pgfqpoint{2.146904in}{0.602114in}}%
\pgfpathlineto{\pgfqpoint{2.147460in}{0.603580in}}%
\pgfpathlineto{\pgfqpoint{2.148016in}{0.601766in}}%
\pgfpathlineto{\pgfqpoint{2.148572in}{0.604962in}}%
\pgfpathlineto{\pgfqpoint{2.149127in}{0.602146in}}%
\pgfpathlineto{\pgfqpoint{2.149683in}{0.601543in}}%
\pgfpathlineto{\pgfqpoint{2.150239in}{0.602606in}}%
\pgfpathlineto{\pgfqpoint{2.150795in}{0.607119in}}%
\pgfpathlineto{\pgfqpoint{2.151351in}{0.603872in}}%
\pgfpathlineto{\pgfqpoint{2.151907in}{0.605552in}}%
\pgfpathlineto{\pgfqpoint{2.152463in}{0.605238in}}%
\pgfpathlineto{\pgfqpoint{2.155798in}{0.601239in}}%
\pgfpathlineto{\pgfqpoint{2.156354in}{0.606884in}}%
\pgfpathlineto{\pgfqpoint{2.156909in}{0.606578in}}%
\pgfpathlineto{\pgfqpoint{2.158021in}{0.602374in}}%
\pgfpathlineto{\pgfqpoint{2.159689in}{0.605371in}}%
\pgfpathlineto{\pgfqpoint{2.161356in}{0.602415in}}%
\pgfpathlineto{\pgfqpoint{2.161912in}{0.603638in}}%
\pgfpathlineto{\pgfqpoint{2.162468in}{0.602487in}}%
\pgfpathlineto{\pgfqpoint{2.163580in}{0.600592in}}%
\pgfpathlineto{\pgfqpoint{2.164136in}{0.601214in}}%
\pgfpathlineto{\pgfqpoint{2.164692in}{0.605189in}}%
\pgfpathlineto{\pgfqpoint{2.165247in}{0.600262in}}%
\pgfpathlineto{\pgfqpoint{2.165803in}{0.605154in}}%
\pgfpathlineto{\pgfqpoint{2.166359in}{0.606051in}}%
\pgfpathlineto{\pgfqpoint{2.166915in}{0.602344in}}%
\pgfpathlineto{\pgfqpoint{2.167471in}{0.604007in}}%
\pgfpathlineto{\pgfqpoint{2.168027in}{0.606976in}}%
\pgfpathlineto{\pgfqpoint{2.168583in}{0.606122in}}%
\pgfpathlineto{\pgfqpoint{2.169138in}{0.606382in}}%
\pgfpathlineto{\pgfqpoint{2.169694in}{0.600873in}}%
\pgfpathlineto{\pgfqpoint{2.170250in}{0.603433in}}%
\pgfpathlineto{\pgfqpoint{2.170806in}{0.605854in}}%
\pgfpathlineto{\pgfqpoint{2.171362in}{0.603940in}}%
\pgfpathlineto{\pgfqpoint{2.171918in}{0.600914in}}%
\pgfpathlineto{\pgfqpoint{2.172474in}{0.604982in}}%
\pgfpathlineto{\pgfqpoint{2.173029in}{0.601936in}}%
\pgfpathlineto{\pgfqpoint{2.173585in}{0.600411in}}%
\pgfpathlineto{\pgfqpoint{2.174141in}{0.601288in}}%
\pgfpathlineto{\pgfqpoint{2.174697in}{0.606654in}}%
\pgfpathlineto{\pgfqpoint{2.175253in}{0.602357in}}%
\pgfpathlineto{\pgfqpoint{2.175809in}{0.605495in}}%
\pgfpathlineto{\pgfqpoint{2.176365in}{0.600135in}}%
\pgfpathlineto{\pgfqpoint{2.176920in}{0.605361in}}%
\pgfpathlineto{\pgfqpoint{2.177476in}{0.600927in}}%
\pgfpathlineto{\pgfqpoint{2.178032in}{0.606526in}}%
\pgfpathlineto{\pgfqpoint{2.178588in}{0.603515in}}%
\pgfpathlineto{\pgfqpoint{2.179144in}{0.604084in}}%
\pgfpathlineto{\pgfqpoint{2.180256in}{0.601694in}}%
\pgfpathlineto{\pgfqpoint{2.180811in}{0.602926in}}%
\pgfpathlineto{\pgfqpoint{2.181367in}{0.602144in}}%
\pgfpathlineto{\pgfqpoint{2.181923in}{0.601833in}}%
\pgfpathlineto{\pgfqpoint{2.183035in}{0.605323in}}%
\pgfpathlineto{\pgfqpoint{2.183591in}{0.603922in}}%
\pgfpathlineto{\pgfqpoint{2.184147in}{0.604598in}}%
\pgfpathlineto{\pgfqpoint{2.184703in}{0.602136in}}%
\pgfpathlineto{\pgfqpoint{2.185258in}{0.604496in}}%
\pgfpathlineto{\pgfqpoint{2.185814in}{0.604072in}}%
\pgfpathlineto{\pgfqpoint{2.186370in}{0.601430in}}%
\pgfpathlineto{\pgfqpoint{2.186926in}{0.602408in}}%
\pgfpathlineto{\pgfqpoint{2.188038in}{0.601234in}}%
\pgfpathlineto{\pgfqpoint{2.189149in}{0.605700in}}%
\pgfpathlineto{\pgfqpoint{2.190817in}{0.601257in}}%
\pgfpathlineto{\pgfqpoint{2.191929in}{0.600204in}}%
\pgfpathlineto{\pgfqpoint{2.192485in}{0.604211in}}%
\pgfpathlineto{\pgfqpoint{2.193040in}{0.601595in}}%
\pgfpathlineto{\pgfqpoint{2.194708in}{0.604249in}}%
\pgfpathlineto{\pgfqpoint{2.195820in}{0.602429in}}%
\pgfpathlineto{\pgfqpoint{2.196931in}{0.605302in}}%
\pgfpathlineto{\pgfqpoint{2.198043in}{0.600895in}}%
\pgfpathlineto{\pgfqpoint{2.198599in}{0.606443in}}%
\pgfpathlineto{\pgfqpoint{2.199155in}{0.601553in}}%
\pgfpathlineto{\pgfqpoint{2.200823in}{0.602170in}}%
\pgfpathlineto{\pgfqpoint{2.201934in}{0.602182in}}%
\pgfpathlineto{\pgfqpoint{2.203046in}{0.603741in}}%
\pgfpathlineto{\pgfqpoint{2.203602in}{0.602791in}}%
\pgfpathlineto{\pgfqpoint{2.204714in}{0.602147in}}%
\pgfpathlineto{\pgfqpoint{2.205269in}{0.604518in}}%
\pgfpathlineto{\pgfqpoint{2.205825in}{0.600837in}}%
\pgfpathlineto{\pgfqpoint{2.206381in}{0.602411in}}%
\pgfpathlineto{\pgfqpoint{2.206937in}{0.601633in}}%
\pgfpathlineto{\pgfqpoint{2.207493in}{0.602259in}}%
\pgfpathlineto{\pgfqpoint{2.208049in}{0.602182in}}%
\pgfpathlineto{\pgfqpoint{2.209160in}{0.603967in}}%
\pgfpathlineto{\pgfqpoint{2.210828in}{0.600689in}}%
\pgfpathlineto{\pgfqpoint{2.211384in}{0.606178in}}%
\pgfpathlineto{\pgfqpoint{2.211940in}{0.601388in}}%
\pgfpathlineto{\pgfqpoint{2.213051in}{0.600312in}}%
\pgfpathlineto{\pgfqpoint{2.214163in}{0.604935in}}%
\pgfpathlineto{\pgfqpoint{2.214719in}{0.603011in}}%
\pgfpathlineto{\pgfqpoint{2.215275in}{0.604802in}}%
\pgfpathlineto{\pgfqpoint{2.215831in}{0.604137in}}%
\pgfpathlineto{\pgfqpoint{2.216387in}{0.604985in}}%
\pgfpathlineto{\pgfqpoint{2.216942in}{0.605023in}}%
\pgfpathlineto{\pgfqpoint{2.217498in}{0.600961in}}%
\pgfpathlineto{\pgfqpoint{2.218054in}{0.601442in}}%
\pgfpathlineto{\pgfqpoint{2.219166in}{0.602477in}}%
\pgfpathlineto{\pgfqpoint{2.219722in}{0.601531in}}%
\pgfpathlineto{\pgfqpoint{2.220278in}{0.605094in}}%
\pgfpathlineto{\pgfqpoint{2.220834in}{0.602708in}}%
\pgfpathlineto{\pgfqpoint{2.222501in}{0.604929in}}%
\pgfpathlineto{\pgfqpoint{2.223057in}{0.602735in}}%
\pgfpathlineto{\pgfqpoint{2.224169in}{0.610099in}}%
\pgfpathlineto{\pgfqpoint{2.224725in}{0.601246in}}%
\pgfpathlineto{\pgfqpoint{2.225280in}{0.602763in}}%
\pgfpathlineto{\pgfqpoint{2.225836in}{0.601596in}}%
\pgfpathlineto{\pgfqpoint{2.227504in}{0.611436in}}%
\pgfpathlineto{\pgfqpoint{2.228060in}{0.602074in}}%
\pgfpathlineto{\pgfqpoint{2.228616in}{0.612693in}}%
\pgfpathlineto{\pgfqpoint{2.229171in}{0.603832in}}%
\pgfpathlineto{\pgfqpoint{2.229727in}{0.603158in}}%
\pgfpathlineto{\pgfqpoint{2.230283in}{0.606626in}}%
\pgfpathlineto{\pgfqpoint{2.230839in}{0.601860in}}%
\pgfpathlineto{\pgfqpoint{2.231395in}{0.605151in}}%
\pgfpathlineto{\pgfqpoint{2.231951in}{0.605611in}}%
\pgfpathlineto{\pgfqpoint{2.232507in}{0.604069in}}%
\pgfpathlineto{\pgfqpoint{2.233062in}{0.609670in}}%
\pgfpathlineto{\pgfqpoint{2.233618in}{0.602659in}}%
\pgfpathlineto{\pgfqpoint{2.234174in}{0.602962in}}%
\pgfpathlineto{\pgfqpoint{2.235286in}{0.605207in}}%
\pgfpathlineto{\pgfqpoint{2.236398in}{0.603532in}}%
\pgfpathlineto{\pgfqpoint{2.238065in}{0.607147in}}%
\pgfpathlineto{\pgfqpoint{2.239177in}{0.603683in}}%
\pgfpathlineto{\pgfqpoint{2.239733in}{0.605730in}}%
\pgfpathlineto{\pgfqpoint{2.240289in}{0.604248in}}%
\pgfpathlineto{\pgfqpoint{2.241400in}{0.606176in}}%
\pgfpathlineto{\pgfqpoint{2.241956in}{0.605185in}}%
\pgfpathlineto{\pgfqpoint{2.242512in}{0.601859in}}%
\pgfpathlineto{\pgfqpoint{2.243068in}{0.602457in}}%
\pgfpathlineto{\pgfqpoint{2.243624in}{0.607640in}}%
\pgfpathlineto{\pgfqpoint{2.244180in}{0.606921in}}%
\pgfpathlineto{\pgfqpoint{2.244736in}{0.606264in}}%
\pgfpathlineto{\pgfqpoint{2.246403in}{0.601696in}}%
\pgfpathlineto{\pgfqpoint{2.246959in}{0.602514in}}%
\pgfpathlineto{\pgfqpoint{2.248071in}{0.607420in}}%
\pgfpathlineto{\pgfqpoint{2.248627in}{0.600232in}}%
\pgfpathlineto{\pgfqpoint{2.249182in}{0.602453in}}%
\pgfpathlineto{\pgfqpoint{2.250294in}{0.605096in}}%
\pgfpathlineto{\pgfqpoint{2.251406in}{0.601192in}}%
\pgfpathlineto{\pgfqpoint{2.251962in}{0.602631in}}%
\pgfpathlineto{\pgfqpoint{2.252518in}{0.601540in}}%
\pgfpathlineto{\pgfqpoint{2.254185in}{0.607123in}}%
\pgfpathlineto{\pgfqpoint{2.255297in}{0.600999in}}%
\pgfpathlineto{\pgfqpoint{2.256409in}{0.602968in}}%
\pgfpathlineto{\pgfqpoint{2.256964in}{0.601651in}}%
\pgfpathlineto{\pgfqpoint{2.259188in}{0.605969in}}%
\pgfpathlineto{\pgfqpoint{2.260300in}{0.602722in}}%
\pgfpathlineto{\pgfqpoint{2.261411in}{0.606088in}}%
\pgfpathlineto{\pgfqpoint{2.263079in}{0.603682in}}%
\pgfpathlineto{\pgfqpoint{2.263635in}{0.604139in}}%
\pgfpathlineto{\pgfqpoint{2.264191in}{0.602129in}}%
\pgfpathlineto{\pgfqpoint{2.265302in}{0.605970in}}%
\pgfpathlineto{\pgfqpoint{2.266970in}{0.602430in}}%
\pgfpathlineto{\pgfqpoint{2.267526in}{0.612749in}}%
\pgfpathlineto{\pgfqpoint{2.268082in}{0.604168in}}%
\pgfpathlineto{\pgfqpoint{2.268638in}{0.604965in}}%
\pgfpathlineto{\pgfqpoint{2.270305in}{0.600979in}}%
\pgfpathlineto{\pgfqpoint{2.271417in}{0.612026in}}%
\pgfpathlineto{\pgfqpoint{2.272529in}{0.609698in}}%
\pgfpathlineto{\pgfqpoint{2.273084in}{0.606774in}}%
\pgfpathlineto{\pgfqpoint{2.273640in}{0.611759in}}%
\pgfpathlineto{\pgfqpoint{2.274196in}{0.602090in}}%
\pgfpathlineto{\pgfqpoint{2.274752in}{0.608045in}}%
\pgfpathlineto{\pgfqpoint{2.275864in}{0.606609in}}%
\pgfpathlineto{\pgfqpoint{2.276420in}{0.601962in}}%
\pgfpathlineto{\pgfqpoint{2.276976in}{0.603344in}}%
\pgfpathlineto{\pgfqpoint{2.277531in}{0.602540in}}%
\pgfpathlineto{\pgfqpoint{2.279199in}{0.608255in}}%
\pgfpathlineto{\pgfqpoint{2.279755in}{0.609773in}}%
\pgfpathlineto{\pgfqpoint{2.280867in}{0.602739in}}%
\pgfpathlineto{\pgfqpoint{2.282534in}{0.608904in}}%
\pgfpathlineto{\pgfqpoint{2.283090in}{0.605659in}}%
\pgfpathlineto{\pgfqpoint{2.283646in}{0.614973in}}%
\pgfpathlineto{\pgfqpoint{2.284758in}{0.603976in}}%
\pgfpathlineto{\pgfqpoint{2.286425in}{0.613712in}}%
\pgfpathlineto{\pgfqpoint{2.286981in}{0.605972in}}%
\pgfpathlineto{\pgfqpoint{2.287537in}{0.612306in}}%
\pgfpathlineto{\pgfqpoint{2.288649in}{0.606056in}}%
\pgfpathlineto{\pgfqpoint{2.289204in}{0.609942in}}%
\pgfpathlineto{\pgfqpoint{2.289760in}{0.604154in}}%
\pgfpathlineto{\pgfqpoint{2.290316in}{0.625159in}}%
\pgfpathlineto{\pgfqpoint{2.290872in}{0.602960in}}%
\pgfpathlineto{\pgfqpoint{2.291428in}{0.609210in}}%
\pgfpathlineto{\pgfqpoint{2.292540in}{0.605228in}}%
\pgfpathlineto{\pgfqpoint{2.293095in}{0.608603in}}%
\pgfpathlineto{\pgfqpoint{2.293651in}{0.603008in}}%
\pgfpathlineto{\pgfqpoint{2.294207in}{0.603989in}}%
\pgfpathlineto{\pgfqpoint{2.294763in}{0.603705in}}%
\pgfpathlineto{\pgfqpoint{2.295319in}{0.619171in}}%
\pgfpathlineto{\pgfqpoint{2.295875in}{0.604392in}}%
\pgfpathlineto{\pgfqpoint{2.296431in}{0.602850in}}%
\pgfpathlineto{\pgfqpoint{2.298098in}{0.610085in}}%
\pgfpathlineto{\pgfqpoint{2.298654in}{0.603160in}}%
\pgfpathlineto{\pgfqpoint{2.299210in}{0.604136in}}%
\pgfpathlineto{\pgfqpoint{2.300878in}{0.612174in}}%
\pgfpathlineto{\pgfqpoint{2.302545in}{0.601164in}}%
\pgfpathlineto{\pgfqpoint{2.303101in}{0.607834in}}%
\pgfpathlineto{\pgfqpoint{2.303657in}{0.607456in}}%
\pgfpathlineto{\pgfqpoint{2.304213in}{0.601401in}}%
\pgfpathlineto{\pgfqpoint{2.304769in}{0.609832in}}%
\pgfpathlineto{\pgfqpoint{2.305324in}{0.608150in}}%
\pgfpathlineto{\pgfqpoint{2.305880in}{0.602519in}}%
\pgfpathlineto{\pgfqpoint{2.306436in}{0.612671in}}%
\pgfpathlineto{\pgfqpoint{2.306992in}{0.606819in}}%
\pgfpathlineto{\pgfqpoint{2.307548in}{0.612338in}}%
\pgfpathlineto{\pgfqpoint{2.309215in}{0.601281in}}%
\pgfpathlineto{\pgfqpoint{2.310883in}{0.609011in}}%
\pgfpathlineto{\pgfqpoint{2.311995in}{0.601772in}}%
\pgfpathlineto{\pgfqpoint{2.312551in}{0.610927in}}%
\pgfpathlineto{\pgfqpoint{2.314218in}{0.600181in}}%
\pgfpathlineto{\pgfqpoint{2.315886in}{0.607989in}}%
\pgfpathlineto{\pgfqpoint{2.316442in}{0.604292in}}%
\pgfpathlineto{\pgfqpoint{2.316998in}{0.606963in}}%
\pgfpathlineto{\pgfqpoint{2.317553in}{0.609267in}}%
\pgfpathlineto{\pgfqpoint{2.318109in}{0.600806in}}%
\pgfpathlineto{\pgfqpoint{2.318665in}{0.610982in}}%
\pgfpathlineto{\pgfqpoint{2.319221in}{0.610166in}}%
\pgfpathlineto{\pgfqpoint{2.320333in}{0.603549in}}%
\pgfpathlineto{\pgfqpoint{2.320889in}{0.610882in}}%
\pgfpathlineto{\pgfqpoint{2.321444in}{0.601132in}}%
\pgfpathlineto{\pgfqpoint{2.322000in}{0.608836in}}%
\pgfpathlineto{\pgfqpoint{2.323668in}{0.605385in}}%
\pgfpathlineto{\pgfqpoint{2.324224in}{0.603377in}}%
\pgfpathlineto{\pgfqpoint{2.325335in}{0.612608in}}%
\pgfpathlineto{\pgfqpoint{2.325891in}{0.603271in}}%
\pgfpathlineto{\pgfqpoint{2.326447in}{0.608303in}}%
\pgfpathlineto{\pgfqpoint{2.327003in}{0.610451in}}%
\pgfpathlineto{\pgfqpoint{2.328115in}{0.604844in}}%
\pgfpathlineto{\pgfqpoint{2.329226in}{0.612309in}}%
\pgfpathlineto{\pgfqpoint{2.329782in}{0.611363in}}%
\pgfpathlineto{\pgfqpoint{2.330338in}{0.607730in}}%
\pgfpathlineto{\pgfqpoint{2.331450in}{0.617953in}}%
\pgfpathlineto{\pgfqpoint{2.332006in}{0.612455in}}%
\pgfpathlineto{\pgfqpoint{2.333118in}{0.611675in}}%
\pgfpathlineto{\pgfqpoint{2.334229in}{0.602719in}}%
\pgfpathlineto{\pgfqpoint{2.334785in}{0.612173in}}%
\pgfpathlineto{\pgfqpoint{2.335341in}{0.610118in}}%
\pgfpathlineto{\pgfqpoint{2.335897in}{0.600675in}}%
\pgfpathlineto{\pgfqpoint{2.336453in}{0.607152in}}%
\pgfpathlineto{\pgfqpoint{2.337009in}{0.611361in}}%
\pgfpathlineto{\pgfqpoint{2.337564in}{0.608009in}}%
\pgfpathlineto{\pgfqpoint{2.338676in}{0.604883in}}%
\pgfpathlineto{\pgfqpoint{2.339232in}{0.618324in}}%
\pgfpathlineto{\pgfqpoint{2.339788in}{0.615389in}}%
\pgfpathlineto{\pgfqpoint{2.340344in}{0.606414in}}%
\pgfpathlineto{\pgfqpoint{2.340900in}{0.615458in}}%
\pgfpathlineto{\pgfqpoint{2.341455in}{0.611966in}}%
\pgfpathlineto{\pgfqpoint{2.342011in}{0.606374in}}%
\pgfpathlineto{\pgfqpoint{2.343679in}{0.621641in}}%
\pgfpathlineto{\pgfqpoint{2.344791in}{0.603294in}}%
\pgfpathlineto{\pgfqpoint{2.346458in}{0.620313in}}%
\pgfpathlineto{\pgfqpoint{2.347014in}{0.615394in}}%
\pgfpathlineto{\pgfqpoint{2.347570in}{0.622568in}}%
\pgfpathlineto{\pgfqpoint{2.348126in}{0.621412in}}%
\pgfpathlineto{\pgfqpoint{2.348682in}{0.610044in}}%
\pgfpathlineto{\pgfqpoint{2.349237in}{0.621461in}}%
\pgfpathlineto{\pgfqpoint{2.349793in}{0.616491in}}%
\pgfpathlineto{\pgfqpoint{2.350349in}{0.619949in}}%
\pgfpathlineto{\pgfqpoint{2.350905in}{0.606843in}}%
\pgfpathlineto{\pgfqpoint{2.351461in}{0.608482in}}%
\pgfpathlineto{\pgfqpoint{2.352017in}{0.607331in}}%
\pgfpathlineto{\pgfqpoint{2.352573in}{0.618035in}}%
\pgfpathlineto{\pgfqpoint{2.353129in}{0.614764in}}%
\pgfpathlineto{\pgfqpoint{2.354796in}{0.602216in}}%
\pgfpathlineto{\pgfqpoint{2.355352in}{0.628430in}}%
\pgfpathlineto{\pgfqpoint{2.355908in}{0.602648in}}%
\pgfpathlineto{\pgfqpoint{2.357575in}{0.606448in}}%
\pgfpathlineto{\pgfqpoint{2.359243in}{0.612644in}}%
\pgfpathlineto{\pgfqpoint{2.360911in}{0.609639in}}%
\pgfpathlineto{\pgfqpoint{2.361466in}{0.610403in}}%
\pgfpathlineto{\pgfqpoint{2.362022in}{0.601425in}}%
\pgfpathlineto{\pgfqpoint{2.362578in}{0.603140in}}%
\pgfpathlineto{\pgfqpoint{2.363690in}{0.618737in}}%
\pgfpathlineto{\pgfqpoint{2.364246in}{0.611214in}}%
\pgfpathlineto{\pgfqpoint{2.364802in}{0.610938in}}%
\pgfpathlineto{\pgfqpoint{2.366469in}{0.602099in}}%
\pgfpathlineto{\pgfqpoint{2.367025in}{0.609051in}}%
\pgfpathlineto{\pgfqpoint{2.367581in}{0.604553in}}%
\pgfpathlineto{\pgfqpoint{2.368137in}{0.608000in}}%
\pgfpathlineto{\pgfqpoint{2.368693in}{0.602184in}}%
\pgfpathlineto{\pgfqpoint{2.369248in}{0.605856in}}%
\pgfpathlineto{\pgfqpoint{2.369804in}{0.614428in}}%
\pgfpathlineto{\pgfqpoint{2.370360in}{0.606136in}}%
\pgfpathlineto{\pgfqpoint{2.370916in}{0.600790in}}%
\pgfpathlineto{\pgfqpoint{2.371472in}{0.603041in}}%
\pgfpathlineto{\pgfqpoint{2.373140in}{0.607319in}}%
\pgfpathlineto{\pgfqpoint{2.373695in}{0.604683in}}%
\pgfpathlineto{\pgfqpoint{2.374251in}{0.608282in}}%
\pgfpathlineto{\pgfqpoint{2.374807in}{0.603328in}}%
\pgfpathlineto{\pgfqpoint{2.375363in}{0.607560in}}%
\pgfpathlineto{\pgfqpoint{2.375919in}{0.605474in}}%
\pgfpathlineto{\pgfqpoint{2.376475in}{0.607290in}}%
\pgfpathlineto{\pgfqpoint{2.377586in}{0.610784in}}%
\pgfpathlineto{\pgfqpoint{2.378142in}{0.608716in}}%
\pgfpathlineto{\pgfqpoint{2.378698in}{0.605735in}}%
\pgfpathlineto{\pgfqpoint{2.379254in}{0.606872in}}%
\pgfpathlineto{\pgfqpoint{2.379810in}{0.608385in}}%
\pgfpathlineto{\pgfqpoint{2.380366in}{0.613983in}}%
\pgfpathlineto{\pgfqpoint{2.380922in}{0.600347in}}%
\pgfpathlineto{\pgfqpoint{2.381477in}{0.609592in}}%
\pgfpathlineto{\pgfqpoint{2.382033in}{0.606373in}}%
\pgfpathlineto{\pgfqpoint{2.382589in}{0.616049in}}%
\pgfpathlineto{\pgfqpoint{2.383145in}{0.611373in}}%
\pgfpathlineto{\pgfqpoint{2.383701in}{0.610390in}}%
\pgfpathlineto{\pgfqpoint{2.384257in}{0.604858in}}%
\pgfpathlineto{\pgfqpoint{2.384813in}{0.612040in}}%
\pgfpathlineto{\pgfqpoint{2.385368in}{0.607250in}}%
\pgfpathlineto{\pgfqpoint{2.385924in}{0.611829in}}%
\pgfpathlineto{\pgfqpoint{2.387036in}{0.604243in}}%
\pgfpathlineto{\pgfqpoint{2.388704in}{0.624462in}}%
\pgfpathlineto{\pgfqpoint{2.389260in}{0.604300in}}%
\pgfpathlineto{\pgfqpoint{2.389815in}{0.618112in}}%
\pgfpathlineto{\pgfqpoint{2.390371in}{0.609279in}}%
\pgfpathlineto{\pgfqpoint{2.390927in}{0.618325in}}%
\pgfpathlineto{\pgfqpoint{2.391483in}{0.602193in}}%
\pgfpathlineto{\pgfqpoint{2.392039in}{0.613239in}}%
\pgfpathlineto{\pgfqpoint{2.392595in}{0.605952in}}%
\pgfpathlineto{\pgfqpoint{2.393151in}{0.616158in}}%
\pgfpathlineto{\pgfqpoint{2.393706in}{0.607772in}}%
\pgfpathlineto{\pgfqpoint{2.394262in}{0.613348in}}%
\pgfpathlineto{\pgfqpoint{2.394818in}{0.610912in}}%
\pgfpathlineto{\pgfqpoint{2.395374in}{0.606073in}}%
\pgfpathlineto{\pgfqpoint{2.395930in}{0.609305in}}%
\pgfpathlineto{\pgfqpoint{2.397042in}{0.616841in}}%
\pgfpathlineto{\pgfqpoint{2.397597in}{0.605508in}}%
\pgfpathlineto{\pgfqpoint{2.398153in}{0.611578in}}%
\pgfpathlineto{\pgfqpoint{2.398709in}{0.626373in}}%
\pgfpathlineto{\pgfqpoint{2.399265in}{0.613665in}}%
\pgfpathlineto{\pgfqpoint{2.400933in}{0.609663in}}%
\pgfpathlineto{\pgfqpoint{2.401488in}{0.616335in}}%
\pgfpathlineto{\pgfqpoint{2.402044in}{0.602041in}}%
\pgfpathlineto{\pgfqpoint{2.402600in}{0.614343in}}%
\pgfpathlineto{\pgfqpoint{2.403156in}{0.623361in}}%
\pgfpathlineto{\pgfqpoint{2.404268in}{0.609631in}}%
\pgfpathlineto{\pgfqpoint{2.404824in}{0.611343in}}%
\pgfpathlineto{\pgfqpoint{2.405379in}{0.631881in}}%
\pgfpathlineto{\pgfqpoint{2.405935in}{0.605761in}}%
\pgfpathlineto{\pgfqpoint{2.406491in}{0.608926in}}%
\pgfpathlineto{\pgfqpoint{2.407047in}{0.608377in}}%
\pgfpathlineto{\pgfqpoint{2.407603in}{0.622676in}}%
\pgfpathlineto{\pgfqpoint{2.408159in}{0.609485in}}%
\pgfpathlineto{\pgfqpoint{2.408715in}{0.610233in}}%
\pgfpathlineto{\pgfqpoint{2.409826in}{0.623722in}}%
\pgfpathlineto{\pgfqpoint{2.410938in}{0.609661in}}%
\pgfpathlineto{\pgfqpoint{2.411494in}{0.614924in}}%
\pgfpathlineto{\pgfqpoint{2.412050in}{0.604959in}}%
\pgfpathlineto{\pgfqpoint{2.412606in}{0.616061in}}%
\pgfpathlineto{\pgfqpoint{2.413162in}{0.605669in}}%
\pgfpathlineto{\pgfqpoint{2.413717in}{0.608258in}}%
\pgfpathlineto{\pgfqpoint{2.414273in}{0.606240in}}%
\pgfpathlineto{\pgfqpoint{2.414829in}{0.605068in}}%
\pgfpathlineto{\pgfqpoint{2.415385in}{0.606268in}}%
\pgfpathlineto{\pgfqpoint{2.417053in}{0.612266in}}%
\pgfpathlineto{\pgfqpoint{2.418720in}{0.604383in}}%
\pgfpathlineto{\pgfqpoint{2.419276in}{0.609893in}}%
\pgfpathlineto{\pgfqpoint{2.419832in}{0.604041in}}%
\pgfpathlineto{\pgfqpoint{2.420388in}{0.609351in}}%
\pgfpathlineto{\pgfqpoint{2.420944in}{0.614535in}}%
\pgfpathlineto{\pgfqpoint{2.422611in}{0.601092in}}%
\pgfpathlineto{\pgfqpoint{2.423167in}{0.611159in}}%
\pgfpathlineto{\pgfqpoint{2.423723in}{0.607181in}}%
\pgfpathlineto{\pgfqpoint{2.425946in}{0.602442in}}%
\pgfpathlineto{\pgfqpoint{2.427614in}{0.610806in}}%
\pgfpathlineto{\pgfqpoint{2.428170in}{0.600362in}}%
\pgfpathlineto{\pgfqpoint{2.428726in}{0.601416in}}%
\pgfpathlineto{\pgfqpoint{2.430393in}{0.605442in}}%
\pgfpathlineto{\pgfqpoint{2.430949in}{0.607442in}}%
\pgfpathlineto{\pgfqpoint{2.431505in}{0.607234in}}%
\pgfpathlineto{\pgfqpoint{2.433173in}{0.601669in}}%
\pgfpathlineto{\pgfqpoint{2.434284in}{0.606290in}}%
\pgfpathlineto{\pgfqpoint{2.434840in}{0.605867in}}%
\pgfpathlineto{\pgfqpoint{2.435396in}{0.600132in}}%
\pgfpathlineto{\pgfqpoint{2.437064in}{0.610640in}}%
\pgfpathlineto{\pgfqpoint{2.438175in}{0.601374in}}%
\pgfpathlineto{\pgfqpoint{2.439843in}{0.609250in}}%
\pgfpathlineto{\pgfqpoint{2.441510in}{0.602397in}}%
\pgfpathlineto{\pgfqpoint{2.442066in}{0.610367in}}%
\pgfpathlineto{\pgfqpoint{2.442622in}{0.606467in}}%
\pgfpathlineto{\pgfqpoint{2.443178in}{0.607939in}}%
\pgfpathlineto{\pgfqpoint{2.443734in}{0.604176in}}%
\pgfpathlineto{\pgfqpoint{2.444290in}{0.606164in}}%
\pgfpathlineto{\pgfqpoint{2.444846in}{0.609642in}}%
\pgfpathlineto{\pgfqpoint{2.445401in}{0.602484in}}%
\pgfpathlineto{\pgfqpoint{2.447069in}{0.611839in}}%
\pgfpathlineto{\pgfqpoint{2.447625in}{0.607649in}}%
\pgfpathlineto{\pgfqpoint{2.448181in}{0.613687in}}%
\pgfpathlineto{\pgfqpoint{2.448737in}{0.603810in}}%
\pgfpathlineto{\pgfqpoint{2.449293in}{0.604938in}}%
\pgfpathlineto{\pgfqpoint{2.450404in}{0.611533in}}%
\pgfpathlineto{\pgfqpoint{2.452072in}{0.601262in}}%
\pgfpathlineto{\pgfqpoint{2.452628in}{0.603571in}}%
\pgfpathlineto{\pgfqpoint{2.453184in}{0.601151in}}%
\pgfpathlineto{\pgfqpoint{2.453739in}{0.602748in}}%
\pgfpathlineto{\pgfqpoint{2.454295in}{0.609805in}}%
\pgfpathlineto{\pgfqpoint{2.454851in}{0.606996in}}%
\pgfpathlineto{\pgfqpoint{2.455407in}{0.605359in}}%
\pgfpathlineto{\pgfqpoint{2.455963in}{0.613313in}}%
\pgfpathlineto{\pgfqpoint{2.456519in}{0.608418in}}%
\pgfpathlineto{\pgfqpoint{2.458186in}{0.600165in}}%
\pgfpathlineto{\pgfqpoint{2.460410in}{0.613200in}}%
\pgfpathlineto{\pgfqpoint{2.461521in}{0.606118in}}%
\pgfpathlineto{\pgfqpoint{2.462077in}{0.609072in}}%
\pgfpathlineto{\pgfqpoint{2.462633in}{0.608204in}}%
\pgfpathlineto{\pgfqpoint{2.463189in}{0.606342in}}%
\pgfpathlineto{\pgfqpoint{2.463745in}{0.600524in}}%
\pgfpathlineto{\pgfqpoint{2.464857in}{0.609207in}}%
\pgfpathlineto{\pgfqpoint{2.467080in}{0.601588in}}%
\pgfpathlineto{\pgfqpoint{2.467636in}{0.611093in}}%
\pgfpathlineto{\pgfqpoint{2.468192in}{0.606432in}}%
\pgfpathlineto{\pgfqpoint{2.470415in}{0.601937in}}%
\pgfpathlineto{\pgfqpoint{2.470971in}{0.608983in}}%
\pgfpathlineto{\pgfqpoint{2.471527in}{0.605724in}}%
\pgfpathlineto{\pgfqpoint{2.473750in}{0.600410in}}%
\pgfpathlineto{\pgfqpoint{2.475418in}{0.605536in}}%
\pgfpathlineto{\pgfqpoint{2.477641in}{0.601684in}}%
\pgfpathlineto{\pgfqpoint{2.478197in}{0.602645in}}%
\pgfpathlineto{\pgfqpoint{2.478753in}{0.600266in}}%
\pgfpathlineto{\pgfqpoint{2.480421in}{0.605198in}}%
\pgfpathlineto{\pgfqpoint{2.481532in}{0.600805in}}%
\pgfpathlineto{\pgfqpoint{2.482088in}{0.601446in}}%
\pgfpathlineto{\pgfqpoint{2.483756in}{0.603894in}}%
\pgfpathlineto{\pgfqpoint{2.484868in}{0.602243in}}%
\pgfpathlineto{\pgfqpoint{2.486535in}{0.600516in}}%
\pgfpathlineto{\pgfqpoint{2.487091in}{0.602269in}}%
\pgfpathlineto{\pgfqpoint{2.487647in}{0.601462in}}%
\pgfpathlineto{\pgfqpoint{2.489870in}{0.602392in}}%
\pgfpathlineto{\pgfqpoint{2.491538in}{0.600986in}}%
\pgfpathlineto{\pgfqpoint{2.493206in}{0.602402in}}%
\pgfpathlineto{\pgfqpoint{2.493761in}{0.601122in}}%
\pgfpathlineto{\pgfqpoint{2.494317in}{0.603436in}}%
\pgfpathlineto{\pgfqpoint{2.494873in}{0.600248in}}%
\pgfpathlineto{\pgfqpoint{2.495429in}{0.601359in}}%
\pgfpathlineto{\pgfqpoint{2.497652in}{0.602313in}}%
\pgfpathlineto{\pgfqpoint{2.498764in}{0.601897in}}%
\pgfpathlineto{\pgfqpoint{2.500432in}{0.602140in}}%
\pgfpathlineto{\pgfqpoint{2.503211in}{0.600534in}}%
\pgfpathlineto{\pgfqpoint{2.503767in}{0.603987in}}%
\pgfpathlineto{\pgfqpoint{2.504323in}{0.601748in}}%
\pgfpathlineto{\pgfqpoint{2.505990in}{0.602902in}}%
\pgfpathlineto{\pgfqpoint{2.506546in}{0.600655in}}%
\pgfpathlineto{\pgfqpoint{2.507102in}{0.603013in}}%
\pgfpathlineto{\pgfqpoint{2.508770in}{0.600304in}}%
\pgfpathlineto{\pgfqpoint{2.509881in}{0.602170in}}%
\pgfpathlineto{\pgfqpoint{2.510437in}{0.600052in}}%
\pgfpathlineto{\pgfqpoint{2.510993in}{0.601868in}}%
\pgfpathlineto{\pgfqpoint{2.512661in}{0.600281in}}%
\pgfpathlineto{\pgfqpoint{2.513772in}{0.602828in}}%
\pgfpathlineto{\pgfqpoint{2.514328in}{0.601650in}}%
\pgfpathlineto{\pgfqpoint{2.515996in}{0.600602in}}%
\pgfpathlineto{\pgfqpoint{2.517663in}{0.601825in}}%
\pgfpathlineto{\pgfqpoint{2.519887in}{0.601317in}}%
\pgfpathlineto{\pgfqpoint{2.520443in}{0.601422in}}%
\pgfpathlineto{\pgfqpoint{2.520999in}{0.600199in}}%
\pgfpathlineto{\pgfqpoint{2.521555in}{0.600474in}}%
\pgfpathlineto{\pgfqpoint{2.522110in}{0.601340in}}%
\pgfpathlineto{\pgfqpoint{2.522666in}{0.600888in}}%
\pgfpathlineto{\pgfqpoint{2.524890in}{0.600583in}}%
\pgfpathlineto{\pgfqpoint{2.527669in}{0.600881in}}%
\pgfpathlineto{\pgfqpoint{2.528781in}{0.600187in}}%
\pgfpathlineto{\pgfqpoint{2.530448in}{0.600628in}}%
\pgfpathlineto{\pgfqpoint{2.533228in}{0.600034in}}%
\pgfpathlineto{\pgfqpoint{2.536007in}{0.600260in}}%
\pgfpathlineto{\pgfqpoint{2.544901in}{0.601201in}}%
\pgfpathlineto{\pgfqpoint{2.545457in}{0.602594in}}%
\pgfpathlineto{\pgfqpoint{2.546012in}{0.600581in}}%
\pgfpathlineto{\pgfqpoint{2.546568in}{0.605534in}}%
\pgfpathlineto{\pgfqpoint{2.547124in}{0.603861in}}%
\pgfpathlineto{\pgfqpoint{2.548792in}{0.599974in}}%
\pgfpathlineto{\pgfqpoint{2.553794in}{0.600173in}}%
\pgfpathlineto{\pgfqpoint{2.556574in}{0.600128in}}%
\pgfpathlineto{\pgfqpoint{2.559909in}{0.600546in}}%
\pgfpathlineto{\pgfqpoint{2.561577in}{0.600125in}}%
\pgfpathlineto{\pgfqpoint{2.563244in}{0.601512in}}%
\pgfpathlineto{\pgfqpoint{2.563800in}{0.600054in}}%
\pgfpathlineto{\pgfqpoint{2.564356in}{0.600875in}}%
\pgfpathlineto{\pgfqpoint{2.566023in}{0.600398in}}%
\pgfpathlineto{\pgfqpoint{2.567691in}{0.600516in}}%
\pgfpathlineto{\pgfqpoint{2.570470in}{0.600930in}}%
\pgfpathlineto{\pgfqpoint{2.571026in}{0.602102in}}%
\pgfpathlineto{\pgfqpoint{2.571582in}{0.601626in}}%
\pgfpathlineto{\pgfqpoint{2.572694in}{0.600239in}}%
\pgfpathlineto{\pgfqpoint{2.573805in}{0.601328in}}%
\pgfpathlineto{\pgfqpoint{2.574361in}{0.600150in}}%
\pgfpathlineto{\pgfqpoint{2.575473in}{0.602499in}}%
\pgfpathlineto{\pgfqpoint{2.576029in}{0.600954in}}%
\pgfpathlineto{\pgfqpoint{2.576585in}{0.601250in}}%
\pgfpathlineto{\pgfqpoint{2.577141in}{0.602226in}}%
\pgfpathlineto{\pgfqpoint{2.577697in}{0.601715in}}%
\pgfpathlineto{\pgfqpoint{2.579920in}{0.601670in}}%
\pgfpathlineto{\pgfqpoint{2.580476in}{0.602012in}}%
\pgfpathlineto{\pgfqpoint{2.581032in}{0.600322in}}%
\pgfpathlineto{\pgfqpoint{2.581588in}{0.601712in}}%
\pgfpathlineto{\pgfqpoint{2.582143in}{0.600697in}}%
\pgfpathlineto{\pgfqpoint{2.582699in}{0.603194in}}%
\pgfpathlineto{\pgfqpoint{2.584367in}{0.600049in}}%
\pgfpathlineto{\pgfqpoint{2.586034in}{0.602228in}}%
\pgfpathlineto{\pgfqpoint{2.586590in}{0.600838in}}%
\pgfpathlineto{\pgfqpoint{2.587146in}{0.601376in}}%
\pgfpathlineto{\pgfqpoint{2.588814in}{0.602560in}}%
\pgfpathlineto{\pgfqpoint{2.589370in}{0.602894in}}%
\pgfpathlineto{\pgfqpoint{2.589925in}{0.601329in}}%
\pgfpathlineto{\pgfqpoint{2.590481in}{0.602512in}}%
\pgfpathlineto{\pgfqpoint{2.591037in}{0.603444in}}%
\pgfpathlineto{\pgfqpoint{2.591593in}{0.602845in}}%
\pgfpathlineto{\pgfqpoint{2.592149in}{0.602731in}}%
\pgfpathlineto{\pgfqpoint{2.593816in}{0.600454in}}%
\pgfpathlineto{\pgfqpoint{2.594928in}{0.601047in}}%
\pgfpathlineto{\pgfqpoint{2.596596in}{0.603909in}}%
\pgfpathlineto{\pgfqpoint{2.598263in}{0.601124in}}%
\pgfpathlineto{\pgfqpoint{2.599375in}{0.602867in}}%
\pgfpathlineto{\pgfqpoint{2.599931in}{0.604246in}}%
\pgfpathlineto{\pgfqpoint{2.600487in}{0.603528in}}%
\pgfpathlineto{\pgfqpoint{2.602154in}{0.601065in}}%
\pgfpathlineto{\pgfqpoint{2.602710in}{0.601697in}}%
\pgfpathlineto{\pgfqpoint{2.603822in}{0.605248in}}%
\pgfpathlineto{\pgfqpoint{2.604934in}{0.601014in}}%
\pgfpathlineto{\pgfqpoint{2.605490in}{0.601513in}}%
\pgfpathlineto{\pgfqpoint{2.606045in}{0.603178in}}%
\pgfpathlineto{\pgfqpoint{2.606601in}{0.600804in}}%
\pgfpathlineto{\pgfqpoint{2.607157in}{0.602991in}}%
\pgfpathlineto{\pgfqpoint{2.607713in}{0.601546in}}%
\pgfpathlineto{\pgfqpoint{2.608269in}{0.603174in}}%
\pgfpathlineto{\pgfqpoint{2.608825in}{0.603427in}}%
\pgfpathlineto{\pgfqpoint{2.609381in}{0.606478in}}%
\pgfpathlineto{\pgfqpoint{2.609936in}{0.600767in}}%
\pgfpathlineto{\pgfqpoint{2.610492in}{0.601150in}}%
\pgfpathlineto{\pgfqpoint{2.611048in}{0.604789in}}%
\pgfpathlineto{\pgfqpoint{2.611604in}{0.602781in}}%
\pgfpathlineto{\pgfqpoint{2.612716in}{0.603655in}}%
\pgfpathlineto{\pgfqpoint{2.613272in}{0.606036in}}%
\pgfpathlineto{\pgfqpoint{2.613827in}{0.601904in}}%
\pgfpathlineto{\pgfqpoint{2.614383in}{0.607955in}}%
\pgfpathlineto{\pgfqpoint{2.614939in}{0.605094in}}%
\pgfpathlineto{\pgfqpoint{2.615495in}{0.607829in}}%
\pgfpathlineto{\pgfqpoint{2.616051in}{0.600755in}}%
\pgfpathlineto{\pgfqpoint{2.616607in}{0.602526in}}%
\pgfpathlineto{\pgfqpoint{2.617163in}{0.607327in}}%
\pgfpathlineto{\pgfqpoint{2.617719in}{0.606040in}}%
\pgfpathlineto{\pgfqpoint{2.618274in}{0.603774in}}%
\pgfpathlineto{\pgfqpoint{2.619386in}{0.610098in}}%
\pgfpathlineto{\pgfqpoint{2.619942in}{0.606424in}}%
\pgfpathlineto{\pgfqpoint{2.620498in}{0.615039in}}%
\pgfpathlineto{\pgfqpoint{2.621054in}{0.609418in}}%
\pgfpathlineto{\pgfqpoint{2.621610in}{0.606329in}}%
\pgfpathlineto{\pgfqpoint{2.622165in}{0.613962in}}%
\pgfpathlineto{\pgfqpoint{2.622721in}{0.608345in}}%
\pgfpathlineto{\pgfqpoint{2.623277in}{0.607648in}}%
\pgfpathlineto{\pgfqpoint{2.623833in}{0.604931in}}%
\pgfpathlineto{\pgfqpoint{2.624945in}{0.613251in}}%
\pgfpathlineto{\pgfqpoint{2.625501in}{0.602742in}}%
\pgfpathlineto{\pgfqpoint{2.626056in}{0.612621in}}%
\pgfpathlineto{\pgfqpoint{2.627168in}{0.602282in}}%
\pgfpathlineto{\pgfqpoint{2.628836in}{0.614983in}}%
\pgfpathlineto{\pgfqpoint{2.629947in}{0.607325in}}%
\pgfpathlineto{\pgfqpoint{2.630503in}{0.612127in}}%
\pgfpathlineto{\pgfqpoint{2.631059in}{0.606412in}}%
\pgfpathlineto{\pgfqpoint{2.631615in}{0.610065in}}%
\pgfpathlineto{\pgfqpoint{2.632171in}{0.620225in}}%
\pgfpathlineto{\pgfqpoint{2.632727in}{0.620163in}}%
\pgfpathlineto{\pgfqpoint{2.633283in}{0.604816in}}%
\pgfpathlineto{\pgfqpoint{2.633838in}{0.610566in}}%
\pgfpathlineto{\pgfqpoint{2.634394in}{0.615075in}}%
\pgfpathlineto{\pgfqpoint{2.634950in}{0.611288in}}%
\pgfpathlineto{\pgfqpoint{2.635506in}{0.613334in}}%
\pgfpathlineto{\pgfqpoint{2.636062in}{0.608551in}}%
\pgfpathlineto{\pgfqpoint{2.636618in}{0.609436in}}%
\pgfpathlineto{\pgfqpoint{2.637730in}{0.616217in}}%
\pgfpathlineto{\pgfqpoint{2.639397in}{0.605990in}}%
\pgfpathlineto{\pgfqpoint{2.639953in}{0.610816in}}%
\pgfpathlineto{\pgfqpoint{2.640509in}{0.609860in}}%
\pgfpathlineto{\pgfqpoint{2.641621in}{0.603678in}}%
\pgfpathlineto{\pgfqpoint{2.642176in}{0.607918in}}%
\pgfpathlineto{\pgfqpoint{2.642732in}{0.606834in}}%
\pgfpathlineto{\pgfqpoint{2.643288in}{0.607166in}}%
\pgfpathlineto{\pgfqpoint{2.643844in}{0.603998in}}%
\pgfpathlineto{\pgfqpoint{2.644400in}{0.609683in}}%
\pgfpathlineto{\pgfqpoint{2.644956in}{0.605058in}}%
\pgfpathlineto{\pgfqpoint{2.647179in}{0.615223in}}%
\pgfpathlineto{\pgfqpoint{2.647735in}{0.614471in}}%
\pgfpathlineto{\pgfqpoint{2.648847in}{0.607423in}}%
\pgfpathlineto{\pgfqpoint{2.650514in}{0.619350in}}%
\pgfpathlineto{\pgfqpoint{2.651626in}{0.611530in}}%
\pgfpathlineto{\pgfqpoint{2.652182in}{0.613934in}}%
\pgfpathlineto{\pgfqpoint{2.652738in}{0.612800in}}%
\pgfpathlineto{\pgfqpoint{2.654405in}{0.604797in}}%
\pgfpathlineto{\pgfqpoint{2.654961in}{0.604523in}}%
\pgfpathlineto{\pgfqpoint{2.657185in}{0.620390in}}%
\pgfpathlineto{\pgfqpoint{2.658296in}{0.600744in}}%
\pgfpathlineto{\pgfqpoint{2.660520in}{0.623399in}}%
\pgfpathlineto{\pgfqpoint{2.662187in}{0.607556in}}%
\pgfpathlineto{\pgfqpoint{2.662743in}{0.615290in}}%
\pgfpathlineto{\pgfqpoint{2.663299in}{0.614293in}}%
\pgfpathlineto{\pgfqpoint{2.664967in}{0.604152in}}%
\pgfpathlineto{\pgfqpoint{2.666634in}{0.618171in}}%
\pgfpathlineto{\pgfqpoint{2.667746in}{0.602219in}}%
\pgfpathlineto{\pgfqpoint{2.668858in}{0.617396in}}%
\pgfpathlineto{\pgfqpoint{2.669414in}{0.603554in}}%
\pgfpathlineto{\pgfqpoint{2.669969in}{0.608561in}}%
\pgfpathlineto{\pgfqpoint{2.670525in}{0.620727in}}%
\pgfpathlineto{\pgfqpoint{2.671081in}{0.616002in}}%
\pgfpathlineto{\pgfqpoint{2.672193in}{0.619426in}}%
\pgfpathlineto{\pgfqpoint{2.672749in}{0.618306in}}%
\pgfpathlineto{\pgfqpoint{2.674416in}{0.608344in}}%
\pgfpathlineto{\pgfqpoint{2.674972in}{0.631067in}}%
\pgfpathlineto{\pgfqpoint{2.675528in}{0.601970in}}%
\pgfpathlineto{\pgfqpoint{2.676084in}{0.632031in}}%
\pgfpathlineto{\pgfqpoint{2.676640in}{0.616147in}}%
\pgfpathlineto{\pgfqpoint{2.677196in}{0.614563in}}%
\pgfpathlineto{\pgfqpoint{2.677752in}{0.615085in}}%
\pgfpathlineto{\pgfqpoint{2.678307in}{0.651549in}}%
\pgfpathlineto{\pgfqpoint{2.678863in}{0.604308in}}%
\pgfpathlineto{\pgfqpoint{2.679419in}{0.639057in}}%
\pgfpathlineto{\pgfqpoint{2.679975in}{0.611945in}}%
\pgfpathlineto{\pgfqpoint{2.680531in}{0.627287in}}%
\pgfpathlineto{\pgfqpoint{2.681087in}{0.612141in}}%
\pgfpathlineto{\pgfqpoint{2.681643in}{0.639192in}}%
\pgfpathlineto{\pgfqpoint{2.682198in}{0.638956in}}%
\pgfpathlineto{\pgfqpoint{2.682754in}{0.607209in}}%
\pgfpathlineto{\pgfqpoint{2.683310in}{0.624415in}}%
\pgfpathlineto{\pgfqpoint{2.684978in}{0.608419in}}%
\pgfpathlineto{\pgfqpoint{2.686089in}{0.647062in}}%
\pgfpathlineto{\pgfqpoint{2.686645in}{0.644232in}}%
\pgfpathlineto{\pgfqpoint{2.687201in}{0.625359in}}%
\pgfpathlineto{\pgfqpoint{2.687757in}{0.630841in}}%
\pgfpathlineto{\pgfqpoint{2.688313in}{0.625605in}}%
\pgfpathlineto{\pgfqpoint{2.688869in}{0.629635in}}%
\pgfpathlineto{\pgfqpoint{2.689425in}{0.630862in}}%
\pgfpathlineto{\pgfqpoint{2.689980in}{0.654310in}}%
\pgfpathlineto{\pgfqpoint{2.690536in}{0.631603in}}%
\pgfpathlineto{\pgfqpoint{2.691648in}{0.623990in}}%
\pgfpathlineto{\pgfqpoint{2.692204in}{0.643124in}}%
\pgfpathlineto{\pgfqpoint{2.692760in}{0.625134in}}%
\pgfpathlineto{\pgfqpoint{2.693316in}{0.628891in}}%
\pgfpathlineto{\pgfqpoint{2.693872in}{0.614961in}}%
\pgfpathlineto{\pgfqpoint{2.694427in}{0.636481in}}%
\pgfpathlineto{\pgfqpoint{2.694983in}{0.624592in}}%
\pgfpathlineto{\pgfqpoint{2.695539in}{0.625448in}}%
\pgfpathlineto{\pgfqpoint{2.697207in}{0.603544in}}%
\pgfpathlineto{\pgfqpoint{2.697763in}{0.632803in}}%
\pgfpathlineto{\pgfqpoint{2.698318in}{0.618761in}}%
\pgfpathlineto{\pgfqpoint{2.699986in}{0.606024in}}%
\pgfpathlineto{\pgfqpoint{2.701654in}{0.621786in}}%
\pgfpathlineto{\pgfqpoint{2.702209in}{0.606824in}}%
\pgfpathlineto{\pgfqpoint{2.702765in}{0.616698in}}%
\pgfpathlineto{\pgfqpoint{2.703321in}{0.609745in}}%
\pgfpathlineto{\pgfqpoint{2.703877in}{0.618702in}}%
\pgfpathlineto{\pgfqpoint{2.704433in}{0.604172in}}%
\pgfpathlineto{\pgfqpoint{2.704989in}{0.606906in}}%
\pgfpathlineto{\pgfqpoint{2.706100in}{0.623124in}}%
\pgfpathlineto{\pgfqpoint{2.707212in}{0.611838in}}%
\pgfpathlineto{\pgfqpoint{2.708880in}{0.634888in}}%
\pgfpathlineto{\pgfqpoint{2.709436in}{0.624525in}}%
\pgfpathlineto{\pgfqpoint{2.709992in}{0.632546in}}%
\pgfpathlineto{\pgfqpoint{2.711103in}{0.635831in}}%
\pgfpathlineto{\pgfqpoint{2.712771in}{0.615562in}}%
\pgfpathlineto{\pgfqpoint{2.713327in}{0.615694in}}%
\pgfpathlineto{\pgfqpoint{2.714438in}{0.609220in}}%
\pgfpathlineto{\pgfqpoint{2.716662in}{0.626557in}}%
\pgfpathlineto{\pgfqpoint{2.717218in}{0.626861in}}%
\pgfpathlineto{\pgfqpoint{2.718329in}{0.612586in}}%
\pgfpathlineto{\pgfqpoint{2.718885in}{0.616478in}}%
\pgfpathlineto{\pgfqpoint{2.719997in}{0.635035in}}%
\pgfpathlineto{\pgfqpoint{2.721665in}{0.603374in}}%
\pgfpathlineto{\pgfqpoint{2.722220in}{0.604793in}}%
\pgfpathlineto{\pgfqpoint{2.722776in}{0.633273in}}%
\pgfpathlineto{\pgfqpoint{2.723332in}{0.623834in}}%
\pgfpathlineto{\pgfqpoint{2.723888in}{0.629412in}}%
\pgfpathlineto{\pgfqpoint{2.724444in}{0.624310in}}%
\pgfpathlineto{\pgfqpoint{2.725000in}{0.609662in}}%
\pgfpathlineto{\pgfqpoint{2.725556in}{0.619710in}}%
\pgfpathlineto{\pgfqpoint{2.726111in}{0.634081in}}%
\pgfpathlineto{\pgfqpoint{2.727223in}{0.608357in}}%
\pgfpathlineto{\pgfqpoint{2.728335in}{0.629530in}}%
\pgfpathlineto{\pgfqpoint{2.728891in}{0.605144in}}%
\pgfpathlineto{\pgfqpoint{2.729447in}{0.626288in}}%
\pgfpathlineto{\pgfqpoint{2.730003in}{0.627078in}}%
\pgfpathlineto{\pgfqpoint{2.730558in}{0.630970in}}%
\pgfpathlineto{\pgfqpoint{2.731114in}{0.607045in}}%
\pgfpathlineto{\pgfqpoint{2.731670in}{0.617817in}}%
\pgfpathlineto{\pgfqpoint{2.732226in}{0.647537in}}%
\pgfpathlineto{\pgfqpoint{2.732782in}{0.619788in}}%
\pgfpathlineto{\pgfqpoint{2.733338in}{0.626554in}}%
\pgfpathlineto{\pgfqpoint{2.733894in}{0.619369in}}%
\pgfpathlineto{\pgfqpoint{2.734449in}{0.645844in}}%
\pgfpathlineto{\pgfqpoint{2.735005in}{0.610717in}}%
\pgfpathlineto{\pgfqpoint{2.735561in}{0.688442in}}%
\pgfpathlineto{\pgfqpoint{2.736117in}{0.617549in}}%
\pgfpathlineto{\pgfqpoint{2.737785in}{0.653898in}}%
\pgfpathlineto{\pgfqpoint{2.738340in}{0.651091in}}%
\pgfpathlineto{\pgfqpoint{2.738896in}{0.653051in}}%
\pgfpathlineto{\pgfqpoint{2.740564in}{0.618047in}}%
\pgfpathlineto{\pgfqpoint{2.741120in}{0.668849in}}%
\pgfpathlineto{\pgfqpoint{2.741676in}{0.643422in}}%
\pgfpathlineto{\pgfqpoint{2.742231in}{0.608778in}}%
\pgfpathlineto{\pgfqpoint{2.743899in}{0.649408in}}%
\pgfpathlineto{\pgfqpoint{2.744455in}{0.611893in}}%
\pgfpathlineto{\pgfqpoint{2.745011in}{0.646249in}}%
\pgfpathlineto{\pgfqpoint{2.745567in}{0.657640in}}%
\pgfpathlineto{\pgfqpoint{2.746678in}{0.621110in}}%
\pgfpathlineto{\pgfqpoint{2.747234in}{0.689568in}}%
\pgfpathlineto{\pgfqpoint{2.747790in}{0.661049in}}%
\pgfpathlineto{\pgfqpoint{2.748346in}{0.627742in}}%
\pgfpathlineto{\pgfqpoint{2.748902in}{0.629706in}}%
\pgfpathlineto{\pgfqpoint{2.749458in}{0.666661in}}%
\pgfpathlineto{\pgfqpoint{2.750014in}{0.632411in}}%
\pgfpathlineto{\pgfqpoint{2.750569in}{0.645745in}}%
\pgfpathlineto{\pgfqpoint{2.751125in}{0.620741in}}%
\pgfpathlineto{\pgfqpoint{2.751681in}{0.633140in}}%
\pgfpathlineto{\pgfqpoint{2.752793in}{0.641495in}}%
\pgfpathlineto{\pgfqpoint{2.753349in}{0.609088in}}%
\pgfpathlineto{\pgfqpoint{2.753905in}{0.624589in}}%
\pgfpathlineto{\pgfqpoint{2.754460in}{0.613966in}}%
\pgfpathlineto{\pgfqpoint{2.755016in}{0.643571in}}%
\pgfpathlineto{\pgfqpoint{2.755572in}{0.609289in}}%
\pgfpathlineto{\pgfqpoint{2.756128in}{0.611826in}}%
\pgfpathlineto{\pgfqpoint{2.756684in}{0.617603in}}%
\pgfpathlineto{\pgfqpoint{2.757240in}{0.614782in}}%
\pgfpathlineto{\pgfqpoint{2.757796in}{0.609658in}}%
\pgfpathlineto{\pgfqpoint{2.758351in}{0.618299in}}%
\pgfpathlineto{\pgfqpoint{2.758907in}{0.601741in}}%
\pgfpathlineto{\pgfqpoint{2.759463in}{0.607107in}}%
\pgfpathlineto{\pgfqpoint{2.760019in}{0.616789in}}%
\pgfpathlineto{\pgfqpoint{2.761687in}{0.602932in}}%
\pgfpathlineto{\pgfqpoint{2.762242in}{0.615567in}}%
\pgfpathlineto{\pgfqpoint{2.762798in}{0.604627in}}%
\pgfpathlineto{\pgfqpoint{2.763354in}{0.609039in}}%
\pgfpathlineto{\pgfqpoint{2.763910in}{0.601252in}}%
\pgfpathlineto{\pgfqpoint{2.764466in}{0.618014in}}%
\pgfpathlineto{\pgfqpoint{2.765022in}{0.607341in}}%
\pgfpathlineto{\pgfqpoint{2.765578in}{0.606958in}}%
\pgfpathlineto{\pgfqpoint{2.767801in}{0.630702in}}%
\pgfpathlineto{\pgfqpoint{2.768357in}{0.628650in}}%
\pgfpathlineto{\pgfqpoint{2.768913in}{0.632720in}}%
\pgfpathlineto{\pgfqpoint{2.769469in}{0.647901in}}%
\pgfpathlineto{\pgfqpoint{2.770025in}{0.637606in}}%
\pgfpathlineto{\pgfqpoint{2.770580in}{0.639602in}}%
\pgfpathlineto{\pgfqpoint{2.771136in}{0.628140in}}%
\pgfpathlineto{\pgfqpoint{2.771692in}{0.641406in}}%
\pgfpathlineto{\pgfqpoint{2.772248in}{0.640370in}}%
\pgfpathlineto{\pgfqpoint{2.773360in}{0.630499in}}%
\pgfpathlineto{\pgfqpoint{2.774471in}{0.608982in}}%
\pgfpathlineto{\pgfqpoint{2.775027in}{0.613872in}}%
\pgfpathlineto{\pgfqpoint{2.775583in}{0.612406in}}%
\pgfpathlineto{\pgfqpoint{2.776695in}{0.649311in}}%
\pgfpathlineto{\pgfqpoint{2.778362in}{0.612670in}}%
\pgfpathlineto{\pgfqpoint{2.780586in}{0.649717in}}%
\pgfpathlineto{\pgfqpoint{2.781142in}{0.611646in}}%
\pgfpathlineto{\pgfqpoint{2.781698in}{0.612527in}}%
\pgfpathlineto{\pgfqpoint{2.783365in}{0.637737in}}%
\pgfpathlineto{\pgfqpoint{2.784477in}{0.607494in}}%
\pgfpathlineto{\pgfqpoint{2.785033in}{0.613762in}}%
\pgfpathlineto{\pgfqpoint{2.785589in}{0.650796in}}%
\pgfpathlineto{\pgfqpoint{2.786145in}{0.627720in}}%
\pgfpathlineto{\pgfqpoint{2.786700in}{0.622899in}}%
\pgfpathlineto{\pgfqpoint{2.787256in}{0.623896in}}%
\pgfpathlineto{\pgfqpoint{2.787812in}{0.633190in}}%
\pgfpathlineto{\pgfqpoint{2.788368in}{0.605966in}}%
\pgfpathlineto{\pgfqpoint{2.788924in}{0.628094in}}%
\pgfpathlineto{\pgfqpoint{2.789480in}{0.626764in}}%
\pgfpathlineto{\pgfqpoint{2.790036in}{0.649254in}}%
\pgfpathlineto{\pgfqpoint{2.790591in}{0.607821in}}%
\pgfpathlineto{\pgfqpoint{2.791147in}{0.631888in}}%
\pgfpathlineto{\pgfqpoint{2.791703in}{0.657708in}}%
\pgfpathlineto{\pgfqpoint{2.792259in}{0.611617in}}%
\pgfpathlineto{\pgfqpoint{2.792815in}{0.665303in}}%
\pgfpathlineto{\pgfqpoint{2.793371in}{0.653952in}}%
\pgfpathlineto{\pgfqpoint{2.793927in}{0.611579in}}%
\pgfpathlineto{\pgfqpoint{2.794482in}{0.647321in}}%
\pgfpathlineto{\pgfqpoint{2.795594in}{0.657228in}}%
\pgfpathlineto{\pgfqpoint{2.796150in}{0.619683in}}%
\pgfpathlineto{\pgfqpoint{2.796706in}{0.668828in}}%
\pgfpathlineto{\pgfqpoint{2.797262in}{0.653497in}}%
\pgfpathlineto{\pgfqpoint{2.797818in}{0.635245in}}%
\pgfpathlineto{\pgfqpoint{2.798373in}{0.662217in}}%
\pgfpathlineto{\pgfqpoint{2.798929in}{0.635854in}}%
\pgfpathlineto{\pgfqpoint{2.799485in}{0.638147in}}%
\pgfpathlineto{\pgfqpoint{2.800597in}{0.622552in}}%
\pgfpathlineto{\pgfqpoint{2.802264in}{0.670466in}}%
\pgfpathlineto{\pgfqpoint{2.802820in}{0.659254in}}%
\pgfpathlineto{\pgfqpoint{2.803376in}{0.631432in}}%
\pgfpathlineto{\pgfqpoint{2.803932in}{0.645999in}}%
\pgfpathlineto{\pgfqpoint{2.804488in}{0.670540in}}%
\pgfpathlineto{\pgfqpoint{2.805044in}{0.661920in}}%
\pgfpathlineto{\pgfqpoint{2.806156in}{0.639581in}}%
\pgfpathlineto{\pgfqpoint{2.806711in}{0.641179in}}%
\pgfpathlineto{\pgfqpoint{2.807267in}{0.655463in}}%
\pgfpathlineto{\pgfqpoint{2.807823in}{0.647714in}}%
\pgfpathlineto{\pgfqpoint{2.808379in}{0.627016in}}%
\pgfpathlineto{\pgfqpoint{2.808935in}{0.628760in}}%
\pgfpathlineto{\pgfqpoint{2.809491in}{0.628595in}}%
\pgfpathlineto{\pgfqpoint{2.810047in}{0.635745in}}%
\pgfpathlineto{\pgfqpoint{2.811714in}{0.601350in}}%
\pgfpathlineto{\pgfqpoint{2.812270in}{0.617284in}}%
\pgfpathlineto{\pgfqpoint{2.812826in}{0.615732in}}%
\pgfpathlineto{\pgfqpoint{2.813382in}{0.615415in}}%
\pgfpathlineto{\pgfqpoint{2.813938in}{0.606486in}}%
\pgfpathlineto{\pgfqpoint{2.814493in}{0.632159in}}%
\pgfpathlineto{\pgfqpoint{2.815049in}{0.613530in}}%
\pgfpathlineto{\pgfqpoint{2.815605in}{0.617523in}}%
\pgfpathlineto{\pgfqpoint{2.816161in}{0.609942in}}%
\pgfpathlineto{\pgfqpoint{2.816717in}{0.615059in}}%
\pgfpathlineto{\pgfqpoint{2.817273in}{0.618497in}}%
\pgfpathlineto{\pgfqpoint{2.817829in}{0.616828in}}%
\pgfpathlineto{\pgfqpoint{2.818384in}{0.611904in}}%
\pgfpathlineto{\pgfqpoint{2.818940in}{0.612832in}}%
\pgfpathlineto{\pgfqpoint{2.819496in}{0.617752in}}%
\pgfpathlineto{\pgfqpoint{2.820052in}{0.610115in}}%
\pgfpathlineto{\pgfqpoint{2.820608in}{0.619293in}}%
\pgfpathlineto{\pgfqpoint{2.822275in}{0.606864in}}%
\pgfpathlineto{\pgfqpoint{2.822831in}{0.602289in}}%
\pgfpathlineto{\pgfqpoint{2.823943in}{0.618623in}}%
\pgfpathlineto{\pgfqpoint{2.824499in}{0.616094in}}%
\pgfpathlineto{\pgfqpoint{2.825055in}{0.615231in}}%
\pgfpathlineto{\pgfqpoint{2.825611in}{0.607891in}}%
\pgfpathlineto{\pgfqpoint{2.826167in}{0.613565in}}%
\pgfpathlineto{\pgfqpoint{2.826722in}{0.612599in}}%
\pgfpathlineto{\pgfqpoint{2.827278in}{0.616004in}}%
\pgfpathlineto{\pgfqpoint{2.828390in}{0.635173in}}%
\pgfpathlineto{\pgfqpoint{2.828946in}{0.635136in}}%
\pgfpathlineto{\pgfqpoint{2.829502in}{0.633163in}}%
\pgfpathlineto{\pgfqpoint{2.830058in}{0.634934in}}%
\pgfpathlineto{\pgfqpoint{2.831169in}{0.649526in}}%
\pgfpathlineto{\pgfqpoint{2.831725in}{0.646106in}}%
\pgfpathlineto{\pgfqpoint{2.832281in}{0.632294in}}%
\pgfpathlineto{\pgfqpoint{2.832837in}{0.633523in}}%
\pgfpathlineto{\pgfqpoint{2.833393in}{0.639038in}}%
\pgfpathlineto{\pgfqpoint{2.835060in}{0.604586in}}%
\pgfpathlineto{\pgfqpoint{2.836728in}{0.640540in}}%
\pgfpathlineto{\pgfqpoint{2.838395in}{0.607347in}}%
\pgfpathlineto{\pgfqpoint{2.840063in}{0.639710in}}%
\pgfpathlineto{\pgfqpoint{2.840619in}{0.638324in}}%
\pgfpathlineto{\pgfqpoint{2.841731in}{0.606560in}}%
\pgfpathlineto{\pgfqpoint{2.842842in}{0.650159in}}%
\pgfpathlineto{\pgfqpoint{2.844510in}{0.612712in}}%
\pgfpathlineto{\pgfqpoint{2.845066in}{0.630771in}}%
\pgfpathlineto{\pgfqpoint{2.845622in}{0.622018in}}%
\pgfpathlineto{\pgfqpoint{2.846178in}{0.627200in}}%
\pgfpathlineto{\pgfqpoint{2.846733in}{0.610610in}}%
\pgfpathlineto{\pgfqpoint{2.847289in}{0.652304in}}%
\pgfpathlineto{\pgfqpoint{2.847845in}{0.606996in}}%
\pgfpathlineto{\pgfqpoint{2.848401in}{0.640587in}}%
\pgfpathlineto{\pgfqpoint{2.850069in}{0.628687in}}%
\pgfpathlineto{\pgfqpoint{2.850624in}{0.675999in}}%
\pgfpathlineto{\pgfqpoint{2.851180in}{0.625071in}}%
\pgfpathlineto{\pgfqpoint{2.851736in}{0.646013in}}%
\pgfpathlineto{\pgfqpoint{2.852292in}{0.651845in}}%
\pgfpathlineto{\pgfqpoint{2.852848in}{0.672789in}}%
\pgfpathlineto{\pgfqpoint{2.853404in}{0.657744in}}%
\pgfpathlineto{\pgfqpoint{2.853960in}{0.670468in}}%
\pgfpathlineto{\pgfqpoint{2.855627in}{0.619243in}}%
\pgfpathlineto{\pgfqpoint{2.856183in}{0.677404in}}%
\pgfpathlineto{\pgfqpoint{2.856739in}{0.664278in}}%
\pgfpathlineto{\pgfqpoint{2.857295in}{0.608096in}}%
\pgfpathlineto{\pgfqpoint{2.857851in}{0.653238in}}%
\pgfpathlineto{\pgfqpoint{2.858406in}{0.622739in}}%
\pgfpathlineto{\pgfqpoint{2.858962in}{0.644072in}}%
\pgfpathlineto{\pgfqpoint{2.859518in}{0.673332in}}%
\pgfpathlineto{\pgfqpoint{2.860074in}{0.666858in}}%
\pgfpathlineto{\pgfqpoint{2.860630in}{0.644895in}}%
\pgfpathlineto{\pgfqpoint{2.861186in}{0.651756in}}%
\pgfpathlineto{\pgfqpoint{2.861742in}{0.653513in}}%
\pgfpathlineto{\pgfqpoint{2.862298in}{0.696812in}}%
\pgfpathlineto{\pgfqpoint{2.863965in}{0.610675in}}%
\pgfpathlineto{\pgfqpoint{2.864521in}{0.659162in}}%
\pgfpathlineto{\pgfqpoint{2.865077in}{0.622525in}}%
\pgfpathlineto{\pgfqpoint{2.865633in}{0.646148in}}%
\pgfpathlineto{\pgfqpoint{2.867300in}{0.608468in}}%
\pgfpathlineto{\pgfqpoint{2.867856in}{0.635145in}}%
\pgfpathlineto{\pgfqpoint{2.868412in}{0.621433in}}%
\pgfpathlineto{\pgfqpoint{2.868968in}{0.609387in}}%
\pgfpathlineto{\pgfqpoint{2.869524in}{0.614155in}}%
\pgfpathlineto{\pgfqpoint{2.870635in}{0.628686in}}%
\pgfpathlineto{\pgfqpoint{2.872303in}{0.605738in}}%
\pgfpathlineto{\pgfqpoint{2.875082in}{0.614877in}}%
\pgfpathlineto{\pgfqpoint{2.876750in}{0.605135in}}%
\pgfpathlineto{\pgfqpoint{2.877306in}{0.608322in}}%
\pgfpathlineto{\pgfqpoint{2.877862in}{0.614809in}}%
\pgfpathlineto{\pgfqpoint{2.878973in}{0.605215in}}%
\pgfpathlineto{\pgfqpoint{2.879529in}{0.614376in}}%
\pgfpathlineto{\pgfqpoint{2.880085in}{0.613107in}}%
\pgfpathlineto{\pgfqpoint{2.880641in}{0.604008in}}%
\pgfpathlineto{\pgfqpoint{2.881197in}{0.609360in}}%
\pgfpathlineto{\pgfqpoint{2.883976in}{0.601485in}}%
\pgfpathlineto{\pgfqpoint{2.885644in}{0.614243in}}%
\pgfpathlineto{\pgfqpoint{2.886200in}{0.607939in}}%
\pgfpathlineto{\pgfqpoint{2.888979in}{0.635013in}}%
\pgfpathlineto{\pgfqpoint{2.889535in}{0.626341in}}%
\pgfpathlineto{\pgfqpoint{2.890646in}{0.646305in}}%
\pgfpathlineto{\pgfqpoint{2.891202in}{0.638911in}}%
\pgfpathlineto{\pgfqpoint{2.891758in}{0.637035in}}%
\pgfpathlineto{\pgfqpoint{2.892314in}{0.645248in}}%
\pgfpathlineto{\pgfqpoint{2.892870in}{0.644122in}}%
\pgfpathlineto{\pgfqpoint{2.894537in}{0.602216in}}%
\pgfpathlineto{\pgfqpoint{2.896761in}{0.653562in}}%
\pgfpathlineto{\pgfqpoint{2.898429in}{0.605040in}}%
\pgfpathlineto{\pgfqpoint{2.899540in}{0.650887in}}%
\pgfpathlineto{\pgfqpoint{2.900096in}{0.641446in}}%
\pgfpathlineto{\pgfqpoint{2.900652in}{0.638576in}}%
\pgfpathlineto{\pgfqpoint{2.901208in}{0.607921in}}%
\pgfpathlineto{\pgfqpoint{2.901764in}{0.614008in}}%
\pgfpathlineto{\pgfqpoint{2.902875in}{0.649141in}}%
\pgfpathlineto{\pgfqpoint{2.903431in}{0.607239in}}%
\pgfpathlineto{\pgfqpoint{2.903987in}{0.609525in}}%
\pgfpathlineto{\pgfqpoint{2.904543in}{0.648772in}}%
\pgfpathlineto{\pgfqpoint{2.905099in}{0.628590in}}%
\pgfpathlineto{\pgfqpoint{2.905655in}{0.629056in}}%
\pgfpathlineto{\pgfqpoint{2.906211in}{0.619914in}}%
\pgfpathlineto{\pgfqpoint{2.906766in}{0.654382in}}%
\pgfpathlineto{\pgfqpoint{2.907322in}{0.609626in}}%
\pgfpathlineto{\pgfqpoint{2.907878in}{0.676793in}}%
\pgfpathlineto{\pgfqpoint{2.908434in}{0.625477in}}%
\pgfpathlineto{\pgfqpoint{2.908990in}{0.623285in}}%
\pgfpathlineto{\pgfqpoint{2.910102in}{0.701068in}}%
\pgfpathlineto{\pgfqpoint{2.910657in}{0.681948in}}%
\pgfpathlineto{\pgfqpoint{2.911213in}{0.626898in}}%
\pgfpathlineto{\pgfqpoint{2.911769in}{0.674447in}}%
\pgfpathlineto{\pgfqpoint{2.912325in}{0.657139in}}%
\pgfpathlineto{\pgfqpoint{2.912881in}{0.690477in}}%
\pgfpathlineto{\pgfqpoint{2.913437in}{0.681295in}}%
\pgfpathlineto{\pgfqpoint{2.913993in}{0.669344in}}%
\pgfpathlineto{\pgfqpoint{2.914548in}{0.621222in}}%
\pgfpathlineto{\pgfqpoint{2.915104in}{0.623796in}}%
\pgfpathlineto{\pgfqpoint{2.916216in}{0.618081in}}%
\pgfpathlineto{\pgfqpoint{2.916772in}{0.631585in}}%
\pgfpathlineto{\pgfqpoint{2.917328in}{0.734600in}}%
\pgfpathlineto{\pgfqpoint{2.917884in}{0.682323in}}%
\pgfpathlineto{\pgfqpoint{2.918440in}{0.632842in}}%
\pgfpathlineto{\pgfqpoint{2.918995in}{0.653211in}}%
\pgfpathlineto{\pgfqpoint{2.919551in}{0.701966in}}%
\pgfpathlineto{\pgfqpoint{2.921219in}{0.628754in}}%
\pgfpathlineto{\pgfqpoint{2.922886in}{0.661001in}}%
\pgfpathlineto{\pgfqpoint{2.924554in}{0.606454in}}%
\pgfpathlineto{\pgfqpoint{2.925110in}{0.628909in}}%
\pgfpathlineto{\pgfqpoint{2.925666in}{0.622117in}}%
\pgfpathlineto{\pgfqpoint{2.926222in}{0.613363in}}%
\pgfpathlineto{\pgfqpoint{2.926777in}{0.620876in}}%
\pgfpathlineto{\pgfqpoint{2.928445in}{0.602859in}}%
\pgfpathlineto{\pgfqpoint{2.929001in}{0.619499in}}%
\pgfpathlineto{\pgfqpoint{2.929557in}{0.612324in}}%
\pgfpathlineto{\pgfqpoint{2.930668in}{0.621632in}}%
\pgfpathlineto{\pgfqpoint{2.931224in}{0.617897in}}%
\pgfpathlineto{\pgfqpoint{2.931780in}{0.602829in}}%
\pgfpathlineto{\pgfqpoint{2.932336in}{0.621499in}}%
\pgfpathlineto{\pgfqpoint{2.932892in}{0.609659in}}%
\pgfpathlineto{\pgfqpoint{2.933448in}{0.617836in}}%
\pgfpathlineto{\pgfqpoint{2.934004in}{0.606887in}}%
\pgfpathlineto{\pgfqpoint{2.934559in}{0.607161in}}%
\pgfpathlineto{\pgfqpoint{2.935115in}{0.608695in}}%
\pgfpathlineto{\pgfqpoint{2.935671in}{0.617261in}}%
\pgfpathlineto{\pgfqpoint{2.936227in}{0.609830in}}%
\pgfpathlineto{\pgfqpoint{2.937339in}{0.602637in}}%
\pgfpathlineto{\pgfqpoint{2.937895in}{0.603841in}}%
\pgfpathlineto{\pgfqpoint{2.938451in}{0.608408in}}%
\pgfpathlineto{\pgfqpoint{2.939006in}{0.607444in}}%
\pgfpathlineto{\pgfqpoint{2.939562in}{0.604426in}}%
\pgfpathlineto{\pgfqpoint{2.940118in}{0.617863in}}%
\pgfpathlineto{\pgfqpoint{2.940674in}{0.607391in}}%
\pgfpathlineto{\pgfqpoint{2.941230in}{0.611710in}}%
\pgfpathlineto{\pgfqpoint{2.941786in}{0.609919in}}%
\pgfpathlineto{\pgfqpoint{2.942897in}{0.609082in}}%
\pgfpathlineto{\pgfqpoint{2.943453in}{0.605072in}}%
\pgfpathlineto{\pgfqpoint{2.944009in}{0.613486in}}%
\pgfpathlineto{\pgfqpoint{2.944565in}{0.606657in}}%
\pgfpathlineto{\pgfqpoint{2.945121in}{0.612411in}}%
\pgfpathlineto{\pgfqpoint{2.945677in}{0.605365in}}%
\pgfpathlineto{\pgfqpoint{2.946233in}{0.608188in}}%
\pgfpathlineto{\pgfqpoint{2.946788in}{0.605627in}}%
\pgfpathlineto{\pgfqpoint{2.949012in}{0.652510in}}%
\pgfpathlineto{\pgfqpoint{2.949568in}{0.631397in}}%
\pgfpathlineto{\pgfqpoint{2.950124in}{0.640406in}}%
\pgfpathlineto{\pgfqpoint{2.951791in}{0.661720in}}%
\pgfpathlineto{\pgfqpoint{2.952347in}{0.661829in}}%
\pgfpathlineto{\pgfqpoint{2.954570in}{0.608246in}}%
\pgfpathlineto{\pgfqpoint{2.955126in}{0.615673in}}%
\pgfpathlineto{\pgfqpoint{2.956794in}{0.673122in}}%
\pgfpathlineto{\pgfqpoint{2.958462in}{0.619979in}}%
\pgfpathlineto{\pgfqpoint{2.959017in}{0.648057in}}%
\pgfpathlineto{\pgfqpoint{2.959573in}{0.641098in}}%
\pgfpathlineto{\pgfqpoint{2.960129in}{0.639264in}}%
\pgfpathlineto{\pgfqpoint{2.961241in}{0.611178in}}%
\pgfpathlineto{\pgfqpoint{2.962353in}{0.671951in}}%
\pgfpathlineto{\pgfqpoint{2.962908in}{0.608101in}}%
\pgfpathlineto{\pgfqpoint{2.963464in}{0.620989in}}%
\pgfpathlineto{\pgfqpoint{2.964020in}{0.659531in}}%
\pgfpathlineto{\pgfqpoint{2.964576in}{0.618950in}}%
\pgfpathlineto{\pgfqpoint{2.965132in}{0.644199in}}%
\pgfpathlineto{\pgfqpoint{2.965688in}{0.663832in}}%
\pgfpathlineto{\pgfqpoint{2.966799in}{0.630832in}}%
\pgfpathlineto{\pgfqpoint{2.967355in}{0.706353in}}%
\pgfpathlineto{\pgfqpoint{2.967911in}{0.682735in}}%
\pgfpathlineto{\pgfqpoint{2.968467in}{0.658078in}}%
\pgfpathlineto{\pgfqpoint{2.969023in}{0.694537in}}%
\pgfpathlineto{\pgfqpoint{2.969579in}{0.645511in}}%
\pgfpathlineto{\pgfqpoint{2.970135in}{0.706129in}}%
\pgfpathlineto{\pgfqpoint{2.970690in}{0.609807in}}%
\pgfpathlineto{\pgfqpoint{2.971246in}{0.683170in}}%
\pgfpathlineto{\pgfqpoint{2.972358in}{0.636748in}}%
\pgfpathlineto{\pgfqpoint{2.972914in}{0.667384in}}%
\pgfpathlineto{\pgfqpoint{2.973470in}{0.614330in}}%
\pgfpathlineto{\pgfqpoint{2.974026in}{0.655615in}}%
\pgfpathlineto{\pgfqpoint{2.974582in}{0.740892in}}%
\pgfpathlineto{\pgfqpoint{2.975137in}{0.681906in}}%
\pgfpathlineto{\pgfqpoint{2.975693in}{0.629702in}}%
\pgfpathlineto{\pgfqpoint{2.976249in}{0.655846in}}%
\pgfpathlineto{\pgfqpoint{2.976805in}{0.660063in}}%
\pgfpathlineto{\pgfqpoint{2.977361in}{0.679114in}}%
\pgfpathlineto{\pgfqpoint{2.977917in}{0.670458in}}%
\pgfpathlineto{\pgfqpoint{2.978473in}{0.660822in}}%
\pgfpathlineto{\pgfqpoint{2.979028in}{0.627276in}}%
\pgfpathlineto{\pgfqpoint{2.979584in}{0.664976in}}%
\pgfpathlineto{\pgfqpoint{2.980140in}{0.627781in}}%
\pgfpathlineto{\pgfqpoint{2.982919in}{0.608009in}}%
\pgfpathlineto{\pgfqpoint{2.984587in}{0.627345in}}%
\pgfpathlineto{\pgfqpoint{2.985143in}{0.604524in}}%
\pgfpathlineto{\pgfqpoint{2.985699in}{0.611296in}}%
\pgfpathlineto{\pgfqpoint{2.986255in}{0.620888in}}%
\pgfpathlineto{\pgfqpoint{2.986810in}{0.617928in}}%
\pgfpathlineto{\pgfqpoint{2.987922in}{0.620138in}}%
\pgfpathlineto{\pgfqpoint{2.988478in}{0.604360in}}%
\pgfpathlineto{\pgfqpoint{2.989034in}{0.619477in}}%
\pgfpathlineto{\pgfqpoint{2.989590in}{0.609965in}}%
\pgfpathlineto{\pgfqpoint{2.990146in}{0.613570in}}%
\pgfpathlineto{\pgfqpoint{2.990701in}{0.615406in}}%
\pgfpathlineto{\pgfqpoint{2.991257in}{0.613143in}}%
\pgfpathlineto{\pgfqpoint{2.991813in}{0.627963in}}%
\pgfpathlineto{\pgfqpoint{2.992925in}{0.604110in}}%
\pgfpathlineto{\pgfqpoint{2.993481in}{0.613760in}}%
\pgfpathlineto{\pgfqpoint{2.994037in}{0.610458in}}%
\pgfpathlineto{\pgfqpoint{2.994593in}{0.604204in}}%
\pgfpathlineto{\pgfqpoint{2.995148in}{0.617360in}}%
\pgfpathlineto{\pgfqpoint{2.995704in}{0.611360in}}%
\pgfpathlineto{\pgfqpoint{2.996260in}{0.609949in}}%
\pgfpathlineto{\pgfqpoint{2.996816in}{0.617404in}}%
\pgfpathlineto{\pgfqpoint{2.997928in}{0.606010in}}%
\pgfpathlineto{\pgfqpoint{2.998484in}{0.619876in}}%
\pgfpathlineto{\pgfqpoint{2.999039in}{0.618174in}}%
\pgfpathlineto{\pgfqpoint{2.999595in}{0.618627in}}%
\pgfpathlineto{\pgfqpoint{3.000151in}{0.616987in}}%
\pgfpathlineto{\pgfqpoint{3.000707in}{0.606090in}}%
\pgfpathlineto{\pgfqpoint{3.001263in}{0.612634in}}%
\pgfpathlineto{\pgfqpoint{3.003486in}{0.604374in}}%
\pgfpathlineto{\pgfqpoint{3.004042in}{0.618723in}}%
\pgfpathlineto{\pgfqpoint{3.004598in}{0.610752in}}%
\pgfpathlineto{\pgfqpoint{3.005154in}{0.616124in}}%
\pgfpathlineto{\pgfqpoint{3.005710in}{0.611696in}}%
\pgfpathlineto{\pgfqpoint{3.006266in}{0.605329in}}%
\pgfpathlineto{\pgfqpoint{3.006821in}{0.619085in}}%
\pgfpathlineto{\pgfqpoint{3.007377in}{0.617637in}}%
\pgfpathlineto{\pgfqpoint{3.007933in}{0.615253in}}%
\pgfpathlineto{\pgfqpoint{3.010712in}{0.643165in}}%
\pgfpathlineto{\pgfqpoint{3.011268in}{0.639424in}}%
\pgfpathlineto{\pgfqpoint{3.011824in}{0.663310in}}%
\pgfpathlineto{\pgfqpoint{3.012380in}{0.645210in}}%
\pgfpathlineto{\pgfqpoint{3.015159in}{0.615041in}}%
\pgfpathlineto{\pgfqpoint{3.015715in}{0.617421in}}%
\pgfpathlineto{\pgfqpoint{3.016271in}{0.653091in}}%
\pgfpathlineto{\pgfqpoint{3.016827in}{0.650177in}}%
\pgfpathlineto{\pgfqpoint{3.017939in}{0.613169in}}%
\pgfpathlineto{\pgfqpoint{3.018495in}{0.624076in}}%
\pgfpathlineto{\pgfqpoint{3.019050in}{0.629476in}}%
\pgfpathlineto{\pgfqpoint{3.019606in}{0.658450in}}%
\pgfpathlineto{\pgfqpoint{3.020718in}{0.609767in}}%
\pgfpathlineto{\pgfqpoint{3.021830in}{0.643342in}}%
\pgfpathlineto{\pgfqpoint{3.022386in}{0.613286in}}%
\pgfpathlineto{\pgfqpoint{3.022941in}{0.655446in}}%
\pgfpathlineto{\pgfqpoint{3.023497in}{0.633464in}}%
\pgfpathlineto{\pgfqpoint{3.024053in}{0.618677in}}%
\pgfpathlineto{\pgfqpoint{3.025165in}{0.667823in}}%
\pgfpathlineto{\pgfqpoint{3.025721in}{0.655979in}}%
\pgfpathlineto{\pgfqpoint{3.026277in}{0.630282in}}%
\pgfpathlineto{\pgfqpoint{3.026832in}{0.633952in}}%
\pgfpathlineto{\pgfqpoint{3.028500in}{0.694493in}}%
\pgfpathlineto{\pgfqpoint{3.029612in}{0.648366in}}%
\pgfpathlineto{\pgfqpoint{3.030168in}{0.610247in}}%
\pgfpathlineto{\pgfqpoint{3.030724in}{0.617412in}}%
\pgfpathlineto{\pgfqpoint{3.031279in}{0.624064in}}%
\pgfpathlineto{\pgfqpoint{3.032391in}{0.705008in}}%
\pgfpathlineto{\pgfqpoint{3.034059in}{0.614452in}}%
\pgfpathlineto{\pgfqpoint{3.034615in}{0.679887in}}%
\pgfpathlineto{\pgfqpoint{3.035170in}{0.649131in}}%
\pgfpathlineto{\pgfqpoint{3.035726in}{0.658490in}}%
\pgfpathlineto{\pgfqpoint{3.037394in}{0.604770in}}%
\pgfpathlineto{\pgfqpoint{3.037950in}{0.627823in}}%
\pgfpathlineto{\pgfqpoint{3.038506in}{0.625973in}}%
\pgfpathlineto{\pgfqpoint{3.039061in}{0.627026in}}%
\pgfpathlineto{\pgfqpoint{3.040173in}{0.607209in}}%
\pgfpathlineto{\pgfqpoint{3.040729in}{0.621325in}}%
\pgfpathlineto{\pgfqpoint{3.041285in}{0.614557in}}%
\pgfpathlineto{\pgfqpoint{3.041841in}{0.619902in}}%
\pgfpathlineto{\pgfqpoint{3.042397in}{0.601650in}}%
\pgfpathlineto{\pgfqpoint{3.042952in}{0.621713in}}%
\pgfpathlineto{\pgfqpoint{3.043508in}{0.614137in}}%
\pgfpathlineto{\pgfqpoint{3.045176in}{0.612285in}}%
\pgfpathlineto{\pgfqpoint{3.046288in}{0.615288in}}%
\pgfpathlineto{\pgfqpoint{3.047399in}{0.605767in}}%
\pgfpathlineto{\pgfqpoint{3.047955in}{0.609873in}}%
\pgfpathlineto{\pgfqpoint{3.048511in}{0.606580in}}%
\pgfpathlineto{\pgfqpoint{3.049067in}{0.604787in}}%
\pgfpathlineto{\pgfqpoint{3.050735in}{0.609115in}}%
\pgfpathlineto{\pgfqpoint{3.051290in}{0.604191in}}%
\pgfpathlineto{\pgfqpoint{3.051846in}{0.605387in}}%
\pgfpathlineto{\pgfqpoint{3.052958in}{0.615080in}}%
\pgfpathlineto{\pgfqpoint{3.054626in}{0.608379in}}%
\pgfpathlineto{\pgfqpoint{3.055181in}{0.613424in}}%
\pgfpathlineto{\pgfqpoint{3.056293in}{0.603705in}}%
\pgfpathlineto{\pgfqpoint{3.056849in}{0.614194in}}%
\pgfpathlineto{\pgfqpoint{3.057405in}{0.611134in}}%
\pgfpathlineto{\pgfqpoint{3.057961in}{0.604520in}}%
\pgfpathlineto{\pgfqpoint{3.059628in}{0.612322in}}%
\pgfpathlineto{\pgfqpoint{3.060184in}{0.617993in}}%
\pgfpathlineto{\pgfqpoint{3.060740in}{0.602877in}}%
\pgfpathlineto{\pgfqpoint{3.061296in}{0.621593in}}%
\pgfpathlineto{\pgfqpoint{3.061852in}{0.613968in}}%
\pgfpathlineto{\pgfqpoint{3.062408in}{0.615096in}}%
\pgfpathlineto{\pgfqpoint{3.062963in}{0.613436in}}%
\pgfpathlineto{\pgfqpoint{3.063519in}{0.607665in}}%
\pgfpathlineto{\pgfqpoint{3.064075in}{0.618608in}}%
\pgfpathlineto{\pgfqpoint{3.064631in}{0.600751in}}%
\pgfpathlineto{\pgfqpoint{3.065187in}{0.622899in}}%
\pgfpathlineto{\pgfqpoint{3.065743in}{0.611137in}}%
\pgfpathlineto{\pgfqpoint{3.066854in}{0.617176in}}%
\pgfpathlineto{\pgfqpoint{3.067410in}{0.610577in}}%
\pgfpathlineto{\pgfqpoint{3.067966in}{0.614213in}}%
\pgfpathlineto{\pgfqpoint{3.068522in}{0.625573in}}%
\pgfpathlineto{\pgfqpoint{3.069078in}{0.613508in}}%
\pgfpathlineto{\pgfqpoint{3.070746in}{0.633031in}}%
\pgfpathlineto{\pgfqpoint{3.072413in}{0.657297in}}%
\pgfpathlineto{\pgfqpoint{3.072969in}{0.658866in}}%
\pgfpathlineto{\pgfqpoint{3.074637in}{0.617310in}}%
\pgfpathlineto{\pgfqpoint{3.075192in}{0.627568in}}%
\pgfpathlineto{\pgfqpoint{3.075748in}{0.661101in}}%
\pgfpathlineto{\pgfqpoint{3.076304in}{0.645234in}}%
\pgfpathlineto{\pgfqpoint{3.077416in}{0.621631in}}%
\pgfpathlineto{\pgfqpoint{3.077972in}{0.625255in}}%
\pgfpathlineto{\pgfqpoint{3.078528in}{0.625805in}}%
\pgfpathlineto{\pgfqpoint{3.079083in}{0.660396in}}%
\pgfpathlineto{\pgfqpoint{3.079639in}{0.620231in}}%
\pgfpathlineto{\pgfqpoint{3.080195in}{0.621823in}}%
\pgfpathlineto{\pgfqpoint{3.081307in}{0.630927in}}%
\pgfpathlineto{\pgfqpoint{3.081863in}{0.617874in}}%
\pgfpathlineto{\pgfqpoint{3.082419in}{0.677336in}}%
\pgfpathlineto{\pgfqpoint{3.082974in}{0.657659in}}%
\pgfpathlineto{\pgfqpoint{3.083530in}{0.619859in}}%
\pgfpathlineto{\pgfqpoint{3.085198in}{0.707400in}}%
\pgfpathlineto{\pgfqpoint{3.086866in}{0.627719in}}%
\pgfpathlineto{\pgfqpoint{3.087421in}{0.683398in}}%
\pgfpathlineto{\pgfqpoint{3.087977in}{0.653432in}}%
\pgfpathlineto{\pgfqpoint{3.088533in}{0.620903in}}%
\pgfpathlineto{\pgfqpoint{3.089089in}{0.653262in}}%
\pgfpathlineto{\pgfqpoint{3.089645in}{0.734084in}}%
\pgfpathlineto{\pgfqpoint{3.090201in}{0.637786in}}%
\pgfpathlineto{\pgfqpoint{3.090757in}{0.638076in}}%
\pgfpathlineto{\pgfqpoint{3.091312in}{0.640564in}}%
\pgfpathlineto{\pgfqpoint{3.091868in}{0.650710in}}%
\pgfpathlineto{\pgfqpoint{3.092424in}{0.623075in}}%
\pgfpathlineto{\pgfqpoint{3.092980in}{0.652083in}}%
\pgfpathlineto{\pgfqpoint{3.094092in}{0.609804in}}%
\pgfpathlineto{\pgfqpoint{3.094648in}{0.610314in}}%
\pgfpathlineto{\pgfqpoint{3.095203in}{0.622250in}}%
\pgfpathlineto{\pgfqpoint{3.095759in}{0.616160in}}%
\pgfpathlineto{\pgfqpoint{3.096315in}{0.610492in}}%
\pgfpathlineto{\pgfqpoint{3.096871in}{0.616483in}}%
\pgfpathlineto{\pgfqpoint{3.097427in}{0.636029in}}%
\pgfpathlineto{\pgfqpoint{3.097983in}{0.606270in}}%
\pgfpathlineto{\pgfqpoint{3.098539in}{0.610812in}}%
\pgfpathlineto{\pgfqpoint{3.099094in}{0.616926in}}%
\pgfpathlineto{\pgfqpoint{3.099650in}{0.602481in}}%
\pgfpathlineto{\pgfqpoint{3.100206in}{0.623165in}}%
\pgfpathlineto{\pgfqpoint{3.100762in}{0.611295in}}%
\pgfpathlineto{\pgfqpoint{3.101318in}{0.622274in}}%
\pgfpathlineto{\pgfqpoint{3.101874in}{0.607806in}}%
\pgfpathlineto{\pgfqpoint{3.102430in}{0.608815in}}%
\pgfpathlineto{\pgfqpoint{3.102985in}{0.625047in}}%
\pgfpathlineto{\pgfqpoint{3.103541in}{0.610132in}}%
\pgfpathlineto{\pgfqpoint{3.104653in}{0.608751in}}%
\pgfpathlineto{\pgfqpoint{3.105209in}{0.603830in}}%
\pgfpathlineto{\pgfqpoint{3.106321in}{0.625575in}}%
\pgfpathlineto{\pgfqpoint{3.106877in}{0.618479in}}%
\pgfpathlineto{\pgfqpoint{3.107432in}{0.612637in}}%
\pgfpathlineto{\pgfqpoint{3.107988in}{0.619006in}}%
\pgfpathlineto{\pgfqpoint{3.108544in}{0.608784in}}%
\pgfpathlineto{\pgfqpoint{3.109100in}{0.622557in}}%
\pgfpathlineto{\pgfqpoint{3.109656in}{0.615027in}}%
\pgfpathlineto{\pgfqpoint{3.110212in}{0.606437in}}%
\pgfpathlineto{\pgfqpoint{3.110768in}{0.623100in}}%
\pgfpathlineto{\pgfqpoint{3.111323in}{0.618528in}}%
\pgfpathlineto{\pgfqpoint{3.112435in}{0.622314in}}%
\pgfpathlineto{\pgfqpoint{3.112991in}{0.606819in}}%
\pgfpathlineto{\pgfqpoint{3.113547in}{0.610348in}}%
\pgfpathlineto{\pgfqpoint{3.114103in}{0.610155in}}%
\pgfpathlineto{\pgfqpoint{3.115214in}{0.614163in}}%
\pgfpathlineto{\pgfqpoint{3.115770in}{0.625364in}}%
\pgfpathlineto{\pgfqpoint{3.116326in}{0.604752in}}%
\pgfpathlineto{\pgfqpoint{3.116882in}{0.616854in}}%
\pgfpathlineto{\pgfqpoint{3.117994in}{0.605949in}}%
\pgfpathlineto{\pgfqpoint{3.119105in}{0.618821in}}%
\pgfpathlineto{\pgfqpoint{3.120773in}{0.604686in}}%
\pgfpathlineto{\pgfqpoint{3.121329in}{0.615422in}}%
\pgfpathlineto{\pgfqpoint{3.121885in}{0.602500in}}%
\pgfpathlineto{\pgfqpoint{3.122441in}{0.607764in}}%
\pgfpathlineto{\pgfqpoint{3.122996in}{0.607901in}}%
\pgfpathlineto{\pgfqpoint{3.123552in}{0.616409in}}%
\pgfpathlineto{\pgfqpoint{3.124664in}{0.600595in}}%
\pgfpathlineto{\pgfqpoint{3.125220in}{0.616203in}}%
\pgfpathlineto{\pgfqpoint{3.125776in}{0.614865in}}%
\pgfpathlineto{\pgfqpoint{3.126888in}{0.606087in}}%
\pgfpathlineto{\pgfqpoint{3.127999in}{0.617018in}}%
\pgfpathlineto{\pgfqpoint{3.128555in}{0.607196in}}%
\pgfpathlineto{\pgfqpoint{3.129111in}{0.616815in}}%
\pgfpathlineto{\pgfqpoint{3.131334in}{0.671098in}}%
\pgfpathlineto{\pgfqpoint{3.133002in}{0.651352in}}%
\pgfpathlineto{\pgfqpoint{3.133558in}{0.652445in}}%
\pgfpathlineto{\pgfqpoint{3.134114in}{0.644526in}}%
\pgfpathlineto{\pgfqpoint{3.134670in}{0.619862in}}%
\pgfpathlineto{\pgfqpoint{3.136337in}{0.655671in}}%
\pgfpathlineto{\pgfqpoint{3.137449in}{0.618774in}}%
\pgfpathlineto{\pgfqpoint{3.138005in}{0.627492in}}%
\pgfpathlineto{\pgfqpoint{3.139672in}{0.665606in}}%
\pgfpathlineto{\pgfqpoint{3.140228in}{0.669749in}}%
\pgfpathlineto{\pgfqpoint{3.140784in}{0.662367in}}%
\pgfpathlineto{\pgfqpoint{3.141340in}{0.666166in}}%
\pgfpathlineto{\pgfqpoint{3.141896in}{0.673705in}}%
\pgfpathlineto{\pgfqpoint{3.142452in}{0.704388in}}%
\pgfpathlineto{\pgfqpoint{3.143007in}{0.661893in}}%
\pgfpathlineto{\pgfqpoint{3.144675in}{0.766332in}}%
\pgfpathlineto{\pgfqpoint{3.145787in}{0.608318in}}%
\pgfpathlineto{\pgfqpoint{3.147454in}{0.719108in}}%
\pgfpathlineto{\pgfqpoint{3.149122in}{0.609004in}}%
\pgfpathlineto{\pgfqpoint{3.149678in}{0.654607in}}%
\pgfpathlineto{\pgfqpoint{3.150234in}{0.639199in}}%
\pgfpathlineto{\pgfqpoint{3.150790in}{0.639031in}}%
\pgfpathlineto{\pgfqpoint{3.151901in}{0.613141in}}%
\pgfpathlineto{\pgfqpoint{3.152457in}{0.621081in}}%
\pgfpathlineto{\pgfqpoint{3.153013in}{0.620501in}}%
\pgfpathlineto{\pgfqpoint{3.154125in}{0.607879in}}%
\pgfpathlineto{\pgfqpoint{3.154681in}{0.624737in}}%
\pgfpathlineto{\pgfqpoint{3.155236in}{0.615916in}}%
\pgfpathlineto{\pgfqpoint{3.155792in}{0.611919in}}%
\pgfpathlineto{\pgfqpoint{3.156348in}{0.626018in}}%
\pgfpathlineto{\pgfqpoint{3.156904in}{0.618462in}}%
\pgfpathlineto{\pgfqpoint{3.157460in}{0.615236in}}%
\pgfpathlineto{\pgfqpoint{3.158016in}{0.635130in}}%
\pgfpathlineto{\pgfqpoint{3.158572in}{0.618183in}}%
\pgfpathlineto{\pgfqpoint{3.159683in}{0.602272in}}%
\pgfpathlineto{\pgfqpoint{3.161351in}{0.615635in}}%
\pgfpathlineto{\pgfqpoint{3.162463in}{0.606406in}}%
\pgfpathlineto{\pgfqpoint{3.164130in}{0.631474in}}%
\pgfpathlineto{\pgfqpoint{3.165798in}{0.601769in}}%
\pgfpathlineto{\pgfqpoint{3.166354in}{0.626239in}}%
\pgfpathlineto{\pgfqpoint{3.166910in}{0.611648in}}%
\pgfpathlineto{\pgfqpoint{3.167465in}{0.612837in}}%
\pgfpathlineto{\pgfqpoint{3.168577in}{0.602651in}}%
\pgfpathlineto{\pgfqpoint{3.169689in}{0.624074in}}%
\pgfpathlineto{\pgfqpoint{3.170245in}{0.603449in}}%
\pgfpathlineto{\pgfqpoint{3.170801in}{0.618542in}}%
\pgfpathlineto{\pgfqpoint{3.171912in}{0.605175in}}%
\pgfpathlineto{\pgfqpoint{3.172468in}{0.628681in}}%
\pgfpathlineto{\pgfqpoint{3.173024in}{0.604324in}}%
\pgfpathlineto{\pgfqpoint{3.173580in}{0.610000in}}%
\pgfpathlineto{\pgfqpoint{3.174136in}{0.609328in}}%
\pgfpathlineto{\pgfqpoint{3.176359in}{0.624238in}}%
\pgfpathlineto{\pgfqpoint{3.176915in}{0.613427in}}%
\pgfpathlineto{\pgfqpoint{3.177471in}{0.614107in}}%
\pgfpathlineto{\pgfqpoint{3.178027in}{0.616927in}}%
\pgfpathlineto{\pgfqpoint{3.178583in}{0.604832in}}%
\pgfpathlineto{\pgfqpoint{3.179138in}{0.618521in}}%
\pgfpathlineto{\pgfqpoint{3.179694in}{0.616523in}}%
\pgfpathlineto{\pgfqpoint{3.180250in}{0.611921in}}%
\pgfpathlineto{\pgfqpoint{3.180806in}{0.621776in}}%
\pgfpathlineto{\pgfqpoint{3.181918in}{0.603576in}}%
\pgfpathlineto{\pgfqpoint{3.183585in}{0.618950in}}%
\pgfpathlineto{\pgfqpoint{3.184141in}{0.601813in}}%
\pgfpathlineto{\pgfqpoint{3.184697in}{0.617764in}}%
\pgfpathlineto{\pgfqpoint{3.186365in}{0.606271in}}%
\pgfpathlineto{\pgfqpoint{3.186921in}{0.623407in}}%
\pgfpathlineto{\pgfqpoint{3.187476in}{0.611625in}}%
\pgfpathlineto{\pgfqpoint{3.188032in}{0.606705in}}%
\pgfpathlineto{\pgfqpoint{3.190256in}{0.647533in}}%
\pgfpathlineto{\pgfqpoint{3.190812in}{0.641731in}}%
\pgfpathlineto{\pgfqpoint{3.192479in}{0.726113in}}%
\pgfpathlineto{\pgfqpoint{3.194703in}{0.613196in}}%
\pgfpathlineto{\pgfqpoint{3.195814in}{0.693822in}}%
\pgfpathlineto{\pgfqpoint{3.196926in}{0.635053in}}%
\pgfpathlineto{\pgfqpoint{3.198038in}{0.690628in}}%
\pgfpathlineto{\pgfqpoint{3.198594in}{0.660713in}}%
\pgfpathlineto{\pgfqpoint{3.200261in}{0.781810in}}%
\pgfpathlineto{\pgfqpoint{3.201373in}{0.633377in}}%
\pgfpathlineto{\pgfqpoint{3.202485in}{0.759492in}}%
\pgfpathlineto{\pgfqpoint{3.204152in}{0.652463in}}%
\pgfpathlineto{\pgfqpoint{3.204708in}{0.796320in}}%
\pgfpathlineto{\pgfqpoint{3.205264in}{0.672427in}}%
\pgfpathlineto{\pgfqpoint{3.205820in}{0.674073in}}%
\pgfpathlineto{\pgfqpoint{3.207487in}{0.608087in}}%
\pgfpathlineto{\pgfqpoint{3.208043in}{0.662160in}}%
\pgfpathlineto{\pgfqpoint{3.208599in}{0.625318in}}%
\pgfpathlineto{\pgfqpoint{3.209155in}{0.635989in}}%
\pgfpathlineto{\pgfqpoint{3.210823in}{0.608329in}}%
\pgfpathlineto{\pgfqpoint{3.211934in}{0.623557in}}%
\pgfpathlineto{\pgfqpoint{3.212490in}{0.607152in}}%
\pgfpathlineto{\pgfqpoint{3.213046in}{0.619402in}}%
\pgfpathlineto{\pgfqpoint{3.213602in}{0.608658in}}%
\pgfpathlineto{\pgfqpoint{3.214714in}{0.642359in}}%
\pgfpathlineto{\pgfqpoint{3.215269in}{0.634787in}}%
\pgfpathlineto{\pgfqpoint{3.215825in}{0.637042in}}%
\pgfpathlineto{\pgfqpoint{3.217493in}{0.617380in}}%
\pgfpathlineto{\pgfqpoint{3.218049in}{0.628015in}}%
\pgfpathlineto{\pgfqpoint{3.219716in}{0.604645in}}%
\pgfpathlineto{\pgfqpoint{3.221384in}{0.612498in}}%
\pgfpathlineto{\pgfqpoint{3.221940in}{0.608582in}}%
\pgfpathlineto{\pgfqpoint{3.222496in}{0.620013in}}%
\pgfpathlineto{\pgfqpoint{3.223052in}{0.616199in}}%
\pgfpathlineto{\pgfqpoint{3.225275in}{0.604605in}}%
\pgfpathlineto{\pgfqpoint{3.225831in}{0.623577in}}%
\pgfpathlineto{\pgfqpoint{3.226387in}{0.616379in}}%
\pgfpathlineto{\pgfqpoint{3.226943in}{0.616266in}}%
\pgfpathlineto{\pgfqpoint{3.227498in}{0.620037in}}%
\pgfpathlineto{\pgfqpoint{3.228054in}{0.607780in}}%
\pgfpathlineto{\pgfqpoint{3.228610in}{0.616468in}}%
\pgfpathlineto{\pgfqpoint{3.229722in}{0.619531in}}%
\pgfpathlineto{\pgfqpoint{3.230834in}{0.636433in}}%
\pgfpathlineto{\pgfqpoint{3.232501in}{0.609023in}}%
\pgfpathlineto{\pgfqpoint{3.233057in}{0.629665in}}%
\pgfpathlineto{\pgfqpoint{3.233613in}{0.607152in}}%
\pgfpathlineto{\pgfqpoint{3.234169in}{0.609665in}}%
\pgfpathlineto{\pgfqpoint{3.234725in}{0.616647in}}%
\pgfpathlineto{\pgfqpoint{3.235280in}{0.612172in}}%
\pgfpathlineto{\pgfqpoint{3.235836in}{0.613198in}}%
\pgfpathlineto{\pgfqpoint{3.236392in}{0.606093in}}%
\pgfpathlineto{\pgfqpoint{3.236948in}{0.616901in}}%
\pgfpathlineto{\pgfqpoint{3.237504in}{0.606981in}}%
\pgfpathlineto{\pgfqpoint{3.238616in}{0.615701in}}%
\pgfpathlineto{\pgfqpoint{3.239172in}{0.604529in}}%
\pgfpathlineto{\pgfqpoint{3.239727in}{0.632730in}}%
\pgfpathlineto{\pgfqpoint{3.240283in}{0.617459in}}%
\pgfpathlineto{\pgfqpoint{3.240839in}{0.604364in}}%
\pgfpathlineto{\pgfqpoint{3.241395in}{0.620083in}}%
\pgfpathlineto{\pgfqpoint{3.241951in}{0.608520in}}%
\pgfpathlineto{\pgfqpoint{3.242507in}{0.602595in}}%
\pgfpathlineto{\pgfqpoint{3.243063in}{0.604118in}}%
\pgfpathlineto{\pgfqpoint{3.243618in}{0.620657in}}%
\pgfpathlineto{\pgfqpoint{3.244174in}{0.614352in}}%
\pgfpathlineto{\pgfqpoint{3.244730in}{0.619150in}}%
\pgfpathlineto{\pgfqpoint{3.245286in}{0.605037in}}%
\pgfpathlineto{\pgfqpoint{3.246954in}{0.623650in}}%
\pgfpathlineto{\pgfqpoint{3.247509in}{0.630031in}}%
\pgfpathlineto{\pgfqpoint{3.248065in}{0.621224in}}%
\pgfpathlineto{\pgfqpoint{3.249733in}{0.639793in}}%
\pgfpathlineto{\pgfqpoint{3.250289in}{0.634242in}}%
\pgfpathlineto{\pgfqpoint{3.252512in}{0.729344in}}%
\pgfpathlineto{\pgfqpoint{3.254180in}{0.636020in}}%
\pgfpathlineto{\pgfqpoint{3.254736in}{0.647005in}}%
\pgfpathlineto{\pgfqpoint{3.256403in}{0.698341in}}%
\pgfpathlineto{\pgfqpoint{3.256959in}{0.676335in}}%
\pgfpathlineto{\pgfqpoint{3.257515in}{0.735202in}}%
\pgfpathlineto{\pgfqpoint{3.258071in}{0.627592in}}%
\pgfpathlineto{\pgfqpoint{3.258627in}{0.689715in}}%
\pgfpathlineto{\pgfqpoint{3.259183in}{0.715884in}}%
\pgfpathlineto{\pgfqpoint{3.259738in}{0.871824in}}%
\pgfpathlineto{\pgfqpoint{3.261406in}{0.650027in}}%
\pgfpathlineto{\pgfqpoint{3.261962in}{0.745036in}}%
\pgfpathlineto{\pgfqpoint{3.263629in}{0.614832in}}%
\pgfpathlineto{\pgfqpoint{3.264185in}{0.647208in}}%
\pgfpathlineto{\pgfqpoint{3.264741in}{0.604233in}}%
\pgfpathlineto{\pgfqpoint{3.265297in}{0.670484in}}%
\pgfpathlineto{\pgfqpoint{3.265853in}{0.618876in}}%
\pgfpathlineto{\pgfqpoint{3.266409in}{0.624461in}}%
\pgfpathlineto{\pgfqpoint{3.268076in}{0.609871in}}%
\pgfpathlineto{\pgfqpoint{3.268632in}{0.609984in}}%
\pgfpathlineto{\pgfqpoint{3.269188in}{0.612925in}}%
\pgfpathlineto{\pgfqpoint{3.269744in}{0.610150in}}%
\pgfpathlineto{\pgfqpoint{3.270300in}{0.636417in}}%
\pgfpathlineto{\pgfqpoint{3.270856in}{0.627038in}}%
\pgfpathlineto{\pgfqpoint{3.271411in}{0.601963in}}%
\pgfpathlineto{\pgfqpoint{3.271967in}{0.624471in}}%
\pgfpathlineto{\pgfqpoint{3.272523in}{0.635858in}}%
\pgfpathlineto{\pgfqpoint{3.274747in}{0.602375in}}%
\pgfpathlineto{\pgfqpoint{3.275303in}{0.625307in}}%
\pgfpathlineto{\pgfqpoint{3.275858in}{0.616759in}}%
\pgfpathlineto{\pgfqpoint{3.277526in}{0.606759in}}%
\pgfpathlineto{\pgfqpoint{3.278082in}{0.622861in}}%
\pgfpathlineto{\pgfqpoint{3.278638in}{0.613642in}}%
\pgfpathlineto{\pgfqpoint{3.279749in}{0.604076in}}%
\pgfpathlineto{\pgfqpoint{3.280305in}{0.630557in}}%
\pgfpathlineto{\pgfqpoint{3.280861in}{0.612091in}}%
\pgfpathlineto{\pgfqpoint{3.281417in}{0.628517in}}%
\pgfpathlineto{\pgfqpoint{3.281973in}{0.601082in}}%
\pgfpathlineto{\pgfqpoint{3.282529in}{0.616225in}}%
\pgfpathlineto{\pgfqpoint{3.284196in}{0.652166in}}%
\pgfpathlineto{\pgfqpoint{3.284752in}{0.610930in}}%
\pgfpathlineto{\pgfqpoint{3.285308in}{0.648079in}}%
\pgfpathlineto{\pgfqpoint{3.286420in}{0.634590in}}%
\pgfpathlineto{\pgfqpoint{3.287531in}{0.618072in}}%
\pgfpathlineto{\pgfqpoint{3.289199in}{0.644014in}}%
\pgfpathlineto{\pgfqpoint{3.289755in}{0.628970in}}%
\pgfpathlineto{\pgfqpoint{3.290311in}{0.673101in}}%
\pgfpathlineto{\pgfqpoint{3.290867in}{0.631842in}}%
\pgfpathlineto{\pgfqpoint{3.291422in}{0.656467in}}%
\pgfpathlineto{\pgfqpoint{3.291978in}{0.634989in}}%
\pgfpathlineto{\pgfqpoint{3.292534in}{0.632565in}}%
\pgfpathlineto{\pgfqpoint{3.293646in}{0.604179in}}%
\pgfpathlineto{\pgfqpoint{3.295314in}{0.638771in}}%
\pgfpathlineto{\pgfqpoint{3.295869in}{0.612405in}}%
\pgfpathlineto{\pgfqpoint{3.296425in}{0.654411in}}%
\pgfpathlineto{\pgfqpoint{3.296981in}{0.622467in}}%
\pgfpathlineto{\pgfqpoint{3.297537in}{0.626832in}}%
\pgfpathlineto{\pgfqpoint{3.298093in}{0.620794in}}%
\pgfpathlineto{\pgfqpoint{3.298649in}{0.635425in}}%
\pgfpathlineto{\pgfqpoint{3.299205in}{0.613911in}}%
\pgfpathlineto{\pgfqpoint{3.299760in}{0.621185in}}%
\pgfpathlineto{\pgfqpoint{3.300316in}{0.614922in}}%
\pgfpathlineto{\pgfqpoint{3.301984in}{0.632429in}}%
\pgfpathlineto{\pgfqpoint{3.302540in}{0.618482in}}%
\pgfpathlineto{\pgfqpoint{3.303096in}{0.644620in}}%
\pgfpathlineto{\pgfqpoint{3.303651in}{0.625693in}}%
\pgfpathlineto{\pgfqpoint{3.304207in}{0.616594in}}%
\pgfpathlineto{\pgfqpoint{3.304763in}{0.632629in}}%
\pgfpathlineto{\pgfqpoint{3.305319in}{0.613360in}}%
\pgfpathlineto{\pgfqpoint{3.305875in}{0.629161in}}%
\pgfpathlineto{\pgfqpoint{3.306431in}{0.635185in}}%
\pgfpathlineto{\pgfqpoint{3.306987in}{0.629482in}}%
\pgfpathlineto{\pgfqpoint{3.308654in}{0.608741in}}%
\pgfpathlineto{\pgfqpoint{3.309210in}{0.631480in}}%
\pgfpathlineto{\pgfqpoint{3.309766in}{0.627987in}}%
\pgfpathlineto{\pgfqpoint{3.310878in}{0.637521in}}%
\pgfpathlineto{\pgfqpoint{3.311989in}{0.661240in}}%
\pgfpathlineto{\pgfqpoint{3.312545in}{0.649659in}}%
\pgfpathlineto{\pgfqpoint{3.313101in}{0.679786in}}%
\pgfpathlineto{\pgfqpoint{3.313657in}{0.666668in}}%
\pgfpathlineto{\pgfqpoint{3.314769in}{0.658003in}}%
\pgfpathlineto{\pgfqpoint{3.315880in}{0.687581in}}%
\pgfpathlineto{\pgfqpoint{3.316436in}{0.651136in}}%
\pgfpathlineto{\pgfqpoint{3.316992in}{0.808851in}}%
\pgfpathlineto{\pgfqpoint{3.317548in}{0.661773in}}%
\pgfpathlineto{\pgfqpoint{3.318104in}{0.625693in}}%
\pgfpathlineto{\pgfqpoint{3.318660in}{0.656588in}}%
\pgfpathlineto{\pgfqpoint{3.319216in}{0.636547in}}%
\pgfpathlineto{\pgfqpoint{3.319771in}{0.640882in}}%
\pgfpathlineto{\pgfqpoint{3.320883in}{0.665806in}}%
\pgfpathlineto{\pgfqpoint{3.321439in}{0.623845in}}%
\pgfpathlineto{\pgfqpoint{3.321995in}{0.659786in}}%
\pgfpathlineto{\pgfqpoint{3.322551in}{0.636333in}}%
\pgfpathlineto{\pgfqpoint{3.323107in}{0.651390in}}%
\pgfpathlineto{\pgfqpoint{3.323662in}{0.656274in}}%
\pgfpathlineto{\pgfqpoint{3.324218in}{0.652100in}}%
\pgfpathlineto{\pgfqpoint{3.324774in}{0.653985in}}%
\pgfpathlineto{\pgfqpoint{3.325330in}{0.644674in}}%
\pgfpathlineto{\pgfqpoint{3.326998in}{0.681536in}}%
\pgfpathlineto{\pgfqpoint{3.328109in}{0.620854in}}%
\pgfpathlineto{\pgfqpoint{3.328665in}{0.629250in}}%
\pgfpathlineto{\pgfqpoint{3.329221in}{0.607293in}}%
\pgfpathlineto{\pgfqpoint{3.329777in}{0.623472in}}%
\pgfpathlineto{\pgfqpoint{3.331444in}{0.677530in}}%
\pgfpathlineto{\pgfqpoint{3.332556in}{0.641388in}}%
\pgfpathlineto{\pgfqpoint{3.333668in}{0.642632in}}%
\pgfpathlineto{\pgfqpoint{3.334224in}{0.633078in}}%
\pgfpathlineto{\pgfqpoint{3.334780in}{0.635981in}}%
\pgfpathlineto{\pgfqpoint{3.335336in}{0.642243in}}%
\pgfpathlineto{\pgfqpoint{3.335891in}{0.614502in}}%
\pgfpathlineto{\pgfqpoint{3.336447in}{0.664772in}}%
\pgfpathlineto{\pgfqpoint{3.337003in}{0.629306in}}%
\pgfpathlineto{\pgfqpoint{3.337559in}{0.626483in}}%
\pgfpathlineto{\pgfqpoint{3.338115in}{0.605736in}}%
\pgfpathlineto{\pgfqpoint{3.339782in}{0.641681in}}%
\pgfpathlineto{\pgfqpoint{3.341450in}{0.612585in}}%
\pgfpathlineto{\pgfqpoint{3.342562in}{0.668321in}}%
\pgfpathlineto{\pgfqpoint{3.343673in}{0.607239in}}%
\pgfpathlineto{\pgfqpoint{3.345341in}{0.657239in}}%
\pgfpathlineto{\pgfqpoint{3.346453in}{0.629802in}}%
\pgfpathlineto{\pgfqpoint{3.347564in}{0.668982in}}%
\pgfpathlineto{\pgfqpoint{3.348120in}{0.654740in}}%
\pgfpathlineto{\pgfqpoint{3.348676in}{0.611188in}}%
\pgfpathlineto{\pgfqpoint{3.349788in}{0.668994in}}%
\pgfpathlineto{\pgfqpoint{3.350344in}{0.635915in}}%
\pgfpathlineto{\pgfqpoint{3.350900in}{0.637913in}}%
\pgfpathlineto{\pgfqpoint{3.351456in}{0.637926in}}%
\pgfpathlineto{\pgfqpoint{3.352011in}{0.658963in}}%
\pgfpathlineto{\pgfqpoint{3.352567in}{0.652527in}}%
\pgfpathlineto{\pgfqpoint{3.353123in}{0.754861in}}%
\pgfpathlineto{\pgfqpoint{3.353679in}{0.628427in}}%
\pgfpathlineto{\pgfqpoint{3.354235in}{0.708979in}}%
\pgfpathlineto{\pgfqpoint{3.355347in}{0.770371in}}%
\pgfpathlineto{\pgfqpoint{3.357014in}{0.874743in}}%
\pgfpathlineto{\pgfqpoint{3.358682in}{0.718927in}}%
\pgfpathlineto{\pgfqpoint{3.359238in}{0.785012in}}%
\pgfpathlineto{\pgfqpoint{3.359793in}{0.774768in}}%
\pgfpathlineto{\pgfqpoint{3.360349in}{0.746891in}}%
\pgfpathlineto{\pgfqpoint{3.360905in}{0.872096in}}%
\pgfpathlineto{\pgfqpoint{3.361461in}{0.698591in}}%
\pgfpathlineto{\pgfqpoint{3.362017in}{0.882982in}}%
\pgfpathlineto{\pgfqpoint{3.362573in}{0.740227in}}%
\pgfpathlineto{\pgfqpoint{3.363129in}{0.863179in}}%
\pgfpathlineto{\pgfqpoint{3.363684in}{0.689152in}}%
\pgfpathlineto{\pgfqpoint{3.364240in}{0.788370in}}%
\pgfpathlineto{\pgfqpoint{3.365352in}{0.671591in}}%
\pgfpathlineto{\pgfqpoint{3.365908in}{0.766098in}}%
\pgfpathlineto{\pgfqpoint{3.366464in}{0.619213in}}%
\pgfpathlineto{\pgfqpoint{3.367020in}{0.817804in}}%
\pgfpathlineto{\pgfqpoint{3.367575in}{0.694102in}}%
\pgfpathlineto{\pgfqpoint{3.368131in}{0.698032in}}%
\pgfpathlineto{\pgfqpoint{3.368687in}{0.675523in}}%
\pgfpathlineto{\pgfqpoint{3.369243in}{0.698770in}}%
\pgfpathlineto{\pgfqpoint{3.369799in}{0.690273in}}%
\pgfpathlineto{\pgfqpoint{3.370355in}{0.621865in}}%
\pgfpathlineto{\pgfqpoint{3.370911in}{0.694847in}}%
\pgfpathlineto{\pgfqpoint{3.371467in}{0.635312in}}%
\pgfpathlineto{\pgfqpoint{3.372022in}{0.677151in}}%
\pgfpathlineto{\pgfqpoint{3.372578in}{0.803874in}}%
\pgfpathlineto{\pgfqpoint{3.373134in}{0.755675in}}%
\pgfpathlineto{\pgfqpoint{3.373690in}{0.761117in}}%
\pgfpathlineto{\pgfqpoint{3.374802in}{0.959602in}}%
\pgfpathlineto{\pgfqpoint{3.376469in}{0.620593in}}%
\pgfpathlineto{\pgfqpoint{3.378137in}{0.719233in}}%
\pgfpathlineto{\pgfqpoint{3.378693in}{0.608801in}}%
\pgfpathlineto{\pgfqpoint{3.379249in}{0.621910in}}%
\pgfpathlineto{\pgfqpoint{3.380360in}{0.668155in}}%
\pgfpathlineto{\pgfqpoint{3.380916in}{0.630726in}}%
\pgfpathlineto{\pgfqpoint{3.381472in}{0.658669in}}%
\pgfpathlineto{\pgfqpoint{3.382028in}{0.656742in}}%
\pgfpathlineto{\pgfqpoint{3.383140in}{0.689819in}}%
\pgfpathlineto{\pgfqpoint{3.383695in}{0.619577in}}%
\pgfpathlineto{\pgfqpoint{3.384251in}{0.668348in}}%
\pgfpathlineto{\pgfqpoint{3.384807in}{0.650657in}}%
\pgfpathlineto{\pgfqpoint{3.385363in}{0.673780in}}%
\pgfpathlineto{\pgfqpoint{3.385919in}{0.658671in}}%
\pgfpathlineto{\pgfqpoint{3.386475in}{0.620587in}}%
\pgfpathlineto{\pgfqpoint{3.387031in}{0.685241in}}%
\pgfpathlineto{\pgfqpoint{3.387586in}{0.619402in}}%
\pgfpathlineto{\pgfqpoint{3.388142in}{0.745569in}}%
\pgfpathlineto{\pgfqpoint{3.388698in}{0.736311in}}%
\pgfpathlineto{\pgfqpoint{3.389254in}{0.711782in}}%
\pgfpathlineto{\pgfqpoint{3.389810in}{0.751041in}}%
\pgfpathlineto{\pgfqpoint{3.390366in}{0.633543in}}%
\pgfpathlineto{\pgfqpoint{3.390922in}{0.726963in}}%
\pgfpathlineto{\pgfqpoint{3.391478in}{0.710637in}}%
\pgfpathlineto{\pgfqpoint{3.392033in}{0.723355in}}%
\pgfpathlineto{\pgfqpoint{3.392589in}{0.753012in}}%
\pgfpathlineto{\pgfqpoint{3.393145in}{0.716574in}}%
\pgfpathlineto{\pgfqpoint{3.393701in}{0.766370in}}%
\pgfpathlineto{\pgfqpoint{3.394813in}{0.615192in}}%
\pgfpathlineto{\pgfqpoint{3.395369in}{0.685964in}}%
\pgfpathlineto{\pgfqpoint{3.395924in}{0.651100in}}%
\pgfpathlineto{\pgfqpoint{3.397036in}{0.638621in}}%
\pgfpathlineto{\pgfqpoint{3.397592in}{0.647732in}}%
\pgfpathlineto{\pgfqpoint{3.398148in}{0.638247in}}%
\pgfpathlineto{\pgfqpoint{3.399815in}{0.715293in}}%
\pgfpathlineto{\pgfqpoint{3.400371in}{0.679929in}}%
\pgfpathlineto{\pgfqpoint{3.400927in}{0.712456in}}%
\pgfpathlineto{\pgfqpoint{3.401483in}{0.721071in}}%
\pgfpathlineto{\pgfqpoint{3.402595in}{0.654834in}}%
\pgfpathlineto{\pgfqpoint{3.404262in}{0.790967in}}%
\pgfpathlineto{\pgfqpoint{3.404818in}{0.753661in}}%
\pgfpathlineto{\pgfqpoint{3.405374in}{0.924833in}}%
\pgfpathlineto{\pgfqpoint{3.405930in}{0.902243in}}%
\pgfpathlineto{\pgfqpoint{3.407598in}{0.705720in}}%
\pgfpathlineto{\pgfqpoint{3.408709in}{0.617108in}}%
\pgfpathlineto{\pgfqpoint{3.409265in}{0.822450in}}%
\pgfpathlineto{\pgfqpoint{3.409821in}{0.699604in}}%
\pgfpathlineto{\pgfqpoint{3.410377in}{0.709908in}}%
\pgfpathlineto{\pgfqpoint{3.410933in}{0.748850in}}%
\pgfpathlineto{\pgfqpoint{3.411489in}{0.871016in}}%
\pgfpathlineto{\pgfqpoint{3.412600in}{0.697686in}}%
\pgfpathlineto{\pgfqpoint{3.413156in}{0.734561in}}%
\pgfpathlineto{\pgfqpoint{3.413712in}{0.867531in}}%
\pgfpathlineto{\pgfqpoint{3.414268in}{0.695839in}}%
\pgfpathlineto{\pgfqpoint{3.414824in}{0.868118in}}%
\pgfpathlineto{\pgfqpoint{3.415380in}{0.701383in}}%
\pgfpathlineto{\pgfqpoint{3.415935in}{0.633691in}}%
\pgfpathlineto{\pgfqpoint{3.417047in}{0.892694in}}%
\pgfpathlineto{\pgfqpoint{3.418715in}{0.638562in}}%
\pgfpathlineto{\pgfqpoint{3.419271in}{0.758445in}}%
\pgfpathlineto{\pgfqpoint{3.420938in}{0.864768in}}%
\pgfpathlineto{\pgfqpoint{3.421494in}{1.117537in}}%
\pgfpathlineto{\pgfqpoint{3.422606in}{0.732049in}}%
\pgfpathlineto{\pgfqpoint{3.423162in}{1.315576in}}%
\pgfpathlineto{\pgfqpoint{3.423717in}{1.154218in}}%
\pgfpathlineto{\pgfqpoint{3.424273in}{1.114064in}}%
\pgfpathlineto{\pgfqpoint{3.425941in}{0.612411in}}%
\pgfpathlineto{\pgfqpoint{3.427053in}{0.895775in}}%
\pgfpathlineto{\pgfqpoint{3.428164in}{0.622166in}}%
\pgfpathlineto{\pgfqpoint{3.428720in}{0.679923in}}%
\pgfpathlineto{\pgfqpoint{3.429276in}{0.738625in}}%
\pgfpathlineto{\pgfqpoint{3.429832in}{0.629125in}}%
\pgfpathlineto{\pgfqpoint{3.430388in}{0.657398in}}%
\pgfpathlineto{\pgfqpoint{3.430944in}{0.661978in}}%
\pgfpathlineto{\pgfqpoint{3.431500in}{0.623656in}}%
\pgfpathlineto{\pgfqpoint{3.431500in}{0.623656in}}%
\pgfusepath{stroke}%
\end{pgfscope}%
\begin{pgfscope}%
\pgfsetrectcap%
\pgfsetmiterjoin%
\pgfsetlinewidth{0.803000pt}%
\definecolor{currentstroke}{rgb}{0.000000,0.000000,0.000000}%
\pgfsetstrokecolor{currentstroke}%
\pgfsetdash{}{0pt}%
\pgfpathmoveto{\pgfqpoint{0.717889in}{0.564143in}}%
\pgfpathlineto{\pgfqpoint{0.717889in}{1.351359in}}%
\pgfusepath{stroke}%
\end{pgfscope}%
\begin{pgfscope}%
\pgfsetrectcap%
\pgfsetmiterjoin%
\pgfsetlinewidth{0.803000pt}%
\definecolor{currentstroke}{rgb}{0.000000,0.000000,0.000000}%
\pgfsetstrokecolor{currentstroke}%
\pgfsetdash{}{0pt}%
\pgfpathmoveto{\pgfqpoint{6.146222in}{0.564143in}}%
\pgfpathlineto{\pgfqpoint{6.146222in}{1.351359in}}%
\pgfusepath{stroke}%
\end{pgfscope}%
\begin{pgfscope}%
\pgfsetrectcap%
\pgfsetmiterjoin%
\pgfsetlinewidth{0.803000pt}%
\definecolor{currentstroke}{rgb}{0.000000,0.000000,0.000000}%
\pgfsetstrokecolor{currentstroke}%
\pgfsetdash{}{0pt}%
\pgfpathmoveto{\pgfqpoint{0.717889in}{0.564143in}}%
\pgfpathlineto{\pgfqpoint{6.146222in}{0.564143in}}%
\pgfusepath{stroke}%
\end{pgfscope}%
\begin{pgfscope}%
\pgfsetrectcap%
\pgfsetmiterjoin%
\pgfsetlinewidth{0.803000pt}%
\definecolor{currentstroke}{rgb}{0.000000,0.000000,0.000000}%
\pgfsetstrokecolor{currentstroke}%
\pgfsetdash{}{0pt}%
\pgfpathmoveto{\pgfqpoint{0.717889in}{1.351359in}}%
\pgfpathlineto{\pgfqpoint{6.146222in}{1.351359in}}%
\pgfusepath{stroke}%
\end{pgfscope}%
\begin{pgfscope}%
\definecolor{textcolor}{rgb}{0.000000,0.000000,0.000000}%
\pgfsetstrokecolor{textcolor}%
\pgfsetfillcolor{textcolor}%
\pgftext[x=3.432055in,y=1.434692in,,base]{\color{textcolor}\rmfamily\fontsize{12.000000}{14.400000}\selectfont Spectrum of Filtered ECG Signal}%
\end{pgfscope}%
\end{pgfpicture}%
\makeatother%
\endgroup%

    }
    \caption{Frequency response of the window FIR filter (top), window filtered ECG signal (middle), and spectrum of filtered ECG signal (bottom).}
    \label{fig:fir-window}
\end{figure}

\subsection{Optimal FIR Filter}
\begin{figure}[H]
    \centering
    \adjustbox{max width=0.75\textwidth}{
    %% Creator: Matplotlib, PGF backend
%%
%% To include the figure in your LaTeX document, write
%%   \input{<filename>.pgf}
%%
%% Make sure the required packages are loaded in your preamble
%%   \usepackage{pgf}
%%
%% Figures using additional raster images can only be included by \input if
%% they are in the same directory as the main LaTeX file. For loading figures
%% from other directories you can use the `import` package
%%   \usepackage{import}
%% and then include the figures with
%%   \import{<path to file>}{<filename>.pgf}
%%
%% Matplotlib used the following preamble
%%
\begingroup%
\makeatletter%
\begin{pgfpicture}%
\pgfpathrectangle{\pgfpointorigin}{\pgfqpoint{6.400000in}{4.800000in}}%
\pgfusepath{use as bounding box, clip}%
\begin{pgfscope}%
\pgfsetbuttcap%
\pgfsetmiterjoin%
\definecolor{currentfill}{rgb}{1.000000,1.000000,1.000000}%
\pgfsetfillcolor{currentfill}%
\pgfsetlinewidth{0.000000pt}%
\definecolor{currentstroke}{rgb}{1.000000,1.000000,1.000000}%
\pgfsetstrokecolor{currentstroke}%
\pgfsetdash{}{0pt}%
\pgfpathmoveto{\pgfqpoint{0.000000in}{0.000000in}}%
\pgfpathlineto{\pgfqpoint{6.400000in}{0.000000in}}%
\pgfpathlineto{\pgfqpoint{6.400000in}{4.800000in}}%
\pgfpathlineto{\pgfqpoint{0.000000in}{4.800000in}}%
\pgfpathclose%
\pgfusepath{fill}%
\end{pgfscope}%
\begin{pgfscope}%
\pgfsetbuttcap%
\pgfsetmiterjoin%
\definecolor{currentfill}{rgb}{1.000000,1.000000,1.000000}%
\pgfsetfillcolor{currentfill}%
\pgfsetlinewidth{0.000000pt}%
\definecolor{currentstroke}{rgb}{0.000000,0.000000,0.000000}%
\pgfsetstrokecolor{currentstroke}%
\pgfsetstrokeopacity{0.000000}%
\pgfsetdash{}{0pt}%
\pgfpathmoveto{\pgfqpoint{0.934300in}{3.664143in}}%
\pgfpathlineto{\pgfqpoint{6.146222in}{3.664143in}}%
\pgfpathlineto{\pgfqpoint{6.146222in}{4.451359in}}%
\pgfpathlineto{\pgfqpoint{0.934300in}{4.451359in}}%
\pgfpathclose%
\pgfusepath{fill}%
\end{pgfscope}%
\begin{pgfscope}%
\pgfsetbuttcap%
\pgfsetroundjoin%
\definecolor{currentfill}{rgb}{0.000000,0.000000,0.000000}%
\pgfsetfillcolor{currentfill}%
\pgfsetlinewidth{0.803000pt}%
\definecolor{currentstroke}{rgb}{0.000000,0.000000,0.000000}%
\pgfsetstrokecolor{currentstroke}%
\pgfsetdash{}{0pt}%
\pgfsys@defobject{currentmarker}{\pgfqpoint{0.000000in}{-0.048611in}}{\pgfqpoint{0.000000in}{0.000000in}}{%
\pgfpathmoveto{\pgfqpoint{0.000000in}{0.000000in}}%
\pgfpathlineto{\pgfqpoint{0.000000in}{-0.048611in}}%
\pgfusepath{stroke,fill}%
}%
\begin{pgfscope}%
\pgfsys@transformshift{0.934300in}{3.664143in}%
\pgfsys@useobject{currentmarker}{}%
\end{pgfscope}%
\end{pgfscope}%
\begin{pgfscope}%
\definecolor{textcolor}{rgb}{0.000000,0.000000,0.000000}%
\pgfsetstrokecolor{textcolor}%
\pgfsetfillcolor{textcolor}%
\pgftext[x=0.934300in,y=3.566921in,,top]{\color{textcolor}\rmfamily\fontsize{10.000000}{12.000000}\selectfont \(\displaystyle 0\)}%
\end{pgfscope}%
\begin{pgfscope}%
\pgfsetbuttcap%
\pgfsetroundjoin%
\definecolor{currentfill}{rgb}{0.000000,0.000000,0.000000}%
\pgfsetfillcolor{currentfill}%
\pgfsetlinewidth{0.803000pt}%
\definecolor{currentstroke}{rgb}{0.000000,0.000000,0.000000}%
\pgfsetstrokecolor{currentstroke}%
\pgfsetdash{}{0pt}%
\pgfsys@defobject{currentmarker}{\pgfqpoint{0.000000in}{-0.048611in}}{\pgfqpoint{0.000000in}{0.000000in}}{%
\pgfpathmoveto{\pgfqpoint{0.000000in}{0.000000in}}%
\pgfpathlineto{\pgfqpoint{0.000000in}{-0.048611in}}%
\pgfusepath{stroke,fill}%
}%
\begin{pgfscope}%
\pgfsys@transformshift{1.976685in}{3.664143in}%
\pgfsys@useobject{currentmarker}{}%
\end{pgfscope}%
\end{pgfscope}%
\begin{pgfscope}%
\definecolor{textcolor}{rgb}{0.000000,0.000000,0.000000}%
\pgfsetstrokecolor{textcolor}%
\pgfsetfillcolor{textcolor}%
\pgftext[x=1.976685in,y=3.566921in,,top]{\color{textcolor}\rmfamily\fontsize{10.000000}{12.000000}\selectfont \(\displaystyle 20\)}%
\end{pgfscope}%
\begin{pgfscope}%
\pgfsetbuttcap%
\pgfsetroundjoin%
\definecolor{currentfill}{rgb}{0.000000,0.000000,0.000000}%
\pgfsetfillcolor{currentfill}%
\pgfsetlinewidth{0.803000pt}%
\definecolor{currentstroke}{rgb}{0.000000,0.000000,0.000000}%
\pgfsetstrokecolor{currentstroke}%
\pgfsetdash{}{0pt}%
\pgfsys@defobject{currentmarker}{\pgfqpoint{0.000000in}{-0.048611in}}{\pgfqpoint{0.000000in}{0.000000in}}{%
\pgfpathmoveto{\pgfqpoint{0.000000in}{0.000000in}}%
\pgfpathlineto{\pgfqpoint{0.000000in}{-0.048611in}}%
\pgfusepath{stroke,fill}%
}%
\begin{pgfscope}%
\pgfsys@transformshift{3.019069in}{3.664143in}%
\pgfsys@useobject{currentmarker}{}%
\end{pgfscope}%
\end{pgfscope}%
\begin{pgfscope}%
\definecolor{textcolor}{rgb}{0.000000,0.000000,0.000000}%
\pgfsetstrokecolor{textcolor}%
\pgfsetfillcolor{textcolor}%
\pgftext[x=3.019069in,y=3.566921in,,top]{\color{textcolor}\rmfamily\fontsize{10.000000}{12.000000}\selectfont \(\displaystyle 40\)}%
\end{pgfscope}%
\begin{pgfscope}%
\pgfsetbuttcap%
\pgfsetroundjoin%
\definecolor{currentfill}{rgb}{0.000000,0.000000,0.000000}%
\pgfsetfillcolor{currentfill}%
\pgfsetlinewidth{0.803000pt}%
\definecolor{currentstroke}{rgb}{0.000000,0.000000,0.000000}%
\pgfsetstrokecolor{currentstroke}%
\pgfsetdash{}{0pt}%
\pgfsys@defobject{currentmarker}{\pgfqpoint{0.000000in}{-0.048611in}}{\pgfqpoint{0.000000in}{0.000000in}}{%
\pgfpathmoveto{\pgfqpoint{0.000000in}{0.000000in}}%
\pgfpathlineto{\pgfqpoint{0.000000in}{-0.048611in}}%
\pgfusepath{stroke,fill}%
}%
\begin{pgfscope}%
\pgfsys@transformshift{4.061453in}{3.664143in}%
\pgfsys@useobject{currentmarker}{}%
\end{pgfscope}%
\end{pgfscope}%
\begin{pgfscope}%
\definecolor{textcolor}{rgb}{0.000000,0.000000,0.000000}%
\pgfsetstrokecolor{textcolor}%
\pgfsetfillcolor{textcolor}%
\pgftext[x=4.061453in,y=3.566921in,,top]{\color{textcolor}\rmfamily\fontsize{10.000000}{12.000000}\selectfont \(\displaystyle 60\)}%
\end{pgfscope}%
\begin{pgfscope}%
\pgfsetbuttcap%
\pgfsetroundjoin%
\definecolor{currentfill}{rgb}{0.000000,0.000000,0.000000}%
\pgfsetfillcolor{currentfill}%
\pgfsetlinewidth{0.803000pt}%
\definecolor{currentstroke}{rgb}{0.000000,0.000000,0.000000}%
\pgfsetstrokecolor{currentstroke}%
\pgfsetdash{}{0pt}%
\pgfsys@defobject{currentmarker}{\pgfqpoint{0.000000in}{-0.048611in}}{\pgfqpoint{0.000000in}{0.000000in}}{%
\pgfpathmoveto{\pgfqpoint{0.000000in}{0.000000in}}%
\pgfpathlineto{\pgfqpoint{0.000000in}{-0.048611in}}%
\pgfusepath{stroke,fill}%
}%
\begin{pgfscope}%
\pgfsys@transformshift{5.103838in}{3.664143in}%
\pgfsys@useobject{currentmarker}{}%
\end{pgfscope}%
\end{pgfscope}%
\begin{pgfscope}%
\definecolor{textcolor}{rgb}{0.000000,0.000000,0.000000}%
\pgfsetstrokecolor{textcolor}%
\pgfsetfillcolor{textcolor}%
\pgftext[x=5.103838in,y=3.566921in,,top]{\color{textcolor}\rmfamily\fontsize{10.000000}{12.000000}\selectfont \(\displaystyle 80\)}%
\end{pgfscope}%
\begin{pgfscope}%
\pgfsetbuttcap%
\pgfsetroundjoin%
\definecolor{currentfill}{rgb}{0.000000,0.000000,0.000000}%
\pgfsetfillcolor{currentfill}%
\pgfsetlinewidth{0.803000pt}%
\definecolor{currentstroke}{rgb}{0.000000,0.000000,0.000000}%
\pgfsetstrokecolor{currentstroke}%
\pgfsetdash{}{0pt}%
\pgfsys@defobject{currentmarker}{\pgfqpoint{0.000000in}{-0.048611in}}{\pgfqpoint{0.000000in}{0.000000in}}{%
\pgfpathmoveto{\pgfqpoint{0.000000in}{0.000000in}}%
\pgfpathlineto{\pgfqpoint{0.000000in}{-0.048611in}}%
\pgfusepath{stroke,fill}%
}%
\begin{pgfscope}%
\pgfsys@transformshift{6.146222in}{3.664143in}%
\pgfsys@useobject{currentmarker}{}%
\end{pgfscope}%
\end{pgfscope}%
\begin{pgfscope}%
\definecolor{textcolor}{rgb}{0.000000,0.000000,0.000000}%
\pgfsetstrokecolor{textcolor}%
\pgfsetfillcolor{textcolor}%
\pgftext[x=6.146222in,y=3.566921in,,top]{\color{textcolor}\rmfamily\fontsize{10.000000}{12.000000}\selectfont \(\displaystyle 100\)}%
\end{pgfscope}%
\begin{pgfscope}%
\definecolor{textcolor}{rgb}{0.000000,0.000000,0.000000}%
\pgfsetstrokecolor{textcolor}%
\pgfsetfillcolor{textcolor}%
\pgftext[x=3.540261in,y=3.387909in,,top]{\color{textcolor}\rmfamily\fontsize{10.000000}{12.000000}\selectfont Frequency (Hz)}%
\end{pgfscope}%
\begin{pgfscope}%
\pgfsetbuttcap%
\pgfsetroundjoin%
\definecolor{currentfill}{rgb}{0.000000,0.000000,0.000000}%
\pgfsetfillcolor{currentfill}%
\pgfsetlinewidth{0.803000pt}%
\definecolor{currentstroke}{rgb}{0.000000,0.000000,0.000000}%
\pgfsetstrokecolor{currentstroke}%
\pgfsetdash{}{0pt}%
\pgfsys@defobject{currentmarker}{\pgfqpoint{-0.048611in}{0.000000in}}{\pgfqpoint{0.000000in}{0.000000in}}{%
\pgfpathmoveto{\pgfqpoint{0.000000in}{0.000000in}}%
\pgfpathlineto{\pgfqpoint{-0.048611in}{0.000000in}}%
\pgfusepath{stroke,fill}%
}%
\begin{pgfscope}%
\pgfsys@transformshift{0.934300in}{3.896166in}%
\pgfsys@useobject{currentmarker}{}%
\end{pgfscope}%
\end{pgfscope}%
\begin{pgfscope}%
\definecolor{textcolor}{rgb}{0.000000,0.000000,0.000000}%
\pgfsetstrokecolor{textcolor}%
\pgfsetfillcolor{textcolor}%
\pgftext[x=0.590164in,y=3.847940in,left,base]{\color{textcolor}\rmfamily\fontsize{10.000000}{12.000000}\selectfont \(\displaystyle -20\)}%
\end{pgfscope}%
\begin{pgfscope}%
\pgfsetbuttcap%
\pgfsetroundjoin%
\definecolor{currentfill}{rgb}{0.000000,0.000000,0.000000}%
\pgfsetfillcolor{currentfill}%
\pgfsetlinewidth{0.803000pt}%
\definecolor{currentstroke}{rgb}{0.000000,0.000000,0.000000}%
\pgfsetstrokecolor{currentstroke}%
\pgfsetdash{}{0pt}%
\pgfsys@defobject{currentmarker}{\pgfqpoint{-0.048611in}{0.000000in}}{\pgfqpoint{0.000000in}{0.000000in}}{%
\pgfpathmoveto{\pgfqpoint{0.000000in}{0.000000in}}%
\pgfpathlineto{\pgfqpoint{-0.048611in}{0.000000in}}%
\pgfusepath{stroke,fill}%
}%
\begin{pgfscope}%
\pgfsys@transformshift{0.934300in}{4.408507in}%
\pgfsys@useobject{currentmarker}{}%
\end{pgfscope}%
\end{pgfscope}%
\begin{pgfscope}%
\definecolor{textcolor}{rgb}{0.000000,0.000000,0.000000}%
\pgfsetstrokecolor{textcolor}%
\pgfsetfillcolor{textcolor}%
\pgftext[x=0.767633in,y=4.360281in,left,base]{\color{textcolor}\rmfamily\fontsize{10.000000}{12.000000}\selectfont \(\displaystyle 0\)}%
\end{pgfscope}%
\begin{pgfscope}%
\definecolor{textcolor}{rgb}{0.000000,0.000000,0.000000}%
\pgfsetstrokecolor{textcolor}%
\pgfsetfillcolor{textcolor}%
\pgftext[x=0.534608in,y=4.057751in,,bottom,rotate=90.000000]{\color{textcolor}\rmfamily\fontsize{10.000000}{12.000000}\selectfont Amplitude (dB)}%
\end{pgfscope}%
\begin{pgfscope}%
\pgfpathrectangle{\pgfqpoint{0.934300in}{3.664143in}}{\pgfqpoint{5.211922in}{0.787215in}}%
\pgfusepath{clip}%
\pgfsetrectcap%
\pgfsetroundjoin%
\pgfsetlinewidth{1.505625pt}%
\definecolor{currentstroke}{rgb}{0.121569,0.466667,0.705882}%
\pgfsetstrokecolor{currentstroke}%
\pgfsetdash{}{0pt}%
\pgfpathmoveto{\pgfqpoint{0.934300in}{4.401233in}}%
\pgfpathlineto{\pgfqpoint{0.986420in}{4.405787in}}%
\pgfpathlineto{\pgfqpoint{1.038539in}{4.413571in}}%
\pgfpathlineto{\pgfqpoint{1.090658in}{4.415005in}}%
\pgfpathlineto{\pgfqpoint{1.194896in}{4.401727in}}%
\pgfpathlineto{\pgfqpoint{1.247016in}{4.403532in}}%
\pgfpathlineto{\pgfqpoint{1.299135in}{4.411452in}}%
\pgfpathlineto{\pgfqpoint{1.351254in}{4.415551in}}%
\pgfpathlineto{\pgfqpoint{1.403373in}{4.410918in}}%
\pgfpathlineto{\pgfqpoint{1.455493in}{4.403102in}}%
\pgfpathlineto{\pgfqpoint{1.507612in}{4.401967in}}%
\pgfpathlineto{\pgfqpoint{1.611850in}{4.415219in}}%
\pgfpathlineto{\pgfqpoint{1.663969in}{4.413026in}}%
\pgfpathlineto{\pgfqpoint{1.716089in}{4.405040in}}%
\pgfpathlineto{\pgfqpoint{1.768208in}{4.401270in}}%
\pgfpathlineto{\pgfqpoint{1.820327in}{4.406706in}}%
\pgfpathlineto{\pgfqpoint{1.872446in}{4.414249in}}%
\pgfpathlineto{\pgfqpoint{1.924566in}{4.414420in}}%
\pgfpathlineto{\pgfqpoint{1.976685in}{4.406945in}}%
\pgfpathlineto{\pgfqpoint{2.028804in}{4.401298in}}%
\pgfpathlineto{\pgfqpoint{2.080923in}{4.405194in}}%
\pgfpathlineto{\pgfqpoint{2.133042in}{4.413406in}}%
\pgfpathlineto{\pgfqpoint{2.185162in}{4.414857in}}%
\pgfpathlineto{\pgfqpoint{2.237281in}{4.407268in}}%
\pgfpathlineto{\pgfqpoint{2.289400in}{4.401234in}}%
\pgfpathlineto{\pgfqpoint{2.341519in}{4.407102in}}%
\pgfpathlineto{\pgfqpoint{2.393638in}{4.415551in}}%
\pgfpathlineto{\pgfqpoint{2.445758in}{4.401233in}}%
\pgfpathlineto{\pgfqpoint{2.497877in}{4.336070in}}%
\pgfpathlineto{\pgfqpoint{2.549996in}{4.182205in}}%
\pgfpathlineto{\pgfqpoint{2.602115in}{3.849026in}}%
\pgfpathlineto{\pgfqpoint{2.654235in}{3.769396in}}%
\pgfpathlineto{\pgfqpoint{2.706354in}{4.142639in}}%
\pgfpathlineto{\pgfqpoint{2.758473in}{4.316544in}}%
\pgfpathlineto{\pgfqpoint{2.810592in}{4.393395in}}%
\pgfpathlineto{\pgfqpoint{2.862711in}{4.415282in}}%
\pgfpathlineto{\pgfqpoint{2.914831in}{4.409693in}}%
\pgfpathlineto{\pgfqpoint{2.966950in}{4.401600in}}%
\pgfpathlineto{\pgfqpoint{3.019069in}{4.404580in}}%
\pgfpathlineto{\pgfqpoint{3.071188in}{4.412958in}}%
\pgfpathlineto{\pgfqpoint{3.123308in}{4.415220in}}%
\pgfpathlineto{\pgfqpoint{3.227546in}{4.402000in}}%
\pgfpathlineto{\pgfqpoint{3.279665in}{4.402949in}}%
\pgfpathlineto{\pgfqpoint{3.331784in}{4.410468in}}%
\pgfpathlineto{\pgfqpoint{3.383904in}{4.415487in}}%
\pgfpathlineto{\pgfqpoint{3.436023in}{4.412349in}}%
\pgfpathlineto{\pgfqpoint{3.488142in}{4.404564in}}%
\pgfpathlineto{\pgfqpoint{3.540261in}{4.401304in}}%
\pgfpathlineto{\pgfqpoint{3.592380in}{4.406717in}}%
\pgfpathlineto{\pgfqpoint{3.644500in}{4.414161in}}%
\pgfpathlineto{\pgfqpoint{3.696619in}{4.414501in}}%
\pgfpathlineto{\pgfqpoint{3.748738in}{4.406940in}}%
\pgfpathlineto{\pgfqpoint{3.800857in}{4.401243in}}%
\pgfpathlineto{\pgfqpoint{3.852977in}{4.406632in}}%
\pgfpathlineto{\pgfqpoint{3.905096in}{4.415305in}}%
\pgfpathlineto{\pgfqpoint{3.957215in}{4.404400in}}%
\pgfpathlineto{\pgfqpoint{4.009334in}{4.347516in}}%
\pgfpathlineto{\pgfqpoint{4.061453in}{4.210087in}}%
\pgfpathlineto{\pgfqpoint{4.113573in}{3.912549in}}%
\pgfpathlineto{\pgfqpoint{4.165692in}{3.699926in}}%
\pgfpathlineto{\pgfqpoint{4.217811in}{4.106462in}}%
\pgfpathlineto{\pgfqpoint{4.269930in}{4.301255in}}%
\pgfpathlineto{\pgfqpoint{4.322050in}{4.388327in}}%
\pgfpathlineto{\pgfqpoint{4.374169in}{4.414894in}}%
\pgfpathlineto{\pgfqpoint{4.426288in}{4.410232in}}%
\pgfpathlineto{\pgfqpoint{4.478407in}{4.401620in}}%
\pgfpathlineto{\pgfqpoint{4.530526in}{4.404791in}}%
\pgfpathlineto{\pgfqpoint{4.582646in}{4.413454in}}%
\pgfpathlineto{\pgfqpoint{4.634765in}{4.414775in}}%
\pgfpathlineto{\pgfqpoint{4.686884in}{4.407324in}}%
\pgfpathlineto{\pgfqpoint{4.739003in}{4.401329in}}%
\pgfpathlineto{\pgfqpoint{4.791123in}{4.405044in}}%
\pgfpathlineto{\pgfqpoint{4.843242in}{4.413201in}}%
\pgfpathlineto{\pgfqpoint{4.895361in}{4.415072in}}%
\pgfpathlineto{\pgfqpoint{4.999599in}{4.401650in}}%
\pgfpathlineto{\pgfqpoint{5.051719in}{4.403858in}}%
\pgfpathlineto{\pgfqpoint{5.103838in}{4.411989in}}%
\pgfpathlineto{\pgfqpoint{5.155957in}{4.415492in}}%
\pgfpathlineto{\pgfqpoint{5.208076in}{4.410045in}}%
\pgfpathlineto{\pgfqpoint{5.260195in}{4.402431in}}%
\pgfpathlineto{\pgfqpoint{5.312315in}{4.402605in}}%
\pgfpathlineto{\pgfqpoint{5.364434in}{4.410323in}}%
\pgfpathlineto{\pgfqpoint{5.416553in}{4.415512in}}%
\pgfpathlineto{\pgfqpoint{5.468672in}{4.411739in}}%
\pgfpathlineto{\pgfqpoint{5.520792in}{4.403661in}}%
\pgfpathlineto{\pgfqpoint{5.572911in}{4.401720in}}%
\pgfpathlineto{\pgfqpoint{5.677149in}{4.415122in}}%
\pgfpathlineto{\pgfqpoint{5.729268in}{4.413259in}}%
\pgfpathlineto{\pgfqpoint{5.781388in}{4.405221in}}%
\pgfpathlineto{\pgfqpoint{5.833507in}{4.401271in}}%
\pgfpathlineto{\pgfqpoint{5.885626in}{4.406682in}}%
\pgfpathlineto{\pgfqpoint{5.937745in}{4.414253in}}%
\pgfpathlineto{\pgfqpoint{5.989865in}{4.414393in}}%
\pgfpathlineto{\pgfqpoint{6.041984in}{4.406923in}}%
\pgfpathlineto{\pgfqpoint{6.094103in}{4.401290in}}%
\pgfpathlineto{\pgfqpoint{6.156222in}{4.406540in}}%
\pgfpathlineto{\pgfqpoint{6.156222in}{4.406540in}}%
\pgfusepath{stroke}%
\end{pgfscope}%
\begin{pgfscope}%
\pgfsetrectcap%
\pgfsetmiterjoin%
\pgfsetlinewidth{0.803000pt}%
\definecolor{currentstroke}{rgb}{0.000000,0.000000,0.000000}%
\pgfsetstrokecolor{currentstroke}%
\pgfsetdash{}{0pt}%
\pgfpathmoveto{\pgfqpoint{0.934300in}{3.664143in}}%
\pgfpathlineto{\pgfqpoint{0.934300in}{4.451359in}}%
\pgfusepath{stroke}%
\end{pgfscope}%
\begin{pgfscope}%
\pgfsetrectcap%
\pgfsetmiterjoin%
\pgfsetlinewidth{0.803000pt}%
\definecolor{currentstroke}{rgb}{0.000000,0.000000,0.000000}%
\pgfsetstrokecolor{currentstroke}%
\pgfsetdash{}{0pt}%
\pgfpathmoveto{\pgfqpoint{6.146222in}{3.664143in}}%
\pgfpathlineto{\pgfqpoint{6.146222in}{4.451359in}}%
\pgfusepath{stroke}%
\end{pgfscope}%
\begin{pgfscope}%
\pgfsetrectcap%
\pgfsetmiterjoin%
\pgfsetlinewidth{0.803000pt}%
\definecolor{currentstroke}{rgb}{0.000000,0.000000,0.000000}%
\pgfsetstrokecolor{currentstroke}%
\pgfsetdash{}{0pt}%
\pgfpathmoveto{\pgfqpoint{0.934300in}{3.664143in}}%
\pgfpathlineto{\pgfqpoint{6.146222in}{3.664143in}}%
\pgfusepath{stroke}%
\end{pgfscope}%
\begin{pgfscope}%
\pgfsetrectcap%
\pgfsetmiterjoin%
\pgfsetlinewidth{0.803000pt}%
\definecolor{currentstroke}{rgb}{0.000000,0.000000,0.000000}%
\pgfsetstrokecolor{currentstroke}%
\pgfsetdash{}{0pt}%
\pgfpathmoveto{\pgfqpoint{0.934300in}{4.451359in}}%
\pgfpathlineto{\pgfqpoint{6.146222in}{4.451359in}}%
\pgfusepath{stroke}%
\end{pgfscope}%
\begin{pgfscope}%
\definecolor{textcolor}{rgb}{0.000000,0.000000,0.000000}%
\pgfsetstrokecolor{textcolor}%
\pgfsetfillcolor{textcolor}%
\pgftext[x=3.540261in,y=4.534692in,,base]{\color{textcolor}\rmfamily\fontsize{12.000000}{14.400000}\selectfont Filter Frequency Response}%
\end{pgfscope}%
\begin{pgfscope}%
\pgfsetbuttcap%
\pgfsetmiterjoin%
\definecolor{currentfill}{rgb}{1.000000,1.000000,1.000000}%
\pgfsetfillcolor{currentfill}%
\pgfsetlinewidth{0.000000pt}%
\definecolor{currentstroke}{rgb}{0.000000,0.000000,0.000000}%
\pgfsetstrokecolor{currentstroke}%
\pgfsetstrokeopacity{0.000000}%
\pgfsetdash{}{0pt}%
\pgfpathmoveto{\pgfqpoint{0.934300in}{2.114143in}}%
\pgfpathlineto{\pgfqpoint{6.146222in}{2.114143in}}%
\pgfpathlineto{\pgfqpoint{6.146222in}{2.901359in}}%
\pgfpathlineto{\pgfqpoint{0.934300in}{2.901359in}}%
\pgfpathclose%
\pgfusepath{fill}%
\end{pgfscope}%
\begin{pgfscope}%
\pgfsetbuttcap%
\pgfsetroundjoin%
\definecolor{currentfill}{rgb}{0.000000,0.000000,0.000000}%
\pgfsetfillcolor{currentfill}%
\pgfsetlinewidth{0.803000pt}%
\definecolor{currentstroke}{rgb}{0.000000,0.000000,0.000000}%
\pgfsetstrokecolor{currentstroke}%
\pgfsetdash{}{0pt}%
\pgfsys@defobject{currentmarker}{\pgfqpoint{0.000000in}{-0.048611in}}{\pgfqpoint{0.000000in}{0.000000in}}{%
\pgfpathmoveto{\pgfqpoint{0.000000in}{0.000000in}}%
\pgfpathlineto{\pgfqpoint{0.000000in}{-0.048611in}}%
\pgfusepath{stroke,fill}%
}%
\begin{pgfscope}%
\pgfsys@transformshift{1.171206in}{2.114143in}%
\pgfsys@useobject{currentmarker}{}%
\end{pgfscope}%
\end{pgfscope}%
\begin{pgfscope}%
\definecolor{textcolor}{rgb}{0.000000,0.000000,0.000000}%
\pgfsetstrokecolor{textcolor}%
\pgfsetfillcolor{textcolor}%
\pgftext[x=1.171206in,y=2.016921in,,top]{\color{textcolor}\rmfamily\fontsize{10.000000}{12.000000}\selectfont \(\displaystyle 0\)}%
\end{pgfscope}%
\begin{pgfscope}%
\pgfsetbuttcap%
\pgfsetroundjoin%
\definecolor{currentfill}{rgb}{0.000000,0.000000,0.000000}%
\pgfsetfillcolor{currentfill}%
\pgfsetlinewidth{0.803000pt}%
\definecolor{currentstroke}{rgb}{0.000000,0.000000,0.000000}%
\pgfsetstrokecolor{currentstroke}%
\pgfsetdash{}{0pt}%
\pgfsys@defobject{currentmarker}{\pgfqpoint{0.000000in}{-0.048611in}}{\pgfqpoint{0.000000in}{0.000000in}}{%
\pgfpathmoveto{\pgfqpoint{0.000000in}{0.000000in}}%
\pgfpathlineto{\pgfqpoint{0.000000in}{-0.048611in}}%
\pgfusepath{stroke,fill}%
}%
\begin{pgfscope}%
\pgfsys@transformshift{2.141590in}{2.114143in}%
\pgfsys@useobject{currentmarker}{}%
\end{pgfscope}%
\end{pgfscope}%
\begin{pgfscope}%
\definecolor{textcolor}{rgb}{0.000000,0.000000,0.000000}%
\pgfsetstrokecolor{textcolor}%
\pgfsetfillcolor{textcolor}%
\pgftext[x=2.141590in,y=2.016921in,,top]{\color{textcolor}\rmfamily\fontsize{10.000000}{12.000000}\selectfont \(\displaystyle 10\)}%
\end{pgfscope}%
\begin{pgfscope}%
\pgfsetbuttcap%
\pgfsetroundjoin%
\definecolor{currentfill}{rgb}{0.000000,0.000000,0.000000}%
\pgfsetfillcolor{currentfill}%
\pgfsetlinewidth{0.803000pt}%
\definecolor{currentstroke}{rgb}{0.000000,0.000000,0.000000}%
\pgfsetstrokecolor{currentstroke}%
\pgfsetdash{}{0pt}%
\pgfsys@defobject{currentmarker}{\pgfqpoint{0.000000in}{-0.048611in}}{\pgfqpoint{0.000000in}{0.000000in}}{%
\pgfpathmoveto{\pgfqpoint{0.000000in}{0.000000in}}%
\pgfpathlineto{\pgfqpoint{0.000000in}{-0.048611in}}%
\pgfusepath{stroke,fill}%
}%
\begin{pgfscope}%
\pgfsys@transformshift{3.111975in}{2.114143in}%
\pgfsys@useobject{currentmarker}{}%
\end{pgfscope}%
\end{pgfscope}%
\begin{pgfscope}%
\definecolor{textcolor}{rgb}{0.000000,0.000000,0.000000}%
\pgfsetstrokecolor{textcolor}%
\pgfsetfillcolor{textcolor}%
\pgftext[x=3.111975in,y=2.016921in,,top]{\color{textcolor}\rmfamily\fontsize{10.000000}{12.000000}\selectfont \(\displaystyle 20\)}%
\end{pgfscope}%
\begin{pgfscope}%
\pgfsetbuttcap%
\pgfsetroundjoin%
\definecolor{currentfill}{rgb}{0.000000,0.000000,0.000000}%
\pgfsetfillcolor{currentfill}%
\pgfsetlinewidth{0.803000pt}%
\definecolor{currentstroke}{rgb}{0.000000,0.000000,0.000000}%
\pgfsetstrokecolor{currentstroke}%
\pgfsetdash{}{0pt}%
\pgfsys@defobject{currentmarker}{\pgfqpoint{0.000000in}{-0.048611in}}{\pgfqpoint{0.000000in}{0.000000in}}{%
\pgfpathmoveto{\pgfqpoint{0.000000in}{0.000000in}}%
\pgfpathlineto{\pgfqpoint{0.000000in}{-0.048611in}}%
\pgfusepath{stroke,fill}%
}%
\begin{pgfscope}%
\pgfsys@transformshift{4.082359in}{2.114143in}%
\pgfsys@useobject{currentmarker}{}%
\end{pgfscope}%
\end{pgfscope}%
\begin{pgfscope}%
\definecolor{textcolor}{rgb}{0.000000,0.000000,0.000000}%
\pgfsetstrokecolor{textcolor}%
\pgfsetfillcolor{textcolor}%
\pgftext[x=4.082359in,y=2.016921in,,top]{\color{textcolor}\rmfamily\fontsize{10.000000}{12.000000}\selectfont \(\displaystyle 30\)}%
\end{pgfscope}%
\begin{pgfscope}%
\pgfsetbuttcap%
\pgfsetroundjoin%
\definecolor{currentfill}{rgb}{0.000000,0.000000,0.000000}%
\pgfsetfillcolor{currentfill}%
\pgfsetlinewidth{0.803000pt}%
\definecolor{currentstroke}{rgb}{0.000000,0.000000,0.000000}%
\pgfsetstrokecolor{currentstroke}%
\pgfsetdash{}{0pt}%
\pgfsys@defobject{currentmarker}{\pgfqpoint{0.000000in}{-0.048611in}}{\pgfqpoint{0.000000in}{0.000000in}}{%
\pgfpathmoveto{\pgfqpoint{0.000000in}{0.000000in}}%
\pgfpathlineto{\pgfqpoint{0.000000in}{-0.048611in}}%
\pgfusepath{stroke,fill}%
}%
\begin{pgfscope}%
\pgfsys@transformshift{5.052744in}{2.114143in}%
\pgfsys@useobject{currentmarker}{}%
\end{pgfscope}%
\end{pgfscope}%
\begin{pgfscope}%
\definecolor{textcolor}{rgb}{0.000000,0.000000,0.000000}%
\pgfsetstrokecolor{textcolor}%
\pgfsetfillcolor{textcolor}%
\pgftext[x=5.052744in,y=2.016921in,,top]{\color{textcolor}\rmfamily\fontsize{10.000000}{12.000000}\selectfont \(\displaystyle 40\)}%
\end{pgfscope}%
\begin{pgfscope}%
\pgfsetbuttcap%
\pgfsetroundjoin%
\definecolor{currentfill}{rgb}{0.000000,0.000000,0.000000}%
\pgfsetfillcolor{currentfill}%
\pgfsetlinewidth{0.803000pt}%
\definecolor{currentstroke}{rgb}{0.000000,0.000000,0.000000}%
\pgfsetstrokecolor{currentstroke}%
\pgfsetdash{}{0pt}%
\pgfsys@defobject{currentmarker}{\pgfqpoint{0.000000in}{-0.048611in}}{\pgfqpoint{0.000000in}{0.000000in}}{%
\pgfpathmoveto{\pgfqpoint{0.000000in}{0.000000in}}%
\pgfpathlineto{\pgfqpoint{0.000000in}{-0.048611in}}%
\pgfusepath{stroke,fill}%
}%
\begin{pgfscope}%
\pgfsys@transformshift{6.023128in}{2.114143in}%
\pgfsys@useobject{currentmarker}{}%
\end{pgfscope}%
\end{pgfscope}%
\begin{pgfscope}%
\definecolor{textcolor}{rgb}{0.000000,0.000000,0.000000}%
\pgfsetstrokecolor{textcolor}%
\pgfsetfillcolor{textcolor}%
\pgftext[x=6.023128in,y=2.016921in,,top]{\color{textcolor}\rmfamily\fontsize{10.000000}{12.000000}\selectfont \(\displaystyle 50\)}%
\end{pgfscope}%
\begin{pgfscope}%
\definecolor{textcolor}{rgb}{0.000000,0.000000,0.000000}%
\pgfsetstrokecolor{textcolor}%
\pgfsetfillcolor{textcolor}%
\pgftext[x=3.540261in,y=1.837909in,,top]{\color{textcolor}\rmfamily\fontsize{10.000000}{12.000000}\selectfont Time (s)}%
\end{pgfscope}%
\begin{pgfscope}%
\pgfsetbuttcap%
\pgfsetroundjoin%
\definecolor{currentfill}{rgb}{0.000000,0.000000,0.000000}%
\pgfsetfillcolor{currentfill}%
\pgfsetlinewidth{0.803000pt}%
\definecolor{currentstroke}{rgb}{0.000000,0.000000,0.000000}%
\pgfsetstrokecolor{currentstroke}%
\pgfsetdash{}{0pt}%
\pgfsys@defobject{currentmarker}{\pgfqpoint{-0.048611in}{0.000000in}}{\pgfqpoint{0.000000in}{0.000000in}}{%
\pgfpathmoveto{\pgfqpoint{0.000000in}{0.000000in}}%
\pgfpathlineto{\pgfqpoint{-0.048611in}{0.000000in}}%
\pgfusepath{stroke,fill}%
}%
\begin{pgfscope}%
\pgfsys@transformshift{0.934300in}{2.445466in}%
\pgfsys@useobject{currentmarker}{}%
\end{pgfscope}%
\end{pgfscope}%
\begin{pgfscope}%
\definecolor{textcolor}{rgb}{0.000000,0.000000,0.000000}%
\pgfsetstrokecolor{textcolor}%
\pgfsetfillcolor{textcolor}%
\pgftext[x=0.767633in,y=2.397241in,left,base]{\color{textcolor}\rmfamily\fontsize{10.000000}{12.000000}\selectfont \(\displaystyle 0\)}%
\end{pgfscope}%
\begin{pgfscope}%
\pgfsetbuttcap%
\pgfsetroundjoin%
\definecolor{currentfill}{rgb}{0.000000,0.000000,0.000000}%
\pgfsetfillcolor{currentfill}%
\pgfsetlinewidth{0.803000pt}%
\definecolor{currentstroke}{rgb}{0.000000,0.000000,0.000000}%
\pgfsetstrokecolor{currentstroke}%
\pgfsetdash{}{0pt}%
\pgfsys@defobject{currentmarker}{\pgfqpoint{-0.048611in}{0.000000in}}{\pgfqpoint{0.000000in}{0.000000in}}{%
\pgfpathmoveto{\pgfqpoint{0.000000in}{0.000000in}}%
\pgfpathlineto{\pgfqpoint{-0.048611in}{0.000000in}}%
\pgfusepath{stroke,fill}%
}%
\begin{pgfscope}%
\pgfsys@transformshift{0.934300in}{2.793097in}%
\pgfsys@useobject{currentmarker}{}%
\end{pgfscope}%
\end{pgfscope}%
\begin{pgfscope}%
\definecolor{textcolor}{rgb}{0.000000,0.000000,0.000000}%
\pgfsetstrokecolor{textcolor}%
\pgfsetfillcolor{textcolor}%
\pgftext[x=0.559300in,y=2.744871in,left,base]{\color{textcolor}\rmfamily\fontsize{10.000000}{12.000000}\selectfont \(\displaystyle 1000\)}%
\end{pgfscope}%
\begin{pgfscope}%
\definecolor{textcolor}{rgb}{0.000000,0.000000,0.000000}%
\pgfsetstrokecolor{textcolor}%
\pgfsetfillcolor{textcolor}%
\pgftext[x=0.503744in,y=2.507751in,,bottom,rotate=90.000000]{\color{textcolor}\rmfamily\fontsize{10.000000}{12.000000}\selectfont ECG Voltage (\(\displaystyle \mu V\))}%
\end{pgfscope}%
\begin{pgfscope}%
\pgfpathrectangle{\pgfqpoint{0.934300in}{2.114143in}}{\pgfqpoint{5.211922in}{0.787215in}}%
\pgfusepath{clip}%
\pgfsetrectcap%
\pgfsetroundjoin%
\pgfsetlinewidth{1.505625pt}%
\definecolor{currentstroke}{rgb}{0.121569,0.466667,0.705882}%
\pgfsetstrokecolor{currentstroke}%
\pgfsetdash{}{0pt}%
\pgfpathmoveto{\pgfqpoint{1.171206in}{2.447024in}}%
\pgfpathlineto{\pgfqpoint{1.171680in}{2.443690in}}%
\pgfpathlineto{\pgfqpoint{1.172438in}{2.444746in}}%
\pgfpathlineto{\pgfqpoint{1.173006in}{2.441287in}}%
\pgfpathlineto{\pgfqpoint{1.173385in}{2.445005in}}%
\pgfpathlineto{\pgfqpoint{1.173859in}{2.449169in}}%
\pgfpathlineto{\pgfqpoint{1.174333in}{2.442721in}}%
\pgfpathlineto{\pgfqpoint{1.174617in}{2.440133in}}%
\pgfpathlineto{\pgfqpoint{1.175091in}{2.446168in}}%
\pgfpathlineto{\pgfqpoint{1.175376in}{2.448857in}}%
\pgfpathlineto{\pgfqpoint{1.175849in}{2.441015in}}%
\pgfpathlineto{\pgfqpoint{1.176134in}{2.437688in}}%
\pgfpathlineto{\pgfqpoint{1.176513in}{2.445797in}}%
\pgfpathlineto{\pgfqpoint{1.176892in}{2.455510in}}%
\pgfpathlineto{\pgfqpoint{1.177460in}{2.440544in}}%
\pgfpathlineto{\pgfqpoint{1.177745in}{2.434290in}}%
\pgfpathlineto{\pgfqpoint{1.178218in}{2.447602in}}%
\pgfpathlineto{\pgfqpoint{1.178503in}{2.454238in}}%
\pgfpathlineto{\pgfqpoint{1.178977in}{2.439567in}}%
\pgfpathlineto{\pgfqpoint{1.179261in}{2.432260in}}%
\pgfpathlineto{\pgfqpoint{1.179735in}{2.452293in}}%
\pgfpathlineto{\pgfqpoint{1.180019in}{2.464120in}}%
\pgfpathlineto{\pgfqpoint{1.180493in}{2.444038in}}%
\pgfpathlineto{\pgfqpoint{1.180872in}{2.425813in}}%
\pgfpathlineto{\pgfqpoint{1.181346in}{2.449902in}}%
\pgfpathlineto{\pgfqpoint{1.181630in}{2.463095in}}%
\pgfpathlineto{\pgfqpoint{1.182199in}{2.432767in}}%
\pgfpathlineto{\pgfqpoint{1.182388in}{2.425700in}}%
\pgfpathlineto{\pgfqpoint{1.182862in}{2.455448in}}%
\pgfpathlineto{\pgfqpoint{1.183241in}{2.474610in}}%
\pgfpathlineto{\pgfqpoint{1.183715in}{2.432713in}}%
\pgfpathlineto{\pgfqpoint{1.183999in}{2.414159in}}%
\pgfpathlineto{\pgfqpoint{1.184473in}{2.451463in}}%
\pgfpathlineto{\pgfqpoint{1.184757in}{2.473921in}}%
\pgfpathlineto{\pgfqpoint{1.185326in}{2.429756in}}%
\pgfpathlineto{\pgfqpoint{1.185515in}{2.417468in}}%
\pgfpathlineto{\pgfqpoint{1.185989in}{2.456230in}}%
\pgfpathlineto{\pgfqpoint{1.186368in}{2.485474in}}%
\pgfpathlineto{\pgfqpoint{1.186842in}{2.427570in}}%
\pgfpathlineto{\pgfqpoint{1.187126in}{2.400113in}}%
\pgfpathlineto{\pgfqpoint{1.187600in}{2.452732in}}%
\pgfpathlineto{\pgfqpoint{1.187979in}{2.487297in}}%
\pgfpathlineto{\pgfqpoint{1.188453in}{2.426894in}}%
\pgfpathlineto{\pgfqpoint{1.188737in}{2.406569in}}%
\pgfpathlineto{\pgfqpoint{1.189211in}{2.471125in}}%
\pgfpathlineto{\pgfqpoint{1.190253in}{2.577935in}}%
\pgfpathlineto{\pgfqpoint{1.189969in}{2.423471in}}%
\pgfpathlineto{\pgfqpoint{1.190633in}{2.523906in}}%
\pgfpathlineto{\pgfqpoint{1.191106in}{2.456809in}}%
\pgfpathlineto{\pgfqpoint{1.191580in}{2.544949in}}%
\pgfpathlineto{\pgfqpoint{1.191770in}{2.558323in}}%
\pgfpathlineto{\pgfqpoint{1.192244in}{2.509112in}}%
\pgfpathlineto{\pgfqpoint{1.192623in}{2.457252in}}%
\pgfpathlineto{\pgfqpoint{1.193191in}{2.519871in}}%
\pgfpathlineto{\pgfqpoint{1.193381in}{2.533061in}}%
\pgfpathlineto{\pgfqpoint{1.193855in}{2.485594in}}%
\pgfpathlineto{\pgfqpoint{1.194234in}{2.451640in}}%
\pgfpathlineto{\pgfqpoint{1.194802in}{2.504124in}}%
\pgfpathlineto{\pgfqpoint{1.194992in}{2.514253in}}%
\pgfpathlineto{\pgfqpoint{1.195466in}{2.471016in}}%
\pgfpathlineto{\pgfqpoint{1.195845in}{2.452766in}}%
\pgfpathlineto{\pgfqpoint{1.196318in}{2.487565in}}%
\pgfpathlineto{\pgfqpoint{1.196508in}{2.498534in}}%
\pgfpathlineto{\pgfqpoint{1.197076in}{2.461620in}}%
\pgfpathlineto{\pgfqpoint{1.197361in}{2.441377in}}%
\pgfpathlineto{\pgfqpoint{1.197929in}{2.473898in}}%
\pgfpathlineto{\pgfqpoint{1.198024in}{2.473528in}}%
\pgfpathlineto{\pgfqpoint{1.200488in}{2.394199in}}%
\pgfpathlineto{\pgfqpoint{1.200772in}{2.405164in}}%
\pgfpathlineto{\pgfqpoint{1.201057in}{2.418702in}}%
\pgfpathlineto{\pgfqpoint{1.201530in}{2.403895in}}%
\pgfpathlineto{\pgfqpoint{1.201815in}{2.408123in}}%
\pgfpathlineto{\pgfqpoint{1.202478in}{2.397759in}}%
\pgfpathlineto{\pgfqpoint{1.202857in}{2.407528in}}%
\pgfpathlineto{\pgfqpoint{1.202952in}{2.410004in}}%
\pgfpathlineto{\pgfqpoint{1.203236in}{2.396390in}}%
\pgfpathlineto{\pgfqpoint{1.203426in}{2.388505in}}%
\pgfpathlineto{\pgfqpoint{1.204184in}{2.403981in}}%
\pgfpathlineto{\pgfqpoint{1.204279in}{2.404002in}}%
\pgfpathlineto{\pgfqpoint{1.205131in}{2.396090in}}%
\pgfpathlineto{\pgfqpoint{1.205416in}{2.403642in}}%
\pgfpathlineto{\pgfqpoint{1.205890in}{2.429974in}}%
\pgfpathlineto{\pgfqpoint{1.206837in}{2.425195in}}%
\pgfpathlineto{\pgfqpoint{1.207027in}{2.427257in}}%
\pgfpathlineto{\pgfqpoint{1.207311in}{2.415015in}}%
\pgfpathlineto{\pgfqpoint{1.208069in}{2.401231in}}%
\pgfpathlineto{\pgfqpoint{1.208448in}{2.413127in}}%
\pgfpathlineto{\pgfqpoint{1.208827in}{2.420215in}}%
\pgfpathlineto{\pgfqpoint{1.209112in}{2.438789in}}%
\pgfpathlineto{\pgfqpoint{1.210059in}{2.436553in}}%
\pgfpathlineto{\pgfqpoint{1.212239in}{2.474322in}}%
\pgfpathlineto{\pgfqpoint{1.210723in}{2.436219in}}%
\pgfpathlineto{\pgfqpoint{1.212713in}{2.456602in}}%
\pgfpathlineto{\pgfqpoint{1.214703in}{2.400711in}}%
\pgfpathlineto{\pgfqpoint{1.213186in}{2.457066in}}%
\pgfpathlineto{\pgfqpoint{1.214892in}{2.405726in}}%
\pgfpathlineto{\pgfqpoint{1.215271in}{2.421475in}}%
\pgfpathlineto{\pgfqpoint{1.215840in}{2.397402in}}%
\pgfpathlineto{\pgfqpoint{1.216124in}{2.377594in}}%
\pgfpathlineto{\pgfqpoint{1.217072in}{2.386043in}}%
\pgfpathlineto{\pgfqpoint{1.217166in}{2.386193in}}%
\pgfpathlineto{\pgfqpoint{1.217261in}{2.384368in}}%
\pgfpathlineto{\pgfqpoint{1.217451in}{2.381151in}}%
\pgfpathlineto{\pgfqpoint{1.217735in}{2.394229in}}%
\pgfpathlineto{\pgfqpoint{1.218398in}{2.443999in}}%
\pgfpathlineto{\pgfqpoint{1.218872in}{2.402898in}}%
\pgfpathlineto{\pgfqpoint{1.219441in}{2.388233in}}%
\pgfpathlineto{\pgfqpoint{1.220009in}{2.396409in}}%
\pgfpathlineto{\pgfqpoint{1.221241in}{2.417321in}}%
\pgfpathlineto{\pgfqpoint{1.222568in}{2.674356in}}%
\pgfpathlineto{\pgfqpoint{1.223705in}{2.587792in}}%
\pgfpathlineto{\pgfqpoint{1.224558in}{2.356655in}}%
\pgfpathlineto{\pgfqpoint{1.225506in}{2.429562in}}%
\pgfpathlineto{\pgfqpoint{1.225600in}{2.428640in}}%
\pgfpathlineto{\pgfqpoint{1.225885in}{2.435585in}}%
\pgfpathlineto{\pgfqpoint{1.227685in}{2.468792in}}%
\pgfpathlineto{\pgfqpoint{1.228822in}{2.429191in}}%
\pgfpathlineto{\pgfqpoint{1.229581in}{2.448167in}}%
\pgfpathlineto{\pgfqpoint{1.230433in}{2.465625in}}%
\pgfpathlineto{\pgfqpoint{1.231002in}{2.456489in}}%
\pgfpathlineto{\pgfqpoint{1.231950in}{2.423406in}}%
\pgfpathlineto{\pgfqpoint{1.232329in}{2.441883in}}%
\pgfpathlineto{\pgfqpoint{1.233750in}{2.460451in}}%
\pgfpathlineto{\pgfqpoint{1.233845in}{2.461507in}}%
\pgfpathlineto{\pgfqpoint{1.234129in}{2.452692in}}%
\pgfpathlineto{\pgfqpoint{1.234698in}{2.432456in}}%
\pgfpathlineto{\pgfqpoint{1.235266in}{2.448049in}}%
\pgfpathlineto{\pgfqpoint{1.235740in}{2.471217in}}%
\pgfpathlineto{\pgfqpoint{1.236877in}{2.466438in}}%
\pgfpathlineto{\pgfqpoint{1.238109in}{2.449585in}}%
\pgfpathlineto{\pgfqpoint{1.238394in}{2.452022in}}%
\pgfpathlineto{\pgfqpoint{1.238773in}{2.466688in}}%
\pgfpathlineto{\pgfqpoint{1.238962in}{2.475299in}}%
\pgfpathlineto{\pgfqpoint{1.239531in}{2.461364in}}%
\pgfpathlineto{\pgfqpoint{1.239720in}{2.462476in}}%
\pgfpathlineto{\pgfqpoint{1.240099in}{2.456892in}}%
\pgfpathlineto{\pgfqpoint{1.242753in}{2.363455in}}%
\pgfpathlineto{\pgfqpoint{1.243227in}{2.381735in}}%
\pgfpathlineto{\pgfqpoint{1.244553in}{2.430018in}}%
\pgfpathlineto{\pgfqpoint{1.245311in}{2.453197in}}%
\pgfpathlineto{\pgfqpoint{1.245690in}{2.436468in}}%
\pgfpathlineto{\pgfqpoint{1.247681in}{2.393376in}}%
\pgfpathlineto{\pgfqpoint{1.247870in}{2.395377in}}%
\pgfpathlineto{\pgfqpoint{1.248154in}{2.400569in}}%
\pgfpathlineto{\pgfqpoint{1.248628in}{2.388323in}}%
\pgfpathlineto{\pgfqpoint{1.248723in}{2.388159in}}%
\pgfpathlineto{\pgfqpoint{1.248818in}{2.389403in}}%
\pgfpathlineto{\pgfqpoint{1.248912in}{2.390420in}}%
\pgfpathlineto{\pgfqpoint{1.249197in}{2.385089in}}%
\pgfpathlineto{\pgfqpoint{1.249386in}{2.379546in}}%
\pgfpathlineto{\pgfqpoint{1.249955in}{2.388822in}}%
\pgfpathlineto{\pgfqpoint{1.250239in}{2.387234in}}%
\pgfpathlineto{\pgfqpoint{1.251376in}{2.367691in}}%
\pgfpathlineto{\pgfqpoint{1.251850in}{2.375569in}}%
\pgfpathlineto{\pgfqpoint{1.252040in}{2.373967in}}%
\pgfpathlineto{\pgfqpoint{1.252324in}{2.367695in}}%
\pgfpathlineto{\pgfqpoint{1.252703in}{2.384415in}}%
\pgfpathlineto{\pgfqpoint{1.252798in}{2.386840in}}%
\pgfpathlineto{\pgfqpoint{1.253461in}{2.376748in}}%
\pgfpathlineto{\pgfqpoint{1.253556in}{2.375936in}}%
\pgfpathlineto{\pgfqpoint{1.253935in}{2.381321in}}%
\pgfpathlineto{\pgfqpoint{1.254504in}{2.377127in}}%
\pgfpathlineto{\pgfqpoint{1.254693in}{2.378239in}}%
\pgfpathlineto{\pgfqpoint{1.254977in}{2.370631in}}%
\pgfpathlineto{\pgfqpoint{1.255262in}{2.358709in}}%
\pgfpathlineto{\pgfqpoint{1.255925in}{2.378396in}}%
\pgfpathlineto{\pgfqpoint{1.258957in}{2.435996in}}%
\pgfpathlineto{\pgfqpoint{1.259242in}{2.452157in}}%
\pgfpathlineto{\pgfqpoint{1.260000in}{2.435480in}}%
\pgfpathlineto{\pgfqpoint{1.260379in}{2.444738in}}%
\pgfpathlineto{\pgfqpoint{1.260663in}{2.434325in}}%
\pgfpathlineto{\pgfqpoint{1.261706in}{2.399622in}}%
\pgfpathlineto{\pgfqpoint{1.261990in}{2.415499in}}%
\pgfpathlineto{\pgfqpoint{1.262179in}{2.421634in}}%
\pgfpathlineto{\pgfqpoint{1.262938in}{2.409580in}}%
\pgfpathlineto{\pgfqpoint{1.264359in}{2.381918in}}%
\pgfpathlineto{\pgfqpoint{1.264549in}{2.384687in}}%
\pgfpathlineto{\pgfqpoint{1.265496in}{2.431131in}}%
\pgfpathlineto{\pgfqpoint{1.266065in}{2.404608in}}%
\pgfpathlineto{\pgfqpoint{1.268150in}{2.370018in}}%
\pgfpathlineto{\pgfqpoint{1.268813in}{2.442849in}}%
\pgfpathlineto{\pgfqpoint{1.269761in}{2.646019in}}%
\pgfpathlineto{\pgfqpoint{1.270519in}{2.612128in}}%
\pgfpathlineto{\pgfqpoint{1.271087in}{2.526862in}}%
\pgfpathlineto{\pgfqpoint{1.271656in}{2.334745in}}%
\pgfpathlineto{\pgfqpoint{1.272509in}{2.400907in}}%
\pgfpathlineto{\pgfqpoint{1.272888in}{2.398331in}}%
\pgfpathlineto{\pgfqpoint{1.273077in}{2.402080in}}%
\pgfpathlineto{\pgfqpoint{1.274120in}{2.425989in}}%
\pgfpathlineto{\pgfqpoint{1.274404in}{2.415334in}}%
\pgfpathlineto{\pgfqpoint{1.274499in}{2.413646in}}%
\pgfpathlineto{\pgfqpoint{1.274783in}{2.426867in}}%
\pgfpathlineto{\pgfqpoint{1.274973in}{2.435252in}}%
\pgfpathlineto{\pgfqpoint{1.275446in}{2.402137in}}%
\pgfpathlineto{\pgfqpoint{1.275636in}{2.388025in}}%
\pgfpathlineto{\pgfqpoint{1.276394in}{2.406320in}}%
\pgfpathlineto{\pgfqpoint{1.277626in}{2.428218in}}%
\pgfpathlineto{\pgfqpoint{1.277816in}{2.425820in}}%
\pgfpathlineto{\pgfqpoint{1.278668in}{2.393497in}}%
\pgfpathlineto{\pgfqpoint{1.279427in}{2.409693in}}%
\pgfpathlineto{\pgfqpoint{1.280279in}{2.434804in}}%
\pgfpathlineto{\pgfqpoint{1.280753in}{2.429627in}}%
\pgfpathlineto{\pgfqpoint{1.281037in}{2.444572in}}%
\pgfpathlineto{\pgfqpoint{1.281606in}{2.418165in}}%
\pgfpathlineto{\pgfqpoint{1.282080in}{2.405106in}}%
\pgfpathlineto{\pgfqpoint{1.282364in}{2.422803in}}%
\pgfpathlineto{\pgfqpoint{1.283501in}{2.446301in}}%
\pgfpathlineto{\pgfqpoint{1.283596in}{2.443199in}}%
\pgfpathlineto{\pgfqpoint{1.284733in}{2.428960in}}%
\pgfpathlineto{\pgfqpoint{1.284259in}{2.455533in}}%
\pgfpathlineto{\pgfqpoint{1.284828in}{2.430031in}}%
\pgfpathlineto{\pgfqpoint{1.285586in}{2.429701in}}%
\pgfpathlineto{\pgfqpoint{1.286913in}{2.461693in}}%
\pgfpathlineto{\pgfqpoint{1.287102in}{2.459519in}}%
\pgfpathlineto{\pgfqpoint{1.287481in}{2.468118in}}%
\pgfpathlineto{\pgfqpoint{1.290514in}{2.531293in}}%
\pgfpathlineto{\pgfqpoint{1.290703in}{2.527177in}}%
\pgfpathlineto{\pgfqpoint{1.290798in}{2.526482in}}%
\pgfpathlineto{\pgfqpoint{1.290988in}{2.531517in}}%
\pgfpathlineto{\pgfqpoint{1.291841in}{2.543835in}}%
\pgfpathlineto{\pgfqpoint{1.292125in}{2.532932in}}%
\pgfpathlineto{\pgfqpoint{1.292220in}{2.531455in}}%
\pgfpathlineto{\pgfqpoint{1.292409in}{2.541428in}}%
\pgfpathlineto{\pgfqpoint{1.292599in}{2.549738in}}%
\pgfpathlineto{\pgfqpoint{1.292978in}{2.525235in}}%
\pgfpathlineto{\pgfqpoint{1.293357in}{2.533052in}}%
\pgfpathlineto{\pgfqpoint{1.296484in}{2.457701in}}%
\pgfpathlineto{\pgfqpoint{1.296579in}{2.457760in}}%
\pgfpathlineto{\pgfqpoint{1.296863in}{2.462922in}}%
\pgfpathlineto{\pgfqpoint{1.297147in}{2.453943in}}%
\pgfpathlineto{\pgfqpoint{1.298000in}{2.432892in}}%
\pgfpathlineto{\pgfqpoint{1.298664in}{2.435634in}}%
\pgfpathlineto{\pgfqpoint{1.300559in}{2.464305in}}%
\pgfpathlineto{\pgfqpoint{1.299327in}{2.433131in}}%
\pgfpathlineto{\pgfqpoint{1.300748in}{2.461675in}}%
\pgfpathlineto{\pgfqpoint{1.301980in}{2.446127in}}%
\pgfpathlineto{\pgfqpoint{1.302170in}{2.452289in}}%
\pgfpathlineto{\pgfqpoint{1.302359in}{2.458294in}}%
\pgfpathlineto{\pgfqpoint{1.303118in}{2.447194in}}%
\pgfpathlineto{\pgfqpoint{1.303212in}{2.447170in}}%
\pgfpathlineto{\pgfqpoint{1.303686in}{2.461064in}}%
\pgfpathlineto{\pgfqpoint{1.304444in}{2.450490in}}%
\pgfpathlineto{\pgfqpoint{1.304634in}{2.447280in}}%
\pgfpathlineto{\pgfqpoint{1.305297in}{2.453879in}}%
\pgfpathlineto{\pgfqpoint{1.306150in}{2.476515in}}%
\pgfpathlineto{\pgfqpoint{1.306719in}{2.465506in}}%
\pgfpathlineto{\pgfqpoint{1.307666in}{2.421603in}}%
\pgfpathlineto{\pgfqpoint{1.308330in}{2.409190in}}%
\pgfpathlineto{\pgfqpoint{1.308803in}{2.415918in}}%
\pgfpathlineto{\pgfqpoint{1.309277in}{2.437941in}}%
\pgfpathlineto{\pgfqpoint{1.309846in}{2.416022in}}%
\pgfpathlineto{\pgfqpoint{1.311267in}{2.395253in}}%
\pgfpathlineto{\pgfqpoint{1.311362in}{2.396977in}}%
\pgfpathlineto{\pgfqpoint{1.312310in}{2.423206in}}%
\pgfpathlineto{\pgfqpoint{1.312594in}{2.407553in}}%
\pgfpathlineto{\pgfqpoint{1.315058in}{2.315491in}}%
\pgfpathlineto{\pgfqpoint{1.315247in}{2.322401in}}%
\pgfpathlineto{\pgfqpoint{1.316953in}{2.595782in}}%
\pgfpathlineto{\pgfqpoint{1.317806in}{2.540575in}}%
\pgfpathlineto{\pgfqpoint{1.318469in}{2.336795in}}%
\pgfpathlineto{\pgfqpoint{1.319322in}{2.396631in}}%
\pgfpathlineto{\pgfqpoint{1.319512in}{2.391623in}}%
\pgfpathlineto{\pgfqpoint{1.319891in}{2.415718in}}%
\pgfpathlineto{\pgfqpoint{1.320175in}{2.408645in}}%
\pgfpathlineto{\pgfqpoint{1.320270in}{2.408009in}}%
\pgfpathlineto{\pgfqpoint{1.320365in}{2.412158in}}%
\pgfpathlineto{\pgfqpoint{1.321407in}{2.458776in}}%
\pgfpathlineto{\pgfqpoint{1.321881in}{2.444364in}}%
\pgfpathlineto{\pgfqpoint{1.321976in}{2.446708in}}%
\pgfpathlineto{\pgfqpoint{1.322260in}{2.435126in}}%
\pgfpathlineto{\pgfqpoint{1.322544in}{2.423078in}}%
\pgfpathlineto{\pgfqpoint{1.323208in}{2.437429in}}%
\pgfpathlineto{\pgfqpoint{1.324913in}{2.497612in}}%
\pgfpathlineto{\pgfqpoint{1.325198in}{2.489826in}}%
\pgfpathlineto{\pgfqpoint{1.325671in}{2.461688in}}%
\pgfpathlineto{\pgfqpoint{1.326335in}{2.483546in}}%
\pgfpathlineto{\pgfqpoint{1.326809in}{2.502258in}}%
\pgfpathlineto{\pgfqpoint{1.327567in}{2.520007in}}%
\pgfpathlineto{\pgfqpoint{1.327946in}{2.504219in}}%
\pgfpathlineto{\pgfqpoint{1.328040in}{2.502653in}}%
\pgfpathlineto{\pgfqpoint{1.328325in}{2.512532in}}%
\pgfpathlineto{\pgfqpoint{1.330694in}{2.563044in}}%
\pgfpathlineto{\pgfqpoint{1.328799in}{2.505619in}}%
\pgfpathlineto{\pgfqpoint{1.331073in}{2.552524in}}%
\pgfpathlineto{\pgfqpoint{1.332305in}{2.533608in}}%
\pgfpathlineto{\pgfqpoint{1.332494in}{2.542852in}}%
\pgfpathlineto{\pgfqpoint{1.332779in}{2.550302in}}%
\pgfpathlineto{\pgfqpoint{1.333442in}{2.540015in}}%
\pgfpathlineto{\pgfqpoint{1.334958in}{2.509290in}}%
\pgfpathlineto{\pgfqpoint{1.333821in}{2.541253in}}%
\pgfpathlineto{\pgfqpoint{1.335148in}{2.515441in}}%
\pgfpathlineto{\pgfqpoint{1.335337in}{2.525887in}}%
\pgfpathlineto{\pgfqpoint{1.335811in}{2.505795in}}%
\pgfpathlineto{\pgfqpoint{1.336190in}{2.515634in}}%
\pgfpathlineto{\pgfqpoint{1.336664in}{2.510799in}}%
\pgfpathlineto{\pgfqpoint{1.337517in}{2.494979in}}%
\pgfpathlineto{\pgfqpoint{1.337801in}{2.505652in}}%
\pgfpathlineto{\pgfqpoint{1.337991in}{2.510766in}}%
\pgfpathlineto{\pgfqpoint{1.338465in}{2.485974in}}%
\pgfpathlineto{\pgfqpoint{1.338559in}{2.487796in}}%
\pgfpathlineto{\pgfqpoint{1.338654in}{2.489336in}}%
\pgfpathlineto{\pgfqpoint{1.338938in}{2.477340in}}%
\pgfpathlineto{\pgfqpoint{1.341307in}{2.399058in}}%
\pgfpathlineto{\pgfqpoint{1.341402in}{2.399776in}}%
\pgfpathlineto{\pgfqpoint{1.341687in}{2.395291in}}%
\pgfpathlineto{\pgfqpoint{1.343108in}{2.366712in}}%
\pgfpathlineto{\pgfqpoint{1.343298in}{2.372694in}}%
\pgfpathlineto{\pgfqpoint{1.343392in}{2.376319in}}%
\pgfpathlineto{\pgfqpoint{1.343866in}{2.355438in}}%
\pgfpathlineto{\pgfqpoint{1.344245in}{2.366656in}}%
\pgfpathlineto{\pgfqpoint{1.345382in}{2.335270in}}%
\pgfpathlineto{\pgfqpoint{1.346140in}{2.347046in}}%
\pgfpathlineto{\pgfqpoint{1.346614in}{2.337339in}}%
\pgfpathlineto{\pgfqpoint{1.347372in}{2.344868in}}%
\pgfpathlineto{\pgfqpoint{1.347562in}{2.347829in}}%
\pgfpathlineto{\pgfqpoint{1.347846in}{2.331743in}}%
\pgfpathlineto{\pgfqpoint{1.348130in}{2.312370in}}%
\pgfpathlineto{\pgfqpoint{1.348889in}{2.332527in}}%
\pgfpathlineto{\pgfqpoint{1.350121in}{2.359758in}}%
\pgfpathlineto{\pgfqpoint{1.350405in}{2.350848in}}%
\pgfpathlineto{\pgfqpoint{1.350500in}{2.350287in}}%
\pgfpathlineto{\pgfqpoint{1.350594in}{2.354851in}}%
\pgfpathlineto{\pgfqpoint{1.350879in}{2.370967in}}%
\pgfpathlineto{\pgfqpoint{1.351258in}{2.338302in}}%
\pgfpathlineto{\pgfqpoint{1.351826in}{2.365443in}}%
\pgfpathlineto{\pgfqpoint{1.352016in}{2.371347in}}%
\pgfpathlineto{\pgfqpoint{1.352395in}{2.392126in}}%
\pgfpathlineto{\pgfqpoint{1.353058in}{2.376193in}}%
\pgfpathlineto{\pgfqpoint{1.353627in}{2.389532in}}%
\pgfpathlineto{\pgfqpoint{1.354574in}{2.358454in}}%
\pgfpathlineto{\pgfqpoint{1.354669in}{2.357637in}}%
\pgfpathlineto{\pgfqpoint{1.354764in}{2.361117in}}%
\pgfpathlineto{\pgfqpoint{1.355712in}{2.394541in}}%
\pgfpathlineto{\pgfqpoint{1.356185in}{2.385546in}}%
\pgfpathlineto{\pgfqpoint{1.357607in}{2.382200in}}%
\pgfpathlineto{\pgfqpoint{1.357796in}{2.384339in}}%
\pgfpathlineto{\pgfqpoint{1.358649in}{2.443120in}}%
\pgfpathlineto{\pgfqpoint{1.359502in}{2.416764in}}%
\pgfpathlineto{\pgfqpoint{1.360829in}{2.403395in}}%
\pgfpathlineto{\pgfqpoint{1.361113in}{2.406498in}}%
\pgfpathlineto{\pgfqpoint{1.361682in}{2.422576in}}%
\pgfpathlineto{\pgfqpoint{1.363198in}{2.688574in}}%
\pgfpathlineto{\pgfqpoint{1.364051in}{2.625717in}}%
\pgfpathlineto{\pgfqpoint{1.364525in}{2.499128in}}%
\pgfpathlineto{\pgfqpoint{1.364904in}{2.368673in}}%
\pgfpathlineto{\pgfqpoint{1.365757in}{2.441347in}}%
\pgfpathlineto{\pgfqpoint{1.365946in}{2.435111in}}%
\pgfpathlineto{\pgfqpoint{1.366420in}{2.454665in}}%
\pgfpathlineto{\pgfqpoint{1.367936in}{2.469783in}}%
\pgfpathlineto{\pgfqpoint{1.367083in}{2.452007in}}%
\pgfpathlineto{\pgfqpoint{1.368031in}{2.469593in}}%
\pgfpathlineto{\pgfqpoint{1.368410in}{2.454330in}}%
\pgfpathlineto{\pgfqpoint{1.368884in}{2.421730in}}%
\pgfpathlineto{\pgfqpoint{1.369737in}{2.427743in}}%
\pgfpathlineto{\pgfqpoint{1.370969in}{2.448266in}}%
\pgfpathlineto{\pgfqpoint{1.371158in}{2.441042in}}%
\pgfpathlineto{\pgfqpoint{1.372485in}{2.386933in}}%
\pgfpathlineto{\pgfqpoint{1.372864in}{2.403697in}}%
\pgfpathlineto{\pgfqpoint{1.373433in}{2.394008in}}%
\pgfpathlineto{\pgfqpoint{1.374191in}{2.412872in}}%
\pgfpathlineto{\pgfqpoint{1.374380in}{2.415893in}}%
\pgfpathlineto{\pgfqpoint{1.374664in}{2.400563in}}%
\pgfpathlineto{\pgfqpoint{1.375328in}{2.373143in}}%
\pgfpathlineto{\pgfqpoint{1.375707in}{2.395656in}}%
\pgfpathlineto{\pgfqpoint{1.375896in}{2.402541in}}%
\pgfpathlineto{\pgfqpoint{1.376654in}{2.390372in}}%
\pgfpathlineto{\pgfqpoint{1.376844in}{2.383794in}}%
\pgfpathlineto{\pgfqpoint{1.377223in}{2.401622in}}%
\pgfpathlineto{\pgfqpoint{1.377697in}{2.388519in}}%
\pgfpathlineto{\pgfqpoint{1.378929in}{2.408580in}}%
\pgfpathlineto{\pgfqpoint{1.379118in}{2.401999in}}%
\pgfpathlineto{\pgfqpoint{1.379308in}{2.395803in}}%
\pgfpathlineto{\pgfqpoint{1.379782in}{2.409164in}}%
\pgfpathlineto{\pgfqpoint{1.380066in}{2.404221in}}%
\pgfpathlineto{\pgfqpoint{1.380445in}{2.421838in}}%
\pgfpathlineto{\pgfqpoint{1.380919in}{2.397995in}}%
\pgfpathlineto{\pgfqpoint{1.381203in}{2.405486in}}%
\pgfpathlineto{\pgfqpoint{1.383762in}{2.445833in}}%
\pgfpathlineto{\pgfqpoint{1.384425in}{2.434906in}}%
\pgfpathlineto{\pgfqpoint{1.385088in}{2.443746in}}%
\pgfpathlineto{\pgfqpoint{1.386415in}{2.395521in}}%
\pgfpathlineto{\pgfqpoint{1.387268in}{2.378988in}}%
\pgfpathlineto{\pgfqpoint{1.387837in}{2.388269in}}%
\pgfpathlineto{\pgfqpoint{1.388121in}{2.403551in}}%
\pgfpathlineto{\pgfqpoint{1.390016in}{2.505622in}}%
\pgfpathlineto{\pgfqpoint{1.390395in}{2.495557in}}%
\pgfpathlineto{\pgfqpoint{1.391343in}{2.489870in}}%
\pgfpathlineto{\pgfqpoint{1.390964in}{2.498027in}}%
\pgfpathlineto{\pgfqpoint{1.391438in}{2.492510in}}%
\pgfpathlineto{\pgfqpoint{1.393143in}{2.541709in}}%
\pgfpathlineto{\pgfqpoint{1.393238in}{2.539559in}}%
\pgfpathlineto{\pgfqpoint{1.393428in}{2.533740in}}%
\pgfpathlineto{\pgfqpoint{1.394281in}{2.541507in}}%
\pgfpathlineto{\pgfqpoint{1.394375in}{2.543009in}}%
\pgfpathlineto{\pgfqpoint{1.394660in}{2.531656in}}%
\pgfpathlineto{\pgfqpoint{1.395039in}{2.521076in}}%
\pgfpathlineto{\pgfqpoint{1.395607in}{2.533790in}}%
\pgfpathlineto{\pgfqpoint{1.397218in}{2.575828in}}%
\pgfpathlineto{\pgfqpoint{1.398545in}{2.558588in}}%
\pgfpathlineto{\pgfqpoint{1.398640in}{2.560634in}}%
\pgfpathlineto{\pgfqpoint{1.399682in}{2.593192in}}%
\pgfpathlineto{\pgfqpoint{1.399872in}{2.579921in}}%
\pgfpathlineto{\pgfqpoint{1.401388in}{2.516483in}}%
\pgfpathlineto{\pgfqpoint{1.401577in}{2.523726in}}%
\pgfpathlineto{\pgfqpoint{1.402336in}{2.560027in}}%
\pgfpathlineto{\pgfqpoint{1.402904in}{2.539987in}}%
\pgfpathlineto{\pgfqpoint{1.404610in}{2.511611in}}%
\pgfpathlineto{\pgfqpoint{1.404705in}{2.513706in}}%
\pgfpathlineto{\pgfqpoint{1.405084in}{2.530280in}}%
\pgfpathlineto{\pgfqpoint{1.405842in}{2.523044in}}%
\pgfpathlineto{\pgfqpoint{1.407832in}{2.443197in}}%
\pgfpathlineto{\pgfqpoint{1.408306in}{2.453869in}}%
\pgfpathlineto{\pgfqpoint{1.409064in}{2.570327in}}%
\pgfpathlineto{\pgfqpoint{1.409632in}{2.677015in}}%
\pgfpathlineto{\pgfqpoint{1.410391in}{2.638525in}}%
\pgfpathlineto{\pgfqpoint{1.411149in}{2.499622in}}%
\pgfpathlineto{\pgfqpoint{1.411717in}{2.337562in}}%
\pgfpathlineto{\pgfqpoint{1.412475in}{2.401771in}}%
\pgfpathlineto{\pgfqpoint{1.412665in}{2.404491in}}%
\pgfpathlineto{\pgfqpoint{1.412949in}{2.390172in}}%
\pgfpathlineto{\pgfqpoint{1.413044in}{2.386108in}}%
\pgfpathlineto{\pgfqpoint{1.413518in}{2.399954in}}%
\pgfpathlineto{\pgfqpoint{1.413802in}{2.397219in}}%
\pgfpathlineto{\pgfqpoint{1.414939in}{2.417210in}}%
\pgfpathlineto{\pgfqpoint{1.415223in}{2.408763in}}%
\pgfpathlineto{\pgfqpoint{1.415603in}{2.377870in}}%
\pgfpathlineto{\pgfqpoint{1.416361in}{2.404217in}}%
\pgfpathlineto{\pgfqpoint{1.417782in}{2.432754in}}%
\pgfpathlineto{\pgfqpoint{1.417877in}{2.429976in}}%
\pgfpathlineto{\pgfqpoint{1.418825in}{2.394173in}}%
\pgfpathlineto{\pgfqpoint{1.419204in}{2.410575in}}%
\pgfpathlineto{\pgfqpoint{1.420246in}{2.436893in}}%
\pgfpathlineto{\pgfqpoint{1.420530in}{2.423812in}}%
\pgfpathlineto{\pgfqpoint{1.420625in}{2.422857in}}%
\pgfpathlineto{\pgfqpoint{1.420815in}{2.429622in}}%
\pgfpathlineto{\pgfqpoint{1.421194in}{2.443262in}}%
\pgfpathlineto{\pgfqpoint{1.421573in}{2.417848in}}%
\pgfpathlineto{\pgfqpoint{1.421762in}{2.409652in}}%
\pgfpathlineto{\pgfqpoint{1.422331in}{2.427353in}}%
\pgfpathlineto{\pgfqpoint{1.423847in}{2.461385in}}%
\pgfpathlineto{\pgfqpoint{1.424510in}{2.468163in}}%
\pgfpathlineto{\pgfqpoint{1.424889in}{2.461692in}}%
\pgfpathlineto{\pgfqpoint{1.424984in}{2.459823in}}%
\pgfpathlineto{\pgfqpoint{1.425268in}{2.473527in}}%
\pgfpathlineto{\pgfqpoint{1.426121in}{2.511480in}}%
\pgfpathlineto{\pgfqpoint{1.426785in}{2.511378in}}%
\pgfpathlineto{\pgfqpoint{1.427164in}{2.512916in}}%
\pgfpathlineto{\pgfqpoint{1.429343in}{2.542013in}}%
\pgfpathlineto{\pgfqpoint{1.429533in}{2.540956in}}%
\pgfpathlineto{\pgfqpoint{1.430196in}{2.529949in}}%
\pgfpathlineto{\pgfqpoint{1.430860in}{2.535865in}}%
\pgfpathlineto{\pgfqpoint{1.431144in}{2.545216in}}%
\pgfpathlineto{\pgfqpoint{1.431523in}{2.526728in}}%
\pgfpathlineto{\pgfqpoint{1.431902in}{2.533714in}}%
\pgfpathlineto{\pgfqpoint{1.433513in}{2.492647in}}%
\pgfpathlineto{\pgfqpoint{1.432565in}{2.535552in}}%
\pgfpathlineto{\pgfqpoint{1.433797in}{2.500410in}}%
\pgfpathlineto{\pgfqpoint{1.433892in}{2.502813in}}%
\pgfpathlineto{\pgfqpoint{1.434176in}{2.491321in}}%
\pgfpathlineto{\pgfqpoint{1.435693in}{2.448471in}}%
\pgfpathlineto{\pgfqpoint{1.437209in}{2.411601in}}%
\pgfpathlineto{\pgfqpoint{1.437588in}{2.420052in}}%
\pgfpathlineto{\pgfqpoint{1.437777in}{2.420572in}}%
\pgfpathlineto{\pgfqpoint{1.438156in}{2.419193in}}%
\pgfpathlineto{\pgfqpoint{1.438630in}{2.404418in}}%
\pgfpathlineto{\pgfqpoint{1.439009in}{2.421945in}}%
\pgfpathlineto{\pgfqpoint{1.439294in}{2.417049in}}%
\pgfpathlineto{\pgfqpoint{1.439673in}{2.429971in}}%
\pgfpathlineto{\pgfqpoint{1.440052in}{2.416487in}}%
\pgfpathlineto{\pgfqpoint{1.440336in}{2.418880in}}%
\pgfpathlineto{\pgfqpoint{1.440999in}{2.386193in}}%
\pgfpathlineto{\pgfqpoint{1.441852in}{2.392666in}}%
\pgfpathlineto{\pgfqpoint{1.442231in}{2.403182in}}%
\pgfpathlineto{\pgfqpoint{1.442800in}{2.386224in}}%
\pgfpathlineto{\pgfqpoint{1.443179in}{2.395198in}}%
\pgfpathlineto{\pgfqpoint{1.443558in}{2.380839in}}%
\pgfpathlineto{\pgfqpoint{1.443937in}{2.362909in}}%
\pgfpathlineto{\pgfqpoint{1.444221in}{2.337283in}}%
\pgfpathlineto{\pgfqpoint{1.445074in}{2.355363in}}%
\pgfpathlineto{\pgfqpoint{1.445169in}{2.355439in}}%
\pgfpathlineto{\pgfqpoint{1.445927in}{2.309039in}}%
\pgfpathlineto{\pgfqpoint{1.447917in}{2.206370in}}%
\pgfpathlineto{\pgfqpoint{1.448012in}{2.207024in}}%
\pgfpathlineto{\pgfqpoint{1.448391in}{2.214958in}}%
\pgfpathlineto{\pgfqpoint{1.448770in}{2.201156in}}%
\pgfpathlineto{\pgfqpoint{1.450286in}{2.161485in}}%
\pgfpathlineto{\pgfqpoint{1.450476in}{2.172858in}}%
\pgfpathlineto{\pgfqpoint{1.451423in}{2.223419in}}%
\pgfpathlineto{\pgfqpoint{1.451992in}{2.212877in}}%
\pgfpathlineto{\pgfqpoint{1.452276in}{2.199878in}}%
\pgfpathlineto{\pgfqpoint{1.452940in}{2.216177in}}%
\pgfpathlineto{\pgfqpoint{1.453034in}{2.217009in}}%
\pgfpathlineto{\pgfqpoint{1.453319in}{2.211292in}}%
\pgfpathlineto{\pgfqpoint{1.453603in}{2.204002in}}%
\pgfpathlineto{\pgfqpoint{1.454077in}{2.222781in}}%
\pgfpathlineto{\pgfqpoint{1.454930in}{2.321805in}}%
\pgfpathlineto{\pgfqpoint{1.455783in}{2.510988in}}%
\pgfpathlineto{\pgfqpoint{1.456541in}{2.465619in}}%
\pgfpathlineto{\pgfqpoint{1.457204in}{2.332669in}}%
\pgfpathlineto{\pgfqpoint{1.457773in}{2.174231in}}%
\pgfpathlineto{\pgfqpoint{1.458531in}{2.235869in}}%
\pgfpathlineto{\pgfqpoint{1.458720in}{2.242036in}}%
\pgfpathlineto{\pgfqpoint{1.461089in}{2.409513in}}%
\pgfpathlineto{\pgfqpoint{1.461468in}{2.400539in}}%
\pgfpathlineto{\pgfqpoint{1.461753in}{2.386455in}}%
\pgfpathlineto{\pgfqpoint{1.462226in}{2.415006in}}%
\pgfpathlineto{\pgfqpoint{1.464311in}{2.499981in}}%
\pgfpathlineto{\pgfqpoint{1.464501in}{2.494710in}}%
\pgfpathlineto{\pgfqpoint{1.464785in}{2.482329in}}%
\pgfpathlineto{\pgfqpoint{1.465259in}{2.513400in}}%
\pgfpathlineto{\pgfqpoint{1.467059in}{2.597385in}}%
\pgfpathlineto{\pgfqpoint{1.467533in}{2.591556in}}%
\pgfpathlineto{\pgfqpoint{1.467912in}{2.582830in}}%
\pgfpathlineto{\pgfqpoint{1.468291in}{2.600734in}}%
\pgfpathlineto{\pgfqpoint{1.469049in}{2.632574in}}%
\pgfpathlineto{\pgfqpoint{1.469902in}{2.626690in}}%
\pgfpathlineto{\pgfqpoint{1.471513in}{2.592236in}}%
\pgfpathlineto{\pgfqpoint{1.471703in}{2.594875in}}%
\pgfpathlineto{\pgfqpoint{1.471798in}{2.595143in}}%
\pgfpathlineto{\pgfqpoint{1.471892in}{2.593993in}}%
\pgfpathlineto{\pgfqpoint{1.474072in}{2.554586in}}%
\pgfpathlineto{\pgfqpoint{1.475020in}{2.568470in}}%
\pgfpathlineto{\pgfqpoint{1.475209in}{2.574481in}}%
\pgfpathlineto{\pgfqpoint{1.475683in}{2.552924in}}%
\pgfpathlineto{\pgfqpoint{1.475778in}{2.552561in}}%
\pgfpathlineto{\pgfqpoint{1.475873in}{2.555031in}}%
\pgfpathlineto{\pgfqpoint{1.476157in}{2.561821in}}%
\pgfpathlineto{\pgfqpoint{1.476631in}{2.543035in}}%
\pgfpathlineto{\pgfqpoint{1.478621in}{2.486929in}}%
\pgfpathlineto{\pgfqpoint{1.478715in}{2.486986in}}%
\pgfpathlineto{\pgfqpoint{1.478905in}{2.483885in}}%
\pgfpathlineto{\pgfqpoint{1.483927in}{2.182311in}}%
\pgfpathlineto{\pgfqpoint{1.484307in}{2.198256in}}%
\pgfpathlineto{\pgfqpoint{1.486202in}{2.173976in}}%
\pgfpathlineto{\pgfqpoint{1.487434in}{2.188348in}}%
\pgfpathlineto{\pgfqpoint{1.489045in}{2.228531in}}%
\pgfpathlineto{\pgfqpoint{1.489519in}{2.225207in}}%
\pgfpathlineto{\pgfqpoint{1.490087in}{2.223899in}}%
\pgfpathlineto{\pgfqpoint{1.490561in}{2.240298in}}%
\pgfpathlineto{\pgfqpoint{1.490940in}{2.261973in}}%
\pgfpathlineto{\pgfqpoint{1.491793in}{2.256482in}}%
\pgfpathlineto{\pgfqpoint{1.493404in}{2.218248in}}%
\pgfpathlineto{\pgfqpoint{1.492267in}{2.260415in}}%
\pgfpathlineto{\pgfqpoint{1.493783in}{2.229545in}}%
\pgfpathlineto{\pgfqpoint{1.494825in}{2.263058in}}%
\pgfpathlineto{\pgfqpoint{1.495868in}{2.251892in}}%
\pgfpathlineto{\pgfqpoint{1.496057in}{2.253573in}}%
\pgfpathlineto{\pgfqpoint{1.497005in}{2.301603in}}%
\pgfpathlineto{\pgfqpoint{1.500890in}{2.554847in}}%
\pgfpathlineto{\pgfqpoint{1.501743in}{2.774333in}}%
\pgfpathlineto{\pgfqpoint{1.502501in}{2.726209in}}%
\pgfpathlineto{\pgfqpoint{1.503354in}{2.540047in}}%
\pgfpathlineto{\pgfqpoint{1.503733in}{2.433053in}}%
\pgfpathlineto{\pgfqpoint{1.504491in}{2.520809in}}%
\pgfpathlineto{\pgfqpoint{1.504870in}{2.527584in}}%
\pgfpathlineto{\pgfqpoint{1.506671in}{2.627021in}}%
\pgfpathlineto{\pgfqpoint{1.507239in}{2.601501in}}%
\pgfpathlineto{\pgfqpoint{1.507429in}{2.604888in}}%
\pgfpathlineto{\pgfqpoint{1.507713in}{2.589411in}}%
\pgfpathlineto{\pgfqpoint{1.507808in}{2.585148in}}%
\pgfpathlineto{\pgfqpoint{1.508282in}{2.605885in}}%
\pgfpathlineto{\pgfqpoint{1.508566in}{2.598254in}}%
\pgfpathlineto{\pgfqpoint{1.508661in}{2.596896in}}%
\pgfpathlineto{\pgfqpoint{1.509040in}{2.605753in}}%
\pgfpathlineto{\pgfqpoint{1.509135in}{2.606178in}}%
\pgfpathlineto{\pgfqpoint{1.509324in}{2.602568in}}%
\pgfpathlineto{\pgfqpoint{1.509798in}{2.606153in}}%
\pgfpathlineto{\pgfqpoint{1.510840in}{2.545068in}}%
\pgfpathlineto{\pgfqpoint{1.510935in}{2.537898in}}%
\pgfpathlineto{\pgfqpoint{1.511409in}{2.577906in}}%
\pgfpathlineto{\pgfqpoint{1.511788in}{2.552638in}}%
\pgfpathlineto{\pgfqpoint{1.512641in}{2.569753in}}%
\pgfpathlineto{\pgfqpoint{1.513020in}{2.562475in}}%
\pgfpathlineto{\pgfqpoint{1.513494in}{2.553892in}}%
\pgfpathlineto{\pgfqpoint{1.513873in}{2.527331in}}%
\pgfpathlineto{\pgfqpoint{1.514726in}{2.541941in}}%
\pgfpathlineto{\pgfqpoint{1.514821in}{2.541079in}}%
\pgfpathlineto{\pgfqpoint{1.515010in}{2.545857in}}%
\pgfpathlineto{\pgfqpoint{1.515294in}{2.552357in}}%
\pgfpathlineto{\pgfqpoint{1.515958in}{2.541075in}}%
\pgfpathlineto{\pgfqpoint{1.518232in}{2.492466in}}%
\pgfpathlineto{\pgfqpoint{1.518422in}{2.497089in}}%
\pgfpathlineto{\pgfqpoint{1.518516in}{2.499068in}}%
\pgfpathlineto{\pgfqpoint{1.518706in}{2.488642in}}%
\pgfpathlineto{\pgfqpoint{1.520506in}{2.401465in}}%
\pgfpathlineto{\pgfqpoint{1.521359in}{2.381258in}}%
\pgfpathlineto{\pgfqpoint{1.524202in}{2.293854in}}%
\pgfpathlineto{\pgfqpoint{1.527708in}{2.225214in}}%
\pgfpathlineto{\pgfqpoint{1.527898in}{2.228089in}}%
\pgfpathlineto{\pgfqpoint{1.528277in}{2.243751in}}%
\pgfpathlineto{\pgfqpoint{1.529035in}{2.230783in}}%
\pgfpathlineto{\pgfqpoint{1.529319in}{2.227404in}}%
\pgfpathlineto{\pgfqpoint{1.529699in}{2.236715in}}%
\pgfpathlineto{\pgfqpoint{1.529983in}{2.231641in}}%
\pgfpathlineto{\pgfqpoint{1.530172in}{2.235527in}}%
\pgfpathlineto{\pgfqpoint{1.530551in}{2.251711in}}%
\pgfpathlineto{\pgfqpoint{1.531215in}{2.238634in}}%
\pgfpathlineto{\pgfqpoint{1.531594in}{2.222659in}}%
\pgfpathlineto{\pgfqpoint{1.532352in}{2.234048in}}%
\pgfpathlineto{\pgfqpoint{1.533015in}{2.249023in}}%
\pgfpathlineto{\pgfqpoint{1.534626in}{2.450994in}}%
\pgfpathlineto{\pgfqpoint{1.536806in}{2.626084in}}%
\pgfpathlineto{\pgfqpoint{1.537185in}{2.636967in}}%
\pgfpathlineto{\pgfqpoint{1.537469in}{2.649674in}}%
\pgfpathlineto{\pgfqpoint{1.538038in}{2.619570in}}%
\pgfpathlineto{\pgfqpoint{1.539080in}{2.569395in}}%
\pgfpathlineto{\pgfqpoint{1.539838in}{2.577848in}}%
\pgfpathlineto{\pgfqpoint{1.540407in}{2.582745in}}%
\pgfpathlineto{\pgfqpoint{1.540691in}{2.575010in}}%
\pgfpathlineto{\pgfqpoint{1.540881in}{2.571942in}}%
\pgfpathlineto{\pgfqpoint{1.541355in}{2.587931in}}%
\pgfpathlineto{\pgfqpoint{1.541828in}{2.578764in}}%
\pgfpathlineto{\pgfqpoint{1.542207in}{2.586809in}}%
\pgfpathlineto{\pgfqpoint{1.543250in}{2.640590in}}%
\pgfpathlineto{\pgfqpoint{1.544197in}{2.622422in}}%
\pgfpathlineto{\pgfqpoint{1.545240in}{2.593231in}}%
\pgfpathlineto{\pgfqpoint{1.545714in}{2.598571in}}%
\pgfpathlineto{\pgfqpoint{1.546093in}{2.587216in}}%
\pgfpathlineto{\pgfqpoint{1.546377in}{2.602758in}}%
\pgfpathlineto{\pgfqpoint{1.547419in}{2.812064in}}%
\pgfpathlineto{\pgfqpoint{1.548367in}{2.742782in}}%
\pgfpathlineto{\pgfqpoint{1.548746in}{2.696369in}}%
\pgfpathlineto{\pgfqpoint{1.549504in}{2.429344in}}%
\pgfpathlineto{\pgfqpoint{1.550452in}{2.474805in}}%
\pgfpathlineto{\pgfqpoint{1.551115in}{2.475166in}}%
\pgfpathlineto{\pgfqpoint{1.551684in}{2.463800in}}%
\pgfpathlineto{\pgfqpoint{1.551968in}{2.468205in}}%
\pgfpathlineto{\pgfqpoint{1.552158in}{2.472157in}}%
\pgfpathlineto{\pgfqpoint{1.552537in}{2.454284in}}%
\pgfpathlineto{\pgfqpoint{1.556801in}{2.298651in}}%
\pgfpathlineto{\pgfqpoint{1.557085in}{2.310437in}}%
\pgfpathlineto{\pgfqpoint{1.557180in}{2.312383in}}%
\pgfpathlineto{\pgfqpoint{1.557654in}{2.299828in}}%
\pgfpathlineto{\pgfqpoint{1.557843in}{2.299533in}}%
\pgfpathlineto{\pgfqpoint{1.559265in}{2.254521in}}%
\pgfpathlineto{\pgfqpoint{1.559549in}{2.234110in}}%
\pgfpathlineto{\pgfqpoint{1.560402in}{2.242162in}}%
\pgfpathlineto{\pgfqpoint{1.561350in}{2.254944in}}%
\pgfpathlineto{\pgfqpoint{1.561539in}{2.250198in}}%
\pgfpathlineto{\pgfqpoint{1.562582in}{2.233133in}}%
\pgfpathlineto{\pgfqpoint{1.562013in}{2.253449in}}%
\pgfpathlineto{\pgfqpoint{1.562771in}{2.241880in}}%
\pgfpathlineto{\pgfqpoint{1.563340in}{2.238111in}}%
\pgfpathlineto{\pgfqpoint{1.564382in}{2.267880in}}%
\pgfpathlineto{\pgfqpoint{1.564477in}{2.268483in}}%
\pgfpathlineto{\pgfqpoint{1.564572in}{2.266060in}}%
\pgfpathlineto{\pgfqpoint{1.565519in}{2.252008in}}%
\pgfpathlineto{\pgfqpoint{1.565804in}{2.258492in}}%
\pgfpathlineto{\pgfqpoint{1.568078in}{2.313918in}}%
\pgfpathlineto{\pgfqpoint{1.568457in}{2.301122in}}%
\pgfpathlineto{\pgfqpoint{1.569310in}{2.309793in}}%
\pgfpathlineto{\pgfqpoint{1.571110in}{2.350888in}}%
\pgfpathlineto{\pgfqpoint{1.571489in}{2.350583in}}%
\pgfpathlineto{\pgfqpoint{1.571679in}{2.357025in}}%
\pgfpathlineto{\pgfqpoint{1.575280in}{2.540090in}}%
\pgfpathlineto{\pgfqpoint{1.575754in}{2.553731in}}%
\pgfpathlineto{\pgfqpoint{1.576038in}{2.570480in}}%
\pgfpathlineto{\pgfqpoint{1.576891in}{2.560748in}}%
\pgfpathlineto{\pgfqpoint{1.578313in}{2.530384in}}%
\pgfpathlineto{\pgfqpoint{1.578976in}{2.531173in}}%
\pgfpathlineto{\pgfqpoint{1.580492in}{2.585505in}}%
\pgfpathlineto{\pgfqpoint{1.580871in}{2.564107in}}%
\pgfpathlineto{\pgfqpoint{1.581061in}{2.554989in}}%
\pgfpathlineto{\pgfqpoint{1.581724in}{2.579129in}}%
\pgfpathlineto{\pgfqpoint{1.582198in}{2.598572in}}%
\pgfpathlineto{\pgfqpoint{1.582672in}{2.575634in}}%
\pgfpathlineto{\pgfqpoint{1.583809in}{2.530826in}}%
\pgfpathlineto{\pgfqpoint{1.584851in}{2.463678in}}%
\pgfpathlineto{\pgfqpoint{1.585420in}{2.471876in}}%
\pgfpathlineto{\pgfqpoint{1.587694in}{2.389068in}}%
\pgfpathlineto{\pgfqpoint{1.588263in}{2.400791in}}%
\pgfpathlineto{\pgfqpoint{1.588642in}{2.420537in}}%
\pgfpathlineto{\pgfqpoint{1.589021in}{2.396586in}}%
\pgfpathlineto{\pgfqpoint{1.590727in}{2.344990in}}%
\pgfpathlineto{\pgfqpoint{1.590916in}{2.349404in}}%
\pgfpathlineto{\pgfqpoint{1.591579in}{2.334933in}}%
\pgfpathlineto{\pgfqpoint{1.592338in}{2.486765in}}%
\pgfpathlineto{\pgfqpoint{1.592906in}{2.580530in}}%
\pgfpathlineto{\pgfqpoint{1.593664in}{2.542788in}}%
\pgfpathlineto{\pgfqpoint{1.594422in}{2.386423in}}%
\pgfpathlineto{\pgfqpoint{1.594991in}{2.226952in}}%
\pgfpathlineto{\pgfqpoint{1.595749in}{2.268365in}}%
\pgfpathlineto{\pgfqpoint{1.596507in}{2.253721in}}%
\pgfpathlineto{\pgfqpoint{1.597171in}{2.258796in}}%
\pgfpathlineto{\pgfqpoint{1.597550in}{2.276540in}}%
\pgfpathlineto{\pgfqpoint{1.598308in}{2.264377in}}%
\pgfpathlineto{\pgfqpoint{1.598592in}{2.246791in}}%
\pgfpathlineto{\pgfqpoint{1.599445in}{2.256547in}}%
\pgfpathlineto{\pgfqpoint{1.600677in}{2.291031in}}%
\pgfpathlineto{\pgfqpoint{1.600961in}{2.301082in}}%
\pgfpathlineto{\pgfqpoint{1.601530in}{2.282806in}}%
\pgfpathlineto{\pgfqpoint{1.601909in}{2.272794in}}%
\pgfpathlineto{\pgfqpoint{1.602383in}{2.290555in}}%
\pgfpathlineto{\pgfqpoint{1.603899in}{2.338638in}}%
\pgfpathlineto{\pgfqpoint{1.603994in}{2.336553in}}%
\pgfpathlineto{\pgfqpoint{1.605415in}{2.289734in}}%
\pgfpathlineto{\pgfqpoint{1.605699in}{2.314770in}}%
\pgfpathlineto{\pgfqpoint{1.609016in}{2.520258in}}%
\pgfpathlineto{\pgfqpoint{1.609395in}{2.539462in}}%
\pgfpathlineto{\pgfqpoint{1.610438in}{2.582139in}}%
\pgfpathlineto{\pgfqpoint{1.610817in}{2.570646in}}%
\pgfpathlineto{\pgfqpoint{1.611101in}{2.571565in}}%
\pgfpathlineto{\pgfqpoint{1.611196in}{2.569124in}}%
\pgfpathlineto{\pgfqpoint{1.611575in}{2.556253in}}%
\pgfpathlineto{\pgfqpoint{1.612428in}{2.560408in}}%
\pgfpathlineto{\pgfqpoint{1.613091in}{2.565919in}}%
\pgfpathlineto{\pgfqpoint{1.613375in}{2.561497in}}%
\pgfpathlineto{\pgfqpoint{1.613754in}{2.544418in}}%
\pgfpathlineto{\pgfqpoint{1.614418in}{2.559740in}}%
\pgfpathlineto{\pgfqpoint{1.615365in}{2.577181in}}%
\pgfpathlineto{\pgfqpoint{1.615650in}{2.565242in}}%
\pgfpathlineto{\pgfqpoint{1.617071in}{2.539624in}}%
\pgfpathlineto{\pgfqpoint{1.617261in}{2.546290in}}%
\pgfpathlineto{\pgfqpoint{1.617355in}{2.549859in}}%
\pgfpathlineto{\pgfqpoint{1.617829in}{2.526332in}}%
\pgfpathlineto{\pgfqpoint{1.618208in}{2.538999in}}%
\pgfpathlineto{\pgfqpoint{1.618492in}{2.528190in}}%
\pgfpathlineto{\pgfqpoint{1.619630in}{2.500714in}}%
\pgfpathlineto{\pgfqpoint{1.619819in}{2.505656in}}%
\pgfpathlineto{\pgfqpoint{1.619914in}{2.506569in}}%
\pgfpathlineto{\pgfqpoint{1.620103in}{2.498965in}}%
\pgfpathlineto{\pgfqpoint{1.620388in}{2.484878in}}%
\pgfpathlineto{\pgfqpoint{1.621335in}{2.489096in}}%
\pgfpathlineto{\pgfqpoint{1.623420in}{2.456363in}}%
\pgfpathlineto{\pgfqpoint{1.623799in}{2.459630in}}%
\pgfpathlineto{\pgfqpoint{1.623989in}{2.457091in}}%
\pgfpathlineto{\pgfqpoint{1.624273in}{2.468971in}}%
\pgfpathlineto{\pgfqpoint{1.624557in}{2.481234in}}%
\pgfpathlineto{\pgfqpoint{1.625410in}{2.476610in}}%
\pgfpathlineto{\pgfqpoint{1.625789in}{2.479072in}}%
\pgfpathlineto{\pgfqpoint{1.625979in}{2.470249in}}%
\pgfpathlineto{\pgfqpoint{1.626358in}{2.446067in}}%
\pgfpathlineto{\pgfqpoint{1.627116in}{2.464775in}}%
\pgfpathlineto{\pgfqpoint{1.627400in}{2.477455in}}%
\pgfpathlineto{\pgfqpoint{1.628064in}{2.464158in}}%
\pgfpathlineto{\pgfqpoint{1.629864in}{2.381884in}}%
\pgfpathlineto{\pgfqpoint{1.630622in}{2.386918in}}%
\pgfpathlineto{\pgfqpoint{1.630907in}{2.399027in}}%
\pgfpathlineto{\pgfqpoint{1.631475in}{2.374046in}}%
\pgfpathlineto{\pgfqpoint{1.632897in}{2.348899in}}%
\pgfpathlineto{\pgfqpoint{1.633086in}{2.351031in}}%
\pgfpathlineto{\pgfqpoint{1.634034in}{2.391703in}}%
\pgfpathlineto{\pgfqpoint{1.634697in}{2.365958in}}%
\pgfpathlineto{\pgfqpoint{1.635740in}{2.354839in}}%
\pgfpathlineto{\pgfqpoint{1.635929in}{2.355524in}}%
\pgfpathlineto{\pgfqpoint{1.636971in}{2.373642in}}%
\pgfpathlineto{\pgfqpoint{1.638488in}{2.643565in}}%
\pgfpathlineto{\pgfqpoint{1.639435in}{2.572730in}}%
\pgfpathlineto{\pgfqpoint{1.640193in}{2.335691in}}%
\pgfpathlineto{\pgfqpoint{1.641236in}{2.417068in}}%
\pgfpathlineto{\pgfqpoint{1.641710in}{2.445105in}}%
\pgfpathlineto{\pgfqpoint{1.643321in}{2.489566in}}%
\pgfpathlineto{\pgfqpoint{1.642184in}{2.444103in}}%
\pgfpathlineto{\pgfqpoint{1.643605in}{2.483737in}}%
\pgfpathlineto{\pgfqpoint{1.644174in}{2.456925in}}%
\pgfpathlineto{\pgfqpoint{1.644837in}{2.467897in}}%
\pgfpathlineto{\pgfqpoint{1.646448in}{2.519799in}}%
\pgfpathlineto{\pgfqpoint{1.646637in}{2.511814in}}%
\pgfpathlineto{\pgfqpoint{1.647396in}{2.488236in}}%
\pgfpathlineto{\pgfqpoint{1.647775in}{2.507370in}}%
\pgfpathlineto{\pgfqpoint{1.649101in}{2.524676in}}%
\pgfpathlineto{\pgfqpoint{1.649386in}{2.525551in}}%
\pgfpathlineto{\pgfqpoint{1.649575in}{2.526208in}}%
\pgfpathlineto{\pgfqpoint{1.649670in}{2.524679in}}%
\pgfpathlineto{\pgfqpoint{1.650238in}{2.500407in}}%
\pgfpathlineto{\pgfqpoint{1.650807in}{2.517546in}}%
\pgfpathlineto{\pgfqpoint{1.651755in}{2.546660in}}%
\pgfpathlineto{\pgfqpoint{1.652323in}{2.537513in}}%
\pgfpathlineto{\pgfqpoint{1.653366in}{2.519464in}}%
\pgfpathlineto{\pgfqpoint{1.653745in}{2.530254in}}%
\pgfpathlineto{\pgfqpoint{1.655356in}{2.556405in}}%
\pgfpathlineto{\pgfqpoint{1.658104in}{2.512592in}}%
\pgfpathlineto{\pgfqpoint{1.658388in}{2.517850in}}%
\pgfpathlineto{\pgfqpoint{1.658672in}{2.523216in}}%
\pgfpathlineto{\pgfqpoint{1.659052in}{2.510251in}}%
\pgfpathlineto{\pgfqpoint{1.659904in}{2.499852in}}%
\pgfpathlineto{\pgfqpoint{1.659525in}{2.513692in}}%
\pgfpathlineto{\pgfqpoint{1.660189in}{2.508085in}}%
\pgfpathlineto{\pgfqpoint{1.660378in}{2.510267in}}%
\pgfpathlineto{\pgfqpoint{1.660852in}{2.503336in}}%
\pgfpathlineto{\pgfqpoint{1.661515in}{2.491023in}}%
\pgfpathlineto{\pgfqpoint{1.661989in}{2.502186in}}%
\pgfpathlineto{\pgfqpoint{1.664358in}{2.456806in}}%
\pgfpathlineto{\pgfqpoint{1.665116in}{2.462817in}}%
\pgfpathlineto{\pgfqpoint{1.665401in}{2.468577in}}%
\pgfpathlineto{\pgfqpoint{1.665685in}{2.459918in}}%
\pgfpathlineto{\pgfqpoint{1.665875in}{2.453958in}}%
\pgfpathlineto{\pgfqpoint{1.666348in}{2.467956in}}%
\pgfpathlineto{\pgfqpoint{1.666822in}{2.458609in}}%
\pgfpathlineto{\pgfqpoint{1.667770in}{2.469502in}}%
\pgfpathlineto{\pgfqpoint{1.667959in}{2.462092in}}%
\pgfpathlineto{\pgfqpoint{1.668812in}{2.453217in}}%
\pgfpathlineto{\pgfqpoint{1.669097in}{2.457429in}}%
\pgfpathlineto{\pgfqpoint{1.671181in}{2.487825in}}%
\pgfpathlineto{\pgfqpoint{1.671276in}{2.485860in}}%
\pgfpathlineto{\pgfqpoint{1.671939in}{2.479280in}}%
\pgfpathlineto{\pgfqpoint{1.672129in}{2.483582in}}%
\pgfpathlineto{\pgfqpoint{1.672603in}{2.523519in}}%
\pgfpathlineto{\pgfqpoint{1.673361in}{2.500794in}}%
\pgfpathlineto{\pgfqpoint{1.674972in}{2.434505in}}%
\pgfpathlineto{\pgfqpoint{1.675730in}{2.452037in}}%
\pgfpathlineto{\pgfqpoint{1.676014in}{2.458411in}}%
\pgfpathlineto{\pgfqpoint{1.676488in}{2.447003in}}%
\pgfpathlineto{\pgfqpoint{1.676962in}{2.427136in}}%
\pgfpathlineto{\pgfqpoint{1.678004in}{2.434026in}}%
\pgfpathlineto{\pgfqpoint{1.678762in}{2.445003in}}%
\pgfpathlineto{\pgfqpoint{1.679142in}{2.459782in}}%
\pgfpathlineto{\pgfqpoint{1.679426in}{2.438298in}}%
\pgfpathlineto{\pgfqpoint{1.680184in}{2.389808in}}%
\pgfpathlineto{\pgfqpoint{1.680942in}{2.392888in}}%
\pgfpathlineto{\pgfqpoint{1.681700in}{2.435568in}}%
\pgfpathlineto{\pgfqpoint{1.682553in}{2.595418in}}%
\pgfpathlineto{\pgfqpoint{1.683501in}{2.749872in}}%
\pgfpathlineto{\pgfqpoint{1.683974in}{2.707971in}}%
\pgfpathlineto{\pgfqpoint{1.684733in}{2.572473in}}%
\pgfpathlineto{\pgfqpoint{1.685206in}{2.420900in}}%
\pgfpathlineto{\pgfqpoint{1.685965in}{2.486803in}}%
\pgfpathlineto{\pgfqpoint{1.686249in}{2.482943in}}%
\pgfpathlineto{\pgfqpoint{1.686533in}{2.490765in}}%
\pgfpathlineto{\pgfqpoint{1.688049in}{2.535876in}}%
\pgfpathlineto{\pgfqpoint{1.688997in}{2.520574in}}%
\pgfpathlineto{\pgfqpoint{1.689376in}{2.506441in}}%
\pgfpathlineto{\pgfqpoint{1.689850in}{2.529355in}}%
\pgfpathlineto{\pgfqpoint{1.690703in}{2.538523in}}%
\pgfpathlineto{\pgfqpoint{1.690229in}{2.520536in}}%
\pgfpathlineto{\pgfqpoint{1.690987in}{2.531285in}}%
\pgfpathlineto{\pgfqpoint{1.691082in}{2.531074in}}%
\pgfpathlineto{\pgfqpoint{1.691177in}{2.533101in}}%
\pgfpathlineto{\pgfqpoint{1.691745in}{2.541753in}}%
\pgfpathlineto{\pgfqpoint{1.691935in}{2.532130in}}%
\pgfpathlineto{\pgfqpoint{1.692219in}{2.510134in}}%
\pgfpathlineto{\pgfqpoint{1.692977in}{2.535692in}}%
\pgfpathlineto{\pgfqpoint{1.693640in}{2.562540in}}%
\pgfpathlineto{\pgfqpoint{1.694683in}{2.561536in}}%
\pgfpathlineto{\pgfqpoint{1.694872in}{2.563447in}}%
\pgfpathlineto{\pgfqpoint{1.695157in}{2.554736in}}%
\pgfpathlineto{\pgfqpoint{1.695725in}{2.546399in}}%
\pgfpathlineto{\pgfqpoint{1.696010in}{2.559231in}}%
\pgfpathlineto{\pgfqpoint{1.696768in}{2.576175in}}%
\pgfpathlineto{\pgfqpoint{1.697241in}{2.569970in}}%
\pgfpathlineto{\pgfqpoint{1.697715in}{2.578944in}}%
\pgfpathlineto{\pgfqpoint{1.698094in}{2.571461in}}%
\pgfpathlineto{\pgfqpoint{1.698473in}{2.562084in}}%
\pgfpathlineto{\pgfqpoint{1.699232in}{2.566867in}}%
\pgfpathlineto{\pgfqpoint{1.700179in}{2.574424in}}%
\pgfpathlineto{\pgfqpoint{1.700274in}{2.571656in}}%
\pgfpathlineto{\pgfqpoint{1.701506in}{2.547719in}}%
\pgfpathlineto{\pgfqpoint{1.701032in}{2.576451in}}%
\pgfpathlineto{\pgfqpoint{1.701695in}{2.550151in}}%
\pgfpathlineto{\pgfqpoint{1.701885in}{2.548208in}}%
\pgfpathlineto{\pgfqpoint{1.701980in}{2.547544in}}%
\pgfpathlineto{\pgfqpoint{1.702169in}{2.553778in}}%
\pgfpathlineto{\pgfqpoint{1.702453in}{2.562791in}}%
\pgfpathlineto{\pgfqpoint{1.703212in}{2.556816in}}%
\pgfpathlineto{\pgfqpoint{1.704538in}{2.549738in}}%
\pgfpathlineto{\pgfqpoint{1.704728in}{2.548598in}}%
\pgfpathlineto{\pgfqpoint{1.705012in}{2.554174in}}%
\pgfpathlineto{\pgfqpoint{1.705296in}{2.564193in}}%
\pgfpathlineto{\pgfqpoint{1.705770in}{2.540650in}}%
\pgfpathlineto{\pgfqpoint{1.706149in}{2.546209in}}%
\pgfpathlineto{\pgfqpoint{1.706339in}{2.541876in}}%
\pgfpathlineto{\pgfqpoint{1.708045in}{2.512709in}}%
\pgfpathlineto{\pgfqpoint{1.708234in}{2.513497in}}%
\pgfpathlineto{\pgfqpoint{1.708329in}{2.512124in}}%
\pgfpathlineto{\pgfqpoint{1.710414in}{2.477084in}}%
\pgfpathlineto{\pgfqpoint{1.710508in}{2.477018in}}%
\pgfpathlineto{\pgfqpoint{1.712309in}{2.508310in}}%
\pgfpathlineto{\pgfqpoint{1.712593in}{2.504535in}}%
\pgfpathlineto{\pgfqpoint{1.713162in}{2.503765in}}%
\pgfpathlineto{\pgfqpoint{1.713351in}{2.505259in}}%
\pgfpathlineto{\pgfqpoint{1.716858in}{2.574013in}}%
\pgfpathlineto{\pgfqpoint{1.717237in}{2.589514in}}%
\pgfpathlineto{\pgfqpoint{1.718090in}{2.583228in}}%
\pgfpathlineto{\pgfqpoint{1.720364in}{2.506672in}}%
\pgfpathlineto{\pgfqpoint{1.720553in}{2.514950in}}%
\pgfpathlineto{\pgfqpoint{1.720743in}{2.520763in}}%
\pgfpathlineto{\pgfqpoint{1.721217in}{2.499280in}}%
\pgfpathlineto{\pgfqpoint{1.721312in}{2.500052in}}%
\pgfpathlineto{\pgfqpoint{1.721596in}{2.487524in}}%
\pgfpathlineto{\pgfqpoint{1.723017in}{2.459650in}}%
\pgfpathlineto{\pgfqpoint{1.723112in}{2.460851in}}%
\pgfpathlineto{\pgfqpoint{1.723870in}{2.498377in}}%
\pgfpathlineto{\pgfqpoint{1.724344in}{2.475512in}}%
\pgfpathlineto{\pgfqpoint{1.725955in}{2.430547in}}%
\pgfpathlineto{\pgfqpoint{1.726713in}{2.429331in}}%
\pgfpathlineto{\pgfqpoint{1.727092in}{2.470970in}}%
\pgfpathlineto{\pgfqpoint{1.728419in}{2.689987in}}%
\pgfpathlineto{\pgfqpoint{1.729082in}{2.643828in}}%
\pgfpathlineto{\pgfqpoint{1.729840in}{2.420868in}}%
\pgfpathlineto{\pgfqpoint{1.730125in}{2.363819in}}%
\pgfpathlineto{\pgfqpoint{1.730788in}{2.444543in}}%
\pgfpathlineto{\pgfqpoint{1.731072in}{2.438977in}}%
\pgfpathlineto{\pgfqpoint{1.731925in}{2.441399in}}%
\pgfpathlineto{\pgfqpoint{1.732683in}{2.466851in}}%
\pgfpathlineto{\pgfqpoint{1.732968in}{2.487319in}}%
\pgfpathlineto{\pgfqpoint{1.733536in}{2.449404in}}%
\pgfpathlineto{\pgfqpoint{1.733631in}{2.450049in}}%
\pgfpathlineto{\pgfqpoint{1.734105in}{2.430523in}}%
\pgfpathlineto{\pgfqpoint{1.735337in}{2.440159in}}%
\pgfpathlineto{\pgfqpoint{1.736190in}{2.458696in}}%
\pgfpathlineto{\pgfqpoint{1.736474in}{2.447749in}}%
\pgfpathlineto{\pgfqpoint{1.737327in}{2.432543in}}%
\pgfpathlineto{\pgfqpoint{1.737706in}{2.438909in}}%
\pgfpathlineto{\pgfqpoint{1.738938in}{2.480382in}}%
\pgfpathlineto{\pgfqpoint{1.739411in}{2.476105in}}%
\pgfpathlineto{\pgfqpoint{1.739791in}{2.456813in}}%
\pgfpathlineto{\pgfqpoint{1.740075in}{2.449222in}}%
\pgfpathlineto{\pgfqpoint{1.740549in}{2.464773in}}%
\pgfpathlineto{\pgfqpoint{1.741970in}{2.493266in}}%
\pgfpathlineto{\pgfqpoint{1.742633in}{2.488268in}}%
\pgfpathlineto{\pgfqpoint{1.743107in}{2.496165in}}%
\pgfpathlineto{\pgfqpoint{1.743581in}{2.479133in}}%
\pgfpathlineto{\pgfqpoint{1.744150in}{2.493520in}}%
\pgfpathlineto{\pgfqpoint{1.744434in}{2.506894in}}%
\pgfpathlineto{\pgfqpoint{1.744813in}{2.487826in}}%
\pgfpathlineto{\pgfqpoint{1.745192in}{2.496525in}}%
\pgfpathlineto{\pgfqpoint{1.746329in}{2.480575in}}%
\pgfpathlineto{\pgfqpoint{1.746614in}{2.490052in}}%
\pgfpathlineto{\pgfqpoint{1.746803in}{2.493302in}}%
\pgfpathlineto{\pgfqpoint{1.747561in}{2.487559in}}%
\pgfpathlineto{\pgfqpoint{1.747846in}{2.480185in}}%
\pgfpathlineto{\pgfqpoint{1.748698in}{2.483982in}}%
\pgfpathlineto{\pgfqpoint{1.752678in}{2.419601in}}%
\pgfpathlineto{\pgfqpoint{1.755237in}{2.299250in}}%
\pgfpathlineto{\pgfqpoint{1.755332in}{2.304287in}}%
\pgfpathlineto{\pgfqpoint{1.756659in}{2.387988in}}%
\pgfpathlineto{\pgfqpoint{1.757227in}{2.385232in}}%
\pgfpathlineto{\pgfqpoint{1.758459in}{2.368493in}}%
\pgfpathlineto{\pgfqpoint{1.758838in}{2.379752in}}%
\pgfpathlineto{\pgfqpoint{1.760260in}{2.407132in}}%
\pgfpathlineto{\pgfqpoint{1.760733in}{2.403314in}}%
\pgfpathlineto{\pgfqpoint{1.761018in}{2.404293in}}%
\pgfpathlineto{\pgfqpoint{1.762250in}{2.428693in}}%
\pgfpathlineto{\pgfqpoint{1.763008in}{2.416749in}}%
\pgfpathlineto{\pgfqpoint{1.764524in}{2.362077in}}%
\pgfpathlineto{\pgfqpoint{1.765187in}{2.384435in}}%
\pgfpathlineto{\pgfqpoint{1.765377in}{2.387527in}}%
\pgfpathlineto{\pgfqpoint{1.765756in}{2.381992in}}%
\pgfpathlineto{\pgfqpoint{1.766040in}{2.382693in}}%
\pgfpathlineto{\pgfqpoint{1.766419in}{2.358611in}}%
\pgfpathlineto{\pgfqpoint{1.767272in}{2.374058in}}%
\pgfpathlineto{\pgfqpoint{1.767651in}{2.387537in}}%
\pgfpathlineto{\pgfqpoint{1.768599in}{2.451723in}}%
\pgfpathlineto{\pgfqpoint{1.769357in}{2.436397in}}%
\pgfpathlineto{\pgfqpoint{1.769641in}{2.422950in}}%
\pgfpathlineto{\pgfqpoint{1.770115in}{2.445653in}}%
\pgfpathlineto{\pgfqpoint{1.770305in}{2.441313in}}%
\pgfpathlineto{\pgfqpoint{1.770494in}{2.444734in}}%
\pgfpathlineto{\pgfqpoint{1.771157in}{2.432343in}}%
\pgfpathlineto{\pgfqpoint{1.771821in}{2.528506in}}%
\pgfpathlineto{\pgfqpoint{1.772958in}{2.727240in}}%
\pgfpathlineto{\pgfqpoint{1.773527in}{2.688032in}}%
\pgfpathlineto{\pgfqpoint{1.774095in}{2.587475in}}%
\pgfpathlineto{\pgfqpoint{1.774759in}{2.393383in}}%
\pgfpathlineto{\pgfqpoint{1.775517in}{2.464249in}}%
\pgfpathlineto{\pgfqpoint{1.775801in}{2.457279in}}%
\pgfpathlineto{\pgfqpoint{1.776275in}{2.471596in}}%
\pgfpathlineto{\pgfqpoint{1.776749in}{2.475556in}}%
\pgfpathlineto{\pgfqpoint{1.777222in}{2.469447in}}%
\pgfpathlineto{\pgfqpoint{1.777507in}{2.472020in}}%
\pgfpathlineto{\pgfqpoint{1.777791in}{2.464865in}}%
\pgfpathlineto{\pgfqpoint{1.779591in}{2.422747in}}%
\pgfpathlineto{\pgfqpoint{1.779971in}{2.429147in}}%
\pgfpathlineto{\pgfqpoint{1.780539in}{2.444093in}}%
\pgfpathlineto{\pgfqpoint{1.780918in}{2.429134in}}%
\pgfpathlineto{\pgfqpoint{1.781866in}{2.382653in}}%
\pgfpathlineto{\pgfqpoint{1.782624in}{2.413221in}}%
\pgfpathlineto{\pgfqpoint{1.782908in}{2.425990in}}%
\pgfpathlineto{\pgfqpoint{1.783382in}{2.407897in}}%
\pgfpathlineto{\pgfqpoint{1.783951in}{2.424477in}}%
\pgfpathlineto{\pgfqpoint{1.785088in}{2.383207in}}%
\pgfpathlineto{\pgfqpoint{1.785941in}{2.403638in}}%
\pgfpathlineto{\pgfqpoint{1.786130in}{2.414270in}}%
\pgfpathlineto{\pgfqpoint{1.786888in}{2.400511in}}%
\pgfpathlineto{\pgfqpoint{1.787836in}{2.391639in}}%
\pgfpathlineto{\pgfqpoint{1.787362in}{2.411718in}}%
\pgfpathlineto{\pgfqpoint{1.788025in}{2.397219in}}%
\pgfpathlineto{\pgfqpoint{1.789068in}{2.409564in}}%
\pgfpathlineto{\pgfqpoint{1.788594in}{2.395861in}}%
\pgfpathlineto{\pgfqpoint{1.789257in}{2.407016in}}%
\pgfpathlineto{\pgfqpoint{1.789921in}{2.407249in}}%
\pgfpathlineto{\pgfqpoint{1.790584in}{2.399982in}}%
\pgfpathlineto{\pgfqpoint{1.790774in}{2.400323in}}%
\pgfpathlineto{\pgfqpoint{1.790963in}{2.401491in}}%
\pgfpathlineto{\pgfqpoint{1.791153in}{2.397166in}}%
\pgfpathlineto{\pgfqpoint{1.791437in}{2.385574in}}%
\pgfpathlineto{\pgfqpoint{1.791911in}{2.402910in}}%
\pgfpathlineto{\pgfqpoint{1.792195in}{2.396899in}}%
\pgfpathlineto{\pgfqpoint{1.792385in}{2.401829in}}%
\pgfpathlineto{\pgfqpoint{1.792764in}{2.423300in}}%
\pgfpathlineto{\pgfqpoint{1.793617in}{2.417172in}}%
\pgfpathlineto{\pgfqpoint{1.793901in}{2.418195in}}%
\pgfpathlineto{\pgfqpoint{1.793996in}{2.418984in}}%
\pgfpathlineto{\pgfqpoint{1.794280in}{2.413005in}}%
\pgfpathlineto{\pgfqpoint{1.795417in}{2.395018in}}%
\pgfpathlineto{\pgfqpoint{1.794849in}{2.414761in}}%
\pgfpathlineto{\pgfqpoint{1.795607in}{2.400024in}}%
\pgfpathlineto{\pgfqpoint{1.795796in}{2.404050in}}%
\pgfpathlineto{\pgfqpoint{1.796175in}{2.391141in}}%
\pgfpathlineto{\pgfqpoint{1.796459in}{2.393173in}}%
\pgfpathlineto{\pgfqpoint{1.799018in}{2.352029in}}%
\pgfpathlineto{\pgfqpoint{1.799492in}{2.353303in}}%
\pgfpathlineto{\pgfqpoint{1.799776in}{2.352446in}}%
\pgfpathlineto{\pgfqpoint{1.800061in}{2.355146in}}%
\pgfpathlineto{\pgfqpoint{1.800440in}{2.365797in}}%
\pgfpathlineto{\pgfqpoint{1.801198in}{2.362534in}}%
\pgfpathlineto{\pgfqpoint{1.802240in}{2.339130in}}%
\pgfpathlineto{\pgfqpoint{1.802714in}{2.343618in}}%
\pgfpathlineto{\pgfqpoint{1.802809in}{2.342618in}}%
\pgfpathlineto{\pgfqpoint{1.802998in}{2.347978in}}%
\pgfpathlineto{\pgfqpoint{1.804799in}{2.383480in}}%
\pgfpathlineto{\pgfqpoint{1.805557in}{2.375579in}}%
\pgfpathlineto{\pgfqpoint{1.806694in}{2.414611in}}%
\pgfpathlineto{\pgfqpoint{1.808400in}{2.355433in}}%
\pgfpathlineto{\pgfqpoint{1.808684in}{2.336075in}}%
\pgfpathlineto{\pgfqpoint{1.809442in}{2.352447in}}%
\pgfpathlineto{\pgfqpoint{1.809726in}{2.362286in}}%
\pgfpathlineto{\pgfqpoint{1.810390in}{2.346425in}}%
\pgfpathlineto{\pgfqpoint{1.810579in}{2.347061in}}%
\pgfpathlineto{\pgfqpoint{1.810769in}{2.344215in}}%
\pgfpathlineto{\pgfqpoint{1.811906in}{2.329200in}}%
\pgfpathlineto{\pgfqpoint{1.812096in}{2.334483in}}%
\pgfpathlineto{\pgfqpoint{1.812759in}{2.377619in}}%
\pgfpathlineto{\pgfqpoint{1.813422in}{2.351619in}}%
\pgfpathlineto{\pgfqpoint{1.814938in}{2.340550in}}%
\pgfpathlineto{\pgfqpoint{1.815223in}{2.344738in}}%
\pgfpathlineto{\pgfqpoint{1.815602in}{2.340261in}}%
\pgfpathlineto{\pgfqpoint{1.815886in}{2.358109in}}%
\pgfpathlineto{\pgfqpoint{1.817497in}{2.624870in}}%
\pgfpathlineto{\pgfqpoint{1.818255in}{2.587279in}}%
\pgfpathlineto{\pgfqpoint{1.819108in}{2.348814in}}%
\pgfpathlineto{\pgfqpoint{1.820151in}{2.418117in}}%
\pgfpathlineto{\pgfqpoint{1.820435in}{2.410697in}}%
\pgfpathlineto{\pgfqpoint{1.820909in}{2.431490in}}%
\pgfpathlineto{\pgfqpoint{1.821193in}{2.427159in}}%
\pgfpathlineto{\pgfqpoint{1.821382in}{2.433725in}}%
\pgfpathlineto{\pgfqpoint{1.821856in}{2.453552in}}%
\pgfpathlineto{\pgfqpoint{1.822425in}{2.435578in}}%
\pgfpathlineto{\pgfqpoint{1.823278in}{2.401405in}}%
\pgfpathlineto{\pgfqpoint{1.823657in}{2.423157in}}%
\pgfpathlineto{\pgfqpoint{1.823941in}{2.433702in}}%
\pgfpathlineto{\pgfqpoint{1.824415in}{2.416276in}}%
\pgfpathlineto{\pgfqpoint{1.824604in}{2.417114in}}%
\pgfpathlineto{\pgfqpoint{1.824699in}{2.417672in}}%
\pgfpathlineto{\pgfqpoint{1.824889in}{2.413966in}}%
\pgfpathlineto{\pgfqpoint{1.825268in}{2.414920in}}%
\pgfpathlineto{\pgfqpoint{1.825742in}{2.377527in}}%
\pgfpathlineto{\pgfqpoint{1.826405in}{2.318835in}}%
\pgfpathlineto{\pgfqpoint{1.827258in}{2.333026in}}%
\pgfpathlineto{\pgfqpoint{1.827542in}{2.325535in}}%
\pgfpathlineto{\pgfqpoint{1.827921in}{2.341756in}}%
\pgfpathlineto{\pgfqpoint{1.828679in}{2.382607in}}%
\pgfpathlineto{\pgfqpoint{1.829343in}{2.364942in}}%
\pgfpathlineto{\pgfqpoint{1.829627in}{2.358730in}}%
\pgfpathlineto{\pgfqpoint{1.830006in}{2.369860in}}%
\pgfpathlineto{\pgfqpoint{1.830101in}{2.371102in}}%
\pgfpathlineto{\pgfqpoint{1.830669in}{2.366633in}}%
\pgfpathlineto{\pgfqpoint{1.832849in}{2.314517in}}%
\pgfpathlineto{\pgfqpoint{1.833607in}{2.314751in}}%
\pgfpathlineto{\pgfqpoint{1.833797in}{2.309586in}}%
\pgfpathlineto{\pgfqpoint{1.835597in}{2.260769in}}%
\pgfpathlineto{\pgfqpoint{1.835692in}{2.261373in}}%
\pgfpathlineto{\pgfqpoint{1.836166in}{2.269706in}}%
\pgfpathlineto{\pgfqpoint{1.836639in}{2.258488in}}%
\pgfpathlineto{\pgfqpoint{1.836829in}{2.254671in}}%
\pgfpathlineto{\pgfqpoint{1.837682in}{2.259465in}}%
\pgfpathlineto{\pgfqpoint{1.838250in}{2.262240in}}%
\pgfpathlineto{\pgfqpoint{1.838440in}{2.258642in}}%
\pgfpathlineto{\pgfqpoint{1.839577in}{2.232343in}}%
\pgfpathlineto{\pgfqpoint{1.839956in}{2.242219in}}%
\pgfpathlineto{\pgfqpoint{1.841852in}{2.215743in}}%
\pgfpathlineto{\pgfqpoint{1.842041in}{2.216669in}}%
\pgfpathlineto{\pgfqpoint{1.842231in}{2.218448in}}%
\pgfpathlineto{\pgfqpoint{1.842515in}{2.212579in}}%
\pgfpathlineto{\pgfqpoint{1.843936in}{2.185770in}}%
\pgfpathlineto{\pgfqpoint{1.844126in}{2.192300in}}%
\pgfpathlineto{\pgfqpoint{1.845358in}{2.207507in}}%
\pgfpathlineto{\pgfqpoint{1.845453in}{2.205698in}}%
\pgfpathlineto{\pgfqpoint{1.845737in}{2.196428in}}%
\pgfpathlineto{\pgfqpoint{1.846211in}{2.213131in}}%
\pgfpathlineto{\pgfqpoint{1.846684in}{2.198577in}}%
\pgfpathlineto{\pgfqpoint{1.846779in}{2.197559in}}%
\pgfpathlineto{\pgfqpoint{1.847158in}{2.203777in}}%
\pgfpathlineto{\pgfqpoint{1.849054in}{2.281341in}}%
\pgfpathlineto{\pgfqpoint{1.849433in}{2.279098in}}%
\pgfpathlineto{\pgfqpoint{1.849622in}{2.278213in}}%
\pgfpathlineto{\pgfqpoint{1.849812in}{2.281289in}}%
\pgfpathlineto{\pgfqpoint{1.851044in}{2.368415in}}%
\pgfpathlineto{\pgfqpoint{1.852181in}{2.346095in}}%
\pgfpathlineto{\pgfqpoint{1.853223in}{2.308047in}}%
\pgfpathlineto{\pgfqpoint{1.853887in}{2.318877in}}%
\pgfpathlineto{\pgfqpoint{1.854455in}{2.311842in}}%
\pgfpathlineto{\pgfqpoint{1.856066in}{2.277655in}}%
\pgfpathlineto{\pgfqpoint{1.854929in}{2.313124in}}%
\pgfpathlineto{\pgfqpoint{1.856256in}{2.280319in}}%
\pgfpathlineto{\pgfqpoint{1.857109in}{2.312104in}}%
\pgfpathlineto{\pgfqpoint{1.857488in}{2.287986in}}%
\pgfpathlineto{\pgfqpoint{1.858340in}{2.250503in}}%
\pgfpathlineto{\pgfqpoint{1.859288in}{2.260044in}}%
\pgfpathlineto{\pgfqpoint{1.859857in}{2.256990in}}%
\pgfpathlineto{\pgfqpoint{1.860615in}{2.365134in}}%
\pgfpathlineto{\pgfqpoint{1.861562in}{2.545884in}}%
\pgfpathlineto{\pgfqpoint{1.862321in}{2.515345in}}%
\pgfpathlineto{\pgfqpoint{1.862889in}{2.406518in}}%
\pgfpathlineto{\pgfqpoint{1.863458in}{2.245127in}}%
\pgfpathlineto{\pgfqpoint{1.864121in}{2.330199in}}%
\pgfpathlineto{\pgfqpoint{1.866301in}{2.381866in}}%
\pgfpathlineto{\pgfqpoint{1.866774in}{2.369150in}}%
\pgfpathlineto{\pgfqpoint{1.867438in}{2.339301in}}%
\pgfpathlineto{\pgfqpoint{1.868101in}{2.359220in}}%
\pgfpathlineto{\pgfqpoint{1.869523in}{2.392787in}}%
\pgfpathlineto{\pgfqpoint{1.870091in}{2.375272in}}%
\pgfpathlineto{\pgfqpoint{1.870470in}{2.358852in}}%
\pgfpathlineto{\pgfqpoint{1.870944in}{2.384155in}}%
\pgfpathlineto{\pgfqpoint{1.871892in}{2.402085in}}%
\pgfpathlineto{\pgfqpoint{1.872176in}{2.399258in}}%
\pgfpathlineto{\pgfqpoint{1.872650in}{2.411015in}}%
\pgfpathlineto{\pgfqpoint{1.872934in}{2.399106in}}%
\pgfpathlineto{\pgfqpoint{1.873313in}{2.383208in}}%
\pgfpathlineto{\pgfqpoint{1.873787in}{2.403994in}}%
\pgfpathlineto{\pgfqpoint{1.873977in}{2.400988in}}%
\pgfpathlineto{\pgfqpoint{1.874071in}{2.399249in}}%
\pgfpathlineto{\pgfqpoint{1.874356in}{2.413481in}}%
\pgfpathlineto{\pgfqpoint{1.875493in}{2.439763in}}%
\pgfpathlineto{\pgfqpoint{1.875682in}{2.436927in}}%
\pgfpathlineto{\pgfqpoint{1.876819in}{2.422814in}}%
\pgfpathlineto{\pgfqpoint{1.877009in}{2.432567in}}%
\pgfpathlineto{\pgfqpoint{1.877767in}{2.455191in}}%
\pgfpathlineto{\pgfqpoint{1.878241in}{2.447938in}}%
\pgfpathlineto{\pgfqpoint{1.878999in}{2.454944in}}%
\pgfpathlineto{\pgfqpoint{1.878715in}{2.444878in}}%
\pgfpathlineto{\pgfqpoint{1.879189in}{2.450633in}}%
\pgfpathlineto{\pgfqpoint{1.879473in}{2.432285in}}%
\pgfpathlineto{\pgfqpoint{1.880326in}{2.447806in}}%
\pgfpathlineto{\pgfqpoint{1.880800in}{2.464737in}}%
\pgfpathlineto{\pgfqpoint{1.881747in}{2.458890in}}%
\pgfpathlineto{\pgfqpoint{1.881937in}{2.456459in}}%
\pgfpathlineto{\pgfqpoint{1.882316in}{2.470289in}}%
\pgfpathlineto{\pgfqpoint{1.883737in}{2.483385in}}%
\pgfpathlineto{\pgfqpoint{1.883927in}{2.476324in}}%
\pgfpathlineto{\pgfqpoint{1.884116in}{2.469083in}}%
\pgfpathlineto{\pgfqpoint{1.884495in}{2.491299in}}%
\pgfpathlineto{\pgfqpoint{1.884969in}{2.476981in}}%
\pgfpathlineto{\pgfqpoint{1.886296in}{2.485107in}}%
\pgfpathlineto{\pgfqpoint{1.887054in}{2.494745in}}%
\pgfpathlineto{\pgfqpoint{1.887244in}{2.489538in}}%
\pgfpathlineto{\pgfqpoint{1.887623in}{2.471913in}}%
\pgfpathlineto{\pgfqpoint{1.888286in}{2.491554in}}%
\pgfpathlineto{\pgfqpoint{1.890465in}{2.541970in}}%
\pgfpathlineto{\pgfqpoint{1.890845in}{2.525979in}}%
\pgfpathlineto{\pgfqpoint{1.890939in}{2.522043in}}%
\pgfpathlineto{\pgfqpoint{1.891413in}{2.544148in}}%
\pgfpathlineto{\pgfqpoint{1.893214in}{2.577647in}}%
\pgfpathlineto{\pgfqpoint{1.893403in}{2.567325in}}%
\pgfpathlineto{\pgfqpoint{1.893593in}{2.559893in}}%
\pgfpathlineto{\pgfqpoint{1.894351in}{2.569326in}}%
\pgfpathlineto{\pgfqpoint{1.895204in}{2.591936in}}%
\pgfpathlineto{\pgfqpoint{1.895488in}{2.579417in}}%
\pgfpathlineto{\pgfqpoint{1.897478in}{2.500661in}}%
\pgfpathlineto{\pgfqpoint{1.897762in}{2.507041in}}%
\pgfpathlineto{\pgfqpoint{1.898141in}{2.513270in}}%
\pgfpathlineto{\pgfqpoint{1.898520in}{2.497951in}}%
\pgfpathlineto{\pgfqpoint{1.900510in}{2.388961in}}%
\pgfpathlineto{\pgfqpoint{1.900700in}{2.389453in}}%
\pgfpathlineto{\pgfqpoint{1.902880in}{2.427021in}}%
\pgfpathlineto{\pgfqpoint{1.903353in}{2.422591in}}%
\pgfpathlineto{\pgfqpoint{1.903543in}{2.421336in}}%
\pgfpathlineto{\pgfqpoint{1.903922in}{2.412183in}}%
\pgfpathlineto{\pgfqpoint{1.904396in}{2.424164in}}%
\pgfpathlineto{\pgfqpoint{1.905059in}{2.546546in}}%
\pgfpathlineto{\pgfqpoint{1.905912in}{2.669766in}}%
\pgfpathlineto{\pgfqpoint{1.906386in}{2.636134in}}%
\pgfpathlineto{\pgfqpoint{1.907049in}{2.556916in}}%
\pgfpathlineto{\pgfqpoint{1.907713in}{2.350240in}}%
\pgfpathlineto{\pgfqpoint{1.908471in}{2.438275in}}%
\pgfpathlineto{\pgfqpoint{1.908850in}{2.430105in}}%
\pgfpathlineto{\pgfqpoint{1.909229in}{2.443030in}}%
\pgfpathlineto{\pgfqpoint{1.909797in}{2.467483in}}%
\pgfpathlineto{\pgfqpoint{1.910650in}{2.510319in}}%
\pgfpathlineto{\pgfqpoint{1.911219in}{2.482682in}}%
\pgfpathlineto{\pgfqpoint{1.911598in}{2.459255in}}%
\pgfpathlineto{\pgfqpoint{1.912261in}{2.482708in}}%
\pgfpathlineto{\pgfqpoint{1.913588in}{2.523608in}}%
\pgfpathlineto{\pgfqpoint{1.913872in}{2.523335in}}%
\pgfpathlineto{\pgfqpoint{1.914157in}{2.516253in}}%
\pgfpathlineto{\pgfqpoint{1.915104in}{2.499608in}}%
\pgfpathlineto{\pgfqpoint{1.915294in}{2.507172in}}%
\pgfpathlineto{\pgfqpoint{1.916810in}{2.548500in}}%
\pgfpathlineto{\pgfqpoint{1.916905in}{2.547049in}}%
\pgfpathlineto{\pgfqpoint{1.917568in}{2.532148in}}%
\pgfpathlineto{\pgfqpoint{1.918231in}{2.537034in}}%
\pgfpathlineto{\pgfqpoint{1.920032in}{2.573363in}}%
\pgfpathlineto{\pgfqpoint{1.920221in}{2.576272in}}%
\pgfpathlineto{\pgfqpoint{1.920506in}{2.560358in}}%
\pgfpathlineto{\pgfqpoint{1.921359in}{2.547008in}}%
\pgfpathlineto{\pgfqpoint{1.921548in}{2.553619in}}%
\pgfpathlineto{\pgfqpoint{1.921832in}{2.563609in}}%
\pgfpathlineto{\pgfqpoint{1.922685in}{2.555285in}}%
\pgfpathlineto{\pgfqpoint{1.923917in}{2.527106in}}%
\pgfpathlineto{\pgfqpoint{1.924865in}{2.538288in}}%
\pgfpathlineto{\pgfqpoint{1.925054in}{2.543773in}}%
\pgfpathlineto{\pgfqpoint{1.925528in}{2.517683in}}%
\pgfpathlineto{\pgfqpoint{1.926665in}{2.536625in}}%
\pgfpathlineto{\pgfqpoint{1.926950in}{2.528175in}}%
\pgfpathlineto{\pgfqpoint{1.929982in}{2.464605in}}%
\pgfpathlineto{\pgfqpoint{1.930266in}{2.466790in}}%
\pgfpathlineto{\pgfqpoint{1.930456in}{2.463353in}}%
\pgfpathlineto{\pgfqpoint{1.932446in}{2.427850in}}%
\pgfpathlineto{\pgfqpoint{1.932636in}{2.432684in}}%
\pgfpathlineto{\pgfqpoint{1.932920in}{2.441553in}}%
\pgfpathlineto{\pgfqpoint{1.933773in}{2.436682in}}%
\pgfpathlineto{\pgfqpoint{1.933867in}{2.436713in}}%
\pgfpathlineto{\pgfqpoint{1.933962in}{2.435685in}}%
\pgfpathlineto{\pgfqpoint{1.935668in}{2.417907in}}%
\pgfpathlineto{\pgfqpoint{1.936142in}{2.423968in}}%
\pgfpathlineto{\pgfqpoint{1.937184in}{2.461365in}}%
\pgfpathlineto{\pgfqpoint{1.937753in}{2.443917in}}%
\pgfpathlineto{\pgfqpoint{1.938227in}{2.447022in}}%
\pgfpathlineto{\pgfqpoint{1.939364in}{2.498976in}}%
\pgfpathlineto{\pgfqpoint{1.940027in}{2.480036in}}%
\pgfpathlineto{\pgfqpoint{1.941638in}{2.425900in}}%
\pgfpathlineto{\pgfqpoint{1.942112in}{2.448657in}}%
\pgfpathlineto{\pgfqpoint{1.942681in}{2.457945in}}%
\pgfpathlineto{\pgfqpoint{1.943060in}{2.449449in}}%
\pgfpathlineto{\pgfqpoint{1.944292in}{2.411375in}}%
\pgfpathlineto{\pgfqpoint{1.944765in}{2.423030in}}%
\pgfpathlineto{\pgfqpoint{1.944955in}{2.421397in}}%
\pgfpathlineto{\pgfqpoint{1.945144in}{2.431360in}}%
\pgfpathlineto{\pgfqpoint{1.945429in}{2.450371in}}%
\pgfpathlineto{\pgfqpoint{1.946092in}{2.420098in}}%
\pgfpathlineto{\pgfqpoint{1.947229in}{2.407806in}}%
\pgfpathlineto{\pgfqpoint{1.947419in}{2.408406in}}%
\pgfpathlineto{\pgfqpoint{1.947703in}{2.406447in}}%
\pgfpathlineto{\pgfqpoint{1.947893in}{2.409551in}}%
\pgfpathlineto{\pgfqpoint{1.949124in}{2.532061in}}%
\pgfpathlineto{\pgfqpoint{1.949788in}{2.678791in}}%
\pgfpathlineto{\pgfqpoint{1.950641in}{2.660453in}}%
\pgfpathlineto{\pgfqpoint{1.951209in}{2.548859in}}%
\pgfpathlineto{\pgfqpoint{1.951778in}{2.378856in}}%
\pgfpathlineto{\pgfqpoint{1.952536in}{2.438226in}}%
\pgfpathlineto{\pgfqpoint{1.952631in}{2.438184in}}%
\pgfpathlineto{\pgfqpoint{1.953010in}{2.448998in}}%
\pgfpathlineto{\pgfqpoint{1.953389in}{2.432482in}}%
\pgfpathlineto{\pgfqpoint{1.953578in}{2.429847in}}%
\pgfpathlineto{\pgfqpoint{1.954052in}{2.437395in}}%
\pgfpathlineto{\pgfqpoint{1.954621in}{2.457929in}}%
\pgfpathlineto{\pgfqpoint{1.955189in}{2.439168in}}%
\pgfpathlineto{\pgfqpoint{1.955853in}{2.404972in}}%
\pgfpathlineto{\pgfqpoint{1.956611in}{2.428233in}}%
\pgfpathlineto{\pgfqpoint{1.957748in}{2.462182in}}%
\pgfpathlineto{\pgfqpoint{1.958127in}{2.451851in}}%
\pgfpathlineto{\pgfqpoint{1.958980in}{2.421452in}}%
\pgfpathlineto{\pgfqpoint{1.959359in}{2.438469in}}%
\pgfpathlineto{\pgfqpoint{1.959928in}{2.453003in}}%
\pgfpathlineto{\pgfqpoint{1.960970in}{2.498300in}}%
\pgfpathlineto{\pgfqpoint{1.961254in}{2.483578in}}%
\pgfpathlineto{\pgfqpoint{1.961444in}{2.472845in}}%
\pgfpathlineto{\pgfqpoint{1.962202in}{2.491246in}}%
\pgfpathlineto{\pgfqpoint{1.964192in}{2.542813in}}%
\pgfpathlineto{\pgfqpoint{1.964381in}{2.536103in}}%
\pgfpathlineto{\pgfqpoint{1.964855in}{2.517762in}}%
\pgfpathlineto{\pgfqpoint{1.965519in}{2.532047in}}%
\pgfpathlineto{\pgfqpoint{1.965708in}{2.534372in}}%
\pgfpathlineto{\pgfqpoint{1.966087in}{2.525251in}}%
\pgfpathlineto{\pgfqpoint{1.966372in}{2.527478in}}%
\pgfpathlineto{\pgfqpoint{1.966751in}{2.501728in}}%
\pgfpathlineto{\pgfqpoint{1.969120in}{2.384107in}}%
\pgfpathlineto{\pgfqpoint{1.969404in}{2.376149in}}%
\pgfpathlineto{\pgfqpoint{1.971299in}{2.306508in}}%
\pgfpathlineto{\pgfqpoint{1.971489in}{2.304447in}}%
\pgfpathlineto{\pgfqpoint{1.975279in}{2.156484in}}%
\pgfpathlineto{\pgfqpoint{1.975753in}{2.181593in}}%
\pgfpathlineto{\pgfqpoint{1.981534in}{2.413286in}}%
\pgfpathlineto{\pgfqpoint{1.981723in}{2.406809in}}%
\pgfpathlineto{\pgfqpoint{1.981818in}{2.404657in}}%
\pgfpathlineto{\pgfqpoint{1.982102in}{2.418065in}}%
\pgfpathlineto{\pgfqpoint{1.984282in}{2.519837in}}%
\pgfpathlineto{\pgfqpoint{1.984566in}{2.512249in}}%
\pgfpathlineto{\pgfqpoint{1.985419in}{2.485229in}}%
\pgfpathlineto{\pgfqpoint{1.985798in}{2.504184in}}%
\pgfpathlineto{\pgfqpoint{1.986935in}{2.546668in}}%
\pgfpathlineto{\pgfqpoint{1.987409in}{2.533179in}}%
\pgfpathlineto{\pgfqpoint{1.987504in}{2.531709in}}%
\pgfpathlineto{\pgfqpoint{1.987788in}{2.538319in}}%
\pgfpathlineto{\pgfqpoint{1.989494in}{2.593825in}}%
\pgfpathlineto{\pgfqpoint{1.989684in}{2.587596in}}%
\pgfpathlineto{\pgfqpoint{1.991389in}{2.531722in}}%
\pgfpathlineto{\pgfqpoint{1.991579in}{2.534526in}}%
\pgfpathlineto{\pgfqpoint{1.992716in}{2.587197in}}%
\pgfpathlineto{\pgfqpoint{1.994137in}{2.810371in}}%
\pgfpathlineto{\pgfqpoint{1.994611in}{2.773593in}}%
\pgfpathlineto{\pgfqpoint{1.995369in}{2.594201in}}%
\pgfpathlineto{\pgfqpoint{1.995748in}{2.494775in}}%
\pgfpathlineto{\pgfqpoint{1.996507in}{2.580067in}}%
\pgfpathlineto{\pgfqpoint{1.998781in}{2.640012in}}%
\pgfpathlineto{\pgfqpoint{1.998876in}{2.639473in}}%
\pgfpathlineto{\pgfqpoint{1.999444in}{2.608827in}}%
\pgfpathlineto{\pgfqpoint{1.999634in}{2.602760in}}%
\pgfpathlineto{\pgfqpoint{2.000392in}{2.610187in}}%
\pgfpathlineto{\pgfqpoint{2.002003in}{2.669286in}}%
\pgfpathlineto{\pgfqpoint{2.002192in}{2.656460in}}%
\pgfpathlineto{\pgfqpoint{2.002950in}{2.638213in}}%
\pgfpathlineto{\pgfqpoint{2.003330in}{2.646651in}}%
\pgfpathlineto{\pgfqpoint{2.004751in}{2.680789in}}%
\pgfpathlineto{\pgfqpoint{2.005225in}{2.664715in}}%
\pgfpathlineto{\pgfqpoint{2.005604in}{2.652737in}}%
\pgfpathlineto{\pgfqpoint{2.006267in}{2.663549in}}%
\pgfpathlineto{\pgfqpoint{2.007594in}{2.702360in}}%
\pgfpathlineto{\pgfqpoint{2.007025in}{2.661994in}}%
\pgfpathlineto{\pgfqpoint{2.007783in}{2.698746in}}%
\pgfpathlineto{\pgfqpoint{2.009015in}{2.674917in}}%
\pgfpathlineto{\pgfqpoint{2.009205in}{2.680299in}}%
\pgfpathlineto{\pgfqpoint{2.010058in}{2.714464in}}%
\pgfpathlineto{\pgfqpoint{2.010721in}{2.703096in}}%
\pgfpathlineto{\pgfqpoint{2.011290in}{2.710607in}}%
\pgfpathlineto{\pgfqpoint{2.011669in}{2.703840in}}%
\pgfpathlineto{\pgfqpoint{2.012522in}{2.688876in}}%
\pgfpathlineto{\pgfqpoint{2.012806in}{2.697395in}}%
\pgfpathlineto{\pgfqpoint{2.012995in}{2.702007in}}%
\pgfpathlineto{\pgfqpoint{2.013375in}{2.696344in}}%
\pgfpathlineto{\pgfqpoint{2.013943in}{2.698365in}}%
\pgfpathlineto{\pgfqpoint{2.014322in}{2.703306in}}%
\pgfpathlineto{\pgfqpoint{2.014512in}{2.696863in}}%
\pgfpathlineto{\pgfqpoint{2.014891in}{2.687863in}}%
\pgfpathlineto{\pgfqpoint{2.015554in}{2.696966in}}%
\pgfpathlineto{\pgfqpoint{2.015744in}{2.694061in}}%
\pgfpathlineto{\pgfqpoint{2.017070in}{2.669608in}}%
\pgfpathlineto{\pgfqpoint{2.017260in}{2.672007in}}%
\pgfpathlineto{\pgfqpoint{2.017449in}{2.673892in}}%
\pgfpathlineto{\pgfqpoint{2.017828in}{2.662782in}}%
\pgfpathlineto{\pgfqpoint{2.020103in}{2.632892in}}%
\pgfpathlineto{\pgfqpoint{2.020292in}{2.635726in}}%
\pgfpathlineto{\pgfqpoint{2.021524in}{2.643038in}}%
\pgfpathlineto{\pgfqpoint{2.021050in}{2.634429in}}%
\pgfpathlineto{\pgfqpoint{2.021619in}{2.642707in}}%
\pgfpathlineto{\pgfqpoint{2.023325in}{2.613459in}}%
\pgfpathlineto{\pgfqpoint{2.023514in}{2.626849in}}%
\pgfpathlineto{\pgfqpoint{2.023799in}{2.650396in}}%
\pgfpathlineto{\pgfqpoint{2.024746in}{2.648060in}}%
\pgfpathlineto{\pgfqpoint{2.024841in}{2.646618in}}%
\pgfpathlineto{\pgfqpoint{2.025125in}{2.656228in}}%
\pgfpathlineto{\pgfqpoint{2.025315in}{2.661939in}}%
\pgfpathlineto{\pgfqpoint{2.025694in}{2.645545in}}%
\pgfpathlineto{\pgfqpoint{2.026168in}{2.653739in}}%
\pgfpathlineto{\pgfqpoint{2.026831in}{2.680193in}}%
\pgfpathlineto{\pgfqpoint{2.027873in}{2.670249in}}%
\pgfpathlineto{\pgfqpoint{2.028347in}{2.671224in}}%
\pgfpathlineto{\pgfqpoint{2.028726in}{2.650207in}}%
\pgfpathlineto{\pgfqpoint{2.029200in}{2.605474in}}%
\pgfpathlineto{\pgfqpoint{2.029958in}{2.625004in}}%
\pgfpathlineto{\pgfqpoint{2.030053in}{2.627996in}}%
\pgfpathlineto{\pgfqpoint{2.030527in}{2.616685in}}%
\pgfpathlineto{\pgfqpoint{2.030716in}{2.617398in}}%
\pgfpathlineto{\pgfqpoint{2.032517in}{2.584060in}}%
\pgfpathlineto{\pgfqpoint{2.032706in}{2.589978in}}%
\pgfpathlineto{\pgfqpoint{2.033275in}{2.621620in}}%
\pgfpathlineto{\pgfqpoint{2.033844in}{2.596197in}}%
\pgfpathlineto{\pgfqpoint{2.034981in}{2.575975in}}%
\pgfpathlineto{\pgfqpoint{2.035170in}{2.577246in}}%
\pgfpathlineto{\pgfqpoint{2.036592in}{2.622527in}}%
\pgfpathlineto{\pgfqpoint{2.037634in}{2.865576in}}%
\pgfpathlineto{\pgfqpoint{2.038582in}{2.823039in}}%
\pgfpathlineto{\pgfqpoint{2.039150in}{2.680309in}}%
\pgfpathlineto{\pgfqpoint{2.039624in}{2.543067in}}%
\pgfpathlineto{\pgfqpoint{2.040382in}{2.606687in}}%
\pgfpathlineto{\pgfqpoint{2.040667in}{2.597438in}}%
\pgfpathlineto{\pgfqpoint{2.041140in}{2.619775in}}%
\pgfpathlineto{\pgfqpoint{2.041235in}{2.619189in}}%
\pgfpathlineto{\pgfqpoint{2.041425in}{2.615395in}}%
\pgfpathlineto{\pgfqpoint{2.041899in}{2.626270in}}%
\pgfpathlineto{\pgfqpoint{2.042183in}{2.618328in}}%
\pgfpathlineto{\pgfqpoint{2.042562in}{2.643893in}}%
\pgfpathlineto{\pgfqpoint{2.042941in}{2.616045in}}%
\pgfpathlineto{\pgfqpoint{2.045121in}{2.519261in}}%
\pgfpathlineto{\pgfqpoint{2.045310in}{2.523286in}}%
\pgfpathlineto{\pgfqpoint{2.045500in}{2.528413in}}%
\pgfpathlineto{\pgfqpoint{2.045973in}{2.507828in}}%
\pgfpathlineto{\pgfqpoint{2.050522in}{2.315519in}}%
\pgfpathlineto{\pgfqpoint{2.052228in}{2.272745in}}%
\pgfpathlineto{\pgfqpoint{2.052796in}{2.286770in}}%
\pgfpathlineto{\pgfqpoint{2.055260in}{2.393311in}}%
\pgfpathlineto{\pgfqpoint{2.055639in}{2.372326in}}%
\pgfpathlineto{\pgfqpoint{2.055734in}{2.372212in}}%
\pgfpathlineto{\pgfqpoint{2.058198in}{2.428347in}}%
\pgfpathlineto{\pgfqpoint{2.058293in}{2.425663in}}%
\pgfpathlineto{\pgfqpoint{2.058577in}{2.413823in}}%
\pgfpathlineto{\pgfqpoint{2.059146in}{2.429386in}}%
\pgfpathlineto{\pgfqpoint{2.059525in}{2.418781in}}%
\pgfpathlineto{\pgfqpoint{2.059619in}{2.418249in}}%
\pgfpathlineto{\pgfqpoint{2.059809in}{2.422551in}}%
\pgfpathlineto{\pgfqpoint{2.061230in}{2.437065in}}%
\pgfpathlineto{\pgfqpoint{2.061325in}{2.434460in}}%
\pgfpathlineto{\pgfqpoint{2.062178in}{2.414147in}}%
\pgfpathlineto{\pgfqpoint{2.062841in}{2.418638in}}%
\pgfpathlineto{\pgfqpoint{2.063505in}{2.409082in}}%
\pgfpathlineto{\pgfqpoint{2.064168in}{2.415644in}}%
\pgfpathlineto{\pgfqpoint{2.064831in}{2.430908in}}%
\pgfpathlineto{\pgfqpoint{2.065211in}{2.416349in}}%
\pgfpathlineto{\pgfqpoint{2.065305in}{2.415128in}}%
\pgfpathlineto{\pgfqpoint{2.065495in}{2.423489in}}%
\pgfpathlineto{\pgfqpoint{2.065779in}{2.435772in}}%
\pgfpathlineto{\pgfqpoint{2.066442in}{2.415645in}}%
\pgfpathlineto{\pgfqpoint{2.066727in}{2.411329in}}%
\pgfpathlineto{\pgfqpoint{2.067011in}{2.422500in}}%
\pgfpathlineto{\pgfqpoint{2.068432in}{2.466119in}}%
\pgfpathlineto{\pgfqpoint{2.068622in}{2.460905in}}%
\pgfpathlineto{\pgfqpoint{2.068812in}{2.456564in}}%
\pgfpathlineto{\pgfqpoint{2.069191in}{2.479360in}}%
\pgfpathlineto{\pgfqpoint{2.069285in}{2.480580in}}%
\pgfpathlineto{\pgfqpoint{2.069475in}{2.472605in}}%
\pgfpathlineto{\pgfqpoint{2.069664in}{2.465530in}}%
\pgfpathlineto{\pgfqpoint{2.070233in}{2.483733in}}%
\pgfpathlineto{\pgfqpoint{2.070707in}{2.512931in}}%
\pgfpathlineto{\pgfqpoint{2.071275in}{2.488919in}}%
\pgfpathlineto{\pgfqpoint{2.075066in}{2.375112in}}%
\pgfpathlineto{\pgfqpoint{2.071749in}{2.491600in}}%
\pgfpathlineto{\pgfqpoint{2.075350in}{2.375304in}}%
\pgfpathlineto{\pgfqpoint{2.076203in}{2.361259in}}%
\pgfpathlineto{\pgfqpoint{2.076487in}{2.372436in}}%
\pgfpathlineto{\pgfqpoint{2.077151in}{2.402734in}}%
\pgfpathlineto{\pgfqpoint{2.077530in}{2.372404in}}%
\pgfpathlineto{\pgfqpoint{2.078857in}{2.345445in}}%
\pgfpathlineto{\pgfqpoint{2.078951in}{2.345901in}}%
\pgfpathlineto{\pgfqpoint{2.080562in}{2.434249in}}%
\pgfpathlineto{\pgfqpoint{2.081605in}{2.611830in}}%
\pgfpathlineto{\pgfqpoint{2.082173in}{2.575590in}}%
\pgfpathlineto{\pgfqpoint{2.082268in}{2.576925in}}%
\pgfpathlineto{\pgfqpoint{2.082458in}{2.565214in}}%
\pgfpathlineto{\pgfqpoint{2.083405in}{2.285307in}}%
\pgfpathlineto{\pgfqpoint{2.084542in}{2.349523in}}%
\pgfpathlineto{\pgfqpoint{2.086343in}{2.398588in}}%
\pgfpathlineto{\pgfqpoint{2.086532in}{2.394500in}}%
\pgfpathlineto{\pgfqpoint{2.087480in}{2.346412in}}%
\pgfpathlineto{\pgfqpoint{2.087954in}{2.368729in}}%
\pgfpathlineto{\pgfqpoint{2.089470in}{2.397443in}}%
\pgfpathlineto{\pgfqpoint{2.090513in}{2.355331in}}%
\pgfpathlineto{\pgfqpoint{2.091460in}{2.376710in}}%
\pgfpathlineto{\pgfqpoint{2.092597in}{2.404206in}}%
\pgfpathlineto{\pgfqpoint{2.093166in}{2.387767in}}%
\pgfpathlineto{\pgfqpoint{2.093735in}{2.373362in}}%
\pgfpathlineto{\pgfqpoint{2.094019in}{2.386411in}}%
\pgfpathlineto{\pgfqpoint{2.094398in}{2.401849in}}%
\pgfpathlineto{\pgfqpoint{2.095156in}{2.389509in}}%
\pgfpathlineto{\pgfqpoint{2.095535in}{2.398266in}}%
\pgfpathlineto{\pgfqpoint{2.096009in}{2.385871in}}%
\pgfpathlineto{\pgfqpoint{2.096483in}{2.394428in}}%
\pgfpathlineto{\pgfqpoint{2.096956in}{2.380083in}}%
\pgfpathlineto{\pgfqpoint{2.097336in}{2.400038in}}%
\pgfpathlineto{\pgfqpoint{2.097430in}{2.402663in}}%
\pgfpathlineto{\pgfqpoint{2.097809in}{2.387833in}}%
\pgfpathlineto{\pgfqpoint{2.098283in}{2.397717in}}%
\pgfpathlineto{\pgfqpoint{2.099705in}{2.379844in}}%
\pgfpathlineto{\pgfqpoint{2.098852in}{2.399695in}}%
\pgfpathlineto{\pgfqpoint{2.100178in}{2.387250in}}%
\pgfpathlineto{\pgfqpoint{2.100558in}{2.399312in}}%
\pgfpathlineto{\pgfqpoint{2.101221in}{2.384833in}}%
\pgfpathlineto{\pgfqpoint{2.101600in}{2.374559in}}%
\pgfpathlineto{\pgfqpoint{2.102169in}{2.385524in}}%
\pgfpathlineto{\pgfqpoint{2.102263in}{2.385106in}}%
\pgfpathlineto{\pgfqpoint{2.102548in}{2.386867in}}%
\pgfpathlineto{\pgfqpoint{2.102642in}{2.385538in}}%
\pgfpathlineto{\pgfqpoint{2.105106in}{2.322164in}}%
\pgfpathlineto{\pgfqpoint{2.105201in}{2.326631in}}%
\pgfpathlineto{\pgfqpoint{2.105485in}{2.339157in}}%
\pgfpathlineto{\pgfqpoint{2.106149in}{2.318687in}}%
\pgfpathlineto{\pgfqpoint{2.107949in}{2.285846in}}%
\pgfpathlineto{\pgfqpoint{2.108044in}{2.288961in}}%
\pgfpathlineto{\pgfqpoint{2.108328in}{2.299726in}}%
\pgfpathlineto{\pgfqpoint{2.109086in}{2.287370in}}%
\pgfpathlineto{\pgfqpoint{2.109276in}{2.284696in}}%
\pgfpathlineto{\pgfqpoint{2.109844in}{2.293261in}}%
\pgfpathlineto{\pgfqpoint{2.110034in}{2.290843in}}%
\pgfpathlineto{\pgfqpoint{2.110508in}{2.266293in}}%
\pgfpathlineto{\pgfqpoint{2.111266in}{2.283567in}}%
\pgfpathlineto{\pgfqpoint{2.112687in}{2.314116in}}%
\pgfpathlineto{\pgfqpoint{2.112877in}{2.311583in}}%
\pgfpathlineto{\pgfqpoint{2.113635in}{2.306175in}}%
\pgfpathlineto{\pgfqpoint{2.113256in}{2.312748in}}%
\pgfpathlineto{\pgfqpoint{2.113730in}{2.309189in}}%
\pgfpathlineto{\pgfqpoint{2.114867in}{2.341436in}}%
\pgfpathlineto{\pgfqpoint{2.115151in}{2.336741in}}%
\pgfpathlineto{\pgfqpoint{2.117046in}{2.270501in}}%
\pgfpathlineto{\pgfqpoint{2.117615in}{2.291857in}}%
\pgfpathlineto{\pgfqpoint{2.117899in}{2.296911in}}%
\pgfpathlineto{\pgfqpoint{2.118278in}{2.280672in}}%
\pgfpathlineto{\pgfqpoint{2.119795in}{2.233569in}}%
\pgfpathlineto{\pgfqpoint{2.119889in}{2.234393in}}%
\pgfpathlineto{\pgfqpoint{2.121311in}{2.290945in}}%
\pgfpathlineto{\pgfqpoint{2.121500in}{2.277505in}}%
\pgfpathlineto{\pgfqpoint{2.122827in}{2.245366in}}%
\pgfpathlineto{\pgfqpoint{2.123111in}{2.245008in}}%
\pgfpathlineto{\pgfqpoint{2.123206in}{2.246189in}}%
\pgfpathlineto{\pgfqpoint{2.124059in}{2.286921in}}%
\pgfpathlineto{\pgfqpoint{2.125291in}{2.530472in}}%
\pgfpathlineto{\pgfqpoint{2.126049in}{2.501162in}}%
\pgfpathlineto{\pgfqpoint{2.126428in}{2.452853in}}%
\pgfpathlineto{\pgfqpoint{2.127186in}{2.230249in}}%
\pgfpathlineto{\pgfqpoint{2.127944in}{2.312206in}}%
\pgfpathlineto{\pgfqpoint{2.128134in}{2.306527in}}%
\pgfpathlineto{\pgfqpoint{2.128608in}{2.325622in}}%
\pgfpathlineto{\pgfqpoint{2.128797in}{2.329984in}}%
\pgfpathlineto{\pgfqpoint{2.129461in}{2.318265in}}%
\pgfpathlineto{\pgfqpoint{2.131166in}{2.264285in}}%
\pgfpathlineto{\pgfqpoint{2.130124in}{2.327059in}}%
\pgfpathlineto{\pgfqpoint{2.131545in}{2.287516in}}%
\pgfpathlineto{\pgfqpoint{2.133251in}{2.404232in}}%
\pgfpathlineto{\pgfqpoint{2.133535in}{2.393562in}}%
\pgfpathlineto{\pgfqpoint{2.134483in}{2.353522in}}%
\pgfpathlineto{\pgfqpoint{2.134862in}{2.379861in}}%
\pgfpathlineto{\pgfqpoint{2.135336in}{2.378474in}}%
\pgfpathlineto{\pgfqpoint{2.136378in}{2.402677in}}%
\pgfpathlineto{\pgfqpoint{2.136663in}{2.406777in}}%
\pgfpathlineto{\pgfqpoint{2.136947in}{2.396272in}}%
\pgfpathlineto{\pgfqpoint{2.137421in}{2.373168in}}%
\pgfpathlineto{\pgfqpoint{2.138084in}{2.392742in}}%
\pgfpathlineto{\pgfqpoint{2.138463in}{2.411204in}}%
\pgfpathlineto{\pgfqpoint{2.139221in}{2.395293in}}%
\pgfpathlineto{\pgfqpoint{2.139695in}{2.396505in}}%
\pgfpathlineto{\pgfqpoint{2.139885in}{2.392720in}}%
\pgfpathlineto{\pgfqpoint{2.140358in}{2.381075in}}%
\pgfpathlineto{\pgfqpoint{2.141117in}{2.388162in}}%
\pgfpathlineto{\pgfqpoint{2.141306in}{2.390262in}}%
\pgfpathlineto{\pgfqpoint{2.141875in}{2.381009in}}%
\pgfpathlineto{\pgfqpoint{2.142917in}{2.367046in}}%
\pgfpathlineto{\pgfqpoint{2.144054in}{2.347212in}}%
\pgfpathlineto{\pgfqpoint{2.144339in}{2.353437in}}%
\pgfpathlineto{\pgfqpoint{2.144433in}{2.355432in}}%
\pgfpathlineto{\pgfqpoint{2.144812in}{2.343705in}}%
\pgfpathlineto{\pgfqpoint{2.144907in}{2.341678in}}%
\pgfpathlineto{\pgfqpoint{2.145286in}{2.354573in}}%
\pgfpathlineto{\pgfqpoint{2.145760in}{2.344750in}}%
\pgfpathlineto{\pgfqpoint{2.146044in}{2.357169in}}%
\pgfpathlineto{\pgfqpoint{2.146613in}{2.332791in}}%
\pgfpathlineto{\pgfqpoint{2.146802in}{2.333387in}}%
\pgfpathlineto{\pgfqpoint{2.147087in}{2.331617in}}%
\pgfpathlineto{\pgfqpoint{2.147940in}{2.332809in}}%
\pgfpathlineto{\pgfqpoint{2.154668in}{2.187265in}}%
\pgfpathlineto{\pgfqpoint{2.156374in}{2.212655in}}%
\pgfpathlineto{\pgfqpoint{2.156563in}{2.204672in}}%
\pgfpathlineto{\pgfqpoint{2.157416in}{2.193199in}}%
\pgfpathlineto{\pgfqpoint{2.157606in}{2.198349in}}%
\pgfpathlineto{\pgfqpoint{2.157985in}{2.217730in}}%
\pgfpathlineto{\pgfqpoint{2.158837in}{2.213428in}}%
\pgfpathlineto{\pgfqpoint{2.160638in}{2.150112in}}%
\pgfpathlineto{\pgfqpoint{2.161207in}{2.165183in}}%
\pgfpathlineto{\pgfqpoint{2.161775in}{2.180344in}}%
\pgfpathlineto{\pgfqpoint{2.162059in}{2.164315in}}%
\pgfpathlineto{\pgfqpoint{2.162344in}{2.149926in}}%
\pgfpathlineto{\pgfqpoint{2.163007in}{2.169121in}}%
\pgfpathlineto{\pgfqpoint{2.163386in}{2.177030in}}%
\pgfpathlineto{\pgfqpoint{2.164713in}{2.238478in}}%
\pgfpathlineto{\pgfqpoint{2.165660in}{2.214926in}}%
\pgfpathlineto{\pgfqpoint{2.165850in}{2.207247in}}%
\pgfpathlineto{\pgfqpoint{2.166324in}{2.234002in}}%
\pgfpathlineto{\pgfqpoint{2.166608in}{2.223045in}}%
\pgfpathlineto{\pgfqpoint{2.166703in}{2.222258in}}%
\pgfpathlineto{\pgfqpoint{2.166987in}{2.227529in}}%
\pgfpathlineto{\pgfqpoint{2.167745in}{2.258570in}}%
\pgfpathlineto{\pgfqpoint{2.168977in}{2.498134in}}%
\pgfpathlineto{\pgfqpoint{2.169735in}{2.452210in}}%
\pgfpathlineto{\pgfqpoint{2.170114in}{2.384068in}}%
\pgfpathlineto{\pgfqpoint{2.170872in}{2.161571in}}%
\pgfpathlineto{\pgfqpoint{2.171631in}{2.234983in}}%
\pgfpathlineto{\pgfqpoint{2.172294in}{2.225377in}}%
\pgfpathlineto{\pgfqpoint{2.172673in}{2.235433in}}%
\pgfpathlineto{\pgfqpoint{2.173715in}{2.248390in}}%
\pgfpathlineto{\pgfqpoint{2.173242in}{2.230707in}}%
\pgfpathlineto{\pgfqpoint{2.174000in}{2.243278in}}%
\pgfpathlineto{\pgfqpoint{2.174284in}{2.247507in}}%
\pgfpathlineto{\pgfqpoint{2.174474in}{2.238229in}}%
\pgfpathlineto{\pgfqpoint{2.174947in}{2.203044in}}%
\pgfpathlineto{\pgfqpoint{2.175800in}{2.214475in}}%
\pgfpathlineto{\pgfqpoint{2.176937in}{2.241541in}}%
\pgfpathlineto{\pgfqpoint{2.177127in}{2.238861in}}%
\pgfpathlineto{\pgfqpoint{2.177980in}{2.219427in}}%
\pgfpathlineto{\pgfqpoint{2.178264in}{2.234233in}}%
\pgfpathlineto{\pgfqpoint{2.179970in}{2.333320in}}%
\pgfpathlineto{\pgfqpoint{2.180349in}{2.331277in}}%
\pgfpathlineto{\pgfqpoint{2.181486in}{2.377622in}}%
\pgfpathlineto{\pgfqpoint{2.183666in}{2.468220in}}%
\pgfpathlineto{\pgfqpoint{2.183760in}{2.468017in}}%
\pgfpathlineto{\pgfqpoint{2.184045in}{2.460918in}}%
\pgfpathlineto{\pgfqpoint{2.184329in}{2.472986in}}%
\pgfpathlineto{\pgfqpoint{2.185845in}{2.523123in}}%
\pgfpathlineto{\pgfqpoint{2.186035in}{2.519384in}}%
\pgfpathlineto{\pgfqpoint{2.186319in}{2.506659in}}%
\pgfpathlineto{\pgfqpoint{2.186793in}{2.528163in}}%
\pgfpathlineto{\pgfqpoint{2.187267in}{2.509153in}}%
\pgfpathlineto{\pgfqpoint{2.190015in}{2.569852in}}%
\pgfpathlineto{\pgfqpoint{2.190678in}{2.562961in}}%
\pgfpathlineto{\pgfqpoint{2.190773in}{2.562371in}}%
\pgfpathlineto{\pgfqpoint{2.190962in}{2.564703in}}%
\pgfpathlineto{\pgfqpoint{2.191910in}{2.582925in}}%
\pgfpathlineto{\pgfqpoint{2.192194in}{2.574115in}}%
\pgfpathlineto{\pgfqpoint{2.192953in}{2.565528in}}%
\pgfpathlineto{\pgfqpoint{2.193237in}{2.571876in}}%
\pgfpathlineto{\pgfqpoint{2.193521in}{2.583706in}}%
\pgfpathlineto{\pgfqpoint{2.193995in}{2.561593in}}%
\pgfpathlineto{\pgfqpoint{2.194090in}{2.559063in}}%
\pgfpathlineto{\pgfqpoint{2.194564in}{2.571996in}}%
\pgfpathlineto{\pgfqpoint{2.194943in}{2.564277in}}%
\pgfpathlineto{\pgfqpoint{2.195795in}{2.593182in}}%
\pgfpathlineto{\pgfqpoint{2.196269in}{2.572361in}}%
\pgfpathlineto{\pgfqpoint{2.196364in}{2.570088in}}%
\pgfpathlineto{\pgfqpoint{2.196743in}{2.586764in}}%
\pgfpathlineto{\pgfqpoint{2.196838in}{2.589205in}}%
\pgfpathlineto{\pgfqpoint{2.197217in}{2.575449in}}%
\pgfpathlineto{\pgfqpoint{2.197691in}{2.584931in}}%
\pgfpathlineto{\pgfqpoint{2.198070in}{2.575252in}}%
\pgfpathlineto{\pgfqpoint{2.198544in}{2.590411in}}%
\pgfpathlineto{\pgfqpoint{2.199776in}{2.628362in}}%
\pgfpathlineto{\pgfqpoint{2.200534in}{2.621752in}}%
\pgfpathlineto{\pgfqpoint{2.200818in}{2.615215in}}%
\pgfpathlineto{\pgfqpoint{2.201102in}{2.624175in}}%
\pgfpathlineto{\pgfqpoint{2.201671in}{2.655730in}}%
\pgfpathlineto{\pgfqpoint{2.202429in}{2.645157in}}%
\pgfpathlineto{\pgfqpoint{2.203471in}{2.612370in}}%
\pgfpathlineto{\pgfqpoint{2.204135in}{2.577780in}}%
\pgfpathlineto{\pgfqpoint{2.204798in}{2.601520in}}%
\pgfpathlineto{\pgfqpoint{2.205082in}{2.607673in}}%
\pgfpathlineto{\pgfqpoint{2.205651in}{2.595639in}}%
\pgfpathlineto{\pgfqpoint{2.206599in}{2.554740in}}%
\pgfpathlineto{\pgfqpoint{2.207357in}{2.563204in}}%
\pgfpathlineto{\pgfqpoint{2.208115in}{2.602427in}}%
\pgfpathlineto{\pgfqpoint{2.208683in}{2.580060in}}%
\pgfpathlineto{\pgfqpoint{2.211052in}{2.506279in}}%
\pgfpathlineto{\pgfqpoint{2.211337in}{2.525374in}}%
\pgfpathlineto{\pgfqpoint{2.212758in}{2.778106in}}%
\pgfpathlineto{\pgfqpoint{2.213516in}{2.751565in}}%
\pgfpathlineto{\pgfqpoint{2.214464in}{2.468377in}}%
\pgfpathlineto{\pgfqpoint{2.215696in}{2.515493in}}%
\pgfpathlineto{\pgfqpoint{2.216359in}{2.513622in}}%
\pgfpathlineto{\pgfqpoint{2.217117in}{2.525128in}}%
\pgfpathlineto{\pgfqpoint{2.217212in}{2.526650in}}%
\pgfpathlineto{\pgfqpoint{2.217496in}{2.515903in}}%
\pgfpathlineto{\pgfqpoint{2.218539in}{2.466790in}}%
\pgfpathlineto{\pgfqpoint{2.219202in}{2.476699in}}%
\pgfpathlineto{\pgfqpoint{2.220339in}{2.501293in}}%
\pgfpathlineto{\pgfqpoint{2.220718in}{2.488437in}}%
\pgfpathlineto{\pgfqpoint{2.221571in}{2.442743in}}%
\pgfpathlineto{\pgfqpoint{2.222140in}{2.460669in}}%
\pgfpathlineto{\pgfqpoint{2.222803in}{2.455200in}}%
\pgfpathlineto{\pgfqpoint{2.223751in}{2.486031in}}%
\pgfpathlineto{\pgfqpoint{2.224130in}{2.478275in}}%
\pgfpathlineto{\pgfqpoint{2.224793in}{2.457125in}}%
\pgfpathlineto{\pgfqpoint{2.225362in}{2.474381in}}%
\pgfpathlineto{\pgfqpoint{2.225551in}{2.473255in}}%
\pgfpathlineto{\pgfqpoint{2.225741in}{2.475513in}}%
\pgfpathlineto{\pgfqpoint{2.226404in}{2.501196in}}%
\pgfpathlineto{\pgfqpoint{2.227162in}{2.488154in}}%
\pgfpathlineto{\pgfqpoint{2.227541in}{2.476765in}}%
\pgfpathlineto{\pgfqpoint{2.227826in}{2.493828in}}%
\pgfpathlineto{\pgfqpoint{2.228584in}{2.506073in}}%
\pgfpathlineto{\pgfqpoint{2.229058in}{2.503534in}}%
\pgfpathlineto{\pgfqpoint{2.230479in}{2.489059in}}%
\pgfpathlineto{\pgfqpoint{2.230763in}{2.497096in}}%
\pgfpathlineto{\pgfqpoint{2.230858in}{2.497501in}}%
\pgfpathlineto{\pgfqpoint{2.230953in}{2.495300in}}%
\pgfpathlineto{\pgfqpoint{2.232280in}{2.476929in}}%
\pgfpathlineto{\pgfqpoint{2.231616in}{2.502228in}}%
\pgfpathlineto{\pgfqpoint{2.232374in}{2.479128in}}%
\pgfpathlineto{\pgfqpoint{2.232659in}{2.487027in}}%
\pgfpathlineto{\pgfqpoint{2.233133in}{2.473104in}}%
\pgfpathlineto{\pgfqpoint{2.233606in}{2.483739in}}%
\pgfpathlineto{\pgfqpoint{2.233796in}{2.481529in}}%
\pgfpathlineto{\pgfqpoint{2.237492in}{2.405386in}}%
\pgfpathlineto{\pgfqpoint{2.237776in}{2.411128in}}%
\pgfpathlineto{\pgfqpoint{2.237871in}{2.412032in}}%
\pgfpathlineto{\pgfqpoint{2.238060in}{2.405082in}}%
\pgfpathlineto{\pgfqpoint{2.238439in}{2.393886in}}%
\pgfpathlineto{\pgfqpoint{2.239008in}{2.408682in}}%
\pgfpathlineto{\pgfqpoint{2.239482in}{2.419310in}}%
\pgfpathlineto{\pgfqpoint{2.239766in}{2.405883in}}%
\pgfpathlineto{\pgfqpoint{2.240050in}{2.391187in}}%
\pgfpathlineto{\pgfqpoint{2.240524in}{2.411358in}}%
\pgfpathlineto{\pgfqpoint{2.240808in}{2.410032in}}%
\pgfpathlineto{\pgfqpoint{2.240903in}{2.410789in}}%
\pgfpathlineto{\pgfqpoint{2.241093in}{2.407948in}}%
\pgfpathlineto{\pgfqpoint{2.241377in}{2.396855in}}%
\pgfpathlineto{\pgfqpoint{2.241756in}{2.414160in}}%
\pgfpathlineto{\pgfqpoint{2.242135in}{2.407448in}}%
\pgfpathlineto{\pgfqpoint{2.243462in}{2.450378in}}%
\pgfpathlineto{\pgfqpoint{2.243936in}{2.445167in}}%
\pgfpathlineto{\pgfqpoint{2.244599in}{2.441668in}}%
\pgfpathlineto{\pgfqpoint{2.244978in}{2.475561in}}%
\pgfpathlineto{\pgfqpoint{2.245736in}{2.497724in}}%
\pgfpathlineto{\pgfqpoint{2.246305in}{2.497065in}}%
\pgfpathlineto{\pgfqpoint{2.246494in}{2.493600in}}%
\pgfpathlineto{\pgfqpoint{2.248010in}{2.428091in}}%
\pgfpathlineto{\pgfqpoint{2.248579in}{2.447320in}}%
\pgfpathlineto{\pgfqpoint{2.248958in}{2.460498in}}%
\pgfpathlineto{\pgfqpoint{2.249337in}{2.440640in}}%
\pgfpathlineto{\pgfqpoint{2.251043in}{2.403778in}}%
\pgfpathlineto{\pgfqpoint{2.251138in}{2.402598in}}%
\pgfpathlineto{\pgfqpoint{2.251327in}{2.413389in}}%
\pgfpathlineto{\pgfqpoint{2.252275in}{2.446963in}}%
\pgfpathlineto{\pgfqpoint{2.252559in}{2.429157in}}%
\pgfpathlineto{\pgfqpoint{2.253791in}{2.396857in}}%
\pgfpathlineto{\pgfqpoint{2.253886in}{2.397508in}}%
\pgfpathlineto{\pgfqpoint{2.255213in}{2.456863in}}%
\pgfpathlineto{\pgfqpoint{2.256350in}{2.669419in}}%
\pgfpathlineto{\pgfqpoint{2.257203in}{2.637871in}}%
\pgfpathlineto{\pgfqpoint{2.257866in}{2.449143in}}%
\pgfpathlineto{\pgfqpoint{2.258245in}{2.370655in}}%
\pgfpathlineto{\pgfqpoint{2.259003in}{2.430512in}}%
\pgfpathlineto{\pgfqpoint{2.259382in}{2.433881in}}%
\pgfpathlineto{\pgfqpoint{2.261467in}{2.489112in}}%
\pgfpathlineto{\pgfqpoint{2.261846in}{2.463659in}}%
\pgfpathlineto{\pgfqpoint{2.262036in}{2.456831in}}%
\pgfpathlineto{\pgfqpoint{2.262699in}{2.469949in}}%
\pgfpathlineto{\pgfqpoint{2.264405in}{2.523703in}}%
\pgfpathlineto{\pgfqpoint{2.264689in}{2.508309in}}%
\pgfpathlineto{\pgfqpoint{2.265163in}{2.489543in}}%
\pgfpathlineto{\pgfqpoint{2.265826in}{2.506403in}}%
\pgfpathlineto{\pgfqpoint{2.267532in}{2.556912in}}%
\pgfpathlineto{\pgfqpoint{2.267721in}{2.548430in}}%
\pgfpathlineto{\pgfqpoint{2.268100in}{2.535545in}}%
\pgfpathlineto{\pgfqpoint{2.268859in}{2.545412in}}%
\pgfpathlineto{\pgfqpoint{2.270754in}{2.590177in}}%
\pgfpathlineto{\pgfqpoint{2.271133in}{2.574232in}}%
\pgfpathlineto{\pgfqpoint{2.271322in}{2.577811in}}%
\pgfpathlineto{\pgfqpoint{2.273123in}{2.616692in}}%
\pgfpathlineto{\pgfqpoint{2.273218in}{2.615143in}}%
\pgfpathlineto{\pgfqpoint{2.274734in}{2.597027in}}%
\pgfpathlineto{\pgfqpoint{2.274829in}{2.597113in}}%
\pgfpathlineto{\pgfqpoint{2.275492in}{2.620077in}}%
\pgfpathlineto{\pgfqpoint{2.276061in}{2.602503in}}%
\pgfpathlineto{\pgfqpoint{2.276155in}{2.601127in}}%
\pgfpathlineto{\pgfqpoint{2.276534in}{2.609762in}}%
\pgfpathlineto{\pgfqpoint{2.276819in}{2.607098in}}%
\pgfpathlineto{\pgfqpoint{2.277956in}{2.617804in}}%
\pgfpathlineto{\pgfqpoint{2.278145in}{2.612853in}}%
\pgfpathlineto{\pgfqpoint{2.279851in}{2.584450in}}%
\pgfpathlineto{\pgfqpoint{2.279946in}{2.585265in}}%
\pgfpathlineto{\pgfqpoint{2.280325in}{2.588827in}}%
\pgfpathlineto{\pgfqpoint{2.280609in}{2.583168in}}%
\pgfpathlineto{\pgfqpoint{2.282220in}{2.558014in}}%
\pgfpathlineto{\pgfqpoint{2.281273in}{2.587774in}}%
\pgfpathlineto{\pgfqpoint{2.282410in}{2.562671in}}%
\pgfpathlineto{\pgfqpoint{2.283168in}{2.567601in}}%
\pgfpathlineto{\pgfqpoint{2.282884in}{2.559627in}}%
\pgfpathlineto{\pgfqpoint{2.283452in}{2.561473in}}%
\pgfpathlineto{\pgfqpoint{2.284116in}{2.558304in}}%
\pgfpathlineto{\pgfqpoint{2.284400in}{2.564184in}}%
\pgfpathlineto{\pgfqpoint{2.284495in}{2.565930in}}%
\pgfpathlineto{\pgfqpoint{2.284968in}{2.557362in}}%
\pgfpathlineto{\pgfqpoint{2.285253in}{2.547225in}}%
\pgfpathlineto{\pgfqpoint{2.285632in}{2.559254in}}%
\pgfpathlineto{\pgfqpoint{2.286106in}{2.550019in}}%
\pgfpathlineto{\pgfqpoint{2.287622in}{2.580103in}}%
\pgfpathlineto{\pgfqpoint{2.287906in}{2.567717in}}%
\pgfpathlineto{\pgfqpoint{2.288380in}{2.569245in}}%
\pgfpathlineto{\pgfqpoint{2.289138in}{2.562048in}}%
\pgfpathlineto{\pgfqpoint{2.289517in}{2.553660in}}%
\pgfpathlineto{\pgfqpoint{2.291223in}{2.467880in}}%
\pgfpathlineto{\pgfqpoint{2.291886in}{2.483443in}}%
\pgfpathlineto{\pgfqpoint{2.292644in}{2.549551in}}%
\pgfpathlineto{\pgfqpoint{2.293402in}{2.533740in}}%
\pgfpathlineto{\pgfqpoint{2.294634in}{2.498042in}}%
\pgfpathlineto{\pgfqpoint{2.294824in}{2.503948in}}%
\pgfpathlineto{\pgfqpoint{2.295393in}{2.538314in}}%
\pgfpathlineto{\pgfqpoint{2.296151in}{2.520410in}}%
\pgfpathlineto{\pgfqpoint{2.297383in}{2.496658in}}%
\pgfpathlineto{\pgfqpoint{2.297477in}{2.497975in}}%
\pgfpathlineto{\pgfqpoint{2.298709in}{2.535200in}}%
\pgfpathlineto{\pgfqpoint{2.300320in}{2.787981in}}%
\pgfpathlineto{\pgfqpoint{2.300984in}{2.713290in}}%
\pgfpathlineto{\pgfqpoint{2.301931in}{2.471284in}}%
\pgfpathlineto{\pgfqpoint{2.303447in}{2.537070in}}%
\pgfpathlineto{\pgfqpoint{2.303637in}{2.534941in}}%
\pgfpathlineto{\pgfqpoint{2.304206in}{2.543436in}}%
\pgfpathlineto{\pgfqpoint{2.305248in}{2.553521in}}%
\pgfpathlineto{\pgfqpoint{2.305532in}{2.547036in}}%
\pgfpathlineto{\pgfqpoint{2.306006in}{2.508134in}}%
\pgfpathlineto{\pgfqpoint{2.306859in}{2.525199in}}%
\pgfpathlineto{\pgfqpoint{2.307049in}{2.523657in}}%
\pgfpathlineto{\pgfqpoint{2.307238in}{2.527905in}}%
\pgfpathlineto{\pgfqpoint{2.307901in}{2.551010in}}%
\pgfpathlineto{\pgfqpoint{2.308470in}{2.536618in}}%
\pgfpathlineto{\pgfqpoint{2.308944in}{2.514317in}}%
\pgfpathlineto{\pgfqpoint{2.309702in}{2.529149in}}%
\pgfpathlineto{\pgfqpoint{2.311029in}{2.553845in}}%
\pgfpathlineto{\pgfqpoint{2.312450in}{2.510987in}}%
\pgfpathlineto{\pgfqpoint{2.312924in}{2.525862in}}%
\pgfpathlineto{\pgfqpoint{2.313208in}{2.521140in}}%
\pgfpathlineto{\pgfqpoint{2.313587in}{2.527760in}}%
\pgfpathlineto{\pgfqpoint{2.313872in}{2.525615in}}%
\pgfpathlineto{\pgfqpoint{2.314156in}{2.534212in}}%
\pgfpathlineto{\pgfqpoint{2.314440in}{2.520208in}}%
\pgfpathlineto{\pgfqpoint{2.315388in}{2.502565in}}%
\pgfpathlineto{\pgfqpoint{2.315577in}{2.506795in}}%
\pgfpathlineto{\pgfqpoint{2.315956in}{2.519367in}}%
\pgfpathlineto{\pgfqpoint{2.316809in}{2.513166in}}%
\pgfpathlineto{\pgfqpoint{2.317283in}{2.504915in}}%
\pgfpathlineto{\pgfqpoint{2.318420in}{2.483705in}}%
\pgfpathlineto{\pgfqpoint{2.318705in}{2.485804in}}%
\pgfpathlineto{\pgfqpoint{2.318799in}{2.485293in}}%
\pgfpathlineto{\pgfqpoint{2.319084in}{2.489392in}}%
\pgfpathlineto{\pgfqpoint{2.319273in}{2.493434in}}%
\pgfpathlineto{\pgfqpoint{2.319747in}{2.481959in}}%
\pgfpathlineto{\pgfqpoint{2.319936in}{2.482726in}}%
\pgfpathlineto{\pgfqpoint{2.320031in}{2.482625in}}%
\pgfpathlineto{\pgfqpoint{2.320126in}{2.483146in}}%
\pgfpathlineto{\pgfqpoint{2.320505in}{2.499899in}}%
\pgfpathlineto{\pgfqpoint{2.320884in}{2.476527in}}%
\pgfpathlineto{\pgfqpoint{2.321642in}{2.466923in}}%
\pgfpathlineto{\pgfqpoint{2.322021in}{2.472398in}}%
\pgfpathlineto{\pgfqpoint{2.326096in}{2.420582in}}%
\pgfpathlineto{\pgfqpoint{2.322685in}{2.472469in}}%
\pgfpathlineto{\pgfqpoint{2.326286in}{2.425585in}}%
\pgfpathlineto{\pgfqpoint{2.326854in}{2.438136in}}%
\pgfpathlineto{\pgfqpoint{2.327518in}{2.432119in}}%
\pgfpathlineto{\pgfqpoint{2.327707in}{2.430488in}}%
\pgfpathlineto{\pgfqpoint{2.328750in}{2.406045in}}%
\pgfpathlineto{\pgfqpoint{2.329318in}{2.413174in}}%
\pgfpathlineto{\pgfqpoint{2.333109in}{2.465132in}}%
\pgfpathlineto{\pgfqpoint{2.333393in}{2.450471in}}%
\pgfpathlineto{\pgfqpoint{2.335288in}{2.382471in}}%
\pgfpathlineto{\pgfqpoint{2.335573in}{2.388171in}}%
\pgfpathlineto{\pgfqpoint{2.336331in}{2.407857in}}%
\pgfpathlineto{\pgfqpoint{2.336710in}{2.392262in}}%
\pgfpathlineto{\pgfqpoint{2.338321in}{2.368359in}}%
\pgfpathlineto{\pgfqpoint{2.338415in}{2.367788in}}%
\pgfpathlineto{\pgfqpoint{2.338510in}{2.370163in}}%
\pgfpathlineto{\pgfqpoint{2.339268in}{2.406749in}}%
\pgfpathlineto{\pgfqpoint{2.339742in}{2.383437in}}%
\pgfpathlineto{\pgfqpoint{2.340500in}{2.366063in}}%
\pgfpathlineto{\pgfqpoint{2.340974in}{2.377593in}}%
\pgfpathlineto{\pgfqpoint{2.341353in}{2.375060in}}%
\pgfpathlineto{\pgfqpoint{2.341637in}{2.378498in}}%
\pgfpathlineto{\pgfqpoint{2.342111in}{2.372115in}}%
\pgfpathlineto{\pgfqpoint{2.342490in}{2.415320in}}%
\pgfpathlineto{\pgfqpoint{2.343817in}{2.630301in}}%
\pgfpathlineto{\pgfqpoint{2.344386in}{2.600710in}}%
\pgfpathlineto{\pgfqpoint{2.344670in}{2.579243in}}%
\pgfpathlineto{\pgfqpoint{2.345523in}{2.312824in}}%
\pgfpathlineto{\pgfqpoint{2.346565in}{2.387433in}}%
\pgfpathlineto{\pgfqpoint{2.346660in}{2.385834in}}%
\pgfpathlineto{\pgfqpoint{2.346944in}{2.395685in}}%
\pgfpathlineto{\pgfqpoint{2.348555in}{2.438334in}}%
\pgfpathlineto{\pgfqpoint{2.348839in}{2.427920in}}%
\pgfpathlineto{\pgfqpoint{2.349692in}{2.393117in}}%
\pgfpathlineto{\pgfqpoint{2.350166in}{2.405629in}}%
\pgfpathlineto{\pgfqpoint{2.351493in}{2.438700in}}%
\pgfpathlineto{\pgfqpoint{2.351967in}{2.426660in}}%
\pgfpathlineto{\pgfqpoint{2.352630in}{2.406824in}}%
\pgfpathlineto{\pgfqpoint{2.353199in}{2.421186in}}%
\pgfpathlineto{\pgfqpoint{2.354810in}{2.444982in}}%
\pgfpathlineto{\pgfqpoint{2.355283in}{2.424510in}}%
\pgfpathlineto{\pgfqpoint{2.355852in}{2.416506in}}%
\pgfpathlineto{\pgfqpoint{2.356231in}{2.426795in}}%
\pgfpathlineto{\pgfqpoint{2.356989in}{2.453247in}}%
\pgfpathlineto{\pgfqpoint{2.358221in}{2.445070in}}%
\pgfpathlineto{\pgfqpoint{2.358600in}{2.421703in}}%
\pgfpathlineto{\pgfqpoint{2.359358in}{2.440074in}}%
\pgfpathlineto{\pgfqpoint{2.359832in}{2.442260in}}%
\pgfpathlineto{\pgfqpoint{2.360022in}{2.439682in}}%
\pgfpathlineto{\pgfqpoint{2.360401in}{2.424416in}}%
\pgfpathlineto{\pgfqpoint{2.360969in}{2.442627in}}%
\pgfpathlineto{\pgfqpoint{2.361254in}{2.443833in}}%
\pgfpathlineto{\pgfqpoint{2.361348in}{2.443107in}}%
\pgfpathlineto{\pgfqpoint{2.361822in}{2.423506in}}%
\pgfpathlineto{\pgfqpoint{2.362580in}{2.433306in}}%
\pgfpathlineto{\pgfqpoint{2.362865in}{2.437793in}}%
\pgfpathlineto{\pgfqpoint{2.363433in}{2.429438in}}%
\pgfpathlineto{\pgfqpoint{2.364570in}{2.423389in}}%
\pgfpathlineto{\pgfqpoint{2.364286in}{2.432778in}}%
\pgfpathlineto{\pgfqpoint{2.364665in}{2.423598in}}%
\pgfpathlineto{\pgfqpoint{2.365044in}{2.440743in}}%
\pgfpathlineto{\pgfqpoint{2.365423in}{2.417755in}}%
\pgfpathlineto{\pgfqpoint{2.365708in}{2.422372in}}%
\pgfpathlineto{\pgfqpoint{2.365897in}{2.417862in}}%
\pgfpathlineto{\pgfqpoint{2.369119in}{2.295287in}}%
\pgfpathlineto{\pgfqpoint{2.369214in}{2.296185in}}%
\pgfpathlineto{\pgfqpoint{2.370067in}{2.340546in}}%
\pgfpathlineto{\pgfqpoint{2.370920in}{2.379777in}}%
\pgfpathlineto{\pgfqpoint{2.372057in}{2.376214in}}%
\pgfpathlineto{\pgfqpoint{2.372910in}{2.356257in}}%
\pgfpathlineto{\pgfqpoint{2.373478in}{2.369617in}}%
\pgfpathlineto{\pgfqpoint{2.373573in}{2.369400in}}%
\pgfpathlineto{\pgfqpoint{2.373668in}{2.370899in}}%
\pgfpathlineto{\pgfqpoint{2.374900in}{2.399171in}}%
\pgfpathlineto{\pgfqpoint{2.375089in}{2.391306in}}%
\pgfpathlineto{\pgfqpoint{2.375373in}{2.379394in}}%
\pgfpathlineto{\pgfqpoint{2.376037in}{2.394539in}}%
\pgfpathlineto{\pgfqpoint{2.377458in}{2.419024in}}%
\pgfpathlineto{\pgfqpoint{2.377553in}{2.416611in}}%
\pgfpathlineto{\pgfqpoint{2.379164in}{2.343857in}}%
\pgfpathlineto{\pgfqpoint{2.379448in}{2.351567in}}%
\pgfpathlineto{\pgfqpoint{2.379922in}{2.367003in}}%
\pgfpathlineto{\pgfqpoint{2.380301in}{2.348414in}}%
\pgfpathlineto{\pgfqpoint{2.381154in}{2.331891in}}%
\pgfpathlineto{\pgfqpoint{2.381723in}{2.334700in}}%
\pgfpathlineto{\pgfqpoint{2.382007in}{2.330888in}}%
\pgfpathlineto{\pgfqpoint{2.382386in}{2.342529in}}%
\pgfpathlineto{\pgfqpoint{2.382955in}{2.373576in}}%
\pgfpathlineto{\pgfqpoint{2.383713in}{2.359585in}}%
\pgfpathlineto{\pgfqpoint{2.384566in}{2.339613in}}%
\pgfpathlineto{\pgfqpoint{2.384850in}{2.348385in}}%
\pgfpathlineto{\pgfqpoint{2.386177in}{2.398876in}}%
\pgfpathlineto{\pgfqpoint{2.387598in}{2.618652in}}%
\pgfpathlineto{\pgfqpoint{2.388261in}{2.583571in}}%
\pgfpathlineto{\pgfqpoint{2.388735in}{2.463272in}}%
\pgfpathlineto{\pgfqpoint{2.389209in}{2.314534in}}%
\pgfpathlineto{\pgfqpoint{2.389967in}{2.376021in}}%
\pgfpathlineto{\pgfqpoint{2.390062in}{2.376316in}}%
\pgfpathlineto{\pgfqpoint{2.390251in}{2.374990in}}%
\pgfpathlineto{\pgfqpoint{2.390441in}{2.371970in}}%
\pgfpathlineto{\pgfqpoint{2.390725in}{2.385005in}}%
\pgfpathlineto{\pgfqpoint{2.392052in}{2.428191in}}%
\pgfpathlineto{\pgfqpoint{2.392241in}{2.418766in}}%
\pgfpathlineto{\pgfqpoint{2.393189in}{2.385632in}}%
\pgfpathlineto{\pgfqpoint{2.393663in}{2.390463in}}%
\pgfpathlineto{\pgfqpoint{2.393758in}{2.388970in}}%
\pgfpathlineto{\pgfqpoint{2.394042in}{2.401450in}}%
\pgfpathlineto{\pgfqpoint{2.395463in}{2.433913in}}%
\pgfpathlineto{\pgfqpoint{2.395558in}{2.431129in}}%
\pgfpathlineto{\pgfqpoint{2.396316in}{2.398536in}}%
\pgfpathlineto{\pgfqpoint{2.396885in}{2.415722in}}%
\pgfpathlineto{\pgfqpoint{2.398401in}{2.446779in}}%
\pgfpathlineto{\pgfqpoint{2.398496in}{2.448048in}}%
\pgfpathlineto{\pgfqpoint{2.398875in}{2.439284in}}%
\pgfpathlineto{\pgfqpoint{2.399444in}{2.429866in}}%
\pgfpathlineto{\pgfqpoint{2.400012in}{2.438651in}}%
\pgfpathlineto{\pgfqpoint{2.400391in}{2.454213in}}%
\pgfpathlineto{\pgfqpoint{2.401434in}{2.469645in}}%
\pgfpathlineto{\pgfqpoint{2.401623in}{2.463801in}}%
\pgfpathlineto{\pgfqpoint{2.402666in}{2.454802in}}%
\pgfpathlineto{\pgfqpoint{2.402286in}{2.465480in}}%
\pgfpathlineto{\pgfqpoint{2.402760in}{2.456672in}}%
\pgfpathlineto{\pgfqpoint{2.404182in}{2.484888in}}%
\pgfpathlineto{\pgfqpoint{2.404371in}{2.481747in}}%
\pgfpathlineto{\pgfqpoint{2.405414in}{2.453126in}}%
\pgfpathlineto{\pgfqpoint{2.404845in}{2.486075in}}%
\pgfpathlineto{\pgfqpoint{2.405887in}{2.466027in}}%
\pgfpathlineto{\pgfqpoint{2.406740in}{2.474382in}}%
\pgfpathlineto{\pgfqpoint{2.407025in}{2.468027in}}%
\pgfpathlineto{\pgfqpoint{2.407878in}{2.456817in}}%
\pgfpathlineto{\pgfqpoint{2.407498in}{2.474656in}}%
\pgfpathlineto{\pgfqpoint{2.408067in}{2.462935in}}%
\pgfpathlineto{\pgfqpoint{2.408920in}{2.476980in}}%
\pgfpathlineto{\pgfqpoint{2.409204in}{2.466088in}}%
\pgfpathlineto{\pgfqpoint{2.410720in}{2.435171in}}%
\pgfpathlineto{\pgfqpoint{2.410815in}{2.435441in}}%
\pgfpathlineto{\pgfqpoint{2.410910in}{2.435830in}}%
\pgfpathlineto{\pgfqpoint{2.411194in}{2.432753in}}%
\pgfpathlineto{\pgfqpoint{2.413753in}{2.401722in}}%
\pgfpathlineto{\pgfqpoint{2.411763in}{2.437309in}}%
\pgfpathlineto{\pgfqpoint{2.413942in}{2.406697in}}%
\pgfpathlineto{\pgfqpoint{2.414227in}{2.414215in}}%
\pgfpathlineto{\pgfqpoint{2.414985in}{2.404187in}}%
\pgfpathlineto{\pgfqpoint{2.415174in}{2.396858in}}%
\pgfpathlineto{\pgfqpoint{2.415648in}{2.410026in}}%
\pgfpathlineto{\pgfqpoint{2.416122in}{2.402876in}}%
\pgfpathlineto{\pgfqpoint{2.416596in}{2.386447in}}%
\pgfpathlineto{\pgfqpoint{2.417164in}{2.403398in}}%
\pgfpathlineto{\pgfqpoint{2.418396in}{2.434951in}}%
\pgfpathlineto{\pgfqpoint{2.419060in}{2.432097in}}%
\pgfpathlineto{\pgfqpoint{2.419249in}{2.431452in}}%
\pgfpathlineto{\pgfqpoint{2.419344in}{2.431677in}}%
\pgfpathlineto{\pgfqpoint{2.420576in}{2.464937in}}%
\pgfpathlineto{\pgfqpoint{2.421239in}{2.452362in}}%
\pgfpathlineto{\pgfqpoint{2.422945in}{2.379461in}}%
\pgfpathlineto{\pgfqpoint{2.423893in}{2.407403in}}%
\pgfpathlineto{\pgfqpoint{2.423987in}{2.408056in}}%
\pgfpathlineto{\pgfqpoint{2.424082in}{2.403913in}}%
\pgfpathlineto{\pgfqpoint{2.425883in}{2.347604in}}%
\pgfpathlineto{\pgfqpoint{2.426451in}{2.367058in}}%
\pgfpathlineto{\pgfqpoint{2.427020in}{2.391322in}}%
\pgfpathlineto{\pgfqpoint{2.427588in}{2.374568in}}%
\pgfpathlineto{\pgfqpoint{2.429010in}{2.340061in}}%
\pgfpathlineto{\pgfqpoint{2.429199in}{2.344670in}}%
\pgfpathlineto{\pgfqpoint{2.430337in}{2.438628in}}%
\pgfpathlineto{\pgfqpoint{2.431095in}{2.589646in}}%
\pgfpathlineto{\pgfqpoint{2.431948in}{2.589037in}}%
\pgfpathlineto{\pgfqpoint{2.432137in}{2.584156in}}%
\pgfpathlineto{\pgfqpoint{2.432706in}{2.458460in}}%
\pgfpathlineto{\pgfqpoint{2.433180in}{2.308368in}}%
\pgfpathlineto{\pgfqpoint{2.433938in}{2.379382in}}%
\pgfpathlineto{\pgfqpoint{2.434222in}{2.389878in}}%
\pgfpathlineto{\pgfqpoint{2.435549in}{2.406400in}}%
\pgfpathlineto{\pgfqpoint{2.436402in}{2.432994in}}%
\pgfpathlineto{\pgfqpoint{2.436686in}{2.415499in}}%
\pgfpathlineto{\pgfqpoint{2.436970in}{2.394263in}}%
\pgfpathlineto{\pgfqpoint{2.437728in}{2.419493in}}%
\pgfpathlineto{\pgfqpoint{2.437918in}{2.421894in}}%
\pgfpathlineto{\pgfqpoint{2.438297in}{2.412558in}}%
\pgfpathlineto{\pgfqpoint{2.438392in}{2.412043in}}%
\pgfpathlineto{\pgfqpoint{2.438486in}{2.413966in}}%
\pgfpathlineto{\pgfqpoint{2.439339in}{2.440143in}}%
\pgfpathlineto{\pgfqpoint{2.439718in}{2.418687in}}%
\pgfpathlineto{\pgfqpoint{2.440666in}{2.404677in}}%
\pgfpathlineto{\pgfqpoint{2.440855in}{2.410424in}}%
\pgfpathlineto{\pgfqpoint{2.442561in}{2.456855in}}%
\pgfpathlineto{\pgfqpoint{2.443509in}{2.426348in}}%
\pgfpathlineto{\pgfqpoint{2.444362in}{2.435062in}}%
\pgfpathlineto{\pgfqpoint{2.445309in}{2.407062in}}%
\pgfpathlineto{\pgfqpoint{2.446352in}{2.378097in}}%
\pgfpathlineto{\pgfqpoint{2.446541in}{2.388787in}}%
\pgfpathlineto{\pgfqpoint{2.448342in}{2.487828in}}%
\pgfpathlineto{\pgfqpoint{2.448437in}{2.484892in}}%
\pgfpathlineto{\pgfqpoint{2.449763in}{2.454239in}}%
\pgfpathlineto{\pgfqpoint{2.449858in}{2.454908in}}%
\pgfpathlineto{\pgfqpoint{2.450995in}{2.474082in}}%
\pgfpathlineto{\pgfqpoint{2.451185in}{2.469980in}}%
\pgfpathlineto{\pgfqpoint{2.452132in}{2.457941in}}%
\pgfpathlineto{\pgfqpoint{2.451753in}{2.470081in}}%
\pgfpathlineto{\pgfqpoint{2.452417in}{2.463567in}}%
\pgfpathlineto{\pgfqpoint{2.452606in}{2.459687in}}%
\pgfpathlineto{\pgfqpoint{2.453743in}{2.434793in}}%
\pgfpathlineto{\pgfqpoint{2.454028in}{2.439877in}}%
\pgfpathlineto{\pgfqpoint{2.454122in}{2.440447in}}%
\pgfpathlineto{\pgfqpoint{2.454501in}{2.437090in}}%
\pgfpathlineto{\pgfqpoint{2.457060in}{2.407549in}}%
\pgfpathlineto{\pgfqpoint{2.457250in}{2.411953in}}%
\pgfpathlineto{\pgfqpoint{2.457439in}{2.416620in}}%
\pgfpathlineto{\pgfqpoint{2.458292in}{2.412012in}}%
\pgfpathlineto{\pgfqpoint{2.458955in}{2.409214in}}%
\pgfpathlineto{\pgfqpoint{2.458671in}{2.414221in}}%
\pgfpathlineto{\pgfqpoint{2.459050in}{2.410741in}}%
\pgfpathlineto{\pgfqpoint{2.459429in}{2.428551in}}%
\pgfpathlineto{\pgfqpoint{2.460282in}{2.419502in}}%
\pgfpathlineto{\pgfqpoint{2.460756in}{2.410111in}}%
\pgfpathlineto{\pgfqpoint{2.461135in}{2.423027in}}%
\pgfpathlineto{\pgfqpoint{2.461419in}{2.421666in}}%
\pgfpathlineto{\pgfqpoint{2.461514in}{2.423623in}}%
\pgfpathlineto{\pgfqpoint{2.462651in}{2.460054in}}%
\pgfpathlineto{\pgfqpoint{2.463125in}{2.453531in}}%
\pgfpathlineto{\pgfqpoint{2.463504in}{2.450353in}}%
\pgfpathlineto{\pgfqpoint{2.463788in}{2.455020in}}%
\pgfpathlineto{\pgfqpoint{2.464926in}{2.481274in}}%
\pgfpathlineto{\pgfqpoint{2.465210in}{2.466814in}}%
\pgfpathlineto{\pgfqpoint{2.465778in}{2.470908in}}%
\pgfpathlineto{\pgfqpoint{2.467105in}{2.390028in}}%
\pgfpathlineto{\pgfqpoint{2.467958in}{2.410543in}}%
\pgfpathlineto{\pgfqpoint{2.468432in}{2.395197in}}%
\pgfpathlineto{\pgfqpoint{2.469948in}{2.366916in}}%
\pgfpathlineto{\pgfqpoint{2.470990in}{2.402226in}}%
\pgfpathlineto{\pgfqpoint{2.471559in}{2.388148in}}%
\pgfpathlineto{\pgfqpoint{2.472033in}{2.368884in}}%
\pgfpathlineto{\pgfqpoint{2.472980in}{2.374756in}}%
\pgfpathlineto{\pgfqpoint{2.473454in}{2.372185in}}%
\pgfpathlineto{\pgfqpoint{2.474212in}{2.420734in}}%
\pgfpathlineto{\pgfqpoint{2.475444in}{2.617708in}}%
\pgfpathlineto{\pgfqpoint{2.476202in}{2.609046in}}%
\pgfpathlineto{\pgfqpoint{2.476582in}{2.550902in}}%
\pgfpathlineto{\pgfqpoint{2.477245in}{2.336676in}}%
\pgfpathlineto{\pgfqpoint{2.478098in}{2.405749in}}%
\pgfpathlineto{\pgfqpoint{2.478193in}{2.405302in}}%
\pgfpathlineto{\pgfqpoint{2.478287in}{2.408806in}}%
\pgfpathlineto{\pgfqpoint{2.480372in}{2.468991in}}%
\pgfpathlineto{\pgfqpoint{2.481414in}{2.428154in}}%
\pgfpathlineto{\pgfqpoint{2.481983in}{2.452297in}}%
\pgfpathlineto{\pgfqpoint{2.483025in}{2.471557in}}%
\pgfpathlineto{\pgfqpoint{2.483689in}{2.467777in}}%
\pgfpathlineto{\pgfqpoint{2.484163in}{2.444555in}}%
\pgfpathlineto{\pgfqpoint{2.485016in}{2.456032in}}%
\pgfpathlineto{\pgfqpoint{2.485395in}{2.472513in}}%
\pgfpathlineto{\pgfqpoint{2.486721in}{2.488565in}}%
\pgfpathlineto{\pgfqpoint{2.486816in}{2.489497in}}%
\pgfpathlineto{\pgfqpoint{2.487006in}{2.483173in}}%
\pgfpathlineto{\pgfqpoint{2.487385in}{2.462343in}}%
\pgfpathlineto{\pgfqpoint{2.488143in}{2.472909in}}%
\pgfpathlineto{\pgfqpoint{2.489280in}{2.491145in}}%
\pgfpathlineto{\pgfqpoint{2.488617in}{2.472487in}}%
\pgfpathlineto{\pgfqpoint{2.490038in}{2.488100in}}%
\pgfpathlineto{\pgfqpoint{2.490417in}{2.465359in}}%
\pgfpathlineto{\pgfqpoint{2.491270in}{2.477579in}}%
\pgfpathlineto{\pgfqpoint{2.491365in}{2.476034in}}%
\pgfpathlineto{\pgfqpoint{2.491839in}{2.485935in}}%
\pgfpathlineto{\pgfqpoint{2.492123in}{2.483352in}}%
\pgfpathlineto{\pgfqpoint{2.492407in}{2.488599in}}%
\pgfpathlineto{\pgfqpoint{2.492502in}{2.489090in}}%
\pgfpathlineto{\pgfqpoint{2.492597in}{2.487214in}}%
\pgfpathlineto{\pgfqpoint{2.493829in}{2.463318in}}%
\pgfpathlineto{\pgfqpoint{2.494018in}{2.469441in}}%
\pgfpathlineto{\pgfqpoint{2.495061in}{2.478441in}}%
\pgfpathlineto{\pgfqpoint{2.494587in}{2.460133in}}%
\pgfpathlineto{\pgfqpoint{2.495250in}{2.476046in}}%
\pgfpathlineto{\pgfqpoint{2.495724in}{2.478664in}}%
\pgfpathlineto{\pgfqpoint{2.498377in}{2.442007in}}%
\pgfpathlineto{\pgfqpoint{2.499988in}{2.418217in}}%
\pgfpathlineto{\pgfqpoint{2.500178in}{2.421190in}}%
\pgfpathlineto{\pgfqpoint{2.500367in}{2.413445in}}%
\pgfpathlineto{\pgfqpoint{2.501504in}{2.391482in}}%
\pgfpathlineto{\pgfqpoint{2.501694in}{2.391913in}}%
\pgfpathlineto{\pgfqpoint{2.502168in}{2.384401in}}%
\pgfpathlineto{\pgfqpoint{2.502547in}{2.393358in}}%
\pgfpathlineto{\pgfqpoint{2.503115in}{2.388439in}}%
\pgfpathlineto{\pgfqpoint{2.503684in}{2.400796in}}%
\pgfpathlineto{\pgfqpoint{2.504158in}{2.387924in}}%
\pgfpathlineto{\pgfqpoint{2.505200in}{2.374844in}}%
\pgfpathlineto{\pgfqpoint{2.505390in}{2.378088in}}%
\pgfpathlineto{\pgfqpoint{2.507569in}{2.423490in}}%
\pgfpathlineto{\pgfqpoint{2.507759in}{2.422429in}}%
\pgfpathlineto{\pgfqpoint{2.508043in}{2.419818in}}%
\pgfpathlineto{\pgfqpoint{2.508327in}{2.427094in}}%
\pgfpathlineto{\pgfqpoint{2.509275in}{2.451722in}}%
\pgfpathlineto{\pgfqpoint{2.509749in}{2.447255in}}%
\pgfpathlineto{\pgfqpoint{2.510602in}{2.427008in}}%
\pgfpathlineto{\pgfqpoint{2.511170in}{2.382952in}}%
\pgfpathlineto{\pgfqpoint{2.512118in}{2.391602in}}%
\pgfpathlineto{\pgfqpoint{2.512308in}{2.396389in}}%
\pgfpathlineto{\pgfqpoint{2.512781in}{2.379293in}}%
\pgfpathlineto{\pgfqpoint{2.513066in}{2.374343in}}%
\pgfpathlineto{\pgfqpoint{2.514013in}{2.361321in}}%
\pgfpathlineto{\pgfqpoint{2.513634in}{2.378293in}}%
\pgfpathlineto{\pgfqpoint{2.514203in}{2.368206in}}%
\pgfpathlineto{\pgfqpoint{2.515435in}{2.402537in}}%
\pgfpathlineto{\pgfqpoint{2.515719in}{2.390656in}}%
\pgfpathlineto{\pgfqpoint{2.517235in}{2.371993in}}%
\pgfpathlineto{\pgfqpoint{2.518467in}{2.421643in}}%
\pgfpathlineto{\pgfqpoint{2.519510in}{2.620477in}}%
\pgfpathlineto{\pgfqpoint{2.520457in}{2.600563in}}%
\pgfpathlineto{\pgfqpoint{2.521026in}{2.482325in}}%
\pgfpathlineto{\pgfqpoint{2.521500in}{2.328797in}}%
\pgfpathlineto{\pgfqpoint{2.522258in}{2.406715in}}%
\pgfpathlineto{\pgfqpoint{2.522353in}{2.407037in}}%
\pgfpathlineto{\pgfqpoint{2.523774in}{2.380459in}}%
\pgfpathlineto{\pgfqpoint{2.523964in}{2.384104in}}%
\pgfpathlineto{\pgfqpoint{2.524248in}{2.392640in}}%
\pgfpathlineto{\pgfqpoint{2.524722in}{2.377528in}}%
\pgfpathlineto{\pgfqpoint{2.525575in}{2.334367in}}%
\pgfpathlineto{\pgfqpoint{2.526048in}{2.359937in}}%
\pgfpathlineto{\pgfqpoint{2.527944in}{2.457482in}}%
\pgfpathlineto{\pgfqpoint{2.528133in}{2.450242in}}%
\pgfpathlineto{\pgfqpoint{2.528512in}{2.419692in}}%
\pgfpathlineto{\pgfqpoint{2.529365in}{2.431683in}}%
\pgfpathlineto{\pgfqpoint{2.529555in}{2.430051in}}%
\pgfpathlineto{\pgfqpoint{2.529744in}{2.436734in}}%
\pgfpathlineto{\pgfqpoint{2.531355in}{2.469935in}}%
\pgfpathlineto{\pgfqpoint{2.531829in}{2.448182in}}%
\pgfpathlineto{\pgfqpoint{2.532492in}{2.461717in}}%
\pgfpathlineto{\pgfqpoint{2.534103in}{2.491713in}}%
\pgfpathlineto{\pgfqpoint{2.533156in}{2.461126in}}%
\pgfpathlineto{\pgfqpoint{2.534198in}{2.490580in}}%
\pgfpathlineto{\pgfqpoint{2.535146in}{2.467880in}}%
\pgfpathlineto{\pgfqpoint{2.535430in}{2.476169in}}%
\pgfpathlineto{\pgfqpoint{2.535809in}{2.482541in}}%
\pgfpathlineto{\pgfqpoint{2.536567in}{2.479587in}}%
\pgfpathlineto{\pgfqpoint{2.536757in}{2.481030in}}%
\pgfpathlineto{\pgfqpoint{2.537041in}{2.475160in}}%
\pgfpathlineto{\pgfqpoint{2.538273in}{2.456496in}}%
\pgfpathlineto{\pgfqpoint{2.538368in}{2.457954in}}%
\pgfpathlineto{\pgfqpoint{2.539410in}{2.478055in}}%
\pgfpathlineto{\pgfqpoint{2.539694in}{2.471637in}}%
\pgfpathlineto{\pgfqpoint{2.539789in}{2.471880in}}%
\pgfpathlineto{\pgfqpoint{2.540073in}{2.486684in}}%
\pgfpathlineto{\pgfqpoint{2.540453in}{2.470454in}}%
\pgfpathlineto{\pgfqpoint{2.540926in}{2.481397in}}%
\pgfpathlineto{\pgfqpoint{2.542443in}{2.463115in}}%
\pgfpathlineto{\pgfqpoint{2.543864in}{2.441459in}}%
\pgfpathlineto{\pgfqpoint{2.543959in}{2.444416in}}%
\pgfpathlineto{\pgfqpoint{2.544148in}{2.451858in}}%
\pgfpathlineto{\pgfqpoint{2.544527in}{2.427589in}}%
\pgfpathlineto{\pgfqpoint{2.545570in}{2.406628in}}%
\pgfpathlineto{\pgfqpoint{2.545854in}{2.409977in}}%
\pgfpathlineto{\pgfqpoint{2.546423in}{2.383611in}}%
\pgfpathlineto{\pgfqpoint{2.546896in}{2.410630in}}%
\pgfpathlineto{\pgfqpoint{2.548128in}{2.389711in}}%
\pgfpathlineto{\pgfqpoint{2.548887in}{2.380497in}}%
\pgfpathlineto{\pgfqpoint{2.549266in}{2.386232in}}%
\pgfpathlineto{\pgfqpoint{2.550308in}{2.395170in}}%
\pgfpathlineto{\pgfqpoint{2.549834in}{2.379486in}}%
\pgfpathlineto{\pgfqpoint{2.550498in}{2.392035in}}%
\pgfpathlineto{\pgfqpoint{2.550592in}{2.389906in}}%
\pgfpathlineto{\pgfqpoint{2.550971in}{2.402481in}}%
\pgfpathlineto{\pgfqpoint{2.551540in}{2.414683in}}%
\pgfpathlineto{\pgfqpoint{2.552014in}{2.401487in}}%
\pgfpathlineto{\pgfqpoint{2.552393in}{2.411612in}}%
\pgfpathlineto{\pgfqpoint{2.553340in}{2.442577in}}%
\pgfpathlineto{\pgfqpoint{2.553814in}{2.423385in}}%
\pgfpathlineto{\pgfqpoint{2.555330in}{2.369065in}}%
\pgfpathlineto{\pgfqpoint{2.554193in}{2.425191in}}%
\pgfpathlineto{\pgfqpoint{2.556183in}{2.387340in}}%
\pgfpathlineto{\pgfqpoint{2.556373in}{2.397948in}}%
\pgfpathlineto{\pgfqpoint{2.557036in}{2.370868in}}%
\pgfpathlineto{\pgfqpoint{2.557415in}{2.358876in}}%
\pgfpathlineto{\pgfqpoint{2.558268in}{2.367011in}}%
\pgfpathlineto{\pgfqpoint{2.558552in}{2.364799in}}%
\pgfpathlineto{\pgfqpoint{2.558837in}{2.374145in}}%
\pgfpathlineto{\pgfqpoint{2.559690in}{2.407547in}}%
\pgfpathlineto{\pgfqpoint{2.560069in}{2.391215in}}%
\pgfpathlineto{\pgfqpoint{2.560922in}{2.385943in}}%
\pgfpathlineto{\pgfqpoint{2.560543in}{2.395106in}}%
\pgfpathlineto{\pgfqpoint{2.561111in}{2.390732in}}%
\pgfpathlineto{\pgfqpoint{2.562912in}{2.482933in}}%
\pgfpathlineto{\pgfqpoint{2.563575in}{2.621213in}}%
\pgfpathlineto{\pgfqpoint{2.564428in}{2.615381in}}%
\pgfpathlineto{\pgfqpoint{2.564617in}{2.614090in}}%
\pgfpathlineto{\pgfqpoint{2.564996in}{2.565550in}}%
\pgfpathlineto{\pgfqpoint{2.565755in}{2.335073in}}%
\pgfpathlineto{\pgfqpoint{2.566607in}{2.403023in}}%
\pgfpathlineto{\pgfqpoint{2.566797in}{2.407983in}}%
\pgfpathlineto{\pgfqpoint{2.568218in}{2.432688in}}%
\pgfpathlineto{\pgfqpoint{2.568787in}{2.450918in}}%
\pgfpathlineto{\pgfqpoint{2.569356in}{2.438178in}}%
\pgfpathlineto{\pgfqpoint{2.569829in}{2.415383in}}%
\pgfpathlineto{\pgfqpoint{2.570303in}{2.440444in}}%
\pgfpathlineto{\pgfqpoint{2.570493in}{2.434365in}}%
\pgfpathlineto{\pgfqpoint{2.570682in}{2.432522in}}%
\pgfpathlineto{\pgfqpoint{2.571061in}{2.440518in}}%
\pgfpathlineto{\pgfqpoint{2.572009in}{2.461659in}}%
\pgfpathlineto{\pgfqpoint{2.572293in}{2.448573in}}%
\pgfpathlineto{\pgfqpoint{2.572957in}{2.430138in}}%
\pgfpathlineto{\pgfqpoint{2.573336in}{2.440887in}}%
\pgfpathlineto{\pgfqpoint{2.574757in}{2.482805in}}%
\pgfpathlineto{\pgfqpoint{2.574947in}{2.480186in}}%
\pgfpathlineto{\pgfqpoint{2.575894in}{2.462802in}}%
\pgfpathlineto{\pgfqpoint{2.576273in}{2.473003in}}%
\pgfpathlineto{\pgfqpoint{2.578074in}{2.509835in}}%
\pgfpathlineto{\pgfqpoint{2.578358in}{2.503310in}}%
\pgfpathlineto{\pgfqpoint{2.579306in}{2.493145in}}%
\pgfpathlineto{\pgfqpoint{2.579495in}{2.500483in}}%
\pgfpathlineto{\pgfqpoint{2.579780in}{2.513428in}}%
\pgfpathlineto{\pgfqpoint{2.580538in}{2.498636in}}%
\pgfpathlineto{\pgfqpoint{2.581296in}{2.488815in}}%
\pgfpathlineto{\pgfqpoint{2.581580in}{2.501327in}}%
\pgfpathlineto{\pgfqpoint{2.581675in}{2.503599in}}%
\pgfpathlineto{\pgfqpoint{2.581959in}{2.489664in}}%
\pgfpathlineto{\pgfqpoint{2.582149in}{2.485087in}}%
\pgfpathlineto{\pgfqpoint{2.582812in}{2.496373in}}%
\pgfpathlineto{\pgfqpoint{2.583665in}{2.511219in}}%
\pgfpathlineto{\pgfqpoint{2.583854in}{2.501739in}}%
\pgfpathlineto{\pgfqpoint{2.584139in}{2.488507in}}%
\pgfpathlineto{\pgfqpoint{2.584518in}{2.507044in}}%
\pgfpathlineto{\pgfqpoint{2.584992in}{2.499174in}}%
\pgfpathlineto{\pgfqpoint{2.585181in}{2.507393in}}%
\pgfpathlineto{\pgfqpoint{2.585750in}{2.484248in}}%
\pgfpathlineto{\pgfqpoint{2.586224in}{2.476236in}}%
\pgfpathlineto{\pgfqpoint{2.586887in}{2.483542in}}%
\pgfpathlineto{\pgfqpoint{2.587171in}{2.481686in}}%
\pgfpathlineto{\pgfqpoint{2.589730in}{2.433613in}}%
\pgfpathlineto{\pgfqpoint{2.590393in}{2.441910in}}%
\pgfpathlineto{\pgfqpoint{2.591436in}{2.459915in}}%
\pgfpathlineto{\pgfqpoint{2.591625in}{2.455198in}}%
\pgfpathlineto{\pgfqpoint{2.592478in}{2.444054in}}%
\pgfpathlineto{\pgfqpoint{2.592857in}{2.449057in}}%
\pgfpathlineto{\pgfqpoint{2.593331in}{2.431399in}}%
\pgfpathlineto{\pgfqpoint{2.593710in}{2.452831in}}%
\pgfpathlineto{\pgfqpoint{2.594279in}{2.441800in}}%
\pgfpathlineto{\pgfqpoint{2.595416in}{2.474480in}}%
\pgfpathlineto{\pgfqpoint{2.595605in}{2.468484in}}%
\pgfpathlineto{\pgfqpoint{2.595700in}{2.466566in}}%
\pgfpathlineto{\pgfqpoint{2.595984in}{2.477789in}}%
\pgfpathlineto{\pgfqpoint{2.597785in}{2.517690in}}%
\pgfpathlineto{\pgfqpoint{2.596553in}{2.472921in}}%
\pgfpathlineto{\pgfqpoint{2.597880in}{2.516470in}}%
\pgfpathlineto{\pgfqpoint{2.602807in}{2.354188in}}%
\pgfpathlineto{\pgfqpoint{2.603186in}{2.377130in}}%
\pgfpathlineto{\pgfqpoint{2.604229in}{2.470869in}}%
\pgfpathlineto{\pgfqpoint{2.604797in}{2.454657in}}%
\pgfpathlineto{\pgfqpoint{2.605082in}{2.464806in}}%
\pgfpathlineto{\pgfqpoint{2.605366in}{2.474474in}}%
\pgfpathlineto{\pgfqpoint{2.605935in}{2.456825in}}%
\pgfpathlineto{\pgfqpoint{2.606314in}{2.441359in}}%
\pgfpathlineto{\pgfqpoint{2.606693in}{2.469385in}}%
\pgfpathlineto{\pgfqpoint{2.607261in}{2.526526in}}%
\pgfpathlineto{\pgfqpoint{2.608304in}{2.709547in}}%
\pgfpathlineto{\pgfqpoint{2.609062in}{2.689336in}}%
\pgfpathlineto{\pgfqpoint{2.609346in}{2.663103in}}%
\pgfpathlineto{\pgfqpoint{2.610199in}{2.391377in}}%
\pgfpathlineto{\pgfqpoint{2.611147in}{2.456242in}}%
\pgfpathlineto{\pgfqpoint{2.613137in}{2.500827in}}%
\pgfpathlineto{\pgfqpoint{2.613421in}{2.491524in}}%
\pgfpathlineto{\pgfqpoint{2.613989in}{2.466696in}}%
\pgfpathlineto{\pgfqpoint{2.614748in}{2.484132in}}%
\pgfpathlineto{\pgfqpoint{2.615695in}{2.500458in}}%
\pgfpathlineto{\pgfqpoint{2.615221in}{2.482314in}}%
\pgfpathlineto{\pgfqpoint{2.615980in}{2.497897in}}%
\pgfpathlineto{\pgfqpoint{2.616359in}{2.515604in}}%
\pgfpathlineto{\pgfqpoint{2.616927in}{2.489623in}}%
\pgfpathlineto{\pgfqpoint{2.617117in}{2.484981in}}%
\pgfpathlineto{\pgfqpoint{2.617591in}{2.499393in}}%
\pgfpathlineto{\pgfqpoint{2.617685in}{2.499386in}}%
\pgfpathlineto{\pgfqpoint{2.617780in}{2.499212in}}%
\pgfpathlineto{\pgfqpoint{2.617875in}{2.500597in}}%
\pgfpathlineto{\pgfqpoint{2.619486in}{2.537550in}}%
\pgfpathlineto{\pgfqpoint{2.619581in}{2.534216in}}%
\pgfpathlineto{\pgfqpoint{2.620433in}{2.511940in}}%
\pgfpathlineto{\pgfqpoint{2.620718in}{2.522409in}}%
\pgfpathlineto{\pgfqpoint{2.621476in}{2.545369in}}%
\pgfpathlineto{\pgfqpoint{2.622234in}{2.544908in}}%
\pgfpathlineto{\pgfqpoint{2.623371in}{2.518086in}}%
\pgfpathlineto{\pgfqpoint{2.623845in}{2.537177in}}%
\pgfpathlineto{\pgfqpoint{2.624414in}{2.538548in}}%
\pgfpathlineto{\pgfqpoint{2.624129in}{2.536199in}}%
\pgfpathlineto{\pgfqpoint{2.624508in}{2.537392in}}%
\pgfpathlineto{\pgfqpoint{2.625077in}{2.520644in}}%
\pgfpathlineto{\pgfqpoint{2.626025in}{2.523447in}}%
\pgfpathlineto{\pgfqpoint{2.626214in}{2.529092in}}%
\pgfpathlineto{\pgfqpoint{2.626688in}{2.517714in}}%
\pgfpathlineto{\pgfqpoint{2.627067in}{2.524414in}}%
\pgfpathlineto{\pgfqpoint{2.627351in}{2.514015in}}%
\pgfpathlineto{\pgfqpoint{2.627730in}{2.525370in}}%
\pgfpathlineto{\pgfqpoint{2.628299in}{2.517189in}}%
\pgfpathlineto{\pgfqpoint{2.629246in}{2.521938in}}%
\pgfpathlineto{\pgfqpoint{2.628773in}{2.514431in}}%
\pgfpathlineto{\pgfqpoint{2.629341in}{2.520417in}}%
\pgfpathlineto{\pgfqpoint{2.631331in}{2.476965in}}%
\pgfpathlineto{\pgfqpoint{2.631616in}{2.480913in}}%
\pgfpathlineto{\pgfqpoint{2.631900in}{2.473745in}}%
\pgfpathlineto{\pgfqpoint{2.633511in}{2.441911in}}%
\pgfpathlineto{\pgfqpoint{2.633700in}{2.445682in}}%
\pgfpathlineto{\pgfqpoint{2.633890in}{2.450816in}}%
\pgfpathlineto{\pgfqpoint{2.634364in}{2.437946in}}%
\pgfpathlineto{\pgfqpoint{2.634743in}{2.423063in}}%
\pgfpathlineto{\pgfqpoint{2.635596in}{2.429608in}}%
\pgfpathlineto{\pgfqpoint{2.635880in}{2.436766in}}%
\pgfpathlineto{\pgfqpoint{2.636449in}{2.426557in}}%
\pgfpathlineto{\pgfqpoint{2.636733in}{2.430401in}}%
\pgfpathlineto{\pgfqpoint{2.637112in}{2.420385in}}%
\pgfpathlineto{\pgfqpoint{2.637491in}{2.432868in}}%
\pgfpathlineto{\pgfqpoint{2.637681in}{2.429151in}}%
\pgfpathlineto{\pgfqpoint{2.637870in}{2.421751in}}%
\pgfpathlineto{\pgfqpoint{2.638249in}{2.429908in}}%
\pgfpathlineto{\pgfqpoint{2.638723in}{2.426774in}}%
\pgfpathlineto{\pgfqpoint{2.640334in}{2.459152in}}%
\pgfpathlineto{\pgfqpoint{2.640429in}{2.457712in}}%
\pgfpathlineto{\pgfqpoint{2.640808in}{2.446927in}}%
\pgfpathlineto{\pgfqpoint{2.641187in}{2.465047in}}%
\pgfpathlineto{\pgfqpoint{2.642324in}{2.490482in}}%
\pgfpathlineto{\pgfqpoint{2.642608in}{2.483397in}}%
\pgfpathlineto{\pgfqpoint{2.644598in}{2.407608in}}%
\pgfpathlineto{\pgfqpoint{2.645072in}{2.425150in}}%
\pgfpathlineto{\pgfqpoint{2.645451in}{2.436457in}}%
\pgfpathlineto{\pgfqpoint{2.645735in}{2.415018in}}%
\pgfpathlineto{\pgfqpoint{2.647062in}{2.391334in}}%
\pgfpathlineto{\pgfqpoint{2.647346in}{2.383055in}}%
\pgfpathlineto{\pgfqpoint{2.647915in}{2.399790in}}%
\pgfpathlineto{\pgfqpoint{2.648484in}{2.417814in}}%
\pgfpathlineto{\pgfqpoint{2.648957in}{2.401966in}}%
\pgfpathlineto{\pgfqpoint{2.650379in}{2.374729in}}%
\pgfpathlineto{\pgfqpoint{2.650758in}{2.381692in}}%
\pgfpathlineto{\pgfqpoint{2.651800in}{2.468524in}}%
\pgfpathlineto{\pgfqpoint{2.652748in}{2.628133in}}%
\pgfpathlineto{\pgfqpoint{2.653222in}{2.608782in}}%
\pgfpathlineto{\pgfqpoint{2.653601in}{2.616208in}}%
\pgfpathlineto{\pgfqpoint{2.653790in}{2.608122in}}%
\pgfpathlineto{\pgfqpoint{2.654738in}{2.330955in}}%
\pgfpathlineto{\pgfqpoint{2.656065in}{2.433082in}}%
\pgfpathlineto{\pgfqpoint{2.656160in}{2.433879in}}%
\pgfpathlineto{\pgfqpoint{2.656539in}{2.428422in}}%
\pgfpathlineto{\pgfqpoint{2.656918in}{2.421803in}}%
\pgfpathlineto{\pgfqpoint{2.657107in}{2.425066in}}%
\pgfpathlineto{\pgfqpoint{2.657676in}{2.459194in}}%
\pgfpathlineto{\pgfqpoint{2.658244in}{2.429372in}}%
\pgfpathlineto{\pgfqpoint{2.658813in}{2.415355in}}%
\pgfpathlineto{\pgfqpoint{2.659192in}{2.432395in}}%
\pgfpathlineto{\pgfqpoint{2.660898in}{2.471947in}}%
\pgfpathlineto{\pgfqpoint{2.661182in}{2.466127in}}%
\pgfpathlineto{\pgfqpoint{2.662224in}{2.445168in}}%
\pgfpathlineto{\pgfqpoint{2.662414in}{2.449605in}}%
\pgfpathlineto{\pgfqpoint{2.662888in}{2.471447in}}%
\pgfpathlineto{\pgfqpoint{2.663741in}{2.468512in}}%
\pgfpathlineto{\pgfqpoint{2.663835in}{2.468662in}}%
\pgfpathlineto{\pgfqpoint{2.664878in}{2.449983in}}%
\pgfpathlineto{\pgfqpoint{2.665162in}{2.460565in}}%
\pgfpathlineto{\pgfqpoint{2.666110in}{2.485394in}}%
\pgfpathlineto{\pgfqpoint{2.666489in}{2.470827in}}%
\pgfpathlineto{\pgfqpoint{2.666584in}{2.469249in}}%
\pgfpathlineto{\pgfqpoint{2.667057in}{2.478369in}}%
\pgfpathlineto{\pgfqpoint{2.667152in}{2.478991in}}%
\pgfpathlineto{\pgfqpoint{2.667626in}{2.474883in}}%
\pgfpathlineto{\pgfqpoint{2.668195in}{2.459868in}}%
\pgfpathlineto{\pgfqpoint{2.668574in}{2.475579in}}%
\pgfpathlineto{\pgfqpoint{2.669521in}{2.483024in}}%
\pgfpathlineto{\pgfqpoint{2.669047in}{2.465782in}}%
\pgfpathlineto{\pgfqpoint{2.669711in}{2.476789in}}%
\pgfpathlineto{\pgfqpoint{2.671132in}{2.440347in}}%
\pgfpathlineto{\pgfqpoint{2.671322in}{2.447777in}}%
\pgfpathlineto{\pgfqpoint{2.672554in}{2.475639in}}%
\pgfpathlineto{\pgfqpoint{2.672838in}{2.473269in}}%
\pgfpathlineto{\pgfqpoint{2.673122in}{2.474816in}}%
\pgfpathlineto{\pgfqpoint{2.673312in}{2.469438in}}%
\pgfpathlineto{\pgfqpoint{2.674165in}{2.460446in}}%
\pgfpathlineto{\pgfqpoint{2.674449in}{2.463882in}}%
\pgfpathlineto{\pgfqpoint{2.674544in}{2.464709in}}%
\pgfpathlineto{\pgfqpoint{2.675018in}{2.459802in}}%
\pgfpathlineto{\pgfqpoint{2.676439in}{2.409218in}}%
\pgfpathlineto{\pgfqpoint{2.678334in}{2.332031in}}%
\pgfpathlineto{\pgfqpoint{2.679092in}{2.344575in}}%
\pgfpathlineto{\pgfqpoint{2.680419in}{2.417110in}}%
\pgfpathlineto{\pgfqpoint{2.681082in}{2.416945in}}%
\pgfpathlineto{\pgfqpoint{2.681272in}{2.426737in}}%
\pgfpathlineto{\pgfqpoint{2.681746in}{2.394478in}}%
\pgfpathlineto{\pgfqpoint{2.681841in}{2.391402in}}%
\pgfpathlineto{\pgfqpoint{2.682599in}{2.401839in}}%
\pgfpathlineto{\pgfqpoint{2.682883in}{2.399465in}}%
\pgfpathlineto{\pgfqpoint{2.683167in}{2.404060in}}%
\pgfpathlineto{\pgfqpoint{2.684683in}{2.425905in}}%
\pgfpathlineto{\pgfqpoint{2.684873in}{2.423436in}}%
\pgfpathlineto{\pgfqpoint{2.685252in}{2.413921in}}%
\pgfpathlineto{\pgfqpoint{2.685726in}{2.429107in}}%
\pgfpathlineto{\pgfqpoint{2.686674in}{2.451801in}}%
\pgfpathlineto{\pgfqpoint{2.686958in}{2.443226in}}%
\pgfpathlineto{\pgfqpoint{2.688853in}{2.393946in}}%
\pgfpathlineto{\pgfqpoint{2.687716in}{2.446716in}}%
\pgfpathlineto{\pgfqpoint{2.688948in}{2.394493in}}%
\pgfpathlineto{\pgfqpoint{2.689232in}{2.395313in}}%
\pgfpathlineto{\pgfqpoint{2.689706in}{2.414786in}}%
\pgfpathlineto{\pgfqpoint{2.689990in}{2.393986in}}%
\pgfpathlineto{\pgfqpoint{2.691412in}{2.366279in}}%
\pgfpathlineto{\pgfqpoint{2.691507in}{2.365111in}}%
\pgfpathlineto{\pgfqpoint{2.691980in}{2.370676in}}%
\pgfpathlineto{\pgfqpoint{2.692454in}{2.398735in}}%
\pgfpathlineto{\pgfqpoint{2.692738in}{2.413650in}}%
\pgfpathlineto{\pgfqpoint{2.693497in}{2.393699in}}%
\pgfpathlineto{\pgfqpoint{2.693781in}{2.385325in}}%
\pgfpathlineto{\pgfqpoint{2.694539in}{2.393849in}}%
\pgfpathlineto{\pgfqpoint{2.694634in}{2.394601in}}%
\pgfpathlineto{\pgfqpoint{2.694918in}{2.388407in}}%
\pgfpathlineto{\pgfqpoint{2.695013in}{2.387043in}}%
\pgfpathlineto{\pgfqpoint{2.695297in}{2.396085in}}%
\pgfpathlineto{\pgfqpoint{2.695866in}{2.427956in}}%
\pgfpathlineto{\pgfqpoint{2.696813in}{2.626420in}}%
\pgfpathlineto{\pgfqpoint{2.697761in}{2.603909in}}%
\pgfpathlineto{\pgfqpoint{2.698045in}{2.618248in}}%
\pgfpathlineto{\pgfqpoint{2.698235in}{2.591463in}}%
\pgfpathlineto{\pgfqpoint{2.699088in}{2.314485in}}%
\pgfpathlineto{\pgfqpoint{2.700035in}{2.387027in}}%
\pgfpathlineto{\pgfqpoint{2.701836in}{2.443237in}}%
\pgfpathlineto{\pgfqpoint{2.702499in}{2.423615in}}%
\pgfpathlineto{\pgfqpoint{2.703163in}{2.405240in}}%
\pgfpathlineto{\pgfqpoint{2.703542in}{2.426449in}}%
\pgfpathlineto{\pgfqpoint{2.705247in}{2.466472in}}%
\pgfpathlineto{\pgfqpoint{2.705342in}{2.464101in}}%
\pgfpathlineto{\pgfqpoint{2.706290in}{2.426734in}}%
\pgfpathlineto{\pgfqpoint{2.706669in}{2.448610in}}%
\pgfpathlineto{\pgfqpoint{2.707995in}{2.463893in}}%
\pgfpathlineto{\pgfqpoint{2.709417in}{2.449684in}}%
\pgfpathlineto{\pgfqpoint{2.709606in}{2.443222in}}%
\pgfpathlineto{\pgfqpoint{2.709986in}{2.466143in}}%
\pgfpathlineto{\pgfqpoint{2.710175in}{2.473556in}}%
\pgfpathlineto{\pgfqpoint{2.710649in}{2.449748in}}%
\pgfpathlineto{\pgfqpoint{2.711028in}{2.465507in}}%
\pgfpathlineto{\pgfqpoint{2.711502in}{2.468613in}}%
\pgfpathlineto{\pgfqpoint{2.711691in}{2.464552in}}%
\pgfpathlineto{\pgfqpoint{2.712070in}{2.450908in}}%
\pgfpathlineto{\pgfqpoint{2.712828in}{2.457701in}}%
\pgfpathlineto{\pgfqpoint{2.714060in}{2.465451in}}%
\pgfpathlineto{\pgfqpoint{2.714155in}{2.465188in}}%
\pgfpathlineto{\pgfqpoint{2.715387in}{2.435911in}}%
\pgfpathlineto{\pgfqpoint{2.716240in}{2.447220in}}%
\pgfpathlineto{\pgfqpoint{2.716429in}{2.444466in}}%
\pgfpathlineto{\pgfqpoint{2.716714in}{2.430753in}}%
\pgfpathlineto{\pgfqpoint{2.717567in}{2.438191in}}%
\pgfpathlineto{\pgfqpoint{2.718325in}{2.445286in}}%
\pgfpathlineto{\pgfqpoint{2.718609in}{2.437623in}}%
\pgfpathlineto{\pgfqpoint{2.720978in}{2.393894in}}%
\pgfpathlineto{\pgfqpoint{2.718988in}{2.439488in}}%
\pgfpathlineto{\pgfqpoint{2.721357in}{2.403020in}}%
\pgfpathlineto{\pgfqpoint{2.721642in}{2.407386in}}%
\pgfpathlineto{\pgfqpoint{2.722210in}{2.397343in}}%
\pgfpathlineto{\pgfqpoint{2.722589in}{2.388211in}}%
\pgfpathlineto{\pgfqpoint{2.722873in}{2.401303in}}%
\pgfpathlineto{\pgfqpoint{2.722968in}{2.404553in}}%
\pgfpathlineto{\pgfqpoint{2.723252in}{2.385539in}}%
\pgfpathlineto{\pgfqpoint{2.723442in}{2.374837in}}%
\pgfpathlineto{\pgfqpoint{2.724200in}{2.391257in}}%
\pgfpathlineto{\pgfqpoint{2.724390in}{2.400456in}}%
\pgfpathlineto{\pgfqpoint{2.724769in}{2.387930in}}%
\pgfpathlineto{\pgfqpoint{2.725337in}{2.394839in}}%
\pgfpathlineto{\pgfqpoint{2.725527in}{2.395633in}}%
\pgfpathlineto{\pgfqpoint{2.725716in}{2.393739in}}%
\pgfpathlineto{\pgfqpoint{2.727043in}{2.377691in}}%
\pgfpathlineto{\pgfqpoint{2.727138in}{2.379626in}}%
\pgfpathlineto{\pgfqpoint{2.728844in}{2.409222in}}%
\pgfpathlineto{\pgfqpoint{2.728938in}{2.408736in}}%
\pgfpathlineto{\pgfqpoint{2.729033in}{2.408592in}}%
\pgfpathlineto{\pgfqpoint{2.729128in}{2.409046in}}%
\pgfpathlineto{\pgfqpoint{2.731023in}{2.457508in}}%
\pgfpathlineto{\pgfqpoint{2.731497in}{2.440079in}}%
\pgfpathlineto{\pgfqpoint{2.731686in}{2.441174in}}%
\pgfpathlineto{\pgfqpoint{2.731781in}{2.440602in}}%
\pgfpathlineto{\pgfqpoint{2.734908in}{2.361594in}}%
\pgfpathlineto{\pgfqpoint{2.735003in}{2.361776in}}%
\pgfpathlineto{\pgfqpoint{2.735382in}{2.372954in}}%
\pgfpathlineto{\pgfqpoint{2.735951in}{2.359614in}}%
\pgfpathlineto{\pgfqpoint{2.736330in}{2.348737in}}%
\pgfpathlineto{\pgfqpoint{2.736709in}{2.367459in}}%
\pgfpathlineto{\pgfqpoint{2.737467in}{2.399071in}}%
\pgfpathlineto{\pgfqpoint{2.737846in}{2.380597in}}%
\pgfpathlineto{\pgfqpoint{2.738225in}{2.353064in}}%
\pgfpathlineto{\pgfqpoint{2.739078in}{2.362534in}}%
\pgfpathlineto{\pgfqpoint{2.739173in}{2.362644in}}%
\pgfpathlineto{\pgfqpoint{2.739362in}{2.361403in}}%
\pgfpathlineto{\pgfqpoint{2.739552in}{2.359075in}}%
\pgfpathlineto{\pgfqpoint{2.739931in}{2.365658in}}%
\pgfpathlineto{\pgfqpoint{2.740500in}{2.410306in}}%
\pgfpathlineto{\pgfqpoint{2.741447in}{2.595826in}}%
\pgfpathlineto{\pgfqpoint{2.742490in}{2.579291in}}%
\pgfpathlineto{\pgfqpoint{2.742963in}{2.494945in}}%
\pgfpathlineto{\pgfqpoint{2.743627in}{2.293772in}}%
\pgfpathlineto{\pgfqpoint{2.744385in}{2.376799in}}%
\pgfpathlineto{\pgfqpoint{2.745143in}{2.387193in}}%
\pgfpathlineto{\pgfqpoint{2.746470in}{2.414395in}}%
\pgfpathlineto{\pgfqpoint{2.745901in}{2.385653in}}%
\pgfpathlineto{\pgfqpoint{2.746754in}{2.406813in}}%
\pgfpathlineto{\pgfqpoint{2.747512in}{2.372181in}}%
\pgfpathlineto{\pgfqpoint{2.748460in}{2.378324in}}%
\pgfpathlineto{\pgfqpoint{2.749502in}{2.373263in}}%
\pgfpathlineto{\pgfqpoint{2.750545in}{2.312147in}}%
\pgfpathlineto{\pgfqpoint{2.751303in}{2.352122in}}%
\pgfpathlineto{\pgfqpoint{2.752914in}{2.442430in}}%
\pgfpathlineto{\pgfqpoint{2.753293in}{2.430937in}}%
\pgfpathlineto{\pgfqpoint{2.753861in}{2.425410in}}%
\pgfpathlineto{\pgfqpoint{2.754240in}{2.432660in}}%
\pgfpathlineto{\pgfqpoint{2.755946in}{2.458638in}}%
\pgfpathlineto{\pgfqpoint{2.756230in}{2.454942in}}%
\pgfpathlineto{\pgfqpoint{2.756420in}{2.451862in}}%
\pgfpathlineto{\pgfqpoint{2.756989in}{2.461540in}}%
\pgfpathlineto{\pgfqpoint{2.757936in}{2.482549in}}%
\pgfpathlineto{\pgfqpoint{2.758220in}{2.469325in}}%
\pgfpathlineto{\pgfqpoint{2.759358in}{2.450219in}}%
\pgfpathlineto{\pgfqpoint{2.759547in}{2.453818in}}%
\pgfpathlineto{\pgfqpoint{2.759737in}{2.458283in}}%
\pgfpathlineto{\pgfqpoint{2.760210in}{2.443011in}}%
\pgfpathlineto{\pgfqpoint{2.760400in}{2.446803in}}%
\pgfpathlineto{\pgfqpoint{2.762011in}{2.483306in}}%
\pgfpathlineto{\pgfqpoint{2.762201in}{2.483125in}}%
\pgfpathlineto{\pgfqpoint{2.762295in}{2.484320in}}%
\pgfpathlineto{\pgfqpoint{2.763243in}{2.505315in}}%
\pgfpathlineto{\pgfqpoint{2.763906in}{2.499428in}}%
\pgfpathlineto{\pgfqpoint{2.765991in}{2.460524in}}%
\pgfpathlineto{\pgfqpoint{2.766181in}{2.463884in}}%
\pgfpathlineto{\pgfqpoint{2.766465in}{2.453468in}}%
\pgfpathlineto{\pgfqpoint{2.767981in}{2.423190in}}%
\pgfpathlineto{\pgfqpoint{2.768360in}{2.438080in}}%
\pgfpathlineto{\pgfqpoint{2.768739in}{2.414156in}}%
\pgfpathlineto{\pgfqpoint{2.768834in}{2.413710in}}%
\pgfpathlineto{\pgfqpoint{2.768929in}{2.416489in}}%
\pgfpathlineto{\pgfqpoint{2.769876in}{2.425209in}}%
\pgfpathlineto{\pgfqpoint{2.769497in}{2.411477in}}%
\pgfpathlineto{\pgfqpoint{2.770066in}{2.419807in}}%
\pgfpathlineto{\pgfqpoint{2.770350in}{2.407487in}}%
\pgfpathlineto{\pgfqpoint{2.770729in}{2.420074in}}%
\pgfpathlineto{\pgfqpoint{2.771203in}{2.415292in}}%
\pgfpathlineto{\pgfqpoint{2.771393in}{2.423013in}}%
\pgfpathlineto{\pgfqpoint{2.771866in}{2.401357in}}%
\pgfpathlineto{\pgfqpoint{2.772056in}{2.403825in}}%
\pgfpathlineto{\pgfqpoint{2.774330in}{2.428894in}}%
\pgfpathlineto{\pgfqpoint{2.774520in}{2.425766in}}%
\pgfpathlineto{\pgfqpoint{2.774804in}{2.413061in}}%
\pgfpathlineto{\pgfqpoint{2.775373in}{2.436163in}}%
\pgfpathlineto{\pgfqpoint{2.775562in}{2.437657in}}%
\pgfpathlineto{\pgfqpoint{2.775941in}{2.454976in}}%
\pgfpathlineto{\pgfqpoint{2.776510in}{2.434591in}}%
\pgfpathlineto{\pgfqpoint{2.776699in}{2.433831in}}%
\pgfpathlineto{\pgfqpoint{2.777647in}{2.388958in}}%
\pgfpathlineto{\pgfqpoint{2.779353in}{2.325996in}}%
\pgfpathlineto{\pgfqpoint{2.779921in}{2.308790in}}%
\pgfpathlineto{\pgfqpoint{2.781343in}{2.286117in}}%
\pgfpathlineto{\pgfqpoint{2.781438in}{2.284629in}}%
\pgfpathlineto{\pgfqpoint{2.781627in}{2.291595in}}%
\pgfpathlineto{\pgfqpoint{2.782101in}{2.331634in}}%
\pgfpathlineto{\pgfqpoint{2.782859in}{2.321095in}}%
\pgfpathlineto{\pgfqpoint{2.783143in}{2.303032in}}%
\pgfpathlineto{\pgfqpoint{2.783996in}{2.308948in}}%
\pgfpathlineto{\pgfqpoint{2.785323in}{2.344174in}}%
\pgfpathlineto{\pgfqpoint{2.786460in}{2.569467in}}%
\pgfpathlineto{\pgfqpoint{2.787882in}{2.504635in}}%
\pgfpathlineto{\pgfqpoint{2.788640in}{2.281658in}}%
\pgfpathlineto{\pgfqpoint{2.789398in}{2.355682in}}%
\pgfpathlineto{\pgfqpoint{2.790156in}{2.389466in}}%
\pgfpathlineto{\pgfqpoint{2.791293in}{2.405332in}}%
\pgfpathlineto{\pgfqpoint{2.790819in}{2.382679in}}%
\pgfpathlineto{\pgfqpoint{2.791388in}{2.405302in}}%
\pgfpathlineto{\pgfqpoint{2.791767in}{2.412740in}}%
\pgfpathlineto{\pgfqpoint{2.791956in}{2.404437in}}%
\pgfpathlineto{\pgfqpoint{2.792430in}{2.368666in}}%
\pgfpathlineto{\pgfqpoint{2.793188in}{2.389537in}}%
\pgfpathlineto{\pgfqpoint{2.794610in}{2.410755in}}%
\pgfpathlineto{\pgfqpoint{2.794705in}{2.410066in}}%
\pgfpathlineto{\pgfqpoint{2.795937in}{2.383039in}}%
\pgfpathlineto{\pgfqpoint{2.796221in}{2.390812in}}%
\pgfpathlineto{\pgfqpoint{2.797927in}{2.426640in}}%
\pgfpathlineto{\pgfqpoint{2.798400in}{2.406351in}}%
\pgfpathlineto{\pgfqpoint{2.799159in}{2.421105in}}%
\pgfpathlineto{\pgfqpoint{2.799917in}{2.437718in}}%
\pgfpathlineto{\pgfqpoint{2.800580in}{2.432457in}}%
\pgfpathlineto{\pgfqpoint{2.801622in}{2.419586in}}%
\pgfpathlineto{\pgfqpoint{2.801243in}{2.433208in}}%
\pgfpathlineto{\pgfqpoint{2.801812in}{2.430548in}}%
\pgfpathlineto{\pgfqpoint{2.802665in}{2.452248in}}%
\pgfpathlineto{\pgfqpoint{2.802286in}{2.426945in}}%
\pgfpathlineto{\pgfqpoint{2.802949in}{2.440004in}}%
\pgfpathlineto{\pgfqpoint{2.803707in}{2.432655in}}%
\pgfpathlineto{\pgfqpoint{2.803328in}{2.449735in}}%
\pgfpathlineto{\pgfqpoint{2.803992in}{2.440506in}}%
\pgfpathlineto{\pgfqpoint{2.804086in}{2.441650in}}%
\pgfpathlineto{\pgfqpoint{2.804276in}{2.434845in}}%
\pgfpathlineto{\pgfqpoint{2.805318in}{2.410583in}}%
\pgfpathlineto{\pgfqpoint{2.805697in}{2.417656in}}%
\pgfpathlineto{\pgfqpoint{2.806266in}{2.414502in}}%
\pgfpathlineto{\pgfqpoint{2.806929in}{2.420597in}}%
\pgfpathlineto{\pgfqpoint{2.807593in}{2.421003in}}%
\pgfpathlineto{\pgfqpoint{2.807877in}{2.412272in}}%
\pgfpathlineto{\pgfqpoint{2.809109in}{2.397458in}}%
\pgfpathlineto{\pgfqpoint{2.808540in}{2.417945in}}%
\pgfpathlineto{\pgfqpoint{2.809204in}{2.399152in}}%
\pgfpathlineto{\pgfqpoint{2.809393in}{2.404022in}}%
\pgfpathlineto{\pgfqpoint{2.809867in}{2.392303in}}%
\pgfpathlineto{\pgfqpoint{2.810151in}{2.398155in}}%
\pgfpathlineto{\pgfqpoint{2.814226in}{2.315982in}}%
\pgfpathlineto{\pgfqpoint{2.814510in}{2.324945in}}%
\pgfpathlineto{\pgfqpoint{2.814795in}{2.334742in}}%
\pgfpathlineto{\pgfqpoint{2.815647in}{2.329916in}}%
\pgfpathlineto{\pgfqpoint{2.816785in}{2.320290in}}%
\pgfpathlineto{\pgfqpoint{2.816311in}{2.335256in}}%
\pgfpathlineto{\pgfqpoint{2.816879in}{2.323603in}}%
\pgfpathlineto{\pgfqpoint{2.817069in}{2.332834in}}%
\pgfpathlineto{\pgfqpoint{2.817543in}{2.318051in}}%
\pgfpathlineto{\pgfqpoint{2.818017in}{2.325308in}}%
\pgfpathlineto{\pgfqpoint{2.818206in}{2.322754in}}%
\pgfpathlineto{\pgfqpoint{2.818396in}{2.332586in}}%
\pgfpathlineto{\pgfqpoint{2.818775in}{2.361761in}}%
\pgfpathlineto{\pgfqpoint{2.819628in}{2.353360in}}%
\pgfpathlineto{\pgfqpoint{2.819912in}{2.345944in}}%
\pgfpathlineto{\pgfqpoint{2.820386in}{2.314048in}}%
\pgfpathlineto{\pgfqpoint{2.821333in}{2.320849in}}%
\pgfpathlineto{\pgfqpoint{2.821902in}{2.307745in}}%
\pgfpathlineto{\pgfqpoint{2.822471in}{2.318755in}}%
\pgfpathlineto{\pgfqpoint{2.822850in}{2.329008in}}%
\pgfpathlineto{\pgfqpoint{2.823323in}{2.312017in}}%
\pgfpathlineto{\pgfqpoint{2.823418in}{2.309849in}}%
\pgfpathlineto{\pgfqpoint{2.823797in}{2.321911in}}%
\pgfpathlineto{\pgfqpoint{2.824555in}{2.354472in}}%
\pgfpathlineto{\pgfqpoint{2.824934in}{2.332768in}}%
\pgfpathlineto{\pgfqpoint{2.826545in}{2.302079in}}%
\pgfpathlineto{\pgfqpoint{2.827303in}{2.315729in}}%
\pgfpathlineto{\pgfqpoint{2.827683in}{2.341806in}}%
\pgfpathlineto{\pgfqpoint{2.828346in}{2.322593in}}%
\pgfpathlineto{\pgfqpoint{2.829578in}{2.304281in}}%
\pgfpathlineto{\pgfqpoint{2.829104in}{2.322649in}}%
\pgfpathlineto{\pgfqpoint{2.829673in}{2.305943in}}%
\pgfpathlineto{\pgfqpoint{2.830999in}{2.393659in}}%
\pgfpathlineto{\pgfqpoint{2.831852in}{2.552201in}}%
\pgfpathlineto{\pgfqpoint{2.832516in}{2.530075in}}%
\pgfpathlineto{\pgfqpoint{2.832989in}{2.517800in}}%
\pgfpathlineto{\pgfqpoint{2.833937in}{2.248805in}}%
\pgfpathlineto{\pgfqpoint{2.835169in}{2.336272in}}%
\pgfpathlineto{\pgfqpoint{2.836969in}{2.379202in}}%
\pgfpathlineto{\pgfqpoint{2.837064in}{2.378962in}}%
\pgfpathlineto{\pgfqpoint{2.837443in}{2.357978in}}%
\pgfpathlineto{\pgfqpoint{2.837728in}{2.341873in}}%
\pgfpathlineto{\pgfqpoint{2.838580in}{2.349548in}}%
\pgfpathlineto{\pgfqpoint{2.840097in}{2.396397in}}%
\pgfpathlineto{\pgfqpoint{2.840286in}{2.392453in}}%
\pgfpathlineto{\pgfqpoint{2.840950in}{2.360039in}}%
\pgfpathlineto{\pgfqpoint{2.841613in}{2.379620in}}%
\pgfpathlineto{\pgfqpoint{2.843129in}{2.411240in}}%
\pgfpathlineto{\pgfqpoint{2.841992in}{2.377243in}}%
\pgfpathlineto{\pgfqpoint{2.843508in}{2.397473in}}%
\pgfpathlineto{\pgfqpoint{2.844077in}{2.381520in}}%
\pgfpathlineto{\pgfqpoint{2.844456in}{2.395122in}}%
\pgfpathlineto{\pgfqpoint{2.845782in}{2.414069in}}%
\pgfpathlineto{\pgfqpoint{2.845877in}{2.414599in}}%
\pgfpathlineto{\pgfqpoint{2.846067in}{2.409497in}}%
\pgfpathlineto{\pgfqpoint{2.846256in}{2.406600in}}%
\pgfpathlineto{\pgfqpoint{2.846635in}{2.416147in}}%
\pgfpathlineto{\pgfqpoint{2.847204in}{2.408765in}}%
\pgfpathlineto{\pgfqpoint{2.847488in}{2.404203in}}%
\pgfpathlineto{\pgfqpoint{2.847867in}{2.411386in}}%
\pgfpathlineto{\pgfqpoint{2.848341in}{2.405547in}}%
\pgfpathlineto{\pgfqpoint{2.848625in}{2.410538in}}%
\pgfpathlineto{\pgfqpoint{2.849004in}{2.402175in}}%
\pgfpathlineto{\pgfqpoint{2.850521in}{2.389837in}}%
\pgfpathlineto{\pgfqpoint{2.849478in}{2.407825in}}%
\pgfpathlineto{\pgfqpoint{2.850710in}{2.391779in}}%
\pgfpathlineto{\pgfqpoint{2.851847in}{2.412884in}}%
\pgfpathlineto{\pgfqpoint{2.852037in}{2.406773in}}%
\pgfpathlineto{\pgfqpoint{2.852890in}{2.395097in}}%
\pgfpathlineto{\pgfqpoint{2.853174in}{2.401667in}}%
\pgfpathlineto{\pgfqpoint{2.853837in}{2.408508in}}%
\pgfpathlineto{\pgfqpoint{2.854027in}{2.403279in}}%
\pgfpathlineto{\pgfqpoint{2.855543in}{2.382194in}}%
\pgfpathlineto{\pgfqpoint{2.856301in}{2.379141in}}%
\pgfpathlineto{\pgfqpoint{2.855922in}{2.384918in}}%
\pgfpathlineto{\pgfqpoint{2.856491in}{2.381232in}}%
\pgfpathlineto{\pgfqpoint{2.856680in}{2.383553in}}%
\pgfpathlineto{\pgfqpoint{2.857249in}{2.376828in}}%
\pgfpathlineto{\pgfqpoint{2.857438in}{2.375426in}}%
\pgfpathlineto{\pgfqpoint{2.858481in}{2.356656in}}%
\pgfpathlineto{\pgfqpoint{2.859144in}{2.360888in}}%
\pgfpathlineto{\pgfqpoint{2.859429in}{2.361540in}}%
\pgfpathlineto{\pgfqpoint{2.859523in}{2.359872in}}%
\pgfpathlineto{\pgfqpoint{2.859713in}{2.354481in}}%
\pgfpathlineto{\pgfqpoint{2.860281in}{2.367832in}}%
\pgfpathlineto{\pgfqpoint{2.860376in}{2.367818in}}%
\pgfpathlineto{\pgfqpoint{2.860850in}{2.376216in}}%
\pgfpathlineto{\pgfqpoint{2.861608in}{2.371045in}}%
\pgfpathlineto{\pgfqpoint{2.862650in}{2.356577in}}%
\pgfpathlineto{\pgfqpoint{2.862840in}{2.360520in}}%
\pgfpathlineto{\pgfqpoint{2.864261in}{2.410541in}}%
\pgfpathlineto{\pgfqpoint{2.864641in}{2.409449in}}%
\pgfpathlineto{\pgfqpoint{2.866346in}{2.454420in}}%
\pgfpathlineto{\pgfqpoint{2.865304in}{2.408785in}}%
\pgfpathlineto{\pgfqpoint{2.866915in}{2.445743in}}%
\pgfpathlineto{\pgfqpoint{2.867389in}{2.446200in}}%
\pgfpathlineto{\pgfqpoint{2.867483in}{2.444502in}}%
\pgfpathlineto{\pgfqpoint{2.868810in}{2.382635in}}%
\pgfpathlineto{\pgfqpoint{2.869379in}{2.402270in}}%
\pgfpathlineto{\pgfqpoint{2.869758in}{2.419974in}}%
\pgfpathlineto{\pgfqpoint{2.870232in}{2.400205in}}%
\pgfpathlineto{\pgfqpoint{2.870516in}{2.407176in}}%
\pgfpathlineto{\pgfqpoint{2.871843in}{2.385223in}}%
\pgfpathlineto{\pgfqpoint{2.872032in}{2.392501in}}%
\pgfpathlineto{\pgfqpoint{2.872601in}{2.425138in}}%
\pgfpathlineto{\pgfqpoint{2.873169in}{2.408944in}}%
\pgfpathlineto{\pgfqpoint{2.873454in}{2.388826in}}%
\pgfpathlineto{\pgfqpoint{2.874401in}{2.393768in}}%
\pgfpathlineto{\pgfqpoint{2.876297in}{2.553058in}}%
\pgfpathlineto{\pgfqpoint{2.877434in}{2.670137in}}%
\pgfpathlineto{\pgfqpoint{2.877813in}{2.654040in}}%
\pgfpathlineto{\pgfqpoint{2.878287in}{2.590524in}}%
\pgfpathlineto{\pgfqpoint{2.879045in}{2.368249in}}%
\pgfpathlineto{\pgfqpoint{2.879803in}{2.448735in}}%
\pgfpathlineto{\pgfqpoint{2.882077in}{2.512230in}}%
\pgfpathlineto{\pgfqpoint{2.882835in}{2.490916in}}%
\pgfpathlineto{\pgfqpoint{2.883214in}{2.481357in}}%
\pgfpathlineto{\pgfqpoint{2.883404in}{2.494185in}}%
\pgfpathlineto{\pgfqpoint{2.883972in}{2.492768in}}%
\pgfpathlineto{\pgfqpoint{2.885015in}{2.539466in}}%
\pgfpathlineto{\pgfqpoint{2.885110in}{2.541450in}}%
\pgfpathlineto{\pgfqpoint{2.885489in}{2.526889in}}%
\pgfpathlineto{\pgfqpoint{2.886057in}{2.495524in}}%
\pgfpathlineto{\pgfqpoint{2.886721in}{2.510869in}}%
\pgfpathlineto{\pgfqpoint{2.887953in}{2.553492in}}%
\pgfpathlineto{\pgfqpoint{2.888332in}{2.547651in}}%
\pgfpathlineto{\pgfqpoint{2.889943in}{2.491711in}}%
\pgfpathlineto{\pgfqpoint{2.890416in}{2.504631in}}%
\pgfpathlineto{\pgfqpoint{2.890606in}{2.501652in}}%
\pgfpathlineto{\pgfqpoint{2.890890in}{2.510560in}}%
\pgfpathlineto{\pgfqpoint{2.892880in}{2.621977in}}%
\pgfpathlineto{\pgfqpoint{2.893070in}{2.617768in}}%
\pgfpathlineto{\pgfqpoint{2.894681in}{2.591195in}}%
\pgfpathlineto{\pgfqpoint{2.894870in}{2.589215in}}%
\pgfpathlineto{\pgfqpoint{2.895628in}{2.593697in}}%
\pgfpathlineto{\pgfqpoint{2.897145in}{2.618173in}}%
\pgfpathlineto{\pgfqpoint{2.897429in}{2.610594in}}%
\pgfpathlineto{\pgfqpoint{2.897808in}{2.604685in}}%
\pgfpathlineto{\pgfqpoint{2.898282in}{2.616190in}}%
\pgfpathlineto{\pgfqpoint{2.898377in}{2.617556in}}%
\pgfpathlineto{\pgfqpoint{2.898850in}{2.610343in}}%
\pgfpathlineto{\pgfqpoint{2.899135in}{2.613013in}}%
\pgfpathlineto{\pgfqpoint{2.899703in}{2.619007in}}%
\pgfpathlineto{\pgfqpoint{2.900272in}{2.613726in}}%
\pgfpathlineto{\pgfqpoint{2.902072in}{2.587498in}}%
\pgfpathlineto{\pgfqpoint{2.902262in}{2.572011in}}%
\pgfpathlineto{\pgfqpoint{2.903210in}{2.574799in}}%
\pgfpathlineto{\pgfqpoint{2.903304in}{2.576235in}}%
\pgfpathlineto{\pgfqpoint{2.903683in}{2.568267in}}%
\pgfpathlineto{\pgfqpoint{2.903873in}{2.569208in}}%
\pgfpathlineto{\pgfqpoint{2.903968in}{2.569171in}}%
\pgfpathlineto{\pgfqpoint{2.905010in}{2.554172in}}%
\pgfpathlineto{\pgfqpoint{2.905768in}{2.557619in}}%
\pgfpathlineto{\pgfqpoint{2.906526in}{2.574599in}}%
\pgfpathlineto{\pgfqpoint{2.907284in}{2.565896in}}%
\pgfpathlineto{\pgfqpoint{2.908422in}{2.544491in}}%
\pgfpathlineto{\pgfqpoint{2.907948in}{2.568712in}}%
\pgfpathlineto{\pgfqpoint{2.908611in}{2.556295in}}%
\pgfpathlineto{\pgfqpoint{2.910222in}{2.592369in}}%
\pgfpathlineto{\pgfqpoint{2.911359in}{2.582050in}}%
\pgfpathlineto{\pgfqpoint{2.911644in}{2.587024in}}%
\pgfpathlineto{\pgfqpoint{2.912686in}{2.605912in}}%
\pgfpathlineto{\pgfqpoint{2.912117in}{2.585375in}}%
\pgfpathlineto{\pgfqpoint{2.912970in}{2.594647in}}%
\pgfpathlineto{\pgfqpoint{2.914676in}{2.527248in}}%
\pgfpathlineto{\pgfqpoint{2.914866in}{2.533367in}}%
\pgfpathlineto{\pgfqpoint{2.915813in}{2.577989in}}%
\pgfpathlineto{\pgfqpoint{2.916382in}{2.553124in}}%
\pgfpathlineto{\pgfqpoint{2.917424in}{2.531181in}}%
\pgfpathlineto{\pgfqpoint{2.917614in}{2.541605in}}%
\pgfpathlineto{\pgfqpoint{2.918751in}{2.579412in}}%
\pgfpathlineto{\pgfqpoint{2.919130in}{2.568735in}}%
\pgfpathlineto{\pgfqpoint{2.919793in}{2.549877in}}%
\pgfpathlineto{\pgfqpoint{2.921215in}{2.553294in}}%
\pgfpathlineto{\pgfqpoint{2.921689in}{2.561676in}}%
\pgfpathlineto{\pgfqpoint{2.922068in}{2.607692in}}%
\pgfpathlineto{\pgfqpoint{2.923205in}{2.846014in}}%
\pgfpathlineto{\pgfqpoint{2.923963in}{2.810350in}}%
\pgfpathlineto{\pgfqpoint{2.924342in}{2.778829in}}%
\pgfpathlineto{\pgfqpoint{2.925100in}{2.529940in}}%
\pgfpathlineto{\pgfqpoint{2.926048in}{2.606966in}}%
\pgfpathlineto{\pgfqpoint{2.926427in}{2.596924in}}%
\pgfpathlineto{\pgfqpoint{2.926711in}{2.609629in}}%
\pgfpathlineto{\pgfqpoint{2.928132in}{2.652073in}}%
\pgfpathlineto{\pgfqpoint{2.928322in}{2.643224in}}%
\pgfpathlineto{\pgfqpoint{2.929175in}{2.601487in}}%
\pgfpathlineto{\pgfqpoint{2.929649in}{2.622392in}}%
\pgfpathlineto{\pgfqpoint{2.931354in}{2.668873in}}%
\pgfpathlineto{\pgfqpoint{2.931734in}{2.647560in}}%
\pgfpathlineto{\pgfqpoint{2.931923in}{2.637936in}}%
\pgfpathlineto{\pgfqpoint{2.932681in}{2.650819in}}%
\pgfpathlineto{\pgfqpoint{2.934197in}{2.689735in}}%
\pgfpathlineto{\pgfqpoint{2.935145in}{2.671400in}}%
\pgfpathlineto{\pgfqpoint{2.935903in}{2.679817in}}%
\pgfpathlineto{\pgfqpoint{2.937040in}{2.697702in}}%
\pgfpathlineto{\pgfqpoint{2.937230in}{2.696458in}}%
\pgfpathlineto{\pgfqpoint{2.938272in}{2.687228in}}%
\pgfpathlineto{\pgfqpoint{2.938746in}{2.687887in}}%
\pgfpathlineto{\pgfqpoint{2.939125in}{2.709315in}}%
\pgfpathlineto{\pgfqpoint{2.939315in}{2.718438in}}%
\pgfpathlineto{\pgfqpoint{2.939883in}{2.702783in}}%
\pgfpathlineto{\pgfqpoint{2.940262in}{2.712203in}}%
\pgfpathlineto{\pgfqpoint{2.940357in}{2.712499in}}%
\pgfpathlineto{\pgfqpoint{2.940452in}{2.710932in}}%
\pgfpathlineto{\pgfqpoint{2.940736in}{2.705196in}}%
\pgfpathlineto{\pgfqpoint{2.941115in}{2.713013in}}%
\pgfpathlineto{\pgfqpoint{2.941589in}{2.707309in}}%
\pgfpathlineto{\pgfqpoint{2.942726in}{2.734952in}}%
\pgfpathlineto{\pgfqpoint{2.942158in}{2.706568in}}%
\pgfpathlineto{\pgfqpoint{2.942916in}{2.724017in}}%
\pgfpathlineto{\pgfqpoint{2.943105in}{2.713239in}}%
\pgfpathlineto{\pgfqpoint{2.943579in}{2.727816in}}%
\pgfpathlineto{\pgfqpoint{2.943958in}{2.724687in}}%
\pgfpathlineto{\pgfqpoint{2.944148in}{2.728808in}}%
\pgfpathlineto{\pgfqpoint{2.944621in}{2.715922in}}%
\pgfpathlineto{\pgfqpoint{2.946138in}{2.698163in}}%
\pgfpathlineto{\pgfqpoint{2.945095in}{2.720026in}}%
\pgfpathlineto{\pgfqpoint{2.946327in}{2.704580in}}%
\pgfpathlineto{\pgfqpoint{2.946517in}{2.710829in}}%
\pgfpathlineto{\pgfqpoint{2.946991in}{2.683205in}}%
\pgfpathlineto{\pgfqpoint{2.947275in}{2.689677in}}%
\pgfpathlineto{\pgfqpoint{2.947749in}{2.673609in}}%
\pgfpathlineto{\pgfqpoint{2.948033in}{2.674920in}}%
\pgfpathlineto{\pgfqpoint{2.948317in}{2.669454in}}%
\pgfpathlineto{\pgfqpoint{2.949454in}{2.643041in}}%
\pgfpathlineto{\pgfqpoint{2.949739in}{2.646976in}}%
\pgfpathlineto{\pgfqpoint{2.949833in}{2.646969in}}%
\pgfpathlineto{\pgfqpoint{2.951634in}{2.616666in}}%
\pgfpathlineto{\pgfqpoint{2.951918in}{2.630452in}}%
\pgfpathlineto{\pgfqpoint{2.952771in}{2.638693in}}%
\pgfpathlineto{\pgfqpoint{2.952392in}{2.623184in}}%
\pgfpathlineto{\pgfqpoint{2.952961in}{2.631139in}}%
\pgfpathlineto{\pgfqpoint{2.955804in}{2.528246in}}%
\pgfpathlineto{\pgfqpoint{2.955898in}{2.527941in}}%
\pgfpathlineto{\pgfqpoint{2.956183in}{2.530120in}}%
\pgfpathlineto{\pgfqpoint{2.957604in}{2.596855in}}%
\pgfpathlineto{\pgfqpoint{2.958647in}{2.655545in}}%
\pgfpathlineto{\pgfqpoint{2.959026in}{2.643129in}}%
\pgfpathlineto{\pgfqpoint{2.960447in}{2.561781in}}%
\pgfpathlineto{\pgfqpoint{2.960826in}{2.530495in}}%
\pgfpathlineto{\pgfqpoint{2.961679in}{2.543139in}}%
\pgfpathlineto{\pgfqpoint{2.961963in}{2.552564in}}%
\pgfpathlineto{\pgfqpoint{2.962437in}{2.532445in}}%
\pgfpathlineto{\pgfqpoint{2.964143in}{2.490200in}}%
\pgfpathlineto{\pgfqpoint{2.964238in}{2.492229in}}%
\pgfpathlineto{\pgfqpoint{2.964901in}{2.530866in}}%
\pgfpathlineto{\pgfqpoint{2.965659in}{2.509053in}}%
\pgfpathlineto{\pgfqpoint{2.967744in}{2.466695in}}%
\pgfpathlineto{\pgfqpoint{2.967933in}{2.474025in}}%
\pgfpathlineto{\pgfqpoint{2.969355in}{2.752702in}}%
\pgfpathlineto{\pgfqpoint{2.970397in}{2.687669in}}%
\pgfpathlineto{\pgfqpoint{2.971250in}{2.438836in}}%
\pgfpathlineto{\pgfqpoint{2.972387in}{2.489518in}}%
\pgfpathlineto{\pgfqpoint{2.974377in}{2.542383in}}%
\pgfpathlineto{\pgfqpoint{2.974662in}{2.525505in}}%
\pgfpathlineto{\pgfqpoint{2.975515in}{2.493099in}}%
\pgfpathlineto{\pgfqpoint{2.975894in}{2.505291in}}%
\pgfpathlineto{\pgfqpoint{2.975988in}{2.505437in}}%
\pgfpathlineto{\pgfqpoint{2.976178in}{2.501366in}}%
\pgfpathlineto{\pgfqpoint{2.976557in}{2.517022in}}%
\pgfpathlineto{\pgfqpoint{2.977410in}{2.522093in}}%
\pgfpathlineto{\pgfqpoint{2.977599in}{2.517233in}}%
\pgfpathlineto{\pgfqpoint{2.978168in}{2.474064in}}%
\pgfpathlineto{\pgfqpoint{2.979021in}{2.491182in}}%
\pgfpathlineto{\pgfqpoint{2.979495in}{2.484945in}}%
\pgfpathlineto{\pgfqpoint{2.980537in}{2.507286in}}%
\pgfpathlineto{\pgfqpoint{2.980632in}{2.508213in}}%
\pgfpathlineto{\pgfqpoint{2.980727in}{2.505427in}}%
\pgfpathlineto{\pgfqpoint{2.981295in}{2.469659in}}%
\pgfpathlineto{\pgfqpoint{2.981959in}{2.482659in}}%
\pgfpathlineto{\pgfqpoint{2.983285in}{2.519639in}}%
\pgfpathlineto{\pgfqpoint{2.983380in}{2.517718in}}%
\pgfpathlineto{\pgfqpoint{2.983759in}{2.505270in}}%
\pgfpathlineto{\pgfqpoint{2.984707in}{2.507658in}}%
\pgfpathlineto{\pgfqpoint{2.985465in}{2.504649in}}%
\pgfpathlineto{\pgfqpoint{2.985749in}{2.507785in}}%
\pgfpathlineto{\pgfqpoint{2.985844in}{2.508170in}}%
\pgfpathlineto{\pgfqpoint{2.986033in}{2.505159in}}%
\pgfpathlineto{\pgfqpoint{2.988687in}{2.471754in}}%
\pgfpathlineto{\pgfqpoint{2.988876in}{2.475503in}}%
\pgfpathlineto{\pgfqpoint{2.989729in}{2.493307in}}%
\pgfpathlineto{\pgfqpoint{2.989350in}{2.475304in}}%
\pgfpathlineto{\pgfqpoint{2.990013in}{2.479365in}}%
\pgfpathlineto{\pgfqpoint{2.990866in}{2.464630in}}%
\pgfpathlineto{\pgfqpoint{2.991151in}{2.475969in}}%
\pgfpathlineto{\pgfqpoint{2.991245in}{2.479093in}}%
\pgfpathlineto{\pgfqpoint{2.991719in}{2.460040in}}%
\pgfpathlineto{\pgfqpoint{2.991814in}{2.459786in}}%
\pgfpathlineto{\pgfqpoint{2.991909in}{2.461272in}}%
\pgfpathlineto{\pgfqpoint{2.992004in}{2.462792in}}%
\pgfpathlineto{\pgfqpoint{2.992477in}{2.453983in}}%
\pgfpathlineto{\pgfqpoint{2.998447in}{2.354307in}}%
\pgfpathlineto{\pgfqpoint{2.993235in}{2.457485in}}%
\pgfpathlineto{\pgfqpoint{2.998637in}{2.361752in}}%
\pgfpathlineto{\pgfqpoint{2.998921in}{2.376325in}}%
\pgfpathlineto{\pgfqpoint{2.999585in}{2.357487in}}%
\pgfpathlineto{\pgfqpoint{3.000438in}{2.339205in}}%
\pgfpathlineto{\pgfqpoint{3.000722in}{2.354706in}}%
\pgfpathlineto{\pgfqpoint{3.001575in}{2.375489in}}%
\pgfpathlineto{\pgfqpoint{3.001859in}{2.362169in}}%
\pgfpathlineto{\pgfqpoint{3.001954in}{2.359632in}}%
\pgfpathlineto{\pgfqpoint{3.002333in}{2.374355in}}%
\pgfpathlineto{\pgfqpoint{3.004418in}{2.421874in}}%
\pgfpathlineto{\pgfqpoint{3.002901in}{2.372794in}}%
\pgfpathlineto{\pgfqpoint{3.004512in}{2.418481in}}%
\pgfpathlineto{\pgfqpoint{3.006218in}{2.365419in}}%
\pgfpathlineto{\pgfqpoint{3.007071in}{2.380687in}}%
\pgfpathlineto{\pgfqpoint{3.007261in}{2.388315in}}%
\pgfpathlineto{\pgfqpoint{3.008113in}{2.379583in}}%
\pgfpathlineto{\pgfqpoint{3.008492in}{2.366756in}}%
\pgfpathlineto{\pgfqpoint{3.009251in}{2.372629in}}%
\pgfpathlineto{\pgfqpoint{3.010577in}{2.411907in}}%
\pgfpathlineto{\pgfqpoint{3.010956in}{2.391207in}}%
\pgfpathlineto{\pgfqpoint{3.012283in}{2.361070in}}%
\pgfpathlineto{\pgfqpoint{3.012378in}{2.361690in}}%
\pgfpathlineto{\pgfqpoint{3.012567in}{2.360649in}}%
\pgfpathlineto{\pgfqpoint{3.012852in}{2.354739in}}%
\pgfpathlineto{\pgfqpoint{3.013231in}{2.365811in}}%
\pgfpathlineto{\pgfqpoint{3.013894in}{2.446854in}}%
\pgfpathlineto{\pgfqpoint{3.014747in}{2.601360in}}%
\pgfpathlineto{\pgfqpoint{3.015505in}{2.583728in}}%
\pgfpathlineto{\pgfqpoint{3.016074in}{2.499653in}}%
\pgfpathlineto{\pgfqpoint{3.016737in}{2.290992in}}%
\pgfpathlineto{\pgfqpoint{3.017495in}{2.355834in}}%
\pgfpathlineto{\pgfqpoint{3.018158in}{2.362999in}}%
\pgfpathlineto{\pgfqpoint{3.018443in}{2.353061in}}%
\pgfpathlineto{\pgfqpoint{3.021191in}{2.272816in}}%
\pgfpathlineto{\pgfqpoint{3.018917in}{2.358925in}}%
\pgfpathlineto{\pgfqpoint{3.021570in}{2.285216in}}%
\pgfpathlineto{\pgfqpoint{3.022991in}{2.389739in}}%
\pgfpathlineto{\pgfqpoint{3.024413in}{2.386441in}}%
\pgfpathlineto{\pgfqpoint{3.024508in}{2.387004in}}%
\pgfpathlineto{\pgfqpoint{3.024697in}{2.384021in}}%
\pgfpathlineto{\pgfqpoint{3.025076in}{2.375084in}}%
\pgfpathlineto{\pgfqpoint{3.025740in}{2.381654in}}%
\pgfpathlineto{\pgfqpoint{3.026119in}{2.393729in}}%
\pgfpathlineto{\pgfqpoint{3.026687in}{2.380288in}}%
\pgfpathlineto{\pgfqpoint{3.026971in}{2.373557in}}%
\pgfpathlineto{\pgfqpoint{3.027066in}{2.370834in}}%
\pgfpathlineto{\pgfqpoint{3.027351in}{2.383161in}}%
\pgfpathlineto{\pgfqpoint{3.027730in}{2.406953in}}%
\pgfpathlineto{\pgfqpoint{3.028488in}{2.394652in}}%
\pgfpathlineto{\pgfqpoint{3.028772in}{2.393038in}}%
\pgfpathlineto{\pgfqpoint{3.028962in}{2.398149in}}%
\pgfpathlineto{\pgfqpoint{3.029246in}{2.408641in}}%
\pgfpathlineto{\pgfqpoint{3.029720in}{2.382050in}}%
\pgfpathlineto{\pgfqpoint{3.029814in}{2.384736in}}%
\pgfpathlineto{\pgfqpoint{3.030952in}{2.400757in}}%
\pgfpathlineto{\pgfqpoint{3.031236in}{2.399578in}}%
\pgfpathlineto{\pgfqpoint{3.031804in}{2.394754in}}%
\pgfpathlineto{\pgfqpoint{3.032183in}{2.396626in}}%
\pgfpathlineto{\pgfqpoint{3.032373in}{2.388825in}}%
\pgfpathlineto{\pgfqpoint{3.032657in}{2.371679in}}%
\pgfpathlineto{\pgfqpoint{3.033510in}{2.383780in}}%
\pgfpathlineto{\pgfqpoint{3.035026in}{2.400363in}}%
\pgfpathlineto{\pgfqpoint{3.035500in}{2.387904in}}%
\pgfpathlineto{\pgfqpoint{3.036069in}{2.401028in}}%
\pgfpathlineto{\pgfqpoint{3.036164in}{2.401143in}}%
\pgfpathlineto{\pgfqpoint{3.036258in}{2.400391in}}%
\pgfpathlineto{\pgfqpoint{3.037206in}{2.389640in}}%
\pgfpathlineto{\pgfqpoint{3.037490in}{2.396875in}}%
\pgfpathlineto{\pgfqpoint{3.037585in}{2.397894in}}%
\pgfpathlineto{\pgfqpoint{3.037775in}{2.389648in}}%
\pgfpathlineto{\pgfqpoint{3.038722in}{2.376351in}}%
\pgfpathlineto{\pgfqpoint{3.039007in}{2.380951in}}%
\pgfpathlineto{\pgfqpoint{3.039101in}{2.381166in}}%
\pgfpathlineto{\pgfqpoint{3.039196in}{2.380039in}}%
\pgfpathlineto{\pgfqpoint{3.040049in}{2.356686in}}%
\pgfpathlineto{\pgfqpoint{3.040428in}{2.372843in}}%
\pgfpathlineto{\pgfqpoint{3.040523in}{2.374189in}}%
\pgfpathlineto{\pgfqpoint{3.040712in}{2.366291in}}%
\pgfpathlineto{\pgfqpoint{3.042134in}{2.339791in}}%
\pgfpathlineto{\pgfqpoint{3.042228in}{2.339874in}}%
\pgfpathlineto{\pgfqpoint{3.042323in}{2.338960in}}%
\pgfpathlineto{\pgfqpoint{3.042702in}{2.331067in}}%
\pgfpathlineto{\pgfqpoint{3.042987in}{2.341210in}}%
\pgfpathlineto{\pgfqpoint{3.043271in}{2.349959in}}%
\pgfpathlineto{\pgfqpoint{3.043745in}{2.327381in}}%
\pgfpathlineto{\pgfqpoint{3.043839in}{2.327635in}}%
\pgfpathlineto{\pgfqpoint{3.044692in}{2.336422in}}%
\pgfpathlineto{\pgfqpoint{3.044977in}{2.329748in}}%
\pgfpathlineto{\pgfqpoint{3.045166in}{2.324058in}}%
\pgfpathlineto{\pgfqpoint{3.045735in}{2.340051in}}%
\pgfpathlineto{\pgfqpoint{3.046114in}{2.344397in}}%
\pgfpathlineto{\pgfqpoint{3.047725in}{2.381975in}}%
\pgfpathlineto{\pgfqpoint{3.047820in}{2.380793in}}%
\pgfpathlineto{\pgfqpoint{3.048483in}{2.371688in}}%
\pgfpathlineto{\pgfqpoint{3.048862in}{2.382305in}}%
\pgfpathlineto{\pgfqpoint{3.049525in}{2.400986in}}%
\pgfpathlineto{\pgfqpoint{3.049999in}{2.384953in}}%
\pgfpathlineto{\pgfqpoint{3.050094in}{2.384888in}}%
\pgfpathlineto{\pgfqpoint{3.050189in}{2.385776in}}%
\pgfpathlineto{\pgfqpoint{3.050378in}{2.387072in}}%
\pgfpathlineto{\pgfqpoint{3.050568in}{2.379530in}}%
\pgfpathlineto{\pgfqpoint{3.051610in}{2.314502in}}%
\pgfpathlineto{\pgfqpoint{3.052273in}{2.333377in}}%
\pgfpathlineto{\pgfqpoint{3.052558in}{2.340560in}}%
\pgfpathlineto{\pgfqpoint{3.052937in}{2.327497in}}%
\pgfpathlineto{\pgfqpoint{3.054358in}{2.297897in}}%
\pgfpathlineto{\pgfqpoint{3.054832in}{2.301492in}}%
\pgfpathlineto{\pgfqpoint{3.055495in}{2.337046in}}%
\pgfpathlineto{\pgfqpoint{3.056348in}{2.314729in}}%
\pgfpathlineto{\pgfqpoint{3.057296in}{2.297681in}}%
\pgfpathlineto{\pgfqpoint{3.057675in}{2.305844in}}%
\pgfpathlineto{\pgfqpoint{3.058338in}{2.298103in}}%
\pgfpathlineto{\pgfqpoint{3.058812in}{2.346031in}}%
\pgfpathlineto{\pgfqpoint{3.060044in}{2.530079in}}%
\pgfpathlineto{\pgfqpoint{3.060897in}{2.516072in}}%
\pgfpathlineto{\pgfqpoint{3.061466in}{2.372967in}}%
\pgfpathlineto{\pgfqpoint{3.061939in}{2.238758in}}%
\pgfpathlineto{\pgfqpoint{3.062603in}{2.328778in}}%
\pgfpathlineto{\pgfqpoint{3.062698in}{2.330274in}}%
\pgfpathlineto{\pgfqpoint{3.063171in}{2.321284in}}%
\pgfpathlineto{\pgfqpoint{3.063266in}{2.321329in}}%
\pgfpathlineto{\pgfqpoint{3.064972in}{2.367435in}}%
\pgfpathlineto{\pgfqpoint{3.065161in}{2.371441in}}%
\pgfpathlineto{\pgfqpoint{3.065540in}{2.350490in}}%
\pgfpathlineto{\pgfqpoint{3.066299in}{2.334393in}}%
\pgfpathlineto{\pgfqpoint{3.066583in}{2.352513in}}%
\pgfpathlineto{\pgfqpoint{3.068004in}{2.375161in}}%
\pgfpathlineto{\pgfqpoint{3.068194in}{2.371343in}}%
\pgfpathlineto{\pgfqpoint{3.068762in}{2.325657in}}%
\pgfpathlineto{\pgfqpoint{3.069426in}{2.360147in}}%
\pgfpathlineto{\pgfqpoint{3.069521in}{2.359187in}}%
\pgfpathlineto{\pgfqpoint{3.069710in}{2.366771in}}%
\pgfpathlineto{\pgfqpoint{3.070468in}{2.361954in}}%
\pgfpathlineto{\pgfqpoint{3.071037in}{2.388595in}}%
\pgfpathlineto{\pgfqpoint{3.071511in}{2.384996in}}%
\pgfpathlineto{\pgfqpoint{3.072079in}{2.355274in}}%
\pgfpathlineto{\pgfqpoint{3.072553in}{2.378947in}}%
\pgfpathlineto{\pgfqpoint{3.073595in}{2.407598in}}%
\pgfpathlineto{\pgfqpoint{3.073785in}{2.402014in}}%
\pgfpathlineto{\pgfqpoint{3.074448in}{2.413040in}}%
\pgfpathlineto{\pgfqpoint{3.075112in}{2.388062in}}%
\pgfpathlineto{\pgfqpoint{3.075206in}{2.388112in}}%
\pgfpathlineto{\pgfqpoint{3.076249in}{2.407920in}}%
\pgfpathlineto{\pgfqpoint{3.075775in}{2.386847in}}%
\pgfpathlineto{\pgfqpoint{3.076628in}{2.396657in}}%
\pgfpathlineto{\pgfqpoint{3.076912in}{2.391344in}}%
\pgfpathlineto{\pgfqpoint{3.077955in}{2.377430in}}%
\pgfpathlineto{\pgfqpoint{3.077481in}{2.397804in}}%
\pgfpathlineto{\pgfqpoint{3.078144in}{2.385489in}}%
\pgfpathlineto{\pgfqpoint{3.079376in}{2.396326in}}%
\pgfpathlineto{\pgfqpoint{3.079471in}{2.395808in}}%
\pgfpathlineto{\pgfqpoint{3.079945in}{2.385558in}}%
\pgfpathlineto{\pgfqpoint{3.080229in}{2.397396in}}%
\pgfpathlineto{\pgfqpoint{3.081177in}{2.403031in}}%
\pgfpathlineto{\pgfqpoint{3.080703in}{2.388619in}}%
\pgfpathlineto{\pgfqpoint{3.081271in}{2.397927in}}%
\pgfpathlineto{\pgfqpoint{3.081650in}{2.380910in}}%
\pgfpathlineto{\pgfqpoint{3.082598in}{2.381872in}}%
\pgfpathlineto{\pgfqpoint{3.084209in}{2.361311in}}%
\pgfpathlineto{\pgfqpoint{3.082977in}{2.385447in}}%
\pgfpathlineto{\pgfqpoint{3.084304in}{2.361353in}}%
\pgfpathlineto{\pgfqpoint{3.084399in}{2.361677in}}%
\pgfpathlineto{\pgfqpoint{3.084493in}{2.360635in}}%
\pgfpathlineto{\pgfqpoint{3.086104in}{2.343514in}}%
\pgfpathlineto{\pgfqpoint{3.086578in}{2.361404in}}%
\pgfpathlineto{\pgfqpoint{3.087336in}{2.346986in}}%
\pgfpathlineto{\pgfqpoint{3.087526in}{2.345297in}}%
\pgfpathlineto{\pgfqpoint{3.087905in}{2.352676in}}%
\pgfpathlineto{\pgfqpoint{3.088094in}{2.351278in}}%
\pgfpathlineto{\pgfqpoint{3.089705in}{2.363488in}}%
\pgfpathlineto{\pgfqpoint{3.088758in}{2.349386in}}%
\pgfpathlineto{\pgfqpoint{3.089990in}{2.357491in}}%
\pgfpathlineto{\pgfqpoint{3.092264in}{2.319474in}}%
\pgfpathlineto{\pgfqpoint{3.092927in}{2.320787in}}%
\pgfpathlineto{\pgfqpoint{3.093591in}{2.339339in}}%
\pgfpathlineto{\pgfqpoint{3.094633in}{2.434011in}}%
\pgfpathlineto{\pgfqpoint{3.095486in}{2.418651in}}%
\pgfpathlineto{\pgfqpoint{3.096434in}{2.376837in}}%
\pgfpathlineto{\pgfqpoint{3.096907in}{2.350055in}}%
\pgfpathlineto{\pgfqpoint{3.097665in}{2.366349in}}%
\pgfpathlineto{\pgfqpoint{3.097855in}{2.372233in}}%
\pgfpathlineto{\pgfqpoint{3.098234in}{2.354231in}}%
\pgfpathlineto{\pgfqpoint{3.099466in}{2.336749in}}%
\pgfpathlineto{\pgfqpoint{3.099561in}{2.338078in}}%
\pgfpathlineto{\pgfqpoint{3.101267in}{2.356574in}}%
\pgfpathlineto{\pgfqpoint{3.100129in}{2.330179in}}%
\pgfpathlineto{\pgfqpoint{3.101456in}{2.353437in}}%
\pgfpathlineto{\pgfqpoint{3.102878in}{2.323292in}}%
\pgfpathlineto{\pgfqpoint{3.102972in}{2.324625in}}%
\pgfpathlineto{\pgfqpoint{3.104204in}{2.402551in}}%
\pgfpathlineto{\pgfqpoint{3.105057in}{2.565503in}}%
\pgfpathlineto{\pgfqpoint{3.105720in}{2.534355in}}%
\pgfpathlineto{\pgfqpoint{3.106099in}{2.546670in}}%
\pgfpathlineto{\pgfqpoint{3.106289in}{2.531833in}}%
\pgfpathlineto{\pgfqpoint{3.107142in}{2.270463in}}%
\pgfpathlineto{\pgfqpoint{3.108279in}{2.345371in}}%
\pgfpathlineto{\pgfqpoint{3.110269in}{2.405455in}}%
\pgfpathlineto{\pgfqpoint{3.110553in}{2.390093in}}%
\pgfpathlineto{\pgfqpoint{3.110838in}{2.373889in}}%
\pgfpathlineto{\pgfqpoint{3.111691in}{2.383596in}}%
\pgfpathlineto{\pgfqpoint{3.112354in}{2.371572in}}%
\pgfpathlineto{\pgfqpoint{3.113017in}{2.400931in}}%
\pgfpathlineto{\pgfqpoint{3.113302in}{2.397120in}}%
\pgfpathlineto{\pgfqpoint{3.114344in}{2.360239in}}%
\pgfpathlineto{\pgfqpoint{3.114913in}{2.384041in}}%
\pgfpathlineto{\pgfqpoint{3.116524in}{2.412387in}}%
\pgfpathlineto{\pgfqpoint{3.117566in}{2.400182in}}%
\pgfpathlineto{\pgfqpoint{3.117850in}{2.407417in}}%
\pgfpathlineto{\pgfqpoint{3.118703in}{2.425082in}}%
\pgfpathlineto{\pgfqpoint{3.119082in}{2.416349in}}%
\pgfpathlineto{\pgfqpoint{3.119177in}{2.415318in}}%
\pgfpathlineto{\pgfqpoint{3.119935in}{2.418007in}}%
\pgfpathlineto{\pgfqpoint{3.120030in}{2.418222in}}%
\pgfpathlineto{\pgfqpoint{3.120219in}{2.416333in}}%
\pgfpathlineto{\pgfqpoint{3.120409in}{2.414162in}}%
\pgfpathlineto{\pgfqpoint{3.120883in}{2.419471in}}%
\pgfpathlineto{\pgfqpoint{3.122020in}{2.442941in}}%
\pgfpathlineto{\pgfqpoint{3.122588in}{2.436797in}}%
\pgfpathlineto{\pgfqpoint{3.122683in}{2.438111in}}%
\pgfpathlineto{\pgfqpoint{3.122873in}{2.429175in}}%
\pgfpathlineto{\pgfqpoint{3.124199in}{2.413641in}}%
\pgfpathlineto{\pgfqpoint{3.124294in}{2.411064in}}%
\pgfpathlineto{\pgfqpoint{3.124673in}{2.426200in}}%
\pgfpathlineto{\pgfqpoint{3.124768in}{2.429223in}}%
\pgfpathlineto{\pgfqpoint{3.125242in}{2.413919in}}%
\pgfpathlineto{\pgfqpoint{3.125526in}{2.409852in}}%
\pgfpathlineto{\pgfqpoint{3.125905in}{2.420943in}}%
\pgfpathlineto{\pgfqpoint{3.126189in}{2.431433in}}%
\pgfpathlineto{\pgfqpoint{3.127042in}{2.425392in}}%
\pgfpathlineto{\pgfqpoint{3.128274in}{2.412462in}}%
\pgfpathlineto{\pgfqpoint{3.127800in}{2.430233in}}%
\pgfpathlineto{\pgfqpoint{3.128464in}{2.414448in}}%
\pgfpathlineto{\pgfqpoint{3.128559in}{2.414800in}}%
\pgfpathlineto{\pgfqpoint{3.128653in}{2.413313in}}%
\pgfpathlineto{\pgfqpoint{3.129791in}{2.393442in}}%
\pgfpathlineto{\pgfqpoint{3.130170in}{2.401955in}}%
\pgfpathlineto{\pgfqpoint{3.130454in}{2.391218in}}%
\pgfpathlineto{\pgfqpoint{3.132160in}{2.352638in}}%
\pgfpathlineto{\pgfqpoint{3.133202in}{2.343822in}}%
\pgfpathlineto{\pgfqpoint{3.132539in}{2.356375in}}%
\pgfpathlineto{\pgfqpoint{3.133392in}{2.348985in}}%
\pgfpathlineto{\pgfqpoint{3.134434in}{2.366878in}}%
\pgfpathlineto{\pgfqpoint{3.134623in}{2.362084in}}%
\pgfpathlineto{\pgfqpoint{3.136234in}{2.333907in}}%
\pgfpathlineto{\pgfqpoint{3.136424in}{2.340329in}}%
\pgfpathlineto{\pgfqpoint{3.138035in}{2.371832in}}%
\pgfpathlineto{\pgfqpoint{3.138130in}{2.371001in}}%
\pgfpathlineto{\pgfqpoint{3.138604in}{2.351028in}}%
\pgfpathlineto{\pgfqpoint{3.139267in}{2.366703in}}%
\pgfpathlineto{\pgfqpoint{3.140120in}{2.395113in}}%
\pgfpathlineto{\pgfqpoint{3.140499in}{2.381143in}}%
\pgfpathlineto{\pgfqpoint{3.142489in}{2.314586in}}%
\pgfpathlineto{\pgfqpoint{3.141352in}{2.384499in}}%
\pgfpathlineto{\pgfqpoint{3.142678in}{2.322704in}}%
\pgfpathlineto{\pgfqpoint{3.142963in}{2.341543in}}%
\pgfpathlineto{\pgfqpoint{3.143816in}{2.327837in}}%
\pgfpathlineto{\pgfqpoint{3.145427in}{2.303786in}}%
\pgfpathlineto{\pgfqpoint{3.145521in}{2.303739in}}%
\pgfpathlineto{\pgfqpoint{3.146564in}{2.337195in}}%
\pgfpathlineto{\pgfqpoint{3.147227in}{2.320557in}}%
\pgfpathlineto{\pgfqpoint{3.147322in}{2.321155in}}%
\pgfpathlineto{\pgfqpoint{3.147511in}{2.318225in}}%
\pgfpathlineto{\pgfqpoint{3.147701in}{2.313405in}}%
\pgfpathlineto{\pgfqpoint{3.148554in}{2.316380in}}%
\pgfpathlineto{\pgfqpoint{3.149407in}{2.353464in}}%
\pgfpathlineto{\pgfqpoint{3.150449in}{2.554422in}}%
\pgfpathlineto{\pgfqpoint{3.151776in}{2.518659in}}%
\pgfpathlineto{\pgfqpoint{3.152629in}{2.269448in}}%
\pgfpathlineto{\pgfqpoint{3.153671in}{2.348531in}}%
\pgfpathlineto{\pgfqpoint{3.155661in}{2.398943in}}%
\pgfpathlineto{\pgfqpoint{3.156040in}{2.387949in}}%
\pgfpathlineto{\pgfqpoint{3.156419in}{2.380106in}}%
\pgfpathlineto{\pgfqpoint{3.157083in}{2.386136in}}%
\pgfpathlineto{\pgfqpoint{3.158788in}{2.437072in}}%
\pgfpathlineto{\pgfqpoint{3.159073in}{2.428735in}}%
\pgfpathlineto{\pgfqpoint{3.159641in}{2.407250in}}%
\pgfpathlineto{\pgfqpoint{3.160210in}{2.418168in}}%
\pgfpathlineto{\pgfqpoint{3.161537in}{2.442836in}}%
\pgfpathlineto{\pgfqpoint{3.161631in}{2.442469in}}%
\pgfpathlineto{\pgfqpoint{3.161821in}{2.442823in}}%
\pgfpathlineto{\pgfqpoint{3.161916in}{2.442166in}}%
\pgfpathlineto{\pgfqpoint{3.163147in}{2.391453in}}%
\pgfpathlineto{\pgfqpoint{3.164379in}{2.399644in}}%
\pgfpathlineto{\pgfqpoint{3.167128in}{2.497552in}}%
\pgfpathlineto{\pgfqpoint{3.167412in}{2.493986in}}%
\pgfpathlineto{\pgfqpoint{3.167507in}{2.492772in}}%
\pgfpathlineto{\pgfqpoint{3.167791in}{2.501075in}}%
\pgfpathlineto{\pgfqpoint{3.167980in}{2.505132in}}%
\pgfpathlineto{\pgfqpoint{3.168454in}{2.487724in}}%
\pgfpathlineto{\pgfqpoint{3.168739in}{2.484551in}}%
\pgfpathlineto{\pgfqpoint{3.169307in}{2.489934in}}%
\pgfpathlineto{\pgfqpoint{3.170255in}{2.513503in}}%
\pgfpathlineto{\pgfqpoint{3.170918in}{2.505720in}}%
\pgfpathlineto{\pgfqpoint{3.171013in}{2.506254in}}%
\pgfpathlineto{\pgfqpoint{3.171297in}{2.501554in}}%
\pgfpathlineto{\pgfqpoint{3.171487in}{2.498543in}}%
\pgfpathlineto{\pgfqpoint{3.171866in}{2.513422in}}%
\pgfpathlineto{\pgfqpoint{3.171961in}{2.514450in}}%
\pgfpathlineto{\pgfqpoint{3.172055in}{2.510859in}}%
\pgfpathlineto{\pgfqpoint{3.172434in}{2.487583in}}%
\pgfpathlineto{\pgfqpoint{3.173192in}{2.503838in}}%
\pgfpathlineto{\pgfqpoint{3.173287in}{2.504570in}}%
\pgfpathlineto{\pgfqpoint{3.173572in}{2.499209in}}%
\pgfpathlineto{\pgfqpoint{3.176509in}{2.430744in}}%
\pgfpathlineto{\pgfqpoint{3.176794in}{2.442256in}}%
\pgfpathlineto{\pgfqpoint{3.176983in}{2.444723in}}%
\pgfpathlineto{\pgfqpoint{3.177552in}{2.436117in}}%
\pgfpathlineto{\pgfqpoint{3.177836in}{2.424389in}}%
\pgfpathlineto{\pgfqpoint{3.178310in}{2.437469in}}%
\pgfpathlineto{\pgfqpoint{3.178689in}{2.431232in}}%
\pgfpathlineto{\pgfqpoint{3.179731in}{2.443121in}}%
\pgfpathlineto{\pgfqpoint{3.179921in}{2.438727in}}%
\pgfpathlineto{\pgfqpoint{3.180584in}{2.438934in}}%
\pgfpathlineto{\pgfqpoint{3.181626in}{2.417460in}}%
\pgfpathlineto{\pgfqpoint{3.182006in}{2.427326in}}%
\pgfpathlineto{\pgfqpoint{3.183237in}{2.450737in}}%
\pgfpathlineto{\pgfqpoint{3.183996in}{2.447256in}}%
\pgfpathlineto{\pgfqpoint{3.184375in}{2.428732in}}%
\pgfpathlineto{\pgfqpoint{3.184848in}{2.449069in}}%
\pgfpathlineto{\pgfqpoint{3.185322in}{2.469460in}}%
\pgfpathlineto{\pgfqpoint{3.185986in}{2.454862in}}%
\pgfpathlineto{\pgfqpoint{3.187691in}{2.382134in}}%
\pgfpathlineto{\pgfqpoint{3.188450in}{2.401627in}}%
\pgfpathlineto{\pgfqpoint{3.188639in}{2.406428in}}%
\pgfpathlineto{\pgfqpoint{3.189397in}{2.397074in}}%
\pgfpathlineto{\pgfqpoint{3.190440in}{2.385052in}}%
\pgfpathlineto{\pgfqpoint{3.190629in}{2.387925in}}%
\pgfpathlineto{\pgfqpoint{3.191008in}{2.386033in}}%
\pgfpathlineto{\pgfqpoint{3.191292in}{2.401051in}}%
\pgfpathlineto{\pgfqpoint{3.191861in}{2.427684in}}%
\pgfpathlineto{\pgfqpoint{3.192430in}{2.407325in}}%
\pgfpathlineto{\pgfqpoint{3.193567in}{2.390507in}}%
\pgfpathlineto{\pgfqpoint{3.193756in}{2.395308in}}%
\pgfpathlineto{\pgfqpoint{3.195557in}{2.602095in}}%
\pgfpathlineto{\pgfqpoint{3.195936in}{2.645762in}}%
\pgfpathlineto{\pgfqpoint{3.196694in}{2.621381in}}%
\pgfpathlineto{\pgfqpoint{3.196789in}{2.622081in}}%
\pgfpathlineto{\pgfqpoint{3.196978in}{2.619062in}}%
\pgfpathlineto{\pgfqpoint{3.197547in}{2.487960in}}%
\pgfpathlineto{\pgfqpoint{3.198021in}{2.347696in}}%
\pgfpathlineto{\pgfqpoint{3.198684in}{2.434028in}}%
\pgfpathlineto{\pgfqpoint{3.201148in}{2.484101in}}%
\pgfpathlineto{\pgfqpoint{3.201243in}{2.484187in}}%
\pgfpathlineto{\pgfqpoint{3.201906in}{2.458665in}}%
\pgfpathlineto{\pgfqpoint{3.202664in}{2.477298in}}%
\pgfpathlineto{\pgfqpoint{3.203896in}{2.501350in}}%
\pgfpathlineto{\pgfqpoint{3.204180in}{2.515030in}}%
\pgfpathlineto{\pgfqpoint{3.204654in}{2.485753in}}%
\pgfpathlineto{\pgfqpoint{3.204844in}{2.489195in}}%
\pgfpathlineto{\pgfqpoint{3.205223in}{2.480399in}}%
\pgfpathlineto{\pgfqpoint{3.205412in}{2.489264in}}%
\pgfpathlineto{\pgfqpoint{3.206929in}{2.525054in}}%
\pgfpathlineto{\pgfqpoint{3.207213in}{2.522227in}}%
\pgfpathlineto{\pgfqpoint{3.207402in}{2.526612in}}%
\pgfpathlineto{\pgfqpoint{3.207592in}{2.529433in}}%
\pgfpathlineto{\pgfqpoint{3.207971in}{2.511273in}}%
\pgfpathlineto{\pgfqpoint{3.208066in}{2.510600in}}%
\pgfpathlineto{\pgfqpoint{3.208160in}{2.513297in}}%
\pgfpathlineto{\pgfqpoint{3.209298in}{2.547542in}}%
\pgfpathlineto{\pgfqpoint{3.209677in}{2.533666in}}%
\pgfpathlineto{\pgfqpoint{3.209961in}{2.540200in}}%
\pgfpathlineto{\pgfqpoint{3.210150in}{2.542761in}}%
\pgfpathlineto{\pgfqpoint{3.210624in}{2.536445in}}%
\pgfpathlineto{\pgfqpoint{3.210909in}{2.522981in}}%
\pgfpathlineto{\pgfqpoint{3.211288in}{2.541448in}}%
\pgfpathlineto{\pgfqpoint{3.211761in}{2.527445in}}%
\pgfpathlineto{\pgfqpoint{3.213183in}{2.547779in}}%
\pgfpathlineto{\pgfqpoint{3.213941in}{2.530621in}}%
\pgfpathlineto{\pgfqpoint{3.214225in}{2.519455in}}%
\pgfpathlineto{\pgfqpoint{3.214983in}{2.532223in}}%
\pgfpathlineto{\pgfqpoint{3.215363in}{2.551890in}}%
\pgfpathlineto{\pgfqpoint{3.216121in}{2.534358in}}%
\pgfpathlineto{\pgfqpoint{3.217163in}{2.545750in}}%
\pgfpathlineto{\pgfqpoint{3.217447in}{2.537806in}}%
\pgfpathlineto{\pgfqpoint{3.217637in}{2.531084in}}%
\pgfpathlineto{\pgfqpoint{3.218490in}{2.539843in}}%
\pgfpathlineto{\pgfqpoint{3.221427in}{2.476635in}}%
\pgfpathlineto{\pgfqpoint{3.222375in}{2.491970in}}%
\pgfpathlineto{\pgfqpoint{3.222565in}{2.498888in}}%
\pgfpathlineto{\pgfqpoint{3.222944in}{2.478748in}}%
\pgfpathlineto{\pgfqpoint{3.223323in}{2.486980in}}%
\pgfpathlineto{\pgfqpoint{3.223607in}{2.475942in}}%
\pgfpathlineto{\pgfqpoint{3.223986in}{2.487661in}}%
\pgfpathlineto{\pgfqpoint{3.224460in}{2.482343in}}%
\pgfpathlineto{\pgfqpoint{3.224744in}{2.494289in}}%
\pgfpathlineto{\pgfqpoint{3.225123in}{2.477955in}}%
\pgfpathlineto{\pgfqpoint{3.225597in}{2.485209in}}%
\pgfpathlineto{\pgfqpoint{3.226639in}{2.460505in}}%
\pgfpathlineto{\pgfqpoint{3.226924in}{2.469722in}}%
\pgfpathlineto{\pgfqpoint{3.227208in}{2.466715in}}%
\pgfpathlineto{\pgfqpoint{3.227492in}{2.474180in}}%
\pgfpathlineto{\pgfqpoint{3.228061in}{2.467030in}}%
\pgfpathlineto{\pgfqpoint{3.228914in}{2.491890in}}%
\pgfpathlineto{\pgfqpoint{3.229009in}{2.492015in}}%
\pgfpathlineto{\pgfqpoint{3.229482in}{2.465898in}}%
\pgfpathlineto{\pgfqpoint{3.230051in}{2.493110in}}%
\pgfpathlineto{\pgfqpoint{3.230525in}{2.518630in}}%
\pgfpathlineto{\pgfqpoint{3.231188in}{2.498486in}}%
\pgfpathlineto{\pgfqpoint{3.233273in}{2.364477in}}%
\pgfpathlineto{\pgfqpoint{3.233557in}{2.374435in}}%
\pgfpathlineto{\pgfqpoint{3.235453in}{2.450919in}}%
\pgfpathlineto{\pgfqpoint{3.236021in}{2.445280in}}%
\pgfpathlineto{\pgfqpoint{3.236400in}{2.441425in}}%
\pgfpathlineto{\pgfqpoint{3.236495in}{2.443993in}}%
\pgfpathlineto{\pgfqpoint{3.237158in}{2.473452in}}%
\pgfpathlineto{\pgfqpoint{3.237632in}{2.447803in}}%
\pgfpathlineto{\pgfqpoint{3.238295in}{2.433138in}}%
\pgfpathlineto{\pgfqpoint{3.238580in}{2.447623in}}%
\pgfpathlineto{\pgfqpoint{3.239054in}{2.436380in}}%
\pgfpathlineto{\pgfqpoint{3.239906in}{2.458493in}}%
\pgfpathlineto{\pgfqpoint{3.240285in}{2.487705in}}%
\pgfpathlineto{\pgfqpoint{3.241328in}{2.684070in}}%
\pgfpathlineto{\pgfqpoint{3.242276in}{2.662319in}}%
\pgfpathlineto{\pgfqpoint{3.242370in}{2.662275in}}%
\pgfpathlineto{\pgfqpoint{3.242749in}{2.584962in}}%
\pgfpathlineto{\pgfqpoint{3.243413in}{2.377104in}}%
\pgfpathlineto{\pgfqpoint{3.244171in}{2.456967in}}%
\pgfpathlineto{\pgfqpoint{3.244929in}{2.472438in}}%
\pgfpathlineto{\pgfqpoint{3.246445in}{2.509138in}}%
\pgfpathlineto{\pgfqpoint{3.246729in}{2.494667in}}%
\pgfpathlineto{\pgfqpoint{3.247582in}{2.471151in}}%
\pgfpathlineto{\pgfqpoint{3.247961in}{2.481294in}}%
\pgfpathlineto{\pgfqpoint{3.249099in}{2.503988in}}%
\pgfpathlineto{\pgfqpoint{3.248530in}{2.475222in}}%
\pgfpathlineto{\pgfqpoint{3.249762in}{2.495695in}}%
\pgfpathlineto{\pgfqpoint{3.250710in}{2.467273in}}%
\pgfpathlineto{\pgfqpoint{3.250994in}{2.481774in}}%
\pgfpathlineto{\pgfqpoint{3.251941in}{2.499215in}}%
\pgfpathlineto{\pgfqpoint{3.252226in}{2.493261in}}%
\pgfpathlineto{\pgfqpoint{3.252605in}{2.505439in}}%
\pgfpathlineto{\pgfqpoint{3.252984in}{2.491070in}}%
\pgfpathlineto{\pgfqpoint{3.253552in}{2.472440in}}%
\pgfpathlineto{\pgfqpoint{3.254216in}{2.486148in}}%
\pgfpathlineto{\pgfqpoint{3.255163in}{2.494928in}}%
\pgfpathlineto{\pgfqpoint{3.255637in}{2.489984in}}%
\pgfpathlineto{\pgfqpoint{3.255732in}{2.489972in}}%
\pgfpathlineto{\pgfqpoint{3.256111in}{2.493479in}}%
\pgfpathlineto{\pgfqpoint{3.256490in}{2.486757in}}%
\pgfpathlineto{\pgfqpoint{3.256869in}{2.477832in}}%
\pgfpathlineto{\pgfqpoint{3.257153in}{2.495500in}}%
\pgfpathlineto{\pgfqpoint{3.257343in}{2.505517in}}%
\pgfpathlineto{\pgfqpoint{3.257722in}{2.491413in}}%
\pgfpathlineto{\pgfqpoint{3.258291in}{2.501431in}}%
\pgfpathlineto{\pgfqpoint{3.259902in}{2.472578in}}%
\pgfpathlineto{\pgfqpoint{3.258764in}{2.504062in}}%
\pgfpathlineto{\pgfqpoint{3.260755in}{2.480852in}}%
\pgfpathlineto{\pgfqpoint{3.261892in}{2.489548in}}%
\pgfpathlineto{\pgfqpoint{3.262081in}{2.483641in}}%
\pgfpathlineto{\pgfqpoint{3.263124in}{2.471217in}}%
\pgfpathlineto{\pgfqpoint{3.262650in}{2.497749in}}%
\pgfpathlineto{\pgfqpoint{3.263218in}{2.473916in}}%
\pgfpathlineto{\pgfqpoint{3.263503in}{2.485754in}}%
\pgfpathlineto{\pgfqpoint{3.264166in}{2.467431in}}%
\pgfpathlineto{\pgfqpoint{3.265682in}{2.442722in}}%
\pgfpathlineto{\pgfqpoint{3.265872in}{2.448006in}}%
\pgfpathlineto{\pgfqpoint{3.266156in}{2.451629in}}%
\pgfpathlineto{\pgfqpoint{3.266535in}{2.442278in}}%
\pgfpathlineto{\pgfqpoint{3.267578in}{2.421959in}}%
\pgfpathlineto{\pgfqpoint{3.267767in}{2.431170in}}%
\pgfpathlineto{\pgfqpoint{3.267862in}{2.435244in}}%
\pgfpathlineto{\pgfqpoint{3.268336in}{2.417944in}}%
\pgfpathlineto{\pgfqpoint{3.268715in}{2.427510in}}%
\pgfpathlineto{\pgfqpoint{3.268999in}{2.411333in}}%
\pgfpathlineto{\pgfqpoint{3.269473in}{2.433151in}}%
\pgfpathlineto{\pgfqpoint{3.269757in}{2.425327in}}%
\pgfpathlineto{\pgfqpoint{3.270136in}{2.439436in}}%
\pgfpathlineto{\pgfqpoint{3.270894in}{2.428049in}}%
\pgfpathlineto{\pgfqpoint{3.270989in}{2.427734in}}%
\pgfpathlineto{\pgfqpoint{3.271179in}{2.429410in}}%
\pgfpathlineto{\pgfqpoint{3.271463in}{2.437593in}}%
\pgfpathlineto{\pgfqpoint{3.271747in}{2.422450in}}%
\pgfpathlineto{\pgfqpoint{3.271937in}{2.411265in}}%
\pgfpathlineto{\pgfqpoint{3.272411in}{2.434367in}}%
\pgfpathlineto{\pgfqpoint{3.272695in}{2.431557in}}%
\pgfpathlineto{\pgfqpoint{3.274022in}{2.470746in}}%
\pgfpathlineto{\pgfqpoint{3.274685in}{2.459832in}}%
\pgfpathlineto{\pgfqpoint{3.274969in}{2.449175in}}%
\pgfpathlineto{\pgfqpoint{3.275632in}{2.469072in}}%
\pgfpathlineto{\pgfqpoint{3.276106in}{2.493174in}}%
\pgfpathlineto{\pgfqpoint{3.276580in}{2.471189in}}%
\pgfpathlineto{\pgfqpoint{3.278286in}{2.395598in}}%
\pgfpathlineto{\pgfqpoint{3.278665in}{2.405417in}}%
\pgfpathlineto{\pgfqpoint{3.278854in}{2.403687in}}%
\pgfpathlineto{\pgfqpoint{3.279044in}{2.407158in}}%
\pgfpathlineto{\pgfqpoint{3.279328in}{2.415610in}}%
\pgfpathlineto{\pgfqpoint{3.279707in}{2.388518in}}%
\pgfpathlineto{\pgfqpoint{3.281697in}{2.367149in}}%
\pgfpathlineto{\pgfqpoint{3.281792in}{2.370495in}}%
\pgfpathlineto{\pgfqpoint{3.282266in}{2.407769in}}%
\pgfpathlineto{\pgfqpoint{3.283024in}{2.394870in}}%
\pgfpathlineto{\pgfqpoint{3.284351in}{2.366708in}}%
\pgfpathlineto{\pgfqpoint{3.284540in}{2.364321in}}%
\pgfpathlineto{\pgfqpoint{3.284825in}{2.379146in}}%
\pgfpathlineto{\pgfqpoint{3.286246in}{2.582322in}}%
\pgfpathlineto{\pgfqpoint{3.286625in}{2.619394in}}%
\pgfpathlineto{\pgfqpoint{3.287383in}{2.592227in}}%
\pgfpathlineto{\pgfqpoint{3.287478in}{2.591959in}}%
\pgfpathlineto{\pgfqpoint{3.287573in}{2.594558in}}%
\pgfpathlineto{\pgfqpoint{3.287668in}{2.597279in}}%
\pgfpathlineto{\pgfqpoint{3.287857in}{2.585142in}}%
\pgfpathlineto{\pgfqpoint{3.288710in}{2.329900in}}%
\pgfpathlineto{\pgfqpoint{3.289752in}{2.407766in}}%
\pgfpathlineto{\pgfqpoint{3.291742in}{2.446235in}}%
\pgfpathlineto{\pgfqpoint{3.292027in}{2.436049in}}%
\pgfpathlineto{\pgfqpoint{3.292880in}{2.408477in}}%
\pgfpathlineto{\pgfqpoint{3.293164in}{2.426179in}}%
\pgfpathlineto{\pgfqpoint{3.294206in}{2.443220in}}%
\pgfpathlineto{\pgfqpoint{3.294396in}{2.442030in}}%
\pgfpathlineto{\pgfqpoint{3.294585in}{2.445677in}}%
\pgfpathlineto{\pgfqpoint{3.294870in}{2.456298in}}%
\pgfpathlineto{\pgfqpoint{3.295438in}{2.434686in}}%
\pgfpathlineto{\pgfqpoint{3.295817in}{2.424308in}}%
\pgfpathlineto{\pgfqpoint{3.296196in}{2.443644in}}%
\pgfpathlineto{\pgfqpoint{3.296386in}{2.450516in}}%
\pgfpathlineto{\pgfqpoint{3.297144in}{2.436509in}}%
\pgfpathlineto{\pgfqpoint{3.298850in}{2.376959in}}%
\pgfpathlineto{\pgfqpoint{3.299134in}{2.379577in}}%
\pgfpathlineto{\pgfqpoint{3.300461in}{2.471363in}}%
\pgfpathlineto{\pgfqpoint{3.301598in}{2.503612in}}%
\pgfpathlineto{\pgfqpoint{3.302072in}{2.502403in}}%
\pgfpathlineto{\pgfqpoint{3.302166in}{2.502475in}}%
\pgfpathlineto{\pgfqpoint{3.302261in}{2.501749in}}%
\pgfpathlineto{\pgfqpoint{3.302451in}{2.500041in}}%
\pgfpathlineto{\pgfqpoint{3.302830in}{2.509663in}}%
\pgfpathlineto{\pgfqpoint{3.303209in}{2.501378in}}%
\pgfpathlineto{\pgfqpoint{3.303588in}{2.511502in}}%
\pgfpathlineto{\pgfqpoint{3.303683in}{2.512327in}}%
\pgfpathlineto{\pgfqpoint{3.304251in}{2.510123in}}%
\pgfpathlineto{\pgfqpoint{3.304725in}{2.489507in}}%
\pgfpathlineto{\pgfqpoint{3.305483in}{2.499611in}}%
\pgfpathlineto{\pgfqpoint{3.306526in}{2.507552in}}%
\pgfpathlineto{\pgfqpoint{3.306810in}{2.505838in}}%
\pgfpathlineto{\pgfqpoint{3.307947in}{2.513322in}}%
\pgfpathlineto{\pgfqpoint{3.308042in}{2.510393in}}%
\pgfpathlineto{\pgfqpoint{3.308610in}{2.512824in}}%
\pgfpathlineto{\pgfqpoint{3.309653in}{2.487290in}}%
\pgfpathlineto{\pgfqpoint{3.309937in}{2.491990in}}%
\pgfpathlineto{\pgfqpoint{3.310127in}{2.485242in}}%
\pgfpathlineto{\pgfqpoint{3.311643in}{2.452731in}}%
\pgfpathlineto{\pgfqpoint{3.311738in}{2.453915in}}%
\pgfpathlineto{\pgfqpoint{3.312117in}{2.461401in}}%
\pgfpathlineto{\pgfqpoint{3.312401in}{2.447727in}}%
\pgfpathlineto{\pgfqpoint{3.312780in}{2.426134in}}%
\pgfpathlineto{\pgfqpoint{3.313538in}{2.442296in}}%
\pgfpathlineto{\pgfqpoint{3.313633in}{2.443958in}}%
\pgfpathlineto{\pgfqpoint{3.313917in}{2.435891in}}%
\pgfpathlineto{\pgfqpoint{3.314770in}{2.420381in}}%
\pgfpathlineto{\pgfqpoint{3.315054in}{2.428778in}}%
\pgfpathlineto{\pgfqpoint{3.315244in}{2.433419in}}%
\pgfpathlineto{\pgfqpoint{3.315812in}{2.418726in}}%
\pgfpathlineto{\pgfqpoint{3.316192in}{2.417478in}}%
\pgfpathlineto{\pgfqpoint{3.316381in}{2.419969in}}%
\pgfpathlineto{\pgfqpoint{3.316665in}{2.424647in}}%
\pgfpathlineto{\pgfqpoint{3.317044in}{2.410690in}}%
\pgfpathlineto{\pgfqpoint{3.317423in}{2.404609in}}%
\pgfpathlineto{\pgfqpoint{3.317897in}{2.414912in}}%
\pgfpathlineto{\pgfqpoint{3.319224in}{2.434549in}}%
\pgfpathlineto{\pgfqpoint{3.319414in}{2.430271in}}%
\pgfpathlineto{\pgfqpoint{3.320266in}{2.417677in}}%
\pgfpathlineto{\pgfqpoint{3.320456in}{2.426522in}}%
\pgfpathlineto{\pgfqpoint{3.320740in}{2.443844in}}%
\pgfpathlineto{\pgfqpoint{3.321688in}{2.442886in}}%
\pgfpathlineto{\pgfqpoint{3.323109in}{2.389229in}}%
\pgfpathlineto{\pgfqpoint{3.323678in}{2.363023in}}%
\pgfpathlineto{\pgfqpoint{3.324341in}{2.376892in}}%
\pgfpathlineto{\pgfqpoint{3.324626in}{2.396421in}}%
\pgfpathlineto{\pgfqpoint{3.325194in}{2.359764in}}%
\pgfpathlineto{\pgfqpoint{3.325478in}{2.368319in}}%
\pgfpathlineto{\pgfqpoint{3.325857in}{2.354780in}}%
\pgfpathlineto{\pgfqpoint{3.326142in}{2.357534in}}%
\pgfpathlineto{\pgfqpoint{3.326995in}{2.347504in}}%
\pgfpathlineto{\pgfqpoint{3.327089in}{2.351048in}}%
\pgfpathlineto{\pgfqpoint{3.327468in}{2.391140in}}%
\pgfpathlineto{\pgfqpoint{3.328227in}{2.357393in}}%
\pgfpathlineto{\pgfqpoint{3.328511in}{2.343054in}}%
\pgfpathlineto{\pgfqpoint{3.329364in}{2.350736in}}%
\pgfpathlineto{\pgfqpoint{3.329459in}{2.350798in}}%
\pgfpathlineto{\pgfqpoint{3.329838in}{2.335540in}}%
\pgfpathlineto{\pgfqpoint{3.330501in}{2.349357in}}%
\pgfpathlineto{\pgfqpoint{3.331069in}{2.441662in}}%
\pgfpathlineto{\pgfqpoint{3.332207in}{2.572225in}}%
\pgfpathlineto{\pgfqpoint{3.332586in}{2.564387in}}%
\pgfpathlineto{\pgfqpoint{3.332965in}{2.546058in}}%
\pgfpathlineto{\pgfqpoint{3.333912in}{2.269893in}}%
\pgfpathlineto{\pgfqpoint{3.335050in}{2.356941in}}%
\pgfpathlineto{\pgfqpoint{3.335144in}{2.355845in}}%
\pgfpathlineto{\pgfqpoint{3.335334in}{2.361299in}}%
\pgfpathlineto{\pgfqpoint{3.336945in}{2.399823in}}%
\pgfpathlineto{\pgfqpoint{3.337229in}{2.387475in}}%
\pgfpathlineto{\pgfqpoint{3.338082in}{2.360024in}}%
\pgfpathlineto{\pgfqpoint{3.338461in}{2.374538in}}%
\pgfpathlineto{\pgfqpoint{3.339977in}{2.409348in}}%
\pgfpathlineto{\pgfqpoint{3.338935in}{2.371921in}}%
\pgfpathlineto{\pgfqpoint{3.340356in}{2.389297in}}%
\pgfpathlineto{\pgfqpoint{3.340925in}{2.364053in}}%
\pgfpathlineto{\pgfqpoint{3.341399in}{2.383842in}}%
\pgfpathlineto{\pgfqpoint{3.343105in}{2.417399in}}%
\pgfpathlineto{\pgfqpoint{3.343578in}{2.416899in}}%
\pgfpathlineto{\pgfqpoint{3.344052in}{2.396655in}}%
\pgfpathlineto{\pgfqpoint{3.344716in}{2.413929in}}%
\pgfpathlineto{\pgfqpoint{3.345000in}{2.430878in}}%
\pgfpathlineto{\pgfqpoint{3.345853in}{2.427443in}}%
\pgfpathlineto{\pgfqpoint{3.346611in}{2.433754in}}%
\pgfpathlineto{\pgfqpoint{3.347274in}{2.410140in}}%
\pgfpathlineto{\pgfqpoint{3.349359in}{2.437126in}}%
\pgfpathlineto{\pgfqpoint{3.349643in}{2.425000in}}%
\pgfpathlineto{\pgfqpoint{3.349833in}{2.420165in}}%
\pgfpathlineto{\pgfqpoint{3.350686in}{2.427961in}}%
\pgfpathlineto{\pgfqpoint{3.351349in}{2.441315in}}%
\pgfpathlineto{\pgfqpoint{3.352581in}{2.435376in}}%
\pgfpathlineto{\pgfqpoint{3.353339in}{2.441216in}}%
\pgfpathlineto{\pgfqpoint{3.354192in}{2.419881in}}%
\pgfpathlineto{\pgfqpoint{3.354571in}{2.423928in}}%
\pgfpathlineto{\pgfqpoint{3.354855in}{2.417822in}}%
\pgfpathlineto{\pgfqpoint{3.355140in}{2.418390in}}%
\pgfpathlineto{\pgfqpoint{3.356656in}{2.392840in}}%
\pgfpathlineto{\pgfqpoint{3.357414in}{2.395104in}}%
\pgfpathlineto{\pgfqpoint{3.358172in}{2.370659in}}%
\pgfpathlineto{\pgfqpoint{3.358267in}{2.369856in}}%
\pgfpathlineto{\pgfqpoint{3.358551in}{2.375550in}}%
\pgfpathlineto{\pgfqpoint{3.358835in}{2.379978in}}%
\pgfpathlineto{\pgfqpoint{3.359309in}{2.367408in}}%
\pgfpathlineto{\pgfqpoint{3.359878in}{2.353005in}}%
\pgfpathlineto{\pgfqpoint{3.360257in}{2.370437in}}%
\pgfpathlineto{\pgfqpoint{3.361110in}{2.380037in}}%
\pgfpathlineto{\pgfqpoint{3.361394in}{2.373260in}}%
\pgfpathlineto{\pgfqpoint{3.362815in}{2.355902in}}%
\pgfpathlineto{\pgfqpoint{3.363005in}{2.360276in}}%
\pgfpathlineto{\pgfqpoint{3.364142in}{2.390043in}}%
\pgfpathlineto{\pgfqpoint{3.363479in}{2.351974in}}%
\pgfpathlineto{\pgfqpoint{3.364616in}{2.375097in}}%
\pgfpathlineto{\pgfqpoint{3.365658in}{2.354490in}}%
\pgfpathlineto{\pgfqpoint{3.365185in}{2.386052in}}%
\pgfpathlineto{\pgfqpoint{3.365943in}{2.359142in}}%
\pgfpathlineto{\pgfqpoint{3.366037in}{2.359425in}}%
\pgfpathlineto{\pgfqpoint{3.366132in}{2.357509in}}%
\pgfpathlineto{\pgfqpoint{3.366417in}{2.351059in}}%
\pgfpathlineto{\pgfqpoint{3.366796in}{2.364724in}}%
\pgfpathlineto{\pgfqpoint{3.368217in}{2.397038in}}%
\pgfpathlineto{\pgfqpoint{3.367364in}{2.352129in}}%
\pgfpathlineto{\pgfqpoint{3.368691in}{2.390472in}}%
\pgfpathlineto{\pgfqpoint{3.369070in}{2.373999in}}%
\pgfpathlineto{\pgfqpoint{3.369544in}{2.394390in}}%
\pgfpathlineto{\pgfqpoint{3.369733in}{2.393723in}}%
\pgfpathlineto{\pgfqpoint{3.369923in}{2.396584in}}%
\pgfpathlineto{\pgfqpoint{3.370112in}{2.400609in}}%
\pgfpathlineto{\pgfqpoint{3.370491in}{2.382679in}}%
\pgfpathlineto{\pgfqpoint{3.371818in}{2.361800in}}%
\pgfpathlineto{\pgfqpoint{3.372008in}{2.366056in}}%
\pgfpathlineto{\pgfqpoint{3.373050in}{2.410668in}}%
\pgfpathlineto{\pgfqpoint{3.373808in}{2.389397in}}%
\pgfpathlineto{\pgfqpoint{3.374945in}{2.371234in}}%
\pgfpathlineto{\pgfqpoint{3.375135in}{2.373498in}}%
\pgfpathlineto{\pgfqpoint{3.376556in}{2.481340in}}%
\pgfpathlineto{\pgfqpoint{3.377314in}{2.633388in}}%
\pgfpathlineto{\pgfqpoint{3.378167in}{2.631539in}}%
\pgfpathlineto{\pgfqpoint{3.378546in}{2.578351in}}%
\pgfpathlineto{\pgfqpoint{3.379399in}{2.337067in}}%
\pgfpathlineto{\pgfqpoint{3.380252in}{2.406890in}}%
\pgfpathlineto{\pgfqpoint{3.380821in}{2.417505in}}%
\pgfpathlineto{\pgfqpoint{3.382621in}{2.461842in}}%
\pgfpathlineto{\pgfqpoint{3.382811in}{2.456425in}}%
\pgfpathlineto{\pgfqpoint{3.383379in}{2.425068in}}%
\pgfpathlineto{\pgfqpoint{3.384232in}{2.437279in}}%
\pgfpathlineto{\pgfqpoint{3.385085in}{2.453907in}}%
\pgfpathlineto{\pgfqpoint{3.385464in}{2.477426in}}%
\pgfpathlineto{\pgfqpoint{3.386033in}{2.443345in}}%
\pgfpathlineto{\pgfqpoint{3.386412in}{2.426188in}}%
\pgfpathlineto{\pgfqpoint{3.387170in}{2.438083in}}%
\pgfpathlineto{\pgfqpoint{3.388497in}{2.460177in}}%
\pgfpathlineto{\pgfqpoint{3.388591in}{2.460864in}}%
\pgfpathlineto{\pgfqpoint{3.388781in}{2.454974in}}%
\pgfpathlineto{\pgfqpoint{3.389728in}{2.438703in}}%
\pgfpathlineto{\pgfqpoint{3.390013in}{2.445253in}}%
\pgfpathlineto{\pgfqpoint{3.391908in}{2.472268in}}%
\pgfpathlineto{\pgfqpoint{3.392003in}{2.470553in}}%
\pgfpathlineto{\pgfqpoint{3.392477in}{2.452222in}}%
\pgfpathlineto{\pgfqpoint{3.393235in}{2.458359in}}%
\pgfpathlineto{\pgfqpoint{3.393614in}{2.470019in}}%
\pgfpathlineto{\pgfqpoint{3.394372in}{2.465991in}}%
\pgfpathlineto{\pgfqpoint{3.395983in}{2.446094in}}%
\pgfpathlineto{\pgfqpoint{3.396078in}{2.445576in}}%
\pgfpathlineto{\pgfqpoint{3.396267in}{2.449094in}}%
\pgfpathlineto{\pgfqpoint{3.397120in}{2.458400in}}%
\pgfpathlineto{\pgfqpoint{3.397404in}{2.450100in}}%
\pgfpathlineto{\pgfqpoint{3.398257in}{2.442372in}}%
\pgfpathlineto{\pgfqpoint{3.397878in}{2.452808in}}%
\pgfpathlineto{\pgfqpoint{3.398352in}{2.446242in}}%
\pgfpathlineto{\pgfqpoint{3.398636in}{2.466048in}}%
\pgfpathlineto{\pgfqpoint{3.399110in}{2.443373in}}%
\pgfpathlineto{\pgfqpoint{3.399489in}{2.452369in}}%
\pgfpathlineto{\pgfqpoint{3.402237in}{2.407002in}}%
\pgfpathlineto{\pgfqpoint{3.403375in}{2.383464in}}%
\pgfpathlineto{\pgfqpoint{3.402901in}{2.407465in}}%
\pgfpathlineto{\pgfqpoint{3.403564in}{2.389506in}}%
\pgfpathlineto{\pgfqpoint{3.404512in}{2.400313in}}%
\pgfpathlineto{\pgfqpoint{3.404133in}{2.379543in}}%
\pgfpathlineto{\pgfqpoint{3.404701in}{2.393347in}}%
\pgfpathlineto{\pgfqpoint{3.405744in}{2.364933in}}%
\pgfpathlineto{\pgfqpoint{3.406028in}{2.373245in}}%
\pgfpathlineto{\pgfqpoint{3.406312in}{2.384458in}}%
\pgfpathlineto{\pgfqpoint{3.407070in}{2.375509in}}%
\pgfpathlineto{\pgfqpoint{3.407923in}{2.376908in}}%
\pgfpathlineto{\pgfqpoint{3.408492in}{2.358517in}}%
\pgfpathlineto{\pgfqpoint{3.408966in}{2.355702in}}%
\pgfpathlineto{\pgfqpoint{3.409724in}{2.369112in}}%
\pgfpathlineto{\pgfqpoint{3.410766in}{2.385521in}}%
\pgfpathlineto{\pgfqpoint{3.410292in}{2.368933in}}%
\pgfpathlineto{\pgfqpoint{3.410956in}{2.379622in}}%
\pgfpathlineto{\pgfqpoint{3.411050in}{2.377331in}}%
\pgfpathlineto{\pgfqpoint{3.411524in}{2.384994in}}%
\pgfpathlineto{\pgfqpoint{3.411903in}{2.380416in}}%
\pgfpathlineto{\pgfqpoint{3.412472in}{2.406924in}}%
\pgfpathlineto{\pgfqpoint{3.412851in}{2.416335in}}%
\pgfpathlineto{\pgfqpoint{3.413325in}{2.404328in}}%
\pgfpathlineto{\pgfqpoint{3.413609in}{2.409513in}}%
\pgfpathlineto{\pgfqpoint{3.414367in}{2.383076in}}%
\pgfpathlineto{\pgfqpoint{3.415125in}{2.356875in}}%
\pgfpathlineto{\pgfqpoint{3.415504in}{2.369541in}}%
\pgfpathlineto{\pgfqpoint{3.415789in}{2.376686in}}%
\pgfpathlineto{\pgfqpoint{3.416547in}{2.369131in}}%
\pgfpathlineto{\pgfqpoint{3.417873in}{2.342686in}}%
\pgfpathlineto{\pgfqpoint{3.418063in}{2.345582in}}%
\pgfpathlineto{\pgfqpoint{3.418537in}{2.358105in}}%
\pgfpathlineto{\pgfqpoint{3.419011in}{2.396161in}}%
\pgfpathlineto{\pgfqpoint{3.419674in}{2.376755in}}%
\pgfpathlineto{\pgfqpoint{3.420527in}{2.356480in}}%
\pgfpathlineto{\pgfqpoint{3.420906in}{2.366224in}}%
\pgfpathlineto{\pgfqpoint{3.421001in}{2.367827in}}%
\pgfpathlineto{\pgfqpoint{3.421569in}{2.359066in}}%
\pgfpathlineto{\pgfqpoint{3.421664in}{2.357777in}}%
\pgfpathlineto{\pgfqpoint{3.421948in}{2.367553in}}%
\pgfpathlineto{\pgfqpoint{3.422706in}{2.509667in}}%
\pgfpathlineto{\pgfqpoint{3.423465in}{2.647012in}}%
\pgfpathlineto{\pgfqpoint{3.424128in}{2.597851in}}%
\pgfpathlineto{\pgfqpoint{3.424507in}{2.571540in}}%
\pgfpathlineto{\pgfqpoint{3.425360in}{2.324947in}}%
\pgfpathlineto{\pgfqpoint{3.426402in}{2.384635in}}%
\pgfpathlineto{\pgfqpoint{3.428487in}{2.451982in}}%
\pgfpathlineto{\pgfqpoint{3.428866in}{2.420258in}}%
\pgfpathlineto{\pgfqpoint{3.429245in}{2.397633in}}%
\pgfpathlineto{\pgfqpoint{3.429908in}{2.418742in}}%
\pgfpathlineto{\pgfqpoint{3.431330in}{2.464169in}}%
\pgfpathlineto{\pgfqpoint{3.431709in}{2.450367in}}%
\pgfpathlineto{\pgfqpoint{3.432562in}{2.411824in}}%
\pgfpathlineto{\pgfqpoint{3.433225in}{2.430478in}}%
\pgfpathlineto{\pgfqpoint{3.433320in}{2.429141in}}%
\pgfpathlineto{\pgfqpoint{3.433509in}{2.438059in}}%
\pgfpathlineto{\pgfqpoint{3.434362in}{2.464960in}}%
\pgfpathlineto{\pgfqpoint{3.434741in}{2.451532in}}%
\pgfpathlineto{\pgfqpoint{3.435310in}{2.426807in}}%
\pgfpathlineto{\pgfqpoint{3.436637in}{2.415034in}}%
\pgfpathlineto{\pgfqpoint{3.437490in}{2.403601in}}%
\pgfpathlineto{\pgfqpoint{3.437111in}{2.417190in}}%
\pgfpathlineto{\pgfqpoint{3.437679in}{2.409460in}}%
\pgfpathlineto{\pgfqpoint{3.439859in}{2.514878in}}%
\pgfpathlineto{\pgfqpoint{3.440048in}{2.511795in}}%
\pgfpathlineto{\pgfqpoint{3.441659in}{2.476570in}}%
\pgfpathlineto{\pgfqpoint{3.441754in}{2.478446in}}%
\pgfpathlineto{\pgfqpoint{3.442702in}{2.507662in}}%
\pgfpathlineto{\pgfqpoint{3.443270in}{2.497692in}}%
\pgfpathlineto{\pgfqpoint{3.443460in}{2.500395in}}%
\pgfpathlineto{\pgfqpoint{3.443744in}{2.510543in}}%
\pgfpathlineto{\pgfqpoint{3.444218in}{2.497071in}}%
\pgfpathlineto{\pgfqpoint{3.444502in}{2.500203in}}%
\pgfpathlineto{\pgfqpoint{3.445545in}{2.491661in}}%
\pgfpathlineto{\pgfqpoint{3.445071in}{2.503312in}}%
\pgfpathlineto{\pgfqpoint{3.445734in}{2.498805in}}%
\pgfpathlineto{\pgfqpoint{3.445924in}{2.502331in}}%
\pgfpathlineto{\pgfqpoint{3.446397in}{2.485813in}}%
\pgfpathlineto{\pgfqpoint{3.446492in}{2.485671in}}%
\pgfpathlineto{\pgfqpoint{3.446682in}{2.486930in}}%
\pgfpathlineto{\pgfqpoint{3.447061in}{2.486299in}}%
\pgfpathlineto{\pgfqpoint{3.447250in}{2.487498in}}%
\pgfpathlineto{\pgfqpoint{3.447535in}{2.482523in}}%
\pgfpathlineto{\pgfqpoint{3.448577in}{2.459918in}}%
\pgfpathlineto{\pgfqpoint{3.448861in}{2.469277in}}%
\pgfpathlineto{\pgfqpoint{3.448956in}{2.469891in}}%
\pgfpathlineto{\pgfqpoint{3.449051in}{2.466826in}}%
\pgfpathlineto{\pgfqpoint{3.450378in}{2.454443in}}%
\pgfpathlineto{\pgfqpoint{3.450662in}{2.455433in}}%
\pgfpathlineto{\pgfqpoint{3.450946in}{2.448873in}}%
\pgfpathlineto{\pgfqpoint{3.451136in}{2.441204in}}%
\pgfpathlineto{\pgfqpoint{3.451515in}{2.452801in}}%
\pgfpathlineto{\pgfqpoint{3.451989in}{2.448510in}}%
\pgfpathlineto{\pgfqpoint{3.452273in}{2.454731in}}%
\pgfpathlineto{\pgfqpoint{3.453126in}{2.449926in}}%
\pgfpathlineto{\pgfqpoint{3.454547in}{2.427972in}}%
\pgfpathlineto{\pgfqpoint{3.453694in}{2.450198in}}%
\pgfpathlineto{\pgfqpoint{3.454831in}{2.432376in}}%
\pgfpathlineto{\pgfqpoint{3.456632in}{2.456350in}}%
\pgfpathlineto{\pgfqpoint{3.457106in}{2.462075in}}%
\pgfpathlineto{\pgfqpoint{3.457390in}{2.470466in}}%
\pgfpathlineto{\pgfqpoint{3.458243in}{2.463734in}}%
\pgfpathlineto{\pgfqpoint{3.458338in}{2.463160in}}%
\pgfpathlineto{\pgfqpoint{3.458432in}{2.465899in}}%
\pgfpathlineto{\pgfqpoint{3.458906in}{2.492817in}}%
\pgfpathlineto{\pgfqpoint{3.459570in}{2.473876in}}%
\pgfpathlineto{\pgfqpoint{3.462981in}{2.385031in}}%
\pgfpathlineto{\pgfqpoint{3.463171in}{2.385930in}}%
\pgfpathlineto{\pgfqpoint{3.463550in}{2.369659in}}%
\pgfpathlineto{\pgfqpoint{3.463834in}{2.358593in}}%
\pgfpathlineto{\pgfqpoint{3.464497in}{2.373067in}}%
\pgfpathlineto{\pgfqpoint{3.464971in}{2.409613in}}%
\pgfpathlineto{\pgfqpoint{3.465635in}{2.380048in}}%
\pgfpathlineto{\pgfqpoint{3.465919in}{2.361122in}}%
\pgfpathlineto{\pgfqpoint{3.466866in}{2.367153in}}%
\pgfpathlineto{\pgfqpoint{3.467056in}{2.364628in}}%
\pgfpathlineto{\pgfqpoint{3.467435in}{2.374157in}}%
\pgfpathlineto{\pgfqpoint{3.467719in}{2.370235in}}%
\pgfpathlineto{\pgfqpoint{3.467909in}{2.375445in}}%
\pgfpathlineto{\pgfqpoint{3.468667in}{2.514932in}}%
\pgfpathlineto{\pgfqpoint{3.469520in}{2.667881in}}%
\pgfpathlineto{\pgfqpoint{3.470088in}{2.622318in}}%
\pgfpathlineto{\pgfqpoint{3.470562in}{2.574490in}}%
\pgfpathlineto{\pgfqpoint{3.471320in}{2.339856in}}%
\pgfpathlineto{\pgfqpoint{3.472173in}{2.418881in}}%
\pgfpathlineto{\pgfqpoint{3.472268in}{2.417887in}}%
\pgfpathlineto{\pgfqpoint{3.472458in}{2.422970in}}%
\pgfpathlineto{\pgfqpoint{3.474163in}{2.492138in}}%
\pgfpathlineto{\pgfqpoint{3.474542in}{2.472813in}}%
\pgfpathlineto{\pgfqpoint{3.475300in}{2.444106in}}%
\pgfpathlineto{\pgfqpoint{3.475774in}{2.465382in}}%
\pgfpathlineto{\pgfqpoint{3.475869in}{2.465516in}}%
\pgfpathlineto{\pgfqpoint{3.476248in}{2.456271in}}%
\pgfpathlineto{\pgfqpoint{3.476532in}{2.472890in}}%
\pgfpathlineto{\pgfqpoint{3.477575in}{2.507496in}}%
\pgfpathlineto{\pgfqpoint{3.477859in}{2.502122in}}%
\pgfpathlineto{\pgfqpoint{3.478333in}{2.481051in}}%
\pgfpathlineto{\pgfqpoint{3.478996in}{2.499064in}}%
\pgfpathlineto{\pgfqpoint{3.480607in}{2.543407in}}%
\pgfpathlineto{\pgfqpoint{3.480986in}{2.534814in}}%
\pgfpathlineto{\pgfqpoint{3.481365in}{2.525314in}}%
\pgfpathlineto{\pgfqpoint{3.482029in}{2.533511in}}%
\pgfpathlineto{\pgfqpoint{3.484019in}{2.579689in}}%
\pgfpathlineto{\pgfqpoint{3.484208in}{2.572196in}}%
\pgfpathlineto{\pgfqpoint{3.484493in}{2.558304in}}%
\pgfpathlineto{\pgfqpoint{3.485156in}{2.580463in}}%
\pgfpathlineto{\pgfqpoint{3.485535in}{2.594604in}}%
\pgfpathlineto{\pgfqpoint{3.486388in}{2.591364in}}%
\pgfpathlineto{\pgfqpoint{3.487525in}{2.571262in}}%
\pgfpathlineto{\pgfqpoint{3.487715in}{2.582133in}}%
\pgfpathlineto{\pgfqpoint{3.488757in}{2.599789in}}%
\pgfpathlineto{\pgfqpoint{3.488283in}{2.576679in}}%
\pgfpathlineto{\pgfqpoint{3.488947in}{2.595537in}}%
\pgfpathlineto{\pgfqpoint{3.489420in}{2.591315in}}%
\pgfpathlineto{\pgfqpoint{3.489894in}{2.596777in}}%
\pgfpathlineto{\pgfqpoint{3.490084in}{2.597396in}}%
\pgfpathlineto{\pgfqpoint{3.490368in}{2.593338in}}%
\pgfpathlineto{\pgfqpoint{3.491316in}{2.574074in}}%
\pgfpathlineto{\pgfqpoint{3.491600in}{2.584789in}}%
\pgfpathlineto{\pgfqpoint{3.491695in}{2.586831in}}%
\pgfpathlineto{\pgfqpoint{3.492074in}{2.573413in}}%
\pgfpathlineto{\pgfqpoint{3.492453in}{2.584765in}}%
\pgfpathlineto{\pgfqpoint{3.493779in}{2.551209in}}%
\pgfpathlineto{\pgfqpoint{3.493874in}{2.552064in}}%
\pgfpathlineto{\pgfqpoint{3.494253in}{2.561915in}}%
\pgfpathlineto{\pgfqpoint{3.494632in}{2.549694in}}%
\pgfpathlineto{\pgfqpoint{3.496054in}{2.529779in}}%
\pgfpathlineto{\pgfqpoint{3.496528in}{2.539925in}}%
\pgfpathlineto{\pgfqpoint{3.496812in}{2.529815in}}%
\pgfpathlineto{\pgfqpoint{3.497570in}{2.538906in}}%
\pgfpathlineto{\pgfqpoint{3.498044in}{2.517282in}}%
\pgfpathlineto{\pgfqpoint{3.499086in}{2.534013in}}%
\pgfpathlineto{\pgfqpoint{3.499750in}{2.524511in}}%
\pgfpathlineto{\pgfqpoint{3.500318in}{2.526642in}}%
\pgfpathlineto{\pgfqpoint{3.501361in}{2.511409in}}%
\pgfpathlineto{\pgfqpoint{3.503066in}{2.549319in}}%
\pgfpathlineto{\pgfqpoint{3.503256in}{2.538098in}}%
\pgfpathlineto{\pgfqpoint{3.503445in}{2.531075in}}%
\pgfpathlineto{\pgfqpoint{3.503824in}{2.550156in}}%
\pgfpathlineto{\pgfqpoint{3.504204in}{2.546779in}}%
\pgfpathlineto{\pgfqpoint{3.504393in}{2.547066in}}%
\pgfpathlineto{\pgfqpoint{3.504488in}{2.545363in}}%
\pgfpathlineto{\pgfqpoint{3.505056in}{2.551188in}}%
\pgfpathlineto{\pgfqpoint{3.506194in}{2.482445in}}%
\pgfpathlineto{\pgfqpoint{3.507426in}{2.406128in}}%
\pgfpathlineto{\pgfqpoint{3.507994in}{2.442274in}}%
\pgfpathlineto{\pgfqpoint{3.508752in}{2.486027in}}%
\pgfpathlineto{\pgfqpoint{3.509510in}{2.476218in}}%
\pgfpathlineto{\pgfqpoint{3.509700in}{2.475053in}}%
\pgfpathlineto{\pgfqpoint{3.510458in}{2.456404in}}%
\pgfpathlineto{\pgfqpoint{3.510742in}{2.471291in}}%
\pgfpathlineto{\pgfqpoint{3.511406in}{2.499488in}}%
\pgfpathlineto{\pgfqpoint{3.511785in}{2.480268in}}%
\pgfpathlineto{\pgfqpoint{3.512922in}{2.436524in}}%
\pgfpathlineto{\pgfqpoint{3.513396in}{2.441118in}}%
\pgfpathlineto{\pgfqpoint{3.514154in}{2.434280in}}%
\pgfpathlineto{\pgfqpoint{3.514343in}{2.440905in}}%
\pgfpathlineto{\pgfqpoint{3.515954in}{2.716986in}}%
\pgfpathlineto{\pgfqpoint{3.516807in}{2.636176in}}%
\pgfpathlineto{\pgfqpoint{3.517660in}{2.383036in}}%
\pgfpathlineto{\pgfqpoint{3.518608in}{2.448138in}}%
\pgfpathlineto{\pgfqpoint{3.518702in}{2.446256in}}%
\pgfpathlineto{\pgfqpoint{3.518987in}{2.459938in}}%
\pgfpathlineto{\pgfqpoint{3.520503in}{2.506360in}}%
\pgfpathlineto{\pgfqpoint{3.520692in}{2.501857in}}%
\pgfpathlineto{\pgfqpoint{3.521830in}{2.467192in}}%
\pgfpathlineto{\pgfqpoint{3.522209in}{2.484946in}}%
\pgfpathlineto{\pgfqpoint{3.523914in}{2.508186in}}%
\pgfpathlineto{\pgfqpoint{3.524578in}{2.482946in}}%
\pgfpathlineto{\pgfqpoint{3.524957in}{2.467019in}}%
\pgfpathlineto{\pgfqpoint{3.525525in}{2.490622in}}%
\pgfpathlineto{\pgfqpoint{3.526852in}{2.520011in}}%
\pgfpathlineto{\pgfqpoint{3.527136in}{2.505671in}}%
\pgfpathlineto{\pgfqpoint{3.528179in}{2.486649in}}%
\pgfpathlineto{\pgfqpoint{3.528368in}{2.492872in}}%
\pgfpathlineto{\pgfqpoint{3.528653in}{2.506817in}}%
\pgfpathlineto{\pgfqpoint{3.529506in}{2.500918in}}%
\pgfpathlineto{\pgfqpoint{3.531022in}{2.476040in}}%
\pgfpathlineto{\pgfqpoint{3.529979in}{2.501510in}}%
\pgfpathlineto{\pgfqpoint{3.531211in}{2.486149in}}%
\pgfpathlineto{\pgfqpoint{3.532064in}{2.505816in}}%
\pgfpathlineto{\pgfqpoint{3.532538in}{2.499456in}}%
\pgfpathlineto{\pgfqpoint{3.532728in}{2.505539in}}%
\pgfpathlineto{\pgfqpoint{3.533391in}{2.491520in}}%
\pgfpathlineto{\pgfqpoint{3.534339in}{2.479647in}}%
\pgfpathlineto{\pgfqpoint{3.534054in}{2.494228in}}%
\pgfpathlineto{\pgfqpoint{3.534528in}{2.485046in}}%
\pgfpathlineto{\pgfqpoint{3.534812in}{2.498551in}}%
\pgfpathlineto{\pgfqpoint{3.535665in}{2.486976in}}%
\pgfpathlineto{\pgfqpoint{3.536802in}{2.498551in}}%
\pgfpathlineto{\pgfqpoint{3.536423in}{2.486007in}}%
\pgfpathlineto{\pgfqpoint{3.537371in}{2.497312in}}%
\pgfpathlineto{\pgfqpoint{3.539835in}{2.449878in}}%
\pgfpathlineto{\pgfqpoint{3.540024in}{2.451303in}}%
\pgfpathlineto{\pgfqpoint{3.540214in}{2.448693in}}%
\pgfpathlineto{\pgfqpoint{3.541541in}{2.413588in}}%
\pgfpathlineto{\pgfqpoint{3.541825in}{2.416307in}}%
\pgfpathlineto{\pgfqpoint{3.542109in}{2.423298in}}%
\pgfpathlineto{\pgfqpoint{3.542393in}{2.407876in}}%
\pgfpathlineto{\pgfqpoint{3.543341in}{2.390924in}}%
\pgfpathlineto{\pgfqpoint{3.543625in}{2.393875in}}%
\pgfpathlineto{\pgfqpoint{3.544194in}{2.369502in}}%
\pgfpathlineto{\pgfqpoint{3.545236in}{2.384377in}}%
\pgfpathlineto{\pgfqpoint{3.546374in}{2.401664in}}%
\pgfpathlineto{\pgfqpoint{3.545994in}{2.381269in}}%
\pgfpathlineto{\pgfqpoint{3.546563in}{2.396259in}}%
\pgfpathlineto{\pgfqpoint{3.546942in}{2.376228in}}%
\pgfpathlineto{\pgfqpoint{3.547795in}{2.377809in}}%
\pgfpathlineto{\pgfqpoint{3.549690in}{2.410552in}}%
\pgfpathlineto{\pgfqpoint{3.550069in}{2.409762in}}%
\pgfpathlineto{\pgfqpoint{3.550543in}{2.401680in}}%
\pgfpathlineto{\pgfqpoint{3.550922in}{2.410683in}}%
\pgfpathlineto{\pgfqpoint{3.551680in}{2.433708in}}%
\pgfpathlineto{\pgfqpoint{3.552344in}{2.419825in}}%
\pgfpathlineto{\pgfqpoint{3.556134in}{2.340318in}}%
\pgfpathlineto{\pgfqpoint{3.556703in}{2.341925in}}%
\pgfpathlineto{\pgfqpoint{3.556798in}{2.341066in}}%
\pgfpathlineto{\pgfqpoint{3.556987in}{2.344166in}}%
\pgfpathlineto{\pgfqpoint{3.557650in}{2.384368in}}%
\pgfpathlineto{\pgfqpoint{3.558598in}{2.369588in}}%
\pgfpathlineto{\pgfqpoint{3.559735in}{2.357339in}}%
\pgfpathlineto{\pgfqpoint{3.559830in}{2.355124in}}%
\pgfpathlineto{\pgfqpoint{3.560304in}{2.368072in}}%
\pgfpathlineto{\pgfqpoint{3.560399in}{2.367867in}}%
\pgfpathlineto{\pgfqpoint{3.560493in}{2.368223in}}%
\pgfpathlineto{\pgfqpoint{3.560967in}{2.406893in}}%
\pgfpathlineto{\pgfqpoint{3.562199in}{2.647162in}}%
\pgfpathlineto{\pgfqpoint{3.563147in}{2.591767in}}%
\pgfpathlineto{\pgfqpoint{3.564000in}{2.333331in}}%
\pgfpathlineto{\pgfqpoint{3.565421in}{2.398438in}}%
\pgfpathlineto{\pgfqpoint{3.565516in}{2.397433in}}%
\pgfpathlineto{\pgfqpoint{3.565611in}{2.400485in}}%
\pgfpathlineto{\pgfqpoint{3.567222in}{2.471407in}}%
\pgfpathlineto{\pgfqpoint{3.567411in}{2.463562in}}%
\pgfpathlineto{\pgfqpoint{3.568075in}{2.429940in}}%
\pgfpathlineto{\pgfqpoint{3.568738in}{2.440629in}}%
\pgfpathlineto{\pgfqpoint{3.568833in}{2.440466in}}%
\pgfpathlineto{\pgfqpoint{3.570254in}{2.476451in}}%
\pgfpathlineto{\pgfqpoint{3.570633in}{2.458192in}}%
\pgfpathlineto{\pgfqpoint{3.571391in}{2.432254in}}%
\pgfpathlineto{\pgfqpoint{3.571770in}{2.450435in}}%
\pgfpathlineto{\pgfqpoint{3.573192in}{2.476010in}}%
\pgfpathlineto{\pgfqpoint{3.573381in}{2.473685in}}%
\pgfpathlineto{\pgfqpoint{3.575182in}{2.412752in}}%
\pgfpathlineto{\pgfqpoint{3.575561in}{2.414232in}}%
\pgfpathlineto{\pgfqpoint{3.575845in}{2.410818in}}%
\pgfpathlineto{\pgfqpoint{3.576129in}{2.421631in}}%
\pgfpathlineto{\pgfqpoint{3.578499in}{2.509160in}}%
\pgfpathlineto{\pgfqpoint{3.578688in}{2.507732in}}%
\pgfpathlineto{\pgfqpoint{3.580299in}{2.480007in}}%
\pgfpathlineto{\pgfqpoint{3.580678in}{2.484518in}}%
\pgfpathlineto{\pgfqpoint{3.581531in}{2.503513in}}%
\pgfpathlineto{\pgfqpoint{3.581815in}{2.490768in}}%
\pgfpathlineto{\pgfqpoint{3.581910in}{2.486893in}}%
\pgfpathlineto{\pgfqpoint{3.582384in}{2.498654in}}%
\pgfpathlineto{\pgfqpoint{3.582763in}{2.496559in}}%
\pgfpathlineto{\pgfqpoint{3.582858in}{2.497515in}}%
\pgfpathlineto{\pgfqpoint{3.583047in}{2.492923in}}%
\pgfpathlineto{\pgfqpoint{3.584279in}{2.471074in}}%
\pgfpathlineto{\pgfqpoint{3.584469in}{2.474641in}}%
\pgfpathlineto{\pgfqpoint{3.585322in}{2.480064in}}%
\pgfpathlineto{\pgfqpoint{3.585037in}{2.464072in}}%
\pgfpathlineto{\pgfqpoint{3.585416in}{2.478181in}}%
\pgfpathlineto{\pgfqpoint{3.587027in}{2.432492in}}%
\pgfpathlineto{\pgfqpoint{3.587406in}{2.436468in}}%
\pgfpathlineto{\pgfqpoint{3.589112in}{2.401351in}}%
\pgfpathlineto{\pgfqpoint{3.590249in}{2.392799in}}%
\pgfpathlineto{\pgfqpoint{3.589491in}{2.404209in}}%
\pgfpathlineto{\pgfqpoint{3.590439in}{2.393877in}}%
\pgfpathlineto{\pgfqpoint{3.592050in}{2.412587in}}%
\pgfpathlineto{\pgfqpoint{3.592239in}{2.410681in}}%
\pgfpathlineto{\pgfqpoint{3.592808in}{2.420247in}}%
\pgfpathlineto{\pgfqpoint{3.593756in}{2.397903in}}%
\pgfpathlineto{\pgfqpoint{3.594988in}{2.428624in}}%
\pgfpathlineto{\pgfqpoint{3.594324in}{2.392223in}}%
\pgfpathlineto{\pgfqpoint{3.595461in}{2.410618in}}%
\pgfpathlineto{\pgfqpoint{3.595556in}{2.409327in}}%
\pgfpathlineto{\pgfqpoint{3.595746in}{2.420190in}}%
\pgfpathlineto{\pgfqpoint{3.596030in}{2.436899in}}%
\pgfpathlineto{\pgfqpoint{3.596693in}{2.410966in}}%
\pgfpathlineto{\pgfqpoint{3.596788in}{2.410130in}}%
\pgfpathlineto{\pgfqpoint{3.596978in}{2.414492in}}%
\pgfpathlineto{\pgfqpoint{3.598115in}{2.445742in}}%
\pgfpathlineto{\pgfqpoint{3.598399in}{2.434510in}}%
\pgfpathlineto{\pgfqpoint{3.600105in}{2.356710in}}%
\pgfpathlineto{\pgfqpoint{3.600863in}{2.380760in}}%
\pgfpathlineto{\pgfqpoint{3.600958in}{2.380884in}}%
\pgfpathlineto{\pgfqpoint{3.603137in}{2.347702in}}%
\pgfpathlineto{\pgfqpoint{3.603327in}{2.351644in}}%
\pgfpathlineto{\pgfqpoint{3.604274in}{2.392430in}}%
\pgfpathlineto{\pgfqpoint{3.604748in}{2.371989in}}%
\pgfpathlineto{\pgfqpoint{3.605791in}{2.356694in}}%
\pgfpathlineto{\pgfqpoint{3.605980in}{2.364092in}}%
\pgfpathlineto{\pgfqpoint{3.607781in}{2.528605in}}%
\pgfpathlineto{\pgfqpoint{3.608634in}{2.644780in}}%
\pgfpathlineto{\pgfqpoint{3.609107in}{2.617024in}}%
\pgfpathlineto{\pgfqpoint{3.609771in}{2.506770in}}%
\pgfpathlineto{\pgfqpoint{3.610339in}{2.317495in}}%
\pgfpathlineto{\pgfqpoint{3.611097in}{2.392384in}}%
\pgfpathlineto{\pgfqpoint{3.611476in}{2.386937in}}%
\pgfpathlineto{\pgfqpoint{3.611950in}{2.392822in}}%
\pgfpathlineto{\pgfqpoint{3.613561in}{2.442429in}}%
\pgfpathlineto{\pgfqpoint{3.614035in}{2.414216in}}%
\pgfpathlineto{\pgfqpoint{3.614698in}{2.398817in}}%
\pgfpathlineto{\pgfqpoint{3.614983in}{2.413102in}}%
\pgfpathlineto{\pgfqpoint{3.616499in}{2.452320in}}%
\pgfpathlineto{\pgfqpoint{3.616783in}{2.442556in}}%
\pgfpathlineto{\pgfqpoint{3.617636in}{2.417797in}}%
\pgfpathlineto{\pgfqpoint{3.617920in}{2.431157in}}%
\pgfpathlineto{\pgfqpoint{3.619437in}{2.453196in}}%
\pgfpathlineto{\pgfqpoint{3.619721in}{2.458860in}}%
\pgfpathlineto{\pgfqpoint{3.620100in}{2.446948in}}%
\pgfpathlineto{\pgfqpoint{3.620574in}{2.438599in}}%
\pgfpathlineto{\pgfqpoint{3.620953in}{2.450813in}}%
\pgfpathlineto{\pgfqpoint{3.622469in}{2.474295in}}%
\pgfpathlineto{\pgfqpoint{3.622564in}{2.473489in}}%
\pgfpathlineto{\pgfqpoint{3.623985in}{2.455344in}}%
\pgfpathlineto{\pgfqpoint{3.624080in}{2.456180in}}%
\pgfpathlineto{\pgfqpoint{3.624838in}{2.475698in}}%
\pgfpathlineto{\pgfqpoint{3.625312in}{2.457564in}}%
\pgfpathlineto{\pgfqpoint{3.625407in}{2.456825in}}%
\pgfpathlineto{\pgfqpoint{3.625596in}{2.461782in}}%
\pgfpathlineto{\pgfqpoint{3.625786in}{2.466662in}}%
\pgfpathlineto{\pgfqpoint{3.626165in}{2.447109in}}%
\pgfpathlineto{\pgfqpoint{3.627018in}{2.444576in}}%
\pgfpathlineto{\pgfqpoint{3.626639in}{2.456383in}}%
\pgfpathlineto{\pgfqpoint{3.627113in}{2.447389in}}%
\pgfpathlineto{\pgfqpoint{3.628629in}{2.467682in}}%
\pgfpathlineto{\pgfqpoint{3.628818in}{2.469184in}}%
\pgfpathlineto{\pgfqpoint{3.629197in}{2.461458in}}%
\pgfpathlineto{\pgfqpoint{3.629292in}{2.460673in}}%
\pgfpathlineto{\pgfqpoint{3.629576in}{2.466835in}}%
\pgfpathlineto{\pgfqpoint{3.629671in}{2.468434in}}%
\pgfpathlineto{\pgfqpoint{3.630050in}{2.456556in}}%
\pgfpathlineto{\pgfqpoint{3.631566in}{2.434770in}}%
\pgfpathlineto{\pgfqpoint{3.631756in}{2.438385in}}%
\pgfpathlineto{\pgfqpoint{3.631851in}{2.440601in}}%
\pgfpathlineto{\pgfqpoint{3.632230in}{2.425441in}}%
\pgfpathlineto{\pgfqpoint{3.632325in}{2.425314in}}%
\pgfpathlineto{\pgfqpoint{3.632609in}{2.435902in}}%
\pgfpathlineto{\pgfqpoint{3.632988in}{2.417991in}}%
\pgfpathlineto{\pgfqpoint{3.633367in}{2.427125in}}%
\pgfpathlineto{\pgfqpoint{3.633746in}{2.400389in}}%
\pgfpathlineto{\pgfqpoint{3.634694in}{2.412510in}}%
\pgfpathlineto{\pgfqpoint{3.634883in}{2.415892in}}%
\pgfpathlineto{\pgfqpoint{3.635357in}{2.401539in}}%
\pgfpathlineto{\pgfqpoint{3.635831in}{2.396757in}}%
\pgfpathlineto{\pgfqpoint{3.635926in}{2.395500in}}%
\pgfpathlineto{\pgfqpoint{3.636305in}{2.404716in}}%
\pgfpathlineto{\pgfqpoint{3.636399in}{2.405988in}}%
\pgfpathlineto{\pgfqpoint{3.636684in}{2.395393in}}%
\pgfpathlineto{\pgfqpoint{3.636779in}{2.394352in}}%
\pgfpathlineto{\pgfqpoint{3.636968in}{2.401890in}}%
\pgfpathlineto{\pgfqpoint{3.637821in}{2.412316in}}%
\pgfpathlineto{\pgfqpoint{3.638010in}{2.407043in}}%
\pgfpathlineto{\pgfqpoint{3.638958in}{2.400516in}}%
\pgfpathlineto{\pgfqpoint{3.638579in}{2.409860in}}%
\pgfpathlineto{\pgfqpoint{3.639148in}{2.404325in}}%
\pgfpathlineto{\pgfqpoint{3.639432in}{2.409000in}}%
\pgfpathlineto{\pgfqpoint{3.640000in}{2.398790in}}%
\pgfpathlineto{\pgfqpoint{3.640095in}{2.398435in}}%
\pgfpathlineto{\pgfqpoint{3.640285in}{2.400362in}}%
\pgfpathlineto{\pgfqpoint{3.642370in}{2.447529in}}%
\pgfpathlineto{\pgfqpoint{3.642464in}{2.446287in}}%
\pgfpathlineto{\pgfqpoint{3.642843in}{2.424788in}}%
\pgfpathlineto{\pgfqpoint{3.643602in}{2.442177in}}%
\pgfpathlineto{\pgfqpoint{3.644360in}{2.458046in}}%
\pgfpathlineto{\pgfqpoint{3.644739in}{2.446823in}}%
\pgfpathlineto{\pgfqpoint{3.646539in}{2.361096in}}%
\pgfpathlineto{\pgfqpoint{3.646918in}{2.366327in}}%
\pgfpathlineto{\pgfqpoint{3.647108in}{2.366960in}}%
\pgfpathlineto{\pgfqpoint{3.647297in}{2.364434in}}%
\pgfpathlineto{\pgfqpoint{3.649382in}{2.271638in}}%
\pgfpathlineto{\pgfqpoint{3.649761in}{2.300564in}}%
\pgfpathlineto{\pgfqpoint{3.650614in}{2.367906in}}%
\pgfpathlineto{\pgfqpoint{3.651277in}{2.365733in}}%
\pgfpathlineto{\pgfqpoint{3.651372in}{2.365473in}}%
\pgfpathlineto{\pgfqpoint{3.652320in}{2.345530in}}%
\pgfpathlineto{\pgfqpoint{3.652604in}{2.359848in}}%
\pgfpathlineto{\pgfqpoint{3.652699in}{2.360291in}}%
\pgfpathlineto{\pgfqpoint{3.652794in}{2.355866in}}%
\pgfpathlineto{\pgfqpoint{3.653078in}{2.337685in}}%
\pgfpathlineto{\pgfqpoint{3.653552in}{2.370711in}}%
\pgfpathlineto{\pgfqpoint{3.654973in}{2.629746in}}%
\pgfpathlineto{\pgfqpoint{3.655826in}{2.563990in}}%
\pgfpathlineto{\pgfqpoint{3.656679in}{2.297218in}}%
\pgfpathlineto{\pgfqpoint{3.658006in}{2.373512in}}%
\pgfpathlineto{\pgfqpoint{3.659711in}{2.420776in}}%
\pgfpathlineto{\pgfqpoint{3.659806in}{2.418107in}}%
\pgfpathlineto{\pgfqpoint{3.660943in}{2.371950in}}%
\pgfpathlineto{\pgfqpoint{3.661322in}{2.393746in}}%
\pgfpathlineto{\pgfqpoint{3.662839in}{2.423867in}}%
\pgfpathlineto{\pgfqpoint{3.662933in}{2.422124in}}%
\pgfpathlineto{\pgfqpoint{3.663502in}{2.388862in}}%
\pgfpathlineto{\pgfqpoint{3.664260in}{2.407371in}}%
\pgfpathlineto{\pgfqpoint{3.666155in}{2.445616in}}%
\pgfpathlineto{\pgfqpoint{3.666345in}{2.443078in}}%
\pgfpathlineto{\pgfqpoint{3.666914in}{2.420732in}}%
\pgfpathlineto{\pgfqpoint{3.667387in}{2.442386in}}%
\pgfpathlineto{\pgfqpoint{3.668714in}{2.456864in}}%
\pgfpathlineto{\pgfqpoint{3.669377in}{2.450166in}}%
\pgfpathlineto{\pgfqpoint{3.670135in}{2.434947in}}%
\pgfpathlineto{\pgfqpoint{3.670894in}{2.443033in}}%
\pgfpathlineto{\pgfqpoint{3.671178in}{2.447119in}}%
\pgfpathlineto{\pgfqpoint{3.671367in}{2.451158in}}%
\pgfpathlineto{\pgfqpoint{3.671841in}{2.440536in}}%
\pgfpathlineto{\pgfqpoint{3.672031in}{2.441782in}}%
\pgfpathlineto{\pgfqpoint{3.672220in}{2.442904in}}%
\pgfpathlineto{\pgfqpoint{3.672410in}{2.438367in}}%
\pgfpathlineto{\pgfqpoint{3.672884in}{2.422271in}}%
\pgfpathlineto{\pgfqpoint{3.673547in}{2.435431in}}%
\pgfpathlineto{\pgfqpoint{3.673642in}{2.434308in}}%
\pgfpathlineto{\pgfqpoint{3.674021in}{2.440628in}}%
\pgfpathlineto{\pgfqpoint{3.674495in}{2.447340in}}%
\pgfpathlineto{\pgfqpoint{3.674779in}{2.437320in}}%
\pgfpathlineto{\pgfqpoint{3.674874in}{2.434883in}}%
\pgfpathlineto{\pgfqpoint{3.675253in}{2.451381in}}%
\pgfpathlineto{\pgfqpoint{3.675348in}{2.452730in}}%
\pgfpathlineto{\pgfqpoint{3.675916in}{2.448079in}}%
\pgfpathlineto{\pgfqpoint{3.677432in}{2.434714in}}%
\pgfpathlineto{\pgfqpoint{3.677622in}{2.437017in}}%
\pgfpathlineto{\pgfqpoint{3.677906in}{2.428520in}}%
\pgfpathlineto{\pgfqpoint{3.679422in}{2.400472in}}%
\pgfpathlineto{\pgfqpoint{3.679707in}{2.406288in}}%
\pgfpathlineto{\pgfqpoint{3.679991in}{2.392272in}}%
\pgfpathlineto{\pgfqpoint{3.681318in}{2.380358in}}%
\pgfpathlineto{\pgfqpoint{3.683308in}{2.367505in}}%
\pgfpathlineto{\pgfqpoint{3.682076in}{2.380990in}}%
\pgfpathlineto{\pgfqpoint{3.683402in}{2.367792in}}%
\pgfpathlineto{\pgfqpoint{3.683592in}{2.368061in}}%
\pgfpathlineto{\pgfqpoint{3.683876in}{2.365784in}}%
\pgfpathlineto{\pgfqpoint{3.683971in}{2.365745in}}%
\pgfpathlineto{\pgfqpoint{3.684445in}{2.384124in}}%
\pgfpathlineto{\pgfqpoint{3.684919in}{2.364193in}}%
\pgfpathlineto{\pgfqpoint{3.685013in}{2.363569in}}%
\pgfpathlineto{\pgfqpoint{3.685203in}{2.366789in}}%
\pgfpathlineto{\pgfqpoint{3.686530in}{2.380951in}}%
\pgfpathlineto{\pgfqpoint{3.686151in}{2.363436in}}%
\pgfpathlineto{\pgfqpoint{3.686719in}{2.379572in}}%
\pgfpathlineto{\pgfqpoint{3.686814in}{2.378374in}}%
\pgfpathlineto{\pgfqpoint{3.687098in}{2.385128in}}%
\pgfpathlineto{\pgfqpoint{3.688425in}{2.407550in}}%
\pgfpathlineto{\pgfqpoint{3.688614in}{2.402935in}}%
\pgfpathlineto{\pgfqpoint{3.688804in}{2.394795in}}%
\pgfpathlineto{\pgfqpoint{3.689278in}{2.421232in}}%
\pgfpathlineto{\pgfqpoint{3.689467in}{2.414717in}}%
\pgfpathlineto{\pgfqpoint{3.689562in}{2.412000in}}%
\pgfpathlineto{\pgfqpoint{3.689846in}{2.427313in}}%
\pgfpathlineto{\pgfqpoint{3.690415in}{2.440586in}}%
\pgfpathlineto{\pgfqpoint{3.690794in}{2.420197in}}%
\pgfpathlineto{\pgfqpoint{3.692310in}{2.359284in}}%
\pgfpathlineto{\pgfqpoint{3.691173in}{2.420744in}}%
\pgfpathlineto{\pgfqpoint{3.692879in}{2.366590in}}%
\pgfpathlineto{\pgfqpoint{3.693258in}{2.387295in}}%
\pgfpathlineto{\pgfqpoint{3.693921in}{2.371327in}}%
\pgfpathlineto{\pgfqpoint{3.695627in}{2.340597in}}%
\pgfpathlineto{\pgfqpoint{3.695722in}{2.340923in}}%
\pgfpathlineto{\pgfqpoint{3.696480in}{2.389026in}}%
\pgfpathlineto{\pgfqpoint{3.697333in}{2.372593in}}%
\pgfpathlineto{\pgfqpoint{3.698470in}{2.363287in}}%
\pgfpathlineto{\pgfqpoint{3.698659in}{2.370519in}}%
\pgfpathlineto{\pgfqpoint{3.699891in}{2.468056in}}%
\pgfpathlineto{\pgfqpoint{3.701029in}{2.653794in}}%
\pgfpathlineto{\pgfqpoint{3.701502in}{2.626787in}}%
\pgfpathlineto{\pgfqpoint{3.702071in}{2.554624in}}%
\pgfpathlineto{\pgfqpoint{3.702734in}{2.338621in}}%
\pgfpathlineto{\pgfqpoint{3.703492in}{2.412842in}}%
\pgfpathlineto{\pgfqpoint{3.704156in}{2.411045in}}%
\pgfpathlineto{\pgfqpoint{3.705009in}{2.442037in}}%
\pgfpathlineto{\pgfqpoint{3.705388in}{2.458829in}}%
\pgfpathlineto{\pgfqpoint{3.705672in}{2.481642in}}%
\pgfpathlineto{\pgfqpoint{3.706335in}{2.450354in}}%
\pgfpathlineto{\pgfqpoint{3.706809in}{2.426855in}}%
\pgfpathlineto{\pgfqpoint{3.707283in}{2.448623in}}%
\pgfpathlineto{\pgfqpoint{3.708894in}{2.490618in}}%
\pgfpathlineto{\pgfqpoint{3.709084in}{2.484365in}}%
\pgfpathlineto{\pgfqpoint{3.710031in}{2.449116in}}%
\pgfpathlineto{\pgfqpoint{3.710315in}{2.462843in}}%
\pgfpathlineto{\pgfqpoint{3.711737in}{2.507062in}}%
\pgfpathlineto{\pgfqpoint{3.712211in}{2.505438in}}%
\pgfpathlineto{\pgfqpoint{3.712685in}{2.490930in}}%
\pgfpathlineto{\pgfqpoint{3.713064in}{2.505742in}}%
\pgfpathlineto{\pgfqpoint{3.713443in}{2.501561in}}%
\pgfpathlineto{\pgfqpoint{3.715148in}{2.532049in}}%
\pgfpathlineto{\pgfqpoint{3.715338in}{2.532973in}}%
\pgfpathlineto{\pgfqpoint{3.715527in}{2.527977in}}%
\pgfpathlineto{\pgfqpoint{3.715717in}{2.524821in}}%
\pgfpathlineto{\pgfqpoint{3.716380in}{2.533984in}}%
\pgfpathlineto{\pgfqpoint{3.717044in}{2.562071in}}%
\pgfpathlineto{\pgfqpoint{3.717991in}{2.554669in}}%
\pgfpathlineto{\pgfqpoint{3.719697in}{2.539013in}}%
\pgfpathlineto{\pgfqpoint{3.718465in}{2.555071in}}%
\pgfpathlineto{\pgfqpoint{3.719887in}{2.545737in}}%
\pgfpathlineto{\pgfqpoint{3.719981in}{2.547384in}}%
\pgfpathlineto{\pgfqpoint{3.720266in}{2.537528in}}%
\pgfpathlineto{\pgfqpoint{3.723298in}{2.459288in}}%
\pgfpathlineto{\pgfqpoint{3.723393in}{2.462473in}}%
\pgfpathlineto{\pgfqpoint{3.724625in}{2.522774in}}%
\pgfpathlineto{\pgfqpoint{3.725288in}{2.519406in}}%
\pgfpathlineto{\pgfqpoint{3.725383in}{2.520183in}}%
\pgfpathlineto{\pgfqpoint{3.725762in}{2.515886in}}%
\pgfpathlineto{\pgfqpoint{3.726899in}{2.491686in}}%
\pgfpathlineto{\pgfqpoint{3.727657in}{2.493012in}}%
\pgfpathlineto{\pgfqpoint{3.728605in}{2.511137in}}%
\pgfpathlineto{\pgfqpoint{3.728226in}{2.490727in}}%
\pgfpathlineto{\pgfqpoint{3.728794in}{2.502618in}}%
\pgfpathlineto{\pgfqpoint{3.729647in}{2.488084in}}%
\pgfpathlineto{\pgfqpoint{3.730121in}{2.489397in}}%
\pgfpathlineto{\pgfqpoint{3.730500in}{2.494493in}}%
\pgfpathlineto{\pgfqpoint{3.731069in}{2.488990in}}%
\pgfpathlineto{\pgfqpoint{3.732680in}{2.474162in}}%
\pgfpathlineto{\pgfqpoint{3.731732in}{2.491244in}}%
\pgfpathlineto{\pgfqpoint{3.732775in}{2.475644in}}%
\pgfpathlineto{\pgfqpoint{3.732964in}{2.480690in}}%
\pgfpathlineto{\pgfqpoint{3.733343in}{2.463308in}}%
\pgfpathlineto{\pgfqpoint{3.733438in}{2.461392in}}%
\pgfpathlineto{\pgfqpoint{3.733627in}{2.475592in}}%
\pgfpathlineto{\pgfqpoint{3.734575in}{2.501008in}}%
\pgfpathlineto{\pgfqpoint{3.734954in}{2.500141in}}%
\pgfpathlineto{\pgfqpoint{3.735049in}{2.500676in}}%
\pgfpathlineto{\pgfqpoint{3.735238in}{2.496560in}}%
\pgfpathlineto{\pgfqpoint{3.735428in}{2.490660in}}%
\pgfpathlineto{\pgfqpoint{3.735807in}{2.497930in}}%
\pgfpathlineto{\pgfqpoint{3.736186in}{2.494789in}}%
\pgfpathlineto{\pgfqpoint{3.736660in}{2.518015in}}%
\pgfpathlineto{\pgfqpoint{3.737418in}{2.501637in}}%
\pgfpathlineto{\pgfqpoint{3.739029in}{2.426717in}}%
\pgfpathlineto{\pgfqpoint{3.739692in}{2.449044in}}%
\pgfpathlineto{\pgfqpoint{3.739977in}{2.453890in}}%
\pgfpathlineto{\pgfqpoint{3.740356in}{2.442153in}}%
\pgfpathlineto{\pgfqpoint{3.742156in}{2.415722in}}%
\pgfpathlineto{\pgfqpoint{3.742346in}{2.418560in}}%
\pgfpathlineto{\pgfqpoint{3.742914in}{2.455917in}}%
\pgfpathlineto{\pgfqpoint{3.743578in}{2.424500in}}%
\pgfpathlineto{\pgfqpoint{3.745189in}{2.397218in}}%
\pgfpathlineto{\pgfqpoint{3.745378in}{2.401921in}}%
\pgfpathlineto{\pgfqpoint{3.746231in}{2.458761in}}%
\pgfpathlineto{\pgfqpoint{3.747558in}{2.687447in}}%
\pgfpathlineto{\pgfqpoint{3.748316in}{2.636167in}}%
\pgfpathlineto{\pgfqpoint{3.749074in}{2.381534in}}%
\pgfpathlineto{\pgfqpoint{3.749264in}{2.354766in}}%
\pgfpathlineto{\pgfqpoint{3.749927in}{2.421021in}}%
\pgfpathlineto{\pgfqpoint{3.750306in}{2.415342in}}%
\pgfpathlineto{\pgfqpoint{3.750590in}{2.421503in}}%
\pgfpathlineto{\pgfqpoint{3.752391in}{2.465054in}}%
\pgfpathlineto{\pgfqpoint{3.754002in}{2.416820in}}%
\pgfpathlineto{\pgfqpoint{3.754381in}{2.436778in}}%
\pgfpathlineto{\pgfqpoint{3.755328in}{2.471604in}}%
\pgfpathlineto{\pgfqpoint{3.755897in}{2.452567in}}%
\pgfpathlineto{\pgfqpoint{3.756371in}{2.427540in}}%
\pgfpathlineto{\pgfqpoint{3.757129in}{2.443820in}}%
\pgfpathlineto{\pgfqpoint{3.758456in}{2.468023in}}%
\pgfpathlineto{\pgfqpoint{3.758740in}{2.456592in}}%
\pgfpathlineto{\pgfqpoint{3.759593in}{2.446898in}}%
\pgfpathlineto{\pgfqpoint{3.759782in}{2.452635in}}%
\pgfpathlineto{\pgfqpoint{3.761299in}{2.515829in}}%
\pgfpathlineto{\pgfqpoint{3.761488in}{2.509093in}}%
\pgfpathlineto{\pgfqpoint{3.761962in}{2.512013in}}%
\pgfpathlineto{\pgfqpoint{3.763004in}{2.493497in}}%
\pgfpathlineto{\pgfqpoint{3.763478in}{2.515275in}}%
\pgfpathlineto{\pgfqpoint{3.764331in}{2.500503in}}%
\pgfpathlineto{\pgfqpoint{3.765847in}{2.483829in}}%
\pgfpathlineto{\pgfqpoint{3.764710in}{2.504032in}}%
\pgfpathlineto{\pgfqpoint{3.765942in}{2.486096in}}%
\pgfpathlineto{\pgfqpoint{3.766984in}{2.506501in}}%
\pgfpathlineto{\pgfqpoint{3.767174in}{2.497363in}}%
\pgfpathlineto{\pgfqpoint{3.767837in}{2.484187in}}%
\pgfpathlineto{\pgfqpoint{3.768501in}{2.486659in}}%
\pgfpathlineto{\pgfqpoint{3.773713in}{2.413149in}}%
\pgfpathlineto{\pgfqpoint{3.768974in}{2.492334in}}%
\pgfpathlineto{\pgfqpoint{3.773902in}{2.416428in}}%
\pgfpathlineto{\pgfqpoint{3.774092in}{2.422065in}}%
\pgfpathlineto{\pgfqpoint{3.774471in}{2.397580in}}%
\pgfpathlineto{\pgfqpoint{3.776082in}{2.370239in}}%
\pgfpathlineto{\pgfqpoint{3.774755in}{2.399143in}}%
\pgfpathlineto{\pgfqpoint{3.776556in}{2.376893in}}%
\pgfpathlineto{\pgfqpoint{3.777408in}{2.381946in}}%
\pgfpathlineto{\pgfqpoint{3.777029in}{2.374582in}}%
\pgfpathlineto{\pgfqpoint{3.777598in}{2.378700in}}%
\pgfpathlineto{\pgfqpoint{3.778735in}{2.356926in}}%
\pgfpathlineto{\pgfqpoint{3.778925in}{2.365558in}}%
\pgfpathlineto{\pgfqpoint{3.780251in}{2.380685in}}%
\pgfpathlineto{\pgfqpoint{3.779493in}{2.358082in}}%
\pgfpathlineto{\pgfqpoint{3.780346in}{2.379914in}}%
\pgfpathlineto{\pgfqpoint{3.780441in}{2.379515in}}%
\pgfpathlineto{\pgfqpoint{3.780630in}{2.382781in}}%
\pgfpathlineto{\pgfqpoint{3.782052in}{2.397030in}}%
\pgfpathlineto{\pgfqpoint{3.782241in}{2.396247in}}%
\pgfpathlineto{\pgfqpoint{3.782431in}{2.398787in}}%
\pgfpathlineto{\pgfqpoint{3.783000in}{2.426113in}}%
\pgfpathlineto{\pgfqpoint{3.783852in}{2.420948in}}%
\pgfpathlineto{\pgfqpoint{3.785179in}{2.356459in}}%
\pgfpathlineto{\pgfqpoint{3.786032in}{2.379142in}}%
\pgfpathlineto{\pgfqpoint{3.786316in}{2.390828in}}%
\pgfpathlineto{\pgfqpoint{3.786885in}{2.369037in}}%
\pgfpathlineto{\pgfqpoint{3.788212in}{2.344654in}}%
\pgfpathlineto{\pgfqpoint{3.788591in}{2.355598in}}%
\pgfpathlineto{\pgfqpoint{3.789349in}{2.391849in}}%
\pgfpathlineto{\pgfqpoint{3.790012in}{2.369092in}}%
\pgfpathlineto{\pgfqpoint{3.792192in}{2.278773in}}%
\pgfpathlineto{\pgfqpoint{3.792476in}{2.301297in}}%
\pgfpathlineto{\pgfqpoint{3.793992in}{2.600129in}}%
\pgfpathlineto{\pgfqpoint{3.794845in}{2.538456in}}%
\pgfpathlineto{\pgfqpoint{3.795603in}{2.298083in}}%
\pgfpathlineto{\pgfqpoint{3.796551in}{2.370814in}}%
\pgfpathlineto{\pgfqpoint{3.796740in}{2.372452in}}%
\pgfpathlineto{\pgfqpoint{3.798636in}{2.425873in}}%
\pgfpathlineto{\pgfqpoint{3.799015in}{2.406726in}}%
\pgfpathlineto{\pgfqpoint{3.799962in}{2.379511in}}%
\pgfpathlineto{\pgfqpoint{3.800247in}{2.393202in}}%
\pgfpathlineto{\pgfqpoint{3.801858in}{2.427995in}}%
\pgfpathlineto{\pgfqpoint{3.802331in}{2.396330in}}%
\pgfpathlineto{\pgfqpoint{3.802616in}{2.389402in}}%
\pgfpathlineto{\pgfqpoint{3.803279in}{2.403518in}}%
\pgfpathlineto{\pgfqpoint{3.804511in}{2.437491in}}%
\pgfpathlineto{\pgfqpoint{3.804795in}{2.431689in}}%
\pgfpathlineto{\pgfqpoint{3.805174in}{2.428461in}}%
\pgfpathlineto{\pgfqpoint{3.805932in}{2.413272in}}%
\pgfpathlineto{\pgfqpoint{3.806122in}{2.422729in}}%
\pgfpathlineto{\pgfqpoint{3.807164in}{2.450029in}}%
\pgfpathlineto{\pgfqpoint{3.807449in}{2.446295in}}%
\pgfpathlineto{\pgfqpoint{3.807543in}{2.445368in}}%
\pgfpathlineto{\pgfqpoint{3.807733in}{2.449257in}}%
\pgfpathlineto{\pgfqpoint{3.808017in}{2.459505in}}%
\pgfpathlineto{\pgfqpoint{3.808775in}{2.448257in}}%
\pgfpathlineto{\pgfqpoint{3.809060in}{2.439691in}}%
\pgfpathlineto{\pgfqpoint{3.809533in}{2.460856in}}%
\pgfpathlineto{\pgfqpoint{3.809628in}{2.460000in}}%
\pgfpathlineto{\pgfqpoint{3.809818in}{2.457293in}}%
\pgfpathlineto{\pgfqpoint{3.810292in}{2.469165in}}%
\pgfpathlineto{\pgfqpoint{3.810671in}{2.461804in}}%
\pgfpathlineto{\pgfqpoint{3.811903in}{2.434824in}}%
\pgfpathlineto{\pgfqpoint{3.812376in}{2.443843in}}%
\pgfpathlineto{\pgfqpoint{3.813135in}{2.459954in}}%
\pgfpathlineto{\pgfqpoint{3.813703in}{2.449871in}}%
\pgfpathlineto{\pgfqpoint{3.813987in}{2.455226in}}%
\pgfpathlineto{\pgfqpoint{3.814461in}{2.443426in}}%
\pgfpathlineto{\pgfqpoint{3.815409in}{2.449668in}}%
\pgfpathlineto{\pgfqpoint{3.815030in}{2.441717in}}%
\pgfpathlineto{\pgfqpoint{3.815598in}{2.445143in}}%
\pgfpathlineto{\pgfqpoint{3.817020in}{2.422319in}}%
\pgfpathlineto{\pgfqpoint{3.818631in}{2.401102in}}%
\pgfpathlineto{\pgfqpoint{3.819105in}{2.403914in}}%
\pgfpathlineto{\pgfqpoint{3.820147in}{2.408039in}}%
\pgfpathlineto{\pgfqpoint{3.819768in}{2.400302in}}%
\pgfpathlineto{\pgfqpoint{3.820242in}{2.406502in}}%
\pgfpathlineto{\pgfqpoint{3.821095in}{2.394990in}}%
\pgfpathlineto{\pgfqpoint{3.821474in}{2.401094in}}%
\pgfpathlineto{\pgfqpoint{3.821663in}{2.402699in}}%
\pgfpathlineto{\pgfqpoint{3.822421in}{2.399619in}}%
\pgfpathlineto{\pgfqpoint{3.822706in}{2.390367in}}%
\pgfpathlineto{\pgfqpoint{3.823180in}{2.400498in}}%
\pgfpathlineto{\pgfqpoint{3.823559in}{2.394974in}}%
\pgfpathlineto{\pgfqpoint{3.824032in}{2.409991in}}%
\pgfpathlineto{\pgfqpoint{3.824791in}{2.401576in}}%
\pgfpathlineto{\pgfqpoint{3.825075in}{2.394907in}}%
\pgfpathlineto{\pgfqpoint{3.825549in}{2.404000in}}%
\pgfpathlineto{\pgfqpoint{3.825738in}{2.403147in}}%
\pgfpathlineto{\pgfqpoint{3.826402in}{2.401943in}}%
\pgfpathlineto{\pgfqpoint{3.827254in}{2.421608in}}%
\pgfpathlineto{\pgfqpoint{3.828392in}{2.435301in}}%
\pgfpathlineto{\pgfqpoint{3.827823in}{2.412378in}}%
\pgfpathlineto{\pgfqpoint{3.828581in}{2.434134in}}%
\pgfpathlineto{\pgfqpoint{3.828960in}{2.430295in}}%
\pgfpathlineto{\pgfqpoint{3.829150in}{2.422861in}}%
\pgfpathlineto{\pgfqpoint{3.829529in}{2.449233in}}%
\pgfpathlineto{\pgfqpoint{3.830476in}{2.467479in}}%
\pgfpathlineto{\pgfqpoint{3.830761in}{2.461419in}}%
\pgfpathlineto{\pgfqpoint{3.832561in}{2.374392in}}%
\pgfpathlineto{\pgfqpoint{3.833509in}{2.395501in}}%
\pgfpathlineto{\pgfqpoint{3.833604in}{2.395773in}}%
\pgfpathlineto{\pgfqpoint{3.833698in}{2.393808in}}%
\pgfpathlineto{\pgfqpoint{3.835404in}{2.345421in}}%
\pgfpathlineto{\pgfqpoint{3.835499in}{2.348004in}}%
\pgfpathlineto{\pgfqpoint{3.836352in}{2.400243in}}%
\pgfpathlineto{\pgfqpoint{3.837110in}{2.378244in}}%
\pgfpathlineto{\pgfqpoint{3.837299in}{2.379247in}}%
\pgfpathlineto{\pgfqpoint{3.837489in}{2.376522in}}%
\pgfpathlineto{\pgfqpoint{3.837773in}{2.368359in}}%
\pgfpathlineto{\pgfqpoint{3.838437in}{2.375934in}}%
\pgfpathlineto{\pgfqpoint{3.839858in}{2.471638in}}%
\pgfpathlineto{\pgfqpoint{3.840900in}{2.659453in}}%
\pgfpathlineto{\pgfqpoint{3.841374in}{2.623736in}}%
\pgfpathlineto{\pgfqpoint{3.842038in}{2.531567in}}%
\pgfpathlineto{\pgfqpoint{3.842701in}{2.340316in}}%
\pgfpathlineto{\pgfqpoint{3.843459in}{2.404620in}}%
\pgfpathlineto{\pgfqpoint{3.843649in}{2.402449in}}%
\pgfpathlineto{\pgfqpoint{3.844028in}{2.416021in}}%
\pgfpathlineto{\pgfqpoint{3.845923in}{2.454896in}}%
\pgfpathlineto{\pgfqpoint{3.846112in}{2.444339in}}%
\pgfpathlineto{\pgfqpoint{3.847060in}{2.410302in}}%
\pgfpathlineto{\pgfqpoint{3.847344in}{2.424344in}}%
\pgfpathlineto{\pgfqpoint{3.848861in}{2.443898in}}%
\pgfpathlineto{\pgfqpoint{3.849050in}{2.447377in}}%
\pgfpathlineto{\pgfqpoint{3.849334in}{2.428251in}}%
\pgfpathlineto{\pgfqpoint{3.849619in}{2.404417in}}%
\pgfpathlineto{\pgfqpoint{3.850377in}{2.425857in}}%
\pgfpathlineto{\pgfqpoint{3.851893in}{2.452674in}}%
\pgfpathlineto{\pgfqpoint{3.851988in}{2.454366in}}%
\pgfpathlineto{\pgfqpoint{3.852272in}{2.443914in}}%
\pgfpathlineto{\pgfqpoint{3.853030in}{2.438637in}}%
\pgfpathlineto{\pgfqpoint{3.853315in}{2.445941in}}%
\pgfpathlineto{\pgfqpoint{3.855020in}{2.478280in}}%
\pgfpathlineto{\pgfqpoint{3.855115in}{2.478021in}}%
\pgfpathlineto{\pgfqpoint{3.855968in}{2.454332in}}%
\pgfpathlineto{\pgfqpoint{3.856347in}{2.471917in}}%
\pgfpathlineto{\pgfqpoint{3.857105in}{2.483545in}}%
\pgfpathlineto{\pgfqpoint{3.857674in}{2.481528in}}%
\pgfpathlineto{\pgfqpoint{3.858242in}{2.477451in}}%
\pgfpathlineto{\pgfqpoint{3.858716in}{2.481860in}}%
\pgfpathlineto{\pgfqpoint{3.859095in}{2.487899in}}%
\pgfpathlineto{\pgfqpoint{3.859285in}{2.480314in}}%
\pgfpathlineto{\pgfqpoint{3.859569in}{2.468760in}}%
\pgfpathlineto{\pgfqpoint{3.859948in}{2.488453in}}%
\pgfpathlineto{\pgfqpoint{3.860327in}{2.481707in}}%
\pgfpathlineto{\pgfqpoint{3.861749in}{2.493327in}}%
\pgfpathlineto{\pgfqpoint{3.861843in}{2.491809in}}%
\pgfpathlineto{\pgfqpoint{3.864591in}{2.404168in}}%
\pgfpathlineto{\pgfqpoint{3.865539in}{2.437152in}}%
\pgfpathlineto{\pgfqpoint{3.867245in}{2.465400in}}%
\pgfpathlineto{\pgfqpoint{3.867340in}{2.465185in}}%
\pgfpathlineto{\pgfqpoint{3.869045in}{2.435717in}}%
\pgfpathlineto{\pgfqpoint{3.869235in}{2.441821in}}%
\pgfpathlineto{\pgfqpoint{3.869519in}{2.459472in}}%
\pgfpathlineto{\pgfqpoint{3.869898in}{2.436430in}}%
\pgfpathlineto{\pgfqpoint{3.870467in}{2.454202in}}%
\pgfpathlineto{\pgfqpoint{3.871794in}{2.436439in}}%
\pgfpathlineto{\pgfqpoint{3.872457in}{2.421537in}}%
\pgfpathlineto{\pgfqpoint{3.872931in}{2.431100in}}%
\pgfpathlineto{\pgfqpoint{3.875205in}{2.455105in}}%
\pgfpathlineto{\pgfqpoint{3.875395in}{2.455588in}}%
\pgfpathlineto{\pgfqpoint{3.875489in}{2.454488in}}%
\pgfpathlineto{\pgfqpoint{3.875868in}{2.445788in}}%
\pgfpathlineto{\pgfqpoint{3.876153in}{2.456603in}}%
\pgfpathlineto{\pgfqpoint{3.876626in}{2.490558in}}%
\pgfpathlineto{\pgfqpoint{3.877290in}{2.463228in}}%
\pgfpathlineto{\pgfqpoint{3.877479in}{2.462060in}}%
\pgfpathlineto{\pgfqpoint{3.879090in}{2.393749in}}%
\pgfpathlineto{\pgfqpoint{3.879564in}{2.423632in}}%
\pgfpathlineto{\pgfqpoint{3.879659in}{2.427601in}}%
\pgfpathlineto{\pgfqpoint{3.880322in}{2.409577in}}%
\pgfpathlineto{\pgfqpoint{3.881554in}{2.372452in}}%
\pgfpathlineto{\pgfqpoint{3.881933in}{2.385544in}}%
\pgfpathlineto{\pgfqpoint{3.882218in}{2.384432in}}%
\pgfpathlineto{\pgfqpoint{3.882407in}{2.389322in}}%
\pgfpathlineto{\pgfqpoint{3.882881in}{2.419000in}}%
\pgfpathlineto{\pgfqpoint{3.883544in}{2.393559in}}%
\pgfpathlineto{\pgfqpoint{3.884776in}{2.376135in}}%
\pgfpathlineto{\pgfqpoint{3.885060in}{2.383312in}}%
\pgfpathlineto{\pgfqpoint{3.885155in}{2.384084in}}%
\pgfpathlineto{\pgfqpoint{3.885345in}{2.378874in}}%
\pgfpathlineto{\pgfqpoint{3.885534in}{2.373040in}}%
\pgfpathlineto{\pgfqpoint{3.885913in}{2.392022in}}%
\pgfpathlineto{\pgfqpoint{3.887145in}{2.640876in}}%
\pgfpathlineto{\pgfqpoint{3.888282in}{2.575037in}}%
\pgfpathlineto{\pgfqpoint{3.889135in}{2.321390in}}%
\pgfpathlineto{\pgfqpoint{3.890083in}{2.400247in}}%
\pgfpathlineto{\pgfqpoint{3.890273in}{2.399220in}}%
\pgfpathlineto{\pgfqpoint{3.890557in}{2.396022in}}%
\pgfpathlineto{\pgfqpoint{3.890936in}{2.401755in}}%
\pgfpathlineto{\pgfqpoint{3.892168in}{2.447837in}}%
\pgfpathlineto{\pgfqpoint{3.892642in}{2.432416in}}%
\pgfpathlineto{\pgfqpoint{3.893210in}{2.411656in}}%
\pgfpathlineto{\pgfqpoint{3.893779in}{2.429836in}}%
\pgfpathlineto{\pgfqpoint{3.893968in}{2.429609in}}%
\pgfpathlineto{\pgfqpoint{3.894063in}{2.430757in}}%
\pgfpathlineto{\pgfqpoint{3.894916in}{2.452146in}}%
\pgfpathlineto{\pgfqpoint{3.895769in}{2.448793in}}%
\pgfpathlineto{\pgfqpoint{3.896337in}{2.413930in}}%
\pgfpathlineto{\pgfqpoint{3.896906in}{2.442855in}}%
\pgfpathlineto{\pgfqpoint{3.898327in}{2.481712in}}%
\pgfpathlineto{\pgfqpoint{3.898707in}{2.467544in}}%
\pgfpathlineto{\pgfqpoint{3.898991in}{2.470134in}}%
\pgfpathlineto{\pgfqpoint{3.899180in}{2.465604in}}%
\pgfpathlineto{\pgfqpoint{3.899370in}{2.460632in}}%
\pgfpathlineto{\pgfqpoint{3.899844in}{2.476028in}}%
\pgfpathlineto{\pgfqpoint{3.901170in}{2.496577in}}%
\pgfpathlineto{\pgfqpoint{3.901644in}{2.489560in}}%
\pgfpathlineto{\pgfqpoint{3.902592in}{2.482937in}}%
\pgfpathlineto{\pgfqpoint{3.902781in}{2.487649in}}%
\pgfpathlineto{\pgfqpoint{3.903160in}{2.500379in}}%
\pgfpathlineto{\pgfqpoint{3.904013in}{2.495925in}}%
\pgfpathlineto{\pgfqpoint{3.905435in}{2.475761in}}%
\pgfpathlineto{\pgfqpoint{3.904582in}{2.496963in}}%
\pgfpathlineto{\pgfqpoint{3.906193in}{2.479268in}}%
\pgfpathlineto{\pgfqpoint{3.907330in}{2.493075in}}%
\pgfpathlineto{\pgfqpoint{3.907520in}{2.490703in}}%
\pgfpathlineto{\pgfqpoint{3.908372in}{2.481762in}}%
\pgfpathlineto{\pgfqpoint{3.908562in}{2.487144in}}%
\pgfpathlineto{\pgfqpoint{3.909320in}{2.491135in}}%
\pgfpathlineto{\pgfqpoint{3.909036in}{2.486680in}}%
\pgfpathlineto{\pgfqpoint{3.909510in}{2.487925in}}%
\pgfpathlineto{\pgfqpoint{3.910742in}{2.473211in}}%
\pgfpathlineto{\pgfqpoint{3.910931in}{2.473922in}}%
\pgfpathlineto{\pgfqpoint{3.911310in}{2.475091in}}%
\pgfpathlineto{\pgfqpoint{3.911594in}{2.472910in}}%
\pgfpathlineto{\pgfqpoint{3.913395in}{2.444570in}}%
\pgfpathlineto{\pgfqpoint{3.913869in}{2.451612in}}%
\pgfpathlineto{\pgfqpoint{3.914153in}{2.443913in}}%
\pgfpathlineto{\pgfqpoint{3.915290in}{2.436703in}}%
\pgfpathlineto{\pgfqpoint{3.914816in}{2.453385in}}%
\pgfpathlineto{\pgfqpoint{3.915385in}{2.436949in}}%
\pgfpathlineto{\pgfqpoint{3.916617in}{2.463107in}}%
\pgfpathlineto{\pgfqpoint{3.916996in}{2.451387in}}%
\pgfpathlineto{\pgfqpoint{3.917659in}{2.449640in}}%
\pgfpathlineto{\pgfqpoint{3.917944in}{2.452599in}}%
\pgfpathlineto{\pgfqpoint{3.918133in}{2.448601in}}%
\pgfpathlineto{\pgfqpoint{3.918512in}{2.429883in}}%
\pgfpathlineto{\pgfqpoint{3.919081in}{2.456767in}}%
\pgfpathlineto{\pgfqpoint{3.919270in}{2.456207in}}%
\pgfpathlineto{\pgfqpoint{3.919460in}{2.459807in}}%
\pgfpathlineto{\pgfqpoint{3.921260in}{2.497924in}}%
\pgfpathlineto{\pgfqpoint{3.921450in}{2.488296in}}%
\pgfpathlineto{\pgfqpoint{3.921639in}{2.481993in}}%
\pgfpathlineto{\pgfqpoint{3.922018in}{2.497051in}}%
\pgfpathlineto{\pgfqpoint{3.922303in}{2.495435in}}%
\pgfpathlineto{\pgfqpoint{3.922777in}{2.511790in}}%
\pgfpathlineto{\pgfqpoint{3.923156in}{2.493735in}}%
\pgfpathlineto{\pgfqpoint{3.925240in}{2.432567in}}%
\pgfpathlineto{\pgfqpoint{3.925525in}{2.448172in}}%
\pgfpathlineto{\pgfqpoint{3.925809in}{2.458068in}}%
\pgfpathlineto{\pgfqpoint{3.926567in}{2.449186in}}%
\pgfpathlineto{\pgfqpoint{3.926946in}{2.429126in}}%
\pgfpathlineto{\pgfqpoint{3.927894in}{2.432075in}}%
\pgfpathlineto{\pgfqpoint{3.927989in}{2.431380in}}%
\pgfpathlineto{\pgfqpoint{3.928178in}{2.435544in}}%
\pgfpathlineto{\pgfqpoint{3.928842in}{2.459153in}}%
\pgfpathlineto{\pgfqpoint{3.929505in}{2.449383in}}%
\pgfpathlineto{\pgfqpoint{3.930547in}{2.426415in}}%
\pgfpathlineto{\pgfqpoint{3.930926in}{2.436765in}}%
\pgfpathlineto{\pgfqpoint{3.931305in}{2.428110in}}%
\pgfpathlineto{\pgfqpoint{3.931874in}{2.436101in}}%
\pgfpathlineto{\pgfqpoint{3.932348in}{2.501139in}}%
\pgfpathlineto{\pgfqpoint{3.933485in}{2.663728in}}%
\pgfpathlineto{\pgfqpoint{3.933864in}{2.637757in}}%
\pgfpathlineto{\pgfqpoint{3.934812in}{2.444916in}}%
\pgfpathlineto{\pgfqpoint{3.935191in}{2.349222in}}%
\pgfpathlineto{\pgfqpoint{3.935949in}{2.422531in}}%
\pgfpathlineto{\pgfqpoint{3.936233in}{2.428249in}}%
\pgfpathlineto{\pgfqpoint{3.936612in}{2.419068in}}%
\pgfpathlineto{\pgfqpoint{3.937086in}{2.423662in}}%
\pgfpathlineto{\pgfqpoint{3.937749in}{2.420551in}}%
\pgfpathlineto{\pgfqpoint{3.937939in}{2.424019in}}%
\pgfpathlineto{\pgfqpoint{3.938128in}{2.426400in}}%
\pgfpathlineto{\pgfqpoint{3.938413in}{2.413680in}}%
\pgfpathlineto{\pgfqpoint{3.939266in}{2.368666in}}%
\pgfpathlineto{\pgfqpoint{3.939739in}{2.388448in}}%
\pgfpathlineto{\pgfqpoint{3.941540in}{2.507532in}}%
\pgfpathlineto{\pgfqpoint{3.942488in}{2.479938in}}%
\pgfpathlineto{\pgfqpoint{3.942677in}{2.479250in}}%
\pgfpathlineto{\pgfqpoint{3.942772in}{2.479900in}}%
\pgfpathlineto{\pgfqpoint{3.944478in}{2.507648in}}%
\pgfpathlineto{\pgfqpoint{3.944667in}{2.504099in}}%
\pgfpathlineto{\pgfqpoint{3.945330in}{2.478873in}}%
\pgfpathlineto{\pgfqpoint{3.945899in}{2.493484in}}%
\pgfpathlineto{\pgfqpoint{3.946657in}{2.508906in}}%
\pgfpathlineto{\pgfqpoint{3.946941in}{2.497448in}}%
\pgfpathlineto{\pgfqpoint{3.947605in}{2.507618in}}%
\pgfpathlineto{\pgfqpoint{3.948173in}{2.488413in}}%
\pgfpathlineto{\pgfqpoint{3.948647in}{2.485344in}}%
\pgfpathlineto{\pgfqpoint{3.948837in}{2.488498in}}%
\pgfpathlineto{\pgfqpoint{3.950353in}{2.511451in}}%
\pgfpathlineto{\pgfqpoint{3.950448in}{2.513155in}}%
\pgfpathlineto{\pgfqpoint{3.950827in}{2.503198in}}%
\pgfpathlineto{\pgfqpoint{3.951964in}{2.489956in}}%
\pgfpathlineto{\pgfqpoint{3.952059in}{2.492025in}}%
\pgfpathlineto{\pgfqpoint{3.952438in}{2.509004in}}%
\pgfpathlineto{\pgfqpoint{3.953196in}{2.495797in}}%
\pgfpathlineto{\pgfqpoint{3.953480in}{2.477374in}}%
\pgfpathlineto{\pgfqpoint{3.954333in}{2.485291in}}%
\pgfpathlineto{\pgfqpoint{3.954617in}{2.494330in}}%
\pgfpathlineto{\pgfqpoint{3.955091in}{2.473480in}}%
\pgfpathlineto{\pgfqpoint{3.955281in}{2.474717in}}%
\pgfpathlineto{\pgfqpoint{3.955470in}{2.476695in}}%
\pgfpathlineto{\pgfqpoint{3.955849in}{2.468845in}}%
\pgfpathlineto{\pgfqpoint{3.955944in}{2.468262in}}%
\pgfpathlineto{\pgfqpoint{3.956039in}{2.470825in}}%
\pgfpathlineto{\pgfqpoint{3.956323in}{2.478557in}}%
\pgfpathlineto{\pgfqpoint{3.956702in}{2.451843in}}%
\pgfpathlineto{\pgfqpoint{3.957176in}{2.456845in}}%
\pgfpathlineto{\pgfqpoint{3.958408in}{2.436571in}}%
\pgfpathlineto{\pgfqpoint{3.958976in}{2.428608in}}%
\pgfpathlineto{\pgfqpoint{3.960019in}{2.429659in}}%
\pgfpathlineto{\pgfqpoint{3.960398in}{2.430417in}}%
\pgfpathlineto{\pgfqpoint{3.960777in}{2.418497in}}%
\pgfpathlineto{\pgfqpoint{3.960872in}{2.415237in}}%
\pgfpathlineto{\pgfqpoint{3.961440in}{2.430603in}}%
\pgfpathlineto{\pgfqpoint{3.961535in}{2.429331in}}%
\pgfpathlineto{\pgfqpoint{3.962009in}{2.425331in}}%
\pgfpathlineto{\pgfqpoint{3.962388in}{2.432721in}}%
\pgfpathlineto{\pgfqpoint{3.964757in}{2.406754in}}%
\pgfpathlineto{\pgfqpoint{3.963146in}{2.436198in}}%
\pgfpathlineto{\pgfqpoint{3.965041in}{2.417669in}}%
\pgfpathlineto{\pgfqpoint{3.966273in}{2.442181in}}%
\pgfpathlineto{\pgfqpoint{3.966652in}{2.434588in}}%
\pgfpathlineto{\pgfqpoint{3.966842in}{2.433308in}}%
\pgfpathlineto{\pgfqpoint{3.967221in}{2.422847in}}%
\pgfpathlineto{\pgfqpoint{3.967505in}{2.441018in}}%
\pgfpathlineto{\pgfqpoint{3.968453in}{2.463465in}}%
\pgfpathlineto{\pgfqpoint{3.968737in}{2.453357in}}%
\pgfpathlineto{\pgfqpoint{3.970443in}{2.387463in}}%
\pgfpathlineto{\pgfqpoint{3.970917in}{2.397808in}}%
\pgfpathlineto{\pgfqpoint{3.971485in}{2.412697in}}%
\pgfpathlineto{\pgfqpoint{3.971675in}{2.419633in}}%
\pgfpathlineto{\pgfqpoint{3.972149in}{2.397817in}}%
\pgfpathlineto{\pgfqpoint{3.972528in}{2.382282in}}%
\pgfpathlineto{\pgfqpoint{3.973002in}{2.403968in}}%
\pgfpathlineto{\pgfqpoint{3.973286in}{2.394636in}}%
\pgfpathlineto{\pgfqpoint{3.973475in}{2.391491in}}%
\pgfpathlineto{\pgfqpoint{3.973854in}{2.402713in}}%
\pgfpathlineto{\pgfqpoint{3.974802in}{2.432740in}}%
\pgfpathlineto{\pgfqpoint{3.975181in}{2.419258in}}%
\pgfpathlineto{\pgfqpoint{3.976413in}{2.399278in}}%
\pgfpathlineto{\pgfqpoint{3.975939in}{2.424455in}}%
\pgfpathlineto{\pgfqpoint{3.976508in}{2.400046in}}%
\pgfpathlineto{\pgfqpoint{3.977929in}{2.464038in}}%
\pgfpathlineto{\pgfqpoint{3.979066in}{2.650524in}}%
\pgfpathlineto{\pgfqpoint{3.979635in}{2.633765in}}%
\pgfpathlineto{\pgfqpoint{3.979825in}{2.628753in}}%
\pgfpathlineto{\pgfqpoint{3.980488in}{2.445742in}}%
\pgfpathlineto{\pgfqpoint{3.980867in}{2.349148in}}%
\pgfpathlineto{\pgfqpoint{3.981625in}{2.411203in}}%
\pgfpathlineto{\pgfqpoint{3.982383in}{2.428798in}}%
\pgfpathlineto{\pgfqpoint{3.983994in}{2.484829in}}%
\pgfpathlineto{\pgfqpoint{3.984089in}{2.482815in}}%
\pgfpathlineto{\pgfqpoint{3.984847in}{2.454017in}}%
\pgfpathlineto{\pgfqpoint{3.985605in}{2.472477in}}%
\pgfpathlineto{\pgfqpoint{3.985889in}{2.482418in}}%
\pgfpathlineto{\pgfqpoint{3.986932in}{2.517955in}}%
\pgfpathlineto{\pgfqpoint{3.987216in}{2.500486in}}%
\pgfpathlineto{\pgfqpoint{3.988164in}{2.470440in}}%
\pgfpathlineto{\pgfqpoint{3.988543in}{2.484956in}}%
\pgfpathlineto{\pgfqpoint{3.989585in}{2.505388in}}%
\pgfpathlineto{\pgfqpoint{3.989870in}{2.498923in}}%
\pgfpathlineto{\pgfqpoint{3.990154in}{2.508658in}}%
\pgfpathlineto{\pgfqpoint{3.990343in}{2.512799in}}%
\pgfpathlineto{\pgfqpoint{3.990817in}{2.494745in}}%
\pgfpathlineto{\pgfqpoint{3.990912in}{2.496383in}}%
\pgfpathlineto{\pgfqpoint{3.991196in}{2.495786in}}%
\pgfpathlineto{\pgfqpoint{3.991575in}{2.512140in}}%
\pgfpathlineto{\pgfqpoint{3.993281in}{2.554774in}}%
\pgfpathlineto{\pgfqpoint{3.993376in}{2.554788in}}%
\pgfpathlineto{\pgfqpoint{3.993850in}{2.537650in}}%
\pgfpathlineto{\pgfqpoint{3.994418in}{2.553423in}}%
\pgfpathlineto{\pgfqpoint{3.995461in}{2.563389in}}%
\pgfpathlineto{\pgfqpoint{3.995650in}{2.560014in}}%
\pgfpathlineto{\pgfqpoint{3.996124in}{2.560301in}}%
\pgfpathlineto{\pgfqpoint{3.996882in}{2.557579in}}%
\pgfpathlineto{\pgfqpoint{3.999062in}{2.537765in}}%
\pgfpathlineto{\pgfqpoint{3.999346in}{2.541874in}}%
\pgfpathlineto{\pgfqpoint{4.000009in}{2.552967in}}%
\pgfpathlineto{\pgfqpoint{4.000388in}{2.541472in}}%
\pgfpathlineto{\pgfqpoint{4.000578in}{2.544248in}}%
\pgfpathlineto{\pgfqpoint{4.001715in}{2.564397in}}%
\pgfpathlineto{\pgfqpoint{4.001241in}{2.542919in}}%
\pgfpathlineto{\pgfqpoint{4.001905in}{2.557844in}}%
\pgfpathlineto{\pgfqpoint{4.002568in}{2.534785in}}%
\pgfpathlineto{\pgfqpoint{4.003137in}{2.546295in}}%
\pgfpathlineto{\pgfqpoint{4.003421in}{2.548593in}}%
\pgfpathlineto{\pgfqpoint{4.003800in}{2.543668in}}%
\pgfpathlineto{\pgfqpoint{4.004748in}{2.539946in}}%
\pgfpathlineto{\pgfqpoint{4.004274in}{2.544802in}}%
\pgfpathlineto{\pgfqpoint{4.005032in}{2.541118in}}%
\pgfpathlineto{\pgfqpoint{4.005411in}{2.542255in}}%
\pgfpathlineto{\pgfqpoint{4.005695in}{2.545796in}}%
\pgfpathlineto{\pgfqpoint{4.005979in}{2.536686in}}%
\pgfpathlineto{\pgfqpoint{4.006169in}{2.530126in}}%
\pgfpathlineto{\pgfqpoint{4.006643in}{2.537310in}}%
\pgfpathlineto{\pgfqpoint{4.007117in}{2.532916in}}%
\pgfpathlineto{\pgfqpoint{4.007306in}{2.531088in}}%
\pgfpathlineto{\pgfqpoint{4.007685in}{2.539319in}}%
\pgfpathlineto{\pgfqpoint{4.007780in}{2.540079in}}%
\pgfpathlineto{\pgfqpoint{4.007970in}{2.533993in}}%
\pgfpathlineto{\pgfqpoint{4.008159in}{2.528805in}}%
\pgfpathlineto{\pgfqpoint{4.008633in}{2.539582in}}%
\pgfpathlineto{\pgfqpoint{4.008822in}{2.539569in}}%
\pgfpathlineto{\pgfqpoint{4.009107in}{2.541686in}}%
\pgfpathlineto{\pgfqpoint{4.009296in}{2.535199in}}%
\pgfpathlineto{\pgfqpoint{4.009960in}{2.509896in}}%
\pgfpathlineto{\pgfqpoint{4.010528in}{2.525112in}}%
\pgfpathlineto{\pgfqpoint{4.011571in}{2.539756in}}%
\pgfpathlineto{\pgfqpoint{4.011950in}{2.539511in}}%
\pgfpathlineto{\pgfqpoint{4.012044in}{2.540121in}}%
\pgfpathlineto{\pgfqpoint{4.012234in}{2.536193in}}%
\pgfpathlineto{\pgfqpoint{4.013561in}{2.504954in}}%
\pgfpathlineto{\pgfqpoint{4.013845in}{2.505847in}}%
\pgfpathlineto{\pgfqpoint{4.014793in}{2.490983in}}%
\pgfpathlineto{\pgfqpoint{4.015077in}{2.501483in}}%
\pgfpathlineto{\pgfqpoint{4.016877in}{2.546346in}}%
\pgfpathlineto{\pgfqpoint{4.017067in}{2.550120in}}%
\pgfpathlineto{\pgfqpoint{4.017541in}{2.536965in}}%
\pgfpathlineto{\pgfqpoint{4.018962in}{2.468490in}}%
\pgfpathlineto{\pgfqpoint{4.019626in}{2.480032in}}%
\pgfpathlineto{\pgfqpoint{4.020194in}{2.503724in}}%
\pgfpathlineto{\pgfqpoint{4.020668in}{2.486527in}}%
\pgfpathlineto{\pgfqpoint{4.021805in}{2.445219in}}%
\pgfpathlineto{\pgfqpoint{4.022089in}{2.457105in}}%
\pgfpathlineto{\pgfqpoint{4.023795in}{2.618397in}}%
\pgfpathlineto{\pgfqpoint{4.024648in}{2.729119in}}%
\pgfpathlineto{\pgfqpoint{4.025217in}{2.696880in}}%
\pgfpathlineto{\pgfqpoint{4.025690in}{2.646577in}}%
\pgfpathlineto{\pgfqpoint{4.026449in}{2.416312in}}%
\pgfpathlineto{\pgfqpoint{4.027396in}{2.485637in}}%
\pgfpathlineto{\pgfqpoint{4.027775in}{2.482575in}}%
\pgfpathlineto{\pgfqpoint{4.027965in}{2.486302in}}%
\pgfpathlineto{\pgfqpoint{4.028249in}{2.495857in}}%
\pgfpathlineto{\pgfqpoint{4.029102in}{2.487720in}}%
\pgfpathlineto{\pgfqpoint{4.029197in}{2.486954in}}%
\pgfpathlineto{\pgfqpoint{4.029386in}{2.492676in}}%
\pgfpathlineto{\pgfqpoint{4.029576in}{2.501275in}}%
\pgfpathlineto{\pgfqpoint{4.030144in}{2.474123in}}%
\pgfpathlineto{\pgfqpoint{4.030713in}{2.455224in}}%
\pgfpathlineto{\pgfqpoint{4.031282in}{2.471930in}}%
\pgfpathlineto{\pgfqpoint{4.031945in}{2.493676in}}%
\pgfpathlineto{\pgfqpoint{4.032703in}{2.509218in}}%
\pgfpathlineto{\pgfqpoint{4.032987in}{2.497708in}}%
\pgfpathlineto{\pgfqpoint{4.033461in}{2.479935in}}%
\pgfpathlineto{\pgfqpoint{4.034030in}{2.497069in}}%
\pgfpathlineto{\pgfqpoint{4.035641in}{2.532815in}}%
\pgfpathlineto{\pgfqpoint{4.035925in}{2.513777in}}%
\pgfpathlineto{\pgfqpoint{4.037062in}{2.493308in}}%
\pgfpathlineto{\pgfqpoint{4.037157in}{2.495324in}}%
\pgfpathlineto{\pgfqpoint{4.037631in}{2.523886in}}%
\pgfpathlineto{\pgfqpoint{4.038484in}{2.509071in}}%
\pgfpathlineto{\pgfqpoint{4.039431in}{2.515413in}}%
\pgfpathlineto{\pgfqpoint{4.039052in}{2.502724in}}%
\pgfpathlineto{\pgfqpoint{4.039526in}{2.511427in}}%
\pgfpathlineto{\pgfqpoint{4.039905in}{2.477342in}}%
\pgfpathlineto{\pgfqpoint{4.040663in}{2.502834in}}%
\pgfpathlineto{\pgfqpoint{4.042179in}{2.485300in}}%
\pgfpathlineto{\pgfqpoint{4.042369in}{2.487169in}}%
\pgfpathlineto{\pgfqpoint{4.043506in}{2.500310in}}%
\pgfpathlineto{\pgfqpoint{4.042937in}{2.483570in}}%
\pgfpathlineto{\pgfqpoint{4.043696in}{2.496289in}}%
\pgfpathlineto{\pgfqpoint{4.044359in}{2.486496in}}%
\pgfpathlineto{\pgfqpoint{4.044738in}{2.492665in}}%
\pgfpathlineto{\pgfqpoint{4.045022in}{2.497821in}}%
\pgfpathlineto{\pgfqpoint{4.045780in}{2.494382in}}%
\pgfpathlineto{\pgfqpoint{4.046633in}{2.483199in}}%
\pgfpathlineto{\pgfqpoint{4.046918in}{2.491764in}}%
\pgfpathlineto{\pgfqpoint{4.047107in}{2.496659in}}%
\pgfpathlineto{\pgfqpoint{4.047581in}{2.476000in}}%
\pgfpathlineto{\pgfqpoint{4.047865in}{2.483572in}}%
\pgfpathlineto{\pgfqpoint{4.047960in}{2.485454in}}%
\pgfpathlineto{\pgfqpoint{4.048244in}{2.472889in}}%
\pgfpathlineto{\pgfqpoint{4.049761in}{2.448742in}}%
\pgfpathlineto{\pgfqpoint{4.048718in}{2.479342in}}%
\pgfpathlineto{\pgfqpoint{4.049855in}{2.450522in}}%
\pgfpathlineto{\pgfqpoint{4.050045in}{2.456231in}}%
\pgfpathlineto{\pgfqpoint{4.050424in}{2.446290in}}%
\pgfpathlineto{\pgfqpoint{4.050898in}{2.453015in}}%
\pgfpathlineto{\pgfqpoint{4.052035in}{2.434270in}}%
\pgfpathlineto{\pgfqpoint{4.052224in}{2.440437in}}%
\pgfpathlineto{\pgfqpoint{4.052982in}{2.454506in}}%
\pgfpathlineto{\pgfqpoint{4.053267in}{2.441407in}}%
\pgfpathlineto{\pgfqpoint{4.053362in}{2.440154in}}%
\pgfpathlineto{\pgfqpoint{4.053646in}{2.450966in}}%
\pgfpathlineto{\pgfqpoint{4.053930in}{2.438294in}}%
\pgfpathlineto{\pgfqpoint{4.054878in}{2.431702in}}%
\pgfpathlineto{\pgfqpoint{4.054404in}{2.452052in}}%
\pgfpathlineto{\pgfqpoint{4.055067in}{2.434455in}}%
\pgfpathlineto{\pgfqpoint{4.057531in}{2.487681in}}%
\pgfpathlineto{\pgfqpoint{4.058100in}{2.473709in}}%
\pgfpathlineto{\pgfqpoint{4.058384in}{2.485564in}}%
\pgfpathlineto{\pgfqpoint{4.059426in}{2.509388in}}%
\pgfpathlineto{\pgfqpoint{4.059616in}{2.502375in}}%
\pgfpathlineto{\pgfqpoint{4.061796in}{2.436514in}}%
\pgfpathlineto{\pgfqpoint{4.061890in}{2.438357in}}%
\pgfpathlineto{\pgfqpoint{4.062364in}{2.462975in}}%
\pgfpathlineto{\pgfqpoint{4.063122in}{2.450481in}}%
\pgfpathlineto{\pgfqpoint{4.064070in}{2.441759in}}%
\pgfpathlineto{\pgfqpoint{4.063596in}{2.452171in}}%
\pgfpathlineto{\pgfqpoint{4.064354in}{2.445847in}}%
\pgfpathlineto{\pgfqpoint{4.065302in}{2.468716in}}%
\pgfpathlineto{\pgfqpoint{4.065586in}{2.485744in}}%
\pgfpathlineto{\pgfqpoint{4.066155in}{2.454886in}}%
\pgfpathlineto{\pgfqpoint{4.067197in}{2.436953in}}%
\pgfpathlineto{\pgfqpoint{4.067387in}{2.439204in}}%
\pgfpathlineto{\pgfqpoint{4.068808in}{2.499613in}}%
\pgfpathlineto{\pgfqpoint{4.069850in}{2.683911in}}%
\pgfpathlineto{\pgfqpoint{4.070609in}{2.658588in}}%
\pgfpathlineto{\pgfqpoint{4.070988in}{2.629146in}}%
\pgfpathlineto{\pgfqpoint{4.071746in}{2.387956in}}%
\pgfpathlineto{\pgfqpoint{4.072883in}{2.458821in}}%
\pgfpathlineto{\pgfqpoint{4.073072in}{2.452805in}}%
\pgfpathlineto{\pgfqpoint{4.073452in}{2.462762in}}%
\pgfpathlineto{\pgfqpoint{4.073925in}{2.459286in}}%
\pgfpathlineto{\pgfqpoint{4.074968in}{2.476694in}}%
\pgfpathlineto{\pgfqpoint{4.075157in}{2.468018in}}%
\pgfpathlineto{\pgfqpoint{4.076200in}{2.417342in}}%
\pgfpathlineto{\pgfqpoint{4.076674in}{2.432411in}}%
\pgfpathlineto{\pgfqpoint{4.076958in}{2.429503in}}%
\pgfpathlineto{\pgfqpoint{4.077242in}{2.435732in}}%
\pgfpathlineto{\pgfqpoint{4.077905in}{2.449422in}}%
\pgfpathlineto{\pgfqpoint{4.078190in}{2.440781in}}%
\pgfpathlineto{\pgfqpoint{4.078948in}{2.408610in}}%
\pgfpathlineto{\pgfqpoint{4.079422in}{2.430951in}}%
\pgfpathlineto{\pgfqpoint{4.079611in}{2.429780in}}%
\pgfpathlineto{\pgfqpoint{4.079706in}{2.429710in}}%
\pgfpathlineto{\pgfqpoint{4.081033in}{2.446816in}}%
\pgfpathlineto{\pgfqpoint{4.081222in}{2.442042in}}%
\pgfpathlineto{\pgfqpoint{4.081506in}{2.430602in}}%
\pgfpathlineto{\pgfqpoint{4.081886in}{2.445492in}}%
\pgfpathlineto{\pgfqpoint{4.082265in}{2.440290in}}%
\pgfpathlineto{\pgfqpoint{4.083117in}{2.459851in}}%
\pgfpathlineto{\pgfqpoint{4.084160in}{2.455546in}}%
\pgfpathlineto{\pgfqpoint{4.085108in}{2.434780in}}%
\pgfpathlineto{\pgfqpoint{4.085392in}{2.448302in}}%
\pgfpathlineto{\pgfqpoint{4.086339in}{2.473295in}}%
\pgfpathlineto{\pgfqpoint{4.086624in}{2.460959in}}%
\pgfpathlineto{\pgfqpoint{4.087950in}{2.444126in}}%
\pgfpathlineto{\pgfqpoint{4.088045in}{2.444471in}}%
\pgfpathlineto{\pgfqpoint{4.088993in}{2.452203in}}%
\pgfpathlineto{\pgfqpoint{4.089182in}{2.450048in}}%
\pgfpathlineto{\pgfqpoint{4.091457in}{2.377146in}}%
\pgfpathlineto{\pgfqpoint{4.091646in}{2.384098in}}%
\pgfpathlineto{\pgfqpoint{4.093352in}{2.451158in}}%
\pgfpathlineto{\pgfqpoint{4.094015in}{2.435726in}}%
\pgfpathlineto{\pgfqpoint{4.094110in}{2.437421in}}%
\pgfpathlineto{\pgfqpoint{4.094773in}{2.433751in}}%
\pgfpathlineto{\pgfqpoint{4.096479in}{2.395985in}}%
\pgfpathlineto{\pgfqpoint{4.096764in}{2.403700in}}%
\pgfpathlineto{\pgfqpoint{4.097616in}{2.408271in}}%
\pgfpathlineto{\pgfqpoint{4.097237in}{2.397211in}}%
\pgfpathlineto{\pgfqpoint{4.097806in}{2.403997in}}%
\pgfpathlineto{\pgfqpoint{4.097901in}{2.403354in}}%
\pgfpathlineto{\pgfqpoint{4.097995in}{2.406536in}}%
\pgfpathlineto{\pgfqpoint{4.098280in}{2.418580in}}%
\pgfpathlineto{\pgfqpoint{4.098564in}{2.403579in}}%
\pgfpathlineto{\pgfqpoint{4.099133in}{2.411947in}}%
\pgfpathlineto{\pgfqpoint{4.099985in}{2.402838in}}%
\pgfpathlineto{\pgfqpoint{4.100270in}{2.409720in}}%
\pgfpathlineto{\pgfqpoint{4.101691in}{2.438370in}}%
\pgfpathlineto{\pgfqpoint{4.102165in}{2.458087in}}%
\pgfpathlineto{\pgfqpoint{4.103018in}{2.454746in}}%
\pgfpathlineto{\pgfqpoint{4.103207in}{2.453222in}}%
\pgfpathlineto{\pgfqpoint{4.103397in}{2.456925in}}%
\pgfpathlineto{\pgfqpoint{4.104534in}{2.505490in}}%
\pgfpathlineto{\pgfqpoint{4.105292in}{2.500541in}}%
\pgfpathlineto{\pgfqpoint{4.105482in}{2.503309in}}%
\pgfpathlineto{\pgfqpoint{4.105766in}{2.491572in}}%
\pgfpathlineto{\pgfqpoint{4.106335in}{2.457213in}}%
\pgfpathlineto{\pgfqpoint{4.106998in}{2.479261in}}%
\pgfpathlineto{\pgfqpoint{4.107188in}{2.476425in}}%
\pgfpathlineto{\pgfqpoint{4.107472in}{2.490665in}}%
\pgfpathlineto{\pgfqpoint{4.107756in}{2.496300in}}%
\pgfpathlineto{\pgfqpoint{4.108230in}{2.481793in}}%
\pgfpathlineto{\pgfqpoint{4.109651in}{2.458521in}}%
\pgfpathlineto{\pgfqpoint{4.110599in}{2.494615in}}%
\pgfpathlineto{\pgfqpoint{4.111073in}{2.467507in}}%
\pgfpathlineto{\pgfqpoint{4.113252in}{2.377461in}}%
\pgfpathlineto{\pgfqpoint{4.113442in}{2.381182in}}%
\pgfpathlineto{\pgfqpoint{4.114295in}{2.509771in}}%
\pgfpathlineto{\pgfqpoint{4.114769in}{2.582593in}}%
\pgfpathlineto{\pgfqpoint{4.115527in}{2.557894in}}%
\pgfpathlineto{\pgfqpoint{4.116095in}{2.501988in}}%
\pgfpathlineto{\pgfqpoint{4.116948in}{2.244257in}}%
\pgfpathlineto{\pgfqpoint{4.117991in}{2.319612in}}%
\pgfpathlineto{\pgfqpoint{4.118180in}{2.315518in}}%
\pgfpathlineto{\pgfqpoint{4.118559in}{2.303034in}}%
\pgfpathlineto{\pgfqpoint{4.119128in}{2.315886in}}%
\pgfpathlineto{\pgfqpoint{4.119981in}{2.340306in}}%
\pgfpathlineto{\pgfqpoint{4.120360in}{2.321615in}}%
\pgfpathlineto{\pgfqpoint{4.121023in}{2.294939in}}%
\pgfpathlineto{\pgfqpoint{4.121686in}{2.309437in}}%
\pgfpathlineto{\pgfqpoint{4.121781in}{2.307718in}}%
\pgfpathlineto{\pgfqpoint{4.122066in}{2.320580in}}%
\pgfpathlineto{\pgfqpoint{4.122918in}{2.342111in}}%
\pgfpathlineto{\pgfqpoint{4.123392in}{2.334660in}}%
\pgfpathlineto{\pgfqpoint{4.123866in}{2.326778in}}%
\pgfpathlineto{\pgfqpoint{4.125856in}{2.383803in}}%
\pgfpathlineto{\pgfqpoint{4.126235in}{2.398373in}}%
\pgfpathlineto{\pgfqpoint{4.126804in}{2.384003in}}%
\pgfpathlineto{\pgfqpoint{4.127183in}{2.369673in}}%
\pgfpathlineto{\pgfqpoint{4.127657in}{2.393481in}}%
\pgfpathlineto{\pgfqpoint{4.128889in}{2.423449in}}%
\pgfpathlineto{\pgfqpoint{4.129741in}{2.409045in}}%
\pgfpathlineto{\pgfqpoint{4.130026in}{2.402195in}}%
\pgfpathlineto{\pgfqpoint{4.130594in}{2.416368in}}%
\pgfpathlineto{\pgfqpoint{4.132111in}{2.439188in}}%
\pgfpathlineto{\pgfqpoint{4.132300in}{2.437807in}}%
\pgfpathlineto{\pgfqpoint{4.132490in}{2.435770in}}%
\pgfpathlineto{\pgfqpoint{4.132869in}{2.441214in}}%
\pgfpathlineto{\pgfqpoint{4.133153in}{2.439473in}}%
\pgfpathlineto{\pgfqpoint{4.134101in}{2.472278in}}%
\pgfpathlineto{\pgfqpoint{4.134669in}{2.471511in}}%
\pgfpathlineto{\pgfqpoint{4.135617in}{2.497358in}}%
\pgfpathlineto{\pgfqpoint{4.137228in}{2.527534in}}%
\pgfpathlineto{\pgfqpoint{4.137323in}{2.526076in}}%
\pgfpathlineto{\pgfqpoint{4.137512in}{2.523060in}}%
\pgfpathlineto{\pgfqpoint{4.138081in}{2.529357in}}%
\pgfpathlineto{\pgfqpoint{4.139313in}{2.540080in}}%
\pgfpathlineto{\pgfqpoint{4.139502in}{2.537838in}}%
\pgfpathlineto{\pgfqpoint{4.140355in}{2.540720in}}%
\pgfpathlineto{\pgfqpoint{4.141587in}{2.510367in}}%
\pgfpathlineto{\pgfqpoint{4.141871in}{2.505989in}}%
\pgfpathlineto{\pgfqpoint{4.142345in}{2.481483in}}%
\pgfpathlineto{\pgfqpoint{4.143198in}{2.490572in}}%
\pgfpathlineto{\pgfqpoint{4.143482in}{2.493473in}}%
\pgfpathlineto{\pgfqpoint{4.143672in}{2.491315in}}%
\pgfpathlineto{\pgfqpoint{4.146515in}{2.421221in}}%
\pgfpathlineto{\pgfqpoint{4.146609in}{2.421402in}}%
\pgfpathlineto{\pgfqpoint{4.147557in}{2.430874in}}%
\pgfpathlineto{\pgfqpoint{4.147178in}{2.414993in}}%
\pgfpathlineto{\pgfqpoint{4.147747in}{2.422517in}}%
\pgfpathlineto{\pgfqpoint{4.148694in}{2.376132in}}%
\pgfpathlineto{\pgfqpoint{4.149073in}{2.398578in}}%
\pgfpathlineto{\pgfqpoint{4.153906in}{2.249997in}}%
\pgfpathlineto{\pgfqpoint{4.154191in}{2.260851in}}%
\pgfpathlineto{\pgfqpoint{4.154285in}{2.261890in}}%
\pgfpathlineto{\pgfqpoint{4.154380in}{2.258512in}}%
\pgfpathlineto{\pgfqpoint{4.154759in}{2.233125in}}%
\pgfpathlineto{\pgfqpoint{4.155328in}{2.268960in}}%
\pgfpathlineto{\pgfqpoint{4.155896in}{2.285590in}}%
\pgfpathlineto{\pgfqpoint{4.156275in}{2.266466in}}%
\pgfpathlineto{\pgfqpoint{4.156654in}{2.241054in}}%
\pgfpathlineto{\pgfqpoint{4.157507in}{2.246145in}}%
\pgfpathlineto{\pgfqpoint{4.157886in}{2.237573in}}%
\pgfpathlineto{\pgfqpoint{4.158171in}{2.245718in}}%
\pgfpathlineto{\pgfqpoint{4.159308in}{2.349802in}}%
\pgfpathlineto{\pgfqpoint{4.160255in}{2.518763in}}%
\pgfpathlineto{\pgfqpoint{4.161014in}{2.510273in}}%
\pgfpathlineto{\pgfqpoint{4.161582in}{2.374079in}}%
\pgfpathlineto{\pgfqpoint{4.162056in}{2.251560in}}%
\pgfpathlineto{\pgfqpoint{4.162814in}{2.306588in}}%
\pgfpathlineto{\pgfqpoint{4.163004in}{2.306403in}}%
\pgfpathlineto{\pgfqpoint{4.163098in}{2.306990in}}%
\pgfpathlineto{\pgfqpoint{4.168026in}{2.412984in}}%
\pgfpathlineto{\pgfqpoint{4.168121in}{2.412642in}}%
\pgfpathlineto{\pgfqpoint{4.168595in}{2.389887in}}%
\pgfpathlineto{\pgfqpoint{4.169163in}{2.356321in}}%
\pgfpathlineto{\pgfqpoint{4.170016in}{2.367526in}}%
\pgfpathlineto{\pgfqpoint{4.170585in}{2.389988in}}%
\pgfpathlineto{\pgfqpoint{4.172954in}{2.510358in}}%
\pgfpathlineto{\pgfqpoint{4.174091in}{2.545356in}}%
\pgfpathlineto{\pgfqpoint{4.174565in}{2.539467in}}%
\pgfpathlineto{\pgfqpoint{4.175418in}{2.534333in}}%
\pgfpathlineto{\pgfqpoint{4.175607in}{2.539168in}}%
\pgfpathlineto{\pgfqpoint{4.176365in}{2.572370in}}%
\pgfpathlineto{\pgfqpoint{4.176744in}{2.547155in}}%
\pgfpathlineto{\pgfqpoint{4.178261in}{2.521622in}}%
\pgfpathlineto{\pgfqpoint{4.177123in}{2.551865in}}%
\pgfpathlineto{\pgfqpoint{4.178450in}{2.526360in}}%
\pgfpathlineto{\pgfqpoint{4.179208in}{2.537382in}}%
\pgfpathlineto{\pgfqpoint{4.179872in}{2.536875in}}%
\pgfpathlineto{\pgfqpoint{4.180345in}{2.544173in}}%
\pgfpathlineto{\pgfqpoint{4.180535in}{2.536057in}}%
\pgfpathlineto{\pgfqpoint{4.181672in}{2.508442in}}%
\pgfpathlineto{\pgfqpoint{4.181198in}{2.537982in}}%
\pgfpathlineto{\pgfqpoint{4.181862in}{2.513573in}}%
\pgfpathlineto{\pgfqpoint{4.182051in}{2.520293in}}%
\pgfpathlineto{\pgfqpoint{4.182620in}{2.500405in}}%
\pgfpathlineto{\pgfqpoint{4.182999in}{2.482399in}}%
\pgfpathlineto{\pgfqpoint{4.185084in}{2.401716in}}%
\pgfpathlineto{\pgfqpoint{4.187453in}{2.327765in}}%
\pgfpathlineto{\pgfqpoint{4.188969in}{2.302925in}}%
\pgfpathlineto{\pgfqpoint{4.190959in}{2.269837in}}%
\pgfpathlineto{\pgfqpoint{4.191907in}{2.277632in}}%
\pgfpathlineto{\pgfqpoint{4.193044in}{2.289369in}}%
\pgfpathlineto{\pgfqpoint{4.193233in}{2.283165in}}%
\pgfpathlineto{\pgfqpoint{4.193518in}{2.263776in}}%
\pgfpathlineto{\pgfqpoint{4.194181in}{2.292752in}}%
\pgfpathlineto{\pgfqpoint{4.194371in}{2.293476in}}%
\pgfpathlineto{\pgfqpoint{4.194939in}{2.318396in}}%
\pgfpathlineto{\pgfqpoint{4.195602in}{2.296901in}}%
\pgfpathlineto{\pgfqpoint{4.196455in}{2.272409in}}%
\pgfpathlineto{\pgfqpoint{4.197119in}{2.249411in}}%
\pgfpathlineto{\pgfqpoint{4.197498in}{2.268566in}}%
\pgfpathlineto{\pgfqpoint{4.197593in}{2.271692in}}%
\pgfpathlineto{\pgfqpoint{4.197972in}{2.258499in}}%
\pgfpathlineto{\pgfqpoint{4.198351in}{2.262515in}}%
\pgfpathlineto{\pgfqpoint{4.199204in}{2.244084in}}%
\pgfpathlineto{\pgfqpoint{4.199583in}{2.254357in}}%
\pgfpathlineto{\pgfqpoint{4.199677in}{2.254721in}}%
\pgfpathlineto{\pgfqpoint{4.199772in}{2.252223in}}%
\pgfpathlineto{\pgfqpoint{4.199962in}{2.245584in}}%
\pgfpathlineto{\pgfqpoint{4.200341in}{2.267700in}}%
\pgfpathlineto{\pgfqpoint{4.201288in}{2.316215in}}%
\pgfpathlineto{\pgfqpoint{4.201667in}{2.291871in}}%
\pgfpathlineto{\pgfqpoint{4.202331in}{2.285900in}}%
\pgfpathlineto{\pgfqpoint{4.202615in}{2.296219in}}%
\pgfpathlineto{\pgfqpoint{4.202710in}{2.299758in}}%
\pgfpathlineto{\pgfqpoint{4.203089in}{2.288924in}}%
\pgfpathlineto{\pgfqpoint{4.203563in}{2.292881in}}%
\pgfpathlineto{\pgfqpoint{4.203657in}{2.290493in}}%
\pgfpathlineto{\pgfqpoint{4.203942in}{2.306821in}}%
\pgfpathlineto{\pgfqpoint{4.205647in}{2.591414in}}%
\pgfpathlineto{\pgfqpoint{4.206500in}{2.532983in}}%
\pgfpathlineto{\pgfqpoint{4.207258in}{2.301414in}}%
\pgfpathlineto{\pgfqpoint{4.208206in}{2.391629in}}%
\pgfpathlineto{\pgfqpoint{4.208301in}{2.391476in}}%
\pgfpathlineto{\pgfqpoint{4.209722in}{2.460258in}}%
\pgfpathlineto{\pgfqpoint{4.210480in}{2.497879in}}%
\pgfpathlineto{\pgfqpoint{4.211049in}{2.472484in}}%
\pgfpathlineto{\pgfqpoint{4.211144in}{2.473260in}}%
\pgfpathlineto{\pgfqpoint{4.211333in}{2.466846in}}%
\pgfpathlineto{\pgfqpoint{4.211428in}{2.463317in}}%
\pgfpathlineto{\pgfqpoint{4.211807in}{2.488139in}}%
\pgfpathlineto{\pgfqpoint{4.213702in}{2.550252in}}%
\pgfpathlineto{\pgfqpoint{4.213892in}{2.543320in}}%
\pgfpathlineto{\pgfqpoint{4.214271in}{2.512740in}}%
\pgfpathlineto{\pgfqpoint{4.215029in}{2.535078in}}%
\pgfpathlineto{\pgfqpoint{4.216451in}{2.570255in}}%
\pgfpathlineto{\pgfqpoint{4.216545in}{2.572377in}}%
\pgfpathlineto{\pgfqpoint{4.216924in}{2.556637in}}%
\pgfpathlineto{\pgfqpoint{4.217398in}{2.545658in}}%
\pgfpathlineto{\pgfqpoint{4.217588in}{2.540836in}}%
\pgfpathlineto{\pgfqpoint{4.218062in}{2.561712in}}%
\pgfpathlineto{\pgfqpoint{4.218251in}{2.555565in}}%
\pgfpathlineto{\pgfqpoint{4.219199in}{2.548961in}}%
\pgfpathlineto{\pgfqpoint{4.218725in}{2.563167in}}%
\pgfpathlineto{\pgfqpoint{4.219388in}{2.553469in}}%
\pgfpathlineto{\pgfqpoint{4.219483in}{2.553813in}}%
\pgfpathlineto{\pgfqpoint{4.219578in}{2.550139in}}%
\pgfpathlineto{\pgfqpoint{4.220810in}{2.497056in}}%
\pgfpathlineto{\pgfqpoint{4.221284in}{2.505417in}}%
\pgfpathlineto{\pgfqpoint{4.224411in}{2.420272in}}%
\pgfpathlineto{\pgfqpoint{4.224506in}{2.420026in}}%
\pgfpathlineto{\pgfqpoint{4.224695in}{2.422055in}}%
\pgfpathlineto{\pgfqpoint{4.224790in}{2.422519in}}%
\pgfpathlineto{\pgfqpoint{4.224885in}{2.421004in}}%
\pgfpathlineto{\pgfqpoint{4.225453in}{2.388251in}}%
\pgfpathlineto{\pgfqpoint{4.226496in}{2.393242in}}%
\pgfpathlineto{\pgfqpoint{4.226590in}{2.393820in}}%
\pgfpathlineto{\pgfqpoint{4.226780in}{2.389832in}}%
\pgfpathlineto{\pgfqpoint{4.229054in}{2.338912in}}%
\pgfpathlineto{\pgfqpoint{4.229338in}{2.341756in}}%
\pgfpathlineto{\pgfqpoint{4.229623in}{2.334959in}}%
\pgfpathlineto{\pgfqpoint{4.229907in}{2.337309in}}%
\pgfpathlineto{\pgfqpoint{4.232655in}{2.289889in}}%
\pgfpathlineto{\pgfqpoint{4.233034in}{2.296444in}}%
\pgfpathlineto{\pgfqpoint{4.234171in}{2.307906in}}%
\pgfpathlineto{\pgfqpoint{4.233603in}{2.289643in}}%
\pgfpathlineto{\pgfqpoint{4.234361in}{2.301961in}}%
\pgfpathlineto{\pgfqpoint{4.235309in}{2.285191in}}%
\pgfpathlineto{\pgfqpoint{4.235688in}{2.292403in}}%
\pgfpathlineto{\pgfqpoint{4.235972in}{2.286710in}}%
\pgfpathlineto{\pgfqpoint{4.236162in}{2.291888in}}%
\pgfpathlineto{\pgfqpoint{4.237867in}{2.337231in}}%
\pgfpathlineto{\pgfqpoint{4.238246in}{2.325027in}}%
\pgfpathlineto{\pgfqpoint{4.238815in}{2.338493in}}%
\pgfpathlineto{\pgfqpoint{4.239668in}{2.374691in}}%
\pgfpathlineto{\pgfqpoint{4.240331in}{2.369241in}}%
\pgfpathlineto{\pgfqpoint{4.240426in}{2.370793in}}%
\pgfpathlineto{\pgfqpoint{4.240710in}{2.358733in}}%
\pgfpathlineto{\pgfqpoint{4.241847in}{2.317131in}}%
\pgfpathlineto{\pgfqpoint{4.242321in}{2.325836in}}%
\pgfpathlineto{\pgfqpoint{4.243364in}{2.328525in}}%
\pgfpathlineto{\pgfqpoint{4.242890in}{2.320879in}}%
\pgfpathlineto{\pgfqpoint{4.243458in}{2.326865in}}%
\pgfpathlineto{\pgfqpoint{4.243932in}{2.315702in}}%
\pgfpathlineto{\pgfqpoint{4.244785in}{2.319047in}}%
\pgfpathlineto{\pgfqpoint{4.245259in}{2.339890in}}%
\pgfpathlineto{\pgfqpoint{4.246017in}{2.366146in}}%
\pgfpathlineto{\pgfqpoint{4.246396in}{2.353543in}}%
\pgfpathlineto{\pgfqpoint{4.247438in}{2.341814in}}%
\pgfpathlineto{\pgfqpoint{4.247059in}{2.356113in}}%
\pgfpathlineto{\pgfqpoint{4.247628in}{2.345067in}}%
\pgfpathlineto{\pgfqpoint{4.248291in}{2.356523in}}%
\pgfpathlineto{\pgfqpoint{4.248576in}{2.344904in}}%
\pgfpathlineto{\pgfqpoint{4.248670in}{2.342270in}}%
\pgfpathlineto{\pgfqpoint{4.248860in}{2.354204in}}%
\pgfpathlineto{\pgfqpoint{4.250376in}{2.592906in}}%
\pgfpathlineto{\pgfqpoint{4.251324in}{2.556909in}}%
\pgfpathlineto{\pgfqpoint{4.251987in}{2.362088in}}%
\pgfpathlineto{\pgfqpoint{4.252840in}{2.451934in}}%
\pgfpathlineto{\pgfqpoint{4.253030in}{2.457181in}}%
\pgfpathlineto{\pgfqpoint{4.253409in}{2.446398in}}%
\pgfpathlineto{\pgfqpoint{4.253882in}{2.449275in}}%
\pgfpathlineto{\pgfqpoint{4.254072in}{2.444484in}}%
\pgfpathlineto{\pgfqpoint{4.254546in}{2.453243in}}%
\pgfpathlineto{\pgfqpoint{4.254830in}{2.453244in}}%
\pgfpathlineto{\pgfqpoint{4.255114in}{2.466007in}}%
\pgfpathlineto{\pgfqpoint{4.255399in}{2.443176in}}%
\pgfpathlineto{\pgfqpoint{4.256536in}{2.377521in}}%
\pgfpathlineto{\pgfqpoint{4.256915in}{2.379582in}}%
\pgfpathlineto{\pgfqpoint{4.259758in}{2.299915in}}%
\pgfpathlineto{\pgfqpoint{4.260705in}{2.310050in}}%
\pgfpathlineto{\pgfqpoint{4.261179in}{2.327315in}}%
\pgfpathlineto{\pgfqpoint{4.261748in}{2.306124in}}%
\pgfpathlineto{\pgfqpoint{4.262032in}{2.298516in}}%
\pgfpathlineto{\pgfqpoint{4.262695in}{2.306466in}}%
\pgfpathlineto{\pgfqpoint{4.263075in}{2.327056in}}%
\pgfpathlineto{\pgfqpoint{4.264022in}{2.320994in}}%
\pgfpathlineto{\pgfqpoint{4.264401in}{2.330790in}}%
\pgfpathlineto{\pgfqpoint{4.264780in}{2.317757in}}%
\pgfpathlineto{\pgfqpoint{4.265444in}{2.305295in}}%
\pgfpathlineto{\pgfqpoint{4.265633in}{2.313064in}}%
\pgfpathlineto{\pgfqpoint{4.266581in}{2.344625in}}%
\pgfpathlineto{\pgfqpoint{4.266960in}{2.332666in}}%
\pgfpathlineto{\pgfqpoint{4.267149in}{2.336782in}}%
\pgfpathlineto{\pgfqpoint{4.267434in}{2.346688in}}%
\pgfpathlineto{\pgfqpoint{4.268287in}{2.338477in}}%
\pgfpathlineto{\pgfqpoint{4.268381in}{2.337277in}}%
\pgfpathlineto{\pgfqpoint{4.268666in}{2.345461in}}%
\pgfpathlineto{\pgfqpoint{4.269518in}{2.361919in}}%
\pgfpathlineto{\pgfqpoint{4.269045in}{2.342284in}}%
\pgfpathlineto{\pgfqpoint{4.270277in}{2.356539in}}%
\pgfpathlineto{\pgfqpoint{4.270371in}{2.356851in}}%
\pgfpathlineto{\pgfqpoint{4.270561in}{2.354823in}}%
\pgfpathlineto{\pgfqpoint{4.271698in}{2.337034in}}%
\pgfpathlineto{\pgfqpoint{4.271982in}{2.343894in}}%
\pgfpathlineto{\pgfqpoint{4.272172in}{2.347149in}}%
\pgfpathlineto{\pgfqpoint{4.272456in}{2.330538in}}%
\pgfpathlineto{\pgfqpoint{4.273878in}{2.309510in}}%
\pgfpathlineto{\pgfqpoint{4.275678in}{2.289504in}}%
\pgfpathlineto{\pgfqpoint{4.275962in}{2.293888in}}%
\pgfpathlineto{\pgfqpoint{4.277100in}{2.300315in}}%
\pgfpathlineto{\pgfqpoint{4.277194in}{2.299735in}}%
\pgfpathlineto{\pgfqpoint{4.277384in}{2.298307in}}%
\pgfpathlineto{\pgfqpoint{4.277668in}{2.304536in}}%
\pgfpathlineto{\pgfqpoint{4.278332in}{2.318620in}}%
\pgfpathlineto{\pgfqpoint{4.278711in}{2.305544in}}%
\pgfpathlineto{\pgfqpoint{4.278805in}{2.304699in}}%
\pgfpathlineto{\pgfqpoint{4.278995in}{2.310077in}}%
\pgfpathlineto{\pgfqpoint{4.279563in}{2.322551in}}%
\pgfpathlineto{\pgfqpoint{4.279943in}{2.307231in}}%
\pgfpathlineto{\pgfqpoint{4.280037in}{2.305196in}}%
\pgfpathlineto{\pgfqpoint{4.280322in}{2.319745in}}%
\pgfpathlineto{\pgfqpoint{4.282122in}{2.365258in}}%
\pgfpathlineto{\pgfqpoint{4.282312in}{2.363691in}}%
\pgfpathlineto{\pgfqpoint{4.282406in}{2.363183in}}%
\pgfpathlineto{\pgfqpoint{4.282596in}{2.367874in}}%
\pgfpathlineto{\pgfqpoint{4.284586in}{2.413001in}}%
\pgfpathlineto{\pgfqpoint{4.283259in}{2.366311in}}%
\pgfpathlineto{\pgfqpoint{4.284681in}{2.412293in}}%
\pgfpathlineto{\pgfqpoint{4.286766in}{2.346655in}}%
\pgfpathlineto{\pgfqpoint{4.287239in}{2.360803in}}%
\pgfpathlineto{\pgfqpoint{4.287618in}{2.386977in}}%
\pgfpathlineto{\pgfqpoint{4.288282in}{2.357267in}}%
\pgfpathlineto{\pgfqpoint{4.288377in}{2.357334in}}%
\pgfpathlineto{\pgfqpoint{4.288566in}{2.356484in}}%
\pgfpathlineto{\pgfqpoint{4.289798in}{2.344826in}}%
\pgfpathlineto{\pgfqpoint{4.289893in}{2.347347in}}%
\pgfpathlineto{\pgfqpoint{4.290935in}{2.388378in}}%
\pgfpathlineto{\pgfqpoint{4.291219in}{2.370340in}}%
\pgfpathlineto{\pgfqpoint{4.291788in}{2.356336in}}%
\pgfpathlineto{\pgfqpoint{4.292451in}{2.360934in}}%
\pgfpathlineto{\pgfqpoint{4.292546in}{2.358467in}}%
\pgfpathlineto{\pgfqpoint{4.292925in}{2.366017in}}%
\pgfpathlineto{\pgfqpoint{4.293304in}{2.363477in}}%
\pgfpathlineto{\pgfqpoint{4.293873in}{2.409969in}}%
\pgfpathlineto{\pgfqpoint{4.295010in}{2.627664in}}%
\pgfpathlineto{\pgfqpoint{4.295768in}{2.600312in}}%
\pgfpathlineto{\pgfqpoint{4.296242in}{2.501916in}}%
\pgfpathlineto{\pgfqpoint{4.296905in}{2.322078in}}%
\pgfpathlineto{\pgfqpoint{4.297663in}{2.382953in}}%
\pgfpathlineto{\pgfqpoint{4.299938in}{2.438812in}}%
\pgfpathlineto{\pgfqpoint{4.298232in}{2.381089in}}%
\pgfpathlineto{\pgfqpoint{4.300222in}{2.433946in}}%
\pgfpathlineto{\pgfqpoint{4.300601in}{2.412901in}}%
\pgfpathlineto{\pgfqpoint{4.300885in}{2.401065in}}%
\pgfpathlineto{\pgfqpoint{4.301549in}{2.420579in}}%
\pgfpathlineto{\pgfqpoint{4.302970in}{2.437816in}}%
\pgfpathlineto{\pgfqpoint{4.303255in}{2.442847in}}%
\pgfpathlineto{\pgfqpoint{4.303539in}{2.427603in}}%
\pgfpathlineto{\pgfqpoint{4.304297in}{2.411085in}}%
\pgfpathlineto{\pgfqpoint{4.304581in}{2.430302in}}%
\pgfpathlineto{\pgfqpoint{4.305813in}{2.450163in}}%
\pgfpathlineto{\pgfqpoint{4.307140in}{2.414137in}}%
\pgfpathlineto{\pgfqpoint{4.307519in}{2.433315in}}%
\pgfpathlineto{\pgfqpoint{4.308751in}{2.455530in}}%
\pgfpathlineto{\pgfqpoint{4.308940in}{2.452458in}}%
\pgfpathlineto{\pgfqpoint{4.310362in}{2.432396in}}%
\pgfpathlineto{\pgfqpoint{4.310551in}{2.440421in}}%
\pgfpathlineto{\pgfqpoint{4.312068in}{2.465230in}}%
\pgfpathlineto{\pgfqpoint{4.312257in}{2.468585in}}%
\pgfpathlineto{\pgfqpoint{4.312636in}{2.454026in}}%
\pgfpathlineto{\pgfqpoint{4.312731in}{2.450316in}}%
\pgfpathlineto{\pgfqpoint{4.313110in}{2.467275in}}%
\pgfpathlineto{\pgfqpoint{4.313489in}{2.459619in}}%
\pgfpathlineto{\pgfqpoint{4.314437in}{2.481529in}}%
\pgfpathlineto{\pgfqpoint{4.315195in}{2.480510in}}%
\pgfpathlineto{\pgfqpoint{4.316142in}{2.492779in}}%
\pgfpathlineto{\pgfqpoint{4.316332in}{2.487302in}}%
\pgfpathlineto{\pgfqpoint{4.317753in}{2.463930in}}%
\pgfpathlineto{\pgfqpoint{4.319554in}{2.438831in}}%
\pgfpathlineto{\pgfqpoint{4.320691in}{2.429171in}}%
\pgfpathlineto{\pgfqpoint{4.320312in}{2.441772in}}%
\pgfpathlineto{\pgfqpoint{4.320786in}{2.429356in}}%
\pgfpathlineto{\pgfqpoint{4.321070in}{2.433868in}}%
\pgfpathlineto{\pgfqpoint{4.321354in}{2.423499in}}%
\pgfpathlineto{\pgfqpoint{4.322302in}{2.402626in}}%
\pgfpathlineto{\pgfqpoint{4.322492in}{2.412134in}}%
\pgfpathlineto{\pgfqpoint{4.323439in}{2.439580in}}%
\pgfpathlineto{\pgfqpoint{4.323724in}{2.429533in}}%
\pgfpathlineto{\pgfqpoint{4.324482in}{2.415697in}}%
\pgfpathlineto{\pgfqpoint{4.324008in}{2.430220in}}%
\pgfpathlineto{\pgfqpoint{4.325335in}{2.423831in}}%
\pgfpathlineto{\pgfqpoint{4.326851in}{2.441166in}}%
\pgfpathlineto{\pgfqpoint{4.326946in}{2.441000in}}%
\pgfpathlineto{\pgfqpoint{4.327704in}{2.420135in}}%
\pgfpathlineto{\pgfqpoint{4.328083in}{2.400035in}}%
\pgfpathlineto{\pgfqpoint{4.328936in}{2.406590in}}%
\pgfpathlineto{\pgfqpoint{4.330452in}{2.458932in}}%
\pgfpathlineto{\pgfqpoint{4.331020in}{2.436796in}}%
\pgfpathlineto{\pgfqpoint{4.331115in}{2.434322in}}%
\pgfpathlineto{\pgfqpoint{4.331494in}{2.443480in}}%
\pgfpathlineto{\pgfqpoint{4.331968in}{2.439421in}}%
\pgfpathlineto{\pgfqpoint{4.332063in}{2.439482in}}%
\pgfpathlineto{\pgfqpoint{4.332252in}{2.438525in}}%
\pgfpathlineto{\pgfqpoint{4.332537in}{2.438889in}}%
\pgfpathlineto{\pgfqpoint{4.332726in}{2.433454in}}%
\pgfpathlineto{\pgfqpoint{4.334242in}{2.402582in}}%
\pgfpathlineto{\pgfqpoint{4.334337in}{2.402829in}}%
\pgfpathlineto{\pgfqpoint{4.335095in}{2.427693in}}%
\pgfpathlineto{\pgfqpoint{4.335474in}{2.445874in}}%
\pgfpathlineto{\pgfqpoint{4.336138in}{2.430187in}}%
\pgfpathlineto{\pgfqpoint{4.337559in}{2.406681in}}%
\pgfpathlineto{\pgfqpoint{4.337654in}{2.406555in}}%
\pgfpathlineto{\pgfqpoint{4.338696in}{2.463110in}}%
\pgfpathlineto{\pgfqpoint{4.340023in}{2.682835in}}%
\pgfpathlineto{\pgfqpoint{4.340876in}{2.638987in}}%
\pgfpathlineto{\pgfqpoint{4.341823in}{2.382141in}}%
\pgfpathlineto{\pgfqpoint{4.342676in}{2.451067in}}%
\pgfpathlineto{\pgfqpoint{4.344477in}{2.507581in}}%
\pgfpathlineto{\pgfqpoint{4.344572in}{2.504076in}}%
\pgfpathlineto{\pgfqpoint{4.345614in}{2.462902in}}%
\pgfpathlineto{\pgfqpoint{4.346277in}{2.471537in}}%
\pgfpathlineto{\pgfqpoint{4.346467in}{2.470371in}}%
\pgfpathlineto{\pgfqpoint{4.346751in}{2.476535in}}%
\pgfpathlineto{\pgfqpoint{4.347699in}{2.506313in}}%
\pgfpathlineto{\pgfqpoint{4.348173in}{2.491406in}}%
\pgfpathlineto{\pgfqpoint{4.349026in}{2.470273in}}%
\pgfpathlineto{\pgfqpoint{4.349405in}{2.481834in}}%
\pgfpathlineto{\pgfqpoint{4.349594in}{2.481636in}}%
\pgfpathlineto{\pgfqpoint{4.349784in}{2.484806in}}%
\pgfpathlineto{\pgfqpoint{4.350731in}{2.502000in}}%
\pgfpathlineto{\pgfqpoint{4.350352in}{2.480155in}}%
\pgfpathlineto{\pgfqpoint{4.351016in}{2.491508in}}%
\pgfpathlineto{\pgfqpoint{4.352248in}{2.475433in}}%
\pgfpathlineto{\pgfqpoint{4.352437in}{2.482474in}}%
\pgfpathlineto{\pgfqpoint{4.353290in}{2.505126in}}%
\pgfpathlineto{\pgfqpoint{4.353859in}{2.502335in}}%
\pgfpathlineto{\pgfqpoint{4.355090in}{2.484431in}}%
\pgfpathlineto{\pgfqpoint{4.355375in}{2.490032in}}%
\pgfpathlineto{\pgfqpoint{4.356322in}{2.512774in}}%
\pgfpathlineto{\pgfqpoint{4.356607in}{2.505103in}}%
\pgfpathlineto{\pgfqpoint{4.356701in}{2.502989in}}%
\pgfpathlineto{\pgfqpoint{4.357081in}{2.513364in}}%
\pgfpathlineto{\pgfqpoint{4.357460in}{2.509559in}}%
\pgfpathlineto{\pgfqpoint{4.358028in}{2.496661in}}%
\pgfpathlineto{\pgfqpoint{4.358692in}{2.506446in}}%
\pgfpathlineto{\pgfqpoint{4.360302in}{2.518451in}}%
\pgfpathlineto{\pgfqpoint{4.360397in}{2.518476in}}%
\pgfpathlineto{\pgfqpoint{4.362198in}{2.490915in}}%
\pgfpathlineto{\pgfqpoint{4.363430in}{2.492692in}}%
\pgfpathlineto{\pgfqpoint{4.363619in}{2.490612in}}%
\pgfpathlineto{\pgfqpoint{4.365515in}{2.436659in}}%
\pgfpathlineto{\pgfqpoint{4.365704in}{2.444565in}}%
\pgfpathlineto{\pgfqpoint{4.365988in}{2.461969in}}%
\pgfpathlineto{\pgfqpoint{4.366652in}{2.439104in}}%
\pgfpathlineto{\pgfqpoint{4.367694in}{2.432260in}}%
\pgfpathlineto{\pgfqpoint{4.367315in}{2.447211in}}%
\pgfpathlineto{\pgfqpoint{4.367789in}{2.433076in}}%
\pgfpathlineto{\pgfqpoint{4.368073in}{2.440595in}}%
\pgfpathlineto{\pgfqpoint{4.368452in}{2.428101in}}%
\pgfpathlineto{\pgfqpoint{4.368926in}{2.438320in}}%
\pgfpathlineto{\pgfqpoint{4.369874in}{2.431116in}}%
\pgfpathlineto{\pgfqpoint{4.369400in}{2.441683in}}%
\pgfpathlineto{\pgfqpoint{4.370063in}{2.434186in}}%
\pgfpathlineto{\pgfqpoint{4.370442in}{2.445514in}}%
\pgfpathlineto{\pgfqpoint{4.371106in}{2.431934in}}%
\pgfpathlineto{\pgfqpoint{4.371485in}{2.424965in}}%
\pgfpathlineto{\pgfqpoint{4.371769in}{2.437922in}}%
\pgfpathlineto{\pgfqpoint{4.373190in}{2.461538in}}%
\pgfpathlineto{\pgfqpoint{4.373380in}{2.458366in}}%
\pgfpathlineto{\pgfqpoint{4.374043in}{2.456119in}}%
\pgfpathlineto{\pgfqpoint{4.374233in}{2.459941in}}%
\pgfpathlineto{\pgfqpoint{4.375844in}{2.485102in}}%
\pgfpathlineto{\pgfqpoint{4.375939in}{2.483788in}}%
\pgfpathlineto{\pgfqpoint{4.377644in}{2.401045in}}%
\pgfpathlineto{\pgfqpoint{4.377929in}{2.415536in}}%
\pgfpathlineto{\pgfqpoint{4.378592in}{2.432599in}}%
\pgfpathlineto{\pgfqpoint{4.379066in}{2.416697in}}%
\pgfpathlineto{\pgfqpoint{4.380108in}{2.389880in}}%
\pgfpathlineto{\pgfqpoint{4.380487in}{2.402185in}}%
\pgfpathlineto{\pgfqpoint{4.381624in}{2.448273in}}%
\pgfpathlineto{\pgfqpoint{4.382098in}{2.421847in}}%
\pgfpathlineto{\pgfqpoint{4.383235in}{2.411809in}}%
\pgfpathlineto{\pgfqpoint{4.383330in}{2.411865in}}%
\pgfpathlineto{\pgfqpoint{4.384657in}{2.441025in}}%
\pgfpathlineto{\pgfqpoint{4.386078in}{2.680986in}}%
\pgfpathlineto{\pgfqpoint{4.386836in}{2.639336in}}%
\pgfpathlineto{\pgfqpoint{4.387310in}{2.491180in}}%
\pgfpathlineto{\pgfqpoint{4.387784in}{2.372246in}}%
\pgfpathlineto{\pgfqpoint{4.388447in}{2.441890in}}%
\pgfpathlineto{\pgfqpoint{4.388637in}{2.443400in}}%
\pgfpathlineto{\pgfqpoint{4.388921in}{2.434599in}}%
\pgfpathlineto{\pgfqpoint{4.389016in}{2.430777in}}%
\pgfpathlineto{\pgfqpoint{4.389395in}{2.455596in}}%
\pgfpathlineto{\pgfqpoint{4.390911in}{2.485747in}}%
\pgfpathlineto{\pgfqpoint{4.391480in}{2.458003in}}%
\pgfpathlineto{\pgfqpoint{4.391764in}{2.436846in}}%
\pgfpathlineto{\pgfqpoint{4.392428in}{2.467423in}}%
\pgfpathlineto{\pgfqpoint{4.393849in}{2.502587in}}%
\pgfpathlineto{\pgfqpoint{4.393091in}{2.464324in}}%
\pgfpathlineto{\pgfqpoint{4.394228in}{2.487706in}}%
\pgfpathlineto{\pgfqpoint{4.394607in}{2.474599in}}%
\pgfpathlineto{\pgfqpoint{4.395365in}{2.478002in}}%
\pgfpathlineto{\pgfqpoint{4.396597in}{2.503171in}}%
\pgfpathlineto{\pgfqpoint{4.396881in}{2.500294in}}%
\pgfpathlineto{\pgfqpoint{4.398019in}{2.483449in}}%
\pgfpathlineto{\pgfqpoint{4.398303in}{2.489694in}}%
\pgfpathlineto{\pgfqpoint{4.399251in}{2.525904in}}%
\pgfpathlineto{\pgfqpoint{4.400009in}{2.518817in}}%
\pgfpathlineto{\pgfqpoint{4.401335in}{2.499493in}}%
\pgfpathlineto{\pgfqpoint{4.401525in}{2.504921in}}%
\pgfpathlineto{\pgfqpoint{4.401904in}{2.521284in}}%
\pgfpathlineto{\pgfqpoint{4.402662in}{2.514589in}}%
\pgfpathlineto{\pgfqpoint{4.403610in}{2.502927in}}%
\pgfpathlineto{\pgfqpoint{4.403231in}{2.515346in}}%
\pgfpathlineto{\pgfqpoint{4.403894in}{2.508779in}}%
\pgfpathlineto{\pgfqpoint{4.404557in}{2.511539in}}%
\pgfpathlineto{\pgfqpoint{4.404273in}{2.506135in}}%
\pgfpathlineto{\pgfqpoint{4.404652in}{2.510742in}}%
\pgfpathlineto{\pgfqpoint{4.407685in}{2.427063in}}%
\pgfpathlineto{\pgfqpoint{4.407779in}{2.430083in}}%
\pgfpathlineto{\pgfqpoint{4.409011in}{2.501188in}}%
\pgfpathlineto{\pgfqpoint{4.409390in}{2.494530in}}%
\pgfpathlineto{\pgfqpoint{4.409675in}{2.497339in}}%
\pgfpathlineto{\pgfqpoint{4.410148in}{2.493240in}}%
\pgfpathlineto{\pgfqpoint{4.410338in}{2.493736in}}%
\pgfpathlineto{\pgfqpoint{4.410812in}{2.479042in}}%
\pgfpathlineto{\pgfqpoint{4.412897in}{2.447560in}}%
\pgfpathlineto{\pgfqpoint{4.412991in}{2.448810in}}%
\pgfpathlineto{\pgfqpoint{4.413276in}{2.456899in}}%
\pgfpathlineto{\pgfqpoint{4.414034in}{2.450796in}}%
\pgfpathlineto{\pgfqpoint{4.414318in}{2.436418in}}%
\pgfpathlineto{\pgfqpoint{4.415171in}{2.441551in}}%
\pgfpathlineto{\pgfqpoint{4.415360in}{2.444492in}}%
\pgfpathlineto{\pgfqpoint{4.415740in}{2.436541in}}%
\pgfpathlineto{\pgfqpoint{4.416213in}{2.442736in}}%
\pgfpathlineto{\pgfqpoint{4.417350in}{2.425035in}}%
\pgfpathlineto{\pgfqpoint{4.416782in}{2.446563in}}%
\pgfpathlineto{\pgfqpoint{4.417635in}{2.436895in}}%
\pgfpathlineto{\pgfqpoint{4.419246in}{2.464007in}}%
\pgfpathlineto{\pgfqpoint{4.419341in}{2.465418in}}%
\pgfpathlineto{\pgfqpoint{4.419625in}{2.456623in}}%
\pgfpathlineto{\pgfqpoint{4.419814in}{2.453810in}}%
\pgfpathlineto{\pgfqpoint{4.420193in}{2.468537in}}%
\pgfpathlineto{\pgfqpoint{4.420572in}{2.465608in}}%
\pgfpathlineto{\pgfqpoint{4.420857in}{2.478686in}}%
\pgfpathlineto{\pgfqpoint{4.421141in}{2.494877in}}%
\pgfpathlineto{\pgfqpoint{4.421899in}{2.481873in}}%
\pgfpathlineto{\pgfqpoint{4.423226in}{2.410313in}}%
\pgfpathlineto{\pgfqpoint{4.423889in}{2.433080in}}%
\pgfpathlineto{\pgfqpoint{4.424363in}{2.445387in}}%
\pgfpathlineto{\pgfqpoint{4.424647in}{2.431381in}}%
\pgfpathlineto{\pgfqpoint{4.425784in}{2.409172in}}%
\pgfpathlineto{\pgfqpoint{4.425974in}{2.411771in}}%
\pgfpathlineto{\pgfqpoint{4.426353in}{2.409699in}}%
\pgfpathlineto{\pgfqpoint{4.426543in}{2.415060in}}%
\pgfpathlineto{\pgfqpoint{4.427490in}{2.448445in}}%
\pgfpathlineto{\pgfqpoint{4.427869in}{2.433613in}}%
\pgfpathlineto{\pgfqpoint{4.429291in}{2.390203in}}%
\pgfpathlineto{\pgfqpoint{4.429386in}{2.391058in}}%
\pgfpathlineto{\pgfqpoint{4.430049in}{2.390731in}}%
\pgfpathlineto{\pgfqpoint{4.430712in}{2.476761in}}%
\pgfpathlineto{\pgfqpoint{4.431849in}{2.672117in}}%
\pgfpathlineto{\pgfqpoint{4.432513in}{2.630097in}}%
\pgfpathlineto{\pgfqpoint{4.433081in}{2.492171in}}%
\pgfpathlineto{\pgfqpoint{4.433650in}{2.350139in}}%
\pgfpathlineto{\pgfqpoint{4.434313in}{2.423765in}}%
\pgfpathlineto{\pgfqpoint{4.436682in}{2.479518in}}%
\pgfpathlineto{\pgfqpoint{4.436872in}{2.476058in}}%
\pgfpathlineto{\pgfqpoint{4.437535in}{2.437317in}}%
\pgfpathlineto{\pgfqpoint{4.438483in}{2.462984in}}%
\pgfpathlineto{\pgfqpoint{4.439715in}{2.491111in}}%
\pgfpathlineto{\pgfqpoint{4.440094in}{2.477645in}}%
\pgfpathlineto{\pgfqpoint{4.441042in}{2.454381in}}%
\pgfpathlineto{\pgfqpoint{4.441421in}{2.464439in}}%
\pgfpathlineto{\pgfqpoint{4.443126in}{2.508095in}}%
\pgfpathlineto{\pgfqpoint{4.443600in}{2.493598in}}%
\pgfpathlineto{\pgfqpoint{4.443884in}{2.486267in}}%
\pgfpathlineto{\pgfqpoint{4.444358in}{2.506110in}}%
\pgfpathlineto{\pgfqpoint{4.444832in}{2.530914in}}%
\pgfpathlineto{\pgfqpoint{4.445685in}{2.522014in}}%
\pgfpathlineto{\pgfqpoint{4.446064in}{2.527302in}}%
\pgfpathlineto{\pgfqpoint{4.446727in}{2.524024in}}%
\pgfpathlineto{\pgfqpoint{4.447106in}{2.516950in}}%
\pgfpathlineto{\pgfqpoint{4.447485in}{2.526367in}}%
\pgfpathlineto{\pgfqpoint{4.447770in}{2.525311in}}%
\pgfpathlineto{\pgfqpoint{4.448149in}{2.534600in}}%
\pgfpathlineto{\pgfqpoint{4.448717in}{2.522765in}}%
\pgfpathlineto{\pgfqpoint{4.448812in}{2.522436in}}%
\pgfpathlineto{\pgfqpoint{4.449096in}{2.524703in}}%
\pgfpathlineto{\pgfqpoint{4.449286in}{2.526248in}}%
\pgfpathlineto{\pgfqpoint{4.449665in}{2.516780in}}%
\pgfpathlineto{\pgfqpoint{4.449760in}{2.516694in}}%
\pgfpathlineto{\pgfqpoint{4.450802in}{2.529596in}}%
\pgfpathlineto{\pgfqpoint{4.451371in}{2.526453in}}%
\pgfpathlineto{\pgfqpoint{4.451466in}{2.526930in}}%
\pgfpathlineto{\pgfqpoint{4.451655in}{2.522354in}}%
\pgfpathlineto{\pgfqpoint{4.451845in}{2.519119in}}%
\pgfpathlineto{\pgfqpoint{4.452318in}{2.532939in}}%
\pgfpathlineto{\pgfqpoint{4.452603in}{2.526696in}}%
\pgfpathlineto{\pgfqpoint{4.452792in}{2.528922in}}%
\pgfpathlineto{\pgfqpoint{4.452982in}{2.532254in}}%
\pgfpathlineto{\pgfqpoint{4.453361in}{2.519460in}}%
\pgfpathlineto{\pgfqpoint{4.454498in}{2.508156in}}%
\pgfpathlineto{\pgfqpoint{4.454024in}{2.522559in}}%
\pgfpathlineto{\pgfqpoint{4.454688in}{2.509953in}}%
\pgfpathlineto{\pgfqpoint{4.454782in}{2.511287in}}%
\pgfpathlineto{\pgfqpoint{4.455067in}{2.502920in}}%
\pgfpathlineto{\pgfqpoint{4.456678in}{2.469504in}}%
\pgfpathlineto{\pgfqpoint{4.457057in}{2.478099in}}%
\pgfpathlineto{\pgfqpoint{4.457625in}{2.466429in}}%
\pgfpathlineto{\pgfqpoint{4.457910in}{2.469747in}}%
\pgfpathlineto{\pgfqpoint{4.458194in}{2.459685in}}%
\pgfpathlineto{\pgfqpoint{4.459141in}{2.444534in}}%
\pgfpathlineto{\pgfqpoint{4.458762in}{2.468943in}}%
\pgfpathlineto{\pgfqpoint{4.459426in}{2.454018in}}%
\pgfpathlineto{\pgfqpoint{4.459521in}{2.455863in}}%
\pgfpathlineto{\pgfqpoint{4.459900in}{2.443526in}}%
\pgfpathlineto{\pgfqpoint{4.460184in}{2.447171in}}%
\pgfpathlineto{\pgfqpoint{4.460373in}{2.443687in}}%
\pgfpathlineto{\pgfqpoint{4.460658in}{2.454738in}}%
\pgfpathlineto{\pgfqpoint{4.460847in}{2.464693in}}%
\pgfpathlineto{\pgfqpoint{4.461605in}{2.444644in}}%
\pgfpathlineto{\pgfqpoint{4.462743in}{2.430667in}}%
\pgfpathlineto{\pgfqpoint{4.462269in}{2.453923in}}%
\pgfpathlineto{\pgfqpoint{4.462932in}{2.434321in}}%
\pgfpathlineto{\pgfqpoint{4.464069in}{2.443690in}}%
\pgfpathlineto{\pgfqpoint{4.464259in}{2.441871in}}%
\pgfpathlineto{\pgfqpoint{4.464543in}{2.437072in}}%
\pgfpathlineto{\pgfqpoint{4.464827in}{2.447736in}}%
\pgfpathlineto{\pgfqpoint{4.465775in}{2.459448in}}%
\pgfpathlineto{\pgfqpoint{4.465964in}{2.449524in}}%
\pgfpathlineto{\pgfqpoint{4.466154in}{2.444648in}}%
\pgfpathlineto{\pgfqpoint{4.466628in}{2.467457in}}%
\pgfpathlineto{\pgfqpoint{4.467102in}{2.501184in}}%
\pgfpathlineto{\pgfqpoint{4.468049in}{2.485004in}}%
\pgfpathlineto{\pgfqpoint{4.469471in}{2.416653in}}%
\pgfpathlineto{\pgfqpoint{4.469850in}{2.439332in}}%
\pgfpathlineto{\pgfqpoint{4.470039in}{2.448971in}}%
\pgfpathlineto{\pgfqpoint{4.470797in}{2.435286in}}%
\pgfpathlineto{\pgfqpoint{4.472314in}{2.402996in}}%
\pgfpathlineto{\pgfqpoint{4.472408in}{2.404805in}}%
\pgfpathlineto{\pgfqpoint{4.473451in}{2.453447in}}%
\pgfpathlineto{\pgfqpoint{4.474019in}{2.434990in}}%
\pgfpathlineto{\pgfqpoint{4.475346in}{2.420468in}}%
\pgfpathlineto{\pgfqpoint{4.475441in}{2.422777in}}%
\pgfpathlineto{\pgfqpoint{4.476768in}{2.504404in}}%
\pgfpathlineto{\pgfqpoint{4.478000in}{2.695008in}}%
\pgfpathlineto{\pgfqpoint{4.478379in}{2.665075in}}%
\pgfpathlineto{\pgfqpoint{4.479326in}{2.443886in}}%
\pgfpathlineto{\pgfqpoint{4.479705in}{2.376716in}}%
\pgfpathlineto{\pgfqpoint{4.480369in}{2.436178in}}%
\pgfpathlineto{\pgfqpoint{4.481885in}{2.463165in}}%
\pgfpathlineto{\pgfqpoint{4.482643in}{2.494582in}}%
\pgfpathlineto{\pgfqpoint{4.483117in}{2.468168in}}%
\pgfpathlineto{\pgfqpoint{4.483780in}{2.436644in}}%
\pgfpathlineto{\pgfqpoint{4.484538in}{2.447117in}}%
\pgfpathlineto{\pgfqpoint{4.484633in}{2.447626in}}%
\pgfpathlineto{\pgfqpoint{4.484823in}{2.443763in}}%
\pgfpathlineto{\pgfqpoint{4.486813in}{2.384237in}}%
\pgfpathlineto{\pgfqpoint{4.486907in}{2.386962in}}%
\pgfpathlineto{\pgfqpoint{4.488613in}{2.511003in}}%
\pgfpathlineto{\pgfqpoint{4.489371in}{2.502524in}}%
\pgfpathlineto{\pgfqpoint{4.489845in}{2.476468in}}%
\pgfpathlineto{\pgfqpoint{4.490698in}{2.493449in}}%
\pgfpathlineto{\pgfqpoint{4.492025in}{2.510847in}}%
\pgfpathlineto{\pgfqpoint{4.492214in}{2.505297in}}%
\pgfpathlineto{\pgfqpoint{4.492593in}{2.489567in}}%
\pgfpathlineto{\pgfqpoint{4.493351in}{2.501290in}}%
\pgfpathlineto{\pgfqpoint{4.493636in}{2.505760in}}%
\pgfpathlineto{\pgfqpoint{4.493920in}{2.516196in}}%
\pgfpathlineto{\pgfqpoint{4.494488in}{2.500511in}}%
\pgfpathlineto{\pgfqpoint{4.494678in}{2.503750in}}%
\pgfpathlineto{\pgfqpoint{4.494962in}{2.506147in}}%
\pgfpathlineto{\pgfqpoint{4.495531in}{2.500337in}}%
\pgfpathlineto{\pgfqpoint{4.495720in}{2.498367in}}%
\pgfpathlineto{\pgfqpoint{4.496194in}{2.503023in}}%
\pgfpathlineto{\pgfqpoint{4.496479in}{2.500722in}}%
\pgfpathlineto{\pgfqpoint{4.498563in}{2.530196in}}%
\pgfpathlineto{\pgfqpoint{4.498753in}{2.523127in}}%
\pgfpathlineto{\pgfqpoint{4.499701in}{2.502386in}}%
\pgfpathlineto{\pgfqpoint{4.499890in}{2.513115in}}%
\pgfpathlineto{\pgfqpoint{4.499985in}{2.516081in}}%
\pgfpathlineto{\pgfqpoint{4.500364in}{2.493438in}}%
\pgfpathlineto{\pgfqpoint{4.501596in}{2.481855in}}%
\pgfpathlineto{\pgfqpoint{4.500743in}{2.495624in}}%
\pgfpathlineto{\pgfqpoint{4.501785in}{2.487380in}}%
\pgfpathlineto{\pgfqpoint{4.501975in}{2.492886in}}%
\pgfpathlineto{\pgfqpoint{4.502543in}{2.477252in}}%
\pgfpathlineto{\pgfqpoint{4.507471in}{2.425438in}}%
\pgfpathlineto{\pgfqpoint{4.507566in}{2.427305in}}%
\pgfpathlineto{\pgfqpoint{4.507850in}{2.438164in}}%
\pgfpathlineto{\pgfqpoint{4.508324in}{2.416532in}}%
\pgfpathlineto{\pgfqpoint{4.508798in}{2.404341in}}%
\pgfpathlineto{\pgfqpoint{4.509461in}{2.410851in}}%
\pgfpathlineto{\pgfqpoint{4.511072in}{2.436638in}}%
\pgfpathlineto{\pgfqpoint{4.511356in}{2.425732in}}%
\pgfpathlineto{\pgfqpoint{4.511451in}{2.423607in}}%
\pgfpathlineto{\pgfqpoint{4.512209in}{2.429096in}}%
\pgfpathlineto{\pgfqpoint{4.512778in}{2.441712in}}%
\pgfpathlineto{\pgfqpoint{4.513536in}{2.450431in}}%
\pgfpathlineto{\pgfqpoint{4.513820in}{2.441384in}}%
\pgfpathlineto{\pgfqpoint{4.515621in}{2.374511in}}%
\pgfpathlineto{\pgfqpoint{4.516189in}{2.393225in}}%
\pgfpathlineto{\pgfqpoint{4.518464in}{2.353522in}}%
\pgfpathlineto{\pgfqpoint{4.518843in}{2.364311in}}%
\pgfpathlineto{\pgfqpoint{4.519411in}{2.391541in}}%
\pgfpathlineto{\pgfqpoint{4.520075in}{2.378817in}}%
\pgfpathlineto{\pgfqpoint{4.521307in}{2.363709in}}%
\pgfpathlineto{\pgfqpoint{4.521401in}{2.364380in}}%
\pgfpathlineto{\pgfqpoint{4.522539in}{2.400481in}}%
\pgfpathlineto{\pgfqpoint{4.523960in}{2.661967in}}%
\pgfpathlineto{\pgfqpoint{4.524813in}{2.601451in}}%
\pgfpathlineto{\pgfqpoint{4.525761in}{2.340229in}}%
\pgfpathlineto{\pgfqpoint{4.526803in}{2.402643in}}%
\pgfpathlineto{\pgfqpoint{4.527372in}{2.393253in}}%
\pgfpathlineto{\pgfqpoint{4.528888in}{2.440038in}}%
\pgfpathlineto{\pgfqpoint{4.528983in}{2.440277in}}%
\pgfpathlineto{\pgfqpoint{4.529077in}{2.438758in}}%
\pgfpathlineto{\pgfqpoint{4.529741in}{2.404414in}}%
\pgfpathlineto{\pgfqpoint{4.530309in}{2.432655in}}%
\pgfpathlineto{\pgfqpoint{4.531731in}{2.449688in}}%
\pgfpathlineto{\pgfqpoint{4.530878in}{2.429470in}}%
\pgfpathlineto{\pgfqpoint{4.532015in}{2.443557in}}%
\pgfpathlineto{\pgfqpoint{4.532584in}{2.413092in}}%
\pgfpathlineto{\pgfqpoint{4.533342in}{2.424511in}}%
\pgfpathlineto{\pgfqpoint{4.534858in}{2.463934in}}%
\pgfpathlineto{\pgfqpoint{4.535048in}{2.456204in}}%
\pgfpathlineto{\pgfqpoint{4.535900in}{2.434610in}}%
\pgfpathlineto{\pgfqpoint{4.536279in}{2.447482in}}%
\pgfpathlineto{\pgfqpoint{4.536469in}{2.446846in}}%
\pgfpathlineto{\pgfqpoint{4.536564in}{2.448098in}}%
\pgfpathlineto{\pgfqpoint{4.537038in}{2.469166in}}%
\pgfpathlineto{\pgfqpoint{4.537890in}{2.461201in}}%
\pgfpathlineto{\pgfqpoint{4.538933in}{2.444417in}}%
\pgfpathlineto{\pgfqpoint{4.539217in}{2.453688in}}%
\pgfpathlineto{\pgfqpoint{4.539975in}{2.464089in}}%
\pgfpathlineto{\pgfqpoint{4.540260in}{2.457031in}}%
\pgfpathlineto{\pgfqpoint{4.540354in}{2.454869in}}%
\pgfpathlineto{\pgfqpoint{4.540733in}{2.469131in}}%
\pgfpathlineto{\pgfqpoint{4.541018in}{2.475804in}}%
\pgfpathlineto{\pgfqpoint{4.541491in}{2.459228in}}%
\pgfpathlineto{\pgfqpoint{4.541871in}{2.450056in}}%
\pgfpathlineto{\pgfqpoint{4.542439in}{2.463555in}}%
\pgfpathlineto{\pgfqpoint{4.542534in}{2.463835in}}%
\pgfpathlineto{\pgfqpoint{4.542629in}{2.461571in}}%
\pgfpathlineto{\pgfqpoint{4.542913in}{2.453831in}}%
\pgfpathlineto{\pgfqpoint{4.543671in}{2.462070in}}%
\pgfpathlineto{\pgfqpoint{4.543766in}{2.464332in}}%
\pgfpathlineto{\pgfqpoint{4.544145in}{2.450941in}}%
\pgfpathlineto{\pgfqpoint{4.544713in}{2.446917in}}%
\pgfpathlineto{\pgfqpoint{4.544998in}{2.454184in}}%
\pgfpathlineto{\pgfqpoint{4.545093in}{2.455840in}}%
\pgfpathlineto{\pgfqpoint{4.545282in}{2.445435in}}%
\pgfpathlineto{\pgfqpoint{4.545472in}{2.432305in}}%
\pgfpathlineto{\pgfqpoint{4.545851in}{2.447796in}}%
\pgfpathlineto{\pgfqpoint{4.546514in}{2.434885in}}%
\pgfpathlineto{\pgfqpoint{4.547935in}{2.410065in}}%
\pgfpathlineto{\pgfqpoint{4.548220in}{2.419973in}}%
\pgfpathlineto{\pgfqpoint{4.548314in}{2.421308in}}%
\pgfpathlineto{\pgfqpoint{4.548599in}{2.410005in}}%
\pgfpathlineto{\pgfqpoint{4.550020in}{2.395032in}}%
\pgfpathlineto{\pgfqpoint{4.550399in}{2.407656in}}%
\pgfpathlineto{\pgfqpoint{4.550778in}{2.391393in}}%
\pgfpathlineto{\pgfqpoint{4.551252in}{2.393059in}}%
\pgfpathlineto{\pgfqpoint{4.552200in}{2.376224in}}%
\pgfpathlineto{\pgfqpoint{4.552389in}{2.379520in}}%
\pgfpathlineto{\pgfqpoint{4.553811in}{2.398049in}}%
\pgfpathlineto{\pgfqpoint{4.553906in}{2.397938in}}%
\pgfpathlineto{\pgfqpoint{4.555422in}{2.375150in}}%
\pgfpathlineto{\pgfqpoint{4.555801in}{2.380357in}}%
\pgfpathlineto{\pgfqpoint{4.557791in}{2.436054in}}%
\pgfpathlineto{\pgfqpoint{4.557886in}{2.432594in}}%
\pgfpathlineto{\pgfqpoint{4.558170in}{2.419199in}}%
\pgfpathlineto{\pgfqpoint{4.558644in}{2.447352in}}%
\pgfpathlineto{\pgfqpoint{4.558739in}{2.445698in}}%
\pgfpathlineto{\pgfqpoint{4.558928in}{2.440618in}}%
\pgfpathlineto{\pgfqpoint{4.559402in}{2.456900in}}%
\pgfpathlineto{\pgfqpoint{4.559781in}{2.447391in}}%
\pgfpathlineto{\pgfqpoint{4.560065in}{2.451413in}}%
\pgfpathlineto{\pgfqpoint{4.560255in}{2.443282in}}%
\pgfpathlineto{\pgfqpoint{4.561581in}{2.373806in}}%
\pgfpathlineto{\pgfqpoint{4.562245in}{2.377269in}}%
\pgfpathlineto{\pgfqpoint{4.562434in}{2.380905in}}%
\pgfpathlineto{\pgfqpoint{4.562719in}{2.366227in}}%
\pgfpathlineto{\pgfqpoint{4.563572in}{2.332539in}}%
\pgfpathlineto{\pgfqpoint{4.563856in}{2.352117in}}%
\pgfpathlineto{\pgfqpoint{4.565562in}{2.471764in}}%
\pgfpathlineto{\pgfqpoint{4.565941in}{2.450024in}}%
\pgfpathlineto{\pgfqpoint{4.567078in}{2.415906in}}%
\pgfpathlineto{\pgfqpoint{4.567362in}{2.422310in}}%
\pgfpathlineto{\pgfqpoint{4.567741in}{2.425639in}}%
\pgfpathlineto{\pgfqpoint{4.568215in}{2.423372in}}%
\pgfpathlineto{\pgfqpoint{4.568784in}{2.495691in}}%
\pgfpathlineto{\pgfqpoint{4.570015in}{2.708939in}}%
\pgfpathlineto{\pgfqpoint{4.570679in}{2.662916in}}%
\pgfpathlineto{\pgfqpoint{4.571058in}{2.578004in}}%
\pgfpathlineto{\pgfqpoint{4.571721in}{2.388277in}}%
\pgfpathlineto{\pgfqpoint{4.572385in}{2.463201in}}%
\pgfpathlineto{\pgfqpoint{4.572764in}{2.448115in}}%
\pgfpathlineto{\pgfqpoint{4.573427in}{2.463799in}}%
\pgfpathlineto{\pgfqpoint{4.574659in}{2.503793in}}%
\pgfpathlineto{\pgfqpoint{4.575227in}{2.488559in}}%
\pgfpathlineto{\pgfqpoint{4.575796in}{2.455026in}}%
\pgfpathlineto{\pgfqpoint{4.576554in}{2.476797in}}%
\pgfpathlineto{\pgfqpoint{4.576649in}{2.475409in}}%
\pgfpathlineto{\pgfqpoint{4.577123in}{2.483268in}}%
\pgfpathlineto{\pgfqpoint{4.577312in}{2.485757in}}%
\pgfpathlineto{\pgfqpoint{4.577691in}{2.499530in}}%
\pgfpathlineto{\pgfqpoint{4.578355in}{2.485538in}}%
\pgfpathlineto{\pgfqpoint{4.578829in}{2.455710in}}%
\pgfpathlineto{\pgfqpoint{4.579492in}{2.479976in}}%
\pgfpathlineto{\pgfqpoint{4.580060in}{2.491509in}}%
\pgfpathlineto{\pgfqpoint{4.580913in}{2.508085in}}%
\pgfpathlineto{\pgfqpoint{4.581292in}{2.497801in}}%
\pgfpathlineto{\pgfqpoint{4.581482in}{2.496727in}}%
\pgfpathlineto{\pgfqpoint{4.581956in}{2.471468in}}%
\pgfpathlineto{\pgfqpoint{4.582714in}{2.491199in}}%
\pgfpathlineto{\pgfqpoint{4.582998in}{2.498496in}}%
\pgfpathlineto{\pgfqpoint{4.583851in}{2.493624in}}%
\pgfpathlineto{\pgfqpoint{4.585083in}{2.465583in}}%
\pgfpathlineto{\pgfqpoint{4.585462in}{2.476057in}}%
\pgfpathlineto{\pgfqpoint{4.585841in}{2.459788in}}%
\pgfpathlineto{\pgfqpoint{4.586315in}{2.481260in}}%
\pgfpathlineto{\pgfqpoint{4.586789in}{2.469974in}}%
\pgfpathlineto{\pgfqpoint{4.587073in}{2.473879in}}%
\pgfpathlineto{\pgfqpoint{4.587452in}{2.465529in}}%
\pgfpathlineto{\pgfqpoint{4.588494in}{2.460202in}}%
\pgfpathlineto{\pgfqpoint{4.588684in}{2.461595in}}%
\pgfpathlineto{\pgfqpoint{4.588968in}{2.465473in}}%
\pgfpathlineto{\pgfqpoint{4.589253in}{2.454315in}}%
\pgfpathlineto{\pgfqpoint{4.590200in}{2.449545in}}%
\pgfpathlineto{\pgfqpoint{4.589821in}{2.467332in}}%
\pgfpathlineto{\pgfqpoint{4.590390in}{2.452881in}}%
\pgfpathlineto{\pgfqpoint{4.591337in}{2.459918in}}%
\pgfpathlineto{\pgfqpoint{4.590864in}{2.446225in}}%
\pgfpathlineto{\pgfqpoint{4.591622in}{2.454807in}}%
\pgfpathlineto{\pgfqpoint{4.591811in}{2.453507in}}%
\pgfpathlineto{\pgfqpoint{4.592285in}{2.458867in}}%
\pgfpathlineto{\pgfqpoint{4.592475in}{2.456132in}}%
\pgfpathlineto{\pgfqpoint{4.593801in}{2.423928in}}%
\pgfpathlineto{\pgfqpoint{4.594086in}{2.432761in}}%
\pgfpathlineto{\pgfqpoint{4.594180in}{2.434587in}}%
\pgfpathlineto{\pgfqpoint{4.594559in}{2.421500in}}%
\pgfpathlineto{\pgfqpoint{4.595697in}{2.403005in}}%
\pgfpathlineto{\pgfqpoint{4.595886in}{2.409430in}}%
\pgfpathlineto{\pgfqpoint{4.596076in}{2.417016in}}%
\pgfpathlineto{\pgfqpoint{4.596644in}{2.399305in}}%
\pgfpathlineto{\pgfqpoint{4.596928in}{2.406682in}}%
\pgfpathlineto{\pgfqpoint{4.597118in}{2.410451in}}%
\pgfpathlineto{\pgfqpoint{4.597497in}{2.392583in}}%
\pgfpathlineto{\pgfqpoint{4.598160in}{2.386175in}}%
\pgfpathlineto{\pgfqpoint{4.597781in}{2.393325in}}%
\pgfpathlineto{\pgfqpoint{4.598824in}{2.387996in}}%
\pgfpathlineto{\pgfqpoint{4.600435in}{2.415612in}}%
\pgfpathlineto{\pgfqpoint{4.600530in}{2.414965in}}%
\pgfpathlineto{\pgfqpoint{4.601477in}{2.398219in}}%
\pgfpathlineto{\pgfqpoint{4.601003in}{2.415796in}}%
\pgfpathlineto{\pgfqpoint{4.601761in}{2.409619in}}%
\pgfpathlineto{\pgfqpoint{4.602614in}{2.424089in}}%
\pgfpathlineto{\pgfqpoint{4.602899in}{2.413910in}}%
\pgfpathlineto{\pgfqpoint{4.602993in}{2.413609in}}%
\pgfpathlineto{\pgfqpoint{4.603467in}{2.437349in}}%
\pgfpathlineto{\pgfqpoint{4.604415in}{2.427691in}}%
\pgfpathlineto{\pgfqpoint{4.604510in}{2.427530in}}%
\pgfpathlineto{\pgfqpoint{4.605742in}{2.465681in}}%
\pgfpathlineto{\pgfqpoint{4.606215in}{2.446539in}}%
\pgfpathlineto{\pgfqpoint{4.607826in}{2.383620in}}%
\pgfpathlineto{\pgfqpoint{4.608205in}{2.404406in}}%
\pgfpathlineto{\pgfqpoint{4.608869in}{2.415443in}}%
\pgfpathlineto{\pgfqpoint{4.609153in}{2.402734in}}%
\pgfpathlineto{\pgfqpoint{4.610859in}{2.372531in}}%
\pgfpathlineto{\pgfqpoint{4.610954in}{2.373930in}}%
\pgfpathlineto{\pgfqpoint{4.611901in}{2.415152in}}%
\pgfpathlineto{\pgfqpoint{4.612375in}{2.391827in}}%
\pgfpathlineto{\pgfqpoint{4.613796in}{2.380230in}}%
\pgfpathlineto{\pgfqpoint{4.612754in}{2.395358in}}%
\pgfpathlineto{\pgfqpoint{4.613986in}{2.385020in}}%
\pgfpathlineto{\pgfqpoint{4.615407in}{2.518738in}}%
\pgfpathlineto{\pgfqpoint{4.616355in}{2.673642in}}%
\pgfpathlineto{\pgfqpoint{4.616829in}{2.642031in}}%
\pgfpathlineto{\pgfqpoint{4.617492in}{2.536250in}}%
\pgfpathlineto{\pgfqpoint{4.618061in}{2.372699in}}%
\pgfpathlineto{\pgfqpoint{4.618819in}{2.434603in}}%
\pgfpathlineto{\pgfqpoint{4.619009in}{2.432640in}}%
\pgfpathlineto{\pgfqpoint{4.619293in}{2.444874in}}%
\pgfpathlineto{\pgfqpoint{4.620051in}{2.455920in}}%
\pgfpathlineto{\pgfqpoint{4.619672in}{2.443544in}}%
\pgfpathlineto{\pgfqpoint{4.620240in}{2.448032in}}%
\pgfpathlineto{\pgfqpoint{4.620430in}{2.440562in}}%
\pgfpathlineto{\pgfqpoint{4.620809in}{2.481887in}}%
\pgfpathlineto{\pgfqpoint{4.620904in}{2.487952in}}%
\pgfpathlineto{\pgfqpoint{4.621662in}{2.468575in}}%
\pgfpathlineto{\pgfqpoint{4.622325in}{2.432178in}}%
\pgfpathlineto{\pgfqpoint{4.623178in}{2.446059in}}%
\pgfpathlineto{\pgfqpoint{4.624221in}{2.466725in}}%
\pgfpathlineto{\pgfqpoint{4.624600in}{2.453437in}}%
\pgfpathlineto{\pgfqpoint{4.625737in}{2.430934in}}%
\pgfpathlineto{\pgfqpoint{4.625926in}{2.436631in}}%
\pgfpathlineto{\pgfqpoint{4.627253in}{2.466630in}}%
\pgfpathlineto{\pgfqpoint{4.627348in}{2.463543in}}%
\pgfpathlineto{\pgfqpoint{4.628390in}{2.435002in}}%
\pgfpathlineto{\pgfqpoint{4.628580in}{2.445525in}}%
\pgfpathlineto{\pgfqpoint{4.629622in}{2.469266in}}%
\pgfpathlineto{\pgfqpoint{4.629906in}{2.467429in}}%
\pgfpathlineto{\pgfqpoint{4.630191in}{2.471498in}}%
\pgfpathlineto{\pgfqpoint{4.630380in}{2.476201in}}%
\pgfpathlineto{\pgfqpoint{4.630854in}{2.458249in}}%
\pgfpathlineto{\pgfqpoint{4.631612in}{2.449154in}}%
\pgfpathlineto{\pgfqpoint{4.631896in}{2.456931in}}%
\pgfpathlineto{\pgfqpoint{4.632465in}{2.468569in}}%
\pgfpathlineto{\pgfqpoint{4.632939in}{2.456151in}}%
\pgfpathlineto{\pgfqpoint{4.633413in}{2.452036in}}%
\pgfpathlineto{\pgfqpoint{4.633602in}{2.456470in}}%
\pgfpathlineto{\pgfqpoint{4.633792in}{2.462102in}}%
\pgfpathlineto{\pgfqpoint{4.634171in}{2.441605in}}%
\pgfpathlineto{\pgfqpoint{4.636161in}{2.380770in}}%
\pgfpathlineto{\pgfqpoint{4.636350in}{2.384670in}}%
\pgfpathlineto{\pgfqpoint{4.636824in}{2.375479in}}%
\pgfpathlineto{\pgfqpoint{4.637582in}{2.416364in}}%
\pgfpathlineto{\pgfqpoint{4.638625in}{2.467153in}}%
\pgfpathlineto{\pgfqpoint{4.639193in}{2.459934in}}%
\pgfpathlineto{\pgfqpoint{4.641562in}{2.421875in}}%
\pgfpathlineto{\pgfqpoint{4.639667in}{2.465970in}}%
\pgfpathlineto{\pgfqpoint{4.641657in}{2.422497in}}%
\pgfpathlineto{\pgfqpoint{4.642889in}{2.435896in}}%
\pgfpathlineto{\pgfqpoint{4.643079in}{2.434574in}}%
\pgfpathlineto{\pgfqpoint{4.644121in}{2.412865in}}%
\pgfpathlineto{\pgfqpoint{4.644879in}{2.420427in}}%
\pgfpathlineto{\pgfqpoint{4.646111in}{2.445652in}}%
\pgfpathlineto{\pgfqpoint{4.645448in}{2.419687in}}%
\pgfpathlineto{\pgfqpoint{4.646490in}{2.429357in}}%
\pgfpathlineto{\pgfqpoint{4.646585in}{2.427245in}}%
\pgfpathlineto{\pgfqpoint{4.647059in}{2.440777in}}%
\pgfpathlineto{\pgfqpoint{4.647153in}{2.441430in}}%
\pgfpathlineto{\pgfqpoint{4.647248in}{2.438871in}}%
\pgfpathlineto{\pgfqpoint{4.647722in}{2.409733in}}%
\pgfpathlineto{\pgfqpoint{4.648385in}{2.432407in}}%
\pgfpathlineto{\pgfqpoint{4.649238in}{2.437951in}}%
\pgfpathlineto{\pgfqpoint{4.648764in}{2.430037in}}%
\pgfpathlineto{\pgfqpoint{4.649902in}{2.435333in}}%
\pgfpathlineto{\pgfqpoint{4.650186in}{2.430566in}}%
\pgfpathlineto{\pgfqpoint{4.650565in}{2.445428in}}%
\pgfpathlineto{\pgfqpoint{4.650944in}{2.435104in}}%
\pgfpathlineto{\pgfqpoint{4.651986in}{2.469719in}}%
\pgfpathlineto{\pgfqpoint{4.652555in}{2.463604in}}%
\pgfpathlineto{\pgfqpoint{4.654166in}{2.383964in}}%
\pgfpathlineto{\pgfqpoint{4.655019in}{2.417711in}}%
\pgfpathlineto{\pgfqpoint{4.655114in}{2.422447in}}%
\pgfpathlineto{\pgfqpoint{4.655682in}{2.399725in}}%
\pgfpathlineto{\pgfqpoint{4.657009in}{2.382141in}}%
\pgfpathlineto{\pgfqpoint{4.657293in}{2.375008in}}%
\pgfpathlineto{\pgfqpoint{4.657767in}{2.395444in}}%
\pgfpathlineto{\pgfqpoint{4.658241in}{2.407268in}}%
\pgfpathlineto{\pgfqpoint{4.658715in}{2.394767in}}%
\pgfpathlineto{\pgfqpoint{4.659757in}{2.363484in}}%
\pgfpathlineto{\pgfqpoint{4.660041in}{2.368591in}}%
\pgfpathlineto{\pgfqpoint{4.660515in}{2.374907in}}%
\pgfpathlineto{\pgfqpoint{4.660894in}{2.365446in}}%
\pgfpathlineto{\pgfqpoint{4.660989in}{2.363521in}}%
\pgfpathlineto{\pgfqpoint{4.661179in}{2.372626in}}%
\pgfpathlineto{\pgfqpoint{4.662505in}{2.638249in}}%
\pgfpathlineto{\pgfqpoint{4.663548in}{2.577845in}}%
\pgfpathlineto{\pgfqpoint{4.664116in}{2.420997in}}%
\pgfpathlineto{\pgfqpoint{4.664495in}{2.336121in}}%
\pgfpathlineto{\pgfqpoint{4.665159in}{2.408537in}}%
\pgfpathlineto{\pgfqpoint{4.665727in}{2.406287in}}%
\pgfpathlineto{\pgfqpoint{4.666770in}{2.433361in}}%
\pgfpathlineto{\pgfqpoint{4.667054in}{2.437374in}}%
\pgfpathlineto{\pgfqpoint{4.667433in}{2.451008in}}%
\pgfpathlineto{\pgfqpoint{4.667812in}{2.426006in}}%
\pgfpathlineto{\pgfqpoint{4.668854in}{2.398066in}}%
\pgfpathlineto{\pgfqpoint{4.669139in}{2.411155in}}%
\pgfpathlineto{\pgfqpoint{4.670276in}{2.440007in}}%
\pgfpathlineto{\pgfqpoint{4.670560in}{2.436961in}}%
\pgfpathlineto{\pgfqpoint{4.670750in}{2.436244in}}%
\pgfpathlineto{\pgfqpoint{4.671887in}{2.410568in}}%
\pgfpathlineto{\pgfqpoint{4.672171in}{2.428555in}}%
\pgfpathlineto{\pgfqpoint{4.673593in}{2.451744in}}%
\pgfpathlineto{\pgfqpoint{4.673877in}{2.459568in}}%
\pgfpathlineto{\pgfqpoint{4.674540in}{2.446678in}}%
\pgfpathlineto{\pgfqpoint{4.674825in}{2.444528in}}%
\pgfpathlineto{\pgfqpoint{4.675109in}{2.447360in}}%
\pgfpathlineto{\pgfqpoint{4.675867in}{2.481829in}}%
\pgfpathlineto{\pgfqpoint{4.677194in}{2.480923in}}%
\pgfpathlineto{\pgfqpoint{4.677762in}{2.462543in}}%
\pgfpathlineto{\pgfqpoint{4.678141in}{2.477702in}}%
\pgfpathlineto{\pgfqpoint{4.678805in}{2.471044in}}%
\pgfpathlineto{\pgfqpoint{4.679373in}{2.487920in}}%
\pgfpathlineto{\pgfqpoint{4.679563in}{2.488274in}}%
\pgfpathlineto{\pgfqpoint{4.679752in}{2.487693in}}%
\pgfpathlineto{\pgfqpoint{4.681269in}{2.464745in}}%
\pgfpathlineto{\pgfqpoint{4.681648in}{2.474491in}}%
\pgfpathlineto{\pgfqpoint{4.682027in}{2.480339in}}%
\pgfpathlineto{\pgfqpoint{4.682690in}{2.475420in}}%
\pgfpathlineto{\pgfqpoint{4.683922in}{2.459843in}}%
\pgfpathlineto{\pgfqpoint{4.684111in}{2.463547in}}%
\pgfpathlineto{\pgfqpoint{4.684206in}{2.465240in}}%
\pgfpathlineto{\pgfqpoint{4.684775in}{2.455698in}}%
\pgfpathlineto{\pgfqpoint{4.685533in}{2.456679in}}%
\pgfpathlineto{\pgfqpoint{4.685912in}{2.445515in}}%
\pgfpathlineto{\pgfqpoint{4.686575in}{2.436652in}}%
\pgfpathlineto{\pgfqpoint{4.687333in}{2.436878in}}%
\pgfpathlineto{\pgfqpoint{4.687712in}{2.432166in}}%
\pgfpathlineto{\pgfqpoint{4.689039in}{2.412448in}}%
\pgfpathlineto{\pgfqpoint{4.688565in}{2.440546in}}%
\pgfpathlineto{\pgfqpoint{4.689134in}{2.414842in}}%
\pgfpathlineto{\pgfqpoint{4.689987in}{2.426439in}}%
\pgfpathlineto{\pgfqpoint{4.690271in}{2.416210in}}%
\pgfpathlineto{\pgfqpoint{4.691029in}{2.407394in}}%
\pgfpathlineto{\pgfqpoint{4.690650in}{2.417969in}}%
\pgfpathlineto{\pgfqpoint{4.691314in}{2.415780in}}%
\pgfpathlineto{\pgfqpoint{4.691408in}{2.417906in}}%
\pgfpathlineto{\pgfqpoint{4.691787in}{2.409290in}}%
\pgfpathlineto{\pgfqpoint{4.692261in}{2.414866in}}%
\pgfpathlineto{\pgfqpoint{4.693209in}{2.413209in}}%
\pgfpathlineto{\pgfqpoint{4.692735in}{2.421494in}}%
\pgfpathlineto{\pgfqpoint{4.693304in}{2.413748in}}%
\pgfpathlineto{\pgfqpoint{4.693588in}{2.419484in}}%
\pgfpathlineto{\pgfqpoint{4.693967in}{2.407381in}}%
\pgfpathlineto{\pgfqpoint{4.694156in}{2.402403in}}%
\pgfpathlineto{\pgfqpoint{4.694820in}{2.416813in}}%
\pgfpathlineto{\pgfqpoint{4.697094in}{2.439520in}}%
\pgfpathlineto{\pgfqpoint{4.697284in}{2.436977in}}%
\pgfpathlineto{\pgfqpoint{4.697378in}{2.436116in}}%
\pgfpathlineto{\pgfqpoint{4.697757in}{2.442399in}}%
\pgfpathlineto{\pgfqpoint{4.698610in}{2.470167in}}%
\pgfpathlineto{\pgfqpoint{4.699274in}{2.459241in}}%
\pgfpathlineto{\pgfqpoint{4.699558in}{2.446002in}}%
\pgfpathlineto{\pgfqpoint{4.700600in}{2.398583in}}%
\pgfpathlineto{\pgfqpoint{4.701074in}{2.409169in}}%
\pgfpathlineto{\pgfqpoint{4.701453in}{2.431156in}}%
\pgfpathlineto{\pgfqpoint{4.702117in}{2.407747in}}%
\pgfpathlineto{\pgfqpoint{4.703538in}{2.369340in}}%
\pgfpathlineto{\pgfqpoint{4.702780in}{2.409077in}}%
\pgfpathlineto{\pgfqpoint{4.703728in}{2.375263in}}%
\pgfpathlineto{\pgfqpoint{4.704581in}{2.413779in}}%
\pgfpathlineto{\pgfqpoint{4.705149in}{2.394563in}}%
\pgfpathlineto{\pgfqpoint{4.706286in}{2.376910in}}%
\pgfpathlineto{\pgfqpoint{4.706476in}{2.383689in}}%
\pgfpathlineto{\pgfqpoint{4.706665in}{2.389831in}}%
\pgfpathlineto{\pgfqpoint{4.707139in}{2.366378in}}%
\pgfpathlineto{\pgfqpoint{4.707329in}{2.361953in}}%
\pgfpathlineto{\pgfqpoint{4.707802in}{2.382057in}}%
\pgfpathlineto{\pgfqpoint{4.709129in}{2.577651in}}%
\pgfpathlineto{\pgfqpoint{4.710077in}{2.511379in}}%
\pgfpathlineto{\pgfqpoint{4.710835in}{2.312102in}}%
\pgfpathlineto{\pgfqpoint{4.711972in}{2.405879in}}%
\pgfpathlineto{\pgfqpoint{4.712067in}{2.406629in}}%
\pgfpathlineto{\pgfqpoint{4.712162in}{2.403660in}}%
\pgfpathlineto{\pgfqpoint{4.713204in}{2.391974in}}%
\pgfpathlineto{\pgfqpoint{4.712825in}{2.410257in}}%
\pgfpathlineto{\pgfqpoint{4.713394in}{2.396787in}}%
\pgfpathlineto{\pgfqpoint{4.713867in}{2.429252in}}%
\pgfpathlineto{\pgfqpoint{4.714436in}{2.398226in}}%
\pgfpathlineto{\pgfqpoint{4.714626in}{2.396999in}}%
\pgfpathlineto{\pgfqpoint{4.715099in}{2.378044in}}%
\pgfpathlineto{\pgfqpoint{4.715668in}{2.392746in}}%
\pgfpathlineto{\pgfqpoint{4.716710in}{2.426064in}}%
\pgfpathlineto{\pgfqpoint{4.717468in}{2.419630in}}%
\pgfpathlineto{\pgfqpoint{4.717942in}{2.378703in}}%
\pgfpathlineto{\pgfqpoint{4.718700in}{2.407551in}}%
\pgfpathlineto{\pgfqpoint{4.720122in}{2.434056in}}%
\pgfpathlineto{\pgfqpoint{4.720501in}{2.432335in}}%
\pgfpathlineto{\pgfqpoint{4.720975in}{2.410116in}}%
\pgfpathlineto{\pgfqpoint{4.721354in}{2.438487in}}%
\pgfpathlineto{\pgfqpoint{4.722965in}{2.478836in}}%
\pgfpathlineto{\pgfqpoint{4.721638in}{2.438229in}}%
\pgfpathlineto{\pgfqpoint{4.723628in}{2.463774in}}%
\pgfpathlineto{\pgfqpoint{4.723818in}{2.455046in}}%
\pgfpathlineto{\pgfqpoint{4.724291in}{2.475017in}}%
\pgfpathlineto{\pgfqpoint{4.724576in}{2.469690in}}%
\pgfpathlineto{\pgfqpoint{4.725902in}{2.495998in}}%
\pgfpathlineto{\pgfqpoint{4.726281in}{2.484120in}}%
\pgfpathlineto{\pgfqpoint{4.727040in}{2.480351in}}%
\pgfpathlineto{\pgfqpoint{4.726661in}{2.485926in}}%
\pgfpathlineto{\pgfqpoint{4.727324in}{2.483253in}}%
\pgfpathlineto{\pgfqpoint{4.728556in}{2.495690in}}%
\pgfpathlineto{\pgfqpoint{4.728745in}{2.489490in}}%
\pgfpathlineto{\pgfqpoint{4.728935in}{2.483853in}}%
\pgfpathlineto{\pgfqpoint{4.729693in}{2.495577in}}%
\pgfpathlineto{\pgfqpoint{4.729788in}{2.496934in}}%
\pgfpathlineto{\pgfqpoint{4.730167in}{2.487554in}}%
\pgfpathlineto{\pgfqpoint{4.733010in}{2.450980in}}%
\pgfpathlineto{\pgfqpoint{4.733199in}{2.456906in}}%
\pgfpathlineto{\pgfqpoint{4.733294in}{2.459189in}}%
\pgfpathlineto{\pgfqpoint{4.733673in}{2.452278in}}%
\pgfpathlineto{\pgfqpoint{4.734147in}{2.453956in}}%
\pgfpathlineto{\pgfqpoint{4.735000in}{2.444451in}}%
\pgfpathlineto{\pgfqpoint{4.734526in}{2.456639in}}%
\pgfpathlineto{\pgfqpoint{4.735189in}{2.450057in}}%
\pgfpathlineto{\pgfqpoint{4.735758in}{2.473549in}}%
\pgfpathlineto{\pgfqpoint{4.736516in}{2.463338in}}%
\pgfpathlineto{\pgfqpoint{4.738885in}{2.482293in}}%
\pgfpathlineto{\pgfqpoint{4.736895in}{2.461919in}}%
\pgfpathlineto{\pgfqpoint{4.738980in}{2.479171in}}%
\pgfpathlineto{\pgfqpoint{4.739738in}{2.482495in}}%
\pgfpathlineto{\pgfqpoint{4.740307in}{2.461074in}}%
\pgfpathlineto{\pgfqpoint{4.740401in}{2.460012in}}%
\pgfpathlineto{\pgfqpoint{4.740686in}{2.467381in}}%
\pgfpathlineto{\pgfqpoint{4.742770in}{2.518400in}}%
\pgfpathlineto{\pgfqpoint{4.743244in}{2.524009in}}%
\pgfpathlineto{\pgfqpoint{4.743434in}{2.518735in}}%
\pgfpathlineto{\pgfqpoint{4.743623in}{2.511666in}}%
\pgfpathlineto{\pgfqpoint{4.744002in}{2.538687in}}%
\pgfpathlineto{\pgfqpoint{4.744666in}{2.560410in}}%
\pgfpathlineto{\pgfqpoint{4.745234in}{2.554331in}}%
\pgfpathlineto{\pgfqpoint{4.749120in}{2.475667in}}%
\pgfpathlineto{\pgfqpoint{4.749404in}{2.478096in}}%
\pgfpathlineto{\pgfqpoint{4.749499in}{2.474102in}}%
\pgfpathlineto{\pgfqpoint{4.749878in}{2.444655in}}%
\pgfpathlineto{\pgfqpoint{4.750352in}{2.485631in}}%
\pgfpathlineto{\pgfqpoint{4.751015in}{2.507547in}}%
\pgfpathlineto{\pgfqpoint{4.751584in}{2.489183in}}%
\pgfpathlineto{\pgfqpoint{4.751678in}{2.489599in}}%
\pgfpathlineto{\pgfqpoint{4.751868in}{2.486521in}}%
\pgfpathlineto{\pgfqpoint{4.752057in}{2.482468in}}%
\pgfpathlineto{\pgfqpoint{4.752436in}{2.492210in}}%
\pgfpathlineto{\pgfqpoint{4.752721in}{2.488717in}}%
\pgfpathlineto{\pgfqpoint{4.753668in}{2.484694in}}%
\pgfpathlineto{\pgfqpoint{4.754142in}{2.555909in}}%
\pgfpathlineto{\pgfqpoint{4.755374in}{2.762136in}}%
\pgfpathlineto{\pgfqpoint{4.756037in}{2.725827in}}%
\pgfpathlineto{\pgfqpoint{4.756416in}{2.673576in}}%
\pgfpathlineto{\pgfqpoint{4.757080in}{2.448445in}}%
\pgfpathlineto{\pgfqpoint{4.757933in}{2.532378in}}%
\pgfpathlineto{\pgfqpoint{4.758027in}{2.532668in}}%
\pgfpathlineto{\pgfqpoint{4.758122in}{2.530455in}}%
\pgfpathlineto{\pgfqpoint{4.758407in}{2.523990in}}%
\pgfpathlineto{\pgfqpoint{4.758880in}{2.538804in}}%
\pgfpathlineto{\pgfqpoint{4.759354in}{2.550276in}}%
\pgfpathlineto{\pgfqpoint{4.760302in}{2.577328in}}%
\pgfpathlineto{\pgfqpoint{4.760586in}{2.561272in}}%
\pgfpathlineto{\pgfqpoint{4.761439in}{2.539421in}}%
\pgfpathlineto{\pgfqpoint{4.761723in}{2.545407in}}%
\pgfpathlineto{\pgfqpoint{4.763429in}{2.591520in}}%
\pgfpathlineto{\pgfqpoint{4.763524in}{2.587290in}}%
\pgfpathlineto{\pgfqpoint{4.764471in}{2.548072in}}%
\pgfpathlineto{\pgfqpoint{4.764756in}{2.561698in}}%
\pgfpathlineto{\pgfqpoint{4.765514in}{2.581142in}}%
\pgfpathlineto{\pgfqpoint{4.766082in}{2.577898in}}%
\pgfpathlineto{\pgfqpoint{4.766272in}{2.585821in}}%
\pgfpathlineto{\pgfqpoint{4.766841in}{2.566611in}}%
\pgfpathlineto{\pgfqpoint{4.767125in}{2.556023in}}%
\pgfpathlineto{\pgfqpoint{4.767883in}{2.561128in}}%
\pgfpathlineto{\pgfqpoint{4.768357in}{2.589932in}}%
\pgfpathlineto{\pgfqpoint{4.769304in}{2.587859in}}%
\pgfpathlineto{\pgfqpoint{4.769494in}{2.591500in}}%
\pgfpathlineto{\pgfqpoint{4.769873in}{2.572785in}}%
\pgfpathlineto{\pgfqpoint{4.769968in}{2.572009in}}%
\pgfpathlineto{\pgfqpoint{4.770063in}{2.575138in}}%
\pgfpathlineto{\pgfqpoint{4.771389in}{2.587971in}}%
\pgfpathlineto{\pgfqpoint{4.771674in}{2.592843in}}%
\pgfpathlineto{\pgfqpoint{4.772242in}{2.584883in}}%
\pgfpathlineto{\pgfqpoint{4.773284in}{2.558329in}}%
\pgfpathlineto{\pgfqpoint{4.773569in}{2.573159in}}%
\pgfpathlineto{\pgfqpoint{4.773758in}{2.579345in}}%
\pgfpathlineto{\pgfqpoint{4.774137in}{2.559314in}}%
\pgfpathlineto{\pgfqpoint{4.774611in}{2.578225in}}%
\pgfpathlineto{\pgfqpoint{4.776317in}{2.547137in}}%
\pgfpathlineto{\pgfqpoint{4.776412in}{2.547511in}}%
\pgfpathlineto{\pgfqpoint{4.776601in}{2.549579in}}%
\pgfpathlineto{\pgfqpoint{4.776886in}{2.541781in}}%
\pgfpathlineto{\pgfqpoint{4.780960in}{2.383874in}}%
\pgfpathlineto{\pgfqpoint{4.781718in}{2.418452in}}%
\pgfpathlineto{\pgfqpoint{4.783045in}{2.441039in}}%
\pgfpathlineto{\pgfqpoint{4.783140in}{2.439518in}}%
\pgfpathlineto{\pgfqpoint{4.783424in}{2.432261in}}%
\pgfpathlineto{\pgfqpoint{4.784088in}{2.440102in}}%
\pgfpathlineto{\pgfqpoint{4.784182in}{2.439950in}}%
\pgfpathlineto{\pgfqpoint{4.784561in}{2.439270in}}%
\pgfpathlineto{\pgfqpoint{4.784846in}{2.433950in}}%
\pgfpathlineto{\pgfqpoint{4.786551in}{2.405111in}}%
\pgfpathlineto{\pgfqpoint{4.786646in}{2.404483in}}%
\pgfpathlineto{\pgfqpoint{4.787025in}{2.409070in}}%
\pgfpathlineto{\pgfqpoint{4.788542in}{2.442470in}}%
\pgfpathlineto{\pgfqpoint{4.788826in}{2.426364in}}%
\pgfpathlineto{\pgfqpoint{4.788921in}{2.423859in}}%
\pgfpathlineto{\pgfqpoint{4.789300in}{2.432509in}}%
\pgfpathlineto{\pgfqpoint{4.789679in}{2.429891in}}%
\pgfpathlineto{\pgfqpoint{4.790816in}{2.460502in}}%
\pgfpathlineto{\pgfqpoint{4.791195in}{2.445170in}}%
\pgfpathlineto{\pgfqpoint{4.792995in}{2.372908in}}%
\pgfpathlineto{\pgfqpoint{4.793469in}{2.386128in}}%
\pgfpathlineto{\pgfqpoint{4.793754in}{2.392755in}}%
\pgfpathlineto{\pgfqpoint{4.794322in}{2.375981in}}%
\pgfpathlineto{\pgfqpoint{4.795933in}{2.347569in}}%
\pgfpathlineto{\pgfqpoint{4.796407in}{2.362897in}}%
\pgfpathlineto{\pgfqpoint{4.796786in}{2.398314in}}%
\pgfpathlineto{\pgfqpoint{4.797449in}{2.367093in}}%
\pgfpathlineto{\pgfqpoint{4.798207in}{2.352981in}}%
\pgfpathlineto{\pgfqpoint{4.798681in}{2.360713in}}%
\pgfpathlineto{\pgfqpoint{4.799724in}{2.342417in}}%
\pgfpathlineto{\pgfqpoint{4.799913in}{2.349662in}}%
\pgfpathlineto{\pgfqpoint{4.801145in}{2.597568in}}%
\pgfpathlineto{\pgfqpoint{4.802282in}{2.533786in}}%
\pgfpathlineto{\pgfqpoint{4.803135in}{2.267532in}}%
\pgfpathlineto{\pgfqpoint{4.804272in}{2.334090in}}%
\pgfpathlineto{\pgfqpoint{4.805694in}{2.371417in}}%
\pgfpathlineto{\pgfqpoint{4.805978in}{2.392082in}}%
\pgfpathlineto{\pgfqpoint{4.806736in}{2.377757in}}%
\pgfpathlineto{\pgfqpoint{4.807305in}{2.353813in}}%
\pgfpathlineto{\pgfqpoint{4.807779in}{2.378777in}}%
\pgfpathlineto{\pgfqpoint{4.809200in}{2.406031in}}%
\pgfpathlineto{\pgfqpoint{4.809390in}{2.410465in}}%
\pgfpathlineto{\pgfqpoint{4.809769in}{2.393851in}}%
\pgfpathlineto{\pgfqpoint{4.810053in}{2.374191in}}%
\pgfpathlineto{\pgfqpoint{4.810811in}{2.392308in}}%
\pgfpathlineto{\pgfqpoint{4.812422in}{2.428894in}}%
\pgfpathlineto{\pgfqpoint{4.812706in}{2.419157in}}%
\pgfpathlineto{\pgfqpoint{4.813559in}{2.397654in}}%
\pgfpathlineto{\pgfqpoint{4.813844in}{2.407616in}}%
\pgfpathlineto{\pgfqpoint{4.814602in}{2.434164in}}%
\pgfpathlineto{\pgfqpoint{4.815265in}{2.429063in}}%
\pgfpathlineto{\pgfqpoint{4.815549in}{2.432578in}}%
\pgfpathlineto{\pgfqpoint{4.816023in}{2.423863in}}%
\pgfpathlineto{\pgfqpoint{4.816876in}{2.416155in}}%
\pgfpathlineto{\pgfqpoint{4.816402in}{2.424904in}}%
\pgfpathlineto{\pgfqpoint{4.817066in}{2.423834in}}%
\pgfpathlineto{\pgfqpoint{4.817824in}{2.440419in}}%
\pgfpathlineto{\pgfqpoint{4.818108in}{2.430733in}}%
\pgfpathlineto{\pgfqpoint{4.819150in}{2.414739in}}%
\pgfpathlineto{\pgfqpoint{4.818676in}{2.433435in}}%
\pgfpathlineto{\pgfqpoint{4.819340in}{2.416419in}}%
\pgfpathlineto{\pgfqpoint{4.821804in}{2.434469in}}%
\pgfpathlineto{\pgfqpoint{4.819719in}{2.415238in}}%
\pgfpathlineto{\pgfqpoint{4.822278in}{2.427654in}}%
\pgfpathlineto{\pgfqpoint{4.822846in}{2.424652in}}%
\pgfpathlineto{\pgfqpoint{4.823225in}{2.428666in}}%
\pgfpathlineto{\pgfqpoint{4.823415in}{2.427579in}}%
\pgfpathlineto{\pgfqpoint{4.826163in}{2.380860in}}%
\pgfpathlineto{\pgfqpoint{4.826447in}{2.390743in}}%
\pgfpathlineto{\pgfqpoint{4.826542in}{2.392740in}}%
\pgfpathlineto{\pgfqpoint{4.826731in}{2.380668in}}%
\pgfpathlineto{\pgfqpoint{4.826921in}{2.367125in}}%
\pgfpathlineto{\pgfqpoint{4.827963in}{2.368542in}}%
\pgfpathlineto{\pgfqpoint{4.829290in}{2.355126in}}%
\pgfpathlineto{\pgfqpoint{4.828627in}{2.374597in}}%
\pgfpathlineto{\pgfqpoint{4.829385in}{2.356723in}}%
\pgfpathlineto{\pgfqpoint{4.829953in}{2.352899in}}%
\pgfpathlineto{\pgfqpoint{4.830806in}{2.367644in}}%
\pgfpathlineto{\pgfqpoint{4.831091in}{2.358275in}}%
\pgfpathlineto{\pgfqpoint{4.831280in}{2.351766in}}%
\pgfpathlineto{\pgfqpoint{4.831754in}{2.363764in}}%
\pgfpathlineto{\pgfqpoint{4.832228in}{2.356152in}}%
\pgfpathlineto{\pgfqpoint{4.832702in}{2.346895in}}%
\pgfpathlineto{\pgfqpoint{4.832986in}{2.358479in}}%
\pgfpathlineto{\pgfqpoint{4.834597in}{2.395282in}}%
\pgfpathlineto{\pgfqpoint{4.834692in}{2.395645in}}%
\pgfpathlineto{\pgfqpoint{4.834786in}{2.393634in}}%
\pgfpathlineto{\pgfqpoint{4.835165in}{2.387064in}}%
\pgfpathlineto{\pgfqpoint{4.835924in}{2.388962in}}%
\pgfpathlineto{\pgfqpoint{4.836303in}{2.402397in}}%
\pgfpathlineto{\pgfqpoint{4.836776in}{2.422090in}}%
\pgfpathlineto{\pgfqpoint{4.837440in}{2.404909in}}%
\pgfpathlineto{\pgfqpoint{4.839240in}{2.348594in}}%
\pgfpathlineto{\pgfqpoint{4.839335in}{2.350558in}}%
\pgfpathlineto{\pgfqpoint{4.840093in}{2.364717in}}%
\pgfpathlineto{\pgfqpoint{4.840377in}{2.353424in}}%
\pgfpathlineto{\pgfqpoint{4.841609in}{2.330164in}}%
\pgfpathlineto{\pgfqpoint{4.841799in}{2.331687in}}%
\pgfpathlineto{\pgfqpoint{4.841894in}{2.331587in}}%
\pgfpathlineto{\pgfqpoint{4.842273in}{2.321662in}}%
\pgfpathlineto{\pgfqpoint{4.842462in}{2.335010in}}%
\pgfpathlineto{\pgfqpoint{4.842936in}{2.378882in}}%
\pgfpathlineto{\pgfqpoint{4.843694in}{2.356860in}}%
\pgfpathlineto{\pgfqpoint{4.845116in}{2.340468in}}%
\pgfpathlineto{\pgfqpoint{4.845495in}{2.345375in}}%
\pgfpathlineto{\pgfqpoint{4.845684in}{2.340083in}}%
\pgfpathlineto{\pgfqpoint{4.845969in}{2.328089in}}%
\pgfpathlineto{\pgfqpoint{4.846348in}{2.365137in}}%
\pgfpathlineto{\pgfqpoint{4.847485in}{2.559033in}}%
\pgfpathlineto{\pgfqpoint{4.848338in}{2.517950in}}%
\pgfpathlineto{\pgfqpoint{4.848622in}{2.498232in}}%
\pgfpathlineto{\pgfqpoint{4.849285in}{2.310121in}}%
\pgfpathlineto{\pgfqpoint{4.850138in}{2.404293in}}%
\pgfpathlineto{\pgfqpoint{4.852318in}{2.454791in}}%
\pgfpathlineto{\pgfqpoint{4.850612in}{2.401535in}}%
\pgfpathlineto{\pgfqpoint{4.852507in}{2.445929in}}%
\pgfpathlineto{\pgfqpoint{4.853455in}{2.410131in}}%
\pgfpathlineto{\pgfqpoint{4.853834in}{2.424843in}}%
\pgfpathlineto{\pgfqpoint{4.855350in}{2.461827in}}%
\pgfpathlineto{\pgfqpoint{4.855635in}{2.455073in}}%
\pgfpathlineto{\pgfqpoint{4.856487in}{2.435163in}}%
\pgfpathlineto{\pgfqpoint{4.856961in}{2.447709in}}%
\pgfpathlineto{\pgfqpoint{4.858193in}{2.463952in}}%
\pgfpathlineto{\pgfqpoint{4.858856in}{2.461642in}}%
\pgfpathlineto{\pgfqpoint{4.859236in}{2.449138in}}%
\pgfpathlineto{\pgfqpoint{4.860088in}{2.455839in}}%
\pgfpathlineto{\pgfqpoint{4.861320in}{2.478750in}}%
\pgfpathlineto{\pgfqpoint{4.861510in}{2.477876in}}%
\pgfpathlineto{\pgfqpoint{4.861889in}{2.479350in}}%
\pgfpathlineto{\pgfqpoint{4.862268in}{2.469053in}}%
\pgfpathlineto{\pgfqpoint{4.862837in}{2.456318in}}%
\pgfpathlineto{\pgfqpoint{4.863121in}{2.473286in}}%
\pgfpathlineto{\pgfqpoint{4.863405in}{2.485089in}}%
\pgfpathlineto{\pgfqpoint{4.864163in}{2.475784in}}%
\pgfpathlineto{\pgfqpoint{4.864353in}{2.471290in}}%
\pgfpathlineto{\pgfqpoint{4.864921in}{2.487585in}}%
\pgfpathlineto{\pgfqpoint{4.865585in}{2.472931in}}%
\pgfpathlineto{\pgfqpoint{4.866438in}{2.477992in}}%
\pgfpathlineto{\pgfqpoint{4.866817in}{2.499246in}}%
\pgfpathlineto{\pgfqpoint{4.867764in}{2.490357in}}%
\pgfpathlineto{\pgfqpoint{4.868049in}{2.487946in}}%
\pgfpathlineto{\pgfqpoint{4.868238in}{2.486654in}}%
\pgfpathlineto{\pgfqpoint{4.868807in}{2.492155in}}%
\pgfpathlineto{\pgfqpoint{4.869091in}{2.490203in}}%
\pgfpathlineto{\pgfqpoint{4.869375in}{2.494221in}}%
\pgfpathlineto{\pgfqpoint{4.869565in}{2.497996in}}%
\pgfpathlineto{\pgfqpoint{4.869849in}{2.481360in}}%
\pgfpathlineto{\pgfqpoint{4.870133in}{2.465647in}}%
\pgfpathlineto{\pgfqpoint{4.870986in}{2.474449in}}%
\pgfpathlineto{\pgfqpoint{4.872503in}{2.457371in}}%
\pgfpathlineto{\pgfqpoint{4.872882in}{2.456602in}}%
\pgfpathlineto{\pgfqpoint{4.874682in}{2.424911in}}%
\pgfpathlineto{\pgfqpoint{4.874777in}{2.426922in}}%
\pgfpathlineto{\pgfqpoint{4.874966in}{2.435263in}}%
\pgfpathlineto{\pgfqpoint{4.875345in}{2.418588in}}%
\pgfpathlineto{\pgfqpoint{4.875914in}{2.432724in}}%
\pgfpathlineto{\pgfqpoint{4.876104in}{2.434678in}}%
\pgfpathlineto{\pgfqpoint{4.877430in}{2.442017in}}%
\pgfpathlineto{\pgfqpoint{4.877525in}{2.442194in}}%
\pgfpathlineto{\pgfqpoint{4.877620in}{2.441118in}}%
\pgfpathlineto{\pgfqpoint{4.878473in}{2.428395in}}%
\pgfpathlineto{\pgfqpoint{4.879231in}{2.430829in}}%
\pgfpathlineto{\pgfqpoint{4.880273in}{2.446358in}}%
\pgfpathlineto{\pgfqpoint{4.881410in}{2.462889in}}%
\pgfpathlineto{\pgfqpoint{4.881600in}{2.462171in}}%
\pgfpathlineto{\pgfqpoint{4.881789in}{2.461217in}}%
\pgfpathlineto{\pgfqpoint{4.882074in}{2.464577in}}%
\pgfpathlineto{\pgfqpoint{4.882737in}{2.492122in}}%
\pgfpathlineto{\pgfqpoint{4.883590in}{2.482409in}}%
\pgfpathlineto{\pgfqpoint{4.884348in}{2.455096in}}%
\pgfpathlineto{\pgfqpoint{4.884917in}{2.418982in}}%
\pgfpathlineto{\pgfqpoint{4.885580in}{2.434091in}}%
\pgfpathlineto{\pgfqpoint{4.885864in}{2.448522in}}%
\pgfpathlineto{\pgfqpoint{4.886528in}{2.429928in}}%
\pgfpathlineto{\pgfqpoint{4.888044in}{2.397440in}}%
\pgfpathlineto{\pgfqpoint{4.888328in}{2.411055in}}%
\pgfpathlineto{\pgfqpoint{4.889086in}{2.448874in}}%
\pgfpathlineto{\pgfqpoint{4.889655in}{2.424904in}}%
\pgfpathlineto{\pgfqpoint{4.890223in}{2.408440in}}%
\pgfpathlineto{\pgfqpoint{4.891171in}{2.410912in}}%
\pgfpathlineto{\pgfqpoint{4.892498in}{2.510366in}}%
\pgfpathlineto{\pgfqpoint{4.893351in}{2.683326in}}%
\pgfpathlineto{\pgfqpoint{4.894109in}{2.651495in}}%
\pgfpathlineto{\pgfqpoint{4.894583in}{2.587989in}}%
\pgfpathlineto{\pgfqpoint{4.895246in}{2.370698in}}%
\pgfpathlineto{\pgfqpoint{4.896099in}{2.434805in}}%
\pgfpathlineto{\pgfqpoint{4.896194in}{2.434932in}}%
\pgfpathlineto{\pgfqpoint{4.898468in}{2.494524in}}%
\pgfpathlineto{\pgfqpoint{4.898657in}{2.483997in}}%
\pgfpathlineto{\pgfqpoint{4.899984in}{2.458194in}}%
\pgfpathlineto{\pgfqpoint{4.900079in}{2.457716in}}%
\pgfpathlineto{\pgfqpoint{4.900174in}{2.461071in}}%
\pgfpathlineto{\pgfqpoint{4.901216in}{2.499530in}}%
\pgfpathlineto{\pgfqpoint{4.901595in}{2.493741in}}%
\pgfpathlineto{\pgfqpoint{4.901690in}{2.494551in}}%
\pgfpathlineto{\pgfqpoint{4.901879in}{2.488269in}}%
\pgfpathlineto{\pgfqpoint{4.902258in}{2.457015in}}%
\pgfpathlineto{\pgfqpoint{4.903111in}{2.471932in}}%
\pgfpathlineto{\pgfqpoint{4.904248in}{2.484010in}}%
\pgfpathlineto{\pgfqpoint{4.904438in}{2.482499in}}%
\pgfpathlineto{\pgfqpoint{4.905765in}{2.451810in}}%
\pgfpathlineto{\pgfqpoint{4.906333in}{2.471505in}}%
\pgfpathlineto{\pgfqpoint{4.906428in}{2.473340in}}%
\pgfpathlineto{\pgfqpoint{4.906807in}{2.470827in}}%
\pgfpathlineto{\pgfqpoint{4.907376in}{2.471327in}}%
\pgfpathlineto{\pgfqpoint{4.907470in}{2.470851in}}%
\pgfpathlineto{\pgfqpoint{4.907660in}{2.473121in}}%
\pgfpathlineto{\pgfqpoint{4.907850in}{2.476141in}}%
\pgfpathlineto{\pgfqpoint{4.908134in}{2.460119in}}%
\pgfpathlineto{\pgfqpoint{4.908418in}{2.444019in}}%
\pgfpathlineto{\pgfqpoint{4.909271in}{2.457438in}}%
\pgfpathlineto{\pgfqpoint{4.910692in}{2.436086in}}%
\pgfpathlineto{\pgfqpoint{4.910977in}{2.444510in}}%
\pgfpathlineto{\pgfqpoint{4.911072in}{2.445194in}}%
\pgfpathlineto{\pgfqpoint{4.911166in}{2.441671in}}%
\pgfpathlineto{\pgfqpoint{4.911640in}{2.412192in}}%
\pgfpathlineto{\pgfqpoint{4.912493in}{2.417881in}}%
\pgfpathlineto{\pgfqpoint{4.912588in}{2.417919in}}%
\pgfpathlineto{\pgfqpoint{4.912682in}{2.417264in}}%
\pgfpathlineto{\pgfqpoint{4.914199in}{2.396663in}}%
\pgfpathlineto{\pgfqpoint{4.914388in}{2.400615in}}%
\pgfpathlineto{\pgfqpoint{4.914673in}{2.408049in}}%
\pgfpathlineto{\pgfqpoint{4.915052in}{2.390490in}}%
\pgfpathlineto{\pgfqpoint{4.917421in}{2.301151in}}%
\pgfpathlineto{\pgfqpoint{4.917515in}{2.301515in}}%
\pgfpathlineto{\pgfqpoint{4.919790in}{2.379673in}}%
\pgfpathlineto{\pgfqpoint{4.920169in}{2.353424in}}%
\pgfpathlineto{\pgfqpoint{4.921685in}{2.326412in}}%
\pgfpathlineto{\pgfqpoint{4.920643in}{2.360794in}}%
\pgfpathlineto{\pgfqpoint{4.921780in}{2.330516in}}%
\pgfpathlineto{\pgfqpoint{4.922727in}{2.358747in}}%
\pgfpathlineto{\pgfqpoint{4.923012in}{2.345200in}}%
\pgfpathlineto{\pgfqpoint{4.923107in}{2.343133in}}%
\pgfpathlineto{\pgfqpoint{4.923391in}{2.351083in}}%
\pgfpathlineto{\pgfqpoint{4.923770in}{2.347672in}}%
\pgfpathlineto{\pgfqpoint{4.924054in}{2.358001in}}%
\pgfpathlineto{\pgfqpoint{4.924528in}{2.334530in}}%
\pgfpathlineto{\pgfqpoint{4.925002in}{2.355975in}}%
\pgfpathlineto{\pgfqpoint{4.925286in}{2.351953in}}%
\pgfpathlineto{\pgfqpoint{4.925570in}{2.361865in}}%
\pgfpathlineto{\pgfqpoint{4.926518in}{2.390323in}}%
\pgfpathlineto{\pgfqpoint{4.926992in}{2.382771in}}%
\pgfpathlineto{\pgfqpoint{4.928698in}{2.444797in}}%
\pgfpathlineto{\pgfqpoint{4.929645in}{2.437404in}}%
\pgfpathlineto{\pgfqpoint{4.930403in}{2.418888in}}%
\pgfpathlineto{\pgfqpoint{4.931067in}{2.399995in}}%
\pgfpathlineto{\pgfqpoint{4.931446in}{2.414754in}}%
\pgfpathlineto{\pgfqpoint{4.931920in}{2.447149in}}%
\pgfpathlineto{\pgfqpoint{4.932678in}{2.428779in}}%
\pgfpathlineto{\pgfqpoint{4.932962in}{2.417512in}}%
\pgfpathlineto{\pgfqpoint{4.933815in}{2.421737in}}%
\pgfpathlineto{\pgfqpoint{4.935426in}{2.473192in}}%
\pgfpathlineto{\pgfqpoint{4.935710in}{2.464474in}}%
\pgfpathlineto{\pgfqpoint{4.936658in}{2.444120in}}%
\pgfpathlineto{\pgfqpoint{4.936279in}{2.464586in}}%
\pgfpathlineto{\pgfqpoint{4.936942in}{2.454158in}}%
\pgfpathlineto{\pgfqpoint{4.937700in}{2.448427in}}%
\pgfpathlineto{\pgfqpoint{4.938458in}{2.539478in}}%
\pgfpathlineto{\pgfqpoint{4.939596in}{2.736320in}}%
\pgfpathlineto{\pgfqpoint{4.940069in}{2.689582in}}%
\pgfpathlineto{\pgfqpoint{4.940448in}{2.660011in}}%
\pgfpathlineto{\pgfqpoint{4.941301in}{2.404446in}}%
\pgfpathlineto{\pgfqpoint{4.942249in}{2.477234in}}%
\pgfpathlineto{\pgfqpoint{4.942628in}{2.472112in}}%
\pgfpathlineto{\pgfqpoint{4.942817in}{2.480199in}}%
\pgfpathlineto{\pgfqpoint{4.944144in}{2.517621in}}%
\pgfpathlineto{\pgfqpoint{4.944239in}{2.515026in}}%
\pgfpathlineto{\pgfqpoint{4.945566in}{2.459921in}}%
\pgfpathlineto{\pgfqpoint{4.946039in}{2.486524in}}%
\pgfpathlineto{\pgfqpoint{4.947271in}{2.514178in}}%
\pgfpathlineto{\pgfqpoint{4.947556in}{2.502423in}}%
\pgfpathlineto{\pgfqpoint{4.948598in}{2.465472in}}%
\pgfpathlineto{\pgfqpoint{4.948977in}{2.475291in}}%
\pgfpathlineto{\pgfqpoint{4.950209in}{2.497131in}}%
\pgfpathlineto{\pgfqpoint{4.950493in}{2.491998in}}%
\pgfpathlineto{\pgfqpoint{4.951346in}{2.475987in}}%
\pgfpathlineto{\pgfqpoint{4.951725in}{2.488481in}}%
\pgfpathlineto{\pgfqpoint{4.952578in}{2.513722in}}%
\pgfpathlineto{\pgfqpoint{4.953810in}{2.506488in}}%
\pgfpathlineto{\pgfqpoint{4.954663in}{2.500667in}}%
\pgfpathlineto{\pgfqpoint{4.954853in}{2.505899in}}%
\pgfpathlineto{\pgfqpoint{4.955042in}{2.510280in}}%
\pgfpathlineto{\pgfqpoint{4.955895in}{2.504020in}}%
\pgfpathlineto{\pgfqpoint{4.956843in}{2.518466in}}%
\pgfpathlineto{\pgfqpoint{4.956464in}{2.503270in}}%
\pgfpathlineto{\pgfqpoint{4.957127in}{2.505633in}}%
\pgfpathlineto{\pgfqpoint{4.957411in}{2.493188in}}%
\pgfpathlineto{\pgfqpoint{4.958264in}{2.497489in}}%
\pgfpathlineto{\pgfqpoint{4.958643in}{2.511758in}}%
\pgfpathlineto{\pgfqpoint{4.959401in}{2.498663in}}%
\pgfpathlineto{\pgfqpoint{4.960254in}{2.485851in}}%
\pgfpathlineto{\pgfqpoint{4.960538in}{2.494452in}}%
\pgfpathlineto{\pgfqpoint{4.960633in}{2.497838in}}%
\pgfpathlineto{\pgfqpoint{4.961107in}{2.477887in}}%
\pgfpathlineto{\pgfqpoint{4.961202in}{2.477347in}}%
\pgfpathlineto{\pgfqpoint{4.961770in}{2.479693in}}%
\pgfpathlineto{\pgfqpoint{4.961865in}{2.480430in}}%
\pgfpathlineto{\pgfqpoint{4.962055in}{2.477226in}}%
\pgfpathlineto{\pgfqpoint{4.964329in}{2.419408in}}%
\pgfpathlineto{\pgfqpoint{4.964613in}{2.426294in}}%
\pgfpathlineto{\pgfqpoint{4.964992in}{2.417899in}}%
\pgfpathlineto{\pgfqpoint{4.965466in}{2.420672in}}%
\pgfpathlineto{\pgfqpoint{4.966414in}{2.403943in}}%
\pgfpathlineto{\pgfqpoint{4.966698in}{2.408365in}}%
\pgfpathlineto{\pgfqpoint{4.967172in}{2.418934in}}%
\pgfpathlineto{\pgfqpoint{4.967551in}{2.403721in}}%
\pgfpathlineto{\pgfqpoint{4.967740in}{2.401057in}}%
\pgfpathlineto{\pgfqpoint{4.968309in}{2.413578in}}%
\pgfpathlineto{\pgfqpoint{4.968878in}{2.421879in}}%
\pgfpathlineto{\pgfqpoint{4.969162in}{2.413387in}}%
\pgfpathlineto{\pgfqpoint{4.970489in}{2.396404in}}%
\pgfpathlineto{\pgfqpoint{4.969825in}{2.418526in}}%
\pgfpathlineto{\pgfqpoint{4.970583in}{2.397119in}}%
\pgfpathlineto{\pgfqpoint{4.972952in}{2.430332in}}%
\pgfpathlineto{\pgfqpoint{4.973332in}{2.416677in}}%
\pgfpathlineto{\pgfqpoint{4.973521in}{2.412371in}}%
\pgfpathlineto{\pgfqpoint{4.973900in}{2.437158in}}%
\pgfpathlineto{\pgfqpoint{4.974658in}{2.455911in}}%
\pgfpathlineto{\pgfqpoint{4.975132in}{2.442346in}}%
\pgfpathlineto{\pgfqpoint{4.975416in}{2.440616in}}%
\pgfpathlineto{\pgfqpoint{4.977217in}{2.377170in}}%
\pgfpathlineto{\pgfqpoint{4.977880in}{2.403587in}}%
\pgfpathlineto{\pgfqpoint{4.978164in}{2.406396in}}%
\pgfpathlineto{\pgfqpoint{4.978544in}{2.399569in}}%
\pgfpathlineto{\pgfqpoint{4.978923in}{2.399434in}}%
\pgfpathlineto{\pgfqpoint{4.979396in}{2.388911in}}%
\pgfpathlineto{\pgfqpoint{4.980155in}{2.391508in}}%
\pgfpathlineto{\pgfqpoint{4.981197in}{2.448112in}}%
\pgfpathlineto{\pgfqpoint{4.981766in}{2.415713in}}%
\pgfpathlineto{\pgfqpoint{4.983282in}{2.408594in}}%
\pgfpathlineto{\pgfqpoint{4.983377in}{2.410335in}}%
\pgfpathlineto{\pgfqpoint{4.984135in}{2.459026in}}%
\pgfpathlineto{\pgfqpoint{4.985556in}{2.687307in}}%
\pgfpathlineto{\pgfqpoint{4.986125in}{2.646642in}}%
\pgfpathlineto{\pgfqpoint{4.986788in}{2.487376in}}%
\pgfpathlineto{\pgfqpoint{4.987262in}{2.372602in}}%
\pgfpathlineto{\pgfqpoint{4.987925in}{2.450646in}}%
\pgfpathlineto{\pgfqpoint{4.989726in}{2.427089in}}%
\pgfpathlineto{\pgfqpoint{4.990200in}{2.442140in}}%
\pgfpathlineto{\pgfqpoint{4.990579in}{2.425970in}}%
\pgfpathlineto{\pgfqpoint{4.990863in}{2.415273in}}%
\pgfpathlineto{\pgfqpoint{4.991431in}{2.435116in}}%
\pgfpathlineto{\pgfqpoint{4.993422in}{2.529518in}}%
\pgfpathlineto{\pgfqpoint{4.994085in}{2.497535in}}%
\pgfpathlineto{\pgfqpoint{4.994559in}{2.475359in}}%
\pgfpathlineto{\pgfqpoint{4.995127in}{2.499500in}}%
\pgfpathlineto{\pgfqpoint{4.996264in}{2.522477in}}%
\pgfpathlineto{\pgfqpoint{4.996643in}{2.509451in}}%
\pgfpathlineto{\pgfqpoint{4.997402in}{2.494636in}}%
\pgfpathlineto{\pgfqpoint{4.997875in}{2.504298in}}%
\pgfpathlineto{\pgfqpoint{4.999107in}{2.525622in}}%
\pgfpathlineto{\pgfqpoint{4.999392in}{2.520543in}}%
\pgfpathlineto{\pgfqpoint{5.000434in}{2.514696in}}%
\pgfpathlineto{\pgfqpoint{4.999865in}{2.528763in}}%
\pgfpathlineto{\pgfqpoint{5.000813in}{2.515555in}}%
\pgfpathlineto{\pgfqpoint{5.002708in}{2.531555in}}%
\pgfpathlineto{\pgfqpoint{5.002993in}{2.529290in}}%
\pgfpathlineto{\pgfqpoint{5.004035in}{2.504943in}}%
\pgfpathlineto{\pgfqpoint{5.004319in}{2.519762in}}%
\pgfpathlineto{\pgfqpoint{5.005172in}{2.525328in}}%
\pgfpathlineto{\pgfqpoint{5.004793in}{2.514786in}}%
\pgfpathlineto{\pgfqpoint{5.005362in}{2.518798in}}%
\pgfpathlineto{\pgfqpoint{5.005646in}{2.506957in}}%
\pgfpathlineto{\pgfqpoint{5.006120in}{2.520468in}}%
\pgfpathlineto{\pgfqpoint{5.006499in}{2.516300in}}%
\pgfpathlineto{\pgfqpoint{5.013891in}{2.417116in}}%
\pgfpathlineto{\pgfqpoint{5.014554in}{2.418636in}}%
\pgfpathlineto{\pgfqpoint{5.015596in}{2.435441in}}%
\pgfpathlineto{\pgfqpoint{5.015786in}{2.425358in}}%
\pgfpathlineto{\pgfqpoint{5.016733in}{2.396975in}}%
\pgfpathlineto{\pgfqpoint{5.017113in}{2.403763in}}%
\pgfpathlineto{\pgfqpoint{5.017397in}{2.409426in}}%
\pgfpathlineto{\pgfqpoint{5.019197in}{2.457042in}}%
\pgfpathlineto{\pgfqpoint{5.019292in}{2.454595in}}%
\pgfpathlineto{\pgfqpoint{5.019671in}{2.432070in}}%
\pgfpathlineto{\pgfqpoint{5.020429in}{2.448320in}}%
\pgfpathlineto{\pgfqpoint{5.021187in}{2.464637in}}%
\pgfpathlineto{\pgfqpoint{5.021472in}{2.454564in}}%
\pgfpathlineto{\pgfqpoint{5.023177in}{2.387376in}}%
\pgfpathlineto{\pgfqpoint{5.023462in}{2.394574in}}%
\pgfpathlineto{\pgfqpoint{5.024315in}{2.417368in}}%
\pgfpathlineto{\pgfqpoint{5.024694in}{2.404919in}}%
\pgfpathlineto{\pgfqpoint{5.026210in}{2.378032in}}%
\pgfpathlineto{\pgfqpoint{5.026305in}{2.378461in}}%
\pgfpathlineto{\pgfqpoint{5.027442in}{2.424391in}}%
\pgfpathlineto{\pgfqpoint{5.027821in}{2.396616in}}%
\pgfpathlineto{\pgfqpoint{5.028484in}{2.384592in}}%
\pgfpathlineto{\pgfqpoint{5.028863in}{2.400896in}}%
\pgfpathlineto{\pgfqpoint{5.028958in}{2.403791in}}%
\pgfpathlineto{\pgfqpoint{5.029337in}{2.384672in}}%
\pgfpathlineto{\pgfqpoint{5.029716in}{2.396617in}}%
\pgfpathlineto{\pgfqpoint{5.030000in}{2.388318in}}%
\pgfpathlineto{\pgfqpoint{5.030285in}{2.405334in}}%
\pgfpathlineto{\pgfqpoint{5.031706in}{2.656729in}}%
\pgfpathlineto{\pgfqpoint{5.032654in}{2.594736in}}%
\pgfpathlineto{\pgfqpoint{5.033507in}{2.337644in}}%
\pgfpathlineto{\pgfqpoint{5.034360in}{2.427277in}}%
\pgfpathlineto{\pgfqpoint{5.034833in}{2.425038in}}%
\pgfpathlineto{\pgfqpoint{5.035686in}{2.446754in}}%
\pgfpathlineto{\pgfqpoint{5.036444in}{2.474306in}}%
\pgfpathlineto{\pgfqpoint{5.036918in}{2.458937in}}%
\pgfpathlineto{\pgfqpoint{5.037392in}{2.433329in}}%
\pgfpathlineto{\pgfqpoint{5.038150in}{2.448699in}}%
\pgfpathlineto{\pgfqpoint{5.039666in}{2.478366in}}%
\pgfpathlineto{\pgfqpoint{5.039856in}{2.462621in}}%
\pgfpathlineto{\pgfqpoint{5.040804in}{2.424675in}}%
\pgfpathlineto{\pgfqpoint{5.041088in}{2.434472in}}%
\pgfpathlineto{\pgfqpoint{5.041941in}{2.458561in}}%
\pgfpathlineto{\pgfqpoint{5.042699in}{2.455566in}}%
\pgfpathlineto{\pgfqpoint{5.042983in}{2.455810in}}%
\pgfpathlineto{\pgfqpoint{5.043078in}{2.454996in}}%
\pgfpathlineto{\pgfqpoint{5.043646in}{2.445840in}}%
\pgfpathlineto{\pgfqpoint{5.044026in}{2.454052in}}%
\pgfpathlineto{\pgfqpoint{5.045257in}{2.486337in}}%
\pgfpathlineto{\pgfqpoint{5.045731in}{2.482161in}}%
\pgfpathlineto{\pgfqpoint{5.046016in}{2.493159in}}%
\pgfpathlineto{\pgfqpoint{5.046395in}{2.469163in}}%
\pgfpathlineto{\pgfqpoint{5.046489in}{2.465794in}}%
\pgfpathlineto{\pgfqpoint{5.046963in}{2.479378in}}%
\pgfpathlineto{\pgfqpoint{5.047153in}{2.477755in}}%
\pgfpathlineto{\pgfqpoint{5.047627in}{2.488002in}}%
\pgfpathlineto{\pgfqpoint{5.048290in}{2.479772in}}%
\pgfpathlineto{\pgfqpoint{5.049617in}{2.442963in}}%
\pgfpathlineto{\pgfqpoint{5.050185in}{2.451291in}}%
\pgfpathlineto{\pgfqpoint{5.051133in}{2.458167in}}%
\pgfpathlineto{\pgfqpoint{5.050754in}{2.445783in}}%
\pgfpathlineto{\pgfqpoint{5.051228in}{2.455559in}}%
\pgfpathlineto{\pgfqpoint{5.051512in}{2.447352in}}%
\pgfpathlineto{\pgfqpoint{5.051986in}{2.461468in}}%
\pgfpathlineto{\pgfqpoint{5.052270in}{2.457480in}}%
\pgfpathlineto{\pgfqpoint{5.054071in}{2.433278in}}%
\pgfpathlineto{\pgfqpoint{5.054165in}{2.433843in}}%
\pgfpathlineto{\pgfqpoint{5.054639in}{2.443632in}}%
\pgfpathlineto{\pgfqpoint{5.055397in}{2.439276in}}%
\pgfpathlineto{\pgfqpoint{5.055492in}{2.439127in}}%
\pgfpathlineto{\pgfqpoint{5.056819in}{2.422010in}}%
\pgfpathlineto{\pgfqpoint{5.057103in}{2.423153in}}%
\pgfpathlineto{\pgfqpoint{5.057293in}{2.425570in}}%
\pgfpathlineto{\pgfqpoint{5.057577in}{2.418273in}}%
\pgfpathlineto{\pgfqpoint{5.059093in}{2.395242in}}%
\pgfpathlineto{\pgfqpoint{5.061367in}{2.315392in}}%
\pgfpathlineto{\pgfqpoint{5.061746in}{2.332681in}}%
\pgfpathlineto{\pgfqpoint{5.065347in}{2.485024in}}%
\pgfpathlineto{\pgfqpoint{5.065632in}{2.472862in}}%
\pgfpathlineto{\pgfqpoint{5.065916in}{2.491614in}}%
\pgfpathlineto{\pgfqpoint{5.067338in}{2.544428in}}%
\pgfpathlineto{\pgfqpoint{5.067432in}{2.542222in}}%
\pgfpathlineto{\pgfqpoint{5.069517in}{2.451703in}}%
\pgfpathlineto{\pgfqpoint{5.069801in}{2.465085in}}%
\pgfpathlineto{\pgfqpoint{5.070086in}{2.475841in}}%
\pgfpathlineto{\pgfqpoint{5.070749in}{2.461904in}}%
\pgfpathlineto{\pgfqpoint{5.072170in}{2.430671in}}%
\pgfpathlineto{\pgfqpoint{5.072455in}{2.443060in}}%
\pgfpathlineto{\pgfqpoint{5.073402in}{2.479575in}}%
\pgfpathlineto{\pgfqpoint{5.073781in}{2.461782in}}%
\pgfpathlineto{\pgfqpoint{5.074824in}{2.421386in}}%
\pgfpathlineto{\pgfqpoint{5.075108in}{2.430558in}}%
\pgfpathlineto{\pgfqpoint{5.075298in}{2.434026in}}%
\pgfpathlineto{\pgfqpoint{5.075866in}{2.418870in}}%
\pgfpathlineto{\pgfqpoint{5.076056in}{2.415430in}}%
\pgfpathlineto{\pgfqpoint{5.076340in}{2.430129in}}%
\pgfpathlineto{\pgfqpoint{5.077762in}{2.672373in}}%
\pgfpathlineto{\pgfqpoint{5.078520in}{2.616623in}}%
\pgfpathlineto{\pgfqpoint{5.079088in}{2.458387in}}%
\pgfpathlineto{\pgfqpoint{5.079562in}{2.354264in}}%
\pgfpathlineto{\pgfqpoint{5.080225in}{2.420867in}}%
\pgfpathlineto{\pgfqpoint{5.080510in}{2.412830in}}%
\pgfpathlineto{\pgfqpoint{5.081457in}{2.416782in}}%
\pgfpathlineto{\pgfqpoint{5.082215in}{2.444055in}}%
\pgfpathlineto{\pgfqpoint{5.082974in}{2.423543in}}%
\pgfpathlineto{\pgfqpoint{5.083447in}{2.389458in}}%
\pgfpathlineto{\pgfqpoint{5.084300in}{2.400701in}}%
\pgfpathlineto{\pgfqpoint{5.085532in}{2.426412in}}%
\pgfpathlineto{\pgfqpoint{5.085911in}{2.418401in}}%
\pgfpathlineto{\pgfqpoint{5.086575in}{2.379801in}}%
\pgfpathlineto{\pgfqpoint{5.087428in}{2.402052in}}%
\pgfpathlineto{\pgfqpoint{5.087522in}{2.402356in}}%
\pgfpathlineto{\pgfqpoint{5.087617in}{2.400096in}}%
\pgfpathlineto{\pgfqpoint{5.087807in}{2.396623in}}%
\pgfpathlineto{\pgfqpoint{5.088091in}{2.417098in}}%
\pgfpathlineto{\pgfqpoint{5.088280in}{2.424398in}}%
\pgfpathlineto{\pgfqpoint{5.089039in}{2.413818in}}%
\pgfpathlineto{\pgfqpoint{5.089891in}{2.402636in}}%
\pgfpathlineto{\pgfqpoint{5.090081in}{2.411604in}}%
\pgfpathlineto{\pgfqpoint{5.091123in}{2.435934in}}%
\pgfpathlineto{\pgfqpoint{5.091408in}{2.435655in}}%
\pgfpathlineto{\pgfqpoint{5.092450in}{2.452854in}}%
\pgfpathlineto{\pgfqpoint{5.092734in}{2.443165in}}%
\pgfpathlineto{\pgfqpoint{5.092924in}{2.436762in}}%
\pgfpathlineto{\pgfqpoint{5.093398in}{2.452241in}}%
\pgfpathlineto{\pgfqpoint{5.093587in}{2.452023in}}%
\pgfpathlineto{\pgfqpoint{5.094061in}{2.468016in}}%
\pgfpathlineto{\pgfqpoint{5.094914in}{2.456852in}}%
\pgfpathlineto{\pgfqpoint{5.095009in}{2.456083in}}%
\pgfpathlineto{\pgfqpoint{5.095198in}{2.460535in}}%
\pgfpathlineto{\pgfqpoint{5.095482in}{2.470123in}}%
\pgfpathlineto{\pgfqpoint{5.095862in}{2.456303in}}%
\pgfpathlineto{\pgfqpoint{5.096430in}{2.468368in}}%
\pgfpathlineto{\pgfqpoint{5.096809in}{2.465164in}}%
\pgfpathlineto{\pgfqpoint{5.097093in}{2.470054in}}%
\pgfpathlineto{\pgfqpoint{5.097852in}{2.483972in}}%
\pgfpathlineto{\pgfqpoint{5.098136in}{2.474950in}}%
\pgfpathlineto{\pgfqpoint{5.098420in}{2.462255in}}%
\pgfpathlineto{\pgfqpoint{5.098894in}{2.479307in}}%
\pgfpathlineto{\pgfqpoint{5.099368in}{2.466843in}}%
\pgfpathlineto{\pgfqpoint{5.099557in}{2.464196in}}%
\pgfpathlineto{\pgfqpoint{5.100126in}{2.473359in}}%
\pgfpathlineto{\pgfqpoint{5.100410in}{2.477547in}}%
\pgfpathlineto{\pgfqpoint{5.100694in}{2.470591in}}%
\pgfpathlineto{\pgfqpoint{5.100979in}{2.464342in}}%
\pgfpathlineto{\pgfqpoint{5.101832in}{2.467699in}}%
\pgfpathlineto{\pgfqpoint{5.102685in}{2.463698in}}%
\pgfpathlineto{\pgfqpoint{5.102305in}{2.467768in}}%
\pgfpathlineto{\pgfqpoint{5.102779in}{2.465915in}}%
\pgfpathlineto{\pgfqpoint{5.103158in}{2.475247in}}%
\pgfpathlineto{\pgfqpoint{5.104011in}{2.472302in}}%
\pgfpathlineto{\pgfqpoint{5.104390in}{2.474770in}}%
\pgfpathlineto{\pgfqpoint{5.104675in}{2.467746in}}%
\pgfpathlineto{\pgfqpoint{5.104959in}{2.457770in}}%
\pgfpathlineto{\pgfqpoint{5.105338in}{2.470192in}}%
\pgfpathlineto{\pgfqpoint{5.105907in}{2.460491in}}%
\pgfpathlineto{\pgfqpoint{5.106759in}{2.454062in}}%
\pgfpathlineto{\pgfqpoint{5.107044in}{2.455867in}}%
\pgfpathlineto{\pgfqpoint{5.107328in}{2.459482in}}%
\pgfpathlineto{\pgfqpoint{5.107612in}{2.450502in}}%
\pgfpathlineto{\pgfqpoint{5.109223in}{2.404283in}}%
\pgfpathlineto{\pgfqpoint{5.109508in}{2.422545in}}%
\pgfpathlineto{\pgfqpoint{5.110550in}{2.443877in}}%
\pgfpathlineto{\pgfqpoint{5.110834in}{2.441107in}}%
\pgfpathlineto{\pgfqpoint{5.111213in}{2.452246in}}%
\pgfpathlineto{\pgfqpoint{5.112161in}{2.448132in}}%
\pgfpathlineto{\pgfqpoint{5.113203in}{2.480620in}}%
\pgfpathlineto{\pgfqpoint{5.113582in}{2.458483in}}%
\pgfpathlineto{\pgfqpoint{5.115288in}{2.382478in}}%
\pgfpathlineto{\pgfqpoint{5.115762in}{2.389446in}}%
\pgfpathlineto{\pgfqpoint{5.116425in}{2.391630in}}%
\pgfpathlineto{\pgfqpoint{5.116615in}{2.388615in}}%
\pgfpathlineto{\pgfqpoint{5.118510in}{2.354582in}}%
\pgfpathlineto{\pgfqpoint{5.118889in}{2.374826in}}%
\pgfpathlineto{\pgfqpoint{5.119268in}{2.398368in}}%
\pgfpathlineto{\pgfqpoint{5.119932in}{2.371395in}}%
\pgfpathlineto{\pgfqpoint{5.121258in}{2.350464in}}%
\pgfpathlineto{\pgfqpoint{5.121448in}{2.351611in}}%
\pgfpathlineto{\pgfqpoint{5.121543in}{2.352423in}}%
\pgfpathlineto{\pgfqpoint{5.121827in}{2.346916in}}%
\pgfpathlineto{\pgfqpoint{5.121922in}{2.344903in}}%
\pgfpathlineto{\pgfqpoint{5.122301in}{2.358178in}}%
\pgfpathlineto{\pgfqpoint{5.123154in}{2.541258in}}%
\pgfpathlineto{\pgfqpoint{5.123912in}{2.613750in}}%
\pgfpathlineto{\pgfqpoint{5.124386in}{2.577444in}}%
\pgfpathlineto{\pgfqpoint{5.125049in}{2.461404in}}%
\pgfpathlineto{\pgfqpoint{5.125617in}{2.299773in}}%
\pgfpathlineto{\pgfqpoint{5.126376in}{2.371947in}}%
\pgfpathlineto{\pgfqpoint{5.128555in}{2.434429in}}%
\pgfpathlineto{\pgfqpoint{5.129029in}{2.421285in}}%
\pgfpathlineto{\pgfqpoint{5.129977in}{2.381168in}}%
\pgfpathlineto{\pgfqpoint{5.130450in}{2.393878in}}%
\pgfpathlineto{\pgfqpoint{5.132346in}{2.349901in}}%
\pgfpathlineto{\pgfqpoint{5.132535in}{2.353953in}}%
\pgfpathlineto{\pgfqpoint{5.134146in}{2.460948in}}%
\pgfpathlineto{\pgfqpoint{5.135094in}{2.457610in}}%
\pgfpathlineto{\pgfqpoint{5.135947in}{2.428845in}}%
\pgfpathlineto{\pgfqpoint{5.136326in}{2.449028in}}%
\pgfpathlineto{\pgfqpoint{5.137558in}{2.467398in}}%
\pgfpathlineto{\pgfqpoint{5.137842in}{2.461499in}}%
\pgfpathlineto{\pgfqpoint{5.139074in}{2.442150in}}%
\pgfpathlineto{\pgfqpoint{5.139358in}{2.453222in}}%
\pgfpathlineto{\pgfqpoint{5.140211in}{2.475627in}}%
\pgfpathlineto{\pgfqpoint{5.140590in}{2.463733in}}%
\pgfpathlineto{\pgfqpoint{5.140874in}{2.473215in}}%
\pgfpathlineto{\pgfqpoint{5.141348in}{2.461937in}}%
\pgfpathlineto{\pgfqpoint{5.141633in}{2.465207in}}%
\pgfpathlineto{\pgfqpoint{5.142580in}{2.454548in}}%
\pgfpathlineto{\pgfqpoint{5.142770in}{2.460384in}}%
\pgfpathlineto{\pgfqpoint{5.143149in}{2.478752in}}%
\pgfpathlineto{\pgfqpoint{5.144002in}{2.469834in}}%
\pgfpathlineto{\pgfqpoint{5.144476in}{2.479059in}}%
\pgfpathlineto{\pgfqpoint{5.145139in}{2.470772in}}%
\pgfpathlineto{\pgfqpoint{5.145802in}{2.467273in}}%
\pgfpathlineto{\pgfqpoint{5.146087in}{2.471313in}}%
\pgfpathlineto{\pgfqpoint{5.146276in}{2.473204in}}%
\pgfpathlineto{\pgfqpoint{5.146655in}{2.463683in}}%
\pgfpathlineto{\pgfqpoint{5.148077in}{2.442975in}}%
\pgfpathlineto{\pgfqpoint{5.148266in}{2.447631in}}%
\pgfpathlineto{\pgfqpoint{5.148550in}{2.455283in}}%
\pgfpathlineto{\pgfqpoint{5.149024in}{2.434102in}}%
\pgfpathlineto{\pgfqpoint{5.150256in}{2.420363in}}%
\pgfpathlineto{\pgfqpoint{5.150446in}{2.422505in}}%
\pgfpathlineto{\pgfqpoint{5.151109in}{2.413028in}}%
\pgfpathlineto{\pgfqpoint{5.151488in}{2.402876in}}%
\pgfpathlineto{\pgfqpoint{5.152246in}{2.410007in}}%
\pgfpathlineto{\pgfqpoint{5.153478in}{2.397229in}}%
\pgfpathlineto{\pgfqpoint{5.153762in}{2.406206in}}%
\pgfpathlineto{\pgfqpoint{5.153952in}{2.409549in}}%
\pgfpathlineto{\pgfqpoint{5.154331in}{2.387586in}}%
\pgfpathlineto{\pgfqpoint{5.155089in}{2.377596in}}%
\pgfpathlineto{\pgfqpoint{5.154615in}{2.387754in}}%
\pgfpathlineto{\pgfqpoint{5.155373in}{2.385853in}}%
\pgfpathlineto{\pgfqpoint{5.156605in}{2.413027in}}%
\pgfpathlineto{\pgfqpoint{5.156795in}{2.410771in}}%
\pgfpathlineto{\pgfqpoint{5.157079in}{2.403907in}}%
\pgfpathlineto{\pgfqpoint{5.157458in}{2.412486in}}%
\pgfpathlineto{\pgfqpoint{5.157932in}{2.408952in}}%
\pgfpathlineto{\pgfqpoint{5.158880in}{2.447430in}}%
\pgfpathlineto{\pgfqpoint{5.159353in}{2.422672in}}%
\pgfpathlineto{\pgfqpoint{5.159448in}{2.421979in}}%
\pgfpathlineto{\pgfqpoint{5.159638in}{2.426551in}}%
\pgfpathlineto{\pgfqpoint{5.159733in}{2.428475in}}%
\pgfpathlineto{\pgfqpoint{5.160017in}{2.418837in}}%
\pgfpathlineto{\pgfqpoint{5.161059in}{2.372948in}}%
\pgfpathlineto{\pgfqpoint{5.161628in}{2.378502in}}%
\pgfpathlineto{\pgfqpoint{5.162102in}{2.402191in}}%
\pgfpathlineto{\pgfqpoint{5.162670in}{2.376163in}}%
\pgfpathlineto{\pgfqpoint{5.163807in}{2.360035in}}%
\pgfpathlineto{\pgfqpoint{5.164092in}{2.360768in}}%
\pgfpathlineto{\pgfqpoint{5.165039in}{2.390589in}}%
\pgfpathlineto{\pgfqpoint{5.165229in}{2.397762in}}%
\pgfpathlineto{\pgfqpoint{5.165797in}{2.373388in}}%
\pgfpathlineto{\pgfqpoint{5.167408in}{2.356585in}}%
\pgfpathlineto{\pgfqpoint{5.167787in}{2.365363in}}%
\pgfpathlineto{\pgfqpoint{5.168546in}{2.456645in}}%
\pgfpathlineto{\pgfqpoint{5.169493in}{2.644593in}}%
\pgfpathlineto{\pgfqpoint{5.170157in}{2.600185in}}%
\pgfpathlineto{\pgfqpoint{5.170725in}{2.519565in}}%
\pgfpathlineto{\pgfqpoint{5.171389in}{2.311459in}}%
\pgfpathlineto{\pgfqpoint{5.172147in}{2.387119in}}%
\pgfpathlineto{\pgfqpoint{5.172431in}{2.378595in}}%
\pgfpathlineto{\pgfqpoint{5.172810in}{2.402809in}}%
\pgfpathlineto{\pgfqpoint{5.174610in}{2.433865in}}%
\pgfpathlineto{\pgfqpoint{5.173189in}{2.398906in}}%
\pgfpathlineto{\pgfqpoint{5.174705in}{2.432622in}}%
\pgfpathlineto{\pgfqpoint{5.175463in}{2.394508in}}%
\pgfpathlineto{\pgfqpoint{5.176032in}{2.417220in}}%
\pgfpathlineto{\pgfqpoint{5.177359in}{2.444706in}}%
\pgfpathlineto{\pgfqpoint{5.176695in}{2.414000in}}%
\pgfpathlineto{\pgfqpoint{5.177548in}{2.440065in}}%
\pgfpathlineto{\pgfqpoint{5.178496in}{2.398589in}}%
\pgfpathlineto{\pgfqpoint{5.178970in}{2.418486in}}%
\pgfpathlineto{\pgfqpoint{5.179349in}{2.429738in}}%
\pgfpathlineto{\pgfqpoint{5.180486in}{2.450061in}}%
\pgfpathlineto{\pgfqpoint{5.179823in}{2.428941in}}%
\pgfpathlineto{\pgfqpoint{5.180865in}{2.442975in}}%
\pgfpathlineto{\pgfqpoint{5.181054in}{2.443223in}}%
\pgfpathlineto{\pgfqpoint{5.181149in}{2.442578in}}%
\pgfpathlineto{\pgfqpoint{5.181623in}{2.413321in}}%
\pgfpathlineto{\pgfqpoint{5.182097in}{2.444299in}}%
\pgfpathlineto{\pgfqpoint{5.183992in}{2.466673in}}%
\pgfpathlineto{\pgfqpoint{5.184182in}{2.460036in}}%
\pgfpathlineto{\pgfqpoint{5.184561in}{2.444823in}}%
\pgfpathlineto{\pgfqpoint{5.185224in}{2.460843in}}%
\pgfpathlineto{\pgfqpoint{5.185793in}{2.475462in}}%
\pgfpathlineto{\pgfqpoint{5.186646in}{2.468165in}}%
\pgfpathlineto{\pgfqpoint{5.187972in}{2.457256in}}%
\pgfpathlineto{\pgfqpoint{5.188067in}{2.458541in}}%
\pgfpathlineto{\pgfqpoint{5.189015in}{2.471766in}}%
\pgfpathlineto{\pgfqpoint{5.188636in}{2.457595in}}%
\pgfpathlineto{\pgfqpoint{5.189299in}{2.461226in}}%
\pgfpathlineto{\pgfqpoint{5.189394in}{2.457696in}}%
\pgfpathlineto{\pgfqpoint{5.189868in}{2.478399in}}%
\pgfpathlineto{\pgfqpoint{5.190247in}{2.466483in}}%
\pgfpathlineto{\pgfqpoint{5.190436in}{2.469479in}}%
\pgfpathlineto{\pgfqpoint{5.190626in}{2.474871in}}%
\pgfpathlineto{\pgfqpoint{5.191099in}{2.462704in}}%
\pgfpathlineto{\pgfqpoint{5.191479in}{2.467491in}}%
\pgfpathlineto{\pgfqpoint{5.193848in}{2.433000in}}%
\pgfpathlineto{\pgfqpoint{5.194890in}{2.422729in}}%
\pgfpathlineto{\pgfqpoint{5.194416in}{2.435163in}}%
\pgfpathlineto{\pgfqpoint{5.195174in}{2.428416in}}%
\pgfpathlineto{\pgfqpoint{5.195364in}{2.432773in}}%
\pgfpathlineto{\pgfqpoint{5.195743in}{2.416460in}}%
\pgfpathlineto{\pgfqpoint{5.195932in}{2.412795in}}%
\pgfpathlineto{\pgfqpoint{5.196311in}{2.436192in}}%
\pgfpathlineto{\pgfqpoint{5.197259in}{2.406191in}}%
\pgfpathlineto{\pgfqpoint{5.197449in}{2.398379in}}%
\pgfpathlineto{\pgfqpoint{5.197828in}{2.412781in}}%
\pgfpathlineto{\pgfqpoint{5.198302in}{2.405720in}}%
\pgfpathlineto{\pgfqpoint{5.198965in}{2.421165in}}%
\pgfpathlineto{\pgfqpoint{5.199818in}{2.416670in}}%
\pgfpathlineto{\pgfqpoint{5.200671in}{2.394768in}}%
\pgfpathlineto{\pgfqpoint{5.201050in}{2.407030in}}%
\pgfpathlineto{\pgfqpoint{5.201144in}{2.407587in}}%
\pgfpathlineto{\pgfqpoint{5.201239in}{2.405137in}}%
\pgfpathlineto{\pgfqpoint{5.203608in}{2.349655in}}%
\pgfpathlineto{\pgfqpoint{5.203703in}{2.350910in}}%
\pgfpathlineto{\pgfqpoint{5.204556in}{2.396390in}}%
\pgfpathlineto{\pgfqpoint{5.205598in}{2.426442in}}%
\pgfpathlineto{\pgfqpoint{5.206072in}{2.426088in}}%
\pgfpathlineto{\pgfqpoint{5.206167in}{2.426174in}}%
\pgfpathlineto{\pgfqpoint{5.209768in}{2.358651in}}%
\pgfpathlineto{\pgfqpoint{5.209863in}{2.359329in}}%
\pgfpathlineto{\pgfqpoint{5.210905in}{2.414651in}}%
\pgfpathlineto{\pgfqpoint{5.211853in}{2.394245in}}%
\pgfpathlineto{\pgfqpoint{5.211948in}{2.395424in}}%
\pgfpathlineto{\pgfqpoint{5.212042in}{2.391751in}}%
\pgfpathlineto{\pgfqpoint{5.212327in}{2.369940in}}%
\pgfpathlineto{\pgfqpoint{5.213179in}{2.376907in}}%
\pgfpathlineto{\pgfqpoint{5.214696in}{2.557601in}}%
\pgfpathlineto{\pgfqpoint{5.215359in}{2.670823in}}%
\pgfpathlineto{\pgfqpoint{5.216022in}{2.641728in}}%
\pgfpathlineto{\pgfqpoint{5.216686in}{2.514912in}}%
\pgfpathlineto{\pgfqpoint{5.217160in}{2.360324in}}%
\pgfpathlineto{\pgfqpoint{5.217918in}{2.440860in}}%
\pgfpathlineto{\pgfqpoint{5.218297in}{2.434882in}}%
\pgfpathlineto{\pgfqpoint{5.218486in}{2.438962in}}%
\pgfpathlineto{\pgfqpoint{5.220287in}{2.495617in}}%
\pgfpathlineto{\pgfqpoint{5.220476in}{2.490449in}}%
\pgfpathlineto{\pgfqpoint{5.221140in}{2.452379in}}%
\pgfpathlineto{\pgfqpoint{5.221708in}{2.480443in}}%
\pgfpathlineto{\pgfqpoint{5.221993in}{2.488837in}}%
\pgfpathlineto{\pgfqpoint{5.223035in}{2.517239in}}%
\pgfpathlineto{\pgfqpoint{5.223319in}{2.509300in}}%
\pgfpathlineto{\pgfqpoint{5.224362in}{2.480434in}}%
\pgfpathlineto{\pgfqpoint{5.224835in}{2.502930in}}%
\pgfpathlineto{\pgfqpoint{5.226257in}{2.542617in}}%
\pgfpathlineto{\pgfqpoint{5.226636in}{2.526985in}}%
\pgfpathlineto{\pgfqpoint{5.227489in}{2.502902in}}%
\pgfpathlineto{\pgfqpoint{5.227773in}{2.518023in}}%
\pgfpathlineto{\pgfqpoint{5.228626in}{2.546474in}}%
\pgfpathlineto{\pgfqpoint{5.229100in}{2.543256in}}%
\pgfpathlineto{\pgfqpoint{5.229479in}{2.562953in}}%
\pgfpathlineto{\pgfqpoint{5.230237in}{2.551249in}}%
\pgfpathlineto{\pgfqpoint{5.230521in}{2.543987in}}%
\pgfpathlineto{\pgfqpoint{5.231374in}{2.547522in}}%
\pgfpathlineto{\pgfqpoint{5.232701in}{2.568890in}}%
\pgfpathlineto{\pgfqpoint{5.232796in}{2.567058in}}%
\pgfpathlineto{\pgfqpoint{5.233269in}{2.543655in}}%
\pgfpathlineto{\pgfqpoint{5.234122in}{2.552519in}}%
\pgfpathlineto{\pgfqpoint{5.234596in}{2.570106in}}%
\pgfpathlineto{\pgfqpoint{5.235354in}{2.560001in}}%
\pgfpathlineto{\pgfqpoint{5.236207in}{2.549758in}}%
\pgfpathlineto{\pgfqpoint{5.236491in}{2.541180in}}%
\pgfpathlineto{\pgfqpoint{5.237250in}{2.547810in}}%
\pgfpathlineto{\pgfqpoint{5.237439in}{2.552103in}}%
\pgfpathlineto{\pgfqpoint{5.238008in}{2.540118in}}%
\pgfpathlineto{\pgfqpoint{5.238955in}{2.516176in}}%
\pgfpathlineto{\pgfqpoint{5.240377in}{2.495160in}}%
\pgfpathlineto{\pgfqpoint{5.240472in}{2.495321in}}%
\pgfpathlineto{\pgfqpoint{5.240566in}{2.495454in}}%
\pgfpathlineto{\pgfqpoint{5.240661in}{2.494130in}}%
\pgfpathlineto{\pgfqpoint{5.241609in}{2.471725in}}%
\pgfpathlineto{\pgfqpoint{5.242367in}{2.473855in}}%
\pgfpathlineto{\pgfqpoint{5.242556in}{2.474968in}}%
\pgfpathlineto{\pgfqpoint{5.242651in}{2.473867in}}%
\pgfpathlineto{\pgfqpoint{5.243125in}{2.456926in}}%
\pgfpathlineto{\pgfqpoint{5.244073in}{2.461347in}}%
\pgfpathlineto{\pgfqpoint{5.245589in}{2.484101in}}%
\pgfpathlineto{\pgfqpoint{5.245684in}{2.480735in}}%
\pgfpathlineto{\pgfqpoint{5.246063in}{2.455636in}}%
\pgfpathlineto{\pgfqpoint{5.246821in}{2.470223in}}%
\pgfpathlineto{\pgfqpoint{5.247768in}{2.490155in}}%
\pgfpathlineto{\pgfqpoint{5.248337in}{2.483472in}}%
\pgfpathlineto{\pgfqpoint{5.248527in}{2.484591in}}%
\pgfpathlineto{\pgfqpoint{5.250327in}{2.528316in}}%
\pgfpathlineto{\pgfqpoint{5.250611in}{2.517252in}}%
\pgfpathlineto{\pgfqpoint{5.252791in}{2.441778in}}%
\pgfpathlineto{\pgfqpoint{5.254023in}{2.460336in}}%
\pgfpathlineto{\pgfqpoint{5.255444in}{2.425153in}}%
\pgfpathlineto{\pgfqpoint{5.256108in}{2.443185in}}%
\pgfpathlineto{\pgfqpoint{5.256392in}{2.458449in}}%
\pgfpathlineto{\pgfqpoint{5.256676in}{2.477580in}}%
\pgfpathlineto{\pgfqpoint{5.257340in}{2.441641in}}%
\pgfpathlineto{\pgfqpoint{5.258192in}{2.444950in}}%
\pgfpathlineto{\pgfqpoint{5.258382in}{2.441752in}}%
\pgfpathlineto{\pgfqpoint{5.258666in}{2.433295in}}%
\pgfpathlineto{\pgfqpoint{5.259140in}{2.447410in}}%
\pgfpathlineto{\pgfqpoint{5.259519in}{2.438856in}}%
\pgfpathlineto{\pgfqpoint{5.259898in}{2.469643in}}%
\pgfpathlineto{\pgfqpoint{5.261130in}{2.699640in}}%
\pgfpathlineto{\pgfqpoint{5.261888in}{2.648864in}}%
\pgfpathlineto{\pgfqpoint{5.262552in}{2.468360in}}%
\pgfpathlineto{\pgfqpoint{5.262931in}{2.376924in}}%
\pgfpathlineto{\pgfqpoint{5.263689in}{2.453118in}}%
\pgfpathlineto{\pgfqpoint{5.263973in}{2.438179in}}%
\pgfpathlineto{\pgfqpoint{5.264826in}{2.448398in}}%
\pgfpathlineto{\pgfqpoint{5.265963in}{2.489618in}}%
\pgfpathlineto{\pgfqpoint{5.266437in}{2.476315in}}%
\pgfpathlineto{\pgfqpoint{5.266911in}{2.436174in}}%
\pgfpathlineto{\pgfqpoint{5.267858in}{2.449877in}}%
\pgfpathlineto{\pgfqpoint{5.269090in}{2.486562in}}%
\pgfpathlineto{\pgfqpoint{5.269375in}{2.482914in}}%
\pgfpathlineto{\pgfqpoint{5.269659in}{2.475684in}}%
\pgfpathlineto{\pgfqpoint{5.270038in}{2.452125in}}%
\pgfpathlineto{\pgfqpoint{5.270607in}{2.487678in}}%
\pgfpathlineto{\pgfqpoint{5.270986in}{2.495923in}}%
\pgfpathlineto{\pgfqpoint{5.272028in}{2.519820in}}%
\pgfpathlineto{\pgfqpoint{5.272407in}{2.508938in}}%
\pgfpathlineto{\pgfqpoint{5.274776in}{2.449673in}}%
\pgfpathlineto{\pgfqpoint{5.275534in}{2.468148in}}%
\pgfpathlineto{\pgfqpoint{5.277145in}{2.548007in}}%
\pgfpathlineto{\pgfqpoint{5.277524in}{2.536281in}}%
\pgfpathlineto{\pgfqpoint{5.278567in}{2.515894in}}%
\pgfpathlineto{\pgfqpoint{5.278093in}{2.542870in}}%
\pgfpathlineto{\pgfqpoint{5.278946in}{2.527857in}}%
\pgfpathlineto{\pgfqpoint{5.279325in}{2.518093in}}%
\pgfpathlineto{\pgfqpoint{5.279988in}{2.526200in}}%
\pgfpathlineto{\pgfqpoint{5.280178in}{2.530141in}}%
\pgfpathlineto{\pgfqpoint{5.280557in}{2.516819in}}%
\pgfpathlineto{\pgfqpoint{5.280652in}{2.514953in}}%
\pgfpathlineto{\pgfqpoint{5.281031in}{2.527690in}}%
\pgfpathlineto{\pgfqpoint{5.281410in}{2.520251in}}%
\pgfpathlineto{\pgfqpoint{5.281694in}{2.526088in}}%
\pgfpathlineto{\pgfqpoint{5.282073in}{2.516231in}}%
\pgfpathlineto{\pgfqpoint{5.283305in}{2.500981in}}%
\pgfpathlineto{\pgfqpoint{5.282926in}{2.517251in}}%
\pgfpathlineto{\pgfqpoint{5.283494in}{2.506642in}}%
\pgfpathlineto{\pgfqpoint{5.283589in}{2.510164in}}%
\pgfpathlineto{\pgfqpoint{5.284063in}{2.490793in}}%
\pgfpathlineto{\pgfqpoint{5.284158in}{2.491425in}}%
\pgfpathlineto{\pgfqpoint{5.284253in}{2.492919in}}%
\pgfpathlineto{\pgfqpoint{5.284537in}{2.483989in}}%
\pgfpathlineto{\pgfqpoint{5.285958in}{2.465902in}}%
\pgfpathlineto{\pgfqpoint{5.288233in}{2.412220in}}%
\pgfpathlineto{\pgfqpoint{5.289180in}{2.421027in}}%
\pgfpathlineto{\pgfqpoint{5.289465in}{2.427285in}}%
\pgfpathlineto{\pgfqpoint{5.290128in}{2.418211in}}%
\pgfpathlineto{\pgfqpoint{5.291928in}{2.388147in}}%
\pgfpathlineto{\pgfqpoint{5.290697in}{2.425055in}}%
\pgfpathlineto{\pgfqpoint{5.292213in}{2.394057in}}%
\pgfpathlineto{\pgfqpoint{5.292592in}{2.402152in}}%
\pgfpathlineto{\pgfqpoint{5.294108in}{2.419746in}}%
\pgfpathlineto{\pgfqpoint{5.294582in}{2.401092in}}%
\pgfpathlineto{\pgfqpoint{5.294961in}{2.423161in}}%
\pgfpathlineto{\pgfqpoint{5.295150in}{2.428014in}}%
\pgfpathlineto{\pgfqpoint{5.295624in}{2.412641in}}%
\pgfpathlineto{\pgfqpoint{5.295909in}{2.418636in}}%
\pgfpathlineto{\pgfqpoint{5.296003in}{2.419527in}}%
\pgfpathlineto{\pgfqpoint{5.296193in}{2.413746in}}%
\pgfpathlineto{\pgfqpoint{5.298467in}{2.326858in}}%
\pgfpathlineto{\pgfqpoint{5.298657in}{2.337695in}}%
\pgfpathlineto{\pgfqpoint{5.298941in}{2.359246in}}%
\pgfpathlineto{\pgfqpoint{5.299604in}{2.331064in}}%
\pgfpathlineto{\pgfqpoint{5.300742in}{2.303033in}}%
\pgfpathlineto{\pgfqpoint{5.300931in}{2.308847in}}%
\pgfpathlineto{\pgfqpoint{5.302163in}{2.357243in}}%
\pgfpathlineto{\pgfqpoint{5.302826in}{2.332519in}}%
\pgfpathlineto{\pgfqpoint{5.303016in}{2.334730in}}%
\pgfpathlineto{\pgfqpoint{5.303300in}{2.323723in}}%
\pgfpathlineto{\pgfqpoint{5.304153in}{2.320572in}}%
\pgfpathlineto{\pgfqpoint{5.303774in}{2.330443in}}%
\pgfpathlineto{\pgfqpoint{5.304248in}{2.322625in}}%
\pgfpathlineto{\pgfqpoint{5.305385in}{2.404986in}}%
\pgfpathlineto{\pgfqpoint{5.306617in}{2.611942in}}%
\pgfpathlineto{\pgfqpoint{5.307185in}{2.582620in}}%
\pgfpathlineto{\pgfqpoint{5.307659in}{2.500612in}}%
\pgfpathlineto{\pgfqpoint{5.308417in}{2.301079in}}%
\pgfpathlineto{\pgfqpoint{5.309081in}{2.374210in}}%
\pgfpathlineto{\pgfqpoint{5.309270in}{2.361783in}}%
\pgfpathlineto{\pgfqpoint{5.310028in}{2.382407in}}%
\pgfpathlineto{\pgfqpoint{5.311545in}{2.407966in}}%
\pgfpathlineto{\pgfqpoint{5.311734in}{2.400657in}}%
\pgfpathlineto{\pgfqpoint{5.312018in}{2.381114in}}%
\pgfpathlineto{\pgfqpoint{5.312871in}{2.392769in}}%
\pgfpathlineto{\pgfqpoint{5.314198in}{2.428728in}}%
\pgfpathlineto{\pgfqpoint{5.313250in}{2.386573in}}%
\pgfpathlineto{\pgfqpoint{5.314956in}{2.415916in}}%
\pgfpathlineto{\pgfqpoint{5.315430in}{2.377648in}}%
\pgfpathlineto{\pgfqpoint{5.316093in}{2.405550in}}%
\pgfpathlineto{\pgfqpoint{5.316283in}{2.408387in}}%
\pgfpathlineto{\pgfqpoint{5.317136in}{2.404163in}}%
\pgfpathlineto{\pgfqpoint{5.317515in}{2.421328in}}%
\pgfpathlineto{\pgfqpoint{5.317704in}{2.428556in}}%
\pgfpathlineto{\pgfqpoint{5.318273in}{2.405763in}}%
\pgfpathlineto{\pgfqpoint{5.318557in}{2.396885in}}%
\pgfpathlineto{\pgfqpoint{5.319126in}{2.413788in}}%
\pgfpathlineto{\pgfqpoint{5.320452in}{2.434161in}}%
\pgfpathlineto{\pgfqpoint{5.320642in}{2.429023in}}%
\pgfpathlineto{\pgfqpoint{5.321021in}{2.413944in}}%
\pgfpathlineto{\pgfqpoint{5.321779in}{2.421157in}}%
\pgfpathlineto{\pgfqpoint{5.323959in}{2.444335in}}%
\pgfpathlineto{\pgfqpoint{5.324053in}{2.443290in}}%
\pgfpathlineto{\pgfqpoint{5.324433in}{2.420554in}}%
\pgfpathlineto{\pgfqpoint{5.325191in}{2.437920in}}%
\pgfpathlineto{\pgfqpoint{5.326044in}{2.448608in}}%
\pgfpathlineto{\pgfqpoint{5.326328in}{2.438965in}}%
\pgfpathlineto{\pgfqpoint{5.326517in}{2.435882in}}%
\pgfpathlineto{\pgfqpoint{5.326991in}{2.449718in}}%
\pgfpathlineto{\pgfqpoint{5.327275in}{2.443832in}}%
\pgfpathlineto{\pgfqpoint{5.329739in}{2.411040in}}%
\pgfpathlineto{\pgfqpoint{5.329834in}{2.412540in}}%
\pgfpathlineto{\pgfqpoint{5.330024in}{2.415377in}}%
\pgfpathlineto{\pgfqpoint{5.330403in}{2.407450in}}%
\pgfpathlineto{\pgfqpoint{5.330687in}{2.409172in}}%
\pgfpathlineto{\pgfqpoint{5.332677in}{2.385860in}}%
\pgfpathlineto{\pgfqpoint{5.332867in}{2.391107in}}%
\pgfpathlineto{\pgfqpoint{5.333056in}{2.397985in}}%
\pgfpathlineto{\pgfqpoint{5.333530in}{2.376079in}}%
\pgfpathlineto{\pgfqpoint{5.334193in}{2.370706in}}%
\pgfpathlineto{\pgfqpoint{5.334478in}{2.376067in}}%
\pgfpathlineto{\pgfqpoint{5.335330in}{2.396002in}}%
\pgfpathlineto{\pgfqpoint{5.336183in}{2.390041in}}%
\pgfpathlineto{\pgfqpoint{5.337036in}{2.365799in}}%
\pgfpathlineto{\pgfqpoint{5.337700in}{2.376202in}}%
\pgfpathlineto{\pgfqpoint{5.339026in}{2.414821in}}%
\pgfpathlineto{\pgfqpoint{5.339405in}{2.414502in}}%
\pgfpathlineto{\pgfqpoint{5.341206in}{2.452999in}}%
\pgfpathlineto{\pgfqpoint{5.339974in}{2.408397in}}%
\pgfpathlineto{\pgfqpoint{5.341585in}{2.443624in}}%
\pgfpathlineto{\pgfqpoint{5.341774in}{2.443784in}}%
\pgfpathlineto{\pgfqpoint{5.341964in}{2.440619in}}%
\pgfpathlineto{\pgfqpoint{5.343291in}{2.383936in}}%
\pgfpathlineto{\pgfqpoint{5.343859in}{2.398863in}}%
\pgfpathlineto{\pgfqpoint{5.344238in}{2.419046in}}%
\pgfpathlineto{\pgfqpoint{5.344807in}{2.396961in}}%
\pgfpathlineto{\pgfqpoint{5.346513in}{2.316332in}}%
\pgfpathlineto{\pgfqpoint{5.347081in}{2.338906in}}%
\pgfpathlineto{\pgfqpoint{5.348692in}{2.374192in}}%
\pgfpathlineto{\pgfqpoint{5.349924in}{2.403979in}}%
\pgfpathlineto{\pgfqpoint{5.350208in}{2.394422in}}%
\pgfpathlineto{\pgfqpoint{5.350303in}{2.393221in}}%
\pgfpathlineto{\pgfqpoint{5.350493in}{2.404105in}}%
\pgfpathlineto{\pgfqpoint{5.352009in}{2.655200in}}%
\pgfpathlineto{\pgfqpoint{5.352767in}{2.595664in}}%
\pgfpathlineto{\pgfqpoint{5.353715in}{2.342470in}}%
\pgfpathlineto{\pgfqpoint{5.355041in}{2.408102in}}%
\pgfpathlineto{\pgfqpoint{5.356842in}{2.469686in}}%
\pgfpathlineto{\pgfqpoint{5.357316in}{2.440965in}}%
\pgfpathlineto{\pgfqpoint{5.357790in}{2.414141in}}%
\pgfpathlineto{\pgfqpoint{5.358548in}{2.431404in}}%
\pgfpathlineto{\pgfqpoint{5.359780in}{2.470340in}}%
\pgfpathlineto{\pgfqpoint{5.360159in}{2.451491in}}%
\pgfpathlineto{\pgfqpoint{5.360917in}{2.414282in}}%
\pgfpathlineto{\pgfqpoint{5.361391in}{2.445723in}}%
\pgfpathlineto{\pgfqpoint{5.362622in}{2.462877in}}%
\pgfpathlineto{\pgfqpoint{5.362812in}{2.468385in}}%
\pgfpathlineto{\pgfqpoint{5.363381in}{2.451083in}}%
\pgfpathlineto{\pgfqpoint{5.363854in}{2.444699in}}%
\pgfpathlineto{\pgfqpoint{5.364233in}{2.453304in}}%
\pgfpathlineto{\pgfqpoint{5.365655in}{2.474552in}}%
\pgfpathlineto{\pgfqpoint{5.365844in}{2.471883in}}%
\pgfpathlineto{\pgfqpoint{5.367266in}{2.434345in}}%
\pgfpathlineto{\pgfqpoint{5.367550in}{2.448560in}}%
\pgfpathlineto{\pgfqpoint{5.368687in}{2.466705in}}%
\pgfpathlineto{\pgfqpoint{5.368782in}{2.465663in}}%
\pgfpathlineto{\pgfqpoint{5.370014in}{2.428729in}}%
\pgfpathlineto{\pgfqpoint{5.370298in}{2.446026in}}%
\pgfpathlineto{\pgfqpoint{5.371151in}{2.456306in}}%
\pgfpathlineto{\pgfqpoint{5.370772in}{2.443684in}}%
\pgfpathlineto{\pgfqpoint{5.371341in}{2.443770in}}%
\pgfpathlineto{\pgfqpoint{5.371530in}{2.434541in}}%
\pgfpathlineto{\pgfqpoint{5.372478in}{2.440203in}}%
\pgfpathlineto{\pgfqpoint{5.372667in}{2.438384in}}%
\pgfpathlineto{\pgfqpoint{5.374468in}{2.419029in}}%
\pgfpathlineto{\pgfqpoint{5.374563in}{2.419449in}}%
\pgfpathlineto{\pgfqpoint{5.375700in}{2.430754in}}%
\pgfpathlineto{\pgfqpoint{5.375226in}{2.410128in}}%
\pgfpathlineto{\pgfqpoint{5.375795in}{2.427458in}}%
\pgfpathlineto{\pgfqpoint{5.376269in}{2.399152in}}%
\pgfpathlineto{\pgfqpoint{5.377216in}{2.404666in}}%
\pgfpathlineto{\pgfqpoint{5.379206in}{2.374625in}}%
\pgfpathlineto{\pgfqpoint{5.377785in}{2.405015in}}%
\pgfpathlineto{\pgfqpoint{5.379396in}{2.377823in}}%
\pgfpathlineto{\pgfqpoint{5.380912in}{2.399794in}}%
\pgfpathlineto{\pgfqpoint{5.381101in}{2.395013in}}%
\pgfpathlineto{\pgfqpoint{5.381954in}{2.396209in}}%
\pgfpathlineto{\pgfqpoint{5.382428in}{2.382676in}}%
\pgfpathlineto{\pgfqpoint{5.382712in}{2.388112in}}%
\pgfpathlineto{\pgfqpoint{5.384323in}{2.422109in}}%
\pgfpathlineto{\pgfqpoint{5.385271in}{2.419876in}}%
\pgfpathlineto{\pgfqpoint{5.385745in}{2.413419in}}%
\pgfpathlineto{\pgfqpoint{5.386029in}{2.424651in}}%
\pgfpathlineto{\pgfqpoint{5.386503in}{2.454794in}}%
\pgfpathlineto{\pgfqpoint{5.387166in}{2.433356in}}%
\pgfpathlineto{\pgfqpoint{5.391052in}{2.350024in}}%
\pgfpathlineto{\pgfqpoint{5.391241in}{2.352602in}}%
\pgfpathlineto{\pgfqpoint{5.393231in}{2.400299in}}%
\pgfpathlineto{\pgfqpoint{5.391810in}{2.351997in}}%
\pgfpathlineto{\pgfqpoint{5.393421in}{2.391802in}}%
\pgfpathlineto{\pgfqpoint{5.394842in}{2.352223in}}%
\pgfpathlineto{\pgfqpoint{5.395032in}{2.349615in}}%
\pgfpathlineto{\pgfqpoint{5.395411in}{2.359454in}}%
\pgfpathlineto{\pgfqpoint{5.395979in}{2.382513in}}%
\pgfpathlineto{\pgfqpoint{5.396927in}{2.594645in}}%
\pgfpathlineto{\pgfqpoint{5.397496in}{2.636221in}}%
\pgfpathlineto{\pgfqpoint{5.397970in}{2.594258in}}%
\pgfpathlineto{\pgfqpoint{5.398822in}{2.409407in}}%
\pgfpathlineto{\pgfqpoint{5.399201in}{2.321141in}}%
\pgfpathlineto{\pgfqpoint{5.399960in}{2.389567in}}%
\pgfpathlineto{\pgfqpoint{5.400244in}{2.368006in}}%
\pgfpathlineto{\pgfqpoint{5.400718in}{2.398249in}}%
\pgfpathlineto{\pgfqpoint{5.400907in}{2.396882in}}%
\pgfpathlineto{\pgfqpoint{5.401855in}{2.423106in}}%
\pgfpathlineto{\pgfqpoint{5.402139in}{2.432063in}}%
\pgfpathlineto{\pgfqpoint{5.402802in}{2.413788in}}%
\pgfpathlineto{\pgfqpoint{5.403182in}{2.392872in}}%
\pgfpathlineto{\pgfqpoint{5.403845in}{2.415237in}}%
\pgfpathlineto{\pgfqpoint{5.404793in}{2.444540in}}%
\pgfpathlineto{\pgfqpoint{5.405551in}{2.427768in}}%
\pgfpathlineto{\pgfqpoint{5.406404in}{2.391862in}}%
\pgfpathlineto{\pgfqpoint{5.406783in}{2.419182in}}%
\pgfpathlineto{\pgfqpoint{5.408204in}{2.454678in}}%
\pgfpathlineto{\pgfqpoint{5.407162in}{2.412340in}}%
\pgfpathlineto{\pgfqpoint{5.408394in}{2.445143in}}%
\pgfpathlineto{\pgfqpoint{5.409341in}{2.438039in}}%
\pgfpathlineto{\pgfqpoint{5.408867in}{2.457540in}}%
\pgfpathlineto{\pgfqpoint{5.409531in}{2.440870in}}%
\pgfpathlineto{\pgfqpoint{5.409720in}{2.440121in}}%
\pgfpathlineto{\pgfqpoint{5.409910in}{2.442864in}}%
\pgfpathlineto{\pgfqpoint{5.410668in}{2.468759in}}%
\pgfpathlineto{\pgfqpoint{5.411426in}{2.466227in}}%
\pgfpathlineto{\pgfqpoint{5.412563in}{2.452181in}}%
\pgfpathlineto{\pgfqpoint{5.412753in}{2.456996in}}%
\pgfpathlineto{\pgfqpoint{5.413890in}{2.470381in}}%
\pgfpathlineto{\pgfqpoint{5.413985in}{2.467367in}}%
\pgfpathlineto{\pgfqpoint{5.414269in}{2.452381in}}%
\pgfpathlineto{\pgfqpoint{5.415217in}{2.454430in}}%
\pgfpathlineto{\pgfqpoint{5.416638in}{2.483959in}}%
\pgfpathlineto{\pgfqpoint{5.416922in}{2.477816in}}%
\pgfpathlineto{\pgfqpoint{5.418154in}{2.464099in}}%
\pgfpathlineto{\pgfqpoint{5.417680in}{2.479106in}}%
\pgfpathlineto{\pgfqpoint{5.418249in}{2.465926in}}%
\pgfpathlineto{\pgfqpoint{5.418439in}{2.470473in}}%
\pgfpathlineto{\pgfqpoint{5.418912in}{2.454953in}}%
\pgfpathlineto{\pgfqpoint{5.419102in}{2.454793in}}%
\pgfpathlineto{\pgfqpoint{5.420429in}{2.399242in}}%
\pgfpathlineto{\pgfqpoint{5.420808in}{2.386758in}}%
\pgfpathlineto{\pgfqpoint{5.421661in}{2.389502in}}%
\pgfpathlineto{\pgfqpoint{5.421850in}{2.391769in}}%
\pgfpathlineto{\pgfqpoint{5.423745in}{2.449480in}}%
\pgfpathlineto{\pgfqpoint{5.423935in}{2.442684in}}%
\pgfpathlineto{\pgfqpoint{5.425072in}{2.416161in}}%
\pgfpathlineto{\pgfqpoint{5.425356in}{2.427035in}}%
\pgfpathlineto{\pgfqpoint{5.425641in}{2.440376in}}%
\pgfpathlineto{\pgfqpoint{5.426494in}{2.438072in}}%
\pgfpathlineto{\pgfqpoint{5.428010in}{2.416465in}}%
\pgfpathlineto{\pgfqpoint{5.428768in}{2.432296in}}%
\pgfpathlineto{\pgfqpoint{5.430379in}{2.451154in}}%
\pgfpathlineto{\pgfqpoint{5.430474in}{2.450701in}}%
\pgfpathlineto{\pgfqpoint{5.430947in}{2.452290in}}%
\pgfpathlineto{\pgfqpoint{5.432085in}{2.476710in}}%
\pgfpathlineto{\pgfqpoint{5.432369in}{2.464393in}}%
\pgfpathlineto{\pgfqpoint{5.434075in}{2.413718in}}%
\pgfpathlineto{\pgfqpoint{5.434264in}{2.416334in}}%
\pgfpathlineto{\pgfqpoint{5.435212in}{2.434140in}}%
\pgfpathlineto{\pgfqpoint{5.434738in}{2.410195in}}%
\pgfpathlineto{\pgfqpoint{5.435496in}{2.421095in}}%
\pgfpathlineto{\pgfqpoint{5.436444in}{2.386030in}}%
\pgfpathlineto{\pgfqpoint{5.437107in}{2.386470in}}%
\pgfpathlineto{\pgfqpoint{5.437297in}{2.389960in}}%
\pgfpathlineto{\pgfqpoint{5.438434in}{2.431617in}}%
\pgfpathlineto{\pgfqpoint{5.438813in}{2.413971in}}%
\pgfpathlineto{\pgfqpoint{5.440140in}{2.401907in}}%
\pgfpathlineto{\pgfqpoint{5.440424in}{2.399155in}}%
\pgfpathlineto{\pgfqpoint{5.440613in}{2.402139in}}%
\pgfpathlineto{\pgfqpoint{5.441182in}{2.393562in}}%
\pgfpathlineto{\pgfqpoint{5.441561in}{2.458214in}}%
\pgfpathlineto{\pgfqpoint{5.442793in}{2.670696in}}%
\pgfpathlineto{\pgfqpoint{5.443267in}{2.629503in}}%
\pgfpathlineto{\pgfqpoint{5.443741in}{2.580066in}}%
\pgfpathlineto{\pgfqpoint{5.444499in}{2.341184in}}%
\pgfpathlineto{\pgfqpoint{5.445352in}{2.411318in}}%
\pgfpathlineto{\pgfqpoint{5.445636in}{2.407629in}}%
\pgfpathlineto{\pgfqpoint{5.445920in}{2.417099in}}%
\pgfpathlineto{\pgfqpoint{5.447531in}{2.456442in}}%
\pgfpathlineto{\pgfqpoint{5.447626in}{2.456759in}}%
\pgfpathlineto{\pgfqpoint{5.448289in}{2.421544in}}%
\pgfpathlineto{\pgfqpoint{5.449142in}{2.436052in}}%
\pgfpathlineto{\pgfqpoint{5.450564in}{2.477265in}}%
\pgfpathlineto{\pgfqpoint{5.451037in}{2.464533in}}%
\pgfpathlineto{\pgfqpoint{5.451322in}{2.451098in}}%
\pgfpathlineto{\pgfqpoint{5.451606in}{2.425079in}}%
\pgfpathlineto{\pgfqpoint{5.452175in}{2.460916in}}%
\pgfpathlineto{\pgfqpoint{5.452364in}{2.456538in}}%
\pgfpathlineto{\pgfqpoint{5.454165in}{2.481251in}}%
\pgfpathlineto{\pgfqpoint{5.454354in}{2.475614in}}%
\pgfpathlineto{\pgfqpoint{5.454638in}{2.463446in}}%
\pgfpathlineto{\pgfqpoint{5.455302in}{2.474953in}}%
\pgfpathlineto{\pgfqpoint{5.456534in}{2.513693in}}%
\pgfpathlineto{\pgfqpoint{5.456723in}{2.504043in}}%
\pgfpathlineto{\pgfqpoint{5.457860in}{2.485679in}}%
\pgfpathlineto{\pgfqpoint{5.457387in}{2.506865in}}%
\pgfpathlineto{\pgfqpoint{5.457955in}{2.489250in}}%
\pgfpathlineto{\pgfqpoint{5.459282in}{2.520194in}}%
\pgfpathlineto{\pgfqpoint{5.459471in}{2.518122in}}%
\pgfpathlineto{\pgfqpoint{5.460609in}{2.501626in}}%
\pgfpathlineto{\pgfqpoint{5.460040in}{2.524039in}}%
\pgfpathlineto{\pgfqpoint{5.461272in}{2.507011in}}%
\pgfpathlineto{\pgfqpoint{5.461746in}{2.523024in}}%
\pgfpathlineto{\pgfqpoint{5.462409in}{2.509316in}}%
\pgfpathlineto{\pgfqpoint{5.462693in}{2.512609in}}%
\pgfpathlineto{\pgfqpoint{5.463072in}{2.506014in}}%
\pgfpathlineto{\pgfqpoint{5.463262in}{2.503775in}}%
\pgfpathlineto{\pgfqpoint{5.463641in}{2.513697in}}%
\pgfpathlineto{\pgfqpoint{5.463736in}{2.515893in}}%
\pgfpathlineto{\pgfqpoint{5.464020in}{2.505202in}}%
\pgfpathlineto{\pgfqpoint{5.465062in}{2.483292in}}%
\pgfpathlineto{\pgfqpoint{5.465347in}{2.487407in}}%
\pgfpathlineto{\pgfqpoint{5.465442in}{2.487326in}}%
\pgfpathlineto{\pgfqpoint{5.466105in}{2.460454in}}%
\pgfpathlineto{\pgfqpoint{5.467147in}{2.446195in}}%
\pgfpathlineto{\pgfqpoint{5.466768in}{2.466087in}}%
\pgfpathlineto{\pgfqpoint{5.467337in}{2.448661in}}%
\pgfpathlineto{\pgfqpoint{5.467432in}{2.450742in}}%
\pgfpathlineto{\pgfqpoint{5.467905in}{2.436769in}}%
\pgfpathlineto{\pgfqpoint{5.468758in}{2.418101in}}%
\pgfpathlineto{\pgfqpoint{5.469137in}{2.431831in}}%
\pgfpathlineto{\pgfqpoint{5.469232in}{2.434237in}}%
\pgfpathlineto{\pgfqpoint{5.469706in}{2.419257in}}%
\pgfpathlineto{\pgfqpoint{5.469895in}{2.423383in}}%
\pgfpathlineto{\pgfqpoint{5.469990in}{2.423852in}}%
\pgfpathlineto{\pgfqpoint{5.470085in}{2.421396in}}%
\pgfpathlineto{\pgfqpoint{5.470275in}{2.414950in}}%
\pgfpathlineto{\pgfqpoint{5.470748in}{2.426896in}}%
\pgfpathlineto{\pgfqpoint{5.471222in}{2.420070in}}%
\pgfpathlineto{\pgfqpoint{5.471412in}{2.422068in}}%
\pgfpathlineto{\pgfqpoint{5.472075in}{2.450278in}}%
\pgfpathlineto{\pgfqpoint{5.473023in}{2.442356in}}%
\pgfpathlineto{\pgfqpoint{5.473307in}{2.430198in}}%
\pgfpathlineto{\pgfqpoint{5.473781in}{2.450952in}}%
\pgfpathlineto{\pgfqpoint{5.473970in}{2.449873in}}%
\pgfpathlineto{\pgfqpoint{5.477287in}{2.511873in}}%
\pgfpathlineto{\pgfqpoint{5.477571in}{2.503921in}}%
\pgfpathlineto{\pgfqpoint{5.479656in}{2.427508in}}%
\pgfpathlineto{\pgfqpoint{5.479751in}{2.429039in}}%
\pgfpathlineto{\pgfqpoint{5.480699in}{2.442314in}}%
\pgfpathlineto{\pgfqpoint{5.480888in}{2.435423in}}%
\pgfpathlineto{\pgfqpoint{5.482120in}{2.400979in}}%
\pgfpathlineto{\pgfqpoint{5.482310in}{2.402077in}}%
\pgfpathlineto{\pgfqpoint{5.482594in}{2.399999in}}%
\pgfpathlineto{\pgfqpoint{5.482783in}{2.405150in}}%
\pgfpathlineto{\pgfqpoint{5.483636in}{2.438432in}}%
\pgfpathlineto{\pgfqpoint{5.483921in}{2.410678in}}%
\pgfpathlineto{\pgfqpoint{5.484300in}{2.419011in}}%
\pgfpathlineto{\pgfqpoint{5.485532in}{2.380277in}}%
\pgfpathlineto{\pgfqpoint{5.486858in}{2.461360in}}%
\pgfpathlineto{\pgfqpoint{5.488185in}{2.650428in}}%
\pgfpathlineto{\pgfqpoint{5.488469in}{2.619775in}}%
\pgfpathlineto{\pgfqpoint{5.489701in}{2.338848in}}%
\pgfpathlineto{\pgfqpoint{5.491407in}{2.394189in}}%
\pgfpathlineto{\pgfqpoint{5.492165in}{2.385959in}}%
\pgfpathlineto{\pgfqpoint{5.493586in}{2.335708in}}%
\pgfpathlineto{\pgfqpoint{5.493871in}{2.346114in}}%
\pgfpathlineto{\pgfqpoint{5.495861in}{2.468589in}}%
\pgfpathlineto{\pgfqpoint{5.496429in}{2.441529in}}%
\pgfpathlineto{\pgfqpoint{5.496903in}{2.421482in}}%
\pgfpathlineto{\pgfqpoint{5.497756in}{2.432002in}}%
\pgfpathlineto{\pgfqpoint{5.499083in}{2.453410in}}%
\pgfpathlineto{\pgfqpoint{5.499178in}{2.453244in}}%
\pgfpathlineto{\pgfqpoint{5.500315in}{2.429573in}}%
\pgfpathlineto{\pgfqpoint{5.500599in}{2.444390in}}%
\pgfpathlineto{\pgfqpoint{5.501831in}{2.468329in}}%
\pgfpathlineto{\pgfqpoint{5.501926in}{2.467442in}}%
\pgfpathlineto{\pgfqpoint{5.503252in}{2.435909in}}%
\pgfpathlineto{\pgfqpoint{5.503537in}{2.453855in}}%
\pgfpathlineto{\pgfqpoint{5.503726in}{2.464755in}}%
\pgfpathlineto{\pgfqpoint{5.504200in}{2.449581in}}%
\pgfpathlineto{\pgfqpoint{5.504579in}{2.453820in}}%
\pgfpathlineto{\pgfqpoint{5.504958in}{2.461992in}}%
\pgfpathlineto{\pgfqpoint{5.505716in}{2.458548in}}%
\pgfpathlineto{\pgfqpoint{5.506001in}{2.446050in}}%
\pgfpathlineto{\pgfqpoint{5.506759in}{2.454704in}}%
\pgfpathlineto{\pgfqpoint{5.507612in}{2.465311in}}%
\pgfpathlineto{\pgfqpoint{5.507233in}{2.450709in}}%
\pgfpathlineto{\pgfqpoint{5.507801in}{2.457059in}}%
\pgfpathlineto{\pgfqpoint{5.507991in}{2.450834in}}%
\pgfpathlineto{\pgfqpoint{5.508749in}{2.460926in}}%
\pgfpathlineto{\pgfqpoint{5.508938in}{2.465210in}}%
\pgfpathlineto{\pgfqpoint{5.509602in}{2.454023in}}%
\pgfpathlineto{\pgfqpoint{5.509981in}{2.440542in}}%
\pgfpathlineto{\pgfqpoint{5.510644in}{2.457544in}}%
\pgfpathlineto{\pgfqpoint{5.511023in}{2.449427in}}%
\pgfpathlineto{\pgfqpoint{5.514150in}{2.397141in}}%
\pgfpathlineto{\pgfqpoint{5.514435in}{2.402442in}}%
\pgfpathlineto{\pgfqpoint{5.514719in}{2.411387in}}%
\pgfpathlineto{\pgfqpoint{5.515193in}{2.395171in}}%
\pgfpathlineto{\pgfqpoint{5.516046in}{2.382724in}}%
\pgfpathlineto{\pgfqpoint{5.516425in}{2.387198in}}%
\pgfpathlineto{\pgfqpoint{5.517751in}{2.401386in}}%
\pgfpathlineto{\pgfqpoint{5.517846in}{2.399835in}}%
\pgfpathlineto{\pgfqpoint{5.518983in}{2.373812in}}%
\pgfpathlineto{\pgfqpoint{5.518509in}{2.404502in}}%
\pgfpathlineto{\pgfqpoint{5.519268in}{2.384202in}}%
\pgfpathlineto{\pgfqpoint{5.520784in}{2.403708in}}%
\pgfpathlineto{\pgfqpoint{5.520879in}{2.403646in}}%
\pgfpathlineto{\pgfqpoint{5.521163in}{2.399910in}}%
\pgfpathlineto{\pgfqpoint{5.521542in}{2.408662in}}%
\pgfpathlineto{\pgfqpoint{5.523437in}{2.454940in}}%
\pgfpathlineto{\pgfqpoint{5.523627in}{2.440416in}}%
\pgfpathlineto{\pgfqpoint{5.525332in}{2.362391in}}%
\pgfpathlineto{\pgfqpoint{5.525427in}{2.361989in}}%
\pgfpathlineto{\pgfqpoint{5.525617in}{2.365362in}}%
\pgfpathlineto{\pgfqpoint{5.526185in}{2.391777in}}%
\pgfpathlineto{\pgfqpoint{5.526943in}{2.379007in}}%
\pgfpathlineto{\pgfqpoint{5.527607in}{2.350867in}}%
\pgfpathlineto{\pgfqpoint{5.528460in}{2.355836in}}%
\pgfpathlineto{\pgfqpoint{5.529502in}{2.416181in}}%
\pgfpathlineto{\pgfqpoint{5.530260in}{2.383263in}}%
\pgfpathlineto{\pgfqpoint{5.531303in}{2.369234in}}%
\pgfpathlineto{\pgfqpoint{5.531587in}{2.379240in}}%
\pgfpathlineto{\pgfqpoint{5.532155in}{2.375346in}}%
\pgfpathlineto{\pgfqpoint{5.532724in}{2.441050in}}%
\pgfpathlineto{\pgfqpoint{5.533956in}{2.653762in}}%
\pgfpathlineto{\pgfqpoint{5.534430in}{2.601578in}}%
\pgfpathlineto{\pgfqpoint{5.535093in}{2.489393in}}%
\pgfpathlineto{\pgfqpoint{5.535567in}{2.337894in}}%
\pgfpathlineto{\pgfqpoint{5.536325in}{2.400701in}}%
\pgfpathlineto{\pgfqpoint{5.536704in}{2.394356in}}%
\pgfpathlineto{\pgfqpoint{5.537841in}{2.424353in}}%
\pgfpathlineto{\pgfqpoint{5.538599in}{2.446808in}}%
\pgfpathlineto{\pgfqpoint{5.538979in}{2.431133in}}%
\pgfpathlineto{\pgfqpoint{5.539926in}{2.404910in}}%
\pgfpathlineto{\pgfqpoint{5.540210in}{2.418410in}}%
\pgfpathlineto{\pgfqpoint{5.541821in}{2.459261in}}%
\pgfpathlineto{\pgfqpoint{5.542864in}{2.424420in}}%
\pgfpathlineto{\pgfqpoint{5.543243in}{2.446958in}}%
\pgfpathlineto{\pgfqpoint{5.544949in}{2.487355in}}%
\pgfpathlineto{\pgfqpoint{5.545043in}{2.481914in}}%
\pgfpathlineto{\pgfqpoint{5.545991in}{2.450488in}}%
\pgfpathlineto{\pgfqpoint{5.546275in}{2.464980in}}%
\pgfpathlineto{\pgfqpoint{5.547223in}{2.488575in}}%
\pgfpathlineto{\pgfqpoint{5.547792in}{2.483896in}}%
\pgfpathlineto{\pgfqpoint{5.549023in}{2.461575in}}%
\pgfpathlineto{\pgfqpoint{5.549213in}{2.468109in}}%
\pgfpathlineto{\pgfqpoint{5.549592in}{2.485580in}}%
\pgfpathlineto{\pgfqpoint{5.550445in}{2.482010in}}%
\pgfpathlineto{\pgfqpoint{5.551014in}{2.476967in}}%
\pgfpathlineto{\pgfqpoint{5.552530in}{2.461225in}}%
\pgfpathlineto{\pgfqpoint{5.552625in}{2.462420in}}%
\pgfpathlineto{\pgfqpoint{5.553098in}{2.484195in}}%
\pgfpathlineto{\pgfqpoint{5.553856in}{2.471723in}}%
\pgfpathlineto{\pgfqpoint{5.553951in}{2.469304in}}%
\pgfpathlineto{\pgfqpoint{5.554804in}{2.474094in}}%
\pgfpathlineto{\pgfqpoint{5.554899in}{2.474470in}}%
\pgfpathlineto{\pgfqpoint{5.554994in}{2.473048in}}%
\pgfpathlineto{\pgfqpoint{5.556320in}{2.452166in}}%
\pgfpathlineto{\pgfqpoint{5.556510in}{2.458057in}}%
\pgfpathlineto{\pgfqpoint{5.556794in}{2.471211in}}%
\pgfpathlineto{\pgfqpoint{5.557268in}{2.455392in}}%
\pgfpathlineto{\pgfqpoint{5.557552in}{2.455911in}}%
\pgfpathlineto{\pgfqpoint{5.558405in}{2.446576in}}%
\pgfpathlineto{\pgfqpoint{5.558689in}{2.451058in}}%
\pgfpathlineto{\pgfqpoint{5.558879in}{2.456098in}}%
\pgfpathlineto{\pgfqpoint{5.559353in}{2.443292in}}%
\pgfpathlineto{\pgfqpoint{5.559732in}{2.454914in}}%
\pgfpathlineto{\pgfqpoint{5.561059in}{2.375625in}}%
\pgfpathlineto{\pgfqpoint{5.562006in}{2.354729in}}%
\pgfpathlineto{\pgfqpoint{5.562290in}{2.360802in}}%
\pgfpathlineto{\pgfqpoint{5.562670in}{2.341957in}}%
\pgfpathlineto{\pgfqpoint{5.563143in}{2.367753in}}%
\pgfpathlineto{\pgfqpoint{5.565323in}{2.431686in}}%
\pgfpathlineto{\pgfqpoint{5.566460in}{2.449512in}}%
\pgfpathlineto{\pgfqpoint{5.565986in}{2.424908in}}%
\pgfpathlineto{\pgfqpoint{5.566650in}{2.441237in}}%
\pgfpathlineto{\pgfqpoint{5.567692in}{2.429625in}}%
\pgfpathlineto{\pgfqpoint{5.567787in}{2.431588in}}%
\pgfpathlineto{\pgfqpoint{5.568829in}{2.462275in}}%
\pgfpathlineto{\pgfqpoint{5.569019in}{2.453365in}}%
\pgfpathlineto{\pgfqpoint{5.570724in}{2.388704in}}%
\pgfpathlineto{\pgfqpoint{5.571483in}{2.410395in}}%
\pgfpathlineto{\pgfqpoint{5.571767in}{2.419242in}}%
\pgfpathlineto{\pgfqpoint{5.572430in}{2.403692in}}%
\pgfpathlineto{\pgfqpoint{5.573473in}{2.385385in}}%
\pgfpathlineto{\pgfqpoint{5.573757in}{2.394615in}}%
\pgfpathlineto{\pgfqpoint{5.574705in}{2.418982in}}%
\pgfpathlineto{\pgfqpoint{5.574989in}{2.434603in}}%
\pgfpathlineto{\pgfqpoint{5.575463in}{2.401291in}}%
\pgfpathlineto{\pgfqpoint{5.576126in}{2.391430in}}%
\pgfpathlineto{\pgfqpoint{5.576695in}{2.399204in}}%
\pgfpathlineto{\pgfqpoint{5.577547in}{2.394822in}}%
\pgfpathlineto{\pgfqpoint{5.578211in}{2.480604in}}%
\pgfpathlineto{\pgfqpoint{5.579064in}{2.671088in}}%
\pgfpathlineto{\pgfqpoint{5.579917in}{2.644098in}}%
\pgfpathlineto{\pgfqpoint{5.580485in}{2.534644in}}%
\pgfpathlineto{\pgfqpoint{5.581149in}{2.348360in}}%
\pgfpathlineto{\pgfqpoint{5.581907in}{2.414738in}}%
\pgfpathlineto{\pgfqpoint{5.582286in}{2.413415in}}%
\pgfpathlineto{\pgfqpoint{5.582380in}{2.414886in}}%
\pgfpathlineto{\pgfqpoint{5.584086in}{2.450467in}}%
\pgfpathlineto{\pgfqpoint{5.584276in}{2.445937in}}%
\pgfpathlineto{\pgfqpoint{5.585223in}{2.399305in}}%
\pgfpathlineto{\pgfqpoint{5.585792in}{2.419103in}}%
\pgfpathlineto{\pgfqpoint{5.586550in}{2.426611in}}%
\pgfpathlineto{\pgfqpoint{5.587024in}{2.447025in}}%
\pgfpathlineto{\pgfqpoint{5.587592in}{2.426437in}}%
\pgfpathlineto{\pgfqpoint{5.588066in}{2.406500in}}%
\pgfpathlineto{\pgfqpoint{5.588256in}{2.398472in}}%
\pgfpathlineto{\pgfqpoint{5.588730in}{2.426146in}}%
\pgfpathlineto{\pgfqpoint{5.590435in}{2.459094in}}%
\pgfpathlineto{\pgfqpoint{5.590814in}{2.446985in}}%
\pgfpathlineto{\pgfqpoint{5.591383in}{2.431933in}}%
\pgfpathlineto{\pgfqpoint{5.591667in}{2.446425in}}%
\pgfpathlineto{\pgfqpoint{5.592615in}{2.466839in}}%
\pgfpathlineto{\pgfqpoint{5.592899in}{2.463523in}}%
\pgfpathlineto{\pgfqpoint{5.593278in}{2.467319in}}%
\pgfpathlineto{\pgfqpoint{5.593468in}{2.463863in}}%
\pgfpathlineto{\pgfqpoint{5.594510in}{2.451802in}}%
\pgfpathlineto{\pgfqpoint{5.594131in}{2.467063in}}%
\pgfpathlineto{\pgfqpoint{5.594700in}{2.456866in}}%
\pgfpathlineto{\pgfqpoint{5.595268in}{2.481469in}}%
\pgfpathlineto{\pgfqpoint{5.596026in}{2.474608in}}%
\pgfpathlineto{\pgfqpoint{5.596406in}{2.478029in}}%
\pgfpathlineto{\pgfqpoint{5.596690in}{2.472590in}}%
\pgfpathlineto{\pgfqpoint{5.596879in}{2.468128in}}%
\pgfpathlineto{\pgfqpoint{5.597353in}{2.482513in}}%
\pgfpathlineto{\pgfqpoint{5.597637in}{2.477379in}}%
\pgfpathlineto{\pgfqpoint{5.598680in}{2.493184in}}%
\pgfpathlineto{\pgfqpoint{5.598964in}{2.485225in}}%
\pgfpathlineto{\pgfqpoint{5.599059in}{2.482239in}}%
\pgfpathlineto{\pgfqpoint{5.599533in}{2.497814in}}%
\pgfpathlineto{\pgfqpoint{5.599628in}{2.497343in}}%
\pgfpathlineto{\pgfqpoint{5.600291in}{2.490885in}}%
\pgfpathlineto{\pgfqpoint{5.600386in}{2.493769in}}%
\pgfpathlineto{\pgfqpoint{5.600670in}{2.509935in}}%
\pgfpathlineto{\pgfqpoint{5.601618in}{2.502648in}}%
\pgfpathlineto{\pgfqpoint{5.601807in}{2.503484in}}%
\pgfpathlineto{\pgfqpoint{5.601997in}{2.501254in}}%
\pgfpathlineto{\pgfqpoint{5.602565in}{2.504564in}}%
\pgfpathlineto{\pgfqpoint{5.603702in}{2.486631in}}%
\pgfpathlineto{\pgfqpoint{5.604555in}{2.477119in}}%
\pgfpathlineto{\pgfqpoint{5.605692in}{2.461702in}}%
\pgfpathlineto{\pgfqpoint{5.605882in}{2.462314in}}%
\pgfpathlineto{\pgfqpoint{5.606071in}{2.463583in}}%
\pgfpathlineto{\pgfqpoint{5.606356in}{2.473760in}}%
\pgfpathlineto{\pgfqpoint{5.606640in}{2.457985in}}%
\pgfpathlineto{\pgfqpoint{5.606924in}{2.442757in}}%
\pgfpathlineto{\pgfqpoint{5.607682in}{2.456242in}}%
\pgfpathlineto{\pgfqpoint{5.608062in}{2.453562in}}%
\pgfpathlineto{\pgfqpoint{5.608346in}{2.463950in}}%
\pgfpathlineto{\pgfqpoint{5.608535in}{2.473690in}}%
\pgfpathlineto{\pgfqpoint{5.609009in}{2.453773in}}%
\pgfpathlineto{\pgfqpoint{5.609388in}{2.461549in}}%
\pgfpathlineto{\pgfqpoint{5.610241in}{2.441882in}}%
\pgfpathlineto{\pgfqpoint{5.610999in}{2.453108in}}%
\pgfpathlineto{\pgfqpoint{5.612136in}{2.468825in}}%
\pgfpathlineto{\pgfqpoint{5.612326in}{2.464245in}}%
\pgfpathlineto{\pgfqpoint{5.612610in}{2.458524in}}%
\pgfpathlineto{\pgfqpoint{5.613179in}{2.470175in}}%
\pgfpathlineto{\pgfqpoint{5.614126in}{2.508192in}}%
\pgfpathlineto{\pgfqpoint{5.614979in}{2.497655in}}%
\pgfpathlineto{\pgfqpoint{5.615548in}{2.466793in}}%
\pgfpathlineto{\pgfqpoint{5.616590in}{2.418094in}}%
\pgfpathlineto{\pgfqpoint{5.616969in}{2.425611in}}%
\pgfpathlineto{\pgfqpoint{5.617443in}{2.437091in}}%
\pgfpathlineto{\pgfqpoint{5.617822in}{2.424056in}}%
\pgfpathlineto{\pgfqpoint{5.618486in}{2.383977in}}%
\pgfpathlineto{\pgfqpoint{5.619433in}{2.388249in}}%
\pgfpathlineto{\pgfqpoint{5.620381in}{2.430113in}}%
\pgfpathlineto{\pgfqpoint{5.620760in}{2.407257in}}%
\pgfpathlineto{\pgfqpoint{5.621708in}{2.393899in}}%
\pgfpathlineto{\pgfqpoint{5.622087in}{2.395906in}}%
\pgfpathlineto{\pgfqpoint{5.622940in}{2.399820in}}%
\pgfpathlineto{\pgfqpoint{5.623034in}{2.398292in}}%
\pgfpathlineto{\pgfqpoint{5.623224in}{2.393438in}}%
\pgfpathlineto{\pgfqpoint{5.623508in}{2.419716in}}%
\pgfpathlineto{\pgfqpoint{5.624930in}{2.672517in}}%
\pgfpathlineto{\pgfqpoint{5.625593in}{2.628538in}}%
\pgfpathlineto{\pgfqpoint{5.626067in}{2.519563in}}%
\pgfpathlineto{\pgfqpoint{5.626635in}{2.352514in}}%
\pgfpathlineto{\pgfqpoint{5.627393in}{2.403327in}}%
\pgfpathlineto{\pgfqpoint{5.628720in}{2.366351in}}%
\pgfpathlineto{\pgfqpoint{5.628910in}{2.370159in}}%
\pgfpathlineto{\pgfqpoint{5.631184in}{2.448794in}}%
\pgfpathlineto{\pgfqpoint{5.631279in}{2.446862in}}%
\pgfpathlineto{\pgfqpoint{5.631468in}{2.438831in}}%
\pgfpathlineto{\pgfqpoint{5.631942in}{2.456936in}}%
\pgfpathlineto{\pgfqpoint{5.632132in}{2.455585in}}%
\pgfpathlineto{\pgfqpoint{5.632511in}{2.471337in}}%
\pgfpathlineto{\pgfqpoint{5.633174in}{2.455656in}}%
\pgfpathlineto{\pgfqpoint{5.633743in}{2.425351in}}%
\pgfpathlineto{\pgfqpoint{5.634311in}{2.453081in}}%
\pgfpathlineto{\pgfqpoint{5.635164in}{2.476722in}}%
\pgfpathlineto{\pgfqpoint{5.636112in}{2.487891in}}%
\pgfpathlineto{\pgfqpoint{5.635733in}{2.467234in}}%
\pgfpathlineto{\pgfqpoint{5.636206in}{2.484274in}}%
\pgfpathlineto{\pgfqpoint{5.636491in}{2.464737in}}%
\pgfpathlineto{\pgfqpoint{5.637344in}{2.479579in}}%
\pgfpathlineto{\pgfqpoint{5.637533in}{2.480779in}}%
\pgfpathlineto{\pgfqpoint{5.639239in}{2.515600in}}%
\pgfpathlineto{\pgfqpoint{5.639523in}{2.505819in}}%
\pgfpathlineto{\pgfqpoint{5.639713in}{2.497815in}}%
\pgfpathlineto{\pgfqpoint{5.640281in}{2.517583in}}%
\pgfpathlineto{\pgfqpoint{5.640471in}{2.513098in}}%
\pgfpathlineto{\pgfqpoint{5.640660in}{2.510363in}}%
\pgfpathlineto{\pgfqpoint{5.640945in}{2.523400in}}%
\pgfpathlineto{\pgfqpoint{5.641134in}{2.530154in}}%
\pgfpathlineto{\pgfqpoint{5.641892in}{2.517287in}}%
\pgfpathlineto{\pgfqpoint{5.642082in}{2.518257in}}%
\pgfpathlineto{\pgfqpoint{5.642556in}{2.515606in}}%
\pgfpathlineto{\pgfqpoint{5.643124in}{2.503208in}}%
\pgfpathlineto{\pgfqpoint{5.643788in}{2.511046in}}%
\pgfpathlineto{\pgfqpoint{5.645304in}{2.522691in}}%
\pgfpathlineto{\pgfqpoint{5.644356in}{2.509417in}}%
\pgfpathlineto{\pgfqpoint{5.645588in}{2.519676in}}%
\pgfpathlineto{\pgfqpoint{5.645872in}{2.521423in}}%
\pgfpathlineto{\pgfqpoint{5.645967in}{2.519638in}}%
\pgfpathlineto{\pgfqpoint{5.647104in}{2.504215in}}%
\pgfpathlineto{\pgfqpoint{5.646631in}{2.525859in}}%
\pgfpathlineto{\pgfqpoint{5.647389in}{2.506357in}}%
\pgfpathlineto{\pgfqpoint{5.647862in}{2.507853in}}%
\pgfpathlineto{\pgfqpoint{5.648147in}{2.499198in}}%
\pgfpathlineto{\pgfqpoint{5.649853in}{2.460506in}}%
\pgfpathlineto{\pgfqpoint{5.649947in}{2.459309in}}%
\pgfpathlineto{\pgfqpoint{5.650232in}{2.468504in}}%
\pgfpathlineto{\pgfqpoint{5.650326in}{2.470371in}}%
\pgfpathlineto{\pgfqpoint{5.650611in}{2.459387in}}%
\pgfpathlineto{\pgfqpoint{5.652032in}{2.443231in}}%
\pgfpathlineto{\pgfqpoint{5.652790in}{2.433774in}}%
\pgfpathlineto{\pgfqpoint{5.653264in}{2.439705in}}%
\pgfpathlineto{\pgfqpoint{5.653548in}{2.441572in}}%
\pgfpathlineto{\pgfqpoint{5.654591in}{2.454144in}}%
\pgfpathlineto{\pgfqpoint{5.654780in}{2.450142in}}%
\pgfpathlineto{\pgfqpoint{5.655823in}{2.433625in}}%
\pgfpathlineto{\pgfqpoint{5.655349in}{2.456148in}}%
\pgfpathlineto{\pgfqpoint{5.656012in}{2.439025in}}%
\pgfpathlineto{\pgfqpoint{5.658571in}{2.481018in}}%
\pgfpathlineto{\pgfqpoint{5.658760in}{2.479820in}}%
\pgfpathlineto{\pgfqpoint{5.659045in}{2.475187in}}%
\pgfpathlineto{\pgfqpoint{5.659329in}{2.483386in}}%
\pgfpathlineto{\pgfqpoint{5.660277in}{2.517721in}}%
\pgfpathlineto{\pgfqpoint{5.660561in}{2.503440in}}%
\pgfpathlineto{\pgfqpoint{5.662267in}{2.416559in}}%
\pgfpathlineto{\pgfqpoint{5.662646in}{2.431373in}}%
\pgfpathlineto{\pgfqpoint{5.663404in}{2.450009in}}%
\pgfpathlineto{\pgfqpoint{5.663783in}{2.438983in}}%
\pgfpathlineto{\pgfqpoint{5.665110in}{2.403396in}}%
\pgfpathlineto{\pgfqpoint{5.665299in}{2.406737in}}%
\pgfpathlineto{\pgfqpoint{5.666626in}{2.456223in}}%
\pgfpathlineto{\pgfqpoint{5.667194in}{2.427005in}}%
\pgfpathlineto{\pgfqpoint{5.667479in}{2.426441in}}%
\pgfpathlineto{\pgfqpoint{5.667573in}{2.425693in}}%
\pgfpathlineto{\pgfqpoint{5.667763in}{2.428959in}}%
\pgfpathlineto{\pgfqpoint{5.668711in}{2.440620in}}%
\pgfpathlineto{\pgfqpoint{5.668332in}{2.427502in}}%
\pgfpathlineto{\pgfqpoint{5.668900in}{2.432898in}}%
\pgfpathlineto{\pgfqpoint{5.669184in}{2.418836in}}%
\pgfpathlineto{\pgfqpoint{5.669469in}{2.445358in}}%
\pgfpathlineto{\pgfqpoint{5.670890in}{2.689876in}}%
\pgfpathlineto{\pgfqpoint{5.671459in}{2.648119in}}%
\pgfpathlineto{\pgfqpoint{5.671743in}{2.629510in}}%
\pgfpathlineto{\pgfqpoint{5.672596in}{2.369251in}}%
\pgfpathlineto{\pgfqpoint{5.673638in}{2.441592in}}%
\pgfpathlineto{\pgfqpoint{5.675628in}{2.486385in}}%
\pgfpathlineto{\pgfqpoint{5.675913in}{2.478643in}}%
\pgfpathlineto{\pgfqpoint{5.677239in}{2.420492in}}%
\pgfpathlineto{\pgfqpoint{5.677429in}{2.427984in}}%
\pgfpathlineto{\pgfqpoint{5.678471in}{2.474808in}}%
\pgfpathlineto{\pgfqpoint{5.678850in}{2.455901in}}%
\pgfpathlineto{\pgfqpoint{5.679703in}{2.433715in}}%
\pgfpathlineto{\pgfqpoint{5.680082in}{2.447664in}}%
\pgfpathlineto{\pgfqpoint{5.680935in}{2.478845in}}%
\pgfpathlineto{\pgfqpoint{5.681693in}{2.472309in}}%
\pgfpathlineto{\pgfqpoint{5.681788in}{2.472193in}}%
\pgfpathlineto{\pgfqpoint{5.683115in}{2.439167in}}%
\pgfpathlineto{\pgfqpoint{5.683494in}{2.455263in}}%
\pgfpathlineto{\pgfqpoint{5.684820in}{2.476004in}}%
\pgfpathlineto{\pgfqpoint{5.685010in}{2.473438in}}%
\pgfpathlineto{\pgfqpoint{5.685863in}{2.447633in}}%
\pgfpathlineto{\pgfqpoint{5.686621in}{2.454087in}}%
\pgfpathlineto{\pgfqpoint{5.686905in}{2.457875in}}%
\pgfpathlineto{\pgfqpoint{5.687379in}{2.448645in}}%
\pgfpathlineto{\pgfqpoint{5.688516in}{2.425967in}}%
\pgfpathlineto{\pgfqpoint{5.688042in}{2.458187in}}%
\pgfpathlineto{\pgfqpoint{5.688611in}{2.430101in}}%
\pgfpathlineto{\pgfqpoint{5.689938in}{2.455523in}}%
\pgfpathlineto{\pgfqpoint{5.690032in}{2.454175in}}%
\pgfpathlineto{\pgfqpoint{5.690791in}{2.445938in}}%
\pgfpathlineto{\pgfqpoint{5.691075in}{2.452578in}}%
\pgfpathlineto{\pgfqpoint{5.691170in}{2.455368in}}%
\pgfpathlineto{\pgfqpoint{5.691643in}{2.441727in}}%
\pgfpathlineto{\pgfqpoint{5.692023in}{2.450035in}}%
\pgfpathlineto{\pgfqpoint{5.693160in}{2.427997in}}%
\pgfpathlineto{\pgfqpoint{5.695339in}{2.355126in}}%
\pgfpathlineto{\pgfqpoint{5.695434in}{2.356078in}}%
\pgfpathlineto{\pgfqpoint{5.696856in}{2.401915in}}%
\pgfpathlineto{\pgfqpoint{5.697519in}{2.412626in}}%
\pgfpathlineto{\pgfqpoint{5.697898in}{2.403801in}}%
\pgfpathlineto{\pgfqpoint{5.699130in}{2.380922in}}%
\pgfpathlineto{\pgfqpoint{5.699509in}{2.381896in}}%
\pgfpathlineto{\pgfqpoint{5.699983in}{2.392737in}}%
\pgfpathlineto{\pgfqpoint{5.700457in}{2.380844in}}%
\pgfpathlineto{\pgfqpoint{5.700551in}{2.379532in}}%
\pgfpathlineto{\pgfqpoint{5.700741in}{2.385227in}}%
\pgfpathlineto{\pgfqpoint{5.701025in}{2.401531in}}%
\pgfpathlineto{\pgfqpoint{5.701594in}{2.373207in}}%
\pgfpathlineto{\pgfqpoint{5.701688in}{2.375320in}}%
\pgfpathlineto{\pgfqpoint{5.701878in}{2.378429in}}%
\pgfpathlineto{\pgfqpoint{5.702447in}{2.368131in}}%
\pgfpathlineto{\pgfqpoint{5.702541in}{2.367627in}}%
\pgfpathlineto{\pgfqpoint{5.702731in}{2.369679in}}%
\pgfpathlineto{\pgfqpoint{5.704626in}{2.420263in}}%
\pgfpathlineto{\pgfqpoint{5.705005in}{2.407664in}}%
\pgfpathlineto{\pgfqpoint{5.705574in}{2.410299in}}%
\pgfpathlineto{\pgfqpoint{5.706521in}{2.449094in}}%
\pgfpathlineto{\pgfqpoint{5.707185in}{2.434269in}}%
\pgfpathlineto{\pgfqpoint{5.708227in}{2.381331in}}%
\pgfpathlineto{\pgfqpoint{5.708891in}{2.393920in}}%
\pgfpathlineto{\pgfqpoint{5.709270in}{2.417982in}}%
\pgfpathlineto{\pgfqpoint{5.710122in}{2.404775in}}%
\pgfpathlineto{\pgfqpoint{5.711354in}{2.376979in}}%
\pgfpathlineto{\pgfqpoint{5.711544in}{2.382502in}}%
\pgfpathlineto{\pgfqpoint{5.712776in}{2.422578in}}%
\pgfpathlineto{\pgfqpoint{5.713250in}{2.397793in}}%
\pgfpathlineto{\pgfqpoint{5.714292in}{2.385617in}}%
\pgfpathlineto{\pgfqpoint{5.713724in}{2.407750in}}%
\pgfpathlineto{\pgfqpoint{5.714576in}{2.387116in}}%
\pgfpathlineto{\pgfqpoint{5.714766in}{2.388466in}}%
\pgfpathlineto{\pgfqpoint{5.715050in}{2.393299in}}%
\pgfpathlineto{\pgfqpoint{5.715335in}{2.377137in}}%
\pgfpathlineto{\pgfqpoint{5.715429in}{2.375214in}}%
\pgfpathlineto{\pgfqpoint{5.715524in}{2.381910in}}%
\pgfpathlineto{\pgfqpoint{5.717135in}{2.652744in}}%
\pgfpathlineto{\pgfqpoint{5.717893in}{2.590767in}}%
\pgfpathlineto{\pgfqpoint{5.718841in}{2.339106in}}%
\pgfpathlineto{\pgfqpoint{5.720167in}{2.396423in}}%
\pgfpathlineto{\pgfqpoint{5.721210in}{2.419131in}}%
\pgfpathlineto{\pgfqpoint{5.721968in}{2.444260in}}%
\pgfpathlineto{\pgfqpoint{5.722347in}{2.426926in}}%
\pgfpathlineto{\pgfqpoint{5.723105in}{2.410741in}}%
\pgfpathlineto{\pgfqpoint{5.723389in}{2.426127in}}%
\pgfpathlineto{\pgfqpoint{5.724811in}{2.460617in}}%
\pgfpathlineto{\pgfqpoint{5.724906in}{2.460300in}}%
\pgfpathlineto{\pgfqpoint{5.725664in}{2.435033in}}%
\pgfpathlineto{\pgfqpoint{5.725948in}{2.422587in}}%
\pgfpathlineto{\pgfqpoint{5.726517in}{2.445381in}}%
\pgfpathlineto{\pgfqpoint{5.728033in}{2.466344in}}%
\pgfpathlineto{\pgfqpoint{5.728128in}{2.465279in}}%
\pgfpathlineto{\pgfqpoint{5.729170in}{2.444101in}}%
\pgfpathlineto{\pgfqpoint{5.729360in}{2.451487in}}%
\pgfpathlineto{\pgfqpoint{5.730781in}{2.482457in}}%
\pgfpathlineto{\pgfqpoint{5.731065in}{2.492246in}}%
\pgfpathlineto{\pgfqpoint{5.731634in}{2.473356in}}%
\pgfpathlineto{\pgfqpoint{5.731823in}{2.467354in}}%
\pgfpathlineto{\pgfqpoint{5.732203in}{2.482109in}}%
\pgfpathlineto{\pgfqpoint{5.732676in}{2.475213in}}%
\pgfpathlineto{\pgfqpoint{5.733055in}{2.502787in}}%
\pgfpathlineto{\pgfqpoint{5.733908in}{2.497958in}}%
\pgfpathlineto{\pgfqpoint{5.734193in}{2.471621in}}%
\pgfpathlineto{\pgfqpoint{5.735140in}{2.481038in}}%
\pgfpathlineto{\pgfqpoint{5.736277in}{2.494063in}}%
\pgfpathlineto{\pgfqpoint{5.735709in}{2.477209in}}%
\pgfpathlineto{\pgfqpoint{5.736467in}{2.486472in}}%
\pgfpathlineto{\pgfqpoint{5.736751in}{2.475218in}}%
\pgfpathlineto{\pgfqpoint{5.737225in}{2.494003in}}%
\pgfpathlineto{\pgfqpoint{5.737699in}{2.477340in}}%
\pgfpathlineto{\pgfqpoint{5.737983in}{2.476685in}}%
\pgfpathlineto{\pgfqpoint{5.739120in}{2.467606in}}%
\pgfpathlineto{\pgfqpoint{5.738646in}{2.481541in}}%
\pgfpathlineto{\pgfqpoint{5.739215in}{2.470904in}}%
\pgfpathlineto{\pgfqpoint{5.739405in}{2.479665in}}%
\pgfpathlineto{\pgfqpoint{5.740068in}{2.462665in}}%
\pgfpathlineto{\pgfqpoint{5.742532in}{2.418990in}}%
\pgfpathlineto{\pgfqpoint{5.742721in}{2.422816in}}%
\pgfpathlineto{\pgfqpoint{5.743385in}{2.425475in}}%
\pgfpathlineto{\pgfqpoint{5.743669in}{2.420794in}}%
\pgfpathlineto{\pgfqpoint{5.745090in}{2.396873in}}%
\pgfpathlineto{\pgfqpoint{5.745280in}{2.406526in}}%
\pgfpathlineto{\pgfqpoint{5.746228in}{2.428575in}}%
\pgfpathlineto{\pgfqpoint{5.746701in}{2.424438in}}%
\pgfpathlineto{\pgfqpoint{5.748028in}{2.409529in}}%
\pgfpathlineto{\pgfqpoint{5.747460in}{2.424805in}}%
\pgfpathlineto{\pgfqpoint{5.748123in}{2.411387in}}%
\pgfpathlineto{\pgfqpoint{5.750776in}{2.447642in}}%
\pgfpathlineto{\pgfqpoint{5.748597in}{2.405226in}}%
\pgfpathlineto{\pgfqpoint{5.750966in}{2.441014in}}%
\pgfpathlineto{\pgfqpoint{5.751250in}{2.427866in}}%
\pgfpathlineto{\pgfqpoint{5.751724in}{2.453585in}}%
\pgfpathlineto{\pgfqpoint{5.752766in}{2.476219in}}%
\pgfpathlineto{\pgfqpoint{5.752956in}{2.468433in}}%
\pgfpathlineto{\pgfqpoint{5.754567in}{2.390354in}}%
\pgfpathlineto{\pgfqpoint{5.754851in}{2.407786in}}%
\pgfpathlineto{\pgfqpoint{5.755230in}{2.435084in}}%
\pgfpathlineto{\pgfqpoint{5.755894in}{2.410023in}}%
\pgfpathlineto{\pgfqpoint{5.757410in}{2.367386in}}%
\pgfpathlineto{\pgfqpoint{5.757505in}{2.367495in}}%
\pgfpathlineto{\pgfqpoint{5.758452in}{2.417080in}}%
\pgfpathlineto{\pgfqpoint{5.759305in}{2.389148in}}%
\pgfpathlineto{\pgfqpoint{5.761390in}{2.314189in}}%
\pgfpathlineto{\pgfqpoint{5.762243in}{2.510290in}}%
\pgfpathlineto{\pgfqpoint{5.763096in}{2.675136in}}%
\pgfpathlineto{\pgfqpoint{5.763664in}{2.650314in}}%
\pgfpathlineto{\pgfqpoint{5.764233in}{2.549602in}}%
\pgfpathlineto{\pgfqpoint{5.764801in}{2.354388in}}%
\pgfpathlineto{\pgfqpoint{5.765654in}{2.417211in}}%
\pgfpathlineto{\pgfqpoint{5.766318in}{2.428653in}}%
\pgfpathlineto{\pgfqpoint{5.767929in}{2.473888in}}%
\pgfpathlineto{\pgfqpoint{5.768308in}{2.448349in}}%
\pgfpathlineto{\pgfqpoint{5.768971in}{2.419470in}}%
\pgfpathlineto{\pgfqpoint{5.769350in}{2.444823in}}%
\pgfpathlineto{\pgfqpoint{5.769919in}{2.443285in}}%
\pgfpathlineto{\pgfqpoint{5.770961in}{2.484141in}}%
\pgfpathlineto{\pgfqpoint{5.771056in}{2.485027in}}%
\pgfpathlineto{\pgfqpoint{5.771245in}{2.479406in}}%
\pgfpathlineto{\pgfqpoint{5.771719in}{2.448238in}}%
\pgfpathlineto{\pgfqpoint{5.772572in}{2.459765in}}%
\pgfpathlineto{\pgfqpoint{5.773709in}{2.490963in}}%
\pgfpathlineto{\pgfqpoint{5.774278in}{2.480799in}}%
\pgfpathlineto{\pgfqpoint{5.775036in}{2.466569in}}%
\pgfpathlineto{\pgfqpoint{5.775415in}{2.478329in}}%
\pgfpathlineto{\pgfqpoint{5.776268in}{2.508753in}}%
\pgfpathlineto{\pgfqpoint{5.777500in}{2.497666in}}%
\pgfpathlineto{\pgfqpoint{5.778163in}{2.495194in}}%
\pgfpathlineto{\pgfqpoint{5.778353in}{2.498121in}}%
\pgfpathlineto{\pgfqpoint{5.779300in}{2.527034in}}%
\pgfpathlineto{\pgfqpoint{5.779585in}{2.506305in}}%
\pgfpathlineto{\pgfqpoint{5.780532in}{2.496237in}}%
\pgfpathlineto{\pgfqpoint{5.780153in}{2.509285in}}%
\pgfpathlineto{\pgfqpoint{5.780722in}{2.500613in}}%
\pgfpathlineto{\pgfqpoint{5.780817in}{2.502475in}}%
\pgfpathlineto{\pgfqpoint{5.781196in}{2.488402in}}%
\pgfpathlineto{\pgfqpoint{5.781385in}{2.484973in}}%
\pgfpathlineto{\pgfqpoint{5.781859in}{2.499071in}}%
\pgfpathlineto{\pgfqpoint{5.782048in}{2.503670in}}%
\pgfpathlineto{\pgfqpoint{5.782333in}{2.485277in}}%
\pgfpathlineto{\pgfqpoint{5.783565in}{2.466533in}}%
\pgfpathlineto{\pgfqpoint{5.783659in}{2.466876in}}%
\pgfpathlineto{\pgfqpoint{5.784702in}{2.471069in}}%
\pgfpathlineto{\pgfqpoint{5.784418in}{2.466520in}}%
\pgfpathlineto{\pgfqpoint{5.784797in}{2.470465in}}%
\pgfpathlineto{\pgfqpoint{5.785744in}{2.460533in}}%
\pgfpathlineto{\pgfqpoint{5.786029in}{2.465622in}}%
\pgfpathlineto{\pgfqpoint{5.786123in}{2.466105in}}%
\pgfpathlineto{\pgfqpoint{5.786218in}{2.464552in}}%
\pgfpathlineto{\pgfqpoint{5.788208in}{2.425265in}}%
\pgfpathlineto{\pgfqpoint{5.788303in}{2.425465in}}%
\pgfpathlineto{\pgfqpoint{5.788587in}{2.427682in}}%
\pgfpathlineto{\pgfqpoint{5.788966in}{2.424060in}}%
\pgfpathlineto{\pgfqpoint{5.789440in}{2.425796in}}%
\pgfpathlineto{\pgfqpoint{5.790767in}{2.406015in}}%
\pgfpathlineto{\pgfqpoint{5.790956in}{2.409039in}}%
\pgfpathlineto{\pgfqpoint{5.791051in}{2.409110in}}%
\pgfpathlineto{\pgfqpoint{5.791335in}{2.400404in}}%
\pgfpathlineto{\pgfqpoint{5.791904in}{2.414302in}}%
\pgfpathlineto{\pgfqpoint{5.791999in}{2.413188in}}%
\pgfpathlineto{\pgfqpoint{5.793894in}{2.384685in}}%
\pgfpathlineto{\pgfqpoint{5.794273in}{2.390535in}}%
\pgfpathlineto{\pgfqpoint{5.796263in}{2.421997in}}%
\pgfpathlineto{\pgfqpoint{5.796358in}{2.420591in}}%
\pgfpathlineto{\pgfqpoint{5.796547in}{2.417949in}}%
\pgfpathlineto{\pgfqpoint{5.796926in}{2.421652in}}%
\pgfpathlineto{\pgfqpoint{5.797211in}{2.421188in}}%
\pgfpathlineto{\pgfqpoint{5.798348in}{2.453223in}}%
\pgfpathlineto{\pgfqpoint{5.798727in}{2.441854in}}%
\pgfpathlineto{\pgfqpoint{5.800527in}{2.390714in}}%
\pgfpathlineto{\pgfqpoint{5.799106in}{2.443528in}}%
\pgfpathlineto{\pgfqpoint{5.801380in}{2.406928in}}%
\pgfpathlineto{\pgfqpoint{5.801570in}{2.409601in}}%
\pgfpathlineto{\pgfqpoint{5.801949in}{2.398726in}}%
\pgfpathlineto{\pgfqpoint{5.803465in}{2.376793in}}%
\pgfpathlineto{\pgfqpoint{5.803560in}{2.377588in}}%
\pgfpathlineto{\pgfqpoint{5.804697in}{2.427662in}}%
\pgfpathlineto{\pgfqpoint{5.805076in}{2.397916in}}%
\pgfpathlineto{\pgfqpoint{5.806024in}{2.384741in}}%
\pgfpathlineto{\pgfqpoint{5.806213in}{2.388550in}}%
\pgfpathlineto{\pgfqpoint{5.806498in}{2.396521in}}%
\pgfpathlineto{\pgfqpoint{5.807066in}{2.380506in}}%
\pgfpathlineto{\pgfqpoint{5.807445in}{2.399017in}}%
\pgfpathlineto{\pgfqpoint{5.808867in}{2.670577in}}%
\pgfpathlineto{\pgfqpoint{5.809814in}{2.590854in}}%
\pgfpathlineto{\pgfqpoint{5.810667in}{2.354255in}}%
\pgfpathlineto{\pgfqpoint{5.811520in}{2.415399in}}%
\pgfpathlineto{\pgfqpoint{5.813510in}{2.466537in}}%
\pgfpathlineto{\pgfqpoint{5.813700in}{2.471337in}}%
\pgfpathlineto{\pgfqpoint{5.813984in}{2.443008in}}%
\pgfpathlineto{\pgfqpoint{5.814837in}{2.410945in}}%
\pgfpathlineto{\pgfqpoint{5.815121in}{2.430918in}}%
\pgfpathlineto{\pgfqpoint{5.816543in}{2.472032in}}%
\pgfpathlineto{\pgfqpoint{5.816637in}{2.471860in}}%
\pgfpathlineto{\pgfqpoint{5.817775in}{2.424534in}}%
\pgfpathlineto{\pgfqpoint{5.818248in}{2.448961in}}%
\pgfpathlineto{\pgfqpoint{5.818817in}{2.448498in}}%
\pgfpathlineto{\pgfqpoint{5.819670in}{2.465913in}}%
\pgfpathlineto{\pgfqpoint{5.819859in}{2.469450in}}%
\pgfpathlineto{\pgfqpoint{5.820428in}{2.457719in}}%
\pgfpathlineto{\pgfqpoint{5.820617in}{2.453313in}}%
\pgfpathlineto{\pgfqpoint{5.821376in}{2.461128in}}%
\pgfpathlineto{\pgfqpoint{5.822228in}{2.486152in}}%
\pgfpathlineto{\pgfqpoint{5.822513in}{2.470453in}}%
\pgfpathlineto{\pgfqpoint{5.824598in}{2.403560in}}%
\pgfpathlineto{\pgfqpoint{5.824787in}{2.409689in}}%
\pgfpathlineto{\pgfqpoint{5.825261in}{2.408327in}}%
\pgfpathlineto{\pgfqpoint{5.828293in}{2.487588in}}%
\pgfpathlineto{\pgfqpoint{5.828767in}{2.471915in}}%
\pgfpathlineto{\pgfqpoint{5.829810in}{2.476041in}}%
\pgfpathlineto{\pgfqpoint{5.830189in}{2.477981in}}%
\pgfpathlineto{\pgfqpoint{5.830378in}{2.476235in}}%
\pgfpathlineto{\pgfqpoint{5.833884in}{2.437133in}}%
\pgfpathlineto{\pgfqpoint{5.834169in}{2.443334in}}%
\pgfpathlineto{\pgfqpoint{5.834643in}{2.430970in}}%
\pgfpathlineto{\pgfqpoint{5.834927in}{2.432362in}}%
\pgfpathlineto{\pgfqpoint{5.835022in}{2.431387in}}%
\pgfpathlineto{\pgfqpoint{5.836538in}{2.395925in}}%
\pgfpathlineto{\pgfqpoint{5.836917in}{2.411151in}}%
\pgfpathlineto{\pgfqpoint{5.837106in}{2.414273in}}%
\pgfpathlineto{\pgfqpoint{5.837770in}{2.408312in}}%
\pgfpathlineto{\pgfqpoint{5.837959in}{2.404256in}}%
\pgfpathlineto{\pgfqpoint{5.838717in}{2.410863in}}%
\pgfpathlineto{\pgfqpoint{5.838907in}{2.415642in}}%
\pgfpathlineto{\pgfqpoint{5.839286in}{2.395802in}}%
\pgfpathlineto{\pgfqpoint{5.840044in}{2.393713in}}%
\pgfpathlineto{\pgfqpoint{5.839665in}{2.403224in}}%
\pgfpathlineto{\pgfqpoint{5.840139in}{2.396472in}}%
\pgfpathlineto{\pgfqpoint{5.841655in}{2.425264in}}%
\pgfpathlineto{\pgfqpoint{5.841845in}{2.420365in}}%
\pgfpathlineto{\pgfqpoint{5.842129in}{2.407701in}}%
\pgfpathlineto{\pgfqpoint{5.842697in}{2.426724in}}%
\pgfpathlineto{\pgfqpoint{5.842887in}{2.424522in}}%
\pgfpathlineto{\pgfqpoint{5.843077in}{2.422289in}}%
\pgfpathlineto{\pgfqpoint{5.843361in}{2.432495in}}%
\pgfpathlineto{\pgfqpoint{5.843740in}{2.445594in}}%
\pgfpathlineto{\pgfqpoint{5.844498in}{2.437535in}}%
\pgfpathlineto{\pgfqpoint{5.845351in}{2.398772in}}%
\pgfpathlineto{\pgfqpoint{5.846014in}{2.356712in}}%
\pgfpathlineto{\pgfqpoint{5.846772in}{2.377082in}}%
\pgfpathlineto{\pgfqpoint{5.847057in}{2.379130in}}%
\pgfpathlineto{\pgfqpoint{5.847246in}{2.375649in}}%
\pgfpathlineto{\pgfqpoint{5.848383in}{2.335684in}}%
\pgfpathlineto{\pgfqpoint{5.848952in}{2.343104in}}%
\pgfpathlineto{\pgfqpoint{5.849994in}{2.387302in}}%
\pgfpathlineto{\pgfqpoint{5.850847in}{2.360264in}}%
\pgfpathlineto{\pgfqpoint{5.851795in}{2.344993in}}%
\pgfpathlineto{\pgfqpoint{5.852079in}{2.355910in}}%
\pgfpathlineto{\pgfqpoint{5.853406in}{2.464793in}}%
\pgfpathlineto{\pgfqpoint{5.854164in}{2.634738in}}%
\pgfpathlineto{\pgfqpoint{5.855017in}{2.596341in}}%
\pgfpathlineto{\pgfqpoint{5.855112in}{2.596699in}}%
\pgfpathlineto{\pgfqpoint{5.855206in}{2.594809in}}%
\pgfpathlineto{\pgfqpoint{5.855680in}{2.456984in}}%
\pgfpathlineto{\pgfqpoint{5.856154in}{2.319521in}}%
\pgfpathlineto{\pgfqpoint{5.856912in}{2.379188in}}%
\pgfpathlineto{\pgfqpoint{5.857386in}{2.373606in}}%
\pgfpathlineto{\pgfqpoint{5.857575in}{2.380097in}}%
\pgfpathlineto{\pgfqpoint{5.859092in}{2.430283in}}%
\pgfpathlineto{\pgfqpoint{5.859281in}{2.425858in}}%
\pgfpathlineto{\pgfqpoint{5.860324in}{2.392649in}}%
\pgfpathlineto{\pgfqpoint{5.860797in}{2.405294in}}%
\pgfpathlineto{\pgfqpoint{5.862219in}{2.449985in}}%
\pgfpathlineto{\pgfqpoint{5.862598in}{2.431966in}}%
\pgfpathlineto{\pgfqpoint{5.863261in}{2.400560in}}%
\pgfpathlineto{\pgfqpoint{5.863640in}{2.429375in}}%
\pgfpathlineto{\pgfqpoint{5.865441in}{2.463232in}}%
\pgfpathlineto{\pgfqpoint{5.864114in}{2.425593in}}%
\pgfpathlineto{\pgfqpoint{5.865630in}{2.461679in}}%
\pgfpathlineto{\pgfqpoint{5.866768in}{2.430068in}}%
\pgfpathlineto{\pgfqpoint{5.867052in}{2.442542in}}%
\pgfpathlineto{\pgfqpoint{5.867999in}{2.468304in}}%
\pgfpathlineto{\pgfqpoint{5.868473in}{2.466216in}}%
\pgfpathlineto{\pgfqpoint{5.869326in}{2.469148in}}%
\pgfpathlineto{\pgfqpoint{5.869042in}{2.460444in}}%
\pgfpathlineto{\pgfqpoint{5.869421in}{2.468482in}}%
\pgfpathlineto{\pgfqpoint{5.869800in}{2.443655in}}%
\pgfpathlineto{\pgfqpoint{5.870274in}{2.478175in}}%
\pgfpathlineto{\pgfqpoint{5.870369in}{2.477772in}}%
\pgfpathlineto{\pgfqpoint{5.870748in}{2.458782in}}%
\pgfpathlineto{\pgfqpoint{5.871601in}{2.468290in}}%
\pgfpathlineto{\pgfqpoint{5.871980in}{2.459573in}}%
\pgfpathlineto{\pgfqpoint{5.872359in}{2.449421in}}%
\pgfpathlineto{\pgfqpoint{5.872832in}{2.465811in}}%
\pgfpathlineto{\pgfqpoint{5.872927in}{2.466107in}}%
\pgfpathlineto{\pgfqpoint{5.873022in}{2.463868in}}%
\pgfpathlineto{\pgfqpoint{5.873212in}{2.461521in}}%
\pgfpathlineto{\pgfqpoint{5.873496in}{2.476002in}}%
\pgfpathlineto{\pgfqpoint{5.873591in}{2.478184in}}%
\pgfpathlineto{\pgfqpoint{5.873970in}{2.462056in}}%
\pgfpathlineto{\pgfqpoint{5.875012in}{2.459963in}}%
\pgfpathlineto{\pgfqpoint{5.874538in}{2.464492in}}%
\pgfpathlineto{\pgfqpoint{5.875107in}{2.460138in}}%
\pgfpathlineto{\pgfqpoint{5.875391in}{2.462686in}}%
\pgfpathlineto{\pgfqpoint{5.875675in}{2.458958in}}%
\pgfpathlineto{\pgfqpoint{5.876718in}{2.445627in}}%
\pgfpathlineto{\pgfqpoint{5.877002in}{2.451937in}}%
\pgfpathlineto{\pgfqpoint{5.877097in}{2.454207in}}%
\pgfpathlineto{\pgfqpoint{5.877476in}{2.438833in}}%
\pgfpathlineto{\pgfqpoint{5.878329in}{2.419086in}}%
\pgfpathlineto{\pgfqpoint{5.878803in}{2.426753in}}%
\pgfpathlineto{\pgfqpoint{5.880603in}{2.399048in}}%
\pgfpathlineto{\pgfqpoint{5.880887in}{2.402750in}}%
\pgfpathlineto{\pgfqpoint{5.881172in}{2.393452in}}%
\pgfpathlineto{\pgfqpoint{5.881930in}{2.376260in}}%
\pgfpathlineto{\pgfqpoint{5.882309in}{2.389130in}}%
\pgfpathlineto{\pgfqpoint{5.883351in}{2.398689in}}%
\pgfpathlineto{\pgfqpoint{5.882783in}{2.386470in}}%
\pgfpathlineto{\pgfqpoint{5.883636in}{2.391504in}}%
\pgfpathlineto{\pgfqpoint{5.885341in}{2.357608in}}%
\pgfpathlineto{\pgfqpoint{5.885531in}{2.371168in}}%
\pgfpathlineto{\pgfqpoint{5.886384in}{2.401197in}}%
\pgfpathlineto{\pgfqpoint{5.886668in}{2.385338in}}%
\pgfpathlineto{\pgfqpoint{5.888089in}{2.346680in}}%
\pgfpathlineto{\pgfqpoint{5.887142in}{2.389405in}}%
\pgfpathlineto{\pgfqpoint{5.888279in}{2.353141in}}%
\pgfpathlineto{\pgfqpoint{5.890553in}{2.412307in}}%
\pgfpathlineto{\pgfqpoint{5.888848in}{2.344002in}}%
\pgfpathlineto{\pgfqpoint{5.890838in}{2.401994in}}%
\pgfpathlineto{\pgfqpoint{5.891691in}{2.379606in}}%
\pgfpathlineto{\pgfqpoint{5.891975in}{2.393669in}}%
\pgfpathlineto{\pgfqpoint{5.892354in}{2.406136in}}%
\pgfpathlineto{\pgfqpoint{5.892922in}{2.389067in}}%
\pgfpathlineto{\pgfqpoint{5.894344in}{2.360428in}}%
\pgfpathlineto{\pgfqpoint{5.894723in}{2.374273in}}%
\pgfpathlineto{\pgfqpoint{5.895765in}{2.404027in}}%
\pgfpathlineto{\pgfqpoint{5.896050in}{2.395513in}}%
\pgfpathlineto{\pgfqpoint{5.897471in}{2.360435in}}%
\pgfpathlineto{\pgfqpoint{5.897566in}{2.360633in}}%
\pgfpathlineto{\pgfqpoint{5.898987in}{2.477652in}}%
\pgfpathlineto{\pgfqpoint{5.900219in}{2.653335in}}%
\pgfpathlineto{\pgfqpoint{5.900693in}{2.608745in}}%
\pgfpathlineto{\pgfqpoint{5.900883in}{2.604085in}}%
\pgfpathlineto{\pgfqpoint{5.901736in}{2.334668in}}%
\pgfpathlineto{\pgfqpoint{5.903157in}{2.418849in}}%
\pgfpathlineto{\pgfqpoint{5.904578in}{2.460266in}}%
\pgfpathlineto{\pgfqpoint{5.904768in}{2.465507in}}%
\pgfpathlineto{\pgfqpoint{5.905147in}{2.442590in}}%
\pgfpathlineto{\pgfqpoint{5.905810in}{2.424382in}}%
\pgfpathlineto{\pgfqpoint{5.906284in}{2.433815in}}%
\pgfpathlineto{\pgfqpoint{5.907800in}{2.463017in}}%
\pgfpathlineto{\pgfqpoint{5.908085in}{2.452780in}}%
\pgfpathlineto{\pgfqpoint{5.909032in}{2.421050in}}%
\pgfpathlineto{\pgfqpoint{5.909317in}{2.430321in}}%
\pgfusepath{stroke}%
\end{pgfscope}%
\begin{pgfscope}%
\pgfsetrectcap%
\pgfsetmiterjoin%
\pgfsetlinewidth{0.803000pt}%
\definecolor{currentstroke}{rgb}{0.000000,0.000000,0.000000}%
\pgfsetstrokecolor{currentstroke}%
\pgfsetdash{}{0pt}%
\pgfpathmoveto{\pgfqpoint{0.934300in}{2.114143in}}%
\pgfpathlineto{\pgfqpoint{0.934300in}{2.901359in}}%
\pgfusepath{stroke}%
\end{pgfscope}%
\begin{pgfscope}%
\pgfsetrectcap%
\pgfsetmiterjoin%
\pgfsetlinewidth{0.803000pt}%
\definecolor{currentstroke}{rgb}{0.000000,0.000000,0.000000}%
\pgfsetstrokecolor{currentstroke}%
\pgfsetdash{}{0pt}%
\pgfpathmoveto{\pgfqpoint{6.146222in}{2.114143in}}%
\pgfpathlineto{\pgfqpoint{6.146222in}{2.901359in}}%
\pgfusepath{stroke}%
\end{pgfscope}%
\begin{pgfscope}%
\pgfsetrectcap%
\pgfsetmiterjoin%
\pgfsetlinewidth{0.803000pt}%
\definecolor{currentstroke}{rgb}{0.000000,0.000000,0.000000}%
\pgfsetstrokecolor{currentstroke}%
\pgfsetdash{}{0pt}%
\pgfpathmoveto{\pgfqpoint{0.934300in}{2.114143in}}%
\pgfpathlineto{\pgfqpoint{6.146222in}{2.114143in}}%
\pgfusepath{stroke}%
\end{pgfscope}%
\begin{pgfscope}%
\pgfsetrectcap%
\pgfsetmiterjoin%
\pgfsetlinewidth{0.803000pt}%
\definecolor{currentstroke}{rgb}{0.000000,0.000000,0.000000}%
\pgfsetstrokecolor{currentstroke}%
\pgfsetdash{}{0pt}%
\pgfpathmoveto{\pgfqpoint{0.934300in}{2.901359in}}%
\pgfpathlineto{\pgfqpoint{6.146222in}{2.901359in}}%
\pgfusepath{stroke}%
\end{pgfscope}%
\begin{pgfscope}%
\definecolor{textcolor}{rgb}{0.000000,0.000000,0.000000}%
\pgfsetstrokecolor{textcolor}%
\pgfsetfillcolor{textcolor}%
\pgftext[x=3.540261in,y=2.984692in,,base]{\color{textcolor}\rmfamily\fontsize{12.000000}{14.400000}\selectfont Filtered ECG Signal}%
\end{pgfscope}%
\begin{pgfscope}%
\pgfsetbuttcap%
\pgfsetmiterjoin%
\definecolor{currentfill}{rgb}{1.000000,1.000000,1.000000}%
\pgfsetfillcolor{currentfill}%
\pgfsetlinewidth{0.000000pt}%
\definecolor{currentstroke}{rgb}{0.000000,0.000000,0.000000}%
\pgfsetstrokecolor{currentstroke}%
\pgfsetstrokeopacity{0.000000}%
\pgfsetdash{}{0pt}%
\pgfpathmoveto{\pgfqpoint{0.934300in}{0.564143in}}%
\pgfpathlineto{\pgfqpoint{6.146222in}{0.564143in}}%
\pgfpathlineto{\pgfqpoint{6.146222in}{1.351359in}}%
\pgfpathlineto{\pgfqpoint{0.934300in}{1.351359in}}%
\pgfpathclose%
\pgfusepath{fill}%
\end{pgfscope}%
\begin{pgfscope}%
\pgfsetbuttcap%
\pgfsetroundjoin%
\definecolor{currentfill}{rgb}{0.000000,0.000000,0.000000}%
\pgfsetfillcolor{currentfill}%
\pgfsetlinewidth{0.803000pt}%
\definecolor{currentstroke}{rgb}{0.000000,0.000000,0.000000}%
\pgfsetstrokecolor{currentstroke}%
\pgfsetdash{}{0pt}%
\pgfsys@defobject{currentmarker}{\pgfqpoint{0.000000in}{-0.048611in}}{\pgfqpoint{0.000000in}{0.000000in}}{%
\pgfpathmoveto{\pgfqpoint{0.000000in}{0.000000in}}%
\pgfpathlineto{\pgfqpoint{0.000000in}{-0.048611in}}%
\pgfusepath{stroke,fill}%
}%
\begin{pgfscope}%
\pgfsys@transformshift{0.934300in}{0.564143in}%
\pgfsys@useobject{currentmarker}{}%
\end{pgfscope}%
\end{pgfscope}%
\begin{pgfscope}%
\definecolor{textcolor}{rgb}{0.000000,0.000000,0.000000}%
\pgfsetstrokecolor{textcolor}%
\pgfsetfillcolor{textcolor}%
\pgftext[x=0.934300in,y=0.466921in,,top]{\color{textcolor}\rmfamily\fontsize{10.000000}{12.000000}\selectfont \(\displaystyle -100\)}%
\end{pgfscope}%
\begin{pgfscope}%
\pgfsetbuttcap%
\pgfsetroundjoin%
\definecolor{currentfill}{rgb}{0.000000,0.000000,0.000000}%
\pgfsetfillcolor{currentfill}%
\pgfsetlinewidth{0.803000pt}%
\definecolor{currentstroke}{rgb}{0.000000,0.000000,0.000000}%
\pgfsetstrokecolor{currentstroke}%
\pgfsetdash{}{0pt}%
\pgfsys@defobject{currentmarker}{\pgfqpoint{0.000000in}{-0.048611in}}{\pgfqpoint{0.000000in}{0.000000in}}{%
\pgfpathmoveto{\pgfqpoint{0.000000in}{0.000000in}}%
\pgfpathlineto{\pgfqpoint{0.000000in}{-0.048611in}}%
\pgfusepath{stroke,fill}%
}%
\begin{pgfscope}%
\pgfsys@transformshift{1.585791in}{0.564143in}%
\pgfsys@useobject{currentmarker}{}%
\end{pgfscope}%
\end{pgfscope}%
\begin{pgfscope}%
\definecolor{textcolor}{rgb}{0.000000,0.000000,0.000000}%
\pgfsetstrokecolor{textcolor}%
\pgfsetfillcolor{textcolor}%
\pgftext[x=1.585791in,y=0.466921in,,top]{\color{textcolor}\rmfamily\fontsize{10.000000}{12.000000}\selectfont \(\displaystyle -75\)}%
\end{pgfscope}%
\begin{pgfscope}%
\pgfsetbuttcap%
\pgfsetroundjoin%
\definecolor{currentfill}{rgb}{0.000000,0.000000,0.000000}%
\pgfsetfillcolor{currentfill}%
\pgfsetlinewidth{0.803000pt}%
\definecolor{currentstroke}{rgb}{0.000000,0.000000,0.000000}%
\pgfsetstrokecolor{currentstroke}%
\pgfsetdash{}{0pt}%
\pgfsys@defobject{currentmarker}{\pgfqpoint{0.000000in}{-0.048611in}}{\pgfqpoint{0.000000in}{0.000000in}}{%
\pgfpathmoveto{\pgfqpoint{0.000000in}{0.000000in}}%
\pgfpathlineto{\pgfqpoint{0.000000in}{-0.048611in}}%
\pgfusepath{stroke,fill}%
}%
\begin{pgfscope}%
\pgfsys@transformshift{2.237281in}{0.564143in}%
\pgfsys@useobject{currentmarker}{}%
\end{pgfscope}%
\end{pgfscope}%
\begin{pgfscope}%
\definecolor{textcolor}{rgb}{0.000000,0.000000,0.000000}%
\pgfsetstrokecolor{textcolor}%
\pgfsetfillcolor{textcolor}%
\pgftext[x=2.237281in,y=0.466921in,,top]{\color{textcolor}\rmfamily\fontsize{10.000000}{12.000000}\selectfont \(\displaystyle -50\)}%
\end{pgfscope}%
\begin{pgfscope}%
\pgfsetbuttcap%
\pgfsetroundjoin%
\definecolor{currentfill}{rgb}{0.000000,0.000000,0.000000}%
\pgfsetfillcolor{currentfill}%
\pgfsetlinewidth{0.803000pt}%
\definecolor{currentstroke}{rgb}{0.000000,0.000000,0.000000}%
\pgfsetstrokecolor{currentstroke}%
\pgfsetdash{}{0pt}%
\pgfsys@defobject{currentmarker}{\pgfqpoint{0.000000in}{-0.048611in}}{\pgfqpoint{0.000000in}{0.000000in}}{%
\pgfpathmoveto{\pgfqpoint{0.000000in}{0.000000in}}%
\pgfpathlineto{\pgfqpoint{0.000000in}{-0.048611in}}%
\pgfusepath{stroke,fill}%
}%
\begin{pgfscope}%
\pgfsys@transformshift{2.888771in}{0.564143in}%
\pgfsys@useobject{currentmarker}{}%
\end{pgfscope}%
\end{pgfscope}%
\begin{pgfscope}%
\definecolor{textcolor}{rgb}{0.000000,0.000000,0.000000}%
\pgfsetstrokecolor{textcolor}%
\pgfsetfillcolor{textcolor}%
\pgftext[x=2.888771in,y=0.466921in,,top]{\color{textcolor}\rmfamily\fontsize{10.000000}{12.000000}\selectfont \(\displaystyle -25\)}%
\end{pgfscope}%
\begin{pgfscope}%
\pgfsetbuttcap%
\pgfsetroundjoin%
\definecolor{currentfill}{rgb}{0.000000,0.000000,0.000000}%
\pgfsetfillcolor{currentfill}%
\pgfsetlinewidth{0.803000pt}%
\definecolor{currentstroke}{rgb}{0.000000,0.000000,0.000000}%
\pgfsetstrokecolor{currentstroke}%
\pgfsetdash{}{0pt}%
\pgfsys@defobject{currentmarker}{\pgfqpoint{0.000000in}{-0.048611in}}{\pgfqpoint{0.000000in}{0.000000in}}{%
\pgfpathmoveto{\pgfqpoint{0.000000in}{0.000000in}}%
\pgfpathlineto{\pgfqpoint{0.000000in}{-0.048611in}}%
\pgfusepath{stroke,fill}%
}%
\begin{pgfscope}%
\pgfsys@transformshift{3.540261in}{0.564143in}%
\pgfsys@useobject{currentmarker}{}%
\end{pgfscope}%
\end{pgfscope}%
\begin{pgfscope}%
\definecolor{textcolor}{rgb}{0.000000,0.000000,0.000000}%
\pgfsetstrokecolor{textcolor}%
\pgfsetfillcolor{textcolor}%
\pgftext[x=3.540261in,y=0.466921in,,top]{\color{textcolor}\rmfamily\fontsize{10.000000}{12.000000}\selectfont \(\displaystyle 0\)}%
\end{pgfscope}%
\begin{pgfscope}%
\pgfsetbuttcap%
\pgfsetroundjoin%
\definecolor{currentfill}{rgb}{0.000000,0.000000,0.000000}%
\pgfsetfillcolor{currentfill}%
\pgfsetlinewidth{0.803000pt}%
\definecolor{currentstroke}{rgb}{0.000000,0.000000,0.000000}%
\pgfsetstrokecolor{currentstroke}%
\pgfsetdash{}{0pt}%
\pgfsys@defobject{currentmarker}{\pgfqpoint{0.000000in}{-0.048611in}}{\pgfqpoint{0.000000in}{0.000000in}}{%
\pgfpathmoveto{\pgfqpoint{0.000000in}{0.000000in}}%
\pgfpathlineto{\pgfqpoint{0.000000in}{-0.048611in}}%
\pgfusepath{stroke,fill}%
}%
\begin{pgfscope}%
\pgfsys@transformshift{4.191751in}{0.564143in}%
\pgfsys@useobject{currentmarker}{}%
\end{pgfscope}%
\end{pgfscope}%
\begin{pgfscope}%
\definecolor{textcolor}{rgb}{0.000000,0.000000,0.000000}%
\pgfsetstrokecolor{textcolor}%
\pgfsetfillcolor{textcolor}%
\pgftext[x=4.191751in,y=0.466921in,,top]{\color{textcolor}\rmfamily\fontsize{10.000000}{12.000000}\selectfont \(\displaystyle 25\)}%
\end{pgfscope}%
\begin{pgfscope}%
\pgfsetbuttcap%
\pgfsetroundjoin%
\definecolor{currentfill}{rgb}{0.000000,0.000000,0.000000}%
\pgfsetfillcolor{currentfill}%
\pgfsetlinewidth{0.803000pt}%
\definecolor{currentstroke}{rgb}{0.000000,0.000000,0.000000}%
\pgfsetstrokecolor{currentstroke}%
\pgfsetdash{}{0pt}%
\pgfsys@defobject{currentmarker}{\pgfqpoint{0.000000in}{-0.048611in}}{\pgfqpoint{0.000000in}{0.000000in}}{%
\pgfpathmoveto{\pgfqpoint{0.000000in}{0.000000in}}%
\pgfpathlineto{\pgfqpoint{0.000000in}{-0.048611in}}%
\pgfusepath{stroke,fill}%
}%
\begin{pgfscope}%
\pgfsys@transformshift{4.843242in}{0.564143in}%
\pgfsys@useobject{currentmarker}{}%
\end{pgfscope}%
\end{pgfscope}%
\begin{pgfscope}%
\definecolor{textcolor}{rgb}{0.000000,0.000000,0.000000}%
\pgfsetstrokecolor{textcolor}%
\pgfsetfillcolor{textcolor}%
\pgftext[x=4.843242in,y=0.466921in,,top]{\color{textcolor}\rmfamily\fontsize{10.000000}{12.000000}\selectfont \(\displaystyle 50\)}%
\end{pgfscope}%
\begin{pgfscope}%
\pgfsetbuttcap%
\pgfsetroundjoin%
\definecolor{currentfill}{rgb}{0.000000,0.000000,0.000000}%
\pgfsetfillcolor{currentfill}%
\pgfsetlinewidth{0.803000pt}%
\definecolor{currentstroke}{rgb}{0.000000,0.000000,0.000000}%
\pgfsetstrokecolor{currentstroke}%
\pgfsetdash{}{0pt}%
\pgfsys@defobject{currentmarker}{\pgfqpoint{0.000000in}{-0.048611in}}{\pgfqpoint{0.000000in}{0.000000in}}{%
\pgfpathmoveto{\pgfqpoint{0.000000in}{0.000000in}}%
\pgfpathlineto{\pgfqpoint{0.000000in}{-0.048611in}}%
\pgfusepath{stroke,fill}%
}%
\begin{pgfscope}%
\pgfsys@transformshift{5.494732in}{0.564143in}%
\pgfsys@useobject{currentmarker}{}%
\end{pgfscope}%
\end{pgfscope}%
\begin{pgfscope}%
\definecolor{textcolor}{rgb}{0.000000,0.000000,0.000000}%
\pgfsetstrokecolor{textcolor}%
\pgfsetfillcolor{textcolor}%
\pgftext[x=5.494732in,y=0.466921in,,top]{\color{textcolor}\rmfamily\fontsize{10.000000}{12.000000}\selectfont \(\displaystyle 75\)}%
\end{pgfscope}%
\begin{pgfscope}%
\pgfsetbuttcap%
\pgfsetroundjoin%
\definecolor{currentfill}{rgb}{0.000000,0.000000,0.000000}%
\pgfsetfillcolor{currentfill}%
\pgfsetlinewidth{0.803000pt}%
\definecolor{currentstroke}{rgb}{0.000000,0.000000,0.000000}%
\pgfsetstrokecolor{currentstroke}%
\pgfsetdash{}{0pt}%
\pgfsys@defobject{currentmarker}{\pgfqpoint{0.000000in}{-0.048611in}}{\pgfqpoint{0.000000in}{0.000000in}}{%
\pgfpathmoveto{\pgfqpoint{0.000000in}{0.000000in}}%
\pgfpathlineto{\pgfqpoint{0.000000in}{-0.048611in}}%
\pgfusepath{stroke,fill}%
}%
\begin{pgfscope}%
\pgfsys@transformshift{6.146222in}{0.564143in}%
\pgfsys@useobject{currentmarker}{}%
\end{pgfscope}%
\end{pgfscope}%
\begin{pgfscope}%
\definecolor{textcolor}{rgb}{0.000000,0.000000,0.000000}%
\pgfsetstrokecolor{textcolor}%
\pgfsetfillcolor{textcolor}%
\pgftext[x=6.146222in,y=0.466921in,,top]{\color{textcolor}\rmfamily\fontsize{10.000000}{12.000000}\selectfont \(\displaystyle 100\)}%
\end{pgfscope}%
\begin{pgfscope}%
\definecolor{textcolor}{rgb}{0.000000,0.000000,0.000000}%
\pgfsetstrokecolor{textcolor}%
\pgfsetfillcolor{textcolor}%
\pgftext[x=3.540261in,y=0.287909in,,top]{\color{textcolor}\rmfamily\fontsize{10.000000}{12.000000}\selectfont Frequency (Hz)}%
\end{pgfscope}%
\begin{pgfscope}%
\pgfsetbuttcap%
\pgfsetroundjoin%
\definecolor{currentfill}{rgb}{0.000000,0.000000,0.000000}%
\pgfsetfillcolor{currentfill}%
\pgfsetlinewidth{0.803000pt}%
\definecolor{currentstroke}{rgb}{0.000000,0.000000,0.000000}%
\pgfsetstrokecolor{currentstroke}%
\pgfsetdash{}{0pt}%
\pgfsys@defobject{currentmarker}{\pgfqpoint{-0.048611in}{0.000000in}}{\pgfqpoint{0.000000in}{0.000000in}}{%
\pgfpathmoveto{\pgfqpoint{0.000000in}{0.000000in}}%
\pgfpathlineto{\pgfqpoint{-0.048611in}{0.000000in}}%
\pgfusepath{stroke,fill}%
}%
\begin{pgfscope}%
\pgfsys@transformshift{0.934300in}{0.599911in}%
\pgfsys@useobject{currentmarker}{}%
\end{pgfscope}%
\end{pgfscope}%
\begin{pgfscope}%
\definecolor{textcolor}{rgb}{0.000000,0.000000,0.000000}%
\pgfsetstrokecolor{textcolor}%
\pgfsetfillcolor{textcolor}%
\pgftext[x=0.767633in,y=0.551686in,left,base]{\color{textcolor}\rmfamily\fontsize{10.000000}{12.000000}\selectfont \(\displaystyle 0\)}%
\end{pgfscope}%
\begin{pgfscope}%
\pgfsetbuttcap%
\pgfsetroundjoin%
\definecolor{currentfill}{rgb}{0.000000,0.000000,0.000000}%
\pgfsetfillcolor{currentfill}%
\pgfsetlinewidth{0.803000pt}%
\definecolor{currentstroke}{rgb}{0.000000,0.000000,0.000000}%
\pgfsetstrokecolor{currentstroke}%
\pgfsetdash{}{0pt}%
\pgfsys@defobject{currentmarker}{\pgfqpoint{-0.048611in}{0.000000in}}{\pgfqpoint{0.000000in}{0.000000in}}{%
\pgfpathmoveto{\pgfqpoint{0.000000in}{0.000000in}}%
\pgfpathlineto{\pgfqpoint{-0.048611in}{0.000000in}}%
\pgfusepath{stroke,fill}%
}%
\begin{pgfscope}%
\pgfsys@transformshift{0.934300in}{1.235388in}%
\pgfsys@useobject{currentmarker}{}%
\end{pgfscope}%
\end{pgfscope}%
\begin{pgfscope}%
\definecolor{textcolor}{rgb}{0.000000,0.000000,0.000000}%
\pgfsetstrokecolor{textcolor}%
\pgfsetfillcolor{textcolor}%
\pgftext[x=0.350966in,y=1.187163in,left,base]{\color{textcolor}\rmfamily\fontsize{10.000000}{12.000000}\selectfont \(\displaystyle 2000000\)}%
\end{pgfscope}%
\begin{pgfscope}%
\definecolor{textcolor}{rgb}{0.000000,0.000000,0.000000}%
\pgfsetstrokecolor{textcolor}%
\pgfsetfillcolor{textcolor}%
\pgftext[x=0.295410in,y=0.957751in,,bottom,rotate=90.000000]{\color{textcolor}\rmfamily\fontsize{10.000000}{12.000000}\selectfont abs(Y(f)) (\(\displaystyle \mu V^2\))}%
\end{pgfscope}%
\begin{pgfscope}%
\pgfpathrectangle{\pgfqpoint{0.934300in}{0.564143in}}{\pgfqpoint{5.211922in}{0.787215in}}%
\pgfusepath{clip}%
\pgfsetrectcap%
\pgfsetroundjoin%
\pgfsetlinewidth{1.505625pt}%
\definecolor{currentstroke}{rgb}{0.121569,0.466667,0.705882}%
\pgfsetstrokecolor{currentstroke}%
\pgfsetdash{}{0pt}%
\pgfpathmoveto{\pgfqpoint{3.540261in}{0.629526in}}%
\pgfpathlineto{\pgfqpoint{3.540795in}{0.623666in}}%
\pgfpathlineto{\pgfqpoint{3.541329in}{0.661903in}}%
\pgfpathlineto{\pgfqpoint{3.541862in}{0.657338in}}%
\pgfpathlineto{\pgfqpoint{3.542396in}{0.628996in}}%
\pgfpathlineto{\pgfqpoint{3.542930in}{0.738287in}}%
\pgfpathlineto{\pgfqpoint{3.543463in}{0.679791in}}%
\pgfpathlineto{\pgfqpoint{3.543997in}{0.622096in}}%
\pgfpathlineto{\pgfqpoint{3.545065in}{0.895303in}}%
\pgfpathlineto{\pgfqpoint{3.545598in}{0.823389in}}%
\pgfpathlineto{\pgfqpoint{3.546132in}{0.612381in}}%
\pgfpathlineto{\pgfqpoint{3.548267in}{1.154160in}}%
\pgfpathlineto{\pgfqpoint{3.548800in}{1.315576in}}%
\pgfpathlineto{\pgfqpoint{3.549334in}{0.732029in}}%
\pgfpathlineto{\pgfqpoint{3.549868in}{0.781923in}}%
\pgfpathlineto{\pgfqpoint{3.550402in}{1.118105in}}%
\pgfpathlineto{\pgfqpoint{3.550935in}{0.865252in}}%
\pgfpathlineto{\pgfqpoint{3.552536in}{0.758969in}}%
\pgfpathlineto{\pgfqpoint{3.553070in}{0.638718in}}%
\pgfpathlineto{\pgfqpoint{3.553604in}{0.711646in}}%
\pgfpathlineto{\pgfqpoint{3.554671in}{0.894025in}}%
\pgfpathlineto{\pgfqpoint{3.555205in}{0.874457in}}%
\pgfpathlineto{\pgfqpoint{3.555739in}{0.633961in}}%
\pgfpathlineto{\pgfqpoint{3.556272in}{0.701991in}}%
\pgfpathlineto{\pgfqpoint{3.556806in}{0.870063in}}%
\pgfpathlineto{\pgfqpoint{3.557340in}{0.696496in}}%
\pgfpathlineto{\pgfqpoint{3.557873in}{0.869635in}}%
\pgfpathlineto{\pgfqpoint{3.558941in}{0.698625in}}%
\pgfpathlineto{\pgfqpoint{3.559474in}{0.703946in}}%
\pgfpathlineto{\pgfqpoint{3.560008in}{0.873871in}}%
\pgfpathlineto{\pgfqpoint{3.560542in}{0.750515in}}%
\pgfpathlineto{\pgfqpoint{3.561609in}{0.700834in}}%
\pgfpathlineto{\pgfqpoint{3.562143in}{0.825430in}}%
\pgfpathlineto{\pgfqpoint{3.562677in}{0.617281in}}%
\pgfpathlineto{\pgfqpoint{3.563210in}{0.652705in}}%
\pgfpathlineto{\pgfqpoint{3.565879in}{0.931036in}}%
\pgfpathlineto{\pgfqpoint{3.567480in}{0.738875in}}%
\pgfpathlineto{\pgfqpoint{3.568014in}{0.790627in}}%
\pgfpathlineto{\pgfqpoint{3.568547in}{0.656170in}}%
\pgfpathlineto{\pgfqpoint{3.569081in}{0.675545in}}%
\pgfpathlineto{\pgfqpoint{3.569615in}{0.723996in}}%
\pgfpathlineto{\pgfqpoint{3.570149in}{0.715297in}}%
\pgfpathlineto{\pgfqpoint{3.570682in}{0.682076in}}%
\pgfpathlineto{\pgfqpoint{3.571216in}{0.718460in}}%
\pgfpathlineto{\pgfqpoint{3.571750in}{0.716560in}}%
\pgfpathlineto{\pgfqpoint{3.572817in}{0.639311in}}%
\pgfpathlineto{\pgfqpoint{3.573351in}{0.649127in}}%
\pgfpathlineto{\pgfqpoint{3.573884in}{0.639824in}}%
\pgfpathlineto{\pgfqpoint{3.574418in}{0.647323in}}%
\pgfpathlineto{\pgfqpoint{3.574952in}{0.652719in}}%
\pgfpathlineto{\pgfqpoint{3.575486in}{0.688962in}}%
\pgfpathlineto{\pgfqpoint{3.576019in}{0.615716in}}%
\pgfpathlineto{\pgfqpoint{3.576553in}{0.644533in}}%
\pgfpathlineto{\pgfqpoint{3.577087in}{0.772404in}}%
\pgfpathlineto{\pgfqpoint{3.577620in}{0.721002in}}%
\pgfpathlineto{\pgfqpoint{3.578154in}{0.758931in}}%
\pgfpathlineto{\pgfqpoint{3.578688in}{0.728230in}}%
\pgfpathlineto{\pgfqpoint{3.579221in}{0.714968in}}%
\pgfpathlineto{\pgfqpoint{3.579755in}{0.732086in}}%
\pgfpathlineto{\pgfqpoint{3.580289in}{0.634990in}}%
\pgfpathlineto{\pgfqpoint{3.580823in}{0.757533in}}%
\pgfpathlineto{\pgfqpoint{3.581356in}{0.716588in}}%
\pgfpathlineto{\pgfqpoint{3.582424in}{0.752025in}}%
\pgfpathlineto{\pgfqpoint{3.582957in}{0.620202in}}%
\pgfpathlineto{\pgfqpoint{3.584025in}{0.621488in}}%
\pgfpathlineto{\pgfqpoint{3.585092in}{0.677360in}}%
\pgfpathlineto{\pgfqpoint{3.586693in}{0.620658in}}%
\pgfpathlineto{\pgfqpoint{3.587227in}{0.694405in}}%
\pgfpathlineto{\pgfqpoint{3.587761in}{0.679897in}}%
\pgfpathlineto{\pgfqpoint{3.589362in}{0.632332in}}%
\pgfpathlineto{\pgfqpoint{3.589895in}{0.671848in}}%
\pgfpathlineto{\pgfqpoint{3.590429in}{0.648262in}}%
\pgfpathlineto{\pgfqpoint{3.591497in}{0.609347in}}%
\pgfpathlineto{\pgfqpoint{3.592030in}{0.726101in}}%
\pgfpathlineto{\pgfqpoint{3.593098in}{0.719787in}}%
\pgfpathlineto{\pgfqpoint{3.593631in}{0.621865in}}%
\pgfpathlineto{\pgfqpoint{3.594165in}{0.710056in}}%
\pgfpathlineto{\pgfqpoint{3.594699in}{0.666694in}}%
\pgfpathlineto{\pgfqpoint{3.595232in}{0.981670in}}%
\pgfpathlineto{\pgfqpoint{3.595766in}{0.904909in}}%
\pgfpathlineto{\pgfqpoint{3.596834in}{0.765525in}}%
\pgfpathlineto{\pgfqpoint{3.597367in}{0.816745in}}%
\pgfpathlineto{\pgfqpoint{3.598435in}{0.637501in}}%
\pgfpathlineto{\pgfqpoint{3.598968in}{0.700964in}}%
\pgfpathlineto{\pgfqpoint{3.599502in}{0.623317in}}%
\pgfpathlineto{\pgfqpoint{3.600036in}{0.696094in}}%
\pgfpathlineto{\pgfqpoint{3.600569in}{0.705217in}}%
\pgfpathlineto{\pgfqpoint{3.601103in}{0.680450in}}%
\pgfpathlineto{\pgfqpoint{3.601637in}{0.704452in}}%
\pgfpathlineto{\pgfqpoint{3.602171in}{0.700368in}}%
\pgfpathlineto{\pgfqpoint{3.602704in}{0.832360in}}%
\pgfpathlineto{\pgfqpoint{3.603238in}{0.620452in}}%
\pgfpathlineto{\pgfqpoint{3.603772in}{0.777356in}}%
\pgfpathlineto{\pgfqpoint{3.604305in}{0.676497in}}%
\pgfpathlineto{\pgfqpoint{3.604839in}{0.726375in}}%
\pgfpathlineto{\pgfqpoint{3.605373in}{0.801228in}}%
\pgfpathlineto{\pgfqpoint{3.605906in}{0.695217in}}%
\pgfpathlineto{\pgfqpoint{3.607508in}{0.902239in}}%
\pgfpathlineto{\pgfqpoint{3.608041in}{0.705258in}}%
\pgfpathlineto{\pgfqpoint{3.608575in}{0.890671in}}%
\pgfpathlineto{\pgfqpoint{3.609109in}{0.756914in}}%
\pgfpathlineto{\pgfqpoint{3.609642in}{0.786788in}}%
\pgfpathlineto{\pgfqpoint{3.610176in}{0.797656in}}%
\pgfpathlineto{\pgfqpoint{3.610710in}{0.727135in}}%
\pgfpathlineto{\pgfqpoint{3.612311in}{0.893532in}}%
\pgfpathlineto{\pgfqpoint{3.614446in}{0.739442in}}%
\pgfpathlineto{\pgfqpoint{3.614979in}{0.716234in}}%
\pgfpathlineto{\pgfqpoint{3.615513in}{0.630254in}}%
\pgfpathlineto{\pgfqpoint{3.616047in}{0.765131in}}%
\pgfpathlineto{\pgfqpoint{3.616580in}{0.655959in}}%
\pgfpathlineto{\pgfqpoint{3.617114in}{0.662895in}}%
\pgfpathlineto{\pgfqpoint{3.618715in}{0.638305in}}%
\pgfpathlineto{\pgfqpoint{3.619249in}{0.673450in}}%
\pgfpathlineto{\pgfqpoint{3.619783in}{0.659521in}}%
\pgfpathlineto{\pgfqpoint{3.620316in}{0.611852in}}%
\pgfpathlineto{\pgfqpoint{3.620850in}{0.658165in}}%
\pgfpathlineto{\pgfqpoint{3.621384in}{0.673293in}}%
\pgfpathlineto{\pgfqpoint{3.622451in}{0.631714in}}%
\pgfpathlineto{\pgfqpoint{3.622985in}{0.636714in}}%
\pgfpathlineto{\pgfqpoint{3.623519in}{0.660816in}}%
\pgfpathlineto{\pgfqpoint{3.625120in}{0.607716in}}%
\pgfpathlineto{\pgfqpoint{3.625653in}{0.621980in}}%
\pgfpathlineto{\pgfqpoint{3.626187in}{0.672373in}}%
\pgfpathlineto{\pgfqpoint{3.626721in}{0.659781in}}%
\pgfpathlineto{\pgfqpoint{3.627254in}{0.613257in}}%
\pgfpathlineto{\pgfqpoint{3.627788in}{0.618423in}}%
\pgfpathlineto{\pgfqpoint{3.628856in}{0.643975in}}%
\pgfpathlineto{\pgfqpoint{3.629389in}{0.633743in}}%
\pgfpathlineto{\pgfqpoint{3.629923in}{0.628264in}}%
\pgfpathlineto{\pgfqpoint{3.630457in}{0.606049in}}%
\pgfpathlineto{\pgfqpoint{3.630990in}{0.627928in}}%
\pgfpathlineto{\pgfqpoint{3.631524in}{0.630881in}}%
\pgfpathlineto{\pgfqpoint{3.632058in}{0.668066in}}%
\pgfpathlineto{\pgfqpoint{3.632592in}{0.615160in}}%
\pgfpathlineto{\pgfqpoint{3.633125in}{0.644260in}}%
\pgfpathlineto{\pgfqpoint{3.634193in}{0.634651in}}%
\pgfpathlineto{\pgfqpoint{3.635260in}{0.662057in}}%
\pgfpathlineto{\pgfqpoint{3.635794in}{0.643240in}}%
\pgfpathlineto{\pgfqpoint{3.636327in}{0.663571in}}%
\pgfpathlineto{\pgfqpoint{3.636861in}{0.680897in}}%
\pgfpathlineto{\pgfqpoint{3.638462in}{0.624426in}}%
\pgfpathlineto{\pgfqpoint{3.638996in}{0.607553in}}%
\pgfpathlineto{\pgfqpoint{3.641131in}{0.684584in}}%
\pgfpathlineto{\pgfqpoint{3.642732in}{0.646174in}}%
\pgfpathlineto{\pgfqpoint{3.643266in}{0.655883in}}%
\pgfpathlineto{\pgfqpoint{3.643799in}{0.653890in}}%
\pgfpathlineto{\pgfqpoint{3.644333in}{0.658105in}}%
\pgfpathlineto{\pgfqpoint{3.645400in}{0.637364in}}%
\pgfpathlineto{\pgfqpoint{3.645934in}{0.661490in}}%
\pgfpathlineto{\pgfqpoint{3.646468in}{0.624564in}}%
\pgfpathlineto{\pgfqpoint{3.647001in}{0.667654in}}%
\pgfpathlineto{\pgfqpoint{3.647535in}{0.666105in}}%
\pgfpathlineto{\pgfqpoint{3.648603in}{0.637393in}}%
\pgfpathlineto{\pgfqpoint{3.649136in}{0.657894in}}%
\pgfpathlineto{\pgfqpoint{3.649670in}{0.626222in}}%
\pgfpathlineto{\pgfqpoint{3.650204in}{0.663152in}}%
\pgfpathlineto{\pgfqpoint{3.650737in}{0.813508in}}%
\pgfpathlineto{\pgfqpoint{3.651271in}{0.652148in}}%
\pgfpathlineto{\pgfqpoint{3.651805in}{0.689365in}}%
\pgfpathlineto{\pgfqpoint{3.652338in}{0.673136in}}%
\pgfpathlineto{\pgfqpoint{3.652872in}{0.659027in}}%
\pgfpathlineto{\pgfqpoint{3.653406in}{0.661497in}}%
\pgfpathlineto{\pgfqpoint{3.654473in}{0.681038in}}%
\pgfpathlineto{\pgfqpoint{3.655007in}{0.650375in}}%
\pgfpathlineto{\pgfqpoint{3.655541in}{0.662189in}}%
\pgfpathlineto{\pgfqpoint{3.658743in}{0.608890in}}%
\pgfpathlineto{\pgfqpoint{3.659277in}{0.621741in}}%
\pgfpathlineto{\pgfqpoint{3.659810in}{0.623656in}}%
\pgfpathlineto{\pgfqpoint{3.660878in}{0.635536in}}%
\pgfpathlineto{\pgfqpoint{3.661411in}{0.629426in}}%
\pgfpathlineto{\pgfqpoint{3.661945in}{0.613368in}}%
\pgfpathlineto{\pgfqpoint{3.662479in}{0.632760in}}%
\pgfpathlineto{\pgfqpoint{3.663012in}{0.616733in}}%
\pgfpathlineto{\pgfqpoint{3.664080in}{0.644773in}}%
\pgfpathlineto{\pgfqpoint{3.664614in}{0.618537in}}%
\pgfpathlineto{\pgfqpoint{3.665147in}{0.632463in}}%
\pgfpathlineto{\pgfqpoint{3.667816in}{0.613843in}}%
\pgfpathlineto{\pgfqpoint{3.668349in}{0.635470in}}%
\pgfpathlineto{\pgfqpoint{3.668883in}{0.620860in}}%
\pgfpathlineto{\pgfqpoint{3.669417in}{0.626872in}}%
\pgfpathlineto{\pgfqpoint{3.669951in}{0.622399in}}%
\pgfpathlineto{\pgfqpoint{3.670484in}{0.654327in}}%
\pgfpathlineto{\pgfqpoint{3.671018in}{0.612309in}}%
\pgfpathlineto{\pgfqpoint{3.671552in}{0.638656in}}%
\pgfpathlineto{\pgfqpoint{3.673153in}{0.604248in}}%
\pgfpathlineto{\pgfqpoint{3.675288in}{0.656408in}}%
\pgfpathlineto{\pgfqpoint{3.675821in}{0.631843in}}%
\pgfpathlineto{\pgfqpoint{3.676355in}{0.672919in}}%
\pgfpathlineto{\pgfqpoint{3.676889in}{0.628926in}}%
\pgfpathlineto{\pgfqpoint{3.677422in}{0.643807in}}%
\pgfpathlineto{\pgfqpoint{3.677956in}{0.634844in}}%
\pgfpathlineto{\pgfqpoint{3.679023in}{0.617944in}}%
\pgfpathlineto{\pgfqpoint{3.679557in}{0.620517in}}%
\pgfpathlineto{\pgfqpoint{3.681158in}{0.648037in}}%
\pgfpathlineto{\pgfqpoint{3.681692in}{0.610898in}}%
\pgfpathlineto{\pgfqpoint{3.682226in}{0.652060in}}%
\pgfpathlineto{\pgfqpoint{3.682759in}{0.632553in}}%
\pgfpathlineto{\pgfqpoint{3.683293in}{0.631497in}}%
\pgfpathlineto{\pgfqpoint{3.684360in}{0.601158in}}%
\pgfpathlineto{\pgfqpoint{3.685962in}{0.630498in}}%
\pgfpathlineto{\pgfqpoint{3.686495in}{0.604022in}}%
\pgfpathlineto{\pgfqpoint{3.687029in}{0.607644in}}%
\pgfpathlineto{\pgfqpoint{3.688096in}{0.622845in}}%
\pgfpathlineto{\pgfqpoint{3.688630in}{0.606785in}}%
\pgfpathlineto{\pgfqpoint{3.689164in}{0.610624in}}%
\pgfpathlineto{\pgfqpoint{3.690231in}{0.616695in}}%
\pgfpathlineto{\pgfqpoint{3.690765in}{0.625299in}}%
\pgfpathlineto{\pgfqpoint{3.691299in}{0.602353in}}%
\pgfpathlineto{\pgfqpoint{3.691832in}{0.617253in}}%
\pgfpathlineto{\pgfqpoint{3.692366in}{0.617005in}}%
\pgfpathlineto{\pgfqpoint{3.693433in}{0.635982in}}%
\pgfpathlineto{\pgfqpoint{3.693967in}{0.624698in}}%
\pgfpathlineto{\pgfqpoint{3.694501in}{0.602042in}}%
\pgfpathlineto{\pgfqpoint{3.695568in}{0.636640in}}%
\pgfpathlineto{\pgfqpoint{3.697169in}{0.609981in}}%
\pgfpathlineto{\pgfqpoint{3.697703in}{0.609884in}}%
\pgfpathlineto{\pgfqpoint{3.699304in}{0.624757in}}%
\pgfpathlineto{\pgfqpoint{3.699838in}{0.619083in}}%
\pgfpathlineto{\pgfqpoint{3.700372in}{0.671343in}}%
\pgfpathlineto{\pgfqpoint{3.700905in}{0.604191in}}%
\pgfpathlineto{\pgfqpoint{3.701439in}{0.647990in}}%
\pgfpathlineto{\pgfqpoint{3.701973in}{0.615001in}}%
\pgfpathlineto{\pgfqpoint{3.702506in}{0.623754in}}%
\pgfpathlineto{\pgfqpoint{3.703040in}{0.632599in}}%
\pgfpathlineto{\pgfqpoint{3.703574in}{0.747727in}}%
\pgfpathlineto{\pgfqpoint{3.704107in}{0.650922in}}%
\pgfpathlineto{\pgfqpoint{3.705175in}{0.692585in}}%
\pgfpathlineto{\pgfqpoint{3.705709in}{0.877713in}}%
\pgfpathlineto{\pgfqpoint{3.706242in}{0.718391in}}%
\pgfpathlineto{\pgfqpoint{3.707310in}{0.628218in}}%
\pgfpathlineto{\pgfqpoint{3.707843in}{0.738418in}}%
\pgfpathlineto{\pgfqpoint{3.708377in}{0.678262in}}%
\pgfpathlineto{\pgfqpoint{3.708911in}{0.700814in}}%
\pgfpathlineto{\pgfqpoint{3.709444in}{0.691910in}}%
\pgfpathlineto{\pgfqpoint{3.711046in}{0.637144in}}%
\pgfpathlineto{\pgfqpoint{3.712647in}{0.733407in}}%
\pgfpathlineto{\pgfqpoint{3.714781in}{0.635414in}}%
\pgfpathlineto{\pgfqpoint{3.715315in}{0.641153in}}%
\pgfpathlineto{\pgfqpoint{3.716916in}{0.622135in}}%
\pgfpathlineto{\pgfqpoint{3.717450in}{0.631119in}}%
\pgfpathlineto{\pgfqpoint{3.717984in}{0.624720in}}%
\pgfpathlineto{\pgfqpoint{3.719585in}{0.605356in}}%
\pgfpathlineto{\pgfqpoint{3.721186in}{0.621711in}}%
\pgfpathlineto{\pgfqpoint{3.722253in}{0.602809in}}%
\pgfpathlineto{\pgfqpoint{3.723321in}{0.621127in}}%
\pgfpathlineto{\pgfqpoint{3.723854in}{0.604651in}}%
\pgfpathlineto{\pgfqpoint{3.724388in}{0.618408in}}%
\pgfpathlineto{\pgfqpoint{3.724922in}{0.634445in}}%
\pgfpathlineto{\pgfqpoint{3.725455in}{0.604859in}}%
\pgfpathlineto{\pgfqpoint{3.725989in}{0.616484in}}%
\pgfpathlineto{\pgfqpoint{3.727057in}{0.607459in}}%
\pgfpathlineto{\pgfqpoint{3.727590in}{0.617937in}}%
\pgfpathlineto{\pgfqpoint{3.728124in}{0.606390in}}%
\pgfpathlineto{\pgfqpoint{3.728658in}{0.613837in}}%
\pgfpathlineto{\pgfqpoint{3.729191in}{0.612747in}}%
\pgfpathlineto{\pgfqpoint{3.729725in}{0.617491in}}%
\pgfpathlineto{\pgfqpoint{3.730792in}{0.607565in}}%
\pgfpathlineto{\pgfqpoint{3.731326in}{0.631313in}}%
\pgfpathlineto{\pgfqpoint{3.731860in}{0.609673in}}%
\pgfpathlineto{\pgfqpoint{3.732394in}{0.614011in}}%
\pgfpathlineto{\pgfqpoint{3.732927in}{0.609654in}}%
\pgfpathlineto{\pgfqpoint{3.733461in}{0.638755in}}%
\pgfpathlineto{\pgfqpoint{3.733995in}{0.631279in}}%
\pgfpathlineto{\pgfqpoint{3.736129in}{0.608335in}}%
\pgfpathlineto{\pgfqpoint{3.736663in}{0.621432in}}%
\pgfpathlineto{\pgfqpoint{3.737197in}{0.617338in}}%
\pgfpathlineto{\pgfqpoint{3.737731in}{0.617395in}}%
\pgfpathlineto{\pgfqpoint{3.738264in}{0.625004in}}%
\pgfpathlineto{\pgfqpoint{3.738798in}{0.604979in}}%
\pgfpathlineto{\pgfqpoint{3.739332in}{0.608530in}}%
\pgfpathlineto{\pgfqpoint{3.741466in}{0.621348in}}%
\pgfpathlineto{\pgfqpoint{3.743068in}{0.608387in}}%
\pgfpathlineto{\pgfqpoint{3.743601in}{0.613122in}}%
\pgfpathlineto{\pgfqpoint{3.744135in}{0.605035in}}%
\pgfpathlineto{\pgfqpoint{3.744669in}{0.605463in}}%
\pgfpathlineto{\pgfqpoint{3.745736in}{0.630005in}}%
\pgfpathlineto{\pgfqpoint{3.746270in}{0.618571in}}%
\pgfpathlineto{\pgfqpoint{3.748938in}{0.645214in}}%
\pgfpathlineto{\pgfqpoint{3.751073in}{0.607744in}}%
\pgfpathlineto{\pgfqpoint{3.751607in}{0.625219in}}%
\pgfpathlineto{\pgfqpoint{3.752140in}{0.621870in}}%
\pgfpathlineto{\pgfqpoint{3.752674in}{0.608789in}}%
\pgfpathlineto{\pgfqpoint{3.754275in}{0.638303in}}%
\pgfpathlineto{\pgfqpoint{3.754809in}{0.627034in}}%
\pgfpathlineto{\pgfqpoint{3.755343in}{0.666169in}}%
\pgfpathlineto{\pgfqpoint{3.755876in}{0.608656in}}%
\pgfpathlineto{\pgfqpoint{3.756410in}{0.659901in}}%
\pgfpathlineto{\pgfqpoint{3.756944in}{0.615323in}}%
\pgfpathlineto{\pgfqpoint{3.758545in}{0.808848in}}%
\pgfpathlineto{\pgfqpoint{3.759079in}{0.655752in}}%
\pgfpathlineto{\pgfqpoint{3.759612in}{0.682783in}}%
\pgfpathlineto{\pgfqpoint{3.760146in}{0.669199in}}%
\pgfpathlineto{\pgfqpoint{3.760680in}{0.769406in}}%
\pgfpathlineto{\pgfqpoint{3.761213in}{0.763341in}}%
\pgfpathlineto{\pgfqpoint{3.761747in}{0.635396in}}%
\pgfpathlineto{\pgfqpoint{3.762281in}{0.731871in}}%
\pgfpathlineto{\pgfqpoint{3.762815in}{0.792667in}}%
\pgfpathlineto{\pgfqpoint{3.763348in}{0.742080in}}%
\pgfpathlineto{\pgfqpoint{3.764416in}{0.664216in}}%
\pgfpathlineto{\pgfqpoint{3.764949in}{0.695938in}}%
\pgfpathlineto{\pgfqpoint{3.766017in}{0.637084in}}%
\pgfpathlineto{\pgfqpoint{3.766550in}{0.668525in}}%
\pgfpathlineto{\pgfqpoint{3.767084in}{0.699013in}}%
\pgfpathlineto{\pgfqpoint{3.767618in}{0.669996in}}%
\pgfpathlineto{\pgfqpoint{3.768152in}{0.613867in}}%
\pgfpathlineto{\pgfqpoint{3.768685in}{0.670127in}}%
\pgfpathlineto{\pgfqpoint{3.769219in}{0.666944in}}%
\pgfpathlineto{\pgfqpoint{3.770286in}{0.732600in}}%
\pgfpathlineto{\pgfqpoint{3.770820in}{0.719367in}}%
\pgfpathlineto{\pgfqpoint{3.774022in}{0.615330in}}%
\pgfpathlineto{\pgfqpoint{3.774556in}{0.607074in}}%
\pgfpathlineto{\pgfqpoint{3.775090in}{0.612215in}}%
\pgfpathlineto{\pgfqpoint{3.775623in}{0.624470in}}%
\pgfpathlineto{\pgfqpoint{3.776157in}{0.606611in}}%
\pgfpathlineto{\pgfqpoint{3.776691in}{0.617352in}}%
\pgfpathlineto{\pgfqpoint{3.777224in}{0.614108in}}%
\pgfpathlineto{\pgfqpoint{3.777758in}{0.618389in}}%
\pgfpathlineto{\pgfqpoint{3.778292in}{0.601907in}}%
\pgfpathlineto{\pgfqpoint{3.778826in}{0.619666in}}%
\pgfpathlineto{\pgfqpoint{3.780427in}{0.603752in}}%
\pgfpathlineto{\pgfqpoint{3.781494in}{0.622519in}}%
\pgfpathlineto{\pgfqpoint{3.782028in}{0.612229in}}%
\pgfpathlineto{\pgfqpoint{3.783095in}{0.619012in}}%
\pgfpathlineto{\pgfqpoint{3.783629in}{0.605038in}}%
\pgfpathlineto{\pgfqpoint{3.784163in}{0.617459in}}%
\pgfpathlineto{\pgfqpoint{3.785230in}{0.613764in}}%
\pgfpathlineto{\pgfqpoint{3.785764in}{0.624844in}}%
\pgfpathlineto{\pgfqpoint{3.786297in}{0.615233in}}%
\pgfpathlineto{\pgfqpoint{3.786831in}{0.617962in}}%
\pgfpathlineto{\pgfqpoint{3.788966in}{0.604413in}}%
\pgfpathlineto{\pgfqpoint{3.789500in}{0.629283in}}%
\pgfpathlineto{\pgfqpoint{3.790033in}{0.605210in}}%
\pgfpathlineto{\pgfqpoint{3.791101in}{0.618830in}}%
\pgfpathlineto{\pgfqpoint{3.791634in}{0.603470in}}%
\pgfpathlineto{\pgfqpoint{3.792168in}{0.624494in}}%
\pgfpathlineto{\pgfqpoint{3.792702in}{0.614281in}}%
\pgfpathlineto{\pgfqpoint{3.793235in}{0.602741in}}%
\pgfpathlineto{\pgfqpoint{3.793769in}{0.605838in}}%
\pgfpathlineto{\pgfqpoint{3.795370in}{0.626547in}}%
\pgfpathlineto{\pgfqpoint{3.795904in}{0.601771in}}%
\pgfpathlineto{\pgfqpoint{3.796438in}{0.619207in}}%
\pgfpathlineto{\pgfqpoint{3.796971in}{0.622045in}}%
\pgfpathlineto{\pgfqpoint{3.797505in}{0.631719in}}%
\pgfpathlineto{\pgfqpoint{3.798039in}{0.629354in}}%
\pgfpathlineto{\pgfqpoint{3.799106in}{0.606410in}}%
\pgfpathlineto{\pgfqpoint{3.799640in}{0.607750in}}%
\pgfpathlineto{\pgfqpoint{3.800174in}{0.615689in}}%
\pgfpathlineto{\pgfqpoint{3.801775in}{0.602280in}}%
\pgfpathlineto{\pgfqpoint{3.802308in}{0.602991in}}%
\pgfpathlineto{\pgfqpoint{3.803376in}{0.635209in}}%
\pgfpathlineto{\pgfqpoint{3.803909in}{0.615233in}}%
\pgfpathlineto{\pgfqpoint{3.804443in}{0.618501in}}%
\pgfpathlineto{\pgfqpoint{3.804977in}{0.626013in}}%
\pgfpathlineto{\pgfqpoint{3.805511in}{0.611889in}}%
\pgfpathlineto{\pgfqpoint{3.806044in}{0.615883in}}%
\pgfpathlineto{\pgfqpoint{3.806578in}{0.624702in}}%
\pgfpathlineto{\pgfqpoint{3.807112in}{0.607874in}}%
\pgfpathlineto{\pgfqpoint{3.807645in}{0.613619in}}%
\pgfpathlineto{\pgfqpoint{3.808713in}{0.621040in}}%
\pgfpathlineto{\pgfqpoint{3.809246in}{0.613086in}}%
\pgfpathlineto{\pgfqpoint{3.809780in}{0.617432in}}%
\pgfpathlineto{\pgfqpoint{3.811381in}{0.654459in}}%
\pgfpathlineto{\pgfqpoint{3.811915in}{0.609022in}}%
\pgfpathlineto{\pgfqpoint{3.813516in}{0.718772in}}%
\pgfpathlineto{\pgfqpoint{3.814050in}{0.711254in}}%
\pgfpathlineto{\pgfqpoint{3.815117in}{0.608329in}}%
\pgfpathlineto{\pgfqpoint{3.815651in}{0.653439in}}%
\pgfpathlineto{\pgfqpoint{3.816185in}{0.765893in}}%
\pgfpathlineto{\pgfqpoint{3.816718in}{0.668463in}}%
\pgfpathlineto{\pgfqpoint{3.817252in}{0.710268in}}%
\pgfpathlineto{\pgfqpoint{3.817786in}{0.661734in}}%
\pgfpathlineto{\pgfqpoint{3.818319in}{0.704114in}}%
\pgfpathlineto{\pgfqpoint{3.819920in}{0.662209in}}%
\pgfpathlineto{\pgfqpoint{3.820454in}{0.669565in}}%
\pgfpathlineto{\pgfqpoint{3.820988in}{0.665488in}}%
\pgfpathlineto{\pgfqpoint{3.822589in}{0.627478in}}%
\pgfpathlineto{\pgfqpoint{3.823123in}{0.618780in}}%
\pgfpathlineto{\pgfqpoint{3.824190in}{0.655601in}}%
\pgfpathlineto{\pgfqpoint{3.825257in}{0.649157in}}%
\pgfpathlineto{\pgfqpoint{3.825791in}{0.619856in}}%
\pgfpathlineto{\pgfqpoint{3.826325in}{0.644570in}}%
\pgfpathlineto{\pgfqpoint{3.827926in}{0.661223in}}%
\pgfpathlineto{\pgfqpoint{3.828460in}{0.657006in}}%
\pgfpathlineto{\pgfqpoint{3.828993in}{0.671329in}}%
\pgfpathlineto{\pgfqpoint{3.830595in}{0.632027in}}%
\pgfpathlineto{\pgfqpoint{3.831662in}{0.607255in}}%
\pgfpathlineto{\pgfqpoint{3.832196in}{0.617133in}}%
\pgfpathlineto{\pgfqpoint{3.833263in}{0.606127in}}%
\pgfpathlineto{\pgfqpoint{3.833797in}{0.609697in}}%
\pgfpathlineto{\pgfqpoint{3.834864in}{0.616347in}}%
\pgfpathlineto{\pgfqpoint{3.835398in}{0.600611in}}%
\pgfpathlineto{\pgfqpoint{3.835932in}{0.607249in}}%
\pgfpathlineto{\pgfqpoint{3.836465in}{0.616588in}}%
\pgfpathlineto{\pgfqpoint{3.836999in}{0.608009in}}%
\pgfpathlineto{\pgfqpoint{3.837533in}{0.607869in}}%
\pgfpathlineto{\pgfqpoint{3.838066in}{0.602505in}}%
\pgfpathlineto{\pgfqpoint{3.838600in}{0.615646in}}%
\pgfpathlineto{\pgfqpoint{3.839667in}{0.615295in}}%
\pgfpathlineto{\pgfqpoint{3.840201in}{0.611446in}}%
\pgfpathlineto{\pgfqpoint{3.840735in}{0.619130in}}%
\pgfpathlineto{\pgfqpoint{3.841269in}{0.613651in}}%
\pgfpathlineto{\pgfqpoint{3.841802in}{0.606066in}}%
\pgfpathlineto{\pgfqpoint{3.842336in}{0.612091in}}%
\pgfpathlineto{\pgfqpoint{3.842870in}{0.617198in}}%
\pgfpathlineto{\pgfqpoint{3.843403in}{0.604829in}}%
\pgfpathlineto{\pgfqpoint{3.843937in}{0.625918in}}%
\pgfpathlineto{\pgfqpoint{3.844471in}{0.614467in}}%
\pgfpathlineto{\pgfqpoint{3.846606in}{0.607009in}}%
\pgfpathlineto{\pgfqpoint{3.847139in}{0.622923in}}%
\pgfpathlineto{\pgfqpoint{3.847673in}{0.622896in}}%
\pgfpathlineto{\pgfqpoint{3.848207in}{0.619047in}}%
\pgfpathlineto{\pgfqpoint{3.848740in}{0.623789in}}%
\pgfpathlineto{\pgfqpoint{3.849274in}{0.606613in}}%
\pgfpathlineto{\pgfqpoint{3.849808in}{0.615476in}}%
\pgfpathlineto{\pgfqpoint{3.850341in}{0.623291in}}%
\pgfpathlineto{\pgfqpoint{3.850875in}{0.609068in}}%
\pgfpathlineto{\pgfqpoint{3.851409in}{0.619658in}}%
\pgfpathlineto{\pgfqpoint{3.851943in}{0.613088in}}%
\pgfpathlineto{\pgfqpoint{3.852476in}{0.619148in}}%
\pgfpathlineto{\pgfqpoint{3.853010in}{0.626513in}}%
\pgfpathlineto{\pgfqpoint{3.854077in}{0.603978in}}%
\pgfpathlineto{\pgfqpoint{3.854611in}{0.609106in}}%
\pgfpathlineto{\pgfqpoint{3.855678in}{0.610547in}}%
\pgfpathlineto{\pgfqpoint{3.856212in}{0.626080in}}%
\pgfpathlineto{\pgfqpoint{3.857280in}{0.608158in}}%
\pgfpathlineto{\pgfqpoint{3.858881in}{0.624216in}}%
\pgfpathlineto{\pgfqpoint{3.859414in}{0.602599in}}%
\pgfpathlineto{\pgfqpoint{3.859948in}{0.617712in}}%
\pgfpathlineto{\pgfqpoint{3.861015in}{0.606568in}}%
\pgfpathlineto{\pgfqpoint{3.861549in}{0.637799in}}%
\pgfpathlineto{\pgfqpoint{3.862083in}{0.617298in}}%
\pgfpathlineto{\pgfqpoint{3.862617in}{0.611043in}}%
\pgfpathlineto{\pgfqpoint{3.863150in}{0.616992in}}%
\pgfpathlineto{\pgfqpoint{3.863684in}{0.623432in}}%
\pgfpathlineto{\pgfqpoint{3.864751in}{0.610307in}}%
\pgfpathlineto{\pgfqpoint{3.865285in}{0.612459in}}%
\pgfpathlineto{\pgfqpoint{3.865819in}{0.654939in}}%
\pgfpathlineto{\pgfqpoint{3.866886in}{0.653552in}}%
\pgfpathlineto{\pgfqpoint{3.868487in}{0.640017in}}%
\pgfpathlineto{\pgfqpoint{3.869021in}{0.741986in}}%
\pgfpathlineto{\pgfqpoint{3.869555in}{0.656426in}}%
\pgfpathlineto{\pgfqpoint{3.870088in}{0.622166in}}%
\pgfpathlineto{\pgfqpoint{3.870622in}{0.656673in}}%
\pgfpathlineto{\pgfqpoint{3.871156in}{0.688494in}}%
\pgfpathlineto{\pgfqpoint{3.871689in}{0.629398in}}%
\pgfpathlineto{\pgfqpoint{3.872223in}{0.653173in}}%
\pgfpathlineto{\pgfqpoint{3.872757in}{0.647186in}}%
\pgfpathlineto{\pgfqpoint{3.873291in}{0.714188in}}%
\pgfpathlineto{\pgfqpoint{3.873824in}{0.661439in}}%
\pgfpathlineto{\pgfqpoint{3.874358in}{0.671049in}}%
\pgfpathlineto{\pgfqpoint{3.874892in}{0.621104in}}%
\pgfpathlineto{\pgfqpoint{3.875425in}{0.661428in}}%
\pgfpathlineto{\pgfqpoint{3.875959in}{0.682351in}}%
\pgfpathlineto{\pgfqpoint{3.876493in}{0.619048in}}%
\pgfpathlineto{\pgfqpoint{3.877026in}{0.632945in}}%
\pgfpathlineto{\pgfqpoint{3.877560in}{0.630642in}}%
\pgfpathlineto{\pgfqpoint{3.878628in}{0.621610in}}%
\pgfpathlineto{\pgfqpoint{3.879161in}{0.664461in}}%
\pgfpathlineto{\pgfqpoint{3.879695in}{0.627497in}}%
\pgfpathlineto{\pgfqpoint{3.880229in}{0.626920in}}%
\pgfpathlineto{\pgfqpoint{3.880762in}{0.623073in}}%
\pgfpathlineto{\pgfqpoint{3.882363in}{0.665310in}}%
\pgfpathlineto{\pgfqpoint{3.883431in}{0.618503in}}%
\pgfpathlineto{\pgfqpoint{3.883965in}{0.630804in}}%
\pgfpathlineto{\pgfqpoint{3.885032in}{0.662856in}}%
\pgfpathlineto{\pgfqpoint{3.885566in}{0.661241in}}%
\pgfpathlineto{\pgfqpoint{3.887700in}{0.629592in}}%
\pgfpathlineto{\pgfqpoint{3.888234in}{0.632479in}}%
\pgfpathlineto{\pgfqpoint{3.888768in}{0.614455in}}%
\pgfpathlineto{\pgfqpoint{3.889835in}{0.615115in}}%
\pgfpathlineto{\pgfqpoint{3.890369in}{0.611345in}}%
\pgfpathlineto{\pgfqpoint{3.890903in}{0.618366in}}%
\pgfpathlineto{\pgfqpoint{3.891436in}{0.617094in}}%
\pgfpathlineto{\pgfqpoint{3.891970in}{0.611912in}}%
\pgfpathlineto{\pgfqpoint{3.892504in}{0.624457in}}%
\pgfpathlineto{\pgfqpoint{3.893037in}{0.600774in}}%
\pgfpathlineto{\pgfqpoint{3.893571in}{0.619791in}}%
\pgfpathlineto{\pgfqpoint{3.894105in}{0.608224in}}%
\pgfpathlineto{\pgfqpoint{3.894639in}{0.614279in}}%
\pgfpathlineto{\pgfqpoint{3.896240in}{0.622907in}}%
\pgfpathlineto{\pgfqpoint{3.896773in}{0.603022in}}%
\pgfpathlineto{\pgfqpoint{3.897307in}{0.619139in}}%
\pgfpathlineto{\pgfqpoint{3.899442in}{0.604864in}}%
\pgfpathlineto{\pgfqpoint{3.900509in}{0.615067in}}%
\pgfpathlineto{\pgfqpoint{3.901043in}{0.603939in}}%
\pgfpathlineto{\pgfqpoint{3.901577in}{0.608680in}}%
\pgfpathlineto{\pgfqpoint{3.902110in}{0.614223in}}%
\pgfpathlineto{\pgfqpoint{3.902644in}{0.608889in}}%
\pgfpathlineto{\pgfqpoint{3.904245in}{0.615965in}}%
\pgfpathlineto{\pgfqpoint{3.904779in}{0.614235in}}%
\pgfpathlineto{\pgfqpoint{3.905846in}{0.604344in}}%
\pgfpathlineto{\pgfqpoint{3.907447in}{0.609664in}}%
\pgfpathlineto{\pgfqpoint{3.907981in}{0.604952in}}%
\pgfpathlineto{\pgfqpoint{3.908515in}{0.606825in}}%
\pgfpathlineto{\pgfqpoint{3.909049in}{0.610285in}}%
\pgfpathlineto{\pgfqpoint{3.909582in}{0.606120in}}%
\pgfpathlineto{\pgfqpoint{3.910116in}{0.607723in}}%
\pgfpathlineto{\pgfqpoint{3.910650in}{0.616027in}}%
\pgfpathlineto{\pgfqpoint{3.911183in}{0.614778in}}%
\pgfpathlineto{\pgfqpoint{3.912784in}{0.612930in}}%
\pgfpathlineto{\pgfqpoint{3.913318in}{0.614695in}}%
\pgfpathlineto{\pgfqpoint{3.913852in}{0.622668in}}%
\pgfpathlineto{\pgfqpoint{3.914386in}{0.601801in}}%
\pgfpathlineto{\pgfqpoint{3.914919in}{0.620682in}}%
\pgfpathlineto{\pgfqpoint{3.915453in}{0.615054in}}%
\pgfpathlineto{\pgfqpoint{3.915987in}{0.622104in}}%
\pgfpathlineto{\pgfqpoint{3.916520in}{0.607568in}}%
\pgfpathlineto{\pgfqpoint{3.917054in}{0.618391in}}%
\pgfpathlineto{\pgfqpoint{3.918655in}{0.628804in}}%
\pgfpathlineto{\pgfqpoint{3.919189in}{0.605000in}}%
\pgfpathlineto{\pgfqpoint{3.920790in}{0.660206in}}%
\pgfpathlineto{\pgfqpoint{3.921324in}{0.650577in}}%
\pgfpathlineto{\pgfqpoint{3.921857in}{0.682110in}}%
\pgfpathlineto{\pgfqpoint{3.922391in}{0.614940in}}%
\pgfpathlineto{\pgfqpoint{3.922925in}{0.645025in}}%
\pgfpathlineto{\pgfqpoint{3.923458in}{0.654782in}}%
\pgfpathlineto{\pgfqpoint{3.923992in}{0.707586in}}%
\pgfpathlineto{\pgfqpoint{3.924526in}{0.675276in}}%
\pgfpathlineto{\pgfqpoint{3.926127in}{0.610431in}}%
\pgfpathlineto{\pgfqpoint{3.927728in}{0.696446in}}%
\pgfpathlineto{\pgfqpoint{3.929863in}{0.630717in}}%
\pgfpathlineto{\pgfqpoint{3.930930in}{0.668758in}}%
\pgfpathlineto{\pgfqpoint{3.931998in}{0.618815in}}%
\pgfpathlineto{\pgfqpoint{3.932531in}{0.633988in}}%
\pgfpathlineto{\pgfqpoint{3.933065in}{0.656216in}}%
\pgfpathlineto{\pgfqpoint{3.933599in}{0.613469in}}%
\pgfpathlineto{\pgfqpoint{3.934132in}{0.643889in}}%
\pgfpathlineto{\pgfqpoint{3.935200in}{0.609741in}}%
\pgfpathlineto{\pgfqpoint{3.935734in}{0.613081in}}%
\pgfpathlineto{\pgfqpoint{3.936267in}{0.658871in}}%
\pgfpathlineto{\pgfqpoint{3.936801in}{0.629783in}}%
\pgfpathlineto{\pgfqpoint{3.937868in}{0.613228in}}%
\pgfpathlineto{\pgfqpoint{3.939469in}{0.653218in}}%
\pgfpathlineto{\pgfqpoint{3.940537in}{0.615144in}}%
\pgfpathlineto{\pgfqpoint{3.941071in}{0.618927in}}%
\pgfpathlineto{\pgfqpoint{3.941604in}{0.619056in}}%
\pgfpathlineto{\pgfqpoint{3.943205in}{0.645146in}}%
\pgfpathlineto{\pgfqpoint{3.943739in}{0.663355in}}%
\pgfpathlineto{\pgfqpoint{3.945340in}{0.627615in}}%
\pgfpathlineto{\pgfqpoint{3.945874in}{0.635418in}}%
\pgfpathlineto{\pgfqpoint{3.946408in}{0.628512in}}%
\pgfpathlineto{\pgfqpoint{3.947475in}{0.615190in}}%
\pgfpathlineto{\pgfqpoint{3.948009in}{0.617485in}}%
\pgfpathlineto{\pgfqpoint{3.948542in}{0.619055in}}%
\pgfpathlineto{\pgfqpoint{3.949076in}{0.605430in}}%
\pgfpathlineto{\pgfqpoint{3.949610in}{0.611733in}}%
\pgfpathlineto{\pgfqpoint{3.950143in}{0.616014in}}%
\pgfpathlineto{\pgfqpoint{3.950677in}{0.610667in}}%
\pgfpathlineto{\pgfqpoint{3.951211in}{0.618631in}}%
\pgfpathlineto{\pgfqpoint{3.951745in}{0.604476in}}%
\pgfpathlineto{\pgfqpoint{3.952278in}{0.608185in}}%
\pgfpathlineto{\pgfqpoint{3.953879in}{0.612648in}}%
\pgfpathlineto{\pgfqpoint{3.954413in}{0.605957in}}%
\pgfpathlineto{\pgfqpoint{3.955480in}{0.618541in}}%
\pgfpathlineto{\pgfqpoint{3.956014in}{0.618153in}}%
\pgfpathlineto{\pgfqpoint{3.956548in}{0.619882in}}%
\pgfpathlineto{\pgfqpoint{3.957082in}{0.606041in}}%
\pgfpathlineto{\pgfqpoint{3.957615in}{0.608624in}}%
\pgfpathlineto{\pgfqpoint{3.958149in}{0.617367in}}%
\pgfpathlineto{\pgfqpoint{3.958683in}{0.610035in}}%
\pgfpathlineto{\pgfqpoint{3.959216in}{0.611447in}}%
\pgfpathlineto{\pgfqpoint{3.959750in}{0.617402in}}%
\pgfpathlineto{\pgfqpoint{3.960284in}{0.604073in}}%
\pgfpathlineto{\pgfqpoint{3.960817in}{0.610557in}}%
\pgfpathlineto{\pgfqpoint{3.961351in}{0.613838in}}%
\pgfpathlineto{\pgfqpoint{3.961885in}{0.604054in}}%
\pgfpathlineto{\pgfqpoint{3.962419in}{0.612418in}}%
\pgfpathlineto{\pgfqpoint{3.962952in}{0.628075in}}%
\pgfpathlineto{\pgfqpoint{3.963486in}{0.613239in}}%
\pgfpathlineto{\pgfqpoint{3.964020in}{0.615506in}}%
\pgfpathlineto{\pgfqpoint{3.964553in}{0.613685in}}%
\pgfpathlineto{\pgfqpoint{3.965087in}{0.610071in}}%
\pgfpathlineto{\pgfqpoint{3.965621in}{0.619617in}}%
\pgfpathlineto{\pgfqpoint{3.966155in}{0.604484in}}%
\pgfpathlineto{\pgfqpoint{3.966688in}{0.620296in}}%
\pgfpathlineto{\pgfqpoint{3.967756in}{0.618067in}}%
\pgfpathlineto{\pgfqpoint{3.968289in}{0.621060in}}%
\pgfpathlineto{\pgfqpoint{3.969357in}{0.604629in}}%
\pgfpathlineto{\pgfqpoint{3.969890in}{0.627612in}}%
\pgfpathlineto{\pgfqpoint{3.970424in}{0.611453in}}%
\pgfpathlineto{\pgfqpoint{3.970958in}{0.617329in}}%
\pgfpathlineto{\pgfqpoint{3.971492in}{0.608099in}}%
\pgfpathlineto{\pgfqpoint{3.972025in}{0.616460in}}%
\pgfpathlineto{\pgfqpoint{3.974160in}{0.627969in}}%
\pgfpathlineto{\pgfqpoint{3.974694in}{0.665728in}}%
\pgfpathlineto{\pgfqpoint{3.975227in}{0.627522in}}%
\pgfpathlineto{\pgfqpoint{3.976829in}{0.680305in}}%
\pgfpathlineto{\pgfqpoint{3.978430in}{0.630090in}}%
\pgfpathlineto{\pgfqpoint{3.979497in}{0.743699in}}%
\pgfpathlineto{\pgfqpoint{3.980564in}{0.614650in}}%
\pgfpathlineto{\pgfqpoint{3.981098in}{0.668969in}}%
\pgfpathlineto{\pgfqpoint{3.982166in}{0.668872in}}%
\pgfpathlineto{\pgfqpoint{3.982699in}{0.685269in}}%
\pgfpathlineto{\pgfqpoint{3.983233in}{0.610151in}}%
\pgfpathlineto{\pgfqpoint{3.983767in}{0.708782in}}%
\pgfpathlineto{\pgfqpoint{3.984300in}{0.646640in}}%
\pgfpathlineto{\pgfqpoint{3.984834in}{0.697018in}}%
\pgfpathlineto{\pgfqpoint{3.985368in}{0.659654in}}%
\pgfpathlineto{\pgfqpoint{3.986435in}{0.709592in}}%
\pgfpathlineto{\pgfqpoint{3.986969in}{0.631918in}}%
\pgfpathlineto{\pgfqpoint{3.987503in}{0.643459in}}%
\pgfpathlineto{\pgfqpoint{3.988036in}{0.665990in}}%
\pgfpathlineto{\pgfqpoint{3.988570in}{0.645659in}}%
\pgfpathlineto{\pgfqpoint{3.989104in}{0.619731in}}%
\pgfpathlineto{\pgfqpoint{3.989637in}{0.661514in}}%
\pgfpathlineto{\pgfqpoint{3.990171in}{0.621627in}}%
\pgfpathlineto{\pgfqpoint{3.990705in}{0.608306in}}%
\pgfpathlineto{\pgfqpoint{3.991238in}{0.674594in}}%
\pgfpathlineto{\pgfqpoint{3.991772in}{0.633490in}}%
\pgfpathlineto{\pgfqpoint{3.992306in}{0.611566in}}%
\pgfpathlineto{\pgfqpoint{3.992840in}{0.627532in}}%
\pgfpathlineto{\pgfqpoint{3.994441in}{0.650006in}}%
\pgfpathlineto{\pgfqpoint{3.994974in}{0.620970in}}%
\pgfpathlineto{\pgfqpoint{3.995508in}{0.625579in}}%
\pgfpathlineto{\pgfqpoint{3.996575in}{0.676632in}}%
\pgfpathlineto{\pgfqpoint{3.998710in}{0.608737in}}%
\pgfpathlineto{\pgfqpoint{3.999244in}{0.620199in}}%
\pgfpathlineto{\pgfqpoint{4.000845in}{0.665170in}}%
\pgfpathlineto{\pgfqpoint{4.001379in}{0.664969in}}%
\pgfpathlineto{\pgfqpoint{4.002980in}{0.642774in}}%
\pgfpathlineto{\pgfqpoint{4.003514in}{0.633212in}}%
\pgfpathlineto{\pgfqpoint{4.004047in}{0.655393in}}%
\pgfpathlineto{\pgfqpoint{4.004581in}{0.644915in}}%
\pgfpathlineto{\pgfqpoint{4.006182in}{0.605834in}}%
\pgfpathlineto{\pgfqpoint{4.006716in}{0.608576in}}%
\pgfpathlineto{\pgfqpoint{4.007249in}{0.605706in}}%
\pgfpathlineto{\pgfqpoint{4.008851in}{0.614423in}}%
\pgfpathlineto{\pgfqpoint{4.009384in}{0.605491in}}%
\pgfpathlineto{\pgfqpoint{4.009918in}{0.609728in}}%
\pgfpathlineto{\pgfqpoint{4.010985in}{0.610666in}}%
\pgfpathlineto{\pgfqpoint{4.011519in}{0.612563in}}%
\pgfpathlineto{\pgfqpoint{4.012053in}{0.607966in}}%
\pgfpathlineto{\pgfqpoint{4.012586in}{0.619132in}}%
\pgfpathlineto{\pgfqpoint{4.013120in}{0.604664in}}%
\pgfpathlineto{\pgfqpoint{4.013654in}{0.608030in}}%
\pgfpathlineto{\pgfqpoint{4.014188in}{0.609035in}}%
\pgfpathlineto{\pgfqpoint{4.015789in}{0.602931in}}%
\pgfpathlineto{\pgfqpoint{4.016856in}{0.618499in}}%
\pgfpathlineto{\pgfqpoint{4.018457in}{0.607453in}}%
\pgfpathlineto{\pgfqpoint{4.020058in}{0.623089in}}%
\pgfpathlineto{\pgfqpoint{4.020592in}{0.602899in}}%
\pgfpathlineto{\pgfqpoint{4.021126in}{0.619236in}}%
\pgfpathlineto{\pgfqpoint{4.021659in}{0.623164in}}%
\pgfpathlineto{\pgfqpoint{4.022193in}{0.620256in}}%
\pgfpathlineto{\pgfqpoint{4.023260in}{0.620868in}}%
\pgfpathlineto{\pgfqpoint{4.023794in}{0.603032in}}%
\pgfpathlineto{\pgfqpoint{4.025395in}{0.622211in}}%
\pgfpathlineto{\pgfqpoint{4.025929in}{0.614366in}}%
\pgfpathlineto{\pgfqpoint{4.026996in}{0.630777in}}%
\pgfpathlineto{\pgfqpoint{4.027530in}{0.606841in}}%
\pgfpathlineto{\pgfqpoint{4.029131in}{0.664957in}}%
\pgfpathlineto{\pgfqpoint{4.030732in}{0.630693in}}%
\pgfpathlineto{\pgfqpoint{4.032333in}{0.708520in}}%
\pgfpathlineto{\pgfqpoint{4.033401in}{0.635031in}}%
\pgfpathlineto{\pgfqpoint{4.034468in}{0.743056in}}%
\pgfpathlineto{\pgfqpoint{4.035536in}{0.619086in}}%
\pgfpathlineto{\pgfqpoint{4.036069in}{0.623496in}}%
\pgfpathlineto{\pgfqpoint{4.036603in}{0.625150in}}%
\pgfpathlineto{\pgfqpoint{4.037137in}{0.622579in}}%
\pgfpathlineto{\pgfqpoint{4.038738in}{0.695653in}}%
\pgfpathlineto{\pgfqpoint{4.040339in}{0.628458in}}%
\pgfpathlineto{\pgfqpoint{4.041406in}{0.706610in}}%
\pgfpathlineto{\pgfqpoint{4.042474in}{0.624458in}}%
\pgfpathlineto{\pgfqpoint{4.043007in}{0.626750in}}%
\pgfpathlineto{\pgfqpoint{4.043541in}{0.680803in}}%
\pgfpathlineto{\pgfqpoint{4.044075in}{0.610019in}}%
\pgfpathlineto{\pgfqpoint{4.044609in}{0.657069in}}%
\pgfpathlineto{\pgfqpoint{4.045142in}{0.620898in}}%
\pgfpathlineto{\pgfqpoint{4.045676in}{0.630403in}}%
\pgfpathlineto{\pgfqpoint{4.046210in}{0.629872in}}%
\pgfpathlineto{\pgfqpoint{4.046743in}{0.651185in}}%
\pgfpathlineto{\pgfqpoint{4.047811in}{0.607559in}}%
\pgfpathlineto{\pgfqpoint{4.048344in}{0.651437in}}%
\pgfpathlineto{\pgfqpoint{4.048878in}{0.641624in}}%
\pgfpathlineto{\pgfqpoint{4.049946in}{0.608349in}}%
\pgfpathlineto{\pgfqpoint{4.051547in}{0.653010in}}%
\pgfpathlineto{\pgfqpoint{4.052614in}{0.605241in}}%
\pgfpathlineto{\pgfqpoint{4.053148in}{0.619322in}}%
\pgfpathlineto{\pgfqpoint{4.054215in}{0.655381in}}%
\pgfpathlineto{\pgfqpoint{4.054749in}{0.649615in}}%
\pgfpathlineto{\pgfqpoint{4.056350in}{0.602367in}}%
\pgfpathlineto{\pgfqpoint{4.058485in}{0.646536in}}%
\pgfpathlineto{\pgfqpoint{4.059018in}{0.638143in}}%
\pgfpathlineto{\pgfqpoint{4.059552in}{0.639948in}}%
\pgfpathlineto{\pgfqpoint{4.060086in}{0.647442in}}%
\pgfpathlineto{\pgfqpoint{4.062221in}{0.626387in}}%
\pgfpathlineto{\pgfqpoint{4.067558in}{0.601512in}}%
\pgfpathlineto{\pgfqpoint{4.068091in}{0.602268in}}%
\pgfpathlineto{\pgfqpoint{4.069159in}{0.609540in}}%
\pgfpathlineto{\pgfqpoint{4.069692in}{0.604094in}}%
\pgfpathlineto{\pgfqpoint{4.070760in}{0.614587in}}%
\pgfpathlineto{\pgfqpoint{4.071294in}{0.605191in}}%
\pgfpathlineto{\pgfqpoint{4.071827in}{0.608055in}}%
\pgfpathlineto{\pgfqpoint{4.072361in}{0.614977in}}%
\pgfpathlineto{\pgfqpoint{4.072895in}{0.608428in}}%
\pgfpathlineto{\pgfqpoint{4.073428in}{0.605206in}}%
\pgfpathlineto{\pgfqpoint{4.075029in}{0.615005in}}%
\pgfpathlineto{\pgfqpoint{4.077698in}{0.605701in}}%
\pgfpathlineto{\pgfqpoint{4.079299in}{0.628690in}}%
\pgfpathlineto{\pgfqpoint{4.080900in}{0.609368in}}%
\pgfpathlineto{\pgfqpoint{4.081968in}{0.635104in}}%
\pgfpathlineto{\pgfqpoint{4.082501in}{0.608494in}}%
\pgfpathlineto{\pgfqpoint{4.084102in}{0.646031in}}%
\pgfpathlineto{\pgfqpoint{4.084636in}{0.622530in}}%
\pgfpathlineto{\pgfqpoint{4.085170in}{0.659018in}}%
\pgfpathlineto{\pgfqpoint{4.085703in}{0.610696in}}%
\pgfpathlineto{\pgfqpoint{4.086237in}{0.648892in}}%
\pgfpathlineto{\pgfqpoint{4.086771in}{0.643578in}}%
\pgfpathlineto{\pgfqpoint{4.087305in}{0.696564in}}%
\pgfpathlineto{\pgfqpoint{4.087838in}{0.653367in}}%
\pgfpathlineto{\pgfqpoint{4.088372in}{0.651670in}}%
\pgfpathlineto{\pgfqpoint{4.088906in}{0.644805in}}%
\pgfpathlineto{\pgfqpoint{4.089973in}{0.673114in}}%
\pgfpathlineto{\pgfqpoint{4.091040in}{0.622671in}}%
\pgfpathlineto{\pgfqpoint{4.091574in}{0.653074in}}%
\pgfpathlineto{\pgfqpoint{4.092108in}{0.608111in}}%
\pgfpathlineto{\pgfqpoint{4.093175in}{0.677213in}}%
\pgfpathlineto{\pgfqpoint{4.093709in}{0.619195in}}%
\pgfpathlineto{\pgfqpoint{4.094776in}{0.622991in}}%
\pgfpathlineto{\pgfqpoint{4.096377in}{0.672731in}}%
\pgfpathlineto{\pgfqpoint{4.097979in}{0.625036in}}%
\pgfpathlineto{\pgfqpoint{4.098512in}{0.676015in}}%
\pgfpathlineto{\pgfqpoint{4.099046in}{0.628708in}}%
\pgfpathlineto{\pgfqpoint{4.100647in}{0.640682in}}%
\pgfpathlineto{\pgfqpoint{4.101181in}{0.606992in}}%
\pgfpathlineto{\pgfqpoint{4.101715in}{0.652487in}}%
\pgfpathlineto{\pgfqpoint{4.102248in}{0.610623in}}%
\pgfpathlineto{\pgfqpoint{4.103849in}{0.630885in}}%
\pgfpathlineto{\pgfqpoint{4.104383in}{0.612803in}}%
\pgfpathlineto{\pgfqpoint{4.104917in}{0.616411in}}%
\pgfpathlineto{\pgfqpoint{4.105984in}{0.650553in}}%
\pgfpathlineto{\pgfqpoint{4.106518in}{0.632828in}}%
\pgfpathlineto{\pgfqpoint{4.107052in}{0.606578in}}%
\pgfpathlineto{\pgfqpoint{4.107585in}{0.626035in}}%
\pgfpathlineto{\pgfqpoint{4.108653in}{0.640085in}}%
\pgfpathlineto{\pgfqpoint{4.110254in}{0.607397in}}%
\pgfpathlineto{\pgfqpoint{4.111855in}{0.641100in}}%
\pgfpathlineto{\pgfqpoint{4.112389in}{0.630763in}}%
\pgfpathlineto{\pgfqpoint{4.112922in}{0.640817in}}%
\pgfpathlineto{\pgfqpoint{4.113456in}{0.604672in}}%
\pgfpathlineto{\pgfqpoint{4.113990in}{0.610224in}}%
\pgfpathlineto{\pgfqpoint{4.114523in}{0.616920in}}%
\pgfpathlineto{\pgfqpoint{4.115057in}{0.639781in}}%
\pgfpathlineto{\pgfqpoint{4.115591in}{0.634160in}}%
\pgfpathlineto{\pgfqpoint{4.116124in}{0.632961in}}%
\pgfpathlineto{\pgfqpoint{4.117192in}{0.650667in}}%
\pgfpathlineto{\pgfqpoint{4.118793in}{0.634003in}}%
\pgfpathlineto{\pgfqpoint{4.119327in}{0.636094in}}%
\pgfpathlineto{\pgfqpoint{4.120394in}{0.635948in}}%
\pgfpathlineto{\pgfqpoint{4.121461in}{0.613015in}}%
\pgfpathlineto{\pgfqpoint{4.121995in}{0.613966in}}%
\pgfpathlineto{\pgfqpoint{4.122529in}{0.608120in}}%
\pgfpathlineto{\pgfqpoint{4.124130in}{0.619260in}}%
\pgfpathlineto{\pgfqpoint{4.124664in}{0.615260in}}%
\pgfpathlineto{\pgfqpoint{4.125197in}{0.602339in}}%
\pgfpathlineto{\pgfqpoint{4.125731in}{0.607116in}}%
\pgfpathlineto{\pgfqpoint{4.126798in}{0.607973in}}%
\pgfpathlineto{\pgfqpoint{4.127332in}{0.619998in}}%
\pgfpathlineto{\pgfqpoint{4.127866in}{0.610455in}}%
\pgfpathlineto{\pgfqpoint{4.128400in}{0.618426in}}%
\pgfpathlineto{\pgfqpoint{4.128933in}{0.613352in}}%
\pgfpathlineto{\pgfqpoint{4.129467in}{0.612346in}}%
\pgfpathlineto{\pgfqpoint{4.130534in}{0.619245in}}%
\pgfpathlineto{\pgfqpoint{4.131602in}{0.610325in}}%
\pgfpathlineto{\pgfqpoint{4.133203in}{0.633617in}}%
\pgfpathlineto{\pgfqpoint{4.133737in}{0.606764in}}%
\pgfpathlineto{\pgfqpoint{4.134270in}{0.616122in}}%
\pgfpathlineto{\pgfqpoint{4.134804in}{0.616441in}}%
\pgfpathlineto{\pgfqpoint{4.135338in}{0.618143in}}%
\pgfpathlineto{\pgfqpoint{4.135871in}{0.601458in}}%
\pgfpathlineto{\pgfqpoint{4.136405in}{0.614618in}}%
\pgfpathlineto{\pgfqpoint{4.136939in}{0.619540in}}%
\pgfpathlineto{\pgfqpoint{4.137472in}{0.637724in}}%
\pgfpathlineto{\pgfqpoint{4.138006in}{0.630150in}}%
\pgfpathlineto{\pgfqpoint{4.138540in}{0.630406in}}%
\pgfpathlineto{\pgfqpoint{4.139074in}{0.628511in}}%
\pgfpathlineto{\pgfqpoint{4.140141in}{0.658631in}}%
\pgfpathlineto{\pgfqpoint{4.141208in}{0.641937in}}%
\pgfpathlineto{\pgfqpoint{4.142809in}{0.674624in}}%
\pgfpathlineto{\pgfqpoint{4.143877in}{0.633398in}}%
\pgfpathlineto{\pgfqpoint{4.144944in}{0.674656in}}%
\pgfpathlineto{\pgfqpoint{4.146545in}{0.623880in}}%
\pgfpathlineto{\pgfqpoint{4.147079in}{0.626485in}}%
\pgfpathlineto{\pgfqpoint{4.148680in}{0.666193in}}%
\pgfpathlineto{\pgfqpoint{4.149214in}{0.637495in}}%
\pgfpathlineto{\pgfqpoint{4.149748in}{0.656962in}}%
\pgfpathlineto{\pgfqpoint{4.150281in}{0.673329in}}%
\pgfpathlineto{\pgfqpoint{4.150815in}{0.620917in}}%
\pgfpathlineto{\pgfqpoint{4.151349in}{0.661010in}}%
\pgfpathlineto{\pgfqpoint{4.152416in}{0.650330in}}%
\pgfpathlineto{\pgfqpoint{4.152950in}{0.612233in}}%
\pgfpathlineto{\pgfqpoint{4.154017in}{0.669673in}}%
\pgfpathlineto{\pgfqpoint{4.154551in}{0.612346in}}%
\pgfpathlineto{\pgfqpoint{4.155085in}{0.661582in}}%
\pgfpathlineto{\pgfqpoint{4.156152in}{0.608270in}}%
\pgfpathlineto{\pgfqpoint{4.156686in}{0.652555in}}%
\pgfpathlineto{\pgfqpoint{4.157219in}{0.628444in}}%
\pgfpathlineto{\pgfqpoint{4.157753in}{0.629863in}}%
\pgfpathlineto{\pgfqpoint{4.158287in}{0.606278in}}%
\pgfpathlineto{\pgfqpoint{4.158820in}{0.635324in}}%
\pgfpathlineto{\pgfqpoint{4.159354in}{0.625529in}}%
\pgfpathlineto{\pgfqpoint{4.159888in}{0.624509in}}%
\pgfpathlineto{\pgfqpoint{4.160422in}{0.629578in}}%
\pgfpathlineto{\pgfqpoint{4.160955in}{0.654144in}}%
\pgfpathlineto{\pgfqpoint{4.162023in}{0.607841in}}%
\pgfpathlineto{\pgfqpoint{4.163090in}{0.640030in}}%
\pgfpathlineto{\pgfqpoint{4.163624in}{0.628927in}}%
\pgfpathlineto{\pgfqpoint{4.164157in}{0.628281in}}%
\pgfpathlineto{\pgfqpoint{4.165225in}{0.612514in}}%
\pgfpathlineto{\pgfqpoint{4.165759in}{0.652916in}}%
\pgfpathlineto{\pgfqpoint{4.166292in}{0.634384in}}%
\pgfpathlineto{\pgfqpoint{4.167893in}{0.613576in}}%
\pgfpathlineto{\pgfqpoint{4.168427in}{0.619776in}}%
\pgfpathlineto{\pgfqpoint{4.169495in}{0.652341in}}%
\pgfpathlineto{\pgfqpoint{4.170028in}{0.648672in}}%
\pgfpathlineto{\pgfqpoint{4.171629in}{0.609618in}}%
\pgfpathlineto{\pgfqpoint{4.172163in}{0.615075in}}%
\pgfpathlineto{\pgfqpoint{4.173764in}{0.642516in}}%
\pgfpathlineto{\pgfqpoint{4.174298in}{0.643672in}}%
\pgfpathlineto{\pgfqpoint{4.174832in}{0.629829in}}%
\pgfpathlineto{\pgfqpoint{4.175365in}{0.641923in}}%
\pgfpathlineto{\pgfqpoint{4.175899in}{0.639634in}}%
\pgfpathlineto{\pgfqpoint{4.176433in}{0.650347in}}%
\pgfpathlineto{\pgfqpoint{4.178567in}{0.619719in}}%
\pgfpathlineto{\pgfqpoint{4.179635in}{0.616142in}}%
\pgfpathlineto{\pgfqpoint{4.180169in}{0.607437in}}%
\pgfpathlineto{\pgfqpoint{4.180702in}{0.607704in}}%
\pgfpathlineto{\pgfqpoint{4.181236in}{0.618744in}}%
\pgfpathlineto{\pgfqpoint{4.181770in}{0.601230in}}%
\pgfpathlineto{\pgfqpoint{4.182303in}{0.609360in}}%
\pgfpathlineto{\pgfqpoint{4.182837in}{0.604698in}}%
\pgfpathlineto{\pgfqpoint{4.183371in}{0.616141in}}%
\pgfpathlineto{\pgfqpoint{4.183904in}{0.602887in}}%
\pgfpathlineto{\pgfqpoint{4.184438in}{0.609897in}}%
\pgfpathlineto{\pgfqpoint{4.184972in}{0.605820in}}%
\pgfpathlineto{\pgfqpoint{4.185506in}{0.617423in}}%
\pgfpathlineto{\pgfqpoint{4.186039in}{0.607281in}}%
\pgfpathlineto{\pgfqpoint{4.186573in}{0.601805in}}%
\pgfpathlineto{\pgfqpoint{4.187107in}{0.618942in}}%
\pgfpathlineto{\pgfqpoint{4.187640in}{0.610078in}}%
\pgfpathlineto{\pgfqpoint{4.188708in}{0.618507in}}%
\pgfpathlineto{\pgfqpoint{4.189775in}{0.609786in}}%
\pgfpathlineto{\pgfqpoint{4.190309in}{0.645608in}}%
\pgfpathlineto{\pgfqpoint{4.190843in}{0.614588in}}%
\pgfpathlineto{\pgfqpoint{4.191376in}{0.625808in}}%
\pgfpathlineto{\pgfqpoint{4.191910in}{0.609618in}}%
\pgfpathlineto{\pgfqpoint{4.192444in}{0.643481in}}%
\pgfpathlineto{\pgfqpoint{4.192977in}{0.638764in}}%
\pgfpathlineto{\pgfqpoint{4.193511in}{0.634719in}}%
\pgfpathlineto{\pgfqpoint{4.194045in}{0.621483in}}%
\pgfpathlineto{\pgfqpoint{4.195646in}{0.669462in}}%
\pgfpathlineto{\pgfqpoint{4.196713in}{0.628862in}}%
\pgfpathlineto{\pgfqpoint{4.197781in}{0.693335in}}%
\pgfpathlineto{\pgfqpoint{4.198314in}{0.622201in}}%
\pgfpathlineto{\pgfqpoint{4.198848in}{0.636503in}}%
\pgfpathlineto{\pgfqpoint{4.199382in}{0.659918in}}%
\pgfpathlineto{\pgfqpoint{4.199915in}{0.648140in}}%
\pgfpathlineto{\pgfqpoint{4.200449in}{0.612300in}}%
\pgfpathlineto{\pgfqpoint{4.200983in}{0.651708in}}%
\pgfpathlineto{\pgfqpoint{4.201517in}{0.637747in}}%
\pgfpathlineto{\pgfqpoint{4.202050in}{0.647670in}}%
\pgfpathlineto{\pgfqpoint{4.202584in}{0.608951in}}%
\pgfpathlineto{\pgfqpoint{4.203118in}{0.645579in}}%
\pgfpathlineto{\pgfqpoint{4.203651in}{0.672250in}}%
\pgfpathlineto{\pgfqpoint{4.204185in}{0.619064in}}%
\pgfpathlineto{\pgfqpoint{4.204719in}{0.642640in}}%
\pgfpathlineto{\pgfqpoint{4.205252in}{0.643920in}}%
\pgfpathlineto{\pgfqpoint{4.206854in}{0.656400in}}%
\pgfpathlineto{\pgfqpoint{4.208455in}{0.618395in}}%
\pgfpathlineto{\pgfqpoint{4.208988in}{0.693467in}}%
\pgfpathlineto{\pgfqpoint{4.209522in}{0.611496in}}%
\pgfpathlineto{\pgfqpoint{4.210056in}{0.648665in}}%
\pgfpathlineto{\pgfqpoint{4.210589in}{0.620597in}}%
\pgfpathlineto{\pgfqpoint{4.211657in}{0.621173in}}%
\pgfpathlineto{\pgfqpoint{4.212191in}{0.650240in}}%
\pgfpathlineto{\pgfqpoint{4.212724in}{0.618870in}}%
\pgfpathlineto{\pgfqpoint{4.213258in}{0.607419in}}%
\pgfpathlineto{\pgfqpoint{4.213792in}{0.632966in}}%
\pgfpathlineto{\pgfqpoint{4.214325in}{0.629119in}}%
\pgfpathlineto{\pgfqpoint{4.214859in}{0.628174in}}%
\pgfpathlineto{\pgfqpoint{4.215393in}{0.605317in}}%
\pgfpathlineto{\pgfqpoint{4.215926in}{0.631721in}}%
\pgfpathlineto{\pgfqpoint{4.216460in}{0.616407in}}%
\pgfpathlineto{\pgfqpoint{4.216994in}{0.608780in}}%
\pgfpathlineto{\pgfqpoint{4.217528in}{0.613531in}}%
\pgfpathlineto{\pgfqpoint{4.218061in}{0.636653in}}%
\pgfpathlineto{\pgfqpoint{4.218595in}{0.621541in}}%
\pgfpathlineto{\pgfqpoint{4.219129in}{0.610731in}}%
\pgfpathlineto{\pgfqpoint{4.220196in}{0.632260in}}%
\pgfpathlineto{\pgfqpoint{4.220730in}{0.626156in}}%
\pgfpathlineto{\pgfqpoint{4.221263in}{0.636149in}}%
\pgfpathlineto{\pgfqpoint{4.222331in}{0.603836in}}%
\pgfpathlineto{\pgfqpoint{4.223932in}{0.639059in}}%
\pgfpathlineto{\pgfqpoint{4.225533in}{0.614330in}}%
\pgfpathlineto{\pgfqpoint{4.226067in}{0.618806in}}%
\pgfpathlineto{\pgfqpoint{4.227134in}{0.630097in}}%
\pgfpathlineto{\pgfqpoint{4.227668in}{0.627919in}}%
\pgfpathlineto{\pgfqpoint{4.228202in}{0.629184in}}%
\pgfpathlineto{\pgfqpoint{4.229269in}{0.610275in}}%
\pgfpathlineto{\pgfqpoint{4.229803in}{0.615892in}}%
\pgfpathlineto{\pgfqpoint{4.230870in}{0.618171in}}%
\pgfpathlineto{\pgfqpoint{4.232471in}{0.641451in}}%
\pgfpathlineto{\pgfqpoint{4.233005in}{0.641876in}}%
\pgfpathlineto{\pgfqpoint{4.234072in}{0.628983in}}%
\pgfpathlineto{\pgfqpoint{4.234606in}{0.641000in}}%
\pgfpathlineto{\pgfqpoint{4.236207in}{0.614351in}}%
\pgfpathlineto{\pgfqpoint{4.236741in}{0.616296in}}%
\pgfpathlineto{\pgfqpoint{4.237275in}{0.627612in}}%
\pgfpathlineto{\pgfqpoint{4.237808in}{0.625691in}}%
\pgfpathlineto{\pgfqpoint{4.238876in}{0.605396in}}%
\pgfpathlineto{\pgfqpoint{4.239409in}{0.622717in}}%
\pgfpathlineto{\pgfqpoint{4.239943in}{0.611881in}}%
\pgfpathlineto{\pgfqpoint{4.240477in}{0.620532in}}%
\pgfpathlineto{\pgfqpoint{4.241010in}{0.608778in}}%
\pgfpathlineto{\pgfqpoint{4.241544in}{0.627230in}}%
\pgfpathlineto{\pgfqpoint{4.242078in}{0.607992in}}%
\pgfpathlineto{\pgfqpoint{4.243145in}{0.607613in}}%
\pgfpathlineto{\pgfqpoint{4.243679in}{0.622440in}}%
\pgfpathlineto{\pgfqpoint{4.244213in}{0.616180in}}%
\pgfpathlineto{\pgfqpoint{4.245280in}{0.642631in}}%
\pgfpathlineto{\pgfqpoint{4.245814in}{0.604940in}}%
\pgfpathlineto{\pgfqpoint{4.246881in}{0.607675in}}%
\pgfpathlineto{\pgfqpoint{4.248482in}{0.648959in}}%
\pgfpathlineto{\pgfqpoint{4.249016in}{0.619789in}}%
\pgfpathlineto{\pgfqpoint{4.249550in}{0.639314in}}%
\pgfpathlineto{\pgfqpoint{4.250083in}{0.633805in}}%
\pgfpathlineto{\pgfqpoint{4.250617in}{0.658968in}}%
\pgfpathlineto{\pgfqpoint{4.251151in}{0.632996in}}%
\pgfpathlineto{\pgfqpoint{4.251684in}{0.634150in}}%
\pgfpathlineto{\pgfqpoint{4.252218in}{0.643587in}}%
\pgfpathlineto{\pgfqpoint{4.252752in}{0.676025in}}%
\pgfpathlineto{\pgfqpoint{4.253286in}{0.643497in}}%
\pgfpathlineto{\pgfqpoint{4.253819in}{0.641813in}}%
\pgfpathlineto{\pgfqpoint{4.254353in}{0.636106in}}%
\pgfpathlineto{\pgfqpoint{4.256488in}{0.668654in}}%
\pgfpathlineto{\pgfqpoint{4.257555in}{0.612150in}}%
\pgfpathlineto{\pgfqpoint{4.258089in}{0.630470in}}%
\pgfpathlineto{\pgfqpoint{4.258623in}{0.624496in}}%
\pgfpathlineto{\pgfqpoint{4.259156in}{0.636394in}}%
\pgfpathlineto{\pgfqpoint{4.259690in}{0.611129in}}%
\pgfpathlineto{\pgfqpoint{4.260757in}{0.659608in}}%
\pgfpathlineto{\pgfqpoint{4.261291in}{0.618626in}}%
\pgfpathlineto{\pgfqpoint{4.262358in}{0.618676in}}%
\pgfpathlineto{\pgfqpoint{4.262892in}{0.661640in}}%
\pgfpathlineto{\pgfqpoint{4.263426in}{0.606787in}}%
\pgfpathlineto{\pgfqpoint{4.263960in}{0.682191in}}%
\pgfpathlineto{\pgfqpoint{4.264493in}{0.623853in}}%
\pgfpathlineto{\pgfqpoint{4.265027in}{0.623483in}}%
\pgfpathlineto{\pgfqpoint{4.265561in}{0.625873in}}%
\pgfpathlineto{\pgfqpoint{4.266094in}{0.651982in}}%
\pgfpathlineto{\pgfqpoint{4.266628in}{0.603635in}}%
\pgfpathlineto{\pgfqpoint{4.267162in}{0.651043in}}%
\pgfpathlineto{\pgfqpoint{4.267695in}{0.614265in}}%
\pgfpathlineto{\pgfqpoint{4.268763in}{0.617036in}}%
\pgfpathlineto{\pgfqpoint{4.269830in}{0.633105in}}%
\pgfpathlineto{\pgfqpoint{4.270898in}{0.627875in}}%
\pgfpathlineto{\pgfqpoint{4.271431in}{0.635863in}}%
\pgfpathlineto{\pgfqpoint{4.272499in}{0.606182in}}%
\pgfpathlineto{\pgfqpoint{4.273032in}{0.630799in}}%
\pgfpathlineto{\pgfqpoint{4.273566in}{0.623049in}}%
\pgfpathlineto{\pgfqpoint{4.274100in}{0.604430in}}%
\pgfpathlineto{\pgfqpoint{4.274634in}{0.616359in}}%
\pgfpathlineto{\pgfqpoint{4.275167in}{0.633546in}}%
\pgfpathlineto{\pgfqpoint{4.276768in}{0.607823in}}%
\pgfpathlineto{\pgfqpoint{4.278903in}{0.629673in}}%
\pgfpathlineto{\pgfqpoint{4.279437in}{0.614376in}}%
\pgfpathlineto{\pgfqpoint{4.279971in}{0.616652in}}%
\pgfpathlineto{\pgfqpoint{4.280504in}{0.617746in}}%
\pgfpathlineto{\pgfqpoint{4.281038in}{0.645939in}}%
\pgfpathlineto{\pgfqpoint{4.281572in}{0.630603in}}%
\pgfpathlineto{\pgfqpoint{4.283173in}{0.601205in}}%
\pgfpathlineto{\pgfqpoint{4.283706in}{0.609663in}}%
\pgfpathlineto{\pgfqpoint{4.284240in}{0.641948in}}%
\pgfpathlineto{\pgfqpoint{4.284774in}{0.634544in}}%
\pgfpathlineto{\pgfqpoint{4.285308in}{0.639014in}}%
\pgfpathlineto{\pgfqpoint{4.286375in}{0.609787in}}%
\pgfpathlineto{\pgfqpoint{4.286909in}{0.609950in}}%
\pgfpathlineto{\pgfqpoint{4.287976in}{0.614955in}}%
\pgfpathlineto{\pgfqpoint{4.289043in}{0.630664in}}%
\pgfpathlineto{\pgfqpoint{4.289577in}{0.625971in}}%
\pgfpathlineto{\pgfqpoint{4.290111in}{0.627291in}}%
\pgfpathlineto{\pgfqpoint{4.290645in}{0.643109in}}%
\pgfpathlineto{\pgfqpoint{4.291178in}{0.635673in}}%
\pgfpathlineto{\pgfqpoint{4.291712in}{0.640991in}}%
\pgfpathlineto{\pgfqpoint{4.292246in}{0.617421in}}%
\pgfpathlineto{\pgfqpoint{4.292779in}{0.619137in}}%
\pgfpathlineto{\pgfqpoint{4.293847in}{0.635574in}}%
\pgfpathlineto{\pgfqpoint{4.294914in}{0.616690in}}%
\pgfpathlineto{\pgfqpoint{4.295448in}{0.619404in}}%
\pgfpathlineto{\pgfqpoint{4.295982in}{0.612414in}}%
\pgfpathlineto{\pgfqpoint{4.296515in}{0.623580in}}%
\pgfpathlineto{\pgfqpoint{4.297049in}{0.610212in}}%
\pgfpathlineto{\pgfqpoint{4.297583in}{0.617868in}}%
\pgfpathlineto{\pgfqpoint{4.298116in}{0.616772in}}%
\pgfpathlineto{\pgfqpoint{4.298650in}{0.620069in}}%
\pgfpathlineto{\pgfqpoint{4.299184in}{0.609698in}}%
\pgfpathlineto{\pgfqpoint{4.299718in}{0.611762in}}%
\pgfpathlineto{\pgfqpoint{4.300785in}{0.627744in}}%
\pgfpathlineto{\pgfqpoint{4.301319in}{0.615845in}}%
\pgfpathlineto{\pgfqpoint{4.301852in}{0.618671in}}%
\pgfpathlineto{\pgfqpoint{4.302386in}{0.618055in}}%
\pgfpathlineto{\pgfqpoint{4.302920in}{0.642363in}}%
\pgfpathlineto{\pgfqpoint{4.303453in}{0.641177in}}%
\pgfpathlineto{\pgfqpoint{4.304521in}{0.623406in}}%
\pgfpathlineto{\pgfqpoint{4.306122in}{0.640968in}}%
\pgfpathlineto{\pgfqpoint{4.307189in}{0.613339in}}%
\pgfpathlineto{\pgfqpoint{4.307723in}{0.656589in}}%
\pgfpathlineto{\pgfqpoint{4.308257in}{0.656555in}}%
\pgfpathlineto{\pgfqpoint{4.309324in}{0.618283in}}%
\pgfpathlineto{\pgfqpoint{4.309858in}{0.635067in}}%
\pgfpathlineto{\pgfqpoint{4.310392in}{0.621394in}}%
\pgfpathlineto{\pgfqpoint{4.311459in}{0.643453in}}%
\pgfpathlineto{\pgfqpoint{4.313060in}{0.606490in}}%
\pgfpathlineto{\pgfqpoint{4.314127in}{0.637533in}}%
\pgfpathlineto{\pgfqpoint{4.314661in}{0.608162in}}%
\pgfpathlineto{\pgfqpoint{4.315195in}{0.639907in}}%
\pgfpathlineto{\pgfqpoint{4.316262in}{0.615703in}}%
\pgfpathlineto{\pgfqpoint{4.316796in}{0.623937in}}%
\pgfpathlineto{\pgfqpoint{4.317330in}{0.626491in}}%
\pgfpathlineto{\pgfqpoint{4.317863in}{0.644189in}}%
\pgfpathlineto{\pgfqpoint{4.318397in}{0.619609in}}%
\pgfpathlineto{\pgfqpoint{4.318931in}{0.630086in}}%
\pgfpathlineto{\pgfqpoint{4.319464in}{0.647703in}}%
\pgfpathlineto{\pgfqpoint{4.319998in}{0.620266in}}%
\pgfpathlineto{\pgfqpoint{4.320532in}{0.632535in}}%
\pgfpathlineto{\pgfqpoint{4.321066in}{0.630967in}}%
\pgfpathlineto{\pgfqpoint{4.321599in}{0.612873in}}%
\pgfpathlineto{\pgfqpoint{4.322133in}{0.619525in}}%
\pgfpathlineto{\pgfqpoint{4.322667in}{0.624669in}}%
\pgfpathlineto{\pgfqpoint{4.323734in}{0.603000in}}%
\pgfpathlineto{\pgfqpoint{4.325335in}{0.627216in}}%
\pgfpathlineto{\pgfqpoint{4.325869in}{0.606932in}}%
\pgfpathlineto{\pgfqpoint{4.326403in}{0.620332in}}%
\pgfpathlineto{\pgfqpoint{4.328004in}{0.609483in}}%
\pgfpathlineto{\pgfqpoint{4.328537in}{0.617313in}}%
\pgfpathlineto{\pgfqpoint{4.329605in}{0.603055in}}%
\pgfpathlineto{\pgfqpoint{4.330138in}{0.623364in}}%
\pgfpathlineto{\pgfqpoint{4.330672in}{0.612030in}}%
\pgfpathlineto{\pgfqpoint{4.331206in}{0.611260in}}%
\pgfpathlineto{\pgfqpoint{4.332807in}{0.603315in}}%
\pgfpathlineto{\pgfqpoint{4.333341in}{0.612194in}}%
\pgfpathlineto{\pgfqpoint{4.333874in}{0.605780in}}%
\pgfpathlineto{\pgfqpoint{4.334408in}{0.604154in}}%
\pgfpathlineto{\pgfqpoint{4.335475in}{0.619229in}}%
\pgfpathlineto{\pgfqpoint{4.336009in}{0.617196in}}%
\pgfpathlineto{\pgfqpoint{4.337077in}{0.603871in}}%
\pgfpathlineto{\pgfqpoint{4.337610in}{0.604134in}}%
\pgfpathlineto{\pgfqpoint{4.339211in}{0.616148in}}%
\pgfpathlineto{\pgfqpoint{4.340812in}{0.604125in}}%
\pgfpathlineto{\pgfqpoint{4.342414in}{0.615589in}}%
\pgfpathlineto{\pgfqpoint{4.345616in}{0.601577in}}%
\pgfpathlineto{\pgfqpoint{4.347751in}{0.614214in}}%
\pgfpathlineto{\pgfqpoint{4.348818in}{0.605376in}}%
\pgfpathlineto{\pgfqpoint{4.349352in}{0.611475in}}%
\pgfpathlineto{\pgfqpoint{4.349885in}{0.610715in}}%
\pgfpathlineto{\pgfqpoint{4.352020in}{0.603383in}}%
\pgfpathlineto{\pgfqpoint{4.352554in}{0.609612in}}%
\pgfpathlineto{\pgfqpoint{4.353088in}{0.603212in}}%
\pgfpathlineto{\pgfqpoint{4.353621in}{0.606000in}}%
\pgfpathlineto{\pgfqpoint{4.354155in}{0.600394in}}%
\pgfpathlineto{\pgfqpoint{4.354689in}{0.602331in}}%
\pgfpathlineto{\pgfqpoint{4.355222in}{0.601215in}}%
\pgfpathlineto{\pgfqpoint{4.355756in}{0.613609in}}%
\pgfpathlineto{\pgfqpoint{4.356290in}{0.603484in}}%
\pgfpathlineto{\pgfqpoint{4.356823in}{0.607428in}}%
\pgfpathlineto{\pgfqpoint{4.357357in}{0.602024in}}%
\pgfpathlineto{\pgfqpoint{4.357891in}{0.608455in}}%
\pgfpathlineto{\pgfqpoint{4.358425in}{0.607586in}}%
\pgfpathlineto{\pgfqpoint{4.358958in}{0.607880in}}%
\pgfpathlineto{\pgfqpoint{4.359492in}{0.603080in}}%
\pgfpathlineto{\pgfqpoint{4.360026in}{0.605992in}}%
\pgfpathlineto{\pgfqpoint{4.361093in}{0.608968in}}%
\pgfpathlineto{\pgfqpoint{4.362160in}{0.604598in}}%
\pgfpathlineto{\pgfqpoint{4.362694in}{0.610595in}}%
\pgfpathlineto{\pgfqpoint{4.363228in}{0.608553in}}%
\pgfpathlineto{\pgfqpoint{4.363762in}{0.600591in}}%
\pgfpathlineto{\pgfqpoint{4.364295in}{0.605878in}}%
\pgfpathlineto{\pgfqpoint{4.365896in}{0.600971in}}%
\pgfpathlineto{\pgfqpoint{4.366964in}{0.608740in}}%
\pgfpathlineto{\pgfqpoint{4.368031in}{0.601327in}}%
\pgfpathlineto{\pgfqpoint{4.368565in}{0.603703in}}%
\pgfpathlineto{\pgfqpoint{4.370166in}{0.602049in}}%
\pgfpathlineto{\pgfqpoint{4.371233in}{0.603919in}}%
\pgfpathlineto{\pgfqpoint{4.372835in}{0.601967in}}%
\pgfpathlineto{\pgfqpoint{4.373368in}{0.603279in}}%
\pgfpathlineto{\pgfqpoint{4.373902in}{0.600445in}}%
\pgfpathlineto{\pgfqpoint{4.374436in}{0.605829in}}%
\pgfpathlineto{\pgfqpoint{4.374969in}{0.601073in}}%
\pgfpathlineto{\pgfqpoint{4.376570in}{0.602837in}}%
\pgfpathlineto{\pgfqpoint{4.378172in}{0.600072in}}%
\pgfpathlineto{\pgfqpoint{4.380306in}{0.601602in}}%
\pgfpathlineto{\pgfqpoint{4.380840in}{0.600182in}}%
\pgfpathlineto{\pgfqpoint{4.381374in}{0.600797in}}%
\pgfpathlineto{\pgfqpoint{4.382441in}{0.600831in}}%
\pgfpathlineto{\pgfqpoint{4.384042in}{0.600185in}}%
\pgfpathlineto{\pgfqpoint{4.385643in}{0.600539in}}%
\pgfpathlineto{\pgfqpoint{4.386711in}{0.600287in}}%
\pgfpathlineto{\pgfqpoint{4.388846in}{0.602342in}}%
\pgfpathlineto{\pgfqpoint{4.389379in}{0.600828in}}%
\pgfpathlineto{\pgfqpoint{4.390447in}{0.615207in}}%
\pgfpathlineto{\pgfqpoint{4.390980in}{0.601449in}}%
\pgfpathlineto{\pgfqpoint{4.391514in}{0.607802in}}%
\pgfpathlineto{\pgfqpoint{4.393115in}{0.602512in}}%
\pgfpathlineto{\pgfqpoint{4.395784in}{0.600126in}}%
\pgfpathlineto{\pgfqpoint{4.397385in}{0.601406in}}%
\pgfpathlineto{\pgfqpoint{4.398986in}{0.600361in}}%
\pgfpathlineto{\pgfqpoint{4.400587in}{0.601393in}}%
\pgfpathlineto{\pgfqpoint{4.402188in}{0.600402in}}%
\pgfpathlineto{\pgfqpoint{4.404323in}{0.602037in}}%
\pgfpathlineto{\pgfqpoint{4.405390in}{0.601365in}}%
\pgfpathlineto{\pgfqpoint{4.406991in}{0.604011in}}%
\pgfpathlineto{\pgfqpoint{4.408059in}{0.600927in}}%
\pgfpathlineto{\pgfqpoint{4.408592in}{0.604116in}}%
\pgfpathlineto{\pgfqpoint{4.409126in}{0.603307in}}%
\pgfpathlineto{\pgfqpoint{4.409660in}{0.603529in}}%
\pgfpathlineto{\pgfqpoint{4.410194in}{0.601901in}}%
\pgfpathlineto{\pgfqpoint{4.411795in}{0.604667in}}%
\pgfpathlineto{\pgfqpoint{4.412328in}{0.601057in}}%
\pgfpathlineto{\pgfqpoint{4.412862in}{0.602811in}}%
\pgfpathlineto{\pgfqpoint{4.413929in}{0.605765in}}%
\pgfpathlineto{\pgfqpoint{4.414997in}{0.600953in}}%
\pgfpathlineto{\pgfqpoint{4.415531in}{0.606241in}}%
\pgfpathlineto{\pgfqpoint{4.416064in}{0.605771in}}%
\pgfpathlineto{\pgfqpoint{4.416598in}{0.604974in}}%
\pgfpathlineto{\pgfqpoint{4.418199in}{0.608004in}}%
\pgfpathlineto{\pgfqpoint{4.419800in}{0.602999in}}%
\pgfpathlineto{\pgfqpoint{4.421401in}{0.607369in}}%
\pgfpathlineto{\pgfqpoint{4.421935in}{0.612550in}}%
\pgfpathlineto{\pgfqpoint{4.423002in}{0.601251in}}%
\pgfpathlineto{\pgfqpoint{4.423536in}{0.606521in}}%
\pgfpathlineto{\pgfqpoint{4.424070in}{0.605709in}}%
\pgfpathlineto{\pgfqpoint{4.424603in}{0.608333in}}%
\pgfpathlineto{\pgfqpoint{4.425137in}{0.600645in}}%
\pgfpathlineto{\pgfqpoint{4.425671in}{0.609567in}}%
\pgfpathlineto{\pgfqpoint{4.426205in}{0.608011in}}%
\pgfpathlineto{\pgfqpoint{4.426738in}{0.601314in}}%
\pgfpathlineto{\pgfqpoint{4.427272in}{0.603832in}}%
\pgfpathlineto{\pgfqpoint{4.427806in}{0.605802in}}%
\pgfpathlineto{\pgfqpoint{4.428339in}{0.612847in}}%
\pgfpathlineto{\pgfqpoint{4.428873in}{0.603203in}}%
\pgfpathlineto{\pgfqpoint{4.429407in}{0.612439in}}%
\pgfpathlineto{\pgfqpoint{4.431008in}{0.607296in}}%
\pgfpathlineto{\pgfqpoint{4.431542in}{0.616609in}}%
\pgfpathlineto{\pgfqpoint{4.432075in}{0.602284in}}%
\pgfpathlineto{\pgfqpoint{4.432609in}{0.607209in}}%
\pgfpathlineto{\pgfqpoint{4.433676in}{0.603168in}}%
\pgfpathlineto{\pgfqpoint{4.434744in}{0.608746in}}%
\pgfpathlineto{\pgfqpoint{4.435278in}{0.608329in}}%
\pgfpathlineto{\pgfqpoint{4.435811in}{0.609674in}}%
\pgfpathlineto{\pgfqpoint{4.436345in}{0.607584in}}%
\pgfpathlineto{\pgfqpoint{4.436879in}{0.607803in}}%
\pgfpathlineto{\pgfqpoint{4.437412in}{0.609172in}}%
\pgfpathlineto{\pgfqpoint{4.438480in}{0.606066in}}%
\pgfpathlineto{\pgfqpoint{4.439013in}{0.607176in}}%
\pgfpathlineto{\pgfqpoint{4.439547in}{0.605730in}}%
\pgfpathlineto{\pgfqpoint{4.440081in}{0.601375in}}%
\pgfpathlineto{\pgfqpoint{4.440615in}{0.613348in}}%
\pgfpathlineto{\pgfqpoint{4.441148in}{0.604544in}}%
\pgfpathlineto{\pgfqpoint{4.442749in}{0.609589in}}%
\pgfpathlineto{\pgfqpoint{4.444350in}{0.603785in}}%
\pgfpathlineto{\pgfqpoint{4.444884in}{0.609101in}}%
\pgfpathlineto{\pgfqpoint{4.445418in}{0.608682in}}%
\pgfpathlineto{\pgfqpoint{4.445952in}{0.604279in}}%
\pgfpathlineto{\pgfqpoint{4.446485in}{0.606759in}}%
\pgfpathlineto{\pgfqpoint{4.447019in}{0.605368in}}%
\pgfpathlineto{\pgfqpoint{4.447553in}{0.608499in}}%
\pgfpathlineto{\pgfqpoint{4.448086in}{0.602220in}}%
\pgfpathlineto{\pgfqpoint{4.448620in}{0.606773in}}%
\pgfpathlineto{\pgfqpoint{4.449154in}{0.603137in}}%
\pgfpathlineto{\pgfqpoint{4.450755in}{0.613779in}}%
\pgfpathlineto{\pgfqpoint{4.452890in}{0.602945in}}%
\pgfpathlineto{\pgfqpoint{4.454491in}{0.617928in}}%
\pgfpathlineto{\pgfqpoint{4.455024in}{0.614094in}}%
\pgfpathlineto{\pgfqpoint{4.455558in}{0.601006in}}%
\pgfpathlineto{\pgfqpoint{4.456092in}{0.609091in}}%
\pgfpathlineto{\pgfqpoint{4.456626in}{0.605970in}}%
\pgfpathlineto{\pgfqpoint{4.457159in}{0.608772in}}%
\pgfpathlineto{\pgfqpoint{4.458760in}{0.618195in}}%
\pgfpathlineto{\pgfqpoint{4.460361in}{0.601415in}}%
\pgfpathlineto{\pgfqpoint{4.461429in}{0.603483in}}%
\pgfpathlineto{\pgfqpoint{4.463030in}{0.628156in}}%
\pgfpathlineto{\pgfqpoint{4.463564in}{0.606247in}}%
\pgfpathlineto{\pgfqpoint{4.464097in}{0.616588in}}%
\pgfpathlineto{\pgfqpoint{4.465165in}{0.617043in}}%
\pgfpathlineto{\pgfqpoint{4.465698in}{0.619650in}}%
\pgfpathlineto{\pgfqpoint{4.466232in}{0.633054in}}%
\pgfpathlineto{\pgfqpoint{4.466766in}{0.604883in}}%
\pgfpathlineto{\pgfqpoint{4.467300in}{0.613346in}}%
\pgfpathlineto{\pgfqpoint{4.467833in}{0.613682in}}%
\pgfpathlineto{\pgfqpoint{4.468901in}{0.626712in}}%
\pgfpathlineto{\pgfqpoint{4.469434in}{0.624449in}}%
\pgfpathlineto{\pgfqpoint{4.469968in}{0.601840in}}%
\pgfpathlineto{\pgfqpoint{4.470502in}{0.617931in}}%
\pgfpathlineto{\pgfqpoint{4.471569in}{0.625590in}}%
\pgfpathlineto{\pgfqpoint{4.472103in}{0.617045in}}%
\pgfpathlineto{\pgfqpoint{4.472637in}{0.617470in}}%
\pgfpathlineto{\pgfqpoint{4.473170in}{0.636543in}}%
\pgfpathlineto{\pgfqpoint{4.473704in}{0.624390in}}%
\pgfpathlineto{\pgfqpoint{4.475305in}{0.600370in}}%
\pgfpathlineto{\pgfqpoint{4.475839in}{0.612245in}}%
\pgfpathlineto{\pgfqpoint{4.476372in}{0.610791in}}%
\pgfpathlineto{\pgfqpoint{4.477440in}{0.634596in}}%
\pgfpathlineto{\pgfqpoint{4.477974in}{0.613773in}}%
\pgfpathlineto{\pgfqpoint{4.478507in}{0.617999in}}%
\pgfpathlineto{\pgfqpoint{4.479041in}{0.624922in}}%
\pgfpathlineto{\pgfqpoint{4.480108in}{0.603028in}}%
\pgfpathlineto{\pgfqpoint{4.481709in}{0.620908in}}%
\pgfpathlineto{\pgfqpoint{4.482243in}{0.606751in}}%
\pgfpathlineto{\pgfqpoint{4.482777in}{0.628170in}}%
\pgfpathlineto{\pgfqpoint{4.483311in}{0.625687in}}%
\pgfpathlineto{\pgfqpoint{4.484378in}{0.609357in}}%
\pgfpathlineto{\pgfqpoint{4.484912in}{0.632305in}}%
\pgfpathlineto{\pgfqpoint{4.485445in}{0.617903in}}%
\pgfpathlineto{\pgfqpoint{4.485979in}{0.627570in}}%
\pgfpathlineto{\pgfqpoint{4.486513in}{0.623441in}}%
\pgfpathlineto{\pgfqpoint{4.487046in}{0.625686in}}%
\pgfpathlineto{\pgfqpoint{4.487580in}{0.605663in}}%
\pgfpathlineto{\pgfqpoint{4.488114in}{0.622124in}}%
\pgfpathlineto{\pgfqpoint{4.489181in}{0.609576in}}%
\pgfpathlineto{\pgfqpoint{4.490782in}{0.622671in}}%
\pgfpathlineto{\pgfqpoint{4.491316in}{0.605161in}}%
\pgfpathlineto{\pgfqpoint{4.491850in}{0.615822in}}%
\pgfpathlineto{\pgfqpoint{4.492383in}{0.616022in}}%
\pgfpathlineto{\pgfqpoint{4.492917in}{0.619947in}}%
\pgfpathlineto{\pgfqpoint{4.494518in}{0.603066in}}%
\pgfpathlineto{\pgfqpoint{4.495586in}{0.621861in}}%
\pgfpathlineto{\pgfqpoint{4.497187in}{0.600524in}}%
\pgfpathlineto{\pgfqpoint{4.498254in}{0.612479in}}%
\pgfpathlineto{\pgfqpoint{4.499322in}{0.603284in}}%
\pgfpathlineto{\pgfqpoint{4.500923in}{0.613816in}}%
\pgfpathlineto{\pgfqpoint{4.501456in}{0.614062in}}%
\pgfpathlineto{\pgfqpoint{4.504125in}{0.600914in}}%
\pgfpathlineto{\pgfqpoint{4.504659in}{0.619822in}}%
\pgfpathlineto{\pgfqpoint{4.505192in}{0.608128in}}%
\pgfpathlineto{\pgfqpoint{4.505726in}{0.613822in}}%
\pgfpathlineto{\pgfqpoint{4.506260in}{0.604488in}}%
\pgfpathlineto{\pgfqpoint{4.506793in}{0.607813in}}%
\pgfpathlineto{\pgfqpoint{4.508928in}{0.619160in}}%
\pgfpathlineto{\pgfqpoint{4.509462in}{0.601768in}}%
\pgfpathlineto{\pgfqpoint{4.509996in}{0.606009in}}%
\pgfpathlineto{\pgfqpoint{4.511063in}{0.624606in}}%
\pgfpathlineto{\pgfqpoint{4.511597in}{0.615868in}}%
\pgfpathlineto{\pgfqpoint{4.512130in}{0.606785in}}%
\pgfpathlineto{\pgfqpoint{4.512664in}{0.616197in}}%
\pgfpathlineto{\pgfqpoint{4.513198in}{0.607393in}}%
\pgfpathlineto{\pgfqpoint{4.513732in}{0.608205in}}%
\pgfpathlineto{\pgfqpoint{4.514799in}{0.619866in}}%
\pgfpathlineto{\pgfqpoint{4.515333in}{0.618201in}}%
\pgfpathlineto{\pgfqpoint{4.516934in}{0.608027in}}%
\pgfpathlineto{\pgfqpoint{4.518001in}{0.612939in}}%
\pgfpathlineto{\pgfqpoint{4.518535in}{0.608734in}}%
\pgfpathlineto{\pgfqpoint{4.519069in}{0.624583in}}%
\pgfpathlineto{\pgfqpoint{4.519602in}{0.607589in}}%
\pgfpathlineto{\pgfqpoint{4.520136in}{0.622363in}}%
\pgfpathlineto{\pgfqpoint{4.520670in}{0.614537in}}%
\pgfpathlineto{\pgfqpoint{4.521203in}{0.618920in}}%
\pgfpathlineto{\pgfqpoint{4.521737in}{0.635272in}}%
\pgfpathlineto{\pgfqpoint{4.522271in}{0.634797in}}%
\pgfpathlineto{\pgfqpoint{4.523338in}{0.613714in}}%
\pgfpathlineto{\pgfqpoint{4.523872in}{0.632966in}}%
\pgfpathlineto{\pgfqpoint{4.525473in}{0.608115in}}%
\pgfpathlineto{\pgfqpoint{4.526007in}{0.645176in}}%
\pgfpathlineto{\pgfqpoint{4.526540in}{0.615811in}}%
\pgfpathlineto{\pgfqpoint{4.527074in}{0.613332in}}%
\pgfpathlineto{\pgfqpoint{4.527608in}{0.615566in}}%
\pgfpathlineto{\pgfqpoint{4.528141in}{0.632404in}}%
\pgfpathlineto{\pgfqpoint{4.528675in}{0.619615in}}%
\pgfpathlineto{\pgfqpoint{4.529209in}{0.602973in}}%
\pgfpathlineto{\pgfqpoint{4.529743in}{0.622014in}}%
\pgfpathlineto{\pgfqpoint{4.530276in}{0.612979in}}%
\pgfpathlineto{\pgfqpoint{4.531877in}{0.618002in}}%
\pgfpathlineto{\pgfqpoint{4.532411in}{0.634697in}}%
\pgfpathlineto{\pgfqpoint{4.533478in}{0.607064in}}%
\pgfpathlineto{\pgfqpoint{4.534012in}{0.621890in}}%
\pgfpathlineto{\pgfqpoint{4.534546in}{0.617985in}}%
\pgfpathlineto{\pgfqpoint{4.535613in}{0.607780in}}%
\pgfpathlineto{\pgfqpoint{4.536681in}{0.616983in}}%
\pgfpathlineto{\pgfqpoint{4.537214in}{0.609972in}}%
\pgfpathlineto{\pgfqpoint{4.537748in}{0.620372in}}%
\pgfpathlineto{\pgfqpoint{4.539349in}{0.602681in}}%
\pgfpathlineto{\pgfqpoint{4.539883in}{0.622456in}}%
\pgfpathlineto{\pgfqpoint{4.540950in}{0.622206in}}%
\pgfpathlineto{\pgfqpoint{4.541484in}{0.605382in}}%
\pgfpathlineto{\pgfqpoint{4.542018in}{0.629785in}}%
\pgfpathlineto{\pgfqpoint{4.542551in}{0.606134in}}%
\pgfpathlineto{\pgfqpoint{4.543085in}{0.619309in}}%
\pgfpathlineto{\pgfqpoint{4.543619in}{0.605234in}}%
\pgfpathlineto{\pgfqpoint{4.544152in}{0.606889in}}%
\pgfpathlineto{\pgfqpoint{4.545754in}{0.614144in}}%
\pgfpathlineto{\pgfqpoint{4.546287in}{0.605796in}}%
\pgfpathlineto{\pgfqpoint{4.546821in}{0.612202in}}%
\pgfpathlineto{\pgfqpoint{4.547355in}{0.613262in}}%
\pgfpathlineto{\pgfqpoint{4.547888in}{0.618718in}}%
\pgfpathlineto{\pgfqpoint{4.549489in}{0.600406in}}%
\pgfpathlineto{\pgfqpoint{4.550023in}{0.615955in}}%
\pgfpathlineto{\pgfqpoint{4.550557in}{0.609513in}}%
\pgfpathlineto{\pgfqpoint{4.551624in}{0.606426in}}%
\pgfpathlineto{\pgfqpoint{4.552692in}{0.612158in}}%
\pgfpathlineto{\pgfqpoint{4.553225in}{0.609836in}}%
\pgfpathlineto{\pgfqpoint{4.555360in}{0.603594in}}%
\pgfpathlineto{\pgfqpoint{4.555894in}{0.609032in}}%
\pgfpathlineto{\pgfqpoint{4.556428in}{0.605039in}}%
\pgfpathlineto{\pgfqpoint{4.556961in}{0.608004in}}%
\pgfpathlineto{\pgfqpoint{4.557495in}{0.605405in}}%
\pgfpathlineto{\pgfqpoint{4.558029in}{0.606502in}}%
\pgfpathlineto{\pgfqpoint{4.559096in}{0.600728in}}%
\pgfpathlineto{\pgfqpoint{4.560163in}{0.615559in}}%
\pgfpathlineto{\pgfqpoint{4.560697in}{0.606297in}}%
\pgfpathlineto{\pgfqpoint{4.561231in}{0.602236in}}%
\pgfpathlineto{\pgfqpoint{4.562832in}{0.609827in}}%
\pgfpathlineto{\pgfqpoint{4.563366in}{0.602189in}}%
\pgfpathlineto{\pgfqpoint{4.563899in}{0.604591in}}%
\pgfpathlineto{\pgfqpoint{4.564433in}{0.605763in}}%
\pgfpathlineto{\pgfqpoint{4.566034in}{0.620029in}}%
\pgfpathlineto{\pgfqpoint{4.567635in}{0.601620in}}%
\pgfpathlineto{\pgfqpoint{4.569236in}{0.612906in}}%
\pgfpathlineto{\pgfqpoint{4.569770in}{0.610652in}}%
\pgfpathlineto{\pgfqpoint{4.570304in}{0.613375in}}%
\pgfpathlineto{\pgfqpoint{4.571905in}{0.606728in}}%
\pgfpathlineto{\pgfqpoint{4.573506in}{0.602874in}}%
\pgfpathlineto{\pgfqpoint{4.574040in}{0.629758in}}%
\pgfpathlineto{\pgfqpoint{4.574573in}{0.602196in}}%
\pgfpathlineto{\pgfqpoint{4.576175in}{0.615330in}}%
\pgfpathlineto{\pgfqpoint{4.576708in}{0.618882in}}%
\pgfpathlineto{\pgfqpoint{4.578309in}{0.607030in}}%
\pgfpathlineto{\pgfqpoint{4.579910in}{0.622386in}}%
\pgfpathlineto{\pgfqpoint{4.580444in}{0.610456in}}%
\pgfpathlineto{\pgfqpoint{4.580978in}{0.622231in}}%
\pgfpathlineto{\pgfqpoint{4.581512in}{0.623327in}}%
\pgfpathlineto{\pgfqpoint{4.583113in}{0.604910in}}%
\pgfpathlineto{\pgfqpoint{4.583646in}{0.620125in}}%
\pgfpathlineto{\pgfqpoint{4.584180in}{0.603333in}}%
\pgfpathlineto{\pgfqpoint{4.584714in}{0.616755in}}%
\pgfpathlineto{\pgfqpoint{4.585247in}{0.622461in}}%
\pgfpathlineto{\pgfqpoint{4.586849in}{0.606664in}}%
\pgfpathlineto{\pgfqpoint{4.587916in}{0.616134in}}%
\pgfpathlineto{\pgfqpoint{4.588450in}{0.606748in}}%
\pgfpathlineto{\pgfqpoint{4.588983in}{0.616050in}}%
\pgfpathlineto{\pgfqpoint{4.589517in}{0.619113in}}%
\pgfpathlineto{\pgfqpoint{4.590051in}{0.605144in}}%
\pgfpathlineto{\pgfqpoint{4.590584in}{0.605950in}}%
\pgfpathlineto{\pgfqpoint{4.591652in}{0.611771in}}%
\pgfpathlineto{\pgfqpoint{4.592719in}{0.600720in}}%
\pgfpathlineto{\pgfqpoint{4.593787in}{0.612592in}}%
\pgfpathlineto{\pgfqpoint{4.594320in}{0.602757in}}%
\pgfpathlineto{\pgfqpoint{4.594854in}{0.604104in}}%
\pgfpathlineto{\pgfqpoint{4.596455in}{0.613046in}}%
\pgfpathlineto{\pgfqpoint{4.596989in}{0.618813in}}%
\pgfpathlineto{\pgfqpoint{4.598056in}{0.608109in}}%
\pgfpathlineto{\pgfqpoint{4.598590in}{0.611790in}}%
\pgfpathlineto{\pgfqpoint{4.599124in}{0.612816in}}%
\pgfpathlineto{\pgfqpoint{4.599657in}{0.610971in}}%
\pgfpathlineto{\pgfqpoint{4.600191in}{0.605017in}}%
\pgfpathlineto{\pgfqpoint{4.600725in}{0.607089in}}%
\pgfpathlineto{\pgfqpoint{4.601258in}{0.610966in}}%
\pgfpathlineto{\pgfqpoint{4.601792in}{0.608758in}}%
\pgfpathlineto{\pgfqpoint{4.602326in}{0.603364in}}%
\pgfpathlineto{\pgfqpoint{4.602860in}{0.613246in}}%
\pgfpathlineto{\pgfqpoint{4.603393in}{0.611769in}}%
\pgfpathlineto{\pgfqpoint{4.603927in}{0.603467in}}%
\pgfpathlineto{\pgfqpoint{4.604461in}{0.605607in}}%
\pgfpathlineto{\pgfqpoint{4.605528in}{0.605845in}}%
\pgfpathlineto{\pgfqpoint{4.606062in}{0.609279in}}%
\pgfpathlineto{\pgfqpoint{4.606595in}{0.601163in}}%
\pgfpathlineto{\pgfqpoint{4.607129in}{0.611491in}}%
\pgfpathlineto{\pgfqpoint{4.607663in}{0.603666in}}%
\pgfpathlineto{\pgfqpoint{4.609264in}{0.611538in}}%
\pgfpathlineto{\pgfqpoint{4.609798in}{0.600875in}}%
\pgfpathlineto{\pgfqpoint{4.610331in}{0.609772in}}%
\pgfpathlineto{\pgfqpoint{4.611399in}{0.604588in}}%
\pgfpathlineto{\pgfqpoint{4.611932in}{0.608481in}}%
\pgfpathlineto{\pgfqpoint{4.612466in}{0.605261in}}%
\pgfpathlineto{\pgfqpoint{4.613534in}{0.600246in}}%
\pgfpathlineto{\pgfqpoint{4.614067in}{0.603054in}}%
\pgfpathlineto{\pgfqpoint{4.614601in}{0.601821in}}%
\pgfpathlineto{\pgfqpoint{4.615135in}{0.611560in}}%
\pgfpathlineto{\pgfqpoint{4.615668in}{0.601891in}}%
\pgfpathlineto{\pgfqpoint{4.616736in}{0.609585in}}%
\pgfpathlineto{\pgfqpoint{4.617269in}{0.606877in}}%
\pgfpathlineto{\pgfqpoint{4.618337in}{0.601390in}}%
\pgfpathlineto{\pgfqpoint{4.619404in}{0.602744in}}%
\pgfpathlineto{\pgfqpoint{4.621005in}{0.613538in}}%
\pgfpathlineto{\pgfqpoint{4.621539in}{0.602708in}}%
\pgfpathlineto{\pgfqpoint{4.622073in}{0.608731in}}%
\pgfpathlineto{\pgfqpoint{4.622606in}{0.610495in}}%
\pgfpathlineto{\pgfqpoint{4.623140in}{0.601481in}}%
\pgfpathlineto{\pgfqpoint{4.623674in}{0.607933in}}%
\pgfpathlineto{\pgfqpoint{4.624208in}{0.608397in}}%
\pgfpathlineto{\pgfqpoint{4.624741in}{0.601185in}}%
\pgfpathlineto{\pgfqpoint{4.625275in}{0.606507in}}%
\pgfpathlineto{\pgfqpoint{4.626342in}{0.612945in}}%
\pgfpathlineto{\pgfqpoint{4.626876in}{0.607972in}}%
\pgfpathlineto{\pgfqpoint{4.627410in}{0.609173in}}%
\pgfpathlineto{\pgfqpoint{4.628477in}{0.603381in}}%
\pgfpathlineto{\pgfqpoint{4.629011in}{0.610779in}}%
\pgfpathlineto{\pgfqpoint{4.629545in}{0.608594in}}%
\pgfpathlineto{\pgfqpoint{4.630078in}{0.608332in}}%
\pgfpathlineto{\pgfqpoint{4.630612in}{0.602997in}}%
\pgfpathlineto{\pgfqpoint{4.631146in}{0.604640in}}%
\pgfpathlineto{\pgfqpoint{4.631679in}{0.620470in}}%
\pgfpathlineto{\pgfqpoint{4.632213in}{0.603909in}}%
\pgfpathlineto{\pgfqpoint{4.632747in}{0.604273in}}%
\pgfpathlineto{\pgfqpoint{4.633280in}{0.603186in}}%
\pgfpathlineto{\pgfqpoint{4.633814in}{0.609187in}}%
\pgfpathlineto{\pgfqpoint{4.634348in}{0.605526in}}%
\pgfpathlineto{\pgfqpoint{4.635415in}{0.609846in}}%
\pgfpathlineto{\pgfqpoint{4.635949in}{0.603177in}}%
\pgfpathlineto{\pgfqpoint{4.636483in}{0.626736in}}%
\pgfpathlineto{\pgfqpoint{4.637016in}{0.604431in}}%
\pgfpathlineto{\pgfqpoint{4.637550in}{0.610613in}}%
\pgfpathlineto{\pgfqpoint{4.638084in}{0.606472in}}%
\pgfpathlineto{\pgfqpoint{4.639151in}{0.613034in}}%
\pgfpathlineto{\pgfqpoint{4.639685in}{0.606298in}}%
\pgfpathlineto{\pgfqpoint{4.640219in}{0.614599in}}%
\pgfpathlineto{\pgfqpoint{4.640752in}{0.613804in}}%
\pgfpathlineto{\pgfqpoint{4.641820in}{0.604231in}}%
\pgfpathlineto{\pgfqpoint{4.642353in}{0.604334in}}%
\pgfpathlineto{\pgfqpoint{4.642887in}{0.615901in}}%
\pgfpathlineto{\pgfqpoint{4.643421in}{0.606023in}}%
\pgfpathlineto{\pgfqpoint{4.643955in}{0.609462in}}%
\pgfpathlineto{\pgfqpoint{4.644488in}{0.607791in}}%
\pgfpathlineto{\pgfqpoint{4.645556in}{0.602854in}}%
\pgfpathlineto{\pgfqpoint{4.646089in}{0.603118in}}%
\pgfpathlineto{\pgfqpoint{4.646623in}{0.610274in}}%
\pgfpathlineto{\pgfqpoint{4.647157in}{0.608739in}}%
\pgfpathlineto{\pgfqpoint{4.648758in}{0.602662in}}%
\pgfpathlineto{\pgfqpoint{4.649292in}{0.603507in}}%
\pgfpathlineto{\pgfqpoint{4.649825in}{0.602091in}}%
\pgfpathlineto{\pgfqpoint{4.651426in}{0.608462in}}%
\pgfpathlineto{\pgfqpoint{4.651960in}{0.602195in}}%
\pgfpathlineto{\pgfqpoint{4.652494in}{0.612306in}}%
\pgfpathlineto{\pgfqpoint{4.653027in}{0.607058in}}%
\pgfpathlineto{\pgfqpoint{4.653561in}{0.610108in}}%
\pgfpathlineto{\pgfqpoint{4.654095in}{0.604574in}}%
\pgfpathlineto{\pgfqpoint{4.654629in}{0.612538in}}%
\pgfpathlineto{\pgfqpoint{4.655162in}{0.604367in}}%
\pgfpathlineto{\pgfqpoint{4.655696in}{0.601054in}}%
\pgfpathlineto{\pgfqpoint{4.656230in}{0.602331in}}%
\pgfpathlineto{\pgfqpoint{4.656763in}{0.602095in}}%
\pgfpathlineto{\pgfqpoint{4.658364in}{0.613209in}}%
\pgfpathlineto{\pgfqpoint{4.658898in}{0.602499in}}%
\pgfpathlineto{\pgfqpoint{4.659432in}{0.604808in}}%
\pgfpathlineto{\pgfqpoint{4.661033in}{0.605365in}}%
\pgfpathlineto{\pgfqpoint{4.661567in}{0.602195in}}%
\pgfpathlineto{\pgfqpoint{4.662100in}{0.604307in}}%
\pgfpathlineto{\pgfqpoint{4.662634in}{0.603764in}}%
\pgfpathlineto{\pgfqpoint{4.663168in}{0.604451in}}%
\pgfpathlineto{\pgfqpoint{4.663701in}{0.604469in}}%
\pgfpathlineto{\pgfqpoint{4.664235in}{0.606229in}}%
\pgfpathlineto{\pgfqpoint{4.664769in}{0.605142in}}%
\pgfpathlineto{\pgfqpoint{4.665303in}{0.602790in}}%
\pgfpathlineto{\pgfqpoint{4.665836in}{0.603358in}}%
\pgfpathlineto{\pgfqpoint{4.666370in}{0.606159in}}%
\pgfpathlineto{\pgfqpoint{4.666904in}{0.605287in}}%
\pgfpathlineto{\pgfqpoint{4.668505in}{0.601682in}}%
\pgfpathlineto{\pgfqpoint{4.669038in}{0.602993in}}%
\pgfpathlineto{\pgfqpoint{4.669572in}{0.601963in}}%
\pgfpathlineto{\pgfqpoint{4.670106in}{0.600987in}}%
\pgfpathlineto{\pgfqpoint{4.670640in}{0.601456in}}%
\pgfpathlineto{\pgfqpoint{4.671173in}{0.607283in}}%
\pgfpathlineto{\pgfqpoint{4.671707in}{0.602776in}}%
\pgfpathlineto{\pgfqpoint{4.672241in}{0.604883in}}%
\pgfpathlineto{\pgfqpoint{4.673842in}{0.601168in}}%
\pgfpathlineto{\pgfqpoint{4.674909in}{0.605192in}}%
\pgfpathlineto{\pgfqpoint{4.675443in}{0.604917in}}%
\pgfpathlineto{\pgfqpoint{4.676510in}{0.600263in}}%
\pgfpathlineto{\pgfqpoint{4.677044in}{0.607485in}}%
\pgfpathlineto{\pgfqpoint{4.677578in}{0.606873in}}%
\pgfpathlineto{\pgfqpoint{4.678645in}{0.601672in}}%
\pgfpathlineto{\pgfqpoint{4.679179in}{0.601999in}}%
\pgfpathlineto{\pgfqpoint{4.679712in}{0.602480in}}%
\pgfpathlineto{\pgfqpoint{4.681314in}{0.607646in}}%
\pgfpathlineto{\pgfqpoint{4.682381in}{0.601850in}}%
\pgfpathlineto{\pgfqpoint{4.683448in}{0.606225in}}%
\pgfpathlineto{\pgfqpoint{4.683982in}{0.605295in}}%
\pgfpathlineto{\pgfqpoint{4.684516in}{0.604258in}}%
\pgfpathlineto{\pgfqpoint{4.685049in}{0.605699in}}%
\pgfpathlineto{\pgfqpoint{4.686117in}{0.603779in}}%
\pgfpathlineto{\pgfqpoint{4.687184in}{0.607236in}}%
\pgfpathlineto{\pgfqpoint{4.688252in}{0.603501in}}%
\pgfpathlineto{\pgfqpoint{4.688785in}{0.604524in}}%
\pgfpathlineto{\pgfqpoint{4.689853in}{0.605138in}}%
\pgfpathlineto{\pgfqpoint{4.690920in}{0.602626in}}%
\pgfpathlineto{\pgfqpoint{4.691454in}{0.609604in}}%
\pgfpathlineto{\pgfqpoint{4.691988in}{0.604047in}}%
\pgfpathlineto{\pgfqpoint{4.692521in}{0.605580in}}%
\pgfpathlineto{\pgfqpoint{4.693055in}{0.605126in}}%
\pgfpathlineto{\pgfqpoint{4.693589in}{0.601827in}}%
\pgfpathlineto{\pgfqpoint{4.694122in}{0.606571in}}%
\pgfpathlineto{\pgfqpoint{4.694656in}{0.603112in}}%
\pgfpathlineto{\pgfqpoint{4.695190in}{0.603835in}}%
\pgfpathlineto{\pgfqpoint{4.695723in}{0.612677in}}%
\pgfpathlineto{\pgfqpoint{4.696257in}{0.602041in}}%
\pgfpathlineto{\pgfqpoint{4.696791in}{0.611379in}}%
\pgfpathlineto{\pgfqpoint{4.698392in}{0.601621in}}%
\pgfpathlineto{\pgfqpoint{4.698926in}{0.602780in}}%
\pgfpathlineto{\pgfqpoint{4.699459in}{0.601255in}}%
\pgfpathlineto{\pgfqpoint{4.699993in}{0.610042in}}%
\pgfpathlineto{\pgfqpoint{4.700527in}{0.605167in}}%
\pgfpathlineto{\pgfqpoint{4.701060in}{0.602685in}}%
\pgfpathlineto{\pgfqpoint{4.702128in}{0.602953in}}%
\pgfpathlineto{\pgfqpoint{4.703729in}{0.605043in}}%
\pgfpathlineto{\pgfqpoint{4.704263in}{0.601479in}}%
\pgfpathlineto{\pgfqpoint{4.704796in}{0.602450in}}%
\pgfpathlineto{\pgfqpoint{4.706398in}{0.600950in}}%
\pgfpathlineto{\pgfqpoint{4.707465in}{0.604950in}}%
\pgfpathlineto{\pgfqpoint{4.709066in}{0.602998in}}%
\pgfpathlineto{\pgfqpoint{4.709600in}{0.604896in}}%
\pgfpathlineto{\pgfqpoint{4.710133in}{0.603848in}}%
\pgfpathlineto{\pgfqpoint{4.710667in}{0.600298in}}%
\pgfpathlineto{\pgfqpoint{4.711201in}{0.601153in}}%
\pgfpathlineto{\pgfqpoint{4.711735in}{0.601344in}}%
\pgfpathlineto{\pgfqpoint{4.712268in}{0.606172in}}%
\pgfpathlineto{\pgfqpoint{4.712802in}{0.600701in}}%
\pgfpathlineto{\pgfqpoint{4.713336in}{0.604283in}}%
\pgfpathlineto{\pgfqpoint{4.713869in}{0.603177in}}%
\pgfpathlineto{\pgfqpoint{4.714403in}{0.603962in}}%
\pgfpathlineto{\pgfqpoint{4.716004in}{0.602303in}}%
\pgfpathlineto{\pgfqpoint{4.717605in}{0.600820in}}%
\pgfpathlineto{\pgfqpoint{4.718139in}{0.604523in}}%
\pgfpathlineto{\pgfqpoint{4.718673in}{0.602140in}}%
\pgfpathlineto{\pgfqpoint{4.720274in}{0.603786in}}%
\pgfpathlineto{\pgfqpoint{4.720807in}{0.602751in}}%
\pgfpathlineto{\pgfqpoint{4.722942in}{0.601763in}}%
\pgfpathlineto{\pgfqpoint{4.723476in}{0.602303in}}%
\pgfpathlineto{\pgfqpoint{4.724010in}{0.601539in}}%
\pgfpathlineto{\pgfqpoint{4.724543in}{0.606600in}}%
\pgfpathlineto{\pgfqpoint{4.725077in}{0.600949in}}%
\pgfpathlineto{\pgfqpoint{4.725611in}{0.601996in}}%
\pgfpathlineto{\pgfqpoint{4.726144in}{0.605452in}}%
\pgfpathlineto{\pgfqpoint{4.726678in}{0.604463in}}%
\pgfpathlineto{\pgfqpoint{4.727212in}{0.602466in}}%
\pgfpathlineto{\pgfqpoint{4.727746in}{0.603204in}}%
\pgfpathlineto{\pgfqpoint{4.728279in}{0.604308in}}%
\pgfpathlineto{\pgfqpoint{4.728813in}{0.603895in}}%
\pgfpathlineto{\pgfqpoint{4.729880in}{0.601631in}}%
\pgfpathlineto{\pgfqpoint{4.730414in}{0.604288in}}%
\pgfpathlineto{\pgfqpoint{4.730948in}{0.600246in}}%
\pgfpathlineto{\pgfqpoint{4.731481in}{0.600549in}}%
\pgfpathlineto{\pgfqpoint{4.733083in}{0.601885in}}%
\pgfpathlineto{\pgfqpoint{4.733616in}{0.605871in}}%
\pgfpathlineto{\pgfqpoint{4.734150in}{0.603812in}}%
\pgfpathlineto{\pgfqpoint{4.734684in}{0.601314in}}%
\pgfpathlineto{\pgfqpoint{4.735217in}{0.601621in}}%
\pgfpathlineto{\pgfqpoint{4.737352in}{0.604657in}}%
\pgfpathlineto{\pgfqpoint{4.737886in}{0.602259in}}%
\pgfpathlineto{\pgfqpoint{4.738420in}{0.604817in}}%
\pgfpathlineto{\pgfqpoint{4.738953in}{0.604045in}}%
\pgfpathlineto{\pgfqpoint{4.739487in}{0.605588in}}%
\pgfpathlineto{\pgfqpoint{4.740554in}{0.601871in}}%
\pgfpathlineto{\pgfqpoint{4.741088in}{0.602197in}}%
\pgfpathlineto{\pgfqpoint{4.741622in}{0.603096in}}%
\pgfpathlineto{\pgfqpoint{4.742689in}{0.601662in}}%
\pgfpathlineto{\pgfqpoint{4.744290in}{0.606816in}}%
\pgfpathlineto{\pgfqpoint{4.745891in}{0.600122in}}%
\pgfpathlineto{\pgfqpoint{4.747492in}{0.607029in}}%
\pgfpathlineto{\pgfqpoint{4.748560in}{0.600395in}}%
\pgfpathlineto{\pgfqpoint{4.749627in}{0.605289in}}%
\pgfpathlineto{\pgfqpoint{4.750161in}{0.600933in}}%
\pgfpathlineto{\pgfqpoint{4.750695in}{0.604181in}}%
\pgfpathlineto{\pgfqpoint{4.751228in}{0.606216in}}%
\pgfpathlineto{\pgfqpoint{4.752296in}{0.600936in}}%
\pgfpathlineto{\pgfqpoint{4.753897in}{0.607443in}}%
\pgfpathlineto{\pgfqpoint{4.754964in}{0.602475in}}%
\pgfpathlineto{\pgfqpoint{4.755498in}{0.606415in}}%
\pgfpathlineto{\pgfqpoint{4.756032in}{0.605500in}}%
\pgfpathlineto{\pgfqpoint{4.756565in}{0.600246in}}%
\pgfpathlineto{\pgfqpoint{4.757099in}{0.605519in}}%
\pgfpathlineto{\pgfqpoint{4.758166in}{0.600653in}}%
\pgfpathlineto{\pgfqpoint{4.759768in}{0.603859in}}%
\pgfpathlineto{\pgfqpoint{4.760301in}{0.602544in}}%
\pgfpathlineto{\pgfqpoint{4.761902in}{0.605735in}}%
\pgfpathlineto{\pgfqpoint{4.763503in}{0.602510in}}%
\pgfpathlineto{\pgfqpoint{4.765105in}{0.607392in}}%
\pgfpathlineto{\pgfqpoint{4.765638in}{0.601352in}}%
\pgfpathlineto{\pgfqpoint{4.766172in}{0.603607in}}%
\pgfpathlineto{\pgfqpoint{4.766706in}{0.602249in}}%
\pgfpathlineto{\pgfqpoint{4.768307in}{0.604929in}}%
\pgfpathlineto{\pgfqpoint{4.769374in}{0.605900in}}%
\pgfpathlineto{\pgfqpoint{4.769908in}{0.604174in}}%
\pgfpathlineto{\pgfqpoint{4.770442in}{0.607600in}}%
\pgfpathlineto{\pgfqpoint{4.771509in}{0.601652in}}%
\pgfpathlineto{\pgfqpoint{4.772043in}{0.602257in}}%
\pgfpathlineto{\pgfqpoint{4.772576in}{0.605314in}}%
\pgfpathlineto{\pgfqpoint{4.773110in}{0.601900in}}%
\pgfpathlineto{\pgfqpoint{4.774178in}{0.602220in}}%
\pgfpathlineto{\pgfqpoint{4.774711in}{0.605702in}}%
\pgfpathlineto{\pgfqpoint{4.775245in}{0.601272in}}%
\pgfpathlineto{\pgfqpoint{4.775779in}{0.601958in}}%
\pgfpathlineto{\pgfqpoint{4.776846in}{0.602471in}}%
\pgfpathlineto{\pgfqpoint{4.777380in}{0.600382in}}%
\pgfpathlineto{\pgfqpoint{4.777913in}{0.604522in}}%
\pgfpathlineto{\pgfqpoint{4.778447in}{0.600966in}}%
\pgfpathlineto{\pgfqpoint{4.779515in}{0.606147in}}%
\pgfpathlineto{\pgfqpoint{4.780048in}{0.604567in}}%
\pgfpathlineto{\pgfqpoint{4.782183in}{0.601016in}}%
\pgfpathlineto{\pgfqpoint{4.783250in}{0.605743in}}%
\pgfpathlineto{\pgfqpoint{4.784852in}{0.602856in}}%
\pgfpathlineto{\pgfqpoint{4.785919in}{0.602487in}}%
\pgfpathlineto{\pgfqpoint{4.786453in}{0.603352in}}%
\pgfpathlineto{\pgfqpoint{4.786986in}{0.602975in}}%
\pgfpathlineto{\pgfqpoint{4.788054in}{0.600634in}}%
\pgfpathlineto{\pgfqpoint{4.789121in}{0.604398in}}%
\pgfpathlineto{\pgfqpoint{4.789655in}{0.601008in}}%
\pgfpathlineto{\pgfqpoint{4.790189in}{0.603232in}}%
\pgfpathlineto{\pgfqpoint{4.790722in}{0.603227in}}%
\pgfpathlineto{\pgfqpoint{4.791790in}{0.600963in}}%
\pgfpathlineto{\pgfqpoint{4.792323in}{0.610021in}}%
\pgfpathlineto{\pgfqpoint{4.792857in}{0.607657in}}%
\pgfpathlineto{\pgfqpoint{4.793924in}{0.602924in}}%
\pgfpathlineto{\pgfqpoint{4.794458in}{0.605450in}}%
\pgfpathlineto{\pgfqpoint{4.794992in}{0.606487in}}%
\pgfpathlineto{\pgfqpoint{4.796593in}{0.603705in}}%
\pgfpathlineto{\pgfqpoint{4.797127in}{0.606793in}}%
\pgfpathlineto{\pgfqpoint{4.797660in}{0.600304in}}%
\pgfpathlineto{\pgfqpoint{4.798194in}{0.602612in}}%
\pgfpathlineto{\pgfqpoint{4.799795in}{0.608667in}}%
\pgfpathlineto{\pgfqpoint{4.800863in}{0.602579in}}%
\pgfpathlineto{\pgfqpoint{4.801396in}{0.604288in}}%
\pgfpathlineto{\pgfqpoint{4.801930in}{0.605584in}}%
\pgfpathlineto{\pgfqpoint{4.803531in}{0.601065in}}%
\pgfpathlineto{\pgfqpoint{4.804065in}{0.602665in}}%
\pgfpathlineto{\pgfqpoint{4.804598in}{0.610907in}}%
\pgfpathlineto{\pgfqpoint{4.805132in}{0.603569in}}%
\pgfpathlineto{\pgfqpoint{4.805666in}{0.608162in}}%
\pgfpathlineto{\pgfqpoint{4.806200in}{0.607195in}}%
\pgfpathlineto{\pgfqpoint{4.806733in}{0.602597in}}%
\pgfpathlineto{\pgfqpoint{4.807267in}{0.604277in}}%
\pgfpathlineto{\pgfqpoint{4.808334in}{0.602105in}}%
\pgfpathlineto{\pgfqpoint{4.808868in}{0.603231in}}%
\pgfpathlineto{\pgfqpoint{4.810469in}{0.607981in}}%
\pgfpathlineto{\pgfqpoint{4.811537in}{0.602168in}}%
\pgfpathlineto{\pgfqpoint{4.813138in}{0.607478in}}%
\pgfpathlineto{\pgfqpoint{4.813671in}{0.604357in}}%
\pgfpathlineto{\pgfqpoint{4.814205in}{0.605859in}}%
\pgfpathlineto{\pgfqpoint{4.814739in}{0.606556in}}%
\pgfpathlineto{\pgfqpoint{4.815806in}{0.603014in}}%
\pgfpathlineto{\pgfqpoint{4.816340in}{0.603520in}}%
\pgfpathlineto{\pgfqpoint{4.816874in}{0.602001in}}%
\pgfpathlineto{\pgfqpoint{4.817941in}{0.611117in}}%
\pgfpathlineto{\pgfqpoint{4.818475in}{0.605734in}}%
\pgfpathlineto{\pgfqpoint{4.820076in}{0.612186in}}%
\pgfpathlineto{\pgfqpoint{4.820609in}{0.601845in}}%
\pgfpathlineto{\pgfqpoint{4.821143in}{0.604173in}}%
\pgfpathlineto{\pgfqpoint{4.821677in}{0.602856in}}%
\pgfpathlineto{\pgfqpoint{4.822744in}{0.612221in}}%
\pgfpathlineto{\pgfqpoint{4.823278in}{0.603454in}}%
\pgfpathlineto{\pgfqpoint{4.823812in}{0.607345in}}%
\pgfpathlineto{\pgfqpoint{4.824879in}{0.605751in}}%
\pgfpathlineto{\pgfqpoint{4.825413in}{0.610846in}}%
\pgfpathlineto{\pgfqpoint{4.825946in}{0.605620in}}%
\pgfpathlineto{\pgfqpoint{4.827548in}{0.601829in}}%
\pgfpathlineto{\pgfqpoint{4.828615in}{0.603984in}}%
\pgfpathlineto{\pgfqpoint{4.829149in}{0.601467in}}%
\pgfpathlineto{\pgfqpoint{4.829682in}{0.604436in}}%
\pgfpathlineto{\pgfqpoint{4.830216in}{0.602792in}}%
\pgfpathlineto{\pgfqpoint{4.831817in}{0.605168in}}%
\pgfpathlineto{\pgfqpoint{4.832351in}{0.603073in}}%
\pgfpathlineto{\pgfqpoint{4.832885in}{0.603745in}}%
\pgfpathlineto{\pgfqpoint{4.833418in}{0.604003in}}%
\pgfpathlineto{\pgfqpoint{4.833952in}{0.601864in}}%
\pgfpathlineto{\pgfqpoint{4.834486in}{0.606653in}}%
\pgfpathlineto{\pgfqpoint{4.835019in}{0.605898in}}%
\pgfpathlineto{\pgfqpoint{4.835553in}{0.600949in}}%
\pgfpathlineto{\pgfqpoint{4.836087in}{0.601727in}}%
\pgfpathlineto{\pgfqpoint{4.836621in}{0.606061in}}%
\pgfpathlineto{\pgfqpoint{4.837154in}{0.602839in}}%
\pgfpathlineto{\pgfqpoint{4.837688in}{0.602203in}}%
\pgfpathlineto{\pgfqpoint{4.838755in}{0.608200in}}%
\pgfpathlineto{\pgfqpoint{4.839289in}{0.600011in}}%
\pgfpathlineto{\pgfqpoint{4.839823in}{0.602526in}}%
\pgfpathlineto{\pgfqpoint{4.840356in}{0.600910in}}%
\pgfpathlineto{\pgfqpoint{4.842491in}{0.607110in}}%
\pgfpathlineto{\pgfqpoint{4.844092in}{0.603272in}}%
\pgfpathlineto{\pgfqpoint{4.844626in}{0.604982in}}%
\pgfpathlineto{\pgfqpoint{4.845693in}{0.600852in}}%
\pgfpathlineto{\pgfqpoint{4.846227in}{0.601862in}}%
\pgfpathlineto{\pgfqpoint{4.846761in}{0.601165in}}%
\pgfpathlineto{\pgfqpoint{4.847828in}{0.608480in}}%
\pgfpathlineto{\pgfqpoint{4.848362in}{0.601520in}}%
\pgfpathlineto{\pgfqpoint{4.848896in}{0.602774in}}%
\pgfpathlineto{\pgfqpoint{4.849429in}{0.605814in}}%
\pgfpathlineto{\pgfqpoint{4.849963in}{0.615814in}}%
\pgfpathlineto{\pgfqpoint{4.850497in}{0.602581in}}%
\pgfpathlineto{\pgfqpoint{4.851564in}{0.602779in}}%
\pgfpathlineto{\pgfqpoint{4.852632in}{0.607550in}}%
\pgfpathlineto{\pgfqpoint{4.853699in}{0.601424in}}%
\pgfpathlineto{\pgfqpoint{4.854233in}{0.602214in}}%
\pgfpathlineto{\pgfqpoint{4.854766in}{0.612195in}}%
\pgfpathlineto{\pgfqpoint{4.855300in}{0.601748in}}%
\pgfpathlineto{\pgfqpoint{4.855834in}{0.601471in}}%
\pgfpathlineto{\pgfqpoint{4.857435in}{0.607849in}}%
\pgfpathlineto{\pgfqpoint{4.858502in}{0.601804in}}%
\pgfpathlineto{\pgfqpoint{4.859036in}{0.605088in}}%
\pgfpathlineto{\pgfqpoint{4.859570in}{0.608416in}}%
\pgfpathlineto{\pgfqpoint{4.860103in}{0.606187in}}%
\pgfpathlineto{\pgfqpoint{4.860637in}{0.607045in}}%
\pgfpathlineto{\pgfqpoint{4.861171in}{0.610442in}}%
\pgfpathlineto{\pgfqpoint{4.861704in}{0.600787in}}%
\pgfpathlineto{\pgfqpoint{4.862238in}{0.606794in}}%
\pgfpathlineto{\pgfqpoint{4.862772in}{0.608368in}}%
\pgfpathlineto{\pgfqpoint{4.863306in}{0.608155in}}%
\pgfpathlineto{\pgfqpoint{4.864373in}{0.603107in}}%
\pgfpathlineto{\pgfqpoint{4.864907in}{0.603293in}}%
\pgfpathlineto{\pgfqpoint{4.865974in}{0.608632in}}%
\pgfpathlineto{\pgfqpoint{4.867041in}{0.600539in}}%
\pgfpathlineto{\pgfqpoint{4.867575in}{0.605787in}}%
\pgfpathlineto{\pgfqpoint{4.868109in}{0.602510in}}%
\pgfpathlineto{\pgfqpoint{4.868643in}{0.601086in}}%
\pgfpathlineto{\pgfqpoint{4.869176in}{0.601910in}}%
\pgfpathlineto{\pgfqpoint{4.870244in}{0.601677in}}%
\pgfpathlineto{\pgfqpoint{4.870777in}{0.602557in}}%
\pgfpathlineto{\pgfqpoint{4.871311in}{0.601721in}}%
\pgfpathlineto{\pgfqpoint{4.872378in}{0.601039in}}%
\pgfpathlineto{\pgfqpoint{4.872912in}{0.601943in}}%
\pgfpathlineto{\pgfqpoint{4.873446in}{0.601507in}}%
\pgfpathlineto{\pgfqpoint{4.877182in}{0.601349in}}%
\pgfpathlineto{\pgfqpoint{4.879850in}{0.603907in}}%
\pgfpathlineto{\pgfqpoint{4.880384in}{0.603763in}}%
\pgfpathlineto{\pgfqpoint{4.880918in}{0.606881in}}%
\pgfpathlineto{\pgfqpoint{4.881451in}{0.604093in}}%
\pgfpathlineto{\pgfqpoint{4.881985in}{0.602644in}}%
\pgfpathlineto{\pgfqpoint{4.883052in}{0.608174in}}%
\pgfpathlineto{\pgfqpoint{4.883586in}{0.601722in}}%
\pgfpathlineto{\pgfqpoint{4.884120in}{0.601997in}}%
\pgfpathlineto{\pgfqpoint{4.885187in}{0.608638in}}%
\pgfpathlineto{\pgfqpoint{4.885721in}{0.607020in}}%
\pgfpathlineto{\pgfqpoint{4.886255in}{0.603648in}}%
\pgfpathlineto{\pgfqpoint{4.886788in}{0.607780in}}%
\pgfpathlineto{\pgfqpoint{4.887322in}{0.602449in}}%
\pgfpathlineto{\pgfqpoint{4.887856in}{0.606317in}}%
\pgfpathlineto{\pgfqpoint{4.888923in}{0.602684in}}%
\pgfpathlineto{\pgfqpoint{4.890524in}{0.606086in}}%
\pgfpathlineto{\pgfqpoint{4.891058in}{0.601889in}}%
\pgfpathlineto{\pgfqpoint{4.891592in}{0.606693in}}%
\pgfpathlineto{\pgfqpoint{4.892125in}{0.606358in}}%
\pgfpathlineto{\pgfqpoint{4.892659in}{0.603054in}}%
\pgfpathlineto{\pgfqpoint{4.893193in}{0.605188in}}%
\pgfpathlineto{\pgfqpoint{4.893726in}{0.605999in}}%
\pgfpathlineto{\pgfqpoint{4.894260in}{0.603458in}}%
\pgfpathlineto{\pgfqpoint{4.894794in}{0.604212in}}%
\pgfpathlineto{\pgfqpoint{4.895328in}{0.603918in}}%
\pgfpathlineto{\pgfqpoint{4.895861in}{0.602032in}}%
\pgfpathlineto{\pgfqpoint{4.896395in}{0.603732in}}%
\pgfpathlineto{\pgfqpoint{4.896929in}{0.606951in}}%
\pgfpathlineto{\pgfqpoint{4.897996in}{0.601614in}}%
\pgfpathlineto{\pgfqpoint{4.899597in}{0.605753in}}%
\pgfpathlineto{\pgfqpoint{4.901198in}{0.601262in}}%
\pgfpathlineto{\pgfqpoint{4.902266in}{0.604310in}}%
\pgfpathlineto{\pgfqpoint{4.902799in}{0.610960in}}%
\pgfpathlineto{\pgfqpoint{4.903333in}{0.607239in}}%
\pgfpathlineto{\pgfqpoint{4.903867in}{0.604582in}}%
\pgfpathlineto{\pgfqpoint{4.905468in}{0.613481in}}%
\pgfpathlineto{\pgfqpoint{4.906002in}{0.603112in}}%
\pgfpathlineto{\pgfqpoint{4.906535in}{0.607509in}}%
\pgfpathlineto{\pgfqpoint{4.908136in}{0.609185in}}%
\pgfpathlineto{\pgfqpoint{4.909204in}{0.605161in}}%
\pgfpathlineto{\pgfqpoint{4.909738in}{0.614150in}}%
\pgfpathlineto{\pgfqpoint{4.910271in}{0.609442in}}%
\pgfpathlineto{\pgfqpoint{4.911872in}{0.606935in}}%
\pgfpathlineto{\pgfqpoint{4.912406in}{0.615144in}}%
\pgfpathlineto{\pgfqpoint{4.914007in}{0.600613in}}%
\pgfpathlineto{\pgfqpoint{4.914541in}{0.616259in}}%
\pgfpathlineto{\pgfqpoint{4.915075in}{0.609105in}}%
\pgfpathlineto{\pgfqpoint{4.915608in}{0.608121in}}%
\pgfpathlineto{\pgfqpoint{4.916142in}{0.603129in}}%
\pgfpathlineto{\pgfqpoint{4.916676in}{0.606252in}}%
\pgfpathlineto{\pgfqpoint{4.918277in}{0.603349in}}%
\pgfpathlineto{\pgfqpoint{4.918810in}{0.603240in}}%
\pgfpathlineto{\pgfqpoint{4.919344in}{0.611338in}}%
\pgfpathlineto{\pgfqpoint{4.919878in}{0.609666in}}%
\pgfpathlineto{\pgfqpoint{4.921479in}{0.602629in}}%
\pgfpathlineto{\pgfqpoint{4.922013in}{0.613755in}}%
\pgfpathlineto{\pgfqpoint{4.922546in}{0.608848in}}%
\pgfpathlineto{\pgfqpoint{4.924147in}{0.600676in}}%
\pgfpathlineto{\pgfqpoint{4.924681in}{0.609327in}}%
\pgfpathlineto{\pgfqpoint{4.925215in}{0.604064in}}%
\pgfpathlineto{\pgfqpoint{4.926282in}{0.602329in}}%
\pgfpathlineto{\pgfqpoint{4.927350in}{0.610664in}}%
\pgfpathlineto{\pgfqpoint{4.927883in}{0.606406in}}%
\pgfpathlineto{\pgfqpoint{4.928951in}{0.616591in}}%
\pgfpathlineto{\pgfqpoint{4.929484in}{0.605699in}}%
\pgfpathlineto{\pgfqpoint{4.930018in}{0.606698in}}%
\pgfpathlineto{\pgfqpoint{4.930552in}{0.613781in}}%
\pgfpathlineto{\pgfqpoint{4.931086in}{0.603491in}}%
\pgfpathlineto{\pgfqpoint{4.931619in}{0.611104in}}%
\pgfpathlineto{\pgfqpoint{4.933220in}{0.612010in}}%
\pgfpathlineto{\pgfqpoint{4.933754in}{0.604717in}}%
\pgfpathlineto{\pgfqpoint{4.934288in}{0.605294in}}%
\pgfpathlineto{\pgfqpoint{4.934821in}{0.613499in}}%
\pgfpathlineto{\pgfqpoint{4.935355in}{0.603471in}}%
\pgfpathlineto{\pgfqpoint{4.935889in}{0.613774in}}%
\pgfpathlineto{\pgfqpoint{4.936956in}{0.602145in}}%
\pgfpathlineto{\pgfqpoint{4.937490in}{0.612937in}}%
\pgfpathlineto{\pgfqpoint{4.938024in}{0.610712in}}%
\pgfpathlineto{\pgfqpoint{4.939625in}{0.601314in}}%
\pgfpathlineto{\pgfqpoint{4.940158in}{0.608065in}}%
\pgfpathlineto{\pgfqpoint{4.940692in}{0.605521in}}%
\pgfpathlineto{\pgfqpoint{4.941226in}{0.606479in}}%
\pgfpathlineto{\pgfqpoint{4.941760in}{0.610931in}}%
\pgfpathlineto{\pgfqpoint{4.942827in}{0.601614in}}%
\pgfpathlineto{\pgfqpoint{4.943361in}{0.603214in}}%
\pgfpathlineto{\pgfqpoint{4.945495in}{0.609226in}}%
\pgfpathlineto{\pgfqpoint{4.946029in}{0.603642in}}%
\pgfpathlineto{\pgfqpoint{4.946563in}{0.606367in}}%
\pgfpathlineto{\pgfqpoint{4.947097in}{0.607145in}}%
\pgfpathlineto{\pgfqpoint{4.947630in}{0.601692in}}%
\pgfpathlineto{\pgfqpoint{4.948164in}{0.605869in}}%
\pgfpathlineto{\pgfqpoint{4.948698in}{0.611383in}}%
\pgfpathlineto{\pgfqpoint{4.949231in}{0.606773in}}%
\pgfpathlineto{\pgfqpoint{4.950832in}{0.603072in}}%
\pgfpathlineto{\pgfqpoint{4.951900in}{0.606386in}}%
\pgfpathlineto{\pgfqpoint{4.952434in}{0.601022in}}%
\pgfpathlineto{\pgfqpoint{4.952967in}{0.602964in}}%
\pgfpathlineto{\pgfqpoint{4.953501in}{0.606784in}}%
\pgfpathlineto{\pgfqpoint{4.954035in}{0.605405in}}%
\pgfpathlineto{\pgfqpoint{4.954568in}{0.602471in}}%
\pgfpathlineto{\pgfqpoint{4.956169in}{0.606850in}}%
\pgfpathlineto{\pgfqpoint{4.957237in}{0.601248in}}%
\pgfpathlineto{\pgfqpoint{4.958304in}{0.613956in}}%
\pgfpathlineto{\pgfqpoint{4.959372in}{0.602758in}}%
\pgfpathlineto{\pgfqpoint{4.959905in}{0.604117in}}%
\pgfpathlineto{\pgfqpoint{4.960439in}{0.616217in}}%
\pgfpathlineto{\pgfqpoint{4.960973in}{0.602543in}}%
\pgfpathlineto{\pgfqpoint{4.961506in}{0.612123in}}%
\pgfpathlineto{\pgfqpoint{4.962040in}{0.616228in}}%
\pgfpathlineto{\pgfqpoint{4.962574in}{0.603908in}}%
\pgfpathlineto{\pgfqpoint{4.963108in}{0.610156in}}%
\pgfpathlineto{\pgfqpoint{4.963641in}{0.611321in}}%
\pgfpathlineto{\pgfqpoint{4.964175in}{0.604191in}}%
\pgfpathlineto{\pgfqpoint{4.964709in}{0.620641in}}%
\pgfpathlineto{\pgfqpoint{4.965242in}{0.605908in}}%
\pgfpathlineto{\pgfqpoint{4.965776in}{0.616724in}}%
\pgfpathlineto{\pgfqpoint{4.966310in}{0.601177in}}%
\pgfpathlineto{\pgfqpoint{4.966843in}{0.604732in}}%
\pgfpathlineto{\pgfqpoint{4.967377in}{0.605299in}}%
\pgfpathlineto{\pgfqpoint{4.967911in}{0.611039in}}%
\pgfpathlineto{\pgfqpoint{4.968445in}{0.601158in}}%
\pgfpathlineto{\pgfqpoint{4.970046in}{0.616241in}}%
\pgfpathlineto{\pgfqpoint{4.971647in}{0.607355in}}%
\pgfpathlineto{\pgfqpoint{4.972181in}{0.606371in}}%
\pgfpathlineto{\pgfqpoint{4.972714in}{0.612974in}}%
\pgfpathlineto{\pgfqpoint{4.973248in}{0.607252in}}%
\pgfpathlineto{\pgfqpoint{4.974315in}{0.605381in}}%
\pgfpathlineto{\pgfqpoint{4.974849in}{0.614503in}}%
\pgfpathlineto{\pgfqpoint{4.975383in}{0.602378in}}%
\pgfpathlineto{\pgfqpoint{4.975916in}{0.609906in}}%
\pgfpathlineto{\pgfqpoint{4.977518in}{0.603850in}}%
\pgfpathlineto{\pgfqpoint{4.978051in}{0.605681in}}%
\pgfpathlineto{\pgfqpoint{4.978585in}{0.614548in}}%
\pgfpathlineto{\pgfqpoint{4.979119in}{0.610564in}}%
\pgfpathlineto{\pgfqpoint{4.980720in}{0.602545in}}%
\pgfpathlineto{\pgfqpoint{4.981253in}{0.608748in}}%
\pgfpathlineto{\pgfqpoint{4.981787in}{0.602749in}}%
\pgfpathlineto{\pgfqpoint{4.984456in}{0.614800in}}%
\pgfpathlineto{\pgfqpoint{4.984989in}{0.607351in}}%
\pgfpathlineto{\pgfqpoint{4.985523in}{0.621488in}}%
\pgfpathlineto{\pgfqpoint{4.986057in}{0.614531in}}%
\pgfpathlineto{\pgfqpoint{4.987124in}{0.602266in}}%
\pgfpathlineto{\pgfqpoint{4.987658in}{0.615991in}}%
\pgfpathlineto{\pgfqpoint{4.988192in}{0.615937in}}%
\pgfpathlineto{\pgfqpoint{4.988725in}{0.606925in}}%
\pgfpathlineto{\pgfqpoint{4.989259in}{0.608314in}}%
\pgfpathlineto{\pgfqpoint{4.989793in}{0.609162in}}%
\pgfpathlineto{\pgfqpoint{4.990326in}{0.618092in}}%
\pgfpathlineto{\pgfqpoint{4.990860in}{0.613534in}}%
\pgfpathlineto{\pgfqpoint{4.991394in}{0.614464in}}%
\pgfpathlineto{\pgfqpoint{4.992461in}{0.607022in}}%
\pgfpathlineto{\pgfqpoint{4.992995in}{0.616055in}}%
\pgfpathlineto{\pgfqpoint{4.993529in}{0.614727in}}%
\pgfpathlineto{\pgfqpoint{4.995130in}{0.604383in}}%
\pgfpathlineto{\pgfqpoint{4.995663in}{0.602475in}}%
\pgfpathlineto{\pgfqpoint{4.996197in}{0.611825in}}%
\pgfpathlineto{\pgfqpoint{4.997264in}{0.611242in}}%
\pgfpathlineto{\pgfqpoint{4.998332in}{0.602822in}}%
\pgfpathlineto{\pgfqpoint{5.000467in}{0.609361in}}%
\pgfpathlineto{\pgfqpoint{5.001000in}{0.607659in}}%
\pgfpathlineto{\pgfqpoint{5.001534in}{0.603892in}}%
\pgfpathlineto{\pgfqpoint{5.002068in}{0.613850in}}%
\pgfpathlineto{\pgfqpoint{5.002601in}{0.607585in}}%
\pgfpathlineto{\pgfqpoint{5.003135in}{0.611579in}}%
\pgfpathlineto{\pgfqpoint{5.003669in}{0.607008in}}%
\pgfpathlineto{\pgfqpoint{5.004203in}{0.612844in}}%
\pgfpathlineto{\pgfqpoint{5.004736in}{0.607721in}}%
\pgfpathlineto{\pgfqpoint{5.006337in}{0.602179in}}%
\pgfpathlineto{\pgfqpoint{5.007405in}{0.608963in}}%
\pgfpathlineto{\pgfqpoint{5.007938in}{0.608238in}}%
\pgfpathlineto{\pgfqpoint{5.008472in}{0.607412in}}%
\pgfpathlineto{\pgfqpoint{5.009006in}{0.611312in}}%
\pgfpathlineto{\pgfqpoint{5.009540in}{0.601856in}}%
\pgfpathlineto{\pgfqpoint{5.010073in}{0.609692in}}%
\pgfpathlineto{\pgfqpoint{5.011674in}{0.602934in}}%
\pgfpathlineto{\pgfqpoint{5.012208in}{0.603727in}}%
\pgfpathlineto{\pgfqpoint{5.013275in}{0.611585in}}%
\pgfpathlineto{\pgfqpoint{5.013809in}{0.610922in}}%
\pgfpathlineto{\pgfqpoint{5.014343in}{0.604002in}}%
\pgfpathlineto{\pgfqpoint{5.014877in}{0.608534in}}%
\pgfpathlineto{\pgfqpoint{5.015410in}{0.610954in}}%
\pgfpathlineto{\pgfqpoint{5.015944in}{0.605883in}}%
\pgfpathlineto{\pgfqpoint{5.016478in}{0.606441in}}%
\pgfpathlineto{\pgfqpoint{5.017011in}{0.617947in}}%
\pgfpathlineto{\pgfqpoint{5.017545in}{0.612870in}}%
\pgfpathlineto{\pgfqpoint{5.018079in}{0.613374in}}%
\pgfpathlineto{\pgfqpoint{5.018612in}{0.609076in}}%
\pgfpathlineto{\pgfqpoint{5.020214in}{0.621356in}}%
\pgfpathlineto{\pgfqpoint{5.021815in}{0.603196in}}%
\pgfpathlineto{\pgfqpoint{5.023416in}{0.618301in}}%
\pgfpathlineto{\pgfqpoint{5.023949in}{0.615900in}}%
\pgfpathlineto{\pgfqpoint{5.024483in}{0.609036in}}%
\pgfpathlineto{\pgfqpoint{5.025017in}{0.611579in}}%
\pgfpathlineto{\pgfqpoint{5.025551in}{0.620839in}}%
\pgfpathlineto{\pgfqpoint{5.026084in}{0.605315in}}%
\pgfpathlineto{\pgfqpoint{5.026618in}{0.612797in}}%
\pgfpathlineto{\pgfqpoint{5.028219in}{0.605892in}}%
\pgfpathlineto{\pgfqpoint{5.030888in}{0.616411in}}%
\pgfpathlineto{\pgfqpoint{5.031421in}{0.603544in}}%
\pgfpathlineto{\pgfqpoint{5.031955in}{0.616474in}}%
\pgfpathlineto{\pgfqpoint{5.032489in}{0.607687in}}%
\pgfpathlineto{\pgfqpoint{5.033556in}{0.607775in}}%
\pgfpathlineto{\pgfqpoint{5.034623in}{0.603042in}}%
\pgfpathlineto{\pgfqpoint{5.035157in}{0.604880in}}%
\pgfpathlineto{\pgfqpoint{5.035691in}{0.619576in}}%
\pgfpathlineto{\pgfqpoint{5.036225in}{0.614310in}}%
\pgfpathlineto{\pgfqpoint{5.037292in}{0.606223in}}%
\pgfpathlineto{\pgfqpoint{5.037826in}{0.608200in}}%
\pgfpathlineto{\pgfqpoint{5.038359in}{0.611917in}}%
\pgfpathlineto{\pgfqpoint{5.038893in}{0.606602in}}%
\pgfpathlineto{\pgfqpoint{5.039427in}{0.616470in}}%
\pgfpathlineto{\pgfqpoint{5.039961in}{0.609046in}}%
\pgfpathlineto{\pgfqpoint{5.041562in}{0.611690in}}%
\pgfpathlineto{\pgfqpoint{5.042095in}{0.619681in}}%
\pgfpathlineto{\pgfqpoint{5.042629in}{0.613965in}}%
\pgfpathlineto{\pgfqpoint{5.043163in}{0.614750in}}%
\pgfpathlineto{\pgfqpoint{5.043696in}{0.605576in}}%
\pgfpathlineto{\pgfqpoint{5.044230in}{0.614424in}}%
\pgfpathlineto{\pgfqpoint{5.044764in}{0.615334in}}%
\pgfpathlineto{\pgfqpoint{5.045298in}{0.613095in}}%
\pgfpathlineto{\pgfqpoint{5.045831in}{0.604525in}}%
\pgfpathlineto{\pgfqpoint{5.046365in}{0.618315in}}%
\pgfpathlineto{\pgfqpoint{5.046899in}{0.615241in}}%
\pgfpathlineto{\pgfqpoint{5.047432in}{0.611563in}}%
\pgfpathlineto{\pgfqpoint{5.047966in}{0.612209in}}%
\pgfpathlineto{\pgfqpoint{5.048500in}{0.615822in}}%
\pgfpathlineto{\pgfqpoint{5.049033in}{0.606717in}}%
\pgfpathlineto{\pgfqpoint{5.049567in}{0.609253in}}%
\pgfpathlineto{\pgfqpoint{5.050101in}{0.610619in}}%
\pgfpathlineto{\pgfqpoint{5.051702in}{0.606139in}}%
\pgfpathlineto{\pgfqpoint{5.052769in}{0.606033in}}%
\pgfpathlineto{\pgfqpoint{5.053303in}{0.603620in}}%
\pgfpathlineto{\pgfqpoint{5.053837in}{0.604790in}}%
\pgfpathlineto{\pgfqpoint{5.054370in}{0.608256in}}%
\pgfpathlineto{\pgfqpoint{5.054904in}{0.603247in}}%
\pgfpathlineto{\pgfqpoint{5.055438in}{0.608819in}}%
\pgfpathlineto{\pgfqpoint{5.055972in}{0.603677in}}%
\pgfpathlineto{\pgfqpoint{5.057039in}{0.611677in}}%
\pgfpathlineto{\pgfqpoint{5.057573in}{0.607937in}}%
\pgfpathlineto{\pgfqpoint{5.058640in}{0.602776in}}%
\pgfpathlineto{\pgfqpoint{5.059174in}{0.613055in}}%
\pgfpathlineto{\pgfqpoint{5.059707in}{0.610943in}}%
\pgfpathlineto{\pgfqpoint{5.060241in}{0.610673in}}%
\pgfpathlineto{\pgfqpoint{5.061309in}{0.604033in}}%
\pgfpathlineto{\pgfqpoint{5.061842in}{0.606245in}}%
\pgfpathlineto{\pgfqpoint{5.063443in}{0.600102in}}%
\pgfpathlineto{\pgfqpoint{5.065044in}{0.610917in}}%
\pgfpathlineto{\pgfqpoint{5.065578in}{0.604717in}}%
\pgfpathlineto{\pgfqpoint{5.066112in}{0.611740in}}%
\pgfpathlineto{\pgfqpoint{5.067179in}{0.603532in}}%
\pgfpathlineto{\pgfqpoint{5.067713in}{0.604657in}}%
\pgfpathlineto{\pgfqpoint{5.068247in}{0.612616in}}%
\pgfpathlineto{\pgfqpoint{5.068780in}{0.606750in}}%
\pgfpathlineto{\pgfqpoint{5.069314in}{0.606520in}}%
\pgfpathlineto{\pgfqpoint{5.069848in}{0.604561in}}%
\pgfpathlineto{\pgfqpoint{5.070381in}{0.608499in}}%
\pgfpathlineto{\pgfqpoint{5.071449in}{0.601354in}}%
\pgfpathlineto{\pgfqpoint{5.072516in}{0.612731in}}%
\pgfpathlineto{\pgfqpoint{5.073050in}{0.612680in}}%
\pgfpathlineto{\pgfqpoint{5.074651in}{0.603540in}}%
\pgfpathlineto{\pgfqpoint{5.075185in}{0.612195in}}%
\pgfpathlineto{\pgfqpoint{5.075718in}{0.610675in}}%
\pgfpathlineto{\pgfqpoint{5.076786in}{0.605270in}}%
\pgfpathlineto{\pgfqpoint{5.077320in}{0.607712in}}%
\pgfpathlineto{\pgfqpoint{5.077853in}{0.601615in}}%
\pgfpathlineto{\pgfqpoint{5.078387in}{0.612949in}}%
\pgfpathlineto{\pgfqpoint{5.078921in}{0.608793in}}%
\pgfpathlineto{\pgfqpoint{5.079454in}{0.606847in}}%
\pgfpathlineto{\pgfqpoint{5.080522in}{0.610779in}}%
\pgfpathlineto{\pgfqpoint{5.081055in}{0.609174in}}%
\pgfpathlineto{\pgfqpoint{5.081589in}{0.609766in}}%
\pgfpathlineto{\pgfqpoint{5.082123in}{0.601553in}}%
\pgfpathlineto{\pgfqpoint{5.082657in}{0.612350in}}%
\pgfpathlineto{\pgfqpoint{5.083190in}{0.601348in}}%
\pgfpathlineto{\pgfqpoint{5.084791in}{0.607133in}}%
\pgfpathlineto{\pgfqpoint{5.085325in}{0.602725in}}%
\pgfpathlineto{\pgfqpoint{5.086392in}{0.602913in}}%
\pgfpathlineto{\pgfqpoint{5.087994in}{0.608630in}}%
\pgfpathlineto{\pgfqpoint{5.088527in}{0.602974in}}%
\pgfpathlineto{\pgfqpoint{5.089595in}{0.603150in}}%
\pgfpathlineto{\pgfqpoint{5.090128in}{0.605408in}}%
\pgfpathlineto{\pgfqpoint{5.090662in}{0.604038in}}%
\pgfpathlineto{\pgfqpoint{5.091196in}{0.601169in}}%
\pgfpathlineto{\pgfqpoint{5.092263in}{0.601431in}}%
\pgfpathlineto{\pgfqpoint{5.092797in}{0.607291in}}%
\pgfpathlineto{\pgfqpoint{5.093331in}{0.606181in}}%
\pgfpathlineto{\pgfqpoint{5.094398in}{0.601968in}}%
\pgfpathlineto{\pgfqpoint{5.095465in}{0.606521in}}%
\pgfpathlineto{\pgfqpoint{5.095999in}{0.606080in}}%
\pgfpathlineto{\pgfqpoint{5.097066in}{0.606523in}}%
\pgfpathlineto{\pgfqpoint{5.098668in}{0.603561in}}%
\pgfpathlineto{\pgfqpoint{5.100802in}{0.608011in}}%
\pgfpathlineto{\pgfqpoint{5.101336in}{0.608024in}}%
\pgfpathlineto{\pgfqpoint{5.102403in}{0.601890in}}%
\pgfpathlineto{\pgfqpoint{5.102937in}{0.603749in}}%
\pgfpathlineto{\pgfqpoint{5.103471in}{0.611968in}}%
\pgfpathlineto{\pgfqpoint{5.104005in}{0.608127in}}%
\pgfpathlineto{\pgfqpoint{5.104538in}{0.602695in}}%
\pgfpathlineto{\pgfqpoint{5.105606in}{0.602885in}}%
\pgfpathlineto{\pgfqpoint{5.106139in}{0.606090in}}%
\pgfpathlineto{\pgfqpoint{5.106673in}{0.601468in}}%
\pgfpathlineto{\pgfqpoint{5.107207in}{0.603189in}}%
\pgfpathlineto{\pgfqpoint{5.108274in}{0.601540in}}%
\pgfpathlineto{\pgfqpoint{5.108808in}{0.603906in}}%
\pgfpathlineto{\pgfqpoint{5.109342in}{0.600134in}}%
\pgfpathlineto{\pgfqpoint{5.109875in}{0.603959in}}%
\pgfpathlineto{\pgfqpoint{5.110943in}{0.602795in}}%
\pgfpathlineto{\pgfqpoint{5.111476in}{0.602359in}}%
\pgfpathlineto{\pgfqpoint{5.112010in}{0.603786in}}%
\pgfpathlineto{\pgfqpoint{5.113611in}{0.601148in}}%
\pgfpathlineto{\pgfqpoint{5.114145in}{0.600442in}}%
\pgfpathlineto{\pgfqpoint{5.114679in}{0.604605in}}%
\pgfpathlineto{\pgfqpoint{5.115212in}{0.601628in}}%
\pgfpathlineto{\pgfqpoint{5.115746in}{0.601865in}}%
\pgfpathlineto{\pgfqpoint{5.116280in}{0.603412in}}%
\pgfpathlineto{\pgfqpoint{5.117881in}{0.601320in}}%
\pgfpathlineto{\pgfqpoint{5.119482in}{0.602335in}}%
\pgfpathlineto{\pgfqpoint{5.120016in}{0.600845in}}%
\pgfpathlineto{\pgfqpoint{5.120549in}{0.603260in}}%
\pgfpathlineto{\pgfqpoint{5.121083in}{0.601983in}}%
\pgfpathlineto{\pgfqpoint{5.121617in}{0.602830in}}%
\pgfpathlineto{\pgfqpoint{5.122150in}{0.602172in}}%
\pgfpathlineto{\pgfqpoint{5.123218in}{0.600633in}}%
\pgfpathlineto{\pgfqpoint{5.123752in}{0.602646in}}%
\pgfpathlineto{\pgfqpoint{5.124285in}{0.602108in}}%
\pgfpathlineto{\pgfqpoint{5.125353in}{0.600541in}}%
\pgfpathlineto{\pgfqpoint{5.125886in}{0.602365in}}%
\pgfpathlineto{\pgfqpoint{5.126420in}{0.601247in}}%
\pgfpathlineto{\pgfqpoint{5.128555in}{0.602092in}}%
\pgfpathlineto{\pgfqpoint{5.129089in}{0.601113in}}%
\pgfpathlineto{\pgfqpoint{5.129622in}{0.602339in}}%
\pgfpathlineto{\pgfqpoint{5.131757in}{0.600088in}}%
\pgfpathlineto{\pgfqpoint{5.132291in}{0.600980in}}%
\pgfpathlineto{\pgfqpoint{5.133358in}{0.601834in}}%
\pgfpathlineto{\pgfqpoint{5.133892in}{0.600653in}}%
\pgfpathlineto{\pgfqpoint{5.134426in}{0.601142in}}%
\pgfpathlineto{\pgfqpoint{5.135493in}{0.601005in}}%
\pgfpathlineto{\pgfqpoint{5.137628in}{0.600692in}}%
\pgfpathlineto{\pgfqpoint{5.138161in}{0.602079in}}%
\pgfpathlineto{\pgfqpoint{5.138695in}{0.601277in}}%
\pgfpathlineto{\pgfqpoint{5.143498in}{0.601474in}}%
\pgfpathlineto{\pgfqpoint{5.145100in}{0.602080in}}%
\pgfpathlineto{\pgfqpoint{5.146167in}{0.603364in}}%
\pgfpathlineto{\pgfqpoint{5.146701in}{0.605873in}}%
\pgfpathlineto{\pgfqpoint{5.147234in}{0.601996in}}%
\pgfpathlineto{\pgfqpoint{5.147768in}{0.619059in}}%
\pgfpathlineto{\pgfqpoint{5.148302in}{0.606420in}}%
\pgfpathlineto{\pgfqpoint{5.148835in}{0.609218in}}%
\pgfpathlineto{\pgfqpoint{5.149369in}{0.619419in}}%
\pgfpathlineto{\pgfqpoint{5.149903in}{0.603037in}}%
\pgfpathlineto{\pgfqpoint{5.150970in}{0.603299in}}%
\pgfpathlineto{\pgfqpoint{5.154172in}{0.601136in}}%
\pgfpathlineto{\pgfqpoint{5.155774in}{0.601703in}}%
\pgfpathlineto{\pgfqpoint{5.156307in}{0.601043in}}%
\pgfpathlineto{\pgfqpoint{5.156841in}{0.601713in}}%
\pgfpathlineto{\pgfqpoint{5.157375in}{0.601693in}}%
\pgfpathlineto{\pgfqpoint{5.157908in}{0.600285in}}%
\pgfpathlineto{\pgfqpoint{5.158442in}{0.601600in}}%
\pgfpathlineto{\pgfqpoint{5.158976in}{0.600831in}}%
\pgfpathlineto{\pgfqpoint{5.159509in}{0.601657in}}%
\pgfpathlineto{\pgfqpoint{5.160043in}{0.601924in}}%
\pgfpathlineto{\pgfqpoint{5.160577in}{0.600034in}}%
\pgfpathlineto{\pgfqpoint{5.161111in}{0.601292in}}%
\pgfpathlineto{\pgfqpoint{5.163779in}{0.601418in}}%
\pgfpathlineto{\pgfqpoint{5.165380in}{0.600905in}}%
\pgfpathlineto{\pgfqpoint{5.166981in}{0.600595in}}%
\pgfpathlineto{\pgfqpoint{5.169650in}{0.602449in}}%
\pgfpathlineto{\pgfqpoint{5.170183in}{0.600797in}}%
\pgfpathlineto{\pgfqpoint{5.170717in}{0.602382in}}%
\pgfpathlineto{\pgfqpoint{5.171251in}{0.602438in}}%
\pgfpathlineto{\pgfqpoint{5.171785in}{0.600623in}}%
\pgfpathlineto{\pgfqpoint{5.172318in}{0.601761in}}%
\pgfpathlineto{\pgfqpoint{5.174453in}{0.602065in}}%
\pgfpathlineto{\pgfqpoint{5.174987in}{0.601198in}}%
\pgfpathlineto{\pgfqpoint{5.175521in}{0.603876in}}%
\pgfpathlineto{\pgfqpoint{5.176054in}{0.600460in}}%
\pgfpathlineto{\pgfqpoint{5.176588in}{0.602173in}}%
\pgfpathlineto{\pgfqpoint{5.177122in}{0.602819in}}%
\pgfpathlineto{\pgfqpoint{5.177655in}{0.600383in}}%
\pgfpathlineto{\pgfqpoint{5.178189in}{0.600957in}}%
\pgfpathlineto{\pgfqpoint{5.178723in}{0.603503in}}%
\pgfpathlineto{\pgfqpoint{5.179256in}{0.602405in}}%
\pgfpathlineto{\pgfqpoint{5.180324in}{0.600595in}}%
\pgfpathlineto{\pgfqpoint{5.180858in}{0.601970in}}%
\pgfpathlineto{\pgfqpoint{5.181391in}{0.601704in}}%
\pgfpathlineto{\pgfqpoint{5.181925in}{0.600892in}}%
\pgfpathlineto{\pgfqpoint{5.183526in}{0.607154in}}%
\pgfpathlineto{\pgfqpoint{5.184060in}{0.600575in}}%
\pgfpathlineto{\pgfqpoint{5.184593in}{0.604332in}}%
\pgfpathlineto{\pgfqpoint{5.185661in}{0.603072in}}%
\pgfpathlineto{\pgfqpoint{5.186195in}{0.601203in}}%
\pgfpathlineto{\pgfqpoint{5.186728in}{0.603609in}}%
\pgfpathlineto{\pgfqpoint{5.187262in}{0.602523in}}%
\pgfpathlineto{\pgfqpoint{5.187796in}{0.603263in}}%
\pgfpathlineto{\pgfqpoint{5.188329in}{0.606265in}}%
\pgfpathlineto{\pgfqpoint{5.188863in}{0.602169in}}%
\pgfpathlineto{\pgfqpoint{5.189930in}{0.602485in}}%
\pgfpathlineto{\pgfqpoint{5.190464in}{0.604115in}}%
\pgfpathlineto{\pgfqpoint{5.191532in}{0.601572in}}%
\pgfpathlineto{\pgfqpoint{5.193133in}{0.609314in}}%
\pgfpathlineto{\pgfqpoint{5.194734in}{0.602289in}}%
\pgfpathlineto{\pgfqpoint{5.195801in}{0.604674in}}%
\pgfpathlineto{\pgfqpoint{5.196869in}{0.601707in}}%
\pgfpathlineto{\pgfqpoint{5.197402in}{0.604322in}}%
\pgfpathlineto{\pgfqpoint{5.197936in}{0.602091in}}%
\pgfpathlineto{\pgfqpoint{5.200071in}{0.607789in}}%
\pgfpathlineto{\pgfqpoint{5.200604in}{0.602849in}}%
\pgfpathlineto{\pgfqpoint{5.201138in}{0.606354in}}%
\pgfpathlineto{\pgfqpoint{5.201672in}{0.606102in}}%
\pgfpathlineto{\pgfqpoint{5.202206in}{0.601668in}}%
\pgfpathlineto{\pgfqpoint{5.203273in}{0.601857in}}%
\pgfpathlineto{\pgfqpoint{5.203807in}{0.606740in}}%
\pgfpathlineto{\pgfqpoint{5.204874in}{0.601044in}}%
\pgfpathlineto{\pgfqpoint{5.205408in}{0.611010in}}%
\pgfpathlineto{\pgfqpoint{5.205941in}{0.602457in}}%
\pgfpathlineto{\pgfqpoint{5.206475in}{0.601755in}}%
\pgfpathlineto{\pgfqpoint{5.207009in}{0.602467in}}%
\pgfpathlineto{\pgfqpoint{5.207543in}{0.609818in}}%
\pgfpathlineto{\pgfqpoint{5.208076in}{0.606840in}}%
\pgfpathlineto{\pgfqpoint{5.208610in}{0.603545in}}%
\pgfpathlineto{\pgfqpoint{5.209144in}{0.604382in}}%
\pgfpathlineto{\pgfqpoint{5.210745in}{0.609322in}}%
\pgfpathlineto{\pgfqpoint{5.211278in}{0.603595in}}%
\pgfpathlineto{\pgfqpoint{5.211812in}{0.605117in}}%
\pgfpathlineto{\pgfqpoint{5.212880in}{0.609379in}}%
\pgfpathlineto{\pgfqpoint{5.213413in}{0.606708in}}%
\pgfpathlineto{\pgfqpoint{5.213947in}{0.618213in}}%
\pgfpathlineto{\pgfqpoint{5.214481in}{0.602622in}}%
\pgfpathlineto{\pgfqpoint{5.215014in}{0.605676in}}%
\pgfpathlineto{\pgfqpoint{5.215548in}{0.607177in}}%
\pgfpathlineto{\pgfqpoint{5.216082in}{0.606336in}}%
\pgfpathlineto{\pgfqpoint{5.216615in}{0.617203in}}%
\pgfpathlineto{\pgfqpoint{5.217149in}{0.606810in}}%
\pgfpathlineto{\pgfqpoint{5.218750in}{0.610175in}}%
\pgfpathlineto{\pgfqpoint{5.219284in}{0.610634in}}%
\pgfpathlineto{\pgfqpoint{5.220351in}{0.602289in}}%
\pgfpathlineto{\pgfqpoint{5.220885in}{0.608681in}}%
\pgfpathlineto{\pgfqpoint{5.221952in}{0.608227in}}%
\pgfpathlineto{\pgfqpoint{5.224087in}{0.603953in}}%
\pgfpathlineto{\pgfqpoint{5.225688in}{0.611505in}}%
\pgfpathlineto{\pgfqpoint{5.226222in}{0.602101in}}%
\pgfpathlineto{\pgfqpoint{5.226756in}{0.608215in}}%
\pgfpathlineto{\pgfqpoint{5.228357in}{0.601752in}}%
\pgfpathlineto{\pgfqpoint{5.228891in}{0.600430in}}%
\pgfpathlineto{\pgfqpoint{5.229424in}{0.610316in}}%
\pgfpathlineto{\pgfqpoint{5.229958in}{0.607182in}}%
\pgfpathlineto{\pgfqpoint{5.230492in}{0.606832in}}%
\pgfpathlineto{\pgfqpoint{5.231559in}{0.610819in}}%
\pgfpathlineto{\pgfqpoint{5.233160in}{0.606523in}}%
\pgfpathlineto{\pgfqpoint{5.233694in}{0.612610in}}%
\pgfpathlineto{\pgfqpoint{5.234228in}{0.606913in}}%
\pgfpathlineto{\pgfqpoint{5.234761in}{0.607016in}}%
\pgfpathlineto{\pgfqpoint{5.235295in}{0.602160in}}%
\pgfpathlineto{\pgfqpoint{5.235829in}{0.608722in}}%
\pgfpathlineto{\pgfqpoint{5.236362in}{0.605643in}}%
\pgfpathlineto{\pgfqpoint{5.236896in}{0.604997in}}%
\pgfpathlineto{\pgfqpoint{5.238497in}{0.613514in}}%
\pgfpathlineto{\pgfqpoint{5.239031in}{0.611202in}}%
\pgfpathlineto{\pgfqpoint{5.240098in}{0.603980in}}%
\pgfpathlineto{\pgfqpoint{5.240632in}{0.607661in}}%
\pgfpathlineto{\pgfqpoint{5.241699in}{0.612108in}}%
\pgfpathlineto{\pgfqpoint{5.242767in}{0.604432in}}%
\pgfpathlineto{\pgfqpoint{5.244368in}{0.615964in}}%
\pgfpathlineto{\pgfqpoint{5.245969in}{0.601807in}}%
\pgfpathlineto{\pgfqpoint{5.246503in}{0.608082in}}%
\pgfpathlineto{\pgfqpoint{5.247036in}{0.607054in}}%
\pgfpathlineto{\pgfqpoint{5.247570in}{0.606541in}}%
\pgfpathlineto{\pgfqpoint{5.248104in}{0.609674in}}%
\pgfpathlineto{\pgfqpoint{5.248638in}{0.603770in}}%
\pgfpathlineto{\pgfqpoint{5.249171in}{0.608760in}}%
\pgfpathlineto{\pgfqpoint{5.249705in}{0.609804in}}%
\pgfpathlineto{\pgfqpoint{5.250239in}{0.615488in}}%
\pgfpathlineto{\pgfqpoint{5.250772in}{0.603668in}}%
\pgfpathlineto{\pgfqpoint{5.251306in}{0.609176in}}%
\pgfpathlineto{\pgfqpoint{5.252373in}{0.602963in}}%
\pgfpathlineto{\pgfqpoint{5.253441in}{0.612244in}}%
\pgfpathlineto{\pgfqpoint{5.253975in}{0.608423in}}%
\pgfpathlineto{\pgfqpoint{5.255042in}{0.608888in}}%
\pgfpathlineto{\pgfqpoint{5.255576in}{0.601226in}}%
\pgfpathlineto{\pgfqpoint{5.257177in}{0.609909in}}%
\pgfpathlineto{\pgfqpoint{5.258778in}{0.603294in}}%
\pgfpathlineto{\pgfqpoint{5.259312in}{0.608042in}}%
\pgfpathlineto{\pgfqpoint{5.259845in}{0.606941in}}%
\pgfpathlineto{\pgfqpoint{5.260379in}{0.605242in}}%
\pgfpathlineto{\pgfqpoint{5.260913in}{0.612607in}}%
\pgfpathlineto{\pgfqpoint{5.261446in}{0.603609in}}%
\pgfpathlineto{\pgfqpoint{5.261980in}{0.605786in}}%
\pgfpathlineto{\pgfqpoint{5.262514in}{0.617765in}}%
\pgfpathlineto{\pgfqpoint{5.263047in}{0.606701in}}%
\pgfpathlineto{\pgfqpoint{5.264115in}{0.601304in}}%
\pgfpathlineto{\pgfqpoint{5.264649in}{0.610421in}}%
\pgfpathlineto{\pgfqpoint{5.265182in}{0.602540in}}%
\pgfpathlineto{\pgfqpoint{5.266783in}{0.609117in}}%
\pgfpathlineto{\pgfqpoint{5.267317in}{0.607016in}}%
\pgfpathlineto{\pgfqpoint{5.267851in}{0.611107in}}%
\pgfpathlineto{\pgfqpoint{5.268384in}{0.602771in}}%
\pgfpathlineto{\pgfqpoint{5.268918in}{0.616579in}}%
\pgfpathlineto{\pgfqpoint{5.269452in}{0.613683in}}%
\pgfpathlineto{\pgfqpoint{5.269986in}{0.601023in}}%
\pgfpathlineto{\pgfqpoint{5.270519in}{0.619384in}}%
\pgfpathlineto{\pgfqpoint{5.271053in}{0.607433in}}%
\pgfpathlineto{\pgfqpoint{5.271587in}{0.608390in}}%
\pgfpathlineto{\pgfqpoint{5.272120in}{0.603182in}}%
\pgfpathlineto{\pgfqpoint{5.272654in}{0.604670in}}%
\pgfpathlineto{\pgfqpoint{5.273188in}{0.610320in}}%
\pgfpathlineto{\pgfqpoint{5.273721in}{0.609693in}}%
\pgfpathlineto{\pgfqpoint{5.274255in}{0.603233in}}%
\pgfpathlineto{\pgfqpoint{5.275323in}{0.620726in}}%
\pgfpathlineto{\pgfqpoint{5.276924in}{0.603148in}}%
\pgfpathlineto{\pgfqpoint{5.277457in}{0.607726in}}%
\pgfpathlineto{\pgfqpoint{5.277991in}{0.604269in}}%
\pgfpathlineto{\pgfqpoint{5.279058in}{0.607773in}}%
\pgfpathlineto{\pgfqpoint{5.279592in}{0.606535in}}%
\pgfpathlineto{\pgfqpoint{5.280126in}{0.607837in}}%
\pgfpathlineto{\pgfqpoint{5.280660in}{0.609912in}}%
\pgfpathlineto{\pgfqpoint{5.281193in}{0.605817in}}%
\pgfpathlineto{\pgfqpoint{5.282261in}{0.606189in}}%
\pgfpathlineto{\pgfqpoint{5.282794in}{0.609139in}}%
\pgfpathlineto{\pgfqpoint{5.283328in}{0.600146in}}%
\pgfpathlineto{\pgfqpoint{5.283862in}{0.604732in}}%
\pgfpathlineto{\pgfqpoint{5.284395in}{0.605773in}}%
\pgfpathlineto{\pgfqpoint{5.285463in}{0.602001in}}%
\pgfpathlineto{\pgfqpoint{5.287064in}{0.613101in}}%
\pgfpathlineto{\pgfqpoint{5.288131in}{0.604446in}}%
\pgfpathlineto{\pgfqpoint{5.288665in}{0.612078in}}%
\pgfpathlineto{\pgfqpoint{5.289199in}{0.609490in}}%
\pgfpathlineto{\pgfqpoint{5.289732in}{0.605069in}}%
\pgfpathlineto{\pgfqpoint{5.290266in}{0.606237in}}%
\pgfpathlineto{\pgfqpoint{5.290800in}{0.605886in}}%
\pgfpathlineto{\pgfqpoint{5.291334in}{0.611336in}}%
\pgfpathlineto{\pgfqpoint{5.291867in}{0.605876in}}%
\pgfpathlineto{\pgfqpoint{5.292401in}{0.605662in}}%
\pgfpathlineto{\pgfqpoint{5.292935in}{0.602867in}}%
\pgfpathlineto{\pgfqpoint{5.293468in}{0.611883in}}%
\pgfpathlineto{\pgfqpoint{5.294002in}{0.609262in}}%
\pgfpathlineto{\pgfqpoint{5.294536in}{0.604950in}}%
\pgfpathlineto{\pgfqpoint{5.295069in}{0.608861in}}%
\pgfpathlineto{\pgfqpoint{5.296137in}{0.608880in}}%
\pgfpathlineto{\pgfqpoint{5.296671in}{0.602438in}}%
\pgfpathlineto{\pgfqpoint{5.297204in}{0.608820in}}%
\pgfpathlineto{\pgfqpoint{5.297738in}{0.608922in}}%
\pgfpathlineto{\pgfqpoint{5.298272in}{0.613148in}}%
\pgfpathlineto{\pgfqpoint{5.298805in}{0.611084in}}%
\pgfpathlineto{\pgfqpoint{5.299339in}{0.604663in}}%
\pgfpathlineto{\pgfqpoint{5.299873in}{0.612820in}}%
\pgfpathlineto{\pgfqpoint{5.300406in}{0.605096in}}%
\pgfpathlineto{\pgfqpoint{5.301474in}{0.605817in}}%
\pgfpathlineto{\pgfqpoint{5.302008in}{0.604589in}}%
\pgfpathlineto{\pgfqpoint{5.302541in}{0.607728in}}%
\pgfpathlineto{\pgfqpoint{5.303075in}{0.605266in}}%
\pgfpathlineto{\pgfqpoint{5.303609in}{0.604762in}}%
\pgfpathlineto{\pgfqpoint{5.304676in}{0.613275in}}%
\pgfpathlineto{\pgfqpoint{5.305210in}{0.609702in}}%
\pgfpathlineto{\pgfqpoint{5.306277in}{0.606355in}}%
\pgfpathlineto{\pgfqpoint{5.306811in}{0.607542in}}%
\pgfpathlineto{\pgfqpoint{5.307345in}{0.604272in}}%
\pgfpathlineto{\pgfqpoint{5.308412in}{0.604542in}}%
\pgfpathlineto{\pgfqpoint{5.308946in}{0.604106in}}%
\pgfpathlineto{\pgfqpoint{5.309479in}{0.602156in}}%
\pgfpathlineto{\pgfqpoint{5.310013in}{0.610618in}}%
\pgfpathlineto{\pgfqpoint{5.310547in}{0.604411in}}%
\pgfpathlineto{\pgfqpoint{5.311614in}{0.610546in}}%
\pgfpathlineto{\pgfqpoint{5.312682in}{0.600905in}}%
\pgfpathlineto{\pgfqpoint{5.313215in}{0.604052in}}%
\pgfpathlineto{\pgfqpoint{5.313749in}{0.606822in}}%
\pgfpathlineto{\pgfqpoint{5.314283in}{0.603568in}}%
\pgfpathlineto{\pgfqpoint{5.314816in}{0.608784in}}%
\pgfpathlineto{\pgfqpoint{5.315884in}{0.602340in}}%
\pgfpathlineto{\pgfqpoint{5.316418in}{0.607872in}}%
\pgfpathlineto{\pgfqpoint{5.316951in}{0.604205in}}%
\pgfpathlineto{\pgfqpoint{5.318019in}{0.607946in}}%
\pgfpathlineto{\pgfqpoint{5.318552in}{0.604138in}}%
\pgfpathlineto{\pgfqpoint{5.319086in}{0.604999in}}%
\pgfpathlineto{\pgfqpoint{5.319620in}{0.611489in}}%
\pgfpathlineto{\pgfqpoint{5.320153in}{0.604647in}}%
\pgfpathlineto{\pgfqpoint{5.321221in}{0.612185in}}%
\pgfpathlineto{\pgfqpoint{5.322288in}{0.605349in}}%
\pgfpathlineto{\pgfqpoint{5.322822in}{0.606317in}}%
\pgfpathlineto{\pgfqpoint{5.323356in}{0.605553in}}%
\pgfpathlineto{\pgfqpoint{5.323889in}{0.608753in}}%
\pgfpathlineto{\pgfqpoint{5.324423in}{0.608026in}}%
\pgfpathlineto{\pgfqpoint{5.324957in}{0.601522in}}%
\pgfpathlineto{\pgfqpoint{5.326024in}{0.614106in}}%
\pgfpathlineto{\pgfqpoint{5.327625in}{0.606037in}}%
\pgfpathlineto{\pgfqpoint{5.328159in}{0.605721in}}%
\pgfpathlineto{\pgfqpoint{5.328693in}{0.608928in}}%
\pgfpathlineto{\pgfqpoint{5.329760in}{0.609192in}}%
\pgfpathlineto{\pgfqpoint{5.330294in}{0.601871in}}%
\pgfpathlineto{\pgfqpoint{5.331895in}{0.611760in}}%
\pgfpathlineto{\pgfqpoint{5.332429in}{0.609128in}}%
\pgfpathlineto{\pgfqpoint{5.332962in}{0.613826in}}%
\pgfpathlineto{\pgfqpoint{5.333496in}{0.605886in}}%
\pgfpathlineto{\pgfqpoint{5.334030in}{0.606669in}}%
\pgfpathlineto{\pgfqpoint{5.334563in}{0.606830in}}%
\pgfpathlineto{\pgfqpoint{5.335097in}{0.602910in}}%
\pgfpathlineto{\pgfqpoint{5.335631in}{0.603943in}}%
\pgfpathlineto{\pgfqpoint{5.336164in}{0.610098in}}%
\pgfpathlineto{\pgfqpoint{5.336698in}{0.601753in}}%
\pgfpathlineto{\pgfqpoint{5.337232in}{0.610867in}}%
\pgfpathlineto{\pgfqpoint{5.337766in}{0.607090in}}%
\pgfpathlineto{\pgfqpoint{5.338833in}{0.601082in}}%
\pgfpathlineto{\pgfqpoint{5.339367in}{0.602957in}}%
\pgfpathlineto{\pgfqpoint{5.339900in}{0.601039in}}%
\pgfpathlineto{\pgfqpoint{5.340434in}{0.601161in}}%
\pgfpathlineto{\pgfqpoint{5.341501in}{0.611838in}}%
\pgfpathlineto{\pgfqpoint{5.342035in}{0.607652in}}%
\pgfpathlineto{\pgfqpoint{5.343103in}{0.602992in}}%
\pgfpathlineto{\pgfqpoint{5.343636in}{0.611088in}}%
\pgfpathlineto{\pgfqpoint{5.344170in}{0.608070in}}%
\pgfpathlineto{\pgfqpoint{5.344704in}{0.610235in}}%
\pgfpathlineto{\pgfqpoint{5.346838in}{0.603301in}}%
\pgfpathlineto{\pgfqpoint{5.347906in}{0.603767in}}%
\pgfpathlineto{\pgfqpoint{5.348440in}{0.605592in}}%
\pgfpathlineto{\pgfqpoint{5.348973in}{0.613234in}}%
\pgfpathlineto{\pgfqpoint{5.349507in}{0.602353in}}%
\pgfpathlineto{\pgfqpoint{5.350041in}{0.604815in}}%
\pgfpathlineto{\pgfqpoint{5.352709in}{0.613739in}}%
\pgfpathlineto{\pgfqpoint{5.353243in}{0.603641in}}%
\pgfpathlineto{\pgfqpoint{5.353777in}{0.612322in}}%
\pgfpathlineto{\pgfqpoint{5.355911in}{0.602773in}}%
\pgfpathlineto{\pgfqpoint{5.356445in}{0.603503in}}%
\pgfpathlineto{\pgfqpoint{5.356979in}{0.606844in}}%
\pgfpathlineto{\pgfqpoint{5.357512in}{0.602242in}}%
\pgfpathlineto{\pgfqpoint{5.358046in}{0.603813in}}%
\pgfpathlineto{\pgfqpoint{5.358580in}{0.605481in}}%
\pgfpathlineto{\pgfqpoint{5.359114in}{0.603019in}}%
\pgfpathlineto{\pgfqpoint{5.359647in}{0.606412in}}%
\pgfpathlineto{\pgfqpoint{5.360181in}{0.600842in}}%
\pgfpathlineto{\pgfqpoint{5.361782in}{0.612725in}}%
\pgfpathlineto{\pgfqpoint{5.362849in}{0.603530in}}%
\pgfpathlineto{\pgfqpoint{5.363383in}{0.604733in}}%
\pgfpathlineto{\pgfqpoint{5.363917in}{0.609530in}}%
\pgfpathlineto{\pgfqpoint{5.364451in}{0.601578in}}%
\pgfpathlineto{\pgfqpoint{5.364984in}{0.611822in}}%
\pgfpathlineto{\pgfqpoint{5.365518in}{0.605853in}}%
\pgfpathlineto{\pgfqpoint{5.366052in}{0.603656in}}%
\pgfpathlineto{\pgfqpoint{5.366585in}{0.605674in}}%
\pgfpathlineto{\pgfqpoint{5.367119in}{0.610058in}}%
\pgfpathlineto{\pgfqpoint{5.368186in}{0.603964in}}%
\pgfpathlineto{\pgfqpoint{5.368720in}{0.618691in}}%
\pgfpathlineto{\pgfqpoint{5.369254in}{0.607562in}}%
\pgfpathlineto{\pgfqpoint{5.370321in}{0.606045in}}%
\pgfpathlineto{\pgfqpoint{5.370855in}{0.606979in}}%
\pgfpathlineto{\pgfqpoint{5.371389in}{0.603646in}}%
\pgfpathlineto{\pgfqpoint{5.371922in}{0.606193in}}%
\pgfpathlineto{\pgfqpoint{5.372456in}{0.607553in}}%
\pgfpathlineto{\pgfqpoint{5.372990in}{0.604001in}}%
\pgfpathlineto{\pgfqpoint{5.373523in}{0.607691in}}%
\pgfpathlineto{\pgfqpoint{5.374591in}{0.604172in}}%
\pgfpathlineto{\pgfqpoint{5.375125in}{0.609445in}}%
\pgfpathlineto{\pgfqpoint{5.375658in}{0.605781in}}%
\pgfpathlineto{\pgfqpoint{5.376192in}{0.608677in}}%
\pgfpathlineto{\pgfqpoint{5.376726in}{0.606208in}}%
\pgfpathlineto{\pgfqpoint{5.377793in}{0.604000in}}%
\pgfpathlineto{\pgfqpoint{5.379394in}{0.607406in}}%
\pgfpathlineto{\pgfqpoint{5.379928in}{0.601063in}}%
\pgfpathlineto{\pgfqpoint{5.380462in}{0.607473in}}%
\pgfpathlineto{\pgfqpoint{5.380995in}{0.610828in}}%
\pgfpathlineto{\pgfqpoint{5.381529in}{0.602210in}}%
\pgfpathlineto{\pgfqpoint{5.382063in}{0.610536in}}%
\pgfpathlineto{\pgfqpoint{5.382596in}{0.606770in}}%
\pgfpathlineto{\pgfqpoint{5.383130in}{0.613654in}}%
\pgfpathlineto{\pgfqpoint{5.383664in}{0.604764in}}%
\pgfpathlineto{\pgfqpoint{5.384198in}{0.608593in}}%
\pgfpathlineto{\pgfqpoint{5.384731in}{0.608842in}}%
\pgfpathlineto{\pgfqpoint{5.386332in}{0.602617in}}%
\pgfpathlineto{\pgfqpoint{5.387933in}{0.606037in}}%
\pgfpathlineto{\pgfqpoint{5.388467in}{0.603151in}}%
\pgfpathlineto{\pgfqpoint{5.390068in}{0.610485in}}%
\pgfpathlineto{\pgfqpoint{5.391136in}{0.610339in}}%
\pgfpathlineto{\pgfqpoint{5.392203in}{0.603642in}}%
\pgfpathlineto{\pgfqpoint{5.392737in}{0.607919in}}%
\pgfpathlineto{\pgfqpoint{5.393270in}{0.611395in}}%
\pgfpathlineto{\pgfqpoint{5.393804in}{0.606165in}}%
\pgfpathlineto{\pgfqpoint{5.394338in}{0.608555in}}%
\pgfpathlineto{\pgfqpoint{5.394872in}{0.606895in}}%
\pgfpathlineto{\pgfqpoint{5.395405in}{0.601409in}}%
\pgfpathlineto{\pgfqpoint{5.395939in}{0.602462in}}%
\pgfpathlineto{\pgfqpoint{5.397006in}{0.610718in}}%
\pgfpathlineto{\pgfqpoint{5.397540in}{0.608492in}}%
\pgfpathlineto{\pgfqpoint{5.398607in}{0.602097in}}%
\pgfpathlineto{\pgfqpoint{5.399141in}{0.609998in}}%
\pgfpathlineto{\pgfqpoint{5.399675in}{0.605884in}}%
\pgfpathlineto{\pgfqpoint{5.400209in}{0.606845in}}%
\pgfpathlineto{\pgfqpoint{5.400742in}{0.601868in}}%
\pgfpathlineto{\pgfqpoint{5.401276in}{0.603597in}}%
\pgfpathlineto{\pgfqpoint{5.402343in}{0.603331in}}%
\pgfpathlineto{\pgfqpoint{5.403944in}{0.607829in}}%
\pgfpathlineto{\pgfqpoint{5.404478in}{0.607234in}}%
\pgfpathlineto{\pgfqpoint{5.405546in}{0.604721in}}%
\pgfpathlineto{\pgfqpoint{5.407680in}{0.610935in}}%
\pgfpathlineto{\pgfqpoint{5.408214in}{0.610042in}}%
\pgfpathlineto{\pgfqpoint{5.408748in}{0.610545in}}%
\pgfpathlineto{\pgfqpoint{5.409815in}{0.601862in}}%
\pgfpathlineto{\pgfqpoint{5.410349in}{0.611195in}}%
\pgfpathlineto{\pgfqpoint{5.410883in}{0.606555in}}%
\pgfpathlineto{\pgfqpoint{5.411416in}{0.607551in}}%
\pgfpathlineto{\pgfqpoint{5.412484in}{0.603253in}}%
\pgfpathlineto{\pgfqpoint{5.413017in}{0.604616in}}%
\pgfpathlineto{\pgfqpoint{5.413551in}{0.607086in}}%
\pgfpathlineto{\pgfqpoint{5.414085in}{0.604980in}}%
\pgfpathlineto{\pgfqpoint{5.414618in}{0.600854in}}%
\pgfpathlineto{\pgfqpoint{5.415152in}{0.601422in}}%
\pgfpathlineto{\pgfqpoint{5.416220in}{0.609692in}}%
\pgfpathlineto{\pgfqpoint{5.417287in}{0.601223in}}%
\pgfpathlineto{\pgfqpoint{5.417821in}{0.602869in}}%
\pgfpathlineto{\pgfqpoint{5.418354in}{0.607034in}}%
\pgfpathlineto{\pgfqpoint{5.418888in}{0.602887in}}%
\pgfpathlineto{\pgfqpoint{5.419955in}{0.605116in}}%
\pgfpathlineto{\pgfqpoint{5.420489in}{0.606518in}}%
\pgfpathlineto{\pgfqpoint{5.421023in}{0.601568in}}%
\pgfpathlineto{\pgfqpoint{5.421557in}{0.604809in}}%
\pgfpathlineto{\pgfqpoint{5.422624in}{0.608476in}}%
\pgfpathlineto{\pgfqpoint{5.423158in}{0.604734in}}%
\pgfpathlineto{\pgfqpoint{5.423691in}{0.610443in}}%
\pgfpathlineto{\pgfqpoint{5.424225in}{0.605834in}}%
\pgfpathlineto{\pgfqpoint{5.424759in}{0.600986in}}%
\pgfpathlineto{\pgfqpoint{5.425292in}{0.602935in}}%
\pgfpathlineto{\pgfqpoint{5.426360in}{0.606439in}}%
\pgfpathlineto{\pgfqpoint{5.427427in}{0.600938in}}%
\pgfpathlineto{\pgfqpoint{5.428495in}{0.606875in}}%
\pgfpathlineto{\pgfqpoint{5.430629in}{0.603126in}}%
\pgfpathlineto{\pgfqpoint{5.432231in}{0.607016in}}%
\pgfpathlineto{\pgfqpoint{5.432764in}{0.604961in}}%
\pgfpathlineto{\pgfqpoint{5.433298in}{0.609795in}}%
\pgfpathlineto{\pgfqpoint{5.433832in}{0.604697in}}%
\pgfpathlineto{\pgfqpoint{5.434365in}{0.606942in}}%
\pgfpathlineto{\pgfqpoint{5.434899in}{0.605119in}}%
\pgfpathlineto{\pgfqpoint{5.435433in}{0.605361in}}%
\pgfpathlineto{\pgfqpoint{5.435966in}{0.611229in}}%
\pgfpathlineto{\pgfqpoint{5.436500in}{0.602051in}}%
\pgfpathlineto{\pgfqpoint{5.437568in}{0.602781in}}%
\pgfpathlineto{\pgfqpoint{5.439169in}{0.610094in}}%
\pgfpathlineto{\pgfqpoint{5.440236in}{0.604151in}}%
\pgfpathlineto{\pgfqpoint{5.440770in}{0.607239in}}%
\pgfpathlineto{\pgfqpoint{5.442371in}{0.601974in}}%
\pgfpathlineto{\pgfqpoint{5.443438in}{0.603857in}}%
\pgfpathlineto{\pgfqpoint{5.444506in}{0.602859in}}%
\pgfpathlineto{\pgfqpoint{5.445039in}{0.607062in}}%
\pgfpathlineto{\pgfqpoint{5.445573in}{0.601604in}}%
\pgfpathlineto{\pgfqpoint{5.446107in}{0.605705in}}%
\pgfpathlineto{\pgfqpoint{5.446641in}{0.606017in}}%
\pgfpathlineto{\pgfqpoint{5.447174in}{0.603528in}}%
\pgfpathlineto{\pgfqpoint{5.447708in}{0.609057in}}%
\pgfpathlineto{\pgfqpoint{5.448242in}{0.606474in}}%
\pgfpathlineto{\pgfqpoint{5.448775in}{0.606612in}}%
\pgfpathlineto{\pgfqpoint{5.449309in}{0.602327in}}%
\pgfpathlineto{\pgfqpoint{5.449843in}{0.603171in}}%
\pgfpathlineto{\pgfqpoint{5.451444in}{0.610771in}}%
\pgfpathlineto{\pgfqpoint{5.453579in}{0.603521in}}%
\pgfpathlineto{\pgfqpoint{5.455180in}{0.606977in}}%
\pgfpathlineto{\pgfqpoint{5.455713in}{0.601759in}}%
\pgfpathlineto{\pgfqpoint{5.456247in}{0.603563in}}%
\pgfpathlineto{\pgfqpoint{5.456781in}{0.605750in}}%
\pgfpathlineto{\pgfqpoint{5.458382in}{0.600755in}}%
\pgfpathlineto{\pgfqpoint{5.459449in}{0.607629in}}%
\pgfpathlineto{\pgfqpoint{5.459983in}{0.604740in}}%
\pgfpathlineto{\pgfqpoint{5.460517in}{0.602069in}}%
\pgfpathlineto{\pgfqpoint{5.462118in}{0.608554in}}%
\pgfpathlineto{\pgfqpoint{5.462652in}{0.606029in}}%
\pgfpathlineto{\pgfqpoint{5.463185in}{0.609653in}}%
\pgfpathlineto{\pgfqpoint{5.464786in}{0.601429in}}%
\pgfpathlineto{\pgfqpoint{5.465854in}{0.605386in}}%
\pgfpathlineto{\pgfqpoint{5.466387in}{0.603213in}}%
\pgfpathlineto{\pgfqpoint{5.467455in}{0.605972in}}%
\pgfpathlineto{\pgfqpoint{5.467989in}{0.600739in}}%
\pgfpathlineto{\pgfqpoint{5.468522in}{0.602830in}}%
\pgfpathlineto{\pgfqpoint{5.469056in}{0.606001in}}%
\pgfpathlineto{\pgfqpoint{5.469590in}{0.603486in}}%
\pgfpathlineto{\pgfqpoint{5.470123in}{0.600738in}}%
\pgfpathlineto{\pgfqpoint{5.470657in}{0.606377in}}%
\pgfpathlineto{\pgfqpoint{5.471191in}{0.605726in}}%
\pgfpathlineto{\pgfqpoint{5.472792in}{0.601650in}}%
\pgfpathlineto{\pgfqpoint{5.473326in}{0.611771in}}%
\pgfpathlineto{\pgfqpoint{5.473859in}{0.606491in}}%
\pgfpathlineto{\pgfqpoint{5.474927in}{0.601568in}}%
\pgfpathlineto{\pgfqpoint{5.475460in}{0.605741in}}%
\pgfpathlineto{\pgfqpoint{5.476528in}{0.605373in}}%
\pgfpathlineto{\pgfqpoint{5.478129in}{0.600683in}}%
\pgfpathlineto{\pgfqpoint{5.478663in}{0.604625in}}%
\pgfpathlineto{\pgfqpoint{5.479196in}{0.604265in}}%
\pgfpathlineto{\pgfqpoint{5.479730in}{0.603324in}}%
\pgfpathlineto{\pgfqpoint{5.480264in}{0.608211in}}%
\pgfpathlineto{\pgfqpoint{5.480797in}{0.606550in}}%
\pgfpathlineto{\pgfqpoint{5.482398in}{0.603443in}}%
\pgfpathlineto{\pgfqpoint{5.482932in}{0.602199in}}%
\pgfpathlineto{\pgfqpoint{5.483466in}{0.607378in}}%
\pgfpathlineto{\pgfqpoint{5.484000in}{0.602992in}}%
\pgfpathlineto{\pgfqpoint{5.486134in}{0.605979in}}%
\pgfpathlineto{\pgfqpoint{5.486668in}{0.607425in}}%
\pgfpathlineto{\pgfqpoint{5.487735in}{0.603197in}}%
\pgfpathlineto{\pgfqpoint{5.488269in}{0.608008in}}%
\pgfpathlineto{\pgfqpoint{5.489337in}{0.601864in}}%
\pgfpathlineto{\pgfqpoint{5.489870in}{0.605112in}}%
\pgfpathlineto{\pgfqpoint{5.490404in}{0.600767in}}%
\pgfpathlineto{\pgfqpoint{5.490938in}{0.610032in}}%
\pgfpathlineto{\pgfqpoint{5.491471in}{0.606916in}}%
\pgfpathlineto{\pgfqpoint{5.492539in}{0.602487in}}%
\pgfpathlineto{\pgfqpoint{5.493072in}{0.608000in}}%
\pgfpathlineto{\pgfqpoint{5.493606in}{0.607765in}}%
\pgfpathlineto{\pgfqpoint{5.495207in}{0.601280in}}%
\pgfpathlineto{\pgfqpoint{5.495741in}{0.605545in}}%
\pgfpathlineto{\pgfqpoint{5.496275in}{0.600194in}}%
\pgfpathlineto{\pgfqpoint{5.497876in}{0.607389in}}%
\pgfpathlineto{\pgfqpoint{5.499477in}{0.602518in}}%
\pgfpathlineto{\pgfqpoint{5.500011in}{0.605035in}}%
\pgfpathlineto{\pgfqpoint{5.500544in}{0.601973in}}%
\pgfpathlineto{\pgfqpoint{5.501078in}{0.604988in}}%
\pgfpathlineto{\pgfqpoint{5.501612in}{0.608973in}}%
\pgfpathlineto{\pgfqpoint{5.502145in}{0.602564in}}%
\pgfpathlineto{\pgfqpoint{5.502679in}{0.604784in}}%
\pgfpathlineto{\pgfqpoint{5.503213in}{0.604706in}}%
\pgfpathlineto{\pgfqpoint{5.503746in}{0.608633in}}%
\pgfpathlineto{\pgfqpoint{5.504280in}{0.605774in}}%
\pgfpathlineto{\pgfqpoint{5.505881in}{0.602023in}}%
\pgfpathlineto{\pgfqpoint{5.507482in}{0.612057in}}%
\pgfpathlineto{\pgfqpoint{5.508550in}{0.602421in}}%
\pgfpathlineto{\pgfqpoint{5.509084in}{0.603106in}}%
\pgfpathlineto{\pgfqpoint{5.509617in}{0.604984in}}%
\pgfpathlineto{\pgfqpoint{5.510151in}{0.600850in}}%
\pgfpathlineto{\pgfqpoint{5.510685in}{0.603597in}}%
\pgfpathlineto{\pgfqpoint{5.511218in}{0.601713in}}%
\pgfpathlineto{\pgfqpoint{5.512286in}{0.605092in}}%
\pgfpathlineto{\pgfqpoint{5.512819in}{0.601172in}}%
\pgfpathlineto{\pgfqpoint{5.513353in}{0.603608in}}%
\pgfpathlineto{\pgfqpoint{5.513887in}{0.606044in}}%
\pgfpathlineto{\pgfqpoint{5.514421in}{0.603591in}}%
\pgfpathlineto{\pgfqpoint{5.516022in}{0.602149in}}%
\pgfpathlineto{\pgfqpoint{5.518156in}{0.609624in}}%
\pgfpathlineto{\pgfqpoint{5.519758in}{0.601032in}}%
\pgfpathlineto{\pgfqpoint{5.520291in}{0.602320in}}%
\pgfpathlineto{\pgfqpoint{5.521892in}{0.607693in}}%
\pgfpathlineto{\pgfqpoint{5.523493in}{0.601415in}}%
\pgfpathlineto{\pgfqpoint{5.524561in}{0.603495in}}%
\pgfpathlineto{\pgfqpoint{5.525095in}{0.600939in}}%
\pgfpathlineto{\pgfqpoint{5.525628in}{0.605729in}}%
\pgfpathlineto{\pgfqpoint{5.526162in}{0.602367in}}%
\pgfpathlineto{\pgfqpoint{5.527763in}{0.607157in}}%
\pgfpathlineto{\pgfqpoint{5.528297in}{0.606276in}}%
\pgfpathlineto{\pgfqpoint{5.529364in}{0.600525in}}%
\pgfpathlineto{\pgfqpoint{5.530432in}{0.602158in}}%
\pgfpathlineto{\pgfqpoint{5.530965in}{0.605682in}}%
\pgfpathlineto{\pgfqpoint{5.531499in}{0.604453in}}%
\pgfpathlineto{\pgfqpoint{5.532033in}{0.601247in}}%
\pgfpathlineto{\pgfqpoint{5.532566in}{0.602446in}}%
\pgfpathlineto{\pgfqpoint{5.533100in}{0.602466in}}%
\pgfpathlineto{\pgfqpoint{5.533634in}{0.600302in}}%
\pgfpathlineto{\pgfqpoint{5.534701in}{0.606497in}}%
\pgfpathlineto{\pgfqpoint{5.535235in}{0.605211in}}%
\pgfpathlineto{\pgfqpoint{5.536836in}{0.601635in}}%
\pgfpathlineto{\pgfqpoint{5.537903in}{0.601877in}}%
\pgfpathlineto{\pgfqpoint{5.538437in}{0.606821in}}%
\pgfpathlineto{\pgfqpoint{5.538971in}{0.602875in}}%
\pgfpathlineto{\pgfqpoint{5.539504in}{0.600901in}}%
\pgfpathlineto{\pgfqpoint{5.541106in}{0.605585in}}%
\pgfpathlineto{\pgfqpoint{5.542173in}{0.601941in}}%
\pgfpathlineto{\pgfqpoint{5.543774in}{0.605333in}}%
\pgfpathlineto{\pgfqpoint{5.544308in}{0.604089in}}%
\pgfpathlineto{\pgfqpoint{5.544841in}{0.608063in}}%
\pgfpathlineto{\pgfqpoint{5.545375in}{0.604549in}}%
\pgfpathlineto{\pgfqpoint{5.545909in}{0.601121in}}%
\pgfpathlineto{\pgfqpoint{5.546443in}{0.607108in}}%
\pgfpathlineto{\pgfqpoint{5.546976in}{0.603739in}}%
\pgfpathlineto{\pgfqpoint{5.549111in}{0.608722in}}%
\pgfpathlineto{\pgfqpoint{5.549645in}{0.604403in}}%
\pgfpathlineto{\pgfqpoint{5.550178in}{0.604736in}}%
\pgfpathlineto{\pgfqpoint{5.550712in}{0.605102in}}%
\pgfpathlineto{\pgfqpoint{5.552313in}{0.602530in}}%
\pgfpathlineto{\pgfqpoint{5.552847in}{0.604156in}}%
\pgfpathlineto{\pgfqpoint{5.553914in}{0.601673in}}%
\pgfpathlineto{\pgfqpoint{5.554448in}{0.604882in}}%
\pgfpathlineto{\pgfqpoint{5.554982in}{0.600430in}}%
\pgfpathlineto{\pgfqpoint{5.555515in}{0.602387in}}%
\pgfpathlineto{\pgfqpoint{5.558718in}{0.603865in}}%
\pgfpathlineto{\pgfqpoint{5.560852in}{0.602199in}}%
\pgfpathlineto{\pgfqpoint{5.562454in}{0.606041in}}%
\pgfpathlineto{\pgfqpoint{5.562987in}{0.604718in}}%
\pgfpathlineto{\pgfqpoint{5.564055in}{0.600507in}}%
\pgfpathlineto{\pgfqpoint{5.564588in}{0.602528in}}%
\pgfpathlineto{\pgfqpoint{5.565656in}{0.607766in}}%
\pgfpathlineto{\pgfqpoint{5.566189in}{0.602112in}}%
\pgfpathlineto{\pgfqpoint{5.566723in}{0.603240in}}%
\pgfpathlineto{\pgfqpoint{5.567257in}{0.603456in}}%
\pgfpathlineto{\pgfqpoint{5.567791in}{0.600681in}}%
\pgfpathlineto{\pgfqpoint{5.568324in}{0.603221in}}%
\pgfpathlineto{\pgfqpoint{5.569392in}{0.604202in}}%
\pgfpathlineto{\pgfqpoint{5.570459in}{0.602273in}}%
\pgfpathlineto{\pgfqpoint{5.570993in}{0.602986in}}%
\pgfpathlineto{\pgfqpoint{5.571526in}{0.604035in}}%
\pgfpathlineto{\pgfqpoint{5.572060in}{0.608144in}}%
\pgfpathlineto{\pgfqpoint{5.572594in}{0.605514in}}%
\pgfpathlineto{\pgfqpoint{5.573128in}{0.603806in}}%
\pgfpathlineto{\pgfqpoint{5.573661in}{0.607548in}}%
\pgfpathlineto{\pgfqpoint{5.575262in}{0.601240in}}%
\pgfpathlineto{\pgfqpoint{5.575796in}{0.603048in}}%
\pgfpathlineto{\pgfqpoint{5.576330in}{0.601891in}}%
\pgfpathlineto{\pgfqpoint{5.576864in}{0.601154in}}%
\pgfpathlineto{\pgfqpoint{5.577397in}{0.604928in}}%
\pgfpathlineto{\pgfqpoint{5.577931in}{0.600519in}}%
\pgfpathlineto{\pgfqpoint{5.578465in}{0.600994in}}%
\pgfpathlineto{\pgfqpoint{5.578998in}{0.604037in}}%
\pgfpathlineto{\pgfqpoint{5.579532in}{0.603280in}}%
\pgfpathlineto{\pgfqpoint{5.580599in}{0.601632in}}%
\pgfpathlineto{\pgfqpoint{5.581133in}{0.603914in}}%
\pgfpathlineto{\pgfqpoint{5.582734in}{0.600802in}}%
\pgfpathlineto{\pgfqpoint{5.583802in}{0.604910in}}%
\pgfpathlineto{\pgfqpoint{5.584335in}{0.601547in}}%
\pgfpathlineto{\pgfqpoint{5.584869in}{0.602150in}}%
\pgfpathlineto{\pgfqpoint{5.586470in}{0.602222in}}%
\pgfpathlineto{\pgfqpoint{5.588071in}{0.603489in}}%
\pgfpathlineto{\pgfqpoint{5.588605in}{0.600631in}}%
\pgfpathlineto{\pgfqpoint{5.589139in}{0.607000in}}%
\pgfpathlineto{\pgfqpoint{5.589672in}{0.605417in}}%
\pgfpathlineto{\pgfqpoint{5.590740in}{0.601371in}}%
\pgfpathlineto{\pgfqpoint{5.592341in}{0.604562in}}%
\pgfpathlineto{\pgfqpoint{5.593408in}{0.600450in}}%
\pgfpathlineto{\pgfqpoint{5.595543in}{0.608630in}}%
\pgfpathlineto{\pgfqpoint{5.596610in}{0.600903in}}%
\pgfpathlineto{\pgfqpoint{5.597678in}{0.603794in}}%
\pgfpathlineto{\pgfqpoint{5.598212in}{0.602081in}}%
\pgfpathlineto{\pgfqpoint{5.598745in}{0.604200in}}%
\pgfpathlineto{\pgfqpoint{5.600346in}{0.600912in}}%
\pgfpathlineto{\pgfqpoint{5.600880in}{0.601092in}}%
\pgfpathlineto{\pgfqpoint{5.601414in}{0.603938in}}%
\pgfpathlineto{\pgfqpoint{5.601947in}{0.601009in}}%
\pgfpathlineto{\pgfqpoint{5.602481in}{0.600067in}}%
\pgfpathlineto{\pgfqpoint{5.603549in}{0.605720in}}%
\pgfpathlineto{\pgfqpoint{5.604082in}{0.605398in}}%
\pgfpathlineto{\pgfqpoint{5.604616in}{0.604631in}}%
\pgfpathlineto{\pgfqpoint{5.606217in}{0.601844in}}%
\pgfpathlineto{\pgfqpoint{5.606751in}{0.600798in}}%
\pgfpathlineto{\pgfqpoint{5.607284in}{0.603242in}}%
\pgfpathlineto{\pgfqpoint{5.607818in}{0.603109in}}%
\pgfpathlineto{\pgfqpoint{5.608352in}{0.601959in}}%
\pgfpathlineto{\pgfqpoint{5.608886in}{0.604738in}}%
\pgfpathlineto{\pgfqpoint{5.609419in}{0.604613in}}%
\pgfpathlineto{\pgfqpoint{5.609953in}{0.601228in}}%
\pgfpathlineto{\pgfqpoint{5.610487in}{0.602104in}}%
\pgfpathlineto{\pgfqpoint{5.612088in}{0.603061in}}%
\pgfpathlineto{\pgfqpoint{5.612621in}{0.601122in}}%
\pgfpathlineto{\pgfqpoint{5.613155in}{0.603962in}}%
\pgfpathlineto{\pgfqpoint{5.613689in}{0.603031in}}%
\pgfpathlineto{\pgfqpoint{5.614223in}{0.600741in}}%
\pgfpathlineto{\pgfqpoint{5.614756in}{0.601991in}}%
\pgfpathlineto{\pgfqpoint{5.615290in}{0.604940in}}%
\pgfpathlineto{\pgfqpoint{5.615824in}{0.603079in}}%
\pgfpathlineto{\pgfqpoint{5.616357in}{0.602363in}}%
\pgfpathlineto{\pgfqpoint{5.617958in}{0.606758in}}%
\pgfpathlineto{\pgfqpoint{5.619560in}{0.602440in}}%
\pgfpathlineto{\pgfqpoint{5.620627in}{0.600655in}}%
\pgfpathlineto{\pgfqpoint{5.622228in}{0.605063in}}%
\pgfpathlineto{\pgfqpoint{5.622762in}{0.605244in}}%
\pgfpathlineto{\pgfqpoint{5.623295in}{0.602203in}}%
\pgfpathlineto{\pgfqpoint{5.623829in}{0.604949in}}%
\pgfpathlineto{\pgfqpoint{5.624363in}{0.606147in}}%
\pgfpathlineto{\pgfqpoint{5.625430in}{0.601838in}}%
\pgfpathlineto{\pgfqpoint{5.626498in}{0.602692in}}%
\pgfpathlineto{\pgfqpoint{5.627565in}{0.603341in}}%
\pgfpathlineto{\pgfqpoint{5.628632in}{0.608999in}}%
\pgfpathlineto{\pgfqpoint{5.629166in}{0.600644in}}%
\pgfpathlineto{\pgfqpoint{5.630234in}{0.600876in}}%
\pgfpathlineto{\pgfqpoint{5.631301in}{0.601123in}}%
\pgfpathlineto{\pgfqpoint{5.632368in}{0.605316in}}%
\pgfpathlineto{\pgfqpoint{5.632902in}{0.601194in}}%
\pgfpathlineto{\pgfqpoint{5.633436in}{0.602208in}}%
\pgfpathlineto{\pgfqpoint{5.633969in}{0.602445in}}%
\pgfpathlineto{\pgfqpoint{5.635571in}{0.600718in}}%
\pgfpathlineto{\pgfqpoint{5.637172in}{0.603954in}}%
\pgfpathlineto{\pgfqpoint{5.638239in}{0.602359in}}%
\pgfpathlineto{\pgfqpoint{5.638773in}{0.606220in}}%
\pgfpathlineto{\pgfqpoint{5.639306in}{0.603118in}}%
\pgfpathlineto{\pgfqpoint{5.640374in}{0.600238in}}%
\pgfpathlineto{\pgfqpoint{5.640908in}{0.603161in}}%
\pgfpathlineto{\pgfqpoint{5.641441in}{0.601999in}}%
\pgfpathlineto{\pgfqpoint{5.642509in}{0.600920in}}%
\pgfpathlineto{\pgfqpoint{5.643042in}{0.604804in}}%
\pgfpathlineto{\pgfqpoint{5.644110in}{0.604549in}}%
\pgfpathlineto{\pgfqpoint{5.645711in}{0.602072in}}%
\pgfpathlineto{\pgfqpoint{5.647312in}{0.604397in}}%
\pgfpathlineto{\pgfqpoint{5.649447in}{0.601436in}}%
\pgfpathlineto{\pgfqpoint{5.649981in}{0.604671in}}%
\pgfpathlineto{\pgfqpoint{5.650514in}{0.600342in}}%
\pgfpathlineto{\pgfqpoint{5.651048in}{0.600473in}}%
\pgfpathlineto{\pgfqpoint{5.651582in}{0.601549in}}%
\pgfpathlineto{\pgfqpoint{5.652115in}{0.606096in}}%
\pgfpathlineto{\pgfqpoint{5.652649in}{0.603582in}}%
\pgfpathlineto{\pgfqpoint{5.653183in}{0.601320in}}%
\pgfpathlineto{\pgfqpoint{5.653716in}{0.603006in}}%
\pgfpathlineto{\pgfqpoint{5.654250in}{0.603840in}}%
\pgfpathlineto{\pgfqpoint{5.654784in}{0.600234in}}%
\pgfpathlineto{\pgfqpoint{5.655318in}{0.602629in}}%
\pgfpathlineto{\pgfqpoint{5.655851in}{0.601121in}}%
\pgfpathlineto{\pgfqpoint{5.656385in}{0.604879in}}%
\pgfpathlineto{\pgfqpoint{5.656919in}{0.600312in}}%
\pgfpathlineto{\pgfqpoint{5.657452in}{0.604514in}}%
\pgfpathlineto{\pgfqpoint{5.657986in}{0.602880in}}%
\pgfpathlineto{\pgfqpoint{5.658520in}{0.604466in}}%
\pgfpathlineto{\pgfqpoint{5.659053in}{0.603700in}}%
\pgfpathlineto{\pgfqpoint{5.659587in}{0.604463in}}%
\pgfpathlineto{\pgfqpoint{5.660121in}{0.605167in}}%
\pgfpathlineto{\pgfqpoint{5.662256in}{0.601521in}}%
\pgfpathlineto{\pgfqpoint{5.663857in}{0.603990in}}%
\pgfpathlineto{\pgfqpoint{5.665992in}{0.600378in}}%
\pgfpathlineto{\pgfqpoint{5.666525in}{0.603868in}}%
\pgfpathlineto{\pgfqpoint{5.667059in}{0.601805in}}%
\pgfpathlineto{\pgfqpoint{5.668126in}{0.604644in}}%
\pgfpathlineto{\pgfqpoint{5.669194in}{0.601420in}}%
\pgfpathlineto{\pgfqpoint{5.669727in}{0.602964in}}%
\pgfpathlineto{\pgfqpoint{5.670795in}{0.604944in}}%
\pgfpathlineto{\pgfqpoint{5.672396in}{0.601208in}}%
\pgfpathlineto{\pgfqpoint{5.673463in}{0.607586in}}%
\pgfpathlineto{\pgfqpoint{5.674531in}{0.601236in}}%
\pgfpathlineto{\pgfqpoint{5.675598in}{0.602969in}}%
\pgfpathlineto{\pgfqpoint{5.676132in}{0.601950in}}%
\pgfpathlineto{\pgfqpoint{5.676666in}{0.602366in}}%
\pgfpathlineto{\pgfqpoint{5.678267in}{0.603705in}}%
\pgfpathlineto{\pgfqpoint{5.679334in}{0.601972in}}%
\pgfpathlineto{\pgfqpoint{5.680401in}{0.606870in}}%
\pgfpathlineto{\pgfqpoint{5.682003in}{0.601058in}}%
\pgfpathlineto{\pgfqpoint{5.682536in}{0.607414in}}%
\pgfpathlineto{\pgfqpoint{5.683070in}{0.600909in}}%
\pgfpathlineto{\pgfqpoint{5.683604in}{0.604350in}}%
\pgfpathlineto{\pgfqpoint{5.684137in}{0.603304in}}%
\pgfpathlineto{\pgfqpoint{5.684671in}{0.602126in}}%
\pgfpathlineto{\pgfqpoint{5.685738in}{0.604645in}}%
\pgfpathlineto{\pgfqpoint{5.686806in}{0.601899in}}%
\pgfpathlineto{\pgfqpoint{5.687340in}{0.603273in}}%
\pgfpathlineto{\pgfqpoint{5.687873in}{0.601964in}}%
\pgfpathlineto{\pgfqpoint{5.688407in}{0.602038in}}%
\pgfpathlineto{\pgfqpoint{5.688941in}{0.600405in}}%
\pgfpathlineto{\pgfqpoint{5.689474in}{0.604167in}}%
\pgfpathlineto{\pgfqpoint{5.690008in}{0.601039in}}%
\pgfpathlineto{\pgfqpoint{5.690542in}{0.603361in}}%
\pgfpathlineto{\pgfqpoint{5.691075in}{0.601718in}}%
\pgfpathlineto{\pgfqpoint{5.692143in}{0.601850in}}%
\pgfpathlineto{\pgfqpoint{5.692677in}{0.603990in}}%
\pgfpathlineto{\pgfqpoint{5.693210in}{0.601845in}}%
\pgfpathlineto{\pgfqpoint{5.693744in}{0.601303in}}%
\pgfpathlineto{\pgfqpoint{5.694278in}{0.601837in}}%
\pgfpathlineto{\pgfqpoint{5.694811in}{0.603826in}}%
\pgfpathlineto{\pgfqpoint{5.695345in}{0.602167in}}%
\pgfpathlineto{\pgfqpoint{5.697480in}{0.600794in}}%
\pgfpathlineto{\pgfqpoint{5.698014in}{0.602538in}}%
\pgfpathlineto{\pgfqpoint{5.698547in}{0.600961in}}%
\pgfpathlineto{\pgfqpoint{5.700682in}{0.603163in}}%
\pgfpathlineto{\pgfqpoint{5.701216in}{0.602149in}}%
\pgfpathlineto{\pgfqpoint{5.701749in}{0.605971in}}%
\pgfpathlineto{\pgfqpoint{5.702283in}{0.603569in}}%
\pgfpathlineto{\pgfqpoint{5.703884in}{0.601817in}}%
\pgfpathlineto{\pgfqpoint{5.704418in}{0.603414in}}%
\pgfpathlineto{\pgfqpoint{5.704952in}{0.601606in}}%
\pgfpathlineto{\pgfqpoint{5.706553in}{0.605230in}}%
\pgfpathlineto{\pgfqpoint{5.708154in}{0.602564in}}%
\pgfpathlineto{\pgfqpoint{5.708688in}{0.601427in}}%
\pgfpathlineto{\pgfqpoint{5.709221in}{0.604959in}}%
\pgfpathlineto{\pgfqpoint{5.709755in}{0.602121in}}%
\pgfpathlineto{\pgfqpoint{5.710289in}{0.604551in}}%
\pgfpathlineto{\pgfqpoint{5.710822in}{0.603840in}}%
\pgfpathlineto{\pgfqpoint{5.711890in}{0.601032in}}%
\pgfpathlineto{\pgfqpoint{5.712424in}{0.601670in}}%
\pgfpathlineto{\pgfqpoint{5.714025in}{0.601890in}}%
\pgfpathlineto{\pgfqpoint{5.714558in}{0.601714in}}%
\pgfpathlineto{\pgfqpoint{5.715092in}{0.604510in}}%
\pgfpathlineto{\pgfqpoint{5.715626in}{0.603720in}}%
\pgfpathlineto{\pgfqpoint{5.716159in}{0.603704in}}%
\pgfpathlineto{\pgfqpoint{5.716693in}{0.600339in}}%
\pgfpathlineto{\pgfqpoint{5.717761in}{0.604488in}}%
\pgfpathlineto{\pgfqpoint{5.719362in}{0.601689in}}%
\pgfpathlineto{\pgfqpoint{5.719895in}{0.602334in}}%
\pgfpathlineto{\pgfqpoint{5.720429in}{0.600471in}}%
\pgfpathlineto{\pgfqpoint{5.720963in}{0.601886in}}%
\pgfpathlineto{\pgfqpoint{5.722030in}{0.603658in}}%
\pgfpathlineto{\pgfqpoint{5.722564in}{0.601147in}}%
\pgfpathlineto{\pgfqpoint{5.723098in}{0.602783in}}%
\pgfpathlineto{\pgfqpoint{5.723631in}{0.602847in}}%
\pgfpathlineto{\pgfqpoint{5.724165in}{0.601380in}}%
\pgfpathlineto{\pgfqpoint{5.724699in}{0.602353in}}%
\pgfpathlineto{\pgfqpoint{5.725232in}{0.605539in}}%
\pgfpathlineto{\pgfqpoint{5.726300in}{0.605233in}}%
\pgfpathlineto{\pgfqpoint{5.726833in}{0.602106in}}%
\pgfpathlineto{\pgfqpoint{5.727367in}{0.602642in}}%
\pgfpathlineto{\pgfqpoint{5.727901in}{0.603990in}}%
\pgfpathlineto{\pgfqpoint{5.729502in}{0.600443in}}%
\pgfpathlineto{\pgfqpoint{5.731103in}{0.604768in}}%
\pgfpathlineto{\pgfqpoint{5.732170in}{0.601140in}}%
\pgfpathlineto{\pgfqpoint{5.732704in}{0.604480in}}%
\pgfpathlineto{\pgfqpoint{5.733238in}{0.601090in}}%
\pgfpathlineto{\pgfqpoint{5.733772in}{0.602492in}}%
\pgfpathlineto{\pgfqpoint{5.734305in}{0.601515in}}%
\pgfpathlineto{\pgfqpoint{5.735373in}{0.602183in}}%
\pgfpathlineto{\pgfqpoint{5.736440in}{0.607233in}}%
\pgfpathlineto{\pgfqpoint{5.736974in}{0.605693in}}%
\pgfpathlineto{\pgfqpoint{5.737507in}{0.602208in}}%
\pgfpathlineto{\pgfqpoint{5.738575in}{0.602496in}}%
\pgfpathlineto{\pgfqpoint{5.739109in}{0.609173in}}%
\pgfpathlineto{\pgfqpoint{5.739642in}{0.603347in}}%
\pgfpathlineto{\pgfqpoint{5.740710in}{0.601077in}}%
\pgfpathlineto{\pgfqpoint{5.741243in}{0.602886in}}%
\pgfpathlineto{\pgfqpoint{5.742311in}{0.604307in}}%
\pgfpathlineto{\pgfqpoint{5.743912in}{0.603335in}}%
\pgfpathlineto{\pgfqpoint{5.747114in}{0.601104in}}%
\pgfpathlineto{\pgfqpoint{5.748715in}{0.603409in}}%
\pgfpathlineto{\pgfqpoint{5.749783in}{0.600504in}}%
\pgfpathlineto{\pgfqpoint{5.750316in}{0.602307in}}%
\pgfpathlineto{\pgfqpoint{5.751384in}{0.602175in}}%
\pgfpathlineto{\pgfqpoint{5.751917in}{0.603591in}}%
\pgfpathlineto{\pgfqpoint{5.752451in}{0.602764in}}%
\pgfpathlineto{\pgfqpoint{5.753518in}{0.600514in}}%
\pgfpathlineto{\pgfqpoint{5.754052in}{0.602234in}}%
\pgfpathlineto{\pgfqpoint{5.755120in}{0.603119in}}%
\pgfpathlineto{\pgfqpoint{5.755653in}{0.604391in}}%
\pgfpathlineto{\pgfqpoint{5.756187in}{0.601573in}}%
\pgfpathlineto{\pgfqpoint{5.756721in}{0.602984in}}%
\pgfpathlineto{\pgfqpoint{5.757254in}{0.605083in}}%
\pgfpathlineto{\pgfqpoint{5.758855in}{0.601289in}}%
\pgfpathlineto{\pgfqpoint{5.759389in}{0.600488in}}%
\pgfpathlineto{\pgfqpoint{5.759923in}{0.601923in}}%
\pgfpathlineto{\pgfqpoint{5.760457in}{0.601142in}}%
\pgfpathlineto{\pgfqpoint{5.761524in}{0.601634in}}%
\pgfpathlineto{\pgfqpoint{5.762058in}{0.604441in}}%
\pgfpathlineto{\pgfqpoint{5.762591in}{0.602234in}}%
\pgfpathlineto{\pgfqpoint{5.763125in}{0.600712in}}%
\pgfpathlineto{\pgfqpoint{5.763659in}{0.604647in}}%
\pgfpathlineto{\pgfqpoint{5.764192in}{0.604493in}}%
\pgfpathlineto{\pgfqpoint{5.765260in}{0.602488in}}%
\pgfpathlineto{\pgfqpoint{5.765794in}{0.604183in}}%
\pgfpathlineto{\pgfqpoint{5.766327in}{0.602648in}}%
\pgfpathlineto{\pgfqpoint{5.766861in}{0.603505in}}%
\pgfpathlineto{\pgfqpoint{5.767395in}{0.601415in}}%
\pgfpathlineto{\pgfqpoint{5.767928in}{0.604093in}}%
\pgfpathlineto{\pgfqpoint{5.768462in}{0.601558in}}%
\pgfpathlineto{\pgfqpoint{5.770063in}{0.603049in}}%
\pgfpathlineto{\pgfqpoint{5.770597in}{0.601883in}}%
\pgfpathlineto{\pgfqpoint{5.771131in}{0.604381in}}%
\pgfpathlineto{\pgfqpoint{5.771664in}{0.603914in}}%
\pgfpathlineto{\pgfqpoint{5.772198in}{0.602898in}}%
\pgfpathlineto{\pgfqpoint{5.772732in}{0.605509in}}%
\pgfpathlineto{\pgfqpoint{5.773265in}{0.602837in}}%
\pgfpathlineto{\pgfqpoint{5.774866in}{0.600692in}}%
\pgfpathlineto{\pgfqpoint{5.775400in}{0.601501in}}%
\pgfpathlineto{\pgfqpoint{5.777001in}{0.604288in}}%
\pgfpathlineto{\pgfqpoint{5.778602in}{0.600578in}}%
\pgfpathlineto{\pgfqpoint{5.779136in}{0.602974in}}%
\pgfpathlineto{\pgfqpoint{5.779670in}{0.602191in}}%
\pgfpathlineto{\pgfqpoint{5.780204in}{0.601859in}}%
\pgfpathlineto{\pgfqpoint{5.781271in}{0.606321in}}%
\pgfpathlineto{\pgfqpoint{5.781805in}{0.605676in}}%
\pgfpathlineto{\pgfqpoint{5.783406in}{0.603496in}}%
\pgfpathlineto{\pgfqpoint{5.783939in}{0.603561in}}%
\pgfpathlineto{\pgfqpoint{5.785007in}{0.601104in}}%
\pgfpathlineto{\pgfqpoint{5.785541in}{0.604110in}}%
\pgfpathlineto{\pgfqpoint{5.786074in}{0.600913in}}%
\pgfpathlineto{\pgfqpoint{5.787142in}{0.604175in}}%
\pgfpathlineto{\pgfqpoint{5.787675in}{0.603021in}}%
\pgfpathlineto{\pgfqpoint{5.789810in}{0.601199in}}%
\pgfpathlineto{\pgfqpoint{5.790344in}{0.605680in}}%
\pgfpathlineto{\pgfqpoint{5.790878in}{0.601686in}}%
\pgfpathlineto{\pgfqpoint{5.791945in}{0.604002in}}%
\pgfpathlineto{\pgfqpoint{5.792479in}{0.601048in}}%
\pgfpathlineto{\pgfqpoint{5.793012in}{0.602568in}}%
\pgfpathlineto{\pgfqpoint{5.794080in}{0.602447in}}%
\pgfpathlineto{\pgfqpoint{5.794613in}{0.600874in}}%
\pgfpathlineto{\pgfqpoint{5.795681in}{0.606874in}}%
\pgfpathlineto{\pgfqpoint{5.796215in}{0.604444in}}%
\pgfpathlineto{\pgfqpoint{5.797282in}{0.602726in}}%
\pgfpathlineto{\pgfqpoint{5.797816in}{0.604646in}}%
\pgfpathlineto{\pgfqpoint{5.798349in}{0.601119in}}%
\pgfpathlineto{\pgfqpoint{5.798883in}{0.605227in}}%
\pgfpathlineto{\pgfqpoint{5.799417in}{0.604577in}}%
\pgfpathlineto{\pgfqpoint{5.801018in}{0.602952in}}%
\pgfpathlineto{\pgfqpoint{5.802085in}{0.600031in}}%
\pgfpathlineto{\pgfqpoint{5.802619in}{0.601954in}}%
\pgfpathlineto{\pgfqpoint{5.803153in}{0.603810in}}%
\pgfpathlineto{\pgfqpoint{5.803686in}{0.603230in}}%
\pgfpathlineto{\pgfqpoint{5.804220in}{0.602679in}}%
\pgfpathlineto{\pgfqpoint{5.804754in}{0.604042in}}%
\pgfpathlineto{\pgfqpoint{5.805287in}{0.600876in}}%
\pgfpathlineto{\pgfqpoint{5.805821in}{0.603225in}}%
\pgfpathlineto{\pgfqpoint{5.806355in}{0.601456in}}%
\pgfpathlineto{\pgfqpoint{5.806889in}{0.602536in}}%
\pgfpathlineto{\pgfqpoint{5.807422in}{0.604357in}}%
\pgfpathlineto{\pgfqpoint{5.808490in}{0.601005in}}%
\pgfpathlineto{\pgfqpoint{5.809023in}{0.601775in}}%
\pgfpathlineto{\pgfqpoint{5.809557in}{0.600874in}}%
\pgfpathlineto{\pgfqpoint{5.810091in}{0.601193in}}%
\pgfpathlineto{\pgfqpoint{5.810624in}{0.602220in}}%
\pgfpathlineto{\pgfqpoint{5.811158in}{0.601942in}}%
\pgfpathlineto{\pgfqpoint{5.811692in}{0.601001in}}%
\pgfpathlineto{\pgfqpoint{5.813293in}{0.603538in}}%
\pgfpathlineto{\pgfqpoint{5.813827in}{0.603844in}}%
\pgfpathlineto{\pgfqpoint{5.815428in}{0.602198in}}%
\pgfpathlineto{\pgfqpoint{5.815961in}{0.603810in}}%
\pgfpathlineto{\pgfqpoint{5.817029in}{0.601813in}}%
\pgfpathlineto{\pgfqpoint{5.817563in}{0.604632in}}%
\pgfpathlineto{\pgfqpoint{5.818096in}{0.603037in}}%
\pgfpathlineto{\pgfqpoint{5.821298in}{0.602071in}}%
\pgfpathlineto{\pgfqpoint{5.821832in}{0.600915in}}%
\pgfpathlineto{\pgfqpoint{5.822366in}{0.601730in}}%
\pgfpathlineto{\pgfqpoint{5.822900in}{0.602699in}}%
\pgfpathlineto{\pgfqpoint{5.824501in}{0.601308in}}%
\pgfpathlineto{\pgfqpoint{5.825034in}{0.606901in}}%
\pgfpathlineto{\pgfqpoint{5.825568in}{0.602051in}}%
\pgfpathlineto{\pgfqpoint{5.826102in}{0.604638in}}%
\pgfpathlineto{\pgfqpoint{5.827703in}{0.601013in}}%
\pgfpathlineto{\pgfqpoint{5.828770in}{0.605117in}}%
\pgfpathlineto{\pgfqpoint{5.829304in}{0.600329in}}%
\pgfpathlineto{\pgfqpoint{5.829838in}{0.601265in}}%
\pgfpathlineto{\pgfqpoint{5.830371in}{0.602825in}}%
\pgfpathlineto{\pgfqpoint{5.830905in}{0.602418in}}%
\pgfpathlineto{\pgfqpoint{5.832506in}{0.600951in}}%
\pgfpathlineto{\pgfqpoint{5.833040in}{0.601985in}}%
\pgfpathlineto{\pgfqpoint{5.833574in}{0.602661in}}%
\pgfpathlineto{\pgfqpoint{5.834107in}{0.602045in}}%
\pgfpathlineto{\pgfqpoint{5.835175in}{0.600984in}}%
\pgfpathlineto{\pgfqpoint{5.836776in}{0.605429in}}%
\pgfpathlineto{\pgfqpoint{5.838377in}{0.601716in}}%
\pgfpathlineto{\pgfqpoint{5.838911in}{0.604372in}}%
\pgfpathlineto{\pgfqpoint{5.839444in}{0.600225in}}%
\pgfpathlineto{\pgfqpoint{5.839978in}{0.602099in}}%
\pgfpathlineto{\pgfqpoint{5.841045in}{0.600412in}}%
\pgfpathlineto{\pgfqpoint{5.841579in}{0.602355in}}%
\pgfpathlineto{\pgfqpoint{5.842113in}{0.600520in}}%
\pgfpathlineto{\pgfqpoint{5.844248in}{0.604609in}}%
\pgfpathlineto{\pgfqpoint{5.844781in}{0.603699in}}%
\pgfpathlineto{\pgfqpoint{5.845315in}{0.604089in}}%
\pgfpathlineto{\pgfqpoint{5.845849in}{0.602129in}}%
\pgfpathlineto{\pgfqpoint{5.846382in}{0.603470in}}%
\pgfpathlineto{\pgfqpoint{5.846916in}{0.606605in}}%
\pgfpathlineto{\pgfqpoint{5.847984in}{0.600835in}}%
\pgfpathlineto{\pgfqpoint{5.849585in}{0.604819in}}%
\pgfpathlineto{\pgfqpoint{5.851186in}{0.601319in}}%
\pgfpathlineto{\pgfqpoint{5.851719in}{0.601876in}}%
\pgfpathlineto{\pgfqpoint{5.852253in}{0.605189in}}%
\pgfpathlineto{\pgfqpoint{5.852787in}{0.602115in}}%
\pgfpathlineto{\pgfqpoint{5.853321in}{0.604416in}}%
\pgfpathlineto{\pgfqpoint{5.853854in}{0.600686in}}%
\pgfpathlineto{\pgfqpoint{5.854388in}{0.603236in}}%
\pgfpathlineto{\pgfqpoint{5.855455in}{0.601811in}}%
\pgfpathlineto{\pgfqpoint{5.855989in}{0.602280in}}%
\pgfpathlineto{\pgfqpoint{5.856523in}{0.601766in}}%
\pgfpathlineto{\pgfqpoint{5.857056in}{0.602705in}}%
\pgfpathlineto{\pgfqpoint{5.857590in}{0.602323in}}%
\pgfpathlineto{\pgfqpoint{5.858124in}{0.603638in}}%
\pgfpathlineto{\pgfqpoint{5.858658in}{0.601395in}}%
\pgfpathlineto{\pgfqpoint{5.859191in}{0.603238in}}%
\pgfpathlineto{\pgfqpoint{5.860259in}{0.603735in}}%
\pgfpathlineto{\pgfqpoint{5.861326in}{0.601746in}}%
\pgfpathlineto{\pgfqpoint{5.862393in}{0.603501in}}%
\pgfpathlineto{\pgfqpoint{5.862927in}{0.600965in}}%
\pgfpathlineto{\pgfqpoint{5.863461in}{0.604707in}}%
\pgfpathlineto{\pgfqpoint{5.863995in}{0.603717in}}%
\pgfpathlineto{\pgfqpoint{5.864528in}{0.601351in}}%
\pgfpathlineto{\pgfqpoint{5.865062in}{0.601893in}}%
\pgfpathlineto{\pgfqpoint{5.865596in}{0.602431in}}%
\pgfpathlineto{\pgfqpoint{5.866129in}{0.600807in}}%
\pgfpathlineto{\pgfqpoint{5.867197in}{0.604537in}}%
\pgfpathlineto{\pgfqpoint{5.868798in}{0.601472in}}%
\pgfpathlineto{\pgfqpoint{5.869332in}{0.602639in}}%
\pgfpathlineto{\pgfqpoint{5.869865in}{0.601924in}}%
\pgfpathlineto{\pgfqpoint{5.870399in}{0.601005in}}%
\pgfpathlineto{\pgfqpoint{5.870933in}{0.603335in}}%
\pgfpathlineto{\pgfqpoint{5.871466in}{0.602798in}}%
\pgfpathlineto{\pgfqpoint{5.872000in}{0.602056in}}%
\pgfpathlineto{\pgfqpoint{5.872534in}{0.603113in}}%
\pgfpathlineto{\pgfqpoint{5.874669in}{0.604033in}}%
\pgfpathlineto{\pgfqpoint{5.876270in}{0.602012in}}%
\pgfpathlineto{\pgfqpoint{5.876803in}{0.602655in}}%
\pgfpathlineto{\pgfqpoint{5.877337in}{0.600749in}}%
\pgfpathlineto{\pgfqpoint{5.877871in}{0.601966in}}%
\pgfpathlineto{\pgfqpoint{5.878404in}{0.604401in}}%
\pgfpathlineto{\pgfqpoint{5.878938in}{0.603198in}}%
\pgfpathlineto{\pgfqpoint{5.880539in}{0.601200in}}%
\pgfpathlineto{\pgfqpoint{5.881073in}{0.602396in}}%
\pgfpathlineto{\pgfqpoint{5.881607in}{0.606044in}}%
\pgfpathlineto{\pgfqpoint{5.882140in}{0.603852in}}%
\pgfpathlineto{\pgfqpoint{5.882674in}{0.604453in}}%
\pgfpathlineto{\pgfqpoint{5.883741in}{0.603442in}}%
\pgfpathlineto{\pgfqpoint{5.884275in}{0.605092in}}%
\pgfpathlineto{\pgfqpoint{5.885343in}{0.601891in}}%
\pgfpathlineto{\pgfqpoint{5.885876in}{0.602063in}}%
\pgfpathlineto{\pgfqpoint{5.886410in}{0.601030in}}%
\pgfpathlineto{\pgfqpoint{5.887477in}{0.606044in}}%
\pgfpathlineto{\pgfqpoint{5.888011in}{0.603891in}}%
\pgfpathlineto{\pgfqpoint{5.888545in}{0.603973in}}%
\pgfpathlineto{\pgfqpoint{5.889612in}{0.600130in}}%
\pgfpathlineto{\pgfqpoint{5.890146in}{0.600933in}}%
\pgfpathlineto{\pgfqpoint{5.891213in}{0.606223in}}%
\pgfpathlineto{\pgfqpoint{5.892281in}{0.601141in}}%
\pgfpathlineto{\pgfqpoint{5.892814in}{0.601524in}}%
\pgfpathlineto{\pgfqpoint{5.893348in}{0.601860in}}%
\pgfpathlineto{\pgfqpoint{5.893882in}{0.603512in}}%
\pgfpathlineto{\pgfqpoint{5.894415in}{0.600758in}}%
\pgfpathlineto{\pgfqpoint{5.894949in}{0.601842in}}%
\pgfpathlineto{\pgfqpoint{5.895483in}{0.601208in}}%
\pgfpathlineto{\pgfqpoint{5.896017in}{0.602179in}}%
\pgfpathlineto{\pgfqpoint{5.896550in}{0.603104in}}%
\pgfpathlineto{\pgfqpoint{5.897084in}{0.601308in}}%
\pgfpathlineto{\pgfqpoint{5.897618in}{0.602011in}}%
\pgfpathlineto{\pgfqpoint{5.898685in}{0.601489in}}%
\pgfpathlineto{\pgfqpoint{5.899219in}{0.602523in}}%
\pgfpathlineto{\pgfqpoint{5.899752in}{0.600989in}}%
\pgfpathlineto{\pgfqpoint{5.900820in}{0.604025in}}%
\pgfpathlineto{\pgfqpoint{5.901354in}{0.602295in}}%
\pgfpathlineto{\pgfqpoint{5.901887in}{0.606151in}}%
\pgfpathlineto{\pgfqpoint{5.902421in}{0.602266in}}%
\pgfpathlineto{\pgfqpoint{5.904022in}{0.600891in}}%
\pgfpathlineto{\pgfqpoint{5.905089in}{0.605218in}}%
\pgfpathlineto{\pgfqpoint{5.905623in}{0.604951in}}%
\pgfpathlineto{\pgfqpoint{5.907224in}{0.601341in}}%
\pgfpathlineto{\pgfqpoint{5.907758in}{0.601994in}}%
\pgfpathlineto{\pgfqpoint{5.908292in}{0.605108in}}%
\pgfpathlineto{\pgfqpoint{5.908825in}{0.600861in}}%
\pgfpathlineto{\pgfqpoint{5.909359in}{0.605319in}}%
\pgfpathlineto{\pgfqpoint{5.911494in}{0.602782in}}%
\pgfpathlineto{\pgfqpoint{5.913095in}{0.603319in}}%
\pgfpathlineto{\pgfqpoint{5.913629in}{0.605142in}}%
\pgfpathlineto{\pgfqpoint{5.914162in}{0.602524in}}%
\pgfpathlineto{\pgfqpoint{5.914696in}{0.603095in}}%
\pgfpathlineto{\pgfqpoint{5.915764in}{0.603076in}}%
\pgfpathlineto{\pgfqpoint{5.916297in}{0.600866in}}%
\pgfpathlineto{\pgfqpoint{5.916831in}{0.604332in}}%
\pgfpathlineto{\pgfqpoint{5.917365in}{0.602037in}}%
\pgfpathlineto{\pgfqpoint{5.917898in}{0.600807in}}%
\pgfpathlineto{\pgfqpoint{5.918432in}{0.602097in}}%
\pgfpathlineto{\pgfqpoint{5.918966in}{0.602329in}}%
\pgfpathlineto{\pgfqpoint{5.919499in}{0.600492in}}%
\pgfpathlineto{\pgfqpoint{5.920033in}{0.601501in}}%
\pgfpathlineto{\pgfqpoint{5.920567in}{0.601820in}}%
\pgfpathlineto{\pgfqpoint{5.921101in}{0.600685in}}%
\pgfpathlineto{\pgfqpoint{5.922702in}{0.603889in}}%
\pgfpathlineto{\pgfqpoint{5.923235in}{0.603662in}}%
\pgfpathlineto{\pgfqpoint{5.923769in}{0.600640in}}%
\pgfpathlineto{\pgfqpoint{5.924303in}{0.603656in}}%
\pgfpathlineto{\pgfqpoint{5.925370in}{0.602676in}}%
\pgfpathlineto{\pgfqpoint{5.927505in}{0.605540in}}%
\pgfpathlineto{\pgfqpoint{5.928039in}{0.602560in}}%
\pgfpathlineto{\pgfqpoint{5.928572in}{0.606981in}}%
\pgfpathlineto{\pgfqpoint{5.929106in}{0.602582in}}%
\pgfpathlineto{\pgfqpoint{5.929640in}{0.603642in}}%
\pgfpathlineto{\pgfqpoint{5.930707in}{0.601250in}}%
\pgfpathlineto{\pgfqpoint{5.931241in}{0.603989in}}%
\pgfpathlineto{\pgfqpoint{5.931775in}{0.601106in}}%
\pgfpathlineto{\pgfqpoint{5.932308in}{0.601288in}}%
\pgfpathlineto{\pgfqpoint{5.933376in}{0.607636in}}%
\pgfpathlineto{\pgfqpoint{5.933909in}{0.602588in}}%
\pgfpathlineto{\pgfqpoint{5.934977in}{0.602815in}}%
\pgfpathlineto{\pgfqpoint{5.936044in}{0.604076in}}%
\pgfpathlineto{\pgfqpoint{5.936578in}{0.608587in}}%
\pgfpathlineto{\pgfqpoint{5.937112in}{0.603163in}}%
\pgfpathlineto{\pgfqpoint{5.937645in}{0.605185in}}%
\pgfpathlineto{\pgfqpoint{5.938713in}{0.601779in}}%
\pgfpathlineto{\pgfqpoint{5.939246in}{0.606575in}}%
\pgfpathlineto{\pgfqpoint{5.939780in}{0.602645in}}%
\pgfpathlineto{\pgfqpoint{5.940314in}{0.605509in}}%
\pgfpathlineto{\pgfqpoint{5.940847in}{0.603574in}}%
\pgfpathlineto{\pgfqpoint{5.941381in}{0.604382in}}%
\pgfpathlineto{\pgfqpoint{5.941915in}{0.600937in}}%
\pgfpathlineto{\pgfqpoint{5.942449in}{0.601879in}}%
\pgfpathlineto{\pgfqpoint{5.943516in}{0.602066in}}%
\pgfpathlineto{\pgfqpoint{5.944050in}{0.603656in}}%
\pgfpathlineto{\pgfqpoint{5.944583in}{0.602546in}}%
\pgfpathlineto{\pgfqpoint{5.945117in}{0.601847in}}%
\pgfpathlineto{\pgfqpoint{5.945651in}{0.602299in}}%
\pgfpathlineto{\pgfqpoint{5.946184in}{0.607758in}}%
\pgfpathlineto{\pgfqpoint{5.946718in}{0.602014in}}%
\pgfpathlineto{\pgfqpoint{5.948319in}{0.601156in}}%
\pgfpathlineto{\pgfqpoint{5.948853in}{0.604383in}}%
\pgfpathlineto{\pgfqpoint{5.949387in}{0.600151in}}%
\pgfpathlineto{\pgfqpoint{5.949920in}{0.603481in}}%
\pgfpathlineto{\pgfqpoint{5.952055in}{0.600747in}}%
\pgfpathlineto{\pgfqpoint{5.953123in}{0.602411in}}%
\pgfpathlineto{\pgfqpoint{5.954190in}{0.601137in}}%
\pgfpathlineto{\pgfqpoint{5.955257in}{0.603830in}}%
\pgfpathlineto{\pgfqpoint{5.956325in}{0.601418in}}%
\pgfpathlineto{\pgfqpoint{5.956858in}{0.607192in}}%
\pgfpathlineto{\pgfqpoint{5.957392in}{0.603644in}}%
\pgfpathlineto{\pgfqpoint{5.957926in}{0.600068in}}%
\pgfpathlineto{\pgfqpoint{5.958460in}{0.600926in}}%
\pgfpathlineto{\pgfqpoint{5.958993in}{0.600634in}}%
\pgfpathlineto{\pgfqpoint{5.960061in}{0.603071in}}%
\pgfpathlineto{\pgfqpoint{5.960594in}{0.601522in}}%
\pgfpathlineto{\pgfqpoint{5.961128in}{0.604757in}}%
\pgfpathlineto{\pgfqpoint{5.961662in}{0.602596in}}%
\pgfpathlineto{\pgfqpoint{5.962729in}{0.601850in}}%
\pgfpathlineto{\pgfqpoint{5.963263in}{0.601149in}}%
\pgfpathlineto{\pgfqpoint{5.963797in}{0.603207in}}%
\pgfpathlineto{\pgfqpoint{5.964330in}{0.601946in}}%
\pgfpathlineto{\pgfqpoint{5.966999in}{0.603533in}}%
\pgfpathlineto{\pgfqpoint{5.968066in}{0.601197in}}%
\pgfpathlineto{\pgfqpoint{5.968600in}{0.601888in}}%
\pgfpathlineto{\pgfqpoint{5.969134in}{0.603890in}}%
\pgfpathlineto{\pgfqpoint{5.969667in}{0.603733in}}%
\pgfpathlineto{\pgfqpoint{5.970735in}{0.602278in}}%
\pgfpathlineto{\pgfqpoint{5.971268in}{0.604753in}}%
\pgfpathlineto{\pgfqpoint{5.971802in}{0.602329in}}%
\pgfpathlineto{\pgfqpoint{5.973403in}{0.602674in}}%
\pgfpathlineto{\pgfqpoint{5.973937in}{0.603682in}}%
\pgfpathlineto{\pgfqpoint{5.975538in}{0.601081in}}%
\pgfpathlineto{\pgfqpoint{5.977673in}{0.603281in}}%
\pgfpathlineto{\pgfqpoint{5.979274in}{0.600554in}}%
\pgfpathlineto{\pgfqpoint{5.981409in}{0.602939in}}%
\pgfpathlineto{\pgfqpoint{5.981942in}{0.601526in}}%
\pgfpathlineto{\pgfqpoint{5.982476in}{0.604268in}}%
\pgfpathlineto{\pgfqpoint{5.983010in}{0.603859in}}%
\pgfpathlineto{\pgfqpoint{5.984611in}{0.600840in}}%
\pgfpathlineto{\pgfqpoint{5.985678in}{0.603794in}}%
\pgfpathlineto{\pgfqpoint{5.986212in}{0.602119in}}%
\pgfpathlineto{\pgfqpoint{5.987279in}{0.601119in}}%
\pgfpathlineto{\pgfqpoint{5.987813in}{0.606484in}}%
\pgfpathlineto{\pgfqpoint{5.988347in}{0.604222in}}%
\pgfpathlineto{\pgfqpoint{5.988881in}{0.600428in}}%
\pgfpathlineto{\pgfqpoint{5.989414in}{0.602143in}}%
\pgfpathlineto{\pgfqpoint{5.991549in}{0.601453in}}%
\pgfpathlineto{\pgfqpoint{5.992616in}{0.603740in}}%
\pgfpathlineto{\pgfqpoint{5.993150in}{0.603131in}}%
\pgfpathlineto{\pgfqpoint{5.993684in}{0.602034in}}%
\pgfpathlineto{\pgfqpoint{5.994218in}{0.603590in}}%
\pgfpathlineto{\pgfqpoint{5.994751in}{0.600286in}}%
\pgfpathlineto{\pgfqpoint{5.995285in}{0.601205in}}%
\pgfpathlineto{\pgfqpoint{5.996886in}{0.602891in}}%
\pgfpathlineto{\pgfqpoint{5.997953in}{0.600980in}}%
\pgfpathlineto{\pgfqpoint{5.999555in}{0.602328in}}%
\pgfpathlineto{\pgfqpoint{6.000088in}{0.601252in}}%
\pgfpathlineto{\pgfqpoint{6.001156in}{0.604311in}}%
\pgfpathlineto{\pgfqpoint{6.001689in}{0.603775in}}%
\pgfpathlineto{\pgfqpoint{6.002223in}{0.603647in}}%
\pgfpathlineto{\pgfqpoint{6.002757in}{0.601465in}}%
\pgfpathlineto{\pgfqpoint{6.003290in}{0.602462in}}%
\pgfpathlineto{\pgfqpoint{6.003824in}{0.601982in}}%
\pgfpathlineto{\pgfqpoint{6.004358in}{0.603351in}}%
\pgfpathlineto{\pgfqpoint{6.005425in}{0.600610in}}%
\pgfpathlineto{\pgfqpoint{6.008627in}{0.602689in}}%
\pgfpathlineto{\pgfqpoint{6.009695in}{0.602600in}}%
\pgfpathlineto{\pgfqpoint{6.010762in}{0.602410in}}%
\pgfpathlineto{\pgfqpoint{6.011296in}{0.601295in}}%
\pgfpathlineto{\pgfqpoint{6.011830in}{0.601695in}}%
\pgfpathlineto{\pgfqpoint{6.013964in}{0.603303in}}%
\pgfpathlineto{\pgfqpoint{6.014498in}{0.601115in}}%
\pgfpathlineto{\pgfqpoint{6.015032in}{0.602940in}}%
\pgfpathlineto{\pgfqpoint{6.017167in}{0.601282in}}%
\pgfpathlineto{\pgfqpoint{6.018234in}{0.603757in}}%
\pgfpathlineto{\pgfqpoint{6.018768in}{0.602671in}}%
\pgfpathlineto{\pgfqpoint{6.019835in}{0.601255in}}%
\pgfpathlineto{\pgfqpoint{6.020903in}{0.603197in}}%
\pgfpathlineto{\pgfqpoint{6.021436in}{0.600719in}}%
\pgfpathlineto{\pgfqpoint{6.021970in}{0.600891in}}%
\pgfpathlineto{\pgfqpoint{6.023037in}{0.603518in}}%
\pgfpathlineto{\pgfqpoint{6.023571in}{0.600333in}}%
\pgfpathlineto{\pgfqpoint{6.024105in}{0.603619in}}%
\pgfpathlineto{\pgfqpoint{6.025706in}{0.600642in}}%
\pgfpathlineto{\pgfqpoint{6.026773in}{0.603892in}}%
\pgfpathlineto{\pgfqpoint{6.027841in}{0.600992in}}%
\pgfpathlineto{\pgfqpoint{6.028374in}{0.601392in}}%
\pgfpathlineto{\pgfqpoint{6.029442in}{0.603627in}}%
\pgfpathlineto{\pgfqpoint{6.029975in}{0.600599in}}%
\pgfpathlineto{\pgfqpoint{6.030509in}{0.602699in}}%
\pgfpathlineto{\pgfqpoint{6.031043in}{0.601200in}}%
\pgfpathlineto{\pgfqpoint{6.031577in}{0.604658in}}%
\pgfpathlineto{\pgfqpoint{6.032110in}{0.602049in}}%
\pgfpathlineto{\pgfqpoint{6.032644in}{0.601816in}}%
\pgfpathlineto{\pgfqpoint{6.033178in}{0.600299in}}%
\pgfpathlineto{\pgfqpoint{6.033711in}{0.600671in}}%
\pgfpathlineto{\pgfqpoint{6.034245in}{0.605329in}}%
\pgfpathlineto{\pgfqpoint{6.034779in}{0.603742in}}%
\pgfpathlineto{\pgfqpoint{6.036380in}{0.600684in}}%
\pgfpathlineto{\pgfqpoint{6.039048in}{0.601223in}}%
\pgfpathlineto{\pgfqpoint{6.039582in}{0.600440in}}%
\pgfpathlineto{\pgfqpoint{6.040116in}{0.603580in}}%
\pgfpathlineto{\pgfqpoint{6.040649in}{0.602467in}}%
\pgfpathlineto{\pgfqpoint{6.042251in}{0.600537in}}%
\pgfpathlineto{\pgfqpoint{6.043852in}{0.603143in}}%
\pgfpathlineto{\pgfqpoint{6.045453in}{0.601306in}}%
\pgfpathlineto{\pgfqpoint{6.045987in}{0.601190in}}%
\pgfpathlineto{\pgfqpoint{6.046520in}{0.602259in}}%
\pgfpathlineto{\pgfqpoint{6.047054in}{0.601249in}}%
\pgfpathlineto{\pgfqpoint{6.047588in}{0.601416in}}%
\pgfpathlineto{\pgfqpoint{6.048121in}{0.603939in}}%
\pgfpathlineto{\pgfqpoint{6.048655in}{0.602063in}}%
\pgfpathlineto{\pgfqpoint{6.050256in}{0.603166in}}%
\pgfpathlineto{\pgfqpoint{6.051324in}{0.600277in}}%
\pgfpathlineto{\pgfqpoint{6.051857in}{0.602269in}}%
\pgfpathlineto{\pgfqpoint{6.052391in}{0.601377in}}%
\pgfpathlineto{\pgfqpoint{6.052925in}{0.605655in}}%
\pgfpathlineto{\pgfqpoint{6.053458in}{0.603659in}}%
\pgfpathlineto{\pgfqpoint{6.053992in}{0.601123in}}%
\pgfpathlineto{\pgfqpoint{6.054526in}{0.602669in}}%
\pgfpathlineto{\pgfqpoint{6.055593in}{0.601206in}}%
\pgfpathlineto{\pgfqpoint{6.057194in}{0.602185in}}%
\pgfpathlineto{\pgfqpoint{6.058262in}{0.601045in}}%
\pgfpathlineto{\pgfqpoint{6.058795in}{0.602767in}}%
\pgfpathlineto{\pgfqpoint{6.059863in}{0.600161in}}%
\pgfpathlineto{\pgfqpoint{6.061464in}{0.603661in}}%
\pgfpathlineto{\pgfqpoint{6.061998in}{0.600905in}}%
\pgfpathlineto{\pgfqpoint{6.062531in}{0.602256in}}%
\pgfpathlineto{\pgfqpoint{6.064132in}{0.604069in}}%
\pgfpathlineto{\pgfqpoint{6.064666in}{0.602326in}}%
\pgfpathlineto{\pgfqpoint{6.065200in}{0.604746in}}%
\pgfpathlineto{\pgfqpoint{6.065733in}{0.604189in}}%
\pgfpathlineto{\pgfqpoint{6.066267in}{0.601142in}}%
\pgfpathlineto{\pgfqpoint{6.066801in}{0.601666in}}%
\pgfpathlineto{\pgfqpoint{6.067335in}{0.604790in}}%
\pgfpathlineto{\pgfqpoint{6.067868in}{0.603341in}}%
\pgfpathlineto{\pgfqpoint{6.068402in}{0.603702in}}%
\pgfpathlineto{\pgfqpoint{6.069469in}{0.600461in}}%
\pgfpathlineto{\pgfqpoint{6.070003in}{0.601349in}}%
\pgfpathlineto{\pgfqpoint{6.071604in}{0.603242in}}%
\pgfpathlineto{\pgfqpoint{6.072672in}{0.601121in}}%
\pgfpathlineto{\pgfqpoint{6.073205in}{0.602549in}}%
\pgfpathlineto{\pgfqpoint{6.073739in}{0.601637in}}%
\pgfpathlineto{\pgfqpoint{6.077475in}{0.602906in}}%
\pgfpathlineto{\pgfqpoint{6.078009in}{0.600592in}}%
\pgfpathlineto{\pgfqpoint{6.078542in}{0.601473in}}%
\pgfpathlineto{\pgfqpoint{6.079076in}{0.601493in}}%
\pgfpathlineto{\pgfqpoint{6.080677in}{0.600589in}}%
\pgfpathlineto{\pgfqpoint{6.081744in}{0.603033in}}%
\pgfpathlineto{\pgfqpoint{6.082278in}{0.601335in}}%
\pgfpathlineto{\pgfqpoint{6.082812in}{0.603040in}}%
\pgfpathlineto{\pgfqpoint{6.083346in}{0.603401in}}%
\pgfpathlineto{\pgfqpoint{6.084947in}{0.601411in}}%
\pgfpathlineto{\pgfqpoint{6.086014in}{0.605244in}}%
\pgfpathlineto{\pgfqpoint{6.087081in}{0.601004in}}%
\pgfpathlineto{\pgfqpoint{6.087615in}{0.603554in}}%
\pgfpathlineto{\pgfqpoint{6.088149in}{0.601898in}}%
\pgfpathlineto{\pgfqpoint{6.088683in}{0.601492in}}%
\pgfpathlineto{\pgfqpoint{6.090284in}{0.602358in}}%
\pgfpathlineto{\pgfqpoint{6.090817in}{0.602924in}}%
\pgfpathlineto{\pgfqpoint{6.091885in}{0.600304in}}%
\pgfpathlineto{\pgfqpoint{6.092418in}{0.601297in}}%
\pgfpathlineto{\pgfqpoint{6.092952in}{0.600766in}}%
\pgfpathlineto{\pgfqpoint{6.093486in}{0.603629in}}%
\pgfpathlineto{\pgfqpoint{6.094020in}{0.600838in}}%
\pgfpathlineto{\pgfqpoint{6.094553in}{0.600274in}}%
\pgfpathlineto{\pgfqpoint{6.095621in}{0.604623in}}%
\pgfpathlineto{\pgfqpoint{6.096154in}{0.600802in}}%
\pgfpathlineto{\pgfqpoint{6.096688in}{0.601635in}}%
\pgfpathlineto{\pgfqpoint{6.098289in}{0.602631in}}%
\pgfpathlineto{\pgfqpoint{6.098823in}{0.605144in}}%
\pgfpathlineto{\pgfqpoint{6.099357in}{0.602373in}}%
\pgfpathlineto{\pgfqpoint{6.099890in}{0.602629in}}%
\pgfpathlineto{\pgfqpoint{6.100424in}{0.601594in}}%
\pgfpathlineto{\pgfqpoint{6.100958in}{0.604630in}}%
\pgfpathlineto{\pgfqpoint{6.101491in}{0.601270in}}%
\pgfpathlineto{\pgfqpoint{6.102025in}{0.601981in}}%
\pgfpathlineto{\pgfqpoint{6.102559in}{0.601072in}}%
\pgfpathlineto{\pgfqpoint{6.103092in}{0.600588in}}%
\pgfpathlineto{\pgfqpoint{6.103626in}{0.604058in}}%
\pgfpathlineto{\pgfqpoint{6.104160in}{0.602343in}}%
\pgfpathlineto{\pgfqpoint{6.105761in}{0.606088in}}%
\pgfpathlineto{\pgfqpoint{6.106828in}{0.601618in}}%
\pgfpathlineto{\pgfqpoint{6.107362in}{0.605600in}}%
\pgfpathlineto{\pgfqpoint{6.107896in}{0.602890in}}%
\pgfpathlineto{\pgfqpoint{6.108963in}{0.601071in}}%
\pgfpathlineto{\pgfqpoint{6.109497in}{0.605469in}}%
\pgfpathlineto{\pgfqpoint{6.110031in}{0.603008in}}%
\pgfpathlineto{\pgfqpoint{6.110564in}{0.600568in}}%
\pgfpathlineto{\pgfqpoint{6.111098in}{0.601915in}}%
\pgfpathlineto{\pgfqpoint{6.112165in}{0.603894in}}%
\pgfpathlineto{\pgfqpoint{6.112699in}{0.601030in}}%
\pgfpathlineto{\pgfqpoint{6.113233in}{0.602011in}}%
\pgfpathlineto{\pgfqpoint{6.115901in}{0.602654in}}%
\pgfpathlineto{\pgfqpoint{6.116435in}{0.600667in}}%
\pgfpathlineto{\pgfqpoint{6.116969in}{0.602054in}}%
\pgfpathlineto{\pgfqpoint{6.117502in}{0.602423in}}%
\pgfpathlineto{\pgfqpoint{6.118036in}{0.601345in}}%
\pgfpathlineto{\pgfqpoint{6.118570in}{0.603037in}}%
\pgfpathlineto{\pgfqpoint{6.119104in}{0.602683in}}%
\pgfpathlineto{\pgfqpoint{6.119637in}{0.600276in}}%
\pgfpathlineto{\pgfqpoint{6.120171in}{0.601474in}}%
\pgfpathlineto{\pgfqpoint{6.121772in}{0.603572in}}%
\pgfpathlineto{\pgfqpoint{6.122306in}{0.603184in}}%
\pgfpathlineto{\pgfqpoint{6.124441in}{0.600617in}}%
\pgfpathlineto{\pgfqpoint{6.125508in}{0.604861in}}%
\pgfpathlineto{\pgfqpoint{6.126042in}{0.602404in}}%
\pgfpathlineto{\pgfqpoint{6.126575in}{0.600146in}}%
\pgfpathlineto{\pgfqpoint{6.127109in}{0.601086in}}%
\pgfpathlineto{\pgfqpoint{6.127643in}{0.600999in}}%
\pgfpathlineto{\pgfqpoint{6.128710in}{0.602663in}}%
\pgfpathlineto{\pgfqpoint{6.129244in}{0.601166in}}%
\pgfpathlineto{\pgfqpoint{6.129778in}{0.602493in}}%
\pgfpathlineto{\pgfqpoint{6.131379in}{0.601266in}}%
\pgfpathlineto{\pgfqpoint{6.131912in}{0.602715in}}%
\pgfpathlineto{\pgfqpoint{6.132446in}{0.601279in}}%
\pgfpathlineto{\pgfqpoint{6.132980in}{0.602343in}}%
\pgfpathlineto{\pgfqpoint{6.133513in}{0.602015in}}%
\pgfpathlineto{\pgfqpoint{6.135115in}{0.601755in}}%
\pgfpathlineto{\pgfqpoint{6.135648in}{0.601866in}}%
\pgfpathlineto{\pgfqpoint{6.136182in}{0.603138in}}%
\pgfpathlineto{\pgfqpoint{6.136716in}{0.602050in}}%
\pgfpathlineto{\pgfqpoint{6.137783in}{0.603411in}}%
\pgfpathlineto{\pgfqpoint{6.138317in}{0.600731in}}%
\pgfpathlineto{\pgfqpoint{6.138850in}{0.602479in}}%
\pgfpathlineto{\pgfqpoint{6.140452in}{0.601017in}}%
\pgfpathlineto{\pgfqpoint{6.141519in}{0.602041in}}%
\pgfpathlineto{\pgfqpoint{6.142053in}{0.600438in}}%
\pgfpathlineto{\pgfqpoint{6.142586in}{0.602942in}}%
\pgfpathlineto{\pgfqpoint{6.143120in}{0.601102in}}%
\pgfpathlineto{\pgfqpoint{6.143654in}{0.600914in}}%
\pgfpathlineto{\pgfqpoint{6.144187in}{0.603147in}}%
\pgfpathlineto{\pgfqpoint{6.144721in}{0.600944in}}%
\pgfpathlineto{\pgfqpoint{6.147390in}{0.602855in}}%
\pgfpathlineto{\pgfqpoint{6.148991in}{0.601010in}}%
\pgfpathlineto{\pgfqpoint{6.150058in}{0.605341in}}%
\pgfpathlineto{\pgfqpoint{6.151659in}{0.601270in}}%
\pgfpathlineto{\pgfqpoint{6.152193in}{0.601159in}}%
\pgfpathlineto{\pgfqpoint{6.152727in}{0.602731in}}%
\pgfpathlineto{\pgfqpoint{6.153260in}{0.602005in}}%
\pgfpathlineto{\pgfqpoint{6.154328in}{0.601412in}}%
\pgfpathlineto{\pgfqpoint{6.155929in}{0.602354in}}%
\pgfpathlineto{\pgfqpoint{6.156222in}{0.601723in}}%
\pgfpathmoveto{\pgfqpoint{6.156222in}{0.599969in}}%
\pgfpathlineto{\pgfqpoint{0.924300in}{0.599971in}}%
\pgfpathmoveto{\pgfqpoint{0.924300in}{0.601723in}}%
\pgfpathlineto{\pgfqpoint{0.925661in}{0.602279in}}%
\pgfpathlineto{\pgfqpoint{0.926728in}{0.601576in}}%
\pgfpathlineto{\pgfqpoint{0.927796in}{0.602731in}}%
\pgfpathlineto{\pgfqpoint{0.928863in}{0.601270in}}%
\pgfpathlineto{\pgfqpoint{0.930464in}{0.605341in}}%
\pgfpathlineto{\pgfqpoint{0.930998in}{0.603321in}}%
\pgfpathlineto{\pgfqpoint{0.931532in}{0.601010in}}%
\pgfpathlineto{\pgfqpoint{0.932065in}{0.601346in}}%
\pgfpathlineto{\pgfqpoint{0.934200in}{0.602239in}}%
\pgfpathlineto{\pgfqpoint{0.935801in}{0.600944in}}%
\pgfpathlineto{\pgfqpoint{0.936335in}{0.603147in}}%
\pgfpathlineto{\pgfqpoint{0.936869in}{0.600914in}}%
\pgfpathlineto{\pgfqpoint{0.937402in}{0.601102in}}%
\pgfpathlineto{\pgfqpoint{0.937936in}{0.602942in}}%
\pgfpathlineto{\pgfqpoint{0.938470in}{0.600438in}}%
\pgfpathlineto{\pgfqpoint{0.939004in}{0.602041in}}%
\pgfpathlineto{\pgfqpoint{0.941138in}{0.601669in}}%
\pgfpathlineto{\pgfqpoint{0.941672in}{0.602479in}}%
\pgfpathlineto{\pgfqpoint{0.942206in}{0.600731in}}%
\pgfpathlineto{\pgfqpoint{0.942739in}{0.603411in}}%
\pgfpathlineto{\pgfqpoint{0.943273in}{0.602698in}}%
\pgfpathlineto{\pgfqpoint{0.945408in}{0.601755in}}%
\pgfpathlineto{\pgfqpoint{0.947543in}{0.602343in}}%
\pgfpathlineto{\pgfqpoint{0.948077in}{0.601279in}}%
\pgfpathlineto{\pgfqpoint{0.949678in}{0.602673in}}%
\pgfpathlineto{\pgfqpoint{0.951279in}{0.601166in}}%
\pgfpathlineto{\pgfqpoint{0.952346in}{0.602586in}}%
\pgfpathlineto{\pgfqpoint{0.953947in}{0.600146in}}%
\pgfpathlineto{\pgfqpoint{0.955015in}{0.604861in}}%
\pgfpathlineto{\pgfqpoint{0.956082in}{0.600617in}}%
\pgfpathlineto{\pgfqpoint{0.956616in}{0.601838in}}%
\pgfpathlineto{\pgfqpoint{0.958217in}{0.603184in}}%
\pgfpathlineto{\pgfqpoint{0.959284in}{0.603254in}}%
\pgfpathlineto{\pgfqpoint{0.960885in}{0.600276in}}%
\pgfpathlineto{\pgfqpoint{0.961953in}{0.603037in}}%
\pgfpathlineto{\pgfqpoint{0.962486in}{0.601345in}}%
\pgfpathlineto{\pgfqpoint{0.963020in}{0.602423in}}%
\pgfpathlineto{\pgfqpoint{0.964088in}{0.600667in}}%
\pgfpathlineto{\pgfqpoint{0.964621in}{0.602654in}}%
\pgfpathlineto{\pgfqpoint{0.967290in}{0.602011in}}%
\pgfpathlineto{\pgfqpoint{0.967823in}{0.601030in}}%
\pgfpathlineto{\pgfqpoint{0.968357in}{0.603894in}}%
\pgfpathlineto{\pgfqpoint{0.968891in}{0.602515in}}%
\pgfpathlineto{\pgfqpoint{0.969958in}{0.600568in}}%
\pgfpathlineto{\pgfqpoint{0.971026in}{0.605469in}}%
\pgfpathlineto{\pgfqpoint{0.971559in}{0.601071in}}%
\pgfpathlineto{\pgfqpoint{0.972093in}{0.602465in}}%
\pgfpathlineto{\pgfqpoint{0.972627in}{0.602890in}}%
\pgfpathlineto{\pgfqpoint{0.973160in}{0.605600in}}%
\pgfpathlineto{\pgfqpoint{0.973694in}{0.601618in}}%
\pgfpathlineto{\pgfqpoint{0.974228in}{0.603161in}}%
\pgfpathlineto{\pgfqpoint{0.974762in}{0.606088in}}%
\pgfpathlineto{\pgfqpoint{0.975295in}{0.603674in}}%
\pgfpathlineto{\pgfqpoint{0.976363in}{0.602343in}}%
\pgfpathlineto{\pgfqpoint{0.976896in}{0.604058in}}%
\pgfpathlineto{\pgfqpoint{0.977430in}{0.600588in}}%
\pgfpathlineto{\pgfqpoint{0.977964in}{0.601072in}}%
\pgfpathlineto{\pgfqpoint{0.979565in}{0.604630in}}%
\pgfpathlineto{\pgfqpoint{0.980099in}{0.601594in}}%
\pgfpathlineto{\pgfqpoint{0.980632in}{0.602629in}}%
\pgfpathlineto{\pgfqpoint{0.981166in}{0.602373in}}%
\pgfpathlineto{\pgfqpoint{0.981700in}{0.605144in}}%
\pgfpathlineto{\pgfqpoint{0.982233in}{0.602631in}}%
\pgfpathlineto{\pgfqpoint{0.983834in}{0.601635in}}%
\pgfpathlineto{\pgfqpoint{0.984368in}{0.600802in}}%
\pgfpathlineto{\pgfqpoint{0.984902in}{0.604623in}}%
\pgfpathlineto{\pgfqpoint{0.985436in}{0.601929in}}%
\pgfpathlineto{\pgfqpoint{0.985969in}{0.600274in}}%
\pgfpathlineto{\pgfqpoint{0.986503in}{0.600838in}}%
\pgfpathlineto{\pgfqpoint{0.987037in}{0.603629in}}%
\pgfpathlineto{\pgfqpoint{0.987570in}{0.600766in}}%
\pgfpathlineto{\pgfqpoint{0.989705in}{0.602924in}}%
\pgfpathlineto{\pgfqpoint{0.990239in}{0.602358in}}%
\pgfpathlineto{\pgfqpoint{0.991840in}{0.601492in}}%
\pgfpathlineto{\pgfqpoint{0.992907in}{0.603554in}}%
\pgfpathlineto{\pgfqpoint{0.993441in}{0.601004in}}%
\pgfpathlineto{\pgfqpoint{0.993975in}{0.602075in}}%
\pgfpathlineto{\pgfqpoint{0.994508in}{0.605244in}}%
\pgfpathlineto{\pgfqpoint{0.995042in}{0.603171in}}%
\pgfpathlineto{\pgfqpoint{0.996643in}{0.601168in}}%
\pgfpathlineto{\pgfqpoint{0.997177in}{0.603401in}}%
\pgfpathlineto{\pgfqpoint{0.997711in}{0.603040in}}%
\pgfpathlineto{\pgfqpoint{0.998244in}{0.601335in}}%
\pgfpathlineto{\pgfqpoint{0.998778in}{0.603033in}}%
\pgfpathlineto{\pgfqpoint{0.999312in}{0.602728in}}%
\pgfpathlineto{\pgfqpoint{1.000913in}{0.600324in}}%
\pgfpathlineto{\pgfqpoint{1.001980in}{0.601473in}}%
\pgfpathlineto{\pgfqpoint{1.002514in}{0.600592in}}%
\pgfpathlineto{\pgfqpoint{1.004115in}{0.602771in}}%
\pgfpathlineto{\pgfqpoint{1.006784in}{0.601637in}}%
\pgfpathlineto{\pgfqpoint{1.007317in}{0.602549in}}%
\pgfpathlineto{\pgfqpoint{1.007851in}{0.601121in}}%
\pgfpathlineto{\pgfqpoint{1.008385in}{0.602015in}}%
\pgfpathlineto{\pgfqpoint{1.008918in}{0.603242in}}%
\pgfpathlineto{\pgfqpoint{1.009452in}{0.602859in}}%
\pgfpathlineto{\pgfqpoint{1.009986in}{0.602834in}}%
\pgfpathlineto{\pgfqpoint{1.011053in}{0.600461in}}%
\pgfpathlineto{\pgfqpoint{1.013188in}{0.604790in}}%
\pgfpathlineto{\pgfqpoint{1.014255in}{0.601142in}}%
\pgfpathlineto{\pgfqpoint{1.015323in}{0.604746in}}%
\pgfpathlineto{\pgfqpoint{1.015857in}{0.602326in}}%
\pgfpathlineto{\pgfqpoint{1.016390in}{0.604069in}}%
\pgfpathlineto{\pgfqpoint{1.017991in}{0.602256in}}%
\pgfpathlineto{\pgfqpoint{1.018525in}{0.600905in}}%
\pgfpathlineto{\pgfqpoint{1.019059in}{0.603661in}}%
\pgfpathlineto{\pgfqpoint{1.019592in}{0.602866in}}%
\pgfpathlineto{\pgfqpoint{1.021194in}{0.600235in}}%
\pgfpathlineto{\pgfqpoint{1.021727in}{0.602767in}}%
\pgfpathlineto{\pgfqpoint{1.022261in}{0.601045in}}%
\pgfpathlineto{\pgfqpoint{1.025997in}{0.602669in}}%
\pgfpathlineto{\pgfqpoint{1.026531in}{0.601123in}}%
\pgfpathlineto{\pgfqpoint{1.027598in}{0.605655in}}%
\pgfpathlineto{\pgfqpoint{1.029199in}{0.600277in}}%
\pgfpathlineto{\pgfqpoint{1.030266in}{0.603166in}}%
\pgfpathlineto{\pgfqpoint{1.030800in}{0.602643in}}%
\pgfpathlineto{\pgfqpoint{1.032401in}{0.603939in}}%
\pgfpathlineto{\pgfqpoint{1.033469in}{0.601249in}}%
\pgfpathlineto{\pgfqpoint{1.034002in}{0.602259in}}%
\pgfpathlineto{\pgfqpoint{1.034536in}{0.601190in}}%
\pgfpathlineto{\pgfqpoint{1.036137in}{0.601796in}}%
\pgfpathlineto{\pgfqpoint{1.037738in}{0.603254in}}%
\pgfpathlineto{\pgfqpoint{1.038806in}{0.600361in}}%
\pgfpathlineto{\pgfqpoint{1.040407in}{0.603580in}}%
\pgfpathlineto{\pgfqpoint{1.040940in}{0.600440in}}%
\pgfpathlineto{\pgfqpoint{1.041474in}{0.601223in}}%
\pgfpathlineto{\pgfqpoint{1.042008in}{0.601126in}}%
\pgfpathlineto{\pgfqpoint{1.042542in}{0.602276in}}%
\pgfpathlineto{\pgfqpoint{1.043075in}{0.601660in}}%
\pgfpathlineto{\pgfqpoint{1.044143in}{0.600684in}}%
\pgfpathlineto{\pgfqpoint{1.046277in}{0.605329in}}%
\pgfpathlineto{\pgfqpoint{1.047345in}{0.600299in}}%
\pgfpathlineto{\pgfqpoint{1.048946in}{0.604658in}}%
\pgfpathlineto{\pgfqpoint{1.050547in}{0.600599in}}%
\pgfpathlineto{\pgfqpoint{1.051081in}{0.603627in}}%
\pgfpathlineto{\pgfqpoint{1.051614in}{0.603119in}}%
\pgfpathlineto{\pgfqpoint{1.053216in}{0.601071in}}%
\pgfpathlineto{\pgfqpoint{1.053749in}{0.603892in}}%
\pgfpathlineto{\pgfqpoint{1.054283in}{0.602062in}}%
\pgfpathlineto{\pgfqpoint{1.054817in}{0.600642in}}%
\pgfpathlineto{\pgfqpoint{1.055350in}{0.601449in}}%
\pgfpathlineto{\pgfqpoint{1.056418in}{0.603619in}}%
\pgfpathlineto{\pgfqpoint{1.056951in}{0.600333in}}%
\pgfpathlineto{\pgfqpoint{1.057485in}{0.603518in}}%
\pgfpathlineto{\pgfqpoint{1.059086in}{0.600719in}}%
\pgfpathlineto{\pgfqpoint{1.059620in}{0.603197in}}%
\pgfpathlineto{\pgfqpoint{1.060154in}{0.602303in}}%
\pgfpathlineto{\pgfqpoint{1.061221in}{0.601184in}}%
\pgfpathlineto{\pgfqpoint{1.062288in}{0.603757in}}%
\pgfpathlineto{\pgfqpoint{1.063356in}{0.601282in}}%
\pgfpathlineto{\pgfqpoint{1.063890in}{0.601656in}}%
\pgfpathlineto{\pgfqpoint{1.065491in}{0.602940in}}%
\pgfpathlineto{\pgfqpoint{1.066024in}{0.601115in}}%
\pgfpathlineto{\pgfqpoint{1.067092in}{0.603315in}}%
\pgfpathlineto{\pgfqpoint{1.069227in}{0.601295in}}%
\pgfpathlineto{\pgfqpoint{1.070828in}{0.602600in}}%
\pgfpathlineto{\pgfqpoint{1.072429in}{0.602323in}}%
\pgfpathlineto{\pgfqpoint{1.074030in}{0.602359in}}%
\pgfpathlineto{\pgfqpoint{1.075631in}{0.600544in}}%
\pgfpathlineto{\pgfqpoint{1.076165in}{0.603351in}}%
\pgfpathlineto{\pgfqpoint{1.076698in}{0.601982in}}%
\pgfpathlineto{\pgfqpoint{1.079367in}{0.604311in}}%
\pgfpathlineto{\pgfqpoint{1.080434in}{0.601252in}}%
\pgfpathlineto{\pgfqpoint{1.080968in}{0.602328in}}%
\pgfpathlineto{\pgfqpoint{1.082569in}{0.600980in}}%
\pgfpathlineto{\pgfqpoint{1.084704in}{0.602796in}}%
\pgfpathlineto{\pgfqpoint{1.085771in}{0.600286in}}%
\pgfpathlineto{\pgfqpoint{1.086305in}{0.603590in}}%
\pgfpathlineto{\pgfqpoint{1.086839in}{0.602034in}}%
\pgfpathlineto{\pgfqpoint{1.087906in}{0.603740in}}%
\pgfpathlineto{\pgfqpoint{1.090041in}{0.601367in}}%
\pgfpathlineto{\pgfqpoint{1.091108in}{0.602143in}}%
\pgfpathlineto{\pgfqpoint{1.091642in}{0.600428in}}%
\pgfpathlineto{\pgfqpoint{1.092709in}{0.606484in}}%
\pgfpathlineto{\pgfqpoint{1.093243in}{0.601119in}}%
\pgfpathlineto{\pgfqpoint{1.093777in}{0.601390in}}%
\pgfpathlineto{\pgfqpoint{1.094844in}{0.603794in}}%
\pgfpathlineto{\pgfqpoint{1.095378in}{0.602148in}}%
\pgfpathlineto{\pgfqpoint{1.095912in}{0.600840in}}%
\pgfpathlineto{\pgfqpoint{1.096445in}{0.602055in}}%
\pgfpathlineto{\pgfqpoint{1.096979in}{0.601861in}}%
\pgfpathlineto{\pgfqpoint{1.098046in}{0.604268in}}%
\pgfpathlineto{\pgfqpoint{1.098580in}{0.601526in}}%
\pgfpathlineto{\pgfqpoint{1.099114in}{0.602939in}}%
\pgfpathlineto{\pgfqpoint{1.101782in}{0.600677in}}%
\pgfpathlineto{\pgfqpoint{1.102316in}{0.600890in}}%
\pgfpathlineto{\pgfqpoint{1.102850in}{0.603281in}}%
\pgfpathlineto{\pgfqpoint{1.103383in}{0.601773in}}%
\pgfpathlineto{\pgfqpoint{1.104985in}{0.601081in}}%
\pgfpathlineto{\pgfqpoint{1.106586in}{0.603682in}}%
\pgfpathlineto{\pgfqpoint{1.107653in}{0.601402in}}%
\pgfpathlineto{\pgfqpoint{1.108187in}{0.602206in}}%
\pgfpathlineto{\pgfqpoint{1.108720in}{0.602329in}}%
\pgfpathlineto{\pgfqpoint{1.109254in}{0.604753in}}%
\pgfpathlineto{\pgfqpoint{1.109788in}{0.602278in}}%
\pgfpathlineto{\pgfqpoint{1.111389in}{0.603890in}}%
\pgfpathlineto{\pgfqpoint{1.112456in}{0.601197in}}%
\pgfpathlineto{\pgfqpoint{1.112990in}{0.601651in}}%
\pgfpathlineto{\pgfqpoint{1.113524in}{0.603533in}}%
\pgfpathlineto{\pgfqpoint{1.114057in}{0.603263in}}%
\pgfpathlineto{\pgfqpoint{1.116192in}{0.601946in}}%
\pgfpathlineto{\pgfqpoint{1.116726in}{0.603207in}}%
\pgfpathlineto{\pgfqpoint{1.117260in}{0.601149in}}%
\pgfpathlineto{\pgfqpoint{1.117793in}{0.601850in}}%
\pgfpathlineto{\pgfqpoint{1.119394in}{0.604757in}}%
\pgfpathlineto{\pgfqpoint{1.119928in}{0.601522in}}%
\pgfpathlineto{\pgfqpoint{1.120462in}{0.603071in}}%
\pgfpathlineto{\pgfqpoint{1.120996in}{0.602686in}}%
\pgfpathlineto{\pgfqpoint{1.122597in}{0.600068in}}%
\pgfpathlineto{\pgfqpoint{1.123664in}{0.607192in}}%
\pgfpathlineto{\pgfqpoint{1.124198in}{0.601418in}}%
\pgfpathlineto{\pgfqpoint{1.124731in}{0.602744in}}%
\pgfpathlineto{\pgfqpoint{1.125265in}{0.603830in}}%
\pgfpathlineto{\pgfqpoint{1.125799in}{0.603254in}}%
\pgfpathlineto{\pgfqpoint{1.126866in}{0.601115in}}%
\pgfpathlineto{\pgfqpoint{1.127934in}{0.602519in}}%
\pgfpathlineto{\pgfqpoint{1.128467in}{0.600747in}}%
\pgfpathlineto{\pgfqpoint{1.129001in}{0.601746in}}%
\pgfpathlineto{\pgfqpoint{1.130602in}{0.603481in}}%
\pgfpathlineto{\pgfqpoint{1.131136in}{0.600151in}}%
\pgfpathlineto{\pgfqpoint{1.131670in}{0.604383in}}%
\pgfpathlineto{\pgfqpoint{1.132203in}{0.601156in}}%
\pgfpathlineto{\pgfqpoint{1.132737in}{0.602569in}}%
\pgfpathlineto{\pgfqpoint{1.133271in}{0.601821in}}%
\pgfpathlineto{\pgfqpoint{1.133804in}{0.602014in}}%
\pgfpathlineto{\pgfqpoint{1.134338in}{0.607758in}}%
\pgfpathlineto{\pgfqpoint{1.134872in}{0.602299in}}%
\pgfpathlineto{\pgfqpoint{1.135405in}{0.601847in}}%
\pgfpathlineto{\pgfqpoint{1.135939in}{0.602546in}}%
\pgfpathlineto{\pgfqpoint{1.136473in}{0.603656in}}%
\pgfpathlineto{\pgfqpoint{1.138074in}{0.601879in}}%
\pgfpathlineto{\pgfqpoint{1.138608in}{0.600937in}}%
\pgfpathlineto{\pgfqpoint{1.140209in}{0.605509in}}%
\pgfpathlineto{\pgfqpoint{1.140742in}{0.602645in}}%
\pgfpathlineto{\pgfqpoint{1.141276in}{0.606575in}}%
\pgfpathlineto{\pgfqpoint{1.141810in}{0.601779in}}%
\pgfpathlineto{\pgfqpoint{1.142344in}{0.604202in}}%
\pgfpathlineto{\pgfqpoint{1.142877in}{0.605185in}}%
\pgfpathlineto{\pgfqpoint{1.143411in}{0.603163in}}%
\pgfpathlineto{\pgfqpoint{1.143945in}{0.608587in}}%
\pgfpathlineto{\pgfqpoint{1.144478in}{0.604076in}}%
\pgfpathlineto{\pgfqpoint{1.146613in}{0.602588in}}%
\pgfpathlineto{\pgfqpoint{1.147147in}{0.607636in}}%
\pgfpathlineto{\pgfqpoint{1.147681in}{0.603009in}}%
\pgfpathlineto{\pgfqpoint{1.148748in}{0.601106in}}%
\pgfpathlineto{\pgfqpoint{1.149282in}{0.603989in}}%
\pgfpathlineto{\pgfqpoint{1.149815in}{0.601250in}}%
\pgfpathlineto{\pgfqpoint{1.150349in}{0.601609in}}%
\pgfpathlineto{\pgfqpoint{1.151950in}{0.606981in}}%
\pgfpathlineto{\pgfqpoint{1.152484in}{0.602560in}}%
\pgfpathlineto{\pgfqpoint{1.153018in}{0.605540in}}%
\pgfpathlineto{\pgfqpoint{1.154085in}{0.602932in}}%
\pgfpathlineto{\pgfqpoint{1.154619in}{0.603531in}}%
\pgfpathlineto{\pgfqpoint{1.155686in}{0.603000in}}%
\pgfpathlineto{\pgfqpoint{1.156220in}{0.603656in}}%
\pgfpathlineto{\pgfqpoint{1.156754in}{0.600640in}}%
\pgfpathlineto{\pgfqpoint{1.157287in}{0.603662in}}%
\pgfpathlineto{\pgfqpoint{1.157821in}{0.603889in}}%
\pgfpathlineto{\pgfqpoint{1.159422in}{0.600685in}}%
\pgfpathlineto{\pgfqpoint{1.159956in}{0.601820in}}%
\pgfpathlineto{\pgfqpoint{1.160489in}{0.601501in}}%
\pgfpathlineto{\pgfqpoint{1.161023in}{0.600492in}}%
\pgfpathlineto{\pgfqpoint{1.161557in}{0.602329in}}%
\pgfpathlineto{\pgfqpoint{1.162091in}{0.602097in}}%
\pgfpathlineto{\pgfqpoint{1.162624in}{0.600807in}}%
\pgfpathlineto{\pgfqpoint{1.163158in}{0.602037in}}%
\pgfpathlineto{\pgfqpoint{1.163692in}{0.604332in}}%
\pgfpathlineto{\pgfqpoint{1.164225in}{0.600866in}}%
\pgfpathlineto{\pgfqpoint{1.164759in}{0.603076in}}%
\pgfpathlineto{\pgfqpoint{1.166360in}{0.602524in}}%
\pgfpathlineto{\pgfqpoint{1.166894in}{0.605142in}}%
\pgfpathlineto{\pgfqpoint{1.167428in}{0.603319in}}%
\pgfpathlineto{\pgfqpoint{1.167961in}{0.601843in}}%
\pgfpathlineto{\pgfqpoint{1.168495in}{0.602720in}}%
\pgfpathlineto{\pgfqpoint{1.169029in}{0.602782in}}%
\pgfpathlineto{\pgfqpoint{1.170096in}{0.604888in}}%
\pgfpathlineto{\pgfqpoint{1.170630in}{0.604217in}}%
\pgfpathlineto{\pgfqpoint{1.171163in}{0.605319in}}%
\pgfpathlineto{\pgfqpoint{1.171697in}{0.600861in}}%
\pgfpathlineto{\pgfqpoint{1.172231in}{0.605108in}}%
\pgfpathlineto{\pgfqpoint{1.173298in}{0.601341in}}%
\pgfpathlineto{\pgfqpoint{1.173832in}{0.601780in}}%
\pgfpathlineto{\pgfqpoint{1.175433in}{0.605218in}}%
\pgfpathlineto{\pgfqpoint{1.176500in}{0.600891in}}%
\pgfpathlineto{\pgfqpoint{1.177034in}{0.603140in}}%
\pgfpathlineto{\pgfqpoint{1.178102in}{0.602266in}}%
\pgfpathlineto{\pgfqpoint{1.178635in}{0.606151in}}%
\pgfpathlineto{\pgfqpoint{1.179169in}{0.602295in}}%
\pgfpathlineto{\pgfqpoint{1.179703in}{0.604025in}}%
\pgfpathlineto{\pgfqpoint{1.180236in}{0.603163in}}%
\pgfpathlineto{\pgfqpoint{1.180770in}{0.600989in}}%
\pgfpathlineto{\pgfqpoint{1.181304in}{0.602523in}}%
\pgfpathlineto{\pgfqpoint{1.182371in}{0.601637in}}%
\pgfpathlineto{\pgfqpoint{1.183972in}{0.603104in}}%
\pgfpathlineto{\pgfqpoint{1.184506in}{0.602179in}}%
\pgfpathlineto{\pgfqpoint{1.186107in}{0.600758in}}%
\pgfpathlineto{\pgfqpoint{1.186641in}{0.603512in}}%
\pgfpathlineto{\pgfqpoint{1.187174in}{0.601860in}}%
\pgfpathlineto{\pgfqpoint{1.188776in}{0.601413in}}%
\pgfpathlineto{\pgfqpoint{1.189309in}{0.606223in}}%
\pgfpathlineto{\pgfqpoint{1.189843in}{0.602908in}}%
\pgfpathlineto{\pgfqpoint{1.190910in}{0.600130in}}%
\pgfpathlineto{\pgfqpoint{1.193045in}{0.606044in}}%
\pgfpathlineto{\pgfqpoint{1.194113in}{0.601030in}}%
\pgfpathlineto{\pgfqpoint{1.194646in}{0.602063in}}%
\pgfpathlineto{\pgfqpoint{1.195714in}{0.602287in}}%
\pgfpathlineto{\pgfqpoint{1.196247in}{0.605092in}}%
\pgfpathlineto{\pgfqpoint{1.196781in}{0.603442in}}%
\pgfpathlineto{\pgfqpoint{1.197315in}{0.603229in}}%
\pgfpathlineto{\pgfqpoint{1.198916in}{0.606044in}}%
\pgfpathlineto{\pgfqpoint{1.199983in}{0.601200in}}%
\pgfpathlineto{\pgfqpoint{1.200517in}{0.601821in}}%
\pgfpathlineto{\pgfqpoint{1.202118in}{0.604401in}}%
\pgfpathlineto{\pgfqpoint{1.203185in}{0.600749in}}%
\pgfpathlineto{\pgfqpoint{1.204787in}{0.603747in}}%
\pgfpathlineto{\pgfqpoint{1.205320in}{0.601777in}}%
\pgfpathlineto{\pgfqpoint{1.205854in}{0.604033in}}%
\pgfpathlineto{\pgfqpoint{1.209056in}{0.602798in}}%
\pgfpathlineto{\pgfqpoint{1.209590in}{0.603335in}}%
\pgfpathlineto{\pgfqpoint{1.210124in}{0.601005in}}%
\pgfpathlineto{\pgfqpoint{1.210657in}{0.601924in}}%
\pgfpathlineto{\pgfqpoint{1.211191in}{0.602639in}}%
\pgfpathlineto{\pgfqpoint{1.212792in}{0.601531in}}%
\pgfpathlineto{\pgfqpoint{1.213326in}{0.604537in}}%
\pgfpathlineto{\pgfqpoint{1.213860in}{0.602914in}}%
\pgfpathlineto{\pgfqpoint{1.214393in}{0.600807in}}%
\pgfpathlineto{\pgfqpoint{1.214927in}{0.602431in}}%
\pgfpathlineto{\pgfqpoint{1.215994in}{0.601351in}}%
\pgfpathlineto{\pgfqpoint{1.217062in}{0.604707in}}%
\pgfpathlineto{\pgfqpoint{1.217595in}{0.600965in}}%
\pgfpathlineto{\pgfqpoint{1.218129in}{0.603501in}}%
\pgfpathlineto{\pgfqpoint{1.218663in}{0.603354in}}%
\pgfpathlineto{\pgfqpoint{1.219730in}{0.601716in}}%
\pgfpathlineto{\pgfqpoint{1.220264in}{0.603735in}}%
\pgfpathlineto{\pgfqpoint{1.220798in}{0.603420in}}%
\pgfpathlineto{\pgfqpoint{1.221331in}{0.603238in}}%
\pgfpathlineto{\pgfqpoint{1.221865in}{0.601395in}}%
\pgfpathlineto{\pgfqpoint{1.222399in}{0.603638in}}%
\pgfpathlineto{\pgfqpoint{1.222932in}{0.602323in}}%
\pgfpathlineto{\pgfqpoint{1.224534in}{0.602280in}}%
\pgfpathlineto{\pgfqpoint{1.225601in}{0.602090in}}%
\pgfpathlineto{\pgfqpoint{1.226135in}{0.603236in}}%
\pgfpathlineto{\pgfqpoint{1.226668in}{0.600686in}}%
\pgfpathlineto{\pgfqpoint{1.228269in}{0.605189in}}%
\pgfpathlineto{\pgfqpoint{1.229337in}{0.601319in}}%
\pgfpathlineto{\pgfqpoint{1.230938in}{0.604819in}}%
\pgfpathlineto{\pgfqpoint{1.232539in}{0.600835in}}%
\pgfpathlineto{\pgfqpoint{1.233606in}{0.606605in}}%
\pgfpathlineto{\pgfqpoint{1.234140in}{0.603470in}}%
\pgfpathlineto{\pgfqpoint{1.234674in}{0.602129in}}%
\pgfpathlineto{\pgfqpoint{1.236275in}{0.604609in}}%
\pgfpathlineto{\pgfqpoint{1.238410in}{0.600520in}}%
\pgfpathlineto{\pgfqpoint{1.238943in}{0.602355in}}%
\pgfpathlineto{\pgfqpoint{1.239477in}{0.600412in}}%
\pgfpathlineto{\pgfqpoint{1.240545in}{0.602099in}}%
\pgfpathlineto{\pgfqpoint{1.241078in}{0.600225in}}%
\pgfpathlineto{\pgfqpoint{1.241612in}{0.604372in}}%
\pgfpathlineto{\pgfqpoint{1.242146in}{0.601716in}}%
\pgfpathlineto{\pgfqpoint{1.244814in}{0.604810in}}%
\pgfpathlineto{\pgfqpoint{1.245348in}{0.600984in}}%
\pgfpathlineto{\pgfqpoint{1.245882in}{0.602053in}}%
\pgfpathlineto{\pgfqpoint{1.250685in}{0.601265in}}%
\pgfpathlineto{\pgfqpoint{1.251219in}{0.600329in}}%
\pgfpathlineto{\pgfqpoint{1.251752in}{0.605117in}}%
\pgfpathlineto{\pgfqpoint{1.252286in}{0.602782in}}%
\pgfpathlineto{\pgfqpoint{1.253887in}{0.601118in}}%
\pgfpathlineto{\pgfqpoint{1.255488in}{0.606901in}}%
\pgfpathlineto{\pgfqpoint{1.257089in}{0.601287in}}%
\pgfpathlineto{\pgfqpoint{1.257623in}{0.602699in}}%
\pgfpathlineto{\pgfqpoint{1.258157in}{0.601730in}}%
\pgfpathlineto{\pgfqpoint{1.258690in}{0.600915in}}%
\pgfpathlineto{\pgfqpoint{1.259224in}{0.602071in}}%
\pgfpathlineto{\pgfqpoint{1.259758in}{0.603576in}}%
\pgfpathlineto{\pgfqpoint{1.260291in}{0.602020in}}%
\pgfpathlineto{\pgfqpoint{1.261893in}{0.602957in}}%
\pgfpathlineto{\pgfqpoint{1.262426in}{0.603037in}}%
\pgfpathlineto{\pgfqpoint{1.262960in}{0.604632in}}%
\pgfpathlineto{\pgfqpoint{1.264027in}{0.601818in}}%
\pgfpathlineto{\pgfqpoint{1.265628in}{0.604104in}}%
\pgfpathlineto{\pgfqpoint{1.266162in}{0.602511in}}%
\pgfpathlineto{\pgfqpoint{1.266696in}{0.603844in}}%
\pgfpathlineto{\pgfqpoint{1.268297in}{0.603317in}}%
\pgfpathlineto{\pgfqpoint{1.268831in}{0.601001in}}%
\pgfpathlineto{\pgfqpoint{1.269364in}{0.601942in}}%
\pgfpathlineto{\pgfqpoint{1.269898in}{0.602220in}}%
\pgfpathlineto{\pgfqpoint{1.270965in}{0.600874in}}%
\pgfpathlineto{\pgfqpoint{1.273100in}{0.604357in}}%
\pgfpathlineto{\pgfqpoint{1.274168in}{0.601456in}}%
\pgfpathlineto{\pgfqpoint{1.274701in}{0.603225in}}%
\pgfpathlineto{\pgfqpoint{1.275235in}{0.600876in}}%
\pgfpathlineto{\pgfqpoint{1.275769in}{0.604042in}}%
\pgfpathlineto{\pgfqpoint{1.276302in}{0.602679in}}%
\pgfpathlineto{\pgfqpoint{1.277370in}{0.603810in}}%
\pgfpathlineto{\pgfqpoint{1.278437in}{0.600031in}}%
\pgfpathlineto{\pgfqpoint{1.280038in}{0.602936in}}%
\pgfpathlineto{\pgfqpoint{1.280572in}{0.602911in}}%
\pgfpathlineto{\pgfqpoint{1.281640in}{0.605227in}}%
\pgfpathlineto{\pgfqpoint{1.282173in}{0.601119in}}%
\pgfpathlineto{\pgfqpoint{1.282707in}{0.604646in}}%
\pgfpathlineto{\pgfqpoint{1.283241in}{0.602726in}}%
\pgfpathlineto{\pgfqpoint{1.283774in}{0.603979in}}%
\pgfpathlineto{\pgfqpoint{1.284308in}{0.604444in}}%
\pgfpathlineto{\pgfqpoint{1.284842in}{0.606874in}}%
\pgfpathlineto{\pgfqpoint{1.285909in}{0.600874in}}%
\pgfpathlineto{\pgfqpoint{1.286443in}{0.602447in}}%
\pgfpathlineto{\pgfqpoint{1.287510in}{0.602568in}}%
\pgfpathlineto{\pgfqpoint{1.288044in}{0.601048in}}%
\pgfpathlineto{\pgfqpoint{1.288578in}{0.604002in}}%
\pgfpathlineto{\pgfqpoint{1.289111in}{0.603230in}}%
\pgfpathlineto{\pgfqpoint{1.289645in}{0.601686in}}%
\pgfpathlineto{\pgfqpoint{1.290179in}{0.605680in}}%
\pgfpathlineto{\pgfqpoint{1.290712in}{0.601199in}}%
\pgfpathlineto{\pgfqpoint{1.291246in}{0.601402in}}%
\pgfpathlineto{\pgfqpoint{1.291780in}{0.603809in}}%
\pgfpathlineto{\pgfqpoint{1.292314in}{0.602828in}}%
\pgfpathlineto{\pgfqpoint{1.293381in}{0.604175in}}%
\pgfpathlineto{\pgfqpoint{1.293915in}{0.602407in}}%
\pgfpathlineto{\pgfqpoint{1.294448in}{0.600913in}}%
\pgfpathlineto{\pgfqpoint{1.294982in}{0.604110in}}%
\pgfpathlineto{\pgfqpoint{1.295516in}{0.601104in}}%
\pgfpathlineto{\pgfqpoint{1.297651in}{0.603750in}}%
\pgfpathlineto{\pgfqpoint{1.298718in}{0.605676in}}%
\pgfpathlineto{\pgfqpoint{1.299252in}{0.606321in}}%
\pgfpathlineto{\pgfqpoint{1.300319in}{0.601859in}}%
\pgfpathlineto{\pgfqpoint{1.300853in}{0.602191in}}%
\pgfpathlineto{\pgfqpoint{1.301386in}{0.602974in}}%
\pgfpathlineto{\pgfqpoint{1.301920in}{0.600578in}}%
\pgfpathlineto{\pgfqpoint{1.302454in}{0.604445in}}%
\pgfpathlineto{\pgfqpoint{1.303521in}{0.604288in}}%
\pgfpathlineto{\pgfqpoint{1.305656in}{0.600692in}}%
\pgfpathlineto{\pgfqpoint{1.307791in}{0.605509in}}%
\pgfpathlineto{\pgfqpoint{1.308325in}{0.602898in}}%
\pgfpathlineto{\pgfqpoint{1.308858in}{0.603914in}}%
\pgfpathlineto{\pgfqpoint{1.309392in}{0.604381in}}%
\pgfpathlineto{\pgfqpoint{1.310993in}{0.601899in}}%
\pgfpathlineto{\pgfqpoint{1.312594in}{0.604093in}}%
\pgfpathlineto{\pgfqpoint{1.313128in}{0.601415in}}%
\pgfpathlineto{\pgfqpoint{1.313662in}{0.603505in}}%
\pgfpathlineto{\pgfqpoint{1.314195in}{0.602648in}}%
\pgfpathlineto{\pgfqpoint{1.314729in}{0.604183in}}%
\pgfpathlineto{\pgfqpoint{1.315263in}{0.602488in}}%
\pgfpathlineto{\pgfqpoint{1.316864in}{0.604647in}}%
\pgfpathlineto{\pgfqpoint{1.317397in}{0.600712in}}%
\pgfpathlineto{\pgfqpoint{1.317931in}{0.602234in}}%
\pgfpathlineto{\pgfqpoint{1.318465in}{0.604441in}}%
\pgfpathlineto{\pgfqpoint{1.320066in}{0.601142in}}%
\pgfpathlineto{\pgfqpoint{1.320600in}{0.601923in}}%
\pgfpathlineto{\pgfqpoint{1.321133in}{0.600488in}}%
\pgfpathlineto{\pgfqpoint{1.321667in}{0.601289in}}%
\pgfpathlineto{\pgfqpoint{1.323268in}{0.605083in}}%
\pgfpathlineto{\pgfqpoint{1.324336in}{0.601573in}}%
\pgfpathlineto{\pgfqpoint{1.324869in}{0.604391in}}%
\pgfpathlineto{\pgfqpoint{1.325403in}{0.603119in}}%
\pgfpathlineto{\pgfqpoint{1.327004in}{0.600514in}}%
\pgfpathlineto{\pgfqpoint{1.327538in}{0.602283in}}%
\pgfpathlineto{\pgfqpoint{1.328605in}{0.603591in}}%
\pgfpathlineto{\pgfqpoint{1.330740in}{0.600504in}}%
\pgfpathlineto{\pgfqpoint{1.331807in}{0.603409in}}%
\pgfpathlineto{\pgfqpoint{1.332341in}{0.602175in}}%
\pgfpathlineto{\pgfqpoint{1.332875in}{0.602426in}}%
\pgfpathlineto{\pgfqpoint{1.334476in}{0.600480in}}%
\pgfpathlineto{\pgfqpoint{1.336077in}{0.602917in}}%
\pgfpathlineto{\pgfqpoint{1.338212in}{0.604307in}}%
\pgfpathlineto{\pgfqpoint{1.338745in}{0.603621in}}%
\pgfpathlineto{\pgfqpoint{1.339813in}{0.601077in}}%
\pgfpathlineto{\pgfqpoint{1.340347in}{0.602763in}}%
\pgfpathlineto{\pgfqpoint{1.340880in}{0.603347in}}%
\pgfpathlineto{\pgfqpoint{1.341414in}{0.609173in}}%
\pgfpathlineto{\pgfqpoint{1.343015in}{0.602208in}}%
\pgfpathlineto{\pgfqpoint{1.344082in}{0.607233in}}%
\pgfpathlineto{\pgfqpoint{1.345684in}{0.601422in}}%
\pgfpathlineto{\pgfqpoint{1.347285in}{0.601090in}}%
\pgfpathlineto{\pgfqpoint{1.347818in}{0.604480in}}%
\pgfpathlineto{\pgfqpoint{1.348352in}{0.601140in}}%
\pgfpathlineto{\pgfqpoint{1.349420in}{0.604768in}}%
\pgfpathlineto{\pgfqpoint{1.349953in}{0.603405in}}%
\pgfpathlineto{\pgfqpoint{1.350487in}{0.603594in}}%
\pgfpathlineto{\pgfqpoint{1.351021in}{0.600443in}}%
\pgfpathlineto{\pgfqpoint{1.351554in}{0.601139in}}%
\pgfpathlineto{\pgfqpoint{1.355290in}{0.605539in}}%
\pgfpathlineto{\pgfqpoint{1.356358in}{0.601380in}}%
\pgfpathlineto{\pgfqpoint{1.357425in}{0.602783in}}%
\pgfpathlineto{\pgfqpoint{1.357959in}{0.601147in}}%
\pgfpathlineto{\pgfqpoint{1.358492in}{0.603658in}}%
\pgfpathlineto{\pgfqpoint{1.359026in}{0.603231in}}%
\pgfpathlineto{\pgfqpoint{1.360094in}{0.600471in}}%
\pgfpathlineto{\pgfqpoint{1.361695in}{0.602565in}}%
\pgfpathlineto{\pgfqpoint{1.362228in}{0.602761in}}%
\pgfpathlineto{\pgfqpoint{1.363296in}{0.604371in}}%
\pgfpathlineto{\pgfqpoint{1.363829in}{0.600339in}}%
\pgfpathlineto{\pgfqpoint{1.364363in}{0.603704in}}%
\pgfpathlineto{\pgfqpoint{1.365431in}{0.604510in}}%
\pgfpathlineto{\pgfqpoint{1.367032in}{0.601213in}}%
\pgfpathlineto{\pgfqpoint{1.367565in}{0.602260in}}%
\pgfpathlineto{\pgfqpoint{1.368099in}{0.601670in}}%
\pgfpathlineto{\pgfqpoint{1.368633in}{0.601032in}}%
\pgfpathlineto{\pgfqpoint{1.370234in}{0.604551in}}%
\pgfpathlineto{\pgfqpoint{1.370768in}{0.602121in}}%
\pgfpathlineto{\pgfqpoint{1.371301in}{0.604959in}}%
\pgfpathlineto{\pgfqpoint{1.371835in}{0.601427in}}%
\pgfpathlineto{\pgfqpoint{1.372369in}{0.602564in}}%
\pgfpathlineto{\pgfqpoint{1.373436in}{0.602620in}}%
\pgfpathlineto{\pgfqpoint{1.373970in}{0.605230in}}%
\pgfpathlineto{\pgfqpoint{1.374503in}{0.603817in}}%
\pgfpathlineto{\pgfqpoint{1.375571in}{0.601606in}}%
\pgfpathlineto{\pgfqpoint{1.376105in}{0.603414in}}%
\pgfpathlineto{\pgfqpoint{1.376638in}{0.601817in}}%
\pgfpathlineto{\pgfqpoint{1.378773in}{0.605971in}}%
\pgfpathlineto{\pgfqpoint{1.380374in}{0.601736in}}%
\pgfpathlineto{\pgfqpoint{1.383576in}{0.602421in}}%
\pgfpathlineto{\pgfqpoint{1.384110in}{0.602823in}}%
\pgfpathlineto{\pgfqpoint{1.384644in}{0.601853in}}%
\pgfpathlineto{\pgfqpoint{1.385177in}{0.602167in}}%
\pgfpathlineto{\pgfqpoint{1.385711in}{0.603826in}}%
\pgfpathlineto{\pgfqpoint{1.386779in}{0.601303in}}%
\pgfpathlineto{\pgfqpoint{1.387312in}{0.601845in}}%
\pgfpathlineto{\pgfqpoint{1.387846in}{0.603990in}}%
\pgfpathlineto{\pgfqpoint{1.388380in}{0.601850in}}%
\pgfpathlineto{\pgfqpoint{1.389447in}{0.601718in}}%
\pgfpathlineto{\pgfqpoint{1.389981in}{0.603361in}}%
\pgfpathlineto{\pgfqpoint{1.390514in}{0.601039in}}%
\pgfpathlineto{\pgfqpoint{1.391048in}{0.604167in}}%
\pgfpathlineto{\pgfqpoint{1.391582in}{0.600405in}}%
\pgfpathlineto{\pgfqpoint{1.392116in}{0.602038in}}%
\pgfpathlineto{\pgfqpoint{1.392649in}{0.601964in}}%
\pgfpathlineto{\pgfqpoint{1.393183in}{0.603273in}}%
\pgfpathlineto{\pgfqpoint{1.393717in}{0.601899in}}%
\pgfpathlineto{\pgfqpoint{1.395318in}{0.604404in}}%
\pgfpathlineto{\pgfqpoint{1.395851in}{0.602126in}}%
\pgfpathlineto{\pgfqpoint{1.396385in}{0.603304in}}%
\pgfpathlineto{\pgfqpoint{1.396919in}{0.604350in}}%
\pgfpathlineto{\pgfqpoint{1.397453in}{0.600909in}}%
\pgfpathlineto{\pgfqpoint{1.397986in}{0.607414in}}%
\pgfpathlineto{\pgfqpoint{1.398520in}{0.601058in}}%
\pgfpathlineto{\pgfqpoint{1.400121in}{0.606870in}}%
\pgfpathlineto{\pgfqpoint{1.401188in}{0.601972in}}%
\pgfpathlineto{\pgfqpoint{1.401722in}{0.602390in}}%
\pgfpathlineto{\pgfqpoint{1.402256in}{0.603705in}}%
\pgfpathlineto{\pgfqpoint{1.402790in}{0.602100in}}%
\pgfpathlineto{\pgfqpoint{1.403323in}{0.602854in}}%
\pgfpathlineto{\pgfqpoint{1.403857in}{0.602366in}}%
\pgfpathlineto{\pgfqpoint{1.404391in}{0.601950in}}%
\pgfpathlineto{\pgfqpoint{1.405458in}{0.602950in}}%
\pgfpathlineto{\pgfqpoint{1.405992in}{0.601236in}}%
\pgfpathlineto{\pgfqpoint{1.406525in}{0.601782in}}%
\pgfpathlineto{\pgfqpoint{1.407059in}{0.607586in}}%
\pgfpathlineto{\pgfqpoint{1.407593in}{0.607033in}}%
\pgfpathlineto{\pgfqpoint{1.408127in}{0.601208in}}%
\pgfpathlineto{\pgfqpoint{1.408660in}{0.603070in}}%
\pgfpathlineto{\pgfqpoint{1.409194in}{0.602613in}}%
\pgfpathlineto{\pgfqpoint{1.410261in}{0.604832in}}%
\pgfpathlineto{\pgfqpoint{1.411329in}{0.601420in}}%
\pgfpathlineto{\pgfqpoint{1.411862in}{0.602773in}}%
\pgfpathlineto{\pgfqpoint{1.412396in}{0.604644in}}%
\pgfpathlineto{\pgfqpoint{1.412930in}{0.603171in}}%
\pgfpathlineto{\pgfqpoint{1.413464in}{0.601805in}}%
\pgfpathlineto{\pgfqpoint{1.413997in}{0.603868in}}%
\pgfpathlineto{\pgfqpoint{1.414531in}{0.600378in}}%
\pgfpathlineto{\pgfqpoint{1.415065in}{0.602440in}}%
\pgfpathlineto{\pgfqpoint{1.415598in}{0.601830in}}%
\pgfpathlineto{\pgfqpoint{1.416132in}{0.602402in}}%
\pgfpathlineto{\pgfqpoint{1.416666in}{0.603990in}}%
\pgfpathlineto{\pgfqpoint{1.417200in}{0.603272in}}%
\pgfpathlineto{\pgfqpoint{1.419334in}{0.601478in}}%
\pgfpathlineto{\pgfqpoint{1.420402in}{0.605167in}}%
\pgfpathlineto{\pgfqpoint{1.420935in}{0.604463in}}%
\pgfpathlineto{\pgfqpoint{1.422537in}{0.602880in}}%
\pgfpathlineto{\pgfqpoint{1.423070in}{0.604514in}}%
\pgfpathlineto{\pgfqpoint{1.423604in}{0.600312in}}%
\pgfpathlineto{\pgfqpoint{1.424138in}{0.604879in}}%
\pgfpathlineto{\pgfqpoint{1.425739in}{0.600234in}}%
\pgfpathlineto{\pgfqpoint{1.426272in}{0.603840in}}%
\pgfpathlineto{\pgfqpoint{1.426806in}{0.603006in}}%
\pgfpathlineto{\pgfqpoint{1.427340in}{0.601320in}}%
\pgfpathlineto{\pgfqpoint{1.428407in}{0.606096in}}%
\pgfpathlineto{\pgfqpoint{1.430008in}{0.600342in}}%
\pgfpathlineto{\pgfqpoint{1.430542in}{0.604671in}}%
\pgfpathlineto{\pgfqpoint{1.431076in}{0.601436in}}%
\pgfpathlineto{\pgfqpoint{1.431609in}{0.604103in}}%
\pgfpathlineto{\pgfqpoint{1.432143in}{0.603030in}}%
\pgfpathlineto{\pgfqpoint{1.433211in}{0.604397in}}%
\pgfpathlineto{\pgfqpoint{1.434812in}{0.602072in}}%
\pgfpathlineto{\pgfqpoint{1.436413in}{0.604549in}}%
\pgfpathlineto{\pgfqpoint{1.438014in}{0.600920in}}%
\pgfpathlineto{\pgfqpoint{1.438548in}{0.601610in}}%
\pgfpathlineto{\pgfqpoint{1.439615in}{0.603161in}}%
\pgfpathlineto{\pgfqpoint{1.440149in}{0.600238in}}%
\pgfpathlineto{\pgfqpoint{1.440682in}{0.602433in}}%
\pgfpathlineto{\pgfqpoint{1.441216in}{0.603118in}}%
\pgfpathlineto{\pgfqpoint{1.441750in}{0.606220in}}%
\pgfpathlineto{\pgfqpoint{1.442283in}{0.602359in}}%
\pgfpathlineto{\pgfqpoint{1.442817in}{0.602513in}}%
\pgfpathlineto{\pgfqpoint{1.443351in}{0.603954in}}%
\pgfpathlineto{\pgfqpoint{1.444952in}{0.600718in}}%
\pgfpathlineto{\pgfqpoint{1.447087in}{0.602208in}}%
\pgfpathlineto{\pgfqpoint{1.447620in}{0.601194in}}%
\pgfpathlineto{\pgfqpoint{1.448154in}{0.605316in}}%
\pgfpathlineto{\pgfqpoint{1.448688in}{0.603998in}}%
\pgfpathlineto{\pgfqpoint{1.450289in}{0.600876in}}%
\pgfpathlineto{\pgfqpoint{1.450823in}{0.601578in}}%
\pgfpathlineto{\pgfqpoint{1.451356in}{0.600644in}}%
\pgfpathlineto{\pgfqpoint{1.451890in}{0.608999in}}%
\pgfpathlineto{\pgfqpoint{1.452424in}{0.608662in}}%
\pgfpathlineto{\pgfqpoint{1.454025in}{0.602692in}}%
\pgfpathlineto{\pgfqpoint{1.454559in}{0.604104in}}%
\pgfpathlineto{\pgfqpoint{1.455092in}{0.601838in}}%
\pgfpathlineto{\pgfqpoint{1.456160in}{0.606147in}}%
\pgfpathlineto{\pgfqpoint{1.456693in}{0.604949in}}%
\pgfpathlineto{\pgfqpoint{1.457227in}{0.602203in}}%
\pgfpathlineto{\pgfqpoint{1.457761in}{0.605244in}}%
\pgfpathlineto{\pgfqpoint{1.458294in}{0.605063in}}%
\pgfpathlineto{\pgfqpoint{1.459896in}{0.600655in}}%
\pgfpathlineto{\pgfqpoint{1.461497in}{0.602763in}}%
\pgfpathlineto{\pgfqpoint{1.462030in}{0.602470in}}%
\pgfpathlineto{\pgfqpoint{1.462564in}{0.606758in}}%
\pgfpathlineto{\pgfqpoint{1.463098in}{0.603994in}}%
\pgfpathlineto{\pgfqpoint{1.463631in}{0.604443in}}%
\pgfpathlineto{\pgfqpoint{1.464165in}{0.602363in}}%
\pgfpathlineto{\pgfqpoint{1.464699in}{0.603079in}}%
\pgfpathlineto{\pgfqpoint{1.465233in}{0.604940in}}%
\pgfpathlineto{\pgfqpoint{1.466300in}{0.600741in}}%
\pgfpathlineto{\pgfqpoint{1.467367in}{0.603962in}}%
\pgfpathlineto{\pgfqpoint{1.467901in}{0.601122in}}%
\pgfpathlineto{\pgfqpoint{1.468435in}{0.603061in}}%
\pgfpathlineto{\pgfqpoint{1.470036in}{0.602104in}}%
\pgfpathlineto{\pgfqpoint{1.470570in}{0.601228in}}%
\pgfpathlineto{\pgfqpoint{1.471637in}{0.604738in}}%
\pgfpathlineto{\pgfqpoint{1.472171in}{0.601959in}}%
\pgfpathlineto{\pgfqpoint{1.472704in}{0.603109in}}%
\pgfpathlineto{\pgfqpoint{1.473238in}{0.603242in}}%
\pgfpathlineto{\pgfqpoint{1.473772in}{0.600798in}}%
\pgfpathlineto{\pgfqpoint{1.474305in}{0.601844in}}%
\pgfpathlineto{\pgfqpoint{1.476440in}{0.605398in}}%
\pgfpathlineto{\pgfqpoint{1.476974in}{0.605720in}}%
\pgfpathlineto{\pgfqpoint{1.478041in}{0.600067in}}%
\pgfpathlineto{\pgfqpoint{1.478575in}{0.601009in}}%
\pgfpathlineto{\pgfqpoint{1.479109in}{0.603938in}}%
\pgfpathlineto{\pgfqpoint{1.479642in}{0.601092in}}%
\pgfpathlineto{\pgfqpoint{1.480176in}{0.600912in}}%
\pgfpathlineto{\pgfqpoint{1.481777in}{0.604200in}}%
\pgfpathlineto{\pgfqpoint{1.482311in}{0.602081in}}%
\pgfpathlineto{\pgfqpoint{1.482845in}{0.603794in}}%
\pgfpathlineto{\pgfqpoint{1.483378in}{0.603262in}}%
\pgfpathlineto{\pgfqpoint{1.483912in}{0.600903in}}%
\pgfpathlineto{\pgfqpoint{1.484446in}{0.602530in}}%
\pgfpathlineto{\pgfqpoint{1.484980in}{0.608630in}}%
\pgfpathlineto{\pgfqpoint{1.485513in}{0.604247in}}%
\pgfpathlineto{\pgfqpoint{1.487114in}{0.600450in}}%
\pgfpathlineto{\pgfqpoint{1.487648in}{0.600932in}}%
\pgfpathlineto{\pgfqpoint{1.488182in}{0.604562in}}%
\pgfpathlineto{\pgfqpoint{1.488715in}{0.602217in}}%
\pgfpathlineto{\pgfqpoint{1.490317in}{0.601762in}}%
\pgfpathlineto{\pgfqpoint{1.491384in}{0.607000in}}%
\pgfpathlineto{\pgfqpoint{1.491918in}{0.600631in}}%
\pgfpathlineto{\pgfqpoint{1.492451in}{0.603489in}}%
\pgfpathlineto{\pgfqpoint{1.494052in}{0.602222in}}%
\pgfpathlineto{\pgfqpoint{1.496187in}{0.601547in}}%
\pgfpathlineto{\pgfqpoint{1.496721in}{0.604910in}}%
\pgfpathlineto{\pgfqpoint{1.497255in}{0.602199in}}%
\pgfpathlineto{\pgfqpoint{1.497788in}{0.600802in}}%
\pgfpathlineto{\pgfqpoint{1.498322in}{0.602370in}}%
\pgfpathlineto{\pgfqpoint{1.498856in}{0.601166in}}%
\pgfpathlineto{\pgfqpoint{1.499389in}{0.603914in}}%
\pgfpathlineto{\pgfqpoint{1.499923in}{0.601632in}}%
\pgfpathlineto{\pgfqpoint{1.500457in}{0.601706in}}%
\pgfpathlineto{\pgfqpoint{1.501524in}{0.604037in}}%
\pgfpathlineto{\pgfqpoint{1.502592in}{0.600519in}}%
\pgfpathlineto{\pgfqpoint{1.503125in}{0.604928in}}%
\pgfpathlineto{\pgfqpoint{1.503659in}{0.601154in}}%
\pgfpathlineto{\pgfqpoint{1.504726in}{0.603048in}}%
\pgfpathlineto{\pgfqpoint{1.506328in}{0.601480in}}%
\pgfpathlineto{\pgfqpoint{1.506861in}{0.607548in}}%
\pgfpathlineto{\pgfqpoint{1.507395in}{0.603806in}}%
\pgfpathlineto{\pgfqpoint{1.508462in}{0.608144in}}%
\pgfpathlineto{\pgfqpoint{1.510063in}{0.602273in}}%
\pgfpathlineto{\pgfqpoint{1.511665in}{0.604383in}}%
\pgfpathlineto{\pgfqpoint{1.512732in}{0.600681in}}%
\pgfpathlineto{\pgfqpoint{1.513266in}{0.603456in}}%
\pgfpathlineto{\pgfqpoint{1.514333in}{0.602112in}}%
\pgfpathlineto{\pgfqpoint{1.514867in}{0.607766in}}%
\pgfpathlineto{\pgfqpoint{1.515400in}{0.603991in}}%
\pgfpathlineto{\pgfqpoint{1.516468in}{0.600507in}}%
\pgfpathlineto{\pgfqpoint{1.517002in}{0.602294in}}%
\pgfpathlineto{\pgfqpoint{1.518069in}{0.606041in}}%
\pgfpathlineto{\pgfqpoint{1.518603in}{0.605152in}}%
\pgfpathlineto{\pgfqpoint{1.519670in}{0.602199in}}%
\pgfpathlineto{\pgfqpoint{1.520204in}{0.602660in}}%
\pgfpathlineto{\pgfqpoint{1.522339in}{0.604290in}}%
\pgfpathlineto{\pgfqpoint{1.524473in}{0.602658in}}%
\pgfpathlineto{\pgfqpoint{1.525007in}{0.602387in}}%
\pgfpathlineto{\pgfqpoint{1.525541in}{0.600430in}}%
\pgfpathlineto{\pgfqpoint{1.526074in}{0.604882in}}%
\pgfpathlineto{\pgfqpoint{1.526608in}{0.601673in}}%
\pgfpathlineto{\pgfqpoint{1.527142in}{0.601606in}}%
\pgfpathlineto{\pgfqpoint{1.528743in}{0.604362in}}%
\pgfpathlineto{\pgfqpoint{1.529277in}{0.603698in}}%
\pgfpathlineto{\pgfqpoint{1.529810in}{0.605102in}}%
\pgfpathlineto{\pgfqpoint{1.530344in}{0.604736in}}%
\pgfpathlineto{\pgfqpoint{1.530878in}{0.604403in}}%
\pgfpathlineto{\pgfqpoint{1.531411in}{0.608722in}}%
\pgfpathlineto{\pgfqpoint{1.531945in}{0.606052in}}%
\pgfpathlineto{\pgfqpoint{1.533013in}{0.606160in}}%
\pgfpathlineto{\pgfqpoint{1.533546in}{0.603739in}}%
\pgfpathlineto{\pgfqpoint{1.534080in}{0.607108in}}%
\pgfpathlineto{\pgfqpoint{1.534614in}{0.601121in}}%
\pgfpathlineto{\pgfqpoint{1.535147in}{0.604549in}}%
\pgfpathlineto{\pgfqpoint{1.535681in}{0.608063in}}%
\pgfpathlineto{\pgfqpoint{1.537282in}{0.604099in}}%
\pgfpathlineto{\pgfqpoint{1.537816in}{0.604337in}}%
\pgfpathlineto{\pgfqpoint{1.538350in}{0.601941in}}%
\pgfpathlineto{\pgfqpoint{1.538883in}{0.603098in}}%
\pgfpathlineto{\pgfqpoint{1.539417in}{0.605585in}}%
\pgfpathlineto{\pgfqpoint{1.539951in}{0.605202in}}%
\pgfpathlineto{\pgfqpoint{1.541018in}{0.600901in}}%
\pgfpathlineto{\pgfqpoint{1.541552in}{0.602875in}}%
\pgfpathlineto{\pgfqpoint{1.542085in}{0.606821in}}%
\pgfpathlineto{\pgfqpoint{1.543687in}{0.601635in}}%
\pgfpathlineto{\pgfqpoint{1.545821in}{0.606497in}}%
\pgfpathlineto{\pgfqpoint{1.546355in}{0.605550in}}%
\pgfpathlineto{\pgfqpoint{1.546889in}{0.600302in}}%
\pgfpathlineto{\pgfqpoint{1.547423in}{0.602466in}}%
\pgfpathlineto{\pgfqpoint{1.547956in}{0.602446in}}%
\pgfpathlineto{\pgfqpoint{1.548490in}{0.601247in}}%
\pgfpathlineto{\pgfqpoint{1.549557in}{0.605682in}}%
\pgfpathlineto{\pgfqpoint{1.551158in}{0.600525in}}%
\pgfpathlineto{\pgfqpoint{1.552760in}{0.607157in}}%
\pgfpathlineto{\pgfqpoint{1.554361in}{0.602367in}}%
\pgfpathlineto{\pgfqpoint{1.554894in}{0.605729in}}%
\pgfpathlineto{\pgfqpoint{1.555428in}{0.600939in}}%
\pgfpathlineto{\pgfqpoint{1.555962in}{0.603495in}}%
\pgfpathlineto{\pgfqpoint{1.557029in}{0.601415in}}%
\pgfpathlineto{\pgfqpoint{1.558630in}{0.607693in}}%
\pgfpathlineto{\pgfqpoint{1.559164in}{0.602128in}}%
\pgfpathlineto{\pgfqpoint{1.560231in}{0.602320in}}%
\pgfpathlineto{\pgfqpoint{1.560765in}{0.601032in}}%
\pgfpathlineto{\pgfqpoint{1.562366in}{0.609624in}}%
\pgfpathlineto{\pgfqpoint{1.564501in}{0.602149in}}%
\pgfpathlineto{\pgfqpoint{1.566636in}{0.606044in}}%
\pgfpathlineto{\pgfqpoint{1.567703in}{0.601172in}}%
\pgfpathlineto{\pgfqpoint{1.568237in}{0.605092in}}%
\pgfpathlineto{\pgfqpoint{1.568771in}{0.604126in}}%
\pgfpathlineto{\pgfqpoint{1.570372in}{0.600850in}}%
\pgfpathlineto{\pgfqpoint{1.570905in}{0.604984in}}%
\pgfpathlineto{\pgfqpoint{1.571439in}{0.603106in}}%
\pgfpathlineto{\pgfqpoint{1.571973in}{0.602421in}}%
\pgfpathlineto{\pgfqpoint{1.573040in}{0.612057in}}%
\pgfpathlineto{\pgfqpoint{1.574641in}{0.602023in}}%
\pgfpathlineto{\pgfqpoint{1.576776in}{0.608633in}}%
\pgfpathlineto{\pgfqpoint{1.578377in}{0.602564in}}%
\pgfpathlineto{\pgfqpoint{1.578911in}{0.608973in}}%
\pgfpathlineto{\pgfqpoint{1.579445in}{0.604988in}}%
\pgfpathlineto{\pgfqpoint{1.579978in}{0.601973in}}%
\pgfpathlineto{\pgfqpoint{1.580512in}{0.605035in}}%
\pgfpathlineto{\pgfqpoint{1.581046in}{0.602518in}}%
\pgfpathlineto{\pgfqpoint{1.581579in}{0.604701in}}%
\pgfpathlineto{\pgfqpoint{1.582113in}{0.602845in}}%
\pgfpathlineto{\pgfqpoint{1.582647in}{0.607389in}}%
\pgfpathlineto{\pgfqpoint{1.583180in}{0.606217in}}%
\pgfpathlineto{\pgfqpoint{1.583714in}{0.606073in}}%
\pgfpathlineto{\pgfqpoint{1.584248in}{0.600194in}}%
\pgfpathlineto{\pgfqpoint{1.584782in}{0.605545in}}%
\pgfpathlineto{\pgfqpoint{1.585315in}{0.601280in}}%
\pgfpathlineto{\pgfqpoint{1.586916in}{0.607765in}}%
\pgfpathlineto{\pgfqpoint{1.587450in}{0.608000in}}%
\pgfpathlineto{\pgfqpoint{1.587984in}{0.602487in}}%
\pgfpathlineto{\pgfqpoint{1.588517in}{0.605699in}}%
\pgfpathlineto{\pgfqpoint{1.589585in}{0.610032in}}%
\pgfpathlineto{\pgfqpoint{1.590119in}{0.600767in}}%
\pgfpathlineto{\pgfqpoint{1.590652in}{0.605112in}}%
\pgfpathlineto{\pgfqpoint{1.591186in}{0.601864in}}%
\pgfpathlineto{\pgfqpoint{1.591720in}{0.602532in}}%
\pgfpathlineto{\pgfqpoint{1.592253in}{0.608008in}}%
\pgfpathlineto{\pgfqpoint{1.592787in}{0.603197in}}%
\pgfpathlineto{\pgfqpoint{1.593854in}{0.607425in}}%
\pgfpathlineto{\pgfqpoint{1.594388in}{0.605979in}}%
\pgfpathlineto{\pgfqpoint{1.595456in}{0.602691in}}%
\pgfpathlineto{\pgfqpoint{1.595989in}{0.603507in}}%
\pgfpathlineto{\pgfqpoint{1.596523in}{0.602992in}}%
\pgfpathlineto{\pgfqpoint{1.597057in}{0.607378in}}%
\pgfpathlineto{\pgfqpoint{1.597590in}{0.602199in}}%
\pgfpathlineto{\pgfqpoint{1.598124in}{0.603443in}}%
\pgfpathlineto{\pgfqpoint{1.599191in}{0.604003in}}%
\pgfpathlineto{\pgfqpoint{1.600259in}{0.608211in}}%
\pgfpathlineto{\pgfqpoint{1.600793in}{0.603324in}}%
\pgfpathlineto{\pgfqpoint{1.601326in}{0.604265in}}%
\pgfpathlineto{\pgfqpoint{1.601860in}{0.604625in}}%
\pgfpathlineto{\pgfqpoint{1.602394in}{0.600683in}}%
\pgfpathlineto{\pgfqpoint{1.602927in}{0.605601in}}%
\pgfpathlineto{\pgfqpoint{1.603995in}{0.605373in}}%
\pgfpathlineto{\pgfqpoint{1.604528in}{0.603994in}}%
\pgfpathlineto{\pgfqpoint{1.605062in}{0.605741in}}%
\pgfpathlineto{\pgfqpoint{1.605596in}{0.601568in}}%
\pgfpathlineto{\pgfqpoint{1.606130in}{0.603458in}}%
\pgfpathlineto{\pgfqpoint{1.607197in}{0.611771in}}%
\pgfpathlineto{\pgfqpoint{1.607731in}{0.601650in}}%
\pgfpathlineto{\pgfqpoint{1.608798in}{0.601752in}}%
\pgfpathlineto{\pgfqpoint{1.609865in}{0.606377in}}%
\pgfpathlineto{\pgfqpoint{1.610399in}{0.600738in}}%
\pgfpathlineto{\pgfqpoint{1.610933in}{0.603486in}}%
\pgfpathlineto{\pgfqpoint{1.611467in}{0.606001in}}%
\pgfpathlineto{\pgfqpoint{1.612534in}{0.600739in}}%
\pgfpathlineto{\pgfqpoint{1.613068in}{0.605972in}}%
\pgfpathlineto{\pgfqpoint{1.613601in}{0.605775in}}%
\pgfpathlineto{\pgfqpoint{1.614135in}{0.603213in}}%
\pgfpathlineto{\pgfqpoint{1.614669in}{0.605386in}}%
\pgfpathlineto{\pgfqpoint{1.615736in}{0.601429in}}%
\pgfpathlineto{\pgfqpoint{1.617337in}{0.609653in}}%
\pgfpathlineto{\pgfqpoint{1.618938in}{0.604104in}}%
\pgfpathlineto{\pgfqpoint{1.619472in}{0.606862in}}%
\pgfpathlineto{\pgfqpoint{1.620006in}{0.602069in}}%
\pgfpathlineto{\pgfqpoint{1.620540in}{0.604740in}}%
\pgfpathlineto{\pgfqpoint{1.621073in}{0.607629in}}%
\pgfpathlineto{\pgfqpoint{1.621607in}{0.604843in}}%
\pgfpathlineto{\pgfqpoint{1.622141in}{0.600755in}}%
\pgfpathlineto{\pgfqpoint{1.622674in}{0.601235in}}%
\pgfpathlineto{\pgfqpoint{1.623742in}{0.605750in}}%
\pgfpathlineto{\pgfqpoint{1.624275in}{0.603563in}}%
\pgfpathlineto{\pgfqpoint{1.624809in}{0.601759in}}%
\pgfpathlineto{\pgfqpoint{1.625343in}{0.606977in}}%
\pgfpathlineto{\pgfqpoint{1.625877in}{0.605172in}}%
\pgfpathlineto{\pgfqpoint{1.626944in}{0.603521in}}%
\pgfpathlineto{\pgfqpoint{1.629079in}{0.610771in}}%
\pgfpathlineto{\pgfqpoint{1.630680in}{0.603171in}}%
\pgfpathlineto{\pgfqpoint{1.631214in}{0.602327in}}%
\pgfpathlineto{\pgfqpoint{1.632815in}{0.609057in}}%
\pgfpathlineto{\pgfqpoint{1.633348in}{0.603528in}}%
\pgfpathlineto{\pgfqpoint{1.633882in}{0.606017in}}%
\pgfpathlineto{\pgfqpoint{1.634416in}{0.605705in}}%
\pgfpathlineto{\pgfqpoint{1.634949in}{0.601604in}}%
\pgfpathlineto{\pgfqpoint{1.635483in}{0.607062in}}%
\pgfpathlineto{\pgfqpoint{1.636017in}{0.602859in}}%
\pgfpathlineto{\pgfqpoint{1.637618in}{0.603617in}}%
\pgfpathlineto{\pgfqpoint{1.638152in}{0.601974in}}%
\pgfpathlineto{\pgfqpoint{1.639753in}{0.607239in}}%
\pgfpathlineto{\pgfqpoint{1.640286in}{0.604151in}}%
\pgfpathlineto{\pgfqpoint{1.640820in}{0.605547in}}%
\pgfpathlineto{\pgfqpoint{1.641354in}{0.610094in}}%
\pgfpathlineto{\pgfqpoint{1.641888in}{0.605825in}}%
\pgfpathlineto{\pgfqpoint{1.642421in}{0.608098in}}%
\pgfpathlineto{\pgfqpoint{1.643489in}{0.608410in}}%
\pgfpathlineto{\pgfqpoint{1.644022in}{0.602051in}}%
\pgfpathlineto{\pgfqpoint{1.644556in}{0.611229in}}%
\pgfpathlineto{\pgfqpoint{1.645090in}{0.605361in}}%
\pgfpathlineto{\pgfqpoint{1.645623in}{0.605119in}}%
\pgfpathlineto{\pgfqpoint{1.646157in}{0.606942in}}%
\pgfpathlineto{\pgfqpoint{1.646691in}{0.604697in}}%
\pgfpathlineto{\pgfqpoint{1.647225in}{0.609795in}}%
\pgfpathlineto{\pgfqpoint{1.647758in}{0.604961in}}%
\pgfpathlineto{\pgfqpoint{1.648292in}{0.607016in}}%
\pgfpathlineto{\pgfqpoint{1.648826in}{0.605092in}}%
\pgfpathlineto{\pgfqpoint{1.649359in}{0.605446in}}%
\pgfpathlineto{\pgfqpoint{1.649893in}{0.603126in}}%
\pgfpathlineto{\pgfqpoint{1.650427in}{0.603661in}}%
\pgfpathlineto{\pgfqpoint{1.652028in}{0.606875in}}%
\pgfpathlineto{\pgfqpoint{1.652562in}{0.606504in}}%
\pgfpathlineto{\pgfqpoint{1.653095in}{0.600938in}}%
\pgfpathlineto{\pgfqpoint{1.653629in}{0.603071in}}%
\pgfpathlineto{\pgfqpoint{1.654163in}{0.606439in}}%
\pgfpathlineto{\pgfqpoint{1.654696in}{0.605670in}}%
\pgfpathlineto{\pgfqpoint{1.655764in}{0.600986in}}%
\pgfpathlineto{\pgfqpoint{1.656831in}{0.610443in}}%
\pgfpathlineto{\pgfqpoint{1.657365in}{0.604734in}}%
\pgfpathlineto{\pgfqpoint{1.657899in}{0.608476in}}%
\pgfpathlineto{\pgfqpoint{1.659500in}{0.601568in}}%
\pgfpathlineto{\pgfqpoint{1.660033in}{0.606518in}}%
\pgfpathlineto{\pgfqpoint{1.660567in}{0.605116in}}%
\pgfpathlineto{\pgfqpoint{1.661634in}{0.602887in}}%
\pgfpathlineto{\pgfqpoint{1.662168in}{0.607034in}}%
\pgfpathlineto{\pgfqpoint{1.662702in}{0.602869in}}%
\pgfpathlineto{\pgfqpoint{1.663236in}{0.601223in}}%
\pgfpathlineto{\pgfqpoint{1.664303in}{0.609692in}}%
\pgfpathlineto{\pgfqpoint{1.664837in}{0.608298in}}%
\pgfpathlineto{\pgfqpoint{1.665904in}{0.600854in}}%
\pgfpathlineto{\pgfqpoint{1.666971in}{0.607086in}}%
\pgfpathlineto{\pgfqpoint{1.667505in}{0.604616in}}%
\pgfpathlineto{\pgfqpoint{1.668039in}{0.603253in}}%
\pgfpathlineto{\pgfqpoint{1.670174in}{0.611195in}}%
\pgfpathlineto{\pgfqpoint{1.670707in}{0.601862in}}%
\pgfpathlineto{\pgfqpoint{1.671241in}{0.606072in}}%
\pgfpathlineto{\pgfqpoint{1.672842in}{0.610935in}}%
\pgfpathlineto{\pgfqpoint{1.674977in}{0.604721in}}%
\pgfpathlineto{\pgfqpoint{1.676578in}{0.607829in}}%
\pgfpathlineto{\pgfqpoint{1.678179in}{0.603331in}}%
\pgfpathlineto{\pgfqpoint{1.679780in}{0.601868in}}%
\pgfpathlineto{\pgfqpoint{1.681381in}{0.609998in}}%
\pgfpathlineto{\pgfqpoint{1.681915in}{0.602097in}}%
\pgfpathlineto{\pgfqpoint{1.682449in}{0.604353in}}%
\pgfpathlineto{\pgfqpoint{1.683516in}{0.610718in}}%
\pgfpathlineto{\pgfqpoint{1.685117in}{0.601409in}}%
\pgfpathlineto{\pgfqpoint{1.686185in}{0.608555in}}%
\pgfpathlineto{\pgfqpoint{1.686718in}{0.606165in}}%
\pgfpathlineto{\pgfqpoint{1.687252in}{0.611395in}}%
\pgfpathlineto{\pgfqpoint{1.687786in}{0.607919in}}%
\pgfpathlineto{\pgfqpoint{1.688320in}{0.603642in}}%
\pgfpathlineto{\pgfqpoint{1.688853in}{0.607761in}}%
\pgfpathlineto{\pgfqpoint{1.690454in}{0.610485in}}%
\pgfpathlineto{\pgfqpoint{1.691522in}{0.608064in}}%
\pgfpathlineto{\pgfqpoint{1.692055in}{0.603151in}}%
\pgfpathlineto{\pgfqpoint{1.692589in}{0.606037in}}%
\pgfpathlineto{\pgfqpoint{1.694190in}{0.602617in}}%
\pgfpathlineto{\pgfqpoint{1.695791in}{0.608842in}}%
\pgfpathlineto{\pgfqpoint{1.696325in}{0.608593in}}%
\pgfpathlineto{\pgfqpoint{1.696859in}{0.604764in}}%
\pgfpathlineto{\pgfqpoint{1.697392in}{0.613654in}}%
\pgfpathlineto{\pgfqpoint{1.697926in}{0.606770in}}%
\pgfpathlineto{\pgfqpoint{1.698460in}{0.610536in}}%
\pgfpathlineto{\pgfqpoint{1.698994in}{0.602210in}}%
\pgfpathlineto{\pgfqpoint{1.699527in}{0.610828in}}%
\pgfpathlineto{\pgfqpoint{1.700595in}{0.601063in}}%
\pgfpathlineto{\pgfqpoint{1.701662in}{0.604133in}}%
\pgfpathlineto{\pgfqpoint{1.702196in}{0.607089in}}%
\pgfpathlineto{\pgfqpoint{1.702729in}{0.604000in}}%
\pgfpathlineto{\pgfqpoint{1.703263in}{0.604120in}}%
\pgfpathlineto{\pgfqpoint{1.704331in}{0.608677in}}%
\pgfpathlineto{\pgfqpoint{1.704864in}{0.605781in}}%
\pgfpathlineto{\pgfqpoint{1.705398in}{0.609445in}}%
\pgfpathlineto{\pgfqpoint{1.705932in}{0.604172in}}%
\pgfpathlineto{\pgfqpoint{1.706465in}{0.605183in}}%
\pgfpathlineto{\pgfqpoint{1.706999in}{0.607691in}}%
\pgfpathlineto{\pgfqpoint{1.707533in}{0.604001in}}%
\pgfpathlineto{\pgfqpoint{1.708066in}{0.607553in}}%
\pgfpathlineto{\pgfqpoint{1.709134in}{0.603646in}}%
\pgfpathlineto{\pgfqpoint{1.710735in}{0.606998in}}%
\pgfpathlineto{\pgfqpoint{1.711269in}{0.607562in}}%
\pgfpathlineto{\pgfqpoint{1.711802in}{0.618691in}}%
\pgfpathlineto{\pgfqpoint{1.712336in}{0.603964in}}%
\pgfpathlineto{\pgfqpoint{1.712870in}{0.604439in}}%
\pgfpathlineto{\pgfqpoint{1.713403in}{0.610058in}}%
\pgfpathlineto{\pgfqpoint{1.713937in}{0.605674in}}%
\pgfpathlineto{\pgfqpoint{1.714471in}{0.603656in}}%
\pgfpathlineto{\pgfqpoint{1.715005in}{0.605853in}}%
\pgfpathlineto{\pgfqpoint{1.715538in}{0.611822in}}%
\pgfpathlineto{\pgfqpoint{1.716072in}{0.601578in}}%
\pgfpathlineto{\pgfqpoint{1.716606in}{0.609530in}}%
\pgfpathlineto{\pgfqpoint{1.717673in}{0.603530in}}%
\pgfpathlineto{\pgfqpoint{1.718207in}{0.604681in}}%
\pgfpathlineto{\pgfqpoint{1.718740in}{0.612725in}}%
\pgfpathlineto{\pgfqpoint{1.719274in}{0.609457in}}%
\pgfpathlineto{\pgfqpoint{1.719808in}{0.608809in}}%
\pgfpathlineto{\pgfqpoint{1.720342in}{0.600842in}}%
\pgfpathlineto{\pgfqpoint{1.720875in}{0.606412in}}%
\pgfpathlineto{\pgfqpoint{1.721409in}{0.603019in}}%
\pgfpathlineto{\pgfqpoint{1.721943in}{0.605481in}}%
\pgfpathlineto{\pgfqpoint{1.723010in}{0.602242in}}%
\pgfpathlineto{\pgfqpoint{1.723544in}{0.606844in}}%
\pgfpathlineto{\pgfqpoint{1.724077in}{0.603503in}}%
\pgfpathlineto{\pgfqpoint{1.724611in}{0.602773in}}%
\pgfpathlineto{\pgfqpoint{1.726212in}{0.608978in}}%
\pgfpathlineto{\pgfqpoint{1.726746in}{0.612322in}}%
\pgfpathlineto{\pgfqpoint{1.727280in}{0.603641in}}%
\pgfpathlineto{\pgfqpoint{1.727813in}{0.613739in}}%
\pgfpathlineto{\pgfqpoint{1.728347in}{0.612822in}}%
\pgfpathlineto{\pgfqpoint{1.729414in}{0.605642in}}%
\pgfpathlineto{\pgfqpoint{1.729948in}{0.606171in}}%
\pgfpathlineto{\pgfqpoint{1.731016in}{0.602353in}}%
\pgfpathlineto{\pgfqpoint{1.731549in}{0.613234in}}%
\pgfpathlineto{\pgfqpoint{1.732083in}{0.605592in}}%
\pgfpathlineto{\pgfqpoint{1.733684in}{0.603301in}}%
\pgfpathlineto{\pgfqpoint{1.735285in}{0.607606in}}%
\pgfpathlineto{\pgfqpoint{1.736886in}{0.611088in}}%
\pgfpathlineto{\pgfqpoint{1.737420in}{0.602992in}}%
\pgfpathlineto{\pgfqpoint{1.737954in}{0.604779in}}%
\pgfpathlineto{\pgfqpoint{1.739021in}{0.611838in}}%
\pgfpathlineto{\pgfqpoint{1.739555in}{0.607667in}}%
\pgfpathlineto{\pgfqpoint{1.740622in}{0.601039in}}%
\pgfpathlineto{\pgfqpoint{1.741156in}{0.602957in}}%
\pgfpathlineto{\pgfqpoint{1.741690in}{0.601082in}}%
\pgfpathlineto{\pgfqpoint{1.742223in}{0.601444in}}%
\pgfpathlineto{\pgfqpoint{1.743291in}{0.610867in}}%
\pgfpathlineto{\pgfqpoint{1.743824in}{0.601753in}}%
\pgfpathlineto{\pgfqpoint{1.744358in}{0.610098in}}%
\pgfpathlineto{\pgfqpoint{1.745425in}{0.602910in}}%
\pgfpathlineto{\pgfqpoint{1.745959in}{0.606830in}}%
\pgfpathlineto{\pgfqpoint{1.746493in}{0.606669in}}%
\pgfpathlineto{\pgfqpoint{1.747027in}{0.605886in}}%
\pgfpathlineto{\pgfqpoint{1.747560in}{0.613826in}}%
\pgfpathlineto{\pgfqpoint{1.748094in}{0.609128in}}%
\pgfpathlineto{\pgfqpoint{1.748628in}{0.611760in}}%
\pgfpathlineto{\pgfqpoint{1.750229in}{0.601871in}}%
\pgfpathlineto{\pgfqpoint{1.750763in}{0.609192in}}%
\pgfpathlineto{\pgfqpoint{1.751830in}{0.608928in}}%
\pgfpathlineto{\pgfqpoint{1.752364in}{0.605721in}}%
\pgfpathlineto{\pgfqpoint{1.753431in}{0.606036in}}%
\pgfpathlineto{\pgfqpoint{1.753965in}{0.606304in}}%
\pgfpathlineto{\pgfqpoint{1.754498in}{0.614106in}}%
\pgfpathlineto{\pgfqpoint{1.755032in}{0.612697in}}%
\pgfpathlineto{\pgfqpoint{1.755566in}{0.601522in}}%
\pgfpathlineto{\pgfqpoint{1.756100in}{0.608026in}}%
\pgfpathlineto{\pgfqpoint{1.756633in}{0.608753in}}%
\pgfpathlineto{\pgfqpoint{1.758234in}{0.605349in}}%
\pgfpathlineto{\pgfqpoint{1.758768in}{0.605783in}}%
\pgfpathlineto{\pgfqpoint{1.759302in}{0.612185in}}%
\pgfpathlineto{\pgfqpoint{1.759835in}{0.607609in}}%
\pgfpathlineto{\pgfqpoint{1.760369in}{0.604647in}}%
\pgfpathlineto{\pgfqpoint{1.760903in}{0.611489in}}%
\pgfpathlineto{\pgfqpoint{1.761437in}{0.604999in}}%
\pgfpathlineto{\pgfqpoint{1.761970in}{0.604138in}}%
\pgfpathlineto{\pgfqpoint{1.762504in}{0.607946in}}%
\pgfpathlineto{\pgfqpoint{1.763038in}{0.607316in}}%
\pgfpathlineto{\pgfqpoint{1.763571in}{0.604205in}}%
\pgfpathlineto{\pgfqpoint{1.764105in}{0.607872in}}%
\pgfpathlineto{\pgfqpoint{1.764639in}{0.602340in}}%
\pgfpathlineto{\pgfqpoint{1.765172in}{0.602671in}}%
\pgfpathlineto{\pgfqpoint{1.765706in}{0.608784in}}%
\pgfpathlineto{\pgfqpoint{1.766240in}{0.603568in}}%
\pgfpathlineto{\pgfqpoint{1.766774in}{0.606822in}}%
\pgfpathlineto{\pgfqpoint{1.767307in}{0.604052in}}%
\pgfpathlineto{\pgfqpoint{1.767841in}{0.600905in}}%
\pgfpathlineto{\pgfqpoint{1.768375in}{0.604058in}}%
\pgfpathlineto{\pgfqpoint{1.768908in}{0.610546in}}%
\pgfpathlineto{\pgfqpoint{1.769442in}{0.607874in}}%
\pgfpathlineto{\pgfqpoint{1.769976in}{0.604411in}}%
\pgfpathlineto{\pgfqpoint{1.770509in}{0.610618in}}%
\pgfpathlineto{\pgfqpoint{1.771043in}{0.602156in}}%
\pgfpathlineto{\pgfqpoint{1.771577in}{0.604106in}}%
\pgfpathlineto{\pgfqpoint{1.772644in}{0.605756in}}%
\pgfpathlineto{\pgfqpoint{1.773178in}{0.604272in}}%
\pgfpathlineto{\pgfqpoint{1.773712in}{0.607542in}}%
\pgfpathlineto{\pgfqpoint{1.774245in}{0.606355in}}%
\pgfpathlineto{\pgfqpoint{1.774779in}{0.606785in}}%
\pgfpathlineto{\pgfqpoint{1.775846in}{0.613275in}}%
\pgfpathlineto{\pgfqpoint{1.776914in}{0.604762in}}%
\pgfpathlineto{\pgfqpoint{1.777448in}{0.605266in}}%
\pgfpathlineto{\pgfqpoint{1.777981in}{0.607728in}}%
\pgfpathlineto{\pgfqpoint{1.778515in}{0.604589in}}%
\pgfpathlineto{\pgfqpoint{1.779049in}{0.605817in}}%
\pgfpathlineto{\pgfqpoint{1.780116in}{0.605096in}}%
\pgfpathlineto{\pgfqpoint{1.780650in}{0.612820in}}%
\pgfpathlineto{\pgfqpoint{1.781183in}{0.604663in}}%
\pgfpathlineto{\pgfqpoint{1.782251in}{0.613148in}}%
\pgfpathlineto{\pgfqpoint{1.783852in}{0.602438in}}%
\pgfpathlineto{\pgfqpoint{1.784919in}{0.609119in}}%
\pgfpathlineto{\pgfqpoint{1.785453in}{0.608861in}}%
\pgfpathlineto{\pgfqpoint{1.785987in}{0.604950in}}%
\pgfpathlineto{\pgfqpoint{1.787054in}{0.611883in}}%
\pgfpathlineto{\pgfqpoint{1.787588in}{0.602867in}}%
\pgfpathlineto{\pgfqpoint{1.788122in}{0.605662in}}%
\pgfpathlineto{\pgfqpoint{1.788655in}{0.605876in}}%
\pgfpathlineto{\pgfqpoint{1.789189in}{0.611336in}}%
\pgfpathlineto{\pgfqpoint{1.789723in}{0.605886in}}%
\pgfpathlineto{\pgfqpoint{1.790256in}{0.606237in}}%
\pgfpathlineto{\pgfqpoint{1.790790in}{0.605069in}}%
\pgfpathlineto{\pgfqpoint{1.791857in}{0.612078in}}%
\pgfpathlineto{\pgfqpoint{1.792391in}{0.604446in}}%
\pgfpathlineto{\pgfqpoint{1.792925in}{0.604610in}}%
\pgfpathlineto{\pgfqpoint{1.793459in}{0.613101in}}%
\pgfpathlineto{\pgfqpoint{1.793992in}{0.604588in}}%
\pgfpathlineto{\pgfqpoint{1.794526in}{0.607385in}}%
\pgfpathlineto{\pgfqpoint{1.795060in}{0.602001in}}%
\pgfpathlineto{\pgfqpoint{1.795593in}{0.603463in}}%
\pgfpathlineto{\pgfqpoint{1.796127in}{0.605773in}}%
\pgfpathlineto{\pgfqpoint{1.796661in}{0.604732in}}%
\pgfpathlineto{\pgfqpoint{1.797194in}{0.600146in}}%
\pgfpathlineto{\pgfqpoint{1.798796in}{0.609144in}}%
\pgfpathlineto{\pgfqpoint{1.799329in}{0.605817in}}%
\pgfpathlineto{\pgfqpoint{1.799863in}{0.609912in}}%
\pgfpathlineto{\pgfqpoint{1.800397in}{0.607837in}}%
\pgfpathlineto{\pgfqpoint{1.800930in}{0.606535in}}%
\pgfpathlineto{\pgfqpoint{1.801464in}{0.607773in}}%
\pgfpathlineto{\pgfqpoint{1.801998in}{0.607014in}}%
\pgfpathlineto{\pgfqpoint{1.802531in}{0.604269in}}%
\pgfpathlineto{\pgfqpoint{1.803065in}{0.607726in}}%
\pgfpathlineto{\pgfqpoint{1.803599in}{0.603148in}}%
\pgfpathlineto{\pgfqpoint{1.804133in}{0.608004in}}%
\pgfpathlineto{\pgfqpoint{1.804666in}{0.609164in}}%
\pgfpathlineto{\pgfqpoint{1.805200in}{0.620726in}}%
\pgfpathlineto{\pgfqpoint{1.805734in}{0.611461in}}%
\pgfpathlineto{\pgfqpoint{1.806267in}{0.603233in}}%
\pgfpathlineto{\pgfqpoint{1.806801in}{0.609693in}}%
\pgfpathlineto{\pgfqpoint{1.807335in}{0.610320in}}%
\pgfpathlineto{\pgfqpoint{1.808402in}{0.603182in}}%
\pgfpathlineto{\pgfqpoint{1.810003in}{0.619384in}}%
\pgfpathlineto{\pgfqpoint{1.810537in}{0.601023in}}%
\pgfpathlineto{\pgfqpoint{1.811071in}{0.613683in}}%
\pgfpathlineto{\pgfqpoint{1.811604in}{0.616579in}}%
\pgfpathlineto{\pgfqpoint{1.812138in}{0.602771in}}%
\pgfpathlineto{\pgfqpoint{1.812672in}{0.611107in}}%
\pgfpathlineto{\pgfqpoint{1.813205in}{0.607016in}}%
\pgfpathlineto{\pgfqpoint{1.813739in}{0.609117in}}%
\pgfpathlineto{\pgfqpoint{1.814807in}{0.607474in}}%
\pgfpathlineto{\pgfqpoint{1.815340in}{0.602540in}}%
\pgfpathlineto{\pgfqpoint{1.815874in}{0.610421in}}%
\pgfpathlineto{\pgfqpoint{1.816408in}{0.601304in}}%
\pgfpathlineto{\pgfqpoint{1.816941in}{0.603049in}}%
\pgfpathlineto{\pgfqpoint{1.818009in}{0.617765in}}%
\pgfpathlineto{\pgfqpoint{1.819076in}{0.603609in}}%
\pgfpathlineto{\pgfqpoint{1.819610in}{0.612607in}}%
\pgfpathlineto{\pgfqpoint{1.820144in}{0.605242in}}%
\pgfpathlineto{\pgfqpoint{1.821211in}{0.608042in}}%
\pgfpathlineto{\pgfqpoint{1.821745in}{0.603294in}}%
\pgfpathlineto{\pgfqpoint{1.822278in}{0.605604in}}%
\pgfpathlineto{\pgfqpoint{1.823346in}{0.609909in}}%
\pgfpathlineto{\pgfqpoint{1.824413in}{0.608103in}}%
\pgfpathlineto{\pgfqpoint{1.824947in}{0.601226in}}%
\pgfpathlineto{\pgfqpoint{1.825481in}{0.608888in}}%
\pgfpathlineto{\pgfqpoint{1.826548in}{0.608423in}}%
\pgfpathlineto{\pgfqpoint{1.827082in}{0.612244in}}%
\pgfpathlineto{\pgfqpoint{1.827615in}{0.608757in}}%
\pgfpathlineto{\pgfqpoint{1.828149in}{0.602963in}}%
\pgfpathlineto{\pgfqpoint{1.828683in}{0.605497in}}%
\pgfpathlineto{\pgfqpoint{1.829217in}{0.609176in}}%
\pgfpathlineto{\pgfqpoint{1.829750in}{0.603668in}}%
\pgfpathlineto{\pgfqpoint{1.830284in}{0.615488in}}%
\pgfpathlineto{\pgfqpoint{1.830818in}{0.609804in}}%
\pgfpathlineto{\pgfqpoint{1.831351in}{0.608760in}}%
\pgfpathlineto{\pgfqpoint{1.831885in}{0.603770in}}%
\pgfpathlineto{\pgfqpoint{1.832419in}{0.609674in}}%
\pgfpathlineto{\pgfqpoint{1.832952in}{0.606541in}}%
\pgfpathlineto{\pgfqpoint{1.834020in}{0.608082in}}%
\pgfpathlineto{\pgfqpoint{1.834554in}{0.601807in}}%
\pgfpathlineto{\pgfqpoint{1.835087in}{0.602156in}}%
\pgfpathlineto{\pgfqpoint{1.836155in}{0.615964in}}%
\pgfpathlineto{\pgfqpoint{1.836688in}{0.612931in}}%
\pgfpathlineto{\pgfqpoint{1.837222in}{0.612903in}}%
\pgfpathlineto{\pgfqpoint{1.837756in}{0.604432in}}%
\pgfpathlineto{\pgfqpoint{1.838289in}{0.609568in}}%
\pgfpathlineto{\pgfqpoint{1.838823in}{0.612108in}}%
\pgfpathlineto{\pgfqpoint{1.839357in}{0.610646in}}%
\pgfpathlineto{\pgfqpoint{1.840424in}{0.603980in}}%
\pgfpathlineto{\pgfqpoint{1.842025in}{0.613514in}}%
\pgfpathlineto{\pgfqpoint{1.843626in}{0.604997in}}%
\pgfpathlineto{\pgfqpoint{1.844160in}{0.605643in}}%
\pgfpathlineto{\pgfqpoint{1.844694in}{0.608722in}}%
\pgfpathlineto{\pgfqpoint{1.845228in}{0.602160in}}%
\pgfpathlineto{\pgfqpoint{1.845761in}{0.607016in}}%
\pgfpathlineto{\pgfqpoint{1.846295in}{0.606913in}}%
\pgfpathlineto{\pgfqpoint{1.846829in}{0.612610in}}%
\pgfpathlineto{\pgfqpoint{1.847362in}{0.606523in}}%
\pgfpathlineto{\pgfqpoint{1.848963in}{0.610819in}}%
\pgfpathlineto{\pgfqpoint{1.849497in}{0.608156in}}%
\pgfpathlineto{\pgfqpoint{1.850031in}{0.606832in}}%
\pgfpathlineto{\pgfqpoint{1.850565in}{0.607182in}}%
\pgfpathlineto{\pgfqpoint{1.851098in}{0.610316in}}%
\pgfpathlineto{\pgfqpoint{1.851632in}{0.600430in}}%
\pgfpathlineto{\pgfqpoint{1.852166in}{0.601752in}}%
\pgfpathlineto{\pgfqpoint{1.853767in}{0.608215in}}%
\pgfpathlineto{\pgfqpoint{1.854300in}{0.602101in}}%
\pgfpathlineto{\pgfqpoint{1.854834in}{0.611505in}}%
\pgfpathlineto{\pgfqpoint{1.855368in}{0.606757in}}%
\pgfpathlineto{\pgfqpoint{1.855902in}{0.608443in}}%
\pgfpathlineto{\pgfqpoint{1.856435in}{0.603953in}}%
\pgfpathlineto{\pgfqpoint{1.856969in}{0.605188in}}%
\pgfpathlineto{\pgfqpoint{1.857503in}{0.605272in}}%
\pgfpathlineto{\pgfqpoint{1.858570in}{0.608227in}}%
\pgfpathlineto{\pgfqpoint{1.859104in}{0.607874in}}%
\pgfpathlineto{\pgfqpoint{1.859637in}{0.608681in}}%
\pgfpathlineto{\pgfqpoint{1.860171in}{0.602289in}}%
\pgfpathlineto{\pgfqpoint{1.860705in}{0.603012in}}%
\pgfpathlineto{\pgfqpoint{1.861239in}{0.610634in}}%
\pgfpathlineto{\pgfqpoint{1.861772in}{0.610175in}}%
\pgfpathlineto{\pgfqpoint{1.863373in}{0.606810in}}%
\pgfpathlineto{\pgfqpoint{1.863907in}{0.617203in}}%
\pgfpathlineto{\pgfqpoint{1.864441in}{0.606336in}}%
\pgfpathlineto{\pgfqpoint{1.864974in}{0.607177in}}%
\pgfpathlineto{\pgfqpoint{1.866042in}{0.602622in}}%
\pgfpathlineto{\pgfqpoint{1.866576in}{0.618213in}}%
\pgfpathlineto{\pgfqpoint{1.867109in}{0.606708in}}%
\pgfpathlineto{\pgfqpoint{1.867643in}{0.609379in}}%
\pgfpathlineto{\pgfqpoint{1.868177in}{0.606659in}}%
\pgfpathlineto{\pgfqpoint{1.869244in}{0.603595in}}%
\pgfpathlineto{\pgfqpoint{1.869778in}{0.609322in}}%
\pgfpathlineto{\pgfqpoint{1.870311in}{0.607189in}}%
\pgfpathlineto{\pgfqpoint{1.871913in}{0.603545in}}%
\pgfpathlineto{\pgfqpoint{1.872980in}{0.609818in}}%
\pgfpathlineto{\pgfqpoint{1.874047in}{0.601755in}}%
\pgfpathlineto{\pgfqpoint{1.874581in}{0.602457in}}%
\pgfpathlineto{\pgfqpoint{1.875115in}{0.611010in}}%
\pgfpathlineto{\pgfqpoint{1.876182in}{0.601056in}}%
\pgfpathlineto{\pgfqpoint{1.876716in}{0.606740in}}%
\pgfpathlineto{\pgfqpoint{1.877250in}{0.601857in}}%
\pgfpathlineto{\pgfqpoint{1.878851in}{0.606102in}}%
\pgfpathlineto{\pgfqpoint{1.879384in}{0.606354in}}%
\pgfpathlineto{\pgfqpoint{1.879918in}{0.602849in}}%
\pgfpathlineto{\pgfqpoint{1.880452in}{0.607789in}}%
\pgfpathlineto{\pgfqpoint{1.880985in}{0.604959in}}%
\pgfpathlineto{\pgfqpoint{1.881519in}{0.606751in}}%
\pgfpathlineto{\pgfqpoint{1.882587in}{0.602091in}}%
\pgfpathlineto{\pgfqpoint{1.883120in}{0.604322in}}%
\pgfpathlineto{\pgfqpoint{1.883654in}{0.601707in}}%
\pgfpathlineto{\pgfqpoint{1.884188in}{0.601880in}}%
\pgfpathlineto{\pgfqpoint{1.884721in}{0.604674in}}%
\pgfpathlineto{\pgfqpoint{1.885255in}{0.603124in}}%
\pgfpathlineto{\pgfqpoint{1.885789in}{0.602289in}}%
\pgfpathlineto{\pgfqpoint{1.887390in}{0.609314in}}%
\pgfpathlineto{\pgfqpoint{1.888991in}{0.601572in}}%
\pgfpathlineto{\pgfqpoint{1.890058in}{0.604115in}}%
\pgfpathlineto{\pgfqpoint{1.890592in}{0.602485in}}%
\pgfpathlineto{\pgfqpoint{1.891660in}{0.602169in}}%
\pgfpathlineto{\pgfqpoint{1.892193in}{0.606265in}}%
\pgfpathlineto{\pgfqpoint{1.893261in}{0.602523in}}%
\pgfpathlineto{\pgfqpoint{1.893794in}{0.603609in}}%
\pgfpathlineto{\pgfqpoint{1.894328in}{0.601203in}}%
\pgfpathlineto{\pgfqpoint{1.894862in}{0.603072in}}%
\pgfpathlineto{\pgfqpoint{1.895929in}{0.604332in}}%
\pgfpathlineto{\pgfqpoint{1.896463in}{0.600575in}}%
\pgfpathlineto{\pgfqpoint{1.896997in}{0.607154in}}%
\pgfpathlineto{\pgfqpoint{1.897530in}{0.603854in}}%
\pgfpathlineto{\pgfqpoint{1.900199in}{0.600595in}}%
\pgfpathlineto{\pgfqpoint{1.901800in}{0.603503in}}%
\pgfpathlineto{\pgfqpoint{1.902867in}{0.600383in}}%
\pgfpathlineto{\pgfqpoint{1.903401in}{0.602819in}}%
\pgfpathlineto{\pgfqpoint{1.903935in}{0.602173in}}%
\pgfpathlineto{\pgfqpoint{1.904468in}{0.600460in}}%
\pgfpathlineto{\pgfqpoint{1.905002in}{0.603876in}}%
\pgfpathlineto{\pgfqpoint{1.905536in}{0.601198in}}%
\pgfpathlineto{\pgfqpoint{1.907137in}{0.601959in}}%
\pgfpathlineto{\pgfqpoint{1.918878in}{0.601113in}}%
\pgfpathlineto{\pgfqpoint{1.919412in}{0.601292in}}%
\pgfpathlineto{\pgfqpoint{1.919946in}{0.600034in}}%
\pgfpathlineto{\pgfqpoint{1.920479in}{0.601924in}}%
\pgfpathlineto{\pgfqpoint{1.921013in}{0.601657in}}%
\pgfpathlineto{\pgfqpoint{1.922614in}{0.600285in}}%
\pgfpathlineto{\pgfqpoint{1.923682in}{0.601713in}}%
\pgfpathlineto{\pgfqpoint{1.925283in}{0.601065in}}%
\pgfpathlineto{\pgfqpoint{1.929552in}{0.603299in}}%
\pgfpathlineto{\pgfqpoint{1.930620in}{0.603037in}}%
\pgfpathlineto{\pgfqpoint{1.931153in}{0.619419in}}%
\pgfpathlineto{\pgfqpoint{1.932221in}{0.606420in}}%
\pgfpathlineto{\pgfqpoint{1.932754in}{0.619059in}}%
\pgfpathlineto{\pgfqpoint{1.933288in}{0.601996in}}%
\pgfpathlineto{\pgfqpoint{1.934356in}{0.603364in}}%
\pgfpathlineto{\pgfqpoint{1.937024in}{0.601474in}}%
\pgfpathlineto{\pgfqpoint{1.939693in}{0.601172in}}%
\pgfpathlineto{\pgfqpoint{1.942361in}{0.602079in}}%
\pgfpathlineto{\pgfqpoint{1.943962in}{0.600868in}}%
\pgfpathlineto{\pgfqpoint{1.948232in}{0.600980in}}%
\pgfpathlineto{\pgfqpoint{1.948765in}{0.600088in}}%
\pgfpathlineto{\pgfqpoint{1.949299in}{0.601311in}}%
\pgfpathlineto{\pgfqpoint{1.950900in}{0.602339in}}%
\pgfpathlineto{\pgfqpoint{1.951434in}{0.601113in}}%
\pgfpathlineto{\pgfqpoint{1.951968in}{0.602092in}}%
\pgfpathlineto{\pgfqpoint{1.952501in}{0.602830in}}%
\pgfpathlineto{\pgfqpoint{1.953035in}{0.600720in}}%
\pgfpathlineto{\pgfqpoint{1.953569in}{0.601626in}}%
\pgfpathlineto{\pgfqpoint{1.954103in}{0.601247in}}%
\pgfpathlineto{\pgfqpoint{1.954636in}{0.602365in}}%
\pgfpathlineto{\pgfqpoint{1.955170in}{0.600541in}}%
\pgfpathlineto{\pgfqpoint{1.955704in}{0.601401in}}%
\pgfpathlineto{\pgfqpoint{1.956771in}{0.602646in}}%
\pgfpathlineto{\pgfqpoint{1.957305in}{0.600633in}}%
\pgfpathlineto{\pgfqpoint{1.957838in}{0.601783in}}%
\pgfpathlineto{\pgfqpoint{1.959973in}{0.603260in}}%
\pgfpathlineto{\pgfqpoint{1.960507in}{0.600845in}}%
\pgfpathlineto{\pgfqpoint{1.961041in}{0.602335in}}%
\pgfpathlineto{\pgfqpoint{1.962642in}{0.601320in}}%
\pgfpathlineto{\pgfqpoint{1.965310in}{0.601628in}}%
\pgfpathlineto{\pgfqpoint{1.965844in}{0.604605in}}%
\pgfpathlineto{\pgfqpoint{1.966378in}{0.600442in}}%
\pgfpathlineto{\pgfqpoint{1.966911in}{0.601148in}}%
\pgfpathlineto{\pgfqpoint{1.968512in}{0.603786in}}%
\pgfpathlineto{\pgfqpoint{1.969046in}{0.602359in}}%
\pgfpathlineto{\pgfqpoint{1.969580in}{0.602795in}}%
\pgfpathlineto{\pgfqpoint{1.970647in}{0.603959in}}%
\pgfpathlineto{\pgfqpoint{1.971181in}{0.600134in}}%
\pgfpathlineto{\pgfqpoint{1.971715in}{0.603906in}}%
\pgfpathlineto{\pgfqpoint{1.972248in}{0.601540in}}%
\pgfpathlineto{\pgfqpoint{1.972782in}{0.602125in}}%
\pgfpathlineto{\pgfqpoint{1.973316in}{0.603189in}}%
\pgfpathlineto{\pgfqpoint{1.973849in}{0.601468in}}%
\pgfpathlineto{\pgfqpoint{1.975451in}{0.607142in}}%
\pgfpathlineto{\pgfqpoint{1.975984in}{0.602695in}}%
\pgfpathlineto{\pgfqpoint{1.977052in}{0.611968in}}%
\pgfpathlineto{\pgfqpoint{1.978119in}{0.601890in}}%
\pgfpathlineto{\pgfqpoint{1.978653in}{0.602216in}}%
\pgfpathlineto{\pgfqpoint{1.979720in}{0.608011in}}%
\pgfpathlineto{\pgfqpoint{1.981855in}{0.603561in}}%
\pgfpathlineto{\pgfqpoint{1.983456in}{0.606523in}}%
\pgfpathlineto{\pgfqpoint{1.985591in}{0.604904in}}%
\pgfpathlineto{\pgfqpoint{1.986125in}{0.601968in}}%
\pgfpathlineto{\pgfqpoint{1.986658in}{0.604476in}}%
\pgfpathlineto{\pgfqpoint{1.987726in}{0.607291in}}%
\pgfpathlineto{\pgfqpoint{1.989327in}{0.601169in}}%
\pgfpathlineto{\pgfqpoint{1.990394in}{0.605408in}}%
\pgfpathlineto{\pgfqpoint{1.991995in}{0.602974in}}%
\pgfpathlineto{\pgfqpoint{1.992529in}{0.608630in}}%
\pgfpathlineto{\pgfqpoint{1.993063in}{0.607619in}}%
\pgfpathlineto{\pgfqpoint{1.994130in}{0.602913in}}%
\pgfpathlineto{\pgfqpoint{1.995197in}{0.602725in}}%
\pgfpathlineto{\pgfqpoint{1.995731in}{0.607133in}}%
\pgfpathlineto{\pgfqpoint{1.997332in}{0.601348in}}%
\pgfpathlineto{\pgfqpoint{1.997866in}{0.612350in}}%
\pgfpathlineto{\pgfqpoint{1.998400in}{0.601553in}}%
\pgfpathlineto{\pgfqpoint{2.000001in}{0.610779in}}%
\pgfpathlineto{\pgfqpoint{2.001068in}{0.606847in}}%
\pgfpathlineto{\pgfqpoint{2.001602in}{0.608793in}}%
\pgfpathlineto{\pgfqpoint{2.002136in}{0.612949in}}%
\pgfpathlineto{\pgfqpoint{2.002669in}{0.601615in}}%
\pgfpathlineto{\pgfqpoint{2.003203in}{0.607712in}}%
\pgfpathlineto{\pgfqpoint{2.003737in}{0.605270in}}%
\pgfpathlineto{\pgfqpoint{2.004270in}{0.607185in}}%
\pgfpathlineto{\pgfqpoint{2.005338in}{0.612195in}}%
\pgfpathlineto{\pgfqpoint{2.005871in}{0.603540in}}%
\pgfpathlineto{\pgfqpoint{2.006939in}{0.604221in}}%
\pgfpathlineto{\pgfqpoint{2.008006in}{0.612731in}}%
\pgfpathlineto{\pgfqpoint{2.008540in}{0.610752in}}%
\pgfpathlineto{\pgfqpoint{2.009074in}{0.601354in}}%
\pgfpathlineto{\pgfqpoint{2.009607in}{0.603135in}}%
\pgfpathlineto{\pgfqpoint{2.010141in}{0.608499in}}%
\pgfpathlineto{\pgfqpoint{2.010675in}{0.604561in}}%
\pgfpathlineto{\pgfqpoint{2.012276in}{0.612616in}}%
\pgfpathlineto{\pgfqpoint{2.013343in}{0.603532in}}%
\pgfpathlineto{\pgfqpoint{2.013877in}{0.603723in}}%
\pgfpathlineto{\pgfqpoint{2.014411in}{0.611740in}}%
\pgfpathlineto{\pgfqpoint{2.014944in}{0.604717in}}%
\pgfpathlineto{\pgfqpoint{2.015478in}{0.610917in}}%
\pgfpathlineto{\pgfqpoint{2.016012in}{0.607876in}}%
\pgfpathlineto{\pgfqpoint{2.016545in}{0.606793in}}%
\pgfpathlineto{\pgfqpoint{2.017079in}{0.600102in}}%
\pgfpathlineto{\pgfqpoint{2.017613in}{0.603410in}}%
\pgfpathlineto{\pgfqpoint{2.018147in}{0.602953in}}%
\pgfpathlineto{\pgfqpoint{2.018680in}{0.606245in}}%
\pgfpathlineto{\pgfqpoint{2.019214in}{0.604033in}}%
\pgfpathlineto{\pgfqpoint{2.019748in}{0.604191in}}%
\pgfpathlineto{\pgfqpoint{2.021349in}{0.613055in}}%
\pgfpathlineto{\pgfqpoint{2.021883in}{0.602776in}}%
\pgfpathlineto{\pgfqpoint{2.022416in}{0.605748in}}%
\pgfpathlineto{\pgfqpoint{2.023484in}{0.611677in}}%
\pgfpathlineto{\pgfqpoint{2.024551in}{0.603677in}}%
\pgfpathlineto{\pgfqpoint{2.025085in}{0.608819in}}%
\pgfpathlineto{\pgfqpoint{2.025618in}{0.603247in}}%
\pgfpathlineto{\pgfqpoint{2.026152in}{0.608256in}}%
\pgfpathlineto{\pgfqpoint{2.026686in}{0.604790in}}%
\pgfpathlineto{\pgfqpoint{2.027220in}{0.603620in}}%
\pgfpathlineto{\pgfqpoint{2.029354in}{0.609160in}}%
\pgfpathlineto{\pgfqpoint{2.029888in}{0.606473in}}%
\pgfpathlineto{\pgfqpoint{2.030422in}{0.610619in}}%
\pgfpathlineto{\pgfqpoint{2.030955in}{0.609253in}}%
\pgfpathlineto{\pgfqpoint{2.031489in}{0.606717in}}%
\pgfpathlineto{\pgfqpoint{2.032023in}{0.615822in}}%
\pgfpathlineto{\pgfqpoint{2.032557in}{0.612209in}}%
\pgfpathlineto{\pgfqpoint{2.033090in}{0.611563in}}%
\pgfpathlineto{\pgfqpoint{2.034158in}{0.618315in}}%
\pgfpathlineto{\pgfqpoint{2.034691in}{0.604525in}}%
\pgfpathlineto{\pgfqpoint{2.035225in}{0.613095in}}%
\pgfpathlineto{\pgfqpoint{2.035759in}{0.615334in}}%
\pgfpathlineto{\pgfqpoint{2.036292in}{0.614424in}}%
\pgfpathlineto{\pgfqpoint{2.036826in}{0.605576in}}%
\pgfpathlineto{\pgfqpoint{2.037360in}{0.614750in}}%
\pgfpathlineto{\pgfqpoint{2.037894in}{0.613965in}}%
\pgfpathlineto{\pgfqpoint{2.038427in}{0.619681in}}%
\pgfpathlineto{\pgfqpoint{2.039495in}{0.609862in}}%
\pgfpathlineto{\pgfqpoint{2.040028in}{0.610351in}}%
\pgfpathlineto{\pgfqpoint{2.040562in}{0.609046in}}%
\pgfpathlineto{\pgfqpoint{2.041096in}{0.616470in}}%
\pgfpathlineto{\pgfqpoint{2.041629in}{0.606602in}}%
\pgfpathlineto{\pgfqpoint{2.042163in}{0.611917in}}%
\pgfpathlineto{\pgfqpoint{2.043231in}{0.606223in}}%
\pgfpathlineto{\pgfqpoint{2.044832in}{0.619576in}}%
\pgfpathlineto{\pgfqpoint{2.045899in}{0.603042in}}%
\pgfpathlineto{\pgfqpoint{2.046433in}{0.605197in}}%
\pgfpathlineto{\pgfqpoint{2.048568in}{0.616474in}}%
\pgfpathlineto{\pgfqpoint{2.049101in}{0.603544in}}%
\pgfpathlineto{\pgfqpoint{2.049635in}{0.616411in}}%
\pgfpathlineto{\pgfqpoint{2.051236in}{0.607446in}}%
\pgfpathlineto{\pgfqpoint{2.051770in}{0.609462in}}%
\pgfpathlineto{\pgfqpoint{2.052303in}{0.605892in}}%
\pgfpathlineto{\pgfqpoint{2.052837in}{0.606088in}}%
\pgfpathlineto{\pgfqpoint{2.053905in}{0.612797in}}%
\pgfpathlineto{\pgfqpoint{2.054438in}{0.605315in}}%
\pgfpathlineto{\pgfqpoint{2.054972in}{0.620839in}}%
\pgfpathlineto{\pgfqpoint{2.055506in}{0.611579in}}%
\pgfpathlineto{\pgfqpoint{2.056039in}{0.609036in}}%
\pgfpathlineto{\pgfqpoint{2.057107in}{0.618301in}}%
\pgfpathlineto{\pgfqpoint{2.058708in}{0.603196in}}%
\pgfpathlineto{\pgfqpoint{2.060309in}{0.621356in}}%
\pgfpathlineto{\pgfqpoint{2.061376in}{0.615852in}}%
\pgfpathlineto{\pgfqpoint{2.061910in}{0.609076in}}%
\pgfpathlineto{\pgfqpoint{2.062444in}{0.613374in}}%
\pgfpathlineto{\pgfqpoint{2.062977in}{0.612870in}}%
\pgfpathlineto{\pgfqpoint{2.063511in}{0.617947in}}%
\pgfpathlineto{\pgfqpoint{2.064579in}{0.605883in}}%
\pgfpathlineto{\pgfqpoint{2.065112in}{0.610954in}}%
\pgfpathlineto{\pgfqpoint{2.065646in}{0.608534in}}%
\pgfpathlineto{\pgfqpoint{2.066180in}{0.604002in}}%
\pgfpathlineto{\pgfqpoint{2.067247in}{0.611585in}}%
\pgfpathlineto{\pgfqpoint{2.068848in}{0.602934in}}%
\pgfpathlineto{\pgfqpoint{2.070449in}{0.609692in}}%
\pgfpathlineto{\pgfqpoint{2.070983in}{0.601856in}}%
\pgfpathlineto{\pgfqpoint{2.071517in}{0.611312in}}%
\pgfpathlineto{\pgfqpoint{2.072050in}{0.607412in}}%
\pgfpathlineto{\pgfqpoint{2.073118in}{0.608963in}}%
\pgfpathlineto{\pgfqpoint{2.074185in}{0.602179in}}%
\pgfpathlineto{\pgfqpoint{2.076320in}{0.612844in}}%
\pgfpathlineto{\pgfqpoint{2.076854in}{0.607008in}}%
\pgfpathlineto{\pgfqpoint{2.077387in}{0.611579in}}%
\pgfpathlineto{\pgfqpoint{2.077921in}{0.607585in}}%
\pgfpathlineto{\pgfqpoint{2.078455in}{0.613850in}}%
\pgfpathlineto{\pgfqpoint{2.078988in}{0.603892in}}%
\pgfpathlineto{\pgfqpoint{2.079522in}{0.607659in}}%
\pgfpathlineto{\pgfqpoint{2.080056in}{0.609361in}}%
\pgfpathlineto{\pgfqpoint{2.081657in}{0.603938in}}%
\pgfpathlineto{\pgfqpoint{2.082724in}{0.602893in}}%
\pgfpathlineto{\pgfqpoint{2.084326in}{0.611825in}}%
\pgfpathlineto{\pgfqpoint{2.084859in}{0.602475in}}%
\pgfpathlineto{\pgfqpoint{2.085393in}{0.604383in}}%
\pgfpathlineto{\pgfqpoint{2.087528in}{0.616055in}}%
\pgfpathlineto{\pgfqpoint{2.088061in}{0.607022in}}%
\pgfpathlineto{\pgfqpoint{2.088595in}{0.608689in}}%
\pgfpathlineto{\pgfqpoint{2.090196in}{0.618092in}}%
\pgfpathlineto{\pgfqpoint{2.091797in}{0.606925in}}%
\pgfpathlineto{\pgfqpoint{2.092865in}{0.615991in}}%
\pgfpathlineto{\pgfqpoint{2.093398in}{0.602266in}}%
\pgfpathlineto{\pgfqpoint{2.093932in}{0.603786in}}%
\pgfpathlineto{\pgfqpoint{2.095000in}{0.621488in}}%
\pgfpathlineto{\pgfqpoint{2.095533in}{0.607351in}}%
\pgfpathlineto{\pgfqpoint{2.096067in}{0.614800in}}%
\pgfpathlineto{\pgfqpoint{2.097668in}{0.606109in}}%
\pgfpathlineto{\pgfqpoint{2.098202in}{0.605980in}}%
\pgfpathlineto{\pgfqpoint{2.098735in}{0.602749in}}%
\pgfpathlineto{\pgfqpoint{2.099269in}{0.608748in}}%
\pgfpathlineto{\pgfqpoint{2.099803in}{0.602545in}}%
\pgfpathlineto{\pgfqpoint{2.100337in}{0.602442in}}%
\pgfpathlineto{\pgfqpoint{2.101938in}{0.614548in}}%
\pgfpathlineto{\pgfqpoint{2.103005in}{0.603850in}}%
\pgfpathlineto{\pgfqpoint{2.103539in}{0.604305in}}%
\pgfpathlineto{\pgfqpoint{2.104606in}{0.609906in}}%
\pgfpathlineto{\pgfqpoint{2.105140in}{0.602378in}}%
\pgfpathlineto{\pgfqpoint{2.105674in}{0.614503in}}%
\pgfpathlineto{\pgfqpoint{2.106207in}{0.605381in}}%
\pgfpathlineto{\pgfqpoint{2.106741in}{0.605611in}}%
\pgfpathlineto{\pgfqpoint{2.107275in}{0.607252in}}%
\pgfpathlineto{\pgfqpoint{2.107808in}{0.612974in}}%
\pgfpathlineto{\pgfqpoint{2.108342in}{0.606371in}}%
\pgfpathlineto{\pgfqpoint{2.108876in}{0.607355in}}%
\pgfpathlineto{\pgfqpoint{2.111544in}{0.616276in}}%
\pgfpathlineto{\pgfqpoint{2.112078in}{0.601158in}}%
\pgfpathlineto{\pgfqpoint{2.112612in}{0.611039in}}%
\pgfpathlineto{\pgfqpoint{2.114213in}{0.601177in}}%
\pgfpathlineto{\pgfqpoint{2.115814in}{0.620641in}}%
\pgfpathlineto{\pgfqpoint{2.116348in}{0.604191in}}%
\pgfpathlineto{\pgfqpoint{2.116881in}{0.611321in}}%
\pgfpathlineto{\pgfqpoint{2.117415in}{0.610156in}}%
\pgfpathlineto{\pgfqpoint{2.117949in}{0.603908in}}%
\pgfpathlineto{\pgfqpoint{2.118482in}{0.616228in}}%
\pgfpathlineto{\pgfqpoint{2.119016in}{0.612123in}}%
\pgfpathlineto{\pgfqpoint{2.119550in}{0.602543in}}%
\pgfpathlineto{\pgfqpoint{2.120083in}{0.616217in}}%
\pgfpathlineto{\pgfqpoint{2.120617in}{0.604117in}}%
\pgfpathlineto{\pgfqpoint{2.121151in}{0.602758in}}%
\pgfpathlineto{\pgfqpoint{2.122218in}{0.613956in}}%
\pgfpathlineto{\pgfqpoint{2.122752in}{0.610248in}}%
\pgfpathlineto{\pgfqpoint{2.123286in}{0.601248in}}%
\pgfpathlineto{\pgfqpoint{2.123819in}{0.604439in}}%
\pgfpathlineto{\pgfqpoint{2.124353in}{0.606850in}}%
\pgfpathlineto{\pgfqpoint{2.124887in}{0.605039in}}%
\pgfpathlineto{\pgfqpoint{2.125420in}{0.606077in}}%
\pgfpathlineto{\pgfqpoint{2.125954in}{0.602471in}}%
\pgfpathlineto{\pgfqpoint{2.126488in}{0.605405in}}%
\pgfpathlineto{\pgfqpoint{2.127022in}{0.606784in}}%
\pgfpathlineto{\pgfqpoint{2.128089in}{0.601022in}}%
\pgfpathlineto{\pgfqpoint{2.129156in}{0.606449in}}%
\pgfpathlineto{\pgfqpoint{2.129690in}{0.603072in}}%
\pgfpathlineto{\pgfqpoint{2.130224in}{0.606727in}}%
\pgfpathlineto{\pgfqpoint{2.130757in}{0.605475in}}%
\pgfpathlineto{\pgfqpoint{2.131291in}{0.606773in}}%
\pgfpathlineto{\pgfqpoint{2.131825in}{0.611383in}}%
\pgfpathlineto{\pgfqpoint{2.132892in}{0.601692in}}%
\pgfpathlineto{\pgfqpoint{2.133426in}{0.607145in}}%
\pgfpathlineto{\pgfqpoint{2.133960in}{0.606367in}}%
\pgfpathlineto{\pgfqpoint{2.134493in}{0.603642in}}%
\pgfpathlineto{\pgfqpoint{2.135027in}{0.609226in}}%
\pgfpathlineto{\pgfqpoint{2.135561in}{0.606734in}}%
\pgfpathlineto{\pgfqpoint{2.137162in}{0.603214in}}%
\pgfpathlineto{\pgfqpoint{2.137696in}{0.601614in}}%
\pgfpathlineto{\pgfqpoint{2.138763in}{0.610931in}}%
\pgfpathlineto{\pgfqpoint{2.139297in}{0.606479in}}%
\pgfpathlineto{\pgfqpoint{2.139830in}{0.605521in}}%
\pgfpathlineto{\pgfqpoint{2.140364in}{0.608065in}}%
\pgfpathlineto{\pgfqpoint{2.140898in}{0.601314in}}%
\pgfpathlineto{\pgfqpoint{2.141431in}{0.601675in}}%
\pgfpathlineto{\pgfqpoint{2.143033in}{0.612937in}}%
\pgfpathlineto{\pgfqpoint{2.143566in}{0.602145in}}%
\pgfpathlineto{\pgfqpoint{2.144100in}{0.606865in}}%
\pgfpathlineto{\pgfqpoint{2.144634in}{0.613774in}}%
\pgfpathlineto{\pgfqpoint{2.145167in}{0.603471in}}%
\pgfpathlineto{\pgfqpoint{2.145701in}{0.613499in}}%
\pgfpathlineto{\pgfqpoint{2.146768in}{0.604717in}}%
\pgfpathlineto{\pgfqpoint{2.147302in}{0.612010in}}%
\pgfpathlineto{\pgfqpoint{2.148370in}{0.611703in}}%
\pgfpathlineto{\pgfqpoint{2.148903in}{0.611104in}}%
\pgfpathlineto{\pgfqpoint{2.149437in}{0.603491in}}%
\pgfpathlineto{\pgfqpoint{2.149971in}{0.613781in}}%
\pgfpathlineto{\pgfqpoint{2.150504in}{0.606698in}}%
\pgfpathlineto{\pgfqpoint{2.151038in}{0.605699in}}%
\pgfpathlineto{\pgfqpoint{2.151572in}{0.616591in}}%
\pgfpathlineto{\pgfqpoint{2.152106in}{0.615570in}}%
\pgfpathlineto{\pgfqpoint{2.152639in}{0.606406in}}%
\pgfpathlineto{\pgfqpoint{2.153173in}{0.610664in}}%
\pgfpathlineto{\pgfqpoint{2.153707in}{0.609768in}}%
\pgfpathlineto{\pgfqpoint{2.154240in}{0.602329in}}%
\pgfpathlineto{\pgfqpoint{2.154774in}{0.603171in}}%
\pgfpathlineto{\pgfqpoint{2.155308in}{0.604064in}}%
\pgfpathlineto{\pgfqpoint{2.155841in}{0.609327in}}%
\pgfpathlineto{\pgfqpoint{2.156375in}{0.600676in}}%
\pgfpathlineto{\pgfqpoint{2.156909in}{0.604008in}}%
\pgfpathlineto{\pgfqpoint{2.157443in}{0.602654in}}%
\pgfpathlineto{\pgfqpoint{2.158510in}{0.613755in}}%
\pgfpathlineto{\pgfqpoint{2.159044in}{0.602629in}}%
\pgfpathlineto{\pgfqpoint{2.159577in}{0.607948in}}%
\pgfpathlineto{\pgfqpoint{2.160111in}{0.605706in}}%
\pgfpathlineto{\pgfqpoint{2.161178in}{0.611338in}}%
\pgfpathlineto{\pgfqpoint{2.161712in}{0.603240in}}%
\pgfpathlineto{\pgfqpoint{2.162246in}{0.603349in}}%
\pgfpathlineto{\pgfqpoint{2.163847in}{0.606252in}}%
\pgfpathlineto{\pgfqpoint{2.164381in}{0.603129in}}%
\pgfpathlineto{\pgfqpoint{2.165982in}{0.616259in}}%
\pgfpathlineto{\pgfqpoint{2.166515in}{0.600613in}}%
\pgfpathlineto{\pgfqpoint{2.167049in}{0.609282in}}%
\pgfpathlineto{\pgfqpoint{2.167583in}{0.603406in}}%
\pgfpathlineto{\pgfqpoint{2.168117in}{0.615144in}}%
\pgfpathlineto{\pgfqpoint{2.168650in}{0.606935in}}%
\pgfpathlineto{\pgfqpoint{2.170785in}{0.614150in}}%
\pgfpathlineto{\pgfqpoint{2.171319in}{0.605161in}}%
\pgfpathlineto{\pgfqpoint{2.171852in}{0.606971in}}%
\pgfpathlineto{\pgfqpoint{2.172386in}{0.609185in}}%
\pgfpathlineto{\pgfqpoint{2.172920in}{0.608615in}}%
\pgfpathlineto{\pgfqpoint{2.173987in}{0.607509in}}%
\pgfpathlineto{\pgfqpoint{2.174521in}{0.603112in}}%
\pgfpathlineto{\pgfqpoint{2.175055in}{0.613481in}}%
\pgfpathlineto{\pgfqpoint{2.175588in}{0.610096in}}%
\pgfpathlineto{\pgfqpoint{2.176122in}{0.609213in}}%
\pgfpathlineto{\pgfqpoint{2.176656in}{0.604582in}}%
\pgfpathlineto{\pgfqpoint{2.177189in}{0.607239in}}%
\pgfpathlineto{\pgfqpoint{2.177723in}{0.610960in}}%
\pgfpathlineto{\pgfqpoint{2.179324in}{0.601262in}}%
\pgfpathlineto{\pgfqpoint{2.180925in}{0.605753in}}%
\pgfpathlineto{\pgfqpoint{2.182526in}{0.601614in}}%
\pgfpathlineto{\pgfqpoint{2.183060in}{0.602378in}}%
\pgfpathlineto{\pgfqpoint{2.183594in}{0.606951in}}%
\pgfpathlineto{\pgfqpoint{2.184128in}{0.603732in}}%
\pgfpathlineto{\pgfqpoint{2.184661in}{0.602032in}}%
\pgfpathlineto{\pgfqpoint{2.185195in}{0.603918in}}%
\pgfpathlineto{\pgfqpoint{2.187330in}{0.605188in}}%
\pgfpathlineto{\pgfqpoint{2.187863in}{0.603054in}}%
\pgfpathlineto{\pgfqpoint{2.188931in}{0.606693in}}%
\pgfpathlineto{\pgfqpoint{2.189465in}{0.601889in}}%
\pgfpathlineto{\pgfqpoint{2.189998in}{0.606086in}}%
\pgfpathlineto{\pgfqpoint{2.191066in}{0.605077in}}%
\pgfpathlineto{\pgfqpoint{2.191599in}{0.602684in}}%
\pgfpathlineto{\pgfqpoint{2.192133in}{0.604163in}}%
\pgfpathlineto{\pgfqpoint{2.192667in}{0.606317in}}%
\pgfpathlineto{\pgfqpoint{2.193200in}{0.602449in}}%
\pgfpathlineto{\pgfqpoint{2.193734in}{0.607780in}}%
\pgfpathlineto{\pgfqpoint{2.194268in}{0.603648in}}%
\pgfpathlineto{\pgfqpoint{2.195335in}{0.608638in}}%
\pgfpathlineto{\pgfqpoint{2.196936in}{0.601722in}}%
\pgfpathlineto{\pgfqpoint{2.197470in}{0.608174in}}%
\pgfpathlineto{\pgfqpoint{2.198004in}{0.607359in}}%
\pgfpathlineto{\pgfqpoint{2.198537in}{0.602644in}}%
\pgfpathlineto{\pgfqpoint{2.199071in}{0.604093in}}%
\pgfpathlineto{\pgfqpoint{2.199605in}{0.606881in}}%
\pgfpathlineto{\pgfqpoint{2.200139in}{0.603763in}}%
\pgfpathlineto{\pgfqpoint{2.201206in}{0.603618in}}%
\pgfpathlineto{\pgfqpoint{2.202807in}{0.602094in}}%
\pgfpathlineto{\pgfqpoint{2.204942in}{0.600599in}}%
\pgfpathlineto{\pgfqpoint{2.206543in}{0.601389in}}%
\pgfpathlineto{\pgfqpoint{2.211346in}{0.601910in}}%
\pgfpathlineto{\pgfqpoint{2.211880in}{0.601086in}}%
\pgfpathlineto{\pgfqpoint{2.212947in}{0.605787in}}%
\pgfpathlineto{\pgfqpoint{2.213481in}{0.600539in}}%
\pgfpathlineto{\pgfqpoint{2.214015in}{0.603182in}}%
\pgfpathlineto{\pgfqpoint{2.214548in}{0.608632in}}%
\pgfpathlineto{\pgfqpoint{2.215082in}{0.605880in}}%
\pgfpathlineto{\pgfqpoint{2.216150in}{0.603107in}}%
\pgfpathlineto{\pgfqpoint{2.216683in}{0.603283in}}%
\pgfpathlineto{\pgfqpoint{2.217751in}{0.608368in}}%
\pgfpathlineto{\pgfqpoint{2.218284in}{0.606794in}}%
\pgfpathlineto{\pgfqpoint{2.218818in}{0.600787in}}%
\pgfpathlineto{\pgfqpoint{2.219352in}{0.610442in}}%
\pgfpathlineto{\pgfqpoint{2.219886in}{0.607045in}}%
\pgfpathlineto{\pgfqpoint{2.220419in}{0.606187in}}%
\pgfpathlineto{\pgfqpoint{2.220953in}{0.608416in}}%
\pgfpathlineto{\pgfqpoint{2.222020in}{0.601804in}}%
\pgfpathlineto{\pgfqpoint{2.222554in}{0.604775in}}%
\pgfpathlineto{\pgfqpoint{2.223088in}{0.607849in}}%
\pgfpathlineto{\pgfqpoint{2.223621in}{0.605035in}}%
\pgfpathlineto{\pgfqpoint{2.224155in}{0.605728in}}%
\pgfpathlineto{\pgfqpoint{2.224689in}{0.601471in}}%
\pgfpathlineto{\pgfqpoint{2.225223in}{0.601748in}}%
\pgfpathlineto{\pgfqpoint{2.225756in}{0.612195in}}%
\pgfpathlineto{\pgfqpoint{2.226290in}{0.602214in}}%
\pgfpathlineto{\pgfqpoint{2.226824in}{0.601424in}}%
\pgfpathlineto{\pgfqpoint{2.227891in}{0.607550in}}%
\pgfpathlineto{\pgfqpoint{2.228425in}{0.605210in}}%
\pgfpathlineto{\pgfqpoint{2.228958in}{0.602779in}}%
\pgfpathlineto{\pgfqpoint{2.230026in}{0.602581in}}%
\pgfpathlineto{\pgfqpoint{2.230560in}{0.615814in}}%
\pgfpathlineto{\pgfqpoint{2.232161in}{0.601520in}}%
\pgfpathlineto{\pgfqpoint{2.232694in}{0.608480in}}%
\pgfpathlineto{\pgfqpoint{2.233228in}{0.604850in}}%
\pgfpathlineto{\pgfqpoint{2.234829in}{0.600852in}}%
\pgfpathlineto{\pgfqpoint{2.238031in}{0.607110in}}%
\pgfpathlineto{\pgfqpoint{2.238565in}{0.602911in}}%
\pgfpathlineto{\pgfqpoint{2.239632in}{0.603076in}}%
\pgfpathlineto{\pgfqpoint{2.241234in}{0.600011in}}%
\pgfpathlineto{\pgfqpoint{2.241767in}{0.608200in}}%
\pgfpathlineto{\pgfqpoint{2.242301in}{0.605223in}}%
\pgfpathlineto{\pgfqpoint{2.242835in}{0.602203in}}%
\pgfpathlineto{\pgfqpoint{2.243368in}{0.602839in}}%
\pgfpathlineto{\pgfqpoint{2.243902in}{0.606061in}}%
\pgfpathlineto{\pgfqpoint{2.244969in}{0.600949in}}%
\pgfpathlineto{\pgfqpoint{2.246037in}{0.606653in}}%
\pgfpathlineto{\pgfqpoint{2.246571in}{0.601864in}}%
\pgfpathlineto{\pgfqpoint{2.247104in}{0.604003in}}%
\pgfpathlineto{\pgfqpoint{2.248172in}{0.603073in}}%
\pgfpathlineto{\pgfqpoint{2.248705in}{0.605168in}}%
\pgfpathlineto{\pgfqpoint{2.249239in}{0.604484in}}%
\pgfpathlineto{\pgfqpoint{2.251374in}{0.601467in}}%
\pgfpathlineto{\pgfqpoint{2.252441in}{0.604014in}}%
\pgfpathlineto{\pgfqpoint{2.252975in}{0.601829in}}%
\pgfpathlineto{\pgfqpoint{2.253509in}{0.603400in}}%
\pgfpathlineto{\pgfqpoint{2.254042in}{0.603409in}}%
\pgfpathlineto{\pgfqpoint{2.255110in}{0.610846in}}%
\pgfpathlineto{\pgfqpoint{2.255643in}{0.605751in}}%
\pgfpathlineto{\pgfqpoint{2.256711in}{0.607345in}}%
\pgfpathlineto{\pgfqpoint{2.257245in}{0.603454in}}%
\pgfpathlineto{\pgfqpoint{2.257778in}{0.612221in}}%
\pgfpathlineto{\pgfqpoint{2.258312in}{0.606297in}}%
\pgfpathlineto{\pgfqpoint{2.259913in}{0.601845in}}%
\pgfpathlineto{\pgfqpoint{2.260447in}{0.612186in}}%
\pgfpathlineto{\pgfqpoint{2.260980in}{0.606709in}}%
\pgfpathlineto{\pgfqpoint{2.261514in}{0.607944in}}%
\pgfpathlineto{\pgfqpoint{2.262048in}{0.605734in}}%
\pgfpathlineto{\pgfqpoint{2.262582in}{0.611117in}}%
\pgfpathlineto{\pgfqpoint{2.263115in}{0.605553in}}%
\pgfpathlineto{\pgfqpoint{2.263649in}{0.602001in}}%
\pgfpathlineto{\pgfqpoint{2.264183in}{0.603520in}}%
\pgfpathlineto{\pgfqpoint{2.264716in}{0.603014in}}%
\pgfpathlineto{\pgfqpoint{2.265250in}{0.603942in}}%
\pgfpathlineto{\pgfqpoint{2.265784in}{0.606556in}}%
\pgfpathlineto{\pgfqpoint{2.266317in}{0.605859in}}%
\pgfpathlineto{\pgfqpoint{2.266851in}{0.604357in}}%
\pgfpathlineto{\pgfqpoint{2.267385in}{0.607478in}}%
\pgfpathlineto{\pgfqpoint{2.268986in}{0.602168in}}%
\pgfpathlineto{\pgfqpoint{2.270053in}{0.607981in}}%
\pgfpathlineto{\pgfqpoint{2.270587in}{0.605327in}}%
\pgfpathlineto{\pgfqpoint{2.271121in}{0.605858in}}%
\pgfpathlineto{\pgfqpoint{2.272188in}{0.602105in}}%
\pgfpathlineto{\pgfqpoint{2.272722in}{0.603054in}}%
\pgfpathlineto{\pgfqpoint{2.273256in}{0.604277in}}%
\pgfpathlineto{\pgfqpoint{2.273789in}{0.602597in}}%
\pgfpathlineto{\pgfqpoint{2.274857in}{0.608162in}}%
\pgfpathlineto{\pgfqpoint{2.275390in}{0.603569in}}%
\pgfpathlineto{\pgfqpoint{2.275924in}{0.610907in}}%
\pgfpathlineto{\pgfqpoint{2.276991in}{0.601065in}}%
\pgfpathlineto{\pgfqpoint{2.277525in}{0.602290in}}%
\pgfpathlineto{\pgfqpoint{2.278059in}{0.602377in}}%
\pgfpathlineto{\pgfqpoint{2.278593in}{0.605584in}}%
\pgfpathlineto{\pgfqpoint{2.279126in}{0.604288in}}%
\pgfpathlineto{\pgfqpoint{2.279660in}{0.602579in}}%
\pgfpathlineto{\pgfqpoint{2.280727in}{0.608667in}}%
\pgfpathlineto{\pgfqpoint{2.281261in}{0.604940in}}%
\pgfpathlineto{\pgfqpoint{2.282862in}{0.600304in}}%
\pgfpathlineto{\pgfqpoint{2.283396in}{0.606793in}}%
\pgfpathlineto{\pgfqpoint{2.283930in}{0.603705in}}%
\pgfpathlineto{\pgfqpoint{2.285531in}{0.606487in}}%
\pgfpathlineto{\pgfqpoint{2.286598in}{0.602924in}}%
\pgfpathlineto{\pgfqpoint{2.287132in}{0.605298in}}%
\pgfpathlineto{\pgfqpoint{2.288199in}{0.610021in}}%
\pgfpathlineto{\pgfqpoint{2.288733in}{0.600963in}}%
\pgfpathlineto{\pgfqpoint{2.289267in}{0.601888in}}%
\pgfpathlineto{\pgfqpoint{2.290334in}{0.603232in}}%
\pgfpathlineto{\pgfqpoint{2.290868in}{0.601008in}}%
\pgfpathlineto{\pgfqpoint{2.291401in}{0.604398in}}%
\pgfpathlineto{\pgfqpoint{2.291935in}{0.602969in}}%
\pgfpathlineto{\pgfqpoint{2.292469in}{0.600634in}}%
\pgfpathlineto{\pgfqpoint{2.293003in}{0.601793in}}%
\pgfpathlineto{\pgfqpoint{2.294070in}{0.603352in}}%
\pgfpathlineto{\pgfqpoint{2.295137in}{0.602112in}}%
\pgfpathlineto{\pgfqpoint{2.295671in}{0.602856in}}%
\pgfpathlineto{\pgfqpoint{2.297272in}{0.605743in}}%
\pgfpathlineto{\pgfqpoint{2.297806in}{0.605371in}}%
\pgfpathlineto{\pgfqpoint{2.298340in}{0.601016in}}%
\pgfpathlineto{\pgfqpoint{2.298873in}{0.602321in}}%
\pgfpathlineto{\pgfqpoint{2.301008in}{0.606147in}}%
\pgfpathlineto{\pgfqpoint{2.302075in}{0.600966in}}%
\pgfpathlineto{\pgfqpoint{2.302609in}{0.604522in}}%
\pgfpathlineto{\pgfqpoint{2.303143in}{0.600382in}}%
\pgfpathlineto{\pgfqpoint{2.303677in}{0.602471in}}%
\pgfpathlineto{\pgfqpoint{2.305278in}{0.601272in}}%
\pgfpathlineto{\pgfqpoint{2.305811in}{0.605702in}}%
\pgfpathlineto{\pgfqpoint{2.306345in}{0.602220in}}%
\pgfpathlineto{\pgfqpoint{2.306879in}{0.603820in}}%
\pgfpathlineto{\pgfqpoint{2.307412in}{0.601900in}}%
\pgfpathlineto{\pgfqpoint{2.307946in}{0.605314in}}%
\pgfpathlineto{\pgfqpoint{2.308480in}{0.602257in}}%
\pgfpathlineto{\pgfqpoint{2.309014in}{0.601652in}}%
\pgfpathlineto{\pgfqpoint{2.309547in}{0.602778in}}%
\pgfpathlineto{\pgfqpoint{2.310081in}{0.607600in}}%
\pgfpathlineto{\pgfqpoint{2.310615in}{0.604174in}}%
\pgfpathlineto{\pgfqpoint{2.311148in}{0.605900in}}%
\pgfpathlineto{\pgfqpoint{2.311682in}{0.605624in}}%
\pgfpathlineto{\pgfqpoint{2.314884in}{0.601352in}}%
\pgfpathlineto{\pgfqpoint{2.315418in}{0.607392in}}%
\pgfpathlineto{\pgfqpoint{2.315952in}{0.607011in}}%
\pgfpathlineto{\pgfqpoint{2.317019in}{0.602510in}}%
\pgfpathlineto{\pgfqpoint{2.317553in}{0.603999in}}%
\pgfpathlineto{\pgfqpoint{2.318086in}{0.602719in}}%
\pgfpathlineto{\pgfqpoint{2.318620in}{0.605735in}}%
\pgfpathlineto{\pgfqpoint{2.319154in}{0.602935in}}%
\pgfpathlineto{\pgfqpoint{2.319688in}{0.604192in}}%
\pgfpathlineto{\pgfqpoint{2.321289in}{0.602619in}}%
\pgfpathlineto{\pgfqpoint{2.322356in}{0.600653in}}%
\pgfpathlineto{\pgfqpoint{2.322890in}{0.601264in}}%
\pgfpathlineto{\pgfqpoint{2.323423in}{0.605519in}}%
\pgfpathlineto{\pgfqpoint{2.323957in}{0.600246in}}%
\pgfpathlineto{\pgfqpoint{2.324491in}{0.605500in}}%
\pgfpathlineto{\pgfqpoint{2.325025in}{0.606415in}}%
\pgfpathlineto{\pgfqpoint{2.325558in}{0.602475in}}%
\pgfpathlineto{\pgfqpoint{2.326092in}{0.604222in}}%
\pgfpathlineto{\pgfqpoint{2.326626in}{0.607443in}}%
\pgfpathlineto{\pgfqpoint{2.327159in}{0.606525in}}%
\pgfpathlineto{\pgfqpoint{2.327693in}{0.606758in}}%
\pgfpathlineto{\pgfqpoint{2.328227in}{0.600936in}}%
\pgfpathlineto{\pgfqpoint{2.328760in}{0.603603in}}%
\pgfpathlineto{\pgfqpoint{2.329294in}{0.606216in}}%
\pgfpathlineto{\pgfqpoint{2.329828in}{0.604181in}}%
\pgfpathlineto{\pgfqpoint{2.330362in}{0.600933in}}%
\pgfpathlineto{\pgfqpoint{2.330895in}{0.605289in}}%
\pgfpathlineto{\pgfqpoint{2.331429in}{0.602008in}}%
\pgfpathlineto{\pgfqpoint{2.331963in}{0.600395in}}%
\pgfpathlineto{\pgfqpoint{2.332496in}{0.601397in}}%
\pgfpathlineto{\pgfqpoint{2.333030in}{0.607029in}}%
\pgfpathlineto{\pgfqpoint{2.333564in}{0.602520in}}%
\pgfpathlineto{\pgfqpoint{2.334097in}{0.605748in}}%
\pgfpathlineto{\pgfqpoint{2.334631in}{0.600122in}}%
\pgfpathlineto{\pgfqpoint{2.335165in}{0.605669in}}%
\pgfpathlineto{\pgfqpoint{2.335699in}{0.600998in}}%
\pgfpathlineto{\pgfqpoint{2.336232in}{0.606816in}}%
\pgfpathlineto{\pgfqpoint{2.336766in}{0.603707in}}%
\pgfpathlineto{\pgfqpoint{2.337300in}{0.604261in}}%
\pgfpathlineto{\pgfqpoint{2.338367in}{0.601729in}}%
\pgfpathlineto{\pgfqpoint{2.338901in}{0.603096in}}%
\pgfpathlineto{\pgfqpoint{2.339434in}{0.602197in}}%
\pgfpathlineto{\pgfqpoint{2.339968in}{0.601871in}}%
\pgfpathlineto{\pgfqpoint{2.341036in}{0.605588in}}%
\pgfpathlineto{\pgfqpoint{2.341569in}{0.604045in}}%
\pgfpathlineto{\pgfqpoint{2.342103in}{0.604817in}}%
\pgfpathlineto{\pgfqpoint{2.342637in}{0.602259in}}%
\pgfpathlineto{\pgfqpoint{2.343170in}{0.604657in}}%
\pgfpathlineto{\pgfqpoint{2.343704in}{0.604194in}}%
\pgfpathlineto{\pgfqpoint{2.345305in}{0.601621in}}%
\pgfpathlineto{\pgfqpoint{2.345839in}{0.601314in}}%
\pgfpathlineto{\pgfqpoint{2.346906in}{0.605871in}}%
\pgfpathlineto{\pgfqpoint{2.348507in}{0.601265in}}%
\pgfpathlineto{\pgfqpoint{2.349575in}{0.600246in}}%
\pgfpathlineto{\pgfqpoint{2.350108in}{0.604288in}}%
\pgfpathlineto{\pgfqpoint{2.350642in}{0.601631in}}%
\pgfpathlineto{\pgfqpoint{2.352243in}{0.604308in}}%
\pgfpathlineto{\pgfqpoint{2.353311in}{0.602466in}}%
\pgfpathlineto{\pgfqpoint{2.354378in}{0.605452in}}%
\pgfpathlineto{\pgfqpoint{2.355446in}{0.600949in}}%
\pgfpathlineto{\pgfqpoint{2.355979in}{0.606600in}}%
\pgfpathlineto{\pgfqpoint{2.356513in}{0.601539in}}%
\pgfpathlineto{\pgfqpoint{2.358114in}{0.602159in}}%
\pgfpathlineto{\pgfqpoint{2.359181in}{0.602243in}}%
\pgfpathlineto{\pgfqpoint{2.360249in}{0.603786in}}%
\pgfpathlineto{\pgfqpoint{2.360783in}{0.602848in}}%
\pgfpathlineto{\pgfqpoint{2.361850in}{0.602140in}}%
\pgfpathlineto{\pgfqpoint{2.362384in}{0.604523in}}%
\pgfpathlineto{\pgfqpoint{2.362917in}{0.600820in}}%
\pgfpathlineto{\pgfqpoint{2.363451in}{0.602464in}}%
\pgfpathlineto{\pgfqpoint{2.363985in}{0.601637in}}%
\pgfpathlineto{\pgfqpoint{2.364518in}{0.602303in}}%
\pgfpathlineto{\pgfqpoint{2.365052in}{0.602226in}}%
\pgfpathlineto{\pgfqpoint{2.366120in}{0.603962in}}%
\pgfpathlineto{\pgfqpoint{2.367721in}{0.600701in}}%
\pgfpathlineto{\pgfqpoint{2.368254in}{0.606172in}}%
\pgfpathlineto{\pgfqpoint{2.368788in}{0.601344in}}%
\pgfpathlineto{\pgfqpoint{2.369855in}{0.600298in}}%
\pgfpathlineto{\pgfqpoint{2.370923in}{0.604896in}}%
\pgfpathlineto{\pgfqpoint{2.371457in}{0.602998in}}%
\pgfpathlineto{\pgfqpoint{2.371990in}{0.604766in}}%
\pgfpathlineto{\pgfqpoint{2.372524in}{0.604091in}}%
\pgfpathlineto{\pgfqpoint{2.373058in}{0.604950in}}%
\pgfpathlineto{\pgfqpoint{2.373591in}{0.604990in}}%
\pgfpathlineto{\pgfqpoint{2.374125in}{0.600950in}}%
\pgfpathlineto{\pgfqpoint{2.374659in}{0.601398in}}%
\pgfpathlineto{\pgfqpoint{2.375726in}{0.602450in}}%
\pgfpathlineto{\pgfqpoint{2.376260in}{0.601479in}}%
\pgfpathlineto{\pgfqpoint{2.376794in}{0.605043in}}%
\pgfpathlineto{\pgfqpoint{2.377327in}{0.602724in}}%
\pgfpathlineto{\pgfqpoint{2.378928in}{0.604868in}}%
\pgfpathlineto{\pgfqpoint{2.379462in}{0.602685in}}%
\pgfpathlineto{\pgfqpoint{2.379996in}{0.605167in}}%
\pgfpathlineto{\pgfqpoint{2.380529in}{0.610042in}}%
\pgfpathlineto{\pgfqpoint{2.381063in}{0.601255in}}%
\pgfpathlineto{\pgfqpoint{2.382131in}{0.601621in}}%
\pgfpathlineto{\pgfqpoint{2.383732in}{0.611379in}}%
\pgfpathlineto{\pgfqpoint{2.384265in}{0.602041in}}%
\pgfpathlineto{\pgfqpoint{2.384799in}{0.612677in}}%
\pgfpathlineto{\pgfqpoint{2.385333in}{0.603835in}}%
\pgfpathlineto{\pgfqpoint{2.385866in}{0.603112in}}%
\pgfpathlineto{\pgfqpoint{2.386400in}{0.606571in}}%
\pgfpathlineto{\pgfqpoint{2.386934in}{0.601827in}}%
\pgfpathlineto{\pgfqpoint{2.387468in}{0.605126in}}%
\pgfpathlineto{\pgfqpoint{2.388001in}{0.605580in}}%
\pgfpathlineto{\pgfqpoint{2.388535in}{0.604047in}}%
\pgfpathlineto{\pgfqpoint{2.389069in}{0.609604in}}%
\pgfpathlineto{\pgfqpoint{2.389602in}{0.602626in}}%
\pgfpathlineto{\pgfqpoint{2.390136in}{0.602947in}}%
\pgfpathlineto{\pgfqpoint{2.391203in}{0.605156in}}%
\pgfpathlineto{\pgfqpoint{2.392271in}{0.603501in}}%
\pgfpathlineto{\pgfqpoint{2.392805in}{0.604981in}}%
\pgfpathlineto{\pgfqpoint{2.393872in}{0.607177in}}%
\pgfpathlineto{\pgfqpoint{2.394939in}{0.603716in}}%
\pgfpathlineto{\pgfqpoint{2.395473in}{0.605699in}}%
\pgfpathlineto{\pgfqpoint{2.396007in}{0.604258in}}%
\pgfpathlineto{\pgfqpoint{2.397074in}{0.606225in}}%
\pgfpathlineto{\pgfqpoint{2.397608in}{0.605164in}}%
\pgfpathlineto{\pgfqpoint{2.398142in}{0.601850in}}%
\pgfpathlineto{\pgfqpoint{2.398675in}{0.602500in}}%
\pgfpathlineto{\pgfqpoint{2.399209in}{0.607646in}}%
\pgfpathlineto{\pgfqpoint{2.399743in}{0.606936in}}%
\pgfpathlineto{\pgfqpoint{2.400276in}{0.606330in}}%
\pgfpathlineto{\pgfqpoint{2.401877in}{0.601672in}}%
\pgfpathlineto{\pgfqpoint{2.402411in}{0.602495in}}%
\pgfpathlineto{\pgfqpoint{2.403479in}{0.607485in}}%
\pgfpathlineto{\pgfqpoint{2.404012in}{0.600263in}}%
\pgfpathlineto{\pgfqpoint{2.404546in}{0.602513in}}%
\pgfpathlineto{\pgfqpoint{2.405613in}{0.605192in}}%
\pgfpathlineto{\pgfqpoint{2.406681in}{0.601168in}}%
\pgfpathlineto{\pgfqpoint{2.407214in}{0.602693in}}%
\pgfpathlineto{\pgfqpoint{2.407748in}{0.601597in}}%
\pgfpathlineto{\pgfqpoint{2.409349in}{0.607283in}}%
\pgfpathlineto{\pgfqpoint{2.410417in}{0.600987in}}%
\pgfpathlineto{\pgfqpoint{2.410950in}{0.601963in}}%
\pgfpathlineto{\pgfqpoint{2.411484in}{0.602993in}}%
\pgfpathlineto{\pgfqpoint{2.412018in}{0.601682in}}%
\pgfpathlineto{\pgfqpoint{2.414153in}{0.606159in}}%
\pgfpathlineto{\pgfqpoint{2.415220in}{0.602790in}}%
\pgfpathlineto{\pgfqpoint{2.416287in}{0.606229in}}%
\pgfpathlineto{\pgfqpoint{2.417888in}{0.603764in}}%
\pgfpathlineto{\pgfqpoint{2.418422in}{0.604307in}}%
\pgfpathlineto{\pgfqpoint{2.418956in}{0.602195in}}%
\pgfpathlineto{\pgfqpoint{2.420023in}{0.606212in}}%
\pgfpathlineto{\pgfqpoint{2.420557in}{0.605040in}}%
\pgfpathlineto{\pgfqpoint{2.421091in}{0.604808in}}%
\pgfpathlineto{\pgfqpoint{2.421624in}{0.602499in}}%
\pgfpathlineto{\pgfqpoint{2.422158in}{0.613209in}}%
\pgfpathlineto{\pgfqpoint{2.422692in}{0.604362in}}%
\pgfpathlineto{\pgfqpoint{2.423226in}{0.605127in}}%
\pgfpathlineto{\pgfqpoint{2.424827in}{0.601054in}}%
\pgfpathlineto{\pgfqpoint{2.425894in}{0.612538in}}%
\pgfpathlineto{\pgfqpoint{2.426961in}{0.610108in}}%
\pgfpathlineto{\pgfqpoint{2.427495in}{0.607058in}}%
\pgfpathlineto{\pgfqpoint{2.428029in}{0.612306in}}%
\pgfpathlineto{\pgfqpoint{2.428563in}{0.602195in}}%
\pgfpathlineto{\pgfqpoint{2.429096in}{0.608462in}}%
\pgfpathlineto{\pgfqpoint{2.430164in}{0.606971in}}%
\pgfpathlineto{\pgfqpoint{2.430697in}{0.602091in}}%
\pgfpathlineto{\pgfqpoint{2.431231in}{0.603507in}}%
\pgfpathlineto{\pgfqpoint{2.431765in}{0.602662in}}%
\pgfpathlineto{\pgfqpoint{2.433366in}{0.608739in}}%
\pgfpathlineto{\pgfqpoint{2.433900in}{0.610274in}}%
\pgfpathlineto{\pgfqpoint{2.434967in}{0.602854in}}%
\pgfpathlineto{\pgfqpoint{2.436568in}{0.609462in}}%
\pgfpathlineto{\pgfqpoint{2.437102in}{0.606023in}}%
\pgfpathlineto{\pgfqpoint{2.437635in}{0.615901in}}%
\pgfpathlineto{\pgfqpoint{2.438703in}{0.604231in}}%
\pgfpathlineto{\pgfqpoint{2.440304in}{0.614599in}}%
\pgfpathlineto{\pgfqpoint{2.440838in}{0.606298in}}%
\pgfpathlineto{\pgfqpoint{2.441371in}{0.613034in}}%
\pgfpathlineto{\pgfqpoint{2.442439in}{0.606472in}}%
\pgfpathlineto{\pgfqpoint{2.442972in}{0.610613in}}%
\pgfpathlineto{\pgfqpoint{2.443506in}{0.604431in}}%
\pgfpathlineto{\pgfqpoint{2.444040in}{0.626736in}}%
\pgfpathlineto{\pgfqpoint{2.444574in}{0.603177in}}%
\pgfpathlineto{\pgfqpoint{2.445107in}{0.609846in}}%
\pgfpathlineto{\pgfqpoint{2.445641in}{0.608284in}}%
\pgfpathlineto{\pgfqpoint{2.446175in}{0.605526in}}%
\pgfpathlineto{\pgfqpoint{2.446708in}{0.609187in}}%
\pgfpathlineto{\pgfqpoint{2.447242in}{0.603186in}}%
\pgfpathlineto{\pgfqpoint{2.447776in}{0.604273in}}%
\pgfpathlineto{\pgfqpoint{2.448309in}{0.603909in}}%
\pgfpathlineto{\pgfqpoint{2.448843in}{0.620470in}}%
\pgfpathlineto{\pgfqpoint{2.449377in}{0.604640in}}%
\pgfpathlineto{\pgfqpoint{2.449911in}{0.602997in}}%
\pgfpathlineto{\pgfqpoint{2.451512in}{0.610779in}}%
\pgfpathlineto{\pgfqpoint{2.452045in}{0.603381in}}%
\pgfpathlineto{\pgfqpoint{2.452579in}{0.604418in}}%
\pgfpathlineto{\pgfqpoint{2.454180in}{0.612945in}}%
\pgfpathlineto{\pgfqpoint{2.455781in}{0.601185in}}%
\pgfpathlineto{\pgfqpoint{2.456315in}{0.608397in}}%
\pgfpathlineto{\pgfqpoint{2.456849in}{0.607933in}}%
\pgfpathlineto{\pgfqpoint{2.457382in}{0.601481in}}%
\pgfpathlineto{\pgfqpoint{2.457916in}{0.610495in}}%
\pgfpathlineto{\pgfqpoint{2.458450in}{0.608731in}}%
\pgfpathlineto{\pgfqpoint{2.458983in}{0.602708in}}%
\pgfpathlineto{\pgfqpoint{2.459517in}{0.613538in}}%
\pgfpathlineto{\pgfqpoint{2.460585in}{0.613105in}}%
\pgfpathlineto{\pgfqpoint{2.462186in}{0.601390in}}%
\pgfpathlineto{\pgfqpoint{2.463787in}{0.609585in}}%
\pgfpathlineto{\pgfqpoint{2.464854in}{0.601891in}}%
\pgfpathlineto{\pgfqpoint{2.465388in}{0.611560in}}%
\pgfpathlineto{\pgfqpoint{2.465922in}{0.601821in}}%
\pgfpathlineto{\pgfqpoint{2.466455in}{0.603054in}}%
\pgfpathlineto{\pgfqpoint{2.466989in}{0.600246in}}%
\pgfpathlineto{\pgfqpoint{2.467523in}{0.602938in}}%
\pgfpathlineto{\pgfqpoint{2.468590in}{0.608481in}}%
\pgfpathlineto{\pgfqpoint{2.469124in}{0.604588in}}%
\pgfpathlineto{\pgfqpoint{2.469657in}{0.607290in}}%
\pgfpathlineto{\pgfqpoint{2.470191in}{0.609772in}}%
\pgfpathlineto{\pgfqpoint{2.470725in}{0.600875in}}%
\pgfpathlineto{\pgfqpoint{2.471259in}{0.611538in}}%
\pgfpathlineto{\pgfqpoint{2.471792in}{0.610727in}}%
\pgfpathlineto{\pgfqpoint{2.472860in}{0.603666in}}%
\pgfpathlineto{\pgfqpoint{2.473393in}{0.611491in}}%
\pgfpathlineto{\pgfqpoint{2.473927in}{0.601163in}}%
\pgfpathlineto{\pgfqpoint{2.474461in}{0.609279in}}%
\pgfpathlineto{\pgfqpoint{2.476062in}{0.605607in}}%
\pgfpathlineto{\pgfqpoint{2.476596in}{0.603467in}}%
\pgfpathlineto{\pgfqpoint{2.477663in}{0.613246in}}%
\pgfpathlineto{\pgfqpoint{2.478197in}{0.603364in}}%
\pgfpathlineto{\pgfqpoint{2.478730in}{0.608758in}}%
\pgfpathlineto{\pgfqpoint{2.479264in}{0.610966in}}%
\pgfpathlineto{\pgfqpoint{2.480331in}{0.605017in}}%
\pgfpathlineto{\pgfqpoint{2.481399in}{0.612816in}}%
\pgfpathlineto{\pgfqpoint{2.481933in}{0.611790in}}%
\pgfpathlineto{\pgfqpoint{2.482466in}{0.608109in}}%
\pgfpathlineto{\pgfqpoint{2.483534in}{0.618813in}}%
\pgfpathlineto{\pgfqpoint{2.484067in}{0.613046in}}%
\pgfpathlineto{\pgfqpoint{2.485135in}{0.612235in}}%
\pgfpathlineto{\pgfqpoint{2.486202in}{0.602757in}}%
\pgfpathlineto{\pgfqpoint{2.486736in}{0.612592in}}%
\pgfpathlineto{\pgfqpoint{2.487270in}{0.610469in}}%
\pgfpathlineto{\pgfqpoint{2.487803in}{0.600720in}}%
\pgfpathlineto{\pgfqpoint{2.488337in}{0.607500in}}%
\pgfpathlineto{\pgfqpoint{2.488871in}{0.611771in}}%
\pgfpathlineto{\pgfqpoint{2.489404in}{0.608407in}}%
\pgfpathlineto{\pgfqpoint{2.490472in}{0.605144in}}%
\pgfpathlineto{\pgfqpoint{2.491006in}{0.619113in}}%
\pgfpathlineto{\pgfqpoint{2.491539in}{0.616050in}}%
\pgfpathlineto{\pgfqpoint{2.492073in}{0.606748in}}%
\pgfpathlineto{\pgfqpoint{2.492607in}{0.616134in}}%
\pgfpathlineto{\pgfqpoint{2.493674in}{0.606664in}}%
\pgfpathlineto{\pgfqpoint{2.495275in}{0.622461in}}%
\pgfpathlineto{\pgfqpoint{2.496343in}{0.603333in}}%
\pgfpathlineto{\pgfqpoint{2.497944in}{0.621151in}}%
\pgfpathlineto{\pgfqpoint{2.498477in}{0.616077in}}%
\pgfpathlineto{\pgfqpoint{2.499011in}{0.623327in}}%
\pgfpathlineto{\pgfqpoint{2.499545in}{0.622231in}}%
\pgfpathlineto{\pgfqpoint{2.500078in}{0.610456in}}%
\pgfpathlineto{\pgfqpoint{2.500612in}{0.622386in}}%
\pgfpathlineto{\pgfqpoint{2.502213in}{0.607030in}}%
\pgfpathlineto{\pgfqpoint{2.502747in}{0.608932in}}%
\pgfpathlineto{\pgfqpoint{2.503281in}{0.607520in}}%
\pgfpathlineto{\pgfqpoint{2.503814in}{0.618882in}}%
\pgfpathlineto{\pgfqpoint{2.504348in}{0.615330in}}%
\pgfpathlineto{\pgfqpoint{2.505949in}{0.602196in}}%
\pgfpathlineto{\pgfqpoint{2.506483in}{0.629758in}}%
\pgfpathlineto{\pgfqpoint{2.507017in}{0.602874in}}%
\pgfpathlineto{\pgfqpoint{2.508618in}{0.606728in}}%
\pgfpathlineto{\pgfqpoint{2.510219in}{0.613375in}}%
\pgfpathlineto{\pgfqpoint{2.511820in}{0.610063in}}%
\pgfpathlineto{\pgfqpoint{2.512354in}{0.611097in}}%
\pgfpathlineto{\pgfqpoint{2.512887in}{0.601620in}}%
\pgfpathlineto{\pgfqpoint{2.513421in}{0.603434in}}%
\pgfpathlineto{\pgfqpoint{2.514488in}{0.620029in}}%
\pgfpathlineto{\pgfqpoint{2.515022in}{0.611905in}}%
\pgfpathlineto{\pgfqpoint{2.515556in}{0.611587in}}%
\pgfpathlineto{\pgfqpoint{2.517157in}{0.602189in}}%
\pgfpathlineto{\pgfqpoint{2.517691in}{0.609827in}}%
\pgfpathlineto{\pgfqpoint{2.518224in}{0.604978in}}%
\pgfpathlineto{\pgfqpoint{2.518758in}{0.608550in}}%
\pgfpathlineto{\pgfqpoint{2.519292in}{0.602236in}}%
\pgfpathlineto{\pgfqpoint{2.519825in}{0.606297in}}%
\pgfpathlineto{\pgfqpoint{2.520359in}{0.615559in}}%
\pgfpathlineto{\pgfqpoint{2.520893in}{0.606672in}}%
\pgfpathlineto{\pgfqpoint{2.521426in}{0.600728in}}%
\pgfpathlineto{\pgfqpoint{2.521960in}{0.603315in}}%
\pgfpathlineto{\pgfqpoint{2.523561in}{0.608004in}}%
\pgfpathlineto{\pgfqpoint{2.524095in}{0.605039in}}%
\pgfpathlineto{\pgfqpoint{2.524629in}{0.609032in}}%
\pgfpathlineto{\pgfqpoint{2.525162in}{0.603594in}}%
\pgfpathlineto{\pgfqpoint{2.525696in}{0.608369in}}%
\pgfpathlineto{\pgfqpoint{2.526230in}{0.605979in}}%
\pgfpathlineto{\pgfqpoint{2.526763in}{0.608259in}}%
\pgfpathlineto{\pgfqpoint{2.527831in}{0.612158in}}%
\pgfpathlineto{\pgfqpoint{2.528365in}{0.609875in}}%
\pgfpathlineto{\pgfqpoint{2.528898in}{0.606426in}}%
\pgfpathlineto{\pgfqpoint{2.529432in}{0.607801in}}%
\pgfpathlineto{\pgfqpoint{2.529966in}{0.609513in}}%
\pgfpathlineto{\pgfqpoint{2.530499in}{0.615955in}}%
\pgfpathlineto{\pgfqpoint{2.531033in}{0.600406in}}%
\pgfpathlineto{\pgfqpoint{2.531567in}{0.611104in}}%
\pgfpathlineto{\pgfqpoint{2.532100in}{0.607354in}}%
\pgfpathlineto{\pgfqpoint{2.532634in}{0.618718in}}%
\pgfpathlineto{\pgfqpoint{2.533168in}{0.613262in}}%
\pgfpathlineto{\pgfqpoint{2.533702in}{0.612202in}}%
\pgfpathlineto{\pgfqpoint{2.534235in}{0.605796in}}%
\pgfpathlineto{\pgfqpoint{2.534769in}{0.614144in}}%
\pgfpathlineto{\pgfqpoint{2.535836in}{0.613958in}}%
\pgfpathlineto{\pgfqpoint{2.536904in}{0.605234in}}%
\pgfpathlineto{\pgfqpoint{2.538505in}{0.629785in}}%
\pgfpathlineto{\pgfqpoint{2.539039in}{0.605382in}}%
\pgfpathlineto{\pgfqpoint{2.539572in}{0.622206in}}%
\pgfpathlineto{\pgfqpoint{2.540640in}{0.622456in}}%
\pgfpathlineto{\pgfqpoint{2.541173in}{0.602681in}}%
\pgfpathlineto{\pgfqpoint{2.542774in}{0.620372in}}%
\pgfpathlineto{\pgfqpoint{2.543308in}{0.609972in}}%
\pgfpathlineto{\pgfqpoint{2.543842in}{0.616983in}}%
\pgfpathlineto{\pgfqpoint{2.544909in}{0.607780in}}%
\pgfpathlineto{\pgfqpoint{2.545443in}{0.612119in}}%
\pgfpathlineto{\pgfqpoint{2.546510in}{0.621890in}}%
\pgfpathlineto{\pgfqpoint{2.547044in}{0.607064in}}%
\pgfpathlineto{\pgfqpoint{2.547578in}{0.615239in}}%
\pgfpathlineto{\pgfqpoint{2.548111in}{0.634697in}}%
\pgfpathlineto{\pgfqpoint{2.548645in}{0.618002in}}%
\pgfpathlineto{\pgfqpoint{2.549713in}{0.614589in}}%
\pgfpathlineto{\pgfqpoint{2.550246in}{0.612979in}}%
\pgfpathlineto{\pgfqpoint{2.550780in}{0.622014in}}%
\pgfpathlineto{\pgfqpoint{2.551314in}{0.602973in}}%
\pgfpathlineto{\pgfqpoint{2.551847in}{0.619615in}}%
\pgfpathlineto{\pgfqpoint{2.552381in}{0.632404in}}%
\pgfpathlineto{\pgfqpoint{2.553448in}{0.613332in}}%
\pgfpathlineto{\pgfqpoint{2.553982in}{0.615811in}}%
\pgfpathlineto{\pgfqpoint{2.554516in}{0.645176in}}%
\pgfpathlineto{\pgfqpoint{2.555050in}{0.608115in}}%
\pgfpathlineto{\pgfqpoint{2.555583in}{0.612617in}}%
\pgfpathlineto{\pgfqpoint{2.556117in}{0.611916in}}%
\pgfpathlineto{\pgfqpoint{2.556651in}{0.632966in}}%
\pgfpathlineto{\pgfqpoint{2.557184in}{0.613714in}}%
\pgfpathlineto{\pgfqpoint{2.557718in}{0.614901in}}%
\pgfpathlineto{\pgfqpoint{2.558786in}{0.635272in}}%
\pgfpathlineto{\pgfqpoint{2.559853in}{0.614537in}}%
\pgfpathlineto{\pgfqpoint{2.560387in}{0.622363in}}%
\pgfpathlineto{\pgfqpoint{2.560920in}{0.607589in}}%
\pgfpathlineto{\pgfqpoint{2.561454in}{0.624583in}}%
\pgfpathlineto{\pgfqpoint{2.561988in}{0.608734in}}%
\pgfpathlineto{\pgfqpoint{2.562521in}{0.612939in}}%
\pgfpathlineto{\pgfqpoint{2.563055in}{0.609919in}}%
\pgfpathlineto{\pgfqpoint{2.563589in}{0.608027in}}%
\pgfpathlineto{\pgfqpoint{2.564123in}{0.609750in}}%
\pgfpathlineto{\pgfqpoint{2.565724in}{0.619866in}}%
\pgfpathlineto{\pgfqpoint{2.567325in}{0.607393in}}%
\pgfpathlineto{\pgfqpoint{2.568392in}{0.606785in}}%
\pgfpathlineto{\pgfqpoint{2.569460in}{0.624606in}}%
\pgfpathlineto{\pgfqpoint{2.571061in}{0.601768in}}%
\pgfpathlineto{\pgfqpoint{2.571594in}{0.619160in}}%
\pgfpathlineto{\pgfqpoint{2.572128in}{0.612584in}}%
\pgfpathlineto{\pgfqpoint{2.574263in}{0.604488in}}%
\pgfpathlineto{\pgfqpoint{2.575864in}{0.619822in}}%
\pgfpathlineto{\pgfqpoint{2.576398in}{0.600914in}}%
\pgfpathlineto{\pgfqpoint{2.576931in}{0.602726in}}%
\pgfpathlineto{\pgfqpoint{2.577465in}{0.610355in}}%
\pgfpathlineto{\pgfqpoint{2.578532in}{0.610225in}}%
\pgfpathlineto{\pgfqpoint{2.579066in}{0.614062in}}%
\pgfpathlineto{\pgfqpoint{2.579600in}{0.613816in}}%
\pgfpathlineto{\pgfqpoint{2.581201in}{0.603284in}}%
\pgfpathlineto{\pgfqpoint{2.582268in}{0.612479in}}%
\pgfpathlineto{\pgfqpoint{2.582802in}{0.611831in}}%
\pgfpathlineto{\pgfqpoint{2.583336in}{0.600524in}}%
\pgfpathlineto{\pgfqpoint{2.584937in}{0.621861in}}%
\pgfpathlineto{\pgfqpoint{2.586004in}{0.603066in}}%
\pgfpathlineto{\pgfqpoint{2.587605in}{0.619947in}}%
\pgfpathlineto{\pgfqpoint{2.589206in}{0.605161in}}%
\pgfpathlineto{\pgfqpoint{2.589740in}{0.622671in}}%
\pgfpathlineto{\pgfqpoint{2.590274in}{0.614367in}}%
\pgfpathlineto{\pgfqpoint{2.590808in}{0.617632in}}%
\pgfpathlineto{\pgfqpoint{2.591341in}{0.609576in}}%
\pgfpathlineto{\pgfqpoint{2.591875in}{0.614136in}}%
\pgfpathlineto{\pgfqpoint{2.592409in}{0.622124in}}%
\pgfpathlineto{\pgfqpoint{2.592942in}{0.605663in}}%
\pgfpathlineto{\pgfqpoint{2.594543in}{0.627570in}}%
\pgfpathlineto{\pgfqpoint{2.595077in}{0.617903in}}%
\pgfpathlineto{\pgfqpoint{2.595611in}{0.632305in}}%
\pgfpathlineto{\pgfqpoint{2.596145in}{0.609357in}}%
\pgfpathlineto{\pgfqpoint{2.596678in}{0.611839in}}%
\pgfpathlineto{\pgfqpoint{2.597746in}{0.628170in}}%
\pgfpathlineto{\pgfqpoint{2.599347in}{0.603186in}}%
\pgfpathlineto{\pgfqpoint{2.599880in}{0.608950in}}%
\pgfpathlineto{\pgfqpoint{2.600414in}{0.603028in}}%
\pgfpathlineto{\pgfqpoint{2.600948in}{0.607007in}}%
\pgfpathlineto{\pgfqpoint{2.601482in}{0.624922in}}%
\pgfpathlineto{\pgfqpoint{2.602015in}{0.617999in}}%
\pgfpathlineto{\pgfqpoint{2.602549in}{0.613773in}}%
\pgfpathlineto{\pgfqpoint{2.603083in}{0.634596in}}%
\pgfpathlineto{\pgfqpoint{2.603616in}{0.622259in}}%
\pgfpathlineto{\pgfqpoint{2.605217in}{0.600370in}}%
\pgfpathlineto{\pgfqpoint{2.607352in}{0.636543in}}%
\pgfpathlineto{\pgfqpoint{2.608420in}{0.617045in}}%
\pgfpathlineto{\pgfqpoint{2.608953in}{0.625590in}}%
\pgfpathlineto{\pgfqpoint{2.609487in}{0.623441in}}%
\pgfpathlineto{\pgfqpoint{2.610554in}{0.601840in}}%
\pgfpathlineto{\pgfqpoint{2.611622in}{0.626712in}}%
\pgfpathlineto{\pgfqpoint{2.612156in}{0.622844in}}%
\pgfpathlineto{\pgfqpoint{2.613757in}{0.604883in}}%
\pgfpathlineto{\pgfqpoint{2.614290in}{0.633054in}}%
\pgfpathlineto{\pgfqpoint{2.614824in}{0.619650in}}%
\pgfpathlineto{\pgfqpoint{2.616959in}{0.606247in}}%
\pgfpathlineto{\pgfqpoint{2.617493in}{0.628156in}}%
\pgfpathlineto{\pgfqpoint{2.618026in}{0.618158in}}%
\pgfpathlineto{\pgfqpoint{2.620161in}{0.601415in}}%
\pgfpathlineto{\pgfqpoint{2.621762in}{0.618195in}}%
\pgfpathlineto{\pgfqpoint{2.623897in}{0.605970in}}%
\pgfpathlineto{\pgfqpoint{2.624431in}{0.609091in}}%
\pgfpathlineto{\pgfqpoint{2.624964in}{0.601006in}}%
\pgfpathlineto{\pgfqpoint{2.626032in}{0.617928in}}%
\pgfpathlineto{\pgfqpoint{2.626566in}{0.617749in}}%
\pgfpathlineto{\pgfqpoint{2.627633in}{0.602945in}}%
\pgfpathlineto{\pgfqpoint{2.628167in}{0.605118in}}%
\pgfpathlineto{\pgfqpoint{2.629768in}{0.613779in}}%
\pgfpathlineto{\pgfqpoint{2.630301in}{0.609117in}}%
\pgfpathlineto{\pgfqpoint{2.630835in}{0.608105in}}%
\pgfpathlineto{\pgfqpoint{2.632436in}{0.602220in}}%
\pgfpathlineto{\pgfqpoint{2.632970in}{0.608499in}}%
\pgfpathlineto{\pgfqpoint{2.633504in}{0.605368in}}%
\pgfpathlineto{\pgfqpoint{2.634037in}{0.606759in}}%
\pgfpathlineto{\pgfqpoint{2.634571in}{0.604279in}}%
\pgfpathlineto{\pgfqpoint{2.635638in}{0.609101in}}%
\pgfpathlineto{\pgfqpoint{2.637240in}{0.603732in}}%
\pgfpathlineto{\pgfqpoint{2.637773in}{0.609589in}}%
\pgfpathlineto{\pgfqpoint{2.638841in}{0.609210in}}%
\pgfpathlineto{\pgfqpoint{2.639374in}{0.604544in}}%
\pgfpathlineto{\pgfqpoint{2.639908in}{0.613348in}}%
\pgfpathlineto{\pgfqpoint{2.640442in}{0.601375in}}%
\pgfpathlineto{\pgfqpoint{2.640975in}{0.605730in}}%
\pgfpathlineto{\pgfqpoint{2.641509in}{0.607176in}}%
\pgfpathlineto{\pgfqpoint{2.642043in}{0.606066in}}%
\pgfpathlineto{\pgfqpoint{2.642577in}{0.606579in}}%
\pgfpathlineto{\pgfqpoint{2.643110in}{0.609172in}}%
\pgfpathlineto{\pgfqpoint{2.643644in}{0.607803in}}%
\pgfpathlineto{\pgfqpoint{2.644178in}{0.607584in}}%
\pgfpathlineto{\pgfqpoint{2.644711in}{0.609674in}}%
\pgfpathlineto{\pgfqpoint{2.645245in}{0.608329in}}%
\pgfpathlineto{\pgfqpoint{2.645779in}{0.608746in}}%
\pgfpathlineto{\pgfqpoint{2.646846in}{0.603168in}}%
\pgfpathlineto{\pgfqpoint{2.647380in}{0.604285in}}%
\pgfpathlineto{\pgfqpoint{2.647914in}{0.607209in}}%
\pgfpathlineto{\pgfqpoint{2.648447in}{0.602284in}}%
\pgfpathlineto{\pgfqpoint{2.648981in}{0.616609in}}%
\pgfpathlineto{\pgfqpoint{2.649515in}{0.607296in}}%
\pgfpathlineto{\pgfqpoint{2.651116in}{0.612439in}}%
\pgfpathlineto{\pgfqpoint{2.651649in}{0.603203in}}%
\pgfpathlineto{\pgfqpoint{2.652183in}{0.612847in}}%
\pgfpathlineto{\pgfqpoint{2.653784in}{0.601314in}}%
\pgfpathlineto{\pgfqpoint{2.654852in}{0.609567in}}%
\pgfpathlineto{\pgfqpoint{2.655385in}{0.600645in}}%
\pgfpathlineto{\pgfqpoint{2.655919in}{0.608333in}}%
\pgfpathlineto{\pgfqpoint{2.657520in}{0.601251in}}%
\pgfpathlineto{\pgfqpoint{2.658588in}{0.612550in}}%
\pgfpathlineto{\pgfqpoint{2.659121in}{0.607369in}}%
\pgfpathlineto{\pgfqpoint{2.660722in}{0.602999in}}%
\pgfpathlineto{\pgfqpoint{2.662323in}{0.608004in}}%
\pgfpathlineto{\pgfqpoint{2.663925in}{0.604974in}}%
\pgfpathlineto{\pgfqpoint{2.664992in}{0.606241in}}%
\pgfpathlineto{\pgfqpoint{2.665526in}{0.600953in}}%
\pgfpathlineto{\pgfqpoint{2.666059in}{0.602338in}}%
\pgfpathlineto{\pgfqpoint{2.666593in}{0.605765in}}%
\pgfpathlineto{\pgfqpoint{2.667127in}{0.604110in}}%
\pgfpathlineto{\pgfqpoint{2.668194in}{0.601057in}}%
\pgfpathlineto{\pgfqpoint{2.668728in}{0.604667in}}%
\pgfpathlineto{\pgfqpoint{2.669262in}{0.602766in}}%
\pgfpathlineto{\pgfqpoint{2.669795in}{0.604150in}}%
\pgfpathlineto{\pgfqpoint{2.670329in}{0.601901in}}%
\pgfpathlineto{\pgfqpoint{2.670863in}{0.603529in}}%
\pgfpathlineto{\pgfqpoint{2.671930in}{0.604116in}}%
\pgfpathlineto{\pgfqpoint{2.672997in}{0.600969in}}%
\pgfpathlineto{\pgfqpoint{2.673531in}{0.604011in}}%
\pgfpathlineto{\pgfqpoint{2.674065in}{0.603049in}}%
\pgfpathlineto{\pgfqpoint{2.674599in}{0.602867in}}%
\pgfpathlineto{\pgfqpoint{2.675132in}{0.601365in}}%
\pgfpathlineto{\pgfqpoint{2.675666in}{0.601768in}}%
\pgfpathlineto{\pgfqpoint{2.676733in}{0.602053in}}%
\pgfpathlineto{\pgfqpoint{2.678334in}{0.600402in}}%
\pgfpathlineto{\pgfqpoint{2.684205in}{0.601158in}}%
\pgfpathlineto{\pgfqpoint{2.684739in}{0.600126in}}%
\pgfpathlineto{\pgfqpoint{2.685273in}{0.600996in}}%
\pgfpathlineto{\pgfqpoint{2.686340in}{0.601017in}}%
\pgfpathlineto{\pgfqpoint{2.688475in}{0.603791in}}%
\pgfpathlineto{\pgfqpoint{2.689009in}{0.607802in}}%
\pgfpathlineto{\pgfqpoint{2.689542in}{0.601449in}}%
\pgfpathlineto{\pgfqpoint{2.690076in}{0.615207in}}%
\pgfpathlineto{\pgfqpoint{2.690610in}{0.610552in}}%
\pgfpathlineto{\pgfqpoint{2.692211in}{0.600208in}}%
\pgfpathlineto{\pgfqpoint{2.693812in}{0.600287in}}%
\pgfpathlineto{\pgfqpoint{2.696480in}{0.600185in}}%
\pgfpathlineto{\pgfqpoint{2.698081in}{0.600831in}}%
\pgfpathlineto{\pgfqpoint{2.701284in}{0.601453in}}%
\pgfpathlineto{\pgfqpoint{2.702351in}{0.600072in}}%
\pgfpathlineto{\pgfqpoint{2.703952in}{0.602837in}}%
\pgfpathlineto{\pgfqpoint{2.705553in}{0.601073in}}%
\pgfpathlineto{\pgfqpoint{2.706087in}{0.605829in}}%
\pgfpathlineto{\pgfqpoint{2.706621in}{0.600445in}}%
\pgfpathlineto{\pgfqpoint{2.707154in}{0.603279in}}%
\pgfpathlineto{\pgfqpoint{2.708755in}{0.602191in}}%
\pgfpathlineto{\pgfqpoint{2.709289in}{0.603919in}}%
\pgfpathlineto{\pgfqpoint{2.709823in}{0.603258in}}%
\pgfpathlineto{\pgfqpoint{2.711424in}{0.602027in}}%
\pgfpathlineto{\pgfqpoint{2.711958in}{0.603703in}}%
\pgfpathlineto{\pgfqpoint{2.712491in}{0.601327in}}%
\pgfpathlineto{\pgfqpoint{2.713559in}{0.608740in}}%
\pgfpathlineto{\pgfqpoint{2.714092in}{0.606433in}}%
\pgfpathlineto{\pgfqpoint{2.714626in}{0.600971in}}%
\pgfpathlineto{\pgfqpoint{2.715160in}{0.601416in}}%
\pgfpathlineto{\pgfqpoint{2.716227in}{0.605878in}}%
\pgfpathlineto{\pgfqpoint{2.716761in}{0.600591in}}%
\pgfpathlineto{\pgfqpoint{2.717828in}{0.610595in}}%
\pgfpathlineto{\pgfqpoint{2.718362in}{0.604598in}}%
\pgfpathlineto{\pgfqpoint{2.718896in}{0.605501in}}%
\pgfpathlineto{\pgfqpoint{2.719429in}{0.608968in}}%
\pgfpathlineto{\pgfqpoint{2.719963in}{0.607551in}}%
\pgfpathlineto{\pgfqpoint{2.721031in}{0.603080in}}%
\pgfpathlineto{\pgfqpoint{2.722632in}{0.608455in}}%
\pgfpathlineto{\pgfqpoint{2.723165in}{0.602024in}}%
\pgfpathlineto{\pgfqpoint{2.723699in}{0.607428in}}%
\pgfpathlineto{\pgfqpoint{2.724233in}{0.603484in}}%
\pgfpathlineto{\pgfqpoint{2.724766in}{0.613609in}}%
\pgfpathlineto{\pgfqpoint{2.726368in}{0.600394in}}%
\pgfpathlineto{\pgfqpoint{2.727969in}{0.609612in}}%
\pgfpathlineto{\pgfqpoint{2.728502in}{0.603383in}}%
\pgfpathlineto{\pgfqpoint{2.729036in}{0.605436in}}%
\pgfpathlineto{\pgfqpoint{2.730637in}{0.610715in}}%
\pgfpathlineto{\pgfqpoint{2.731171in}{0.611475in}}%
\pgfpathlineto{\pgfqpoint{2.731705in}{0.605376in}}%
\pgfpathlineto{\pgfqpoint{2.732238in}{0.610743in}}%
\pgfpathlineto{\pgfqpoint{2.732772in}{0.614214in}}%
\pgfpathlineto{\pgfqpoint{2.733306in}{0.611217in}}%
\pgfpathlineto{\pgfqpoint{2.733839in}{0.610665in}}%
\pgfpathlineto{\pgfqpoint{2.734907in}{0.601577in}}%
\pgfpathlineto{\pgfqpoint{2.735440in}{0.602299in}}%
\pgfpathlineto{\pgfqpoint{2.735974in}{0.601826in}}%
\pgfpathlineto{\pgfqpoint{2.738109in}{0.615589in}}%
\pgfpathlineto{\pgfqpoint{2.739710in}{0.604125in}}%
\pgfpathlineto{\pgfqpoint{2.740777in}{0.610783in}}%
\pgfpathlineto{\pgfqpoint{2.741311in}{0.616148in}}%
\pgfpathlineto{\pgfqpoint{2.741845in}{0.613685in}}%
\pgfpathlineto{\pgfqpoint{2.743446in}{0.603871in}}%
\pgfpathlineto{\pgfqpoint{2.743980in}{0.606843in}}%
\pgfpathlineto{\pgfqpoint{2.745047in}{0.619229in}}%
\pgfpathlineto{\pgfqpoint{2.746114in}{0.604154in}}%
\pgfpathlineto{\pgfqpoint{2.746648in}{0.605780in}}%
\pgfpathlineto{\pgfqpoint{2.747182in}{0.612194in}}%
\pgfpathlineto{\pgfqpoint{2.747716in}{0.603315in}}%
\pgfpathlineto{\pgfqpoint{2.748249in}{0.610520in}}%
\pgfpathlineto{\pgfqpoint{2.748783in}{0.605974in}}%
\pgfpathlineto{\pgfqpoint{2.750384in}{0.623364in}}%
\pgfpathlineto{\pgfqpoint{2.750918in}{0.603055in}}%
\pgfpathlineto{\pgfqpoint{2.751451in}{0.603923in}}%
\pgfpathlineto{\pgfqpoint{2.751985in}{0.617313in}}%
\pgfpathlineto{\pgfqpoint{2.752519in}{0.609483in}}%
\pgfpathlineto{\pgfqpoint{2.753586in}{0.613015in}}%
\pgfpathlineto{\pgfqpoint{2.754120in}{0.620332in}}%
\pgfpathlineto{\pgfqpoint{2.754654in}{0.606932in}}%
\pgfpathlineto{\pgfqpoint{2.755187in}{0.627216in}}%
\pgfpathlineto{\pgfqpoint{2.756255in}{0.626743in}}%
\pgfpathlineto{\pgfqpoint{2.756789in}{0.603000in}}%
\pgfpathlineto{\pgfqpoint{2.757322in}{0.608313in}}%
\pgfpathlineto{\pgfqpoint{2.757856in}{0.624669in}}%
\pgfpathlineto{\pgfqpoint{2.758390in}{0.619525in}}%
\pgfpathlineto{\pgfqpoint{2.758923in}{0.612873in}}%
\pgfpathlineto{\pgfqpoint{2.759991in}{0.632535in}}%
\pgfpathlineto{\pgfqpoint{2.760524in}{0.620266in}}%
\pgfpathlineto{\pgfqpoint{2.761058in}{0.647703in}}%
\pgfpathlineto{\pgfqpoint{2.761592in}{0.630086in}}%
\pgfpathlineto{\pgfqpoint{2.762126in}{0.619609in}}%
\pgfpathlineto{\pgfqpoint{2.762659in}{0.644189in}}%
\pgfpathlineto{\pgfqpoint{2.763193in}{0.626491in}}%
\pgfpathlineto{\pgfqpoint{2.763727in}{0.623937in}}%
\pgfpathlineto{\pgfqpoint{2.764260in}{0.615703in}}%
\pgfpathlineto{\pgfqpoint{2.765328in}{0.639907in}}%
\pgfpathlineto{\pgfqpoint{2.765861in}{0.608162in}}%
\pgfpathlineto{\pgfqpoint{2.766395in}{0.637533in}}%
\pgfpathlineto{\pgfqpoint{2.767463in}{0.606490in}}%
\pgfpathlineto{\pgfqpoint{2.769064in}{0.643453in}}%
\pgfpathlineto{\pgfqpoint{2.770131in}{0.621394in}}%
\pgfpathlineto{\pgfqpoint{2.770665in}{0.635067in}}%
\pgfpathlineto{\pgfqpoint{2.771198in}{0.618283in}}%
\pgfpathlineto{\pgfqpoint{2.771732in}{0.628216in}}%
\pgfpathlineto{\pgfqpoint{2.772800in}{0.656589in}}%
\pgfpathlineto{\pgfqpoint{2.773333in}{0.613339in}}%
\pgfpathlineto{\pgfqpoint{2.773867in}{0.628812in}}%
\pgfpathlineto{\pgfqpoint{2.774401in}{0.640968in}}%
\pgfpathlineto{\pgfqpoint{2.774934in}{0.631103in}}%
\pgfpathlineto{\pgfqpoint{2.775468in}{0.635675in}}%
\pgfpathlineto{\pgfqpoint{2.776002in}{0.623406in}}%
\pgfpathlineto{\pgfqpoint{2.776535in}{0.624803in}}%
\pgfpathlineto{\pgfqpoint{2.777603in}{0.642363in}}%
\pgfpathlineto{\pgfqpoint{2.779204in}{0.615845in}}%
\pgfpathlineto{\pgfqpoint{2.779738in}{0.627744in}}%
\pgfpathlineto{\pgfqpoint{2.780271in}{0.625115in}}%
\pgfpathlineto{\pgfqpoint{2.781339in}{0.609698in}}%
\pgfpathlineto{\pgfqpoint{2.781872in}{0.620069in}}%
\pgfpathlineto{\pgfqpoint{2.782406in}{0.616772in}}%
\pgfpathlineto{\pgfqpoint{2.782940in}{0.617868in}}%
\pgfpathlineto{\pgfqpoint{2.783474in}{0.610212in}}%
\pgfpathlineto{\pgfqpoint{2.784007in}{0.623580in}}%
\pgfpathlineto{\pgfqpoint{2.784541in}{0.612414in}}%
\pgfpathlineto{\pgfqpoint{2.786676in}{0.635574in}}%
\pgfpathlineto{\pgfqpoint{2.787209in}{0.633280in}}%
\pgfpathlineto{\pgfqpoint{2.788277in}{0.617421in}}%
\pgfpathlineto{\pgfqpoint{2.789878in}{0.643109in}}%
\pgfpathlineto{\pgfqpoint{2.790945in}{0.625971in}}%
\pgfpathlineto{\pgfqpoint{2.791479in}{0.630664in}}%
\pgfpathlineto{\pgfqpoint{2.792013in}{0.627633in}}%
\pgfpathlineto{\pgfqpoint{2.793614in}{0.609950in}}%
\pgfpathlineto{\pgfqpoint{2.794148in}{0.609787in}}%
\pgfpathlineto{\pgfqpoint{2.796282in}{0.641948in}}%
\pgfpathlineto{\pgfqpoint{2.797350in}{0.601205in}}%
\pgfpathlineto{\pgfqpoint{2.799485in}{0.645939in}}%
\pgfpathlineto{\pgfqpoint{2.801086in}{0.614376in}}%
\pgfpathlineto{\pgfqpoint{2.801619in}{0.629673in}}%
\pgfpathlineto{\pgfqpoint{2.802153in}{0.627698in}}%
\pgfpathlineto{\pgfqpoint{2.803754in}{0.607823in}}%
\pgfpathlineto{\pgfqpoint{2.805355in}{0.633546in}}%
\pgfpathlineto{\pgfqpoint{2.806423in}{0.604430in}}%
\pgfpathlineto{\pgfqpoint{2.807490in}{0.630799in}}%
\pgfpathlineto{\pgfqpoint{2.808024in}{0.606182in}}%
\pgfpathlineto{\pgfqpoint{2.808557in}{0.614879in}}%
\pgfpathlineto{\pgfqpoint{2.809091in}{0.635863in}}%
\pgfpathlineto{\pgfqpoint{2.809625in}{0.627875in}}%
\pgfpathlineto{\pgfqpoint{2.810692in}{0.633105in}}%
\pgfpathlineto{\pgfqpoint{2.811226in}{0.631450in}}%
\pgfpathlineto{\pgfqpoint{2.812827in}{0.614265in}}%
\pgfpathlineto{\pgfqpoint{2.813361in}{0.651043in}}%
\pgfpathlineto{\pgfqpoint{2.813894in}{0.603635in}}%
\pgfpathlineto{\pgfqpoint{2.814428in}{0.651982in}}%
\pgfpathlineto{\pgfqpoint{2.815496in}{0.623483in}}%
\pgfpathlineto{\pgfqpoint{2.816029in}{0.623853in}}%
\pgfpathlineto{\pgfqpoint{2.816563in}{0.682191in}}%
\pgfpathlineto{\pgfqpoint{2.817097in}{0.606787in}}%
\pgfpathlineto{\pgfqpoint{2.817630in}{0.661640in}}%
\pgfpathlineto{\pgfqpoint{2.819231in}{0.618626in}}%
\pgfpathlineto{\pgfqpoint{2.819765in}{0.659608in}}%
\pgfpathlineto{\pgfqpoint{2.820299in}{0.659010in}}%
\pgfpathlineto{\pgfqpoint{2.820833in}{0.611129in}}%
\pgfpathlineto{\pgfqpoint{2.821366in}{0.636394in}}%
\pgfpathlineto{\pgfqpoint{2.822967in}{0.612150in}}%
\pgfpathlineto{\pgfqpoint{2.824035in}{0.668654in}}%
\pgfpathlineto{\pgfqpoint{2.824569in}{0.664327in}}%
\pgfpathlineto{\pgfqpoint{2.826170in}{0.636106in}}%
\pgfpathlineto{\pgfqpoint{2.827237in}{0.643497in}}%
\pgfpathlineto{\pgfqpoint{2.827771in}{0.676025in}}%
\pgfpathlineto{\pgfqpoint{2.828304in}{0.643587in}}%
\pgfpathlineto{\pgfqpoint{2.829372in}{0.632996in}}%
\pgfpathlineto{\pgfqpoint{2.829906in}{0.658968in}}%
\pgfpathlineto{\pgfqpoint{2.830439in}{0.633805in}}%
\pgfpathlineto{\pgfqpoint{2.830973in}{0.639314in}}%
\pgfpathlineto{\pgfqpoint{2.831507in}{0.619789in}}%
\pgfpathlineto{\pgfqpoint{2.832040in}{0.648959in}}%
\pgfpathlineto{\pgfqpoint{2.832574in}{0.632364in}}%
\pgfpathlineto{\pgfqpoint{2.833108in}{0.633949in}}%
\pgfpathlineto{\pgfqpoint{2.834709in}{0.604940in}}%
\pgfpathlineto{\pgfqpoint{2.835243in}{0.642631in}}%
\pgfpathlineto{\pgfqpoint{2.835776in}{0.624470in}}%
\pgfpathlineto{\pgfqpoint{2.837377in}{0.607613in}}%
\pgfpathlineto{\pgfqpoint{2.838978in}{0.627230in}}%
\pgfpathlineto{\pgfqpoint{2.839512in}{0.608778in}}%
\pgfpathlineto{\pgfqpoint{2.840046in}{0.620532in}}%
\pgfpathlineto{\pgfqpoint{2.840580in}{0.611881in}}%
\pgfpathlineto{\pgfqpoint{2.841113in}{0.622717in}}%
\pgfpathlineto{\pgfqpoint{2.841647in}{0.605396in}}%
\pgfpathlineto{\pgfqpoint{2.842181in}{0.608496in}}%
\pgfpathlineto{\pgfqpoint{2.843248in}{0.627612in}}%
\pgfpathlineto{\pgfqpoint{2.844315in}{0.614351in}}%
\pgfpathlineto{\pgfqpoint{2.845917in}{0.641000in}}%
\pgfpathlineto{\pgfqpoint{2.846450in}{0.628983in}}%
\pgfpathlineto{\pgfqpoint{2.846984in}{0.638485in}}%
\pgfpathlineto{\pgfqpoint{2.847518in}{0.641876in}}%
\pgfpathlineto{\pgfqpoint{2.848051in}{0.641451in}}%
\pgfpathlineto{\pgfqpoint{2.850186in}{0.617948in}}%
\pgfpathlineto{\pgfqpoint{2.851254in}{0.610275in}}%
\pgfpathlineto{\pgfqpoint{2.853388in}{0.630097in}}%
\pgfpathlineto{\pgfqpoint{2.853922in}{0.630002in}}%
\pgfpathlineto{\pgfqpoint{2.854989in}{0.614330in}}%
\pgfpathlineto{\pgfqpoint{2.855523in}{0.618145in}}%
\pgfpathlineto{\pgfqpoint{2.856591in}{0.639059in}}%
\pgfpathlineto{\pgfqpoint{2.858192in}{0.603836in}}%
\pgfpathlineto{\pgfqpoint{2.858725in}{0.605013in}}%
\pgfpathlineto{\pgfqpoint{2.859259in}{0.636149in}}%
\pgfpathlineto{\pgfqpoint{2.859793in}{0.626156in}}%
\pgfpathlineto{\pgfqpoint{2.860326in}{0.632260in}}%
\pgfpathlineto{\pgfqpoint{2.860860in}{0.626494in}}%
\pgfpathlineto{\pgfqpoint{2.861394in}{0.610731in}}%
\pgfpathlineto{\pgfqpoint{2.861928in}{0.621541in}}%
\pgfpathlineto{\pgfqpoint{2.862461in}{0.636653in}}%
\pgfpathlineto{\pgfqpoint{2.863529in}{0.608780in}}%
\pgfpathlineto{\pgfqpoint{2.864596in}{0.631721in}}%
\pgfpathlineto{\pgfqpoint{2.865130in}{0.605317in}}%
\pgfpathlineto{\pgfqpoint{2.865663in}{0.628174in}}%
\pgfpathlineto{\pgfqpoint{2.866197in}{0.629119in}}%
\pgfpathlineto{\pgfqpoint{2.866731in}{0.632966in}}%
\pgfpathlineto{\pgfqpoint{2.867265in}{0.607419in}}%
\pgfpathlineto{\pgfqpoint{2.867798in}{0.618870in}}%
\pgfpathlineto{\pgfqpoint{2.868332in}{0.650240in}}%
\pgfpathlineto{\pgfqpoint{2.868866in}{0.621173in}}%
\pgfpathlineto{\pgfqpoint{2.869933in}{0.620597in}}%
\pgfpathlineto{\pgfqpoint{2.870467in}{0.648665in}}%
\pgfpathlineto{\pgfqpoint{2.871000in}{0.611496in}}%
\pgfpathlineto{\pgfqpoint{2.871534in}{0.693467in}}%
\pgfpathlineto{\pgfqpoint{2.872068in}{0.618395in}}%
\pgfpathlineto{\pgfqpoint{2.873669in}{0.656400in}}%
\pgfpathlineto{\pgfqpoint{2.874203in}{0.653668in}}%
\pgfpathlineto{\pgfqpoint{2.874736in}{0.655466in}}%
\pgfpathlineto{\pgfqpoint{2.876337in}{0.619064in}}%
\pgfpathlineto{\pgfqpoint{2.876871in}{0.672250in}}%
\pgfpathlineto{\pgfqpoint{2.877405in}{0.645579in}}%
\pgfpathlineto{\pgfqpoint{2.877939in}{0.608951in}}%
\pgfpathlineto{\pgfqpoint{2.878472in}{0.647670in}}%
\pgfpathlineto{\pgfqpoint{2.879006in}{0.637747in}}%
\pgfpathlineto{\pgfqpoint{2.879540in}{0.651708in}}%
\pgfpathlineto{\pgfqpoint{2.880073in}{0.612300in}}%
\pgfpathlineto{\pgfqpoint{2.880607in}{0.648140in}}%
\pgfpathlineto{\pgfqpoint{2.881141in}{0.659918in}}%
\pgfpathlineto{\pgfqpoint{2.882208in}{0.622201in}}%
\pgfpathlineto{\pgfqpoint{2.882742in}{0.693335in}}%
\pgfpathlineto{\pgfqpoint{2.883276in}{0.663483in}}%
\pgfpathlineto{\pgfqpoint{2.883809in}{0.628862in}}%
\pgfpathlineto{\pgfqpoint{2.884343in}{0.631144in}}%
\pgfpathlineto{\pgfqpoint{2.884877in}{0.669462in}}%
\pgfpathlineto{\pgfqpoint{2.885410in}{0.633761in}}%
\pgfpathlineto{\pgfqpoint{2.885944in}{0.647867in}}%
\pgfpathlineto{\pgfqpoint{2.886478in}{0.621483in}}%
\pgfpathlineto{\pgfqpoint{2.887011in}{0.634719in}}%
\pgfpathlineto{\pgfqpoint{2.888079in}{0.643481in}}%
\pgfpathlineto{\pgfqpoint{2.888613in}{0.609618in}}%
\pgfpathlineto{\pgfqpoint{2.889146in}{0.625808in}}%
\pgfpathlineto{\pgfqpoint{2.889680in}{0.614588in}}%
\pgfpathlineto{\pgfqpoint{2.890214in}{0.645608in}}%
\pgfpathlineto{\pgfqpoint{2.890747in}{0.609786in}}%
\pgfpathlineto{\pgfqpoint{2.891281in}{0.612165in}}%
\pgfpathlineto{\pgfqpoint{2.891815in}{0.618507in}}%
\pgfpathlineto{\pgfqpoint{2.892349in}{0.615253in}}%
\pgfpathlineto{\pgfqpoint{2.892882in}{0.610078in}}%
\pgfpathlineto{\pgfqpoint{2.893416in}{0.618942in}}%
\pgfpathlineto{\pgfqpoint{2.893950in}{0.601805in}}%
\pgfpathlineto{\pgfqpoint{2.894483in}{0.607281in}}%
\pgfpathlineto{\pgfqpoint{2.895017in}{0.617423in}}%
\pgfpathlineto{\pgfqpoint{2.896618in}{0.602887in}}%
\pgfpathlineto{\pgfqpoint{2.897152in}{0.616141in}}%
\pgfpathlineto{\pgfqpoint{2.897686in}{0.604698in}}%
\pgfpathlineto{\pgfqpoint{2.898219in}{0.609360in}}%
\pgfpathlineto{\pgfqpoint{2.898753in}{0.601230in}}%
\pgfpathlineto{\pgfqpoint{2.899287in}{0.618744in}}%
\pgfpathlineto{\pgfqpoint{2.899820in}{0.607704in}}%
\pgfpathlineto{\pgfqpoint{2.900354in}{0.607437in}}%
\pgfpathlineto{\pgfqpoint{2.902489in}{0.632342in}}%
\pgfpathlineto{\pgfqpoint{2.903023in}{0.630351in}}%
\pgfpathlineto{\pgfqpoint{2.903556in}{0.634475in}}%
\pgfpathlineto{\pgfqpoint{2.904090in}{0.650347in}}%
\pgfpathlineto{\pgfqpoint{2.904624in}{0.639634in}}%
\pgfpathlineto{\pgfqpoint{2.905157in}{0.641923in}}%
\pgfpathlineto{\pgfqpoint{2.905691in}{0.629829in}}%
\pgfpathlineto{\pgfqpoint{2.906225in}{0.643672in}}%
\pgfpathlineto{\pgfqpoint{2.906758in}{0.642516in}}%
\pgfpathlineto{\pgfqpoint{2.907826in}{0.632411in}}%
\pgfpathlineto{\pgfqpoint{2.908893in}{0.609618in}}%
\pgfpathlineto{\pgfqpoint{2.909427in}{0.614797in}}%
\pgfpathlineto{\pgfqpoint{2.909961in}{0.613072in}}%
\pgfpathlineto{\pgfqpoint{2.911028in}{0.652341in}}%
\pgfpathlineto{\pgfqpoint{2.912629in}{0.613576in}}%
\pgfpathlineto{\pgfqpoint{2.914764in}{0.652916in}}%
\pgfpathlineto{\pgfqpoint{2.915298in}{0.612514in}}%
\pgfpathlineto{\pgfqpoint{2.915831in}{0.613429in}}%
\pgfpathlineto{\pgfqpoint{2.917432in}{0.640030in}}%
\pgfpathlineto{\pgfqpoint{2.918500in}{0.607841in}}%
\pgfpathlineto{\pgfqpoint{2.919034in}{0.614528in}}%
\pgfpathlineto{\pgfqpoint{2.919567in}{0.654144in}}%
\pgfpathlineto{\pgfqpoint{2.920101in}{0.629578in}}%
\pgfpathlineto{\pgfqpoint{2.920635in}{0.624509in}}%
\pgfpathlineto{\pgfqpoint{2.921168in}{0.625529in}}%
\pgfpathlineto{\pgfqpoint{2.921702in}{0.635324in}}%
\pgfpathlineto{\pgfqpoint{2.922236in}{0.606278in}}%
\pgfpathlineto{\pgfqpoint{2.922769in}{0.629863in}}%
\pgfpathlineto{\pgfqpoint{2.923303in}{0.628444in}}%
\pgfpathlineto{\pgfqpoint{2.923837in}{0.652555in}}%
\pgfpathlineto{\pgfqpoint{2.924371in}{0.608270in}}%
\pgfpathlineto{\pgfqpoint{2.924904in}{0.634088in}}%
\pgfpathlineto{\pgfqpoint{2.925438in}{0.661582in}}%
\pgfpathlineto{\pgfqpoint{2.925972in}{0.612346in}}%
\pgfpathlineto{\pgfqpoint{2.926505in}{0.669673in}}%
\pgfpathlineto{\pgfqpoint{2.927039in}{0.657359in}}%
\pgfpathlineto{\pgfqpoint{2.927573in}{0.612233in}}%
\pgfpathlineto{\pgfqpoint{2.928106in}{0.650330in}}%
\pgfpathlineto{\pgfqpoint{2.929174in}{0.661010in}}%
\pgfpathlineto{\pgfqpoint{2.929708in}{0.620917in}}%
\pgfpathlineto{\pgfqpoint{2.930241in}{0.673329in}}%
\pgfpathlineto{\pgfqpoint{2.930775in}{0.656962in}}%
\pgfpathlineto{\pgfqpoint{2.931309in}{0.637495in}}%
\pgfpathlineto{\pgfqpoint{2.931842in}{0.666193in}}%
\pgfpathlineto{\pgfqpoint{2.932376in}{0.638030in}}%
\pgfpathlineto{\pgfqpoint{2.932910in}{0.640470in}}%
\pgfpathlineto{\pgfqpoint{2.933977in}{0.623880in}}%
\pgfpathlineto{\pgfqpoint{2.935578in}{0.674656in}}%
\pgfpathlineto{\pgfqpoint{2.936112in}{0.662777in}}%
\pgfpathlineto{\pgfqpoint{2.936646in}{0.633398in}}%
\pgfpathlineto{\pgfqpoint{2.937179in}{0.648749in}}%
\pgfpathlineto{\pgfqpoint{2.937713in}{0.674624in}}%
\pgfpathlineto{\pgfqpoint{2.938247in}{0.665558in}}%
\pgfpathlineto{\pgfqpoint{2.939314in}{0.641937in}}%
\pgfpathlineto{\pgfqpoint{2.939848in}{0.643505in}}%
\pgfpathlineto{\pgfqpoint{2.940382in}{0.658631in}}%
\pgfpathlineto{\pgfqpoint{2.940915in}{0.650424in}}%
\pgfpathlineto{\pgfqpoint{2.941449in}{0.628511in}}%
\pgfpathlineto{\pgfqpoint{2.942516in}{0.630150in}}%
\pgfpathlineto{\pgfqpoint{2.943050in}{0.637724in}}%
\pgfpathlineto{\pgfqpoint{2.944651in}{0.601458in}}%
\pgfpathlineto{\pgfqpoint{2.945185in}{0.618143in}}%
\pgfpathlineto{\pgfqpoint{2.945719in}{0.616441in}}%
\pgfpathlineto{\pgfqpoint{2.946252in}{0.616122in}}%
\pgfpathlineto{\pgfqpoint{2.946786in}{0.606764in}}%
\pgfpathlineto{\pgfqpoint{2.947320in}{0.633617in}}%
\pgfpathlineto{\pgfqpoint{2.947853in}{0.614119in}}%
\pgfpathlineto{\pgfqpoint{2.948387in}{0.618263in}}%
\pgfpathlineto{\pgfqpoint{2.948921in}{0.610325in}}%
\pgfpathlineto{\pgfqpoint{2.949454in}{0.615660in}}%
\pgfpathlineto{\pgfqpoint{2.949988in}{0.619245in}}%
\pgfpathlineto{\pgfqpoint{2.950522in}{0.617543in}}%
\pgfpathlineto{\pgfqpoint{2.951056in}{0.612346in}}%
\pgfpathlineto{\pgfqpoint{2.951589in}{0.613352in}}%
\pgfpathlineto{\pgfqpoint{2.952123in}{0.618426in}}%
\pgfpathlineto{\pgfqpoint{2.952657in}{0.610455in}}%
\pgfpathlineto{\pgfqpoint{2.953190in}{0.619998in}}%
\pgfpathlineto{\pgfqpoint{2.954791in}{0.607116in}}%
\pgfpathlineto{\pgfqpoint{2.955325in}{0.602339in}}%
\pgfpathlineto{\pgfqpoint{2.956393in}{0.619260in}}%
\pgfpathlineto{\pgfqpoint{2.956926in}{0.616565in}}%
\pgfpathlineto{\pgfqpoint{2.957460in}{0.615686in}}%
\pgfpathlineto{\pgfqpoint{2.957994in}{0.608120in}}%
\pgfpathlineto{\pgfqpoint{2.958527in}{0.613966in}}%
\pgfpathlineto{\pgfqpoint{2.959061in}{0.613015in}}%
\pgfpathlineto{\pgfqpoint{2.959595in}{0.616444in}}%
\pgfpathlineto{\pgfqpoint{2.960662in}{0.636108in}}%
\pgfpathlineto{\pgfqpoint{2.961196in}{0.636094in}}%
\pgfpathlineto{\pgfqpoint{2.961730in}{0.634003in}}%
\pgfpathlineto{\pgfqpoint{2.962263in}{0.635734in}}%
\pgfpathlineto{\pgfqpoint{2.963331in}{0.650667in}}%
\pgfpathlineto{\pgfqpoint{2.963864in}{0.647149in}}%
\pgfpathlineto{\pgfqpoint{2.964398in}{0.632961in}}%
\pgfpathlineto{\pgfqpoint{2.964932in}{0.634160in}}%
\pgfpathlineto{\pgfqpoint{2.965466in}{0.639781in}}%
\pgfpathlineto{\pgfqpoint{2.967067in}{0.604672in}}%
\pgfpathlineto{\pgfqpoint{2.968668in}{0.641100in}}%
\pgfpathlineto{\pgfqpoint{2.970269in}{0.607397in}}%
\pgfpathlineto{\pgfqpoint{2.971870in}{0.640085in}}%
\pgfpathlineto{\pgfqpoint{2.972404in}{0.638666in}}%
\pgfpathlineto{\pgfqpoint{2.973471in}{0.606578in}}%
\pgfpathlineto{\pgfqpoint{2.974538in}{0.650553in}}%
\pgfpathlineto{\pgfqpoint{2.975072in}{0.632238in}}%
\pgfpathlineto{\pgfqpoint{2.976140in}{0.612803in}}%
\pgfpathlineto{\pgfqpoint{2.976673in}{0.630885in}}%
\pgfpathlineto{\pgfqpoint{2.977207in}{0.622118in}}%
\pgfpathlineto{\pgfqpoint{2.977741in}{0.627305in}}%
\pgfpathlineto{\pgfqpoint{2.978274in}{0.610623in}}%
\pgfpathlineto{\pgfqpoint{2.978808in}{0.652487in}}%
\pgfpathlineto{\pgfqpoint{2.979342in}{0.606992in}}%
\pgfpathlineto{\pgfqpoint{2.979875in}{0.640682in}}%
\pgfpathlineto{\pgfqpoint{2.981477in}{0.628708in}}%
\pgfpathlineto{\pgfqpoint{2.982010in}{0.676015in}}%
\pgfpathlineto{\pgfqpoint{2.982544in}{0.625036in}}%
\pgfpathlineto{\pgfqpoint{2.984145in}{0.672731in}}%
\pgfpathlineto{\pgfqpoint{2.986814in}{0.619195in}}%
\pgfpathlineto{\pgfqpoint{2.987347in}{0.677213in}}%
\pgfpathlineto{\pgfqpoint{2.987881in}{0.664094in}}%
\pgfpathlineto{\pgfqpoint{2.988415in}{0.608111in}}%
\pgfpathlineto{\pgfqpoint{2.988948in}{0.653074in}}%
\pgfpathlineto{\pgfqpoint{2.989482in}{0.622671in}}%
\pgfpathlineto{\pgfqpoint{2.990016in}{0.643927in}}%
\pgfpathlineto{\pgfqpoint{2.990549in}{0.673114in}}%
\pgfpathlineto{\pgfqpoint{2.991083in}{0.666659in}}%
\pgfpathlineto{\pgfqpoint{2.991617in}{0.644805in}}%
\pgfpathlineto{\pgfqpoint{2.992151in}{0.651670in}}%
\pgfpathlineto{\pgfqpoint{2.992684in}{0.653367in}}%
\pgfpathlineto{\pgfqpoint{2.993218in}{0.696564in}}%
\pgfpathlineto{\pgfqpoint{2.994819in}{0.610696in}}%
\pgfpathlineto{\pgfqpoint{2.995353in}{0.659018in}}%
\pgfpathlineto{\pgfqpoint{2.995886in}{0.622530in}}%
\pgfpathlineto{\pgfqpoint{2.996420in}{0.646031in}}%
\pgfpathlineto{\pgfqpoint{2.998021in}{0.608494in}}%
\pgfpathlineto{\pgfqpoint{2.998555in}{0.635104in}}%
\pgfpathlineto{\pgfqpoint{2.999089in}{0.621383in}}%
\pgfpathlineto{\pgfqpoint{2.999622in}{0.609368in}}%
\pgfpathlineto{\pgfqpoint{3.000156in}{0.614136in}}%
\pgfpathlineto{\pgfqpoint{3.001223in}{0.628690in}}%
\pgfpathlineto{\pgfqpoint{3.002825in}{0.605701in}}%
\pgfpathlineto{\pgfqpoint{3.005493in}{0.615005in}}%
\pgfpathlineto{\pgfqpoint{3.007094in}{0.605206in}}%
\pgfpathlineto{\pgfqpoint{3.007628in}{0.608428in}}%
\pgfpathlineto{\pgfqpoint{3.008162in}{0.614977in}}%
\pgfpathlineto{\pgfqpoint{3.008695in}{0.608055in}}%
\pgfpathlineto{\pgfqpoint{3.009229in}{0.605191in}}%
\pgfpathlineto{\pgfqpoint{3.009763in}{0.614587in}}%
\pgfpathlineto{\pgfqpoint{3.010296in}{0.613306in}}%
\pgfpathlineto{\pgfqpoint{3.010830in}{0.604094in}}%
\pgfpathlineto{\pgfqpoint{3.011364in}{0.609540in}}%
\pgfpathlineto{\pgfqpoint{3.014032in}{0.601572in}}%
\pgfpathlineto{\pgfqpoint{3.015633in}{0.614425in}}%
\pgfpathlineto{\pgfqpoint{3.016167in}{0.608151in}}%
\pgfpathlineto{\pgfqpoint{3.018836in}{0.635905in}}%
\pgfpathlineto{\pgfqpoint{3.019369in}{0.627013in}}%
\pgfpathlineto{\pgfqpoint{3.020437in}{0.647442in}}%
\pgfpathlineto{\pgfqpoint{3.020970in}{0.639948in}}%
\pgfpathlineto{\pgfqpoint{3.021504in}{0.638143in}}%
\pgfpathlineto{\pgfqpoint{3.022038in}{0.646536in}}%
\pgfpathlineto{\pgfqpoint{3.022571in}{0.645349in}}%
\pgfpathlineto{\pgfqpoint{3.024173in}{0.602367in}}%
\pgfpathlineto{\pgfqpoint{3.026307in}{0.655381in}}%
\pgfpathlineto{\pgfqpoint{3.027909in}{0.605241in}}%
\pgfpathlineto{\pgfqpoint{3.028976in}{0.653010in}}%
\pgfpathlineto{\pgfqpoint{3.029510in}{0.643106in}}%
\pgfpathlineto{\pgfqpoint{3.030043in}{0.640090in}}%
\pgfpathlineto{\pgfqpoint{3.030577in}{0.608349in}}%
\pgfpathlineto{\pgfqpoint{3.031111in}{0.614545in}}%
\pgfpathlineto{\pgfqpoint{3.032178in}{0.651437in}}%
\pgfpathlineto{\pgfqpoint{3.032712in}{0.607559in}}%
\pgfpathlineto{\pgfqpoint{3.033246in}{0.609873in}}%
\pgfpathlineto{\pgfqpoint{3.033779in}{0.651185in}}%
\pgfpathlineto{\pgfqpoint{3.034313in}{0.629872in}}%
\pgfpathlineto{\pgfqpoint{3.034847in}{0.630403in}}%
\pgfpathlineto{\pgfqpoint{3.035380in}{0.620898in}}%
\pgfpathlineto{\pgfqpoint{3.035914in}{0.657069in}}%
\pgfpathlineto{\pgfqpoint{3.036448in}{0.610019in}}%
\pgfpathlineto{\pgfqpoint{3.036981in}{0.680803in}}%
\pgfpathlineto{\pgfqpoint{3.037515in}{0.626750in}}%
\pgfpathlineto{\pgfqpoint{3.038049in}{0.624458in}}%
\pgfpathlineto{\pgfqpoint{3.039116in}{0.706610in}}%
\pgfpathlineto{\pgfqpoint{3.039650in}{0.686602in}}%
\pgfpathlineto{\pgfqpoint{3.040184in}{0.628458in}}%
\pgfpathlineto{\pgfqpoint{3.040717in}{0.678607in}}%
\pgfpathlineto{\pgfqpoint{3.041251in}{0.660422in}}%
\pgfpathlineto{\pgfqpoint{3.041785in}{0.695653in}}%
\pgfpathlineto{\pgfqpoint{3.042318in}{0.685998in}}%
\pgfpathlineto{\pgfqpoint{3.042852in}{0.673385in}}%
\pgfpathlineto{\pgfqpoint{3.043386in}{0.622579in}}%
\pgfpathlineto{\pgfqpoint{3.044453in}{0.623496in}}%
\pgfpathlineto{\pgfqpoint{3.044987in}{0.619086in}}%
\pgfpathlineto{\pgfqpoint{3.045521in}{0.633444in}}%
\pgfpathlineto{\pgfqpoint{3.046054in}{0.743056in}}%
\pgfpathlineto{\pgfqpoint{3.046588in}{0.687637in}}%
\pgfpathlineto{\pgfqpoint{3.047122in}{0.635031in}}%
\pgfpathlineto{\pgfqpoint{3.047655in}{0.656517in}}%
\pgfpathlineto{\pgfqpoint{3.048189in}{0.708520in}}%
\pgfpathlineto{\pgfqpoint{3.049790in}{0.630693in}}%
\pgfpathlineto{\pgfqpoint{3.051391in}{0.664957in}}%
\pgfpathlineto{\pgfqpoint{3.052992in}{0.606841in}}%
\pgfpathlineto{\pgfqpoint{3.053526in}{0.630777in}}%
\pgfpathlineto{\pgfqpoint{3.054060in}{0.623626in}}%
\pgfpathlineto{\pgfqpoint{3.054594in}{0.614366in}}%
\pgfpathlineto{\pgfqpoint{3.055127in}{0.622211in}}%
\pgfpathlineto{\pgfqpoint{3.056728in}{0.603032in}}%
\pgfpathlineto{\pgfqpoint{3.057262in}{0.620868in}}%
\pgfpathlineto{\pgfqpoint{3.058329in}{0.620256in}}%
\pgfpathlineto{\pgfqpoint{3.058863in}{0.623164in}}%
\pgfpathlineto{\pgfqpoint{3.059397in}{0.619236in}}%
\pgfpathlineto{\pgfqpoint{3.059931in}{0.602899in}}%
\pgfpathlineto{\pgfqpoint{3.060464in}{0.623089in}}%
\pgfpathlineto{\pgfqpoint{3.060998in}{0.610261in}}%
\pgfpathlineto{\pgfqpoint{3.061532in}{0.619173in}}%
\pgfpathlineto{\pgfqpoint{3.062065in}{0.607453in}}%
\pgfpathlineto{\pgfqpoint{3.062599in}{0.607765in}}%
\pgfpathlineto{\pgfqpoint{3.063133in}{0.609405in}}%
\pgfpathlineto{\pgfqpoint{3.063666in}{0.618499in}}%
\pgfpathlineto{\pgfqpoint{3.064200in}{0.610613in}}%
\pgfpathlineto{\pgfqpoint{3.065268in}{0.602931in}}%
\pgfpathlineto{\pgfqpoint{3.065801in}{0.604192in}}%
\pgfpathlineto{\pgfqpoint{3.066335in}{0.609035in}}%
\pgfpathlineto{\pgfqpoint{3.066869in}{0.608030in}}%
\pgfpathlineto{\pgfqpoint{3.067402in}{0.604664in}}%
\pgfpathlineto{\pgfqpoint{3.067936in}{0.619132in}}%
\pgfpathlineto{\pgfqpoint{3.068470in}{0.607966in}}%
\pgfpathlineto{\pgfqpoint{3.069003in}{0.612563in}}%
\pgfpathlineto{\pgfqpoint{3.069537in}{0.610666in}}%
\pgfpathlineto{\pgfqpoint{3.070605in}{0.609728in}}%
\pgfpathlineto{\pgfqpoint{3.071138in}{0.605491in}}%
\pgfpathlineto{\pgfqpoint{3.071672in}{0.614423in}}%
\pgfpathlineto{\pgfqpoint{3.072206in}{0.607150in}}%
\pgfpathlineto{\pgfqpoint{3.072739in}{0.613252in}}%
\pgfpathlineto{\pgfqpoint{3.073273in}{0.605706in}}%
\pgfpathlineto{\pgfqpoint{3.074340in}{0.605834in}}%
\pgfpathlineto{\pgfqpoint{3.076475in}{0.655393in}}%
\pgfpathlineto{\pgfqpoint{3.077009in}{0.633212in}}%
\pgfpathlineto{\pgfqpoint{3.077543in}{0.642774in}}%
\pgfpathlineto{\pgfqpoint{3.079144in}{0.664969in}}%
\pgfpathlineto{\pgfqpoint{3.079677in}{0.665170in}}%
\pgfpathlineto{\pgfqpoint{3.081812in}{0.608737in}}%
\pgfpathlineto{\pgfqpoint{3.082346in}{0.616327in}}%
\pgfpathlineto{\pgfqpoint{3.083947in}{0.676632in}}%
\pgfpathlineto{\pgfqpoint{3.085548in}{0.620970in}}%
\pgfpathlineto{\pgfqpoint{3.086082in}{0.650006in}}%
\pgfpathlineto{\pgfqpoint{3.086616in}{0.642764in}}%
\pgfpathlineto{\pgfqpoint{3.087149in}{0.641019in}}%
\pgfpathlineto{\pgfqpoint{3.088217in}{0.611566in}}%
\pgfpathlineto{\pgfqpoint{3.089284in}{0.674594in}}%
\pgfpathlineto{\pgfqpoint{3.089818in}{0.608306in}}%
\pgfpathlineto{\pgfqpoint{3.090351in}{0.621627in}}%
\pgfpathlineto{\pgfqpoint{3.090885in}{0.661514in}}%
\pgfpathlineto{\pgfqpoint{3.091419in}{0.619731in}}%
\pgfpathlineto{\pgfqpoint{3.092486in}{0.665990in}}%
\pgfpathlineto{\pgfqpoint{3.093554in}{0.631918in}}%
\pgfpathlineto{\pgfqpoint{3.094087in}{0.709592in}}%
\pgfpathlineto{\pgfqpoint{3.094621in}{0.685155in}}%
\pgfpathlineto{\pgfqpoint{3.095155in}{0.659654in}}%
\pgfpathlineto{\pgfqpoint{3.095689in}{0.697018in}}%
\pgfpathlineto{\pgfqpoint{3.096222in}{0.646640in}}%
\pgfpathlineto{\pgfqpoint{3.096756in}{0.708782in}}%
\pgfpathlineto{\pgfqpoint{3.097290in}{0.610151in}}%
\pgfpathlineto{\pgfqpoint{3.097823in}{0.685269in}}%
\pgfpathlineto{\pgfqpoint{3.099958in}{0.614650in}}%
\pgfpathlineto{\pgfqpoint{3.101026in}{0.743699in}}%
\pgfpathlineto{\pgfqpoint{3.101559in}{0.683437in}}%
\pgfpathlineto{\pgfqpoint{3.102093in}{0.630090in}}%
\pgfpathlineto{\pgfqpoint{3.102627in}{0.656655in}}%
\pgfpathlineto{\pgfqpoint{3.103160in}{0.661085in}}%
\pgfpathlineto{\pgfqpoint{3.103694in}{0.680305in}}%
\pgfpathlineto{\pgfqpoint{3.104228in}{0.671377in}}%
\pgfpathlineto{\pgfqpoint{3.104761in}{0.661639in}}%
\pgfpathlineto{\pgfqpoint{3.105295in}{0.627522in}}%
\pgfpathlineto{\pgfqpoint{3.105829in}{0.665728in}}%
\pgfpathlineto{\pgfqpoint{3.106363in}{0.627969in}}%
\pgfpathlineto{\pgfqpoint{3.109031in}{0.608099in}}%
\pgfpathlineto{\pgfqpoint{3.110632in}{0.627612in}}%
\pgfpathlineto{\pgfqpoint{3.111166in}{0.604629in}}%
\pgfpathlineto{\pgfqpoint{3.111700in}{0.611408in}}%
\pgfpathlineto{\pgfqpoint{3.112233in}{0.621060in}}%
\pgfpathlineto{\pgfqpoint{3.112767in}{0.618067in}}%
\pgfpathlineto{\pgfqpoint{3.113834in}{0.620296in}}%
\pgfpathlineto{\pgfqpoint{3.114368in}{0.604484in}}%
\pgfpathlineto{\pgfqpoint{3.114902in}{0.619617in}}%
\pgfpathlineto{\pgfqpoint{3.115435in}{0.610071in}}%
\pgfpathlineto{\pgfqpoint{3.115969in}{0.613685in}}%
\pgfpathlineto{\pgfqpoint{3.116503in}{0.615506in}}%
\pgfpathlineto{\pgfqpoint{3.117037in}{0.613239in}}%
\pgfpathlineto{\pgfqpoint{3.117570in}{0.628075in}}%
\pgfpathlineto{\pgfqpoint{3.118104in}{0.612418in}}%
\pgfpathlineto{\pgfqpoint{3.118638in}{0.604054in}}%
\pgfpathlineto{\pgfqpoint{3.119171in}{0.613838in}}%
\pgfpathlineto{\pgfqpoint{3.119705in}{0.610557in}}%
\pgfpathlineto{\pgfqpoint{3.120239in}{0.604073in}}%
\pgfpathlineto{\pgfqpoint{3.120772in}{0.617402in}}%
\pgfpathlineto{\pgfqpoint{3.121306in}{0.611447in}}%
\pgfpathlineto{\pgfqpoint{3.121840in}{0.610035in}}%
\pgfpathlineto{\pgfqpoint{3.122374in}{0.617367in}}%
\pgfpathlineto{\pgfqpoint{3.123441in}{0.606041in}}%
\pgfpathlineto{\pgfqpoint{3.123975in}{0.619882in}}%
\pgfpathlineto{\pgfqpoint{3.124508in}{0.618153in}}%
\pgfpathlineto{\pgfqpoint{3.125042in}{0.618541in}}%
\pgfpathlineto{\pgfqpoint{3.125576in}{0.617003in}}%
\pgfpathlineto{\pgfqpoint{3.126109in}{0.605957in}}%
\pgfpathlineto{\pgfqpoint{3.126643in}{0.612648in}}%
\pgfpathlineto{\pgfqpoint{3.128778in}{0.604476in}}%
\pgfpathlineto{\pgfqpoint{3.129312in}{0.618631in}}%
\pgfpathlineto{\pgfqpoint{3.129845in}{0.610667in}}%
\pgfpathlineto{\pgfqpoint{3.130379in}{0.616014in}}%
\pgfpathlineto{\pgfqpoint{3.130913in}{0.611733in}}%
\pgfpathlineto{\pgfqpoint{3.131446in}{0.605430in}}%
\pgfpathlineto{\pgfqpoint{3.131980in}{0.619055in}}%
\pgfpathlineto{\pgfqpoint{3.132514in}{0.617485in}}%
\pgfpathlineto{\pgfqpoint{3.133048in}{0.615190in}}%
\pgfpathlineto{\pgfqpoint{3.136783in}{0.663355in}}%
\pgfpathlineto{\pgfqpoint{3.138918in}{0.619056in}}%
\pgfpathlineto{\pgfqpoint{3.139452in}{0.618927in}}%
\pgfpathlineto{\pgfqpoint{3.139986in}{0.615144in}}%
\pgfpathlineto{\pgfqpoint{3.140519in}{0.617470in}}%
\pgfpathlineto{\pgfqpoint{3.141053in}{0.653218in}}%
\pgfpathlineto{\pgfqpoint{3.141587in}{0.650431in}}%
\pgfpathlineto{\pgfqpoint{3.142654in}{0.613228in}}%
\pgfpathlineto{\pgfqpoint{3.143188in}{0.624346in}}%
\pgfpathlineto{\pgfqpoint{3.143722in}{0.629783in}}%
\pgfpathlineto{\pgfqpoint{3.144255in}{0.658871in}}%
\pgfpathlineto{\pgfqpoint{3.145323in}{0.609741in}}%
\pgfpathlineto{\pgfqpoint{3.146390in}{0.643889in}}%
\pgfpathlineto{\pgfqpoint{3.146924in}{0.613469in}}%
\pgfpathlineto{\pgfqpoint{3.147457in}{0.656216in}}%
\pgfpathlineto{\pgfqpoint{3.147991in}{0.633988in}}%
\pgfpathlineto{\pgfqpoint{3.148525in}{0.618815in}}%
\pgfpathlineto{\pgfqpoint{3.149592in}{0.668758in}}%
\pgfpathlineto{\pgfqpoint{3.150126in}{0.656852in}}%
\pgfpathlineto{\pgfqpoint{3.150660in}{0.630717in}}%
\pgfpathlineto{\pgfqpoint{3.151193in}{0.634523in}}%
\pgfpathlineto{\pgfqpoint{3.152794in}{0.696446in}}%
\pgfpathlineto{\pgfqpoint{3.153862in}{0.649407in}}%
\pgfpathlineto{\pgfqpoint{3.154396in}{0.610431in}}%
\pgfpathlineto{\pgfqpoint{3.154929in}{0.617776in}}%
\pgfpathlineto{\pgfqpoint{3.155463in}{0.624702in}}%
\pgfpathlineto{\pgfqpoint{3.156530in}{0.707586in}}%
\pgfpathlineto{\pgfqpoint{3.158131in}{0.614940in}}%
\pgfpathlineto{\pgfqpoint{3.158665in}{0.682110in}}%
\pgfpathlineto{\pgfqpoint{3.159199in}{0.650577in}}%
\pgfpathlineto{\pgfqpoint{3.159733in}{0.660206in}}%
\pgfpathlineto{\pgfqpoint{3.161334in}{0.605000in}}%
\pgfpathlineto{\pgfqpoint{3.161867in}{0.628804in}}%
\pgfpathlineto{\pgfqpoint{3.162935in}{0.628079in}}%
\pgfpathlineto{\pgfqpoint{3.164002in}{0.607568in}}%
\pgfpathlineto{\pgfqpoint{3.164536in}{0.622104in}}%
\pgfpathlineto{\pgfqpoint{3.165070in}{0.615054in}}%
\pgfpathlineto{\pgfqpoint{3.165603in}{0.620682in}}%
\pgfpathlineto{\pgfqpoint{3.166137in}{0.601801in}}%
\pgfpathlineto{\pgfqpoint{3.166671in}{0.622668in}}%
\pgfpathlineto{\pgfqpoint{3.167204in}{0.614695in}}%
\pgfpathlineto{\pgfqpoint{3.168806in}{0.612924in}}%
\pgfpathlineto{\pgfqpoint{3.169873in}{0.616027in}}%
\pgfpathlineto{\pgfqpoint{3.170940in}{0.606120in}}%
\pgfpathlineto{\pgfqpoint{3.171474in}{0.610285in}}%
\pgfpathlineto{\pgfqpoint{3.172008in}{0.606825in}}%
\pgfpathlineto{\pgfqpoint{3.172541in}{0.604952in}}%
\pgfpathlineto{\pgfqpoint{3.174143in}{0.609575in}}%
\pgfpathlineto{\pgfqpoint{3.174676in}{0.604344in}}%
\pgfpathlineto{\pgfqpoint{3.175210in}{0.605610in}}%
\pgfpathlineto{\pgfqpoint{3.176277in}{0.615965in}}%
\pgfpathlineto{\pgfqpoint{3.177878in}{0.608889in}}%
\pgfpathlineto{\pgfqpoint{3.178412in}{0.614223in}}%
\pgfpathlineto{\pgfqpoint{3.178946in}{0.608680in}}%
\pgfpathlineto{\pgfqpoint{3.179480in}{0.603939in}}%
\pgfpathlineto{\pgfqpoint{3.180013in}{0.615067in}}%
\pgfpathlineto{\pgfqpoint{3.180547in}{0.611734in}}%
\pgfpathlineto{\pgfqpoint{3.181081in}{0.604864in}}%
\pgfpathlineto{\pgfqpoint{3.181614in}{0.612174in}}%
\pgfpathlineto{\pgfqpoint{3.182148in}{0.605201in}}%
\pgfpathlineto{\pgfqpoint{3.183215in}{0.619139in}}%
\pgfpathlineto{\pgfqpoint{3.183749in}{0.603022in}}%
\pgfpathlineto{\pgfqpoint{3.184283in}{0.622907in}}%
\pgfpathlineto{\pgfqpoint{3.184817in}{0.614919in}}%
\pgfpathlineto{\pgfqpoint{3.185350in}{0.616130in}}%
\pgfpathlineto{\pgfqpoint{3.185884in}{0.614279in}}%
\pgfpathlineto{\pgfqpoint{3.186418in}{0.608224in}}%
\pgfpathlineto{\pgfqpoint{3.186951in}{0.619791in}}%
\pgfpathlineto{\pgfqpoint{3.187485in}{0.600774in}}%
\pgfpathlineto{\pgfqpoint{3.188019in}{0.624457in}}%
\pgfpathlineto{\pgfqpoint{3.188552in}{0.611912in}}%
\pgfpathlineto{\pgfqpoint{3.189620in}{0.618366in}}%
\pgfpathlineto{\pgfqpoint{3.190154in}{0.611345in}}%
\pgfpathlineto{\pgfqpoint{3.190687in}{0.615115in}}%
\pgfpathlineto{\pgfqpoint{3.191755in}{0.614455in}}%
\pgfpathlineto{\pgfqpoint{3.192288in}{0.632479in}}%
\pgfpathlineto{\pgfqpoint{3.192822in}{0.629592in}}%
\pgfpathlineto{\pgfqpoint{3.195491in}{0.662856in}}%
\pgfpathlineto{\pgfqpoint{3.197092in}{0.618503in}}%
\pgfpathlineto{\pgfqpoint{3.197625in}{0.629455in}}%
\pgfpathlineto{\pgfqpoint{3.198159in}{0.665310in}}%
\pgfpathlineto{\pgfqpoint{3.198693in}{0.648341in}}%
\pgfpathlineto{\pgfqpoint{3.199760in}{0.623073in}}%
\pgfpathlineto{\pgfqpoint{3.200294in}{0.626920in}}%
\pgfpathlineto{\pgfqpoint{3.200828in}{0.627497in}}%
\pgfpathlineto{\pgfqpoint{3.201361in}{0.664461in}}%
\pgfpathlineto{\pgfqpoint{3.201895in}{0.621610in}}%
\pgfpathlineto{\pgfqpoint{3.202429in}{0.623317in}}%
\pgfpathlineto{\pgfqpoint{3.203496in}{0.632945in}}%
\pgfpathlineto{\pgfqpoint{3.204030in}{0.619048in}}%
\pgfpathlineto{\pgfqpoint{3.204563in}{0.682351in}}%
\pgfpathlineto{\pgfqpoint{3.205097in}{0.661428in}}%
\pgfpathlineto{\pgfqpoint{3.205631in}{0.621104in}}%
\pgfpathlineto{\pgfqpoint{3.207232in}{0.714188in}}%
\pgfpathlineto{\pgfqpoint{3.208833in}{0.629398in}}%
\pgfpathlineto{\pgfqpoint{3.209367in}{0.688494in}}%
\pgfpathlineto{\pgfqpoint{3.209900in}{0.656673in}}%
\pgfpathlineto{\pgfqpoint{3.210434in}{0.622166in}}%
\pgfpathlineto{\pgfqpoint{3.210968in}{0.656426in}}%
\pgfpathlineto{\pgfqpoint{3.211502in}{0.741986in}}%
\pgfpathlineto{\pgfqpoint{3.212035in}{0.640017in}}%
\pgfpathlineto{\pgfqpoint{3.213103in}{0.642885in}}%
\pgfpathlineto{\pgfqpoint{3.213636in}{0.653552in}}%
\pgfpathlineto{\pgfqpoint{3.214704in}{0.654939in}}%
\pgfpathlineto{\pgfqpoint{3.215237in}{0.612459in}}%
\pgfpathlineto{\pgfqpoint{3.215771in}{0.610307in}}%
\pgfpathlineto{\pgfqpoint{3.216305in}{0.610837in}}%
\pgfpathlineto{\pgfqpoint{3.216839in}{0.623432in}}%
\pgfpathlineto{\pgfqpoint{3.217372in}{0.616992in}}%
\pgfpathlineto{\pgfqpoint{3.217906in}{0.611043in}}%
\pgfpathlineto{\pgfqpoint{3.218440in}{0.617298in}}%
\pgfpathlineto{\pgfqpoint{3.218973in}{0.637799in}}%
\pgfpathlineto{\pgfqpoint{3.219507in}{0.606568in}}%
\pgfpathlineto{\pgfqpoint{3.220041in}{0.611346in}}%
\pgfpathlineto{\pgfqpoint{3.220574in}{0.617712in}}%
\pgfpathlineto{\pgfqpoint{3.221108in}{0.602599in}}%
\pgfpathlineto{\pgfqpoint{3.221642in}{0.624216in}}%
\pgfpathlineto{\pgfqpoint{3.222709in}{0.623237in}}%
\pgfpathlineto{\pgfqpoint{3.223243in}{0.608158in}}%
\pgfpathlineto{\pgfqpoint{3.223777in}{0.609204in}}%
\pgfpathlineto{\pgfqpoint{3.224310in}{0.626080in}}%
\pgfpathlineto{\pgfqpoint{3.224844in}{0.610547in}}%
\pgfpathlineto{\pgfqpoint{3.225912in}{0.609106in}}%
\pgfpathlineto{\pgfqpoint{3.226445in}{0.603978in}}%
\pgfpathlineto{\pgfqpoint{3.227513in}{0.626513in}}%
\pgfpathlineto{\pgfqpoint{3.228046in}{0.619148in}}%
\pgfpathlineto{\pgfqpoint{3.229114in}{0.619658in}}%
\pgfpathlineto{\pgfqpoint{3.229647in}{0.609068in}}%
\pgfpathlineto{\pgfqpoint{3.230181in}{0.623291in}}%
\pgfpathlineto{\pgfqpoint{3.230715in}{0.615476in}}%
\pgfpathlineto{\pgfqpoint{3.231249in}{0.606613in}}%
\pgfpathlineto{\pgfqpoint{3.231782in}{0.623789in}}%
\pgfpathlineto{\pgfqpoint{3.232850in}{0.622896in}}%
\pgfpathlineto{\pgfqpoint{3.233383in}{0.622923in}}%
\pgfpathlineto{\pgfqpoint{3.233917in}{0.607009in}}%
\pgfpathlineto{\pgfqpoint{3.234451in}{0.610619in}}%
\pgfpathlineto{\pgfqpoint{3.234984in}{0.610391in}}%
\pgfpathlineto{\pgfqpoint{3.236052in}{0.614467in}}%
\pgfpathlineto{\pgfqpoint{3.236586in}{0.625918in}}%
\pgfpathlineto{\pgfqpoint{3.237119in}{0.604829in}}%
\pgfpathlineto{\pgfqpoint{3.237653in}{0.617198in}}%
\pgfpathlineto{\pgfqpoint{3.238720in}{0.606066in}}%
\pgfpathlineto{\pgfqpoint{3.239788in}{0.619130in}}%
\pgfpathlineto{\pgfqpoint{3.241389in}{0.604727in}}%
\pgfpathlineto{\pgfqpoint{3.241923in}{0.615646in}}%
\pgfpathlineto{\pgfqpoint{3.242456in}{0.602505in}}%
\pgfpathlineto{\pgfqpoint{3.242990in}{0.607869in}}%
\pgfpathlineto{\pgfqpoint{3.243524in}{0.608009in}}%
\pgfpathlineto{\pgfqpoint{3.244057in}{0.616588in}}%
\pgfpathlineto{\pgfqpoint{3.245125in}{0.600611in}}%
\pgfpathlineto{\pgfqpoint{3.245658in}{0.616347in}}%
\pgfpathlineto{\pgfqpoint{3.246192in}{0.615009in}}%
\pgfpathlineto{\pgfqpoint{3.247260in}{0.606127in}}%
\pgfpathlineto{\pgfqpoint{3.247793in}{0.610058in}}%
\pgfpathlineto{\pgfqpoint{3.248327in}{0.617133in}}%
\pgfpathlineto{\pgfqpoint{3.248861in}{0.607255in}}%
\pgfpathlineto{\pgfqpoint{3.249394in}{0.616906in}}%
\pgfpathlineto{\pgfqpoint{3.251529in}{0.671329in}}%
\pgfpathlineto{\pgfqpoint{3.253130in}{0.651421in}}%
\pgfpathlineto{\pgfqpoint{3.253664in}{0.652526in}}%
\pgfpathlineto{\pgfqpoint{3.254198in}{0.644570in}}%
\pgfpathlineto{\pgfqpoint{3.254731in}{0.619856in}}%
\pgfpathlineto{\pgfqpoint{3.256332in}{0.655601in}}%
\pgfpathlineto{\pgfqpoint{3.257400in}{0.618780in}}%
\pgfpathlineto{\pgfqpoint{3.257934in}{0.627478in}}%
\pgfpathlineto{\pgfqpoint{3.259535in}{0.665488in}}%
\pgfpathlineto{\pgfqpoint{3.260068in}{0.669565in}}%
\pgfpathlineto{\pgfqpoint{3.260602in}{0.662209in}}%
\pgfpathlineto{\pgfqpoint{3.261136in}{0.665978in}}%
\pgfpathlineto{\pgfqpoint{3.261669in}{0.673489in}}%
\pgfpathlineto{\pgfqpoint{3.262203in}{0.704114in}}%
\pgfpathlineto{\pgfqpoint{3.262737in}{0.661734in}}%
\pgfpathlineto{\pgfqpoint{3.264338in}{0.765893in}}%
\pgfpathlineto{\pgfqpoint{3.265405in}{0.608329in}}%
\pgfpathlineto{\pgfqpoint{3.267006in}{0.718772in}}%
\pgfpathlineto{\pgfqpoint{3.268608in}{0.609022in}}%
\pgfpathlineto{\pgfqpoint{3.269141in}{0.654459in}}%
\pgfpathlineto{\pgfqpoint{3.269675in}{0.639124in}}%
\pgfpathlineto{\pgfqpoint{3.270209in}{0.638940in}}%
\pgfpathlineto{\pgfqpoint{3.271276in}{0.613086in}}%
\pgfpathlineto{\pgfqpoint{3.271810in}{0.621040in}}%
\pgfpathlineto{\pgfqpoint{3.272343in}{0.620517in}}%
\pgfpathlineto{\pgfqpoint{3.273411in}{0.607874in}}%
\pgfpathlineto{\pgfqpoint{3.273945in}{0.624702in}}%
\pgfpathlineto{\pgfqpoint{3.274478in}{0.615883in}}%
\pgfpathlineto{\pgfqpoint{3.275012in}{0.611889in}}%
\pgfpathlineto{\pgfqpoint{3.275546in}{0.626013in}}%
\pgfpathlineto{\pgfqpoint{3.276079in}{0.618501in}}%
\pgfpathlineto{\pgfqpoint{3.276613in}{0.615233in}}%
\pgfpathlineto{\pgfqpoint{3.277147in}{0.635209in}}%
\pgfpathlineto{\pgfqpoint{3.277680in}{0.618215in}}%
\pgfpathlineto{\pgfqpoint{3.278748in}{0.602280in}}%
\pgfpathlineto{\pgfqpoint{3.280349in}{0.615689in}}%
\pgfpathlineto{\pgfqpoint{3.281416in}{0.606410in}}%
\pgfpathlineto{\pgfqpoint{3.283017in}{0.631719in}}%
\pgfpathlineto{\pgfqpoint{3.284619in}{0.601771in}}%
\pgfpathlineto{\pgfqpoint{3.285152in}{0.626547in}}%
\pgfpathlineto{\pgfqpoint{3.285686in}{0.611734in}}%
\pgfpathlineto{\pgfqpoint{3.286220in}{0.613020in}}%
\pgfpathlineto{\pgfqpoint{3.287287in}{0.602741in}}%
\pgfpathlineto{\pgfqpoint{3.288354in}{0.624494in}}%
\pgfpathlineto{\pgfqpoint{3.288888in}{0.603470in}}%
\pgfpathlineto{\pgfqpoint{3.289422in}{0.618830in}}%
\pgfpathlineto{\pgfqpoint{3.290489in}{0.605210in}}%
\pgfpathlineto{\pgfqpoint{3.291023in}{0.629283in}}%
\pgfpathlineto{\pgfqpoint{3.291557in}{0.604413in}}%
\pgfpathlineto{\pgfqpoint{3.293692in}{0.617962in}}%
\pgfpathlineto{\pgfqpoint{3.294225in}{0.615233in}}%
\pgfpathlineto{\pgfqpoint{3.294759in}{0.624844in}}%
\pgfpathlineto{\pgfqpoint{3.295293in}{0.613764in}}%
\pgfpathlineto{\pgfqpoint{3.295826in}{0.614429in}}%
\pgfpathlineto{\pgfqpoint{3.296360in}{0.617459in}}%
\pgfpathlineto{\pgfqpoint{3.296894in}{0.605038in}}%
\pgfpathlineto{\pgfqpoint{3.297427in}{0.619012in}}%
\pgfpathlineto{\pgfqpoint{3.297961in}{0.617004in}}%
\pgfpathlineto{\pgfqpoint{3.298495in}{0.612229in}}%
\pgfpathlineto{\pgfqpoint{3.299029in}{0.622519in}}%
\pgfpathlineto{\pgfqpoint{3.299562in}{0.612322in}}%
\pgfpathlineto{\pgfqpoint{3.300096in}{0.603752in}}%
\pgfpathlineto{\pgfqpoint{3.301697in}{0.619666in}}%
\pgfpathlineto{\pgfqpoint{3.302231in}{0.601907in}}%
\pgfpathlineto{\pgfqpoint{3.302764in}{0.618389in}}%
\pgfpathlineto{\pgfqpoint{3.304366in}{0.606611in}}%
\pgfpathlineto{\pgfqpoint{3.304899in}{0.624470in}}%
\pgfpathlineto{\pgfqpoint{3.305433in}{0.612215in}}%
\pgfpathlineto{\pgfqpoint{3.305967in}{0.607074in}}%
\pgfpathlineto{\pgfqpoint{3.308101in}{0.649881in}}%
\pgfpathlineto{\pgfqpoint{3.308635in}{0.643685in}}%
\pgfpathlineto{\pgfqpoint{3.310236in}{0.732600in}}%
\pgfpathlineto{\pgfqpoint{3.312371in}{0.613867in}}%
\pgfpathlineto{\pgfqpoint{3.313438in}{0.699013in}}%
\pgfpathlineto{\pgfqpoint{3.313972in}{0.668525in}}%
\pgfpathlineto{\pgfqpoint{3.314506in}{0.637084in}}%
\pgfpathlineto{\pgfqpoint{3.315573in}{0.695938in}}%
\pgfpathlineto{\pgfqpoint{3.316107in}{0.664216in}}%
\pgfpathlineto{\pgfqpoint{3.317708in}{0.792667in}}%
\pgfpathlineto{\pgfqpoint{3.318775in}{0.635396in}}%
\pgfpathlineto{\pgfqpoint{3.319843in}{0.769406in}}%
\pgfpathlineto{\pgfqpoint{3.321444in}{0.655752in}}%
\pgfpathlineto{\pgfqpoint{3.321978in}{0.808848in}}%
\pgfpathlineto{\pgfqpoint{3.322511in}{0.677183in}}%
\pgfpathlineto{\pgfqpoint{3.323045in}{0.678850in}}%
\pgfpathlineto{\pgfqpoint{3.324646in}{0.608656in}}%
\pgfpathlineto{\pgfqpoint{3.325180in}{0.666169in}}%
\pgfpathlineto{\pgfqpoint{3.325714in}{0.627034in}}%
\pgfpathlineto{\pgfqpoint{3.326247in}{0.638303in}}%
\pgfpathlineto{\pgfqpoint{3.327848in}{0.608789in}}%
\pgfpathlineto{\pgfqpoint{3.328916in}{0.625219in}}%
\pgfpathlineto{\pgfqpoint{3.329449in}{0.607744in}}%
\pgfpathlineto{\pgfqpoint{3.329983in}{0.620774in}}%
\pgfpathlineto{\pgfqpoint{3.330517in}{0.609234in}}%
\pgfpathlineto{\pgfqpoint{3.331584in}{0.645214in}}%
\pgfpathlineto{\pgfqpoint{3.332118in}{0.637097in}}%
\pgfpathlineto{\pgfqpoint{3.332652in}{0.639480in}}%
\pgfpathlineto{\pgfqpoint{3.334253in}{0.618571in}}%
\pgfpathlineto{\pgfqpoint{3.334786in}{0.630005in}}%
\pgfpathlineto{\pgfqpoint{3.335320in}{0.617628in}}%
\pgfpathlineto{\pgfqpoint{3.336388in}{0.605035in}}%
\pgfpathlineto{\pgfqpoint{3.337989in}{0.613446in}}%
\pgfpathlineto{\pgfqpoint{3.338522in}{0.609247in}}%
\pgfpathlineto{\pgfqpoint{3.339056in}{0.621348in}}%
\pgfpathlineto{\pgfqpoint{3.339590in}{0.617171in}}%
\pgfpathlineto{\pgfqpoint{3.341725in}{0.604979in}}%
\pgfpathlineto{\pgfqpoint{3.342258in}{0.625004in}}%
\pgfpathlineto{\pgfqpoint{3.342792in}{0.617395in}}%
\pgfpathlineto{\pgfqpoint{3.343326in}{0.617338in}}%
\pgfpathlineto{\pgfqpoint{3.343859in}{0.621432in}}%
\pgfpathlineto{\pgfqpoint{3.344393in}{0.608335in}}%
\pgfpathlineto{\pgfqpoint{3.344927in}{0.617396in}}%
\pgfpathlineto{\pgfqpoint{3.345994in}{0.620744in}}%
\pgfpathlineto{\pgfqpoint{3.347062in}{0.638755in}}%
\pgfpathlineto{\pgfqpoint{3.348663in}{0.609673in}}%
\pgfpathlineto{\pgfqpoint{3.349196in}{0.631313in}}%
\pgfpathlineto{\pgfqpoint{3.349730in}{0.607565in}}%
\pgfpathlineto{\pgfqpoint{3.350264in}{0.610324in}}%
\pgfpathlineto{\pgfqpoint{3.350797in}{0.617491in}}%
\pgfpathlineto{\pgfqpoint{3.351331in}{0.612747in}}%
\pgfpathlineto{\pgfqpoint{3.351865in}{0.613837in}}%
\pgfpathlineto{\pgfqpoint{3.352399in}{0.606390in}}%
\pgfpathlineto{\pgfqpoint{3.352932in}{0.617937in}}%
\pgfpathlineto{\pgfqpoint{3.353466in}{0.607459in}}%
\pgfpathlineto{\pgfqpoint{3.354533in}{0.616484in}}%
\pgfpathlineto{\pgfqpoint{3.355067in}{0.604859in}}%
\pgfpathlineto{\pgfqpoint{3.355601in}{0.634445in}}%
\pgfpathlineto{\pgfqpoint{3.356134in}{0.618408in}}%
\pgfpathlineto{\pgfqpoint{3.356668in}{0.604651in}}%
\pgfpathlineto{\pgfqpoint{3.357202in}{0.621127in}}%
\pgfpathlineto{\pgfqpoint{3.357736in}{0.608852in}}%
\pgfpathlineto{\pgfqpoint{3.358269in}{0.602809in}}%
\pgfpathlineto{\pgfqpoint{3.358803in}{0.604384in}}%
\pgfpathlineto{\pgfqpoint{3.359337in}{0.621711in}}%
\pgfpathlineto{\pgfqpoint{3.359870in}{0.615005in}}%
\pgfpathlineto{\pgfqpoint{3.360404in}{0.619928in}}%
\pgfpathlineto{\pgfqpoint{3.360938in}{0.605356in}}%
\pgfpathlineto{\pgfqpoint{3.361472in}{0.620596in}}%
\pgfpathlineto{\pgfqpoint{3.362005in}{0.612468in}}%
\pgfpathlineto{\pgfqpoint{3.363073in}{0.631119in}}%
\pgfpathlineto{\pgfqpoint{3.363606in}{0.622135in}}%
\pgfpathlineto{\pgfqpoint{3.365207in}{0.641153in}}%
\pgfpathlineto{\pgfqpoint{3.365741in}{0.635414in}}%
\pgfpathlineto{\pgfqpoint{3.367876in}{0.733407in}}%
\pgfpathlineto{\pgfqpoint{3.369477in}{0.637144in}}%
\pgfpathlineto{\pgfqpoint{3.370011in}{0.648448in}}%
\pgfpathlineto{\pgfqpoint{3.371612in}{0.700814in}}%
\pgfpathlineto{\pgfqpoint{3.372146in}{0.678262in}}%
\pgfpathlineto{\pgfqpoint{3.372679in}{0.738418in}}%
\pgfpathlineto{\pgfqpoint{3.373213in}{0.628218in}}%
\pgfpathlineto{\pgfqpoint{3.373747in}{0.691867in}}%
\pgfpathlineto{\pgfqpoint{3.374280in}{0.718391in}}%
\pgfpathlineto{\pgfqpoint{3.374814in}{0.877713in}}%
\pgfpathlineto{\pgfqpoint{3.376415in}{0.650922in}}%
\pgfpathlineto{\pgfqpoint{3.376949in}{0.747727in}}%
\pgfpathlineto{\pgfqpoint{3.378550in}{0.615001in}}%
\pgfpathlineto{\pgfqpoint{3.379084in}{0.647990in}}%
\pgfpathlineto{\pgfqpoint{3.379617in}{0.604191in}}%
\pgfpathlineto{\pgfqpoint{3.380151in}{0.671343in}}%
\pgfpathlineto{\pgfqpoint{3.380685in}{0.619083in}}%
\pgfpathlineto{\pgfqpoint{3.381218in}{0.624757in}}%
\pgfpathlineto{\pgfqpoint{3.382820in}{0.609884in}}%
\pgfpathlineto{\pgfqpoint{3.383353in}{0.609981in}}%
\pgfpathlineto{\pgfqpoint{3.383887in}{0.612972in}}%
\pgfpathlineto{\pgfqpoint{3.384421in}{0.610134in}}%
\pgfpathlineto{\pgfqpoint{3.384954in}{0.636640in}}%
\pgfpathlineto{\pgfqpoint{3.385488in}{0.627318in}}%
\pgfpathlineto{\pgfqpoint{3.386022in}{0.602042in}}%
\pgfpathlineto{\pgfqpoint{3.386555in}{0.624698in}}%
\pgfpathlineto{\pgfqpoint{3.387089in}{0.635982in}}%
\pgfpathlineto{\pgfqpoint{3.389224in}{0.602353in}}%
\pgfpathlineto{\pgfqpoint{3.389758in}{0.625299in}}%
\pgfpathlineto{\pgfqpoint{3.390291in}{0.616695in}}%
\pgfpathlineto{\pgfqpoint{3.391892in}{0.606785in}}%
\pgfpathlineto{\pgfqpoint{3.392426in}{0.622845in}}%
\pgfpathlineto{\pgfqpoint{3.392960in}{0.613617in}}%
\pgfpathlineto{\pgfqpoint{3.394027in}{0.604022in}}%
\pgfpathlineto{\pgfqpoint{3.394561in}{0.630498in}}%
\pgfpathlineto{\pgfqpoint{3.395628in}{0.628386in}}%
\pgfpathlineto{\pgfqpoint{3.396162in}{0.601158in}}%
\pgfpathlineto{\pgfqpoint{3.396696in}{0.616163in}}%
\pgfpathlineto{\pgfqpoint{3.398297in}{0.652060in}}%
\pgfpathlineto{\pgfqpoint{3.398831in}{0.610898in}}%
\pgfpathlineto{\pgfqpoint{3.399364in}{0.648037in}}%
\pgfpathlineto{\pgfqpoint{3.400432in}{0.634512in}}%
\pgfpathlineto{\pgfqpoint{3.401499in}{0.617944in}}%
\pgfpathlineto{\pgfqpoint{3.403100in}{0.643807in}}%
\pgfpathlineto{\pgfqpoint{3.403634in}{0.628926in}}%
\pgfpathlineto{\pgfqpoint{3.404168in}{0.672919in}}%
\pgfpathlineto{\pgfqpoint{3.404701in}{0.631843in}}%
\pgfpathlineto{\pgfqpoint{3.405235in}{0.656408in}}%
\pgfpathlineto{\pgfqpoint{3.405769in}{0.634871in}}%
\pgfpathlineto{\pgfqpoint{3.406302in}{0.632447in}}%
\pgfpathlineto{\pgfqpoint{3.407370in}{0.604248in}}%
\pgfpathlineto{\pgfqpoint{3.408971in}{0.638656in}}%
\pgfpathlineto{\pgfqpoint{3.409505in}{0.612309in}}%
\pgfpathlineto{\pgfqpoint{3.410038in}{0.654327in}}%
\pgfpathlineto{\pgfqpoint{3.410572in}{0.622399in}}%
\pgfpathlineto{\pgfqpoint{3.411106in}{0.626872in}}%
\pgfpathlineto{\pgfqpoint{3.411639in}{0.620860in}}%
\pgfpathlineto{\pgfqpoint{3.412173in}{0.635470in}}%
\pgfpathlineto{\pgfqpoint{3.412707in}{0.613843in}}%
\pgfpathlineto{\pgfqpoint{3.413774in}{0.614988in}}%
\pgfpathlineto{\pgfqpoint{3.415375in}{0.632463in}}%
\pgfpathlineto{\pgfqpoint{3.415909in}{0.618537in}}%
\pgfpathlineto{\pgfqpoint{3.416443in}{0.644773in}}%
\pgfpathlineto{\pgfqpoint{3.416976in}{0.625898in}}%
\pgfpathlineto{\pgfqpoint{3.417510in}{0.616733in}}%
\pgfpathlineto{\pgfqpoint{3.418044in}{0.632760in}}%
\pgfpathlineto{\pgfqpoint{3.418577in}{0.613368in}}%
\pgfpathlineto{\pgfqpoint{3.419111in}{0.629426in}}%
\pgfpathlineto{\pgfqpoint{3.419645in}{0.635536in}}%
\pgfpathlineto{\pgfqpoint{3.420179in}{0.629778in}}%
\pgfpathlineto{\pgfqpoint{3.421780in}{0.608890in}}%
\pgfpathlineto{\pgfqpoint{3.422313in}{0.631865in}}%
\pgfpathlineto{\pgfqpoint{3.423381in}{0.631804in}}%
\pgfpathlineto{\pgfqpoint{3.424982in}{0.662189in}}%
\pgfpathlineto{\pgfqpoint{3.425516in}{0.650375in}}%
\pgfpathlineto{\pgfqpoint{3.426049in}{0.681038in}}%
\pgfpathlineto{\pgfqpoint{3.426583in}{0.667853in}}%
\pgfpathlineto{\pgfqpoint{3.427650in}{0.659027in}}%
\pgfpathlineto{\pgfqpoint{3.428718in}{0.689365in}}%
\pgfpathlineto{\pgfqpoint{3.429252in}{0.652148in}}%
\pgfpathlineto{\pgfqpoint{3.429785in}{0.813508in}}%
\pgfpathlineto{\pgfqpoint{3.430319in}{0.663152in}}%
\pgfpathlineto{\pgfqpoint{3.430853in}{0.626222in}}%
\pgfpathlineto{\pgfqpoint{3.431386in}{0.657894in}}%
\pgfpathlineto{\pgfqpoint{3.431920in}{0.637393in}}%
\pgfpathlineto{\pgfqpoint{3.432454in}{0.641944in}}%
\pgfpathlineto{\pgfqpoint{3.433521in}{0.667654in}}%
\pgfpathlineto{\pgfqpoint{3.434055in}{0.624564in}}%
\pgfpathlineto{\pgfqpoint{3.434589in}{0.661490in}}%
\pgfpathlineto{\pgfqpoint{3.435122in}{0.637364in}}%
\pgfpathlineto{\pgfqpoint{3.435656in}{0.652954in}}%
\pgfpathlineto{\pgfqpoint{3.436190in}{0.658105in}}%
\pgfpathlineto{\pgfqpoint{3.436723in}{0.653890in}}%
\pgfpathlineto{\pgfqpoint{3.437257in}{0.655883in}}%
\pgfpathlineto{\pgfqpoint{3.437791in}{0.646174in}}%
\pgfpathlineto{\pgfqpoint{3.439392in}{0.684584in}}%
\pgfpathlineto{\pgfqpoint{3.440459in}{0.621656in}}%
\pgfpathlineto{\pgfqpoint{3.440993in}{0.630487in}}%
\pgfpathlineto{\pgfqpoint{3.441527in}{0.607553in}}%
\pgfpathlineto{\pgfqpoint{3.442060in}{0.624426in}}%
\pgfpathlineto{\pgfqpoint{3.443661in}{0.680897in}}%
\pgfpathlineto{\pgfqpoint{3.444729in}{0.643240in}}%
\pgfpathlineto{\pgfqpoint{3.445796in}{0.644719in}}%
\pgfpathlineto{\pgfqpoint{3.446330in}{0.634651in}}%
\pgfpathlineto{\pgfqpoint{3.446864in}{0.637660in}}%
\pgfpathlineto{\pgfqpoint{3.447397in}{0.644260in}}%
\pgfpathlineto{\pgfqpoint{3.447931in}{0.615160in}}%
\pgfpathlineto{\pgfqpoint{3.448465in}{0.668066in}}%
\pgfpathlineto{\pgfqpoint{3.448998in}{0.630881in}}%
\pgfpathlineto{\pgfqpoint{3.449532in}{0.627928in}}%
\pgfpathlineto{\pgfqpoint{3.450066in}{0.606049in}}%
\pgfpathlineto{\pgfqpoint{3.450600in}{0.628264in}}%
\pgfpathlineto{\pgfqpoint{3.451667in}{0.643975in}}%
\pgfpathlineto{\pgfqpoint{3.453268in}{0.613257in}}%
\pgfpathlineto{\pgfqpoint{3.454335in}{0.672373in}}%
\pgfpathlineto{\pgfqpoint{3.455403in}{0.607716in}}%
\pgfpathlineto{\pgfqpoint{3.457004in}{0.660816in}}%
\pgfpathlineto{\pgfqpoint{3.458071in}{0.631714in}}%
\pgfpathlineto{\pgfqpoint{3.459139in}{0.673293in}}%
\pgfpathlineto{\pgfqpoint{3.459672in}{0.658165in}}%
\pgfpathlineto{\pgfqpoint{3.460206in}{0.611852in}}%
\pgfpathlineto{\pgfqpoint{3.460740in}{0.659521in}}%
\pgfpathlineto{\pgfqpoint{3.461274in}{0.673450in}}%
\pgfpathlineto{\pgfqpoint{3.461807in}{0.638305in}}%
\pgfpathlineto{\pgfqpoint{3.462875in}{0.640356in}}%
\pgfpathlineto{\pgfqpoint{3.463408in}{0.662895in}}%
\pgfpathlineto{\pgfqpoint{3.463942in}{0.655959in}}%
\pgfpathlineto{\pgfqpoint{3.464476in}{0.765131in}}%
\pgfpathlineto{\pgfqpoint{3.465009in}{0.630254in}}%
\pgfpathlineto{\pgfqpoint{3.465543in}{0.716234in}}%
\pgfpathlineto{\pgfqpoint{3.466611in}{0.781903in}}%
\pgfpathlineto{\pgfqpoint{3.468212in}{0.893532in}}%
\pgfpathlineto{\pgfqpoint{3.469813in}{0.727135in}}%
\pgfpathlineto{\pgfqpoint{3.470346in}{0.797656in}}%
\pgfpathlineto{\pgfqpoint{3.470880in}{0.786788in}}%
\pgfpathlineto{\pgfqpoint{3.471414in}{0.756914in}}%
\pgfpathlineto{\pgfqpoint{3.471948in}{0.890671in}}%
\pgfpathlineto{\pgfqpoint{3.472481in}{0.705258in}}%
\pgfpathlineto{\pgfqpoint{3.473015in}{0.902239in}}%
\pgfpathlineto{\pgfqpoint{3.474616in}{0.695217in}}%
\pgfpathlineto{\pgfqpoint{3.475150in}{0.801228in}}%
\pgfpathlineto{\pgfqpoint{3.475683in}{0.726375in}}%
\pgfpathlineto{\pgfqpoint{3.476217in}{0.676497in}}%
\pgfpathlineto{\pgfqpoint{3.476751in}{0.777356in}}%
\pgfpathlineto{\pgfqpoint{3.477285in}{0.620452in}}%
\pgfpathlineto{\pgfqpoint{3.477818in}{0.832360in}}%
\pgfpathlineto{\pgfqpoint{3.478352in}{0.700368in}}%
\pgfpathlineto{\pgfqpoint{3.478886in}{0.704452in}}%
\pgfpathlineto{\pgfqpoint{3.479419in}{0.680450in}}%
\pgfpathlineto{\pgfqpoint{3.479953in}{0.705217in}}%
\pgfpathlineto{\pgfqpoint{3.480487in}{0.696094in}}%
\pgfpathlineto{\pgfqpoint{3.481020in}{0.623317in}}%
\pgfpathlineto{\pgfqpoint{3.481554in}{0.700964in}}%
\pgfpathlineto{\pgfqpoint{3.482088in}{0.637501in}}%
\pgfpathlineto{\pgfqpoint{3.482622in}{0.682040in}}%
\pgfpathlineto{\pgfqpoint{3.483155in}{0.816745in}}%
\pgfpathlineto{\pgfqpoint{3.483689in}{0.765525in}}%
\pgfpathlineto{\pgfqpoint{3.484223in}{0.771217in}}%
\pgfpathlineto{\pgfqpoint{3.485290in}{0.981670in}}%
\pgfpathlineto{\pgfqpoint{3.486891in}{0.621865in}}%
\pgfpathlineto{\pgfqpoint{3.488492in}{0.726101in}}%
\pgfpathlineto{\pgfqpoint{3.489026in}{0.609347in}}%
\pgfpathlineto{\pgfqpoint{3.489560in}{0.623138in}}%
\pgfpathlineto{\pgfqpoint{3.490627in}{0.671848in}}%
\pgfpathlineto{\pgfqpoint{3.491161in}{0.632332in}}%
\pgfpathlineto{\pgfqpoint{3.491694in}{0.661911in}}%
\pgfpathlineto{\pgfqpoint{3.492228in}{0.659845in}}%
\pgfpathlineto{\pgfqpoint{3.493296in}{0.694405in}}%
\pgfpathlineto{\pgfqpoint{3.493829in}{0.620658in}}%
\pgfpathlineto{\pgfqpoint{3.494363in}{0.671873in}}%
\pgfpathlineto{\pgfqpoint{3.494897in}{0.653220in}}%
\pgfpathlineto{\pgfqpoint{3.495430in}{0.677360in}}%
\pgfpathlineto{\pgfqpoint{3.495964in}{0.661407in}}%
\pgfpathlineto{\pgfqpoint{3.496498in}{0.621488in}}%
\pgfpathlineto{\pgfqpoint{3.497565in}{0.620202in}}%
\pgfpathlineto{\pgfqpoint{3.498099in}{0.752025in}}%
\pgfpathlineto{\pgfqpoint{3.499166in}{0.716588in}}%
\pgfpathlineto{\pgfqpoint{3.499700in}{0.757533in}}%
\pgfpathlineto{\pgfqpoint{3.500234in}{0.634990in}}%
\pgfpathlineto{\pgfqpoint{3.500767in}{0.732086in}}%
\pgfpathlineto{\pgfqpoint{3.501301in}{0.714968in}}%
\pgfpathlineto{\pgfqpoint{3.501835in}{0.728230in}}%
\pgfpathlineto{\pgfqpoint{3.502369in}{0.758931in}}%
\pgfpathlineto{\pgfqpoint{3.502902in}{0.721002in}}%
\pgfpathlineto{\pgfqpoint{3.503436in}{0.772404in}}%
\pgfpathlineto{\pgfqpoint{3.504503in}{0.615716in}}%
\pgfpathlineto{\pgfqpoint{3.505037in}{0.688962in}}%
\pgfpathlineto{\pgfqpoint{3.505571in}{0.652719in}}%
\pgfpathlineto{\pgfqpoint{3.506638in}{0.639824in}}%
\pgfpathlineto{\pgfqpoint{3.507706in}{0.639311in}}%
\pgfpathlineto{\pgfqpoint{3.508239in}{0.671660in}}%
\pgfpathlineto{\pgfqpoint{3.509307in}{0.718460in}}%
\pgfpathlineto{\pgfqpoint{3.509840in}{0.682076in}}%
\pgfpathlineto{\pgfqpoint{3.510374in}{0.715297in}}%
\pgfpathlineto{\pgfqpoint{3.510908in}{0.723996in}}%
\pgfpathlineto{\pgfqpoint{3.511975in}{0.656170in}}%
\pgfpathlineto{\pgfqpoint{3.513576in}{0.794791in}}%
\pgfpathlineto{\pgfqpoint{3.514110in}{0.756756in}}%
\pgfpathlineto{\pgfqpoint{3.514644in}{0.931036in}}%
\pgfpathlineto{\pgfqpoint{3.515177in}{0.907741in}}%
\pgfpathlineto{\pgfqpoint{3.516778in}{0.707442in}}%
\pgfpathlineto{\pgfqpoint{3.517846in}{0.617281in}}%
\pgfpathlineto{\pgfqpoint{3.518380in}{0.825430in}}%
\pgfpathlineto{\pgfqpoint{3.518913in}{0.700834in}}%
\pgfpathlineto{\pgfqpoint{3.519447in}{0.711166in}}%
\pgfpathlineto{\pgfqpoint{3.519981in}{0.750515in}}%
\pgfpathlineto{\pgfqpoint{3.520514in}{0.873871in}}%
\pgfpathlineto{\pgfqpoint{3.521582in}{0.698625in}}%
\pgfpathlineto{\pgfqpoint{3.522115in}{0.735827in}}%
\pgfpathlineto{\pgfqpoint{3.522649in}{0.869635in}}%
\pgfpathlineto{\pgfqpoint{3.523183in}{0.696496in}}%
\pgfpathlineto{\pgfqpoint{3.523717in}{0.870063in}}%
\pgfpathlineto{\pgfqpoint{3.524250in}{0.701991in}}%
\pgfpathlineto{\pgfqpoint{3.524784in}{0.633961in}}%
\pgfpathlineto{\pgfqpoint{3.525851in}{0.894025in}}%
\pgfpathlineto{\pgfqpoint{3.527452in}{0.638718in}}%
\pgfpathlineto{\pgfqpoint{3.527986in}{0.758969in}}%
\pgfpathlineto{\pgfqpoint{3.529587in}{0.865252in}}%
\pgfpathlineto{\pgfqpoint{3.530121in}{1.118105in}}%
\pgfpathlineto{\pgfqpoint{3.531188in}{0.732029in}}%
\pgfpathlineto{\pgfqpoint{3.531722in}{1.315576in}}%
\pgfpathlineto{\pgfqpoint{3.532256in}{1.154160in}}%
\pgfpathlineto{\pgfqpoint{3.532789in}{1.113822in}}%
\pgfpathlineto{\pgfqpoint{3.534391in}{0.612381in}}%
\pgfpathlineto{\pgfqpoint{3.535458in}{0.895303in}}%
\pgfpathlineto{\pgfqpoint{3.536525in}{0.622096in}}%
\pgfpathlineto{\pgfqpoint{3.537059in}{0.679791in}}%
\pgfpathlineto{\pgfqpoint{3.537593in}{0.738287in}}%
\pgfpathlineto{\pgfqpoint{3.538126in}{0.628996in}}%
\pgfpathlineto{\pgfqpoint{3.538660in}{0.657338in}}%
\pgfpathlineto{\pgfqpoint{3.539194in}{0.661903in}}%
\pgfpathlineto{\pgfqpoint{3.539728in}{0.623666in}}%
\pgfpathlineto{\pgfqpoint{3.539728in}{0.623666in}}%
\pgfusepath{stroke}%
\end{pgfscope}%
\begin{pgfscope}%
\pgfsetrectcap%
\pgfsetmiterjoin%
\pgfsetlinewidth{0.803000pt}%
\definecolor{currentstroke}{rgb}{0.000000,0.000000,0.000000}%
\pgfsetstrokecolor{currentstroke}%
\pgfsetdash{}{0pt}%
\pgfpathmoveto{\pgfqpoint{0.934300in}{0.564143in}}%
\pgfpathlineto{\pgfqpoint{0.934300in}{1.351359in}}%
\pgfusepath{stroke}%
\end{pgfscope}%
\begin{pgfscope}%
\pgfsetrectcap%
\pgfsetmiterjoin%
\pgfsetlinewidth{0.803000pt}%
\definecolor{currentstroke}{rgb}{0.000000,0.000000,0.000000}%
\pgfsetstrokecolor{currentstroke}%
\pgfsetdash{}{0pt}%
\pgfpathmoveto{\pgfqpoint{6.146222in}{0.564143in}}%
\pgfpathlineto{\pgfqpoint{6.146222in}{1.351359in}}%
\pgfusepath{stroke}%
\end{pgfscope}%
\begin{pgfscope}%
\pgfsetrectcap%
\pgfsetmiterjoin%
\pgfsetlinewidth{0.803000pt}%
\definecolor{currentstroke}{rgb}{0.000000,0.000000,0.000000}%
\pgfsetstrokecolor{currentstroke}%
\pgfsetdash{}{0pt}%
\pgfpathmoveto{\pgfqpoint{0.934300in}{0.564143in}}%
\pgfpathlineto{\pgfqpoint{6.146222in}{0.564143in}}%
\pgfusepath{stroke}%
\end{pgfscope}%
\begin{pgfscope}%
\pgfsetrectcap%
\pgfsetmiterjoin%
\pgfsetlinewidth{0.803000pt}%
\definecolor{currentstroke}{rgb}{0.000000,0.000000,0.000000}%
\pgfsetstrokecolor{currentstroke}%
\pgfsetdash{}{0pt}%
\pgfpathmoveto{\pgfqpoint{0.934300in}{1.351359in}}%
\pgfpathlineto{\pgfqpoint{6.146222in}{1.351359in}}%
\pgfusepath{stroke}%
\end{pgfscope}%
\begin{pgfscope}%
\definecolor{textcolor}{rgb}{0.000000,0.000000,0.000000}%
\pgfsetstrokecolor{textcolor}%
\pgfsetfillcolor{textcolor}%
\pgftext[x=3.540261in,y=1.434692in,,base]{\color{textcolor}\rmfamily\fontsize{12.000000}{14.400000}\selectfont Spectrum of Filtered ECG Signal}%
\end{pgfscope}%
\end{pgfpicture}%
\makeatother%
\endgroup%

    }
    \caption{Frequency response of the optimal FIR filter (top), optimal filtered ECG signal (middle), and spectrum of filtered ECG signal (bottom).}
    \label{fig:fir-optimal}
\end{figure}

\subsection{Frequency Sampled FIR Filter}
\begin{figure}[H]
    \centering
    \adjustbox{max width=0.75\textwidth}{
    %% Creator: Matplotlib, PGF backend
%%
%% To include the figure in your LaTeX document, write
%%   \input{<filename>.pgf}
%%
%% Make sure the required packages are loaded in your preamble
%%   \usepackage{pgf}
%%
%% and, on pdftex
%%   \usepackage[utf8]{inputenc}\DeclareUnicodeCharacter{2212}{-}
%%
%% or, on luatex and xetex
%%   \usepackage{unicode-math}
%%
%% Figures using additional raster images can only be included by \input if
%% they are in the same directory as the main LaTeX file. For loading figures
%% from other directories you can use the `import` package
%%   \usepackage{import}
%%
%% and then include the figures with
%%   \import{<path to file>}{<filename>.pgf}
%%
%% Matplotlib used the following preamble
%%
\begingroup%
\makeatletter%
\begin{pgfpicture}%
\pgfpathrectangle{\pgfpointorigin}{\pgfqpoint{6.400000in}{4.800000in}}%
\pgfusepath{use as bounding box, clip}%
\begin{pgfscope}%
\pgfsetbuttcap%
\pgfsetmiterjoin%
\definecolor{currentfill}{rgb}{1.000000,1.000000,1.000000}%
\pgfsetfillcolor{currentfill}%
\pgfsetlinewidth{0.000000pt}%
\definecolor{currentstroke}{rgb}{1.000000,1.000000,1.000000}%
\pgfsetstrokecolor{currentstroke}%
\pgfsetdash{}{0pt}%
\pgfpathmoveto{\pgfqpoint{0.000000in}{0.000000in}}%
\pgfpathlineto{\pgfqpoint{6.400000in}{0.000000in}}%
\pgfpathlineto{\pgfqpoint{6.400000in}{4.800000in}}%
\pgfpathlineto{\pgfqpoint{0.000000in}{4.800000in}}%
\pgfpathclose%
\pgfusepath{fill}%
\end{pgfscope}%
\begin{pgfscope}%
\pgfsetbuttcap%
\pgfsetmiterjoin%
\definecolor{currentfill}{rgb}{1.000000,1.000000,1.000000}%
\pgfsetfillcolor{currentfill}%
\pgfsetlinewidth{0.000000pt}%
\definecolor{currentstroke}{rgb}{0.000000,0.000000,0.000000}%
\pgfsetstrokecolor{currentstroke}%
\pgfsetstrokeopacity{0.000000}%
\pgfsetdash{}{0pt}%
\pgfpathmoveto{\pgfqpoint{0.717889in}{3.664143in}}%
\pgfpathlineto{\pgfqpoint{6.146222in}{3.664143in}}%
\pgfpathlineto{\pgfqpoint{6.146222in}{4.451359in}}%
\pgfpathlineto{\pgfqpoint{0.717889in}{4.451359in}}%
\pgfpathclose%
\pgfusepath{fill}%
\end{pgfscope}%
\begin{pgfscope}%
\pgfsetbuttcap%
\pgfsetroundjoin%
\definecolor{currentfill}{rgb}{0.000000,0.000000,0.000000}%
\pgfsetfillcolor{currentfill}%
\pgfsetlinewidth{0.803000pt}%
\definecolor{currentstroke}{rgb}{0.000000,0.000000,0.000000}%
\pgfsetstrokecolor{currentstroke}%
\pgfsetdash{}{0pt}%
\pgfsys@defobject{currentmarker}{\pgfqpoint{0.000000in}{-0.048611in}}{\pgfqpoint{0.000000in}{0.000000in}}{%
\pgfpathmoveto{\pgfqpoint{0.000000in}{0.000000in}}%
\pgfpathlineto{\pgfqpoint{0.000000in}{-0.048611in}}%
\pgfusepath{stroke,fill}%
}%
\begin{pgfscope}%
\pgfsys@transformshift{0.717889in}{3.664143in}%
\pgfsys@useobject{currentmarker}{}%
\end{pgfscope}%
\end{pgfscope}%
\begin{pgfscope}%
\definecolor{textcolor}{rgb}{0.000000,0.000000,0.000000}%
\pgfsetstrokecolor{textcolor}%
\pgfsetfillcolor{textcolor}%
\pgftext[x=0.717889in,y=3.566921in,,top]{\color{textcolor}\rmfamily\fontsize{10.000000}{12.000000}\selectfont \(\displaystyle {0}\)}%
\end{pgfscope}%
\begin{pgfscope}%
\pgfsetbuttcap%
\pgfsetroundjoin%
\definecolor{currentfill}{rgb}{0.000000,0.000000,0.000000}%
\pgfsetfillcolor{currentfill}%
\pgfsetlinewidth{0.803000pt}%
\definecolor{currentstroke}{rgb}{0.000000,0.000000,0.000000}%
\pgfsetstrokecolor{currentstroke}%
\pgfsetdash{}{0pt}%
\pgfsys@defobject{currentmarker}{\pgfqpoint{0.000000in}{-0.048611in}}{\pgfqpoint{0.000000in}{0.000000in}}{%
\pgfpathmoveto{\pgfqpoint{0.000000in}{0.000000in}}%
\pgfpathlineto{\pgfqpoint{0.000000in}{-0.048611in}}%
\pgfusepath{stroke,fill}%
}%
\begin{pgfscope}%
\pgfsys@transformshift{1.803555in}{3.664143in}%
\pgfsys@useobject{currentmarker}{}%
\end{pgfscope}%
\end{pgfscope}%
\begin{pgfscope}%
\definecolor{textcolor}{rgb}{0.000000,0.000000,0.000000}%
\pgfsetstrokecolor{textcolor}%
\pgfsetfillcolor{textcolor}%
\pgftext[x=1.803555in,y=3.566921in,,top]{\color{textcolor}\rmfamily\fontsize{10.000000}{12.000000}\selectfont \(\displaystyle {20}\)}%
\end{pgfscope}%
\begin{pgfscope}%
\pgfsetbuttcap%
\pgfsetroundjoin%
\definecolor{currentfill}{rgb}{0.000000,0.000000,0.000000}%
\pgfsetfillcolor{currentfill}%
\pgfsetlinewidth{0.803000pt}%
\definecolor{currentstroke}{rgb}{0.000000,0.000000,0.000000}%
\pgfsetstrokecolor{currentstroke}%
\pgfsetdash{}{0pt}%
\pgfsys@defobject{currentmarker}{\pgfqpoint{0.000000in}{-0.048611in}}{\pgfqpoint{0.000000in}{0.000000in}}{%
\pgfpathmoveto{\pgfqpoint{0.000000in}{0.000000in}}%
\pgfpathlineto{\pgfqpoint{0.000000in}{-0.048611in}}%
\pgfusepath{stroke,fill}%
}%
\begin{pgfscope}%
\pgfsys@transformshift{2.889222in}{3.664143in}%
\pgfsys@useobject{currentmarker}{}%
\end{pgfscope}%
\end{pgfscope}%
\begin{pgfscope}%
\definecolor{textcolor}{rgb}{0.000000,0.000000,0.000000}%
\pgfsetstrokecolor{textcolor}%
\pgfsetfillcolor{textcolor}%
\pgftext[x=2.889222in,y=3.566921in,,top]{\color{textcolor}\rmfamily\fontsize{10.000000}{12.000000}\selectfont \(\displaystyle {40}\)}%
\end{pgfscope}%
\begin{pgfscope}%
\pgfsetbuttcap%
\pgfsetroundjoin%
\definecolor{currentfill}{rgb}{0.000000,0.000000,0.000000}%
\pgfsetfillcolor{currentfill}%
\pgfsetlinewidth{0.803000pt}%
\definecolor{currentstroke}{rgb}{0.000000,0.000000,0.000000}%
\pgfsetstrokecolor{currentstroke}%
\pgfsetdash{}{0pt}%
\pgfsys@defobject{currentmarker}{\pgfqpoint{0.000000in}{-0.048611in}}{\pgfqpoint{0.000000in}{0.000000in}}{%
\pgfpathmoveto{\pgfqpoint{0.000000in}{0.000000in}}%
\pgfpathlineto{\pgfqpoint{0.000000in}{-0.048611in}}%
\pgfusepath{stroke,fill}%
}%
\begin{pgfscope}%
\pgfsys@transformshift{3.974889in}{3.664143in}%
\pgfsys@useobject{currentmarker}{}%
\end{pgfscope}%
\end{pgfscope}%
\begin{pgfscope}%
\definecolor{textcolor}{rgb}{0.000000,0.000000,0.000000}%
\pgfsetstrokecolor{textcolor}%
\pgfsetfillcolor{textcolor}%
\pgftext[x=3.974889in,y=3.566921in,,top]{\color{textcolor}\rmfamily\fontsize{10.000000}{12.000000}\selectfont \(\displaystyle {60}\)}%
\end{pgfscope}%
\begin{pgfscope}%
\pgfsetbuttcap%
\pgfsetroundjoin%
\definecolor{currentfill}{rgb}{0.000000,0.000000,0.000000}%
\pgfsetfillcolor{currentfill}%
\pgfsetlinewidth{0.803000pt}%
\definecolor{currentstroke}{rgb}{0.000000,0.000000,0.000000}%
\pgfsetstrokecolor{currentstroke}%
\pgfsetdash{}{0pt}%
\pgfsys@defobject{currentmarker}{\pgfqpoint{0.000000in}{-0.048611in}}{\pgfqpoint{0.000000in}{0.000000in}}{%
\pgfpathmoveto{\pgfqpoint{0.000000in}{0.000000in}}%
\pgfpathlineto{\pgfqpoint{0.000000in}{-0.048611in}}%
\pgfusepath{stroke,fill}%
}%
\begin{pgfscope}%
\pgfsys@transformshift{5.060555in}{3.664143in}%
\pgfsys@useobject{currentmarker}{}%
\end{pgfscope}%
\end{pgfscope}%
\begin{pgfscope}%
\definecolor{textcolor}{rgb}{0.000000,0.000000,0.000000}%
\pgfsetstrokecolor{textcolor}%
\pgfsetfillcolor{textcolor}%
\pgftext[x=5.060555in,y=3.566921in,,top]{\color{textcolor}\rmfamily\fontsize{10.000000}{12.000000}\selectfont \(\displaystyle {80}\)}%
\end{pgfscope}%
\begin{pgfscope}%
\pgfsetbuttcap%
\pgfsetroundjoin%
\definecolor{currentfill}{rgb}{0.000000,0.000000,0.000000}%
\pgfsetfillcolor{currentfill}%
\pgfsetlinewidth{0.803000pt}%
\definecolor{currentstroke}{rgb}{0.000000,0.000000,0.000000}%
\pgfsetstrokecolor{currentstroke}%
\pgfsetdash{}{0pt}%
\pgfsys@defobject{currentmarker}{\pgfqpoint{0.000000in}{-0.048611in}}{\pgfqpoint{0.000000in}{0.000000in}}{%
\pgfpathmoveto{\pgfqpoint{0.000000in}{0.000000in}}%
\pgfpathlineto{\pgfqpoint{0.000000in}{-0.048611in}}%
\pgfusepath{stroke,fill}%
}%
\begin{pgfscope}%
\pgfsys@transformshift{6.146222in}{3.664143in}%
\pgfsys@useobject{currentmarker}{}%
\end{pgfscope}%
\end{pgfscope}%
\begin{pgfscope}%
\definecolor{textcolor}{rgb}{0.000000,0.000000,0.000000}%
\pgfsetstrokecolor{textcolor}%
\pgfsetfillcolor{textcolor}%
\pgftext[x=6.146222in,y=3.566921in,,top]{\color{textcolor}\rmfamily\fontsize{10.000000}{12.000000}\selectfont \(\displaystyle {100}\)}%
\end{pgfscope}%
\begin{pgfscope}%
\definecolor{textcolor}{rgb}{0.000000,0.000000,0.000000}%
\pgfsetstrokecolor{textcolor}%
\pgfsetfillcolor{textcolor}%
\pgftext[x=3.432055in,y=3.387909in,,top]{\color{textcolor}\rmfamily\fontsize{10.000000}{12.000000}\selectfont Frequency (Hz)}%
\end{pgfscope}%
\begin{pgfscope}%
\pgfsetbuttcap%
\pgfsetroundjoin%
\definecolor{currentfill}{rgb}{0.000000,0.000000,0.000000}%
\pgfsetfillcolor{currentfill}%
\pgfsetlinewidth{0.803000pt}%
\definecolor{currentstroke}{rgb}{0.000000,0.000000,0.000000}%
\pgfsetstrokecolor{currentstroke}%
\pgfsetdash{}{0pt}%
\pgfsys@defobject{currentmarker}{\pgfqpoint{-0.048611in}{0.000000in}}{\pgfqpoint{0.000000in}{0.000000in}}{%
\pgfpathmoveto{\pgfqpoint{0.000000in}{0.000000in}}%
\pgfpathlineto{\pgfqpoint{-0.048611in}{0.000000in}}%
\pgfusepath{stroke,fill}%
}%
\begin{pgfscope}%
\pgfsys@transformshift{0.717889in}{3.706274in}%
\pgfsys@useobject{currentmarker}{}%
\end{pgfscope}%
\end{pgfscope}%
\begin{pgfscope}%
\definecolor{textcolor}{rgb}{0.000000,0.000000,0.000000}%
\pgfsetstrokecolor{textcolor}%
\pgfsetfillcolor{textcolor}%
\pgftext[x=0.373752in, y=3.658049in, left, base]{\color{textcolor}\rmfamily\fontsize{10.000000}{12.000000}\selectfont \(\displaystyle {-20}\)}%
\end{pgfscope}%
\begin{pgfscope}%
\pgfsetbuttcap%
\pgfsetroundjoin%
\definecolor{currentfill}{rgb}{0.000000,0.000000,0.000000}%
\pgfsetfillcolor{currentfill}%
\pgfsetlinewidth{0.803000pt}%
\definecolor{currentstroke}{rgb}{0.000000,0.000000,0.000000}%
\pgfsetstrokecolor{currentstroke}%
\pgfsetdash{}{0pt}%
\pgfsys@defobject{currentmarker}{\pgfqpoint{-0.048611in}{0.000000in}}{\pgfqpoint{0.000000in}{0.000000in}}{%
\pgfpathmoveto{\pgfqpoint{0.000000in}{0.000000in}}%
\pgfpathlineto{\pgfqpoint{-0.048611in}{0.000000in}}%
\pgfusepath{stroke,fill}%
}%
\begin{pgfscope}%
\pgfsys@transformshift{0.717889in}{4.415419in}%
\pgfsys@useobject{currentmarker}{}%
\end{pgfscope}%
\end{pgfscope}%
\begin{pgfscope}%
\definecolor{textcolor}{rgb}{0.000000,0.000000,0.000000}%
\pgfsetstrokecolor{textcolor}%
\pgfsetfillcolor{textcolor}%
\pgftext[x=0.551222in, y=4.367194in, left, base]{\color{textcolor}\rmfamily\fontsize{10.000000}{12.000000}\selectfont \(\displaystyle {0}\)}%
\end{pgfscope}%
\begin{pgfscope}%
\definecolor{textcolor}{rgb}{0.000000,0.000000,0.000000}%
\pgfsetstrokecolor{textcolor}%
\pgfsetfillcolor{textcolor}%
\pgftext[x=0.318197in,y=4.057751in,,bottom,rotate=90.000000]{\color{textcolor}\rmfamily\fontsize{10.000000}{12.000000}\selectfont Amplitude (dB)}%
\end{pgfscope}%
\begin{pgfscope}%
\pgfpathrectangle{\pgfqpoint{0.717889in}{3.664143in}}{\pgfqpoint{5.428334in}{0.787215in}}%
\pgfusepath{clip}%
\pgfsetrectcap%
\pgfsetroundjoin%
\pgfsetlinewidth{1.505625pt}%
\definecolor{currentstroke}{rgb}{0.121569,0.466667,0.705882}%
\pgfsetstrokecolor{currentstroke}%
\pgfsetdash{}{0pt}%
\pgfpathmoveto{\pgfqpoint{0.717889in}{4.415414in}}%
\pgfpathlineto{\pgfqpoint{1.966405in}{4.415434in}}%
\pgfpathlineto{\pgfqpoint{2.020689in}{4.414335in}}%
\pgfpathlineto{\pgfqpoint{2.074972in}{4.409747in}}%
\pgfpathlineto{\pgfqpoint{2.129255in}{4.396470in}}%
\pgfpathlineto{\pgfqpoint{2.183539in}{4.366419in}}%
\pgfpathlineto{\pgfqpoint{2.237822in}{4.308998in}}%
\pgfpathlineto{\pgfqpoint{2.292105in}{4.211117in}}%
\pgfpathlineto{\pgfqpoint{2.346389in}{4.057389in}}%
\pgfpathlineto{\pgfqpoint{2.400672in}{3.844698in}}%
\pgfpathlineto{\pgfqpoint{2.454955in}{3.699926in}}%
\pgfpathlineto{\pgfqpoint{2.509239in}{3.844707in}}%
\pgfpathlineto{\pgfqpoint{2.563522in}{4.057375in}}%
\pgfpathlineto{\pgfqpoint{2.617805in}{4.211101in}}%
\pgfpathlineto{\pgfqpoint{2.672089in}{4.309001in}}%
\pgfpathlineto{\pgfqpoint{2.726372in}{4.366438in}}%
\pgfpathlineto{\pgfqpoint{2.780655in}{4.396480in}}%
\pgfpathlineto{\pgfqpoint{2.834939in}{4.409732in}}%
\pgfpathlineto{\pgfqpoint{2.889222in}{4.414310in}}%
\pgfpathlineto{\pgfqpoint{2.997789in}{4.415576in}}%
\pgfpathlineto{\pgfqpoint{3.703472in}{4.413543in}}%
\pgfpathlineto{\pgfqpoint{3.757755in}{4.407193in}}%
\pgfpathlineto{\pgfqpoint{3.812039in}{4.390177in}}%
\pgfpathlineto{\pgfqpoint{3.866322in}{4.353633in}}%
\pgfpathlineto{\pgfqpoint{3.920605in}{4.286297in}}%
\pgfpathlineto{\pgfqpoint{3.974889in}{4.174342in}}%
\pgfpathlineto{\pgfqpoint{4.029172in}{4.002695in}}%
\pgfpathlineto{\pgfqpoint{4.083455in}{3.784463in}}%
\pgfpathlineto{\pgfqpoint{4.137739in}{3.714520in}}%
\pgfpathlineto{\pgfqpoint{4.192022in}{3.907846in}}%
\pgfpathlineto{\pgfqpoint{4.246305in}{4.106891in}}%
\pgfpathlineto{\pgfqpoint{4.300589in}{4.243464in}}%
\pgfpathlineto{\pgfqpoint{4.354872in}{4.328545in}}%
\pgfpathlineto{\pgfqpoint{4.409155in}{4.377101in}}%
\pgfpathlineto{\pgfqpoint{4.463439in}{4.401482in}}%
\pgfpathlineto{\pgfqpoint{4.517722in}{4.411622in}}%
\pgfpathlineto{\pgfqpoint{4.572005in}{4.414838in}}%
\pgfpathlineto{\pgfqpoint{4.680572in}{4.415536in}}%
\pgfpathlineto{\pgfqpoint{5.331972in}{4.415387in}}%
\pgfpathlineto{\pgfqpoint{6.156222in}{4.415408in}}%
\pgfpathlineto{\pgfqpoint{6.156222in}{4.415408in}}%
\pgfusepath{stroke}%
\end{pgfscope}%
\begin{pgfscope}%
\pgfsetrectcap%
\pgfsetmiterjoin%
\pgfsetlinewidth{0.803000pt}%
\definecolor{currentstroke}{rgb}{0.000000,0.000000,0.000000}%
\pgfsetstrokecolor{currentstroke}%
\pgfsetdash{}{0pt}%
\pgfpathmoveto{\pgfqpoint{0.717889in}{3.664143in}}%
\pgfpathlineto{\pgfqpoint{0.717889in}{4.451359in}}%
\pgfusepath{stroke}%
\end{pgfscope}%
\begin{pgfscope}%
\pgfsetrectcap%
\pgfsetmiterjoin%
\pgfsetlinewidth{0.803000pt}%
\definecolor{currentstroke}{rgb}{0.000000,0.000000,0.000000}%
\pgfsetstrokecolor{currentstroke}%
\pgfsetdash{}{0pt}%
\pgfpathmoveto{\pgfqpoint{6.146222in}{3.664143in}}%
\pgfpathlineto{\pgfqpoint{6.146222in}{4.451359in}}%
\pgfusepath{stroke}%
\end{pgfscope}%
\begin{pgfscope}%
\pgfsetrectcap%
\pgfsetmiterjoin%
\pgfsetlinewidth{0.803000pt}%
\definecolor{currentstroke}{rgb}{0.000000,0.000000,0.000000}%
\pgfsetstrokecolor{currentstroke}%
\pgfsetdash{}{0pt}%
\pgfpathmoveto{\pgfqpoint{0.717889in}{3.664143in}}%
\pgfpathlineto{\pgfqpoint{6.146222in}{3.664143in}}%
\pgfusepath{stroke}%
\end{pgfscope}%
\begin{pgfscope}%
\pgfsetrectcap%
\pgfsetmiterjoin%
\pgfsetlinewidth{0.803000pt}%
\definecolor{currentstroke}{rgb}{0.000000,0.000000,0.000000}%
\pgfsetstrokecolor{currentstroke}%
\pgfsetdash{}{0pt}%
\pgfpathmoveto{\pgfqpoint{0.717889in}{4.451359in}}%
\pgfpathlineto{\pgfqpoint{6.146222in}{4.451359in}}%
\pgfusepath{stroke}%
\end{pgfscope}%
\begin{pgfscope}%
\definecolor{textcolor}{rgb}{0.000000,0.000000,0.000000}%
\pgfsetstrokecolor{textcolor}%
\pgfsetfillcolor{textcolor}%
\pgftext[x=3.432055in,y=4.534692in,,base]{\color{textcolor}\rmfamily\fontsize{12.000000}{14.400000}\selectfont Filter Frequency Response}%
\end{pgfscope}%
\begin{pgfscope}%
\pgfsetbuttcap%
\pgfsetmiterjoin%
\definecolor{currentfill}{rgb}{1.000000,1.000000,1.000000}%
\pgfsetfillcolor{currentfill}%
\pgfsetlinewidth{0.000000pt}%
\definecolor{currentstroke}{rgb}{0.000000,0.000000,0.000000}%
\pgfsetstrokecolor{currentstroke}%
\pgfsetstrokeopacity{0.000000}%
\pgfsetdash{}{0pt}%
\pgfpathmoveto{\pgfqpoint{0.717889in}{2.114143in}}%
\pgfpathlineto{\pgfqpoint{6.146222in}{2.114143in}}%
\pgfpathlineto{\pgfqpoint{6.146222in}{2.901359in}}%
\pgfpathlineto{\pgfqpoint{0.717889in}{2.901359in}}%
\pgfpathclose%
\pgfusepath{fill}%
\end{pgfscope}%
\begin{pgfscope}%
\pgfsetbuttcap%
\pgfsetroundjoin%
\definecolor{currentfill}{rgb}{0.000000,0.000000,0.000000}%
\pgfsetfillcolor{currentfill}%
\pgfsetlinewidth{0.803000pt}%
\definecolor{currentstroke}{rgb}{0.000000,0.000000,0.000000}%
\pgfsetstrokecolor{currentstroke}%
\pgfsetdash{}{0pt}%
\pgfsys@defobject{currentmarker}{\pgfqpoint{0.000000in}{-0.048611in}}{\pgfqpoint{0.000000in}{0.000000in}}{%
\pgfpathmoveto{\pgfqpoint{0.000000in}{0.000000in}}%
\pgfpathlineto{\pgfqpoint{0.000000in}{-0.048611in}}%
\pgfusepath{stroke,fill}%
}%
\begin{pgfscope}%
\pgfsys@transformshift{0.964631in}{2.114143in}%
\pgfsys@useobject{currentmarker}{}%
\end{pgfscope}%
\end{pgfscope}%
\begin{pgfscope}%
\definecolor{textcolor}{rgb}{0.000000,0.000000,0.000000}%
\pgfsetstrokecolor{textcolor}%
\pgfsetfillcolor{textcolor}%
\pgftext[x=0.964631in,y=2.016921in,,top]{\color{textcolor}\rmfamily\fontsize{10.000000}{12.000000}\selectfont \(\displaystyle {0}\)}%
\end{pgfscope}%
\begin{pgfscope}%
\pgfsetbuttcap%
\pgfsetroundjoin%
\definecolor{currentfill}{rgb}{0.000000,0.000000,0.000000}%
\pgfsetfillcolor{currentfill}%
\pgfsetlinewidth{0.803000pt}%
\definecolor{currentstroke}{rgb}{0.000000,0.000000,0.000000}%
\pgfsetstrokecolor{currentstroke}%
\pgfsetdash{}{0pt}%
\pgfsys@defobject{currentmarker}{\pgfqpoint{0.000000in}{-0.048611in}}{\pgfqpoint{0.000000in}{0.000000in}}{%
\pgfpathmoveto{\pgfqpoint{0.000000in}{0.000000in}}%
\pgfpathlineto{\pgfqpoint{0.000000in}{-0.048611in}}%
\pgfusepath{stroke,fill}%
}%
\begin{pgfscope}%
\pgfsys@transformshift{1.975308in}{2.114143in}%
\pgfsys@useobject{currentmarker}{}%
\end{pgfscope}%
\end{pgfscope}%
\begin{pgfscope}%
\definecolor{textcolor}{rgb}{0.000000,0.000000,0.000000}%
\pgfsetstrokecolor{textcolor}%
\pgfsetfillcolor{textcolor}%
\pgftext[x=1.975308in,y=2.016921in,,top]{\color{textcolor}\rmfamily\fontsize{10.000000}{12.000000}\selectfont \(\displaystyle {10}\)}%
\end{pgfscope}%
\begin{pgfscope}%
\pgfsetbuttcap%
\pgfsetroundjoin%
\definecolor{currentfill}{rgb}{0.000000,0.000000,0.000000}%
\pgfsetfillcolor{currentfill}%
\pgfsetlinewidth{0.803000pt}%
\definecolor{currentstroke}{rgb}{0.000000,0.000000,0.000000}%
\pgfsetstrokecolor{currentstroke}%
\pgfsetdash{}{0pt}%
\pgfsys@defobject{currentmarker}{\pgfqpoint{0.000000in}{-0.048611in}}{\pgfqpoint{0.000000in}{0.000000in}}{%
\pgfpathmoveto{\pgfqpoint{0.000000in}{0.000000in}}%
\pgfpathlineto{\pgfqpoint{0.000000in}{-0.048611in}}%
\pgfusepath{stroke,fill}%
}%
\begin{pgfscope}%
\pgfsys@transformshift{2.985986in}{2.114143in}%
\pgfsys@useobject{currentmarker}{}%
\end{pgfscope}%
\end{pgfscope}%
\begin{pgfscope}%
\definecolor{textcolor}{rgb}{0.000000,0.000000,0.000000}%
\pgfsetstrokecolor{textcolor}%
\pgfsetfillcolor{textcolor}%
\pgftext[x=2.985986in,y=2.016921in,,top]{\color{textcolor}\rmfamily\fontsize{10.000000}{12.000000}\selectfont \(\displaystyle {20}\)}%
\end{pgfscope}%
\begin{pgfscope}%
\pgfsetbuttcap%
\pgfsetroundjoin%
\definecolor{currentfill}{rgb}{0.000000,0.000000,0.000000}%
\pgfsetfillcolor{currentfill}%
\pgfsetlinewidth{0.803000pt}%
\definecolor{currentstroke}{rgb}{0.000000,0.000000,0.000000}%
\pgfsetstrokecolor{currentstroke}%
\pgfsetdash{}{0pt}%
\pgfsys@defobject{currentmarker}{\pgfqpoint{0.000000in}{-0.048611in}}{\pgfqpoint{0.000000in}{0.000000in}}{%
\pgfpathmoveto{\pgfqpoint{0.000000in}{0.000000in}}%
\pgfpathlineto{\pgfqpoint{0.000000in}{-0.048611in}}%
\pgfusepath{stroke,fill}%
}%
\begin{pgfscope}%
\pgfsys@transformshift{3.996663in}{2.114143in}%
\pgfsys@useobject{currentmarker}{}%
\end{pgfscope}%
\end{pgfscope}%
\begin{pgfscope}%
\definecolor{textcolor}{rgb}{0.000000,0.000000,0.000000}%
\pgfsetstrokecolor{textcolor}%
\pgfsetfillcolor{textcolor}%
\pgftext[x=3.996663in,y=2.016921in,,top]{\color{textcolor}\rmfamily\fontsize{10.000000}{12.000000}\selectfont \(\displaystyle {30}\)}%
\end{pgfscope}%
\begin{pgfscope}%
\pgfsetbuttcap%
\pgfsetroundjoin%
\definecolor{currentfill}{rgb}{0.000000,0.000000,0.000000}%
\pgfsetfillcolor{currentfill}%
\pgfsetlinewidth{0.803000pt}%
\definecolor{currentstroke}{rgb}{0.000000,0.000000,0.000000}%
\pgfsetstrokecolor{currentstroke}%
\pgfsetdash{}{0pt}%
\pgfsys@defobject{currentmarker}{\pgfqpoint{0.000000in}{-0.048611in}}{\pgfqpoint{0.000000in}{0.000000in}}{%
\pgfpathmoveto{\pgfqpoint{0.000000in}{0.000000in}}%
\pgfpathlineto{\pgfqpoint{0.000000in}{-0.048611in}}%
\pgfusepath{stroke,fill}%
}%
\begin{pgfscope}%
\pgfsys@transformshift{5.007340in}{2.114143in}%
\pgfsys@useobject{currentmarker}{}%
\end{pgfscope}%
\end{pgfscope}%
\begin{pgfscope}%
\definecolor{textcolor}{rgb}{0.000000,0.000000,0.000000}%
\pgfsetstrokecolor{textcolor}%
\pgfsetfillcolor{textcolor}%
\pgftext[x=5.007340in,y=2.016921in,,top]{\color{textcolor}\rmfamily\fontsize{10.000000}{12.000000}\selectfont \(\displaystyle {40}\)}%
\end{pgfscope}%
\begin{pgfscope}%
\pgfsetbuttcap%
\pgfsetroundjoin%
\definecolor{currentfill}{rgb}{0.000000,0.000000,0.000000}%
\pgfsetfillcolor{currentfill}%
\pgfsetlinewidth{0.803000pt}%
\definecolor{currentstroke}{rgb}{0.000000,0.000000,0.000000}%
\pgfsetstrokecolor{currentstroke}%
\pgfsetdash{}{0pt}%
\pgfsys@defobject{currentmarker}{\pgfqpoint{0.000000in}{-0.048611in}}{\pgfqpoint{0.000000in}{0.000000in}}{%
\pgfpathmoveto{\pgfqpoint{0.000000in}{0.000000in}}%
\pgfpathlineto{\pgfqpoint{0.000000in}{-0.048611in}}%
\pgfusepath{stroke,fill}%
}%
\begin{pgfscope}%
\pgfsys@transformshift{6.018017in}{2.114143in}%
\pgfsys@useobject{currentmarker}{}%
\end{pgfscope}%
\end{pgfscope}%
\begin{pgfscope}%
\definecolor{textcolor}{rgb}{0.000000,0.000000,0.000000}%
\pgfsetstrokecolor{textcolor}%
\pgfsetfillcolor{textcolor}%
\pgftext[x=6.018017in,y=2.016921in,,top]{\color{textcolor}\rmfamily\fontsize{10.000000}{12.000000}\selectfont \(\displaystyle {50}\)}%
\end{pgfscope}%
\begin{pgfscope}%
\definecolor{textcolor}{rgb}{0.000000,0.000000,0.000000}%
\pgfsetstrokecolor{textcolor}%
\pgfsetfillcolor{textcolor}%
\pgftext[x=3.432055in,y=1.837909in,,top]{\color{textcolor}\rmfamily\fontsize{10.000000}{12.000000}\selectfont Time (s)}%
\end{pgfscope}%
\begin{pgfscope}%
\pgfsetbuttcap%
\pgfsetroundjoin%
\definecolor{currentfill}{rgb}{0.000000,0.000000,0.000000}%
\pgfsetfillcolor{currentfill}%
\pgfsetlinewidth{0.803000pt}%
\definecolor{currentstroke}{rgb}{0.000000,0.000000,0.000000}%
\pgfsetstrokecolor{currentstroke}%
\pgfsetdash{}{0pt}%
\pgfsys@defobject{currentmarker}{\pgfqpoint{-0.048611in}{0.000000in}}{\pgfqpoint{0.000000in}{0.000000in}}{%
\pgfpathmoveto{\pgfqpoint{0.000000in}{0.000000in}}%
\pgfpathlineto{\pgfqpoint{-0.048611in}{0.000000in}}%
\pgfusepath{stroke,fill}%
}%
\begin{pgfscope}%
\pgfsys@transformshift{0.717889in}{2.450015in}%
\pgfsys@useobject{currentmarker}{}%
\end{pgfscope}%
\end{pgfscope}%
\begin{pgfscope}%
\definecolor{textcolor}{rgb}{0.000000,0.000000,0.000000}%
\pgfsetstrokecolor{textcolor}%
\pgfsetfillcolor{textcolor}%
\pgftext[x=0.551222in, y=2.401790in, left, base]{\color{textcolor}\rmfamily\fontsize{10.000000}{12.000000}\selectfont \(\displaystyle {0}\)}%
\end{pgfscope}%
\begin{pgfscope}%
\pgfsetbuttcap%
\pgfsetroundjoin%
\definecolor{currentfill}{rgb}{0.000000,0.000000,0.000000}%
\pgfsetfillcolor{currentfill}%
\pgfsetlinewidth{0.803000pt}%
\definecolor{currentstroke}{rgb}{0.000000,0.000000,0.000000}%
\pgfsetstrokecolor{currentstroke}%
\pgfsetdash{}{0pt}%
\pgfsys@defobject{currentmarker}{\pgfqpoint{-0.048611in}{0.000000in}}{\pgfqpoint{0.000000in}{0.000000in}}{%
\pgfpathmoveto{\pgfqpoint{0.000000in}{0.000000in}}%
\pgfpathlineto{\pgfqpoint{-0.048611in}{0.000000in}}%
\pgfusepath{stroke,fill}%
}%
\begin{pgfscope}%
\pgfsys@transformshift{0.717889in}{2.791960in}%
\pgfsys@useobject{currentmarker}{}%
\end{pgfscope}%
\end{pgfscope}%
\begin{pgfscope}%
\definecolor{textcolor}{rgb}{0.000000,0.000000,0.000000}%
\pgfsetstrokecolor{textcolor}%
\pgfsetfillcolor{textcolor}%
\pgftext[x=0.342888in, y=2.743734in, left, base]{\color{textcolor}\rmfamily\fontsize{10.000000}{12.000000}\selectfont \(\displaystyle {1000}\)}%
\end{pgfscope}%
\begin{pgfscope}%
\definecolor{textcolor}{rgb}{0.000000,0.000000,0.000000}%
\pgfsetstrokecolor{textcolor}%
\pgfsetfillcolor{textcolor}%
\pgftext[x=0.287332in,y=2.507751in,,bottom,rotate=90.000000]{\color{textcolor}\rmfamily\fontsize{10.000000}{12.000000}\selectfont ECG Voltage (\(\displaystyle \mu V\))}%
\end{pgfscope}%
\begin{pgfscope}%
\pgfpathrectangle{\pgfqpoint{0.717889in}{2.114143in}}{\pgfqpoint{5.428334in}{0.787215in}}%
\pgfusepath{clip}%
\pgfsetrectcap%
\pgfsetroundjoin%
\pgfsetlinewidth{1.505625pt}%
\definecolor{currentstroke}{rgb}{0.121569,0.466667,0.705882}%
\pgfsetstrokecolor{currentstroke}%
\pgfsetdash{}{0pt}%
\pgfpathmoveto{\pgfqpoint{0.964631in}{2.450015in}}%
\pgfpathlineto{\pgfqpoint{0.970652in}{2.451011in}}%
\pgfpathlineto{\pgfqpoint{0.970948in}{2.451502in}}%
\pgfpathlineto{\pgfqpoint{0.971343in}{2.449686in}}%
\pgfpathlineto{\pgfqpoint{0.971836in}{2.447503in}}%
\pgfpathlineto{\pgfqpoint{0.972231in}{2.450012in}}%
\pgfpathlineto{\pgfqpoint{0.972626in}{2.452177in}}%
\pgfpathlineto{\pgfqpoint{0.973119in}{2.448756in}}%
\pgfpathlineto{\pgfqpoint{0.973317in}{2.447770in}}%
\pgfpathlineto{\pgfqpoint{0.973711in}{2.451015in}}%
\pgfpathlineto{\pgfqpoint{0.974106in}{2.455552in}}%
\pgfpathlineto{\pgfqpoint{0.974600in}{2.447993in}}%
\pgfpathlineto{\pgfqpoint{0.974994in}{2.441356in}}%
\pgfpathlineto{\pgfqpoint{0.975488in}{2.451635in}}%
\pgfpathlineto{\pgfqpoint{0.975784in}{2.457608in}}%
\pgfpathlineto{\pgfqpoint{0.976278in}{2.448032in}}%
\pgfpathlineto{\pgfqpoint{0.976574in}{2.442588in}}%
\pgfpathlineto{\pgfqpoint{0.976968in}{2.452833in}}%
\pgfpathlineto{\pgfqpoint{0.977363in}{2.464367in}}%
\pgfpathlineto{\pgfqpoint{0.977758in}{2.449508in}}%
\pgfpathlineto{\pgfqpoint{0.978252in}{2.429209in}}%
\pgfpathlineto{\pgfqpoint{0.978745in}{2.457266in}}%
\pgfpathlineto{\pgfqpoint{0.979041in}{2.469329in}}%
\pgfpathlineto{\pgfqpoint{0.979535in}{2.443493in}}%
\pgfpathlineto{\pgfqpoint{0.979831in}{2.432277in}}%
\pgfpathlineto{\pgfqpoint{0.980225in}{2.456162in}}%
\pgfpathlineto{\pgfqpoint{0.980620in}{2.478185in}}%
\pgfpathlineto{\pgfqpoint{0.981015in}{2.446138in}}%
\pgfpathlineto{\pgfqpoint{0.981410in}{2.411122in}}%
\pgfpathlineto{\pgfqpoint{0.981903in}{2.455378in}}%
\pgfpathlineto{\pgfqpoint{0.982298in}{2.486762in}}%
\pgfpathlineto{\pgfqpoint{0.982792in}{2.434925in}}%
\pgfpathlineto{\pgfqpoint{0.983088in}{2.417212in}}%
\pgfpathlineto{\pgfqpoint{0.983483in}{2.460655in}}%
\pgfpathlineto{\pgfqpoint{0.984470in}{2.584535in}}%
\pgfpathlineto{\pgfqpoint{0.984272in}{2.440583in}}%
\pgfpathlineto{\pgfqpoint{0.984864in}{2.535216in}}%
\pgfpathlineto{\pgfqpoint{0.985358in}{2.458162in}}%
\pgfpathlineto{\pgfqpoint{0.985851in}{2.543693in}}%
\pgfpathlineto{\pgfqpoint{0.986049in}{2.568465in}}%
\pgfpathlineto{\pgfqpoint{0.986641in}{2.501757in}}%
\pgfpathlineto{\pgfqpoint{0.987036in}{2.461765in}}%
\pgfpathlineto{\pgfqpoint{0.987529in}{2.521629in}}%
\pgfpathlineto{\pgfqpoint{0.987825in}{2.535395in}}%
\pgfpathlineto{\pgfqpoint{0.988220in}{2.494769in}}%
\pgfpathlineto{\pgfqpoint{0.988615in}{2.457621in}}%
\pgfpathlineto{\pgfqpoint{0.989207in}{2.504256in}}%
\pgfpathlineto{\pgfqpoint{0.989404in}{2.516420in}}%
\pgfpathlineto{\pgfqpoint{0.989898in}{2.478104in}}%
\pgfpathlineto{\pgfqpoint{0.990293in}{2.460233in}}%
\pgfpathlineto{\pgfqpoint{0.990786in}{2.486564in}}%
\pgfpathlineto{\pgfqpoint{0.991082in}{2.500507in}}%
\pgfpathlineto{\pgfqpoint{0.991576in}{2.469342in}}%
\pgfpathlineto{\pgfqpoint{0.991971in}{2.446037in}}%
\pgfpathlineto{\pgfqpoint{0.992563in}{2.478969in}}%
\pgfpathlineto{\pgfqpoint{0.992958in}{2.459299in}}%
\pgfpathlineto{\pgfqpoint{0.993649in}{2.416280in}}%
\pgfpathlineto{\pgfqpoint{0.994339in}{2.437786in}}%
\pgfpathlineto{\pgfqpoint{0.994438in}{2.437865in}}%
\pgfpathlineto{\pgfqpoint{0.994932in}{2.407101in}}%
\pgfpathlineto{\pgfqpoint{0.995228in}{2.394456in}}%
\pgfpathlineto{\pgfqpoint{0.995721in}{2.421825in}}%
\pgfpathlineto{\pgfqpoint{0.995919in}{2.425544in}}%
\pgfpathlineto{\pgfqpoint{0.996313in}{2.411271in}}%
\pgfpathlineto{\pgfqpoint{0.996511in}{2.411928in}}%
\pgfpathlineto{\pgfqpoint{0.997004in}{2.399692in}}%
\pgfpathlineto{\pgfqpoint{0.997498in}{2.410266in}}%
\pgfpathlineto{\pgfqpoint{0.997794in}{2.421289in}}%
\pgfpathlineto{\pgfqpoint{0.998189in}{2.392482in}}%
\pgfpathlineto{\pgfqpoint{0.998386in}{2.386270in}}%
\pgfpathlineto{\pgfqpoint{0.998880in}{2.405589in}}%
\pgfpathlineto{\pgfqpoint{0.999373in}{2.420840in}}%
\pgfpathlineto{\pgfqpoint{0.999867in}{2.405663in}}%
\pgfpathlineto{\pgfqpoint{1.000163in}{2.398644in}}%
\pgfpathlineto{\pgfqpoint{1.000557in}{2.421625in}}%
\pgfpathlineto{\pgfqpoint{1.000854in}{2.442695in}}%
\pgfpathlineto{\pgfqpoint{1.001446in}{2.411893in}}%
\pgfpathlineto{\pgfqpoint{1.001643in}{2.408135in}}%
\pgfpathlineto{\pgfqpoint{1.002137in}{2.422471in}}%
\pgfpathlineto{\pgfqpoint{1.002729in}{2.436131in}}%
\pgfpathlineto{\pgfqpoint{1.002926in}{2.426924in}}%
\pgfpathlineto{\pgfqpoint{1.003321in}{2.407130in}}%
\pgfpathlineto{\pgfqpoint{1.003913in}{2.428676in}}%
\pgfpathlineto{\pgfqpoint{1.004209in}{2.452416in}}%
\pgfpathlineto{\pgfqpoint{1.004900in}{2.424694in}}%
\pgfpathlineto{\pgfqpoint{1.004999in}{2.424060in}}%
\pgfpathlineto{\pgfqpoint{1.005196in}{2.426166in}}%
\pgfpathlineto{\pgfqpoint{1.007368in}{2.483704in}}%
\pgfpathlineto{\pgfqpoint{1.007466in}{2.485677in}}%
\pgfpathlineto{\pgfqpoint{1.007763in}{2.474095in}}%
\pgfpathlineto{\pgfqpoint{1.008256in}{2.439200in}}%
\pgfpathlineto{\pgfqpoint{1.009046in}{2.447481in}}%
\pgfpathlineto{\pgfqpoint{1.009144in}{2.447602in}}%
\pgfpathlineto{\pgfqpoint{1.009638in}{2.417751in}}%
\pgfpathlineto{\pgfqpoint{1.010033in}{2.398064in}}%
\pgfpathlineto{\pgfqpoint{1.010526in}{2.430448in}}%
\pgfpathlineto{\pgfqpoint{1.010723in}{2.436254in}}%
\pgfpathlineto{\pgfqpoint{1.011118in}{2.410364in}}%
\pgfpathlineto{\pgfqpoint{1.011612in}{2.363755in}}%
\pgfpathlineto{\pgfqpoint{1.012303in}{2.399242in}}%
\pgfpathlineto{\pgfqpoint{1.013981in}{2.463486in}}%
\pgfpathlineto{\pgfqpoint{1.012994in}{2.387202in}}%
\pgfpathlineto{\pgfqpoint{1.014178in}{2.441243in}}%
\pgfpathlineto{\pgfqpoint{1.014968in}{2.371919in}}%
\pgfpathlineto{\pgfqpoint{1.015461in}{2.402862in}}%
\pgfpathlineto{\pgfqpoint{1.017632in}{2.573250in}}%
\pgfpathlineto{\pgfqpoint{1.018718in}{2.657223in}}%
\pgfpathlineto{\pgfqpoint{1.019014in}{2.644853in}}%
\pgfpathlineto{\pgfqpoint{1.019902in}{2.432360in}}%
\pgfpathlineto{\pgfqpoint{1.020199in}{2.375641in}}%
\pgfpathlineto{\pgfqpoint{1.020889in}{2.446408in}}%
\pgfpathlineto{\pgfqpoint{1.021383in}{2.413867in}}%
\pgfpathlineto{\pgfqpoint{1.021876in}{2.442236in}}%
\pgfpathlineto{\pgfqpoint{1.023554in}{2.484388in}}%
\pgfpathlineto{\pgfqpoint{1.023653in}{2.485352in}}%
\pgfpathlineto{\pgfqpoint{1.023850in}{2.477800in}}%
\pgfpathlineto{\pgfqpoint{1.024739in}{2.420922in}}%
\pgfpathlineto{\pgfqpoint{1.025331in}{2.452426in}}%
\pgfpathlineto{\pgfqpoint{1.027009in}{2.472907in}}%
\pgfpathlineto{\pgfqpoint{1.026022in}{2.452246in}}%
\pgfpathlineto{\pgfqpoint{1.027107in}{2.471607in}}%
\pgfpathlineto{\pgfqpoint{1.028094in}{2.419185in}}%
\pgfpathlineto{\pgfqpoint{1.028489in}{2.450988in}}%
\pgfpathlineto{\pgfqpoint{1.028687in}{2.456440in}}%
\pgfpathlineto{\pgfqpoint{1.029279in}{2.444005in}}%
\pgfpathlineto{\pgfqpoint{1.029476in}{2.435246in}}%
\pgfpathlineto{\pgfqpoint{1.029970in}{2.458758in}}%
\pgfpathlineto{\pgfqpoint{1.030266in}{2.466001in}}%
\pgfpathlineto{\pgfqpoint{1.030759in}{2.448903in}}%
\pgfpathlineto{\pgfqpoint{1.031352in}{2.442027in}}%
\pgfpathlineto{\pgfqpoint{1.031648in}{2.455241in}}%
\pgfpathlineto{\pgfqpoint{1.032042in}{2.478524in}}%
\pgfpathlineto{\pgfqpoint{1.032635in}{2.448592in}}%
\pgfpathlineto{\pgfqpoint{1.032733in}{2.447039in}}%
\pgfpathlineto{\pgfqpoint{1.033029in}{2.455176in}}%
\pgfpathlineto{\pgfqpoint{1.033819in}{2.481747in}}%
\pgfpathlineto{\pgfqpoint{1.034214in}{2.459221in}}%
\pgfpathlineto{\pgfqpoint{1.034609in}{2.444805in}}%
\pgfpathlineto{\pgfqpoint{1.035102in}{2.470264in}}%
\pgfpathlineto{\pgfqpoint{1.035300in}{2.483448in}}%
\pgfpathlineto{\pgfqpoint{1.035990in}{2.454487in}}%
\pgfpathlineto{\pgfqpoint{1.036188in}{2.452998in}}%
\pgfpathlineto{\pgfqpoint{1.036583in}{2.459919in}}%
\pgfpathlineto{\pgfqpoint{1.036780in}{2.461430in}}%
\pgfpathlineto{\pgfqpoint{1.037273in}{2.455158in}}%
\pgfpathlineto{\pgfqpoint{1.039247in}{2.360459in}}%
\pgfpathlineto{\pgfqpoint{1.039445in}{2.364480in}}%
\pgfpathlineto{\pgfqpoint{1.041912in}{2.466940in}}%
\pgfpathlineto{\pgfqpoint{1.042110in}{2.458490in}}%
\pgfpathlineto{\pgfqpoint{1.042899in}{2.410273in}}%
\pgfpathlineto{\pgfqpoint{1.043590in}{2.425347in}}%
\pgfpathlineto{\pgfqpoint{1.044380in}{2.384852in}}%
\pgfpathlineto{\pgfqpoint{1.044873in}{2.407231in}}%
\pgfpathlineto{\pgfqpoint{1.045071in}{2.409933in}}%
\pgfpathlineto{\pgfqpoint{1.045465in}{2.397652in}}%
\pgfpathlineto{\pgfqpoint{1.046156in}{2.375428in}}%
\pgfpathlineto{\pgfqpoint{1.046452in}{2.393113in}}%
\pgfpathlineto{\pgfqpoint{1.046749in}{2.406552in}}%
\pgfpathlineto{\pgfqpoint{1.047341in}{2.377924in}}%
\pgfpathlineto{\pgfqpoint{1.047637in}{2.363144in}}%
\pgfpathlineto{\pgfqpoint{1.048328in}{2.386384in}}%
\pgfpathlineto{\pgfqpoint{1.048624in}{2.394078in}}%
\pgfpathlineto{\pgfqpoint{1.049019in}{2.376500in}}%
\pgfpathlineto{\pgfqpoint{1.049216in}{2.367521in}}%
\pgfpathlineto{\pgfqpoint{1.049710in}{2.394290in}}%
\pgfpathlineto{\pgfqpoint{1.049907in}{2.398724in}}%
\pgfpathlineto{\pgfqpoint{1.050400in}{2.385706in}}%
\pgfpathlineto{\pgfqpoint{1.050993in}{2.369102in}}%
\pgfpathlineto{\pgfqpoint{1.051387in}{2.380591in}}%
\pgfpathlineto{\pgfqpoint{1.051782in}{2.398119in}}%
\pgfpathlineto{\pgfqpoint{1.052276in}{2.366543in}}%
\pgfpathlineto{\pgfqpoint{1.052473in}{2.362096in}}%
\pgfpathlineto{\pgfqpoint{1.052967in}{2.382045in}}%
\pgfpathlineto{\pgfqpoint{1.053361in}{2.401615in}}%
\pgfpathlineto{\pgfqpoint{1.053954in}{2.378519in}}%
\pgfpathlineto{\pgfqpoint{1.054151in}{2.373977in}}%
\pgfpathlineto{\pgfqpoint{1.054546in}{2.388601in}}%
\pgfpathlineto{\pgfqpoint{1.056322in}{2.454822in}}%
\pgfpathlineto{\pgfqpoint{1.056520in}{2.462394in}}%
\pgfpathlineto{\pgfqpoint{1.057013in}{2.438423in}}%
\pgfpathlineto{\pgfqpoint{1.057309in}{2.421892in}}%
\pgfpathlineto{\pgfqpoint{1.058099in}{2.435950in}}%
\pgfpathlineto{\pgfqpoint{1.058198in}{2.436241in}}%
\pgfpathlineto{\pgfqpoint{1.058395in}{2.434464in}}%
\pgfpathlineto{\pgfqpoint{1.058987in}{2.398654in}}%
\pgfpathlineto{\pgfqpoint{1.059678in}{2.427033in}}%
\pgfpathlineto{\pgfqpoint{1.059974in}{2.429706in}}%
\pgfpathlineto{\pgfqpoint{1.060073in}{2.427828in}}%
\pgfpathlineto{\pgfqpoint{1.060961in}{2.374767in}}%
\pgfpathlineto{\pgfqpoint{1.061455in}{2.408084in}}%
\pgfpathlineto{\pgfqpoint{1.061553in}{2.408596in}}%
\pgfpathlineto{\pgfqpoint{1.061652in}{2.405516in}}%
\pgfpathlineto{\pgfqpoint{1.062244in}{2.391368in}}%
\pgfpathlineto{\pgfqpoint{1.062540in}{2.406784in}}%
\pgfpathlineto{\pgfqpoint{1.062935in}{2.450005in}}%
\pgfpathlineto{\pgfqpoint{1.063527in}{2.407960in}}%
\pgfpathlineto{\pgfqpoint{1.064120in}{2.356423in}}%
\pgfpathlineto{\pgfqpoint{1.064712in}{2.397673in}}%
\pgfpathlineto{\pgfqpoint{1.065600in}{2.369146in}}%
\pgfpathlineto{\pgfqpoint{1.065896in}{2.393065in}}%
\pgfpathlineto{\pgfqpoint{1.067673in}{2.631104in}}%
\pgfpathlineto{\pgfqpoint{1.068265in}{2.611853in}}%
\pgfpathlineto{\pgfqpoint{1.068956in}{2.451821in}}%
\pgfpathlineto{\pgfqpoint{1.069351in}{2.357326in}}%
\pgfpathlineto{\pgfqpoint{1.070140in}{2.404657in}}%
\pgfpathlineto{\pgfqpoint{1.070535in}{2.386577in}}%
\pgfpathlineto{\pgfqpoint{1.071029in}{2.414312in}}%
\pgfpathlineto{\pgfqpoint{1.071325in}{2.427347in}}%
\pgfpathlineto{\pgfqpoint{1.072114in}{2.417854in}}%
\pgfpathlineto{\pgfqpoint{1.072213in}{2.415228in}}%
\pgfpathlineto{\pgfqpoint{1.072509in}{2.431571in}}%
\pgfpathlineto{\pgfqpoint{1.072904in}{2.456426in}}%
\pgfpathlineto{\pgfqpoint{1.073299in}{2.414969in}}%
\pgfpathlineto{\pgfqpoint{1.073595in}{2.389176in}}%
\pgfpathlineto{\pgfqpoint{1.074384in}{2.416461in}}%
\pgfpathlineto{\pgfqpoint{1.074680in}{2.430841in}}%
\pgfpathlineto{\pgfqpoint{1.075174in}{2.411367in}}%
\pgfpathlineto{\pgfqpoint{1.075569in}{2.423819in}}%
\pgfpathlineto{\pgfqpoint{1.076161in}{2.437371in}}%
\pgfpathlineto{\pgfqpoint{1.076457in}{2.423368in}}%
\pgfpathlineto{\pgfqpoint{1.077247in}{2.400328in}}%
\pgfpathlineto{\pgfqpoint{1.077543in}{2.420267in}}%
\pgfpathlineto{\pgfqpoint{1.077937in}{2.440384in}}%
\pgfpathlineto{\pgfqpoint{1.078628in}{2.418023in}}%
\pgfpathlineto{\pgfqpoint{1.078727in}{2.416210in}}%
\pgfpathlineto{\pgfqpoint{1.078924in}{2.425471in}}%
\pgfpathlineto{\pgfqpoint{1.079319in}{2.448490in}}%
\pgfpathlineto{\pgfqpoint{1.080010in}{2.423708in}}%
\pgfpathlineto{\pgfqpoint{1.080306in}{2.411053in}}%
\pgfpathlineto{\pgfqpoint{1.080800in}{2.441897in}}%
\pgfpathlineto{\pgfqpoint{1.081096in}{2.449120in}}%
\pgfpathlineto{\pgfqpoint{1.081688in}{2.442682in}}%
\pgfpathlineto{\pgfqpoint{1.082083in}{2.416932in}}%
\pgfpathlineto{\pgfqpoint{1.082478in}{2.457502in}}%
\pgfpathlineto{\pgfqpoint{1.082576in}{2.461597in}}%
\pgfpathlineto{\pgfqpoint{1.083267in}{2.449764in}}%
\pgfpathlineto{\pgfqpoint{1.083859in}{2.425646in}}%
\pgfpathlineto{\pgfqpoint{1.084155in}{2.448499in}}%
\pgfpathlineto{\pgfqpoint{1.084452in}{2.471935in}}%
\pgfpathlineto{\pgfqpoint{1.085241in}{2.450908in}}%
\pgfpathlineto{\pgfqpoint{1.085340in}{2.449403in}}%
\pgfpathlineto{\pgfqpoint{1.085636in}{2.460951in}}%
\pgfpathlineto{\pgfqpoint{1.086228in}{2.495849in}}%
\pgfpathlineto{\pgfqpoint{1.086919in}{2.477108in}}%
\pgfpathlineto{\pgfqpoint{1.089584in}{2.548350in}}%
\pgfpathlineto{\pgfqpoint{1.089781in}{2.542979in}}%
\pgfpathlineto{\pgfqpoint{1.090077in}{2.529928in}}%
\pgfpathlineto{\pgfqpoint{1.090768in}{2.540114in}}%
\pgfpathlineto{\pgfqpoint{1.091163in}{2.562723in}}%
\pgfpathlineto{\pgfqpoint{1.091558in}{2.521288in}}%
\pgfpathlineto{\pgfqpoint{1.091657in}{2.518216in}}%
\pgfpathlineto{\pgfqpoint{1.092347in}{2.529710in}}%
\pgfpathlineto{\pgfqpoint{1.092644in}{2.541918in}}%
\pgfpathlineto{\pgfqpoint{1.093137in}{2.511778in}}%
\pgfpathlineto{\pgfqpoint{1.095210in}{2.447210in}}%
\pgfpathlineto{\pgfqpoint{1.095308in}{2.449754in}}%
\pgfpathlineto{\pgfqpoint{1.095703in}{2.472557in}}%
\pgfpathlineto{\pgfqpoint{1.096394in}{2.449931in}}%
\pgfpathlineto{\pgfqpoint{1.096789in}{2.424931in}}%
\pgfpathlineto{\pgfqpoint{1.097381in}{2.453112in}}%
\pgfpathlineto{\pgfqpoint{1.097677in}{2.461943in}}%
\pgfpathlineto{\pgfqpoint{1.097973in}{2.448090in}}%
\pgfpathlineto{\pgfqpoint{1.098269in}{2.429888in}}%
\pgfpathlineto{\pgfqpoint{1.098862in}{2.454853in}}%
\pgfpathlineto{\pgfqpoint{1.099256in}{2.475092in}}%
\pgfpathlineto{\pgfqpoint{1.099750in}{2.452217in}}%
\pgfpathlineto{\pgfqpoint{1.100046in}{2.437729in}}%
\pgfpathlineto{\pgfqpoint{1.100540in}{2.467198in}}%
\pgfpathlineto{\pgfqpoint{1.100737in}{2.471643in}}%
\pgfpathlineto{\pgfqpoint{1.101329in}{2.463061in}}%
\pgfpathlineto{\pgfqpoint{1.101724in}{2.443105in}}%
\pgfpathlineto{\pgfqpoint{1.102316in}{2.459968in}}%
\pgfpathlineto{\pgfqpoint{1.102612in}{2.472015in}}%
\pgfpathlineto{\pgfqpoint{1.103106in}{2.444595in}}%
\pgfpathlineto{\pgfqpoint{1.103303in}{2.442421in}}%
\pgfpathlineto{\pgfqpoint{1.103599in}{2.448107in}}%
\pgfpathlineto{\pgfqpoint{1.104191in}{2.474352in}}%
\pgfpathlineto{\pgfqpoint{1.104784in}{2.455437in}}%
\pgfpathlineto{\pgfqpoint{1.104882in}{2.453476in}}%
\pgfpathlineto{\pgfqpoint{1.105178in}{2.469042in}}%
\pgfpathlineto{\pgfqpoint{1.105573in}{2.482703in}}%
\pgfpathlineto{\pgfqpoint{1.106067in}{2.465598in}}%
\pgfpathlineto{\pgfqpoint{1.106758in}{2.421693in}}%
\pgfpathlineto{\pgfqpoint{1.107547in}{2.433949in}}%
\pgfpathlineto{\pgfqpoint{1.108041in}{2.417171in}}%
\pgfpathlineto{\pgfqpoint{1.108435in}{2.438026in}}%
\pgfpathlineto{\pgfqpoint{1.108732in}{2.443615in}}%
\pgfpathlineto{\pgfqpoint{1.109126in}{2.431059in}}%
\pgfpathlineto{\pgfqpoint{1.109916in}{2.389059in}}%
\pgfpathlineto{\pgfqpoint{1.110409in}{2.412033in}}%
\pgfpathlineto{\pgfqpoint{1.111791in}{2.431035in}}%
\pgfpathlineto{\pgfqpoint{1.111890in}{2.428913in}}%
\pgfpathlineto{\pgfqpoint{1.114653in}{2.326892in}}%
\pgfpathlineto{\pgfqpoint{1.114851in}{2.345645in}}%
\pgfpathlineto{\pgfqpoint{1.117121in}{2.593709in}}%
\pgfpathlineto{\pgfqpoint{1.117220in}{2.588433in}}%
\pgfpathlineto{\pgfqpoint{1.118108in}{2.348017in}}%
\pgfpathlineto{\pgfqpoint{1.119786in}{2.398806in}}%
\pgfpathlineto{\pgfqpoint{1.119885in}{2.397109in}}%
\pgfpathlineto{\pgfqpoint{1.120082in}{2.407437in}}%
\pgfpathlineto{\pgfqpoint{1.121069in}{2.463662in}}%
\pgfpathlineto{\pgfqpoint{1.121464in}{2.447722in}}%
\pgfpathlineto{\pgfqpoint{1.121562in}{2.447777in}}%
\pgfpathlineto{\pgfqpoint{1.121858in}{2.460117in}}%
\pgfpathlineto{\pgfqpoint{1.122352in}{2.437531in}}%
\pgfpathlineto{\pgfqpoint{1.122944in}{2.422123in}}%
\pgfpathlineto{\pgfqpoint{1.123240in}{2.438055in}}%
\pgfpathlineto{\pgfqpoint{1.125116in}{2.500041in}}%
\pgfpathlineto{\pgfqpoint{1.126300in}{2.472104in}}%
\pgfpathlineto{\pgfqpoint{1.126596in}{2.486499in}}%
\pgfpathlineto{\pgfqpoint{1.127188in}{2.521677in}}%
\pgfpathlineto{\pgfqpoint{1.127780in}{2.500281in}}%
\pgfpathlineto{\pgfqpoint{1.127978in}{2.491723in}}%
\pgfpathlineto{\pgfqpoint{1.128373in}{2.520765in}}%
\pgfpathlineto{\pgfqpoint{1.128570in}{2.532473in}}%
\pgfpathlineto{\pgfqpoint{1.129458in}{2.519723in}}%
\pgfpathlineto{\pgfqpoint{1.129656in}{2.516980in}}%
\pgfpathlineto{\pgfqpoint{1.129853in}{2.525982in}}%
\pgfpathlineto{\pgfqpoint{1.130445in}{2.561566in}}%
\pgfpathlineto{\pgfqpoint{1.131037in}{2.543876in}}%
\pgfpathlineto{\pgfqpoint{1.131235in}{2.538101in}}%
\pgfpathlineto{\pgfqpoint{1.131827in}{2.554688in}}%
\pgfpathlineto{\pgfqpoint{1.132024in}{2.559712in}}%
\pgfpathlineto{\pgfqpoint{1.132321in}{2.545236in}}%
\pgfpathlineto{\pgfqpoint{1.132617in}{2.529579in}}%
\pgfpathlineto{\pgfqpoint{1.133308in}{2.551755in}}%
\pgfpathlineto{\pgfqpoint{1.133505in}{2.556948in}}%
\pgfpathlineto{\pgfqpoint{1.133998in}{2.539264in}}%
\pgfpathlineto{\pgfqpoint{1.134492in}{2.509461in}}%
\pgfpathlineto{\pgfqpoint{1.134887in}{2.542034in}}%
\pgfpathlineto{\pgfqpoint{1.134985in}{2.544304in}}%
\pgfpathlineto{\pgfqpoint{1.135380in}{2.528215in}}%
\pgfpathlineto{\pgfqpoint{1.135578in}{2.532575in}}%
\pgfpathlineto{\pgfqpoint{1.135874in}{2.513789in}}%
\pgfpathlineto{\pgfqpoint{1.136071in}{2.498168in}}%
\pgfpathlineto{\pgfqpoint{1.136762in}{2.529337in}}%
\pgfpathlineto{\pgfqpoint{1.136861in}{2.529537in}}%
\pgfpathlineto{\pgfqpoint{1.136959in}{2.527436in}}%
\pgfpathlineto{\pgfqpoint{1.137848in}{2.487847in}}%
\pgfpathlineto{\pgfqpoint{1.138243in}{2.511957in}}%
\pgfpathlineto{\pgfqpoint{1.138440in}{2.522878in}}%
\pgfpathlineto{\pgfqpoint{1.139032in}{2.486347in}}%
\pgfpathlineto{\pgfqpoint{1.140611in}{2.440644in}}%
\pgfpathlineto{\pgfqpoint{1.141105in}{2.407055in}}%
\pgfpathlineto{\pgfqpoint{1.141993in}{2.412889in}}%
\pgfpathlineto{\pgfqpoint{1.142881in}{2.375353in}}%
\pgfpathlineto{\pgfqpoint{1.143375in}{2.403654in}}%
\pgfpathlineto{\pgfqpoint{1.146040in}{2.341694in}}%
\pgfpathlineto{\pgfqpoint{1.146336in}{2.353993in}}%
\pgfpathlineto{\pgfqpoint{1.146731in}{2.366197in}}%
\pgfpathlineto{\pgfqpoint{1.147125in}{2.347098in}}%
\pgfpathlineto{\pgfqpoint{1.147520in}{2.330705in}}%
\pgfpathlineto{\pgfqpoint{1.148112in}{2.352103in}}%
\pgfpathlineto{\pgfqpoint{1.148409in}{2.363601in}}%
\pgfpathlineto{\pgfqpoint{1.148803in}{2.334020in}}%
\pgfpathlineto{\pgfqpoint{1.149099in}{2.319538in}}%
\pgfpathlineto{\pgfqpoint{1.149692in}{2.346229in}}%
\pgfpathlineto{\pgfqpoint{1.149988in}{2.357941in}}%
\pgfpathlineto{\pgfqpoint{1.150481in}{2.332738in}}%
\pgfpathlineto{\pgfqpoint{1.150580in}{2.329552in}}%
\pgfpathlineto{\pgfqpoint{1.150975in}{2.350163in}}%
\pgfpathlineto{\pgfqpoint{1.151863in}{2.387299in}}%
\pgfpathlineto{\pgfqpoint{1.152060in}{2.366061in}}%
\pgfpathlineto{\pgfqpoint{1.152258in}{2.346969in}}%
\pgfpathlineto{\pgfqpoint{1.152949in}{2.380131in}}%
\pgfpathlineto{\pgfqpoint{1.153343in}{2.405415in}}%
\pgfpathlineto{\pgfqpoint{1.153936in}{2.373644in}}%
\pgfpathlineto{\pgfqpoint{1.154232in}{2.363976in}}%
\pgfpathlineto{\pgfqpoint{1.154528in}{2.381837in}}%
\pgfpathlineto{\pgfqpoint{1.154824in}{2.401442in}}%
\pgfpathlineto{\pgfqpoint{1.155515in}{2.373352in}}%
\pgfpathlineto{\pgfqpoint{1.155811in}{2.360822in}}%
\pgfpathlineto{\pgfqpoint{1.156206in}{2.394095in}}%
\pgfpathlineto{\pgfqpoint{1.156699in}{2.404134in}}%
\pgfpathlineto{\pgfqpoint{1.157094in}{2.390639in}}%
\pgfpathlineto{\pgfqpoint{1.157390in}{2.375915in}}%
\pgfpathlineto{\pgfqpoint{1.158081in}{2.394802in}}%
\pgfpathlineto{\pgfqpoint{1.158278in}{2.396682in}}%
\pgfpathlineto{\pgfqpoint{1.158673in}{2.387469in}}%
\pgfpathlineto{\pgfqpoint{1.158969in}{2.378782in}}%
\pgfpathlineto{\pgfqpoint{1.159265in}{2.395002in}}%
\pgfpathlineto{\pgfqpoint{1.159956in}{2.463322in}}%
\pgfpathlineto{\pgfqpoint{1.160548in}{2.417356in}}%
\pgfpathlineto{\pgfqpoint{1.160943in}{2.404532in}}%
\pgfpathlineto{\pgfqpoint{1.161535in}{2.418832in}}%
\pgfpathlineto{\pgfqpoint{1.161634in}{2.417830in}}%
\pgfpathlineto{\pgfqpoint{1.162226in}{2.391795in}}%
\pgfpathlineto{\pgfqpoint{1.162720in}{2.416433in}}%
\pgfpathlineto{\pgfqpoint{1.163904in}{2.565262in}}%
\pgfpathlineto{\pgfqpoint{1.164694in}{2.692502in}}%
\pgfpathlineto{\pgfqpoint{1.165187in}{2.644457in}}%
\pgfpathlineto{\pgfqpoint{1.166569in}{2.396010in}}%
\pgfpathlineto{\pgfqpoint{1.167556in}{2.433507in}}%
\pgfpathlineto{\pgfqpoint{1.168050in}{2.463976in}}%
\pgfpathlineto{\pgfqpoint{1.168642in}{2.434692in}}%
\pgfpathlineto{\pgfqpoint{1.168740in}{2.432823in}}%
\pgfpathlineto{\pgfqpoint{1.168938in}{2.441335in}}%
\pgfpathlineto{\pgfqpoint{1.169629in}{2.492876in}}%
\pgfpathlineto{\pgfqpoint{1.170122in}{2.459023in}}%
\pgfpathlineto{\pgfqpoint{1.170616in}{2.418497in}}%
\pgfpathlineto{\pgfqpoint{1.171405in}{2.434378in}}%
\pgfpathlineto{\pgfqpoint{1.172096in}{2.416934in}}%
\pgfpathlineto{\pgfqpoint{1.172392in}{2.430367in}}%
\pgfpathlineto{\pgfqpoint{1.172787in}{2.465109in}}%
\pgfpathlineto{\pgfqpoint{1.173379in}{2.423333in}}%
\pgfpathlineto{\pgfqpoint{1.173774in}{2.398374in}}%
\pgfpathlineto{\pgfqpoint{1.174564in}{2.407417in}}%
\pgfpathlineto{\pgfqpoint{1.174761in}{2.409794in}}%
\pgfpathlineto{\pgfqpoint{1.174959in}{2.403830in}}%
\pgfpathlineto{\pgfqpoint{1.175353in}{2.379575in}}%
\pgfpathlineto{\pgfqpoint{1.175847in}{2.417184in}}%
\pgfpathlineto{\pgfqpoint{1.176340in}{2.431337in}}%
\pgfpathlineto{\pgfqpoint{1.176636in}{2.409141in}}%
\pgfpathlineto{\pgfqpoint{1.177031in}{2.380332in}}%
\pgfpathlineto{\pgfqpoint{1.177623in}{2.404931in}}%
\pgfpathlineto{\pgfqpoint{1.177821in}{2.415822in}}%
\pgfpathlineto{\pgfqpoint{1.178413in}{2.387055in}}%
\pgfpathlineto{\pgfqpoint{1.178808in}{2.378239in}}%
\pgfpathlineto{\pgfqpoint{1.179005in}{2.387533in}}%
\pgfpathlineto{\pgfqpoint{1.179400in}{2.417217in}}%
\pgfpathlineto{\pgfqpoint{1.180190in}{2.401457in}}%
\pgfpathlineto{\pgfqpoint{1.180387in}{2.398462in}}%
\pgfpathlineto{\pgfqpoint{1.180683in}{2.406969in}}%
\pgfpathlineto{\pgfqpoint{1.180979in}{2.422597in}}%
\pgfpathlineto{\pgfqpoint{1.181571in}{2.393935in}}%
\pgfpathlineto{\pgfqpoint{1.182262in}{2.407731in}}%
\pgfpathlineto{\pgfqpoint{1.182657in}{2.441568in}}%
\pgfpathlineto{\pgfqpoint{1.183348in}{2.410474in}}%
\pgfpathlineto{\pgfqpoint{1.183743in}{2.414206in}}%
\pgfpathlineto{\pgfqpoint{1.184236in}{2.438281in}}%
\pgfpathlineto{\pgfqpoint{1.184927in}{2.424345in}}%
\pgfpathlineto{\pgfqpoint{1.185125in}{2.419036in}}%
\pgfpathlineto{\pgfqpoint{1.185519in}{2.433263in}}%
\pgfpathlineto{\pgfqpoint{1.186111in}{2.463929in}}%
\pgfpathlineto{\pgfqpoint{1.186605in}{2.442725in}}%
\pgfpathlineto{\pgfqpoint{1.187000in}{2.424921in}}%
\pgfpathlineto{\pgfqpoint{1.187395in}{2.452402in}}%
\pgfpathlineto{\pgfqpoint{1.187592in}{2.458484in}}%
\pgfpathlineto{\pgfqpoint{1.187987in}{2.430595in}}%
\pgfpathlineto{\pgfqpoint{1.188678in}{2.398594in}}%
\pgfpathlineto{\pgfqpoint{1.189467in}{2.400189in}}%
\pgfpathlineto{\pgfqpoint{1.189961in}{2.380740in}}%
\pgfpathlineto{\pgfqpoint{1.190553in}{2.401093in}}%
\pgfpathlineto{\pgfqpoint{1.192527in}{2.520555in}}%
\pgfpathlineto{\pgfqpoint{1.192626in}{2.520585in}}%
\pgfpathlineto{\pgfqpoint{1.193317in}{2.483009in}}%
\pgfpathlineto{\pgfqpoint{1.194106in}{2.510088in}}%
\pgfpathlineto{\pgfqpoint{1.194797in}{2.508898in}}%
\pgfpathlineto{\pgfqpoint{1.195784in}{2.552245in}}%
\pgfpathlineto{\pgfqpoint{1.195883in}{2.552651in}}%
\pgfpathlineto{\pgfqpoint{1.195981in}{2.548522in}}%
\pgfpathlineto{\pgfqpoint{1.196771in}{2.523466in}}%
\pgfpathlineto{\pgfqpoint{1.197067in}{2.538793in}}%
\pgfpathlineto{\pgfqpoint{1.197363in}{2.550630in}}%
\pgfpathlineto{\pgfqpoint{1.197955in}{2.530341in}}%
\pgfpathlineto{\pgfqpoint{1.198251in}{2.521957in}}%
\pgfpathlineto{\pgfqpoint{1.198646in}{2.544224in}}%
\pgfpathlineto{\pgfqpoint{1.199238in}{2.565941in}}%
\pgfpathlineto{\pgfqpoint{1.199929in}{2.553884in}}%
\pgfpathlineto{\pgfqpoint{1.200916in}{2.589314in}}%
\pgfpathlineto{\pgfqpoint{1.201311in}{2.569721in}}%
\pgfpathlineto{\pgfqpoint{1.201607in}{2.558136in}}%
\pgfpathlineto{\pgfqpoint{1.202101in}{2.588855in}}%
\pgfpathlineto{\pgfqpoint{1.202594in}{2.594492in}}%
\pgfpathlineto{\pgfqpoint{1.202693in}{2.591178in}}%
\pgfpathlineto{\pgfqpoint{1.203186in}{2.548564in}}%
\pgfpathlineto{\pgfqpoint{1.203976in}{2.572609in}}%
\pgfpathlineto{\pgfqpoint{1.204075in}{2.572301in}}%
\pgfpathlineto{\pgfqpoint{1.204568in}{2.527317in}}%
\pgfpathlineto{\pgfqpoint{1.205259in}{2.557185in}}%
\pgfpathlineto{\pgfqpoint{1.205456in}{2.568561in}}%
\pgfpathlineto{\pgfqpoint{1.206049in}{2.537335in}}%
\pgfpathlineto{\pgfqpoint{1.206542in}{2.500599in}}%
\pgfpathlineto{\pgfqpoint{1.207134in}{2.535476in}}%
\pgfpathlineto{\pgfqpoint{1.207332in}{2.540362in}}%
\pgfpathlineto{\pgfqpoint{1.207727in}{2.522529in}}%
\pgfpathlineto{\pgfqpoint{1.207924in}{2.515373in}}%
\pgfpathlineto{\pgfqpoint{1.208516in}{2.535544in}}%
\pgfpathlineto{\pgfqpoint{1.208911in}{2.545861in}}%
\pgfpathlineto{\pgfqpoint{1.209108in}{2.534865in}}%
\pgfpathlineto{\pgfqpoint{1.209701in}{2.460078in}}%
\pgfpathlineto{\pgfqpoint{1.210490in}{2.482000in}}%
\pgfpathlineto{\pgfqpoint{1.210688in}{2.487298in}}%
\pgfpathlineto{\pgfqpoint{1.210984in}{2.461431in}}%
\pgfpathlineto{\pgfqpoint{1.211280in}{2.444477in}}%
\pgfpathlineto{\pgfqpoint{1.211773in}{2.477247in}}%
\pgfpathlineto{\pgfqpoint{1.213451in}{2.670814in}}%
\pgfpathlineto{\pgfqpoint{1.213846in}{2.648571in}}%
\pgfpathlineto{\pgfqpoint{1.214635in}{2.500118in}}%
\pgfpathlineto{\pgfqpoint{1.215129in}{2.362541in}}%
\pgfpathlineto{\pgfqpoint{1.215919in}{2.403976in}}%
\pgfpathlineto{\pgfqpoint{1.216609in}{2.385233in}}%
\pgfpathlineto{\pgfqpoint{1.216906in}{2.406291in}}%
\pgfpathlineto{\pgfqpoint{1.217399in}{2.404533in}}%
\pgfpathlineto{\pgfqpoint{1.218583in}{2.440019in}}%
\pgfpathlineto{\pgfqpoint{1.218682in}{2.440859in}}%
\pgfpathlineto{\pgfqpoint{1.218781in}{2.437690in}}%
\pgfpathlineto{\pgfqpoint{1.219373in}{2.380106in}}%
\pgfpathlineto{\pgfqpoint{1.220064in}{2.412151in}}%
\pgfpathlineto{\pgfqpoint{1.220459in}{2.425252in}}%
\pgfpathlineto{\pgfqpoint{1.221051in}{2.405583in}}%
\pgfpathlineto{\pgfqpoint{1.221150in}{2.405813in}}%
\pgfpathlineto{\pgfqpoint{1.221643in}{2.437795in}}%
\pgfpathlineto{\pgfqpoint{1.222334in}{2.421084in}}%
\pgfpathlineto{\pgfqpoint{1.222729in}{2.400215in}}%
\pgfpathlineto{\pgfqpoint{1.223321in}{2.424561in}}%
\pgfpathlineto{\pgfqpoint{1.223518in}{2.421309in}}%
\pgfpathlineto{\pgfqpoint{1.223814in}{2.430986in}}%
\pgfpathlineto{\pgfqpoint{1.225196in}{2.456692in}}%
\pgfpathlineto{\pgfqpoint{1.224505in}{2.412365in}}%
\pgfpathlineto{\pgfqpoint{1.225295in}{2.453961in}}%
\pgfpathlineto{\pgfqpoint{1.225986in}{2.422371in}}%
\pgfpathlineto{\pgfqpoint{1.226479in}{2.445676in}}%
\pgfpathlineto{\pgfqpoint{1.226874in}{2.461911in}}%
\pgfpathlineto{\pgfqpoint{1.227565in}{2.448010in}}%
\pgfpathlineto{\pgfqpoint{1.227664in}{2.446845in}}%
\pgfpathlineto{\pgfqpoint{1.227960in}{2.454125in}}%
\pgfpathlineto{\pgfqpoint{1.230230in}{2.518697in}}%
\pgfpathlineto{\pgfqpoint{1.231020in}{2.499280in}}%
\pgfpathlineto{\pgfqpoint{1.231414in}{2.515886in}}%
\pgfpathlineto{\pgfqpoint{1.231908in}{2.542945in}}%
\pgfpathlineto{\pgfqpoint{1.232599in}{2.522814in}}%
\pgfpathlineto{\pgfqpoint{1.232697in}{2.522831in}}%
\pgfpathlineto{\pgfqpoint{1.233586in}{2.548810in}}%
\pgfpathlineto{\pgfqpoint{1.233980in}{2.531314in}}%
\pgfpathlineto{\pgfqpoint{1.234375in}{2.515771in}}%
\pgfpathlineto{\pgfqpoint{1.234869in}{2.539895in}}%
\pgfpathlineto{\pgfqpoint{1.235362in}{2.552599in}}%
\pgfpathlineto{\pgfqpoint{1.235658in}{2.532262in}}%
\pgfpathlineto{\pgfqpoint{1.235954in}{2.516411in}}%
\pgfpathlineto{\pgfqpoint{1.236645in}{2.532182in}}%
\pgfpathlineto{\pgfqpoint{1.236843in}{2.540941in}}%
\pgfpathlineto{\pgfqpoint{1.237238in}{2.511822in}}%
\pgfpathlineto{\pgfqpoint{1.237731in}{2.484742in}}%
\pgfpathlineto{\pgfqpoint{1.238225in}{2.510342in}}%
\pgfpathlineto{\pgfqpoint{1.238422in}{2.515640in}}%
\pgfpathlineto{\pgfqpoint{1.238718in}{2.489399in}}%
\pgfpathlineto{\pgfqpoint{1.239310in}{2.450300in}}%
\pgfpathlineto{\pgfqpoint{1.240001in}{2.460195in}}%
\pgfpathlineto{\pgfqpoint{1.240199in}{2.456122in}}%
\pgfpathlineto{\pgfqpoint{1.240791in}{2.419018in}}%
\pgfpathlineto{\pgfqpoint{1.241679in}{2.429532in}}%
\pgfpathlineto{\pgfqpoint{1.242271in}{2.421708in}}%
\pgfpathlineto{\pgfqpoint{1.242567in}{2.416342in}}%
\pgfpathlineto{\pgfqpoint{1.242863in}{2.421821in}}%
\pgfpathlineto{\pgfqpoint{1.243357in}{2.421610in}}%
\pgfpathlineto{\pgfqpoint{1.243554in}{2.427114in}}%
\pgfpathlineto{\pgfqpoint{1.243949in}{2.406347in}}%
\pgfpathlineto{\pgfqpoint{1.244344in}{2.420307in}}%
\pgfpathlineto{\pgfqpoint{1.244541in}{2.418815in}}%
\pgfpathlineto{\pgfqpoint{1.244640in}{2.419434in}}%
\pgfpathlineto{\pgfqpoint{1.244936in}{2.431844in}}%
\pgfpathlineto{\pgfqpoint{1.245331in}{2.404455in}}%
\pgfpathlineto{\pgfqpoint{1.245726in}{2.384669in}}%
\pgfpathlineto{\pgfqpoint{1.246417in}{2.402352in}}%
\pgfpathlineto{\pgfqpoint{1.246811in}{2.408808in}}%
\pgfpathlineto{\pgfqpoint{1.247009in}{2.402186in}}%
\pgfpathlineto{\pgfqpoint{1.247502in}{2.377307in}}%
\pgfpathlineto{\pgfqpoint{1.247996in}{2.403878in}}%
\pgfpathlineto{\pgfqpoint{1.248094in}{2.405202in}}%
\pgfpathlineto{\pgfqpoint{1.248391in}{2.395181in}}%
\pgfpathlineto{\pgfqpoint{1.248983in}{2.345926in}}%
\pgfpathlineto{\pgfqpoint{1.249081in}{2.341396in}}%
\pgfpathlineto{\pgfqpoint{1.249476in}{2.366695in}}%
\pgfpathlineto{\pgfqpoint{1.249674in}{2.371281in}}%
\pgfpathlineto{\pgfqpoint{1.250167in}{2.360960in}}%
\pgfpathlineto{\pgfqpoint{1.252437in}{2.221175in}}%
\pgfpathlineto{\pgfqpoint{1.252733in}{2.224021in}}%
\pgfpathlineto{\pgfqpoint{1.253325in}{2.235092in}}%
\pgfpathlineto{\pgfqpoint{1.253523in}{2.227842in}}%
\pgfpathlineto{\pgfqpoint{1.255398in}{2.167445in}}%
\pgfpathlineto{\pgfqpoint{1.256089in}{2.212703in}}%
\pgfpathlineto{\pgfqpoint{1.256583in}{2.253106in}}%
\pgfpathlineto{\pgfqpoint{1.257175in}{2.218423in}}%
\pgfpathlineto{\pgfqpoint{1.257471in}{2.200426in}}%
\pgfpathlineto{\pgfqpoint{1.258162in}{2.227744in}}%
\pgfpathlineto{\pgfqpoint{1.258359in}{2.224431in}}%
\pgfpathlineto{\pgfqpoint{1.258853in}{2.199075in}}%
\pgfpathlineto{\pgfqpoint{1.259247in}{2.226487in}}%
\pgfpathlineto{\pgfqpoint{1.261221in}{2.514527in}}%
\pgfpathlineto{\pgfqpoint{1.262011in}{2.443485in}}%
\pgfpathlineto{\pgfqpoint{1.262702in}{2.313531in}}%
\pgfpathlineto{\pgfqpoint{1.263294in}{2.208464in}}%
\pgfpathlineto{\pgfqpoint{1.263985in}{2.234501in}}%
\pgfpathlineto{\pgfqpoint{1.264380in}{2.271189in}}%
\pgfpathlineto{\pgfqpoint{1.266452in}{2.432052in}}%
\pgfpathlineto{\pgfqpoint{1.266650in}{2.424311in}}%
\pgfpathlineto{\pgfqpoint{1.267341in}{2.386823in}}%
\pgfpathlineto{\pgfqpoint{1.267736in}{2.417772in}}%
\pgfpathlineto{\pgfqpoint{1.269611in}{2.506298in}}%
\pgfpathlineto{\pgfqpoint{1.268426in}{2.414869in}}%
\pgfpathlineto{\pgfqpoint{1.269808in}{2.503856in}}%
\pgfpathlineto{\pgfqpoint{1.270203in}{2.489358in}}%
\pgfpathlineto{\pgfqpoint{1.270400in}{2.477784in}}%
\pgfpathlineto{\pgfqpoint{1.270894in}{2.513719in}}%
\pgfpathlineto{\pgfqpoint{1.272868in}{2.601868in}}%
\pgfpathlineto{\pgfqpoint{1.273065in}{2.597191in}}%
\pgfpathlineto{\pgfqpoint{1.273657in}{2.574060in}}%
\pgfpathlineto{\pgfqpoint{1.274052in}{2.596915in}}%
\pgfpathlineto{\pgfqpoint{1.274842in}{2.617334in}}%
\pgfpathlineto{\pgfqpoint{1.275138in}{2.604009in}}%
\pgfpathlineto{\pgfqpoint{1.275335in}{2.595840in}}%
\pgfpathlineto{\pgfqpoint{1.275829in}{2.625236in}}%
\pgfpathlineto{\pgfqpoint{1.275928in}{2.624007in}}%
\pgfpathlineto{\pgfqpoint{1.278395in}{2.564294in}}%
\pgfpathlineto{\pgfqpoint{1.278889in}{2.582641in}}%
\pgfpathlineto{\pgfqpoint{1.279185in}{2.590165in}}%
\pgfpathlineto{\pgfqpoint{1.279579in}{2.574902in}}%
\pgfpathlineto{\pgfqpoint{1.280172in}{2.546098in}}%
\pgfpathlineto{\pgfqpoint{1.280764in}{2.568157in}}%
\pgfpathlineto{\pgfqpoint{1.281257in}{2.575573in}}%
\pgfpathlineto{\pgfqpoint{1.281455in}{2.568401in}}%
\pgfpathlineto{\pgfqpoint{1.281849in}{2.540187in}}%
\pgfpathlineto{\pgfqpoint{1.282738in}{2.553222in}}%
\pgfpathlineto{\pgfqpoint{1.283922in}{2.517140in}}%
\pgfpathlineto{\pgfqpoint{1.284218in}{2.527448in}}%
\pgfpathlineto{\pgfqpoint{1.284317in}{2.529623in}}%
\pgfpathlineto{\pgfqpoint{1.284514in}{2.518395in}}%
\pgfpathlineto{\pgfqpoint{1.286883in}{2.364134in}}%
\pgfpathlineto{\pgfqpoint{1.287179in}{2.372552in}}%
\pgfpathlineto{\pgfqpoint{1.287475in}{2.359539in}}%
\pgfpathlineto{\pgfqpoint{1.290338in}{2.199384in}}%
\pgfpathlineto{\pgfqpoint{1.290436in}{2.202706in}}%
\pgfpathlineto{\pgfqpoint{1.290831in}{2.229147in}}%
\pgfpathlineto{\pgfqpoint{1.291325in}{2.193010in}}%
\pgfpathlineto{\pgfqpoint{1.291917in}{2.184809in}}%
\pgfpathlineto{\pgfqpoint{1.292114in}{2.192122in}}%
\pgfpathlineto{\pgfqpoint{1.292410in}{2.201569in}}%
\pgfpathlineto{\pgfqpoint{1.292904in}{2.178659in}}%
\pgfpathlineto{\pgfqpoint{1.293002in}{2.178283in}}%
\pgfpathlineto{\pgfqpoint{1.293101in}{2.179921in}}%
\pgfpathlineto{\pgfqpoint{1.297543in}{2.284246in}}%
\pgfpathlineto{\pgfqpoint{1.297937in}{2.257272in}}%
\pgfpathlineto{\pgfqpoint{1.298135in}{2.246603in}}%
\pgfpathlineto{\pgfqpoint{1.298826in}{2.264512in}}%
\pgfpathlineto{\pgfqpoint{1.299122in}{2.271858in}}%
\pgfpathlineto{\pgfqpoint{1.299418in}{2.252185in}}%
\pgfpathlineto{\pgfqpoint{1.300109in}{2.241286in}}%
\pgfpathlineto{\pgfqpoint{1.300405in}{2.250607in}}%
\pgfpathlineto{\pgfqpoint{1.301688in}{2.261738in}}%
\pgfpathlineto{\pgfqpoint{1.301984in}{2.267777in}}%
\pgfpathlineto{\pgfqpoint{1.302478in}{2.252851in}}%
\pgfpathlineto{\pgfqpoint{1.302971in}{2.251502in}}%
\pgfpathlineto{\pgfqpoint{1.303168in}{2.258539in}}%
\pgfpathlineto{\pgfqpoint{1.305833in}{2.355127in}}%
\pgfpathlineto{\pgfqpoint{1.306228in}{2.353644in}}%
\pgfpathlineto{\pgfqpoint{1.306623in}{2.381149in}}%
\pgfpathlineto{\pgfqpoint{1.308992in}{2.767008in}}%
\pgfpathlineto{\pgfqpoint{1.309683in}{2.718536in}}%
\pgfpathlineto{\pgfqpoint{1.310867in}{2.478645in}}%
\pgfpathlineto{\pgfqpoint{1.311163in}{2.436962in}}%
\pgfpathlineto{\pgfqpoint{1.311755in}{2.515240in}}%
\pgfpathlineto{\pgfqpoint{1.314025in}{2.632941in}}%
\pgfpathlineto{\pgfqpoint{1.314618in}{2.589432in}}%
\pgfpathlineto{\pgfqpoint{1.315703in}{2.602487in}}%
\pgfpathlineto{\pgfqpoint{1.315901in}{2.593720in}}%
\pgfpathlineto{\pgfqpoint{1.316197in}{2.575996in}}%
\pgfpathlineto{\pgfqpoint{1.316789in}{2.603258in}}%
\pgfpathlineto{\pgfqpoint{1.317282in}{2.606327in}}%
\pgfpathlineto{\pgfqpoint{1.317480in}{2.598268in}}%
\pgfpathlineto{\pgfqpoint{1.318565in}{2.553293in}}%
\pgfpathlineto{\pgfqpoint{1.318763in}{2.564367in}}%
\pgfpathlineto{\pgfqpoint{1.318960in}{2.579912in}}%
\pgfpathlineto{\pgfqpoint{1.319454in}{2.532769in}}%
\pgfpathlineto{\pgfqpoint{1.319651in}{2.543643in}}%
\pgfpathlineto{\pgfqpoint{1.320342in}{2.571473in}}%
\pgfpathlineto{\pgfqpoint{1.320836in}{2.549732in}}%
\pgfpathlineto{\pgfqpoint{1.321526in}{2.537100in}}%
\pgfpathlineto{\pgfqpoint{1.321823in}{2.551190in}}%
\pgfpathlineto{\pgfqpoint{1.322119in}{2.565063in}}%
\pgfpathlineto{\pgfqpoint{1.322513in}{2.531011in}}%
\pgfpathlineto{\pgfqpoint{1.322711in}{2.524267in}}%
\pgfpathlineto{\pgfqpoint{1.323303in}{2.547486in}}%
\pgfpathlineto{\pgfqpoint{1.323500in}{2.550717in}}%
\pgfpathlineto{\pgfqpoint{1.323895in}{2.537289in}}%
\pgfpathlineto{\pgfqpoint{1.324586in}{2.521589in}}%
\pgfpathlineto{\pgfqpoint{1.324882in}{2.534271in}}%
\pgfpathlineto{\pgfqpoint{1.325080in}{2.543567in}}%
\pgfpathlineto{\pgfqpoint{1.325672in}{2.516817in}}%
\pgfpathlineto{\pgfqpoint{1.327646in}{2.432870in}}%
\pgfpathlineto{\pgfqpoint{1.328238in}{2.433233in}}%
\pgfpathlineto{\pgfqpoint{1.328633in}{2.425904in}}%
\pgfpathlineto{\pgfqpoint{1.331298in}{2.335100in}}%
\pgfpathlineto{\pgfqpoint{1.331692in}{2.343537in}}%
\pgfpathlineto{\pgfqpoint{1.331989in}{2.332135in}}%
\pgfpathlineto{\pgfqpoint{1.332679in}{2.288646in}}%
\pgfpathlineto{\pgfqpoint{1.333469in}{2.297366in}}%
\pgfpathlineto{\pgfqpoint{1.334456in}{2.259269in}}%
\pgfpathlineto{\pgfqpoint{1.334851in}{2.277897in}}%
\pgfpathlineto{\pgfqpoint{1.334950in}{2.280282in}}%
\pgfpathlineto{\pgfqpoint{1.335246in}{2.266673in}}%
\pgfpathlineto{\pgfqpoint{1.335937in}{2.225367in}}%
\pgfpathlineto{\pgfqpoint{1.336430in}{2.252339in}}%
\pgfpathlineto{\pgfqpoint{1.336726in}{2.264547in}}%
\pgfpathlineto{\pgfqpoint{1.337318in}{2.238551in}}%
\pgfpathlineto{\pgfqpoint{1.337516in}{2.236624in}}%
\pgfpathlineto{\pgfqpoint{1.337812in}{2.245816in}}%
\pgfpathlineto{\pgfqpoint{1.338207in}{2.266467in}}%
\pgfpathlineto{\pgfqpoint{1.338897in}{2.254224in}}%
\pgfpathlineto{\pgfqpoint{1.340575in}{2.235805in}}%
\pgfpathlineto{\pgfqpoint{1.340674in}{2.233966in}}%
\pgfpathlineto{\pgfqpoint{1.340970in}{2.242660in}}%
\pgfpathlineto{\pgfqpoint{1.343043in}{2.430624in}}%
\pgfpathlineto{\pgfqpoint{1.345214in}{2.615258in}}%
\pgfpathlineto{\pgfqpoint{1.345609in}{2.611909in}}%
\pgfpathlineto{\pgfqpoint{1.346102in}{2.643520in}}%
\pgfpathlineto{\pgfqpoint{1.346300in}{2.649676in}}%
\pgfpathlineto{\pgfqpoint{1.346695in}{2.626779in}}%
\pgfpathlineto{\pgfqpoint{1.347682in}{2.581808in}}%
\pgfpathlineto{\pgfqpoint{1.348175in}{2.589727in}}%
\pgfpathlineto{\pgfqpoint{1.348866in}{2.567690in}}%
\pgfpathlineto{\pgfqpoint{1.349754in}{2.575812in}}%
\pgfpathlineto{\pgfqpoint{1.350050in}{2.578990in}}%
\pgfpathlineto{\pgfqpoint{1.350347in}{2.573525in}}%
\pgfpathlineto{\pgfqpoint{1.350643in}{2.566907in}}%
\pgfpathlineto{\pgfqpoint{1.351037in}{2.585510in}}%
\pgfpathlineto{\pgfqpoint{1.352419in}{2.633585in}}%
\pgfpathlineto{\pgfqpoint{1.352715in}{2.628069in}}%
\pgfpathlineto{\pgfqpoint{1.353702in}{2.577201in}}%
\pgfpathlineto{\pgfqpoint{1.354591in}{2.601111in}}%
\pgfpathlineto{\pgfqpoint{1.355183in}{2.589888in}}%
\pgfpathlineto{\pgfqpoint{1.355775in}{2.657845in}}%
\pgfpathlineto{\pgfqpoint{1.356466in}{2.805635in}}%
\pgfpathlineto{\pgfqpoint{1.357255in}{2.755103in}}%
\pgfpathlineto{\pgfqpoint{1.358736in}{2.435198in}}%
\pgfpathlineto{\pgfqpoint{1.360414in}{2.454267in}}%
\pgfpathlineto{\pgfqpoint{1.360710in}{2.450435in}}%
\pgfpathlineto{\pgfqpoint{1.361006in}{2.459269in}}%
\pgfpathlineto{\pgfqpoint{1.361500in}{2.483046in}}%
\pgfpathlineto{\pgfqpoint{1.361993in}{2.454479in}}%
\pgfpathlineto{\pgfqpoint{1.363276in}{2.400286in}}%
\pgfpathlineto{\pgfqpoint{1.363967in}{2.368749in}}%
\pgfpathlineto{\pgfqpoint{1.364362in}{2.391753in}}%
\pgfpathlineto{\pgfqpoint{1.364559in}{2.402159in}}%
\pgfpathlineto{\pgfqpoint{1.365151in}{2.374325in}}%
\pgfpathlineto{\pgfqpoint{1.367125in}{2.291651in}}%
\pgfpathlineto{\pgfqpoint{1.367520in}{2.304821in}}%
\pgfpathlineto{\pgfqpoint{1.367718in}{2.306965in}}%
\pgfpathlineto{\pgfqpoint{1.368112in}{2.296946in}}%
\pgfpathlineto{\pgfqpoint{1.370086in}{2.240046in}}%
\pgfpathlineto{\pgfqpoint{1.370284in}{2.243275in}}%
\pgfpathlineto{\pgfqpoint{1.370876in}{2.264609in}}%
\pgfpathlineto{\pgfqpoint{1.371073in}{2.273128in}}%
\pgfpathlineto{\pgfqpoint{1.371567in}{2.255206in}}%
\pgfpathlineto{\pgfqpoint{1.371764in}{2.256880in}}%
\pgfpathlineto{\pgfqpoint{1.371863in}{2.256925in}}%
\pgfpathlineto{\pgfqpoint{1.372159in}{2.251428in}}%
\pgfpathlineto{\pgfqpoint{1.372455in}{2.264879in}}%
\pgfpathlineto{\pgfqpoint{1.372751in}{2.278579in}}%
\pgfpathlineto{\pgfqpoint{1.373245in}{2.243709in}}%
\pgfpathlineto{\pgfqpoint{1.373343in}{2.243179in}}%
\pgfpathlineto{\pgfqpoint{1.373541in}{2.247167in}}%
\pgfpathlineto{\pgfqpoint{1.374133in}{2.279755in}}%
\pgfpathlineto{\pgfqpoint{1.374330in}{2.289033in}}%
\pgfpathlineto{\pgfqpoint{1.374824in}{2.270119in}}%
\pgfpathlineto{\pgfqpoint{1.375021in}{2.270578in}}%
\pgfpathlineto{\pgfqpoint{1.375317in}{2.264095in}}%
\pgfpathlineto{\pgfqpoint{1.375613in}{2.275597in}}%
\pgfpathlineto{\pgfqpoint{1.375910in}{2.292650in}}%
\pgfpathlineto{\pgfqpoint{1.376699in}{2.276797in}}%
\pgfpathlineto{\pgfqpoint{1.376897in}{2.274222in}}%
\pgfpathlineto{\pgfqpoint{1.377094in}{2.284290in}}%
\pgfpathlineto{\pgfqpoint{1.377982in}{2.321579in}}%
\pgfpathlineto{\pgfqpoint{1.378377in}{2.302510in}}%
\pgfpathlineto{\pgfqpoint{1.378476in}{2.299825in}}%
\pgfpathlineto{\pgfqpoint{1.378772in}{2.315816in}}%
\pgfpathlineto{\pgfqpoint{1.379561in}{2.328202in}}%
\pgfpathlineto{\pgfqpoint{1.379858in}{2.321033in}}%
\pgfpathlineto{\pgfqpoint{1.380055in}{2.317096in}}%
\pgfpathlineto{\pgfqpoint{1.380351in}{2.331570in}}%
\pgfpathlineto{\pgfqpoint{1.380943in}{2.367038in}}%
\pgfpathlineto{\pgfqpoint{1.381634in}{2.346956in}}%
\pgfpathlineto{\pgfqpoint{1.381832in}{2.352769in}}%
\pgfpathlineto{\pgfqpoint{1.384398in}{2.484028in}}%
\pgfpathlineto{\pgfqpoint{1.384990in}{2.503844in}}%
\pgfpathlineto{\pgfqpoint{1.386273in}{2.562748in}}%
\pgfpathlineto{\pgfqpoint{1.386668in}{2.550410in}}%
\pgfpathlineto{\pgfqpoint{1.386766in}{2.550552in}}%
\pgfpathlineto{\pgfqpoint{1.387260in}{2.566614in}}%
\pgfpathlineto{\pgfqpoint{1.387753in}{2.552733in}}%
\pgfpathlineto{\pgfqpoint{1.388346in}{2.527885in}}%
\pgfpathlineto{\pgfqpoint{1.388938in}{2.544492in}}%
\pgfpathlineto{\pgfqpoint{1.389037in}{2.545379in}}%
\pgfpathlineto{\pgfqpoint{1.389234in}{2.538231in}}%
\pgfpathlineto{\pgfqpoint{1.389530in}{2.527133in}}%
\pgfpathlineto{\pgfqpoint{1.390221in}{2.540613in}}%
\pgfpathlineto{\pgfqpoint{1.390813in}{2.584555in}}%
\pgfpathlineto{\pgfqpoint{1.391405in}{2.550560in}}%
\pgfpathlineto{\pgfqpoint{1.391603in}{2.545052in}}%
\pgfpathlineto{\pgfqpoint{1.391899in}{2.571812in}}%
\pgfpathlineto{\pgfqpoint{1.392590in}{2.596885in}}%
\pgfpathlineto{\pgfqpoint{1.392984in}{2.576591in}}%
\pgfpathlineto{\pgfqpoint{1.395057in}{2.471374in}}%
\pgfpathlineto{\pgfqpoint{1.395353in}{2.476690in}}%
\pgfpathlineto{\pgfqpoint{1.395847in}{2.478690in}}%
\pgfpathlineto{\pgfqpoint{1.396044in}{2.475747in}}%
\pgfpathlineto{\pgfqpoint{1.398117in}{2.390012in}}%
\pgfpathlineto{\pgfqpoint{1.398512in}{2.393961in}}%
\pgfpathlineto{\pgfqpoint{1.399499in}{2.426936in}}%
\pgfpathlineto{\pgfqpoint{1.399795in}{2.406972in}}%
\pgfpathlineto{\pgfqpoint{1.401473in}{2.339738in}}%
\pgfpathlineto{\pgfqpoint{1.401571in}{2.340396in}}%
\pgfpathlineto{\pgfqpoint{1.402953in}{2.414058in}}%
\pgfpathlineto{\pgfqpoint{1.403940in}{2.583430in}}%
\pgfpathlineto{\pgfqpoint{1.404631in}{2.537910in}}%
\pgfpathlineto{\pgfqpoint{1.405618in}{2.371167in}}%
\pgfpathlineto{\pgfqpoint{1.406111in}{2.247285in}}%
\pgfpathlineto{\pgfqpoint{1.406901in}{2.281559in}}%
\pgfpathlineto{\pgfqpoint{1.407395in}{2.267358in}}%
\pgfpathlineto{\pgfqpoint{1.407789in}{2.246879in}}%
\pgfpathlineto{\pgfqpoint{1.408382in}{2.266514in}}%
\pgfpathlineto{\pgfqpoint{1.408776in}{2.303987in}}%
\pgfpathlineto{\pgfqpoint{1.409467in}{2.282480in}}%
\pgfpathlineto{\pgfqpoint{1.409862in}{2.256984in}}%
\pgfpathlineto{\pgfqpoint{1.410750in}{2.263601in}}%
\pgfpathlineto{\pgfqpoint{1.411145in}{2.251118in}}%
\pgfpathlineto{\pgfqpoint{1.411540in}{2.270851in}}%
\pgfpathlineto{\pgfqpoint{1.412231in}{2.321785in}}%
\pgfpathlineto{\pgfqpoint{1.412823in}{2.288668in}}%
\pgfpathlineto{\pgfqpoint{1.413020in}{2.285650in}}%
\pgfpathlineto{\pgfqpoint{1.413514in}{2.297742in}}%
\pgfpathlineto{\pgfqpoint{1.415290in}{2.354864in}}%
\pgfpathlineto{\pgfqpoint{1.415587in}{2.342521in}}%
\pgfpathlineto{\pgfqpoint{1.416179in}{2.304375in}}%
\pgfpathlineto{\pgfqpoint{1.417067in}{2.312306in}}%
\pgfpathlineto{\pgfqpoint{1.422101in}{2.585502in}}%
\pgfpathlineto{\pgfqpoint{1.423186in}{2.570270in}}%
\pgfpathlineto{\pgfqpoint{1.423285in}{2.570264in}}%
\pgfpathlineto{\pgfqpoint{1.423680in}{2.577642in}}%
\pgfpathlineto{\pgfqpoint{1.423877in}{2.569438in}}%
\pgfpathlineto{\pgfqpoint{1.424272in}{2.552621in}}%
\pgfpathlineto{\pgfqpoint{1.424864in}{2.573121in}}%
\pgfpathlineto{\pgfqpoint{1.425062in}{2.575578in}}%
\pgfpathlineto{\pgfqpoint{1.425358in}{2.560830in}}%
\pgfpathlineto{\pgfqpoint{1.425851in}{2.532721in}}%
\pgfpathlineto{\pgfqpoint{1.426443in}{2.562031in}}%
\pgfpathlineto{\pgfqpoint{1.427134in}{2.573733in}}%
\pgfpathlineto{\pgfqpoint{1.427430in}{2.566183in}}%
\pgfpathlineto{\pgfqpoint{1.429108in}{2.526454in}}%
\pgfpathlineto{\pgfqpoint{1.428121in}{2.567952in}}%
\pgfpathlineto{\pgfqpoint{1.429207in}{2.528239in}}%
\pgfpathlineto{\pgfqpoint{1.430293in}{2.543828in}}%
\pgfpathlineto{\pgfqpoint{1.430391in}{2.539969in}}%
\pgfpathlineto{\pgfqpoint{1.430984in}{2.498857in}}%
\pgfpathlineto{\pgfqpoint{1.431872in}{2.509723in}}%
\pgfpathlineto{\pgfqpoint{1.431971in}{2.509909in}}%
\pgfpathlineto{\pgfqpoint{1.432563in}{2.481427in}}%
\pgfpathlineto{\pgfqpoint{1.433254in}{2.505012in}}%
\pgfpathlineto{\pgfqpoint{1.433550in}{2.497459in}}%
\pgfpathlineto{\pgfqpoint{1.434043in}{2.471863in}}%
\pgfpathlineto{\pgfqpoint{1.434932in}{2.479029in}}%
\pgfpathlineto{\pgfqpoint{1.435622in}{2.453880in}}%
\pgfpathlineto{\pgfqpoint{1.436313in}{2.472160in}}%
\pgfpathlineto{\pgfqpoint{1.436807in}{2.488848in}}%
\pgfpathlineto{\pgfqpoint{1.437103in}{2.469935in}}%
\pgfpathlineto{\pgfqpoint{1.437300in}{2.461852in}}%
\pgfpathlineto{\pgfqpoint{1.437991in}{2.479677in}}%
\pgfpathlineto{\pgfqpoint{1.438189in}{2.484479in}}%
\pgfpathlineto{\pgfqpoint{1.438485in}{2.466551in}}%
\pgfpathlineto{\pgfqpoint{1.438781in}{2.448825in}}%
\pgfpathlineto{\pgfqpoint{1.439472in}{2.475081in}}%
\pgfpathlineto{\pgfqpoint{1.439768in}{2.496596in}}%
\pgfpathlineto{\pgfqpoint{1.440459in}{2.463832in}}%
\pgfpathlineto{\pgfqpoint{1.442433in}{2.393403in}}%
\pgfpathlineto{\pgfqpoint{1.442630in}{2.401562in}}%
\pgfpathlineto{\pgfqpoint{1.443420in}{2.411586in}}%
\pgfpathlineto{\pgfqpoint{1.443617in}{2.403039in}}%
\pgfpathlineto{\pgfqpoint{1.444308in}{2.364366in}}%
\pgfpathlineto{\pgfqpoint{1.445098in}{2.370372in}}%
\pgfpathlineto{\pgfqpoint{1.445492in}{2.353989in}}%
\pgfpathlineto{\pgfqpoint{1.445986in}{2.375864in}}%
\pgfpathlineto{\pgfqpoint{1.446677in}{2.412045in}}%
\pgfpathlineto{\pgfqpoint{1.447072in}{2.386114in}}%
\pgfpathlineto{\pgfqpoint{1.447664in}{2.356555in}}%
\pgfpathlineto{\pgfqpoint{1.448453in}{2.358952in}}%
\pgfpathlineto{\pgfqpoint{1.448749in}{2.354640in}}%
\pgfpathlineto{\pgfqpoint{1.449045in}{2.364857in}}%
\pgfpathlineto{\pgfqpoint{1.450723in}{2.560816in}}%
\pgfpathlineto{\pgfqpoint{1.451414in}{2.645576in}}%
\pgfpathlineto{\pgfqpoint{1.451908in}{2.602430in}}%
\pgfpathlineto{\pgfqpoint{1.452796in}{2.472443in}}%
\pgfpathlineto{\pgfqpoint{1.453290in}{2.362060in}}%
\pgfpathlineto{\pgfqpoint{1.453980in}{2.402764in}}%
\pgfpathlineto{\pgfqpoint{1.456349in}{2.511115in}}%
\pgfpathlineto{\pgfqpoint{1.456547in}{2.503917in}}%
\pgfpathlineto{\pgfqpoint{1.457238in}{2.456395in}}%
\pgfpathlineto{\pgfqpoint{1.457928in}{2.476487in}}%
\pgfpathlineto{\pgfqpoint{1.458521in}{2.470156in}}%
\pgfpathlineto{\pgfqpoint{1.459310in}{2.519512in}}%
\pgfpathlineto{\pgfqpoint{1.459606in}{2.532488in}}%
\pgfpathlineto{\pgfqpoint{1.460100in}{2.501916in}}%
\pgfpathlineto{\pgfqpoint{1.460495in}{2.497507in}}%
\pgfpathlineto{\pgfqpoint{1.460889in}{2.509148in}}%
\pgfpathlineto{\pgfqpoint{1.461284in}{2.531115in}}%
\pgfpathlineto{\pgfqpoint{1.461876in}{2.500375in}}%
\pgfpathlineto{\pgfqpoint{1.462172in}{2.511370in}}%
\pgfpathlineto{\pgfqpoint{1.462765in}{2.532280in}}%
\pgfpathlineto{\pgfqpoint{1.463258in}{2.514474in}}%
\pgfpathlineto{\pgfqpoint{1.463554in}{2.499741in}}%
\pgfpathlineto{\pgfqpoint{1.464048in}{2.527044in}}%
\pgfpathlineto{\pgfqpoint{1.464541in}{2.542807in}}%
\pgfpathlineto{\pgfqpoint{1.465232in}{2.534361in}}%
\pgfpathlineto{\pgfqpoint{1.465430in}{2.533978in}}%
\pgfpathlineto{\pgfqpoint{1.465627in}{2.535422in}}%
\pgfpathlineto{\pgfqpoint{1.465923in}{2.540709in}}%
\pgfpathlineto{\pgfqpoint{1.466515in}{2.535283in}}%
\pgfpathlineto{\pgfqpoint{1.466910in}{2.521269in}}%
\pgfpathlineto{\pgfqpoint{1.467305in}{2.544139in}}%
\pgfpathlineto{\pgfqpoint{1.467700in}{2.560045in}}%
\pgfpathlineto{\pgfqpoint{1.468292in}{2.537363in}}%
\pgfpathlineto{\pgfqpoint{1.468489in}{2.533666in}}%
\pgfpathlineto{\pgfqpoint{1.468884in}{2.553883in}}%
\pgfpathlineto{\pgfqpoint{1.469081in}{2.558215in}}%
\pgfpathlineto{\pgfqpoint{1.469575in}{2.545893in}}%
\pgfpathlineto{\pgfqpoint{1.469970in}{2.534706in}}%
\pgfpathlineto{\pgfqpoint{1.470661in}{2.541898in}}%
\pgfpathlineto{\pgfqpoint{1.470957in}{2.549610in}}%
\pgfpathlineto{\pgfqpoint{1.471351in}{2.531631in}}%
\pgfpathlineto{\pgfqpoint{1.471845in}{2.505416in}}%
\pgfpathlineto{\pgfqpoint{1.472437in}{2.528565in}}%
\pgfpathlineto{\pgfqpoint{1.472536in}{2.528969in}}%
\pgfpathlineto{\pgfqpoint{1.472635in}{2.526413in}}%
\pgfpathlineto{\pgfqpoint{1.473622in}{2.502441in}}%
\pgfpathlineto{\pgfqpoint{1.473918in}{2.515031in}}%
\pgfpathlineto{\pgfqpoint{1.474214in}{2.525845in}}%
\pgfpathlineto{\pgfqpoint{1.474707in}{2.505289in}}%
\pgfpathlineto{\pgfqpoint{1.475201in}{2.486984in}}%
\pgfpathlineto{\pgfqpoint{1.475694in}{2.504038in}}%
\pgfpathlineto{\pgfqpoint{1.475892in}{2.511126in}}%
\pgfpathlineto{\pgfqpoint{1.476385in}{2.483187in}}%
\pgfpathlineto{\pgfqpoint{1.476681in}{2.474269in}}%
\pgfpathlineto{\pgfqpoint{1.477175in}{2.491228in}}%
\pgfpathlineto{\pgfqpoint{1.477273in}{2.492557in}}%
\pgfpathlineto{\pgfqpoint{1.477767in}{2.485210in}}%
\pgfpathlineto{\pgfqpoint{1.478359in}{2.452389in}}%
\pgfpathlineto{\pgfqpoint{1.479050in}{2.479311in}}%
\pgfpathlineto{\pgfqpoint{1.479445in}{2.484253in}}%
\pgfpathlineto{\pgfqpoint{1.479642in}{2.476202in}}%
\pgfpathlineto{\pgfqpoint{1.479938in}{2.459934in}}%
\pgfpathlineto{\pgfqpoint{1.480530in}{2.480852in}}%
\pgfpathlineto{\pgfqpoint{1.480728in}{2.477405in}}%
\pgfpathlineto{\pgfqpoint{1.481616in}{2.455781in}}%
\pgfpathlineto{\pgfqpoint{1.482110in}{2.466338in}}%
\pgfpathlineto{\pgfqpoint{1.482208in}{2.466084in}}%
\pgfpathlineto{\pgfqpoint{1.482307in}{2.467680in}}%
\pgfpathlineto{\pgfqpoint{1.482603in}{2.479199in}}%
\pgfpathlineto{\pgfqpoint{1.483195in}{2.460957in}}%
\pgfpathlineto{\pgfqpoint{1.483491in}{2.466465in}}%
\pgfpathlineto{\pgfqpoint{1.483886in}{2.486367in}}%
\pgfpathlineto{\pgfqpoint{1.484676in}{2.473714in}}%
\pgfpathlineto{\pgfqpoint{1.484972in}{2.463588in}}%
\pgfpathlineto{\pgfqpoint{1.485268in}{2.484709in}}%
\pgfpathlineto{\pgfqpoint{1.485860in}{2.506901in}}%
\pgfpathlineto{\pgfqpoint{1.486354in}{2.488996in}}%
\pgfpathlineto{\pgfqpoint{1.486551in}{2.495240in}}%
\pgfpathlineto{\pgfqpoint{1.487045in}{2.529723in}}%
\pgfpathlineto{\pgfqpoint{1.487637in}{2.508816in}}%
\pgfpathlineto{\pgfqpoint{1.488624in}{2.472754in}}%
\pgfpathlineto{\pgfqpoint{1.488920in}{2.493329in}}%
\pgfpathlineto{\pgfqpoint{1.489019in}{2.496151in}}%
\pgfpathlineto{\pgfqpoint{1.489216in}{2.477111in}}%
\pgfpathlineto{\pgfqpoint{1.489512in}{2.452093in}}%
\pgfpathlineto{\pgfqpoint{1.490302in}{2.463940in}}%
\pgfpathlineto{\pgfqpoint{1.490499in}{2.471600in}}%
\pgfpathlineto{\pgfqpoint{1.490993in}{2.448589in}}%
\pgfpathlineto{\pgfqpoint{1.491486in}{2.416261in}}%
\pgfpathlineto{\pgfqpoint{1.492078in}{2.449643in}}%
\pgfpathlineto{\pgfqpoint{1.492374in}{2.456356in}}%
\pgfpathlineto{\pgfqpoint{1.492868in}{2.447412in}}%
\pgfpathlineto{\pgfqpoint{1.493065in}{2.448245in}}%
\pgfpathlineto{\pgfqpoint{1.493263in}{2.449308in}}%
\pgfpathlineto{\pgfqpoint{1.493756in}{2.478462in}}%
\pgfpathlineto{\pgfqpoint{1.494052in}{2.451622in}}%
\pgfpathlineto{\pgfqpoint{1.494743in}{2.378788in}}%
\pgfpathlineto{\pgfqpoint{1.495335in}{2.414919in}}%
\pgfpathlineto{\pgfqpoint{1.495533in}{2.412628in}}%
\pgfpathlineto{\pgfqpoint{1.495631in}{2.411202in}}%
\pgfpathlineto{\pgfqpoint{1.495927in}{2.420141in}}%
\pgfpathlineto{\pgfqpoint{1.496914in}{2.524744in}}%
\pgfpathlineto{\pgfqpoint{1.498395in}{2.744304in}}%
\pgfpathlineto{\pgfqpoint{1.498888in}{2.709008in}}%
\pgfpathlineto{\pgfqpoint{1.499481in}{2.612395in}}%
\pgfpathlineto{\pgfqpoint{1.500073in}{2.441516in}}%
\pgfpathlineto{\pgfqpoint{1.500862in}{2.486854in}}%
\pgfpathlineto{\pgfqpoint{1.501159in}{2.474654in}}%
\pgfpathlineto{\pgfqpoint{1.501652in}{2.495752in}}%
\pgfpathlineto{\pgfqpoint{1.503626in}{2.551337in}}%
\pgfpathlineto{\pgfqpoint{1.503823in}{2.544365in}}%
\pgfpathlineto{\pgfqpoint{1.504514in}{2.505313in}}%
\pgfpathlineto{\pgfqpoint{1.505008in}{2.537042in}}%
\pgfpathlineto{\pgfqpoint{1.506093in}{2.526753in}}%
\pgfpathlineto{\pgfqpoint{1.505699in}{2.537739in}}%
\pgfpathlineto{\pgfqpoint{1.506291in}{2.530797in}}%
\pgfpathlineto{\pgfqpoint{1.506883in}{2.554354in}}%
\pgfpathlineto{\pgfqpoint{1.507278in}{2.527973in}}%
\pgfpathlineto{\pgfqpoint{1.507475in}{2.520371in}}%
\pgfpathlineto{\pgfqpoint{1.508166in}{2.540192in}}%
\pgfpathlineto{\pgfqpoint{1.508758in}{2.565253in}}%
\pgfpathlineto{\pgfqpoint{1.509252in}{2.543968in}}%
\pgfpathlineto{\pgfqpoint{1.509351in}{2.542808in}}%
\pgfpathlineto{\pgfqpoint{1.509647in}{2.550635in}}%
\pgfpathlineto{\pgfqpoint{1.510239in}{2.574889in}}%
\pgfpathlineto{\pgfqpoint{1.510831in}{2.559553in}}%
\pgfpathlineto{\pgfqpoint{1.511028in}{2.553095in}}%
\pgfpathlineto{\pgfqpoint{1.511423in}{2.569289in}}%
\pgfpathlineto{\pgfqpoint{1.511917in}{2.580119in}}%
\pgfpathlineto{\pgfqpoint{1.512410in}{2.566686in}}%
\pgfpathlineto{\pgfqpoint{1.512608in}{2.560076in}}%
\pgfpathlineto{\pgfqpoint{1.513101in}{2.577205in}}%
\pgfpathlineto{\pgfqpoint{1.513397in}{2.586750in}}%
\pgfpathlineto{\pgfqpoint{1.514088in}{2.576095in}}%
\pgfpathlineto{\pgfqpoint{1.514483in}{2.565814in}}%
\pgfpathlineto{\pgfqpoint{1.515075in}{2.576173in}}%
\pgfpathlineto{\pgfqpoint{1.515272in}{2.574239in}}%
\pgfpathlineto{\pgfqpoint{1.515667in}{2.571304in}}%
\pgfpathlineto{\pgfqpoint{1.516062in}{2.548638in}}%
\pgfpathlineto{\pgfqpoint{1.516457in}{2.582825in}}%
\pgfpathlineto{\pgfqpoint{1.516556in}{2.587871in}}%
\pgfpathlineto{\pgfqpoint{1.517246in}{2.566592in}}%
\pgfpathlineto{\pgfqpoint{1.517641in}{2.553552in}}%
\pgfpathlineto{\pgfqpoint{1.518036in}{2.567777in}}%
\pgfpathlineto{\pgfqpoint{1.518332in}{2.567556in}}%
\pgfpathlineto{\pgfqpoint{1.518530in}{2.570531in}}%
\pgfpathlineto{\pgfqpoint{1.518826in}{2.559178in}}%
\pgfpathlineto{\pgfqpoint{1.519220in}{2.544094in}}%
\pgfpathlineto{\pgfqpoint{1.519813in}{2.563943in}}%
\pgfpathlineto{\pgfqpoint{1.520010in}{2.568218in}}%
\pgfpathlineto{\pgfqpoint{1.520602in}{2.554920in}}%
\pgfpathlineto{\pgfqpoint{1.520701in}{2.554852in}}%
\pgfpathlineto{\pgfqpoint{1.520997in}{2.560447in}}%
\pgfpathlineto{\pgfqpoint{1.521392in}{2.548849in}}%
\pgfpathlineto{\pgfqpoint{1.521787in}{2.556165in}}%
\pgfpathlineto{\pgfqpoint{1.522181in}{2.537206in}}%
\pgfpathlineto{\pgfqpoint{1.522576in}{2.520543in}}%
\pgfpathlineto{\pgfqpoint{1.523168in}{2.538868in}}%
\pgfpathlineto{\pgfqpoint{1.523267in}{2.539751in}}%
\pgfpathlineto{\pgfqpoint{1.523563in}{2.532505in}}%
\pgfpathlineto{\pgfqpoint{1.524550in}{2.497069in}}%
\pgfpathlineto{\pgfqpoint{1.525044in}{2.510257in}}%
\pgfpathlineto{\pgfqpoint{1.525537in}{2.484053in}}%
\pgfpathlineto{\pgfqpoint{1.526327in}{2.496187in}}%
\pgfpathlineto{\pgfqpoint{1.526820in}{2.511594in}}%
\pgfpathlineto{\pgfqpoint{1.527215in}{2.492714in}}%
\pgfpathlineto{\pgfqpoint{1.527511in}{2.483707in}}%
\pgfpathlineto{\pgfqpoint{1.528005in}{2.506416in}}%
\pgfpathlineto{\pgfqpoint{1.528301in}{2.514501in}}%
\pgfpathlineto{\pgfqpoint{1.528794in}{2.501902in}}%
\pgfpathlineto{\pgfqpoint{1.528992in}{2.502884in}}%
\pgfpathlineto{\pgfqpoint{1.532940in}{2.588693in}}%
\pgfpathlineto{\pgfqpoint{1.533236in}{2.604970in}}%
\pgfpathlineto{\pgfqpoint{1.533828in}{2.570870in}}%
\pgfpathlineto{\pgfqpoint{1.535802in}{2.523341in}}%
\pgfpathlineto{\pgfqpoint{1.534321in}{2.576654in}}%
\pgfpathlineto{\pgfqpoint{1.536197in}{2.539058in}}%
\pgfpathlineto{\pgfqpoint{1.536295in}{2.540696in}}%
\pgfpathlineto{\pgfqpoint{1.536493in}{2.532020in}}%
\pgfpathlineto{\pgfqpoint{1.538269in}{2.484010in}}%
\pgfpathlineto{\pgfqpoint{1.538368in}{2.485405in}}%
\pgfpathlineto{\pgfqpoint{1.538664in}{2.474190in}}%
\pgfpathlineto{\pgfqpoint{1.538862in}{2.463379in}}%
\pgfpathlineto{\pgfqpoint{1.539552in}{2.480676in}}%
\pgfpathlineto{\pgfqpoint{1.540046in}{2.506116in}}%
\pgfpathlineto{\pgfqpoint{1.540638in}{2.488284in}}%
\pgfpathlineto{\pgfqpoint{1.542415in}{2.433398in}}%
\pgfpathlineto{\pgfqpoint{1.543698in}{2.477979in}}%
\pgfpathlineto{\pgfqpoint{1.544981in}{2.687726in}}%
\pgfpathlineto{\pgfqpoint{1.545770in}{2.639014in}}%
\pgfpathlineto{\pgfqpoint{1.546363in}{2.509002in}}%
\pgfpathlineto{\pgfqpoint{1.546856in}{2.376697in}}%
\pgfpathlineto{\pgfqpoint{1.547547in}{2.449941in}}%
\pgfpathlineto{\pgfqpoint{1.548041in}{2.450822in}}%
\pgfpathlineto{\pgfqpoint{1.548238in}{2.448615in}}%
\pgfpathlineto{\pgfqpoint{1.548731in}{2.432883in}}%
\pgfpathlineto{\pgfqpoint{1.549225in}{2.452221in}}%
\pgfpathlineto{\pgfqpoint{1.549817in}{2.502514in}}%
\pgfpathlineto{\pgfqpoint{1.550311in}{2.452027in}}%
\pgfpathlineto{\pgfqpoint{1.550903in}{2.446596in}}%
\pgfpathlineto{\pgfqpoint{1.550607in}{2.452443in}}%
\pgfpathlineto{\pgfqpoint{1.551199in}{2.451770in}}%
\pgfpathlineto{\pgfqpoint{1.551298in}{2.453473in}}%
\pgfpathlineto{\pgfqpoint{1.551594in}{2.445031in}}%
\pgfpathlineto{\pgfqpoint{1.552087in}{2.425124in}}%
\pgfpathlineto{\pgfqpoint{1.552482in}{2.447695in}}%
\pgfpathlineto{\pgfqpoint{1.553074in}{2.466344in}}%
\pgfpathlineto{\pgfqpoint{1.553469in}{2.447056in}}%
\pgfpathlineto{\pgfqpoint{1.553666in}{2.440620in}}%
\pgfpathlineto{\pgfqpoint{1.554357in}{2.455050in}}%
\pgfpathlineto{\pgfqpoint{1.554456in}{2.455570in}}%
\pgfpathlineto{\pgfqpoint{1.554752in}{2.452145in}}%
\pgfpathlineto{\pgfqpoint{1.555147in}{2.454478in}}%
\pgfpathlineto{\pgfqpoint{1.555443in}{2.445128in}}%
\pgfpathlineto{\pgfqpoint{1.555739in}{2.462317in}}%
\pgfpathlineto{\pgfqpoint{1.556035in}{2.485126in}}%
\pgfpathlineto{\pgfqpoint{1.556825in}{2.460866in}}%
\pgfpathlineto{\pgfqpoint{1.557022in}{2.453433in}}%
\pgfpathlineto{\pgfqpoint{1.557516in}{2.475107in}}%
\pgfpathlineto{\pgfqpoint{1.557910in}{2.495559in}}%
\pgfpathlineto{\pgfqpoint{1.558404in}{2.465166in}}%
\pgfpathlineto{\pgfqpoint{1.558503in}{2.462592in}}%
\pgfpathlineto{\pgfqpoint{1.558897in}{2.477578in}}%
\pgfpathlineto{\pgfqpoint{1.559391in}{2.500420in}}%
\pgfpathlineto{\pgfqpoint{1.560279in}{2.497238in}}%
\pgfpathlineto{\pgfqpoint{1.561069in}{2.502318in}}%
\pgfpathlineto{\pgfqpoint{1.561266in}{2.497845in}}%
\pgfpathlineto{\pgfqpoint{1.562154in}{2.481283in}}%
\pgfpathlineto{\pgfqpoint{1.562352in}{2.492843in}}%
\pgfpathlineto{\pgfqpoint{1.562648in}{2.503991in}}%
\pgfpathlineto{\pgfqpoint{1.563339in}{2.491705in}}%
\pgfpathlineto{\pgfqpoint{1.563635in}{2.485631in}}%
\pgfpathlineto{\pgfqpoint{1.563931in}{2.500733in}}%
\pgfpathlineto{\pgfqpoint{1.564227in}{2.512124in}}%
\pgfpathlineto{\pgfqpoint{1.564819in}{2.490281in}}%
\pgfpathlineto{\pgfqpoint{1.565214in}{2.471834in}}%
\pgfpathlineto{\pgfqpoint{1.565806in}{2.494180in}}%
\pgfpathlineto{\pgfqpoint{1.566004in}{2.495776in}}%
\pgfpathlineto{\pgfqpoint{1.566300in}{2.488207in}}%
\pgfpathlineto{\pgfqpoint{1.566892in}{2.478545in}}%
\pgfpathlineto{\pgfqpoint{1.567386in}{2.486038in}}%
\pgfpathlineto{\pgfqpoint{1.567484in}{2.486340in}}%
\pgfpathlineto{\pgfqpoint{1.567583in}{2.484079in}}%
\pgfpathlineto{\pgfqpoint{1.568669in}{2.448369in}}%
\pgfpathlineto{\pgfqpoint{1.569063in}{2.460236in}}%
\pgfpathlineto{\pgfqpoint{1.569360in}{2.470424in}}%
\pgfpathlineto{\pgfqpoint{1.569754in}{2.438328in}}%
\pgfpathlineto{\pgfqpoint{1.571926in}{2.333183in}}%
\pgfpathlineto{\pgfqpoint{1.572123in}{2.335137in}}%
\pgfpathlineto{\pgfqpoint{1.572419in}{2.342242in}}%
\pgfpathlineto{\pgfqpoint{1.572814in}{2.324739in}}%
\pgfpathlineto{\pgfqpoint{1.573011in}{2.314533in}}%
\pgfpathlineto{\pgfqpoint{1.573505in}{2.350231in}}%
\pgfpathlineto{\pgfqpoint{1.574294in}{2.391057in}}%
\pgfpathlineto{\pgfqpoint{1.574985in}{2.374225in}}%
\pgfpathlineto{\pgfqpoint{1.575874in}{2.406576in}}%
\pgfpathlineto{\pgfqpoint{1.576268in}{2.385149in}}%
\pgfpathlineto{\pgfqpoint{1.576565in}{2.377548in}}%
\pgfpathlineto{\pgfqpoint{1.577157in}{2.391509in}}%
\pgfpathlineto{\pgfqpoint{1.579229in}{2.432468in}}%
\pgfpathlineto{\pgfqpoint{1.579427in}{2.426962in}}%
\pgfpathlineto{\pgfqpoint{1.579723in}{2.417755in}}%
\pgfpathlineto{\pgfqpoint{1.580414in}{2.430906in}}%
\pgfpathlineto{\pgfqpoint{1.580512in}{2.431257in}}%
\pgfpathlineto{\pgfqpoint{1.580710in}{2.428950in}}%
\pgfpathlineto{\pgfqpoint{1.582881in}{2.377004in}}%
\pgfpathlineto{\pgfqpoint{1.583375in}{2.385555in}}%
\pgfpathlineto{\pgfqpoint{1.583671in}{2.393144in}}%
\pgfpathlineto{\pgfqpoint{1.584263in}{2.382844in}}%
\pgfpathlineto{\pgfqpoint{1.584658in}{2.347066in}}%
\pgfpathlineto{\pgfqpoint{1.585250in}{2.387096in}}%
\pgfpathlineto{\pgfqpoint{1.586928in}{2.463651in}}%
\pgfpathlineto{\pgfqpoint{1.587718in}{2.428380in}}%
\pgfpathlineto{\pgfqpoint{1.588014in}{2.409360in}}%
\pgfpathlineto{\pgfqpoint{1.588507in}{2.447827in}}%
\pgfpathlineto{\pgfqpoint{1.589099in}{2.459912in}}%
\pgfpathlineto{\pgfqpoint{1.589395in}{2.444280in}}%
\pgfpathlineto{\pgfqpoint{1.589593in}{2.435670in}}%
\pgfpathlineto{\pgfqpoint{1.589988in}{2.473830in}}%
\pgfpathlineto{\pgfqpoint{1.591567in}{2.712666in}}%
\pgfpathlineto{\pgfqpoint{1.591962in}{2.693680in}}%
\pgfpathlineto{\pgfqpoint{1.592652in}{2.586219in}}%
\pgfpathlineto{\pgfqpoint{1.593343in}{2.408495in}}%
\pgfpathlineto{\pgfqpoint{1.594034in}{2.472028in}}%
\pgfpathlineto{\pgfqpoint{1.594528in}{2.449388in}}%
\pgfpathlineto{\pgfqpoint{1.595120in}{2.472344in}}%
\pgfpathlineto{\pgfqpoint{1.595416in}{2.477343in}}%
\pgfpathlineto{\pgfqpoint{1.595910in}{2.467782in}}%
\pgfpathlineto{\pgfqpoint{1.596107in}{2.469961in}}%
\pgfpathlineto{\pgfqpoint{1.596304in}{2.471200in}}%
\pgfpathlineto{\pgfqpoint{1.596897in}{2.467740in}}%
\pgfpathlineto{\pgfqpoint{1.598278in}{2.421851in}}%
\pgfpathlineto{\pgfqpoint{1.598870in}{2.430906in}}%
\pgfpathlineto{\pgfqpoint{1.599167in}{2.435423in}}%
\pgfpathlineto{\pgfqpoint{1.599265in}{2.437408in}}%
\pgfpathlineto{\pgfqpoint{1.599759in}{2.427597in}}%
\pgfpathlineto{\pgfqpoint{1.600055in}{2.432115in}}%
\pgfpathlineto{\pgfqpoint{1.600351in}{2.420624in}}%
\pgfpathlineto{\pgfqpoint{1.600844in}{2.397272in}}%
\pgfpathlineto{\pgfqpoint{1.601535in}{2.411384in}}%
\pgfpathlineto{\pgfqpoint{1.601831in}{2.426501in}}%
\pgfpathlineto{\pgfqpoint{1.602424in}{2.401037in}}%
\pgfpathlineto{\pgfqpoint{1.603608in}{2.421671in}}%
\pgfpathlineto{\pgfqpoint{1.603904in}{2.409349in}}%
\pgfpathlineto{\pgfqpoint{1.604792in}{2.392928in}}%
\pgfpathlineto{\pgfqpoint{1.604990in}{2.404151in}}%
\pgfpathlineto{\pgfqpoint{1.605187in}{2.416313in}}%
\pgfpathlineto{\pgfqpoint{1.605878in}{2.393376in}}%
\pgfpathlineto{\pgfqpoint{1.606076in}{2.388053in}}%
\pgfpathlineto{\pgfqpoint{1.606372in}{2.407619in}}%
\pgfpathlineto{\pgfqpoint{1.606569in}{2.420228in}}%
\pgfpathlineto{\pgfqpoint{1.607457in}{2.410391in}}%
\pgfpathlineto{\pgfqpoint{1.607753in}{2.395492in}}%
\pgfpathlineto{\pgfqpoint{1.608346in}{2.411699in}}%
\pgfpathlineto{\pgfqpoint{1.608543in}{2.409013in}}%
\pgfpathlineto{\pgfqpoint{1.609333in}{2.398586in}}%
\pgfpathlineto{\pgfqpoint{1.609629in}{2.404313in}}%
\pgfpathlineto{\pgfqpoint{1.610221in}{2.424410in}}%
\pgfpathlineto{\pgfqpoint{1.610616in}{2.401456in}}%
\pgfpathlineto{\pgfqpoint{1.610813in}{2.391606in}}%
\pgfpathlineto{\pgfqpoint{1.611603in}{2.404102in}}%
\pgfpathlineto{\pgfqpoint{1.611997in}{2.428401in}}%
\pgfpathlineto{\pgfqpoint{1.612491in}{2.399474in}}%
\pgfpathlineto{\pgfqpoint{1.613083in}{2.426046in}}%
\pgfpathlineto{\pgfqpoint{1.613379in}{2.434435in}}%
\pgfpathlineto{\pgfqpoint{1.613774in}{2.417725in}}%
\pgfpathlineto{\pgfqpoint{1.614761in}{2.400665in}}%
\pgfpathlineto{\pgfqpoint{1.615057in}{2.408306in}}%
\pgfpathlineto{\pgfqpoint{1.615156in}{2.410856in}}%
\pgfpathlineto{\pgfqpoint{1.615551in}{2.393919in}}%
\pgfpathlineto{\pgfqpoint{1.615748in}{2.388806in}}%
\pgfpathlineto{\pgfqpoint{1.616538in}{2.395364in}}%
\pgfpathlineto{\pgfqpoint{1.616735in}{2.399625in}}%
\pgfpathlineto{\pgfqpoint{1.617130in}{2.383151in}}%
\pgfpathlineto{\pgfqpoint{1.617623in}{2.358536in}}%
\pgfpathlineto{\pgfqpoint{1.618610in}{2.363970in}}%
\pgfpathlineto{\pgfqpoint{1.619202in}{2.359884in}}%
\pgfpathlineto{\pgfqpoint{1.619499in}{2.364329in}}%
\pgfpathlineto{\pgfqpoint{1.619992in}{2.377929in}}%
\pgfpathlineto{\pgfqpoint{1.620387in}{2.362546in}}%
\pgfpathlineto{\pgfqpoint{1.621078in}{2.347930in}}%
\pgfpathlineto{\pgfqpoint{1.621374in}{2.360088in}}%
\pgfpathlineto{\pgfqpoint{1.621571in}{2.366140in}}%
\pgfpathlineto{\pgfqpoint{1.621966in}{2.352085in}}%
\pgfpathlineto{\pgfqpoint{1.622361in}{2.355344in}}%
\pgfpathlineto{\pgfqpoint{1.622558in}{2.351949in}}%
\pgfpathlineto{\pgfqpoint{1.622953in}{2.367532in}}%
\pgfpathlineto{\pgfqpoint{1.623545in}{2.375113in}}%
\pgfpathlineto{\pgfqpoint{1.623841in}{2.364085in}}%
\pgfpathlineto{\pgfqpoint{1.624039in}{2.358020in}}%
\pgfpathlineto{\pgfqpoint{1.624532in}{2.379033in}}%
\pgfpathlineto{\pgfqpoint{1.626111in}{2.410100in}}%
\pgfpathlineto{\pgfqpoint{1.626506in}{2.421796in}}%
\pgfpathlineto{\pgfqpoint{1.626901in}{2.399814in}}%
\pgfpathlineto{\pgfqpoint{1.627493in}{2.371427in}}%
\pgfpathlineto{\pgfqpoint{1.627987in}{2.390280in}}%
\pgfpathlineto{\pgfqpoint{1.628085in}{2.393309in}}%
\pgfpathlineto{\pgfqpoint{1.628381in}{2.370825in}}%
\pgfpathlineto{\pgfqpoint{1.628678in}{2.349431in}}%
\pgfpathlineto{\pgfqpoint{1.629467in}{2.360145in}}%
\pgfpathlineto{\pgfqpoint{1.629763in}{2.372440in}}%
\pgfpathlineto{\pgfqpoint{1.630257in}{2.348200in}}%
\pgfpathlineto{\pgfqpoint{1.631046in}{2.331538in}}%
\pgfpathlineto{\pgfqpoint{1.631342in}{2.341575in}}%
\pgfpathlineto{\pgfqpoint{1.632922in}{2.393199in}}%
\pgfpathlineto{\pgfqpoint{1.632132in}{2.335633in}}%
\pgfpathlineto{\pgfqpoint{1.633119in}{2.387108in}}%
\pgfpathlineto{\pgfqpoint{1.634007in}{2.332000in}}%
\pgfpathlineto{\pgfqpoint{1.634600in}{2.350719in}}%
\pgfpathlineto{\pgfqpoint{1.634797in}{2.357317in}}%
\pgfpathlineto{\pgfqpoint{1.635290in}{2.340378in}}%
\pgfpathlineto{\pgfqpoint{1.635488in}{2.341785in}}%
\pgfpathlineto{\pgfqpoint{1.636080in}{2.363926in}}%
\pgfpathlineto{\pgfqpoint{1.637955in}{2.626271in}}%
\pgfpathlineto{\pgfqpoint{1.638646in}{2.575510in}}%
\pgfpathlineto{\pgfqpoint{1.639534in}{2.368760in}}%
\pgfpathlineto{\pgfqpoint{1.640521in}{2.410012in}}%
\pgfpathlineto{\pgfqpoint{1.640719in}{2.409087in}}%
\pgfpathlineto{\pgfqpoint{1.640916in}{2.411960in}}%
\pgfpathlineto{\pgfqpoint{1.642693in}{2.459043in}}%
\pgfpathlineto{\pgfqpoint{1.642792in}{2.456731in}}%
\pgfpathlineto{\pgfqpoint{1.643779in}{2.400263in}}%
\pgfpathlineto{\pgfqpoint{1.644272in}{2.430049in}}%
\pgfpathlineto{\pgfqpoint{1.644469in}{2.438872in}}%
\pgfpathlineto{\pgfqpoint{1.644963in}{2.410925in}}%
\pgfpathlineto{\pgfqpoint{1.645160in}{2.405440in}}%
\pgfpathlineto{\pgfqpoint{1.645851in}{2.414100in}}%
\pgfpathlineto{\pgfqpoint{1.646049in}{2.417913in}}%
\pgfpathlineto{\pgfqpoint{1.646345in}{2.393711in}}%
\pgfpathlineto{\pgfqpoint{1.647134in}{2.324796in}}%
\pgfpathlineto{\pgfqpoint{1.647924in}{2.337548in}}%
\pgfpathlineto{\pgfqpoint{1.648417in}{2.315965in}}%
\pgfpathlineto{\pgfqpoint{1.648911in}{2.342194in}}%
\pgfpathlineto{\pgfqpoint{1.649503in}{2.386846in}}%
\pgfpathlineto{\pgfqpoint{1.650391in}{2.367433in}}%
\pgfpathlineto{\pgfqpoint{1.650490in}{2.365498in}}%
\pgfpathlineto{\pgfqpoint{1.650885in}{2.377726in}}%
\pgfpathlineto{\pgfqpoint{1.650984in}{2.378433in}}%
\pgfpathlineto{\pgfqpoint{1.651181in}{2.374514in}}%
\pgfpathlineto{\pgfqpoint{1.655228in}{2.282733in}}%
\pgfpathlineto{\pgfqpoint{1.655622in}{2.295839in}}%
\pgfpathlineto{\pgfqpoint{1.655721in}{2.296036in}}%
\pgfpathlineto{\pgfqpoint{1.655820in}{2.294172in}}%
\pgfpathlineto{\pgfqpoint{1.656807in}{2.267992in}}%
\pgfpathlineto{\pgfqpoint{1.657399in}{2.280677in}}%
\pgfpathlineto{\pgfqpoint{1.657695in}{2.272993in}}%
\pgfpathlineto{\pgfqpoint{1.658485in}{2.250983in}}%
\pgfpathlineto{\pgfqpoint{1.658879in}{2.263199in}}%
\pgfpathlineto{\pgfqpoint{1.659373in}{2.275970in}}%
\pgfpathlineto{\pgfqpoint{1.659768in}{2.260190in}}%
\pgfpathlineto{\pgfqpoint{1.660755in}{2.242989in}}%
\pgfpathlineto{\pgfqpoint{1.661051in}{2.247492in}}%
\pgfpathlineto{\pgfqpoint{1.661150in}{2.247785in}}%
\pgfpathlineto{\pgfqpoint{1.661248in}{2.245768in}}%
\pgfpathlineto{\pgfqpoint{1.661742in}{2.219784in}}%
\pgfpathlineto{\pgfqpoint{1.662433in}{2.239088in}}%
\pgfpathlineto{\pgfqpoint{1.664900in}{2.192390in}}%
\pgfpathlineto{\pgfqpoint{1.665492in}{2.204504in}}%
\pgfpathlineto{\pgfqpoint{1.665887in}{2.222601in}}%
\pgfpathlineto{\pgfqpoint{1.666381in}{2.193128in}}%
\pgfpathlineto{\pgfqpoint{1.666479in}{2.192068in}}%
\pgfpathlineto{\pgfqpoint{1.666677in}{2.198038in}}%
\pgfpathlineto{\pgfqpoint{1.667762in}{2.216054in}}%
\pgfpathlineto{\pgfqpoint{1.667960in}{2.210959in}}%
\pgfpathlineto{\pgfqpoint{1.668256in}{2.204542in}}%
\pgfpathlineto{\pgfqpoint{1.668749in}{2.220947in}}%
\pgfpathlineto{\pgfqpoint{1.670723in}{2.284277in}}%
\pgfpathlineto{\pgfqpoint{1.671217in}{2.283555in}}%
\pgfpathlineto{\pgfqpoint{1.671513in}{2.289235in}}%
\pgfpathlineto{\pgfqpoint{1.672697in}{2.368548in}}%
\pgfpathlineto{\pgfqpoint{1.673191in}{2.327731in}}%
\pgfpathlineto{\pgfqpoint{1.673290in}{2.325388in}}%
\pgfpathlineto{\pgfqpoint{1.673586in}{2.341630in}}%
\pgfpathlineto{\pgfqpoint{1.673783in}{2.348832in}}%
\pgfpathlineto{\pgfqpoint{1.674375in}{2.332104in}}%
\pgfpathlineto{\pgfqpoint{1.674869in}{2.316844in}}%
\pgfpathlineto{\pgfqpoint{1.675362in}{2.330963in}}%
\pgfpathlineto{\pgfqpoint{1.675560in}{2.335221in}}%
\pgfpathlineto{\pgfqpoint{1.676053in}{2.321638in}}%
\pgfpathlineto{\pgfqpoint{1.678027in}{2.281022in}}%
\pgfpathlineto{\pgfqpoint{1.678323in}{2.290249in}}%
\pgfpathlineto{\pgfqpoint{1.679014in}{2.329328in}}%
\pgfpathlineto{\pgfqpoint{1.679409in}{2.299805in}}%
\pgfpathlineto{\pgfqpoint{1.680297in}{2.246782in}}%
\pgfpathlineto{\pgfqpoint{1.680791in}{2.264302in}}%
\pgfpathlineto{\pgfqpoint{1.680889in}{2.264273in}}%
\pgfpathlineto{\pgfqpoint{1.681185in}{2.254157in}}%
\pgfpathlineto{\pgfqpoint{1.681679in}{2.271586in}}%
\pgfpathlineto{\pgfqpoint{1.681975in}{2.276200in}}%
\pgfpathlineto{\pgfqpoint{1.683258in}{2.479764in}}%
\pgfpathlineto{\pgfqpoint{1.683850in}{2.542676in}}%
\pgfpathlineto{\pgfqpoint{1.684442in}{2.511168in}}%
\pgfpathlineto{\pgfqpoint{1.685232in}{2.387863in}}%
\pgfpathlineto{\pgfqpoint{1.685726in}{2.265454in}}%
\pgfpathlineto{\pgfqpoint{1.686515in}{2.323519in}}%
\pgfpathlineto{\pgfqpoint{1.687009in}{2.349650in}}%
\pgfpathlineto{\pgfqpoint{1.687798in}{2.342450in}}%
\pgfpathlineto{\pgfqpoint{1.687996in}{2.338426in}}%
\pgfpathlineto{\pgfqpoint{1.688292in}{2.362367in}}%
\pgfpathlineto{\pgfqpoint{1.688687in}{2.399494in}}%
\pgfpathlineto{\pgfqpoint{1.689377in}{2.361463in}}%
\pgfpathlineto{\pgfqpoint{1.689772in}{2.340035in}}%
\pgfpathlineto{\pgfqpoint{1.690364in}{2.364374in}}%
\pgfpathlineto{\pgfqpoint{1.692042in}{2.404193in}}%
\pgfpathlineto{\pgfqpoint{1.690957in}{2.359810in}}%
\pgfpathlineto{\pgfqpoint{1.692437in}{2.392442in}}%
\pgfpathlineto{\pgfqpoint{1.692931in}{2.363175in}}%
\pgfpathlineto{\pgfqpoint{1.693424in}{2.391438in}}%
\pgfpathlineto{\pgfqpoint{1.693720in}{2.409428in}}%
\pgfpathlineto{\pgfqpoint{1.694214in}{2.386821in}}%
\pgfpathlineto{\pgfqpoint{1.694510in}{2.394194in}}%
\pgfpathlineto{\pgfqpoint{1.694806in}{2.396290in}}%
\pgfpathlineto{\pgfqpoint{1.695299in}{2.417044in}}%
\pgfpathlineto{\pgfqpoint{1.695793in}{2.391651in}}%
\pgfpathlineto{\pgfqpoint{1.695990in}{2.386752in}}%
\pgfpathlineto{\pgfqpoint{1.696286in}{2.404891in}}%
\pgfpathlineto{\pgfqpoint{1.697273in}{2.430652in}}%
\pgfpathlineto{\pgfqpoint{1.697471in}{2.421954in}}%
\pgfpathlineto{\pgfqpoint{1.697668in}{2.412058in}}%
\pgfpathlineto{\pgfqpoint{1.698260in}{2.439556in}}%
\pgfpathlineto{\pgfqpoint{1.698458in}{2.441219in}}%
\pgfpathlineto{\pgfqpoint{1.698951in}{2.437146in}}%
\pgfpathlineto{\pgfqpoint{1.699247in}{2.439369in}}%
\pgfpathlineto{\pgfqpoint{1.699543in}{2.432410in}}%
\pgfpathlineto{\pgfqpoint{1.699741in}{2.437208in}}%
\pgfpathlineto{\pgfqpoint{1.700136in}{2.466259in}}%
\pgfpathlineto{\pgfqpoint{1.700827in}{2.445606in}}%
\pgfpathlineto{\pgfqpoint{1.701024in}{2.438632in}}%
\pgfpathlineto{\pgfqpoint{1.701715in}{2.453880in}}%
\pgfpathlineto{\pgfqpoint{1.701912in}{2.464144in}}%
\pgfpathlineto{\pgfqpoint{1.702406in}{2.437998in}}%
\pgfpathlineto{\pgfqpoint{1.702800in}{2.453223in}}%
\pgfpathlineto{\pgfqpoint{1.703689in}{2.473384in}}%
\pgfpathlineto{\pgfqpoint{1.703886in}{2.464661in}}%
\pgfpathlineto{\pgfqpoint{1.704281in}{2.440351in}}%
\pgfpathlineto{\pgfqpoint{1.704972in}{2.461410in}}%
\pgfpathlineto{\pgfqpoint{1.706847in}{2.493792in}}%
\pgfpathlineto{\pgfqpoint{1.707341in}{2.467395in}}%
\pgfpathlineto{\pgfqpoint{1.708130in}{2.480778in}}%
\pgfpathlineto{\pgfqpoint{1.708426in}{2.493172in}}%
\pgfpathlineto{\pgfqpoint{1.709216in}{2.482879in}}%
\pgfpathlineto{\pgfqpoint{1.709413in}{2.484537in}}%
\pgfpathlineto{\pgfqpoint{1.710104in}{2.502348in}}%
\pgfpathlineto{\pgfqpoint{1.710499in}{2.490162in}}%
\pgfpathlineto{\pgfqpoint{1.710894in}{2.468527in}}%
\pgfpathlineto{\pgfqpoint{1.711387in}{2.501123in}}%
\pgfpathlineto{\pgfqpoint{1.713559in}{2.530790in}}%
\pgfpathlineto{\pgfqpoint{1.713855in}{2.535896in}}%
\pgfpathlineto{\pgfqpoint{1.714250in}{2.524566in}}%
\pgfpathlineto{\pgfqpoint{1.714348in}{2.524125in}}%
\pgfpathlineto{\pgfqpoint{1.714447in}{2.528640in}}%
\pgfpathlineto{\pgfqpoint{1.714940in}{2.561226in}}%
\pgfpathlineto{\pgfqpoint{1.715730in}{2.546971in}}%
\pgfpathlineto{\pgfqpoint{1.715927in}{2.549153in}}%
\pgfpathlineto{\pgfqpoint{1.718198in}{2.594656in}}%
\pgfpathlineto{\pgfqpoint{1.718296in}{2.593741in}}%
\pgfpathlineto{\pgfqpoint{1.718592in}{2.593924in}}%
\pgfpathlineto{\pgfqpoint{1.718888in}{2.586426in}}%
\pgfpathlineto{\pgfqpoint{1.720764in}{2.514502in}}%
\pgfpathlineto{\pgfqpoint{1.721060in}{2.514963in}}%
\pgfpathlineto{\pgfqpoint{1.721652in}{2.528574in}}%
\pgfpathlineto{\pgfqpoint{1.722047in}{2.515715in}}%
\pgfpathlineto{\pgfqpoint{1.724119in}{2.395507in}}%
\pgfpathlineto{\pgfqpoint{1.724514in}{2.407440in}}%
\pgfpathlineto{\pgfqpoint{1.725106in}{2.436281in}}%
\pgfpathlineto{\pgfqpoint{1.725797in}{2.420703in}}%
\pgfpathlineto{\pgfqpoint{1.726192in}{2.410623in}}%
\pgfpathlineto{\pgfqpoint{1.726686in}{2.425031in}}%
\pgfpathlineto{\pgfqpoint{1.726784in}{2.424958in}}%
\pgfpathlineto{\pgfqpoint{1.727179in}{2.419092in}}%
\pgfpathlineto{\pgfqpoint{1.727673in}{2.426397in}}%
\pgfpathlineto{\pgfqpoint{1.728561in}{2.463273in}}%
\pgfpathlineto{\pgfqpoint{1.729943in}{2.665381in}}%
\pgfpathlineto{\pgfqpoint{1.730930in}{2.606633in}}%
\pgfpathlineto{\pgfqpoint{1.731818in}{2.371605in}}%
\pgfpathlineto{\pgfqpoint{1.733200in}{2.440990in}}%
\pgfpathlineto{\pgfqpoint{1.734285in}{2.485388in}}%
\pgfpathlineto{\pgfqpoint{1.734779in}{2.529549in}}%
\pgfpathlineto{\pgfqpoint{1.735371in}{2.492335in}}%
\pgfpathlineto{\pgfqpoint{1.735766in}{2.464427in}}%
\pgfpathlineto{\pgfqpoint{1.736457in}{2.486871in}}%
\pgfpathlineto{\pgfqpoint{1.737937in}{2.537113in}}%
\pgfpathlineto{\pgfqpoint{1.737148in}{2.481989in}}%
\pgfpathlineto{\pgfqpoint{1.738530in}{2.515530in}}%
\pgfpathlineto{\pgfqpoint{1.738826in}{2.505784in}}%
\pgfpathlineto{\pgfqpoint{1.739319in}{2.518292in}}%
\pgfpathlineto{\pgfqpoint{1.739516in}{2.516994in}}%
\pgfpathlineto{\pgfqpoint{1.740503in}{2.534759in}}%
\pgfpathlineto{\pgfqpoint{1.741194in}{2.554556in}}%
\pgfpathlineto{\pgfqpoint{1.741589in}{2.538628in}}%
\pgfpathlineto{\pgfqpoint{1.741885in}{2.528588in}}%
\pgfpathlineto{\pgfqpoint{1.742477in}{2.548925in}}%
\pgfpathlineto{\pgfqpoint{1.743168in}{2.566527in}}%
\pgfpathlineto{\pgfqpoint{1.743464in}{2.552548in}}%
\pgfpathlineto{\pgfqpoint{1.743662in}{2.544712in}}%
\pgfpathlineto{\pgfqpoint{1.744155in}{2.573008in}}%
\pgfpathlineto{\pgfqpoint{1.744353in}{2.576203in}}%
\pgfpathlineto{\pgfqpoint{1.744945in}{2.569024in}}%
\pgfpathlineto{\pgfqpoint{1.745340in}{2.548101in}}%
\pgfpathlineto{\pgfqpoint{1.746129in}{2.563784in}}%
\pgfpathlineto{\pgfqpoint{1.746425in}{2.567996in}}%
\pgfpathlineto{\pgfqpoint{1.746722in}{2.555224in}}%
\pgfpathlineto{\pgfqpoint{1.748597in}{2.527404in}}%
\pgfpathlineto{\pgfqpoint{1.747511in}{2.559830in}}%
\pgfpathlineto{\pgfqpoint{1.748696in}{2.529850in}}%
\pgfpathlineto{\pgfqpoint{1.749090in}{2.549652in}}%
\pgfpathlineto{\pgfqpoint{1.749979in}{2.540641in}}%
\pgfpathlineto{\pgfqpoint{1.750373in}{2.517908in}}%
\pgfpathlineto{\pgfqpoint{1.751163in}{2.525409in}}%
\pgfpathlineto{\pgfqpoint{1.751558in}{2.531620in}}%
\pgfpathlineto{\pgfqpoint{1.751953in}{2.521817in}}%
\pgfpathlineto{\pgfqpoint{1.752249in}{2.527206in}}%
\pgfpathlineto{\pgfqpoint{1.752643in}{2.533082in}}%
\pgfpathlineto{\pgfqpoint{1.752940in}{2.523929in}}%
\pgfpathlineto{\pgfqpoint{1.754815in}{2.465892in}}%
\pgfpathlineto{\pgfqpoint{1.755012in}{2.461720in}}%
\pgfpathlineto{\pgfqpoint{1.755703in}{2.471593in}}%
\pgfpathlineto{\pgfqpoint{1.755901in}{2.473408in}}%
\pgfpathlineto{\pgfqpoint{1.756098in}{2.469331in}}%
\pgfpathlineto{\pgfqpoint{1.756888in}{2.437591in}}%
\pgfpathlineto{\pgfqpoint{1.757677in}{2.442571in}}%
\pgfpathlineto{\pgfqpoint{1.757973in}{2.445718in}}%
\pgfpathlineto{\pgfqpoint{1.758269in}{2.437750in}}%
\pgfpathlineto{\pgfqpoint{1.758467in}{2.433897in}}%
\pgfpathlineto{\pgfqpoint{1.758960in}{2.446106in}}%
\pgfpathlineto{\pgfqpoint{1.759059in}{2.447063in}}%
\pgfpathlineto{\pgfqpoint{1.759355in}{2.441929in}}%
\pgfpathlineto{\pgfqpoint{1.759947in}{2.425938in}}%
\pgfpathlineto{\pgfqpoint{1.760441in}{2.440577in}}%
\pgfpathlineto{\pgfqpoint{1.760539in}{2.440693in}}%
\pgfpathlineto{\pgfqpoint{1.760638in}{2.439525in}}%
\pgfpathlineto{\pgfqpoint{1.761428in}{2.423872in}}%
\pgfpathlineto{\pgfqpoint{1.761724in}{2.435660in}}%
\pgfpathlineto{\pgfqpoint{1.762415in}{2.470330in}}%
\pgfpathlineto{\pgfqpoint{1.762908in}{2.448970in}}%
\pgfpathlineto{\pgfqpoint{1.763106in}{2.443445in}}%
\pgfpathlineto{\pgfqpoint{1.763599in}{2.456999in}}%
\pgfpathlineto{\pgfqpoint{1.764685in}{2.497465in}}%
\pgfpathlineto{\pgfqpoint{1.765080in}{2.486501in}}%
\pgfpathlineto{\pgfqpoint{1.765376in}{2.487079in}}%
\pgfpathlineto{\pgfqpoint{1.766067in}{2.475552in}}%
\pgfpathlineto{\pgfqpoint{1.766560in}{2.440582in}}%
\pgfpathlineto{\pgfqpoint{1.767350in}{2.453220in}}%
\pgfpathlineto{\pgfqpoint{1.767646in}{2.461814in}}%
\pgfpathlineto{\pgfqpoint{1.768534in}{2.457838in}}%
\pgfpathlineto{\pgfqpoint{1.769916in}{2.410502in}}%
\pgfpathlineto{\pgfqpoint{1.770409in}{2.438879in}}%
\pgfpathlineto{\pgfqpoint{1.770804in}{2.451573in}}%
\pgfpathlineto{\pgfqpoint{1.771100in}{2.466174in}}%
\pgfpathlineto{\pgfqpoint{1.771594in}{2.438799in}}%
\pgfpathlineto{\pgfqpoint{1.772975in}{2.398879in}}%
\pgfpathlineto{\pgfqpoint{1.773272in}{2.404464in}}%
\pgfpathlineto{\pgfqpoint{1.774456in}{2.457520in}}%
\pgfpathlineto{\pgfqpoint{1.775640in}{2.672882in}}%
\pgfpathlineto{\pgfqpoint{1.776726in}{2.636144in}}%
\pgfpathlineto{\pgfqpoint{1.777417in}{2.459774in}}%
\pgfpathlineto{\pgfqpoint{1.777713in}{2.397894in}}%
\pgfpathlineto{\pgfqpoint{1.778503in}{2.439608in}}%
\pgfpathlineto{\pgfqpoint{1.778601in}{2.439521in}}%
\pgfpathlineto{\pgfqpoint{1.778897in}{2.447741in}}%
\pgfpathlineto{\pgfqpoint{1.779193in}{2.433939in}}%
\pgfpathlineto{\pgfqpoint{1.779588in}{2.414630in}}%
\pgfpathlineto{\pgfqpoint{1.780082in}{2.436256in}}%
\pgfpathlineto{\pgfqpoint{1.780674in}{2.475519in}}%
\pgfpathlineto{\pgfqpoint{1.781167in}{2.446729in}}%
\pgfpathlineto{\pgfqpoint{1.781957in}{2.410287in}}%
\pgfpathlineto{\pgfqpoint{1.782747in}{2.415427in}}%
\pgfpathlineto{\pgfqpoint{1.783931in}{2.472862in}}%
\pgfpathlineto{\pgfqpoint{1.784523in}{2.439808in}}%
\pgfpathlineto{\pgfqpoint{1.785115in}{2.432129in}}%
\pgfpathlineto{\pgfqpoint{1.785510in}{2.444267in}}%
\pgfpathlineto{\pgfqpoint{1.785609in}{2.444506in}}%
\pgfpathlineto{\pgfqpoint{1.786004in}{2.431784in}}%
\pgfpathlineto{\pgfqpoint{1.786300in}{2.449510in}}%
\pgfpathlineto{\pgfqpoint{1.787287in}{2.502624in}}%
\pgfpathlineto{\pgfqpoint{1.787682in}{2.473322in}}%
\pgfpathlineto{\pgfqpoint{1.787780in}{2.469967in}}%
\pgfpathlineto{\pgfqpoint{1.788076in}{2.487847in}}%
\pgfpathlineto{\pgfqpoint{1.789656in}{2.523804in}}%
\pgfpathlineto{\pgfqpoint{1.789952in}{2.527399in}}%
\pgfpathlineto{\pgfqpoint{1.790544in}{2.546755in}}%
\pgfpathlineto{\pgfqpoint{1.790939in}{2.527468in}}%
\pgfpathlineto{\pgfqpoint{1.791235in}{2.519499in}}%
\pgfpathlineto{\pgfqpoint{1.791728in}{2.536010in}}%
\pgfpathlineto{\pgfqpoint{1.792024in}{2.544058in}}%
\pgfpathlineto{\pgfqpoint{1.792419in}{2.528183in}}%
\pgfpathlineto{\pgfqpoint{1.796466in}{2.356830in}}%
\pgfpathlineto{\pgfqpoint{1.796663in}{2.359523in}}%
\pgfpathlineto{\pgfqpoint{1.796762in}{2.361224in}}%
\pgfpathlineto{\pgfqpoint{1.796959in}{2.354031in}}%
\pgfpathlineto{\pgfqpoint{1.798538in}{2.310032in}}%
\pgfpathlineto{\pgfqpoint{1.798835in}{2.312887in}}%
\pgfpathlineto{\pgfqpoint{1.798933in}{2.309518in}}%
\pgfpathlineto{\pgfqpoint{1.801302in}{2.173244in}}%
\pgfpathlineto{\pgfqpoint{1.801499in}{2.176409in}}%
\pgfpathlineto{\pgfqpoint{1.804954in}{2.301458in}}%
\pgfpathlineto{\pgfqpoint{1.802289in}{2.174021in}}%
\pgfpathlineto{\pgfqpoint{1.805349in}{2.289078in}}%
\pgfpathlineto{\pgfqpoint{1.806336in}{2.322111in}}%
\pgfpathlineto{\pgfqpoint{1.811567in}{2.531122in}}%
\pgfpathlineto{\pgfqpoint{1.811665in}{2.531035in}}%
\pgfpathlineto{\pgfqpoint{1.812652in}{2.497767in}}%
\pgfpathlineto{\pgfqpoint{1.813047in}{2.515171in}}%
\pgfpathlineto{\pgfqpoint{1.814429in}{2.548241in}}%
\pgfpathlineto{\pgfqpoint{1.814626in}{2.545198in}}%
\pgfpathlineto{\pgfqpoint{1.815021in}{2.538093in}}%
\pgfpathlineto{\pgfqpoint{1.815416in}{2.549635in}}%
\pgfpathlineto{\pgfqpoint{1.816995in}{2.601212in}}%
\pgfpathlineto{\pgfqpoint{1.815909in}{2.543412in}}%
\pgfpathlineto{\pgfqpoint{1.817489in}{2.573432in}}%
\pgfpathlineto{\pgfqpoint{1.818969in}{2.523481in}}%
\pgfpathlineto{\pgfqpoint{1.819561in}{2.547673in}}%
\pgfpathlineto{\pgfqpoint{1.820844in}{2.708450in}}%
\pgfpathlineto{\pgfqpoint{1.821338in}{2.800092in}}%
\pgfpathlineto{\pgfqpoint{1.822128in}{2.777538in}}%
\pgfpathlineto{\pgfqpoint{1.823509in}{2.514167in}}%
\pgfpathlineto{\pgfqpoint{1.824792in}{2.593279in}}%
\pgfpathlineto{\pgfqpoint{1.825582in}{2.585369in}}%
\pgfpathlineto{\pgfqpoint{1.825088in}{2.594003in}}%
\pgfpathlineto{\pgfqpoint{1.825681in}{2.588826in}}%
\pgfpathlineto{\pgfqpoint{1.826569in}{2.657054in}}%
\pgfpathlineto{\pgfqpoint{1.827260in}{2.622927in}}%
\pgfpathlineto{\pgfqpoint{1.828247in}{2.607099in}}%
\pgfpathlineto{\pgfqpoint{1.828444in}{2.609370in}}%
\pgfpathlineto{\pgfqpoint{1.829530in}{2.661587in}}%
\pgfpathlineto{\pgfqpoint{1.829925in}{2.680316in}}%
\pgfpathlineto{\pgfqpoint{1.830320in}{2.654799in}}%
\pgfpathlineto{\pgfqpoint{1.830517in}{2.647368in}}%
\pgfpathlineto{\pgfqpoint{1.831208in}{2.656803in}}%
\pgfpathlineto{\pgfqpoint{1.831405in}{2.655263in}}%
\pgfpathlineto{\pgfqpoint{1.832096in}{2.648108in}}%
\pgfpathlineto{\pgfqpoint{1.832392in}{2.655767in}}%
\pgfpathlineto{\pgfqpoint{1.832886in}{2.690848in}}%
\pgfpathlineto{\pgfqpoint{1.833478in}{2.665252in}}%
\pgfpathlineto{\pgfqpoint{1.833675in}{2.659842in}}%
\pgfpathlineto{\pgfqpoint{1.834267in}{2.673594in}}%
\pgfpathlineto{\pgfqpoint{1.834662in}{2.687582in}}%
\pgfpathlineto{\pgfqpoint{1.834958in}{2.672172in}}%
\pgfpathlineto{\pgfqpoint{1.835254in}{2.653530in}}%
\pgfpathlineto{\pgfqpoint{1.835748in}{2.702741in}}%
\pgfpathlineto{\pgfqpoint{1.835945in}{2.710308in}}%
\pgfpathlineto{\pgfqpoint{1.836439in}{2.694147in}}%
\pgfpathlineto{\pgfqpoint{1.836735in}{2.698723in}}%
\pgfpathlineto{\pgfqpoint{1.836834in}{2.700294in}}%
\pgfpathlineto{\pgfqpoint{1.837130in}{2.691758in}}%
\pgfpathlineto{\pgfqpoint{1.837327in}{2.687415in}}%
\pgfpathlineto{\pgfqpoint{1.837821in}{2.705125in}}%
\pgfpathlineto{\pgfqpoint{1.838808in}{2.698437in}}%
\pgfpathlineto{\pgfqpoint{1.838413in}{2.707660in}}%
\pgfpathlineto{\pgfqpoint{1.838906in}{2.700535in}}%
\pgfpathlineto{\pgfqpoint{1.839499in}{2.723505in}}%
\pgfpathlineto{\pgfqpoint{1.840091in}{2.708314in}}%
\pgfpathlineto{\pgfqpoint{1.840979in}{2.702368in}}%
\pgfpathlineto{\pgfqpoint{1.841275in}{2.705944in}}%
\pgfpathlineto{\pgfqpoint{1.841374in}{2.706509in}}%
\pgfpathlineto{\pgfqpoint{1.841473in}{2.704794in}}%
\pgfpathlineto{\pgfqpoint{1.841867in}{2.694521in}}%
\pgfpathlineto{\pgfqpoint{1.842361in}{2.709392in}}%
\pgfpathlineto{\pgfqpoint{1.842756in}{2.717593in}}%
\pgfpathlineto{\pgfqpoint{1.843052in}{2.704867in}}%
\pgfpathlineto{\pgfqpoint{1.843545in}{2.684826in}}%
\pgfpathlineto{\pgfqpoint{1.844137in}{2.703522in}}%
\pgfpathlineto{\pgfqpoint{1.844236in}{2.704577in}}%
\pgfpathlineto{\pgfqpoint{1.844433in}{2.698820in}}%
\pgfpathlineto{\pgfqpoint{1.844927in}{2.675641in}}%
\pgfpathlineto{\pgfqpoint{1.845717in}{2.684084in}}%
\pgfpathlineto{\pgfqpoint{1.846013in}{2.689762in}}%
\pgfpathlineto{\pgfqpoint{1.846309in}{2.677422in}}%
\pgfpathlineto{\pgfqpoint{1.846802in}{2.657155in}}%
\pgfpathlineto{\pgfqpoint{1.847592in}{2.658627in}}%
\pgfpathlineto{\pgfqpoint{1.847888in}{2.662574in}}%
\pgfpathlineto{\pgfqpoint{1.848283in}{2.655512in}}%
\pgfpathlineto{\pgfqpoint{1.848579in}{2.649176in}}%
\pgfpathlineto{\pgfqpoint{1.849171in}{2.659992in}}%
\pgfpathlineto{\pgfqpoint{1.849270in}{2.660355in}}%
\pgfpathlineto{\pgfqpoint{1.849368in}{2.658923in}}%
\pgfpathlineto{\pgfqpoint{1.849961in}{2.630945in}}%
\pgfpathlineto{\pgfqpoint{1.850849in}{2.643532in}}%
\pgfpathlineto{\pgfqpoint{1.851145in}{2.641050in}}%
\pgfpathlineto{\pgfqpoint{1.852231in}{2.634946in}}%
\pgfpathlineto{\pgfqpoint{1.851737in}{2.647480in}}%
\pgfpathlineto{\pgfqpoint{1.852329in}{2.639019in}}%
\pgfpathlineto{\pgfqpoint{1.852724in}{2.660287in}}%
\pgfpathlineto{\pgfqpoint{1.853218in}{2.630467in}}%
\pgfpathlineto{\pgfqpoint{1.853415in}{2.638154in}}%
\pgfpathlineto{\pgfqpoint{1.854303in}{2.670294in}}%
\pgfpathlineto{\pgfqpoint{1.855093in}{2.667788in}}%
\pgfpathlineto{\pgfqpoint{1.855784in}{2.693257in}}%
\pgfpathlineto{\pgfqpoint{1.856179in}{2.673676in}}%
\pgfpathlineto{\pgfqpoint{1.856968in}{2.666541in}}%
\pgfpathlineto{\pgfqpoint{1.857264in}{2.672869in}}%
\pgfpathlineto{\pgfqpoint{1.857462in}{2.677189in}}%
\pgfpathlineto{\pgfqpoint{1.857857in}{2.659275in}}%
\pgfpathlineto{\pgfqpoint{1.858251in}{2.618329in}}%
\pgfpathlineto{\pgfqpoint{1.859140in}{2.635126in}}%
\pgfpathlineto{\pgfqpoint{1.859238in}{2.635889in}}%
\pgfpathlineto{\pgfqpoint{1.859337in}{2.632614in}}%
\pgfpathlineto{\pgfqpoint{1.861015in}{2.592785in}}%
\pgfpathlineto{\pgfqpoint{1.861114in}{2.593082in}}%
\pgfpathlineto{\pgfqpoint{1.861706in}{2.589148in}}%
\pgfpathlineto{\pgfqpoint{1.862199in}{2.613624in}}%
\pgfpathlineto{\pgfqpoint{1.862594in}{2.632171in}}%
\pgfpathlineto{\pgfqpoint{1.863088in}{2.601263in}}%
\pgfpathlineto{\pgfqpoint{1.863482in}{2.579406in}}%
\pgfpathlineto{\pgfqpoint{1.864371in}{2.579587in}}%
\pgfpathlineto{\pgfqpoint{1.864963in}{2.575843in}}%
\pgfpathlineto{\pgfqpoint{1.865160in}{2.581984in}}%
\pgfpathlineto{\pgfqpoint{1.866641in}{2.766474in}}%
\pgfpathlineto{\pgfqpoint{1.867134in}{2.865576in}}%
\pgfpathlineto{\pgfqpoint{1.867924in}{2.824762in}}%
\pgfpathlineto{\pgfqpoint{1.868713in}{2.693202in}}%
\pgfpathlineto{\pgfqpoint{1.869207in}{2.562849in}}%
\pgfpathlineto{\pgfqpoint{1.869997in}{2.606973in}}%
\pgfpathlineto{\pgfqpoint{1.870194in}{2.603282in}}%
\pgfpathlineto{\pgfqpoint{1.870589in}{2.621649in}}%
\pgfpathlineto{\pgfqpoint{1.870687in}{2.624991in}}%
\pgfpathlineto{\pgfqpoint{1.871181in}{2.606036in}}%
\pgfpathlineto{\pgfqpoint{1.871674in}{2.620852in}}%
\pgfpathlineto{\pgfqpoint{1.871970in}{2.633454in}}%
\pgfpathlineto{\pgfqpoint{1.872267in}{2.661205in}}%
\pgfpathlineto{\pgfqpoint{1.872859in}{2.610091in}}%
\pgfpathlineto{\pgfqpoint{1.873352in}{2.578288in}}%
\pgfpathlineto{\pgfqpoint{1.874833in}{2.515932in}}%
\pgfpathlineto{\pgfqpoint{1.873846in}{2.581746in}}%
\pgfpathlineto{\pgfqpoint{1.875228in}{2.536577in}}%
\pgfpathlineto{\pgfqpoint{1.875326in}{2.540040in}}%
\pgfpathlineto{\pgfqpoint{1.875721in}{2.522610in}}%
\pgfpathlineto{\pgfqpoint{1.878189in}{2.399826in}}%
\pgfpathlineto{\pgfqpoint{1.878682in}{2.407055in}}%
\pgfpathlineto{\pgfqpoint{1.879274in}{2.385438in}}%
\pgfpathlineto{\pgfqpoint{1.881248in}{2.279032in}}%
\pgfpathlineto{\pgfqpoint{1.883025in}{2.309907in}}%
\pgfpathlineto{\pgfqpoint{1.884900in}{2.385061in}}%
\pgfpathlineto{\pgfqpoint{1.885196in}{2.392744in}}%
\pgfpathlineto{\pgfqpoint{1.885492in}{2.408224in}}%
\pgfpathlineto{\pgfqpoint{1.885986in}{2.385244in}}%
\pgfpathlineto{\pgfqpoint{1.886282in}{2.395494in}}%
\pgfpathlineto{\pgfqpoint{1.886381in}{2.395554in}}%
\pgfpathlineto{\pgfqpoint{1.886479in}{2.394371in}}%
\pgfpathlineto{\pgfqpoint{1.886578in}{2.393400in}}%
\pgfpathlineto{\pgfqpoint{1.886775in}{2.399021in}}%
\pgfpathlineto{\pgfqpoint{1.887071in}{2.409990in}}%
\pgfpathlineto{\pgfqpoint{1.887466in}{2.398349in}}%
\pgfpathlineto{\pgfqpoint{1.887861in}{2.402231in}}%
\pgfpathlineto{\pgfqpoint{1.888058in}{2.404557in}}%
\pgfpathlineto{\pgfqpoint{1.888552in}{2.441756in}}%
\pgfpathlineto{\pgfqpoint{1.889638in}{2.433480in}}%
\pgfpathlineto{\pgfqpoint{1.890723in}{2.423823in}}%
\pgfpathlineto{\pgfqpoint{1.891118in}{2.427175in}}%
\pgfpathlineto{\pgfqpoint{1.891217in}{2.426414in}}%
\pgfpathlineto{\pgfqpoint{1.891315in}{2.428367in}}%
\pgfpathlineto{\pgfqpoint{1.891710in}{2.450114in}}%
\pgfpathlineto{\pgfqpoint{1.892401in}{2.437309in}}%
\pgfpathlineto{\pgfqpoint{1.892697in}{2.417066in}}%
\pgfpathlineto{\pgfqpoint{1.893586in}{2.427900in}}%
\pgfpathlineto{\pgfqpoint{1.894079in}{2.413635in}}%
\pgfpathlineto{\pgfqpoint{1.894671in}{2.424808in}}%
\pgfpathlineto{\pgfqpoint{1.895263in}{2.438724in}}%
\pgfpathlineto{\pgfqpoint{1.895658in}{2.427483in}}%
\pgfpathlineto{\pgfqpoint{1.895954in}{2.411162in}}%
\pgfpathlineto{\pgfqpoint{1.896547in}{2.433967in}}%
\pgfpathlineto{\pgfqpoint{1.896645in}{2.432610in}}%
\pgfpathlineto{\pgfqpoint{1.897435in}{2.416626in}}%
\pgfpathlineto{\pgfqpoint{1.897731in}{2.428874in}}%
\pgfpathlineto{\pgfqpoint{1.899211in}{2.456089in}}%
\pgfpathlineto{\pgfqpoint{1.899507in}{2.447854in}}%
\pgfpathlineto{\pgfqpoint{1.899804in}{2.462690in}}%
\pgfpathlineto{\pgfqpoint{1.901481in}{2.519258in}}%
\pgfpathlineto{\pgfqpoint{1.901580in}{2.519836in}}%
\pgfpathlineto{\pgfqpoint{1.901679in}{2.517019in}}%
\pgfpathlineto{\pgfqpoint{1.904146in}{2.411966in}}%
\pgfpathlineto{\pgfqpoint{1.904541in}{2.421038in}}%
\pgfpathlineto{\pgfqpoint{1.905035in}{2.432193in}}%
\pgfpathlineto{\pgfqpoint{1.905232in}{2.422043in}}%
\pgfpathlineto{\pgfqpoint{1.906022in}{2.364221in}}%
\pgfpathlineto{\pgfqpoint{1.906811in}{2.380846in}}%
\pgfpathlineto{\pgfqpoint{1.907305in}{2.366392in}}%
\pgfpathlineto{\pgfqpoint{1.907601in}{2.380140in}}%
\pgfpathlineto{\pgfqpoint{1.908292in}{2.421885in}}%
\pgfpathlineto{\pgfqpoint{1.908588in}{2.389968in}}%
\pgfpathlineto{\pgfqpoint{1.909279in}{2.351698in}}%
\pgfpathlineto{\pgfqpoint{1.909871in}{2.355509in}}%
\pgfpathlineto{\pgfqpoint{1.910167in}{2.350326in}}%
\pgfpathlineto{\pgfqpoint{1.910660in}{2.361268in}}%
\pgfpathlineto{\pgfqpoint{1.910759in}{2.360246in}}%
\pgfpathlineto{\pgfqpoint{1.910858in}{2.358975in}}%
\pgfpathlineto{\pgfqpoint{1.911154in}{2.367647in}}%
\pgfpathlineto{\pgfqpoint{1.912931in}{2.607904in}}%
\pgfpathlineto{\pgfqpoint{1.913918in}{2.539821in}}%
\pgfpathlineto{\pgfqpoint{1.914806in}{2.310772in}}%
\pgfpathlineto{\pgfqpoint{1.915694in}{2.338327in}}%
\pgfpathlineto{\pgfqpoint{1.915793in}{2.337963in}}%
\pgfpathlineto{\pgfqpoint{1.917865in}{2.415687in}}%
\pgfpathlineto{\pgfqpoint{1.918162in}{2.401484in}}%
\pgfpathlineto{\pgfqpoint{1.918852in}{2.345374in}}%
\pgfpathlineto{\pgfqpoint{1.919445in}{2.371794in}}%
\pgfpathlineto{\pgfqpoint{1.920925in}{2.397693in}}%
\pgfpathlineto{\pgfqpoint{1.921123in}{2.405487in}}%
\pgfpathlineto{\pgfqpoint{1.921616in}{2.383358in}}%
\pgfpathlineto{\pgfqpoint{1.922110in}{2.359920in}}%
\pgfpathlineto{\pgfqpoint{1.922800in}{2.376813in}}%
\pgfpathlineto{\pgfqpoint{1.924478in}{2.405064in}}%
\pgfpathlineto{\pgfqpoint{1.924676in}{2.401643in}}%
\pgfpathlineto{\pgfqpoint{1.925465in}{2.381237in}}%
\pgfpathlineto{\pgfqpoint{1.925860in}{2.397684in}}%
\pgfpathlineto{\pgfqpoint{1.926156in}{2.410838in}}%
\pgfpathlineto{\pgfqpoint{1.926748in}{2.390520in}}%
\pgfpathlineto{\pgfqpoint{1.927045in}{2.381531in}}%
\pgfpathlineto{\pgfqpoint{1.927538in}{2.394308in}}%
\pgfpathlineto{\pgfqpoint{1.927834in}{2.388613in}}%
\pgfpathlineto{\pgfqpoint{1.928031in}{2.391021in}}%
\pgfpathlineto{\pgfqpoint{1.929413in}{2.411048in}}%
\pgfpathlineto{\pgfqpoint{1.928821in}{2.387765in}}%
\pgfpathlineto{\pgfqpoint{1.929512in}{2.408245in}}%
\pgfpathlineto{\pgfqpoint{1.929907in}{2.386687in}}%
\pgfpathlineto{\pgfqpoint{1.930696in}{2.394444in}}%
\pgfpathlineto{\pgfqpoint{1.930992in}{2.401349in}}%
\pgfpathlineto{\pgfqpoint{1.931585in}{2.386362in}}%
\pgfpathlineto{\pgfqpoint{1.931782in}{2.385111in}}%
\pgfpathlineto{\pgfqpoint{1.932078in}{2.388366in}}%
\pgfpathlineto{\pgfqpoint{1.932572in}{2.410005in}}%
\pgfpathlineto{\pgfqpoint{1.933164in}{2.388507in}}%
\pgfpathlineto{\pgfqpoint{1.933756in}{2.368907in}}%
\pgfpathlineto{\pgfqpoint{1.934250in}{2.386940in}}%
\pgfpathlineto{\pgfqpoint{1.934644in}{2.389553in}}%
\pgfpathlineto{\pgfqpoint{1.934940in}{2.382024in}}%
\pgfpathlineto{\pgfqpoint{1.936618in}{2.343782in}}%
\pgfpathlineto{\pgfqpoint{1.936816in}{2.344686in}}%
\pgfpathlineto{\pgfqpoint{1.937013in}{2.341272in}}%
\pgfpathlineto{\pgfqpoint{1.937803in}{2.345529in}}%
\pgfpathlineto{\pgfqpoint{1.938395in}{2.322297in}}%
\pgfpathlineto{\pgfqpoint{1.940270in}{2.286276in}}%
\pgfpathlineto{\pgfqpoint{1.940566in}{2.297783in}}%
\pgfpathlineto{\pgfqpoint{1.940764in}{2.305980in}}%
\pgfpathlineto{\pgfqpoint{1.941455in}{2.292519in}}%
\pgfpathlineto{\pgfqpoint{1.941751in}{2.282907in}}%
\pgfpathlineto{\pgfqpoint{1.942343in}{2.298633in}}%
\pgfpathlineto{\pgfqpoint{1.942442in}{2.300751in}}%
\pgfpathlineto{\pgfqpoint{1.942738in}{2.290925in}}%
\pgfpathlineto{\pgfqpoint{1.943034in}{2.274432in}}%
\pgfpathlineto{\pgfqpoint{1.943823in}{2.285708in}}%
\pgfpathlineto{\pgfqpoint{1.945304in}{2.310410in}}%
\pgfpathlineto{\pgfqpoint{1.947278in}{2.339366in}}%
\pgfpathlineto{\pgfqpoint{1.947574in}{2.351193in}}%
\pgfpathlineto{\pgfqpoint{1.948067in}{2.330776in}}%
\pgfpathlineto{\pgfqpoint{1.949943in}{2.274245in}}%
\pgfpathlineto{\pgfqpoint{1.950140in}{2.280888in}}%
\pgfpathlineto{\pgfqpoint{1.950732in}{2.312112in}}%
\pgfpathlineto{\pgfqpoint{1.951127in}{2.289273in}}%
\pgfpathlineto{\pgfqpoint{1.952608in}{2.247021in}}%
\pgfpathlineto{\pgfqpoint{1.952805in}{2.241370in}}%
\pgfpathlineto{\pgfqpoint{1.953298in}{2.267504in}}%
\pgfpathlineto{\pgfqpoint{1.954187in}{2.309821in}}%
\pgfpathlineto{\pgfqpoint{1.954483in}{2.283312in}}%
\pgfpathlineto{\pgfqpoint{1.954878in}{2.237590in}}%
\pgfpathlineto{\pgfqpoint{1.955766in}{2.248330in}}%
\pgfpathlineto{\pgfqpoint{1.956161in}{2.244586in}}%
\pgfpathlineto{\pgfqpoint{1.956358in}{2.248918in}}%
\pgfpathlineto{\pgfqpoint{1.957246in}{2.334870in}}%
\pgfpathlineto{\pgfqpoint{1.958628in}{2.517784in}}%
\pgfpathlineto{\pgfqpoint{1.959122in}{2.497145in}}%
\pgfpathlineto{\pgfqpoint{1.959615in}{2.447233in}}%
\pgfpathlineto{\pgfqpoint{1.960405in}{2.257539in}}%
\pgfpathlineto{\pgfqpoint{1.961194in}{2.307578in}}%
\pgfpathlineto{\pgfqpoint{1.961490in}{2.296540in}}%
\pgfpathlineto{\pgfqpoint{1.961984in}{2.323571in}}%
\pgfpathlineto{\pgfqpoint{1.962083in}{2.324952in}}%
\pgfpathlineto{\pgfqpoint{1.962379in}{2.314960in}}%
\pgfpathlineto{\pgfqpoint{1.962872in}{2.311941in}}%
\pgfpathlineto{\pgfqpoint{1.963070in}{2.316481in}}%
\pgfpathlineto{\pgfqpoint{1.963464in}{2.342387in}}%
\pgfpathlineto{\pgfqpoint{1.963958in}{2.314656in}}%
\pgfpathlineto{\pgfqpoint{1.964550in}{2.262852in}}%
\pgfpathlineto{\pgfqpoint{1.965142in}{2.305499in}}%
\pgfpathlineto{\pgfqpoint{1.966820in}{2.409359in}}%
\pgfpathlineto{\pgfqpoint{1.967116in}{2.398311in}}%
\pgfpathlineto{\pgfqpoint{1.968005in}{2.352713in}}%
\pgfpathlineto{\pgfqpoint{1.968893in}{2.375201in}}%
\pgfpathlineto{\pgfqpoint{1.970275in}{2.409151in}}%
\pgfpathlineto{\pgfqpoint{1.970768in}{2.387489in}}%
\pgfpathlineto{\pgfqpoint{1.971163in}{2.374094in}}%
\pgfpathlineto{\pgfqpoint{1.971755in}{2.390965in}}%
\pgfpathlineto{\pgfqpoint{1.972150in}{2.410128in}}%
\pgfpathlineto{\pgfqpoint{1.972742in}{2.388389in}}%
\pgfpathlineto{\pgfqpoint{1.972940in}{2.387877in}}%
\pgfpathlineto{\pgfqpoint{1.973236in}{2.390552in}}%
\pgfpathlineto{\pgfqpoint{1.973532in}{2.395974in}}%
\pgfpathlineto{\pgfqpoint{1.974025in}{2.384847in}}%
\pgfpathlineto{\pgfqpoint{1.974617in}{2.378781in}}%
\pgfpathlineto{\pgfqpoint{1.974913in}{2.385365in}}%
\pgfpathlineto{\pgfqpoint{1.975111in}{2.389712in}}%
\pgfpathlineto{\pgfqpoint{1.975802in}{2.379419in}}%
\pgfpathlineto{\pgfqpoint{1.977184in}{2.362603in}}%
\pgfpathlineto{\pgfqpoint{1.977381in}{2.362782in}}%
\pgfpathlineto{\pgfqpoint{1.977973in}{2.345713in}}%
\pgfpathlineto{\pgfqpoint{1.979158in}{2.352742in}}%
\pgfpathlineto{\pgfqpoint{1.979256in}{2.353545in}}%
\pgfpathlineto{\pgfqpoint{1.979355in}{2.350824in}}%
\pgfpathlineto{\pgfqpoint{1.980046in}{2.357134in}}%
\pgfpathlineto{\pgfqpoint{1.980934in}{2.326991in}}%
\pgfpathlineto{\pgfqpoint{1.981329in}{2.325363in}}%
\pgfpathlineto{\pgfqpoint{1.981526in}{2.327071in}}%
\pgfpathlineto{\pgfqpoint{1.982020in}{2.343920in}}%
\pgfpathlineto{\pgfqpoint{1.982217in}{2.328125in}}%
\pgfpathlineto{\pgfqpoint{1.983402in}{2.275435in}}%
\pgfpathlineto{\pgfqpoint{1.983599in}{2.281804in}}%
\pgfpathlineto{\pgfqpoint{1.983698in}{2.282794in}}%
\pgfpathlineto{\pgfqpoint{1.983895in}{2.276462in}}%
\pgfpathlineto{\pgfqpoint{1.986066in}{2.219832in}}%
\pgfpathlineto{\pgfqpoint{1.986165in}{2.220810in}}%
\pgfpathlineto{\pgfqpoint{1.986461in}{2.227646in}}%
\pgfpathlineto{\pgfqpoint{1.987053in}{2.217327in}}%
\pgfpathlineto{\pgfqpoint{1.987547in}{2.197261in}}%
\pgfpathlineto{\pgfqpoint{1.988337in}{2.210206in}}%
\pgfpathlineto{\pgfqpoint{1.988534in}{2.211561in}}%
\pgfpathlineto{\pgfqpoint{1.988731in}{2.206405in}}%
\pgfpathlineto{\pgfqpoint{1.989225in}{2.188104in}}%
\pgfpathlineto{\pgfqpoint{1.989817in}{2.202863in}}%
\pgfpathlineto{\pgfqpoint{1.989916in}{2.203674in}}%
\pgfpathlineto{\pgfqpoint{1.990212in}{2.197513in}}%
\pgfpathlineto{\pgfqpoint{1.990311in}{2.196065in}}%
\pgfpathlineto{\pgfqpoint{1.990607in}{2.203106in}}%
\pgfpathlineto{\pgfqpoint{1.991594in}{2.221049in}}%
\pgfpathlineto{\pgfqpoint{1.991100in}{2.200674in}}%
\pgfpathlineto{\pgfqpoint{1.991890in}{2.212926in}}%
\pgfpathlineto{\pgfqpoint{1.992778in}{2.202030in}}%
\pgfpathlineto{\pgfqpoint{1.992383in}{2.214237in}}%
\pgfpathlineto{\pgfqpoint{1.992975in}{2.206313in}}%
\pgfpathlineto{\pgfqpoint{1.993271in}{2.218910in}}%
\pgfpathlineto{\pgfqpoint{1.993765in}{2.204731in}}%
\pgfpathlineto{\pgfqpoint{1.993962in}{2.205050in}}%
\pgfpathlineto{\pgfqpoint{1.994357in}{2.185705in}}%
\pgfpathlineto{\pgfqpoint{1.995640in}{2.160781in}}%
\pgfpathlineto{\pgfqpoint{1.995838in}{2.161441in}}%
\pgfpathlineto{\pgfqpoint{1.996430in}{2.188127in}}%
\pgfpathlineto{\pgfqpoint{1.996825in}{2.164280in}}%
\pgfpathlineto{\pgfqpoint{1.997121in}{2.149926in}}%
\pgfpathlineto{\pgfqpoint{1.997713in}{2.174026in}}%
\pgfpathlineto{\pgfqpoint{1.999588in}{2.251074in}}%
\pgfpathlineto{\pgfqpoint{2.000180in}{2.233867in}}%
\pgfpathlineto{\pgfqpoint{2.000674in}{2.203852in}}%
\pgfpathlineto{\pgfqpoint{2.001167in}{2.238673in}}%
\pgfpathlineto{\pgfqpoint{2.001562in}{2.229697in}}%
\pgfpathlineto{\pgfqpoint{2.002154in}{2.237811in}}%
\pgfpathlineto{\pgfqpoint{2.002549in}{2.258877in}}%
\pgfpathlineto{\pgfqpoint{2.003832in}{2.489808in}}%
\pgfpathlineto{\pgfqpoint{2.004721in}{2.449957in}}%
\pgfpathlineto{\pgfqpoint{2.005115in}{2.385315in}}%
\pgfpathlineto{\pgfqpoint{2.005806in}{2.180139in}}%
\pgfpathlineto{\pgfqpoint{2.006596in}{2.250281in}}%
\pgfpathlineto{\pgfqpoint{2.007484in}{2.221866in}}%
\pgfpathlineto{\pgfqpoint{2.007879in}{2.235862in}}%
\pgfpathlineto{\pgfqpoint{2.009458in}{2.263383in}}%
\pgfpathlineto{\pgfqpoint{2.008274in}{2.232943in}}%
\pgfpathlineto{\pgfqpoint{2.009656in}{2.255247in}}%
\pgfpathlineto{\pgfqpoint{2.010840in}{2.203291in}}%
\pgfpathlineto{\pgfqpoint{2.011136in}{2.219219in}}%
\pgfpathlineto{\pgfqpoint{2.012518in}{2.245592in}}%
\pgfpathlineto{\pgfqpoint{2.012913in}{2.255305in}}%
\pgfpathlineto{\pgfqpoint{2.013209in}{2.244342in}}%
\pgfpathlineto{\pgfqpoint{2.013406in}{2.236946in}}%
\pgfpathlineto{\pgfqpoint{2.014097in}{2.254903in}}%
\pgfpathlineto{\pgfqpoint{2.019427in}{2.485178in}}%
\pgfpathlineto{\pgfqpoint{2.019723in}{2.481829in}}%
\pgfpathlineto{\pgfqpoint{2.020019in}{2.487243in}}%
\pgfpathlineto{\pgfqpoint{2.020315in}{2.478589in}}%
\pgfpathlineto{\pgfqpoint{2.020512in}{2.474454in}}%
\pgfpathlineto{\pgfqpoint{2.020907in}{2.498050in}}%
\pgfpathlineto{\pgfqpoint{2.022585in}{2.541683in}}%
\pgfpathlineto{\pgfqpoint{2.023079in}{2.525443in}}%
\pgfpathlineto{\pgfqpoint{2.023868in}{2.532020in}}%
\pgfpathlineto{\pgfqpoint{2.025053in}{2.558499in}}%
\pgfpathlineto{\pgfqpoint{2.025349in}{2.557946in}}%
\pgfpathlineto{\pgfqpoint{2.025842in}{2.576886in}}%
\pgfpathlineto{\pgfqpoint{2.026533in}{2.561768in}}%
\pgfpathlineto{\pgfqpoint{2.026730in}{2.556283in}}%
\pgfpathlineto{\pgfqpoint{2.027323in}{2.570212in}}%
\pgfpathlineto{\pgfqpoint{2.027816in}{2.590390in}}%
\pgfpathlineto{\pgfqpoint{2.028310in}{2.570201in}}%
\pgfpathlineto{\pgfqpoint{2.028803in}{2.565241in}}%
\pgfpathlineto{\pgfqpoint{2.029099in}{2.572364in}}%
\pgfpathlineto{\pgfqpoint{2.029494in}{2.588201in}}%
\pgfpathlineto{\pgfqpoint{2.029889in}{2.563567in}}%
\pgfpathlineto{\pgfqpoint{2.030086in}{2.553567in}}%
\pgfpathlineto{\pgfqpoint{2.030777in}{2.578684in}}%
\pgfpathlineto{\pgfqpoint{2.031073in}{2.581362in}}%
\pgfpathlineto{\pgfqpoint{2.031665in}{2.589951in}}%
\pgfpathlineto{\pgfqpoint{2.032060in}{2.583286in}}%
\pgfpathlineto{\pgfqpoint{2.033343in}{2.566760in}}%
\pgfpathlineto{\pgfqpoint{2.032948in}{2.583479in}}%
\pgfpathlineto{\pgfqpoint{2.033442in}{2.569192in}}%
\pgfpathlineto{\pgfqpoint{2.034528in}{2.595293in}}%
\pgfpathlineto{\pgfqpoint{2.034824in}{2.592868in}}%
\pgfpathlineto{\pgfqpoint{2.034922in}{2.592968in}}%
\pgfpathlineto{\pgfqpoint{2.037489in}{2.652302in}}%
\pgfpathlineto{\pgfqpoint{2.037785in}{2.663747in}}%
\pgfpathlineto{\pgfqpoint{2.038377in}{2.644915in}}%
\pgfpathlineto{\pgfqpoint{2.040548in}{2.597769in}}%
\pgfpathlineto{\pgfqpoint{2.041239in}{2.608451in}}%
\pgfpathlineto{\pgfqpoint{2.041831in}{2.599292in}}%
\pgfpathlineto{\pgfqpoint{2.042522in}{2.571818in}}%
\pgfpathlineto{\pgfqpoint{2.043016in}{2.551313in}}%
\pgfpathlineto{\pgfqpoint{2.043608in}{2.568889in}}%
\pgfpathlineto{\pgfqpoint{2.044694in}{2.611302in}}%
\pgfpathlineto{\pgfqpoint{2.045187in}{2.586355in}}%
\pgfpathlineto{\pgfqpoint{2.046766in}{2.527273in}}%
\pgfpathlineto{\pgfqpoint{2.047161in}{2.538123in}}%
\pgfpathlineto{\pgfqpoint{2.047359in}{2.541427in}}%
\pgfpathlineto{\pgfqpoint{2.047753in}{2.526153in}}%
\pgfpathlineto{\pgfqpoint{2.047852in}{2.525701in}}%
\pgfpathlineto{\pgfqpoint{2.047951in}{2.530209in}}%
\pgfpathlineto{\pgfqpoint{2.049530in}{2.770952in}}%
\pgfpathlineto{\pgfqpoint{2.050517in}{2.711253in}}%
\pgfpathlineto{\pgfqpoint{2.051306in}{2.483543in}}%
\pgfpathlineto{\pgfqpoint{2.052195in}{2.524650in}}%
\pgfpathlineto{\pgfqpoint{2.053280in}{2.508728in}}%
\pgfpathlineto{\pgfqpoint{2.053577in}{2.518818in}}%
\pgfpathlineto{\pgfqpoint{2.054169in}{2.541702in}}%
\pgfpathlineto{\pgfqpoint{2.054564in}{2.517249in}}%
\pgfpathlineto{\pgfqpoint{2.056241in}{2.476806in}}%
\pgfpathlineto{\pgfqpoint{2.056340in}{2.476533in}}%
\pgfpathlineto{\pgfqpoint{2.056439in}{2.478980in}}%
\pgfpathlineto{\pgfqpoint{2.057426in}{2.509423in}}%
\pgfpathlineto{\pgfqpoint{2.057722in}{2.496131in}}%
\pgfpathlineto{\pgfqpoint{2.058413in}{2.458819in}}%
\pgfpathlineto{\pgfqpoint{2.059104in}{2.467685in}}%
\pgfpathlineto{\pgfqpoint{2.059400in}{2.470055in}}%
\pgfpathlineto{\pgfqpoint{2.059597in}{2.463772in}}%
\pgfpathlineto{\pgfqpoint{2.059893in}{2.452568in}}%
\pgfpathlineto{\pgfqpoint{2.060485in}{2.469167in}}%
\pgfpathlineto{\pgfqpoint{2.060979in}{2.482715in}}%
\pgfpathlineto{\pgfqpoint{2.061966in}{2.480510in}}%
\pgfpathlineto{\pgfqpoint{2.062361in}{2.486639in}}%
\pgfpathlineto{\pgfqpoint{2.062657in}{2.479815in}}%
\pgfpathlineto{\pgfqpoint{2.062953in}{2.471184in}}%
\pgfpathlineto{\pgfqpoint{2.063348in}{2.488904in}}%
\pgfpathlineto{\pgfqpoint{2.063743in}{2.505604in}}%
\pgfpathlineto{\pgfqpoint{2.064236in}{2.481847in}}%
\pgfpathlineto{\pgfqpoint{2.064335in}{2.481994in}}%
\pgfpathlineto{\pgfqpoint{2.065420in}{2.523559in}}%
\pgfpathlineto{\pgfqpoint{2.066111in}{2.503662in}}%
\pgfpathlineto{\pgfqpoint{2.066309in}{2.500589in}}%
\pgfpathlineto{\pgfqpoint{2.067000in}{2.507580in}}%
\pgfpathlineto{\pgfqpoint{2.067888in}{2.492636in}}%
\pgfpathlineto{\pgfqpoint{2.068184in}{2.506666in}}%
\pgfpathlineto{\pgfqpoint{2.068381in}{2.515869in}}%
\pgfpathlineto{\pgfqpoint{2.069171in}{2.508906in}}%
\pgfpathlineto{\pgfqpoint{2.070849in}{2.473623in}}%
\pgfpathlineto{\pgfqpoint{2.070948in}{2.474864in}}%
\pgfpathlineto{\pgfqpoint{2.071737in}{2.488163in}}%
\pgfpathlineto{\pgfqpoint{2.072033in}{2.480327in}}%
\pgfpathlineto{\pgfqpoint{2.073810in}{2.436354in}}%
\pgfpathlineto{\pgfqpoint{2.073909in}{2.436253in}}%
\pgfpathlineto{\pgfqpoint{2.074007in}{2.437711in}}%
\pgfpathlineto{\pgfqpoint{2.074106in}{2.438604in}}%
\pgfpathlineto{\pgfqpoint{2.074303in}{2.433523in}}%
\pgfpathlineto{\pgfqpoint{2.075389in}{2.419800in}}%
\pgfpathlineto{\pgfqpoint{2.074896in}{2.437284in}}%
\pgfpathlineto{\pgfqpoint{2.075586in}{2.420669in}}%
\pgfpathlineto{\pgfqpoint{2.076179in}{2.398448in}}%
\pgfpathlineto{\pgfqpoint{2.076771in}{2.419255in}}%
\pgfpathlineto{\pgfqpoint{2.077264in}{2.421882in}}%
\pgfpathlineto{\pgfqpoint{2.077462in}{2.419172in}}%
\pgfpathlineto{\pgfqpoint{2.077856in}{2.396041in}}%
\pgfpathlineto{\pgfqpoint{2.078449in}{2.420256in}}%
\pgfpathlineto{\pgfqpoint{2.078547in}{2.418378in}}%
\pgfpathlineto{\pgfqpoint{2.079337in}{2.399776in}}%
\pgfpathlineto{\pgfqpoint{2.079534in}{2.411057in}}%
\pgfpathlineto{\pgfqpoint{2.080521in}{2.432914in}}%
\pgfpathlineto{\pgfqpoint{2.080719in}{2.423570in}}%
\pgfpathlineto{\pgfqpoint{2.080817in}{2.419230in}}%
\pgfpathlineto{\pgfqpoint{2.081212in}{2.443854in}}%
\pgfpathlineto{\pgfqpoint{2.081508in}{2.456634in}}%
\pgfpathlineto{\pgfqpoint{2.082397in}{2.452292in}}%
\pgfpathlineto{\pgfqpoint{2.082594in}{2.448470in}}%
\pgfpathlineto{\pgfqpoint{2.082890in}{2.463114in}}%
\pgfpathlineto{\pgfqpoint{2.083384in}{2.502119in}}%
\pgfpathlineto{\pgfqpoint{2.084173in}{2.494589in}}%
\pgfpathlineto{\pgfqpoint{2.084272in}{2.494665in}}%
\pgfpathlineto{\pgfqpoint{2.084568in}{2.498642in}}%
\pgfpathlineto{\pgfqpoint{2.084963in}{2.486144in}}%
\pgfpathlineto{\pgfqpoint{2.085456in}{2.467978in}}%
\pgfpathlineto{\pgfqpoint{2.086049in}{2.439718in}}%
\pgfpathlineto{\pgfqpoint{2.086641in}{2.455799in}}%
\pgfpathlineto{\pgfqpoint{2.087233in}{2.468246in}}%
\pgfpathlineto{\pgfqpoint{2.087529in}{2.451124in}}%
\pgfpathlineto{\pgfqpoint{2.089009in}{2.409017in}}%
\pgfpathlineto{\pgfqpoint{2.089108in}{2.409538in}}%
\pgfpathlineto{\pgfqpoint{2.089700in}{2.429210in}}%
\pgfpathlineto{\pgfqpoint{2.090095in}{2.467020in}}%
\pgfpathlineto{\pgfqpoint{2.090885in}{2.443432in}}%
\pgfpathlineto{\pgfqpoint{2.092267in}{2.394807in}}%
\pgfpathlineto{\pgfqpoint{2.092365in}{2.395544in}}%
\pgfpathlineto{\pgfqpoint{2.093648in}{2.467298in}}%
\pgfpathlineto{\pgfqpoint{2.094931in}{2.665112in}}%
\pgfpathlineto{\pgfqpoint{2.096017in}{2.610532in}}%
\pgfpathlineto{\pgfqpoint{2.096905in}{2.396264in}}%
\pgfpathlineto{\pgfqpoint{2.098090in}{2.437037in}}%
\pgfpathlineto{\pgfqpoint{2.099373in}{2.477025in}}%
\pgfpathlineto{\pgfqpoint{2.099965in}{2.507045in}}%
\pgfpathlineto{\pgfqpoint{2.100459in}{2.486297in}}%
\pgfpathlineto{\pgfqpoint{2.100853in}{2.459380in}}%
\pgfpathlineto{\pgfqpoint{2.101544in}{2.476875in}}%
\pgfpathlineto{\pgfqpoint{2.102038in}{2.473875in}}%
\pgfpathlineto{\pgfqpoint{2.102827in}{2.513239in}}%
\pgfpathlineto{\pgfqpoint{2.103222in}{2.537061in}}%
\pgfpathlineto{\pgfqpoint{2.103716in}{2.505579in}}%
\pgfpathlineto{\pgfqpoint{2.104012in}{2.494902in}}%
\pgfpathlineto{\pgfqpoint{2.104604in}{2.513825in}}%
\pgfpathlineto{\pgfqpoint{2.105887in}{2.533935in}}%
\pgfpathlineto{\pgfqpoint{2.106479in}{2.563989in}}%
\pgfpathlineto{\pgfqpoint{2.106973in}{2.537641in}}%
\pgfpathlineto{\pgfqpoint{2.107071in}{2.535772in}}%
\pgfpathlineto{\pgfqpoint{2.107466in}{2.546317in}}%
\pgfpathlineto{\pgfqpoint{2.109835in}{2.592307in}}%
\pgfpathlineto{\pgfqpoint{2.110131in}{2.581466in}}%
\pgfpathlineto{\pgfqpoint{2.110328in}{2.573211in}}%
\pgfpathlineto{\pgfqpoint{2.110921in}{2.598202in}}%
\pgfpathlineto{\pgfqpoint{2.111513in}{2.610755in}}%
\pgfpathlineto{\pgfqpoint{2.111908in}{2.596389in}}%
\pgfpathlineto{\pgfqpoint{2.112204in}{2.609492in}}%
\pgfpathlineto{\pgfqpoint{2.112500in}{2.619317in}}%
\pgfpathlineto{\pgfqpoint{2.113289in}{2.607986in}}%
\pgfpathlineto{\pgfqpoint{2.113586in}{2.599747in}}%
\pgfpathlineto{\pgfqpoint{2.114276in}{2.610831in}}%
\pgfpathlineto{\pgfqpoint{2.114671in}{2.629145in}}%
\pgfpathlineto{\pgfqpoint{2.115263in}{2.612805in}}%
\pgfpathlineto{\pgfqpoint{2.115559in}{2.604285in}}%
\pgfpathlineto{\pgfqpoint{2.115954in}{2.617301in}}%
\pgfpathlineto{\pgfqpoint{2.116349in}{2.612202in}}%
\pgfpathlineto{\pgfqpoint{2.116546in}{2.612840in}}%
\pgfpathlineto{\pgfqpoint{2.116744in}{2.610891in}}%
\pgfpathlineto{\pgfqpoint{2.116941in}{2.607666in}}%
\pgfpathlineto{\pgfqpoint{2.117336in}{2.618254in}}%
\pgfpathlineto{\pgfqpoint{2.117533in}{2.621656in}}%
\pgfpathlineto{\pgfqpoint{2.118126in}{2.612991in}}%
\pgfpathlineto{\pgfqpoint{2.120396in}{2.578656in}}%
\pgfpathlineto{\pgfqpoint{2.120494in}{2.580945in}}%
\pgfpathlineto{\pgfqpoint{2.121087in}{2.600676in}}%
\pgfpathlineto{\pgfqpoint{2.121580in}{2.580738in}}%
\pgfpathlineto{\pgfqpoint{2.121876in}{2.565751in}}%
\pgfpathlineto{\pgfqpoint{2.122765in}{2.571904in}}%
\pgfpathlineto{\pgfqpoint{2.122863in}{2.572978in}}%
\pgfpathlineto{\pgfqpoint{2.123061in}{2.566508in}}%
\pgfpathlineto{\pgfqpoint{2.123357in}{2.556476in}}%
\pgfpathlineto{\pgfqpoint{2.124048in}{2.566885in}}%
\pgfpathlineto{\pgfqpoint{2.124344in}{2.577210in}}%
\pgfpathlineto{\pgfqpoint{2.124936in}{2.560010in}}%
\pgfpathlineto{\pgfqpoint{2.125824in}{2.552927in}}%
\pgfpathlineto{\pgfqpoint{2.125429in}{2.565982in}}%
\pgfpathlineto{\pgfqpoint{2.126022in}{2.559593in}}%
\pgfpathlineto{\pgfqpoint{2.127601in}{2.592119in}}%
\pgfpathlineto{\pgfqpoint{2.126515in}{2.550267in}}%
\pgfpathlineto{\pgfqpoint{2.127699in}{2.588381in}}%
\pgfpathlineto{\pgfqpoint{2.129279in}{2.563834in}}%
\pgfpathlineto{\pgfqpoint{2.131253in}{2.485339in}}%
\pgfpathlineto{\pgfqpoint{2.132141in}{2.510198in}}%
\pgfpathlineto{\pgfqpoint{2.132634in}{2.557539in}}%
\pgfpathlineto{\pgfqpoint{2.133424in}{2.538530in}}%
\pgfpathlineto{\pgfqpoint{2.134904in}{2.504293in}}%
\pgfpathlineto{\pgfqpoint{2.134214in}{2.539196in}}%
\pgfpathlineto{\pgfqpoint{2.135003in}{2.507390in}}%
\pgfpathlineto{\pgfqpoint{2.135694in}{2.550569in}}%
\pgfpathlineto{\pgfqpoint{2.136385in}{2.526880in}}%
\pgfpathlineto{\pgfqpoint{2.136780in}{2.501036in}}%
\pgfpathlineto{\pgfqpoint{2.137668in}{2.503570in}}%
\pgfpathlineto{\pgfqpoint{2.137865in}{2.502835in}}%
\pgfpathlineto{\pgfqpoint{2.138063in}{2.505171in}}%
\pgfpathlineto{\pgfqpoint{2.138951in}{2.545890in}}%
\pgfpathlineto{\pgfqpoint{2.140728in}{2.786660in}}%
\pgfpathlineto{\pgfqpoint{2.141221in}{2.721995in}}%
\pgfpathlineto{\pgfqpoint{2.142406in}{2.498289in}}%
\pgfpathlineto{\pgfqpoint{2.144083in}{2.534805in}}%
\pgfpathlineto{\pgfqpoint{2.144380in}{2.530685in}}%
\pgfpathlineto{\pgfqpoint{2.144873in}{2.541185in}}%
\pgfpathlineto{\pgfqpoint{2.145761in}{2.573321in}}%
\pgfpathlineto{\pgfqpoint{2.146156in}{2.556622in}}%
\pgfpathlineto{\pgfqpoint{2.146650in}{2.512815in}}%
\pgfpathlineto{\pgfqpoint{2.147637in}{2.518037in}}%
\pgfpathlineto{\pgfqpoint{2.147735in}{2.515823in}}%
\pgfpathlineto{\pgfqpoint{2.148031in}{2.525047in}}%
\pgfpathlineto{\pgfqpoint{2.148722in}{2.563311in}}%
\pgfpathlineto{\pgfqpoint{2.149315in}{2.545503in}}%
\pgfpathlineto{\pgfqpoint{2.149709in}{2.522756in}}%
\pgfpathlineto{\pgfqpoint{2.150499in}{2.530927in}}%
\pgfpathlineto{\pgfqpoint{2.151881in}{2.558859in}}%
\pgfpathlineto{\pgfqpoint{2.151289in}{2.521388in}}%
\pgfpathlineto{\pgfqpoint{2.151979in}{2.558088in}}%
\pgfpathlineto{\pgfqpoint{2.153361in}{2.518258in}}%
\pgfpathlineto{\pgfqpoint{2.153657in}{2.524443in}}%
\pgfpathlineto{\pgfqpoint{2.153855in}{2.527112in}}%
\pgfpathlineto{\pgfqpoint{2.154249in}{2.517027in}}%
\pgfpathlineto{\pgfqpoint{2.154546in}{2.521276in}}%
\pgfpathlineto{\pgfqpoint{2.154743in}{2.520714in}}%
\pgfpathlineto{\pgfqpoint{2.154842in}{2.522857in}}%
\pgfpathlineto{\pgfqpoint{2.155236in}{2.540634in}}%
\pgfpathlineto{\pgfqpoint{2.155829in}{2.520362in}}%
\pgfpathlineto{\pgfqpoint{2.155927in}{2.520719in}}%
\pgfpathlineto{\pgfqpoint{2.156125in}{2.517329in}}%
\pgfpathlineto{\pgfqpoint{2.156421in}{2.508943in}}%
\pgfpathlineto{\pgfqpoint{2.157013in}{2.521317in}}%
\pgfpathlineto{\pgfqpoint{2.157408in}{2.510785in}}%
\pgfpathlineto{\pgfqpoint{2.157803in}{2.509803in}}%
\pgfpathlineto{\pgfqpoint{2.158197in}{2.513392in}}%
\pgfpathlineto{\pgfqpoint{2.158296in}{2.513551in}}%
\pgfpathlineto{\pgfqpoint{2.158395in}{2.512669in}}%
\pgfpathlineto{\pgfqpoint{2.159678in}{2.489402in}}%
\pgfpathlineto{\pgfqpoint{2.158987in}{2.513963in}}%
\pgfpathlineto{\pgfqpoint{2.160270in}{2.495973in}}%
\pgfpathlineto{\pgfqpoint{2.160468in}{2.499452in}}%
\pgfpathlineto{\pgfqpoint{2.160862in}{2.485416in}}%
\pgfpathlineto{\pgfqpoint{2.161060in}{2.483366in}}%
\pgfpathlineto{\pgfqpoint{2.161455in}{2.489420in}}%
\pgfpathlineto{\pgfqpoint{2.161849in}{2.510114in}}%
\pgfpathlineto{\pgfqpoint{2.162343in}{2.478007in}}%
\pgfpathlineto{\pgfqpoint{2.162639in}{2.475452in}}%
\pgfpathlineto{\pgfqpoint{2.162935in}{2.464536in}}%
\pgfpathlineto{\pgfqpoint{2.163428in}{2.475902in}}%
\pgfpathlineto{\pgfqpoint{2.163725in}{2.474373in}}%
\pgfpathlineto{\pgfqpoint{2.164021in}{2.478256in}}%
\pgfpathlineto{\pgfqpoint{2.164317in}{2.467688in}}%
\pgfpathlineto{\pgfqpoint{2.166488in}{2.427019in}}%
\pgfpathlineto{\pgfqpoint{2.167080in}{2.449167in}}%
\pgfpathlineto{\pgfqpoint{2.168462in}{2.446009in}}%
\pgfpathlineto{\pgfqpoint{2.169548in}{2.417961in}}%
\pgfpathlineto{\pgfqpoint{2.170337in}{2.425196in}}%
\pgfpathlineto{\pgfqpoint{2.170535in}{2.426348in}}%
\pgfpathlineto{\pgfqpoint{2.170930in}{2.423072in}}%
\pgfpathlineto{\pgfqpoint{2.171226in}{2.421780in}}%
\pgfpathlineto{\pgfqpoint{2.171620in}{2.426991in}}%
\pgfpathlineto{\pgfqpoint{2.172114in}{2.434900in}}%
\pgfpathlineto{\pgfqpoint{2.172410in}{2.427987in}}%
\pgfpathlineto{\pgfqpoint{2.172706in}{2.419525in}}%
\pgfpathlineto{\pgfqpoint{2.173200in}{2.443995in}}%
\pgfpathlineto{\pgfqpoint{2.173496in}{2.448514in}}%
\pgfpathlineto{\pgfqpoint{2.174878in}{2.468293in}}%
\pgfpathlineto{\pgfqpoint{2.177542in}{2.397740in}}%
\pgfpathlineto{\pgfqpoint{2.178036in}{2.410490in}}%
\pgfpathlineto{\pgfqpoint{2.178233in}{2.414849in}}%
\pgfpathlineto{\pgfqpoint{2.178628in}{2.394045in}}%
\pgfpathlineto{\pgfqpoint{2.179220in}{2.366388in}}%
\pgfpathlineto{\pgfqpoint{2.179813in}{2.383080in}}%
\pgfpathlineto{\pgfqpoint{2.181392in}{2.419026in}}%
\pgfpathlineto{\pgfqpoint{2.180503in}{2.382742in}}%
\pgfpathlineto{\pgfqpoint{2.181589in}{2.409173in}}%
\pgfpathlineto{\pgfqpoint{2.182477in}{2.357491in}}%
\pgfpathlineto{\pgfqpoint{2.182971in}{2.382375in}}%
\pgfpathlineto{\pgfqpoint{2.183958in}{2.388612in}}%
\pgfpathlineto{\pgfqpoint{2.184155in}{2.384606in}}%
\pgfpathlineto{\pgfqpoint{2.184254in}{2.383369in}}%
\pgfpathlineto{\pgfqpoint{2.184353in}{2.386097in}}%
\pgfpathlineto{\pgfqpoint{2.186031in}{2.624474in}}%
\pgfpathlineto{\pgfqpoint{2.187018in}{2.557850in}}%
\pgfpathlineto{\pgfqpoint{2.187807in}{2.329715in}}%
\pgfpathlineto{\pgfqpoint{2.188695in}{2.392922in}}%
\pgfpathlineto{\pgfqpoint{2.188992in}{2.384137in}}%
\pgfpathlineto{\pgfqpoint{2.189386in}{2.404894in}}%
\pgfpathlineto{\pgfqpoint{2.190965in}{2.444776in}}%
\pgfpathlineto{\pgfqpoint{2.191064in}{2.444717in}}%
\pgfpathlineto{\pgfqpoint{2.192150in}{2.398265in}}%
\pgfpathlineto{\pgfqpoint{2.193038in}{2.415283in}}%
\pgfpathlineto{\pgfqpoint{2.194025in}{2.434503in}}%
\pgfpathlineto{\pgfqpoint{2.195012in}{2.425258in}}%
\pgfpathlineto{\pgfqpoint{2.195407in}{2.420418in}}%
\pgfpathlineto{\pgfqpoint{2.195900in}{2.427674in}}%
\pgfpathlineto{\pgfqpoint{2.196394in}{2.434499in}}%
\pgfpathlineto{\pgfqpoint{2.196789in}{2.426794in}}%
\pgfpathlineto{\pgfqpoint{2.196986in}{2.421091in}}%
\pgfpathlineto{\pgfqpoint{2.197578in}{2.438514in}}%
\pgfpathlineto{\pgfqpoint{2.198664in}{2.431274in}}%
\pgfpathlineto{\pgfqpoint{2.199256in}{2.432074in}}%
\pgfpathlineto{\pgfqpoint{2.199750in}{2.453435in}}%
\pgfpathlineto{\pgfqpoint{2.200243in}{2.432409in}}%
\pgfpathlineto{\pgfqpoint{2.200441in}{2.430399in}}%
\pgfpathlineto{\pgfqpoint{2.200737in}{2.438413in}}%
\pgfpathlineto{\pgfqpoint{2.201822in}{2.451070in}}%
\pgfpathlineto{\pgfqpoint{2.202020in}{2.448330in}}%
\pgfpathlineto{\pgfqpoint{2.203402in}{2.422184in}}%
\pgfpathlineto{\pgfqpoint{2.203599in}{2.428180in}}%
\pgfpathlineto{\pgfqpoint{2.204389in}{2.449679in}}%
\pgfpathlineto{\pgfqpoint{2.205079in}{2.444135in}}%
\pgfpathlineto{\pgfqpoint{2.206955in}{2.418854in}}%
\pgfpathlineto{\pgfqpoint{2.207053in}{2.420549in}}%
\pgfpathlineto{\pgfqpoint{2.208040in}{2.450261in}}%
\pgfpathlineto{\pgfqpoint{2.208337in}{2.434233in}}%
\pgfpathlineto{\pgfqpoint{2.209718in}{2.407606in}}%
\pgfpathlineto{\pgfqpoint{2.212186in}{2.302541in}}%
\pgfpathlineto{\pgfqpoint{2.212975in}{2.332444in}}%
\pgfpathlineto{\pgfqpoint{2.214258in}{2.389326in}}%
\pgfpathlineto{\pgfqpoint{2.214752in}{2.375160in}}%
\pgfpathlineto{\pgfqpoint{2.214949in}{2.372549in}}%
\pgfpathlineto{\pgfqpoint{2.215640in}{2.378148in}}%
\pgfpathlineto{\pgfqpoint{2.215936in}{2.383782in}}%
\pgfpathlineto{\pgfqpoint{2.216331in}{2.367790in}}%
\pgfpathlineto{\pgfqpoint{2.216430in}{2.366876in}}%
\pgfpathlineto{\pgfqpoint{2.216726in}{2.370881in}}%
\pgfpathlineto{\pgfqpoint{2.217022in}{2.370103in}}%
\pgfpathlineto{\pgfqpoint{2.217614in}{2.396912in}}%
\pgfpathlineto{\pgfqpoint{2.218601in}{2.386783in}}%
\pgfpathlineto{\pgfqpoint{2.218799in}{2.382504in}}%
\pgfpathlineto{\pgfqpoint{2.219193in}{2.399990in}}%
\pgfpathlineto{\pgfqpoint{2.220378in}{2.414059in}}%
\pgfpathlineto{\pgfqpoint{2.220575in}{2.412686in}}%
\pgfpathlineto{\pgfqpoint{2.220773in}{2.415672in}}%
\pgfpathlineto{\pgfqpoint{2.220970in}{2.422726in}}%
\pgfpathlineto{\pgfqpoint{2.221365in}{2.398607in}}%
\pgfpathlineto{\pgfqpoint{2.222944in}{2.358953in}}%
\pgfpathlineto{\pgfqpoint{2.223635in}{2.373969in}}%
\pgfpathlineto{\pgfqpoint{2.223931in}{2.363748in}}%
\pgfpathlineto{\pgfqpoint{2.224918in}{2.322751in}}%
\pgfpathlineto{\pgfqpoint{2.225411in}{2.339180in}}%
\pgfpathlineto{\pgfqpoint{2.226596in}{2.380761in}}%
\pgfpathlineto{\pgfqpoint{2.226793in}{2.386676in}}%
\pgfpathlineto{\pgfqpoint{2.227484in}{2.373526in}}%
\pgfpathlineto{\pgfqpoint{2.228372in}{2.331187in}}%
\pgfpathlineto{\pgfqpoint{2.228767in}{2.349258in}}%
\pgfpathlineto{\pgfqpoint{2.230544in}{2.490839in}}%
\pgfpathlineto{\pgfqpoint{2.231629in}{2.603172in}}%
\pgfpathlineto{\pgfqpoint{2.232024in}{2.584514in}}%
\pgfpathlineto{\pgfqpoint{2.232419in}{2.577808in}}%
\pgfpathlineto{\pgfqpoint{2.233307in}{2.332873in}}%
\pgfpathlineto{\pgfqpoint{2.234887in}{2.376218in}}%
\pgfpathlineto{\pgfqpoint{2.236268in}{2.440529in}}%
\pgfpathlineto{\pgfqpoint{2.236564in}{2.421406in}}%
\pgfpathlineto{\pgfqpoint{2.238045in}{2.385431in}}%
\pgfpathlineto{\pgfqpoint{2.238144in}{2.385781in}}%
\pgfpathlineto{\pgfqpoint{2.239723in}{2.435673in}}%
\pgfpathlineto{\pgfqpoint{2.240216in}{2.422520in}}%
\pgfpathlineto{\pgfqpoint{2.240710in}{2.410588in}}%
\pgfpathlineto{\pgfqpoint{2.241401in}{2.420265in}}%
\pgfpathlineto{\pgfqpoint{2.242388in}{2.448110in}}%
\pgfpathlineto{\pgfqpoint{2.242881in}{2.442990in}}%
\pgfpathlineto{\pgfqpoint{2.243177in}{2.439701in}}%
\pgfpathlineto{\pgfqpoint{2.243375in}{2.437269in}}%
\pgfpathlineto{\pgfqpoint{2.244066in}{2.443658in}}%
\pgfpathlineto{\pgfqpoint{2.244263in}{2.444348in}}%
\pgfpathlineto{\pgfqpoint{2.244559in}{2.440046in}}%
\pgfpathlineto{\pgfqpoint{2.244658in}{2.439200in}}%
\pgfpathlineto{\pgfqpoint{2.244855in}{2.445021in}}%
\pgfpathlineto{\pgfqpoint{2.245349in}{2.473223in}}%
\pgfpathlineto{\pgfqpoint{2.246138in}{2.462565in}}%
\pgfpathlineto{\pgfqpoint{2.246434in}{2.450215in}}%
\pgfpathlineto{\pgfqpoint{2.247026in}{2.473577in}}%
\pgfpathlineto{\pgfqpoint{2.248013in}{2.466433in}}%
\pgfpathlineto{\pgfqpoint{2.248211in}{2.471401in}}%
\pgfpathlineto{\pgfqpoint{2.248902in}{2.490931in}}%
\pgfpathlineto{\pgfqpoint{2.249494in}{2.478371in}}%
\pgfpathlineto{\pgfqpoint{2.249593in}{2.480651in}}%
\pgfpathlineto{\pgfqpoint{2.249889in}{2.468802in}}%
\pgfpathlineto{\pgfqpoint{2.250086in}{2.459307in}}%
\pgfpathlineto{\pgfqpoint{2.250678in}{2.478074in}}%
\pgfpathlineto{\pgfqpoint{2.250876in}{2.476589in}}%
\pgfpathlineto{\pgfqpoint{2.251271in}{2.473309in}}%
\pgfpathlineto{\pgfqpoint{2.251665in}{2.479401in}}%
\pgfpathlineto{\pgfqpoint{2.252060in}{2.474450in}}%
\pgfpathlineto{\pgfqpoint{2.252258in}{2.478275in}}%
\pgfpathlineto{\pgfqpoint{2.252554in}{2.464902in}}%
\pgfpathlineto{\pgfqpoint{2.252850in}{2.450775in}}%
\pgfpathlineto{\pgfqpoint{2.253245in}{2.471561in}}%
\pgfpathlineto{\pgfqpoint{2.253541in}{2.469767in}}%
\pgfpathlineto{\pgfqpoint{2.253837in}{2.483200in}}%
\pgfpathlineto{\pgfqpoint{2.254330in}{2.460583in}}%
\pgfpathlineto{\pgfqpoint{2.256304in}{2.425087in}}%
\pgfpathlineto{\pgfqpoint{2.256798in}{2.441740in}}%
\pgfpathlineto{\pgfqpoint{2.256896in}{2.442168in}}%
\pgfpathlineto{\pgfqpoint{2.256995in}{2.439851in}}%
\pgfpathlineto{\pgfqpoint{2.258081in}{2.416417in}}%
\pgfpathlineto{\pgfqpoint{2.258377in}{2.424940in}}%
\pgfpathlineto{\pgfqpoint{2.258476in}{2.425224in}}%
\pgfpathlineto{\pgfqpoint{2.258574in}{2.423071in}}%
\pgfpathlineto{\pgfqpoint{2.259660in}{2.403157in}}%
\pgfpathlineto{\pgfqpoint{2.259857in}{2.408955in}}%
\pgfpathlineto{\pgfqpoint{2.260055in}{2.413594in}}%
\pgfpathlineto{\pgfqpoint{2.260450in}{2.405225in}}%
\pgfpathlineto{\pgfqpoint{2.260943in}{2.409104in}}%
\pgfpathlineto{\pgfqpoint{2.261733in}{2.401684in}}%
\pgfpathlineto{\pgfqpoint{2.262226in}{2.405091in}}%
\pgfpathlineto{\pgfqpoint{2.262424in}{2.403363in}}%
\pgfpathlineto{\pgfqpoint{2.262818in}{2.409466in}}%
\pgfpathlineto{\pgfqpoint{2.265977in}{2.460105in}}%
\pgfpathlineto{\pgfqpoint{2.266174in}{2.452773in}}%
\pgfpathlineto{\pgfqpoint{2.266668in}{2.459222in}}%
\pgfpathlineto{\pgfqpoint{2.269036in}{2.392830in}}%
\pgfpathlineto{\pgfqpoint{2.269332in}{2.404736in}}%
\pgfpathlineto{\pgfqpoint{2.269530in}{2.412578in}}%
\pgfpathlineto{\pgfqpoint{2.270122in}{2.387793in}}%
\pgfpathlineto{\pgfqpoint{2.270616in}{2.364082in}}%
\pgfpathlineto{\pgfqpoint{2.270813in}{2.357350in}}%
\pgfpathlineto{\pgfqpoint{2.271701in}{2.363577in}}%
\pgfpathlineto{\pgfqpoint{2.272688in}{2.401470in}}%
\pgfpathlineto{\pgfqpoint{2.273280in}{2.382917in}}%
\pgfpathlineto{\pgfqpoint{2.274662in}{2.342362in}}%
\pgfpathlineto{\pgfqpoint{2.274860in}{2.346532in}}%
\pgfpathlineto{\pgfqpoint{2.276143in}{2.454383in}}%
\pgfpathlineto{\pgfqpoint{2.276932in}{2.584767in}}%
\pgfpathlineto{\pgfqpoint{2.277722in}{2.574862in}}%
\pgfpathlineto{\pgfqpoint{2.277919in}{2.580746in}}%
\pgfpathlineto{\pgfqpoint{2.278215in}{2.563961in}}%
\pgfpathlineto{\pgfqpoint{2.279104in}{2.328472in}}%
\pgfpathlineto{\pgfqpoint{2.280485in}{2.396328in}}%
\pgfpathlineto{\pgfqpoint{2.280880in}{2.380780in}}%
\pgfpathlineto{\pgfqpoint{2.281374in}{2.400564in}}%
\pgfpathlineto{\pgfqpoint{2.282459in}{2.445737in}}%
\pgfpathlineto{\pgfqpoint{2.282756in}{2.427611in}}%
\pgfpathlineto{\pgfqpoint{2.283052in}{2.406281in}}%
\pgfpathlineto{\pgfqpoint{2.283940in}{2.411446in}}%
\pgfpathlineto{\pgfqpoint{2.284335in}{2.406254in}}%
\pgfpathlineto{\pgfqpoint{2.284532in}{2.409130in}}%
\pgfpathlineto{\pgfqpoint{2.285519in}{2.448076in}}%
\pgfpathlineto{\pgfqpoint{2.285914in}{2.427301in}}%
\pgfpathlineto{\pgfqpoint{2.287000in}{2.404801in}}%
\pgfpathlineto{\pgfqpoint{2.286407in}{2.428754in}}%
\pgfpathlineto{\pgfqpoint{2.287296in}{2.412158in}}%
\pgfpathlineto{\pgfqpoint{2.288776in}{2.458924in}}%
\pgfpathlineto{\pgfqpoint{2.289467in}{2.449674in}}%
\pgfpathlineto{\pgfqpoint{2.292132in}{2.386897in}}%
\pgfpathlineto{\pgfqpoint{2.292921in}{2.394718in}}%
\pgfpathlineto{\pgfqpoint{2.294895in}{2.495312in}}%
\pgfpathlineto{\pgfqpoint{2.295093in}{2.483540in}}%
\pgfpathlineto{\pgfqpoint{2.295389in}{2.464200in}}%
\pgfpathlineto{\pgfqpoint{2.295882in}{2.484227in}}%
\pgfpathlineto{\pgfqpoint{2.296376in}{2.465503in}}%
\pgfpathlineto{\pgfqpoint{2.296475in}{2.464838in}}%
\pgfpathlineto{\pgfqpoint{2.296672in}{2.468960in}}%
\pgfpathlineto{\pgfqpoint{2.296771in}{2.470560in}}%
\pgfpathlineto{\pgfqpoint{2.297067in}{2.458229in}}%
\pgfpathlineto{\pgfqpoint{2.297264in}{2.453976in}}%
\pgfpathlineto{\pgfqpoint{2.297659in}{2.479660in}}%
\pgfpathlineto{\pgfqpoint{2.297758in}{2.480491in}}%
\pgfpathlineto{\pgfqpoint{2.297955in}{2.475399in}}%
\pgfpathlineto{\pgfqpoint{2.300423in}{2.435153in}}%
\pgfpathlineto{\pgfqpoint{2.300620in}{2.439079in}}%
\pgfpathlineto{\pgfqpoint{2.301015in}{2.449094in}}%
\pgfpathlineto{\pgfqpoint{2.301607in}{2.437468in}}%
\pgfpathlineto{\pgfqpoint{2.303384in}{2.415617in}}%
\pgfpathlineto{\pgfqpoint{2.303976in}{2.417777in}}%
\pgfpathlineto{\pgfqpoint{2.304371in}{2.433362in}}%
\pgfpathlineto{\pgfqpoint{2.305061in}{2.417098in}}%
\pgfpathlineto{\pgfqpoint{2.305358in}{2.408296in}}%
\pgfpathlineto{\pgfqpoint{2.306048in}{2.415278in}}%
\pgfpathlineto{\pgfqpoint{2.307430in}{2.431881in}}%
\pgfpathlineto{\pgfqpoint{2.307726in}{2.428158in}}%
\pgfpathlineto{\pgfqpoint{2.308516in}{2.417472in}}%
\pgfpathlineto{\pgfqpoint{2.308121in}{2.429147in}}%
\pgfpathlineto{\pgfqpoint{2.308713in}{2.422512in}}%
\pgfpathlineto{\pgfqpoint{2.309700in}{2.462587in}}%
\pgfpathlineto{\pgfqpoint{2.310293in}{2.455912in}}%
\pgfpathlineto{\pgfqpoint{2.312266in}{2.478175in}}%
\pgfpathlineto{\pgfqpoint{2.316609in}{2.385430in}}%
\pgfpathlineto{\pgfqpoint{2.317300in}{2.377988in}}%
\pgfpathlineto{\pgfqpoint{2.317498in}{2.382377in}}%
\pgfpathlineto{\pgfqpoint{2.318583in}{2.408849in}}%
\pgfpathlineto{\pgfqpoint{2.318879in}{2.402212in}}%
\pgfpathlineto{\pgfqpoint{2.319274in}{2.389329in}}%
\pgfpathlineto{\pgfqpoint{2.319669in}{2.372108in}}%
\pgfpathlineto{\pgfqpoint{2.320459in}{2.373556in}}%
\pgfpathlineto{\pgfqpoint{2.321939in}{2.453217in}}%
\pgfpathlineto{\pgfqpoint{2.322827in}{2.613226in}}%
\pgfpathlineto{\pgfqpoint{2.323617in}{2.597752in}}%
\pgfpathlineto{\pgfqpoint{2.323913in}{2.602147in}}%
\pgfpathlineto{\pgfqpoint{2.324012in}{2.599535in}}%
\pgfpathlineto{\pgfqpoint{2.324604in}{2.453185in}}%
\pgfpathlineto{\pgfqpoint{2.324999in}{2.352541in}}%
\pgfpathlineto{\pgfqpoint{2.325788in}{2.420224in}}%
\pgfpathlineto{\pgfqpoint{2.325986in}{2.414228in}}%
\pgfpathlineto{\pgfqpoint{2.326775in}{2.421282in}}%
\pgfpathlineto{\pgfqpoint{2.328256in}{2.478576in}}%
\pgfpathlineto{\pgfqpoint{2.327269in}{2.420024in}}%
\pgfpathlineto{\pgfqpoint{2.329045in}{2.453038in}}%
\pgfpathlineto{\pgfqpoint{2.329440in}{2.439204in}}%
\pgfpathlineto{\pgfqpoint{2.330230in}{2.446778in}}%
\pgfpathlineto{\pgfqpoint{2.330723in}{2.464426in}}%
\pgfpathlineto{\pgfqpoint{2.331019in}{2.474888in}}%
\pgfpathlineto{\pgfqpoint{2.331809in}{2.471013in}}%
\pgfpathlineto{\pgfqpoint{2.333191in}{2.450172in}}%
\pgfpathlineto{\pgfqpoint{2.333289in}{2.451855in}}%
\pgfpathlineto{\pgfqpoint{2.335066in}{2.492349in}}%
\pgfpathlineto{\pgfqpoint{2.336744in}{2.464990in}}%
\pgfpathlineto{\pgfqpoint{2.337040in}{2.476017in}}%
\pgfpathlineto{\pgfqpoint{2.338323in}{2.493103in}}%
\pgfpathlineto{\pgfqpoint{2.339705in}{2.469338in}}%
\pgfpathlineto{\pgfqpoint{2.340001in}{2.476558in}}%
\pgfpathlineto{\pgfqpoint{2.340790in}{2.493439in}}%
\pgfpathlineto{\pgfqpoint{2.341185in}{2.483652in}}%
\pgfpathlineto{\pgfqpoint{2.343061in}{2.456907in}}%
\pgfpathlineto{\pgfqpoint{2.341679in}{2.487089in}}%
\pgfpathlineto{\pgfqpoint{2.343357in}{2.473026in}}%
\pgfpathlineto{\pgfqpoint{2.344048in}{2.484388in}}%
\pgfpathlineto{\pgfqpoint{2.344541in}{2.479415in}}%
\pgfpathlineto{\pgfqpoint{2.345528in}{2.471334in}}%
\pgfpathlineto{\pgfqpoint{2.347403in}{2.443234in}}%
\pgfpathlineto{\pgfqpoint{2.347601in}{2.440522in}}%
\pgfpathlineto{\pgfqpoint{2.349377in}{2.396924in}}%
\pgfpathlineto{\pgfqpoint{2.349476in}{2.399362in}}%
\pgfpathlineto{\pgfqpoint{2.349871in}{2.414143in}}%
\pgfpathlineto{\pgfqpoint{2.350660in}{2.406889in}}%
\pgfpathlineto{\pgfqpoint{2.351746in}{2.388207in}}%
\pgfpathlineto{\pgfqpoint{2.352042in}{2.393686in}}%
\pgfpathlineto{\pgfqpoint{2.352338in}{2.398324in}}%
\pgfpathlineto{\pgfqpoint{2.352832in}{2.389920in}}%
\pgfpathlineto{\pgfqpoint{2.353029in}{2.387663in}}%
\pgfpathlineto{\pgfqpoint{2.353523in}{2.396264in}}%
\pgfpathlineto{\pgfqpoint{2.353720in}{2.400066in}}%
\pgfpathlineto{\pgfqpoint{2.354214in}{2.386892in}}%
\pgfpathlineto{\pgfqpoint{2.354411in}{2.385591in}}%
\pgfpathlineto{\pgfqpoint{2.354707in}{2.392000in}}%
\pgfpathlineto{\pgfqpoint{2.356681in}{2.437630in}}%
\pgfpathlineto{\pgfqpoint{2.358458in}{2.453449in}}%
\pgfpathlineto{\pgfqpoint{2.358754in}{2.449646in}}%
\pgfpathlineto{\pgfqpoint{2.362504in}{2.363173in}}%
\pgfpathlineto{\pgfqpoint{2.362800in}{2.375901in}}%
\pgfpathlineto{\pgfqpoint{2.363689in}{2.388113in}}%
\pgfpathlineto{\pgfqpoint{2.363294in}{2.375207in}}%
\pgfpathlineto{\pgfqpoint{2.363985in}{2.380303in}}%
\pgfpathlineto{\pgfqpoint{2.364281in}{2.391316in}}%
\pgfpathlineto{\pgfqpoint{2.364774in}{2.414736in}}%
\pgfpathlineto{\pgfqpoint{2.365268in}{2.392140in}}%
\pgfpathlineto{\pgfqpoint{2.365761in}{2.373736in}}%
\pgfpathlineto{\pgfqpoint{2.366650in}{2.379056in}}%
\pgfpathlineto{\pgfqpoint{2.367834in}{2.420565in}}%
\pgfpathlineto{\pgfqpoint{2.369018in}{2.614337in}}%
\pgfpathlineto{\pgfqpoint{2.369907in}{2.603213in}}%
\pgfpathlineto{\pgfqpoint{2.370598in}{2.483717in}}%
\pgfpathlineto{\pgfqpoint{2.371091in}{2.345157in}}%
\pgfpathlineto{\pgfqpoint{2.371782in}{2.412051in}}%
\pgfpathlineto{\pgfqpoint{2.371881in}{2.415455in}}%
\pgfpathlineto{\pgfqpoint{2.372374in}{2.395489in}}%
\pgfpathlineto{\pgfqpoint{2.372670in}{2.398131in}}%
\pgfpathlineto{\pgfqpoint{2.372868in}{2.395837in}}%
\pgfpathlineto{\pgfqpoint{2.373460in}{2.376877in}}%
\pgfpathlineto{\pgfqpoint{2.373953in}{2.394781in}}%
\pgfpathlineto{\pgfqpoint{2.374151in}{2.396776in}}%
\pgfpathlineto{\pgfqpoint{2.374447in}{2.389479in}}%
\pgfpathlineto{\pgfqpoint{2.375335in}{2.347408in}}%
\pgfpathlineto{\pgfqpoint{2.375927in}{2.370376in}}%
\pgfpathlineto{\pgfqpoint{2.377901in}{2.465454in}}%
\pgfpathlineto{\pgfqpoint{2.378099in}{2.454443in}}%
\pgfpathlineto{\pgfqpoint{2.379480in}{2.426756in}}%
\pgfpathlineto{\pgfqpoint{2.379875in}{2.444584in}}%
\pgfpathlineto{\pgfqpoint{2.381356in}{2.481847in}}%
\pgfpathlineto{\pgfqpoint{2.380270in}{2.444375in}}%
\pgfpathlineto{\pgfqpoint{2.381751in}{2.466204in}}%
\pgfpathlineto{\pgfqpoint{2.382441in}{2.458692in}}%
\pgfpathlineto{\pgfqpoint{2.382047in}{2.467147in}}%
\pgfpathlineto{\pgfqpoint{2.382738in}{2.466289in}}%
\pgfpathlineto{\pgfqpoint{2.384317in}{2.493458in}}%
\pgfpathlineto{\pgfqpoint{2.383231in}{2.458946in}}%
\pgfpathlineto{\pgfqpoint{2.384909in}{2.490224in}}%
\pgfpathlineto{\pgfqpoint{2.385402in}{2.474439in}}%
\pgfpathlineto{\pgfqpoint{2.386883in}{2.479849in}}%
\pgfpathlineto{\pgfqpoint{2.386982in}{2.480027in}}%
\pgfpathlineto{\pgfqpoint{2.387179in}{2.478140in}}%
\pgfpathlineto{\pgfqpoint{2.387475in}{2.471096in}}%
\pgfpathlineto{\pgfqpoint{2.387969in}{2.486522in}}%
\pgfpathlineto{\pgfqpoint{2.388067in}{2.486952in}}%
\pgfpathlineto{\pgfqpoint{2.388166in}{2.484419in}}%
\pgfpathlineto{\pgfqpoint{2.388659in}{2.462341in}}%
\pgfpathlineto{\pgfqpoint{2.389449in}{2.469483in}}%
\pgfpathlineto{\pgfqpoint{2.390535in}{2.487336in}}%
\pgfpathlineto{\pgfqpoint{2.390732in}{2.477872in}}%
\pgfpathlineto{\pgfqpoint{2.391226in}{2.490479in}}%
\pgfpathlineto{\pgfqpoint{2.392311in}{2.460737in}}%
\pgfpathlineto{\pgfqpoint{2.392805in}{2.468294in}}%
\pgfpathlineto{\pgfqpoint{2.393101in}{2.461758in}}%
\pgfpathlineto{\pgfqpoint{2.394285in}{2.447461in}}%
\pgfpathlineto{\pgfqpoint{2.394384in}{2.447776in}}%
\pgfpathlineto{\pgfqpoint{2.394680in}{2.456889in}}%
\pgfpathlineto{\pgfqpoint{2.394976in}{2.437680in}}%
\pgfpathlineto{\pgfqpoint{2.396062in}{2.416764in}}%
\pgfpathlineto{\pgfqpoint{2.396161in}{2.418011in}}%
\pgfpathlineto{\pgfqpoint{2.396457in}{2.424768in}}%
\pgfpathlineto{\pgfqpoint{2.396753in}{2.410278in}}%
\pgfpathlineto{\pgfqpoint{2.397542in}{2.417332in}}%
\pgfpathlineto{\pgfqpoint{2.398036in}{2.393051in}}%
\pgfpathlineto{\pgfqpoint{2.398233in}{2.394100in}}%
\pgfpathlineto{\pgfqpoint{2.398431in}{2.391911in}}%
\pgfpathlineto{\pgfqpoint{2.398825in}{2.383282in}}%
\pgfpathlineto{\pgfqpoint{2.399418in}{2.392973in}}%
\pgfpathlineto{\pgfqpoint{2.400109in}{2.403210in}}%
\pgfpathlineto{\pgfqpoint{2.400405in}{2.393700in}}%
\pgfpathlineto{\pgfqpoint{2.400701in}{2.384665in}}%
\pgfpathlineto{\pgfqpoint{2.401096in}{2.398889in}}%
\pgfpathlineto{\pgfqpoint{2.401490in}{2.390111in}}%
\pgfpathlineto{\pgfqpoint{2.402181in}{2.408292in}}%
\pgfpathlineto{\pgfqpoint{2.404155in}{2.443885in}}%
\pgfpathlineto{\pgfqpoint{2.404254in}{2.445435in}}%
\pgfpathlineto{\pgfqpoint{2.404550in}{2.436504in}}%
\pgfpathlineto{\pgfqpoint{2.406228in}{2.391097in}}%
\pgfpathlineto{\pgfqpoint{2.407017in}{2.381544in}}%
\pgfpathlineto{\pgfqpoint{2.406623in}{2.392661in}}%
\pgfpathlineto{\pgfqpoint{2.407215in}{2.392478in}}%
\pgfpathlineto{\pgfqpoint{2.407412in}{2.404149in}}%
\pgfpathlineto{\pgfqpoint{2.408004in}{2.372358in}}%
\pgfpathlineto{\pgfqpoint{2.408498in}{2.352920in}}%
\pgfpathlineto{\pgfqpoint{2.408991in}{2.373020in}}%
\pgfpathlineto{\pgfqpoint{2.410768in}{2.420256in}}%
\pgfpathlineto{\pgfqpoint{2.411163in}{2.401859in}}%
\pgfpathlineto{\pgfqpoint{2.411558in}{2.385661in}}%
\pgfpathlineto{\pgfqpoint{2.412249in}{2.393362in}}%
\pgfpathlineto{\pgfqpoint{2.414617in}{2.590409in}}%
\pgfpathlineto{\pgfqpoint{2.415703in}{2.616841in}}%
\pgfpathlineto{\pgfqpoint{2.415900in}{2.615372in}}%
\pgfpathlineto{\pgfqpoint{2.416295in}{2.585125in}}%
\pgfpathlineto{\pgfqpoint{2.417183in}{2.352983in}}%
\pgfpathlineto{\pgfqpoint{2.418170in}{2.403641in}}%
\pgfpathlineto{\pgfqpoint{2.420342in}{2.460215in}}%
\pgfpathlineto{\pgfqpoint{2.420441in}{2.458842in}}%
\pgfpathlineto{\pgfqpoint{2.421428in}{2.422674in}}%
\pgfpathlineto{\pgfqpoint{2.422513in}{2.435893in}}%
\pgfpathlineto{\pgfqpoint{2.423698in}{2.464914in}}%
\pgfpathlineto{\pgfqpoint{2.424191in}{2.448721in}}%
\pgfpathlineto{\pgfqpoint{2.424882in}{2.442437in}}%
\pgfpathlineto{\pgfqpoint{2.425079in}{2.447493in}}%
\pgfpathlineto{\pgfqpoint{2.426560in}{2.476021in}}%
\pgfpathlineto{\pgfqpoint{2.425869in}{2.447211in}}%
\pgfpathlineto{\pgfqpoint{2.426659in}{2.475904in}}%
\pgfpathlineto{\pgfqpoint{2.426856in}{2.474673in}}%
\pgfpathlineto{\pgfqpoint{2.427152in}{2.479162in}}%
\pgfpathlineto{\pgfqpoint{2.429323in}{2.502123in}}%
\pgfpathlineto{\pgfqpoint{2.429620in}{2.496335in}}%
\pgfpathlineto{\pgfqpoint{2.430113in}{2.506662in}}%
\pgfpathlineto{\pgfqpoint{2.430508in}{2.508622in}}%
\pgfpathlineto{\pgfqpoint{2.431396in}{2.504542in}}%
\pgfpathlineto{\pgfqpoint{2.431791in}{2.514926in}}%
\pgfpathlineto{\pgfqpoint{2.432778in}{2.490536in}}%
\pgfpathlineto{\pgfqpoint{2.433469in}{2.495340in}}%
\pgfpathlineto{\pgfqpoint{2.433765in}{2.514966in}}%
\pgfpathlineto{\pgfqpoint{2.434653in}{2.503987in}}%
\pgfpathlineto{\pgfqpoint{2.435048in}{2.500719in}}%
\pgfpathlineto{\pgfqpoint{2.435541in}{2.504741in}}%
\pgfpathlineto{\pgfqpoint{2.435838in}{2.509861in}}%
\pgfpathlineto{\pgfqpoint{2.436035in}{2.499281in}}%
\pgfpathlineto{\pgfqpoint{2.436232in}{2.485921in}}%
\pgfpathlineto{\pgfqpoint{2.436726in}{2.509592in}}%
\pgfpathlineto{\pgfqpoint{2.437022in}{2.500474in}}%
\pgfpathlineto{\pgfqpoint{2.437417in}{2.514214in}}%
\pgfpathlineto{\pgfqpoint{2.437713in}{2.499439in}}%
\pgfpathlineto{\pgfqpoint{2.438404in}{2.479374in}}%
\pgfpathlineto{\pgfqpoint{2.438996in}{2.485534in}}%
\pgfpathlineto{\pgfqpoint{2.439489in}{2.482268in}}%
\pgfpathlineto{\pgfqpoint{2.442154in}{2.441806in}}%
\pgfpathlineto{\pgfqpoint{2.442845in}{2.450589in}}%
\pgfpathlineto{\pgfqpoint{2.443931in}{2.467195in}}%
\pgfpathlineto{\pgfqpoint{2.444030in}{2.464368in}}%
\pgfpathlineto{\pgfqpoint{2.445017in}{2.443448in}}%
\pgfpathlineto{\pgfqpoint{2.445313in}{2.451466in}}%
\pgfpathlineto{\pgfqpoint{2.445411in}{2.453096in}}%
\pgfpathlineto{\pgfqpoint{2.445806in}{2.441574in}}%
\pgfpathlineto{\pgfqpoint{2.445905in}{2.441541in}}%
\pgfpathlineto{\pgfqpoint{2.447188in}{2.472725in}}%
\pgfpathlineto{\pgfqpoint{2.447385in}{2.463843in}}%
\pgfpathlineto{\pgfqpoint{2.447583in}{2.451357in}}%
\pgfpathlineto{\pgfqpoint{2.448076in}{2.470839in}}%
\pgfpathlineto{\pgfqpoint{2.448372in}{2.466975in}}%
\pgfpathlineto{\pgfqpoint{2.450445in}{2.519151in}}%
\pgfpathlineto{\pgfqpoint{2.450741in}{2.510372in}}%
\pgfpathlineto{\pgfqpoint{2.452320in}{2.465268in}}%
\pgfpathlineto{\pgfqpoint{2.452715in}{2.467462in}}%
\pgfpathlineto{\pgfqpoint{2.453899in}{2.428148in}}%
\pgfpathlineto{\pgfqpoint{2.455577in}{2.370687in}}%
\pgfpathlineto{\pgfqpoint{2.455775in}{2.370175in}}%
\pgfpathlineto{\pgfqpoint{2.455873in}{2.370740in}}%
\pgfpathlineto{\pgfqpoint{2.456367in}{2.420169in}}%
\pgfpathlineto{\pgfqpoint{2.457157in}{2.478358in}}%
\pgfpathlineto{\pgfqpoint{2.457650in}{2.453876in}}%
\pgfpathlineto{\pgfqpoint{2.457847in}{2.447919in}}%
\pgfpathlineto{\pgfqpoint{2.458242in}{2.469322in}}%
\pgfpathlineto{\pgfqpoint{2.458440in}{2.477436in}}%
\pgfpathlineto{\pgfqpoint{2.459131in}{2.459347in}}%
\pgfpathlineto{\pgfqpoint{2.459328in}{2.449989in}}%
\pgfpathlineto{\pgfqpoint{2.459723in}{2.477811in}}%
\pgfpathlineto{\pgfqpoint{2.461697in}{2.706475in}}%
\pgfpathlineto{\pgfqpoint{2.462684in}{2.639925in}}%
\pgfpathlineto{\pgfqpoint{2.463473in}{2.411353in}}%
\pgfpathlineto{\pgfqpoint{2.464362in}{2.454270in}}%
\pgfpathlineto{\pgfqpoint{2.464559in}{2.459217in}}%
\pgfpathlineto{\pgfqpoint{2.466533in}{2.515325in}}%
\pgfpathlineto{\pgfqpoint{2.467421in}{2.469907in}}%
\pgfpathlineto{\pgfqpoint{2.468211in}{2.491329in}}%
\pgfpathlineto{\pgfqpoint{2.468408in}{2.499475in}}%
\pgfpathlineto{\pgfqpoint{2.468803in}{2.478802in}}%
\pgfpathlineto{\pgfqpoint{2.469395in}{2.498296in}}%
\pgfpathlineto{\pgfqpoint{2.469889in}{2.522306in}}%
\pgfpathlineto{\pgfqpoint{2.470382in}{2.499596in}}%
\pgfpathlineto{\pgfqpoint{2.470678in}{2.493612in}}%
\pgfpathlineto{\pgfqpoint{2.471073in}{2.510109in}}%
\pgfpathlineto{\pgfqpoint{2.473146in}{2.537624in}}%
\pgfpathlineto{\pgfqpoint{2.471468in}{2.508673in}}%
\pgfpathlineto{\pgfqpoint{2.473244in}{2.535057in}}%
\pgfpathlineto{\pgfqpoint{2.473541in}{2.524109in}}%
\pgfpathlineto{\pgfqpoint{2.474330in}{2.529236in}}%
\pgfpathlineto{\pgfqpoint{2.475218in}{2.541114in}}%
\pgfpathlineto{\pgfqpoint{2.475515in}{2.533165in}}%
\pgfpathlineto{\pgfqpoint{2.475712in}{2.535330in}}%
\pgfpathlineto{\pgfqpoint{2.476502in}{2.542912in}}%
\pgfpathlineto{\pgfqpoint{2.476798in}{2.537184in}}%
\pgfpathlineto{\pgfqpoint{2.477094in}{2.529496in}}%
\pgfpathlineto{\pgfqpoint{2.477587in}{2.547160in}}%
\pgfpathlineto{\pgfqpoint{2.477686in}{2.548906in}}%
\pgfpathlineto{\pgfqpoint{2.478278in}{2.539842in}}%
\pgfpathlineto{\pgfqpoint{2.478870in}{2.516965in}}%
\pgfpathlineto{\pgfqpoint{2.479561in}{2.533516in}}%
\pgfpathlineto{\pgfqpoint{2.479857in}{2.529875in}}%
\pgfpathlineto{\pgfqpoint{2.480153in}{2.539115in}}%
\pgfpathlineto{\pgfqpoint{2.480252in}{2.539697in}}%
\pgfpathlineto{\pgfqpoint{2.480351in}{2.537333in}}%
\pgfpathlineto{\pgfqpoint{2.481338in}{2.521439in}}%
\pgfpathlineto{\pgfqpoint{2.480943in}{2.537805in}}%
\pgfpathlineto{\pgfqpoint{2.481634in}{2.527067in}}%
\pgfpathlineto{\pgfqpoint{2.481831in}{2.522620in}}%
\pgfpathlineto{\pgfqpoint{2.482127in}{2.512736in}}%
\pgfpathlineto{\pgfqpoint{2.482917in}{2.520336in}}%
\pgfpathlineto{\pgfqpoint{2.483312in}{2.527795in}}%
\pgfpathlineto{\pgfqpoint{2.483805in}{2.514617in}}%
\pgfpathlineto{\pgfqpoint{2.484003in}{2.517035in}}%
\pgfpathlineto{\pgfqpoint{2.484200in}{2.512265in}}%
\pgfpathlineto{\pgfqpoint{2.485483in}{2.475388in}}%
\pgfpathlineto{\pgfqpoint{2.485779in}{2.482719in}}%
\pgfpathlineto{\pgfqpoint{2.485878in}{2.483745in}}%
\pgfpathlineto{\pgfqpoint{2.486174in}{2.477576in}}%
\pgfpathlineto{\pgfqpoint{2.488938in}{2.431755in}}%
\pgfpathlineto{\pgfqpoint{2.489036in}{2.432325in}}%
\pgfpathlineto{\pgfqpoint{2.489431in}{2.445840in}}%
\pgfpathlineto{\pgfqpoint{2.490319in}{2.441116in}}%
\pgfpathlineto{\pgfqpoint{2.491504in}{2.418309in}}%
\pgfpathlineto{\pgfqpoint{2.491800in}{2.428066in}}%
\pgfpathlineto{\pgfqpoint{2.492688in}{2.447747in}}%
\pgfpathlineto{\pgfqpoint{2.493083in}{2.436563in}}%
\pgfpathlineto{\pgfqpoint{2.493774in}{2.434116in}}%
\pgfpathlineto{\pgfqpoint{2.495452in}{2.459718in}}%
\pgfpathlineto{\pgfqpoint{2.496932in}{2.490105in}}%
\pgfpathlineto{\pgfqpoint{2.497327in}{2.478341in}}%
\pgfpathlineto{\pgfqpoint{2.502065in}{2.396060in}}%
\pgfpathlineto{\pgfqpoint{2.502163in}{2.397161in}}%
\pgfpathlineto{\pgfqpoint{2.503249in}{2.428057in}}%
\pgfpathlineto{\pgfqpoint{2.503742in}{2.409734in}}%
\pgfpathlineto{\pgfqpoint{2.505322in}{2.381176in}}%
\pgfpathlineto{\pgfqpoint{2.505914in}{2.406006in}}%
\pgfpathlineto{\pgfqpoint{2.507789in}{2.622910in}}%
\pgfpathlineto{\pgfqpoint{2.508973in}{2.596487in}}%
\pgfpathlineto{\pgfqpoint{2.509862in}{2.346439in}}%
\pgfpathlineto{\pgfqpoint{2.510947in}{2.414947in}}%
\pgfpathlineto{\pgfqpoint{2.511244in}{2.434704in}}%
\pgfpathlineto{\pgfqpoint{2.512132in}{2.422292in}}%
\pgfpathlineto{\pgfqpoint{2.512724in}{2.463826in}}%
\pgfpathlineto{\pgfqpoint{2.512921in}{2.469852in}}%
\pgfpathlineto{\pgfqpoint{2.513316in}{2.445079in}}%
\pgfpathlineto{\pgfqpoint{2.514106in}{2.423979in}}%
\pgfpathlineto{\pgfqpoint{2.514501in}{2.437211in}}%
\pgfpathlineto{\pgfqpoint{2.514895in}{2.435648in}}%
\pgfpathlineto{\pgfqpoint{2.515488in}{2.448274in}}%
\pgfpathlineto{\pgfqpoint{2.516080in}{2.474946in}}%
\pgfpathlineto{\pgfqpoint{2.516672in}{2.462838in}}%
\pgfpathlineto{\pgfqpoint{2.517758in}{2.454128in}}%
\pgfpathlineto{\pgfqpoint{2.517363in}{2.468303in}}%
\pgfpathlineto{\pgfqpoint{2.517856in}{2.455047in}}%
\pgfpathlineto{\pgfqpoint{2.518251in}{2.467801in}}%
\pgfpathlineto{\pgfqpoint{2.519238in}{2.467745in}}%
\pgfpathlineto{\pgfqpoint{2.519534in}{2.458493in}}%
\pgfpathlineto{\pgfqpoint{2.519732in}{2.454386in}}%
\pgfpathlineto{\pgfqpoint{2.520126in}{2.467954in}}%
\pgfpathlineto{\pgfqpoint{2.520423in}{2.465599in}}%
\pgfpathlineto{\pgfqpoint{2.521706in}{2.480149in}}%
\pgfpathlineto{\pgfqpoint{2.521804in}{2.478248in}}%
\pgfpathlineto{\pgfqpoint{2.522199in}{2.467469in}}%
\pgfpathlineto{\pgfqpoint{2.522890in}{2.475733in}}%
\pgfpathlineto{\pgfqpoint{2.524371in}{2.484444in}}%
\pgfpathlineto{\pgfqpoint{2.524469in}{2.480666in}}%
\pgfpathlineto{\pgfqpoint{2.524864in}{2.461944in}}%
\pgfpathlineto{\pgfqpoint{2.525752in}{2.468205in}}%
\pgfpathlineto{\pgfqpoint{2.526838in}{2.456082in}}%
\pgfpathlineto{\pgfqpoint{2.527035in}{2.459695in}}%
\pgfpathlineto{\pgfqpoint{2.528417in}{2.473848in}}%
\pgfpathlineto{\pgfqpoint{2.529009in}{2.474958in}}%
\pgfpathlineto{\pgfqpoint{2.529207in}{2.469724in}}%
\pgfpathlineto{\pgfqpoint{2.529503in}{2.462995in}}%
\pgfpathlineto{\pgfqpoint{2.530292in}{2.469753in}}%
\pgfpathlineto{\pgfqpoint{2.530490in}{2.473114in}}%
\pgfpathlineto{\pgfqpoint{2.530983in}{2.463493in}}%
\pgfpathlineto{\pgfqpoint{2.534438in}{2.336956in}}%
\pgfpathlineto{\pgfqpoint{2.535227in}{2.356003in}}%
\pgfpathlineto{\pgfqpoint{2.536708in}{2.422724in}}%
\pgfpathlineto{\pgfqpoint{2.537004in}{2.416593in}}%
\pgfpathlineto{\pgfqpoint{2.537103in}{2.415236in}}%
\pgfpathlineto{\pgfqpoint{2.537399in}{2.424074in}}%
\pgfpathlineto{\pgfqpoint{2.537497in}{2.426113in}}%
\pgfpathlineto{\pgfqpoint{2.537695in}{2.416924in}}%
\pgfpathlineto{\pgfqpoint{2.538090in}{2.392592in}}%
\pgfpathlineto{\pgfqpoint{2.538781in}{2.415667in}}%
\pgfpathlineto{\pgfqpoint{2.539077in}{2.410566in}}%
\pgfpathlineto{\pgfqpoint{2.539373in}{2.405944in}}%
\pgfpathlineto{\pgfqpoint{2.540064in}{2.409508in}}%
\pgfpathlineto{\pgfqpoint{2.543123in}{2.453942in}}%
\pgfpathlineto{\pgfqpoint{2.543222in}{2.453402in}}%
\pgfpathlineto{\pgfqpoint{2.545788in}{2.405547in}}%
\pgfpathlineto{\pgfqpoint{2.545986in}{2.409579in}}%
\pgfpathlineto{\pgfqpoint{2.546282in}{2.420124in}}%
\pgfpathlineto{\pgfqpoint{2.546676in}{2.391564in}}%
\pgfpathlineto{\pgfqpoint{2.547565in}{2.368066in}}%
\pgfpathlineto{\pgfqpoint{2.548157in}{2.375741in}}%
\pgfpathlineto{\pgfqpoint{2.549440in}{2.424823in}}%
\pgfpathlineto{\pgfqpoint{2.550131in}{2.398951in}}%
\pgfpathlineto{\pgfqpoint{2.550526in}{2.379527in}}%
\pgfpathlineto{\pgfqpoint{2.551217in}{2.395309in}}%
\pgfpathlineto{\pgfqpoint{2.552796in}{2.461117in}}%
\pgfpathlineto{\pgfqpoint{2.553684in}{2.617223in}}%
\pgfpathlineto{\pgfqpoint{2.554572in}{2.600097in}}%
\pgfpathlineto{\pgfqpoint{2.554770in}{2.606511in}}%
\pgfpathlineto{\pgfqpoint{2.554967in}{2.617075in}}%
\pgfpathlineto{\pgfqpoint{2.555263in}{2.567899in}}%
\pgfpathlineto{\pgfqpoint{2.556053in}{2.329473in}}%
\pgfpathlineto{\pgfqpoint{2.556941in}{2.388086in}}%
\pgfpathlineto{\pgfqpoint{2.557040in}{2.387546in}}%
\pgfpathlineto{\pgfqpoint{2.557139in}{2.389803in}}%
\pgfpathlineto{\pgfqpoint{2.558915in}{2.452374in}}%
\pgfpathlineto{\pgfqpoint{2.559409in}{2.441110in}}%
\pgfpathlineto{\pgfqpoint{2.560297in}{2.411816in}}%
\pgfpathlineto{\pgfqpoint{2.560692in}{2.429256in}}%
\pgfpathlineto{\pgfqpoint{2.562468in}{2.466699in}}%
\pgfpathlineto{\pgfqpoint{2.561284in}{2.427453in}}%
\pgfpathlineto{\pgfqpoint{2.562567in}{2.464161in}}%
\pgfpathlineto{\pgfqpoint{2.563554in}{2.438381in}}%
\pgfpathlineto{\pgfqpoint{2.563850in}{2.448036in}}%
\pgfpathlineto{\pgfqpoint{2.564837in}{2.465042in}}%
\pgfpathlineto{\pgfqpoint{2.565232in}{2.463735in}}%
\pgfpathlineto{\pgfqpoint{2.565725in}{2.455050in}}%
\pgfpathlineto{\pgfqpoint{2.566219in}{2.462446in}}%
\pgfpathlineto{\pgfqpoint{2.566515in}{2.472633in}}%
\pgfpathlineto{\pgfqpoint{2.567008in}{2.452522in}}%
\pgfpathlineto{\pgfqpoint{2.567107in}{2.452312in}}%
\pgfpathlineto{\pgfqpoint{2.567601in}{2.473077in}}%
\pgfpathlineto{\pgfqpoint{2.568094in}{2.451636in}}%
\pgfpathlineto{\pgfqpoint{2.568489in}{2.464241in}}%
\pgfpathlineto{\pgfqpoint{2.569377in}{2.457831in}}%
\pgfpathlineto{\pgfqpoint{2.569673in}{2.462687in}}%
\pgfpathlineto{\pgfqpoint{2.570660in}{2.466274in}}%
\pgfpathlineto{\pgfqpoint{2.570858in}{2.465077in}}%
\pgfpathlineto{\pgfqpoint{2.571549in}{2.466486in}}%
\pgfpathlineto{\pgfqpoint{2.573029in}{2.448799in}}%
\pgfpathlineto{\pgfqpoint{2.573523in}{2.457390in}}%
\pgfpathlineto{\pgfqpoint{2.574115in}{2.448476in}}%
\pgfpathlineto{\pgfqpoint{2.574510in}{2.434008in}}%
\pgfpathlineto{\pgfqpoint{2.575398in}{2.438042in}}%
\pgfpathlineto{\pgfqpoint{2.576089in}{2.449645in}}%
\pgfpathlineto{\pgfqpoint{2.576878in}{2.442465in}}%
\pgfpathlineto{\pgfqpoint{2.578161in}{2.418971in}}%
\pgfpathlineto{\pgfqpoint{2.578359in}{2.419355in}}%
\pgfpathlineto{\pgfqpoint{2.578852in}{2.397063in}}%
\pgfpathlineto{\pgfqpoint{2.579642in}{2.412098in}}%
\pgfpathlineto{\pgfqpoint{2.580530in}{2.392874in}}%
\pgfpathlineto{\pgfqpoint{2.580826in}{2.405054in}}%
\pgfpathlineto{\pgfqpoint{2.580925in}{2.408395in}}%
\pgfpathlineto{\pgfqpoint{2.581320in}{2.384403in}}%
\pgfpathlineto{\pgfqpoint{2.581419in}{2.382221in}}%
\pgfpathlineto{\pgfqpoint{2.581813in}{2.391877in}}%
\pgfpathlineto{\pgfqpoint{2.582011in}{2.391218in}}%
\pgfpathlineto{\pgfqpoint{2.582406in}{2.408271in}}%
\pgfpathlineto{\pgfqpoint{2.583294in}{2.400339in}}%
\pgfpathlineto{\pgfqpoint{2.583985in}{2.385781in}}%
\pgfpathlineto{\pgfqpoint{2.584182in}{2.382480in}}%
\pgfpathlineto{\pgfqpoint{2.584676in}{2.394130in}}%
\pgfpathlineto{\pgfqpoint{2.584774in}{2.394748in}}%
\pgfpathlineto{\pgfqpoint{2.585070in}{2.390203in}}%
\pgfpathlineto{\pgfqpoint{2.585169in}{2.389971in}}%
\pgfpathlineto{\pgfqpoint{2.585268in}{2.391460in}}%
\pgfpathlineto{\pgfqpoint{2.586452in}{2.411748in}}%
\pgfpathlineto{\pgfqpoint{2.586946in}{2.408399in}}%
\pgfpathlineto{\pgfqpoint{2.587242in}{2.405020in}}%
\pgfpathlineto{\pgfqpoint{2.587538in}{2.412308in}}%
\pgfpathlineto{\pgfqpoint{2.589314in}{2.460036in}}%
\pgfpathlineto{\pgfqpoint{2.589413in}{2.459602in}}%
\pgfpathlineto{\pgfqpoint{2.591881in}{2.391062in}}%
\pgfpathlineto{\pgfqpoint{2.592078in}{2.395060in}}%
\pgfpathlineto{\pgfqpoint{2.592275in}{2.402129in}}%
\pgfpathlineto{\pgfqpoint{2.592868in}{2.388055in}}%
\pgfpathlineto{\pgfqpoint{2.593460in}{2.358558in}}%
\pgfpathlineto{\pgfqpoint{2.594545in}{2.366321in}}%
\pgfpathlineto{\pgfqpoint{2.594842in}{2.361082in}}%
\pgfpathlineto{\pgfqpoint{2.595138in}{2.372345in}}%
\pgfpathlineto{\pgfqpoint{2.596026in}{2.412044in}}%
\pgfpathlineto{\pgfqpoint{2.596421in}{2.384112in}}%
\pgfpathlineto{\pgfqpoint{2.596816in}{2.346378in}}%
\pgfpathlineto{\pgfqpoint{2.597605in}{2.365906in}}%
\pgfpathlineto{\pgfqpoint{2.598691in}{2.383460in}}%
\pgfpathlineto{\pgfqpoint{2.599579in}{2.510864in}}%
\pgfpathlineto{\pgfqpoint{2.600171in}{2.584796in}}%
\pgfpathlineto{\pgfqpoint{2.600862in}{2.568192in}}%
\pgfpathlineto{\pgfqpoint{2.601158in}{2.578051in}}%
\pgfpathlineto{\pgfqpoint{2.601454in}{2.560513in}}%
\pgfpathlineto{\pgfqpoint{2.602441in}{2.313304in}}%
\pgfpathlineto{\pgfqpoint{2.603626in}{2.382024in}}%
\pgfpathlineto{\pgfqpoint{2.605205in}{2.416027in}}%
\pgfpathlineto{\pgfqpoint{2.605501in}{2.427235in}}%
\pgfpathlineto{\pgfqpoint{2.605995in}{2.403443in}}%
\pgfpathlineto{\pgfqpoint{2.606488in}{2.377465in}}%
\pgfpathlineto{\pgfqpoint{2.607475in}{2.379871in}}%
\pgfpathlineto{\pgfqpoint{2.609153in}{2.354221in}}%
\pgfpathlineto{\pgfqpoint{2.609646in}{2.323805in}}%
\pgfpathlineto{\pgfqpoint{2.610239in}{2.350162in}}%
\pgfpathlineto{\pgfqpoint{2.614187in}{2.451794in}}%
\pgfpathlineto{\pgfqpoint{2.614483in}{2.443825in}}%
\pgfpathlineto{\pgfqpoint{2.614680in}{2.440118in}}%
\pgfpathlineto{\pgfqpoint{2.615174in}{2.454162in}}%
\pgfpathlineto{\pgfqpoint{2.617345in}{2.482747in}}%
\pgfpathlineto{\pgfqpoint{2.617542in}{2.476305in}}%
\pgfpathlineto{\pgfqpoint{2.618727in}{2.451613in}}%
\pgfpathlineto{\pgfqpoint{2.618924in}{2.455009in}}%
\pgfpathlineto{\pgfqpoint{2.619812in}{2.453762in}}%
\pgfpathlineto{\pgfqpoint{2.620405in}{2.473520in}}%
\pgfpathlineto{\pgfqpoint{2.620997in}{2.461972in}}%
\pgfpathlineto{\pgfqpoint{2.621293in}{2.472690in}}%
\pgfpathlineto{\pgfqpoint{2.622872in}{2.507713in}}%
\pgfpathlineto{\pgfqpoint{2.622971in}{2.504562in}}%
\pgfpathlineto{\pgfqpoint{2.623958in}{2.488815in}}%
\pgfpathlineto{\pgfqpoint{2.624155in}{2.495428in}}%
\pgfpathlineto{\pgfqpoint{2.624254in}{2.496360in}}%
\pgfpathlineto{\pgfqpoint{2.624353in}{2.492887in}}%
\pgfpathlineto{\pgfqpoint{2.625636in}{2.463537in}}%
\pgfpathlineto{\pgfqpoint{2.625833in}{2.465620in}}%
\pgfpathlineto{\pgfqpoint{2.625932in}{2.467454in}}%
\pgfpathlineto{\pgfqpoint{2.626228in}{2.457196in}}%
\pgfpathlineto{\pgfqpoint{2.627807in}{2.423453in}}%
\pgfpathlineto{\pgfqpoint{2.628202in}{2.440419in}}%
\pgfpathlineto{\pgfqpoint{2.628597in}{2.422597in}}%
\pgfpathlineto{\pgfqpoint{2.629090in}{2.428742in}}%
\pgfpathlineto{\pgfqpoint{2.630275in}{2.409903in}}%
\pgfpathlineto{\pgfqpoint{2.630373in}{2.412123in}}%
\pgfpathlineto{\pgfqpoint{2.631459in}{2.423463in}}%
\pgfpathlineto{\pgfqpoint{2.630965in}{2.406014in}}%
\pgfpathlineto{\pgfqpoint{2.631558in}{2.420380in}}%
\pgfpathlineto{\pgfqpoint{2.631755in}{2.414010in}}%
\pgfpathlineto{\pgfqpoint{2.632643in}{2.421423in}}%
\pgfpathlineto{\pgfqpoint{2.632939in}{2.419929in}}%
\pgfpathlineto{\pgfqpoint{2.633137in}{2.422094in}}%
\pgfpathlineto{\pgfqpoint{2.633926in}{2.418747in}}%
\pgfpathlineto{\pgfqpoint{2.634519in}{2.428346in}}%
\pgfpathlineto{\pgfqpoint{2.634617in}{2.428384in}}%
\pgfpathlineto{\pgfqpoint{2.634815in}{2.424943in}}%
\pgfpathlineto{\pgfqpoint{2.635111in}{2.434383in}}%
\pgfpathlineto{\pgfqpoint{2.636098in}{2.456312in}}%
\pgfpathlineto{\pgfqpoint{2.636295in}{2.448613in}}%
\pgfpathlineto{\pgfqpoint{2.638861in}{2.354210in}}%
\pgfpathlineto{\pgfqpoint{2.639157in}{2.354529in}}%
\pgfpathlineto{\pgfqpoint{2.639355in}{2.347143in}}%
\pgfpathlineto{\pgfqpoint{2.640539in}{2.294435in}}%
\pgfpathlineto{\pgfqpoint{2.641131in}{2.303378in}}%
\pgfpathlineto{\pgfqpoint{2.641427in}{2.305001in}}%
\pgfpathlineto{\pgfqpoint{2.641625in}{2.303635in}}%
\pgfpathlineto{\pgfqpoint{2.641822in}{2.299069in}}%
\pgfpathlineto{\pgfqpoint{2.642118in}{2.313752in}}%
\pgfpathlineto{\pgfqpoint{2.642513in}{2.343584in}}%
\pgfpathlineto{\pgfqpoint{2.643303in}{2.323983in}}%
\pgfpathlineto{\pgfqpoint{2.643599in}{2.300311in}}%
\pgfpathlineto{\pgfqpoint{2.644487in}{2.315907in}}%
\pgfpathlineto{\pgfqpoint{2.645968in}{2.371113in}}%
\pgfpathlineto{\pgfqpoint{2.647053in}{2.562512in}}%
\pgfpathlineto{\pgfqpoint{2.648139in}{2.538951in}}%
\pgfpathlineto{\pgfqpoint{2.648336in}{2.534995in}}%
\pgfpathlineto{\pgfqpoint{2.648929in}{2.375474in}}%
\pgfpathlineto{\pgfqpoint{2.649323in}{2.297793in}}%
\pgfpathlineto{\pgfqpoint{2.650014in}{2.358106in}}%
\pgfpathlineto{\pgfqpoint{2.650113in}{2.356252in}}%
\pgfpathlineto{\pgfqpoint{2.650409in}{2.368465in}}%
\pgfpathlineto{\pgfqpoint{2.652580in}{2.421615in}}%
\pgfpathlineto{\pgfqpoint{2.653271in}{2.374610in}}%
\pgfpathlineto{\pgfqpoint{2.654554in}{2.390401in}}%
\pgfpathlineto{\pgfqpoint{2.654653in}{2.388800in}}%
\pgfpathlineto{\pgfqpoint{2.655048in}{2.396794in}}%
\pgfpathlineto{\pgfqpoint{2.655541in}{2.415095in}}%
\pgfpathlineto{\pgfqpoint{2.656035in}{2.395962in}}%
\pgfpathlineto{\pgfqpoint{2.656331in}{2.404531in}}%
\pgfpathlineto{\pgfqpoint{2.656430in}{2.406070in}}%
\pgfpathlineto{\pgfqpoint{2.656825in}{2.395089in}}%
\pgfpathlineto{\pgfqpoint{2.656923in}{2.394117in}}%
\pgfpathlineto{\pgfqpoint{2.657417in}{2.399985in}}%
\pgfpathlineto{\pgfqpoint{2.658996in}{2.423537in}}%
\pgfpathlineto{\pgfqpoint{2.659391in}{2.409980in}}%
\pgfpathlineto{\pgfqpoint{2.660082in}{2.423266in}}%
\pgfpathlineto{\pgfqpoint{2.661069in}{2.432030in}}%
\pgfpathlineto{\pgfqpoint{2.660674in}{2.421210in}}%
\pgfpathlineto{\pgfqpoint{2.661266in}{2.426915in}}%
\pgfpathlineto{\pgfqpoint{2.661463in}{2.422064in}}%
\pgfpathlineto{\pgfqpoint{2.661759in}{2.429033in}}%
\pgfpathlineto{\pgfqpoint{2.662253in}{2.425177in}}%
\pgfpathlineto{\pgfqpoint{2.663240in}{2.450644in}}%
\pgfpathlineto{\pgfqpoint{2.663536in}{2.434576in}}%
\pgfpathlineto{\pgfqpoint{2.663635in}{2.434692in}}%
\pgfpathlineto{\pgfqpoint{2.663931in}{2.450024in}}%
\pgfpathlineto{\pgfqpoint{2.664326in}{2.433005in}}%
\pgfpathlineto{\pgfqpoint{2.664819in}{2.440439in}}%
\pgfpathlineto{\pgfqpoint{2.665806in}{2.429478in}}%
\pgfpathlineto{\pgfqpoint{2.666004in}{2.430564in}}%
\pgfpathlineto{\pgfqpoint{2.666201in}{2.432371in}}%
\pgfpathlineto{\pgfqpoint{2.666596in}{2.424861in}}%
\pgfpathlineto{\pgfqpoint{2.667681in}{2.413116in}}%
\pgfpathlineto{\pgfqpoint{2.667978in}{2.417436in}}%
\pgfpathlineto{\pgfqpoint{2.669162in}{2.424031in}}%
\pgfpathlineto{\pgfqpoint{2.669261in}{2.421508in}}%
\pgfpathlineto{\pgfqpoint{2.670642in}{2.398302in}}%
\pgfpathlineto{\pgfqpoint{2.670050in}{2.422639in}}%
\pgfpathlineto{\pgfqpoint{2.670840in}{2.401660in}}%
\pgfpathlineto{\pgfqpoint{2.670938in}{2.403243in}}%
\pgfpathlineto{\pgfqpoint{2.671333in}{2.392394in}}%
\pgfpathlineto{\pgfqpoint{2.671432in}{2.391776in}}%
\pgfpathlineto{\pgfqpoint{2.671629in}{2.396940in}}%
\pgfpathlineto{\pgfqpoint{2.671728in}{2.398324in}}%
\pgfpathlineto{\pgfqpoint{2.672024in}{2.388930in}}%
\pgfpathlineto{\pgfqpoint{2.674294in}{2.346927in}}%
\pgfpathlineto{\pgfqpoint{2.674590in}{2.347172in}}%
\pgfpathlineto{\pgfqpoint{2.676071in}{2.328887in}}%
\pgfpathlineto{\pgfqpoint{2.676170in}{2.331548in}}%
\pgfpathlineto{\pgfqpoint{2.676466in}{2.341830in}}%
\pgfpathlineto{\pgfqpoint{2.677058in}{2.327519in}}%
\pgfpathlineto{\pgfqpoint{2.677157in}{2.328484in}}%
\pgfpathlineto{\pgfqpoint{2.678242in}{2.334686in}}%
\pgfpathlineto{\pgfqpoint{2.677749in}{2.322102in}}%
\pgfpathlineto{\pgfqpoint{2.678341in}{2.331876in}}%
\pgfpathlineto{\pgfqpoint{2.678538in}{2.325421in}}%
\pgfpathlineto{\pgfqpoint{2.679032in}{2.345806in}}%
\pgfpathlineto{\pgfqpoint{2.679427in}{2.330015in}}%
\pgfpathlineto{\pgfqpoint{2.679723in}{2.332289in}}%
\pgfpathlineto{\pgfqpoint{2.679821in}{2.331074in}}%
\pgfpathlineto{\pgfqpoint{2.680117in}{2.323457in}}%
\pgfpathlineto{\pgfqpoint{2.680414in}{2.341932in}}%
\pgfpathlineto{\pgfqpoint{2.680710in}{2.361270in}}%
\pgfpathlineto{\pgfqpoint{2.681598in}{2.355422in}}%
\pgfpathlineto{\pgfqpoint{2.681697in}{2.356313in}}%
\pgfpathlineto{\pgfqpoint{2.681993in}{2.349248in}}%
\pgfpathlineto{\pgfqpoint{2.683572in}{2.314545in}}%
\pgfpathlineto{\pgfqpoint{2.683868in}{2.309272in}}%
\pgfpathlineto{\pgfqpoint{2.684460in}{2.315678in}}%
\pgfpathlineto{\pgfqpoint{2.686730in}{2.357490in}}%
\pgfpathlineto{\pgfqpoint{2.686829in}{2.355663in}}%
\pgfpathlineto{\pgfqpoint{2.687915in}{2.304512in}}%
\pgfpathlineto{\pgfqpoint{2.688704in}{2.312477in}}%
\pgfpathlineto{\pgfqpoint{2.688803in}{2.311608in}}%
\pgfpathlineto{\pgfqpoint{2.689395in}{2.315342in}}%
\pgfpathlineto{\pgfqpoint{2.690086in}{2.351713in}}%
\pgfpathlineto{\pgfqpoint{2.690678in}{2.323861in}}%
\pgfpathlineto{\pgfqpoint{2.691073in}{2.306685in}}%
\pgfpathlineto{\pgfqpoint{2.691567in}{2.324881in}}%
\pgfpathlineto{\pgfqpoint{2.691961in}{2.307713in}}%
\pgfpathlineto{\pgfqpoint{2.692554in}{2.322609in}}%
\pgfpathlineto{\pgfqpoint{2.693244in}{2.371711in}}%
\pgfpathlineto{\pgfqpoint{2.694330in}{2.546462in}}%
\pgfpathlineto{\pgfqpoint{2.695317in}{2.522104in}}%
\pgfpathlineto{\pgfqpoint{2.695712in}{2.476134in}}%
\pgfpathlineto{\pgfqpoint{2.696501in}{2.266359in}}%
\pgfpathlineto{\pgfqpoint{2.697291in}{2.335323in}}%
\pgfpathlineto{\pgfqpoint{2.697488in}{2.327135in}}%
\pgfpathlineto{\pgfqpoint{2.697982in}{2.343859in}}%
\pgfpathlineto{\pgfqpoint{2.698179in}{2.342818in}}%
\pgfpathlineto{\pgfqpoint{2.698278in}{2.342157in}}%
\pgfpathlineto{\pgfqpoint{2.698574in}{2.346190in}}%
\pgfpathlineto{\pgfqpoint{2.699166in}{2.365488in}}%
\pgfpathlineto{\pgfqpoint{2.699660in}{2.390247in}}%
\pgfpathlineto{\pgfqpoint{2.700153in}{2.365942in}}%
\pgfpathlineto{\pgfqpoint{2.701239in}{2.347695in}}%
\pgfpathlineto{\pgfqpoint{2.701338in}{2.349306in}}%
\pgfpathlineto{\pgfqpoint{2.702917in}{2.400018in}}%
\pgfpathlineto{\pgfqpoint{2.703410in}{2.382289in}}%
\pgfpathlineto{\pgfqpoint{2.703904in}{2.371140in}}%
\pgfpathlineto{\pgfqpoint{2.704496in}{2.380562in}}%
\pgfpathlineto{\pgfqpoint{2.704694in}{2.377637in}}%
\pgfpathlineto{\pgfqpoint{2.704891in}{2.372237in}}%
\pgfpathlineto{\pgfqpoint{2.705384in}{2.383869in}}%
\pgfpathlineto{\pgfqpoint{2.705582in}{2.381527in}}%
\pgfpathlineto{\pgfqpoint{2.706075in}{2.410629in}}%
\pgfpathlineto{\pgfqpoint{2.707062in}{2.394191in}}%
\pgfpathlineto{\pgfqpoint{2.707260in}{2.395513in}}%
\pgfpathlineto{\pgfqpoint{2.708938in}{2.412821in}}%
\pgfpathlineto{\pgfqpoint{2.709234in}{2.406595in}}%
\pgfpathlineto{\pgfqpoint{2.709628in}{2.419419in}}%
\pgfpathlineto{\pgfqpoint{2.709826in}{2.423071in}}%
\pgfpathlineto{\pgfqpoint{2.710517in}{2.413994in}}%
\pgfpathlineto{\pgfqpoint{2.711405in}{2.402775in}}%
\pgfpathlineto{\pgfqpoint{2.711701in}{2.407798in}}%
\pgfpathlineto{\pgfqpoint{2.712688in}{2.413644in}}%
\pgfpathlineto{\pgfqpoint{2.712293in}{2.403546in}}%
\pgfpathlineto{\pgfqpoint{2.712886in}{2.410344in}}%
\pgfpathlineto{\pgfqpoint{2.713873in}{2.396812in}}%
\pgfpathlineto{\pgfqpoint{2.714169in}{2.403562in}}%
\pgfpathlineto{\pgfqpoint{2.715156in}{2.414225in}}%
\pgfpathlineto{\pgfqpoint{2.714662in}{2.398280in}}%
\pgfpathlineto{\pgfqpoint{2.715353in}{2.408753in}}%
\pgfpathlineto{\pgfqpoint{2.716241in}{2.402215in}}%
\pgfpathlineto{\pgfqpoint{2.715846in}{2.409810in}}%
\pgfpathlineto{\pgfqpoint{2.716439in}{2.405124in}}%
\pgfpathlineto{\pgfqpoint{2.717228in}{2.408314in}}%
\pgfpathlineto{\pgfqpoint{2.717426in}{2.403384in}}%
\pgfpathlineto{\pgfqpoint{2.717722in}{2.397486in}}%
\pgfpathlineto{\pgfqpoint{2.718610in}{2.398915in}}%
\pgfpathlineto{\pgfqpoint{2.721374in}{2.372219in}}%
\pgfpathlineto{\pgfqpoint{2.721571in}{2.374322in}}%
\pgfpathlineto{\pgfqpoint{2.721867in}{2.368155in}}%
\pgfpathlineto{\pgfqpoint{2.722065in}{2.365016in}}%
\pgfpathlineto{\pgfqpoint{2.722459in}{2.379017in}}%
\pgfpathlineto{\pgfqpoint{2.722755in}{2.374469in}}%
\pgfpathlineto{\pgfqpoint{2.723052in}{2.373439in}}%
\pgfpathlineto{\pgfqpoint{2.723446in}{2.361255in}}%
\pgfpathlineto{\pgfqpoint{2.724137in}{2.370038in}}%
\pgfpathlineto{\pgfqpoint{2.724532in}{2.378244in}}%
\pgfpathlineto{\pgfqpoint{2.725322in}{2.374016in}}%
\pgfpathlineto{\pgfqpoint{2.725519in}{2.371175in}}%
\pgfpathlineto{\pgfqpoint{2.725914in}{2.382123in}}%
\pgfpathlineto{\pgfqpoint{2.726012in}{2.383553in}}%
\pgfpathlineto{\pgfqpoint{2.726309in}{2.373017in}}%
\pgfpathlineto{\pgfqpoint{2.726506in}{2.367491in}}%
\pgfpathlineto{\pgfqpoint{2.726999in}{2.385372in}}%
\pgfpathlineto{\pgfqpoint{2.727098in}{2.385090in}}%
\pgfpathlineto{\pgfqpoint{2.727394in}{2.379269in}}%
\pgfpathlineto{\pgfqpoint{2.727690in}{2.395556in}}%
\pgfpathlineto{\pgfqpoint{2.728875in}{2.422463in}}%
\pgfpathlineto{\pgfqpoint{2.729072in}{2.421621in}}%
\pgfpathlineto{\pgfqpoint{2.730257in}{2.454966in}}%
\pgfpathlineto{\pgfqpoint{2.731441in}{2.444357in}}%
\pgfpathlineto{\pgfqpoint{2.732823in}{2.396271in}}%
\pgfpathlineto{\pgfqpoint{2.733316in}{2.406302in}}%
\pgfpathlineto{\pgfqpoint{2.733612in}{2.416952in}}%
\pgfpathlineto{\pgfqpoint{2.733810in}{2.425135in}}%
\pgfpathlineto{\pgfqpoint{2.734303in}{2.404796in}}%
\pgfpathlineto{\pgfqpoint{2.734501in}{2.408402in}}%
\pgfpathlineto{\pgfqpoint{2.734599in}{2.408222in}}%
\pgfpathlineto{\pgfqpoint{2.735981in}{2.394239in}}%
\pgfpathlineto{\pgfqpoint{2.736080in}{2.395987in}}%
\pgfpathlineto{\pgfqpoint{2.736968in}{2.436513in}}%
\pgfpathlineto{\pgfqpoint{2.737363in}{2.419186in}}%
\pgfpathlineto{\pgfqpoint{2.737758in}{2.392390in}}%
\pgfpathlineto{\pgfqpoint{2.738646in}{2.394559in}}%
\pgfpathlineto{\pgfqpoint{2.740719in}{2.581995in}}%
\pgfpathlineto{\pgfqpoint{2.741804in}{2.663163in}}%
\pgfpathlineto{\pgfqpoint{2.742100in}{2.651989in}}%
\pgfpathlineto{\pgfqpoint{2.742791in}{2.566066in}}%
\pgfpathlineto{\pgfqpoint{2.743482in}{2.388505in}}%
\pgfpathlineto{\pgfqpoint{2.744173in}{2.456086in}}%
\pgfpathlineto{\pgfqpoint{2.745061in}{2.452874in}}%
\pgfpathlineto{\pgfqpoint{2.744667in}{2.461912in}}%
\pgfpathlineto{\pgfqpoint{2.745160in}{2.454051in}}%
\pgfpathlineto{\pgfqpoint{2.746542in}{2.525165in}}%
\pgfpathlineto{\pgfqpoint{2.747529in}{2.499029in}}%
\pgfpathlineto{\pgfqpoint{2.747825in}{2.489073in}}%
\pgfpathlineto{\pgfqpoint{2.748220in}{2.507214in}}%
\pgfpathlineto{\pgfqpoint{2.748713in}{2.493219in}}%
\pgfpathlineto{\pgfqpoint{2.749799in}{2.546751in}}%
\pgfpathlineto{\pgfqpoint{2.750292in}{2.530721in}}%
\pgfpathlineto{\pgfqpoint{2.750885in}{2.509624in}}%
\pgfpathlineto{\pgfqpoint{2.751576in}{2.515948in}}%
\pgfpathlineto{\pgfqpoint{2.752760in}{2.554923in}}%
\pgfpathlineto{\pgfqpoint{2.753056in}{2.550802in}}%
\pgfpathlineto{\pgfqpoint{2.754833in}{2.490990in}}%
\pgfpathlineto{\pgfqpoint{2.755721in}{2.508595in}}%
\pgfpathlineto{\pgfqpoint{2.757892in}{2.621396in}}%
\pgfpathlineto{\pgfqpoint{2.758583in}{2.612536in}}%
\pgfpathlineto{\pgfqpoint{2.758682in}{2.612685in}}%
\pgfpathlineto{\pgfqpoint{2.758781in}{2.610827in}}%
\pgfpathlineto{\pgfqpoint{2.759866in}{2.596920in}}%
\pgfpathlineto{\pgfqpoint{2.760162in}{2.600671in}}%
\pgfpathlineto{\pgfqpoint{2.762334in}{2.624759in}}%
\pgfpathlineto{\pgfqpoint{2.762728in}{2.612975in}}%
\pgfpathlineto{\pgfqpoint{2.763025in}{2.609238in}}%
\pgfpathlineto{\pgfqpoint{2.763419in}{2.617204in}}%
\pgfpathlineto{\pgfqpoint{2.763617in}{2.621086in}}%
\pgfpathlineto{\pgfqpoint{2.764110in}{2.611901in}}%
\pgfpathlineto{\pgfqpoint{2.764406in}{2.614209in}}%
\pgfpathlineto{\pgfqpoint{2.765097in}{2.623334in}}%
\pgfpathlineto{\pgfqpoint{2.765788in}{2.616968in}}%
\pgfpathlineto{\pgfqpoint{2.766874in}{2.598032in}}%
\pgfpathlineto{\pgfqpoint{2.767071in}{2.603466in}}%
\pgfpathlineto{\pgfqpoint{2.767170in}{2.607015in}}%
\pgfpathlineto{\pgfqpoint{2.767565in}{2.581457in}}%
\pgfpathlineto{\pgfqpoint{2.767663in}{2.575809in}}%
\pgfpathlineto{\pgfqpoint{2.768058in}{2.590795in}}%
\pgfpathlineto{\pgfqpoint{2.768650in}{2.582355in}}%
\pgfpathlineto{\pgfqpoint{2.768749in}{2.583972in}}%
\pgfpathlineto{\pgfqpoint{2.769243in}{2.575076in}}%
\pgfpathlineto{\pgfqpoint{2.769341in}{2.575371in}}%
\pgfpathlineto{\pgfqpoint{2.769539in}{2.573437in}}%
\pgfpathlineto{\pgfqpoint{2.770427in}{2.565511in}}%
\pgfpathlineto{\pgfqpoint{2.770723in}{2.570273in}}%
\pgfpathlineto{\pgfqpoint{2.772105in}{2.581644in}}%
\pgfpathlineto{\pgfqpoint{2.771315in}{2.569520in}}%
\pgfpathlineto{\pgfqpoint{2.772204in}{2.581007in}}%
\pgfpathlineto{\pgfqpoint{2.774079in}{2.558375in}}%
\pgfpathlineto{\pgfqpoint{2.774178in}{2.563061in}}%
\pgfpathlineto{\pgfqpoint{2.775165in}{2.585856in}}%
\pgfpathlineto{\pgfqpoint{2.775362in}{2.577789in}}%
\pgfpathlineto{\pgfqpoint{2.775559in}{2.572430in}}%
\pgfpathlineto{\pgfqpoint{2.775954in}{2.592225in}}%
\pgfpathlineto{\pgfqpoint{2.776250in}{2.584522in}}%
\pgfpathlineto{\pgfqpoint{2.776349in}{2.584461in}}%
\pgfpathlineto{\pgfqpoint{2.777533in}{2.606199in}}%
\pgfpathlineto{\pgfqpoint{2.777928in}{2.598035in}}%
\pgfpathlineto{\pgfqpoint{2.778027in}{2.597996in}}%
\pgfpathlineto{\pgfqpoint{2.778520in}{2.611737in}}%
\pgfpathlineto{\pgfqpoint{2.778816in}{2.598579in}}%
\pgfpathlineto{\pgfqpoint{2.780593in}{2.546138in}}%
\pgfpathlineto{\pgfqpoint{2.781580in}{2.579303in}}%
\pgfpathlineto{\pgfqpoint{2.781777in}{2.588742in}}%
\pgfpathlineto{\pgfqpoint{2.782370in}{2.556270in}}%
\pgfpathlineto{\pgfqpoint{2.782666in}{2.539281in}}%
\pgfpathlineto{\pgfqpoint{2.783554in}{2.547290in}}%
\pgfpathlineto{\pgfqpoint{2.784837in}{2.597046in}}%
\pgfpathlineto{\pgfqpoint{2.785133in}{2.588557in}}%
\pgfpathlineto{\pgfqpoint{2.785923in}{2.547086in}}%
\pgfpathlineto{\pgfqpoint{2.786811in}{2.563762in}}%
\pgfpathlineto{\pgfqpoint{2.787403in}{2.562989in}}%
\pgfpathlineto{\pgfqpoint{2.788094in}{2.597613in}}%
\pgfpathlineto{\pgfqpoint{2.789575in}{2.838909in}}%
\pgfpathlineto{\pgfqpoint{2.790463in}{2.803762in}}%
\pgfpathlineto{\pgfqpoint{2.790956in}{2.687504in}}%
\pgfpathlineto{\pgfqpoint{2.791450in}{2.555512in}}%
\pgfpathlineto{\pgfqpoint{2.792239in}{2.608476in}}%
\pgfpathlineto{\pgfqpoint{2.792832in}{2.599541in}}%
\pgfpathlineto{\pgfqpoint{2.793029in}{2.607328in}}%
\pgfpathlineto{\pgfqpoint{2.794608in}{2.669087in}}%
\pgfpathlineto{\pgfqpoint{2.794806in}{2.660316in}}%
\pgfpathlineto{\pgfqpoint{2.795694in}{2.609010in}}%
\pgfpathlineto{\pgfqpoint{2.796187in}{2.629706in}}%
\pgfpathlineto{\pgfqpoint{2.797767in}{2.671548in}}%
\pgfpathlineto{\pgfqpoint{2.798063in}{2.678428in}}%
\pgfpathlineto{\pgfqpoint{2.798359in}{2.658450in}}%
\pgfpathlineto{\pgfqpoint{2.798556in}{2.649594in}}%
\pgfpathlineto{\pgfqpoint{2.799346in}{2.660685in}}%
\pgfpathlineto{\pgfqpoint{2.800826in}{2.689086in}}%
\pgfpathlineto{\pgfqpoint{2.801517in}{2.693620in}}%
\pgfpathlineto{\pgfqpoint{2.801715in}{2.689727in}}%
\pgfpathlineto{\pgfqpoint{2.801912in}{2.684674in}}%
\pgfpathlineto{\pgfqpoint{2.802307in}{2.692061in}}%
\pgfpathlineto{\pgfqpoint{2.802702in}{2.687385in}}%
\pgfpathlineto{\pgfqpoint{2.802998in}{2.697501in}}%
\pgfpathlineto{\pgfqpoint{2.803985in}{2.694568in}}%
\pgfpathlineto{\pgfqpoint{2.804281in}{2.691998in}}%
\pgfpathlineto{\pgfqpoint{2.804577in}{2.697148in}}%
\pgfpathlineto{\pgfqpoint{2.804774in}{2.702414in}}%
\pgfpathlineto{\pgfqpoint{2.805663in}{2.697731in}}%
\pgfpathlineto{\pgfqpoint{2.805860in}{2.701550in}}%
\pgfpathlineto{\pgfqpoint{2.806255in}{2.721565in}}%
\pgfpathlineto{\pgfqpoint{2.807044in}{2.708455in}}%
\pgfpathlineto{\pgfqpoint{2.808130in}{2.724176in}}%
\pgfpathlineto{\pgfqpoint{2.808426in}{2.719154in}}%
\pgfpathlineto{\pgfqpoint{2.808525in}{2.718808in}}%
\pgfpathlineto{\pgfqpoint{2.808624in}{2.720194in}}%
\pgfpathlineto{\pgfqpoint{2.809808in}{2.740607in}}%
\pgfpathlineto{\pgfqpoint{2.809216in}{2.715335in}}%
\pgfpathlineto{\pgfqpoint{2.810005in}{2.728502in}}%
\pgfpathlineto{\pgfqpoint{2.810301in}{2.715998in}}%
\pgfpathlineto{\pgfqpoint{2.811091in}{2.729030in}}%
\pgfpathlineto{\pgfqpoint{2.811288in}{2.734233in}}%
\pgfpathlineto{\pgfqpoint{2.811782in}{2.720010in}}%
\pgfpathlineto{\pgfqpoint{2.813361in}{2.701809in}}%
\pgfpathlineto{\pgfqpoint{2.812275in}{2.723538in}}%
\pgfpathlineto{\pgfqpoint{2.813558in}{2.706370in}}%
\pgfpathlineto{\pgfqpoint{2.813756in}{2.711768in}}%
\pgfpathlineto{\pgfqpoint{2.814151in}{2.687890in}}%
\pgfpathlineto{\pgfqpoint{2.814249in}{2.688034in}}%
\pgfpathlineto{\pgfqpoint{2.814545in}{2.696381in}}%
\pgfpathlineto{\pgfqpoint{2.815039in}{2.678390in}}%
\pgfpathlineto{\pgfqpoint{2.816519in}{2.655020in}}%
\pgfpathlineto{\pgfqpoint{2.816816in}{2.645560in}}%
\pgfpathlineto{\pgfqpoint{2.817704in}{2.647608in}}%
\pgfpathlineto{\pgfqpoint{2.818000in}{2.649936in}}%
\pgfpathlineto{\pgfqpoint{2.818197in}{2.645417in}}%
\pgfpathlineto{\pgfqpoint{2.819086in}{2.630111in}}%
\pgfpathlineto{\pgfqpoint{2.819283in}{2.637100in}}%
\pgfpathlineto{\pgfqpoint{2.819480in}{2.646653in}}%
\pgfpathlineto{\pgfqpoint{2.819875in}{2.628137in}}%
\pgfpathlineto{\pgfqpoint{2.820369in}{2.638189in}}%
\pgfpathlineto{\pgfqpoint{2.823527in}{2.529459in}}%
\pgfpathlineto{\pgfqpoint{2.823724in}{2.531178in}}%
\pgfpathlineto{\pgfqpoint{2.825008in}{2.585259in}}%
\pgfpathlineto{\pgfqpoint{2.826291in}{2.658115in}}%
\pgfpathlineto{\pgfqpoint{2.826982in}{2.638907in}}%
\pgfpathlineto{\pgfqpoint{2.827475in}{2.614461in}}%
\pgfpathlineto{\pgfqpoint{2.828659in}{2.544637in}}%
\pgfpathlineto{\pgfqpoint{2.829449in}{2.550088in}}%
\pgfpathlineto{\pgfqpoint{2.829844in}{2.558978in}}%
\pgfpathlineto{\pgfqpoint{2.830140in}{2.547301in}}%
\pgfpathlineto{\pgfqpoint{2.832015in}{2.501394in}}%
\pgfpathlineto{\pgfqpoint{2.832114in}{2.501317in}}%
\pgfpathlineto{\pgfqpoint{2.832903in}{2.547013in}}%
\pgfpathlineto{\pgfqpoint{2.833594in}{2.515959in}}%
\pgfpathlineto{\pgfqpoint{2.835075in}{2.480033in}}%
\pgfpathlineto{\pgfqpoint{2.835371in}{2.489442in}}%
\pgfpathlineto{\pgfqpoint{2.835864in}{2.487041in}}%
\pgfpathlineto{\pgfqpoint{2.836555in}{2.578758in}}%
\pgfpathlineto{\pgfqpoint{2.837542in}{2.749438in}}%
\pgfpathlineto{\pgfqpoint{2.838332in}{2.706069in}}%
\pgfpathlineto{\pgfqpoint{2.838727in}{2.669660in}}%
\pgfpathlineto{\pgfqpoint{2.839516in}{2.461820in}}%
\pgfpathlineto{\pgfqpoint{2.840405in}{2.499235in}}%
\pgfpathlineto{\pgfqpoint{2.840602in}{2.493336in}}%
\pgfpathlineto{\pgfqpoint{2.841194in}{2.499306in}}%
\pgfpathlineto{\pgfqpoint{2.841589in}{2.495025in}}%
\pgfpathlineto{\pgfqpoint{2.841688in}{2.495282in}}%
\pgfpathlineto{\pgfqpoint{2.842773in}{2.556234in}}%
\pgfpathlineto{\pgfqpoint{2.843267in}{2.512024in}}%
\pgfpathlineto{\pgfqpoint{2.844649in}{2.496381in}}%
\pgfpathlineto{\pgfqpoint{2.844747in}{2.495763in}}%
\pgfpathlineto{\pgfqpoint{2.844846in}{2.499567in}}%
\pgfpathlineto{\pgfqpoint{2.845833in}{2.532321in}}%
\pgfpathlineto{\pgfqpoint{2.846228in}{2.519252in}}%
\pgfpathlineto{\pgfqpoint{2.846721in}{2.486015in}}%
\pgfpathlineto{\pgfqpoint{2.847610in}{2.493313in}}%
\pgfpathlineto{\pgfqpoint{2.847708in}{2.493392in}}%
\pgfpathlineto{\pgfqpoint{2.848103in}{2.477941in}}%
\pgfpathlineto{\pgfqpoint{2.848597in}{2.498862in}}%
\pgfpathlineto{\pgfqpoint{2.848794in}{2.497983in}}%
\pgfpathlineto{\pgfqpoint{2.848991in}{2.503338in}}%
\pgfpathlineto{\pgfqpoint{2.849287in}{2.511616in}}%
\pgfpathlineto{\pgfqpoint{2.849682in}{2.486907in}}%
\pgfpathlineto{\pgfqpoint{2.849880in}{2.479820in}}%
\pgfpathlineto{\pgfqpoint{2.850669in}{2.490524in}}%
\pgfpathlineto{\pgfqpoint{2.852051in}{2.521603in}}%
\pgfpathlineto{\pgfqpoint{2.852248in}{2.516813in}}%
\pgfpathlineto{\pgfqpoint{2.852545in}{2.508668in}}%
\pgfpathlineto{\pgfqpoint{2.853235in}{2.519810in}}%
\pgfpathlineto{\pgfqpoint{2.853532in}{2.518895in}}%
\pgfpathlineto{\pgfqpoint{2.857776in}{2.474011in}}%
\pgfpathlineto{\pgfqpoint{2.858368in}{2.480268in}}%
\pgfpathlineto{\pgfqpoint{2.858763in}{2.499594in}}%
\pgfpathlineto{\pgfqpoint{2.859256in}{2.473858in}}%
\pgfpathlineto{\pgfqpoint{2.859848in}{2.466965in}}%
\pgfpathlineto{\pgfqpoint{2.860144in}{2.475561in}}%
\pgfpathlineto{\pgfqpoint{2.860440in}{2.485013in}}%
\pgfpathlineto{\pgfqpoint{2.860934in}{2.463490in}}%
\pgfpathlineto{\pgfqpoint{2.861033in}{2.464352in}}%
\pgfpathlineto{\pgfqpoint{2.861131in}{2.465503in}}%
\pgfpathlineto{\pgfqpoint{2.861625in}{2.459304in}}%
\pgfpathlineto{\pgfqpoint{2.862020in}{2.453376in}}%
\pgfpathlineto{\pgfqpoint{2.862316in}{2.460096in}}%
\pgfpathlineto{\pgfqpoint{2.862414in}{2.462367in}}%
\pgfpathlineto{\pgfqpoint{2.862711in}{2.449016in}}%
\pgfpathlineto{\pgfqpoint{2.864092in}{2.425095in}}%
\pgfpathlineto{\pgfqpoint{2.864487in}{2.411892in}}%
\pgfpathlineto{\pgfqpoint{2.866362in}{2.368670in}}%
\pgfpathlineto{\pgfqpoint{2.866461in}{2.370858in}}%
\pgfpathlineto{\pgfqpoint{2.866856in}{2.387236in}}%
\pgfpathlineto{\pgfqpoint{2.867448in}{2.369505in}}%
\pgfpathlineto{\pgfqpoint{2.867843in}{2.356187in}}%
\pgfpathlineto{\pgfqpoint{2.868238in}{2.376653in}}%
\pgfpathlineto{\pgfqpoint{2.868336in}{2.378802in}}%
\pgfpathlineto{\pgfqpoint{2.868731in}{2.364543in}}%
\pgfpathlineto{\pgfqpoint{2.869916in}{2.350932in}}%
\pgfpathlineto{\pgfqpoint{2.870113in}{2.357232in}}%
\pgfpathlineto{\pgfqpoint{2.871100in}{2.373979in}}%
\pgfpathlineto{\pgfqpoint{2.871297in}{2.366562in}}%
\pgfpathlineto{\pgfqpoint{2.871495in}{2.360211in}}%
\pgfpathlineto{\pgfqpoint{2.871988in}{2.381015in}}%
\pgfpathlineto{\pgfqpoint{2.872383in}{2.379088in}}%
\pgfpathlineto{\pgfqpoint{2.873864in}{2.417161in}}%
\pgfpathlineto{\pgfqpoint{2.874061in}{2.421807in}}%
\pgfpathlineto{\pgfqpoint{2.874357in}{2.400469in}}%
\pgfpathlineto{\pgfqpoint{2.874850in}{2.412219in}}%
\pgfpathlineto{\pgfqpoint{2.875936in}{2.374026in}}%
\pgfpathlineto{\pgfqpoint{2.877022in}{2.396652in}}%
\pgfpathlineto{\pgfqpoint{2.877614in}{2.385537in}}%
\pgfpathlineto{\pgfqpoint{2.878305in}{2.366880in}}%
\pgfpathlineto{\pgfqpoint{2.879095in}{2.376896in}}%
\pgfpathlineto{\pgfqpoint{2.880378in}{2.424186in}}%
\pgfpathlineto{\pgfqpoint{2.880772in}{2.404513in}}%
\pgfpathlineto{\pgfqpoint{2.882253in}{2.360249in}}%
\pgfpathlineto{\pgfqpoint{2.882352in}{2.360471in}}%
\pgfpathlineto{\pgfqpoint{2.882648in}{2.358936in}}%
\pgfpathlineto{\pgfqpoint{2.882746in}{2.358135in}}%
\pgfpathlineto{\pgfqpoint{2.882944in}{2.362939in}}%
\pgfpathlineto{\pgfqpoint{2.884029in}{2.477966in}}%
\pgfpathlineto{\pgfqpoint{2.884819in}{2.594874in}}%
\pgfpathlineto{\pgfqpoint{2.885510in}{2.579887in}}%
\pgfpathlineto{\pgfqpoint{2.886300in}{2.473963in}}%
\pgfpathlineto{\pgfqpoint{2.886892in}{2.311287in}}%
\pgfpathlineto{\pgfqpoint{2.887681in}{2.354591in}}%
\pgfpathlineto{\pgfqpoint{2.888372in}{2.364083in}}%
\pgfpathlineto{\pgfqpoint{2.888570in}{2.355956in}}%
\pgfpathlineto{\pgfqpoint{2.888866in}{2.344165in}}%
\pgfpathlineto{\pgfqpoint{2.889557in}{2.360109in}}%
\pgfpathlineto{\pgfqpoint{2.889853in}{2.365109in}}%
\pgfpathlineto{\pgfqpoint{2.890149in}{2.354052in}}%
\pgfpathlineto{\pgfqpoint{2.890938in}{2.277689in}}%
\pgfpathlineto{\pgfqpoint{2.892123in}{2.293075in}}%
\pgfpathlineto{\pgfqpoint{2.893307in}{2.396961in}}%
\pgfpathlineto{\pgfqpoint{2.894294in}{2.374570in}}%
\pgfpathlineto{\pgfqpoint{2.894886in}{2.392301in}}%
\pgfpathlineto{\pgfqpoint{2.895380in}{2.374021in}}%
\pgfpathlineto{\pgfqpoint{2.895577in}{2.369699in}}%
\pgfpathlineto{\pgfqpoint{2.896169in}{2.382255in}}%
\pgfpathlineto{\pgfqpoint{2.896663in}{2.395371in}}%
\pgfpathlineto{\pgfqpoint{2.897453in}{2.385946in}}%
\pgfpathlineto{\pgfqpoint{2.897650in}{2.381891in}}%
\pgfpathlineto{\pgfqpoint{2.897946in}{2.392671in}}%
\pgfpathlineto{\pgfqpoint{2.898242in}{2.408131in}}%
\pgfpathlineto{\pgfqpoint{2.898933in}{2.392444in}}%
\pgfpathlineto{\pgfqpoint{2.899032in}{2.392485in}}%
\pgfpathlineto{\pgfqpoint{2.899427in}{2.394562in}}%
\pgfpathlineto{\pgfqpoint{2.899920in}{2.411583in}}%
\pgfpathlineto{\pgfqpoint{2.900216in}{2.393100in}}%
\pgfpathlineto{\pgfqpoint{2.900414in}{2.386952in}}%
\pgfpathlineto{\pgfqpoint{2.900907in}{2.402748in}}%
\pgfpathlineto{\pgfqpoint{2.901104in}{2.400989in}}%
\pgfpathlineto{\pgfqpoint{2.901499in}{2.402691in}}%
\pgfpathlineto{\pgfqpoint{2.901697in}{2.401084in}}%
\pgfpathlineto{\pgfqpoint{2.902091in}{2.396107in}}%
\pgfpathlineto{\pgfqpoint{2.902782in}{2.399740in}}%
\pgfpathlineto{\pgfqpoint{2.902980in}{2.402946in}}%
\pgfpathlineto{\pgfqpoint{2.903276in}{2.388211in}}%
\pgfpathlineto{\pgfqpoint{2.903572in}{2.377908in}}%
\pgfpathlineto{\pgfqpoint{2.904263in}{2.390688in}}%
\pgfpathlineto{\pgfqpoint{2.904756in}{2.406013in}}%
\pgfpathlineto{\pgfqpoint{2.905151in}{2.391205in}}%
\pgfpathlineto{\pgfqpoint{2.905348in}{2.385078in}}%
\pgfpathlineto{\pgfqpoint{2.905941in}{2.403291in}}%
\pgfpathlineto{\pgfqpoint{2.906039in}{2.403567in}}%
\pgfpathlineto{\pgfqpoint{2.906138in}{2.402149in}}%
\pgfpathlineto{\pgfqpoint{2.906533in}{2.392350in}}%
\pgfpathlineto{\pgfqpoint{2.907421in}{2.396796in}}%
\pgfpathlineto{\pgfqpoint{2.908803in}{2.387256in}}%
\pgfpathlineto{\pgfqpoint{2.909000in}{2.377645in}}%
\pgfpathlineto{\pgfqpoint{2.909494in}{2.391064in}}%
\pgfpathlineto{\pgfqpoint{2.909987in}{2.379019in}}%
\pgfpathlineto{\pgfqpoint{2.910086in}{2.379736in}}%
\pgfpathlineto{\pgfqpoint{2.910382in}{2.374318in}}%
\pgfpathlineto{\pgfqpoint{2.911172in}{2.362150in}}%
\pgfpathlineto{\pgfqpoint{2.911468in}{2.371451in}}%
\pgfpathlineto{\pgfqpoint{2.911665in}{2.374721in}}%
\pgfpathlineto{\pgfqpoint{2.911961in}{2.358682in}}%
\pgfpathlineto{\pgfqpoint{2.912652in}{2.360417in}}%
\pgfpathlineto{\pgfqpoint{2.913244in}{2.348805in}}%
\pgfpathlineto{\pgfqpoint{2.913935in}{2.336968in}}%
\pgfpathlineto{\pgfqpoint{2.914231in}{2.347332in}}%
\pgfpathlineto{\pgfqpoint{2.914429in}{2.355045in}}%
\pgfpathlineto{\pgfqpoint{2.914922in}{2.331211in}}%
\pgfpathlineto{\pgfqpoint{2.915120in}{2.326235in}}%
\pgfpathlineto{\pgfqpoint{2.915811in}{2.335617in}}%
\pgfpathlineto{\pgfqpoint{2.917094in}{2.345745in}}%
\pgfpathlineto{\pgfqpoint{2.916501in}{2.334816in}}%
\pgfpathlineto{\pgfqpoint{2.917192in}{2.345271in}}%
\pgfpathlineto{\pgfqpoint{2.917291in}{2.344437in}}%
\pgfpathlineto{\pgfqpoint{2.917488in}{2.347542in}}%
\pgfpathlineto{\pgfqpoint{2.919265in}{2.385241in}}%
\pgfpathlineto{\pgfqpoint{2.919759in}{2.385803in}}%
\pgfpathlineto{\pgfqpoint{2.919462in}{2.384687in}}%
\pgfpathlineto{\pgfqpoint{2.919857in}{2.385524in}}%
\pgfpathlineto{\pgfqpoint{2.920055in}{2.383421in}}%
\pgfpathlineto{\pgfqpoint{2.920449in}{2.390598in}}%
\pgfpathlineto{\pgfqpoint{2.921042in}{2.403582in}}%
\pgfpathlineto{\pgfqpoint{2.921436in}{2.387856in}}%
\pgfpathlineto{\pgfqpoint{2.923312in}{2.327532in}}%
\pgfpathlineto{\pgfqpoint{2.923805in}{2.333791in}}%
\pgfpathlineto{\pgfqpoint{2.924299in}{2.348477in}}%
\pgfpathlineto{\pgfqpoint{2.924595in}{2.332993in}}%
\pgfpathlineto{\pgfqpoint{2.925088in}{2.300452in}}%
\pgfpathlineto{\pgfqpoint{2.925878in}{2.309905in}}%
\pgfpathlineto{\pgfqpoint{2.926569in}{2.308606in}}%
\pgfpathlineto{\pgfqpoint{2.926273in}{2.310844in}}%
\pgfpathlineto{\pgfqpoint{2.926667in}{2.310227in}}%
\pgfpathlineto{\pgfqpoint{2.927457in}{2.351202in}}%
\pgfpathlineto{\pgfqpoint{2.927951in}{2.326976in}}%
\pgfpathlineto{\pgfqpoint{2.928444in}{2.300316in}}%
\pgfpathlineto{\pgfqpoint{2.929234in}{2.305211in}}%
\pgfpathlineto{\pgfqpoint{2.930122in}{2.305080in}}%
\pgfpathlineto{\pgfqpoint{2.930615in}{2.351052in}}%
\pgfpathlineto{\pgfqpoint{2.932195in}{2.525530in}}%
\pgfpathlineto{\pgfqpoint{2.932589in}{2.510576in}}%
\pgfpathlineto{\pgfqpoint{2.932787in}{2.512479in}}%
\pgfpathlineto{\pgfqpoint{2.932885in}{2.510348in}}%
\pgfpathlineto{\pgfqpoint{2.933478in}{2.381230in}}%
\pgfpathlineto{\pgfqpoint{2.933971in}{2.261922in}}%
\pgfpathlineto{\pgfqpoint{2.934662in}{2.326399in}}%
\pgfpathlineto{\pgfqpoint{2.935254in}{2.320221in}}%
\pgfpathlineto{\pgfqpoint{2.935452in}{2.324399in}}%
\pgfpathlineto{\pgfqpoint{2.937228in}{2.385976in}}%
\pgfpathlineto{\pgfqpoint{2.937623in}{2.364774in}}%
\pgfpathlineto{\pgfqpoint{2.938413in}{2.337359in}}%
\pgfpathlineto{\pgfqpoint{2.938807in}{2.354242in}}%
\pgfpathlineto{\pgfqpoint{2.940387in}{2.381809in}}%
\pgfpathlineto{\pgfqpoint{2.940485in}{2.379200in}}%
\pgfpathlineto{\pgfqpoint{2.941176in}{2.334912in}}%
\pgfpathlineto{\pgfqpoint{2.941867in}{2.362710in}}%
\pgfpathlineto{\pgfqpoint{2.942163in}{2.374692in}}%
\pgfpathlineto{\pgfqpoint{2.942854in}{2.358677in}}%
\pgfpathlineto{\pgfqpoint{2.943446in}{2.386701in}}%
\pgfpathlineto{\pgfqpoint{2.943841in}{2.391232in}}%
\pgfpathlineto{\pgfqpoint{2.944038in}{2.386824in}}%
\pgfpathlineto{\pgfqpoint{2.944532in}{2.366102in}}%
\pgfpathlineto{\pgfqpoint{2.945025in}{2.382836in}}%
\pgfpathlineto{\pgfqpoint{2.946111in}{2.403558in}}%
\pgfpathlineto{\pgfqpoint{2.946309in}{2.396819in}}%
\pgfpathlineto{\pgfqpoint{2.946506in}{2.386863in}}%
\pgfpathlineto{\pgfqpoint{2.947098in}{2.414874in}}%
\pgfpathlineto{\pgfqpoint{2.948184in}{2.394891in}}%
\pgfpathlineto{\pgfqpoint{2.948381in}{2.388364in}}%
\pgfpathlineto{\pgfqpoint{2.948875in}{2.409126in}}%
\pgfpathlineto{\pgfqpoint{2.949171in}{2.399321in}}%
\pgfpathlineto{\pgfqpoint{2.949763in}{2.384263in}}%
\pgfpathlineto{\pgfqpoint{2.950158in}{2.400245in}}%
\pgfpathlineto{\pgfqpoint{2.950256in}{2.400048in}}%
\pgfpathlineto{\pgfqpoint{2.950553in}{2.387503in}}%
\pgfpathlineto{\pgfqpoint{2.951046in}{2.400226in}}%
\pgfpathlineto{\pgfqpoint{2.951441in}{2.394105in}}%
\pgfpathlineto{\pgfqpoint{2.952132in}{2.401849in}}%
\pgfpathlineto{\pgfqpoint{2.952428in}{2.395429in}}%
\pgfpathlineto{\pgfqpoint{2.952724in}{2.385359in}}%
\pgfpathlineto{\pgfqpoint{2.953217in}{2.398623in}}%
\pgfpathlineto{\pgfqpoint{2.953514in}{2.390821in}}%
\pgfpathlineto{\pgfqpoint{2.953908in}{2.408177in}}%
\pgfpathlineto{\pgfqpoint{2.954303in}{2.385794in}}%
\pgfpathlineto{\pgfqpoint{2.954501in}{2.379887in}}%
\pgfpathlineto{\pgfqpoint{2.955093in}{2.393558in}}%
\pgfpathlineto{\pgfqpoint{2.955290in}{2.397234in}}%
\pgfpathlineto{\pgfqpoint{2.955685in}{2.385052in}}%
\pgfpathlineto{\pgfqpoint{2.955882in}{2.387113in}}%
\pgfpathlineto{\pgfqpoint{2.957856in}{2.349718in}}%
\pgfpathlineto{\pgfqpoint{2.958152in}{2.353075in}}%
\pgfpathlineto{\pgfqpoint{2.958646in}{2.358411in}}%
\pgfpathlineto{\pgfqpoint{2.958942in}{2.352717in}}%
\pgfpathlineto{\pgfqpoint{2.959139in}{2.347266in}}%
\pgfpathlineto{\pgfqpoint{2.959633in}{2.361390in}}%
\pgfpathlineto{\pgfqpoint{2.959929in}{2.355787in}}%
\pgfpathlineto{\pgfqpoint{2.960028in}{2.355871in}}%
\pgfpathlineto{\pgfqpoint{2.960324in}{2.361656in}}%
\pgfpathlineto{\pgfqpoint{2.961015in}{2.354978in}}%
\pgfpathlineto{\pgfqpoint{2.961212in}{2.350504in}}%
\pgfpathlineto{\pgfqpoint{2.961706in}{2.358727in}}%
\pgfpathlineto{\pgfqpoint{2.962002in}{2.357813in}}%
\pgfpathlineto{\pgfqpoint{2.962693in}{2.354962in}}%
\pgfpathlineto{\pgfqpoint{2.963581in}{2.372143in}}%
\pgfpathlineto{\pgfqpoint{2.963778in}{2.367336in}}%
\pgfpathlineto{\pgfqpoint{2.964667in}{2.326655in}}%
\pgfpathlineto{\pgfqpoint{2.965357in}{2.332187in}}%
\pgfpathlineto{\pgfqpoint{2.966048in}{2.316473in}}%
\pgfpathlineto{\pgfqpoint{2.966443in}{2.329277in}}%
\pgfpathlineto{\pgfqpoint{2.968121in}{2.430384in}}%
\pgfpathlineto{\pgfqpoint{2.968911in}{2.417013in}}%
\pgfpathlineto{\pgfqpoint{2.970983in}{2.360520in}}%
\pgfpathlineto{\pgfqpoint{2.971082in}{2.362131in}}%
\pgfpathlineto{\pgfqpoint{2.971477in}{2.376287in}}%
\pgfpathlineto{\pgfqpoint{2.971872in}{2.354551in}}%
\pgfpathlineto{\pgfqpoint{2.972365in}{2.334568in}}%
\pgfpathlineto{\pgfqpoint{2.973056in}{2.344278in}}%
\pgfpathlineto{\pgfqpoint{2.973846in}{2.339552in}}%
\pgfpathlineto{\pgfqpoint{2.974536in}{2.362616in}}%
\pgfpathlineto{\pgfqpoint{2.974931in}{2.367212in}}%
\pgfpathlineto{\pgfqpoint{2.975129in}{2.359725in}}%
\pgfpathlineto{\pgfqpoint{2.975721in}{2.320484in}}%
\pgfpathlineto{\pgfqpoint{2.976510in}{2.331124in}}%
\pgfpathlineto{\pgfqpoint{2.976708in}{2.332565in}}%
\pgfpathlineto{\pgfqpoint{2.977892in}{2.393419in}}%
\pgfpathlineto{\pgfqpoint{2.978879in}{2.555088in}}%
\pgfpathlineto{\pgfqpoint{2.979471in}{2.531852in}}%
\pgfpathlineto{\pgfqpoint{2.979570in}{2.530506in}}%
\pgfpathlineto{\pgfqpoint{2.979767in}{2.537717in}}%
\pgfpathlineto{\pgfqpoint{2.979965in}{2.545876in}}%
\pgfpathlineto{\pgfqpoint{2.980261in}{2.514540in}}%
\pgfpathlineto{\pgfqpoint{2.981051in}{2.287156in}}%
\pgfpathlineto{\pgfqpoint{2.982038in}{2.345963in}}%
\pgfpathlineto{\pgfqpoint{2.982235in}{2.341950in}}%
\pgfpathlineto{\pgfqpoint{2.982630in}{2.359230in}}%
\pgfpathlineto{\pgfqpoint{2.984308in}{2.413907in}}%
\pgfpathlineto{\pgfqpoint{2.984604in}{2.400011in}}%
\pgfpathlineto{\pgfqpoint{2.985591in}{2.374554in}}%
\pgfpathlineto{\pgfqpoint{2.985788in}{2.380232in}}%
\pgfpathlineto{\pgfqpoint{2.987170in}{2.403084in}}%
\pgfpathlineto{\pgfqpoint{2.986479in}{2.372390in}}%
\pgfpathlineto{\pgfqpoint{2.987269in}{2.402384in}}%
\pgfpathlineto{\pgfqpoint{2.988749in}{2.369219in}}%
\pgfpathlineto{\pgfqpoint{2.989341in}{2.387893in}}%
\pgfpathlineto{\pgfqpoint{2.991611in}{2.416994in}}%
\pgfpathlineto{\pgfqpoint{2.991710in}{2.415872in}}%
\pgfpathlineto{\pgfqpoint{2.992105in}{2.406391in}}%
\pgfpathlineto{\pgfqpoint{2.992796in}{2.416563in}}%
\pgfpathlineto{\pgfqpoint{2.993092in}{2.426655in}}%
\pgfpathlineto{\pgfqpoint{2.993783in}{2.412379in}}%
\pgfpathlineto{\pgfqpoint{2.994079in}{2.413219in}}%
\pgfpathlineto{\pgfqpoint{2.995855in}{2.428900in}}%
\pgfpathlineto{\pgfqpoint{2.995954in}{2.428228in}}%
\pgfpathlineto{\pgfqpoint{2.996053in}{2.427264in}}%
\pgfpathlineto{\pgfqpoint{2.996250in}{2.431987in}}%
\pgfpathlineto{\pgfqpoint{2.996546in}{2.446771in}}%
\pgfpathlineto{\pgfqpoint{2.996941in}{2.430819in}}%
\pgfpathlineto{\pgfqpoint{2.997336in}{2.434221in}}%
\pgfpathlineto{\pgfqpoint{2.999014in}{2.410007in}}%
\pgfpathlineto{\pgfqpoint{2.999507in}{2.432783in}}%
\pgfpathlineto{\pgfqpoint{3.000099in}{2.411327in}}%
\pgfpathlineto{\pgfqpoint{3.000297in}{2.408017in}}%
\pgfpathlineto{\pgfqpoint{3.000593in}{2.418364in}}%
\pgfpathlineto{\pgfqpoint{3.000988in}{2.434176in}}%
\pgfpathlineto{\pgfqpoint{3.001777in}{2.425620in}}%
\pgfpathlineto{\pgfqpoint{3.002073in}{2.413604in}}%
\pgfpathlineto{\pgfqpoint{3.002567in}{2.431870in}}%
\pgfpathlineto{\pgfqpoint{3.002962in}{2.419644in}}%
\pgfpathlineto{\pgfqpoint{3.004738in}{2.401153in}}%
\pgfpathlineto{\pgfqpoint{3.004936in}{2.403272in}}%
\pgfpathlineto{\pgfqpoint{3.005034in}{2.403324in}}%
\pgfpathlineto{\pgfqpoint{3.007107in}{2.354207in}}%
\pgfpathlineto{\pgfqpoint{3.007502in}{2.367085in}}%
\pgfpathlineto{\pgfqpoint{3.007699in}{2.369359in}}%
\pgfpathlineto{\pgfqpoint{3.008094in}{2.356548in}}%
\pgfpathlineto{\pgfqpoint{3.008291in}{2.352321in}}%
\pgfpathlineto{\pgfqpoint{3.008982in}{2.359851in}}%
\pgfpathlineto{\pgfqpoint{3.009476in}{2.372980in}}%
\pgfpathlineto{\pgfqpoint{3.009871in}{2.358425in}}%
\pgfpathlineto{\pgfqpoint{3.010463in}{2.339934in}}%
\pgfpathlineto{\pgfqpoint{3.010660in}{2.352650in}}%
\pgfpathlineto{\pgfqpoint{3.010956in}{2.372165in}}%
\pgfpathlineto{\pgfqpoint{3.011450in}{2.347690in}}%
\pgfpathlineto{\pgfqpoint{3.011647in}{2.348918in}}%
\pgfpathlineto{\pgfqpoint{3.011943in}{2.345600in}}%
\pgfpathlineto{\pgfqpoint{3.012141in}{2.351368in}}%
\pgfpathlineto{\pgfqpoint{3.013128in}{2.369590in}}%
\pgfpathlineto{\pgfqpoint{3.013424in}{2.362362in}}%
\pgfpathlineto{\pgfqpoint{3.013720in}{2.350874in}}%
\pgfpathlineto{\pgfqpoint{3.014213in}{2.380593in}}%
\pgfpathlineto{\pgfqpoint{3.015496in}{2.393529in}}%
\pgfpathlineto{\pgfqpoint{3.015694in}{2.389416in}}%
\pgfpathlineto{\pgfqpoint{3.017964in}{2.330860in}}%
\pgfpathlineto{\pgfqpoint{3.018260in}{2.341870in}}%
\pgfpathlineto{\pgfqpoint{3.018951in}{2.343680in}}%
\pgfpathlineto{\pgfqpoint{3.018655in}{2.338558in}}%
\pgfpathlineto{\pgfqpoint{3.019148in}{2.340609in}}%
\pgfpathlineto{\pgfqpoint{3.019938in}{2.300906in}}%
\pgfpathlineto{\pgfqpoint{3.020530in}{2.322956in}}%
\pgfpathlineto{\pgfqpoint{3.020629in}{2.322975in}}%
\pgfpathlineto{\pgfqpoint{3.020728in}{2.322205in}}%
\pgfpathlineto{\pgfqpoint{3.021024in}{2.318806in}}%
\pgfpathlineto{\pgfqpoint{3.021320in}{2.326858in}}%
\pgfpathlineto{\pgfqpoint{3.022109in}{2.347683in}}%
\pgfpathlineto{\pgfqpoint{3.022504in}{2.333085in}}%
\pgfpathlineto{\pgfqpoint{3.023294in}{2.307087in}}%
\pgfpathlineto{\pgfqpoint{3.023787in}{2.320743in}}%
\pgfpathlineto{\pgfqpoint{3.024281in}{2.329486in}}%
\pgfpathlineto{\pgfqpoint{3.024972in}{2.356920in}}%
\pgfpathlineto{\pgfqpoint{3.026057in}{2.550436in}}%
\pgfpathlineto{\pgfqpoint{3.027538in}{2.522390in}}%
\pgfpathlineto{\pgfqpoint{3.028426in}{2.285233in}}%
\pgfpathlineto{\pgfqpoint{3.029610in}{2.350903in}}%
\pgfpathlineto{\pgfqpoint{3.030894in}{2.391784in}}%
\pgfpathlineto{\pgfqpoint{3.031584in}{2.407224in}}%
\pgfpathlineto{\pgfqpoint{3.031979in}{2.397448in}}%
\pgfpathlineto{\pgfqpoint{3.033065in}{2.384724in}}%
\pgfpathlineto{\pgfqpoint{3.033262in}{2.391857in}}%
\pgfpathlineto{\pgfqpoint{3.034841in}{2.437309in}}%
\pgfpathlineto{\pgfqpoint{3.035039in}{2.434767in}}%
\pgfpathlineto{\pgfqpoint{3.036223in}{2.416838in}}%
\pgfpathlineto{\pgfqpoint{3.036421in}{2.421663in}}%
\pgfpathlineto{\pgfqpoint{3.037605in}{2.445605in}}%
\pgfpathlineto{\pgfqpoint{3.037802in}{2.440897in}}%
\pgfpathlineto{\pgfqpoint{3.039875in}{2.385809in}}%
\pgfpathlineto{\pgfqpoint{3.040665in}{2.405420in}}%
\pgfpathlineto{\pgfqpoint{3.044317in}{2.504771in}}%
\pgfpathlineto{\pgfqpoint{3.044415in}{2.503883in}}%
\pgfpathlineto{\pgfqpoint{3.044909in}{2.489229in}}%
\pgfpathlineto{\pgfqpoint{3.045600in}{2.500536in}}%
\pgfpathlineto{\pgfqpoint{3.045698in}{2.500584in}}%
\pgfpathlineto{\pgfqpoint{3.045896in}{2.499879in}}%
\pgfpathlineto{\pgfqpoint{3.046093in}{2.499067in}}%
\pgfpathlineto{\pgfqpoint{3.046389in}{2.502543in}}%
\pgfpathlineto{\pgfqpoint{3.046883in}{2.515814in}}%
\pgfpathlineto{\pgfqpoint{3.047574in}{2.506459in}}%
\pgfpathlineto{\pgfqpoint{3.048067in}{2.496078in}}%
\pgfpathlineto{\pgfqpoint{3.048363in}{2.511603in}}%
\pgfpathlineto{\pgfqpoint{3.048561in}{2.520374in}}%
\pgfpathlineto{\pgfqpoint{3.049153in}{2.492017in}}%
\pgfpathlineto{\pgfqpoint{3.050140in}{2.505010in}}%
\pgfpathlineto{\pgfqpoint{3.050535in}{2.498801in}}%
\pgfpathlineto{\pgfqpoint{3.052607in}{2.453573in}}%
\pgfpathlineto{\pgfqpoint{3.053101in}{2.436805in}}%
\pgfpathlineto{\pgfqpoint{3.053199in}{2.433629in}}%
\pgfpathlineto{\pgfqpoint{3.053594in}{2.450849in}}%
\pgfpathlineto{\pgfqpoint{3.053693in}{2.453792in}}%
\pgfpathlineto{\pgfqpoint{3.054285in}{2.441911in}}%
\pgfpathlineto{\pgfqpoint{3.054680in}{2.429510in}}%
\pgfpathlineto{\pgfqpoint{3.055075in}{2.451257in}}%
\pgfpathlineto{\pgfqpoint{3.055173in}{2.453604in}}%
\pgfpathlineto{\pgfqpoint{3.055667in}{2.442878in}}%
\pgfpathlineto{\pgfqpoint{3.055766in}{2.442991in}}%
\pgfpathlineto{\pgfqpoint{3.055963in}{2.440817in}}%
\pgfpathlineto{\pgfqpoint{3.056259in}{2.434715in}}%
\pgfpathlineto{\pgfqpoint{3.056753in}{2.447725in}}%
\pgfpathlineto{\pgfqpoint{3.057246in}{2.435699in}}%
\pgfpathlineto{\pgfqpoint{3.057641in}{2.434598in}}%
\pgfpathlineto{\pgfqpoint{3.057937in}{2.431221in}}%
\pgfpathlineto{\pgfqpoint{3.058332in}{2.438064in}}%
\pgfpathlineto{\pgfqpoint{3.058431in}{2.438651in}}%
\pgfpathlineto{\pgfqpoint{3.058727in}{2.434000in}}%
\pgfpathlineto{\pgfqpoint{3.059418in}{2.423337in}}%
\pgfpathlineto{\pgfqpoint{3.059714in}{2.432293in}}%
\pgfpathlineto{\pgfqpoint{3.060306in}{2.455457in}}%
\pgfpathlineto{\pgfqpoint{3.061194in}{2.448311in}}%
\pgfpathlineto{\pgfqpoint{3.061392in}{2.444842in}}%
\pgfpathlineto{\pgfqpoint{3.061688in}{2.455055in}}%
\pgfpathlineto{\pgfqpoint{3.062378in}{2.469672in}}%
\pgfpathlineto{\pgfqpoint{3.062773in}{2.459636in}}%
\pgfpathlineto{\pgfqpoint{3.064056in}{2.437158in}}%
\pgfpathlineto{\pgfqpoint{3.064846in}{2.402630in}}%
\pgfpathlineto{\pgfqpoint{3.065734in}{2.405165in}}%
\pgfpathlineto{\pgfqpoint{3.066524in}{2.408174in}}%
\pgfpathlineto{\pgfqpoint{3.066623in}{2.406881in}}%
\pgfpathlineto{\pgfqpoint{3.067116in}{2.388202in}}%
\pgfpathlineto{\pgfqpoint{3.067906in}{2.396643in}}%
\pgfpathlineto{\pgfqpoint{3.069287in}{2.435595in}}%
\pgfpathlineto{\pgfqpoint{3.069584in}{2.427029in}}%
\pgfpathlineto{\pgfqpoint{3.071064in}{2.391737in}}%
\pgfpathlineto{\pgfqpoint{3.071163in}{2.393801in}}%
\pgfpathlineto{\pgfqpoint{3.073038in}{2.586154in}}%
\pgfpathlineto{\pgfqpoint{3.073531in}{2.643418in}}%
\pgfpathlineto{\pgfqpoint{3.074222in}{2.613014in}}%
\pgfpathlineto{\pgfqpoint{3.074518in}{2.617155in}}%
\pgfpathlineto{\pgfqpoint{3.074716in}{2.611949in}}%
\pgfpathlineto{\pgfqpoint{3.075308in}{2.464479in}}%
\pgfpathlineto{\pgfqpoint{3.075703in}{2.364161in}}%
\pgfpathlineto{\pgfqpoint{3.076394in}{2.441305in}}%
\pgfpathlineto{\pgfqpoint{3.077381in}{2.455388in}}%
\pgfpathlineto{\pgfqpoint{3.077578in}{2.448530in}}%
\pgfpathlineto{\pgfqpoint{3.077677in}{2.446369in}}%
\pgfpathlineto{\pgfqpoint{3.077973in}{2.463513in}}%
\pgfpathlineto{\pgfqpoint{3.078466in}{2.493287in}}%
\pgfpathlineto{\pgfqpoint{3.079256in}{2.484649in}}%
\pgfpathlineto{\pgfqpoint{3.079651in}{2.470197in}}%
\pgfpathlineto{\pgfqpoint{3.080539in}{2.472862in}}%
\pgfpathlineto{\pgfqpoint{3.080835in}{2.469563in}}%
\pgfpathlineto{\pgfqpoint{3.081033in}{2.478054in}}%
\pgfpathlineto{\pgfqpoint{3.082020in}{2.520124in}}%
\pgfpathlineto{\pgfqpoint{3.082316in}{2.501672in}}%
\pgfpathlineto{\pgfqpoint{3.082513in}{2.489869in}}%
\pgfpathlineto{\pgfqpoint{3.082908in}{2.502811in}}%
\pgfpathlineto{\pgfqpoint{3.083401in}{2.501840in}}%
\pgfpathlineto{\pgfqpoint{3.084981in}{2.530470in}}%
\pgfpathlineto{\pgfqpoint{3.083994in}{2.498331in}}%
\pgfpathlineto{\pgfqpoint{3.085178in}{2.524410in}}%
\pgfpathlineto{\pgfqpoint{3.085375in}{2.521099in}}%
\pgfpathlineto{\pgfqpoint{3.085671in}{2.527416in}}%
\pgfpathlineto{\pgfqpoint{3.086165in}{2.525953in}}%
\pgfpathlineto{\pgfqpoint{3.086560in}{2.543734in}}%
\pgfpathlineto{\pgfqpoint{3.087053in}{2.518246in}}%
\pgfpathlineto{\pgfqpoint{3.087547in}{2.540400in}}%
\pgfpathlineto{\pgfqpoint{3.089027in}{2.525999in}}%
\pgfpathlineto{\pgfqpoint{3.088238in}{2.546567in}}%
\pgfpathlineto{\pgfqpoint{3.089126in}{2.526824in}}%
\pgfpathlineto{\pgfqpoint{3.089619in}{2.559356in}}%
\pgfpathlineto{\pgfqpoint{3.090409in}{2.539220in}}%
\pgfpathlineto{\pgfqpoint{3.090705in}{2.531685in}}%
\pgfpathlineto{\pgfqpoint{3.091100in}{2.547083in}}%
\pgfpathlineto{\pgfqpoint{3.091396in}{2.555655in}}%
\pgfpathlineto{\pgfqpoint{3.091988in}{2.538777in}}%
\pgfpathlineto{\pgfqpoint{3.092482in}{2.527992in}}%
\pgfpathlineto{\pgfqpoint{3.092778in}{2.542760in}}%
\pgfpathlineto{\pgfqpoint{3.092975in}{2.549030in}}%
\pgfpathlineto{\pgfqpoint{3.093370in}{2.537764in}}%
\pgfpathlineto{\pgfqpoint{3.093863in}{2.542344in}}%
\pgfpathlineto{\pgfqpoint{3.094160in}{2.535731in}}%
\pgfpathlineto{\pgfqpoint{3.094850in}{2.545920in}}%
\pgfpathlineto{\pgfqpoint{3.095245in}{2.535324in}}%
\pgfpathlineto{\pgfqpoint{3.095344in}{2.534208in}}%
\pgfpathlineto{\pgfqpoint{3.095541in}{2.540892in}}%
\pgfpathlineto{\pgfqpoint{3.095739in}{2.547167in}}%
\pgfpathlineto{\pgfqpoint{3.096528in}{2.540245in}}%
\pgfpathlineto{\pgfqpoint{3.097417in}{2.528040in}}%
\pgfpathlineto{\pgfqpoint{3.097614in}{2.536030in}}%
\pgfpathlineto{\pgfqpoint{3.097811in}{2.541536in}}%
\pgfpathlineto{\pgfqpoint{3.098108in}{2.519094in}}%
\pgfpathlineto{\pgfqpoint{3.098996in}{2.504024in}}%
\pgfpathlineto{\pgfqpoint{3.099292in}{2.512206in}}%
\pgfpathlineto{\pgfqpoint{3.099391in}{2.513975in}}%
\pgfpathlineto{\pgfqpoint{3.099687in}{2.503280in}}%
\pgfpathlineto{\pgfqpoint{3.100180in}{2.476155in}}%
\pgfpathlineto{\pgfqpoint{3.100871in}{2.486991in}}%
\pgfpathlineto{\pgfqpoint{3.101266in}{2.510238in}}%
\pgfpathlineto{\pgfqpoint{3.101661in}{2.484982in}}%
\pgfpathlineto{\pgfqpoint{3.101957in}{2.491211in}}%
\pgfpathlineto{\pgfqpoint{3.102746in}{2.503850in}}%
\pgfpathlineto{\pgfqpoint{3.102352in}{2.487677in}}%
\pgfpathlineto{\pgfqpoint{3.102944in}{2.496648in}}%
\pgfpathlineto{\pgfqpoint{3.103832in}{2.478612in}}%
\pgfpathlineto{\pgfqpoint{3.104128in}{2.490788in}}%
\pgfpathlineto{\pgfqpoint{3.104326in}{2.495852in}}%
\pgfpathlineto{\pgfqpoint{3.104819in}{2.476370in}}%
\pgfpathlineto{\pgfqpoint{3.104918in}{2.477391in}}%
\pgfpathlineto{\pgfqpoint{3.105016in}{2.477411in}}%
\pgfpathlineto{\pgfqpoint{3.105411in}{2.469215in}}%
\pgfpathlineto{\pgfqpoint{3.105707in}{2.482560in}}%
\pgfpathlineto{\pgfqpoint{3.105905in}{2.487804in}}%
\pgfpathlineto{\pgfqpoint{3.106300in}{2.478599in}}%
\pgfpathlineto{\pgfqpoint{3.106596in}{2.479960in}}%
\pgfpathlineto{\pgfqpoint{3.106990in}{2.464092in}}%
\pgfpathlineto{\pgfqpoint{3.107484in}{2.486376in}}%
\pgfpathlineto{\pgfqpoint{3.107879in}{2.496767in}}%
\pgfpathlineto{\pgfqpoint{3.108175in}{2.483488in}}%
\pgfpathlineto{\pgfqpoint{3.108372in}{2.470888in}}%
\pgfpathlineto{\pgfqpoint{3.108964in}{2.505730in}}%
\pgfpathlineto{\pgfqpoint{3.109458in}{2.529744in}}%
\pgfpathlineto{\pgfqpoint{3.110050in}{2.506090in}}%
\pgfpathlineto{\pgfqpoint{3.112419in}{2.383657in}}%
\pgfpathlineto{\pgfqpoint{3.113208in}{2.413245in}}%
\pgfpathlineto{\pgfqpoint{3.114689in}{2.452232in}}%
\pgfpathlineto{\pgfqpoint{3.115084in}{2.447409in}}%
\pgfpathlineto{\pgfqpoint{3.116367in}{2.485061in}}%
\pgfpathlineto{\pgfqpoint{3.116762in}{2.467256in}}%
\pgfpathlineto{\pgfqpoint{3.117650in}{2.435359in}}%
\pgfpathlineto{\pgfqpoint{3.118045in}{2.451215in}}%
\pgfpathlineto{\pgfqpoint{3.118439in}{2.433557in}}%
\pgfpathlineto{\pgfqpoint{3.118736in}{2.452938in}}%
\pgfpathlineto{\pgfqpoint{3.120808in}{2.683460in}}%
\pgfpathlineto{\pgfqpoint{3.122091in}{2.639623in}}%
\pgfpathlineto{\pgfqpoint{3.122980in}{2.396418in}}%
\pgfpathlineto{\pgfqpoint{3.124065in}{2.472426in}}%
\pgfpathlineto{\pgfqpoint{3.124263in}{2.475221in}}%
\pgfpathlineto{\pgfqpoint{3.124855in}{2.467032in}}%
\pgfpathlineto{\pgfqpoint{3.125052in}{2.463957in}}%
\pgfpathlineto{\pgfqpoint{3.125348in}{2.479537in}}%
\pgfpathlineto{\pgfqpoint{3.126138in}{2.523770in}}%
\pgfpathlineto{\pgfqpoint{3.126533in}{2.489954in}}%
\pgfpathlineto{\pgfqpoint{3.126829in}{2.477041in}}%
\pgfpathlineto{\pgfqpoint{3.127619in}{2.484609in}}%
\pgfpathlineto{\pgfqpoint{3.127915in}{2.477000in}}%
\pgfpathlineto{\pgfqpoint{3.128211in}{2.464465in}}%
\pgfpathlineto{\pgfqpoint{3.128605in}{2.488288in}}%
\pgfpathlineto{\pgfqpoint{3.129000in}{2.514999in}}%
\pgfpathlineto{\pgfqpoint{3.129691in}{2.493043in}}%
\pgfpathlineto{\pgfqpoint{3.130086in}{2.475503in}}%
\pgfpathlineto{\pgfqpoint{3.130777in}{2.485350in}}%
\pgfpathlineto{\pgfqpoint{3.131468in}{2.477705in}}%
\pgfpathlineto{\pgfqpoint{3.132455in}{2.508127in}}%
\pgfpathlineto{\pgfqpoint{3.132553in}{2.508818in}}%
\pgfpathlineto{\pgfqpoint{3.132652in}{2.506085in}}%
\pgfpathlineto{\pgfqpoint{3.133442in}{2.485832in}}%
\pgfpathlineto{\pgfqpoint{3.133837in}{2.499812in}}%
\pgfpathlineto{\pgfqpoint{3.133935in}{2.500882in}}%
\pgfpathlineto{\pgfqpoint{3.134133in}{2.494495in}}%
\pgfpathlineto{\pgfqpoint{3.134626in}{2.485058in}}%
\pgfpathlineto{\pgfqpoint{3.135120in}{2.493981in}}%
\pgfpathlineto{\pgfqpoint{3.135416in}{2.499952in}}%
\pgfpathlineto{\pgfqpoint{3.136205in}{2.493317in}}%
\pgfpathlineto{\pgfqpoint{3.136403in}{2.493270in}}%
\pgfpathlineto{\pgfqpoint{3.136501in}{2.493814in}}%
\pgfpathlineto{\pgfqpoint{3.137390in}{2.510969in}}%
\pgfpathlineto{\pgfqpoint{3.137686in}{2.496220in}}%
\pgfpathlineto{\pgfqpoint{3.137883in}{2.484553in}}%
\pgfpathlineto{\pgfqpoint{3.138574in}{2.507071in}}%
\pgfpathlineto{\pgfqpoint{3.138870in}{2.513206in}}%
\pgfpathlineto{\pgfqpoint{3.139166in}{2.504196in}}%
\pgfpathlineto{\pgfqpoint{3.139956in}{2.485501in}}%
\pgfpathlineto{\pgfqpoint{3.140351in}{2.495408in}}%
\pgfpathlineto{\pgfqpoint{3.140449in}{2.498146in}}%
\pgfpathlineto{\pgfqpoint{3.140943in}{2.481713in}}%
\pgfpathlineto{\pgfqpoint{3.141140in}{2.478476in}}%
\pgfpathlineto{\pgfqpoint{3.141634in}{2.491650in}}%
\pgfpathlineto{\pgfqpoint{3.142127in}{2.498122in}}%
\pgfpathlineto{\pgfqpoint{3.142423in}{2.488568in}}%
\pgfpathlineto{\pgfqpoint{3.142719in}{2.476633in}}%
\pgfpathlineto{\pgfqpoint{3.143114in}{2.496103in}}%
\pgfpathlineto{\pgfqpoint{3.143608in}{2.482678in}}%
\pgfpathlineto{\pgfqpoint{3.143904in}{2.493569in}}%
\pgfpathlineto{\pgfqpoint{3.144496in}{2.469241in}}%
\pgfpathlineto{\pgfqpoint{3.144693in}{2.463194in}}%
\pgfpathlineto{\pgfqpoint{3.145088in}{2.472915in}}%
\pgfpathlineto{\pgfqpoint{3.145582in}{2.470272in}}%
\pgfpathlineto{\pgfqpoint{3.145878in}{2.457505in}}%
\pgfpathlineto{\pgfqpoint{3.146174in}{2.440738in}}%
\pgfpathlineto{\pgfqpoint{3.146865in}{2.458256in}}%
\pgfpathlineto{\pgfqpoint{3.146963in}{2.456182in}}%
\pgfpathlineto{\pgfqpoint{3.148049in}{2.424463in}}%
\pgfpathlineto{\pgfqpoint{3.148345in}{2.440269in}}%
\pgfpathlineto{\pgfqpoint{3.148543in}{2.447282in}}%
\pgfpathlineto{\pgfqpoint{3.148937in}{2.428917in}}%
\pgfpathlineto{\pgfqpoint{3.149234in}{2.435476in}}%
\pgfpathlineto{\pgfqpoint{3.149628in}{2.417620in}}%
\pgfpathlineto{\pgfqpoint{3.150023in}{2.444062in}}%
\pgfpathlineto{\pgfqpoint{3.150122in}{2.445089in}}%
\pgfpathlineto{\pgfqpoint{3.150319in}{2.437364in}}%
\pgfpathlineto{\pgfqpoint{3.151306in}{2.426993in}}%
\pgfpathlineto{\pgfqpoint{3.150813in}{2.439863in}}%
\pgfpathlineto{\pgfqpoint{3.151602in}{2.432058in}}%
\pgfpathlineto{\pgfqpoint{3.152195in}{2.444273in}}%
\pgfpathlineto{\pgfqpoint{3.152392in}{2.436432in}}%
\pgfpathlineto{\pgfqpoint{3.152688in}{2.417403in}}%
\pgfpathlineto{\pgfqpoint{3.153182in}{2.444769in}}%
\pgfpathlineto{\pgfqpoint{3.153379in}{2.442654in}}%
\pgfpathlineto{\pgfqpoint{3.154267in}{2.436143in}}%
\pgfpathlineto{\pgfqpoint{3.153872in}{2.447768in}}%
\pgfpathlineto{\pgfqpoint{3.154366in}{2.438453in}}%
\pgfpathlineto{\pgfqpoint{3.155254in}{2.475020in}}%
\pgfpathlineto{\pgfqpoint{3.155748in}{2.458532in}}%
\pgfpathlineto{\pgfqpoint{3.155846in}{2.457204in}}%
\pgfpathlineto{\pgfqpoint{3.156044in}{2.464797in}}%
\pgfpathlineto{\pgfqpoint{3.157031in}{2.498436in}}%
\pgfpathlineto{\pgfqpoint{3.157327in}{2.484627in}}%
\pgfpathlineto{\pgfqpoint{3.159301in}{2.404609in}}%
\pgfpathlineto{\pgfqpoint{3.160288in}{2.425329in}}%
\pgfpathlineto{\pgfqpoint{3.160485in}{2.417305in}}%
\pgfpathlineto{\pgfqpoint{3.161374in}{2.375565in}}%
\pgfpathlineto{\pgfqpoint{3.161768in}{2.390053in}}%
\pgfpathlineto{\pgfqpoint{3.161867in}{2.390561in}}%
\pgfpathlineto{\pgfqpoint{3.161966in}{2.387380in}}%
\pgfpathlineto{\pgfqpoint{3.162755in}{2.372894in}}%
\pgfpathlineto{\pgfqpoint{3.162953in}{2.378685in}}%
\pgfpathlineto{\pgfqpoint{3.163446in}{2.426573in}}%
\pgfpathlineto{\pgfqpoint{3.164137in}{2.401938in}}%
\pgfpathlineto{\pgfqpoint{3.164532in}{2.364432in}}%
\pgfpathlineto{\pgfqpoint{3.165618in}{2.367748in}}%
\pgfpathlineto{\pgfqpoint{3.165815in}{2.362946in}}%
\pgfpathlineto{\pgfqpoint{3.166111in}{2.382205in}}%
\pgfpathlineto{\pgfqpoint{3.167986in}{2.613844in}}%
\pgfpathlineto{\pgfqpoint{3.168875in}{2.583871in}}%
\pgfpathlineto{\pgfqpoint{3.169072in}{2.586617in}}%
\pgfpathlineto{\pgfqpoint{3.169171in}{2.585985in}}%
\pgfpathlineto{\pgfqpoint{3.169664in}{2.475625in}}%
\pgfpathlineto{\pgfqpoint{3.170158in}{2.357019in}}%
\pgfpathlineto{\pgfqpoint{3.170947in}{2.404128in}}%
\pgfpathlineto{\pgfqpoint{3.171243in}{2.409648in}}%
\pgfpathlineto{\pgfqpoint{3.171737in}{2.431281in}}%
\pgfpathlineto{\pgfqpoint{3.172230in}{2.406069in}}%
\pgfpathlineto{\pgfqpoint{3.172329in}{2.404543in}}%
\pgfpathlineto{\pgfqpoint{3.172527in}{2.415689in}}%
\pgfpathlineto{\pgfqpoint{3.173316in}{2.468486in}}%
\pgfpathlineto{\pgfqpoint{3.173810in}{2.431858in}}%
\pgfpathlineto{\pgfqpoint{3.174402in}{2.416197in}}%
\pgfpathlineto{\pgfqpoint{3.174797in}{2.435886in}}%
\pgfpathlineto{\pgfqpoint{3.175487in}{2.419621in}}%
\pgfpathlineto{\pgfqpoint{3.176376in}{2.464351in}}%
\pgfpathlineto{\pgfqpoint{3.176573in}{2.470224in}}%
\pgfpathlineto{\pgfqpoint{3.176968in}{2.447578in}}%
\pgfpathlineto{\pgfqpoint{3.177461in}{2.432034in}}%
\pgfpathlineto{\pgfqpoint{3.177856in}{2.449497in}}%
\pgfpathlineto{\pgfqpoint{3.178054in}{2.460047in}}%
\pgfpathlineto{\pgfqpoint{3.178745in}{2.434825in}}%
\pgfpathlineto{\pgfqpoint{3.180719in}{2.385632in}}%
\pgfpathlineto{\pgfqpoint{3.181113in}{2.397019in}}%
\pgfpathlineto{\pgfqpoint{3.182989in}{2.507626in}}%
\pgfpathlineto{\pgfqpoint{3.183778in}{2.505283in}}%
\pgfpathlineto{\pgfqpoint{3.184666in}{2.512579in}}%
\pgfpathlineto{\pgfqpoint{3.184765in}{2.513896in}}%
\pgfpathlineto{\pgfqpoint{3.185061in}{2.505474in}}%
\pgfpathlineto{\pgfqpoint{3.185357in}{2.496592in}}%
\pgfpathlineto{\pgfqpoint{3.185950in}{2.511497in}}%
\pgfpathlineto{\pgfqpoint{3.186246in}{2.517474in}}%
\pgfpathlineto{\pgfqpoint{3.186542in}{2.507241in}}%
\pgfpathlineto{\pgfqpoint{3.186838in}{2.494211in}}%
\pgfpathlineto{\pgfqpoint{3.187529in}{2.505194in}}%
\pgfpathlineto{\pgfqpoint{3.187825in}{2.514495in}}%
\pgfpathlineto{\pgfqpoint{3.188614in}{2.504146in}}%
\pgfpathlineto{\pgfqpoint{3.188911in}{2.502753in}}%
\pgfpathlineto{\pgfqpoint{3.189108in}{2.505227in}}%
\pgfpathlineto{\pgfqpoint{3.189503in}{2.517064in}}%
\pgfpathlineto{\pgfqpoint{3.190194in}{2.511213in}}%
\pgfpathlineto{\pgfqpoint{3.190490in}{2.499336in}}%
\pgfpathlineto{\pgfqpoint{3.190983in}{2.516332in}}%
\pgfpathlineto{\pgfqpoint{3.191279in}{2.508642in}}%
\pgfpathlineto{\pgfqpoint{3.191477in}{2.509869in}}%
\pgfpathlineto{\pgfqpoint{3.191575in}{2.508031in}}%
\pgfpathlineto{\pgfqpoint{3.193549in}{2.461941in}}%
\pgfpathlineto{\pgfqpoint{3.193944in}{2.452580in}}%
\pgfpathlineto{\pgfqpoint{3.194339in}{2.467269in}}%
\pgfpathlineto{\pgfqpoint{3.194536in}{2.470849in}}%
\pgfpathlineto{\pgfqpoint{3.194832in}{2.454191in}}%
\pgfpathlineto{\pgfqpoint{3.195227in}{2.425963in}}%
\pgfpathlineto{\pgfqpoint{3.195918in}{2.448539in}}%
\pgfpathlineto{\pgfqpoint{3.196214in}{2.455604in}}%
\pgfpathlineto{\pgfqpoint{3.196609in}{2.433575in}}%
\pgfpathlineto{\pgfqpoint{3.197201in}{2.424146in}}%
\pgfpathlineto{\pgfqpoint{3.197596in}{2.436830in}}%
\pgfpathlineto{\pgfqpoint{3.197793in}{2.442291in}}%
\pgfpathlineto{\pgfqpoint{3.198287in}{2.419899in}}%
\pgfpathlineto{\pgfqpoint{3.198682in}{2.410889in}}%
\pgfpathlineto{\pgfqpoint{3.199175in}{2.426314in}}%
\pgfpathlineto{\pgfqpoint{3.199373in}{2.430826in}}%
\pgfpathlineto{\pgfqpoint{3.199866in}{2.415201in}}%
\pgfpathlineto{\pgfqpoint{3.200162in}{2.409393in}}%
\pgfpathlineto{\pgfqpoint{3.200656in}{2.420555in}}%
\pgfpathlineto{\pgfqpoint{3.201248in}{2.431230in}}%
\pgfpathlineto{\pgfqpoint{3.201939in}{2.427351in}}%
\pgfpathlineto{\pgfqpoint{3.202136in}{2.425311in}}%
\pgfpathlineto{\pgfqpoint{3.202334in}{2.422577in}}%
\pgfpathlineto{\pgfqpoint{3.202728in}{2.436458in}}%
\pgfpathlineto{\pgfqpoint{3.202827in}{2.436467in}}%
\pgfpathlineto{\pgfqpoint{3.203024in}{2.433178in}}%
\pgfpathlineto{\pgfqpoint{3.203321in}{2.443990in}}%
\pgfpathlineto{\pgfqpoint{3.204406in}{2.450992in}}%
\pgfpathlineto{\pgfqpoint{3.203913in}{2.435173in}}%
\pgfpathlineto{\pgfqpoint{3.204505in}{2.449350in}}%
\pgfpathlineto{\pgfqpoint{3.206676in}{2.372943in}}%
\pgfpathlineto{\pgfqpoint{3.207170in}{2.379965in}}%
\pgfpathlineto{\pgfqpoint{3.207565in}{2.408629in}}%
\pgfpathlineto{\pgfqpoint{3.207959in}{2.371382in}}%
\pgfpathlineto{\pgfqpoint{3.208749in}{2.351910in}}%
\pgfpathlineto{\pgfqpoint{3.209045in}{2.365137in}}%
\pgfpathlineto{\pgfqpoint{3.209144in}{2.368121in}}%
\pgfpathlineto{\pgfqpoint{3.209835in}{2.358290in}}%
\pgfpathlineto{\pgfqpoint{3.210032in}{2.352620in}}%
\pgfpathlineto{\pgfqpoint{3.210328in}{2.381447in}}%
\pgfpathlineto{\pgfqpoint{3.210624in}{2.406855in}}%
\pgfpathlineto{\pgfqpoint{3.211217in}{2.371863in}}%
\pgfpathlineto{\pgfqpoint{3.211611in}{2.334842in}}%
\pgfpathlineto{\pgfqpoint{3.212401in}{2.354499in}}%
\pgfpathlineto{\pgfqpoint{3.212993in}{2.339564in}}%
\pgfpathlineto{\pgfqpoint{3.214474in}{2.503377in}}%
\pgfpathlineto{\pgfqpoint{3.215559in}{2.569694in}}%
\pgfpathlineto{\pgfqpoint{3.215855in}{2.565544in}}%
\pgfpathlineto{\pgfqpoint{3.216250in}{2.542895in}}%
\pgfpathlineto{\pgfqpoint{3.217237in}{2.295062in}}%
\pgfpathlineto{\pgfqpoint{3.218422in}{2.351598in}}%
\pgfpathlineto{\pgfqpoint{3.218619in}{2.355021in}}%
\pgfpathlineto{\pgfqpoint{3.220494in}{2.415146in}}%
\pgfpathlineto{\pgfqpoint{3.220593in}{2.411551in}}%
\pgfpathlineto{\pgfqpoint{3.221580in}{2.360032in}}%
\pgfpathlineto{\pgfqpoint{3.222073in}{2.379279in}}%
\pgfpathlineto{\pgfqpoint{3.222271in}{2.377244in}}%
\pgfpathlineto{\pgfqpoint{3.222468in}{2.374483in}}%
\pgfpathlineto{\pgfqpoint{3.222962in}{2.381029in}}%
\pgfpathlineto{\pgfqpoint{3.223653in}{2.414990in}}%
\pgfpathlineto{\pgfqpoint{3.224146in}{2.388925in}}%
\pgfpathlineto{\pgfqpoint{3.224640in}{2.370993in}}%
\pgfpathlineto{\pgfqpoint{3.225133in}{2.392185in}}%
\pgfpathlineto{\pgfqpoint{3.227206in}{2.426749in}}%
\pgfpathlineto{\pgfqpoint{3.227304in}{2.427186in}}%
\pgfpathlineto{\pgfqpoint{3.227403in}{2.424812in}}%
\pgfpathlineto{\pgfqpoint{3.228193in}{2.403668in}}%
\pgfpathlineto{\pgfqpoint{3.228489in}{2.416483in}}%
\pgfpathlineto{\pgfqpoint{3.228785in}{2.433944in}}%
\pgfpathlineto{\pgfqpoint{3.229575in}{2.422341in}}%
\pgfpathlineto{\pgfqpoint{3.229969in}{2.410750in}}%
\pgfpathlineto{\pgfqpoint{3.230265in}{2.428718in}}%
\pgfpathlineto{\pgfqpoint{3.230562in}{2.443086in}}%
\pgfpathlineto{\pgfqpoint{3.231154in}{2.416989in}}%
\pgfpathlineto{\pgfqpoint{3.231351in}{2.416007in}}%
\pgfpathlineto{\pgfqpoint{3.231548in}{2.420705in}}%
\pgfpathlineto{\pgfqpoint{3.232437in}{2.437303in}}%
\pgfpathlineto{\pgfqpoint{3.232733in}{2.428039in}}%
\pgfpathlineto{\pgfqpoint{3.232930in}{2.421682in}}%
\pgfpathlineto{\pgfqpoint{3.233424in}{2.436585in}}%
\pgfpathlineto{\pgfqpoint{3.233720in}{2.431507in}}%
\pgfpathlineto{\pgfqpoint{3.235398in}{2.449555in}}%
\pgfpathlineto{\pgfqpoint{3.234509in}{2.429020in}}%
\pgfpathlineto{\pgfqpoint{3.235595in}{2.441838in}}%
\pgfpathlineto{\pgfqpoint{3.236385in}{2.429444in}}%
\pgfpathlineto{\pgfqpoint{3.236780in}{2.436010in}}%
\pgfpathlineto{\pgfqpoint{3.236977in}{2.433041in}}%
\pgfpathlineto{\pgfqpoint{3.237273in}{2.442608in}}%
\pgfpathlineto{\pgfqpoint{3.237372in}{2.445777in}}%
\pgfpathlineto{\pgfqpoint{3.237767in}{2.428943in}}%
\pgfpathlineto{\pgfqpoint{3.238754in}{2.432579in}}%
\pgfpathlineto{\pgfqpoint{3.239148in}{2.418008in}}%
\pgfpathlineto{\pgfqpoint{3.241221in}{2.384037in}}%
\pgfpathlineto{\pgfqpoint{3.241418in}{2.391242in}}%
\pgfpathlineto{\pgfqpoint{3.241813in}{2.402058in}}%
\pgfpathlineto{\pgfqpoint{3.242307in}{2.382648in}}%
\pgfpathlineto{\pgfqpoint{3.242701in}{2.368907in}}%
\pgfpathlineto{\pgfqpoint{3.243294in}{2.385759in}}%
\pgfpathlineto{\pgfqpoint{3.243392in}{2.386391in}}%
\pgfpathlineto{\pgfqpoint{3.243688in}{2.382277in}}%
\pgfpathlineto{\pgfqpoint{3.244379in}{2.360683in}}%
\pgfpathlineto{\pgfqpoint{3.244774in}{2.380295in}}%
\pgfpathlineto{\pgfqpoint{3.245465in}{2.386369in}}%
\pgfpathlineto{\pgfqpoint{3.245662in}{2.381198in}}%
\pgfpathlineto{\pgfqpoint{3.246255in}{2.362023in}}%
\pgfpathlineto{\pgfqpoint{3.246847in}{2.377307in}}%
\pgfpathlineto{\pgfqpoint{3.246946in}{2.379743in}}%
\pgfpathlineto{\pgfqpoint{3.247439in}{2.369310in}}%
\pgfpathlineto{\pgfqpoint{3.247636in}{2.370553in}}%
\pgfpathlineto{\pgfqpoint{3.248031in}{2.355287in}}%
\pgfpathlineto{\pgfqpoint{3.248327in}{2.376364in}}%
\pgfpathlineto{\pgfqpoint{3.248722in}{2.396452in}}%
\pgfpathlineto{\pgfqpoint{3.249314in}{2.363646in}}%
\pgfpathlineto{\pgfqpoint{3.249413in}{2.360767in}}%
\pgfpathlineto{\pgfqpoint{3.249709in}{2.380795in}}%
\pgfpathlineto{\pgfqpoint{3.249906in}{2.388478in}}%
\pgfpathlineto{\pgfqpoint{3.250301in}{2.372264in}}%
\pgfpathlineto{\pgfqpoint{3.250696in}{2.373272in}}%
\pgfpathlineto{\pgfqpoint{3.251091in}{2.354331in}}%
\pgfpathlineto{\pgfqpoint{3.251683in}{2.374737in}}%
\pgfpathlineto{\pgfqpoint{3.252078in}{2.359277in}}%
\pgfpathlineto{\pgfqpoint{3.252177in}{2.357894in}}%
\pgfpathlineto{\pgfqpoint{3.252473in}{2.366179in}}%
\pgfpathlineto{\pgfqpoint{3.253460in}{2.406294in}}%
\pgfpathlineto{\pgfqpoint{3.253854in}{2.388748in}}%
\pgfpathlineto{\pgfqpoint{3.253953in}{2.385792in}}%
\pgfpathlineto{\pgfqpoint{3.254644in}{2.397932in}}%
\pgfpathlineto{\pgfqpoint{3.254940in}{2.411434in}}%
\pgfpathlineto{\pgfqpoint{3.255434in}{2.385687in}}%
\pgfpathlineto{\pgfqpoint{3.255927in}{2.356993in}}%
\pgfpathlineto{\pgfqpoint{3.256618in}{2.373969in}}%
\pgfpathlineto{\pgfqpoint{3.257408in}{2.371333in}}%
\pgfpathlineto{\pgfqpoint{3.257901in}{2.418294in}}%
\pgfpathlineto{\pgfqpoint{3.258000in}{2.422272in}}%
\pgfpathlineto{\pgfqpoint{3.258592in}{2.403913in}}%
\pgfpathlineto{\pgfqpoint{3.259184in}{2.370049in}}%
\pgfpathlineto{\pgfqpoint{3.260072in}{2.380583in}}%
\pgfpathlineto{\pgfqpoint{3.260467in}{2.393658in}}%
\pgfpathlineto{\pgfqpoint{3.261652in}{2.499375in}}%
\pgfpathlineto{\pgfqpoint{3.263330in}{2.631385in}}%
\pgfpathlineto{\pgfqpoint{3.263724in}{2.574132in}}%
\pgfpathlineto{\pgfqpoint{3.264613in}{2.356697in}}%
\pgfpathlineto{\pgfqpoint{3.265402in}{2.409424in}}%
\pgfpathlineto{\pgfqpoint{3.265600in}{2.403357in}}%
\pgfpathlineto{\pgfqpoint{3.266192in}{2.420856in}}%
\pgfpathlineto{\pgfqpoint{3.267968in}{2.474662in}}%
\pgfpathlineto{\pgfqpoint{3.268166in}{2.469712in}}%
\pgfpathlineto{\pgfqpoint{3.268857in}{2.427140in}}%
\pgfpathlineto{\pgfqpoint{3.269745in}{2.439141in}}%
\pgfpathlineto{\pgfqpoint{3.271028in}{2.477307in}}%
\pgfpathlineto{\pgfqpoint{3.271522in}{2.455036in}}%
\pgfpathlineto{\pgfqpoint{3.272509in}{2.428107in}}%
\pgfpathlineto{\pgfqpoint{3.272706in}{2.438822in}}%
\pgfpathlineto{\pgfqpoint{3.273002in}{2.461585in}}%
\pgfpathlineto{\pgfqpoint{3.273890in}{2.447394in}}%
\pgfpathlineto{\pgfqpoint{3.274877in}{2.462488in}}%
\pgfpathlineto{\pgfqpoint{3.275173in}{2.453386in}}%
\pgfpathlineto{\pgfqpoint{3.275568in}{2.445493in}}%
\pgfpathlineto{\pgfqpoint{3.276160in}{2.454431in}}%
\pgfpathlineto{\pgfqpoint{3.276753in}{2.466540in}}%
\pgfpathlineto{\pgfqpoint{3.277049in}{2.451961in}}%
\pgfpathlineto{\pgfqpoint{3.277147in}{2.448527in}}%
\pgfpathlineto{\pgfqpoint{3.277542in}{2.467925in}}%
\pgfpathlineto{\pgfqpoint{3.277740in}{2.474496in}}%
\pgfpathlineto{\pgfqpoint{3.278332in}{2.465764in}}%
\pgfpathlineto{\pgfqpoint{3.278529in}{2.465830in}}%
\pgfpathlineto{\pgfqpoint{3.278924in}{2.456215in}}%
\pgfpathlineto{\pgfqpoint{3.279319in}{2.468938in}}%
\pgfpathlineto{\pgfqpoint{3.279417in}{2.470972in}}%
\pgfpathlineto{\pgfqpoint{3.279812in}{2.463367in}}%
\pgfpathlineto{\pgfqpoint{3.280108in}{2.465948in}}%
\pgfpathlineto{\pgfqpoint{3.280602in}{2.445280in}}%
\pgfpathlineto{\pgfqpoint{3.281095in}{2.466897in}}%
\pgfpathlineto{\pgfqpoint{3.281194in}{2.468369in}}%
\pgfpathlineto{\pgfqpoint{3.281688in}{2.459717in}}%
\pgfpathlineto{\pgfqpoint{3.282082in}{2.448771in}}%
\pgfpathlineto{\pgfqpoint{3.282675in}{2.460228in}}%
\pgfpathlineto{\pgfqpoint{3.283069in}{2.465358in}}%
\pgfpathlineto{\pgfqpoint{3.283267in}{2.458527in}}%
\pgfpathlineto{\pgfqpoint{3.283563in}{2.444496in}}%
\pgfpathlineto{\pgfqpoint{3.284352in}{2.453263in}}%
\pgfpathlineto{\pgfqpoint{3.284649in}{2.474760in}}%
\pgfpathlineto{\pgfqpoint{3.285142in}{2.443025in}}%
\pgfpathlineto{\pgfqpoint{3.285339in}{2.445683in}}%
\pgfpathlineto{\pgfqpoint{3.286030in}{2.450413in}}%
\pgfpathlineto{\pgfqpoint{3.286425in}{2.445859in}}%
\pgfpathlineto{\pgfqpoint{3.288399in}{2.408250in}}%
\pgfpathlineto{\pgfqpoint{3.289189in}{2.412188in}}%
\pgfpathlineto{\pgfqpoint{3.289682in}{2.392650in}}%
\pgfpathlineto{\pgfqpoint{3.290867in}{2.408267in}}%
\pgfpathlineto{\pgfqpoint{3.290373in}{2.379274in}}%
\pgfpathlineto{\pgfqpoint{3.290965in}{2.404564in}}%
\pgfpathlineto{\pgfqpoint{3.292051in}{2.370600in}}%
\pgfpathlineto{\pgfqpoint{3.292347in}{2.382208in}}%
\pgfpathlineto{\pgfqpoint{3.292643in}{2.395566in}}%
\pgfpathlineto{\pgfqpoint{3.293334in}{2.377643in}}%
\pgfpathlineto{\pgfqpoint{3.293828in}{2.357763in}}%
\pgfpathlineto{\pgfqpoint{3.294222in}{2.380549in}}%
\pgfpathlineto{\pgfqpoint{3.294321in}{2.384361in}}%
\pgfpathlineto{\pgfqpoint{3.294913in}{2.367110in}}%
\pgfpathlineto{\pgfqpoint{3.295209in}{2.364878in}}%
\pgfpathlineto{\pgfqpoint{3.295407in}{2.362997in}}%
\pgfpathlineto{\pgfqpoint{3.295900in}{2.369602in}}%
\pgfpathlineto{\pgfqpoint{3.296098in}{2.372012in}}%
\pgfpathlineto{\pgfqpoint{3.297282in}{2.386228in}}%
\pgfpathlineto{\pgfqpoint{3.296788in}{2.365073in}}%
\pgfpathlineto{\pgfqpoint{3.297479in}{2.385152in}}%
\pgfpathlineto{\pgfqpoint{3.297578in}{2.385115in}}%
\pgfpathlineto{\pgfqpoint{3.299355in}{2.424439in}}%
\pgfpathlineto{\pgfqpoint{3.299749in}{2.408602in}}%
\pgfpathlineto{\pgfqpoint{3.301723in}{2.365138in}}%
\pgfpathlineto{\pgfqpoint{3.301822in}{2.364123in}}%
\pgfpathlineto{\pgfqpoint{3.302020in}{2.367789in}}%
\pgfpathlineto{\pgfqpoint{3.302513in}{2.389617in}}%
\pgfpathlineto{\pgfqpoint{3.303204in}{2.371821in}}%
\pgfpathlineto{\pgfqpoint{3.303697in}{2.345075in}}%
\pgfpathlineto{\pgfqpoint{3.304684in}{2.349478in}}%
\pgfpathlineto{\pgfqpoint{3.304981in}{2.349425in}}%
\pgfpathlineto{\pgfqpoint{3.305079in}{2.349648in}}%
\pgfpathlineto{\pgfqpoint{3.305474in}{2.378675in}}%
\pgfpathlineto{\pgfqpoint{3.305869in}{2.414824in}}%
\pgfpathlineto{\pgfqpoint{3.306461in}{2.385227in}}%
\pgfpathlineto{\pgfqpoint{3.306856in}{2.354864in}}%
\pgfpathlineto{\pgfqpoint{3.307645in}{2.365056in}}%
\pgfpathlineto{\pgfqpoint{3.308336in}{2.362546in}}%
\pgfpathlineto{\pgfqpoint{3.309422in}{2.452517in}}%
\pgfpathlineto{\pgfqpoint{3.310606in}{2.642318in}}%
\pgfpathlineto{\pgfqpoint{3.311199in}{2.591897in}}%
\pgfpathlineto{\pgfqpoint{3.311692in}{2.545747in}}%
\pgfpathlineto{\pgfqpoint{3.312482in}{2.351089in}}%
\pgfpathlineto{\pgfqpoint{3.313370in}{2.381802in}}%
\pgfpathlineto{\pgfqpoint{3.313469in}{2.380593in}}%
\pgfpathlineto{\pgfqpoint{3.313666in}{2.387090in}}%
\pgfpathlineto{\pgfqpoint{3.315739in}{2.470332in}}%
\pgfpathlineto{\pgfqpoint{3.315936in}{2.457180in}}%
\pgfpathlineto{\pgfqpoint{3.316528in}{2.397730in}}%
\pgfpathlineto{\pgfqpoint{3.317219in}{2.425458in}}%
\pgfpathlineto{\pgfqpoint{3.317515in}{2.429799in}}%
\pgfpathlineto{\pgfqpoint{3.317910in}{2.418981in}}%
\pgfpathlineto{\pgfqpoint{3.318009in}{2.415744in}}%
\pgfpathlineto{\pgfqpoint{3.318305in}{2.430573in}}%
\pgfpathlineto{\pgfqpoint{3.318798in}{2.474827in}}%
\pgfpathlineto{\pgfqpoint{3.319489in}{2.440548in}}%
\pgfpathlineto{\pgfqpoint{3.319983in}{2.417600in}}%
\pgfpathlineto{\pgfqpoint{3.320476in}{2.444168in}}%
\pgfpathlineto{\pgfqpoint{3.320674in}{2.435389in}}%
\pgfpathlineto{\pgfqpoint{3.320772in}{2.431910in}}%
\pgfpathlineto{\pgfqpoint{3.321167in}{2.442788in}}%
\pgfpathlineto{\pgfqpoint{3.321463in}{2.441083in}}%
\pgfpathlineto{\pgfqpoint{3.321957in}{2.468965in}}%
\pgfpathlineto{\pgfqpoint{3.322648in}{2.448278in}}%
\pgfpathlineto{\pgfqpoint{3.323042in}{2.425613in}}%
\pgfpathlineto{\pgfqpoint{3.324029in}{2.426039in}}%
\pgfpathlineto{\pgfqpoint{3.325115in}{2.406272in}}%
\pgfpathlineto{\pgfqpoint{3.325312in}{2.416385in}}%
\pgfpathlineto{\pgfqpoint{3.327484in}{2.513084in}}%
\pgfpathlineto{\pgfqpoint{3.327681in}{2.511390in}}%
\pgfpathlineto{\pgfqpoint{3.328076in}{2.495039in}}%
\pgfpathlineto{\pgfqpoint{3.328570in}{2.516605in}}%
\pgfpathlineto{\pgfqpoint{3.328668in}{2.517632in}}%
\pgfpathlineto{\pgfqpoint{3.328964in}{2.511579in}}%
\pgfpathlineto{\pgfqpoint{3.329458in}{2.481632in}}%
\pgfpathlineto{\pgfqpoint{3.330149in}{2.501445in}}%
\pgfpathlineto{\pgfqpoint{3.330544in}{2.514719in}}%
\pgfpathlineto{\pgfqpoint{3.331136in}{2.495727in}}%
\pgfpathlineto{\pgfqpoint{3.331234in}{2.495447in}}%
\pgfpathlineto{\pgfqpoint{3.331333in}{2.498058in}}%
\pgfpathlineto{\pgfqpoint{3.331728in}{2.513255in}}%
\pgfpathlineto{\pgfqpoint{3.332419in}{2.504063in}}%
\pgfpathlineto{\pgfqpoint{3.332715in}{2.495120in}}%
\pgfpathlineto{\pgfqpoint{3.333603in}{2.501333in}}%
\pgfpathlineto{\pgfqpoint{3.333801in}{2.508960in}}%
\pgfpathlineto{\pgfqpoint{3.334294in}{2.485838in}}%
\pgfpathlineto{\pgfqpoint{3.334491in}{2.482288in}}%
\pgfpathlineto{\pgfqpoint{3.335084in}{2.491699in}}%
\pgfpathlineto{\pgfqpoint{3.335281in}{2.495602in}}%
\pgfpathlineto{\pgfqpoint{3.335676in}{2.482872in}}%
\pgfpathlineto{\pgfqpoint{3.336465in}{2.462623in}}%
\pgfpathlineto{\pgfqpoint{3.336860in}{2.473991in}}%
\pgfpathlineto{\pgfqpoint{3.337058in}{2.478404in}}%
\pgfpathlineto{\pgfqpoint{3.337452in}{2.457114in}}%
\pgfpathlineto{\pgfqpoint{3.338045in}{2.455893in}}%
\pgfpathlineto{\pgfqpoint{3.337749in}{2.459221in}}%
\pgfpathlineto{\pgfqpoint{3.338242in}{2.457939in}}%
\pgfpathlineto{\pgfqpoint{3.338834in}{2.465492in}}%
\pgfpathlineto{\pgfqpoint{3.339130in}{2.454723in}}%
\pgfpathlineto{\pgfqpoint{3.339328in}{2.444974in}}%
\pgfpathlineto{\pgfqpoint{3.339821in}{2.462835in}}%
\pgfpathlineto{\pgfqpoint{3.340117in}{2.462374in}}%
\pgfpathlineto{\pgfqpoint{3.340413in}{2.466718in}}%
\pgfpathlineto{\pgfqpoint{3.340710in}{2.455988in}}%
\pgfpathlineto{\pgfqpoint{3.341499in}{2.447148in}}%
\pgfpathlineto{\pgfqpoint{3.341795in}{2.451998in}}%
\pgfpathlineto{\pgfqpoint{3.341993in}{2.455972in}}%
\pgfpathlineto{\pgfqpoint{3.342486in}{2.441562in}}%
\pgfpathlineto{\pgfqpoint{3.342782in}{2.434074in}}%
\pgfpathlineto{\pgfqpoint{3.343276in}{2.451138in}}%
\pgfpathlineto{\pgfqpoint{3.343572in}{2.456874in}}%
\pgfpathlineto{\pgfqpoint{3.344065in}{2.445998in}}%
\pgfpathlineto{\pgfqpoint{3.344361in}{2.439782in}}%
\pgfpathlineto{\pgfqpoint{3.344855in}{2.456285in}}%
\pgfpathlineto{\pgfqpoint{3.346533in}{2.485499in}}%
\pgfpathlineto{\pgfqpoint{3.346730in}{2.482896in}}%
\pgfpathlineto{\pgfqpoint{3.347322in}{2.490076in}}%
\pgfpathlineto{\pgfqpoint{3.348112in}{2.475315in}}%
\pgfpathlineto{\pgfqpoint{3.348605in}{2.455519in}}%
\pgfpathlineto{\pgfqpoint{3.349691in}{2.418228in}}%
\pgfpathlineto{\pgfqpoint{3.349987in}{2.425283in}}%
\pgfpathlineto{\pgfqpoint{3.350283in}{2.430328in}}%
\pgfpathlineto{\pgfqpoint{3.350678in}{2.418814in}}%
\pgfpathlineto{\pgfqpoint{3.352455in}{2.364162in}}%
\pgfpathlineto{\pgfqpoint{3.352751in}{2.370834in}}%
\pgfpathlineto{\pgfqpoint{3.353639in}{2.424260in}}%
\pgfpathlineto{\pgfqpoint{3.354231in}{2.391598in}}%
\pgfpathlineto{\pgfqpoint{3.354725in}{2.364996in}}%
\pgfpathlineto{\pgfqpoint{3.356008in}{2.369345in}}%
\pgfpathlineto{\pgfqpoint{3.357488in}{2.495185in}}%
\pgfpathlineto{\pgfqpoint{3.358475in}{2.665638in}}%
\pgfpathlineto{\pgfqpoint{3.359068in}{2.614529in}}%
\pgfpathlineto{\pgfqpoint{3.359561in}{2.579194in}}%
\pgfpathlineto{\pgfqpoint{3.360351in}{2.362152in}}%
\pgfpathlineto{\pgfqpoint{3.361239in}{2.423607in}}%
\pgfpathlineto{\pgfqpoint{3.361338in}{2.423578in}}%
\pgfpathlineto{\pgfqpoint{3.362917in}{2.477801in}}%
\pgfpathlineto{\pgfqpoint{3.363312in}{2.510181in}}%
\pgfpathlineto{\pgfqpoint{3.363904in}{2.472422in}}%
\pgfpathlineto{\pgfqpoint{3.364397in}{2.453459in}}%
\pgfpathlineto{\pgfqpoint{3.364891in}{2.474772in}}%
\pgfpathlineto{\pgfqpoint{3.364989in}{2.476124in}}%
\pgfpathlineto{\pgfqpoint{3.365187in}{2.468110in}}%
\pgfpathlineto{\pgfqpoint{3.365483in}{2.451376in}}%
\pgfpathlineto{\pgfqpoint{3.365976in}{2.485904in}}%
\pgfpathlineto{\pgfqpoint{3.366766in}{2.517001in}}%
\pgfpathlineto{\pgfqpoint{3.367260in}{2.500281in}}%
\pgfpathlineto{\pgfqpoint{3.367556in}{2.490089in}}%
\pgfpathlineto{\pgfqpoint{3.368049in}{2.511820in}}%
\pgfpathlineto{\pgfqpoint{3.368148in}{2.511998in}}%
\pgfpathlineto{\pgfqpoint{3.368247in}{2.510702in}}%
\pgfpathlineto{\pgfqpoint{3.368839in}{2.504558in}}%
\pgfpathlineto{\pgfqpoint{3.369036in}{2.508604in}}%
\pgfpathlineto{\pgfqpoint{3.369924in}{2.548692in}}%
\pgfpathlineto{\pgfqpoint{3.370517in}{2.532809in}}%
\pgfpathlineto{\pgfqpoint{3.370714in}{2.531355in}}%
\pgfpathlineto{\pgfqpoint{3.370911in}{2.538383in}}%
\pgfpathlineto{\pgfqpoint{3.371208in}{2.552021in}}%
\pgfpathlineto{\pgfqpoint{3.372096in}{2.549166in}}%
\pgfpathlineto{\pgfqpoint{3.372293in}{2.547969in}}%
\pgfpathlineto{\pgfqpoint{3.372491in}{2.550321in}}%
\pgfpathlineto{\pgfqpoint{3.373576in}{2.579096in}}%
\pgfpathlineto{\pgfqpoint{3.373872in}{2.566495in}}%
\pgfpathlineto{\pgfqpoint{3.373971in}{2.563270in}}%
\pgfpathlineto{\pgfqpoint{3.374267in}{2.580466in}}%
\pgfpathlineto{\pgfqpoint{3.374958in}{2.594564in}}%
\pgfpathlineto{\pgfqpoint{3.375353in}{2.579721in}}%
\pgfpathlineto{\pgfqpoint{3.375550in}{2.575734in}}%
\pgfpathlineto{\pgfqpoint{3.375945in}{2.592874in}}%
\pgfpathlineto{\pgfqpoint{3.376142in}{2.597832in}}%
\pgfpathlineto{\pgfqpoint{3.376537in}{2.584764in}}%
\pgfpathlineto{\pgfqpoint{3.376833in}{2.588011in}}%
\pgfpathlineto{\pgfqpoint{3.377228in}{2.575543in}}%
\pgfpathlineto{\pgfqpoint{3.377426in}{2.591155in}}%
\pgfpathlineto{\pgfqpoint{3.377623in}{2.607977in}}%
\pgfpathlineto{\pgfqpoint{3.378116in}{2.588237in}}%
\pgfpathlineto{\pgfqpoint{3.378511in}{2.595821in}}%
\pgfpathlineto{\pgfqpoint{3.378807in}{2.587960in}}%
\pgfpathlineto{\pgfqpoint{3.379400in}{2.598986in}}%
\pgfpathlineto{\pgfqpoint{3.379696in}{2.603296in}}%
\pgfpathlineto{\pgfqpoint{3.380090in}{2.594748in}}%
\pgfpathlineto{\pgfqpoint{3.380485in}{2.581333in}}%
\pgfpathlineto{\pgfqpoint{3.381373in}{2.587659in}}%
\pgfpathlineto{\pgfqpoint{3.381472in}{2.590207in}}%
\pgfpathlineto{\pgfqpoint{3.381768in}{2.577793in}}%
\pgfpathlineto{\pgfqpoint{3.382755in}{2.568336in}}%
\pgfpathlineto{\pgfqpoint{3.382360in}{2.586040in}}%
\pgfpathlineto{\pgfqpoint{3.382854in}{2.572144in}}%
\pgfpathlineto{\pgfqpoint{3.383150in}{2.582194in}}%
\pgfpathlineto{\pgfqpoint{3.383545in}{2.555861in}}%
\pgfpathlineto{\pgfqpoint{3.383742in}{2.550423in}}%
\pgfpathlineto{\pgfqpoint{3.384236in}{2.569777in}}%
\pgfpathlineto{\pgfqpoint{3.384334in}{2.570559in}}%
\pgfpathlineto{\pgfqpoint{3.384532in}{2.563385in}}%
\pgfpathlineto{\pgfqpoint{3.385025in}{2.537158in}}%
\pgfpathlineto{\pgfqpoint{3.386012in}{2.539566in}}%
\pgfpathlineto{\pgfqpoint{3.386111in}{2.539345in}}%
\pgfpathlineto{\pgfqpoint{3.386308in}{2.541601in}}%
\pgfpathlineto{\pgfqpoint{3.386506in}{2.544144in}}%
\pgfpathlineto{\pgfqpoint{3.386802in}{2.536365in}}%
\pgfpathlineto{\pgfqpoint{3.387098in}{2.523568in}}%
\pgfpathlineto{\pgfqpoint{3.387592in}{2.550298in}}%
\pgfpathlineto{\pgfqpoint{3.387690in}{2.553553in}}%
\pgfpathlineto{\pgfqpoint{3.388085in}{2.531591in}}%
\pgfpathlineto{\pgfqpoint{3.388776in}{2.522773in}}%
\pgfpathlineto{\pgfqpoint{3.389072in}{2.535403in}}%
\pgfpathlineto{\pgfqpoint{3.389269in}{2.541131in}}%
\pgfpathlineto{\pgfqpoint{3.389862in}{2.526352in}}%
\pgfpathlineto{\pgfqpoint{3.390158in}{2.517729in}}%
\pgfpathlineto{\pgfqpoint{3.390750in}{2.532305in}}%
\pgfpathlineto{\pgfqpoint{3.391046in}{2.535721in}}%
\pgfpathlineto{\pgfqpoint{3.391145in}{2.537628in}}%
\pgfpathlineto{\pgfqpoint{3.391441in}{2.528480in}}%
\pgfpathlineto{\pgfqpoint{3.391638in}{2.519259in}}%
\pgfpathlineto{\pgfqpoint{3.392329in}{2.536437in}}%
\pgfpathlineto{\pgfqpoint{3.392625in}{2.545405in}}%
\pgfpathlineto{\pgfqpoint{3.393513in}{2.541480in}}%
\pgfpathlineto{\pgfqpoint{3.393711in}{2.532407in}}%
\pgfpathlineto{\pgfqpoint{3.394106in}{2.560960in}}%
\pgfpathlineto{\pgfqpoint{3.394303in}{2.569012in}}%
\pgfpathlineto{\pgfqpoint{3.394994in}{2.550062in}}%
\pgfpathlineto{\pgfqpoint{3.398054in}{2.423422in}}%
\pgfpathlineto{\pgfqpoint{3.395487in}{2.551978in}}%
\pgfpathlineto{\pgfqpoint{3.398547in}{2.446390in}}%
\pgfpathlineto{\pgfqpoint{3.399238in}{2.495139in}}%
\pgfpathlineto{\pgfqpoint{3.399929in}{2.466691in}}%
\pgfpathlineto{\pgfqpoint{3.401607in}{2.491615in}}%
\pgfpathlineto{\pgfqpoint{3.402199in}{2.513531in}}%
\pgfpathlineto{\pgfqpoint{3.402495in}{2.495585in}}%
\pgfpathlineto{\pgfqpoint{3.403581in}{2.433935in}}%
\pgfpathlineto{\pgfqpoint{3.403976in}{2.449266in}}%
\pgfpathlineto{\pgfqpoint{3.404963in}{2.446124in}}%
\pgfpathlineto{\pgfqpoint{3.405555in}{2.526455in}}%
\pgfpathlineto{\pgfqpoint{3.406838in}{2.705128in}}%
\pgfpathlineto{\pgfqpoint{3.407331in}{2.675303in}}%
\pgfpathlineto{\pgfqpoint{3.407825in}{2.615518in}}%
\pgfpathlineto{\pgfqpoint{3.408614in}{2.405909in}}%
\pgfpathlineto{\pgfqpoint{3.409404in}{2.459461in}}%
\pgfpathlineto{\pgfqpoint{3.409700in}{2.441354in}}%
\pgfpathlineto{\pgfqpoint{3.410292in}{2.467860in}}%
\pgfpathlineto{\pgfqpoint{3.410391in}{2.467546in}}%
\pgfpathlineto{\pgfqpoint{3.410490in}{2.467079in}}%
\pgfpathlineto{\pgfqpoint{3.410687in}{2.470057in}}%
\pgfpathlineto{\pgfqpoint{3.411674in}{2.516665in}}%
\pgfpathlineto{\pgfqpoint{3.412365in}{2.500016in}}%
\pgfpathlineto{\pgfqpoint{3.412957in}{2.468664in}}%
\pgfpathlineto{\pgfqpoint{3.413648in}{2.489572in}}%
\pgfpathlineto{\pgfqpoint{3.415227in}{2.514584in}}%
\pgfpathlineto{\pgfqpoint{3.415622in}{2.505457in}}%
\pgfpathlineto{\pgfqpoint{3.416313in}{2.471939in}}%
\pgfpathlineto{\pgfqpoint{3.417004in}{2.493695in}}%
\pgfpathlineto{\pgfqpoint{3.418188in}{2.515541in}}%
\pgfpathlineto{\pgfqpoint{3.418386in}{2.512847in}}%
\pgfpathlineto{\pgfqpoint{3.419669in}{2.489739in}}%
\pgfpathlineto{\pgfqpoint{3.419767in}{2.492753in}}%
\pgfpathlineto{\pgfqpoint{3.420754in}{2.505603in}}%
\pgfpathlineto{\pgfqpoint{3.420952in}{2.499818in}}%
\pgfpathlineto{\pgfqpoint{3.421149in}{2.493434in}}%
\pgfpathlineto{\pgfqpoint{3.421840in}{2.501011in}}%
\pgfpathlineto{\pgfqpoint{3.422037in}{2.508187in}}%
\pgfpathlineto{\pgfqpoint{3.422531in}{2.487860in}}%
\pgfpathlineto{\pgfqpoint{3.422728in}{2.491688in}}%
\pgfpathlineto{\pgfqpoint{3.423617in}{2.508766in}}%
\pgfpathlineto{\pgfqpoint{3.423123in}{2.488645in}}%
\pgfpathlineto{\pgfqpoint{3.424308in}{2.505469in}}%
\pgfpathlineto{\pgfqpoint{3.424702in}{2.492743in}}%
\pgfpathlineto{\pgfqpoint{3.425393in}{2.499318in}}%
\pgfpathlineto{\pgfqpoint{3.425591in}{2.509259in}}%
\pgfpathlineto{\pgfqpoint{3.426084in}{2.484028in}}%
\pgfpathlineto{\pgfqpoint{3.426479in}{2.499604in}}%
\pgfpathlineto{\pgfqpoint{3.426775in}{2.498407in}}%
\pgfpathlineto{\pgfqpoint{3.426972in}{2.501109in}}%
\pgfpathlineto{\pgfqpoint{3.427071in}{2.501381in}}%
\pgfpathlineto{\pgfqpoint{3.427170in}{2.499506in}}%
\pgfpathlineto{\pgfqpoint{3.428058in}{2.484690in}}%
\pgfpathlineto{\pgfqpoint{3.428354in}{2.496133in}}%
\pgfpathlineto{\pgfqpoint{3.428650in}{2.505596in}}%
\pgfpathlineto{\pgfqpoint{3.429242in}{2.493373in}}%
\pgfpathlineto{\pgfqpoint{3.429539in}{2.480080in}}%
\pgfpathlineto{\pgfqpoint{3.429933in}{2.497892in}}%
\pgfpathlineto{\pgfqpoint{3.430427in}{2.488074in}}%
\pgfpathlineto{\pgfqpoint{3.433388in}{2.423739in}}%
\pgfpathlineto{\pgfqpoint{3.433881in}{2.426000in}}%
\pgfpathlineto{\pgfqpoint{3.433980in}{2.426859in}}%
\pgfpathlineto{\pgfqpoint{3.434177in}{2.422584in}}%
\pgfpathlineto{\pgfqpoint{3.434572in}{2.397144in}}%
\pgfpathlineto{\pgfqpoint{3.435559in}{2.401935in}}%
\pgfpathlineto{\pgfqpoint{3.435757in}{2.399014in}}%
\pgfpathlineto{\pgfqpoint{3.436151in}{2.377967in}}%
\pgfpathlineto{\pgfqpoint{3.436645in}{2.403659in}}%
\pgfpathlineto{\pgfqpoint{3.436744in}{2.406758in}}%
\pgfpathlineto{\pgfqpoint{3.437237in}{2.389934in}}%
\pgfpathlineto{\pgfqpoint{3.438027in}{2.383909in}}%
\pgfpathlineto{\pgfqpoint{3.437632in}{2.392387in}}%
\pgfpathlineto{\pgfqpoint{3.438224in}{2.388414in}}%
\pgfpathlineto{\pgfqpoint{3.438520in}{2.407612in}}%
\pgfpathlineto{\pgfqpoint{3.439112in}{2.379889in}}%
\pgfpathlineto{\pgfqpoint{3.439310in}{2.386947in}}%
\pgfpathlineto{\pgfqpoint{3.440395in}{2.402499in}}%
\pgfpathlineto{\pgfqpoint{3.440593in}{2.396298in}}%
\pgfpathlineto{\pgfqpoint{3.440889in}{2.383922in}}%
\pgfpathlineto{\pgfqpoint{3.441580in}{2.398722in}}%
\pgfpathlineto{\pgfqpoint{3.442962in}{2.417850in}}%
\pgfpathlineto{\pgfqpoint{3.443850in}{2.431929in}}%
\pgfpathlineto{\pgfqpoint{3.444245in}{2.427666in}}%
\pgfpathlineto{\pgfqpoint{3.445725in}{2.388195in}}%
\pgfpathlineto{\pgfqpoint{3.446120in}{2.376296in}}%
\pgfpathlineto{\pgfqpoint{3.446613in}{2.389294in}}%
\pgfpathlineto{\pgfqpoint{3.446910in}{2.384726in}}%
\pgfpathlineto{\pgfqpoint{3.447206in}{2.389672in}}%
\pgfpathlineto{\pgfqpoint{3.447403in}{2.382493in}}%
\pgfpathlineto{\pgfqpoint{3.448686in}{2.344380in}}%
\pgfpathlineto{\pgfqpoint{3.448884in}{2.346246in}}%
\pgfpathlineto{\pgfqpoint{3.449871in}{2.377875in}}%
\pgfpathlineto{\pgfqpoint{3.450265in}{2.398287in}}%
\pgfpathlineto{\pgfqpoint{3.450956in}{2.378391in}}%
\pgfpathlineto{\pgfqpoint{3.452535in}{2.354541in}}%
\pgfpathlineto{\pgfqpoint{3.452634in}{2.356627in}}%
\pgfpathlineto{\pgfqpoint{3.454115in}{2.493195in}}%
\pgfpathlineto{\pgfqpoint{3.455003in}{2.645029in}}%
\pgfpathlineto{\pgfqpoint{3.455595in}{2.610834in}}%
\pgfpathlineto{\pgfqpoint{3.456385in}{2.501050in}}%
\pgfpathlineto{\pgfqpoint{3.456977in}{2.353125in}}%
\pgfpathlineto{\pgfqpoint{3.457766in}{2.397163in}}%
\pgfpathlineto{\pgfqpoint{3.460135in}{2.482174in}}%
\pgfpathlineto{\pgfqpoint{3.460333in}{2.477534in}}%
\pgfpathlineto{\pgfqpoint{3.460925in}{2.436698in}}%
\pgfpathlineto{\pgfqpoint{3.461813in}{2.441397in}}%
\pgfpathlineto{\pgfqpoint{3.462011in}{2.436848in}}%
\pgfpathlineto{\pgfqpoint{3.462504in}{2.447991in}}%
\pgfpathlineto{\pgfqpoint{3.463294in}{2.484171in}}%
\pgfpathlineto{\pgfqpoint{3.463787in}{2.459308in}}%
\pgfpathlineto{\pgfqpoint{3.464577in}{2.446581in}}%
\pgfpathlineto{\pgfqpoint{3.464873in}{2.453175in}}%
\pgfpathlineto{\pgfqpoint{3.466452in}{2.480294in}}%
\pgfpathlineto{\pgfqpoint{3.465465in}{2.442346in}}%
\pgfpathlineto{\pgfqpoint{3.466748in}{2.469898in}}%
\pgfpathlineto{\pgfqpoint{3.468525in}{2.408003in}}%
\pgfpathlineto{\pgfqpoint{3.468722in}{2.406209in}}%
\pgfpathlineto{\pgfqpoint{3.469216in}{2.410394in}}%
\pgfpathlineto{\pgfqpoint{3.471387in}{2.509009in}}%
\pgfpathlineto{\pgfqpoint{3.472867in}{2.505620in}}%
\pgfpathlineto{\pgfqpoint{3.473262in}{2.503975in}}%
\pgfpathlineto{\pgfqpoint{3.473854in}{2.486004in}}%
\pgfpathlineto{\pgfqpoint{3.474348in}{2.499751in}}%
\pgfpathlineto{\pgfqpoint{3.474545in}{2.504623in}}%
\pgfpathlineto{\pgfqpoint{3.475236in}{2.497563in}}%
\pgfpathlineto{\pgfqpoint{3.475532in}{2.484641in}}%
\pgfpathlineto{\pgfqpoint{3.476026in}{2.505114in}}%
\pgfpathlineto{\pgfqpoint{3.476223in}{2.502937in}}%
\pgfpathlineto{\pgfqpoint{3.476717in}{2.493867in}}%
\pgfpathlineto{\pgfqpoint{3.478098in}{2.474889in}}%
\pgfpathlineto{\pgfqpoint{3.478296in}{2.478108in}}%
\pgfpathlineto{\pgfqpoint{3.478493in}{2.472616in}}%
\pgfpathlineto{\pgfqpoint{3.479678in}{2.453835in}}%
\pgfpathlineto{\pgfqpoint{3.479184in}{2.483806in}}%
\pgfpathlineto{\pgfqpoint{3.479776in}{2.455019in}}%
\pgfpathlineto{\pgfqpoint{3.479875in}{2.455377in}}%
\pgfpathlineto{\pgfqpoint{3.479974in}{2.453149in}}%
\pgfpathlineto{\pgfqpoint{3.482244in}{2.415590in}}%
\pgfpathlineto{\pgfqpoint{3.482639in}{2.429400in}}%
\pgfpathlineto{\pgfqpoint{3.482935in}{2.417042in}}%
\pgfpathlineto{\pgfqpoint{3.483823in}{2.398549in}}%
\pgfpathlineto{\pgfqpoint{3.484119in}{2.404361in}}%
\pgfpathlineto{\pgfqpoint{3.484613in}{2.412663in}}%
\pgfpathlineto{\pgfqpoint{3.485402in}{2.408074in}}%
\pgfpathlineto{\pgfqpoint{3.485600in}{2.406969in}}%
\pgfpathlineto{\pgfqpoint{3.485797in}{2.410528in}}%
\pgfpathlineto{\pgfqpoint{3.486093in}{2.420266in}}%
\pgfpathlineto{\pgfqpoint{3.486587in}{2.406856in}}%
\pgfpathlineto{\pgfqpoint{3.486981in}{2.414567in}}%
\pgfpathlineto{\pgfqpoint{3.488462in}{2.393564in}}%
\pgfpathlineto{\pgfqpoint{3.487475in}{2.418120in}}%
\pgfpathlineto{\pgfqpoint{3.488659in}{2.403985in}}%
\pgfpathlineto{\pgfqpoint{3.489153in}{2.432874in}}%
\pgfpathlineto{\pgfqpoint{3.489745in}{2.407989in}}%
\pgfpathlineto{\pgfqpoint{3.492410in}{2.450409in}}%
\pgfpathlineto{\pgfqpoint{3.492706in}{2.438935in}}%
\pgfpathlineto{\pgfqpoint{3.494779in}{2.371243in}}%
\pgfpathlineto{\pgfqpoint{3.494976in}{2.375538in}}%
\pgfpathlineto{\pgfqpoint{3.495371in}{2.386597in}}%
\pgfpathlineto{\pgfqpoint{3.495864in}{2.374616in}}%
\pgfpathlineto{\pgfqpoint{3.496753in}{2.343543in}}%
\pgfpathlineto{\pgfqpoint{3.497246in}{2.362083in}}%
\pgfpathlineto{\pgfqpoint{3.498529in}{2.393879in}}%
\pgfpathlineto{\pgfqpoint{3.498825in}{2.407602in}}%
\pgfpathlineto{\pgfqpoint{3.499319in}{2.378112in}}%
\pgfpathlineto{\pgfqpoint{3.499714in}{2.349959in}}%
\pgfpathlineto{\pgfqpoint{3.500503in}{2.363317in}}%
\pgfpathlineto{\pgfqpoint{3.502872in}{2.611159in}}%
\pgfpathlineto{\pgfqpoint{3.503464in}{2.632769in}}%
\pgfpathlineto{\pgfqpoint{3.503859in}{2.613814in}}%
\pgfpathlineto{\pgfqpoint{3.504550in}{2.512014in}}%
\pgfpathlineto{\pgfqpoint{3.505142in}{2.337279in}}%
\pgfpathlineto{\pgfqpoint{3.505932in}{2.396983in}}%
\pgfpathlineto{\pgfqpoint{3.506326in}{2.382579in}}%
\pgfpathlineto{\pgfqpoint{3.507017in}{2.396106in}}%
\pgfpathlineto{\pgfqpoint{3.508498in}{2.453762in}}%
\pgfpathlineto{\pgfqpoint{3.508794in}{2.443958in}}%
\pgfpathlineto{\pgfqpoint{3.509682in}{2.400014in}}%
\pgfpathlineto{\pgfqpoint{3.510077in}{2.417644in}}%
\pgfpathlineto{\pgfqpoint{3.511656in}{2.453303in}}%
\pgfpathlineto{\pgfqpoint{3.511755in}{2.451556in}}%
\pgfpathlineto{\pgfqpoint{3.512742in}{2.427720in}}%
\pgfpathlineto{\pgfqpoint{3.513235in}{2.435899in}}%
\pgfpathlineto{\pgfqpoint{3.513334in}{2.435587in}}%
\pgfpathlineto{\pgfqpoint{3.513531in}{2.437964in}}%
\pgfpathlineto{\pgfqpoint{3.515111in}{2.457989in}}%
\pgfpathlineto{\pgfqpoint{3.516591in}{2.452012in}}%
\pgfpathlineto{\pgfqpoint{3.516690in}{2.452769in}}%
\pgfpathlineto{\pgfqpoint{3.517085in}{2.469805in}}%
\pgfpathlineto{\pgfqpoint{3.517973in}{2.463333in}}%
\pgfpathlineto{\pgfqpoint{3.518170in}{2.459182in}}%
\pgfpathlineto{\pgfqpoint{3.518664in}{2.475433in}}%
\pgfpathlineto{\pgfqpoint{3.518762in}{2.475834in}}%
\pgfpathlineto{\pgfqpoint{3.518960in}{2.473710in}}%
\pgfpathlineto{\pgfqpoint{3.519651in}{2.457828in}}%
\pgfpathlineto{\pgfqpoint{3.520046in}{2.472103in}}%
\pgfpathlineto{\pgfqpoint{3.520243in}{2.477773in}}%
\pgfpathlineto{\pgfqpoint{3.520736in}{2.460324in}}%
\pgfpathlineto{\pgfqpoint{3.521625in}{2.452577in}}%
\pgfpathlineto{\pgfqpoint{3.521230in}{2.464434in}}%
\pgfpathlineto{\pgfqpoint{3.521822in}{2.458219in}}%
\pgfpathlineto{\pgfqpoint{3.522118in}{2.471819in}}%
\pgfpathlineto{\pgfqpoint{3.522513in}{2.454324in}}%
\pgfpathlineto{\pgfqpoint{3.522908in}{2.460294in}}%
\pgfpathlineto{\pgfqpoint{3.523500in}{2.471850in}}%
\pgfpathlineto{\pgfqpoint{3.524388in}{2.465421in}}%
\pgfpathlineto{\pgfqpoint{3.524783in}{2.460199in}}%
\pgfpathlineto{\pgfqpoint{3.525079in}{2.468539in}}%
\pgfpathlineto{\pgfqpoint{3.525277in}{2.474478in}}%
\pgfpathlineto{\pgfqpoint{3.525770in}{2.454861in}}%
\pgfpathlineto{\pgfqpoint{3.527941in}{2.424246in}}%
\pgfpathlineto{\pgfqpoint{3.528040in}{2.425704in}}%
\pgfpathlineto{\pgfqpoint{3.528336in}{2.441684in}}%
\pgfpathlineto{\pgfqpoint{3.528731in}{2.425343in}}%
\pgfpathlineto{\pgfqpoint{3.529126in}{2.428757in}}%
\pgfpathlineto{\pgfqpoint{3.529521in}{2.401073in}}%
\pgfpathlineto{\pgfqpoint{3.530409in}{2.413710in}}%
\pgfpathlineto{\pgfqpoint{3.530606in}{2.417467in}}%
\pgfpathlineto{\pgfqpoint{3.531001in}{2.405281in}}%
\pgfpathlineto{\pgfqpoint{3.531198in}{2.402121in}}%
\pgfpathlineto{\pgfqpoint{3.531988in}{2.407758in}}%
\pgfpathlineto{\pgfqpoint{3.533074in}{2.419415in}}%
\pgfpathlineto{\pgfqpoint{3.532679in}{2.401132in}}%
\pgfpathlineto{\pgfqpoint{3.533271in}{2.416447in}}%
\pgfpathlineto{\pgfqpoint{3.534159in}{2.399871in}}%
\pgfpathlineto{\pgfqpoint{3.533666in}{2.417340in}}%
\pgfpathlineto{\pgfqpoint{3.534949in}{2.403652in}}%
\pgfpathlineto{\pgfqpoint{3.536726in}{2.429183in}}%
\pgfpathlineto{\pgfqpoint{3.536923in}{2.424916in}}%
\pgfpathlineto{\pgfqpoint{3.537219in}{2.411799in}}%
\pgfpathlineto{\pgfqpoint{3.537811in}{2.434128in}}%
\pgfpathlineto{\pgfqpoint{3.538502in}{2.451512in}}%
\pgfpathlineto{\pgfqpoint{3.538897in}{2.435331in}}%
\pgfpathlineto{\pgfqpoint{3.538996in}{2.433057in}}%
\pgfpathlineto{\pgfqpoint{3.539292in}{2.446537in}}%
\pgfpathlineto{\pgfqpoint{3.540279in}{2.458009in}}%
\pgfpathlineto{\pgfqpoint{3.540575in}{2.456030in}}%
\pgfpathlineto{\pgfqpoint{3.541858in}{2.417982in}}%
\pgfpathlineto{\pgfqpoint{3.542845in}{2.374160in}}%
\pgfpathlineto{\pgfqpoint{3.543338in}{2.378539in}}%
\pgfpathlineto{\pgfqpoint{3.543733in}{2.366916in}}%
\pgfpathlineto{\pgfqpoint{3.545806in}{2.279962in}}%
\pgfpathlineto{\pgfqpoint{3.546003in}{2.287806in}}%
\pgfpathlineto{\pgfqpoint{3.547089in}{2.379067in}}%
\pgfpathlineto{\pgfqpoint{3.547780in}{2.361314in}}%
\pgfpathlineto{\pgfqpoint{3.547879in}{2.361138in}}%
\pgfpathlineto{\pgfqpoint{3.548866in}{2.345024in}}%
\pgfpathlineto{\pgfqpoint{3.548471in}{2.361418in}}%
\pgfpathlineto{\pgfqpoint{3.549063in}{2.354167in}}%
\pgfpathlineto{\pgfqpoint{3.549557in}{2.352873in}}%
\pgfpathlineto{\pgfqpoint{3.550741in}{2.512525in}}%
\pgfpathlineto{\pgfqpoint{3.551629in}{2.625048in}}%
\pgfpathlineto{\pgfqpoint{3.552024in}{2.583409in}}%
\pgfpathlineto{\pgfqpoint{3.553110in}{2.377319in}}%
\pgfpathlineto{\pgfqpoint{3.553406in}{2.321794in}}%
\pgfpathlineto{\pgfqpoint{3.554195in}{2.369406in}}%
\pgfpathlineto{\pgfqpoint{3.554294in}{2.369477in}}%
\pgfpathlineto{\pgfqpoint{3.554393in}{2.368263in}}%
\pgfpathlineto{\pgfqpoint{3.554491in}{2.366727in}}%
\pgfpathlineto{\pgfqpoint{3.554788in}{2.375998in}}%
\pgfpathlineto{\pgfqpoint{3.556465in}{2.439379in}}%
\pgfpathlineto{\pgfqpoint{3.555380in}{2.375053in}}%
\pgfpathlineto{\pgfqpoint{3.556762in}{2.428537in}}%
\pgfpathlineto{\pgfqpoint{3.557354in}{2.378186in}}%
\pgfpathlineto{\pgfqpoint{3.558143in}{2.390847in}}%
\pgfpathlineto{\pgfqpoint{3.559821in}{2.434313in}}%
\pgfpathlineto{\pgfqpoint{3.558736in}{2.386005in}}%
\pgfpathlineto{\pgfqpoint{3.559920in}{2.431896in}}%
\pgfpathlineto{\pgfqpoint{3.560512in}{2.395842in}}%
\pgfpathlineto{\pgfqpoint{3.561302in}{2.416111in}}%
\pgfpathlineto{\pgfqpoint{3.562486in}{2.428572in}}%
\pgfpathlineto{\pgfqpoint{3.563276in}{2.448523in}}%
\pgfpathlineto{\pgfqpoint{3.563670in}{2.439935in}}%
\pgfpathlineto{\pgfqpoint{3.563967in}{2.429987in}}%
\pgfpathlineto{\pgfqpoint{3.564460in}{2.449609in}}%
\pgfpathlineto{\pgfqpoint{3.564559in}{2.449749in}}%
\pgfpathlineto{\pgfqpoint{3.564657in}{2.448283in}}%
\pgfpathlineto{\pgfqpoint{3.565447in}{2.439194in}}%
\pgfpathlineto{\pgfqpoint{3.565644in}{2.445359in}}%
\pgfpathlineto{\pgfqpoint{3.565941in}{2.455907in}}%
\pgfpathlineto{\pgfqpoint{3.566829in}{2.450922in}}%
\pgfpathlineto{\pgfqpoint{3.568309in}{2.440797in}}%
\pgfpathlineto{\pgfqpoint{3.568507in}{2.442115in}}%
\pgfpathlineto{\pgfqpoint{3.569592in}{2.449670in}}%
\pgfpathlineto{\pgfqpoint{3.569691in}{2.448158in}}%
\pgfpathlineto{\pgfqpoint{3.570283in}{2.428832in}}%
\pgfpathlineto{\pgfqpoint{3.570777in}{2.447156in}}%
\pgfpathlineto{\pgfqpoint{3.570875in}{2.447556in}}%
\pgfpathlineto{\pgfqpoint{3.571073in}{2.444751in}}%
\pgfpathlineto{\pgfqpoint{3.572356in}{2.436443in}}%
\pgfpathlineto{\pgfqpoint{3.571961in}{2.445997in}}%
\pgfpathlineto{\pgfqpoint{3.572455in}{2.438692in}}%
\pgfpathlineto{\pgfqpoint{3.572849in}{2.457687in}}%
\pgfpathlineto{\pgfqpoint{3.573540in}{2.444482in}}%
\pgfpathlineto{\pgfqpoint{3.573836in}{2.439517in}}%
\pgfpathlineto{\pgfqpoint{3.574429in}{2.449964in}}%
\pgfpathlineto{\pgfqpoint{3.574527in}{2.450189in}}%
\pgfpathlineto{\pgfqpoint{3.574626in}{2.448919in}}%
\pgfpathlineto{\pgfqpoint{3.577094in}{2.401116in}}%
\pgfpathlineto{\pgfqpoint{3.577390in}{2.409979in}}%
\pgfpathlineto{\pgfqpoint{3.577883in}{2.393592in}}%
\pgfpathlineto{\pgfqpoint{3.578080in}{2.394651in}}%
\pgfpathlineto{\pgfqpoint{3.578278in}{2.391327in}}%
\pgfpathlineto{\pgfqpoint{3.579067in}{2.385864in}}%
\pgfpathlineto{\pgfqpoint{3.579462in}{2.387003in}}%
\pgfpathlineto{\pgfqpoint{3.580054in}{2.382492in}}%
\pgfpathlineto{\pgfqpoint{3.580351in}{2.375915in}}%
\pgfpathlineto{\pgfqpoint{3.581140in}{2.381566in}}%
\pgfpathlineto{\pgfqpoint{3.581831in}{2.366704in}}%
\pgfpathlineto{\pgfqpoint{3.582127in}{2.378102in}}%
\pgfpathlineto{\pgfqpoint{3.582423in}{2.387953in}}%
\pgfpathlineto{\pgfqpoint{3.583015in}{2.366793in}}%
\pgfpathlineto{\pgfqpoint{3.584200in}{2.383889in}}%
\pgfpathlineto{\pgfqpoint{3.584496in}{2.393992in}}%
\pgfpathlineto{\pgfqpoint{3.584891in}{2.381040in}}%
\pgfpathlineto{\pgfqpoint{3.585384in}{2.388339in}}%
\pgfpathlineto{\pgfqpoint{3.587062in}{2.413754in}}%
\pgfpathlineto{\pgfqpoint{3.587358in}{2.438736in}}%
\pgfpathlineto{\pgfqpoint{3.588345in}{2.435691in}}%
\pgfpathlineto{\pgfqpoint{3.588543in}{2.438000in}}%
\pgfpathlineto{\pgfqpoint{3.588740in}{2.434550in}}%
\pgfpathlineto{\pgfqpoint{3.590418in}{2.374738in}}%
\pgfpathlineto{\pgfqpoint{3.590714in}{2.381420in}}%
\pgfpathlineto{\pgfqpoint{3.590813in}{2.381600in}}%
\pgfpathlineto{\pgfqpoint{3.590911in}{2.379639in}}%
\pgfpathlineto{\pgfqpoint{3.591010in}{2.377979in}}%
\pgfpathlineto{\pgfqpoint{3.591306in}{2.386790in}}%
\pgfpathlineto{\pgfqpoint{3.591504in}{2.390579in}}%
\pgfpathlineto{\pgfqpoint{3.591898in}{2.376725in}}%
\pgfpathlineto{\pgfqpoint{3.592096in}{2.377820in}}%
\pgfpathlineto{\pgfqpoint{3.592392in}{2.367059in}}%
\pgfpathlineto{\pgfqpoint{3.593379in}{2.350378in}}%
\pgfpathlineto{\pgfqpoint{3.593576in}{2.354729in}}%
\pgfpathlineto{\pgfqpoint{3.594859in}{2.399607in}}%
\pgfpathlineto{\pgfqpoint{3.595452in}{2.381539in}}%
\pgfpathlineto{\pgfqpoint{3.596142in}{2.363340in}}%
\pgfpathlineto{\pgfqpoint{3.596932in}{2.367740in}}%
\pgfpathlineto{\pgfqpoint{3.598314in}{2.458576in}}%
\pgfpathlineto{\pgfqpoint{3.599597in}{2.645485in}}%
\pgfpathlineto{\pgfqpoint{3.600090in}{2.620197in}}%
\pgfpathlineto{\pgfqpoint{3.600584in}{2.586060in}}%
\pgfpathlineto{\pgfqpoint{3.601373in}{2.356895in}}%
\pgfpathlineto{\pgfqpoint{3.602262in}{2.421867in}}%
\pgfpathlineto{\pgfqpoint{3.604433in}{2.493581in}}%
\pgfpathlineto{\pgfqpoint{3.602854in}{2.408650in}}%
\pgfpathlineto{\pgfqpoint{3.604729in}{2.476568in}}%
\pgfpathlineto{\pgfqpoint{3.605618in}{2.437600in}}%
\pgfpathlineto{\pgfqpoint{3.606111in}{2.450754in}}%
\pgfpathlineto{\pgfqpoint{3.607098in}{2.472173in}}%
\pgfpathlineto{\pgfqpoint{3.607789in}{2.495192in}}%
\pgfpathlineto{\pgfqpoint{3.608184in}{2.472456in}}%
\pgfpathlineto{\pgfqpoint{3.608973in}{2.464705in}}%
\pgfpathlineto{\pgfqpoint{3.609269in}{2.470026in}}%
\pgfpathlineto{\pgfqpoint{3.610355in}{2.496646in}}%
\pgfpathlineto{\pgfqpoint{3.610750in}{2.508121in}}%
\pgfpathlineto{\pgfqpoint{3.611441in}{2.501630in}}%
\pgfpathlineto{\pgfqpoint{3.611638in}{2.499655in}}%
\pgfpathlineto{\pgfqpoint{3.611836in}{2.508247in}}%
\pgfpathlineto{\pgfqpoint{3.612132in}{2.522403in}}%
\pgfpathlineto{\pgfqpoint{3.612526in}{2.508006in}}%
\pgfpathlineto{\pgfqpoint{3.613119in}{2.520437in}}%
\pgfpathlineto{\pgfqpoint{3.613513in}{2.526609in}}%
\pgfpathlineto{\pgfqpoint{3.614500in}{2.531412in}}%
\pgfpathlineto{\pgfqpoint{3.614698in}{2.529744in}}%
\pgfpathlineto{\pgfqpoint{3.614895in}{2.531978in}}%
\pgfpathlineto{\pgfqpoint{3.616277in}{2.557189in}}%
\pgfpathlineto{\pgfqpoint{3.616474in}{2.554118in}}%
\pgfpathlineto{\pgfqpoint{3.616770in}{2.548603in}}%
\pgfpathlineto{\pgfqpoint{3.617165in}{2.558180in}}%
\pgfpathlineto{\pgfqpoint{3.617560in}{2.553064in}}%
\pgfpathlineto{\pgfqpoint{3.618646in}{2.564533in}}%
\pgfpathlineto{\pgfqpoint{3.618152in}{2.548889in}}%
\pgfpathlineto{\pgfqpoint{3.618744in}{2.560776in}}%
\pgfpathlineto{\pgfqpoint{3.620422in}{2.522495in}}%
\pgfpathlineto{\pgfqpoint{3.622791in}{2.463252in}}%
\pgfpathlineto{\pgfqpoint{3.622890in}{2.466549in}}%
\pgfpathlineto{\pgfqpoint{3.624074in}{2.523163in}}%
\pgfpathlineto{\pgfqpoint{3.624765in}{2.518215in}}%
\pgfpathlineto{\pgfqpoint{3.625061in}{2.523895in}}%
\pgfpathlineto{\pgfqpoint{3.625555in}{2.513943in}}%
\pgfpathlineto{\pgfqpoint{3.626542in}{2.496952in}}%
\pgfpathlineto{\pgfqpoint{3.626739in}{2.501687in}}%
\pgfpathlineto{\pgfqpoint{3.626936in}{2.505245in}}%
\pgfpathlineto{\pgfqpoint{3.627233in}{2.495420in}}%
\pgfpathlineto{\pgfqpoint{3.627627in}{2.498605in}}%
\pgfpathlineto{\pgfqpoint{3.627825in}{2.492262in}}%
\pgfpathlineto{\pgfqpoint{3.628220in}{2.513870in}}%
\pgfpathlineto{\pgfqpoint{3.628318in}{2.517271in}}%
\pgfpathlineto{\pgfqpoint{3.628713in}{2.503289in}}%
\pgfpathlineto{\pgfqpoint{3.629009in}{2.505018in}}%
\pgfpathlineto{\pgfqpoint{3.631279in}{2.481761in}}%
\pgfpathlineto{\pgfqpoint{3.631674in}{2.492766in}}%
\pgfpathlineto{\pgfqpoint{3.632464in}{2.485338in}}%
\pgfpathlineto{\pgfqpoint{3.632562in}{2.485131in}}%
\pgfpathlineto{\pgfqpoint{3.632661in}{2.486775in}}%
\pgfpathlineto{\pgfqpoint{3.632858in}{2.491771in}}%
\pgfpathlineto{\pgfqpoint{3.633155in}{2.479327in}}%
\pgfpathlineto{\pgfqpoint{3.633352in}{2.469646in}}%
\pgfpathlineto{\pgfqpoint{3.633747in}{2.496385in}}%
\pgfpathlineto{\pgfqpoint{3.634141in}{2.484720in}}%
\pgfpathlineto{\pgfqpoint{3.636609in}{2.524937in}}%
\pgfpathlineto{\pgfqpoint{3.636806in}{2.519626in}}%
\pgfpathlineto{\pgfqpoint{3.638879in}{2.451094in}}%
\pgfpathlineto{\pgfqpoint{3.639274in}{2.439048in}}%
\pgfpathlineto{\pgfqpoint{3.639767in}{2.456073in}}%
\pgfpathlineto{\pgfqpoint{3.640063in}{2.461530in}}%
\pgfpathlineto{\pgfqpoint{3.640458in}{2.449389in}}%
\pgfpathlineto{\pgfqpoint{3.641445in}{2.417753in}}%
\pgfpathlineto{\pgfqpoint{3.641939in}{2.426949in}}%
\pgfpathlineto{\pgfqpoint{3.642037in}{2.427295in}}%
\pgfpathlineto{\pgfqpoint{3.642235in}{2.425174in}}%
\pgfpathlineto{\pgfqpoint{3.642432in}{2.423886in}}%
\pgfpathlineto{\pgfqpoint{3.642630in}{2.427879in}}%
\pgfpathlineto{\pgfqpoint{3.643222in}{2.471465in}}%
\pgfpathlineto{\pgfqpoint{3.643814in}{2.433188in}}%
\pgfpathlineto{\pgfqpoint{3.645591in}{2.400188in}}%
\pgfpathlineto{\pgfqpoint{3.645689in}{2.402692in}}%
\pgfpathlineto{\pgfqpoint{3.646775in}{2.494342in}}%
\pgfpathlineto{\pgfqpoint{3.648157in}{2.682378in}}%
\pgfpathlineto{\pgfqpoint{3.648848in}{2.631291in}}%
\pgfpathlineto{\pgfqpoint{3.649539in}{2.434473in}}%
\pgfpathlineto{\pgfqpoint{3.649835in}{2.379065in}}%
\pgfpathlineto{\pgfqpoint{3.650624in}{2.422600in}}%
\pgfpathlineto{\pgfqpoint{3.650920in}{2.415786in}}%
\pgfpathlineto{\pgfqpoint{3.651315in}{2.428537in}}%
\pgfpathlineto{\pgfqpoint{3.653092in}{2.481496in}}%
\pgfpathlineto{\pgfqpoint{3.653190in}{2.478030in}}%
\pgfpathlineto{\pgfqpoint{3.654770in}{2.419865in}}%
\pgfpathlineto{\pgfqpoint{3.654967in}{2.423538in}}%
\pgfpathlineto{\pgfqpoint{3.656151in}{2.478747in}}%
\pgfpathlineto{\pgfqpoint{3.656744in}{2.461018in}}%
\pgfpathlineto{\pgfqpoint{3.657237in}{2.438213in}}%
\pgfpathlineto{\pgfqpoint{3.658125in}{2.446834in}}%
\pgfpathlineto{\pgfqpoint{3.658323in}{2.446376in}}%
\pgfpathlineto{\pgfqpoint{3.658520in}{2.447719in}}%
\pgfpathlineto{\pgfqpoint{3.659408in}{2.468534in}}%
\pgfpathlineto{\pgfqpoint{3.660395in}{2.462563in}}%
\pgfpathlineto{\pgfqpoint{3.660593in}{2.458991in}}%
\pgfpathlineto{\pgfqpoint{3.660889in}{2.470481in}}%
\pgfpathlineto{\pgfqpoint{3.662369in}{2.511926in}}%
\pgfpathlineto{\pgfqpoint{3.662468in}{2.510353in}}%
\pgfpathlineto{\pgfqpoint{3.663554in}{2.505453in}}%
\pgfpathlineto{\pgfqpoint{3.663060in}{2.516012in}}%
\pgfpathlineto{\pgfqpoint{3.663652in}{2.506237in}}%
\pgfpathlineto{\pgfqpoint{3.664639in}{2.515007in}}%
\pgfpathlineto{\pgfqpoint{3.664146in}{2.500640in}}%
\pgfpathlineto{\pgfqpoint{3.664837in}{2.508767in}}%
\pgfpathlineto{\pgfqpoint{3.665626in}{2.498387in}}%
\pgfpathlineto{\pgfqpoint{3.665923in}{2.507805in}}%
\pgfpathlineto{\pgfqpoint{3.666021in}{2.508727in}}%
\pgfpathlineto{\pgfqpoint{3.666219in}{2.501503in}}%
\pgfpathlineto{\pgfqpoint{3.667107in}{2.493433in}}%
\pgfpathlineto{\pgfqpoint{3.667304in}{2.499132in}}%
\pgfpathlineto{\pgfqpoint{3.668291in}{2.507873in}}%
\pgfpathlineto{\pgfqpoint{3.667798in}{2.492229in}}%
\pgfpathlineto{\pgfqpoint{3.668390in}{2.503993in}}%
\pgfpathlineto{\pgfqpoint{3.669180in}{2.486749in}}%
\pgfpathlineto{\pgfqpoint{3.669574in}{2.496139in}}%
\pgfpathlineto{\pgfqpoint{3.671351in}{2.471155in}}%
\pgfpathlineto{\pgfqpoint{3.672831in}{2.451131in}}%
\pgfpathlineto{\pgfqpoint{3.674608in}{2.424855in}}%
\pgfpathlineto{\pgfqpoint{3.676582in}{2.396500in}}%
\pgfpathlineto{\pgfqpoint{3.676878in}{2.379111in}}%
\pgfpathlineto{\pgfqpoint{3.677766in}{2.383900in}}%
\pgfpathlineto{\pgfqpoint{3.678753in}{2.375911in}}%
\pgfpathlineto{\pgfqpoint{3.679050in}{2.382749in}}%
\pgfpathlineto{\pgfqpoint{3.679148in}{2.384034in}}%
\pgfpathlineto{\pgfqpoint{3.679444in}{2.376836in}}%
\pgfpathlineto{\pgfqpoint{3.680431in}{2.366514in}}%
\pgfpathlineto{\pgfqpoint{3.680037in}{2.378319in}}%
\pgfpathlineto{\pgfqpoint{3.680530in}{2.368619in}}%
\pgfpathlineto{\pgfqpoint{3.680925in}{2.386239in}}%
\pgfpathlineto{\pgfqpoint{3.681418in}{2.368204in}}%
\pgfpathlineto{\pgfqpoint{3.681714in}{2.374895in}}%
\pgfpathlineto{\pgfqpoint{3.684083in}{2.415336in}}%
\pgfpathlineto{\pgfqpoint{3.684281in}{2.413602in}}%
\pgfpathlineto{\pgfqpoint{3.684675in}{2.424651in}}%
\pgfpathlineto{\pgfqpoint{3.684873in}{2.427951in}}%
\pgfpathlineto{\pgfqpoint{3.685465in}{2.422850in}}%
\pgfpathlineto{\pgfqpoint{3.685662in}{2.423587in}}%
\pgfpathlineto{\pgfqpoint{3.685958in}{2.416740in}}%
\pgfpathlineto{\pgfqpoint{3.687242in}{2.372895in}}%
\pgfpathlineto{\pgfqpoint{3.687735in}{2.380861in}}%
\pgfpathlineto{\pgfqpoint{3.687932in}{2.379126in}}%
\pgfpathlineto{\pgfqpoint{3.688130in}{2.386114in}}%
\pgfpathlineto{\pgfqpoint{3.688426in}{2.396187in}}%
\pgfpathlineto{\pgfqpoint{3.688919in}{2.374775in}}%
\pgfpathlineto{\pgfqpoint{3.690203in}{2.352853in}}%
\pgfpathlineto{\pgfqpoint{3.690499in}{2.357213in}}%
\pgfpathlineto{\pgfqpoint{3.691584in}{2.405252in}}%
\pgfpathlineto{\pgfqpoint{3.692176in}{2.375743in}}%
\pgfpathlineto{\pgfqpoint{3.694447in}{2.296989in}}%
\pgfpathlineto{\pgfqpoint{3.694841in}{2.322299in}}%
\pgfpathlineto{\pgfqpoint{3.696421in}{2.590297in}}%
\pgfpathlineto{\pgfqpoint{3.697210in}{2.555437in}}%
\pgfpathlineto{\pgfqpoint{3.698098in}{2.319452in}}%
\pgfpathlineto{\pgfqpoint{3.699382in}{2.373786in}}%
\pgfpathlineto{\pgfqpoint{3.701257in}{2.439132in}}%
\pgfpathlineto{\pgfqpoint{3.699875in}{2.372144in}}%
\pgfpathlineto{\pgfqpoint{3.701750in}{2.409531in}}%
\pgfpathlineto{\pgfqpoint{3.702737in}{2.384126in}}%
\pgfpathlineto{\pgfqpoint{3.703033in}{2.395400in}}%
\pgfpathlineto{\pgfqpoint{3.704613in}{2.433243in}}%
\pgfpathlineto{\pgfqpoint{3.703428in}{2.392944in}}%
\pgfpathlineto{\pgfqpoint{3.705007in}{2.411597in}}%
\pgfpathlineto{\pgfqpoint{3.705303in}{2.402361in}}%
\pgfpathlineto{\pgfqpoint{3.706093in}{2.406756in}}%
\pgfpathlineto{\pgfqpoint{3.707376in}{2.437352in}}%
\pgfpathlineto{\pgfqpoint{3.706784in}{2.406565in}}%
\pgfpathlineto{\pgfqpoint{3.708166in}{2.431603in}}%
\pgfpathlineto{\pgfqpoint{3.709350in}{2.445844in}}%
\pgfpathlineto{\pgfqpoint{3.708857in}{2.428213in}}%
\pgfpathlineto{\pgfqpoint{3.709547in}{2.440192in}}%
\pgfpathlineto{\pgfqpoint{3.709646in}{2.437682in}}%
\pgfpathlineto{\pgfqpoint{3.710238in}{2.446793in}}%
\pgfpathlineto{\pgfqpoint{3.710436in}{2.445103in}}%
\pgfpathlineto{\pgfqpoint{3.710534in}{2.444394in}}%
\pgfpathlineto{\pgfqpoint{3.710732in}{2.448037in}}%
\pgfpathlineto{\pgfqpoint{3.711719in}{2.462140in}}%
\pgfpathlineto{\pgfqpoint{3.711916in}{2.456378in}}%
\pgfpathlineto{\pgfqpoint{3.712114in}{2.451927in}}%
\pgfpathlineto{\pgfqpoint{3.712508in}{2.462975in}}%
\pgfpathlineto{\pgfqpoint{3.713002in}{2.453040in}}%
\pgfpathlineto{\pgfqpoint{3.713495in}{2.466818in}}%
\pgfpathlineto{\pgfqpoint{3.714088in}{2.456377in}}%
\pgfpathlineto{\pgfqpoint{3.715075in}{2.447137in}}%
\pgfpathlineto{\pgfqpoint{3.714581in}{2.458643in}}%
\pgfpathlineto{\pgfqpoint{3.715371in}{2.451683in}}%
\pgfpathlineto{\pgfqpoint{3.716456in}{2.461181in}}%
\pgfpathlineto{\pgfqpoint{3.716654in}{2.458083in}}%
\pgfpathlineto{\pgfqpoint{3.717740in}{2.445315in}}%
\pgfpathlineto{\pgfqpoint{3.718036in}{2.448913in}}%
\pgfpathlineto{\pgfqpoint{3.718332in}{2.445404in}}%
\pgfpathlineto{\pgfqpoint{3.718628in}{2.450969in}}%
\pgfpathlineto{\pgfqpoint{3.718726in}{2.453090in}}%
\pgfpathlineto{\pgfqpoint{3.719023in}{2.441318in}}%
\pgfpathlineto{\pgfqpoint{3.720306in}{2.426640in}}%
\pgfpathlineto{\pgfqpoint{3.722082in}{2.405828in}}%
\pgfpathlineto{\pgfqpoint{3.722773in}{2.407909in}}%
\pgfpathlineto{\pgfqpoint{3.722971in}{2.406440in}}%
\pgfpathlineto{\pgfqpoint{3.723661in}{2.411346in}}%
\pgfpathlineto{\pgfqpoint{3.724550in}{2.397984in}}%
\pgfpathlineto{\pgfqpoint{3.724846in}{2.402150in}}%
\pgfpathlineto{\pgfqpoint{3.725833in}{2.409308in}}%
\pgfpathlineto{\pgfqpoint{3.726030in}{2.407270in}}%
\pgfpathlineto{\pgfqpoint{3.726326in}{2.399790in}}%
\pgfpathlineto{\pgfqpoint{3.726820in}{2.411484in}}%
\pgfpathlineto{\pgfqpoint{3.727215in}{2.401746in}}%
\pgfpathlineto{\pgfqpoint{3.727609in}{2.411512in}}%
\pgfpathlineto{\pgfqpoint{3.728399in}{2.405866in}}%
\pgfpathlineto{\pgfqpoint{3.728794in}{2.396832in}}%
\pgfpathlineto{\pgfqpoint{3.729287in}{2.409203in}}%
\pgfpathlineto{\pgfqpoint{3.729781in}{2.418981in}}%
\pgfpathlineto{\pgfqpoint{3.730669in}{2.413377in}}%
\pgfpathlineto{\pgfqpoint{3.730768in}{2.410994in}}%
\pgfpathlineto{\pgfqpoint{3.731163in}{2.425354in}}%
\pgfpathlineto{\pgfqpoint{3.732742in}{2.443345in}}%
\pgfpathlineto{\pgfqpoint{3.731656in}{2.414059in}}%
\pgfpathlineto{\pgfqpoint{3.732939in}{2.440642in}}%
\pgfpathlineto{\pgfqpoint{3.733038in}{2.438173in}}%
\pgfpathlineto{\pgfqpoint{3.733334in}{2.451875in}}%
\pgfpathlineto{\pgfqpoint{3.733630in}{2.467268in}}%
\pgfpathlineto{\pgfqpoint{3.734518in}{2.465553in}}%
\pgfpathlineto{\pgfqpoint{3.735209in}{2.437045in}}%
\pgfpathlineto{\pgfqpoint{3.736591in}{2.388161in}}%
\pgfpathlineto{\pgfqpoint{3.737085in}{2.403092in}}%
\pgfpathlineto{\pgfqpoint{3.737183in}{2.403184in}}%
\pgfpathlineto{\pgfqpoint{3.739453in}{2.356593in}}%
\pgfpathlineto{\pgfqpoint{3.739848in}{2.369769in}}%
\pgfpathlineto{\pgfqpoint{3.740539in}{2.411969in}}%
\pgfpathlineto{\pgfqpoint{3.741131in}{2.386307in}}%
\pgfpathlineto{\pgfqpoint{3.742019in}{2.365398in}}%
\pgfpathlineto{\pgfqpoint{3.742513in}{2.373410in}}%
\pgfpathlineto{\pgfqpoint{3.743796in}{2.417633in}}%
\pgfpathlineto{\pgfqpoint{3.744487in}{2.563069in}}%
\pgfpathlineto{\pgfqpoint{3.745277in}{2.651417in}}%
\pgfpathlineto{\pgfqpoint{3.745770in}{2.617695in}}%
\pgfpathlineto{\pgfqpoint{3.746461in}{2.536783in}}%
\pgfpathlineto{\pgfqpoint{3.747152in}{2.361533in}}%
\pgfpathlineto{\pgfqpoint{3.747941in}{2.405824in}}%
\pgfpathlineto{\pgfqpoint{3.748139in}{2.402088in}}%
\pgfpathlineto{\pgfqpoint{3.748632in}{2.416145in}}%
\pgfpathlineto{\pgfqpoint{3.748830in}{2.415356in}}%
\pgfpathlineto{\pgfqpoint{3.748928in}{2.414935in}}%
\pgfpathlineto{\pgfqpoint{3.749323in}{2.416873in}}%
\pgfpathlineto{\pgfqpoint{3.750508in}{2.466858in}}%
\pgfpathlineto{\pgfqpoint{3.750804in}{2.440772in}}%
\pgfpathlineto{\pgfqpoint{3.751692in}{2.414619in}}%
\pgfpathlineto{\pgfqpoint{3.751988in}{2.424128in}}%
\pgfpathlineto{\pgfqpoint{3.753765in}{2.454642in}}%
\pgfpathlineto{\pgfqpoint{3.752383in}{2.423755in}}%
\pgfpathlineto{\pgfqpoint{3.753863in}{2.452499in}}%
\pgfpathlineto{\pgfqpoint{3.754357in}{2.415867in}}%
\pgfpathlineto{\pgfqpoint{3.755245in}{2.432648in}}%
\pgfpathlineto{\pgfqpoint{3.756824in}{2.456294in}}%
\pgfpathlineto{\pgfqpoint{3.755739in}{2.426143in}}%
\pgfpathlineto{\pgfqpoint{3.756923in}{2.455612in}}%
\pgfpathlineto{\pgfqpoint{3.757219in}{2.446053in}}%
\pgfpathlineto{\pgfqpoint{3.758107in}{2.451312in}}%
\pgfpathlineto{\pgfqpoint{3.760081in}{2.479198in}}%
\pgfpathlineto{\pgfqpoint{3.760377in}{2.473604in}}%
\pgfpathlineto{\pgfqpoint{3.760970in}{2.464665in}}%
\pgfpathlineto{\pgfqpoint{3.761266in}{2.474948in}}%
\pgfpathlineto{\pgfqpoint{3.762549in}{2.482348in}}%
\pgfpathlineto{\pgfqpoint{3.764227in}{2.497679in}}%
\pgfpathlineto{\pgfqpoint{3.764424in}{2.489804in}}%
\pgfpathlineto{\pgfqpoint{3.764720in}{2.476358in}}%
\pgfpathlineto{\pgfqpoint{3.765115in}{2.491824in}}%
\pgfpathlineto{\pgfqpoint{3.765510in}{2.483988in}}%
\pgfpathlineto{\pgfqpoint{3.766990in}{2.496134in}}%
\pgfpathlineto{\pgfqpoint{3.767089in}{2.494217in}}%
\pgfpathlineto{\pgfqpoint{3.769754in}{2.412088in}}%
\pgfpathlineto{\pgfqpoint{3.769951in}{2.409898in}}%
\pgfpathlineto{\pgfqpoint{3.770346in}{2.420949in}}%
\pgfpathlineto{\pgfqpoint{3.772715in}{2.469000in}}%
\pgfpathlineto{\pgfqpoint{3.772814in}{2.469064in}}%
\pgfpathlineto{\pgfqpoint{3.773899in}{2.442669in}}%
\pgfpathlineto{\pgfqpoint{3.774195in}{2.460386in}}%
\pgfpathlineto{\pgfqpoint{3.774294in}{2.460620in}}%
\pgfpathlineto{\pgfqpoint{3.775577in}{2.443237in}}%
\pgfpathlineto{\pgfqpoint{3.775084in}{2.469063in}}%
\pgfpathlineto{\pgfqpoint{3.775676in}{2.446063in}}%
\pgfpathlineto{\pgfqpoint{3.775972in}{2.456463in}}%
\pgfpathlineto{\pgfqpoint{3.776663in}{2.443636in}}%
\pgfpathlineto{\pgfqpoint{3.777748in}{2.431113in}}%
\pgfpathlineto{\pgfqpoint{3.777156in}{2.449481in}}%
\pgfpathlineto{\pgfqpoint{3.778242in}{2.435730in}}%
\pgfpathlineto{\pgfqpoint{3.778637in}{2.440711in}}%
\pgfpathlineto{\pgfqpoint{3.779426in}{2.437335in}}%
\pgfpathlineto{\pgfqpoint{3.779624in}{2.434656in}}%
\pgfpathlineto{\pgfqpoint{3.779920in}{2.441364in}}%
\pgfpathlineto{\pgfqpoint{3.781302in}{2.469283in}}%
\pgfpathlineto{\pgfqpoint{3.781499in}{2.465812in}}%
\pgfpathlineto{\pgfqpoint{3.781795in}{2.460395in}}%
\pgfpathlineto{\pgfqpoint{3.782190in}{2.474449in}}%
\pgfpathlineto{\pgfqpoint{3.782486in}{2.488778in}}%
\pgfpathlineto{\pgfqpoint{3.783078in}{2.463366in}}%
\pgfpathlineto{\pgfqpoint{3.783276in}{2.462795in}}%
\pgfpathlineto{\pgfqpoint{3.783670in}{2.445126in}}%
\pgfpathlineto{\pgfqpoint{3.785151in}{2.405979in}}%
\pgfpathlineto{\pgfqpoint{3.785250in}{2.408428in}}%
\pgfpathlineto{\pgfqpoint{3.785644in}{2.431256in}}%
\pgfpathlineto{\pgfqpoint{3.786335in}{2.412921in}}%
\pgfpathlineto{\pgfqpoint{3.786829in}{2.376514in}}%
\pgfpathlineto{\pgfqpoint{3.787816in}{2.389064in}}%
\pgfpathlineto{\pgfqpoint{3.789000in}{2.430113in}}%
\pgfpathlineto{\pgfqpoint{3.789296in}{2.411514in}}%
\pgfpathlineto{\pgfqpoint{3.790086in}{2.378755in}}%
\pgfpathlineto{\pgfqpoint{3.790875in}{2.379890in}}%
\pgfpathlineto{\pgfqpoint{3.791961in}{2.393516in}}%
\pgfpathlineto{\pgfqpoint{3.792948in}{2.559191in}}%
\pgfpathlineto{\pgfqpoint{3.793442in}{2.632300in}}%
\pgfpathlineto{\pgfqpoint{3.794231in}{2.605995in}}%
\pgfpathlineto{\pgfqpoint{3.794725in}{2.555035in}}%
\pgfpathlineto{\pgfqpoint{3.795514in}{2.343101in}}%
\pgfpathlineto{\pgfqpoint{3.796304in}{2.399130in}}%
\pgfpathlineto{\pgfqpoint{3.796501in}{2.400924in}}%
\pgfpathlineto{\pgfqpoint{3.796995in}{2.394839in}}%
\pgfpathlineto{\pgfqpoint{3.797093in}{2.394624in}}%
\pgfpathlineto{\pgfqpoint{3.797291in}{2.396409in}}%
\pgfpathlineto{\pgfqpoint{3.798278in}{2.442025in}}%
\pgfpathlineto{\pgfqpoint{3.798673in}{2.461677in}}%
\pgfpathlineto{\pgfqpoint{3.799265in}{2.437055in}}%
\pgfpathlineto{\pgfqpoint{3.799758in}{2.419424in}}%
\pgfpathlineto{\pgfqpoint{3.800548in}{2.427783in}}%
\pgfpathlineto{\pgfqpoint{3.800745in}{2.429658in}}%
\pgfpathlineto{\pgfqpoint{3.802325in}{2.457852in}}%
\pgfpathlineto{\pgfqpoint{3.802522in}{2.453145in}}%
\pgfpathlineto{\pgfqpoint{3.803015in}{2.428326in}}%
\pgfpathlineto{\pgfqpoint{3.803706in}{2.445476in}}%
\pgfpathlineto{\pgfqpoint{3.804792in}{2.475708in}}%
\pgfpathlineto{\pgfqpoint{3.805878in}{2.481072in}}%
\pgfpathlineto{\pgfqpoint{3.805483in}{2.470302in}}%
\pgfpathlineto{\pgfqpoint{3.805976in}{2.479861in}}%
\pgfpathlineto{\pgfqpoint{3.806174in}{2.476661in}}%
\pgfpathlineto{\pgfqpoint{3.806569in}{2.483193in}}%
\pgfpathlineto{\pgfqpoint{3.806963in}{2.481399in}}%
\pgfpathlineto{\pgfqpoint{3.808049in}{2.495679in}}%
\pgfpathlineto{\pgfqpoint{3.808345in}{2.491583in}}%
\pgfpathlineto{\pgfqpoint{3.808543in}{2.489321in}}%
\pgfpathlineto{\pgfqpoint{3.809036in}{2.495651in}}%
\pgfpathlineto{\pgfqpoint{3.810023in}{2.501041in}}%
\pgfpathlineto{\pgfqpoint{3.810220in}{2.498051in}}%
\pgfpathlineto{\pgfqpoint{3.810615in}{2.489111in}}%
\pgfpathlineto{\pgfqpoint{3.811306in}{2.495506in}}%
\pgfpathlineto{\pgfqpoint{3.811602in}{2.498925in}}%
\pgfpathlineto{\pgfqpoint{3.811997in}{2.493758in}}%
\pgfpathlineto{\pgfqpoint{3.813181in}{2.483719in}}%
\pgfpathlineto{\pgfqpoint{3.813379in}{2.486325in}}%
\pgfpathlineto{\pgfqpoint{3.814464in}{2.496486in}}%
\pgfpathlineto{\pgfqpoint{3.814662in}{2.492646in}}%
\pgfpathlineto{\pgfqpoint{3.815451in}{2.485305in}}%
\pgfpathlineto{\pgfqpoint{3.815748in}{2.492032in}}%
\pgfpathlineto{\pgfqpoint{3.815945in}{2.496292in}}%
\pgfpathlineto{\pgfqpoint{3.816735in}{2.489905in}}%
\pgfpathlineto{\pgfqpoint{3.818116in}{2.476146in}}%
\pgfpathlineto{\pgfqpoint{3.818807in}{2.478320in}}%
\pgfpathlineto{\pgfqpoint{3.818906in}{2.477251in}}%
\pgfpathlineto{\pgfqpoint{3.820781in}{2.449951in}}%
\pgfpathlineto{\pgfqpoint{3.820880in}{2.450587in}}%
\pgfpathlineto{\pgfqpoint{3.821275in}{2.456000in}}%
\pgfpathlineto{\pgfqpoint{3.821571in}{2.447391in}}%
\pgfpathlineto{\pgfqpoint{3.821768in}{2.444361in}}%
\pgfpathlineto{\pgfqpoint{3.822163in}{2.459170in}}%
\pgfpathlineto{\pgfqpoint{3.822262in}{2.460877in}}%
\pgfpathlineto{\pgfqpoint{3.822656in}{2.450076in}}%
\pgfpathlineto{\pgfqpoint{3.822755in}{2.448680in}}%
\pgfpathlineto{\pgfqpoint{3.823150in}{2.456283in}}%
\pgfpathlineto{\pgfqpoint{3.824137in}{2.467323in}}%
\pgfpathlineto{\pgfqpoint{3.823643in}{2.451607in}}%
\pgfpathlineto{\pgfqpoint{3.824334in}{2.461919in}}%
\pgfpathlineto{\pgfqpoint{3.825223in}{2.448680in}}%
\pgfpathlineto{\pgfqpoint{3.825519in}{2.455972in}}%
\pgfpathlineto{\pgfqpoint{3.825617in}{2.457542in}}%
\pgfpathlineto{\pgfqpoint{3.825914in}{2.447961in}}%
\pgfpathlineto{\pgfqpoint{3.826111in}{2.443361in}}%
\pgfpathlineto{\pgfqpoint{3.826604in}{2.463672in}}%
\pgfpathlineto{\pgfqpoint{3.826901in}{2.459563in}}%
\pgfpathlineto{\pgfqpoint{3.827197in}{2.464555in}}%
\pgfpathlineto{\pgfqpoint{3.829664in}{2.506213in}}%
\pgfpathlineto{\pgfqpoint{3.829763in}{2.506392in}}%
\pgfpathlineto{\pgfqpoint{3.830059in}{2.499767in}}%
\pgfpathlineto{\pgfqpoint{3.830454in}{2.513572in}}%
\pgfpathlineto{\pgfqpoint{3.830552in}{2.514854in}}%
\pgfpathlineto{\pgfqpoint{3.830750in}{2.508455in}}%
\pgfpathlineto{\pgfqpoint{3.832526in}{2.450471in}}%
\pgfpathlineto{\pgfqpoint{3.832822in}{2.450932in}}%
\pgfpathlineto{\pgfqpoint{3.833119in}{2.441134in}}%
\pgfpathlineto{\pgfqpoint{3.833513in}{2.461051in}}%
\pgfpathlineto{\pgfqpoint{3.833711in}{2.466290in}}%
\pgfpathlineto{\pgfqpoint{3.834402in}{2.456469in}}%
\pgfpathlineto{\pgfqpoint{3.834895in}{2.425679in}}%
\pgfpathlineto{\pgfqpoint{3.835685in}{2.442350in}}%
\pgfpathlineto{\pgfqpoint{3.835981in}{2.439963in}}%
\pgfpathlineto{\pgfqpoint{3.836178in}{2.442990in}}%
\pgfpathlineto{\pgfqpoint{3.836968in}{2.474165in}}%
\pgfpathlineto{\pgfqpoint{3.837560in}{2.455380in}}%
\pgfpathlineto{\pgfqpoint{3.838646in}{2.427771in}}%
\pgfpathlineto{\pgfqpoint{3.838843in}{2.434763in}}%
\pgfpathlineto{\pgfqpoint{3.839435in}{2.431383in}}%
\pgfpathlineto{\pgfqpoint{3.840521in}{2.518158in}}%
\pgfpathlineto{\pgfqpoint{3.841804in}{2.658394in}}%
\pgfpathlineto{\pgfqpoint{3.842100in}{2.633377in}}%
\pgfpathlineto{\pgfqpoint{3.843186in}{2.424391in}}%
\pgfpathlineto{\pgfqpoint{3.843482in}{2.372531in}}%
\pgfpathlineto{\pgfqpoint{3.844272in}{2.425090in}}%
\pgfpathlineto{\pgfqpoint{3.844469in}{2.428136in}}%
\pgfpathlineto{\pgfqpoint{3.844864in}{2.419322in}}%
\pgfpathlineto{\pgfqpoint{3.845259in}{2.424559in}}%
\pgfpathlineto{\pgfqpoint{3.845653in}{2.418645in}}%
\pgfpathlineto{\pgfqpoint{3.846147in}{2.426114in}}%
\pgfpathlineto{\pgfqpoint{3.846542in}{2.440075in}}%
\pgfpathlineto{\pgfqpoint{3.846936in}{2.421423in}}%
\pgfpathlineto{\pgfqpoint{3.847726in}{2.376429in}}%
\pgfpathlineto{\pgfqpoint{3.848318in}{2.396640in}}%
\pgfpathlineto{\pgfqpoint{3.849503in}{2.478154in}}%
\pgfpathlineto{\pgfqpoint{3.850193in}{2.513775in}}%
\pgfpathlineto{\pgfqpoint{3.850588in}{2.487858in}}%
\pgfpathlineto{\pgfqpoint{3.850687in}{2.484173in}}%
\pgfpathlineto{\pgfqpoint{3.851082in}{2.490391in}}%
\pgfpathlineto{\pgfqpoint{3.851575in}{2.488542in}}%
\pgfpathlineto{\pgfqpoint{3.853253in}{2.507126in}}%
\pgfpathlineto{\pgfqpoint{3.853451in}{2.503404in}}%
\pgfpathlineto{\pgfqpoint{3.854043in}{2.492278in}}%
\pgfpathlineto{\pgfqpoint{3.854635in}{2.498577in}}%
\pgfpathlineto{\pgfqpoint{3.855425in}{2.508286in}}%
\pgfpathlineto{\pgfqpoint{3.855622in}{2.502769in}}%
\pgfpathlineto{\pgfqpoint{3.855918in}{2.488126in}}%
\pgfpathlineto{\pgfqpoint{3.856510in}{2.507655in}}%
\pgfpathlineto{\pgfqpoint{3.856708in}{2.503379in}}%
\pgfpathlineto{\pgfqpoint{3.857596in}{2.494979in}}%
\pgfpathlineto{\pgfqpoint{3.857793in}{2.498510in}}%
\pgfpathlineto{\pgfqpoint{3.859274in}{2.512504in}}%
\pgfpathlineto{\pgfqpoint{3.859372in}{2.514059in}}%
\pgfpathlineto{\pgfqpoint{3.859866in}{2.504541in}}%
\pgfpathlineto{\pgfqpoint{3.860952in}{2.496742in}}%
\pgfpathlineto{\pgfqpoint{3.860458in}{2.510765in}}%
\pgfpathlineto{\pgfqpoint{3.861050in}{2.497794in}}%
\pgfpathlineto{\pgfqpoint{3.861445in}{2.512572in}}%
\pgfpathlineto{\pgfqpoint{3.862136in}{2.505420in}}%
\pgfpathlineto{\pgfqpoint{3.862531in}{2.478580in}}%
\pgfpathlineto{\pgfqpoint{3.863419in}{2.489972in}}%
\pgfpathlineto{\pgfqpoint{3.863715in}{2.497787in}}%
\pgfpathlineto{\pgfqpoint{3.864110in}{2.476775in}}%
\pgfpathlineto{\pgfqpoint{3.864209in}{2.474301in}}%
\pgfpathlineto{\pgfqpoint{3.864702in}{2.480196in}}%
\pgfpathlineto{\pgfqpoint{3.865196in}{2.476740in}}%
\pgfpathlineto{\pgfqpoint{3.865393in}{2.482971in}}%
\pgfpathlineto{\pgfqpoint{3.865788in}{2.459083in}}%
\pgfpathlineto{\pgfqpoint{3.866972in}{2.444836in}}%
\pgfpathlineto{\pgfqpoint{3.866380in}{2.459873in}}%
\pgfpathlineto{\pgfqpoint{3.867071in}{2.444891in}}%
\pgfpathlineto{\pgfqpoint{3.867170in}{2.445768in}}%
\pgfpathlineto{\pgfqpoint{3.867565in}{2.439986in}}%
\pgfpathlineto{\pgfqpoint{3.869045in}{2.433390in}}%
\pgfpathlineto{\pgfqpoint{3.869144in}{2.433492in}}%
\pgfpathlineto{\pgfqpoint{3.869341in}{2.433063in}}%
\pgfpathlineto{\pgfqpoint{3.869538in}{2.434362in}}%
\pgfpathlineto{\pgfqpoint{3.869835in}{2.439792in}}%
\pgfpathlineto{\pgfqpoint{3.870229in}{2.427468in}}%
\pgfpathlineto{\pgfqpoint{3.870328in}{2.427567in}}%
\pgfpathlineto{\pgfqpoint{3.870723in}{2.439984in}}%
\pgfpathlineto{\pgfqpoint{3.871809in}{2.438700in}}%
\pgfpathlineto{\pgfqpoint{3.872894in}{2.423559in}}%
\pgfpathlineto{\pgfqpoint{3.873092in}{2.429178in}}%
\pgfpathlineto{\pgfqpoint{3.873289in}{2.436615in}}%
\pgfpathlineto{\pgfqpoint{3.873980in}{2.423752in}}%
\pgfpathlineto{\pgfqpoint{3.874276in}{2.414766in}}%
\pgfpathlineto{\pgfqpoint{3.874868in}{2.429762in}}%
\pgfpathlineto{\pgfqpoint{3.876447in}{2.442221in}}%
\pgfpathlineto{\pgfqpoint{3.875460in}{2.426395in}}%
\pgfpathlineto{\pgfqpoint{3.876645in}{2.439965in}}%
\pgfpathlineto{\pgfqpoint{3.876744in}{2.438166in}}%
\pgfpathlineto{\pgfqpoint{3.877040in}{2.449373in}}%
\pgfpathlineto{\pgfqpoint{3.878125in}{2.468418in}}%
\pgfpathlineto{\pgfqpoint{3.878323in}{2.462491in}}%
\pgfpathlineto{\pgfqpoint{3.880198in}{2.403668in}}%
\pgfpathlineto{\pgfqpoint{3.880395in}{2.404769in}}%
\pgfpathlineto{\pgfqpoint{3.881481in}{2.428806in}}%
\pgfpathlineto{\pgfqpoint{3.881777in}{2.415270in}}%
\pgfpathlineto{\pgfqpoint{3.882369in}{2.378880in}}%
\pgfpathlineto{\pgfqpoint{3.882962in}{2.409057in}}%
\pgfpathlineto{\pgfqpoint{3.883356in}{2.403785in}}%
\pgfpathlineto{\pgfqpoint{3.883751in}{2.412105in}}%
\pgfpathlineto{\pgfqpoint{3.884738in}{2.448410in}}%
\pgfpathlineto{\pgfqpoint{3.885133in}{2.425107in}}%
\pgfpathlineto{\pgfqpoint{3.885528in}{2.397310in}}%
\pgfpathlineto{\pgfqpoint{3.886416in}{2.405225in}}%
\pgfpathlineto{\pgfqpoint{3.887206in}{2.420215in}}%
\pgfpathlineto{\pgfqpoint{3.888291in}{2.564647in}}%
\pgfpathlineto{\pgfqpoint{3.889180in}{2.644530in}}%
\pgfpathlineto{\pgfqpoint{3.889574in}{2.633468in}}%
\pgfpathlineto{\pgfqpoint{3.890068in}{2.616450in}}%
\pgfpathlineto{\pgfqpoint{3.891055in}{2.371937in}}%
\pgfpathlineto{\pgfqpoint{3.892338in}{2.416771in}}%
\pgfpathlineto{\pgfqpoint{3.894016in}{2.487341in}}%
\pgfpathlineto{\pgfqpoint{3.894312in}{2.499778in}}%
\pgfpathlineto{\pgfqpoint{3.894904in}{2.478358in}}%
\pgfpathlineto{\pgfqpoint{3.895200in}{2.459437in}}%
\pgfpathlineto{\pgfqpoint{3.895694in}{2.482541in}}%
\pgfpathlineto{\pgfqpoint{3.895990in}{2.475539in}}%
\pgfpathlineto{\pgfqpoint{3.896089in}{2.475660in}}%
\pgfpathlineto{\pgfqpoint{3.897372in}{2.523430in}}%
\pgfpathlineto{\pgfqpoint{3.898062in}{2.500838in}}%
\pgfpathlineto{\pgfqpoint{3.898753in}{2.479447in}}%
\pgfpathlineto{\pgfqpoint{3.899247in}{2.497874in}}%
\pgfpathlineto{\pgfqpoint{3.899346in}{2.497875in}}%
\pgfpathlineto{\pgfqpoint{3.899642in}{2.491823in}}%
\pgfpathlineto{\pgfqpoint{3.900135in}{2.503434in}}%
\pgfpathlineto{\pgfqpoint{3.900431in}{2.497040in}}%
\pgfpathlineto{\pgfqpoint{3.905465in}{2.565067in}}%
\pgfpathlineto{\pgfqpoint{3.906353in}{2.562374in}}%
\pgfpathlineto{\pgfqpoint{3.906649in}{2.556423in}}%
\pgfpathlineto{\pgfqpoint{3.907340in}{2.562834in}}%
\pgfpathlineto{\pgfqpoint{3.907735in}{2.568843in}}%
\pgfpathlineto{\pgfqpoint{3.908130in}{2.559162in}}%
\pgfpathlineto{\pgfqpoint{3.908426in}{2.547396in}}%
\pgfpathlineto{\pgfqpoint{3.908821in}{2.561368in}}%
\pgfpathlineto{\pgfqpoint{3.909314in}{2.547696in}}%
\pgfpathlineto{\pgfqpoint{3.909610in}{2.552666in}}%
\pgfpathlineto{\pgfqpoint{3.910005in}{2.538460in}}%
\pgfpathlineto{\pgfqpoint{3.910301in}{2.542564in}}%
\pgfpathlineto{\pgfqpoint{3.910992in}{2.559321in}}%
\pgfpathlineto{\pgfqpoint{3.911979in}{2.551944in}}%
\pgfpathlineto{\pgfqpoint{3.912176in}{2.547506in}}%
\pgfpathlineto{\pgfqpoint{3.912571in}{2.564694in}}%
\pgfpathlineto{\pgfqpoint{3.912769in}{2.569342in}}%
\pgfpathlineto{\pgfqpoint{3.913163in}{2.550061in}}%
\pgfpathlineto{\pgfqpoint{3.913657in}{2.537143in}}%
\pgfpathlineto{\pgfqpoint{3.914150in}{2.551026in}}%
\pgfpathlineto{\pgfqpoint{3.914447in}{2.553284in}}%
\pgfpathlineto{\pgfqpoint{3.914841in}{2.548385in}}%
\pgfpathlineto{\pgfqpoint{3.915137in}{2.544181in}}%
\pgfpathlineto{\pgfqpoint{3.915532in}{2.549602in}}%
\pgfpathlineto{\pgfqpoint{3.915927in}{2.546816in}}%
\pgfpathlineto{\pgfqpoint{3.916519in}{2.543803in}}%
\pgfpathlineto{\pgfqpoint{3.916717in}{2.545789in}}%
\pgfpathlineto{\pgfqpoint{3.917013in}{2.549083in}}%
\pgfpathlineto{\pgfqpoint{3.917407in}{2.540715in}}%
\pgfpathlineto{\pgfqpoint{3.917901in}{2.548707in}}%
\pgfpathlineto{\pgfqpoint{3.919480in}{2.532921in}}%
\pgfpathlineto{\pgfqpoint{3.919678in}{2.534111in}}%
\pgfpathlineto{\pgfqpoint{3.920566in}{2.544185in}}%
\pgfpathlineto{\pgfqpoint{3.920862in}{2.535689in}}%
\pgfpathlineto{\pgfqpoint{3.921454in}{2.520220in}}%
\pgfpathlineto{\pgfqpoint{3.922046in}{2.532543in}}%
\pgfpathlineto{\pgfqpoint{3.922441in}{2.543379in}}%
\pgfpathlineto{\pgfqpoint{3.923132in}{2.536847in}}%
\pgfpathlineto{\pgfqpoint{3.926389in}{2.489197in}}%
\pgfpathlineto{\pgfqpoint{3.923626in}{2.540212in}}%
\pgfpathlineto{\pgfqpoint{3.926586in}{2.492700in}}%
\pgfpathlineto{\pgfqpoint{3.928659in}{2.555033in}}%
\pgfpathlineto{\pgfqpoint{3.928758in}{2.557160in}}%
\pgfpathlineto{\pgfqpoint{3.929153in}{2.547392in}}%
\pgfpathlineto{\pgfqpoint{3.930732in}{2.481990in}}%
\pgfpathlineto{\pgfqpoint{3.931423in}{2.485570in}}%
\pgfpathlineto{\pgfqpoint{3.932015in}{2.518347in}}%
\pgfpathlineto{\pgfqpoint{3.932508in}{2.495354in}}%
\pgfpathlineto{\pgfqpoint{3.933693in}{2.452143in}}%
\pgfpathlineto{\pgfqpoint{3.933890in}{2.459016in}}%
\pgfpathlineto{\pgfqpoint{3.935765in}{2.626346in}}%
\pgfpathlineto{\pgfqpoint{3.936752in}{2.725122in}}%
\pgfpathlineto{\pgfqpoint{3.937246in}{2.695926in}}%
\pgfpathlineto{\pgfqpoint{3.937838in}{2.624579in}}%
\pgfpathlineto{\pgfqpoint{3.938529in}{2.439037in}}%
\pgfpathlineto{\pgfqpoint{3.939417in}{2.485586in}}%
\pgfpathlineto{\pgfqpoint{3.939812in}{2.483074in}}%
\pgfpathlineto{\pgfqpoint{3.940108in}{2.488581in}}%
\pgfpathlineto{\pgfqpoint{3.941786in}{2.516801in}}%
\pgfpathlineto{\pgfqpoint{3.941885in}{2.516361in}}%
\pgfpathlineto{\pgfqpoint{3.942971in}{2.460082in}}%
\pgfpathlineto{\pgfqpoint{3.943760in}{2.475265in}}%
\pgfpathlineto{\pgfqpoint{3.944747in}{2.508238in}}%
\pgfpathlineto{\pgfqpoint{3.945043in}{2.516545in}}%
\pgfpathlineto{\pgfqpoint{3.945438in}{2.502417in}}%
\pgfpathlineto{\pgfqpoint{3.945833in}{2.491216in}}%
\pgfpathlineto{\pgfqpoint{3.946425in}{2.502449in}}%
\pgfpathlineto{\pgfqpoint{3.946721in}{2.512952in}}%
\pgfpathlineto{\pgfqpoint{3.947609in}{2.507126in}}%
\pgfpathlineto{\pgfqpoint{3.948103in}{2.533012in}}%
\pgfpathlineto{\pgfqpoint{3.948991in}{2.512655in}}%
\pgfpathlineto{\pgfqpoint{3.949485in}{2.501706in}}%
\pgfpathlineto{\pgfqpoint{3.949583in}{2.499640in}}%
\pgfpathlineto{\pgfqpoint{3.949879in}{2.511916in}}%
\pgfpathlineto{\pgfqpoint{3.950077in}{2.524987in}}%
\pgfpathlineto{\pgfqpoint{3.950669in}{2.505778in}}%
\pgfpathlineto{\pgfqpoint{3.950866in}{2.506039in}}%
\pgfpathlineto{\pgfqpoint{3.950965in}{2.505782in}}%
\pgfpathlineto{\pgfqpoint{3.951163in}{2.507947in}}%
\pgfpathlineto{\pgfqpoint{3.952051in}{2.528594in}}%
\pgfpathlineto{\pgfqpoint{3.951557in}{2.507906in}}%
\pgfpathlineto{\pgfqpoint{3.952248in}{2.515964in}}%
\pgfpathlineto{\pgfqpoint{3.952643in}{2.485888in}}%
\pgfpathlineto{\pgfqpoint{3.953433in}{2.503678in}}%
\pgfpathlineto{\pgfqpoint{3.954814in}{2.493640in}}%
\pgfpathlineto{\pgfqpoint{3.955012in}{2.496634in}}%
\pgfpathlineto{\pgfqpoint{3.955308in}{2.505219in}}%
\pgfpathlineto{\pgfqpoint{3.955801in}{2.494207in}}%
\pgfpathlineto{\pgfqpoint{3.956295in}{2.504698in}}%
\pgfpathlineto{\pgfqpoint{3.957381in}{2.490614in}}%
\pgfpathlineto{\pgfqpoint{3.957874in}{2.499537in}}%
\pgfpathlineto{\pgfqpoint{3.958071in}{2.499286in}}%
\pgfpathlineto{\pgfqpoint{3.958170in}{2.499761in}}%
\pgfpathlineto{\pgfqpoint{3.958466in}{2.504690in}}%
\pgfpathlineto{\pgfqpoint{3.958960in}{2.496520in}}%
\pgfpathlineto{\pgfqpoint{3.959552in}{2.484938in}}%
\pgfpathlineto{\pgfqpoint{3.959848in}{2.498202in}}%
\pgfpathlineto{\pgfqpoint{3.960045in}{2.505001in}}%
\pgfpathlineto{\pgfqpoint{3.960539in}{2.480785in}}%
\pgfpathlineto{\pgfqpoint{3.961329in}{2.474746in}}%
\pgfpathlineto{\pgfqpoint{3.960934in}{2.486610in}}%
\pgfpathlineto{\pgfqpoint{3.961526in}{2.480561in}}%
\pgfpathlineto{\pgfqpoint{3.961723in}{2.488503in}}%
\pgfpathlineto{\pgfqpoint{3.962217in}{2.459142in}}%
\pgfpathlineto{\pgfqpoint{3.962710in}{2.449903in}}%
\pgfpathlineto{\pgfqpoint{3.963105in}{2.462720in}}%
\pgfpathlineto{\pgfqpoint{3.963204in}{2.463430in}}%
\pgfpathlineto{\pgfqpoint{3.963401in}{2.457502in}}%
\pgfpathlineto{\pgfqpoint{3.964289in}{2.442963in}}%
\pgfpathlineto{\pgfqpoint{3.963796in}{2.460459in}}%
\pgfpathlineto{\pgfqpoint{3.964586in}{2.455871in}}%
\pgfpathlineto{\pgfqpoint{3.964783in}{2.463014in}}%
\pgfpathlineto{\pgfqpoint{3.965178in}{2.450509in}}%
\pgfpathlineto{\pgfqpoint{3.965573in}{2.453960in}}%
\pgfpathlineto{\pgfqpoint{3.966560in}{2.448205in}}%
\pgfpathlineto{\pgfqpoint{3.966165in}{2.456876in}}%
\pgfpathlineto{\pgfqpoint{3.966658in}{2.451325in}}%
\pgfpathlineto{\pgfqpoint{3.966856in}{2.457545in}}%
\pgfpathlineto{\pgfqpoint{3.967250in}{2.434225in}}%
\pgfpathlineto{\pgfqpoint{3.967744in}{2.454397in}}%
\pgfpathlineto{\pgfqpoint{3.968139in}{2.447426in}}%
\pgfpathlineto{\pgfqpoint{3.968534in}{2.458023in}}%
\pgfpathlineto{\pgfqpoint{3.968731in}{2.454897in}}%
\pgfpathlineto{\pgfqpoint{3.968928in}{2.449534in}}%
\pgfpathlineto{\pgfqpoint{3.969422in}{2.464149in}}%
\pgfpathlineto{\pgfqpoint{3.972087in}{2.509202in}}%
\pgfpathlineto{\pgfqpoint{3.972876in}{2.510678in}}%
\pgfpathlineto{\pgfqpoint{3.974357in}{2.465632in}}%
\pgfpathlineto{\pgfqpoint{3.975442in}{2.442842in}}%
\pgfpathlineto{\pgfqpoint{3.975640in}{2.453236in}}%
\pgfpathlineto{\pgfqpoint{3.975936in}{2.465566in}}%
\pgfpathlineto{\pgfqpoint{3.976726in}{2.453624in}}%
\pgfpathlineto{\pgfqpoint{3.977022in}{2.444009in}}%
\pgfpathlineto{\pgfqpoint{3.977811in}{2.451166in}}%
\pgfpathlineto{\pgfqpoint{3.979390in}{2.497331in}}%
\pgfpathlineto{\pgfqpoint{3.979588in}{2.489419in}}%
\pgfpathlineto{\pgfqpoint{3.980180in}{2.435417in}}%
\pgfpathlineto{\pgfqpoint{3.981167in}{2.444618in}}%
\pgfpathlineto{\pgfqpoint{3.982746in}{2.534800in}}%
\pgfpathlineto{\pgfqpoint{3.983832in}{2.671864in}}%
\pgfpathlineto{\pgfqpoint{3.984325in}{2.660335in}}%
\pgfpathlineto{\pgfqpoint{3.984918in}{2.630558in}}%
\pgfpathlineto{\pgfqpoint{3.985707in}{2.405031in}}%
\pgfpathlineto{\pgfqpoint{3.986892in}{2.453505in}}%
\pgfpathlineto{\pgfqpoint{3.987089in}{2.446460in}}%
\pgfpathlineto{\pgfqpoint{3.987484in}{2.458113in}}%
\pgfpathlineto{\pgfqpoint{3.987879in}{2.457384in}}%
\pgfpathlineto{\pgfqpoint{3.989063in}{2.488842in}}%
\pgfpathlineto{\pgfqpoint{3.989359in}{2.473084in}}%
\pgfpathlineto{\pgfqpoint{3.990346in}{2.416965in}}%
\pgfpathlineto{\pgfqpoint{3.990938in}{2.429907in}}%
\pgfpathlineto{\pgfqpoint{3.991037in}{2.429552in}}%
\pgfpathlineto{\pgfqpoint{3.991234in}{2.432863in}}%
\pgfpathlineto{\pgfqpoint{3.992221in}{2.454319in}}%
\pgfpathlineto{\pgfqpoint{3.992517in}{2.444547in}}%
\pgfpathlineto{\pgfqpoint{3.993208in}{2.420906in}}%
\pgfpathlineto{\pgfqpoint{3.993998in}{2.425861in}}%
\pgfpathlineto{\pgfqpoint{3.995478in}{2.446362in}}%
\pgfpathlineto{\pgfqpoint{3.995676in}{2.441813in}}%
\pgfpathlineto{\pgfqpoint{3.995774in}{2.439962in}}%
\pgfpathlineto{\pgfqpoint{3.996071in}{2.453656in}}%
\pgfpathlineto{\pgfqpoint{3.996268in}{2.462257in}}%
\pgfpathlineto{\pgfqpoint{3.996860in}{2.446432in}}%
\pgfpathlineto{\pgfqpoint{3.997058in}{2.449151in}}%
\pgfpathlineto{\pgfqpoint{3.997650in}{2.460155in}}%
\pgfpathlineto{\pgfqpoint{3.998045in}{2.448980in}}%
\pgfpathlineto{\pgfqpoint{3.998143in}{2.448287in}}%
\pgfpathlineto{\pgfqpoint{3.998341in}{2.453570in}}%
\pgfpathlineto{\pgfqpoint{3.999229in}{2.465572in}}%
\pgfpathlineto{\pgfqpoint{3.998834in}{2.450534in}}%
\pgfpathlineto{\pgfqpoint{3.999426in}{2.459550in}}%
\pgfpathlineto{\pgfqpoint{4.000315in}{2.447817in}}%
\pgfpathlineto{\pgfqpoint{4.000512in}{2.452307in}}%
\pgfpathlineto{\pgfqpoint{4.000907in}{2.477346in}}%
\pgfpathlineto{\pgfqpoint{4.001598in}{2.455361in}}%
\pgfpathlineto{\pgfqpoint{4.001696in}{2.454847in}}%
\pgfpathlineto{\pgfqpoint{4.001894in}{2.458861in}}%
\pgfpathlineto{\pgfqpoint{4.002881in}{2.465691in}}%
\pgfpathlineto{\pgfqpoint{4.003078in}{2.463322in}}%
\pgfpathlineto{\pgfqpoint{4.006237in}{2.386170in}}%
\pgfpathlineto{\pgfqpoint{4.006730in}{2.392957in}}%
\pgfpathlineto{\pgfqpoint{4.008112in}{2.451225in}}%
\pgfpathlineto{\pgfqpoint{4.009494in}{2.441771in}}%
\pgfpathlineto{\pgfqpoint{4.011468in}{2.397908in}}%
\pgfpathlineto{\pgfqpoint{4.011862in}{2.410623in}}%
\pgfpathlineto{\pgfqpoint{4.012553in}{2.419403in}}%
\pgfpathlineto{\pgfqpoint{4.012257in}{2.410028in}}%
\pgfpathlineto{\pgfqpoint{4.012849in}{2.410015in}}%
\pgfpathlineto{\pgfqpoint{4.013639in}{2.405344in}}%
\pgfpathlineto{\pgfqpoint{4.013343in}{2.416491in}}%
\pgfpathlineto{\pgfqpoint{4.013836in}{2.413153in}}%
\pgfpathlineto{\pgfqpoint{4.014034in}{2.422891in}}%
\pgfpathlineto{\pgfqpoint{4.014527in}{2.402947in}}%
\pgfpathlineto{\pgfqpoint{4.014823in}{2.406274in}}%
\pgfpathlineto{\pgfqpoint{4.015021in}{2.404720in}}%
\pgfpathlineto{\pgfqpoint{4.015218in}{2.410254in}}%
\pgfpathlineto{\pgfqpoint{4.015613in}{2.424715in}}%
\pgfpathlineto{\pgfqpoint{4.016403in}{2.415139in}}%
\pgfpathlineto{\pgfqpoint{4.016797in}{2.429863in}}%
\pgfpathlineto{\pgfqpoint{4.017390in}{2.460669in}}%
\pgfpathlineto{\pgfqpoint{4.018179in}{2.455463in}}%
\pgfpathlineto{\pgfqpoint{4.018574in}{2.463499in}}%
\pgfpathlineto{\pgfqpoint{4.019758in}{2.499852in}}%
\pgfpathlineto{\pgfqpoint{4.020252in}{2.494815in}}%
\pgfpathlineto{\pgfqpoint{4.020745in}{2.503410in}}%
\pgfpathlineto{\pgfqpoint{4.021041in}{2.495336in}}%
\pgfpathlineto{\pgfqpoint{4.021732in}{2.463684in}}%
\pgfpathlineto{\pgfqpoint{4.022127in}{2.488937in}}%
\pgfpathlineto{\pgfqpoint{4.023213in}{2.494467in}}%
\pgfpathlineto{\pgfqpoint{4.022621in}{2.483085in}}%
\pgfpathlineto{\pgfqpoint{4.023311in}{2.494150in}}%
\pgfpathlineto{\pgfqpoint{4.023509in}{2.491659in}}%
\pgfpathlineto{\pgfqpoint{4.025187in}{2.464672in}}%
\pgfpathlineto{\pgfqpoint{4.026174in}{2.501347in}}%
\pgfpathlineto{\pgfqpoint{4.026470in}{2.484542in}}%
\pgfpathlineto{\pgfqpoint{4.028641in}{2.402922in}}%
\pgfpathlineto{\pgfqpoint{4.028937in}{2.393391in}}%
\pgfpathlineto{\pgfqpoint{4.029332in}{2.412421in}}%
\pgfpathlineto{\pgfqpoint{4.030220in}{2.555675in}}%
\pgfpathlineto{\pgfqpoint{4.030516in}{2.580546in}}%
\pgfpathlineto{\pgfqpoint{4.031207in}{2.549078in}}%
\pgfpathlineto{\pgfqpoint{4.031800in}{2.517674in}}%
\pgfpathlineto{\pgfqpoint{4.032490in}{2.317140in}}%
\pgfpathlineto{\pgfqpoint{4.032787in}{2.262111in}}%
\pgfpathlineto{\pgfqpoint{4.033576in}{2.313381in}}%
\pgfpathlineto{\pgfqpoint{4.033872in}{2.325267in}}%
\pgfpathlineto{\pgfqpoint{4.034366in}{2.300159in}}%
\pgfpathlineto{\pgfqpoint{4.034563in}{2.297160in}}%
\pgfpathlineto{\pgfqpoint{4.034958in}{2.311412in}}%
\pgfpathlineto{\pgfqpoint{4.035846in}{2.355879in}}%
\pgfpathlineto{\pgfqpoint{4.036241in}{2.335683in}}%
\pgfpathlineto{\pgfqpoint{4.037820in}{2.304331in}}%
\pgfpathlineto{\pgfqpoint{4.038018in}{2.310759in}}%
\pgfpathlineto{\pgfqpoint{4.038906in}{2.355775in}}%
\pgfpathlineto{\pgfqpoint{4.039399in}{2.338694in}}%
\pgfpathlineto{\pgfqpoint{4.040288in}{2.354636in}}%
\pgfpathlineto{\pgfqpoint{4.042360in}{2.401899in}}%
\pgfpathlineto{\pgfqpoint{4.042656in}{2.394340in}}%
\pgfpathlineto{\pgfqpoint{4.043347in}{2.388849in}}%
\pgfpathlineto{\pgfqpoint{4.043643in}{2.395924in}}%
\pgfpathlineto{\pgfqpoint{4.045223in}{2.431026in}}%
\pgfpathlineto{\pgfqpoint{4.044334in}{2.391708in}}%
\pgfpathlineto{\pgfqpoint{4.045420in}{2.426369in}}%
\pgfpathlineto{\pgfqpoint{4.046210in}{2.409308in}}%
\pgfpathlineto{\pgfqpoint{4.046604in}{2.419859in}}%
\pgfpathlineto{\pgfqpoint{4.047591in}{2.419772in}}%
\pgfpathlineto{\pgfqpoint{4.048282in}{2.443316in}}%
\pgfpathlineto{\pgfqpoint{4.048578in}{2.446706in}}%
\pgfpathlineto{\pgfqpoint{4.048874in}{2.439200in}}%
\pgfpathlineto{\pgfqpoint{4.049072in}{2.435256in}}%
\pgfpathlineto{\pgfqpoint{4.049664in}{2.446108in}}%
\pgfpathlineto{\pgfqpoint{4.051638in}{2.487696in}}%
\pgfpathlineto{\pgfqpoint{4.055093in}{2.540572in}}%
\pgfpathlineto{\pgfqpoint{4.056474in}{2.533605in}}%
\pgfpathlineto{\pgfqpoint{4.057659in}{2.510247in}}%
\pgfpathlineto{\pgfqpoint{4.057165in}{2.537129in}}%
\pgfpathlineto{\pgfqpoint{4.058053in}{2.517807in}}%
\pgfpathlineto{\pgfqpoint{4.058152in}{2.518380in}}%
\pgfpathlineto{\pgfqpoint{4.058547in}{2.514143in}}%
\pgfpathlineto{\pgfqpoint{4.060916in}{2.475458in}}%
\pgfpathlineto{\pgfqpoint{4.061014in}{2.475661in}}%
\pgfpathlineto{\pgfqpoint{4.061409in}{2.483939in}}%
\pgfpathlineto{\pgfqpoint{4.061804in}{2.468208in}}%
\pgfpathlineto{\pgfqpoint{4.064272in}{2.409728in}}%
\pgfpathlineto{\pgfqpoint{4.064469in}{2.417255in}}%
\pgfpathlineto{\pgfqpoint{4.064765in}{2.430561in}}%
\pgfpathlineto{\pgfqpoint{4.065160in}{2.406863in}}%
\pgfpathlineto{\pgfqpoint{4.065357in}{2.408031in}}%
\pgfpathlineto{\pgfqpoint{4.066245in}{2.410128in}}%
\pgfpathlineto{\pgfqpoint{4.068417in}{2.328180in}}%
\pgfpathlineto{\pgfqpoint{4.069009in}{2.287575in}}%
\pgfpathlineto{\pgfqpoint{4.069799in}{2.297550in}}%
\pgfpathlineto{\pgfqpoint{4.069996in}{2.298526in}}%
\pgfpathlineto{\pgfqpoint{4.070292in}{2.292780in}}%
\pgfpathlineto{\pgfqpoint{4.072167in}{2.242249in}}%
\pgfpathlineto{\pgfqpoint{4.072266in}{2.243558in}}%
\pgfpathlineto{\pgfqpoint{4.073154in}{2.307033in}}%
\pgfpathlineto{\pgfqpoint{4.073747in}{2.274171in}}%
\pgfpathlineto{\pgfqpoint{4.074141in}{2.241896in}}%
\pgfpathlineto{\pgfqpoint{4.074931in}{2.254036in}}%
\pgfpathlineto{\pgfqpoint{4.075030in}{2.254271in}}%
\pgfpathlineto{\pgfqpoint{4.075128in}{2.252149in}}%
\pgfpathlineto{\pgfqpoint{4.075424in}{2.241136in}}%
\pgfpathlineto{\pgfqpoint{4.075819in}{2.259903in}}%
\pgfpathlineto{\pgfqpoint{4.077201in}{2.428550in}}%
\pgfpathlineto{\pgfqpoint{4.077892in}{2.521812in}}%
\pgfpathlineto{\pgfqpoint{4.078583in}{2.507080in}}%
\pgfpathlineto{\pgfqpoint{4.078978in}{2.465591in}}%
\pgfpathlineto{\pgfqpoint{4.079767in}{2.279747in}}%
\pgfpathlineto{\pgfqpoint{4.080754in}{2.308450in}}%
\pgfpathlineto{\pgfqpoint{4.082531in}{2.375089in}}%
\pgfpathlineto{\pgfqpoint{4.082926in}{2.403862in}}%
\pgfpathlineto{\pgfqpoint{4.083518in}{2.374320in}}%
\pgfpathlineto{\pgfqpoint{4.083913in}{2.362083in}}%
\pgfpathlineto{\pgfqpoint{4.084406in}{2.379757in}}%
\pgfpathlineto{\pgfqpoint{4.086084in}{2.421785in}}%
\pgfpathlineto{\pgfqpoint{4.086281in}{2.416318in}}%
\pgfpathlineto{\pgfqpoint{4.087170in}{2.359904in}}%
\pgfpathlineto{\pgfqpoint{4.088058in}{2.369671in}}%
\pgfpathlineto{\pgfqpoint{4.088354in}{2.365953in}}%
\pgfpathlineto{\pgfqpoint{4.088551in}{2.372716in}}%
\pgfpathlineto{\pgfqpoint{4.091216in}{2.517724in}}%
\pgfpathlineto{\pgfqpoint{4.091414in}{2.513907in}}%
\pgfpathlineto{\pgfqpoint{4.091611in}{2.508462in}}%
\pgfpathlineto{\pgfqpoint{4.092006in}{2.526911in}}%
\pgfpathlineto{\pgfqpoint{4.092598in}{2.540514in}}%
\pgfpathlineto{\pgfqpoint{4.093190in}{2.534338in}}%
\pgfpathlineto{\pgfqpoint{4.093881in}{2.542013in}}%
\pgfpathlineto{\pgfqpoint{4.094572in}{2.569773in}}%
\pgfpathlineto{\pgfqpoint{4.094967in}{2.543774in}}%
\pgfpathlineto{\pgfqpoint{4.095164in}{2.533701in}}%
\pgfpathlineto{\pgfqpoint{4.095559in}{2.546412in}}%
\pgfpathlineto{\pgfqpoint{4.095954in}{2.542642in}}%
\pgfpathlineto{\pgfqpoint{4.096151in}{2.547008in}}%
\pgfpathlineto{\pgfqpoint{4.096447in}{2.529961in}}%
\pgfpathlineto{\pgfqpoint{4.096645in}{2.521419in}}%
\pgfpathlineto{\pgfqpoint{4.097336in}{2.539066in}}%
\pgfpathlineto{\pgfqpoint{4.097533in}{2.543161in}}%
\pgfpathlineto{\pgfqpoint{4.098125in}{2.531515in}}%
\pgfpathlineto{\pgfqpoint{4.098323in}{2.529720in}}%
\pgfpathlineto{\pgfqpoint{4.098619in}{2.535164in}}%
\pgfpathlineto{\pgfqpoint{4.098816in}{2.540746in}}%
\pgfpathlineto{\pgfqpoint{4.099310in}{2.527105in}}%
\pgfpathlineto{\pgfqpoint{4.099704in}{2.535590in}}%
\pgfpathlineto{\pgfqpoint{4.101876in}{2.461272in}}%
\pgfpathlineto{\pgfqpoint{4.102073in}{2.464311in}}%
\pgfpathlineto{\pgfqpoint{4.102172in}{2.465034in}}%
\pgfpathlineto{\pgfqpoint{4.102369in}{2.460294in}}%
\pgfpathlineto{\pgfqpoint{4.108587in}{2.286115in}}%
\pgfpathlineto{\pgfqpoint{4.108686in}{2.287299in}}%
\pgfpathlineto{\pgfqpoint{4.109081in}{2.296144in}}%
\pgfpathlineto{\pgfqpoint{4.109673in}{2.286084in}}%
\pgfpathlineto{\pgfqpoint{4.109969in}{2.279076in}}%
\pgfpathlineto{\pgfqpoint{4.110660in}{2.287511in}}%
\pgfpathlineto{\pgfqpoint{4.110759in}{2.287797in}}%
\pgfpathlineto{\pgfqpoint{4.110956in}{2.286786in}}%
\pgfpathlineto{\pgfqpoint{4.111548in}{2.275370in}}%
\pgfpathlineto{\pgfqpoint{4.111844in}{2.284519in}}%
\pgfpathlineto{\pgfqpoint{4.113127in}{2.303894in}}%
\pgfpathlineto{\pgfqpoint{4.113325in}{2.300967in}}%
\pgfpathlineto{\pgfqpoint{4.113621in}{2.308578in}}%
\pgfpathlineto{\pgfqpoint{4.114016in}{2.329918in}}%
\pgfpathlineto{\pgfqpoint{4.114509in}{2.301242in}}%
\pgfpathlineto{\pgfqpoint{4.116385in}{2.261319in}}%
\pgfpathlineto{\pgfqpoint{4.116779in}{2.280929in}}%
\pgfpathlineto{\pgfqpoint{4.117668in}{2.267013in}}%
\pgfpathlineto{\pgfqpoint{4.118359in}{2.244677in}}%
\pgfpathlineto{\pgfqpoint{4.118753in}{2.265945in}}%
\pgfpathlineto{\pgfqpoint{4.120530in}{2.337485in}}%
\pgfpathlineto{\pgfqpoint{4.119346in}{2.261155in}}%
\pgfpathlineto{\pgfqpoint{4.120727in}{2.330856in}}%
\pgfpathlineto{\pgfqpoint{4.121616in}{2.282556in}}%
\pgfpathlineto{\pgfqpoint{4.122010in}{2.304042in}}%
\pgfpathlineto{\pgfqpoint{4.122208in}{2.309765in}}%
\pgfpathlineto{\pgfqpoint{4.122603in}{2.296030in}}%
\pgfpathlineto{\pgfqpoint{4.122899in}{2.298136in}}%
\pgfpathlineto{\pgfqpoint{4.122997in}{2.296819in}}%
\pgfpathlineto{\pgfqpoint{4.123195in}{2.301399in}}%
\pgfpathlineto{\pgfqpoint{4.124478in}{2.536306in}}%
\pgfpathlineto{\pgfqpoint{4.125267in}{2.589214in}}%
\pgfpathlineto{\pgfqpoint{4.125662in}{2.567522in}}%
\pgfpathlineto{\pgfqpoint{4.126353in}{2.447407in}}%
\pgfpathlineto{\pgfqpoint{4.126847in}{2.323149in}}%
\pgfpathlineto{\pgfqpoint{4.127538in}{2.400341in}}%
\pgfpathlineto{\pgfqpoint{4.127932in}{2.382705in}}%
\pgfpathlineto{\pgfqpoint{4.128426in}{2.414355in}}%
\pgfpathlineto{\pgfqpoint{4.130202in}{2.509403in}}%
\pgfpathlineto{\pgfqpoint{4.130597in}{2.477873in}}%
\pgfpathlineto{\pgfqpoint{4.131189in}{2.460183in}}%
\pgfpathlineto{\pgfqpoint{4.131485in}{2.476666in}}%
\pgfpathlineto{\pgfqpoint{4.133163in}{2.537133in}}%
\pgfpathlineto{\pgfqpoint{4.133657in}{2.555712in}}%
\pgfpathlineto{\pgfqpoint{4.133854in}{2.540049in}}%
\pgfpathlineto{\pgfqpoint{4.134644in}{2.515821in}}%
\pgfpathlineto{\pgfqpoint{4.134940in}{2.533561in}}%
\pgfpathlineto{\pgfqpoint{4.135236in}{2.553423in}}%
\pgfpathlineto{\pgfqpoint{4.136026in}{2.535708in}}%
\pgfpathlineto{\pgfqpoint{4.136618in}{2.565770in}}%
\pgfpathlineto{\pgfqpoint{4.137407in}{2.552434in}}%
\pgfpathlineto{\pgfqpoint{4.137605in}{2.544229in}}%
\pgfpathlineto{\pgfqpoint{4.138000in}{2.559184in}}%
\pgfpathlineto{\pgfqpoint{4.138493in}{2.552408in}}%
\pgfpathlineto{\pgfqpoint{4.138789in}{2.559406in}}%
\pgfpathlineto{\pgfqpoint{4.139085in}{2.544348in}}%
\pgfpathlineto{\pgfqpoint{4.139974in}{2.531913in}}%
\pgfpathlineto{\pgfqpoint{4.140270in}{2.536271in}}%
\pgfpathlineto{\pgfqpoint{4.140664in}{2.514723in}}%
\pgfpathlineto{\pgfqpoint{4.142046in}{2.492678in}}%
\pgfpathlineto{\pgfqpoint{4.145797in}{2.384266in}}%
\pgfpathlineto{\pgfqpoint{4.146290in}{2.394877in}}%
\pgfpathlineto{\pgfqpoint{4.146883in}{2.404013in}}%
\pgfpathlineto{\pgfqpoint{4.147179in}{2.393442in}}%
\pgfpathlineto{\pgfqpoint{4.148560in}{2.372641in}}%
\pgfpathlineto{\pgfqpoint{4.150929in}{2.317130in}}%
\pgfpathlineto{\pgfqpoint{4.151028in}{2.320034in}}%
\pgfpathlineto{\pgfqpoint{4.151324in}{2.331675in}}%
\pgfpathlineto{\pgfqpoint{4.152015in}{2.321130in}}%
\pgfpathlineto{\pgfqpoint{4.152410in}{2.295548in}}%
\pgfpathlineto{\pgfqpoint{4.153199in}{2.312881in}}%
\pgfpathlineto{\pgfqpoint{4.154285in}{2.289859in}}%
\pgfpathlineto{\pgfqpoint{4.154581in}{2.304516in}}%
\pgfpathlineto{\pgfqpoint{4.154877in}{2.317841in}}%
\pgfpathlineto{\pgfqpoint{4.155469in}{2.297576in}}%
\pgfpathlineto{\pgfqpoint{4.155568in}{2.298113in}}%
\pgfpathlineto{\pgfqpoint{4.155963in}{2.296405in}}%
\pgfpathlineto{\pgfqpoint{4.156160in}{2.302381in}}%
\pgfpathlineto{\pgfqpoint{4.156456in}{2.315760in}}%
\pgfpathlineto{\pgfqpoint{4.156851in}{2.302276in}}%
\pgfpathlineto{\pgfqpoint{4.157345in}{2.307442in}}%
\pgfpathlineto{\pgfqpoint{4.157542in}{2.302145in}}%
\pgfpathlineto{\pgfqpoint{4.157937in}{2.319589in}}%
\pgfpathlineto{\pgfqpoint{4.160108in}{2.374943in}}%
\pgfpathlineto{\pgfqpoint{4.160207in}{2.373503in}}%
\pgfpathlineto{\pgfqpoint{4.160996in}{2.360936in}}%
\pgfpathlineto{\pgfqpoint{4.161293in}{2.373151in}}%
\pgfpathlineto{\pgfqpoint{4.161391in}{2.375531in}}%
\pgfpathlineto{\pgfqpoint{4.161687in}{2.361625in}}%
\pgfpathlineto{\pgfqpoint{4.162576in}{2.329358in}}%
\pgfpathlineto{\pgfqpoint{4.163069in}{2.335911in}}%
\pgfpathlineto{\pgfqpoint{4.163267in}{2.336603in}}%
\pgfpathlineto{\pgfqpoint{4.163464in}{2.335087in}}%
\pgfpathlineto{\pgfqpoint{4.165241in}{2.314727in}}%
\pgfpathlineto{\pgfqpoint{4.166030in}{2.328074in}}%
\pgfpathlineto{\pgfqpoint{4.166918in}{2.370016in}}%
\pgfpathlineto{\pgfqpoint{4.167609in}{2.354685in}}%
\pgfpathlineto{\pgfqpoint{4.168695in}{2.334729in}}%
\pgfpathlineto{\pgfqpoint{4.168202in}{2.355737in}}%
\pgfpathlineto{\pgfqpoint{4.168991in}{2.339718in}}%
\pgfpathlineto{\pgfqpoint{4.170274in}{2.377680in}}%
\pgfpathlineto{\pgfqpoint{4.171656in}{2.585986in}}%
\pgfpathlineto{\pgfqpoint{4.172742in}{2.557450in}}%
\pgfpathlineto{\pgfqpoint{4.173531in}{2.375805in}}%
\pgfpathlineto{\pgfqpoint{4.174321in}{2.452739in}}%
\pgfpathlineto{\pgfqpoint{4.174518in}{2.459876in}}%
\pgfpathlineto{\pgfqpoint{4.175209in}{2.441584in}}%
\pgfpathlineto{\pgfqpoint{4.175604in}{2.431150in}}%
\pgfpathlineto{\pgfqpoint{4.175900in}{2.446650in}}%
\pgfpathlineto{\pgfqpoint{4.176690in}{2.473506in}}%
\pgfpathlineto{\pgfqpoint{4.176887in}{2.457701in}}%
\pgfpathlineto{\pgfqpoint{4.178565in}{2.370005in}}%
\pgfpathlineto{\pgfqpoint{4.178861in}{2.360246in}}%
\pgfpathlineto{\pgfqpoint{4.179354in}{2.378170in}}%
\pgfpathlineto{\pgfqpoint{4.179552in}{2.381455in}}%
\pgfpathlineto{\pgfqpoint{4.179848in}{2.371849in}}%
\pgfpathlineto{\pgfqpoint{4.181723in}{2.299304in}}%
\pgfpathlineto{\pgfqpoint{4.182217in}{2.298129in}}%
\pgfpathlineto{\pgfqpoint{4.182414in}{2.303205in}}%
\pgfpathlineto{\pgfqpoint{4.182908in}{2.332282in}}%
\pgfpathlineto{\pgfqpoint{4.183500in}{2.308050in}}%
\pgfpathlineto{\pgfqpoint{4.183697in}{2.300768in}}%
\pgfpathlineto{\pgfqpoint{4.184191in}{2.322550in}}%
\pgfpathlineto{\pgfqpoint{4.184882in}{2.325209in}}%
\pgfpathlineto{\pgfqpoint{4.184586in}{2.321307in}}%
\pgfpathlineto{\pgfqpoint{4.184980in}{2.323099in}}%
\pgfpathlineto{\pgfqpoint{4.185375in}{2.306114in}}%
\pgfpathlineto{\pgfqpoint{4.185967in}{2.327015in}}%
\pgfpathlineto{\pgfqpoint{4.186263in}{2.334093in}}%
\pgfpathlineto{\pgfqpoint{4.186658in}{2.319092in}}%
\pgfpathlineto{\pgfqpoint{4.186856in}{2.315779in}}%
\pgfpathlineto{\pgfqpoint{4.187546in}{2.326087in}}%
\pgfpathlineto{\pgfqpoint{4.187941in}{2.344698in}}%
\pgfpathlineto{\pgfqpoint{4.188830in}{2.336426in}}%
\pgfpathlineto{\pgfqpoint{4.188928in}{2.335478in}}%
\pgfpathlineto{\pgfqpoint{4.189126in}{2.339541in}}%
\pgfpathlineto{\pgfqpoint{4.189520in}{2.354718in}}%
\pgfpathlineto{\pgfqpoint{4.190310in}{2.345561in}}%
\pgfpathlineto{\pgfqpoint{4.190606in}{2.351911in}}%
\pgfpathlineto{\pgfqpoint{4.190902in}{2.364478in}}%
\pgfpathlineto{\pgfqpoint{4.191297in}{2.350903in}}%
\pgfpathlineto{\pgfqpoint{4.191791in}{2.356061in}}%
\pgfpathlineto{\pgfqpoint{4.191988in}{2.351693in}}%
\pgfpathlineto{\pgfqpoint{4.192481in}{2.366625in}}%
\pgfpathlineto{\pgfqpoint{4.192580in}{2.367683in}}%
\pgfpathlineto{\pgfqpoint{4.192876in}{2.360524in}}%
\pgfpathlineto{\pgfqpoint{4.193271in}{2.344208in}}%
\pgfpathlineto{\pgfqpoint{4.194061in}{2.347264in}}%
\pgfpathlineto{\pgfqpoint{4.194357in}{2.357790in}}%
\pgfpathlineto{\pgfqpoint{4.194752in}{2.333091in}}%
\pgfpathlineto{\pgfqpoint{4.195048in}{2.318872in}}%
\pgfpathlineto{\pgfqpoint{4.195936in}{2.325058in}}%
\pgfpathlineto{\pgfqpoint{4.198206in}{2.289327in}}%
\pgfpathlineto{\pgfqpoint{4.198502in}{2.295351in}}%
\pgfpathlineto{\pgfqpoint{4.199390in}{2.313308in}}%
\pgfpathlineto{\pgfqpoint{4.199785in}{2.305634in}}%
\pgfpathlineto{\pgfqpoint{4.199983in}{2.302673in}}%
\pgfpathlineto{\pgfqpoint{4.200476in}{2.313121in}}%
\pgfpathlineto{\pgfqpoint{4.200871in}{2.326819in}}%
\pgfpathlineto{\pgfqpoint{4.201266in}{2.305715in}}%
\pgfpathlineto{\pgfqpoint{4.201463in}{2.301164in}}%
\pgfpathlineto{\pgfqpoint{4.201957in}{2.314615in}}%
\pgfpathlineto{\pgfqpoint{4.203042in}{2.331907in}}%
\pgfpathlineto{\pgfqpoint{4.203240in}{2.328994in}}%
\pgfpathlineto{\pgfqpoint{4.203338in}{2.327560in}}%
\pgfpathlineto{\pgfqpoint{4.203634in}{2.337536in}}%
\pgfpathlineto{\pgfqpoint{4.204128in}{2.362384in}}%
\pgfpathlineto{\pgfqpoint{4.204918in}{2.355295in}}%
\pgfpathlineto{\pgfqpoint{4.205016in}{2.354891in}}%
\pgfpathlineto{\pgfqpoint{4.205115in}{2.356325in}}%
\pgfpathlineto{\pgfqpoint{4.207385in}{2.418415in}}%
\pgfpathlineto{\pgfqpoint{4.207484in}{2.416915in}}%
\pgfpathlineto{\pgfqpoint{4.209852in}{2.351021in}}%
\pgfpathlineto{\pgfqpoint{4.210149in}{2.363973in}}%
\pgfpathlineto{\pgfqpoint{4.210543in}{2.396667in}}%
\pgfpathlineto{\pgfqpoint{4.211136in}{2.354743in}}%
\pgfpathlineto{\pgfqpoint{4.211530in}{2.345008in}}%
\pgfpathlineto{\pgfqpoint{4.212123in}{2.357005in}}%
\pgfpathlineto{\pgfqpoint{4.213899in}{2.403667in}}%
\pgfpathlineto{\pgfqpoint{4.212813in}{2.350537in}}%
\pgfpathlineto{\pgfqpoint{4.214097in}{2.396220in}}%
\pgfpathlineto{\pgfqpoint{4.214787in}{2.344113in}}%
\pgfpathlineto{\pgfqpoint{4.215380in}{2.371547in}}%
\pgfpathlineto{\pgfqpoint{4.215478in}{2.371484in}}%
\pgfpathlineto{\pgfqpoint{4.216268in}{2.363514in}}%
\pgfpathlineto{\pgfqpoint{4.216465in}{2.367487in}}%
\pgfpathlineto{\pgfqpoint{4.217354in}{2.493188in}}%
\pgfpathlineto{\pgfqpoint{4.218341in}{2.615108in}}%
\pgfpathlineto{\pgfqpoint{4.218933in}{2.601956in}}%
\pgfpathlineto{\pgfqpoint{4.219328in}{2.554725in}}%
\pgfpathlineto{\pgfqpoint{4.220216in}{2.345117in}}%
\pgfpathlineto{\pgfqpoint{4.221005in}{2.380315in}}%
\pgfpathlineto{\pgfqpoint{4.221302in}{2.383417in}}%
\pgfpathlineto{\pgfqpoint{4.221499in}{2.379263in}}%
\pgfpathlineto{\pgfqpoint{4.221598in}{2.377891in}}%
\pgfpathlineto{\pgfqpoint{4.221795in}{2.386880in}}%
\pgfpathlineto{\pgfqpoint{4.223473in}{2.452644in}}%
\pgfpathlineto{\pgfqpoint{4.223572in}{2.451936in}}%
\pgfpathlineto{\pgfqpoint{4.223966in}{2.433175in}}%
\pgfpathlineto{\pgfqpoint{4.224361in}{2.402960in}}%
\pgfpathlineto{\pgfqpoint{4.225052in}{2.422414in}}%
\pgfpathlineto{\pgfqpoint{4.226335in}{2.432364in}}%
\pgfpathlineto{\pgfqpoint{4.226829in}{2.454298in}}%
\pgfpathlineto{\pgfqpoint{4.227322in}{2.430701in}}%
\pgfpathlineto{\pgfqpoint{4.227914in}{2.414186in}}%
\pgfpathlineto{\pgfqpoint{4.228210in}{2.431656in}}%
\pgfpathlineto{\pgfqpoint{4.228507in}{2.447022in}}%
\pgfpathlineto{\pgfqpoint{4.229395in}{2.441881in}}%
\pgfpathlineto{\pgfqpoint{4.229494in}{2.442049in}}%
\pgfpathlineto{\pgfqpoint{4.229592in}{2.440663in}}%
\pgfpathlineto{\pgfqpoint{4.229888in}{2.434710in}}%
\pgfpathlineto{\pgfqpoint{4.230184in}{2.450363in}}%
\pgfpathlineto{\pgfqpoint{4.230283in}{2.454449in}}%
\pgfpathlineto{\pgfqpoint{4.230678in}{2.432406in}}%
\pgfpathlineto{\pgfqpoint{4.230875in}{2.424380in}}%
\pgfpathlineto{\pgfqpoint{4.231468in}{2.443324in}}%
\pgfpathlineto{\pgfqpoint{4.231764in}{2.453195in}}%
\pgfpathlineto{\pgfqpoint{4.232553in}{2.448196in}}%
\pgfpathlineto{\pgfqpoint{4.232849in}{2.442332in}}%
\pgfpathlineto{\pgfqpoint{4.233244in}{2.453166in}}%
\pgfpathlineto{\pgfqpoint{4.233639in}{2.461243in}}%
\pgfpathlineto{\pgfqpoint{4.234034in}{2.447272in}}%
\pgfpathlineto{\pgfqpoint{4.234231in}{2.437453in}}%
\pgfpathlineto{\pgfqpoint{4.234922in}{2.453067in}}%
\pgfpathlineto{\pgfqpoint{4.236402in}{2.468388in}}%
\pgfpathlineto{\pgfqpoint{4.236501in}{2.467110in}}%
\pgfpathlineto{\pgfqpoint{4.236699in}{2.465150in}}%
\pgfpathlineto{\pgfqpoint{4.236896in}{2.474758in}}%
\pgfpathlineto{\pgfqpoint{4.237093in}{2.484075in}}%
\pgfpathlineto{\pgfqpoint{4.237488in}{2.464817in}}%
\pgfpathlineto{\pgfqpoint{4.237982in}{2.477065in}}%
\pgfpathlineto{\pgfqpoint{4.238179in}{2.478568in}}%
\pgfpathlineto{\pgfqpoint{4.238475in}{2.490171in}}%
\pgfpathlineto{\pgfqpoint{4.239067in}{2.471906in}}%
\pgfpathlineto{\pgfqpoint{4.239758in}{2.487109in}}%
\pgfpathlineto{\pgfqpoint{4.240252in}{2.504227in}}%
\pgfpathlineto{\pgfqpoint{4.240548in}{2.484114in}}%
\pgfpathlineto{\pgfqpoint{4.240844in}{2.468658in}}%
\pgfpathlineto{\pgfqpoint{4.241732in}{2.473301in}}%
\pgfpathlineto{\pgfqpoint{4.241930in}{2.471718in}}%
\pgfpathlineto{\pgfqpoint{4.243015in}{2.453870in}}%
\pgfpathlineto{\pgfqpoint{4.243213in}{2.460487in}}%
\pgfpathlineto{\pgfqpoint{4.243410in}{2.468615in}}%
\pgfpathlineto{\pgfqpoint{4.243805in}{2.443278in}}%
\pgfpathlineto{\pgfqpoint{4.244101in}{2.427526in}}%
\pgfpathlineto{\pgfqpoint{4.244693in}{2.444005in}}%
\pgfpathlineto{\pgfqpoint{4.244891in}{2.440944in}}%
\pgfpathlineto{\pgfqpoint{4.245878in}{2.417985in}}%
\pgfpathlineto{\pgfqpoint{4.245285in}{2.441615in}}%
\pgfpathlineto{\pgfqpoint{4.246865in}{2.428281in}}%
\pgfpathlineto{\pgfqpoint{4.247852in}{2.439913in}}%
\pgfpathlineto{\pgfqpoint{4.247457in}{2.420801in}}%
\pgfpathlineto{\pgfqpoint{4.248049in}{2.437643in}}%
\pgfpathlineto{\pgfqpoint{4.248937in}{2.413201in}}%
\pgfpathlineto{\pgfqpoint{4.249431in}{2.427524in}}%
\pgfpathlineto{\pgfqpoint{4.249924in}{2.446874in}}%
\pgfpathlineto{\pgfqpoint{4.250615in}{2.431221in}}%
\pgfpathlineto{\pgfqpoint{4.250911in}{2.426333in}}%
\pgfpathlineto{\pgfqpoint{4.251207in}{2.438637in}}%
\pgfpathlineto{\pgfqpoint{4.251503in}{2.446947in}}%
\pgfpathlineto{\pgfqpoint{4.252096in}{2.429351in}}%
\pgfpathlineto{\pgfqpoint{4.252589in}{2.409880in}}%
\pgfpathlineto{\pgfqpoint{4.253083in}{2.430094in}}%
\pgfpathlineto{\pgfqpoint{4.253181in}{2.429199in}}%
\pgfpathlineto{\pgfqpoint{4.253872in}{2.411158in}}%
\pgfpathlineto{\pgfqpoint{4.254366in}{2.425622in}}%
\pgfpathlineto{\pgfqpoint{4.256340in}{2.462742in}}%
\pgfpathlineto{\pgfqpoint{4.256537in}{2.458773in}}%
\pgfpathlineto{\pgfqpoint{4.258906in}{2.402340in}}%
\pgfpathlineto{\pgfqpoint{4.259202in}{2.407565in}}%
\pgfpathlineto{\pgfqpoint{4.260288in}{2.454668in}}%
\pgfpathlineto{\pgfqpoint{4.260979in}{2.442893in}}%
\pgfpathlineto{\pgfqpoint{4.262064in}{2.405245in}}%
\pgfpathlineto{\pgfqpoint{4.262558in}{2.407006in}}%
\pgfpathlineto{\pgfqpoint{4.263841in}{2.486296in}}%
\pgfpathlineto{\pgfqpoint{4.265025in}{2.678053in}}%
\pgfpathlineto{\pgfqpoint{4.265913in}{2.650485in}}%
\pgfpathlineto{\pgfqpoint{4.266999in}{2.394803in}}%
\pgfpathlineto{\pgfqpoint{4.268578in}{2.448843in}}%
\pgfpathlineto{\pgfqpoint{4.269368in}{2.485856in}}%
\pgfpathlineto{\pgfqpoint{4.269763in}{2.522282in}}%
\pgfpathlineto{\pgfqpoint{4.270454in}{2.499012in}}%
\pgfpathlineto{\pgfqpoint{4.271934in}{2.460891in}}%
\pgfpathlineto{\pgfqpoint{4.272033in}{2.461084in}}%
\pgfpathlineto{\pgfqpoint{4.272724in}{2.497006in}}%
\pgfpathlineto{\pgfqpoint{4.273020in}{2.514921in}}%
\pgfpathlineto{\pgfqpoint{4.273612in}{2.491171in}}%
\pgfpathlineto{\pgfqpoint{4.273908in}{2.479821in}}%
\pgfpathlineto{\pgfqpoint{4.274698in}{2.490712in}}%
\pgfpathlineto{\pgfqpoint{4.274994in}{2.481765in}}%
\pgfpathlineto{\pgfqpoint{4.275191in}{2.473144in}}%
\pgfpathlineto{\pgfqpoint{4.275981in}{2.488442in}}%
\pgfpathlineto{\pgfqpoint{4.276277in}{2.508482in}}%
\pgfpathlineto{\pgfqpoint{4.276968in}{2.483043in}}%
\pgfpathlineto{\pgfqpoint{4.277066in}{2.481624in}}%
\pgfpathlineto{\pgfqpoint{4.277363in}{2.490016in}}%
\pgfpathlineto{\pgfqpoint{4.277659in}{2.499298in}}%
\pgfpathlineto{\pgfqpoint{4.278350in}{2.486743in}}%
\pgfpathlineto{\pgfqpoint{4.278547in}{2.481675in}}%
\pgfpathlineto{\pgfqpoint{4.278942in}{2.499749in}}%
\pgfpathlineto{\pgfqpoint{4.279435in}{2.510033in}}%
\pgfpathlineto{\pgfqpoint{4.279929in}{2.495886in}}%
\pgfpathlineto{\pgfqpoint{4.280422in}{2.496087in}}%
\pgfpathlineto{\pgfqpoint{4.280225in}{2.496826in}}%
\pgfpathlineto{\pgfqpoint{4.280521in}{2.496685in}}%
\pgfpathlineto{\pgfqpoint{4.281311in}{2.511306in}}%
\pgfpathlineto{\pgfqpoint{4.281607in}{2.501266in}}%
\pgfpathlineto{\pgfqpoint{4.281804in}{2.494234in}}%
\pgfpathlineto{\pgfqpoint{4.282495in}{2.510032in}}%
\pgfpathlineto{\pgfqpoint{4.282890in}{2.521605in}}%
\pgfpathlineto{\pgfqpoint{4.283482in}{2.507176in}}%
\pgfpathlineto{\pgfqpoint{4.283679in}{2.504297in}}%
\pgfpathlineto{\pgfqpoint{4.283975in}{2.514131in}}%
\pgfpathlineto{\pgfqpoint{4.284271in}{2.525697in}}%
\pgfpathlineto{\pgfqpoint{4.284962in}{2.510902in}}%
\pgfpathlineto{\pgfqpoint{4.285258in}{2.504499in}}%
\pgfpathlineto{\pgfqpoint{4.285752in}{2.519171in}}%
\pgfpathlineto{\pgfqpoint{4.286147in}{2.525022in}}%
\pgfpathlineto{\pgfqpoint{4.286542in}{2.513229in}}%
\pgfpathlineto{\pgfqpoint{4.286936in}{2.505477in}}%
\pgfpathlineto{\pgfqpoint{4.287430in}{2.517634in}}%
\pgfpathlineto{\pgfqpoint{4.287529in}{2.519316in}}%
\pgfpathlineto{\pgfqpoint{4.287825in}{2.511312in}}%
\pgfpathlineto{\pgfqpoint{4.288318in}{2.487606in}}%
\pgfpathlineto{\pgfqpoint{4.289009in}{2.505998in}}%
\pgfpathlineto{\pgfqpoint{4.289601in}{2.498172in}}%
\pgfpathlineto{\pgfqpoint{4.291674in}{2.433927in}}%
\pgfpathlineto{\pgfqpoint{4.291871in}{2.443053in}}%
\pgfpathlineto{\pgfqpoint{4.292266in}{2.469108in}}%
\pgfpathlineto{\pgfqpoint{4.292957in}{2.442184in}}%
\pgfpathlineto{\pgfqpoint{4.293154in}{2.436704in}}%
\pgfpathlineto{\pgfqpoint{4.293648in}{2.449016in}}%
\pgfpathlineto{\pgfqpoint{4.293944in}{2.446363in}}%
\pgfpathlineto{\pgfqpoint{4.294339in}{2.457818in}}%
\pgfpathlineto{\pgfqpoint{4.294635in}{2.439577in}}%
\pgfpathlineto{\pgfqpoint{4.294832in}{2.432995in}}%
\pgfpathlineto{\pgfqpoint{4.295523in}{2.445745in}}%
\pgfpathlineto{\pgfqpoint{4.295819in}{2.453519in}}%
\pgfpathlineto{\pgfqpoint{4.296214in}{2.433407in}}%
\pgfpathlineto{\pgfqpoint{4.296411in}{2.429851in}}%
\pgfpathlineto{\pgfqpoint{4.296905in}{2.443591in}}%
\pgfpathlineto{\pgfqpoint{4.297398in}{2.452086in}}%
\pgfpathlineto{\pgfqpoint{4.297793in}{2.439707in}}%
\pgfpathlineto{\pgfqpoint{4.297991in}{2.434035in}}%
\pgfpathlineto{\pgfqpoint{4.298682in}{2.443540in}}%
\pgfpathlineto{\pgfqpoint{4.299471in}{2.458945in}}%
\pgfpathlineto{\pgfqpoint{4.299965in}{2.450199in}}%
\pgfpathlineto{\pgfqpoint{4.302432in}{2.493621in}}%
\pgfpathlineto{\pgfqpoint{4.304406in}{2.417540in}}%
\pgfpathlineto{\pgfqpoint{4.305097in}{2.430816in}}%
\pgfpathlineto{\pgfqpoint{4.305393in}{2.444033in}}%
\pgfpathlineto{\pgfqpoint{4.305887in}{2.417929in}}%
\pgfpathlineto{\pgfqpoint{4.306577in}{2.390185in}}%
\pgfpathlineto{\pgfqpoint{4.307071in}{2.411135in}}%
\pgfpathlineto{\pgfqpoint{4.308551in}{2.462829in}}%
\pgfpathlineto{\pgfqpoint{4.308749in}{2.453388in}}%
\pgfpathlineto{\pgfqpoint{4.309637in}{2.404537in}}%
\pgfpathlineto{\pgfqpoint{4.310229in}{2.417648in}}%
\pgfpathlineto{\pgfqpoint{4.312006in}{2.538555in}}%
\pgfpathlineto{\pgfqpoint{4.313092in}{2.666702in}}%
\pgfpathlineto{\pgfqpoint{4.313486in}{2.634429in}}%
\pgfpathlineto{\pgfqpoint{4.313881in}{2.644328in}}%
\pgfpathlineto{\pgfqpoint{4.314079in}{2.612571in}}%
\pgfpathlineto{\pgfqpoint{4.314868in}{2.388128in}}%
\pgfpathlineto{\pgfqpoint{4.315658in}{2.449163in}}%
\pgfpathlineto{\pgfqpoint{4.316250in}{2.421012in}}%
\pgfpathlineto{\pgfqpoint{4.316546in}{2.448356in}}%
\pgfpathlineto{\pgfqpoint{4.318125in}{2.494264in}}%
\pgfpathlineto{\pgfqpoint{4.318421in}{2.491693in}}%
\pgfpathlineto{\pgfqpoint{4.319112in}{2.445201in}}%
\pgfpathlineto{\pgfqpoint{4.320099in}{2.471247in}}%
\pgfpathlineto{\pgfqpoint{4.320297in}{2.470533in}}%
\pgfpathlineto{\pgfqpoint{4.320395in}{2.471629in}}%
\pgfpathlineto{\pgfqpoint{4.321185in}{2.501514in}}%
\pgfpathlineto{\pgfqpoint{4.322172in}{2.493181in}}%
\pgfpathlineto{\pgfqpoint{4.322468in}{2.485612in}}%
\pgfpathlineto{\pgfqpoint{4.322764in}{2.475300in}}%
\pgfpathlineto{\pgfqpoint{4.323258in}{2.496445in}}%
\pgfpathlineto{\pgfqpoint{4.323850in}{2.507373in}}%
\pgfpathlineto{\pgfqpoint{4.324245in}{2.496854in}}%
\pgfpathlineto{\pgfqpoint{4.324541in}{2.490752in}}%
\pgfpathlineto{\pgfqpoint{4.324935in}{2.506663in}}%
\pgfpathlineto{\pgfqpoint{4.325133in}{2.509415in}}%
\pgfpathlineto{\pgfqpoint{4.325626in}{2.497584in}}%
\pgfpathlineto{\pgfqpoint{4.325824in}{2.494850in}}%
\pgfpathlineto{\pgfqpoint{4.326416in}{2.504027in}}%
\pgfpathlineto{\pgfqpoint{4.326909in}{2.531031in}}%
\pgfpathlineto{\pgfqpoint{4.327600in}{2.511122in}}%
\pgfpathlineto{\pgfqpoint{4.327896in}{2.506598in}}%
\pgfpathlineto{\pgfqpoint{4.328291in}{2.520589in}}%
\pgfpathlineto{\pgfqpoint{4.328587in}{2.530589in}}%
\pgfpathlineto{\pgfqpoint{4.329081in}{2.505072in}}%
\pgfpathlineto{\pgfqpoint{4.330068in}{2.520608in}}%
\pgfpathlineto{\pgfqpoint{4.330265in}{2.524086in}}%
\pgfpathlineto{\pgfqpoint{4.330759in}{2.509607in}}%
\pgfpathlineto{\pgfqpoint{4.331252in}{2.505496in}}%
\pgfpathlineto{\pgfqpoint{4.331548in}{2.518850in}}%
\pgfpathlineto{\pgfqpoint{4.331746in}{2.527266in}}%
\pgfpathlineto{\pgfqpoint{4.332437in}{2.514207in}}%
\pgfpathlineto{\pgfqpoint{4.334411in}{2.455916in}}%
\pgfpathlineto{\pgfqpoint{4.334509in}{2.457418in}}%
\pgfpathlineto{\pgfqpoint{4.335101in}{2.470881in}}%
\pgfpathlineto{\pgfqpoint{4.335299in}{2.459341in}}%
\pgfpathlineto{\pgfqpoint{4.335595in}{2.430330in}}%
\pgfpathlineto{\pgfqpoint{4.336187in}{2.480199in}}%
\pgfpathlineto{\pgfqpoint{4.336385in}{2.477881in}}%
\pgfpathlineto{\pgfqpoint{4.336582in}{2.486177in}}%
\pgfpathlineto{\pgfqpoint{4.336878in}{2.508533in}}%
\pgfpathlineto{\pgfqpoint{4.337766in}{2.494426in}}%
\pgfpathlineto{\pgfqpoint{4.338358in}{2.505484in}}%
\pgfpathlineto{\pgfqpoint{4.338655in}{2.494654in}}%
\pgfpathlineto{\pgfqpoint{4.339148in}{2.473894in}}%
\pgfpathlineto{\pgfqpoint{4.340036in}{2.474147in}}%
\pgfpathlineto{\pgfqpoint{4.340826in}{2.448932in}}%
\pgfpathlineto{\pgfqpoint{4.340925in}{2.446667in}}%
\pgfpathlineto{\pgfqpoint{4.341221in}{2.458349in}}%
\pgfpathlineto{\pgfqpoint{4.341517in}{2.471300in}}%
\pgfpathlineto{\pgfqpoint{4.342208in}{2.458832in}}%
\pgfpathlineto{\pgfqpoint{4.342504in}{2.439779in}}%
\pgfpathlineto{\pgfqpoint{4.343392in}{2.450976in}}%
\pgfpathlineto{\pgfqpoint{4.343491in}{2.451176in}}%
\pgfpathlineto{\pgfqpoint{4.343590in}{2.449187in}}%
\pgfpathlineto{\pgfqpoint{4.343984in}{2.429915in}}%
\pgfpathlineto{\pgfqpoint{4.344577in}{2.446909in}}%
\pgfpathlineto{\pgfqpoint{4.345070in}{2.464661in}}%
\pgfpathlineto{\pgfqpoint{4.345465in}{2.440077in}}%
\pgfpathlineto{\pgfqpoint{4.345662in}{2.430404in}}%
\pgfpathlineto{\pgfqpoint{4.346254in}{2.456051in}}%
\pgfpathlineto{\pgfqpoint{4.346551in}{2.462430in}}%
\pgfpathlineto{\pgfqpoint{4.347044in}{2.449279in}}%
\pgfpathlineto{\pgfqpoint{4.347241in}{2.443960in}}%
\pgfpathlineto{\pgfqpoint{4.347636in}{2.461059in}}%
\pgfpathlineto{\pgfqpoint{4.348623in}{2.484672in}}%
\pgfpathlineto{\pgfqpoint{4.348919in}{2.472459in}}%
\pgfpathlineto{\pgfqpoint{4.349117in}{2.468750in}}%
\pgfpathlineto{\pgfqpoint{4.349413in}{2.488228in}}%
\pgfpathlineto{\pgfqpoint{4.349709in}{2.501545in}}%
\pgfpathlineto{\pgfqpoint{4.350301in}{2.481944in}}%
\pgfpathlineto{\pgfqpoint{4.351979in}{2.428594in}}%
\pgfpathlineto{\pgfqpoint{4.352275in}{2.431588in}}%
\pgfpathlineto{\pgfqpoint{4.352769in}{2.448070in}}%
\pgfpathlineto{\pgfqpoint{4.352966in}{2.456163in}}%
\pgfpathlineto{\pgfqpoint{4.353361in}{2.432198in}}%
\pgfpathlineto{\pgfqpoint{4.354150in}{2.409192in}}%
\pgfpathlineto{\pgfqpoint{4.354545in}{2.422914in}}%
\pgfpathlineto{\pgfqpoint{4.356223in}{2.466961in}}%
\pgfpathlineto{\pgfqpoint{4.356420in}{2.462973in}}%
\pgfpathlineto{\pgfqpoint{4.357111in}{2.394883in}}%
\pgfpathlineto{\pgfqpoint{4.358296in}{2.405237in}}%
\pgfpathlineto{\pgfqpoint{4.358888in}{2.396760in}}%
\pgfpathlineto{\pgfqpoint{4.359283in}{2.445546in}}%
\pgfpathlineto{\pgfqpoint{4.360862in}{2.663156in}}%
\pgfpathlineto{\pgfqpoint{4.361355in}{2.638089in}}%
\pgfpathlineto{\pgfqpoint{4.361750in}{2.584650in}}%
\pgfpathlineto{\pgfqpoint{4.362540in}{2.373295in}}%
\pgfpathlineto{\pgfqpoint{4.363329in}{2.428359in}}%
\pgfpathlineto{\pgfqpoint{4.363724in}{2.425934in}}%
\pgfpathlineto{\pgfqpoint{4.364020in}{2.432077in}}%
\pgfpathlineto{\pgfqpoint{4.365895in}{2.494864in}}%
\pgfpathlineto{\pgfqpoint{4.366192in}{2.489167in}}%
\pgfpathlineto{\pgfqpoint{4.366784in}{2.440925in}}%
\pgfpathlineto{\pgfqpoint{4.367573in}{2.466189in}}%
\pgfpathlineto{\pgfqpoint{4.369153in}{2.496738in}}%
\pgfpathlineto{\pgfqpoint{4.369745in}{2.481362in}}%
\pgfpathlineto{\pgfqpoint{4.370337in}{2.457275in}}%
\pgfpathlineto{\pgfqpoint{4.370929in}{2.474936in}}%
\pgfpathlineto{\pgfqpoint{4.372706in}{2.519496in}}%
\pgfpathlineto{\pgfqpoint{4.373397in}{2.495406in}}%
\pgfpathlineto{\pgfqpoint{4.373890in}{2.512973in}}%
\pgfpathlineto{\pgfqpoint{4.374285in}{2.534209in}}%
\pgfpathlineto{\pgfqpoint{4.374976in}{2.514341in}}%
\pgfpathlineto{\pgfqpoint{4.375173in}{2.513359in}}%
\pgfpathlineto{\pgfqpoint{4.375371in}{2.517607in}}%
\pgfpathlineto{\pgfqpoint{4.376160in}{2.540664in}}%
\pgfpathlineto{\pgfqpoint{4.376555in}{2.525561in}}%
\pgfpathlineto{\pgfqpoint{4.376752in}{2.521685in}}%
\pgfpathlineto{\pgfqpoint{4.377443in}{2.529364in}}%
\pgfpathlineto{\pgfqpoint{4.377739in}{2.539681in}}%
\pgfpathlineto{\pgfqpoint{4.378233in}{2.521515in}}%
\pgfpathlineto{\pgfqpoint{4.378430in}{2.516826in}}%
\pgfpathlineto{\pgfqpoint{4.378924in}{2.532202in}}%
\pgfpathlineto{\pgfqpoint{4.379615in}{2.535634in}}%
\pgfpathlineto{\pgfqpoint{4.379812in}{2.530210in}}%
\pgfpathlineto{\pgfqpoint{4.380108in}{2.518104in}}%
\pgfpathlineto{\pgfqpoint{4.380700in}{2.532687in}}%
\pgfpathlineto{\pgfqpoint{4.380799in}{2.532212in}}%
\pgfpathlineto{\pgfqpoint{4.381095in}{2.535342in}}%
\pgfpathlineto{\pgfqpoint{4.381293in}{2.529845in}}%
\pgfpathlineto{\pgfqpoint{4.381589in}{2.515645in}}%
\pgfpathlineto{\pgfqpoint{4.382280in}{2.534548in}}%
\pgfpathlineto{\pgfqpoint{4.382674in}{2.541988in}}%
\pgfpathlineto{\pgfqpoint{4.382970in}{2.532880in}}%
\pgfpathlineto{\pgfqpoint{4.383365in}{2.508234in}}%
\pgfpathlineto{\pgfqpoint{4.384155in}{2.523315in}}%
\pgfpathlineto{\pgfqpoint{4.386622in}{2.465378in}}%
\pgfpathlineto{\pgfqpoint{4.386918in}{2.475556in}}%
\pgfpathlineto{\pgfqpoint{4.387214in}{2.480622in}}%
\pgfpathlineto{\pgfqpoint{4.387905in}{2.473813in}}%
\pgfpathlineto{\pgfqpoint{4.388300in}{2.454436in}}%
\pgfpathlineto{\pgfqpoint{4.388794in}{2.477982in}}%
\pgfpathlineto{\pgfqpoint{4.389287in}{2.460499in}}%
\pgfpathlineto{\pgfqpoint{4.389485in}{2.464944in}}%
\pgfpathlineto{\pgfqpoint{4.389879in}{2.447827in}}%
\pgfpathlineto{\pgfqpoint{4.389978in}{2.446433in}}%
\pgfpathlineto{\pgfqpoint{4.390373in}{2.455439in}}%
\pgfpathlineto{\pgfqpoint{4.390570in}{2.457418in}}%
\pgfpathlineto{\pgfqpoint{4.390965in}{2.474321in}}%
\pgfpathlineto{\pgfqpoint{4.391360in}{2.445272in}}%
\pgfpathlineto{\pgfqpoint{4.391952in}{2.439667in}}%
\pgfpathlineto{\pgfqpoint{4.392149in}{2.447599in}}%
\pgfpathlineto{\pgfqpoint{4.392446in}{2.464754in}}%
\pgfpathlineto{\pgfqpoint{4.393038in}{2.436819in}}%
\pgfpathlineto{\pgfqpoint{4.393136in}{2.439319in}}%
\pgfpathlineto{\pgfqpoint{4.394123in}{2.452387in}}%
\pgfpathlineto{\pgfqpoint{4.394419in}{2.446308in}}%
\pgfpathlineto{\pgfqpoint{4.394814in}{2.432530in}}%
\pgfpathlineto{\pgfqpoint{4.395209in}{2.453570in}}%
\pgfpathlineto{\pgfqpoint{4.395999in}{2.469941in}}%
\pgfpathlineto{\pgfqpoint{4.396295in}{2.457725in}}%
\pgfpathlineto{\pgfqpoint{4.396492in}{2.452011in}}%
\pgfpathlineto{\pgfqpoint{4.396986in}{2.478341in}}%
\pgfpathlineto{\pgfqpoint{4.397479in}{2.508973in}}%
\pgfpathlineto{\pgfqpoint{4.397874in}{2.473671in}}%
\pgfpathlineto{\pgfqpoint{4.399848in}{2.423584in}}%
\pgfpathlineto{\pgfqpoint{4.398367in}{2.483590in}}%
\pgfpathlineto{\pgfqpoint{4.400144in}{2.434703in}}%
\pgfpathlineto{\pgfqpoint{4.400539in}{2.464903in}}%
\pgfpathlineto{\pgfqpoint{4.401230in}{2.438405in}}%
\pgfpathlineto{\pgfqpoint{4.401723in}{2.412382in}}%
\pgfpathlineto{\pgfqpoint{4.402315in}{2.434749in}}%
\pgfpathlineto{\pgfqpoint{4.402414in}{2.435828in}}%
\pgfpathlineto{\pgfqpoint{4.402612in}{2.427125in}}%
\pgfpathlineto{\pgfqpoint{4.402908in}{2.406284in}}%
\pgfpathlineto{\pgfqpoint{4.403500in}{2.442109in}}%
\pgfpathlineto{\pgfqpoint{4.404092in}{2.473726in}}%
\pgfpathlineto{\pgfqpoint{4.404487in}{2.445239in}}%
\pgfpathlineto{\pgfqpoint{4.405079in}{2.420697in}}%
\pgfpathlineto{\pgfqpoint{4.405671in}{2.434546in}}%
\pgfpathlineto{\pgfqpoint{4.406066in}{2.418194in}}%
\pgfpathlineto{\pgfqpoint{4.406658in}{2.433185in}}%
\pgfpathlineto{\pgfqpoint{4.407546in}{2.520756in}}%
\pgfpathlineto{\pgfqpoint{4.408830in}{2.694403in}}%
\pgfpathlineto{\pgfqpoint{4.409224in}{2.656896in}}%
\pgfpathlineto{\pgfqpoint{4.410606in}{2.404452in}}%
\pgfpathlineto{\pgfqpoint{4.411889in}{2.450053in}}%
\pgfpathlineto{\pgfqpoint{4.411988in}{2.450097in}}%
\pgfpathlineto{\pgfqpoint{4.412087in}{2.449527in}}%
\pgfpathlineto{\pgfqpoint{4.412481in}{2.441328in}}%
\pgfpathlineto{\pgfqpoint{4.412777in}{2.448561in}}%
\pgfpathlineto{\pgfqpoint{4.413666in}{2.516822in}}%
\pgfpathlineto{\pgfqpoint{4.414159in}{2.477721in}}%
\pgfpathlineto{\pgfqpoint{4.414751in}{2.440677in}}%
\pgfpathlineto{\pgfqpoint{4.415541in}{2.450454in}}%
\pgfpathlineto{\pgfqpoint{4.416133in}{2.426475in}}%
\pgfpathlineto{\pgfqpoint{4.416627in}{2.446774in}}%
\pgfpathlineto{\pgfqpoint{4.416824in}{2.449441in}}%
\pgfpathlineto{\pgfqpoint{4.417120in}{2.440803in}}%
\pgfpathlineto{\pgfqpoint{4.417910in}{2.391585in}}%
\pgfpathlineto{\pgfqpoint{4.418305in}{2.419300in}}%
\pgfpathlineto{\pgfqpoint{4.419983in}{2.513700in}}%
\pgfpathlineto{\pgfqpoint{4.421167in}{2.482431in}}%
\pgfpathlineto{\pgfqpoint{4.421759in}{2.497516in}}%
\pgfpathlineto{\pgfqpoint{4.421957in}{2.499519in}}%
\pgfpathlineto{\pgfqpoint{4.422253in}{2.493749in}}%
\pgfpathlineto{\pgfqpoint{4.422450in}{2.489585in}}%
\pgfpathlineto{\pgfqpoint{4.423042in}{2.500710in}}%
\pgfpathlineto{\pgfqpoint{4.423536in}{2.518098in}}%
\pgfpathlineto{\pgfqpoint{4.423930in}{2.496285in}}%
\pgfpathlineto{\pgfqpoint{4.424029in}{2.494335in}}%
\pgfpathlineto{\pgfqpoint{4.424523in}{2.502453in}}%
\pgfpathlineto{\pgfqpoint{4.425312in}{2.520788in}}%
\pgfpathlineto{\pgfqpoint{4.425608in}{2.511174in}}%
\pgfpathlineto{\pgfqpoint{4.426003in}{2.493747in}}%
\pgfpathlineto{\pgfqpoint{4.426595in}{2.514522in}}%
\pgfpathlineto{\pgfqpoint{4.426793in}{2.516723in}}%
\pgfpathlineto{\pgfqpoint{4.427188in}{2.506354in}}%
\pgfpathlineto{\pgfqpoint{4.427484in}{2.501647in}}%
\pgfpathlineto{\pgfqpoint{4.428076in}{2.510830in}}%
\pgfpathlineto{\pgfqpoint{4.428471in}{2.524145in}}%
\pgfpathlineto{\pgfqpoint{4.429063in}{2.509214in}}%
\pgfpathlineto{\pgfqpoint{4.429260in}{2.505814in}}%
\pgfpathlineto{\pgfqpoint{4.429655in}{2.518530in}}%
\pgfpathlineto{\pgfqpoint{4.430247in}{2.536759in}}%
\pgfpathlineto{\pgfqpoint{4.430445in}{2.524173in}}%
\pgfpathlineto{\pgfqpoint{4.430741in}{2.498931in}}%
\pgfpathlineto{\pgfqpoint{4.431530in}{2.517384in}}%
\pgfpathlineto{\pgfqpoint{4.431728in}{2.528248in}}%
\pgfpathlineto{\pgfqpoint{4.432221in}{2.492674in}}%
\pgfpathlineto{\pgfqpoint{4.432320in}{2.491114in}}%
\pgfpathlineto{\pgfqpoint{4.433011in}{2.497252in}}%
\pgfpathlineto{\pgfqpoint{4.433109in}{2.497388in}}%
\pgfpathlineto{\pgfqpoint{4.436169in}{2.445714in}}%
\pgfpathlineto{\pgfqpoint{4.436367in}{2.453812in}}%
\pgfpathlineto{\pgfqpoint{4.436564in}{2.463364in}}%
\pgfpathlineto{\pgfqpoint{4.437156in}{2.437401in}}%
\pgfpathlineto{\pgfqpoint{4.437354in}{2.435663in}}%
\pgfpathlineto{\pgfqpoint{4.437748in}{2.442620in}}%
\pgfpathlineto{\pgfqpoint{4.438341in}{2.457367in}}%
\pgfpathlineto{\pgfqpoint{4.438637in}{2.438993in}}%
\pgfpathlineto{\pgfqpoint{4.439426in}{2.427549in}}%
\pgfpathlineto{\pgfqpoint{4.439722in}{2.437630in}}%
\pgfpathlineto{\pgfqpoint{4.439920in}{2.444561in}}%
\pgfpathlineto{\pgfqpoint{4.440315in}{2.421476in}}%
\pgfpathlineto{\pgfqpoint{4.440709in}{2.407065in}}%
\pgfpathlineto{\pgfqpoint{4.441203in}{2.426133in}}%
\pgfpathlineto{\pgfqpoint{4.441400in}{2.430545in}}%
\pgfpathlineto{\pgfqpoint{4.441992in}{2.421877in}}%
\pgfpathlineto{\pgfqpoint{4.442486in}{2.411397in}}%
\pgfpathlineto{\pgfqpoint{4.442881in}{2.423142in}}%
\pgfpathlineto{\pgfqpoint{4.443275in}{2.442473in}}%
\pgfpathlineto{\pgfqpoint{4.444065in}{2.429270in}}%
\pgfpathlineto{\pgfqpoint{4.445052in}{2.454544in}}%
\pgfpathlineto{\pgfqpoint{4.445447in}{2.441291in}}%
\pgfpathlineto{\pgfqpoint{4.445940in}{2.444422in}}%
\pgfpathlineto{\pgfqpoint{4.449494in}{2.378762in}}%
\pgfpathlineto{\pgfqpoint{4.449691in}{2.379151in}}%
\pgfpathlineto{\pgfqpoint{4.449790in}{2.378848in}}%
\pgfpathlineto{\pgfqpoint{4.450678in}{2.360510in}}%
\pgfpathlineto{\pgfqpoint{4.451171in}{2.374118in}}%
\pgfpathlineto{\pgfqpoint{4.451764in}{2.401686in}}%
\pgfpathlineto{\pgfqpoint{4.452454in}{2.387654in}}%
\pgfpathlineto{\pgfqpoint{4.453836in}{2.359349in}}%
\pgfpathlineto{\pgfqpoint{4.454132in}{2.367797in}}%
\pgfpathlineto{\pgfqpoint{4.455712in}{2.502852in}}%
\pgfpathlineto{\pgfqpoint{4.456600in}{2.655045in}}%
\pgfpathlineto{\pgfqpoint{4.457093in}{2.607565in}}%
\pgfpathlineto{\pgfqpoint{4.457488in}{2.607633in}}%
\pgfpathlineto{\pgfqpoint{4.457982in}{2.495041in}}%
\pgfpathlineto{\pgfqpoint{4.458574in}{2.352784in}}%
\pgfpathlineto{\pgfqpoint{4.459265in}{2.404222in}}%
\pgfpathlineto{\pgfqpoint{4.459758in}{2.411314in}}%
\pgfpathlineto{\pgfqpoint{4.459956in}{2.405652in}}%
\pgfpathlineto{\pgfqpoint{4.460350in}{2.381729in}}%
\pgfpathlineto{\pgfqpoint{4.460745in}{2.415086in}}%
\pgfpathlineto{\pgfqpoint{4.461732in}{2.447008in}}%
\pgfpathlineto{\pgfqpoint{4.462028in}{2.440818in}}%
\pgfpathlineto{\pgfqpoint{4.462719in}{2.419248in}}%
\pgfpathlineto{\pgfqpoint{4.463706in}{2.419807in}}%
\pgfpathlineto{\pgfqpoint{4.463805in}{2.419264in}}%
\pgfpathlineto{\pgfqpoint{4.463904in}{2.420839in}}%
\pgfpathlineto{\pgfqpoint{4.464693in}{2.457664in}}%
\pgfpathlineto{\pgfqpoint{4.465187in}{2.435156in}}%
\pgfpathlineto{\pgfqpoint{4.465483in}{2.420227in}}%
\pgfpathlineto{\pgfqpoint{4.466272in}{2.432945in}}%
\pgfpathlineto{\pgfqpoint{4.466865in}{2.423439in}}%
\pgfpathlineto{\pgfqpoint{4.467161in}{2.430323in}}%
\pgfpathlineto{\pgfqpoint{4.467950in}{2.466883in}}%
\pgfpathlineto{\pgfqpoint{4.468641in}{2.451106in}}%
\pgfpathlineto{\pgfqpoint{4.469431in}{2.464034in}}%
\pgfpathlineto{\pgfqpoint{4.469727in}{2.452415in}}%
\pgfpathlineto{\pgfqpoint{4.469924in}{2.447711in}}%
\pgfpathlineto{\pgfqpoint{4.470615in}{2.459782in}}%
\pgfpathlineto{\pgfqpoint{4.470911in}{2.469321in}}%
\pgfpathlineto{\pgfqpoint{4.471503in}{2.454920in}}%
\pgfpathlineto{\pgfqpoint{4.471799in}{2.450021in}}%
\pgfpathlineto{\pgfqpoint{4.472293in}{2.459274in}}%
\pgfpathlineto{\pgfqpoint{4.472786in}{2.476753in}}%
\pgfpathlineto{\pgfqpoint{4.473280in}{2.455708in}}%
\pgfpathlineto{\pgfqpoint{4.473675in}{2.449139in}}%
\pgfpathlineto{\pgfqpoint{4.473872in}{2.453963in}}%
\pgfpathlineto{\pgfqpoint{4.474464in}{2.482801in}}%
\pgfpathlineto{\pgfqpoint{4.474958in}{2.459897in}}%
\pgfpathlineto{\pgfqpoint{4.475155in}{2.454593in}}%
\pgfpathlineto{\pgfqpoint{4.475649in}{2.470317in}}%
\pgfpathlineto{\pgfqpoint{4.475945in}{2.481857in}}%
\pgfpathlineto{\pgfqpoint{4.476438in}{2.454917in}}%
\pgfpathlineto{\pgfqpoint{4.476636in}{2.452627in}}%
\pgfpathlineto{\pgfqpoint{4.477129in}{2.459346in}}%
\pgfpathlineto{\pgfqpoint{4.477425in}{2.471176in}}%
\pgfpathlineto{\pgfqpoint{4.478018in}{2.451535in}}%
\pgfpathlineto{\pgfqpoint{4.478412in}{2.443032in}}%
\pgfpathlineto{\pgfqpoint{4.478807in}{2.457019in}}%
\pgfpathlineto{\pgfqpoint{4.479202in}{2.444481in}}%
\pgfpathlineto{\pgfqpoint{4.479498in}{2.454237in}}%
\pgfpathlineto{\pgfqpoint{4.479991in}{2.432379in}}%
\pgfpathlineto{\pgfqpoint{4.480189in}{2.431285in}}%
\pgfpathlineto{\pgfqpoint{4.480485in}{2.434654in}}%
\pgfpathlineto{\pgfqpoint{4.480781in}{2.439397in}}%
\pgfpathlineto{\pgfqpoint{4.481275in}{2.428892in}}%
\pgfpathlineto{\pgfqpoint{4.481669in}{2.407941in}}%
\pgfpathlineto{\pgfqpoint{4.482656in}{2.411298in}}%
\pgfpathlineto{\pgfqpoint{4.483051in}{2.399986in}}%
\pgfpathlineto{\pgfqpoint{4.483841in}{2.403824in}}%
\pgfpathlineto{\pgfqpoint{4.484236in}{2.420315in}}%
\pgfpathlineto{\pgfqpoint{4.484630in}{2.395089in}}%
\pgfpathlineto{\pgfqpoint{4.484828in}{2.387048in}}%
\pgfpathlineto{\pgfqpoint{4.485223in}{2.395339in}}%
\pgfpathlineto{\pgfqpoint{4.485617in}{2.395185in}}%
\pgfpathlineto{\pgfqpoint{4.485815in}{2.398996in}}%
\pgfpathlineto{\pgfqpoint{4.486308in}{2.385894in}}%
\pgfpathlineto{\pgfqpoint{4.486802in}{2.384754in}}%
\pgfpathlineto{\pgfqpoint{4.487197in}{2.400488in}}%
\pgfpathlineto{\pgfqpoint{4.487493in}{2.407537in}}%
\pgfpathlineto{\pgfqpoint{4.487986in}{2.392947in}}%
\pgfpathlineto{\pgfqpoint{4.488282in}{2.381037in}}%
\pgfpathlineto{\pgfqpoint{4.488776in}{2.408718in}}%
\pgfpathlineto{\pgfqpoint{4.488874in}{2.410644in}}%
\pgfpathlineto{\pgfqpoint{4.489170in}{2.398653in}}%
\pgfpathlineto{\pgfqpoint{4.489763in}{2.377552in}}%
\pgfpathlineto{\pgfqpoint{4.490256in}{2.396869in}}%
\pgfpathlineto{\pgfqpoint{4.492724in}{2.455590in}}%
\pgfpathlineto{\pgfqpoint{4.492822in}{2.453697in}}%
\pgfpathlineto{\pgfqpoint{4.493118in}{2.437869in}}%
\pgfpathlineto{\pgfqpoint{4.493711in}{2.461634in}}%
\pgfpathlineto{\pgfqpoint{4.493809in}{2.459609in}}%
\pgfpathlineto{\pgfqpoint{4.494599in}{2.430431in}}%
\pgfpathlineto{\pgfqpoint{4.496376in}{2.379130in}}%
\pgfpathlineto{\pgfqpoint{4.497955in}{2.326258in}}%
\pgfpathlineto{\pgfqpoint{4.496869in}{2.388128in}}%
\pgfpathlineto{\pgfqpoint{4.498251in}{2.350578in}}%
\pgfpathlineto{\pgfqpoint{4.500126in}{2.481957in}}%
\pgfpathlineto{\pgfqpoint{4.500323in}{2.471915in}}%
\pgfpathlineto{\pgfqpoint{4.501508in}{2.410098in}}%
\pgfpathlineto{\pgfqpoint{4.501903in}{2.427842in}}%
\pgfpathlineto{\pgfqpoint{4.503284in}{2.493289in}}%
\pgfpathlineto{\pgfqpoint{4.504666in}{2.692711in}}%
\pgfpathlineto{\pgfqpoint{4.505258in}{2.662262in}}%
\pgfpathlineto{\pgfqpoint{4.505357in}{2.663902in}}%
\pgfpathlineto{\pgfqpoint{4.505456in}{2.658951in}}%
\pgfpathlineto{\pgfqpoint{4.506443in}{2.402001in}}%
\pgfpathlineto{\pgfqpoint{4.507825in}{2.444020in}}%
\pgfpathlineto{\pgfqpoint{4.509601in}{2.508609in}}%
\pgfpathlineto{\pgfqpoint{4.510391in}{2.483468in}}%
\pgfpathlineto{\pgfqpoint{4.510786in}{2.459438in}}%
\pgfpathlineto{\pgfqpoint{4.511575in}{2.471649in}}%
\pgfpathlineto{\pgfqpoint{4.512562in}{2.498938in}}%
\pgfpathlineto{\pgfqpoint{4.513253in}{2.494770in}}%
\pgfpathlineto{\pgfqpoint{4.513352in}{2.496350in}}%
\pgfpathlineto{\pgfqpoint{4.513549in}{2.487276in}}%
\pgfpathlineto{\pgfqpoint{4.513845in}{2.470407in}}%
\pgfpathlineto{\pgfqpoint{4.514734in}{2.474740in}}%
\pgfpathlineto{\pgfqpoint{4.516707in}{2.506455in}}%
\pgfpathlineto{\pgfqpoint{4.516806in}{2.503461in}}%
\pgfpathlineto{\pgfqpoint{4.517694in}{2.479846in}}%
\pgfpathlineto{\pgfqpoint{4.517991in}{2.489424in}}%
\pgfpathlineto{\pgfqpoint{4.518484in}{2.502817in}}%
\pgfpathlineto{\pgfqpoint{4.519076in}{2.489158in}}%
\pgfpathlineto{\pgfqpoint{4.519471in}{2.474469in}}%
\pgfpathlineto{\pgfqpoint{4.519965in}{2.497008in}}%
\pgfpathlineto{\pgfqpoint{4.520261in}{2.484941in}}%
\pgfpathlineto{\pgfqpoint{4.521149in}{2.454716in}}%
\pgfpathlineto{\pgfqpoint{4.521445in}{2.475871in}}%
\pgfpathlineto{\pgfqpoint{4.521642in}{2.488616in}}%
\pgfpathlineto{\pgfqpoint{4.522432in}{2.472104in}}%
\pgfpathlineto{\pgfqpoint{4.522827in}{2.467255in}}%
\pgfpathlineto{\pgfqpoint{4.523222in}{2.477389in}}%
\pgfpathlineto{\pgfqpoint{4.523419in}{2.482634in}}%
\pgfpathlineto{\pgfqpoint{4.523913in}{2.466534in}}%
\pgfpathlineto{\pgfqpoint{4.524801in}{2.460659in}}%
\pgfpathlineto{\pgfqpoint{4.524900in}{2.463732in}}%
\pgfpathlineto{\pgfqpoint{4.525196in}{2.478676in}}%
\pgfpathlineto{\pgfqpoint{4.525689in}{2.447930in}}%
\pgfpathlineto{\pgfqpoint{4.525788in}{2.447234in}}%
\pgfpathlineto{\pgfqpoint{4.526084in}{2.450887in}}%
\pgfpathlineto{\pgfqpoint{4.526281in}{2.449798in}}%
\pgfpathlineto{\pgfqpoint{4.526775in}{2.468555in}}%
\pgfpathlineto{\pgfqpoint{4.527367in}{2.451223in}}%
\pgfpathlineto{\pgfqpoint{4.527564in}{2.452369in}}%
\pgfpathlineto{\pgfqpoint{4.528058in}{2.464945in}}%
\pgfpathlineto{\pgfqpoint{4.528453in}{2.450206in}}%
\pgfpathlineto{\pgfqpoint{4.529341in}{2.422655in}}%
\pgfpathlineto{\pgfqpoint{4.529736in}{2.440075in}}%
\pgfpathlineto{\pgfqpoint{4.529933in}{2.443863in}}%
\pgfpathlineto{\pgfqpoint{4.530328in}{2.422431in}}%
\pgfpathlineto{\pgfqpoint{4.530624in}{2.407321in}}%
\pgfpathlineto{\pgfqpoint{4.531512in}{2.417564in}}%
\pgfpathlineto{\pgfqpoint{4.531808in}{2.427918in}}%
\pgfpathlineto{\pgfqpoint{4.532203in}{2.402724in}}%
\pgfpathlineto{\pgfqpoint{4.532401in}{2.396707in}}%
\pgfpathlineto{\pgfqpoint{4.532993in}{2.416907in}}%
\pgfpathlineto{\pgfqpoint{4.533980in}{2.391963in}}%
\pgfpathlineto{\pgfqpoint{4.534572in}{2.403049in}}%
\pgfpathlineto{\pgfqpoint{4.535065in}{2.421307in}}%
\pgfpathlineto{\pgfqpoint{4.535559in}{2.403192in}}%
\pgfpathlineto{\pgfqpoint{4.535756in}{2.399036in}}%
\pgfpathlineto{\pgfqpoint{4.536151in}{2.412883in}}%
\pgfpathlineto{\pgfqpoint{4.536941in}{2.423649in}}%
\pgfpathlineto{\pgfqpoint{4.537237in}{2.413960in}}%
\pgfpathlineto{\pgfqpoint{4.537434in}{2.406601in}}%
\pgfpathlineto{\pgfqpoint{4.537928in}{2.430307in}}%
\pgfpathlineto{\pgfqpoint{4.538026in}{2.429234in}}%
\pgfpathlineto{\pgfqpoint{4.539013in}{2.406199in}}%
\pgfpathlineto{\pgfqpoint{4.539211in}{2.419601in}}%
\pgfpathlineto{\pgfqpoint{4.539606in}{2.441702in}}%
\pgfpathlineto{\pgfqpoint{4.540494in}{2.436162in}}%
\pgfpathlineto{\pgfqpoint{4.540593in}{2.436094in}}%
\pgfpathlineto{\pgfqpoint{4.541185in}{2.466019in}}%
\pgfpathlineto{\pgfqpoint{4.542073in}{2.455634in}}%
\pgfpathlineto{\pgfqpoint{4.544047in}{2.392578in}}%
\pgfpathlineto{\pgfqpoint{4.544639in}{2.423720in}}%
\pgfpathlineto{\pgfqpoint{4.545330in}{2.405768in}}%
\pgfpathlineto{\pgfqpoint{4.546218in}{2.381549in}}%
\pgfpathlineto{\pgfqpoint{4.546515in}{2.392958in}}%
\pgfpathlineto{\pgfqpoint{4.546613in}{2.393073in}}%
\pgfpathlineto{\pgfqpoint{4.547205in}{2.376445in}}%
\pgfpathlineto{\pgfqpoint{4.547502in}{2.393145in}}%
\pgfpathlineto{\pgfqpoint{4.548291in}{2.429071in}}%
\pgfpathlineto{\pgfqpoint{4.548587in}{2.407284in}}%
\pgfpathlineto{\pgfqpoint{4.548883in}{2.385530in}}%
\pgfpathlineto{\pgfqpoint{4.549772in}{2.391950in}}%
\pgfpathlineto{\pgfqpoint{4.549870in}{2.392859in}}%
\pgfpathlineto{\pgfqpoint{4.550068in}{2.387116in}}%
\pgfpathlineto{\pgfqpoint{4.550364in}{2.378326in}}%
\pgfpathlineto{\pgfqpoint{4.550759in}{2.402249in}}%
\pgfpathlineto{\pgfqpoint{4.551844in}{2.498268in}}%
\pgfpathlineto{\pgfqpoint{4.552930in}{2.673316in}}%
\pgfpathlineto{\pgfqpoint{4.553522in}{2.623801in}}%
\pgfpathlineto{\pgfqpoint{4.554114in}{2.540167in}}%
\pgfpathlineto{\pgfqpoint{4.554805in}{2.393999in}}%
\pgfpathlineto{\pgfqpoint{4.555496in}{2.428814in}}%
\pgfpathlineto{\pgfqpoint{4.555595in}{2.427832in}}%
\pgfpathlineto{\pgfqpoint{4.555792in}{2.434536in}}%
\pgfpathlineto{\pgfqpoint{4.556088in}{2.453018in}}%
\pgfpathlineto{\pgfqpoint{4.556878in}{2.443842in}}%
\pgfpathlineto{\pgfqpoint{4.557075in}{2.434436in}}%
\pgfpathlineto{\pgfqpoint{4.557371in}{2.461599in}}%
\pgfpathlineto{\pgfqpoint{4.557766in}{2.506901in}}%
\pgfpathlineto{\pgfqpoint{4.558457in}{2.469573in}}%
\pgfpathlineto{\pgfqpoint{4.560036in}{2.437868in}}%
\pgfpathlineto{\pgfqpoint{4.560332in}{2.445806in}}%
\pgfpathlineto{\pgfqpoint{4.561122in}{2.479615in}}%
\pgfpathlineto{\pgfqpoint{4.561517in}{2.459304in}}%
\pgfpathlineto{\pgfqpoint{4.561912in}{2.435447in}}%
\pgfpathlineto{\pgfqpoint{4.562800in}{2.442569in}}%
\pgfpathlineto{\pgfqpoint{4.563096in}{2.444081in}}%
\pgfpathlineto{\pgfqpoint{4.563195in}{2.442631in}}%
\pgfpathlineto{\pgfqpoint{4.563589in}{2.432206in}}%
\pgfpathlineto{\pgfqpoint{4.563886in}{2.447807in}}%
\pgfpathlineto{\pgfqpoint{4.564280in}{2.474083in}}%
\pgfpathlineto{\pgfqpoint{4.564774in}{2.442791in}}%
\pgfpathlineto{\pgfqpoint{4.564971in}{2.445697in}}%
\pgfpathlineto{\pgfqpoint{4.565860in}{2.471375in}}%
\pgfpathlineto{\pgfqpoint{4.565366in}{2.444504in}}%
\pgfpathlineto{\pgfqpoint{4.566452in}{2.458972in}}%
\pgfpathlineto{\pgfqpoint{4.566847in}{2.458566in}}%
\pgfpathlineto{\pgfqpoint{4.567143in}{2.464651in}}%
\pgfpathlineto{\pgfqpoint{4.567636in}{2.484371in}}%
\pgfpathlineto{\pgfqpoint{4.568130in}{2.461769in}}%
\pgfpathlineto{\pgfqpoint{4.568722in}{2.457336in}}%
\pgfpathlineto{\pgfqpoint{4.569018in}{2.464355in}}%
\pgfpathlineto{\pgfqpoint{4.569314in}{2.469394in}}%
\pgfpathlineto{\pgfqpoint{4.569808in}{2.460319in}}%
\pgfpathlineto{\pgfqpoint{4.570104in}{2.450152in}}%
\pgfpathlineto{\pgfqpoint{4.570795in}{2.465326in}}%
\pgfpathlineto{\pgfqpoint{4.571091in}{2.475564in}}%
\pgfpathlineto{\pgfqpoint{4.571485in}{2.449807in}}%
\pgfpathlineto{\pgfqpoint{4.573459in}{2.380099in}}%
\pgfpathlineto{\pgfqpoint{4.573657in}{2.383822in}}%
\pgfpathlineto{\pgfqpoint{4.576026in}{2.472512in}}%
\pgfpathlineto{\pgfqpoint{4.576223in}{2.468314in}}%
\pgfpathlineto{\pgfqpoint{4.576815in}{2.455786in}}%
\pgfpathlineto{\pgfqpoint{4.577111in}{2.470285in}}%
\pgfpathlineto{\pgfqpoint{4.577309in}{2.476304in}}%
\pgfpathlineto{\pgfqpoint{4.577802in}{2.451191in}}%
\pgfpathlineto{\pgfqpoint{4.579283in}{2.429419in}}%
\pgfpathlineto{\pgfqpoint{4.579480in}{2.432796in}}%
\pgfpathlineto{\pgfqpoint{4.579579in}{2.433252in}}%
\pgfpathlineto{\pgfqpoint{4.579776in}{2.428992in}}%
\pgfpathlineto{\pgfqpoint{4.579974in}{2.424486in}}%
\pgfpathlineto{\pgfqpoint{4.580368in}{2.438578in}}%
\pgfpathlineto{\pgfqpoint{4.580664in}{2.447712in}}%
\pgfpathlineto{\pgfqpoint{4.581257in}{2.431716in}}%
\pgfpathlineto{\pgfqpoint{4.581750in}{2.416021in}}%
\pgfpathlineto{\pgfqpoint{4.582145in}{2.436556in}}%
\pgfpathlineto{\pgfqpoint{4.582342in}{2.440956in}}%
\pgfpathlineto{\pgfqpoint{4.582934in}{2.428828in}}%
\pgfpathlineto{\pgfqpoint{4.583231in}{2.417924in}}%
\pgfpathlineto{\pgfqpoint{4.583625in}{2.439232in}}%
\pgfpathlineto{\pgfqpoint{4.583921in}{2.454447in}}%
\pgfpathlineto{\pgfqpoint{4.584514in}{2.427032in}}%
\pgfpathlineto{\pgfqpoint{4.584612in}{2.426859in}}%
\pgfpathlineto{\pgfqpoint{4.584711in}{2.428866in}}%
\pgfpathlineto{\pgfqpoint{4.585994in}{2.439662in}}%
\pgfpathlineto{\pgfqpoint{4.585599in}{2.427811in}}%
\pgfpathlineto{\pgfqpoint{4.586093in}{2.439456in}}%
\pgfpathlineto{\pgfqpoint{4.586685in}{2.426567in}}%
\pgfpathlineto{\pgfqpoint{4.587080in}{2.441521in}}%
\pgfpathlineto{\pgfqpoint{4.587277in}{2.444641in}}%
\pgfpathlineto{\pgfqpoint{4.587771in}{2.434605in}}%
\pgfpathlineto{\pgfqpoint{4.588067in}{2.424356in}}%
\pgfpathlineto{\pgfqpoint{4.588462in}{2.447942in}}%
\pgfpathlineto{\pgfqpoint{4.589251in}{2.464322in}}%
\pgfpathlineto{\pgfqpoint{4.589547in}{2.450845in}}%
\pgfpathlineto{\pgfqpoint{4.589646in}{2.450890in}}%
\pgfpathlineto{\pgfqpoint{4.590534in}{2.470111in}}%
\pgfpathlineto{\pgfqpoint{4.590830in}{2.457932in}}%
\pgfpathlineto{\pgfqpoint{4.592508in}{2.403805in}}%
\pgfpathlineto{\pgfqpoint{4.592903in}{2.398104in}}%
\pgfpathlineto{\pgfqpoint{4.593100in}{2.408951in}}%
\pgfpathlineto{\pgfqpoint{4.593397in}{2.425165in}}%
\pgfpathlineto{\pgfqpoint{4.594087in}{2.401192in}}%
\pgfpathlineto{\pgfqpoint{4.594581in}{2.376697in}}%
\pgfpathlineto{\pgfqpoint{4.595272in}{2.396453in}}%
\pgfpathlineto{\pgfqpoint{4.595469in}{2.394049in}}%
\pgfpathlineto{\pgfqpoint{4.595963in}{2.401201in}}%
\pgfpathlineto{\pgfqpoint{4.596752in}{2.420577in}}%
\pgfpathlineto{\pgfqpoint{4.597048in}{2.408059in}}%
\pgfpathlineto{\pgfqpoint{4.598035in}{2.355015in}}%
\pgfpathlineto{\pgfqpoint{4.598430in}{2.373642in}}%
\pgfpathlineto{\pgfqpoint{4.600404in}{2.549010in}}%
\pgfpathlineto{\pgfqpoint{4.600898in}{2.624183in}}%
\pgfpathlineto{\pgfqpoint{4.601687in}{2.604563in}}%
\pgfpathlineto{\pgfqpoint{4.602576in}{2.470111in}}%
\pgfpathlineto{\pgfqpoint{4.603069in}{2.353463in}}%
\pgfpathlineto{\pgfqpoint{4.603760in}{2.417199in}}%
\pgfpathlineto{\pgfqpoint{4.603859in}{2.418212in}}%
\pgfpathlineto{\pgfqpoint{4.604056in}{2.412054in}}%
\pgfpathlineto{\pgfqpoint{4.604352in}{2.398330in}}%
\pgfpathlineto{\pgfqpoint{4.604944in}{2.422397in}}%
\pgfpathlineto{\pgfqpoint{4.606129in}{2.458629in}}%
\pgfpathlineto{\pgfqpoint{4.606425in}{2.444364in}}%
\pgfpathlineto{\pgfqpoint{4.607708in}{2.396241in}}%
\pgfpathlineto{\pgfqpoint{4.608004in}{2.413748in}}%
\pgfpathlineto{\pgfqpoint{4.608991in}{2.444522in}}%
\pgfpathlineto{\pgfqpoint{4.609386in}{2.437825in}}%
\pgfpathlineto{\pgfqpoint{4.609583in}{2.438359in}}%
\pgfpathlineto{\pgfqpoint{4.609879in}{2.436286in}}%
\pgfpathlineto{\pgfqpoint{4.610274in}{2.439428in}}%
\pgfpathlineto{\pgfqpoint{4.610471in}{2.433435in}}%
\pgfpathlineto{\pgfqpoint{4.610866in}{2.416402in}}%
\pgfpathlineto{\pgfqpoint{4.611458in}{2.440011in}}%
\pgfpathlineto{\pgfqpoint{4.613531in}{2.464360in}}%
\pgfpathlineto{\pgfqpoint{4.613630in}{2.464010in}}%
\pgfpathlineto{\pgfqpoint{4.614222in}{2.450137in}}%
\pgfpathlineto{\pgfqpoint{4.614617in}{2.464593in}}%
\pgfpathlineto{\pgfqpoint{4.615012in}{2.484494in}}%
\pgfpathlineto{\pgfqpoint{4.615703in}{2.468830in}}%
\pgfpathlineto{\pgfqpoint{4.615900in}{2.463527in}}%
\pgfpathlineto{\pgfqpoint{4.616393in}{2.484107in}}%
\pgfpathlineto{\pgfqpoint{4.616492in}{2.484535in}}%
\pgfpathlineto{\pgfqpoint{4.616690in}{2.481725in}}%
\pgfpathlineto{\pgfqpoint{4.617973in}{2.469484in}}%
\pgfpathlineto{\pgfqpoint{4.617479in}{2.485489in}}%
\pgfpathlineto{\pgfqpoint{4.618071in}{2.473165in}}%
\pgfpathlineto{\pgfqpoint{4.618565in}{2.494475in}}%
\pgfpathlineto{\pgfqpoint{4.619157in}{2.479687in}}%
\pgfpathlineto{\pgfqpoint{4.619552in}{2.474710in}}%
\pgfpathlineto{\pgfqpoint{4.619947in}{2.485458in}}%
\pgfpathlineto{\pgfqpoint{4.620144in}{2.491000in}}%
\pgfpathlineto{\pgfqpoint{4.620637in}{2.469831in}}%
\pgfpathlineto{\pgfqpoint{4.620736in}{2.469747in}}%
\pgfpathlineto{\pgfqpoint{4.621526in}{2.487761in}}%
\pgfpathlineto{\pgfqpoint{4.622019in}{2.478907in}}%
\pgfpathlineto{\pgfqpoint{4.622414in}{2.458422in}}%
\pgfpathlineto{\pgfqpoint{4.623204in}{2.470245in}}%
\pgfpathlineto{\pgfqpoint{4.623302in}{2.470240in}}%
\pgfpathlineto{\pgfqpoint{4.623500in}{2.472972in}}%
\pgfpathlineto{\pgfqpoint{4.623796in}{2.466457in}}%
\pgfpathlineto{\pgfqpoint{4.624191in}{2.454152in}}%
\pgfpathlineto{\pgfqpoint{4.624882in}{2.463562in}}%
\pgfpathlineto{\pgfqpoint{4.624980in}{2.465949in}}%
\pgfpathlineto{\pgfqpoint{4.625276in}{2.454052in}}%
\pgfpathlineto{\pgfqpoint{4.625967in}{2.435694in}}%
\pgfpathlineto{\pgfqpoint{4.626362in}{2.446908in}}%
\pgfpathlineto{\pgfqpoint{4.626559in}{2.450522in}}%
\pgfpathlineto{\pgfqpoint{4.627053in}{2.440093in}}%
\pgfpathlineto{\pgfqpoint{4.627645in}{2.418997in}}%
\pgfpathlineto{\pgfqpoint{4.627941in}{2.437406in}}%
\pgfpathlineto{\pgfqpoint{4.628139in}{2.449516in}}%
\pgfpathlineto{\pgfqpoint{4.628731in}{2.414903in}}%
\pgfpathlineto{\pgfqpoint{4.629718in}{2.434286in}}%
\pgfpathlineto{\pgfqpoint{4.630211in}{2.425890in}}%
\pgfpathlineto{\pgfqpoint{4.630310in}{2.425820in}}%
\pgfpathlineto{\pgfqpoint{4.630705in}{2.411134in}}%
\pgfpathlineto{\pgfqpoint{4.631198in}{2.428257in}}%
\pgfpathlineto{\pgfqpoint{4.631494in}{2.422263in}}%
\pgfpathlineto{\pgfqpoint{4.631790in}{2.421386in}}%
\pgfpathlineto{\pgfqpoint{4.632185in}{2.408943in}}%
\pgfpathlineto{\pgfqpoint{4.632876in}{2.418465in}}%
\pgfpathlineto{\pgfqpoint{4.633370in}{2.428170in}}%
\pgfpathlineto{\pgfqpoint{4.633666in}{2.417736in}}%
\pgfpathlineto{\pgfqpoint{4.633962in}{2.407925in}}%
\pgfpathlineto{\pgfqpoint{4.634554in}{2.427796in}}%
\pgfpathlineto{\pgfqpoint{4.634751in}{2.429945in}}%
\pgfpathlineto{\pgfqpoint{4.635146in}{2.422669in}}%
\pgfpathlineto{\pgfqpoint{4.635442in}{2.423984in}}%
\pgfpathlineto{\pgfqpoint{4.635738in}{2.418801in}}%
\pgfpathlineto{\pgfqpoint{4.635936in}{2.422697in}}%
\pgfpathlineto{\pgfqpoint{4.636923in}{2.446386in}}%
\pgfpathlineto{\pgfqpoint{4.637318in}{2.443801in}}%
\pgfpathlineto{\pgfqpoint{4.638305in}{2.469456in}}%
\pgfpathlineto{\pgfqpoint{4.638897in}{2.454032in}}%
\pgfpathlineto{\pgfqpoint{4.640673in}{2.406655in}}%
\pgfpathlineto{\pgfqpoint{4.639390in}{2.460644in}}%
\pgfpathlineto{\pgfqpoint{4.641266in}{2.428643in}}%
\pgfpathlineto{\pgfqpoint{4.641463in}{2.439350in}}%
\pgfpathlineto{\pgfqpoint{4.642055in}{2.413203in}}%
\pgfpathlineto{\pgfqpoint{4.643733in}{2.372324in}}%
\pgfpathlineto{\pgfqpoint{4.642943in}{2.415039in}}%
\pgfpathlineto{\pgfqpoint{4.643930in}{2.379024in}}%
\pgfpathlineto{\pgfqpoint{4.644819in}{2.429075in}}%
\pgfpathlineto{\pgfqpoint{4.645312in}{2.400585in}}%
\pgfpathlineto{\pgfqpoint{4.645806in}{2.377803in}}%
\pgfpathlineto{\pgfqpoint{4.646694in}{2.380244in}}%
\pgfpathlineto{\pgfqpoint{4.647484in}{2.374747in}}%
\pgfpathlineto{\pgfqpoint{4.648569in}{2.450692in}}%
\pgfpathlineto{\pgfqpoint{4.649556in}{2.578563in}}%
\pgfpathlineto{\pgfqpoint{4.650050in}{2.538416in}}%
\pgfpathlineto{\pgfqpoint{4.651432in}{2.337848in}}%
\pgfpathlineto{\pgfqpoint{4.652320in}{2.396794in}}%
\pgfpathlineto{\pgfqpoint{4.652616in}{2.411985in}}%
\pgfpathlineto{\pgfqpoint{4.653011in}{2.395312in}}%
\pgfpathlineto{\pgfqpoint{4.653406in}{2.401425in}}%
\pgfpathlineto{\pgfqpoint{4.653800in}{2.387527in}}%
\pgfpathlineto{\pgfqpoint{4.654096in}{2.414569in}}%
\pgfpathlineto{\pgfqpoint{4.654491in}{2.449467in}}%
\pgfpathlineto{\pgfqpoint{4.655083in}{2.402965in}}%
\pgfpathlineto{\pgfqpoint{4.655676in}{2.386752in}}%
\pgfpathlineto{\pgfqpoint{4.656465in}{2.394282in}}%
\pgfpathlineto{\pgfqpoint{4.657057in}{2.406344in}}%
\pgfpathlineto{\pgfqpoint{4.657551in}{2.436598in}}%
\pgfpathlineto{\pgfqpoint{4.658242in}{2.425700in}}%
\pgfpathlineto{\pgfqpoint{4.658735in}{2.388012in}}%
\pgfpathlineto{\pgfqpoint{4.659426in}{2.413464in}}%
\pgfpathlineto{\pgfqpoint{4.659525in}{2.413887in}}%
\pgfpathlineto{\pgfqpoint{4.659722in}{2.410220in}}%
\pgfpathlineto{\pgfqpoint{4.660018in}{2.401420in}}%
\pgfpathlineto{\pgfqpoint{4.660512in}{2.414672in}}%
\pgfpathlineto{\pgfqpoint{4.661005in}{2.443817in}}%
\pgfpathlineto{\pgfqpoint{4.661696in}{2.423457in}}%
\pgfpathlineto{\pgfqpoint{4.661894in}{2.418060in}}%
\pgfpathlineto{\pgfqpoint{4.662190in}{2.444711in}}%
\pgfpathlineto{\pgfqpoint{4.662979in}{2.456984in}}%
\pgfpathlineto{\pgfqpoint{4.663275in}{2.444107in}}%
\pgfpathlineto{\pgfqpoint{4.663374in}{2.440462in}}%
\pgfpathlineto{\pgfqpoint{4.663670in}{2.459625in}}%
\pgfpathlineto{\pgfqpoint{4.664065in}{2.483073in}}%
\pgfpathlineto{\pgfqpoint{4.664756in}{2.465029in}}%
\pgfpathlineto{\pgfqpoint{4.664953in}{2.460214in}}%
\pgfpathlineto{\pgfqpoint{4.665447in}{2.483087in}}%
\pgfpathlineto{\pgfqpoint{4.665644in}{2.480252in}}%
\pgfpathlineto{\pgfqpoint{4.665940in}{2.491749in}}%
\pgfpathlineto{\pgfqpoint{4.667125in}{2.499675in}}%
\pgfpathlineto{\pgfqpoint{4.666631in}{2.477965in}}%
\pgfpathlineto{\pgfqpoint{4.667223in}{2.499607in}}%
\pgfpathlineto{\pgfqpoint{4.668309in}{2.484601in}}%
\pgfpathlineto{\pgfqpoint{4.668704in}{2.492074in}}%
\pgfpathlineto{\pgfqpoint{4.669099in}{2.501747in}}%
\pgfpathlineto{\pgfqpoint{4.669790in}{2.494295in}}%
\pgfpathlineto{\pgfqpoint{4.670184in}{2.484344in}}%
\pgfpathlineto{\pgfqpoint{4.670678in}{2.499718in}}%
\pgfpathlineto{\pgfqpoint{4.670974in}{2.505460in}}%
\pgfpathlineto{\pgfqpoint{4.671270in}{2.495932in}}%
\pgfpathlineto{\pgfqpoint{4.672158in}{2.472206in}}%
\pgfpathlineto{\pgfqpoint{4.672553in}{2.485418in}}%
\pgfpathlineto{\pgfqpoint{4.675021in}{2.450157in}}%
\pgfpathlineto{\pgfqpoint{4.675218in}{2.454105in}}%
\pgfpathlineto{\pgfqpoint{4.676008in}{2.464899in}}%
\pgfpathlineto{\pgfqpoint{4.676205in}{2.457682in}}%
\pgfpathlineto{\pgfqpoint{4.676501in}{2.444464in}}%
\pgfpathlineto{\pgfqpoint{4.676995in}{2.470292in}}%
\pgfpathlineto{\pgfqpoint{4.677291in}{2.485111in}}%
\pgfpathlineto{\pgfqpoint{4.677883in}{2.460519in}}%
\pgfpathlineto{\pgfqpoint{4.678672in}{2.473784in}}%
\pgfpathlineto{\pgfqpoint{4.679166in}{2.490886in}}%
\pgfpathlineto{\pgfqpoint{4.679758in}{2.473828in}}%
\pgfpathlineto{\pgfqpoint{4.679956in}{2.472057in}}%
\pgfpathlineto{\pgfqpoint{4.680252in}{2.479628in}}%
\pgfpathlineto{\pgfqpoint{4.680548in}{2.491341in}}%
\pgfpathlineto{\pgfqpoint{4.681041in}{2.464341in}}%
\pgfpathlineto{\pgfqpoint{4.685088in}{2.531881in}}%
\pgfpathlineto{\pgfqpoint{4.686963in}{2.561310in}}%
\pgfpathlineto{\pgfqpoint{4.687062in}{2.562086in}}%
\pgfpathlineto{\pgfqpoint{4.687259in}{2.557588in}}%
\pgfpathlineto{\pgfqpoint{4.688148in}{2.510162in}}%
\pgfpathlineto{\pgfqpoint{4.689036in}{2.514632in}}%
\pgfpathlineto{\pgfqpoint{4.689233in}{2.516558in}}%
\pgfpathlineto{\pgfqpoint{4.689529in}{2.509214in}}%
\pgfpathlineto{\pgfqpoint{4.691207in}{2.468355in}}%
\pgfpathlineto{\pgfqpoint{4.690122in}{2.511473in}}%
\pgfpathlineto{\pgfqpoint{4.691405in}{2.475431in}}%
\pgfpathlineto{\pgfqpoint{4.693280in}{2.518101in}}%
\pgfpathlineto{\pgfqpoint{4.691997in}{2.463203in}}%
\pgfpathlineto{\pgfqpoint{4.693379in}{2.514413in}}%
\pgfpathlineto{\pgfqpoint{4.694267in}{2.474040in}}%
\pgfpathlineto{\pgfqpoint{4.694859in}{2.489832in}}%
\pgfpathlineto{\pgfqpoint{4.696340in}{2.550335in}}%
\pgfpathlineto{\pgfqpoint{4.697425in}{2.752127in}}%
\pgfpathlineto{\pgfqpoint{4.698215in}{2.728549in}}%
\pgfpathlineto{\pgfqpoint{4.698708in}{2.704797in}}%
\pgfpathlineto{\pgfqpoint{4.699498in}{2.464425in}}%
\pgfpathlineto{\pgfqpoint{4.700485in}{2.536304in}}%
\pgfpathlineto{\pgfqpoint{4.700880in}{2.520396in}}%
\pgfpathlineto{\pgfqpoint{4.701472in}{2.535574in}}%
\pgfpathlineto{\pgfqpoint{4.702854in}{2.583561in}}%
\pgfpathlineto{\pgfqpoint{4.703249in}{2.561540in}}%
\pgfpathlineto{\pgfqpoint{4.704235in}{2.539568in}}%
\pgfpathlineto{\pgfqpoint{4.704532in}{2.548942in}}%
\pgfpathlineto{\pgfqpoint{4.706111in}{2.589005in}}%
\pgfpathlineto{\pgfqpoint{4.706308in}{2.579310in}}%
\pgfpathlineto{\pgfqpoint{4.707295in}{2.559453in}}%
\pgfpathlineto{\pgfqpoint{4.707493in}{2.563394in}}%
\pgfpathlineto{\pgfqpoint{4.708381in}{2.581856in}}%
\pgfpathlineto{\pgfqpoint{4.709072in}{2.580228in}}%
\pgfpathlineto{\pgfqpoint{4.709565in}{2.570381in}}%
\pgfpathlineto{\pgfqpoint{4.710157in}{2.578063in}}%
\pgfpathlineto{\pgfqpoint{4.710256in}{2.579063in}}%
\pgfpathlineto{\pgfqpoint{4.710454in}{2.572348in}}%
\pgfpathlineto{\pgfqpoint{4.710750in}{2.562430in}}%
\pgfpathlineto{\pgfqpoint{4.711342in}{2.584645in}}%
\pgfpathlineto{\pgfqpoint{4.711638in}{2.586620in}}%
\pgfpathlineto{\pgfqpoint{4.712033in}{2.581837in}}%
\pgfpathlineto{\pgfqpoint{4.712329in}{2.584001in}}%
\pgfpathlineto{\pgfqpoint{4.712625in}{2.576812in}}%
\pgfpathlineto{\pgfqpoint{4.712822in}{2.572600in}}%
\pgfpathlineto{\pgfqpoint{4.713118in}{2.593173in}}%
\pgfpathlineto{\pgfqpoint{4.713316in}{2.606063in}}%
\pgfpathlineto{\pgfqpoint{4.714007in}{2.583062in}}%
\pgfpathlineto{\pgfqpoint{4.714204in}{2.579933in}}%
\pgfpathlineto{\pgfqpoint{4.714599in}{2.589491in}}%
\pgfpathlineto{\pgfqpoint{4.714796in}{2.593664in}}%
\pgfpathlineto{\pgfqpoint{4.715388in}{2.583600in}}%
\pgfpathlineto{\pgfqpoint{4.715783in}{2.567769in}}%
\pgfpathlineto{\pgfqpoint{4.716474in}{2.577468in}}%
\pgfpathlineto{\pgfqpoint{4.716770in}{2.597161in}}%
\pgfpathlineto{\pgfqpoint{4.717362in}{2.562849in}}%
\pgfpathlineto{\pgfqpoint{4.717757in}{2.577811in}}%
\pgfpathlineto{\pgfqpoint{4.718646in}{2.570202in}}%
\pgfpathlineto{\pgfqpoint{4.719435in}{2.550856in}}%
\pgfpathlineto{\pgfqpoint{4.719929in}{2.559625in}}%
\pgfpathlineto{\pgfqpoint{4.723975in}{2.403870in}}%
\pgfpathlineto{\pgfqpoint{4.724271in}{2.387667in}}%
\pgfpathlineto{\pgfqpoint{4.724962in}{2.410796in}}%
\pgfpathlineto{\pgfqpoint{4.726541in}{2.458820in}}%
\pgfpathlineto{\pgfqpoint{4.726640in}{2.457004in}}%
\pgfpathlineto{\pgfqpoint{4.727134in}{2.442955in}}%
\pgfpathlineto{\pgfqpoint{4.727923in}{2.447423in}}%
\pgfpathlineto{\pgfqpoint{4.728022in}{2.447782in}}%
\pgfpathlineto{\pgfqpoint{4.728219in}{2.446012in}}%
\pgfpathlineto{\pgfqpoint{4.728812in}{2.415846in}}%
\pgfpathlineto{\pgfqpoint{4.729700in}{2.430939in}}%
\pgfpathlineto{\pgfqpoint{4.729799in}{2.431796in}}%
\pgfpathlineto{\pgfqpoint{4.729996in}{2.427723in}}%
\pgfpathlineto{\pgfqpoint{4.730687in}{2.412638in}}%
\pgfpathlineto{\pgfqpoint{4.731082in}{2.427428in}}%
\pgfpathlineto{\pgfqpoint{4.731772in}{2.415860in}}%
\pgfpathlineto{\pgfqpoint{4.732266in}{2.437289in}}%
\pgfpathlineto{\pgfqpoint{4.732562in}{2.428040in}}%
\pgfpathlineto{\pgfqpoint{4.732858in}{2.441422in}}%
\pgfpathlineto{\pgfqpoint{4.733056in}{2.450478in}}%
\pgfpathlineto{\pgfqpoint{4.733845in}{2.438946in}}%
\pgfpathlineto{\pgfqpoint{4.733944in}{2.436493in}}%
\pgfpathlineto{\pgfqpoint{4.734240in}{2.446905in}}%
\pgfpathlineto{\pgfqpoint{4.734635in}{2.467181in}}%
\pgfpathlineto{\pgfqpoint{4.735128in}{2.440957in}}%
\pgfpathlineto{\pgfqpoint{4.737004in}{2.381917in}}%
\pgfpathlineto{\pgfqpoint{4.737300in}{2.386360in}}%
\pgfpathlineto{\pgfqpoint{4.737793in}{2.402703in}}%
\pgfpathlineto{\pgfqpoint{4.738287in}{2.382219in}}%
\pgfpathlineto{\pgfqpoint{4.738978in}{2.355818in}}%
\pgfpathlineto{\pgfqpoint{4.739866in}{2.360198in}}%
\pgfpathlineto{\pgfqpoint{4.740162in}{2.355155in}}%
\pgfpathlineto{\pgfqpoint{4.740458in}{2.367470in}}%
\pgfpathlineto{\pgfqpoint{4.740951in}{2.414419in}}%
\pgfpathlineto{\pgfqpoint{4.741544in}{2.375826in}}%
\pgfpathlineto{\pgfqpoint{4.742235in}{2.346343in}}%
\pgfpathlineto{\pgfqpoint{4.742728in}{2.364716in}}%
\pgfpathlineto{\pgfqpoint{4.742827in}{2.365736in}}%
\pgfpathlineto{\pgfqpoint{4.743024in}{2.359221in}}%
\pgfpathlineto{\pgfqpoint{4.743419in}{2.347943in}}%
\pgfpathlineto{\pgfqpoint{4.744011in}{2.359778in}}%
\pgfpathlineto{\pgfqpoint{4.745886in}{2.589482in}}%
\pgfpathlineto{\pgfqpoint{4.746577in}{2.529876in}}%
\pgfpathlineto{\pgfqpoint{4.747466in}{2.292784in}}%
\pgfpathlineto{\pgfqpoint{4.748650in}{2.329016in}}%
\pgfpathlineto{\pgfqpoint{4.750032in}{2.359660in}}%
\pgfpathlineto{\pgfqpoint{4.750920in}{2.404036in}}%
\pgfpathlineto{\pgfqpoint{4.751315in}{2.384267in}}%
\pgfpathlineto{\pgfqpoint{4.751907in}{2.354905in}}%
\pgfpathlineto{\pgfqpoint{4.752401in}{2.385096in}}%
\pgfpathlineto{\pgfqpoint{4.754078in}{2.421653in}}%
\pgfpathlineto{\pgfqpoint{4.754276in}{2.416950in}}%
\pgfpathlineto{\pgfqpoint{4.754671in}{2.388401in}}%
\pgfpathlineto{\pgfqpoint{4.755559in}{2.397551in}}%
\pgfpathlineto{\pgfqpoint{4.757138in}{2.428908in}}%
\pgfpathlineto{\pgfqpoint{4.757237in}{2.430946in}}%
\pgfpathlineto{\pgfqpoint{4.757730in}{2.418183in}}%
\pgfpathlineto{\pgfqpoint{4.758421in}{2.402855in}}%
\pgfpathlineto{\pgfqpoint{4.758816in}{2.416171in}}%
\pgfpathlineto{\pgfqpoint{4.759408in}{2.433716in}}%
\pgfpathlineto{\pgfqpoint{4.760000in}{2.421599in}}%
\pgfpathlineto{\pgfqpoint{4.761086in}{2.438086in}}%
\pgfpathlineto{\pgfqpoint{4.761481in}{2.428397in}}%
\pgfpathlineto{\pgfqpoint{4.761777in}{2.415508in}}%
\pgfpathlineto{\pgfqpoint{4.762369in}{2.436286in}}%
\pgfpathlineto{\pgfqpoint{4.762665in}{2.440153in}}%
\pgfpathlineto{\pgfqpoint{4.762764in}{2.442057in}}%
\pgfpathlineto{\pgfqpoint{4.763060in}{2.428771in}}%
\pgfpathlineto{\pgfqpoint{4.763257in}{2.417913in}}%
\pgfpathlineto{\pgfqpoint{4.763751in}{2.436586in}}%
\pgfpathlineto{\pgfqpoint{4.764146in}{2.429884in}}%
\pgfpathlineto{\pgfqpoint{4.764343in}{2.431527in}}%
\pgfpathlineto{\pgfqpoint{4.764639in}{2.424645in}}%
\pgfpathlineto{\pgfqpoint{4.765034in}{2.416637in}}%
\pgfpathlineto{\pgfqpoint{4.765528in}{2.427283in}}%
\pgfpathlineto{\pgfqpoint{4.765922in}{2.435693in}}%
\pgfpathlineto{\pgfqpoint{4.766613in}{2.428956in}}%
\pgfpathlineto{\pgfqpoint{4.767205in}{2.437015in}}%
\pgfpathlineto{\pgfqpoint{4.767699in}{2.429674in}}%
\pgfpathlineto{\pgfqpoint{4.768094in}{2.420167in}}%
\pgfpathlineto{\pgfqpoint{4.768785in}{2.428349in}}%
\pgfpathlineto{\pgfqpoint{4.769278in}{2.433040in}}%
\pgfpathlineto{\pgfqpoint{4.769377in}{2.430224in}}%
\pgfpathlineto{\pgfqpoint{4.770660in}{2.407297in}}%
\pgfpathlineto{\pgfqpoint{4.770857in}{2.408406in}}%
\pgfpathlineto{\pgfqpoint{4.770956in}{2.408257in}}%
\pgfpathlineto{\pgfqpoint{4.773423in}{2.367257in}}%
\pgfpathlineto{\pgfqpoint{4.773522in}{2.367752in}}%
\pgfpathlineto{\pgfqpoint{4.773917in}{2.382631in}}%
\pgfpathlineto{\pgfqpoint{4.774410in}{2.365277in}}%
\pgfpathlineto{\pgfqpoint{4.774707in}{2.356230in}}%
\pgfpathlineto{\pgfqpoint{4.775397in}{2.367704in}}%
\pgfpathlineto{\pgfqpoint{4.775694in}{2.374189in}}%
\pgfpathlineto{\pgfqpoint{4.776384in}{2.364708in}}%
\pgfpathlineto{\pgfqpoint{4.776779in}{2.351970in}}%
\pgfpathlineto{\pgfqpoint{4.777273in}{2.367901in}}%
\pgfpathlineto{\pgfqpoint{4.777668in}{2.358441in}}%
\pgfpathlineto{\pgfqpoint{4.778161in}{2.354885in}}%
\pgfpathlineto{\pgfqpoint{4.778358in}{2.359250in}}%
\pgfpathlineto{\pgfqpoint{4.778753in}{2.385345in}}%
\pgfpathlineto{\pgfqpoint{4.779543in}{2.367607in}}%
\pgfpathlineto{\pgfqpoint{4.779740in}{2.364359in}}%
\pgfpathlineto{\pgfqpoint{4.779938in}{2.373827in}}%
\pgfpathlineto{\pgfqpoint{4.780332in}{2.398302in}}%
\pgfpathlineto{\pgfqpoint{4.781221in}{2.395881in}}%
\pgfpathlineto{\pgfqpoint{4.782306in}{2.428795in}}%
\pgfpathlineto{\pgfqpoint{4.782997in}{2.412661in}}%
\pgfpathlineto{\pgfqpoint{4.784971in}{2.363527in}}%
\pgfpathlineto{\pgfqpoint{4.785267in}{2.370092in}}%
\pgfpathlineto{\pgfqpoint{4.785465in}{2.374717in}}%
\pgfpathlineto{\pgfqpoint{4.786057in}{2.364038in}}%
\pgfpathlineto{\pgfqpoint{4.787537in}{2.334184in}}%
\pgfpathlineto{\pgfqpoint{4.787636in}{2.335153in}}%
\pgfpathlineto{\pgfqpoint{4.788130in}{2.332415in}}%
\pgfpathlineto{\pgfqpoint{4.788524in}{2.363562in}}%
\pgfpathlineto{\pgfqpoint{4.788919in}{2.394013in}}%
\pgfpathlineto{\pgfqpoint{4.789610in}{2.360383in}}%
\pgfpathlineto{\pgfqpoint{4.791091in}{2.341261in}}%
\pgfpathlineto{\pgfqpoint{4.791189in}{2.341317in}}%
\pgfpathlineto{\pgfqpoint{4.792571in}{2.390513in}}%
\pgfpathlineto{\pgfqpoint{4.793657in}{2.557463in}}%
\pgfpathlineto{\pgfqpoint{4.794644in}{2.511740in}}%
\pgfpathlineto{\pgfqpoint{4.794940in}{2.485498in}}%
\pgfpathlineto{\pgfqpoint{4.795532in}{2.331880in}}%
\pgfpathlineto{\pgfqpoint{4.796322in}{2.396518in}}%
\pgfpathlineto{\pgfqpoint{4.797309in}{2.423509in}}%
\pgfpathlineto{\pgfqpoint{4.797703in}{2.411011in}}%
\pgfpathlineto{\pgfqpoint{4.798296in}{2.451536in}}%
\pgfpathlineto{\pgfqpoint{4.798690in}{2.471163in}}%
\pgfpathlineto{\pgfqpoint{4.799085in}{2.437691in}}%
\pgfpathlineto{\pgfqpoint{4.799776in}{2.415791in}}%
\pgfpathlineto{\pgfqpoint{4.800171in}{2.431196in}}%
\pgfpathlineto{\pgfqpoint{4.801849in}{2.472168in}}%
\pgfpathlineto{\pgfqpoint{4.802046in}{2.466241in}}%
\pgfpathlineto{\pgfqpoint{4.802836in}{2.443060in}}%
\pgfpathlineto{\pgfqpoint{4.803231in}{2.454384in}}%
\pgfpathlineto{\pgfqpoint{4.803428in}{2.460001in}}%
\pgfpathlineto{\pgfqpoint{4.804020in}{2.448356in}}%
\pgfpathlineto{\pgfqpoint{4.804316in}{2.440796in}}%
\pgfpathlineto{\pgfqpoint{4.804711in}{2.462995in}}%
\pgfpathlineto{\pgfqpoint{4.804908in}{2.466291in}}%
\pgfpathlineto{\pgfqpoint{4.805599in}{2.457154in}}%
\pgfpathlineto{\pgfqpoint{4.805797in}{2.449629in}}%
\pgfpathlineto{\pgfqpoint{4.806389in}{2.469961in}}%
\pgfpathlineto{\pgfqpoint{4.807573in}{2.464618in}}%
\pgfpathlineto{\pgfqpoint{4.807080in}{2.471871in}}%
\pgfpathlineto{\pgfqpoint{4.807672in}{2.465970in}}%
\pgfpathlineto{\pgfqpoint{4.808264in}{2.481562in}}%
\pgfpathlineto{\pgfqpoint{4.808955in}{2.471846in}}%
\pgfpathlineto{\pgfqpoint{4.809152in}{2.468158in}}%
\pgfpathlineto{\pgfqpoint{4.809745in}{2.473521in}}%
\pgfpathlineto{\pgfqpoint{4.810139in}{2.492478in}}%
\pgfpathlineto{\pgfqpoint{4.810830in}{2.474601in}}%
\pgfpathlineto{\pgfqpoint{4.811126in}{2.470975in}}%
\pgfpathlineto{\pgfqpoint{4.811324in}{2.476219in}}%
\pgfpathlineto{\pgfqpoint{4.811719in}{2.493539in}}%
\pgfpathlineto{\pgfqpoint{4.812410in}{2.477509in}}%
\pgfpathlineto{\pgfqpoint{4.812607in}{2.480426in}}%
\pgfpathlineto{\pgfqpoint{4.813002in}{2.498429in}}%
\pgfpathlineto{\pgfqpoint{4.813890in}{2.495985in}}%
\pgfpathlineto{\pgfqpoint{4.814285in}{2.483456in}}%
\pgfpathlineto{\pgfqpoint{4.814976in}{2.494592in}}%
\pgfpathlineto{\pgfqpoint{4.815469in}{2.488405in}}%
\pgfpathlineto{\pgfqpoint{4.815963in}{2.493954in}}%
\pgfpathlineto{\pgfqpoint{4.816654in}{2.506520in}}%
\pgfpathlineto{\pgfqpoint{4.816851in}{2.495112in}}%
\pgfpathlineto{\pgfqpoint{4.817246in}{2.462985in}}%
\pgfpathlineto{\pgfqpoint{4.818035in}{2.482877in}}%
\pgfpathlineto{\pgfqpoint{4.820503in}{2.439492in}}%
\pgfpathlineto{\pgfqpoint{4.821490in}{2.450562in}}%
\pgfpathlineto{\pgfqpoint{4.821589in}{2.452569in}}%
\pgfpathlineto{\pgfqpoint{4.821786in}{2.442565in}}%
\pgfpathlineto{\pgfqpoint{4.822674in}{2.430799in}}%
\pgfpathlineto{\pgfqpoint{4.822872in}{2.440205in}}%
\pgfpathlineto{\pgfqpoint{4.823069in}{2.449552in}}%
\pgfpathlineto{\pgfqpoint{4.823859in}{2.434745in}}%
\pgfpathlineto{\pgfqpoint{4.823957in}{2.433607in}}%
\pgfpathlineto{\pgfqpoint{4.824253in}{2.439764in}}%
\pgfpathlineto{\pgfqpoint{4.824846in}{2.448408in}}%
\pgfpathlineto{\pgfqpoint{4.825240in}{2.438399in}}%
\pgfpathlineto{\pgfqpoint{4.825536in}{2.433801in}}%
\pgfpathlineto{\pgfqpoint{4.826030in}{2.443459in}}%
\pgfpathlineto{\pgfqpoint{4.826326in}{2.449179in}}%
\pgfpathlineto{\pgfqpoint{4.826820in}{2.435015in}}%
\pgfpathlineto{\pgfqpoint{4.827214in}{2.431058in}}%
\pgfpathlineto{\pgfqpoint{4.827609in}{2.438422in}}%
\pgfpathlineto{\pgfqpoint{4.829484in}{2.481176in}}%
\pgfpathlineto{\pgfqpoint{4.831064in}{2.490401in}}%
\pgfpathlineto{\pgfqpoint{4.831261in}{2.488434in}}%
\pgfpathlineto{\pgfqpoint{4.832149in}{2.451932in}}%
\pgfpathlineto{\pgfqpoint{4.832544in}{2.435439in}}%
\pgfpathlineto{\pgfqpoint{4.833432in}{2.442847in}}%
\pgfpathlineto{\pgfqpoint{4.833630in}{2.447446in}}%
\pgfpathlineto{\pgfqpoint{4.834321in}{2.436563in}}%
\pgfpathlineto{\pgfqpoint{4.835110in}{2.412499in}}%
\pgfpathlineto{\pgfqpoint{4.835801in}{2.407377in}}%
\pgfpathlineto{\pgfqpoint{4.835999in}{2.413309in}}%
\pgfpathlineto{\pgfqpoint{4.836986in}{2.455860in}}%
\pgfpathlineto{\pgfqpoint{4.837479in}{2.435338in}}%
\pgfpathlineto{\pgfqpoint{4.838367in}{2.406284in}}%
\pgfpathlineto{\pgfqpoint{4.839157in}{2.414264in}}%
\pgfpathlineto{\pgfqpoint{4.840835in}{2.598474in}}%
\pgfpathlineto{\pgfqpoint{4.841427in}{2.677130in}}%
\pgfpathlineto{\pgfqpoint{4.842019in}{2.645744in}}%
\pgfpathlineto{\pgfqpoint{4.842710in}{2.591908in}}%
\pgfpathlineto{\pgfqpoint{4.843401in}{2.387786in}}%
\pgfpathlineto{\pgfqpoint{4.844289in}{2.441690in}}%
\pgfpathlineto{\pgfqpoint{4.844388in}{2.441066in}}%
\pgfpathlineto{\pgfqpoint{4.844585in}{2.445867in}}%
\pgfpathlineto{\pgfqpoint{4.846757in}{2.501909in}}%
\pgfpathlineto{\pgfqpoint{4.845178in}{2.437002in}}%
\pgfpathlineto{\pgfqpoint{4.847053in}{2.479604in}}%
\pgfpathlineto{\pgfqpoint{4.848435in}{2.447669in}}%
\pgfpathlineto{\pgfqpoint{4.848731in}{2.467504in}}%
\pgfpathlineto{\pgfqpoint{4.849619in}{2.507262in}}%
\pgfpathlineto{\pgfqpoint{4.850113in}{2.496402in}}%
\pgfpathlineto{\pgfqpoint{4.850310in}{2.492419in}}%
\pgfpathlineto{\pgfqpoint{4.851790in}{2.464607in}}%
\pgfpathlineto{\pgfqpoint{4.851988in}{2.463444in}}%
\pgfpathlineto{\pgfqpoint{4.852185in}{2.466960in}}%
\pgfpathlineto{\pgfqpoint{4.852679in}{2.488391in}}%
\pgfpathlineto{\pgfqpoint{4.853370in}{2.470940in}}%
\pgfpathlineto{\pgfqpoint{4.854653in}{2.459979in}}%
\pgfpathlineto{\pgfqpoint{4.854060in}{2.471889in}}%
\pgfpathlineto{\pgfqpoint{4.854850in}{2.460887in}}%
\pgfpathlineto{\pgfqpoint{4.855837in}{2.477921in}}%
\pgfpathlineto{\pgfqpoint{4.856627in}{2.472158in}}%
\pgfpathlineto{\pgfqpoint{4.858305in}{2.443741in}}%
\pgfpathlineto{\pgfqpoint{4.858403in}{2.444291in}}%
\pgfpathlineto{\pgfqpoint{4.858995in}{2.452764in}}%
\pgfpathlineto{\pgfqpoint{4.859390in}{2.445450in}}%
\pgfpathlineto{\pgfqpoint{4.861858in}{2.412704in}}%
\pgfpathlineto{\pgfqpoint{4.859884in}{2.448427in}}%
\pgfpathlineto{\pgfqpoint{4.862055in}{2.414389in}}%
\pgfpathlineto{\pgfqpoint{4.862549in}{2.425953in}}%
\pgfpathlineto{\pgfqpoint{4.862845in}{2.413818in}}%
\pgfpathlineto{\pgfqpoint{4.864917in}{2.339263in}}%
\pgfpathlineto{\pgfqpoint{4.865016in}{2.341501in}}%
\pgfpathlineto{\pgfqpoint{4.865213in}{2.346235in}}%
\pgfpathlineto{\pgfqpoint{4.865707in}{2.330535in}}%
\pgfpathlineto{\pgfqpoint{4.866595in}{2.304930in}}%
\pgfpathlineto{\pgfqpoint{4.866891in}{2.322073in}}%
\pgfpathlineto{\pgfqpoint{4.868569in}{2.375032in}}%
\pgfpathlineto{\pgfqpoint{4.868964in}{2.393340in}}%
\pgfpathlineto{\pgfqpoint{4.869359in}{2.364037in}}%
\pgfpathlineto{\pgfqpoint{4.870148in}{2.353505in}}%
\pgfpathlineto{\pgfqpoint{4.869852in}{2.364396in}}%
\pgfpathlineto{\pgfqpoint{4.870445in}{2.362218in}}%
\pgfpathlineto{\pgfqpoint{4.870543in}{2.363108in}}%
\pgfpathlineto{\pgfqpoint{4.870642in}{2.359181in}}%
\pgfpathlineto{\pgfqpoint{4.870938in}{2.341311in}}%
\pgfpathlineto{\pgfqpoint{4.871728in}{2.349282in}}%
\pgfpathlineto{\pgfqpoint{4.872024in}{2.368123in}}%
\pgfpathlineto{\pgfqpoint{4.872813in}{2.350679in}}%
\pgfpathlineto{\pgfqpoint{4.873011in}{2.343308in}}%
\pgfpathlineto{\pgfqpoint{4.873504in}{2.361989in}}%
\pgfpathlineto{\pgfqpoint{4.873800in}{2.352867in}}%
\pgfpathlineto{\pgfqpoint{4.873998in}{2.357407in}}%
\pgfpathlineto{\pgfqpoint{4.874787in}{2.355847in}}%
\pgfpathlineto{\pgfqpoint{4.875676in}{2.389009in}}%
\pgfpathlineto{\pgfqpoint{4.875873in}{2.391139in}}%
\pgfpathlineto{\pgfqpoint{4.876169in}{2.383527in}}%
\pgfpathlineto{\pgfqpoint{4.876366in}{2.377430in}}%
\pgfpathlineto{\pgfqpoint{4.876761in}{2.402210in}}%
\pgfpathlineto{\pgfqpoint{4.878340in}{2.445841in}}%
\pgfpathlineto{\pgfqpoint{4.878439in}{2.445551in}}%
\pgfpathlineto{\pgfqpoint{4.878735in}{2.448319in}}%
\pgfpathlineto{\pgfqpoint{4.879031in}{2.443178in}}%
\pgfpathlineto{\pgfqpoint{4.880611in}{2.419904in}}%
\pgfpathlineto{\pgfqpoint{4.880907in}{2.415108in}}%
\pgfpathlineto{\pgfqpoint{4.881203in}{2.427798in}}%
\pgfpathlineto{\pgfqpoint{4.881598in}{2.451725in}}%
\pgfpathlineto{\pgfqpoint{4.882288in}{2.434452in}}%
\pgfpathlineto{\pgfqpoint{4.882782in}{2.407075in}}%
\pgfpathlineto{\pgfqpoint{4.883473in}{2.428539in}}%
\pgfpathlineto{\pgfqpoint{4.885151in}{2.483084in}}%
\pgfpathlineto{\pgfqpoint{4.885348in}{2.479822in}}%
\pgfpathlineto{\pgfqpoint{4.886532in}{2.441824in}}%
\pgfpathlineto{\pgfqpoint{4.886730in}{2.453009in}}%
\pgfpathlineto{\pgfqpoint{4.889592in}{2.722488in}}%
\pgfpathlineto{\pgfqpoint{4.890480in}{2.662294in}}%
\pgfpathlineto{\pgfqpoint{4.891369in}{2.421215in}}%
\pgfpathlineto{\pgfqpoint{4.892356in}{2.472578in}}%
\pgfpathlineto{\pgfqpoint{4.892652in}{2.464674in}}%
\pgfpathlineto{\pgfqpoint{4.893047in}{2.483381in}}%
\pgfpathlineto{\pgfqpoint{4.894330in}{2.524035in}}%
\pgfpathlineto{\pgfqpoint{4.894527in}{2.518901in}}%
\pgfpathlineto{\pgfqpoint{4.895810in}{2.460318in}}%
\pgfpathlineto{\pgfqpoint{4.896402in}{2.486921in}}%
\pgfpathlineto{\pgfqpoint{4.897587in}{2.513790in}}%
\pgfpathlineto{\pgfqpoint{4.898080in}{2.496764in}}%
\pgfpathlineto{\pgfqpoint{4.899166in}{2.471436in}}%
\pgfpathlineto{\pgfqpoint{4.899659in}{2.484530in}}%
\pgfpathlineto{\pgfqpoint{4.900153in}{2.496871in}}%
\pgfpathlineto{\pgfqpoint{4.900844in}{2.489033in}}%
\pgfpathlineto{\pgfqpoint{4.901041in}{2.484685in}}%
\pgfpathlineto{\pgfqpoint{4.901535in}{2.491900in}}%
\pgfpathlineto{\pgfqpoint{4.901831in}{2.491429in}}%
\pgfpathlineto{\pgfqpoint{4.903213in}{2.516000in}}%
\pgfpathlineto{\pgfqpoint{4.903509in}{2.506588in}}%
\pgfpathlineto{\pgfqpoint{4.903805in}{2.497605in}}%
\pgfpathlineto{\pgfqpoint{4.904200in}{2.506782in}}%
\pgfpathlineto{\pgfqpoint{4.904594in}{2.502710in}}%
\pgfpathlineto{\pgfqpoint{4.904989in}{2.519466in}}%
\pgfpathlineto{\pgfqpoint{4.905779in}{2.511039in}}%
\pgfpathlineto{\pgfqpoint{4.907161in}{2.502270in}}%
\pgfpathlineto{\pgfqpoint{4.907358in}{2.506415in}}%
\pgfpathlineto{\pgfqpoint{4.907654in}{2.517167in}}%
\pgfpathlineto{\pgfqpoint{4.908049in}{2.505707in}}%
\pgfpathlineto{\pgfqpoint{4.908542in}{2.513309in}}%
\pgfpathlineto{\pgfqpoint{4.910615in}{2.492626in}}%
\pgfpathlineto{\pgfqpoint{4.909431in}{2.515800in}}%
\pgfpathlineto{\pgfqpoint{4.910714in}{2.492689in}}%
\pgfpathlineto{\pgfqpoint{4.910812in}{2.492592in}}%
\pgfpathlineto{\pgfqpoint{4.911108in}{2.488105in}}%
\pgfpathlineto{\pgfqpoint{4.911306in}{2.496156in}}%
\pgfpathlineto{\pgfqpoint{4.911503in}{2.506859in}}%
\pgfpathlineto{\pgfqpoint{4.912095in}{2.481478in}}%
\pgfpathlineto{\pgfqpoint{4.912392in}{2.480632in}}%
\pgfpathlineto{\pgfqpoint{4.912688in}{2.483942in}}%
\pgfpathlineto{\pgfqpoint{4.912885in}{2.486011in}}%
\pgfpathlineto{\pgfqpoint{4.913181in}{2.475321in}}%
\pgfpathlineto{\pgfqpoint{4.915353in}{2.425131in}}%
\pgfpathlineto{\pgfqpoint{4.915550in}{2.427504in}}%
\pgfpathlineto{\pgfqpoint{4.916340in}{2.432078in}}%
\pgfpathlineto{\pgfqpoint{4.916043in}{2.422745in}}%
\pgfpathlineto{\pgfqpoint{4.916537in}{2.426702in}}%
\pgfpathlineto{\pgfqpoint{4.917425in}{2.408201in}}%
\pgfpathlineto{\pgfqpoint{4.917721in}{2.415724in}}%
\pgfpathlineto{\pgfqpoint{4.918314in}{2.435458in}}%
\pgfpathlineto{\pgfqpoint{4.918708in}{2.415460in}}%
\pgfpathlineto{\pgfqpoint{4.918906in}{2.410238in}}%
\pgfpathlineto{\pgfqpoint{4.919597in}{2.420646in}}%
\pgfpathlineto{\pgfqpoint{4.919991in}{2.423765in}}%
\pgfpathlineto{\pgfqpoint{4.920287in}{2.416938in}}%
\pgfpathlineto{\pgfqpoint{4.920584in}{2.403406in}}%
\pgfpathlineto{\pgfqpoint{4.921176in}{2.426424in}}%
\pgfpathlineto{\pgfqpoint{4.922261in}{2.407924in}}%
\pgfpathlineto{\pgfqpoint{4.922952in}{2.416038in}}%
\pgfpathlineto{\pgfqpoint{4.923347in}{2.422995in}}%
\pgfpathlineto{\pgfqpoint{4.923643in}{2.414890in}}%
\pgfpathlineto{\pgfqpoint{4.923742in}{2.412522in}}%
\pgfpathlineto{\pgfqpoint{4.924137in}{2.428521in}}%
\pgfpathlineto{\pgfqpoint{4.925420in}{2.450208in}}%
\pgfpathlineto{\pgfqpoint{4.925519in}{2.447814in}}%
\pgfpathlineto{\pgfqpoint{4.925716in}{2.445271in}}%
\pgfpathlineto{\pgfqpoint{4.926111in}{2.460150in}}%
\pgfpathlineto{\pgfqpoint{4.928183in}{2.406425in}}%
\pgfpathlineto{\pgfqpoint{4.928776in}{2.385275in}}%
\pgfpathlineto{\pgfqpoint{4.929170in}{2.407168in}}%
\pgfpathlineto{\pgfqpoint{4.929664in}{2.414093in}}%
\pgfpathlineto{\pgfqpoint{4.930059in}{2.401449in}}%
\pgfpathlineto{\pgfqpoint{4.930355in}{2.394962in}}%
\pgfpathlineto{\pgfqpoint{4.931144in}{2.402279in}}%
\pgfpathlineto{\pgfqpoint{4.932822in}{2.463563in}}%
\pgfpathlineto{\pgfqpoint{4.931835in}{2.399220in}}%
\pgfpathlineto{\pgfqpoint{4.933217in}{2.433321in}}%
\pgfpathlineto{\pgfqpoint{4.933711in}{2.408263in}}%
\pgfpathlineto{\pgfqpoint{4.934401in}{2.417698in}}%
\pgfpathlineto{\pgfqpoint{4.934500in}{2.418607in}}%
\pgfpathlineto{\pgfqpoint{4.934895in}{2.413291in}}%
\pgfpathlineto{\pgfqpoint{4.935092in}{2.410792in}}%
\pgfpathlineto{\pgfqpoint{4.935388in}{2.422410in}}%
\pgfpathlineto{\pgfqpoint{4.936277in}{2.535873in}}%
\pgfpathlineto{\pgfqpoint{4.937461in}{2.682522in}}%
\pgfpathlineto{\pgfqpoint{4.937955in}{2.650755in}}%
\pgfpathlineto{\pgfqpoint{4.938843in}{2.463392in}}%
\pgfpathlineto{\pgfqpoint{4.939238in}{2.397807in}}%
\pgfpathlineto{\pgfqpoint{4.939929in}{2.453661in}}%
\pgfpathlineto{\pgfqpoint{4.941409in}{2.428578in}}%
\pgfpathlineto{\pgfqpoint{4.941705in}{2.424630in}}%
\pgfpathlineto{\pgfqpoint{4.941903in}{2.431083in}}%
\pgfpathlineto{\pgfqpoint{4.942396in}{2.459121in}}%
\pgfpathlineto{\pgfqpoint{4.942890in}{2.432510in}}%
\pgfpathlineto{\pgfqpoint{4.943186in}{2.425548in}}%
\pgfpathlineto{\pgfqpoint{4.943580in}{2.438594in}}%
\pgfpathlineto{\pgfqpoint{4.945653in}{2.539206in}}%
\pgfpathlineto{\pgfqpoint{4.945752in}{2.539356in}}%
\pgfpathlineto{\pgfqpoint{4.946344in}{2.510560in}}%
\pgfpathlineto{\pgfqpoint{4.946838in}{2.481901in}}%
\pgfpathlineto{\pgfqpoint{4.947528in}{2.499502in}}%
\pgfpathlineto{\pgfqpoint{4.947923in}{2.493635in}}%
\pgfpathlineto{\pgfqpoint{4.948219in}{2.501972in}}%
\pgfpathlineto{\pgfqpoint{4.948713in}{2.519963in}}%
\pgfpathlineto{\pgfqpoint{4.949502in}{2.512122in}}%
\pgfpathlineto{\pgfqpoint{4.949897in}{2.505833in}}%
\pgfpathlineto{\pgfqpoint{4.950489in}{2.512698in}}%
\pgfpathlineto{\pgfqpoint{4.950785in}{2.518040in}}%
\pgfpathlineto{\pgfqpoint{4.951674in}{2.517744in}}%
\pgfpathlineto{\pgfqpoint{4.951871in}{2.516706in}}%
\pgfpathlineto{\pgfqpoint{4.952069in}{2.521089in}}%
\pgfpathlineto{\pgfqpoint{4.952463in}{2.539710in}}%
\pgfpathlineto{\pgfqpoint{4.953154in}{2.524457in}}%
\pgfpathlineto{\pgfqpoint{4.953450in}{2.520090in}}%
\pgfpathlineto{\pgfqpoint{4.953845in}{2.532088in}}%
\pgfpathlineto{\pgfqpoint{4.954240in}{2.524359in}}%
\pgfpathlineto{\pgfqpoint{4.955128in}{2.530471in}}%
\pgfpathlineto{\pgfqpoint{4.955622in}{2.541018in}}%
\pgfpathlineto{\pgfqpoint{4.956115in}{2.527166in}}%
\pgfpathlineto{\pgfqpoint{4.956707in}{2.508997in}}%
\pgfpathlineto{\pgfqpoint{4.957102in}{2.528143in}}%
\pgfpathlineto{\pgfqpoint{4.957201in}{2.529242in}}%
\pgfpathlineto{\pgfqpoint{4.957398in}{2.522490in}}%
\pgfpathlineto{\pgfqpoint{4.958287in}{2.507063in}}%
\pgfpathlineto{\pgfqpoint{4.957892in}{2.525793in}}%
\pgfpathlineto{\pgfqpoint{4.958583in}{2.517098in}}%
\pgfpathlineto{\pgfqpoint{4.958879in}{2.528423in}}%
\pgfpathlineto{\pgfqpoint{4.959570in}{2.513021in}}%
\pgfpathlineto{\pgfqpoint{4.961149in}{2.489883in}}%
\pgfpathlineto{\pgfqpoint{4.961642in}{2.491965in}}%
\pgfpathlineto{\pgfqpoint{4.962728in}{2.459956in}}%
\pgfpathlineto{\pgfqpoint{4.963419in}{2.448091in}}%
\pgfpathlineto{\pgfqpoint{4.963715in}{2.457631in}}%
\pgfpathlineto{\pgfqpoint{4.964011in}{2.470876in}}%
\pgfpathlineto{\pgfqpoint{4.964505in}{2.439189in}}%
\pgfpathlineto{\pgfqpoint{4.965196in}{2.434480in}}%
\pgfpathlineto{\pgfqpoint{4.965393in}{2.440008in}}%
\pgfpathlineto{\pgfqpoint{4.965492in}{2.442108in}}%
\pgfpathlineto{\pgfqpoint{4.965886in}{2.430519in}}%
\pgfpathlineto{\pgfqpoint{4.965985in}{2.428695in}}%
\pgfpathlineto{\pgfqpoint{4.966380in}{2.431839in}}%
\pgfpathlineto{\pgfqpoint{4.966873in}{2.429688in}}%
\pgfpathlineto{\pgfqpoint{4.967268in}{2.438056in}}%
\pgfpathlineto{\pgfqpoint{4.967663in}{2.426536in}}%
\pgfpathlineto{\pgfqpoint{4.967762in}{2.425299in}}%
\pgfpathlineto{\pgfqpoint{4.968058in}{2.430885in}}%
\pgfpathlineto{\pgfqpoint{4.968354in}{2.430377in}}%
\pgfpathlineto{\pgfqpoint{4.968749in}{2.440819in}}%
\pgfpathlineto{\pgfqpoint{4.969045in}{2.420895in}}%
\pgfpathlineto{\pgfqpoint{4.969834in}{2.405745in}}%
\pgfpathlineto{\pgfqpoint{4.970130in}{2.415880in}}%
\pgfpathlineto{\pgfqpoint{4.971117in}{2.433146in}}%
\pgfpathlineto{\pgfqpoint{4.971512in}{2.428375in}}%
\pgfpathlineto{\pgfqpoint{4.971611in}{2.428300in}}%
\pgfpathlineto{\pgfqpoint{4.971710in}{2.429585in}}%
\pgfpathlineto{\pgfqpoint{4.973486in}{2.469232in}}%
\pgfpathlineto{\pgfqpoint{4.973782in}{2.466546in}}%
\pgfpathlineto{\pgfqpoint{4.974078in}{2.466699in}}%
\pgfpathlineto{\pgfqpoint{4.974177in}{2.465861in}}%
\pgfpathlineto{\pgfqpoint{4.974868in}{2.453915in}}%
\pgfpathlineto{\pgfqpoint{4.976250in}{2.397036in}}%
\pgfpathlineto{\pgfqpoint{4.976842in}{2.410325in}}%
\pgfpathlineto{\pgfqpoint{4.977829in}{2.419693in}}%
\pgfpathlineto{\pgfqpoint{4.978026in}{2.416069in}}%
\pgfpathlineto{\pgfqpoint{4.979211in}{2.385908in}}%
\pgfpathlineto{\pgfqpoint{4.979704in}{2.387769in}}%
\pgfpathlineto{\pgfqpoint{4.980000in}{2.397505in}}%
\pgfpathlineto{\pgfqpoint{4.980494in}{2.439424in}}%
\pgfpathlineto{\pgfqpoint{4.981283in}{2.419757in}}%
\pgfpathlineto{\pgfqpoint{4.982172in}{2.381690in}}%
\pgfpathlineto{\pgfqpoint{4.982567in}{2.397533in}}%
\pgfpathlineto{\pgfqpoint{4.983060in}{2.388036in}}%
\pgfpathlineto{\pgfqpoint{4.984442in}{2.480540in}}%
\pgfpathlineto{\pgfqpoint{4.985527in}{2.648942in}}%
\pgfpathlineto{\pgfqpoint{4.986021in}{2.619059in}}%
\pgfpathlineto{\pgfqpoint{4.986613in}{2.577840in}}%
\pgfpathlineto{\pgfqpoint{4.987403in}{2.357081in}}%
\pgfpathlineto{\pgfqpoint{4.988291in}{2.427585in}}%
\pgfpathlineto{\pgfqpoint{4.988883in}{2.424384in}}%
\pgfpathlineto{\pgfqpoint{4.989870in}{2.461862in}}%
\pgfpathlineto{\pgfqpoint{4.990462in}{2.489610in}}%
\pgfpathlineto{\pgfqpoint{4.990956in}{2.462428in}}%
\pgfpathlineto{\pgfqpoint{4.991449in}{2.438792in}}%
\pgfpathlineto{\pgfqpoint{4.992239in}{2.448798in}}%
\pgfpathlineto{\pgfqpoint{4.992634in}{2.445225in}}%
\pgfpathlineto{\pgfqpoint{4.992831in}{2.451758in}}%
\pgfpathlineto{\pgfqpoint{4.993720in}{2.484417in}}%
\pgfpathlineto{\pgfqpoint{4.994016in}{2.464114in}}%
\pgfpathlineto{\pgfqpoint{4.994410in}{2.430748in}}%
\pgfpathlineto{\pgfqpoint{4.995200in}{2.442445in}}%
\pgfpathlineto{\pgfqpoint{4.996582in}{2.458973in}}%
\pgfpathlineto{\pgfqpoint{4.995693in}{2.437823in}}%
\pgfpathlineto{\pgfqpoint{4.997273in}{2.451600in}}%
\pgfpathlineto{\pgfqpoint{4.997470in}{2.451034in}}%
\pgfpathlineto{\pgfqpoint{4.997766in}{2.453901in}}%
\pgfpathlineto{\pgfqpoint{5.000431in}{2.490411in}}%
\pgfpathlineto{\pgfqpoint{5.000530in}{2.488579in}}%
\pgfpathlineto{\pgfqpoint{5.000925in}{2.470281in}}%
\pgfpathlineto{\pgfqpoint{5.001418in}{2.495491in}}%
\pgfpathlineto{\pgfqpoint{5.001517in}{2.494135in}}%
\pgfpathlineto{\pgfqpoint{5.004083in}{2.448899in}}%
\pgfpathlineto{\pgfqpoint{5.004576in}{2.464231in}}%
\pgfpathlineto{\pgfqpoint{5.004971in}{2.472771in}}%
\pgfpathlineto{\pgfqpoint{5.005267in}{2.457144in}}%
\pgfpathlineto{\pgfqpoint{5.006057in}{2.450980in}}%
\pgfpathlineto{\pgfqpoint{5.005761in}{2.459354in}}%
\pgfpathlineto{\pgfqpoint{5.006254in}{2.455477in}}%
\pgfpathlineto{\pgfqpoint{5.006550in}{2.466097in}}%
\pgfpathlineto{\pgfqpoint{5.007143in}{2.450407in}}%
\pgfpathlineto{\pgfqpoint{5.007340in}{2.447543in}}%
\pgfpathlineto{\pgfqpoint{5.007735in}{2.457353in}}%
\pgfpathlineto{\pgfqpoint{5.008031in}{2.452699in}}%
\pgfpathlineto{\pgfqpoint{5.008327in}{2.458836in}}%
\pgfpathlineto{\pgfqpoint{5.008623in}{2.446482in}}%
\pgfpathlineto{\pgfqpoint{5.008919in}{2.435515in}}%
\pgfpathlineto{\pgfqpoint{5.009511in}{2.446698in}}%
\pgfpathlineto{\pgfqpoint{5.009709in}{2.444527in}}%
\pgfpathlineto{\pgfqpoint{5.010696in}{2.425281in}}%
\pgfpathlineto{\pgfqpoint{5.010992in}{2.437213in}}%
\pgfpathlineto{\pgfqpoint{5.011189in}{2.443490in}}%
\pgfpathlineto{\pgfqpoint{5.011880in}{2.427966in}}%
\pgfpathlineto{\pgfqpoint{5.014940in}{2.369438in}}%
\pgfpathlineto{\pgfqpoint{5.016420in}{2.322594in}}%
\pgfpathlineto{\pgfqpoint{5.016519in}{2.322885in}}%
\pgfpathlineto{\pgfqpoint{5.017802in}{2.384725in}}%
\pgfpathlineto{\pgfqpoint{5.019579in}{2.460823in}}%
\pgfpathlineto{\pgfqpoint{5.020960in}{2.488152in}}%
\pgfpathlineto{\pgfqpoint{5.022638in}{2.543164in}}%
\pgfpathlineto{\pgfqpoint{5.022836in}{2.538344in}}%
\pgfpathlineto{\pgfqpoint{5.025007in}{2.458935in}}%
\pgfpathlineto{\pgfqpoint{5.025698in}{2.474078in}}%
\pgfpathlineto{\pgfqpoint{5.026290in}{2.463130in}}%
\pgfpathlineto{\pgfqpoint{5.027672in}{2.442094in}}%
\pgfpathlineto{\pgfqpoint{5.027771in}{2.443482in}}%
\pgfpathlineto{\pgfqpoint{5.029054in}{2.486623in}}%
\pgfpathlineto{\pgfqpoint{5.029350in}{2.470240in}}%
\pgfpathlineto{\pgfqpoint{5.030436in}{2.418847in}}%
\pgfpathlineto{\pgfqpoint{5.030732in}{2.432529in}}%
\pgfpathlineto{\pgfqpoint{5.030929in}{2.438388in}}%
\pgfpathlineto{\pgfqpoint{5.031620in}{2.424656in}}%
\pgfpathlineto{\pgfqpoint{5.031719in}{2.424148in}}%
\pgfpathlineto{\pgfqpoint{5.031817in}{2.425789in}}%
\pgfpathlineto{\pgfqpoint{5.032311in}{2.502525in}}%
\pgfpathlineto{\pgfqpoint{5.033396in}{2.661956in}}%
\pgfpathlineto{\pgfqpoint{5.033890in}{2.633567in}}%
\pgfpathlineto{\pgfqpoint{5.034778in}{2.503392in}}%
\pgfpathlineto{\pgfqpoint{5.035370in}{2.369022in}}%
\pgfpathlineto{\pgfqpoint{5.036061in}{2.430073in}}%
\pgfpathlineto{\pgfqpoint{5.037048in}{2.406763in}}%
\pgfpathlineto{\pgfqpoint{5.037443in}{2.415455in}}%
\pgfpathlineto{\pgfqpoint{5.038134in}{2.450779in}}%
\pgfpathlineto{\pgfqpoint{5.039022in}{2.429413in}}%
\pgfpathlineto{\pgfqpoint{5.040108in}{2.394190in}}%
\pgfpathlineto{\pgfqpoint{5.040404in}{2.399037in}}%
\pgfpathlineto{\pgfqpoint{5.041588in}{2.430821in}}%
\pgfpathlineto{\pgfqpoint{5.042181in}{2.424242in}}%
\pgfpathlineto{\pgfqpoint{5.043266in}{2.392320in}}%
\pgfpathlineto{\pgfqpoint{5.043859in}{2.394813in}}%
\pgfpathlineto{\pgfqpoint{5.044056in}{2.401042in}}%
\pgfpathlineto{\pgfqpoint{5.044451in}{2.426789in}}%
\pgfpathlineto{\pgfqpoint{5.045339in}{2.417317in}}%
\pgfpathlineto{\pgfqpoint{5.045536in}{2.421526in}}%
\pgfpathlineto{\pgfqpoint{5.045833in}{2.432331in}}%
\pgfpathlineto{\pgfqpoint{5.046227in}{2.417065in}}%
\pgfpathlineto{\pgfqpoint{5.046721in}{2.426489in}}%
\pgfpathlineto{\pgfqpoint{5.046918in}{2.418536in}}%
\pgfpathlineto{\pgfqpoint{5.047412in}{2.437705in}}%
\pgfpathlineto{\pgfqpoint{5.047609in}{2.437366in}}%
\pgfpathlineto{\pgfqpoint{5.048596in}{2.456363in}}%
\pgfpathlineto{\pgfqpoint{5.048892in}{2.466557in}}%
\pgfpathlineto{\pgfqpoint{5.049386in}{2.450483in}}%
\pgfpathlineto{\pgfqpoint{5.049583in}{2.452382in}}%
\pgfpathlineto{\pgfqpoint{5.049682in}{2.452600in}}%
\pgfpathlineto{\pgfqpoint{5.049781in}{2.451314in}}%
\pgfpathlineto{\pgfqpoint{5.049978in}{2.446996in}}%
\pgfpathlineto{\pgfqpoint{5.050373in}{2.463342in}}%
\pgfpathlineto{\pgfqpoint{5.050570in}{2.468539in}}%
\pgfpathlineto{\pgfqpoint{5.051360in}{2.461231in}}%
\pgfpathlineto{\pgfqpoint{5.051458in}{2.460336in}}%
\pgfpathlineto{\pgfqpoint{5.051656in}{2.466115in}}%
\pgfpathlineto{\pgfqpoint{5.051952in}{2.480754in}}%
\pgfpathlineto{\pgfqpoint{5.052840in}{2.475118in}}%
\pgfpathlineto{\pgfqpoint{5.053334in}{2.462611in}}%
\pgfpathlineto{\pgfqpoint{5.053728in}{2.476340in}}%
\pgfpathlineto{\pgfqpoint{5.054321in}{2.489553in}}%
\pgfpathlineto{\pgfqpoint{5.054715in}{2.477301in}}%
\pgfpathlineto{\pgfqpoint{5.055012in}{2.464692in}}%
\pgfpathlineto{\pgfqpoint{5.055505in}{2.484643in}}%
\pgfpathlineto{\pgfqpoint{5.055801in}{2.477540in}}%
\pgfpathlineto{\pgfqpoint{5.056295in}{2.462303in}}%
\pgfpathlineto{\pgfqpoint{5.056887in}{2.475568in}}%
\pgfpathlineto{\pgfqpoint{5.057183in}{2.483253in}}%
\pgfpathlineto{\pgfqpoint{5.057775in}{2.469916in}}%
\pgfpathlineto{\pgfqpoint{5.058170in}{2.467829in}}%
\pgfpathlineto{\pgfqpoint{5.058466in}{2.471352in}}%
\pgfpathlineto{\pgfqpoint{5.058762in}{2.470127in}}%
\pgfpathlineto{\pgfqpoint{5.058960in}{2.471364in}}%
\pgfpathlineto{\pgfqpoint{5.059157in}{2.469328in}}%
\pgfpathlineto{\pgfqpoint{5.059453in}{2.461336in}}%
\pgfpathlineto{\pgfqpoint{5.060045in}{2.473391in}}%
\pgfpathlineto{\pgfqpoint{5.060637in}{2.481435in}}%
\pgfpathlineto{\pgfqpoint{5.061328in}{2.475132in}}%
\pgfpathlineto{\pgfqpoint{5.061723in}{2.461123in}}%
\pgfpathlineto{\pgfqpoint{5.062315in}{2.478057in}}%
\pgfpathlineto{\pgfqpoint{5.063204in}{2.458822in}}%
\pgfpathlineto{\pgfqpoint{5.064092in}{2.463046in}}%
\pgfpathlineto{\pgfqpoint{5.064289in}{2.461144in}}%
\pgfpathlineto{\pgfqpoint{5.066263in}{2.411634in}}%
\pgfpathlineto{\pgfqpoint{5.066461in}{2.420325in}}%
\pgfpathlineto{\pgfqpoint{5.066855in}{2.444782in}}%
\pgfpathlineto{\pgfqpoint{5.067744in}{2.438301in}}%
\pgfpathlineto{\pgfqpoint{5.067941in}{2.435061in}}%
\pgfpathlineto{\pgfqpoint{5.068336in}{2.449307in}}%
\pgfpathlineto{\pgfqpoint{5.068533in}{2.447034in}}%
\pgfpathlineto{\pgfqpoint{5.068632in}{2.447259in}}%
\pgfpathlineto{\pgfqpoint{5.070409in}{2.485659in}}%
\pgfpathlineto{\pgfqpoint{5.070606in}{2.477529in}}%
\pgfpathlineto{\pgfqpoint{5.072481in}{2.396865in}}%
\pgfpathlineto{\pgfqpoint{5.072580in}{2.396138in}}%
\pgfpathlineto{\pgfqpoint{5.073073in}{2.399774in}}%
\pgfpathlineto{\pgfqpoint{5.073567in}{2.403318in}}%
\pgfpathlineto{\pgfqpoint{5.073863in}{2.399462in}}%
\pgfpathlineto{\pgfqpoint{5.074751in}{2.360546in}}%
\pgfpathlineto{\pgfqpoint{5.075640in}{2.368137in}}%
\pgfpathlineto{\pgfqpoint{5.075936in}{2.365474in}}%
\pgfpathlineto{\pgfqpoint{5.076133in}{2.371112in}}%
\pgfpathlineto{\pgfqpoint{5.076824in}{2.416823in}}%
\pgfpathlineto{\pgfqpoint{5.077318in}{2.384673in}}%
\pgfpathlineto{\pgfqpoint{5.077811in}{2.344193in}}%
\pgfpathlineto{\pgfqpoint{5.078601in}{2.360291in}}%
\pgfpathlineto{\pgfqpoint{5.078798in}{2.357022in}}%
\pgfpathlineto{\pgfqpoint{5.079588in}{2.360350in}}%
\pgfpathlineto{\pgfqpoint{5.080377in}{2.454079in}}%
\pgfpathlineto{\pgfqpoint{5.081660in}{2.608544in}}%
\pgfpathlineto{\pgfqpoint{5.082055in}{2.576918in}}%
\pgfpathlineto{\pgfqpoint{5.082845in}{2.438596in}}%
\pgfpathlineto{\pgfqpoint{5.083338in}{2.325589in}}%
\pgfpathlineto{\pgfqpoint{5.084128in}{2.368526in}}%
\pgfpathlineto{\pgfqpoint{5.084720in}{2.374291in}}%
\pgfpathlineto{\pgfqpoint{5.086497in}{2.451218in}}%
\pgfpathlineto{\pgfqpoint{5.086891in}{2.434605in}}%
\pgfpathlineto{\pgfqpoint{5.087878in}{2.383911in}}%
\pgfpathlineto{\pgfqpoint{5.088273in}{2.398025in}}%
\pgfpathlineto{\pgfqpoint{5.088470in}{2.393559in}}%
\pgfpathlineto{\pgfqpoint{5.088865in}{2.368443in}}%
\pgfpathlineto{\pgfqpoint{5.089556in}{2.389710in}}%
\pgfpathlineto{\pgfqpoint{5.089754in}{2.388195in}}%
\pgfpathlineto{\pgfqpoint{5.090444in}{2.359582in}}%
\pgfpathlineto{\pgfqpoint{5.090938in}{2.385106in}}%
\pgfpathlineto{\pgfqpoint{5.093109in}{2.462941in}}%
\pgfpathlineto{\pgfqpoint{5.093504in}{2.451406in}}%
\pgfpathlineto{\pgfqpoint{5.094096in}{2.437595in}}%
\pgfpathlineto{\pgfqpoint{5.094392in}{2.451974in}}%
\pgfpathlineto{\pgfqpoint{5.095676in}{2.459669in}}%
\pgfpathlineto{\pgfqpoint{5.096959in}{2.463131in}}%
\pgfpathlineto{\pgfqpoint{5.096070in}{2.457029in}}%
\pgfpathlineto{\pgfqpoint{5.097057in}{2.461933in}}%
\pgfpathlineto{\pgfqpoint{5.097452in}{2.450874in}}%
\pgfpathlineto{\pgfqpoint{5.097847in}{2.470867in}}%
\pgfpathlineto{\pgfqpoint{5.098439in}{2.476182in}}%
\pgfpathlineto{\pgfqpoint{5.098143in}{2.470698in}}%
\pgfpathlineto{\pgfqpoint{5.098636in}{2.471183in}}%
\pgfpathlineto{\pgfqpoint{5.098933in}{2.459182in}}%
\pgfpathlineto{\pgfqpoint{5.099722in}{2.466100in}}%
\pgfpathlineto{\pgfqpoint{5.100018in}{2.473201in}}%
\pgfpathlineto{\pgfqpoint{5.100808in}{2.465422in}}%
\pgfpathlineto{\pgfqpoint{5.100907in}{2.464301in}}%
\pgfpathlineto{\pgfqpoint{5.101203in}{2.469949in}}%
\pgfpathlineto{\pgfqpoint{5.101499in}{2.485813in}}%
\pgfpathlineto{\pgfqpoint{5.101992in}{2.460862in}}%
\pgfpathlineto{\pgfqpoint{5.102190in}{2.462363in}}%
\pgfpathlineto{\pgfqpoint{5.102979in}{2.479727in}}%
\pgfpathlineto{\pgfqpoint{5.103868in}{2.470734in}}%
\pgfpathlineto{\pgfqpoint{5.104657in}{2.479355in}}%
\pgfpathlineto{\pgfqpoint{5.104756in}{2.480866in}}%
\pgfpathlineto{\pgfqpoint{5.105052in}{2.473328in}}%
\pgfpathlineto{\pgfqpoint{5.106631in}{2.450368in}}%
\pgfpathlineto{\pgfqpoint{5.108901in}{2.415820in}}%
\pgfpathlineto{\pgfqpoint{5.107125in}{2.457954in}}%
\pgfpathlineto{\pgfqpoint{5.109099in}{2.418537in}}%
\pgfpathlineto{\pgfqpoint{5.109789in}{2.425161in}}%
\pgfpathlineto{\pgfqpoint{5.110086in}{2.418545in}}%
\pgfpathlineto{\pgfqpoint{5.110382in}{2.413573in}}%
\pgfpathlineto{\pgfqpoint{5.110974in}{2.421138in}}%
\pgfpathlineto{\pgfqpoint{5.111073in}{2.420904in}}%
\pgfpathlineto{\pgfqpoint{5.112356in}{2.392051in}}%
\pgfpathlineto{\pgfqpoint{5.112750in}{2.411594in}}%
\pgfpathlineto{\pgfqpoint{5.112849in}{2.414004in}}%
\pgfpathlineto{\pgfqpoint{5.113244in}{2.399119in}}%
\pgfpathlineto{\pgfqpoint{5.114132in}{2.388407in}}%
\pgfpathlineto{\pgfqpoint{5.113540in}{2.400023in}}%
\pgfpathlineto{\pgfqpoint{5.114330in}{2.394827in}}%
\pgfpathlineto{\pgfqpoint{5.115119in}{2.389518in}}%
\pgfpathlineto{\pgfqpoint{5.115613in}{2.407521in}}%
\pgfpathlineto{\pgfqpoint{5.117981in}{2.449040in}}%
\pgfpathlineto{\pgfqpoint{5.118278in}{2.429847in}}%
\pgfpathlineto{\pgfqpoint{5.119561in}{2.409873in}}%
\pgfpathlineto{\pgfqpoint{5.119758in}{2.408544in}}%
\pgfpathlineto{\pgfqpoint{5.120350in}{2.385181in}}%
\pgfpathlineto{\pgfqpoint{5.121041in}{2.395203in}}%
\pgfpathlineto{\pgfqpoint{5.121337in}{2.406844in}}%
\pgfpathlineto{\pgfqpoint{5.121831in}{2.378568in}}%
\pgfpathlineto{\pgfqpoint{5.122127in}{2.365674in}}%
\pgfpathlineto{\pgfqpoint{5.122620in}{2.381009in}}%
\pgfpathlineto{\pgfqpoint{5.123015in}{2.371929in}}%
\pgfpathlineto{\pgfqpoint{5.123410in}{2.369536in}}%
\pgfpathlineto{\pgfqpoint{5.123607in}{2.372589in}}%
\pgfpathlineto{\pgfqpoint{5.124594in}{2.410914in}}%
\pgfpathlineto{\pgfqpoint{5.124890in}{2.388812in}}%
\pgfpathlineto{\pgfqpoint{5.125581in}{2.365515in}}%
\pgfpathlineto{\pgfqpoint{5.126075in}{2.376603in}}%
\pgfpathlineto{\pgfqpoint{5.126173in}{2.377755in}}%
\pgfpathlineto{\pgfqpoint{5.126470in}{2.369457in}}%
\pgfpathlineto{\pgfqpoint{5.126864in}{2.357432in}}%
\pgfpathlineto{\pgfqpoint{5.127259in}{2.375461in}}%
\pgfpathlineto{\pgfqpoint{5.128542in}{2.575211in}}%
\pgfpathlineto{\pgfqpoint{5.129036in}{2.640574in}}%
\pgfpathlineto{\pgfqpoint{5.129727in}{2.594240in}}%
\pgfpathlineto{\pgfqpoint{5.130418in}{2.487993in}}%
\pgfpathlineto{\pgfqpoint{5.131010in}{2.334854in}}%
\pgfpathlineto{\pgfqpoint{5.131799in}{2.388329in}}%
\pgfpathlineto{\pgfqpoint{5.132095in}{2.380235in}}%
\pgfpathlineto{\pgfqpoint{5.132490in}{2.406415in}}%
\pgfpathlineto{\pgfqpoint{5.132589in}{2.408144in}}%
\pgfpathlineto{\pgfqpoint{5.132885in}{2.395673in}}%
\pgfpathlineto{\pgfqpoint{5.132984in}{2.394514in}}%
\pgfpathlineto{\pgfqpoint{5.133181in}{2.400838in}}%
\pgfpathlineto{\pgfqpoint{5.134366in}{2.449266in}}%
\pgfpathlineto{\pgfqpoint{5.134662in}{2.428981in}}%
\pgfpathlineto{\pgfqpoint{5.135254in}{2.402339in}}%
\pgfpathlineto{\pgfqpoint{5.135747in}{2.417002in}}%
\pgfpathlineto{\pgfqpoint{5.136043in}{2.424284in}}%
\pgfpathlineto{\pgfqpoint{5.136537in}{2.404774in}}%
\pgfpathlineto{\pgfqpoint{5.136932in}{2.426514in}}%
\pgfpathlineto{\pgfqpoint{5.137326in}{2.451563in}}%
\pgfpathlineto{\pgfqpoint{5.138017in}{2.430308in}}%
\pgfpathlineto{\pgfqpoint{5.138412in}{2.412082in}}%
\pgfpathlineto{\pgfqpoint{5.139202in}{2.426733in}}%
\pgfpathlineto{\pgfqpoint{5.141176in}{2.453744in}}%
\pgfpathlineto{\pgfqpoint{5.139794in}{2.418440in}}%
\pgfpathlineto{\pgfqpoint{5.141373in}{2.445953in}}%
\pgfpathlineto{\pgfqpoint{5.141669in}{2.427773in}}%
\pgfpathlineto{\pgfqpoint{5.142163in}{2.452518in}}%
\pgfpathlineto{\pgfqpoint{5.142459in}{2.446765in}}%
\pgfpathlineto{\pgfqpoint{5.142656in}{2.448037in}}%
\pgfpathlineto{\pgfqpoint{5.144137in}{2.474067in}}%
\pgfpathlineto{\pgfqpoint{5.144433in}{2.464974in}}%
\pgfpathlineto{\pgfqpoint{5.144828in}{2.456051in}}%
\pgfpathlineto{\pgfqpoint{5.145518in}{2.465884in}}%
\pgfpathlineto{\pgfqpoint{5.145913in}{2.470367in}}%
\pgfpathlineto{\pgfqpoint{5.146308in}{2.463797in}}%
\pgfpathlineto{\pgfqpoint{5.146900in}{2.468763in}}%
\pgfpathlineto{\pgfqpoint{5.146999in}{2.468697in}}%
\pgfpathlineto{\pgfqpoint{5.147098in}{2.469715in}}%
\pgfpathlineto{\pgfqpoint{5.147591in}{2.478304in}}%
\pgfpathlineto{\pgfqpoint{5.148085in}{2.469514in}}%
\pgfpathlineto{\pgfqpoint{5.148973in}{2.461667in}}%
\pgfpathlineto{\pgfqpoint{5.149170in}{2.466127in}}%
\pgfpathlineto{\pgfqpoint{5.150355in}{2.481843in}}%
\pgfpathlineto{\pgfqpoint{5.149763in}{2.454685in}}%
\pgfpathlineto{\pgfqpoint{5.150552in}{2.476679in}}%
\pgfpathlineto{\pgfqpoint{5.153118in}{2.446705in}}%
\pgfpathlineto{\pgfqpoint{5.151046in}{2.480505in}}%
\pgfpathlineto{\pgfqpoint{5.153316in}{2.450645in}}%
\pgfpathlineto{\pgfqpoint{5.153612in}{2.459552in}}%
\pgfpathlineto{\pgfqpoint{5.154105in}{2.444509in}}%
\pgfpathlineto{\pgfqpoint{5.155487in}{2.427276in}}%
\pgfpathlineto{\pgfqpoint{5.155684in}{2.429602in}}%
\pgfpathlineto{\pgfqpoint{5.155981in}{2.434960in}}%
\pgfpathlineto{\pgfqpoint{5.156375in}{2.417880in}}%
\pgfpathlineto{\pgfqpoint{5.156474in}{2.415128in}}%
\pgfpathlineto{\pgfqpoint{5.156770in}{2.431513in}}%
\pgfpathlineto{\pgfqpoint{5.156968in}{2.445702in}}%
\pgfpathlineto{\pgfqpoint{5.157461in}{2.422921in}}%
\pgfpathlineto{\pgfqpoint{5.157757in}{2.424267in}}%
\pgfpathlineto{\pgfqpoint{5.158152in}{2.404230in}}%
\pgfpathlineto{\pgfqpoint{5.159040in}{2.413078in}}%
\pgfpathlineto{\pgfqpoint{5.160521in}{2.424438in}}%
\pgfpathlineto{\pgfqpoint{5.160718in}{2.417674in}}%
\pgfpathlineto{\pgfqpoint{5.161508in}{2.403494in}}%
\pgfpathlineto{\pgfqpoint{5.161804in}{2.414367in}}%
\pgfpathlineto{\pgfqpoint{5.162001in}{2.418036in}}%
\pgfpathlineto{\pgfqpoint{5.162396in}{2.403523in}}%
\pgfpathlineto{\pgfqpoint{5.162593in}{2.404101in}}%
\pgfpathlineto{\pgfqpoint{5.162890in}{2.395708in}}%
\pgfpathlineto{\pgfqpoint{5.164469in}{2.357959in}}%
\pgfpathlineto{\pgfqpoint{5.164765in}{2.365171in}}%
\pgfpathlineto{\pgfqpoint{5.167232in}{2.429075in}}%
\pgfpathlineto{\pgfqpoint{5.167331in}{2.427046in}}%
\pgfpathlineto{\pgfqpoint{5.168121in}{2.392667in}}%
\pgfpathlineto{\pgfqpoint{5.169108in}{2.397366in}}%
\pgfpathlineto{\pgfqpoint{5.169798in}{2.370287in}}%
\pgfpathlineto{\pgfqpoint{5.170588in}{2.375300in}}%
\pgfpathlineto{\pgfqpoint{5.170983in}{2.362344in}}%
\pgfpathlineto{\pgfqpoint{5.171575in}{2.391927in}}%
\pgfpathlineto{\pgfqpoint{5.172167in}{2.434725in}}%
\pgfpathlineto{\pgfqpoint{5.172759in}{2.403753in}}%
\pgfpathlineto{\pgfqpoint{5.173648in}{2.368001in}}%
\pgfpathlineto{\pgfqpoint{5.174240in}{2.371763in}}%
\pgfpathlineto{\pgfqpoint{5.174339in}{2.369600in}}%
\pgfpathlineto{\pgfqpoint{5.174536in}{2.378844in}}%
\pgfpathlineto{\pgfqpoint{5.176806in}{2.663389in}}%
\pgfpathlineto{\pgfqpoint{5.177793in}{2.604724in}}%
\pgfpathlineto{\pgfqpoint{5.178780in}{2.388278in}}%
\pgfpathlineto{\pgfqpoint{5.180162in}{2.442720in}}%
\pgfpathlineto{\pgfqpoint{5.181938in}{2.515912in}}%
\pgfpathlineto{\pgfqpoint{5.182333in}{2.488658in}}%
\pgfpathlineto{\pgfqpoint{5.182827in}{2.455462in}}%
\pgfpathlineto{\pgfqpoint{5.183518in}{2.480825in}}%
\pgfpathlineto{\pgfqpoint{5.183616in}{2.480829in}}%
\pgfpathlineto{\pgfqpoint{5.184801in}{2.527390in}}%
\pgfpathlineto{\pgfqpoint{5.185590in}{2.506726in}}%
\pgfpathlineto{\pgfqpoint{5.186084in}{2.487953in}}%
\pgfpathlineto{\pgfqpoint{5.186577in}{2.506074in}}%
\pgfpathlineto{\pgfqpoint{5.188156in}{2.547881in}}%
\pgfpathlineto{\pgfqpoint{5.187071in}{2.504823in}}%
\pgfpathlineto{\pgfqpoint{5.188551in}{2.529960in}}%
\pgfpathlineto{\pgfqpoint{5.189440in}{2.511918in}}%
\pgfpathlineto{\pgfqpoint{5.189736in}{2.527331in}}%
\pgfpathlineto{\pgfqpoint{5.190032in}{2.544515in}}%
\pgfpathlineto{\pgfqpoint{5.190920in}{2.540506in}}%
\pgfpathlineto{\pgfqpoint{5.191117in}{2.544483in}}%
\pgfpathlineto{\pgfqpoint{5.191512in}{2.565766in}}%
\pgfpathlineto{\pgfqpoint{5.192302in}{2.554016in}}%
\pgfpathlineto{\pgfqpoint{5.193387in}{2.549044in}}%
\pgfpathlineto{\pgfqpoint{5.192993in}{2.557292in}}%
\pgfpathlineto{\pgfqpoint{5.193486in}{2.549835in}}%
\pgfpathlineto{\pgfqpoint{5.194868in}{2.573628in}}%
\pgfpathlineto{\pgfqpoint{5.195065in}{2.566130in}}%
\pgfpathlineto{\pgfqpoint{5.195460in}{2.547388in}}%
\pgfpathlineto{\pgfqpoint{5.196151in}{2.560834in}}%
\pgfpathlineto{\pgfqpoint{5.196348in}{2.562900in}}%
\pgfpathlineto{\pgfqpoint{5.196743in}{2.576297in}}%
\pgfpathlineto{\pgfqpoint{5.197434in}{2.563417in}}%
\pgfpathlineto{\pgfqpoint{5.198421in}{2.555178in}}%
\pgfpathlineto{\pgfqpoint{5.198816in}{2.539035in}}%
\pgfpathlineto{\pgfqpoint{5.199507in}{2.551997in}}%
\pgfpathlineto{\pgfqpoint{5.199803in}{2.560739in}}%
\pgfpathlineto{\pgfqpoint{5.200296in}{2.545629in}}%
\pgfpathlineto{\pgfqpoint{5.205626in}{2.465492in}}%
\pgfpathlineto{\pgfqpoint{5.205824in}{2.471578in}}%
\pgfpathlineto{\pgfqpoint{5.206218in}{2.488873in}}%
\pgfpathlineto{\pgfqpoint{5.206712in}{2.470328in}}%
\pgfpathlineto{\pgfqpoint{5.207107in}{2.481749in}}%
\pgfpathlineto{\pgfqpoint{5.207304in}{2.479042in}}%
\pgfpathlineto{\pgfqpoint{5.207699in}{2.469491in}}%
\pgfpathlineto{\pgfqpoint{5.208192in}{2.482634in}}%
\pgfpathlineto{\pgfqpoint{5.208291in}{2.484482in}}%
\pgfpathlineto{\pgfqpoint{5.208488in}{2.475610in}}%
\pgfpathlineto{\pgfqpoint{5.208686in}{2.463260in}}%
\pgfpathlineto{\pgfqpoint{5.209278in}{2.485133in}}%
\pgfpathlineto{\pgfqpoint{5.209377in}{2.484773in}}%
\pgfpathlineto{\pgfqpoint{5.209673in}{2.488098in}}%
\pgfpathlineto{\pgfqpoint{5.209870in}{2.482920in}}%
\pgfpathlineto{\pgfqpoint{5.210166in}{2.474555in}}%
\pgfpathlineto{\pgfqpoint{5.210561in}{2.486486in}}%
\pgfpathlineto{\pgfqpoint{5.210956in}{2.483963in}}%
\pgfpathlineto{\pgfqpoint{5.211153in}{2.483360in}}%
\pgfpathlineto{\pgfqpoint{5.211351in}{2.485387in}}%
\pgfpathlineto{\pgfqpoint{5.213127in}{2.533356in}}%
\pgfpathlineto{\pgfqpoint{5.213325in}{2.527400in}}%
\pgfpathlineto{\pgfqpoint{5.215299in}{2.466837in}}%
\pgfpathlineto{\pgfqpoint{5.215693in}{2.461598in}}%
\pgfpathlineto{\pgfqpoint{5.216088in}{2.469165in}}%
\pgfpathlineto{\pgfqpoint{5.216384in}{2.474822in}}%
\pgfpathlineto{\pgfqpoint{5.216878in}{2.462125in}}%
\pgfpathlineto{\pgfqpoint{5.218457in}{2.434040in}}%
\pgfpathlineto{\pgfqpoint{5.218852in}{2.449389in}}%
\pgfpathlineto{\pgfqpoint{5.219839in}{2.488714in}}%
\pgfpathlineto{\pgfqpoint{5.220135in}{2.460388in}}%
\pgfpathlineto{\pgfqpoint{5.220727in}{2.439415in}}%
\pgfpathlineto{\pgfqpoint{5.221319in}{2.446004in}}%
\pgfpathlineto{\pgfqpoint{5.221418in}{2.446914in}}%
\pgfpathlineto{\pgfqpoint{5.221714in}{2.442212in}}%
\pgfpathlineto{\pgfqpoint{5.221911in}{2.438431in}}%
\pgfpathlineto{\pgfqpoint{5.222306in}{2.457293in}}%
\pgfpathlineto{\pgfqpoint{5.223392in}{2.516455in}}%
\pgfpathlineto{\pgfqpoint{5.224478in}{2.696947in}}%
\pgfpathlineto{\pgfqpoint{5.225070in}{2.659215in}}%
\pgfpathlineto{\pgfqpoint{5.225761in}{2.558563in}}%
\pgfpathlineto{\pgfqpoint{5.226353in}{2.394925in}}%
\pgfpathlineto{\pgfqpoint{5.227143in}{2.453178in}}%
\pgfpathlineto{\pgfqpoint{5.227439in}{2.442541in}}%
\pgfpathlineto{\pgfqpoint{5.227833in}{2.456701in}}%
\pgfpathlineto{\pgfqpoint{5.228327in}{2.446002in}}%
\pgfpathlineto{\pgfqpoint{5.229511in}{2.501310in}}%
\pgfpathlineto{\pgfqpoint{5.230005in}{2.478276in}}%
\pgfpathlineto{\pgfqpoint{5.230498in}{2.444567in}}%
\pgfpathlineto{\pgfqpoint{5.231485in}{2.448669in}}%
\pgfpathlineto{\pgfqpoint{5.232670in}{2.491817in}}%
\pgfpathlineto{\pgfqpoint{5.233262in}{2.482075in}}%
\pgfpathlineto{\pgfqpoint{5.233657in}{2.465049in}}%
\pgfpathlineto{\pgfqpoint{5.234051in}{2.490939in}}%
\pgfpathlineto{\pgfqpoint{5.235335in}{2.502691in}}%
\pgfpathlineto{\pgfqpoint{5.235828in}{2.520649in}}%
\pgfpathlineto{\pgfqpoint{5.236322in}{2.503537in}}%
\pgfpathlineto{\pgfqpoint{5.237999in}{2.453361in}}%
\pgfpathlineto{\pgfqpoint{5.238295in}{2.456912in}}%
\pgfpathlineto{\pgfqpoint{5.238592in}{2.450356in}}%
\pgfpathlineto{\pgfqpoint{5.238690in}{2.448900in}}%
\pgfpathlineto{\pgfqpoint{5.238986in}{2.459865in}}%
\pgfpathlineto{\pgfqpoint{5.241059in}{2.545281in}}%
\pgfpathlineto{\pgfqpoint{5.242342in}{2.535862in}}%
\pgfpathlineto{\pgfqpoint{5.242638in}{2.520376in}}%
\pgfpathlineto{\pgfqpoint{5.243428in}{2.529899in}}%
\pgfpathlineto{\pgfqpoint{5.243823in}{2.536845in}}%
\pgfpathlineto{\pgfqpoint{5.244415in}{2.529286in}}%
\pgfpathlineto{\pgfqpoint{5.244810in}{2.510272in}}%
\pgfpathlineto{\pgfqpoint{5.245599in}{2.523101in}}%
\pgfpathlineto{\pgfqpoint{5.245895in}{2.529481in}}%
\pgfpathlineto{\pgfqpoint{5.246389in}{2.516222in}}%
\pgfpathlineto{\pgfqpoint{5.246685in}{2.512451in}}%
\pgfpathlineto{\pgfqpoint{5.247080in}{2.523101in}}%
\pgfpathlineto{\pgfqpoint{5.247178in}{2.523081in}}%
\pgfpathlineto{\pgfqpoint{5.249054in}{2.482506in}}%
\pgfpathlineto{\pgfqpoint{5.249251in}{2.484641in}}%
\pgfpathlineto{\pgfqpoint{5.249350in}{2.485742in}}%
\pgfpathlineto{\pgfqpoint{5.249547in}{2.480631in}}%
\pgfpathlineto{\pgfqpoint{5.251225in}{2.443255in}}%
\pgfpathlineto{\pgfqpoint{5.251422in}{2.445451in}}%
\pgfpathlineto{\pgfqpoint{5.251817in}{2.438332in}}%
\pgfpathlineto{\pgfqpoint{5.252113in}{2.447979in}}%
\pgfpathlineto{\pgfqpoint{5.252212in}{2.448631in}}%
\pgfpathlineto{\pgfqpoint{5.252311in}{2.445201in}}%
\pgfpathlineto{\pgfqpoint{5.252706in}{2.425176in}}%
\pgfpathlineto{\pgfqpoint{5.253594in}{2.429492in}}%
\pgfpathlineto{\pgfqpoint{5.253890in}{2.433876in}}%
\pgfpathlineto{\pgfqpoint{5.254186in}{2.422504in}}%
\pgfpathlineto{\pgfqpoint{5.254877in}{2.410683in}}%
\pgfpathlineto{\pgfqpoint{5.255173in}{2.422625in}}%
\pgfpathlineto{\pgfqpoint{5.255370in}{2.429106in}}%
\pgfpathlineto{\pgfqpoint{5.255765in}{2.414879in}}%
\pgfpathlineto{\pgfqpoint{5.256061in}{2.417038in}}%
\pgfpathlineto{\pgfqpoint{5.257739in}{2.398294in}}%
\pgfpathlineto{\pgfqpoint{5.257838in}{2.398830in}}%
\pgfpathlineto{\pgfqpoint{5.259911in}{2.441060in}}%
\pgfpathlineto{\pgfqpoint{5.260009in}{2.437914in}}%
\pgfpathlineto{\pgfqpoint{5.261391in}{2.397340in}}%
\pgfpathlineto{\pgfqpoint{5.261687in}{2.402598in}}%
\pgfpathlineto{\pgfqpoint{5.261786in}{2.403778in}}%
\pgfpathlineto{\pgfqpoint{5.262082in}{2.394774in}}%
\pgfpathlineto{\pgfqpoint{5.263365in}{2.337633in}}%
\pgfpathlineto{\pgfqpoint{5.263661in}{2.354627in}}%
\pgfpathlineto{\pgfqpoint{5.263859in}{2.364075in}}%
\pgfpathlineto{\pgfqpoint{5.264451in}{2.333825in}}%
\pgfpathlineto{\pgfqpoint{5.264944in}{2.311880in}}%
\pgfpathlineto{\pgfqpoint{5.265931in}{2.318129in}}%
\pgfpathlineto{\pgfqpoint{5.266918in}{2.361757in}}%
\pgfpathlineto{\pgfqpoint{5.267214in}{2.368379in}}%
\pgfpathlineto{\pgfqpoint{5.267510in}{2.351450in}}%
\pgfpathlineto{\pgfqpoint{5.268497in}{2.326544in}}%
\pgfpathlineto{\pgfqpoint{5.268793in}{2.334582in}}%
\pgfpathlineto{\pgfqpoint{5.268892in}{2.335010in}}%
\pgfpathlineto{\pgfqpoint{5.268991in}{2.331635in}}%
\pgfpathlineto{\pgfqpoint{5.269287in}{2.321094in}}%
\pgfpathlineto{\pgfqpoint{5.269780in}{2.342541in}}%
\pgfpathlineto{\pgfqpoint{5.270274in}{2.367734in}}%
\pgfpathlineto{\pgfqpoint{5.271853in}{2.610975in}}%
\pgfpathlineto{\pgfqpoint{5.272544in}{2.567228in}}%
\pgfpathlineto{\pgfqpoint{5.273136in}{2.442916in}}%
\pgfpathlineto{\pgfqpoint{5.273728in}{2.321655in}}%
\pgfpathlineto{\pgfqpoint{5.274419in}{2.377119in}}%
\pgfpathlineto{\pgfqpoint{5.274617in}{2.365369in}}%
\pgfpathlineto{\pgfqpoint{5.275110in}{2.384402in}}%
\pgfpathlineto{\pgfqpoint{5.275505in}{2.379024in}}%
\pgfpathlineto{\pgfqpoint{5.275900in}{2.374356in}}%
\pgfpathlineto{\pgfqpoint{5.275998in}{2.378131in}}%
\pgfpathlineto{\pgfqpoint{5.276985in}{2.422348in}}%
\pgfpathlineto{\pgfqpoint{5.277282in}{2.401603in}}%
\pgfpathlineto{\pgfqpoint{5.277479in}{2.388530in}}%
\pgfpathlineto{\pgfqpoint{5.278367in}{2.398721in}}%
\pgfpathlineto{\pgfqpoint{5.278663in}{2.386425in}}%
\pgfpathlineto{\pgfqpoint{5.278861in}{2.378348in}}%
\pgfpathlineto{\pgfqpoint{5.279354in}{2.406684in}}%
\pgfpathlineto{\pgfqpoint{5.279848in}{2.435447in}}%
\pgfpathlineto{\pgfqpoint{5.280539in}{2.424276in}}%
\pgfpathlineto{\pgfqpoint{5.281032in}{2.390434in}}%
\pgfpathlineto{\pgfqpoint{5.281822in}{2.406813in}}%
\pgfpathlineto{\pgfqpoint{5.282315in}{2.396201in}}%
\pgfpathlineto{\pgfqpoint{5.282907in}{2.403412in}}%
\pgfpathlineto{\pgfqpoint{5.283401in}{2.433471in}}%
\pgfpathlineto{\pgfqpoint{5.284092in}{2.413457in}}%
\pgfpathlineto{\pgfqpoint{5.284289in}{2.410034in}}%
\pgfpathlineto{\pgfqpoint{5.284783in}{2.418411in}}%
\pgfpathlineto{\pgfqpoint{5.284980in}{2.417002in}}%
\pgfpathlineto{\pgfqpoint{5.285079in}{2.416568in}}%
\pgfpathlineto{\pgfqpoint{5.285375in}{2.420053in}}%
\pgfpathlineto{\pgfqpoint{5.286362in}{2.435166in}}%
\pgfpathlineto{\pgfqpoint{5.286658in}{2.425241in}}%
\pgfpathlineto{\pgfqpoint{5.286855in}{2.422318in}}%
\pgfpathlineto{\pgfqpoint{5.287349in}{2.433181in}}%
\pgfpathlineto{\pgfqpoint{5.287448in}{2.432834in}}%
\pgfpathlineto{\pgfqpoint{5.287645in}{2.432137in}}%
\pgfpathlineto{\pgfqpoint{5.287941in}{2.435694in}}%
\pgfpathlineto{\pgfqpoint{5.288040in}{2.436384in}}%
\pgfpathlineto{\pgfqpoint{5.288435in}{2.431687in}}%
\pgfpathlineto{\pgfqpoint{5.288829in}{2.425266in}}%
\pgfpathlineto{\pgfqpoint{5.289125in}{2.432357in}}%
\pgfpathlineto{\pgfqpoint{5.289915in}{2.455168in}}%
\pgfpathlineto{\pgfqpoint{5.290211in}{2.442786in}}%
\pgfpathlineto{\pgfqpoint{5.290409in}{2.431519in}}%
\pgfpathlineto{\pgfqpoint{5.291001in}{2.445198in}}%
\pgfpathlineto{\pgfqpoint{5.291198in}{2.445082in}}%
\pgfpathlineto{\pgfqpoint{5.291396in}{2.444303in}}%
\pgfpathlineto{\pgfqpoint{5.292481in}{2.436869in}}%
\pgfpathlineto{\pgfqpoint{5.292086in}{2.445425in}}%
\pgfpathlineto{\pgfqpoint{5.292580in}{2.438507in}}%
\pgfpathlineto{\pgfqpoint{5.293073in}{2.458358in}}%
\pgfpathlineto{\pgfqpoint{5.293666in}{2.441972in}}%
\pgfpathlineto{\pgfqpoint{5.295738in}{2.414710in}}%
\pgfpathlineto{\pgfqpoint{5.295837in}{2.414901in}}%
\pgfpathlineto{\pgfqpoint{5.296232in}{2.424799in}}%
\pgfpathlineto{\pgfqpoint{5.296824in}{2.413760in}}%
\pgfpathlineto{\pgfqpoint{5.298897in}{2.390630in}}%
\pgfpathlineto{\pgfqpoint{5.299094in}{2.393492in}}%
\pgfpathlineto{\pgfqpoint{5.299489in}{2.408675in}}%
\pgfpathlineto{\pgfqpoint{5.299982in}{2.386137in}}%
\pgfpathlineto{\pgfqpoint{5.300673in}{2.377218in}}%
\pgfpathlineto{\pgfqpoint{5.301364in}{2.381519in}}%
\pgfpathlineto{\pgfqpoint{5.301759in}{2.397155in}}%
\pgfpathlineto{\pgfqpoint{5.302154in}{2.379041in}}%
\pgfpathlineto{\pgfqpoint{5.302647in}{2.394225in}}%
\pgfpathlineto{\pgfqpoint{5.303634in}{2.375336in}}%
\pgfpathlineto{\pgfqpoint{5.303042in}{2.396681in}}%
\pgfpathlineto{\pgfqpoint{5.304325in}{2.385078in}}%
\pgfpathlineto{\pgfqpoint{5.305608in}{2.415065in}}%
\pgfpathlineto{\pgfqpoint{5.305806in}{2.414009in}}%
\pgfpathlineto{\pgfqpoint{5.307878in}{2.456265in}}%
\pgfpathlineto{\pgfqpoint{5.308372in}{2.445182in}}%
\pgfpathlineto{\pgfqpoint{5.308668in}{2.439704in}}%
\pgfpathlineto{\pgfqpoint{5.310346in}{2.389938in}}%
\pgfpathlineto{\pgfqpoint{5.310543in}{2.395718in}}%
\pgfpathlineto{\pgfqpoint{5.311037in}{2.428127in}}%
\pgfpathlineto{\pgfqpoint{5.311629in}{2.402929in}}%
\pgfpathlineto{\pgfqpoint{5.313405in}{2.320155in}}%
\pgfpathlineto{\pgfqpoint{5.313701in}{2.331705in}}%
\pgfpathlineto{\pgfqpoint{5.314688in}{2.377907in}}%
\pgfpathlineto{\pgfqpoint{5.315083in}{2.360153in}}%
\pgfpathlineto{\pgfqpoint{5.315281in}{2.354902in}}%
\pgfpathlineto{\pgfqpoint{5.315774in}{2.375790in}}%
\pgfpathlineto{\pgfqpoint{5.318242in}{2.559188in}}%
\pgfpathlineto{\pgfqpoint{5.319130in}{2.646543in}}%
\pgfpathlineto{\pgfqpoint{5.319525in}{2.604033in}}%
\pgfpathlineto{\pgfqpoint{5.320413in}{2.463874in}}%
\pgfpathlineto{\pgfqpoint{5.320907in}{2.369512in}}%
\pgfpathlineto{\pgfqpoint{5.321597in}{2.415540in}}%
\pgfpathlineto{\pgfqpoint{5.321992in}{2.400624in}}%
\pgfpathlineto{\pgfqpoint{5.322584in}{2.414646in}}%
\pgfpathlineto{\pgfqpoint{5.324164in}{2.488092in}}%
\pgfpathlineto{\pgfqpoint{5.323078in}{2.411854in}}%
\pgfpathlineto{\pgfqpoint{5.324558in}{2.459484in}}%
\pgfpathlineto{\pgfqpoint{5.325151in}{2.416662in}}%
\pgfpathlineto{\pgfqpoint{5.325940in}{2.427012in}}%
\pgfpathlineto{\pgfqpoint{5.326828in}{2.463163in}}%
\pgfpathlineto{\pgfqpoint{5.327223in}{2.480891in}}%
\pgfpathlineto{\pgfqpoint{5.327815in}{2.456283in}}%
\pgfpathlineto{\pgfqpoint{5.328408in}{2.421198in}}%
\pgfpathlineto{\pgfqpoint{5.329197in}{2.444583in}}%
\pgfpathlineto{\pgfqpoint{5.330086in}{2.457321in}}%
\pgfpathlineto{\pgfqpoint{5.330480in}{2.472472in}}%
\pgfpathlineto{\pgfqpoint{5.331073in}{2.457125in}}%
\pgfpathlineto{\pgfqpoint{5.331467in}{2.452197in}}%
\pgfpathlineto{\pgfqpoint{5.331961in}{2.458588in}}%
\pgfpathlineto{\pgfqpoint{5.333441in}{2.474205in}}%
\pgfpathlineto{\pgfqpoint{5.332948in}{2.458098in}}%
\pgfpathlineto{\pgfqpoint{5.334033in}{2.467569in}}%
\pgfpathlineto{\pgfqpoint{5.335020in}{2.437959in}}%
\pgfpathlineto{\pgfqpoint{5.335415in}{2.457925in}}%
\pgfpathlineto{\pgfqpoint{5.336501in}{2.469477in}}%
\pgfpathlineto{\pgfqpoint{5.336698in}{2.467050in}}%
\pgfpathlineto{\pgfqpoint{5.337981in}{2.434597in}}%
\pgfpathlineto{\pgfqpoint{5.338080in}{2.440763in}}%
\pgfpathlineto{\pgfqpoint{5.339067in}{2.462362in}}%
\pgfpathlineto{\pgfqpoint{5.339265in}{2.450021in}}%
\pgfpathlineto{\pgfqpoint{5.339462in}{2.440183in}}%
\pgfpathlineto{\pgfqpoint{5.340449in}{2.444631in}}%
\pgfpathlineto{\pgfqpoint{5.340745in}{2.438935in}}%
\pgfpathlineto{\pgfqpoint{5.341535in}{2.428811in}}%
\pgfpathlineto{\pgfqpoint{5.341929in}{2.435764in}}%
\pgfpathlineto{\pgfqpoint{5.342028in}{2.437330in}}%
\pgfpathlineto{\pgfqpoint{5.342522in}{2.427699in}}%
\pgfpathlineto{\pgfqpoint{5.342620in}{2.427496in}}%
\pgfpathlineto{\pgfqpoint{5.342719in}{2.428922in}}%
\pgfpathlineto{\pgfqpoint{5.342818in}{2.430411in}}%
\pgfpathlineto{\pgfqpoint{5.343114in}{2.421072in}}%
\pgfpathlineto{\pgfqpoint{5.344496in}{2.397813in}}%
\pgfpathlineto{\pgfqpoint{5.343805in}{2.434724in}}%
\pgfpathlineto{\pgfqpoint{5.344594in}{2.398896in}}%
\pgfpathlineto{\pgfqpoint{5.345285in}{2.413735in}}%
\pgfpathlineto{\pgfqpoint{5.346173in}{2.406546in}}%
\pgfpathlineto{\pgfqpoint{5.347555in}{2.382330in}}%
\pgfpathlineto{\pgfqpoint{5.347851in}{2.387306in}}%
\pgfpathlineto{\pgfqpoint{5.348147in}{2.392332in}}%
\pgfpathlineto{\pgfqpoint{5.348444in}{2.405935in}}%
\pgfpathlineto{\pgfqpoint{5.349332in}{2.400882in}}%
\pgfpathlineto{\pgfqpoint{5.349825in}{2.382139in}}%
\pgfpathlineto{\pgfqpoint{5.350418in}{2.400595in}}%
\pgfpathlineto{\pgfqpoint{5.350812in}{2.391288in}}%
\pgfpathlineto{\pgfqpoint{5.351602in}{2.398396in}}%
\pgfpathlineto{\pgfqpoint{5.354760in}{2.452010in}}%
\pgfpathlineto{\pgfqpoint{5.355056in}{2.463179in}}%
\pgfpathlineto{\pgfqpoint{5.355649in}{2.439743in}}%
\pgfpathlineto{\pgfqpoint{5.357721in}{2.374680in}}%
\pgfpathlineto{\pgfqpoint{5.357820in}{2.375868in}}%
\pgfpathlineto{\pgfqpoint{5.358511in}{2.397773in}}%
\pgfpathlineto{\pgfqpoint{5.358906in}{2.384509in}}%
\pgfpathlineto{\pgfqpoint{5.359794in}{2.343974in}}%
\pgfpathlineto{\pgfqpoint{5.360189in}{2.361669in}}%
\pgfpathlineto{\pgfqpoint{5.361965in}{2.413799in}}%
\pgfpathlineto{\pgfqpoint{5.362261in}{2.397711in}}%
\pgfpathlineto{\pgfqpoint{5.362656in}{2.348327in}}%
\pgfpathlineto{\pgfqpoint{5.363544in}{2.356382in}}%
\pgfpathlineto{\pgfqpoint{5.363939in}{2.355915in}}%
\pgfpathlineto{\pgfqpoint{5.364828in}{2.394846in}}%
\pgfpathlineto{\pgfqpoint{5.366604in}{2.630194in}}%
\pgfpathlineto{\pgfqpoint{5.367394in}{2.558527in}}%
\pgfpathlineto{\pgfqpoint{5.368282in}{2.343982in}}%
\pgfpathlineto{\pgfqpoint{5.369565in}{2.373568in}}%
\pgfpathlineto{\pgfqpoint{5.371243in}{2.443421in}}%
\pgfpathlineto{\pgfqpoint{5.371342in}{2.445496in}}%
\pgfpathlineto{\pgfqpoint{5.371835in}{2.435108in}}%
\pgfpathlineto{\pgfqpoint{5.372427in}{2.391427in}}%
\pgfpathlineto{\pgfqpoint{5.373118in}{2.419876in}}%
\pgfpathlineto{\pgfqpoint{5.374204in}{2.446821in}}%
\pgfpathlineto{\pgfqpoint{5.374895in}{2.434966in}}%
\pgfpathlineto{\pgfqpoint{5.375783in}{2.398001in}}%
\pgfpathlineto{\pgfqpoint{5.376178in}{2.426553in}}%
\pgfpathlineto{\pgfqpoint{5.377658in}{2.453148in}}%
\pgfpathlineto{\pgfqpoint{5.376573in}{2.418133in}}%
\pgfpathlineto{\pgfqpoint{5.377757in}{2.450896in}}%
\pgfpathlineto{\pgfqpoint{5.377955in}{2.440479in}}%
\pgfpathlineto{\pgfqpoint{5.378448in}{2.461059in}}%
\pgfpathlineto{\pgfqpoint{5.378941in}{2.446360in}}%
\pgfpathlineto{\pgfqpoint{5.380126in}{2.469499in}}%
\pgfpathlineto{\pgfqpoint{5.380521in}{2.457495in}}%
\pgfpathlineto{\pgfqpoint{5.380619in}{2.454775in}}%
\pgfpathlineto{\pgfqpoint{5.381014in}{2.463942in}}%
\pgfpathlineto{\pgfqpoint{5.381409in}{2.458651in}}%
\pgfpathlineto{\pgfqpoint{5.382791in}{2.475725in}}%
\pgfpathlineto{\pgfqpoint{5.382889in}{2.473478in}}%
\pgfpathlineto{\pgfqpoint{5.383975in}{2.451197in}}%
\pgfpathlineto{\pgfqpoint{5.384271in}{2.462197in}}%
\pgfpathlineto{\pgfqpoint{5.385258in}{2.470775in}}%
\pgfpathlineto{\pgfqpoint{5.384765in}{2.459417in}}%
\pgfpathlineto{\pgfqpoint{5.385554in}{2.470612in}}%
\pgfpathlineto{\pgfqpoint{5.386245in}{2.491165in}}%
\pgfpathlineto{\pgfqpoint{5.386837in}{2.476362in}}%
\pgfpathlineto{\pgfqpoint{5.387923in}{2.466426in}}%
\pgfpathlineto{\pgfqpoint{5.387528in}{2.476437in}}%
\pgfpathlineto{\pgfqpoint{5.388022in}{2.467275in}}%
\pgfpathlineto{\pgfqpoint{5.388318in}{2.475211in}}%
\pgfpathlineto{\pgfqpoint{5.388811in}{2.458507in}}%
\pgfpathlineto{\pgfqpoint{5.389009in}{2.458616in}}%
\pgfpathlineto{\pgfqpoint{5.389996in}{2.429803in}}%
\pgfpathlineto{\pgfqpoint{5.390785in}{2.385627in}}%
\pgfpathlineto{\pgfqpoint{5.391378in}{2.400545in}}%
\pgfpathlineto{\pgfqpoint{5.391772in}{2.397936in}}%
\pgfpathlineto{\pgfqpoint{5.391871in}{2.399501in}}%
\pgfpathlineto{\pgfqpoint{5.393845in}{2.446213in}}%
\pgfpathlineto{\pgfqpoint{5.393944in}{2.443956in}}%
\pgfpathlineto{\pgfqpoint{5.395326in}{2.429884in}}%
\pgfpathlineto{\pgfqpoint{5.394536in}{2.448172in}}%
\pgfpathlineto{\pgfqpoint{5.395424in}{2.431989in}}%
\pgfpathlineto{\pgfqpoint{5.395819in}{2.446109in}}%
\pgfpathlineto{\pgfqpoint{5.396313in}{2.431636in}}%
\pgfpathlineto{\pgfqpoint{5.396609in}{2.437147in}}%
\pgfpathlineto{\pgfqpoint{5.397003in}{2.418061in}}%
\pgfpathlineto{\pgfqpoint{5.397596in}{2.437893in}}%
\pgfpathlineto{\pgfqpoint{5.397892in}{2.438607in}}%
\pgfpathlineto{\pgfqpoint{5.398089in}{2.437668in}}%
\pgfpathlineto{\pgfqpoint{5.398780in}{2.429831in}}%
\pgfpathlineto{\pgfqpoint{5.399076in}{2.435800in}}%
\pgfpathlineto{\pgfqpoint{5.399964in}{2.444385in}}%
\pgfpathlineto{\pgfqpoint{5.400162in}{2.439534in}}%
\pgfpathlineto{\pgfqpoint{5.400260in}{2.437511in}}%
\pgfpathlineto{\pgfqpoint{5.400655in}{2.449308in}}%
\pgfpathlineto{\pgfqpoint{5.401741in}{2.478387in}}%
\pgfpathlineto{\pgfqpoint{5.402136in}{2.468753in}}%
\pgfpathlineto{\pgfqpoint{5.402234in}{2.468383in}}%
\pgfpathlineto{\pgfqpoint{5.402333in}{2.471456in}}%
\pgfpathlineto{\pgfqpoint{5.402531in}{2.478340in}}%
\pgfpathlineto{\pgfqpoint{5.402925in}{2.457137in}}%
\pgfpathlineto{\pgfqpoint{5.404011in}{2.440496in}}%
\pgfpathlineto{\pgfqpoint{5.404307in}{2.441207in}}%
\pgfpathlineto{\pgfqpoint{5.405294in}{2.417772in}}%
\pgfpathlineto{\pgfqpoint{5.405590in}{2.430123in}}%
\pgfpathlineto{\pgfqpoint{5.405788in}{2.439719in}}%
\pgfpathlineto{\pgfqpoint{5.406281in}{2.411781in}}%
\pgfpathlineto{\pgfqpoint{5.406972in}{2.382887in}}%
\pgfpathlineto{\pgfqpoint{5.407465in}{2.404644in}}%
\pgfpathlineto{\pgfqpoint{5.407762in}{2.402545in}}%
\pgfpathlineto{\pgfqpoint{5.408156in}{2.411586in}}%
\pgfpathlineto{\pgfqpoint{5.409045in}{2.442425in}}%
\pgfpathlineto{\pgfqpoint{5.409439in}{2.425443in}}%
\pgfpathlineto{\pgfqpoint{5.410032in}{2.395271in}}%
\pgfpathlineto{\pgfqpoint{5.410723in}{2.407721in}}%
\pgfpathlineto{\pgfqpoint{5.411611in}{2.415749in}}%
\pgfpathlineto{\pgfqpoint{5.411216in}{2.407283in}}%
\pgfpathlineto{\pgfqpoint{5.411808in}{2.412380in}}%
\pgfpathlineto{\pgfqpoint{5.412006in}{2.408299in}}%
\pgfpathlineto{\pgfqpoint{5.412203in}{2.429415in}}%
\pgfpathlineto{\pgfqpoint{5.413782in}{2.665572in}}%
\pgfpathlineto{\pgfqpoint{5.414177in}{2.629516in}}%
\pgfpathlineto{\pgfqpoint{5.414671in}{2.582232in}}%
\pgfpathlineto{\pgfqpoint{5.415460in}{2.358328in}}%
\pgfpathlineto{\pgfqpoint{5.416348in}{2.414093in}}%
\pgfpathlineto{\pgfqpoint{5.416644in}{2.408058in}}%
\pgfpathlineto{\pgfqpoint{5.417039in}{2.424514in}}%
\pgfpathlineto{\pgfqpoint{5.417631in}{2.422153in}}%
\pgfpathlineto{\pgfqpoint{5.418717in}{2.465389in}}%
\pgfpathlineto{\pgfqpoint{5.419408in}{2.432383in}}%
\pgfpathlineto{\pgfqpoint{5.420395in}{2.441210in}}%
\pgfpathlineto{\pgfqpoint{5.421777in}{2.476112in}}%
\pgfpathlineto{\pgfqpoint{5.422369in}{2.472746in}}%
\pgfpathlineto{\pgfqpoint{5.422863in}{2.441238in}}%
\pgfpathlineto{\pgfqpoint{5.423850in}{2.462618in}}%
\pgfpathlineto{\pgfqpoint{5.424047in}{2.471086in}}%
\pgfpathlineto{\pgfqpoint{5.424540in}{2.453106in}}%
\pgfpathlineto{\pgfqpoint{5.425034in}{2.468279in}}%
\pgfpathlineto{\pgfqpoint{5.425133in}{2.467817in}}%
\pgfpathlineto{\pgfqpoint{5.425231in}{2.469657in}}%
\pgfpathlineto{\pgfqpoint{5.425626in}{2.490061in}}%
\pgfpathlineto{\pgfqpoint{5.426514in}{2.480414in}}%
\pgfpathlineto{\pgfqpoint{5.426613in}{2.479300in}}%
\pgfpathlineto{\pgfqpoint{5.426909in}{2.488152in}}%
\pgfpathlineto{\pgfqpoint{5.427896in}{2.505726in}}%
\pgfpathlineto{\pgfqpoint{5.428192in}{2.496103in}}%
\pgfpathlineto{\pgfqpoint{5.428390in}{2.488904in}}%
\pgfpathlineto{\pgfqpoint{5.428784in}{2.515214in}}%
\pgfpathlineto{\pgfqpoint{5.428883in}{2.518791in}}%
\pgfpathlineto{\pgfqpoint{5.429377in}{2.504014in}}%
\pgfpathlineto{\pgfqpoint{5.429870in}{2.516550in}}%
\pgfpathlineto{\pgfqpoint{5.430265in}{2.506325in}}%
\pgfpathlineto{\pgfqpoint{5.430857in}{2.518142in}}%
\pgfpathlineto{\pgfqpoint{5.431153in}{2.513068in}}%
\pgfpathlineto{\pgfqpoint{5.431351in}{2.514434in}}%
\pgfpathlineto{\pgfqpoint{5.432535in}{2.526447in}}%
\pgfpathlineto{\pgfqpoint{5.432634in}{2.524788in}}%
\pgfpathlineto{\pgfqpoint{5.433325in}{2.525249in}}%
\pgfpathlineto{\pgfqpoint{5.434114in}{2.509781in}}%
\pgfpathlineto{\pgfqpoint{5.434312in}{2.510005in}}%
\pgfpathlineto{\pgfqpoint{5.434706in}{2.505503in}}%
\pgfpathlineto{\pgfqpoint{5.434904in}{2.503451in}}%
\pgfpathlineto{\pgfqpoint{5.435200in}{2.512734in}}%
\pgfpathlineto{\pgfqpoint{5.435496in}{2.525734in}}%
\pgfpathlineto{\pgfqpoint{5.435989in}{2.497713in}}%
\pgfpathlineto{\pgfqpoint{5.436878in}{2.484860in}}%
\pgfpathlineto{\pgfqpoint{5.437273in}{2.488912in}}%
\pgfpathlineto{\pgfqpoint{5.440727in}{2.421726in}}%
\pgfpathlineto{\pgfqpoint{5.441023in}{2.429818in}}%
\pgfpathlineto{\pgfqpoint{5.442010in}{2.441633in}}%
\pgfpathlineto{\pgfqpoint{5.442208in}{2.434690in}}%
\pgfpathlineto{\pgfqpoint{5.443195in}{2.418907in}}%
\pgfpathlineto{\pgfqpoint{5.443491in}{2.422839in}}%
\pgfpathlineto{\pgfqpoint{5.445070in}{2.463353in}}%
\pgfpathlineto{\pgfqpoint{5.445267in}{2.454167in}}%
\pgfpathlineto{\pgfqpoint{5.446155in}{2.445789in}}%
\pgfpathlineto{\pgfqpoint{5.445761in}{2.454896in}}%
\pgfpathlineto{\pgfqpoint{5.446353in}{2.449364in}}%
\pgfpathlineto{\pgfqpoint{5.448031in}{2.480955in}}%
\pgfpathlineto{\pgfqpoint{5.448820in}{2.511478in}}%
\pgfpathlineto{\pgfqpoint{5.449906in}{2.500781in}}%
\pgfpathlineto{\pgfqpoint{5.452176in}{2.440216in}}%
\pgfpathlineto{\pgfqpoint{5.452472in}{2.442778in}}%
\pgfpathlineto{\pgfqpoint{5.452768in}{2.438132in}}%
\pgfpathlineto{\pgfqpoint{5.452867in}{2.437267in}}%
\pgfpathlineto{\pgfqpoint{5.453163in}{2.444102in}}%
\pgfpathlineto{\pgfqpoint{5.453361in}{2.439037in}}%
\pgfpathlineto{\pgfqpoint{5.454545in}{2.407557in}}%
\pgfpathlineto{\pgfqpoint{5.454742in}{2.411385in}}%
\pgfpathlineto{\pgfqpoint{5.456223in}{2.445829in}}%
\pgfpathlineto{\pgfqpoint{5.456420in}{2.429987in}}%
\pgfpathlineto{\pgfqpoint{5.457407in}{2.388746in}}%
\pgfpathlineto{\pgfqpoint{5.457703in}{2.404414in}}%
\pgfpathlineto{\pgfqpoint{5.457901in}{2.396831in}}%
\pgfpathlineto{\pgfqpoint{5.458098in}{2.383545in}}%
\pgfpathlineto{\pgfqpoint{5.458986in}{2.395875in}}%
\pgfpathlineto{\pgfqpoint{5.459677in}{2.498536in}}%
\pgfpathlineto{\pgfqpoint{5.460960in}{2.641787in}}%
\pgfpathlineto{\pgfqpoint{5.461158in}{2.624720in}}%
\pgfpathlineto{\pgfqpoint{5.462540in}{2.355643in}}%
\pgfpathlineto{\pgfqpoint{5.463823in}{2.407522in}}%
\pgfpathlineto{\pgfqpoint{5.465402in}{2.380400in}}%
\pgfpathlineto{\pgfqpoint{5.465797in}{2.391328in}}%
\pgfpathlineto{\pgfqpoint{5.465994in}{2.384597in}}%
\pgfpathlineto{\pgfqpoint{5.466586in}{2.352569in}}%
\pgfpathlineto{\pgfqpoint{5.467178in}{2.372244in}}%
\pgfpathlineto{\pgfqpoint{5.468264in}{2.431651in}}%
\pgfpathlineto{\pgfqpoint{5.468955in}{2.471797in}}%
\pgfpathlineto{\pgfqpoint{5.469646in}{2.455113in}}%
\pgfpathlineto{\pgfqpoint{5.470830in}{2.421686in}}%
\pgfpathlineto{\pgfqpoint{5.471620in}{2.436734in}}%
\pgfpathlineto{\pgfqpoint{5.473199in}{2.460243in}}%
\pgfpathlineto{\pgfqpoint{5.473692in}{2.437734in}}%
\pgfpathlineto{\pgfqpoint{5.474679in}{2.451585in}}%
\pgfpathlineto{\pgfqpoint{5.475173in}{2.466935in}}%
\pgfpathlineto{\pgfqpoint{5.475765in}{2.451182in}}%
\pgfpathlineto{\pgfqpoint{5.475864in}{2.451375in}}%
\pgfpathlineto{\pgfqpoint{5.476259in}{2.468227in}}%
\pgfpathlineto{\pgfqpoint{5.476752in}{2.449602in}}%
\pgfpathlineto{\pgfqpoint{5.477147in}{2.461974in}}%
\pgfpathlineto{\pgfqpoint{5.477640in}{2.443280in}}%
\pgfpathlineto{\pgfqpoint{5.478233in}{2.460695in}}%
\pgfpathlineto{\pgfqpoint{5.479220in}{2.468611in}}%
\pgfpathlineto{\pgfqpoint{5.478825in}{2.460311in}}%
\pgfpathlineto{\pgfqpoint{5.479318in}{2.466085in}}%
\pgfpathlineto{\pgfqpoint{5.480799in}{2.447596in}}%
\pgfpathlineto{\pgfqpoint{5.481194in}{2.468330in}}%
\pgfpathlineto{\pgfqpoint{5.482477in}{2.467354in}}%
\pgfpathlineto{\pgfqpoint{5.482575in}{2.468956in}}%
\pgfpathlineto{\pgfqpoint{5.483168in}{2.462729in}}%
\pgfpathlineto{\pgfqpoint{5.483661in}{2.437635in}}%
\pgfpathlineto{\pgfqpoint{5.484253in}{2.456784in}}%
\pgfpathlineto{\pgfqpoint{5.484451in}{2.461132in}}%
\pgfpathlineto{\pgfqpoint{5.485142in}{2.450487in}}%
\pgfpathlineto{\pgfqpoint{5.487116in}{2.405848in}}%
\pgfpathlineto{\pgfqpoint{5.488103in}{2.408834in}}%
\pgfpathlineto{\pgfqpoint{5.488596in}{2.417441in}}%
\pgfpathlineto{\pgfqpoint{5.488892in}{2.408819in}}%
\pgfpathlineto{\pgfqpoint{5.490175in}{2.389143in}}%
\pgfpathlineto{\pgfqpoint{5.490373in}{2.390951in}}%
\pgfpathlineto{\pgfqpoint{5.490965in}{2.407351in}}%
\pgfpathlineto{\pgfqpoint{5.491853in}{2.400401in}}%
\pgfpathlineto{\pgfqpoint{5.493037in}{2.385225in}}%
\pgfpathlineto{\pgfqpoint{5.492544in}{2.408582in}}%
\pgfpathlineto{\pgfqpoint{5.493136in}{2.386684in}}%
\pgfpathlineto{\pgfqpoint{5.494814in}{2.407104in}}%
\pgfpathlineto{\pgfqpoint{5.495110in}{2.401847in}}%
\pgfpathlineto{\pgfqpoint{5.495308in}{2.397489in}}%
\pgfpathlineto{\pgfqpoint{5.495702in}{2.410504in}}%
\pgfpathlineto{\pgfqpoint{5.497676in}{2.460711in}}%
\pgfpathlineto{\pgfqpoint{5.499848in}{2.376508in}}%
\pgfpathlineto{\pgfqpoint{5.500243in}{2.387186in}}%
\pgfpathlineto{\pgfqpoint{5.500637in}{2.400710in}}%
\pgfpathlineto{\pgfqpoint{5.501229in}{2.388605in}}%
\pgfpathlineto{\pgfqpoint{5.501920in}{2.344990in}}%
\pgfpathlineto{\pgfqpoint{5.502611in}{2.371731in}}%
\pgfpathlineto{\pgfqpoint{5.502809in}{2.368380in}}%
\pgfpathlineto{\pgfqpoint{5.503105in}{2.377365in}}%
\pgfpathlineto{\pgfqpoint{5.503993in}{2.432485in}}%
\pgfpathlineto{\pgfqpoint{5.504487in}{2.396012in}}%
\pgfpathlineto{\pgfqpoint{5.505177in}{2.365682in}}%
\pgfpathlineto{\pgfqpoint{5.505868in}{2.372843in}}%
\pgfpathlineto{\pgfqpoint{5.507349in}{2.460587in}}%
\pgfpathlineto{\pgfqpoint{5.508632in}{2.638880in}}%
\pgfpathlineto{\pgfqpoint{5.509125in}{2.597802in}}%
\pgfpathlineto{\pgfqpoint{5.509816in}{2.501124in}}%
\pgfpathlineto{\pgfqpoint{5.510310in}{2.360788in}}%
\pgfpathlineto{\pgfqpoint{5.511099in}{2.403312in}}%
\pgfpathlineto{\pgfqpoint{5.511494in}{2.386347in}}%
\pgfpathlineto{\pgfqpoint{5.511889in}{2.405350in}}%
\pgfpathlineto{\pgfqpoint{5.513468in}{2.460754in}}%
\pgfpathlineto{\pgfqpoint{5.513863in}{2.443984in}}%
\pgfpathlineto{\pgfqpoint{5.514850in}{2.402983in}}%
\pgfpathlineto{\pgfqpoint{5.515245in}{2.421414in}}%
\pgfpathlineto{\pgfqpoint{5.516824in}{2.464663in}}%
\pgfpathlineto{\pgfqpoint{5.517021in}{2.458178in}}%
\pgfpathlineto{\pgfqpoint{5.517910in}{2.430539in}}%
\pgfpathlineto{\pgfqpoint{5.518304in}{2.445187in}}%
\pgfpathlineto{\pgfqpoint{5.520081in}{2.484840in}}%
\pgfpathlineto{\pgfqpoint{5.520180in}{2.480218in}}%
\pgfpathlineto{\pgfqpoint{5.521167in}{2.459323in}}%
\pgfpathlineto{\pgfqpoint{5.521364in}{2.464389in}}%
\pgfpathlineto{\pgfqpoint{5.522450in}{2.491628in}}%
\pgfpathlineto{\pgfqpoint{5.522943in}{2.484069in}}%
\pgfpathlineto{\pgfqpoint{5.523437in}{2.471443in}}%
\pgfpathlineto{\pgfqpoint{5.524424in}{2.472659in}}%
\pgfpathlineto{\pgfqpoint{5.525608in}{2.487802in}}%
\pgfpathlineto{\pgfqpoint{5.525904in}{2.485295in}}%
\pgfpathlineto{\pgfqpoint{5.526595in}{2.471546in}}%
\pgfpathlineto{\pgfqpoint{5.527483in}{2.478605in}}%
\pgfpathlineto{\pgfqpoint{5.527582in}{2.479218in}}%
\pgfpathlineto{\pgfqpoint{5.527780in}{2.475596in}}%
\pgfpathlineto{\pgfqpoint{5.527977in}{2.469280in}}%
\pgfpathlineto{\pgfqpoint{5.528470in}{2.489280in}}%
\pgfpathlineto{\pgfqpoint{5.528569in}{2.491003in}}%
\pgfpathlineto{\pgfqpoint{5.529161in}{2.483574in}}%
\pgfpathlineto{\pgfqpoint{5.529556in}{2.466572in}}%
\pgfpathlineto{\pgfqpoint{5.530346in}{2.474828in}}%
\pgfpathlineto{\pgfqpoint{5.530444in}{2.476541in}}%
\pgfpathlineto{\pgfqpoint{5.530839in}{2.467029in}}%
\pgfpathlineto{\pgfqpoint{5.531925in}{2.460505in}}%
\pgfpathlineto{\pgfqpoint{5.531530in}{2.473610in}}%
\pgfpathlineto{\pgfqpoint{5.532024in}{2.461772in}}%
\pgfpathlineto{\pgfqpoint{5.532320in}{2.475631in}}%
\pgfpathlineto{\pgfqpoint{5.532912in}{2.451980in}}%
\pgfpathlineto{\pgfqpoint{5.533504in}{2.452154in}}%
\pgfpathlineto{\pgfqpoint{5.534688in}{2.461703in}}%
\pgfpathlineto{\pgfqpoint{5.534787in}{2.459292in}}%
\pgfpathlineto{\pgfqpoint{5.538538in}{2.350846in}}%
\pgfpathlineto{\pgfqpoint{5.535478in}{2.462848in}}%
\pgfpathlineto{\pgfqpoint{5.538834in}{2.361253in}}%
\pgfpathlineto{\pgfqpoint{5.540413in}{2.446620in}}%
\pgfpathlineto{\pgfqpoint{5.541597in}{2.439790in}}%
\pgfpathlineto{\pgfqpoint{5.542880in}{2.427913in}}%
\pgfpathlineto{\pgfqpoint{5.542387in}{2.444526in}}%
\pgfpathlineto{\pgfqpoint{5.542979in}{2.429855in}}%
\pgfpathlineto{\pgfqpoint{5.544262in}{2.467756in}}%
\pgfpathlineto{\pgfqpoint{5.544558in}{2.454106in}}%
\pgfpathlineto{\pgfqpoint{5.546927in}{2.408981in}}%
\pgfpathlineto{\pgfqpoint{5.544953in}{2.463722in}}%
\pgfpathlineto{\pgfqpoint{5.547717in}{2.416341in}}%
\pgfpathlineto{\pgfqpoint{5.548013in}{2.423325in}}%
\pgfpathlineto{\pgfqpoint{5.548506in}{2.409160in}}%
\pgfpathlineto{\pgfqpoint{5.549098in}{2.387245in}}%
\pgfpathlineto{\pgfqpoint{5.549888in}{2.399389in}}%
\pgfpathlineto{\pgfqpoint{5.551369in}{2.443309in}}%
\pgfpathlineto{\pgfqpoint{5.551566in}{2.433604in}}%
\pgfpathlineto{\pgfqpoint{5.552454in}{2.386165in}}%
\pgfpathlineto{\pgfqpoint{5.552948in}{2.401355in}}%
\pgfpathlineto{\pgfqpoint{5.554428in}{2.444265in}}%
\pgfpathlineto{\pgfqpoint{5.555613in}{2.661837in}}%
\pgfpathlineto{\pgfqpoint{5.556698in}{2.626479in}}%
\pgfpathlineto{\pgfqpoint{5.557784in}{2.361883in}}%
\pgfpathlineto{\pgfqpoint{5.559264in}{2.422538in}}%
\pgfpathlineto{\pgfqpoint{5.560844in}{2.458369in}}%
\pgfpathlineto{\pgfqpoint{5.559758in}{2.418406in}}%
\pgfpathlineto{\pgfqpoint{5.561041in}{2.452837in}}%
\pgfpathlineto{\pgfqpoint{5.562028in}{2.409675in}}%
\pgfpathlineto{\pgfqpoint{5.562620in}{2.421571in}}%
\pgfpathlineto{\pgfqpoint{5.562719in}{2.420961in}}%
\pgfpathlineto{\pgfqpoint{5.563015in}{2.424410in}}%
\pgfpathlineto{\pgfqpoint{5.563805in}{2.448052in}}%
\pgfpathlineto{\pgfqpoint{5.564397in}{2.429443in}}%
\pgfpathlineto{\pgfqpoint{5.564693in}{2.431511in}}%
\pgfpathlineto{\pgfqpoint{5.564890in}{2.427456in}}%
\pgfpathlineto{\pgfqpoint{5.565186in}{2.416067in}}%
\pgfpathlineto{\pgfqpoint{5.565680in}{2.433460in}}%
\pgfpathlineto{\pgfqpoint{5.565877in}{2.432151in}}%
\pgfpathlineto{\pgfqpoint{5.568937in}{2.468932in}}%
\pgfpathlineto{\pgfqpoint{5.569924in}{2.463676in}}%
\pgfpathlineto{\pgfqpoint{5.570714in}{2.456266in}}%
\pgfpathlineto{\pgfqpoint{5.570911in}{2.459774in}}%
\pgfpathlineto{\pgfqpoint{5.571404in}{2.481416in}}%
\pgfpathlineto{\pgfqpoint{5.572391in}{2.480840in}}%
\pgfpathlineto{\pgfqpoint{5.574069in}{2.471650in}}%
\pgfpathlineto{\pgfqpoint{5.574267in}{2.477022in}}%
\pgfpathlineto{\pgfqpoint{5.574662in}{2.500239in}}%
\pgfpathlineto{\pgfqpoint{5.575550in}{2.491023in}}%
\pgfpathlineto{\pgfqpoint{5.576537in}{2.483827in}}%
\pgfpathlineto{\pgfqpoint{5.576043in}{2.494617in}}%
\pgfpathlineto{\pgfqpoint{5.576635in}{2.486867in}}%
\pgfpathlineto{\pgfqpoint{5.578116in}{2.519067in}}%
\pgfpathlineto{\pgfqpoint{5.578313in}{2.513265in}}%
\pgfpathlineto{\pgfqpoint{5.578708in}{2.498982in}}%
\pgfpathlineto{\pgfqpoint{5.579596in}{2.500018in}}%
\pgfpathlineto{\pgfqpoint{5.580485in}{2.490228in}}%
\pgfpathlineto{\pgfqpoint{5.580090in}{2.503509in}}%
\pgfpathlineto{\pgfqpoint{5.580781in}{2.495309in}}%
\pgfpathlineto{\pgfqpoint{5.580978in}{2.497959in}}%
\pgfpathlineto{\pgfqpoint{5.581669in}{2.490571in}}%
\pgfpathlineto{\pgfqpoint{5.582261in}{2.470060in}}%
\pgfpathlineto{\pgfqpoint{5.582459in}{2.463116in}}%
\pgfpathlineto{\pgfqpoint{5.582854in}{2.477059in}}%
\pgfpathlineto{\pgfqpoint{5.583446in}{2.463853in}}%
\pgfpathlineto{\pgfqpoint{5.583841in}{2.472131in}}%
\pgfpathlineto{\pgfqpoint{5.584137in}{2.484525in}}%
\pgfpathlineto{\pgfqpoint{5.584630in}{2.459848in}}%
\pgfpathlineto{\pgfqpoint{5.584828in}{2.464904in}}%
\pgfpathlineto{\pgfqpoint{5.584926in}{2.466111in}}%
\pgfpathlineto{\pgfqpoint{5.585321in}{2.457925in}}%
\pgfpathlineto{\pgfqpoint{5.585814in}{2.451709in}}%
\pgfpathlineto{\pgfqpoint{5.586012in}{2.456958in}}%
\pgfpathlineto{\pgfqpoint{5.586308in}{2.472938in}}%
\pgfpathlineto{\pgfqpoint{5.586801in}{2.451884in}}%
\pgfpathlineto{\pgfqpoint{5.587196in}{2.465494in}}%
\pgfpathlineto{\pgfqpoint{5.587394in}{2.467969in}}%
\pgfpathlineto{\pgfqpoint{5.587887in}{2.462270in}}%
\pgfpathlineto{\pgfqpoint{5.588677in}{2.445056in}}%
\pgfpathlineto{\pgfqpoint{5.589072in}{2.455389in}}%
\pgfpathlineto{\pgfqpoint{5.589565in}{2.454929in}}%
\pgfpathlineto{\pgfqpoint{5.591243in}{2.491107in}}%
\pgfpathlineto{\pgfqpoint{5.591342in}{2.491384in}}%
\pgfpathlineto{\pgfqpoint{5.591440in}{2.490241in}}%
\pgfpathlineto{\pgfqpoint{5.591736in}{2.483204in}}%
\pgfpathlineto{\pgfqpoint{5.592131in}{2.501725in}}%
\pgfpathlineto{\pgfqpoint{5.592427in}{2.489127in}}%
\pgfpathlineto{\pgfqpoint{5.593020in}{2.498398in}}%
\pgfpathlineto{\pgfqpoint{5.594796in}{2.426928in}}%
\pgfpathlineto{\pgfqpoint{5.594993in}{2.425546in}}%
\pgfpathlineto{\pgfqpoint{5.595388in}{2.432119in}}%
\pgfpathlineto{\pgfqpoint{5.595684in}{2.439335in}}%
\pgfpathlineto{\pgfqpoint{5.596079in}{2.422831in}}%
\pgfpathlineto{\pgfqpoint{5.596671in}{2.386328in}}%
\pgfpathlineto{\pgfqpoint{5.597461in}{2.398093in}}%
\pgfpathlineto{\pgfqpoint{5.597658in}{2.401116in}}%
\pgfpathlineto{\pgfqpoint{5.598645in}{2.438027in}}%
\pgfpathlineto{\pgfqpoint{5.599040in}{2.418392in}}%
\pgfpathlineto{\pgfqpoint{5.600027in}{2.393964in}}%
\pgfpathlineto{\pgfqpoint{5.600521in}{2.399950in}}%
\pgfpathlineto{\pgfqpoint{5.601902in}{2.435210in}}%
\pgfpathlineto{\pgfqpoint{5.603383in}{2.665299in}}%
\pgfpathlineto{\pgfqpoint{5.603975in}{2.628458in}}%
\pgfpathlineto{\pgfqpoint{5.604370in}{2.587553in}}%
\pgfpathlineto{\pgfqpoint{5.605159in}{2.370784in}}%
\pgfpathlineto{\pgfqpoint{5.606146in}{2.395762in}}%
\pgfpathlineto{\pgfqpoint{5.607331in}{2.364751in}}%
\pgfpathlineto{\pgfqpoint{5.607430in}{2.367150in}}%
\pgfpathlineto{\pgfqpoint{5.609897in}{2.447833in}}%
\pgfpathlineto{\pgfqpoint{5.609996in}{2.443738in}}%
\pgfpathlineto{\pgfqpoint{5.610193in}{2.433216in}}%
\pgfpathlineto{\pgfqpoint{5.610884in}{2.456450in}}%
\pgfpathlineto{\pgfqpoint{5.611279in}{2.475865in}}%
\pgfpathlineto{\pgfqpoint{5.612167in}{2.462705in}}%
\pgfpathlineto{\pgfqpoint{5.612562in}{2.442927in}}%
\pgfpathlineto{\pgfqpoint{5.613450in}{2.450317in}}%
\pgfpathlineto{\pgfqpoint{5.613746in}{2.453212in}}%
\pgfpathlineto{\pgfqpoint{5.615819in}{2.495684in}}%
\pgfpathlineto{\pgfqpoint{5.615918in}{2.495763in}}%
\pgfpathlineto{\pgfqpoint{5.616510in}{2.476626in}}%
\pgfpathlineto{\pgfqpoint{5.616905in}{2.491072in}}%
\pgfpathlineto{\pgfqpoint{5.618385in}{2.522202in}}%
\pgfpathlineto{\pgfqpoint{5.618484in}{2.520754in}}%
\pgfpathlineto{\pgfqpoint{5.619767in}{2.504646in}}%
\pgfpathlineto{\pgfqpoint{5.619273in}{2.526037in}}%
\pgfpathlineto{\pgfqpoint{5.619866in}{2.505563in}}%
\pgfpathlineto{\pgfqpoint{5.620359in}{2.526042in}}%
\pgfpathlineto{\pgfqpoint{5.621346in}{2.524803in}}%
\pgfpathlineto{\pgfqpoint{5.621741in}{2.527225in}}%
\pgfpathlineto{\pgfqpoint{5.622037in}{2.523672in}}%
\pgfpathlineto{\pgfqpoint{5.623024in}{2.508379in}}%
\pgfpathlineto{\pgfqpoint{5.623517in}{2.509775in}}%
\pgfpathlineto{\pgfqpoint{5.625195in}{2.527226in}}%
\pgfpathlineto{\pgfqpoint{5.625393in}{2.521104in}}%
\pgfpathlineto{\pgfqpoint{5.626478in}{2.500029in}}%
\pgfpathlineto{\pgfqpoint{5.625985in}{2.522694in}}%
\pgfpathlineto{\pgfqpoint{5.626676in}{2.504477in}}%
\pgfpathlineto{\pgfqpoint{5.627268in}{2.515211in}}%
\pgfpathlineto{\pgfqpoint{5.627564in}{2.506530in}}%
\pgfpathlineto{\pgfqpoint{5.629341in}{2.459000in}}%
\pgfpathlineto{\pgfqpoint{5.629439in}{2.457223in}}%
\pgfpathlineto{\pgfqpoint{5.629736in}{2.467243in}}%
\pgfpathlineto{\pgfqpoint{5.629933in}{2.470867in}}%
\pgfpathlineto{\pgfqpoint{5.630328in}{2.460413in}}%
\pgfpathlineto{\pgfqpoint{5.630723in}{2.463949in}}%
\pgfpathlineto{\pgfqpoint{5.632499in}{2.441572in}}%
\pgfpathlineto{\pgfqpoint{5.632598in}{2.442382in}}%
\pgfpathlineto{\pgfqpoint{5.633585in}{2.459475in}}%
\pgfpathlineto{\pgfqpoint{5.634374in}{2.456388in}}%
\pgfpathlineto{\pgfqpoint{5.634670in}{2.441413in}}%
\pgfpathlineto{\pgfqpoint{5.635065in}{2.457868in}}%
\pgfpathlineto{\pgfqpoint{5.635657in}{2.446485in}}%
\pgfpathlineto{\pgfqpoint{5.637335in}{2.475928in}}%
\pgfpathlineto{\pgfqpoint{5.637434in}{2.473573in}}%
\pgfpathlineto{\pgfqpoint{5.637829in}{2.461568in}}%
\pgfpathlineto{\pgfqpoint{5.638322in}{2.481970in}}%
\pgfpathlineto{\pgfqpoint{5.640198in}{2.523717in}}%
\pgfpathlineto{\pgfqpoint{5.640395in}{2.516104in}}%
\pgfpathlineto{\pgfqpoint{5.642369in}{2.431723in}}%
\pgfpathlineto{\pgfqpoint{5.642468in}{2.434317in}}%
\pgfpathlineto{\pgfqpoint{5.643455in}{2.461313in}}%
\pgfpathlineto{\pgfqpoint{5.643751in}{2.451053in}}%
\pgfpathlineto{\pgfqpoint{5.645133in}{2.411620in}}%
\pgfpathlineto{\pgfqpoint{5.645330in}{2.415542in}}%
\pgfpathlineto{\pgfqpoint{5.646712in}{2.472753in}}%
\pgfpathlineto{\pgfqpoint{5.647107in}{2.446204in}}%
\pgfpathlineto{\pgfqpoint{5.647797in}{2.415094in}}%
\pgfpathlineto{\pgfqpoint{5.648291in}{2.433788in}}%
\pgfpathlineto{\pgfqpoint{5.648488in}{2.431323in}}%
\pgfpathlineto{\pgfqpoint{5.648686in}{2.434157in}}%
\pgfpathlineto{\pgfqpoint{5.650364in}{2.586507in}}%
\pgfpathlineto{\pgfqpoint{5.651351in}{2.678558in}}%
\pgfpathlineto{\pgfqpoint{5.651745in}{2.647783in}}%
\pgfpathlineto{\pgfqpoint{5.652239in}{2.610769in}}%
\pgfpathlineto{\pgfqpoint{5.653028in}{2.393212in}}%
\pgfpathlineto{\pgfqpoint{5.653917in}{2.435326in}}%
\pgfpathlineto{\pgfqpoint{5.654015in}{2.433992in}}%
\pgfpathlineto{\pgfqpoint{5.654312in}{2.441409in}}%
\pgfpathlineto{\pgfqpoint{5.656187in}{2.500370in}}%
\pgfpathlineto{\pgfqpoint{5.656286in}{2.499708in}}%
\pgfpathlineto{\pgfqpoint{5.656779in}{2.456320in}}%
\pgfpathlineto{\pgfqpoint{5.657865in}{2.420654in}}%
\pgfpathlineto{\pgfqpoint{5.658062in}{2.428305in}}%
\pgfpathlineto{\pgfqpoint{5.659148in}{2.479733in}}%
\pgfpathlineto{\pgfqpoint{5.659543in}{2.462346in}}%
\pgfpathlineto{\pgfqpoint{5.660530in}{2.441378in}}%
\pgfpathlineto{\pgfqpoint{5.660826in}{2.452811in}}%
\pgfpathlineto{\pgfqpoint{5.661714in}{2.480121in}}%
\pgfpathlineto{\pgfqpoint{5.662405in}{2.469679in}}%
\pgfpathlineto{\pgfqpoint{5.662504in}{2.470413in}}%
\pgfpathlineto{\pgfqpoint{5.662800in}{2.466294in}}%
\pgfpathlineto{\pgfqpoint{5.664083in}{2.446800in}}%
\pgfpathlineto{\pgfqpoint{5.664280in}{2.452854in}}%
\pgfpathlineto{\pgfqpoint{5.664774in}{2.472679in}}%
\pgfpathlineto{\pgfqpoint{5.665662in}{2.470247in}}%
\pgfpathlineto{\pgfqpoint{5.665761in}{2.471770in}}%
\pgfpathlineto{\pgfqpoint{5.666353in}{2.466145in}}%
\pgfpathlineto{\pgfqpoint{5.667439in}{2.455983in}}%
\pgfpathlineto{\pgfqpoint{5.667833in}{2.461631in}}%
\pgfpathlineto{\pgfqpoint{5.668031in}{2.462429in}}%
\pgfpathlineto{\pgfqpoint{5.668327in}{2.458314in}}%
\pgfpathlineto{\pgfqpoint{5.669511in}{2.431407in}}%
\pgfpathlineto{\pgfqpoint{5.669709in}{2.438643in}}%
\pgfpathlineto{\pgfqpoint{5.671090in}{2.464167in}}%
\pgfpathlineto{\pgfqpoint{5.672867in}{2.444633in}}%
\pgfpathlineto{\pgfqpoint{5.673064in}{2.450406in}}%
\pgfpathlineto{\pgfqpoint{5.673262in}{2.455292in}}%
\pgfpathlineto{\pgfqpoint{5.673755in}{2.442729in}}%
\pgfpathlineto{\pgfqpoint{5.674051in}{2.446164in}}%
\pgfpathlineto{\pgfqpoint{5.674150in}{2.446842in}}%
\pgfpathlineto{\pgfqpoint{5.674347in}{2.442303in}}%
\pgfpathlineto{\pgfqpoint{5.676716in}{2.363365in}}%
\pgfpathlineto{\pgfqpoint{5.676914in}{2.365441in}}%
\pgfpathlineto{\pgfqpoint{5.679184in}{2.411953in}}%
\pgfpathlineto{\pgfqpoint{5.680072in}{2.409767in}}%
\pgfpathlineto{\pgfqpoint{5.681750in}{2.387357in}}%
\pgfpathlineto{\pgfqpoint{5.682145in}{2.375066in}}%
\pgfpathlineto{\pgfqpoint{5.682539in}{2.398289in}}%
\pgfpathlineto{\pgfqpoint{5.682737in}{2.403316in}}%
\pgfpathlineto{\pgfqpoint{5.683132in}{2.381663in}}%
\pgfpathlineto{\pgfqpoint{5.683428in}{2.389338in}}%
\pgfpathlineto{\pgfqpoint{5.683625in}{2.391388in}}%
\pgfpathlineto{\pgfqpoint{5.684119in}{2.383478in}}%
\pgfpathlineto{\pgfqpoint{5.685106in}{2.367223in}}%
\pgfpathlineto{\pgfqpoint{5.685303in}{2.376877in}}%
\pgfpathlineto{\pgfqpoint{5.686389in}{2.428927in}}%
\pgfpathlineto{\pgfqpoint{5.687080in}{2.426711in}}%
\pgfpathlineto{\pgfqpoint{5.687376in}{2.424872in}}%
\pgfpathlineto{\pgfqpoint{5.687474in}{2.426450in}}%
\pgfpathlineto{\pgfqpoint{5.688363in}{2.442447in}}%
\pgfpathlineto{\pgfqpoint{5.688757in}{2.434614in}}%
\pgfpathlineto{\pgfqpoint{5.688955in}{2.435430in}}%
\pgfpathlineto{\pgfqpoint{5.689054in}{2.434266in}}%
\pgfpathlineto{\pgfqpoint{5.690139in}{2.395496in}}%
\pgfpathlineto{\pgfqpoint{5.690830in}{2.403455in}}%
\pgfpathlineto{\pgfqpoint{5.691225in}{2.420625in}}%
\pgfpathlineto{\pgfqpoint{5.692015in}{2.410020in}}%
\pgfpathlineto{\pgfqpoint{5.693396in}{2.385446in}}%
\pgfpathlineto{\pgfqpoint{5.693495in}{2.387725in}}%
\pgfpathlineto{\pgfqpoint{5.694581in}{2.429898in}}%
\pgfpathlineto{\pgfqpoint{5.695173in}{2.415486in}}%
\pgfpathlineto{\pgfqpoint{5.696456in}{2.387149in}}%
\pgfpathlineto{\pgfqpoint{5.696555in}{2.387372in}}%
\pgfpathlineto{\pgfqpoint{5.696851in}{2.387156in}}%
\pgfpathlineto{\pgfqpoint{5.697838in}{2.415322in}}%
\pgfpathlineto{\pgfqpoint{5.699417in}{2.649164in}}%
\pgfpathlineto{\pgfqpoint{5.699812in}{2.618211in}}%
\pgfpathlineto{\pgfqpoint{5.701194in}{2.359566in}}%
\pgfpathlineto{\pgfqpoint{5.702773in}{2.397215in}}%
\pgfpathlineto{\pgfqpoint{5.703069in}{2.392114in}}%
\pgfpathlineto{\pgfqpoint{5.703464in}{2.401096in}}%
\pgfpathlineto{\pgfqpoint{5.704352in}{2.458603in}}%
\pgfpathlineto{\pgfqpoint{5.704944in}{2.429921in}}%
\pgfpathlineto{\pgfqpoint{5.705635in}{2.419479in}}%
\pgfpathlineto{\pgfqpoint{5.705931in}{2.428715in}}%
\pgfpathlineto{\pgfqpoint{5.707510in}{2.469092in}}%
\pgfpathlineto{\pgfqpoint{5.707708in}{2.466491in}}%
\pgfpathlineto{\pgfqpoint{5.709879in}{2.435550in}}%
\pgfpathlineto{\pgfqpoint{5.709978in}{2.437446in}}%
\pgfpathlineto{\pgfqpoint{5.710767in}{2.471062in}}%
\pgfpathlineto{\pgfqpoint{5.711656in}{2.467879in}}%
\pgfpathlineto{\pgfqpoint{5.711952in}{2.458476in}}%
\pgfpathlineto{\pgfqpoint{5.712445in}{2.472610in}}%
\pgfpathlineto{\pgfqpoint{5.712840in}{2.464765in}}%
\pgfpathlineto{\pgfqpoint{5.713630in}{2.482237in}}%
\pgfpathlineto{\pgfqpoint{5.714024in}{2.497346in}}%
\pgfpathlineto{\pgfqpoint{5.714617in}{2.479528in}}%
\pgfpathlineto{\pgfqpoint{5.714715in}{2.478684in}}%
\pgfpathlineto{\pgfqpoint{5.714814in}{2.481795in}}%
\pgfpathlineto{\pgfqpoint{5.715998in}{2.495742in}}%
\pgfpathlineto{\pgfqpoint{5.715505in}{2.477177in}}%
\pgfpathlineto{\pgfqpoint{5.716097in}{2.490911in}}%
\pgfpathlineto{\pgfqpoint{5.716393in}{2.472068in}}%
\pgfpathlineto{\pgfqpoint{5.716788in}{2.502540in}}%
\pgfpathlineto{\pgfqpoint{5.717281in}{2.482725in}}%
\pgfpathlineto{\pgfqpoint{5.717676in}{2.505376in}}%
\pgfpathlineto{\pgfqpoint{5.718466in}{2.492738in}}%
\pgfpathlineto{\pgfqpoint{5.719749in}{2.475177in}}%
\pgfpathlineto{\pgfqpoint{5.719946in}{2.480174in}}%
\pgfpathlineto{\pgfqpoint{5.720341in}{2.501341in}}%
\pgfpathlineto{\pgfqpoint{5.721032in}{2.483071in}}%
\pgfpathlineto{\pgfqpoint{5.722315in}{2.467578in}}%
\pgfpathlineto{\pgfqpoint{5.722513in}{2.475141in}}%
\pgfpathlineto{\pgfqpoint{5.722611in}{2.478378in}}%
\pgfpathlineto{\pgfqpoint{5.723105in}{2.465202in}}%
\pgfpathlineto{\pgfqpoint{5.723302in}{2.466922in}}%
\pgfpathlineto{\pgfqpoint{5.723697in}{2.456694in}}%
\pgfpathlineto{\pgfqpoint{5.725868in}{2.419800in}}%
\pgfpathlineto{\pgfqpoint{5.726362in}{2.426606in}}%
\pgfpathlineto{\pgfqpoint{5.726855in}{2.434111in}}%
\pgfpathlineto{\pgfqpoint{5.727250in}{2.427847in}}%
\pgfpathlineto{\pgfqpoint{5.728533in}{2.407417in}}%
\pgfpathlineto{\pgfqpoint{5.728731in}{2.418401in}}%
\pgfpathlineto{\pgfqpoint{5.730014in}{2.433677in}}%
\pgfpathlineto{\pgfqpoint{5.729224in}{2.413990in}}%
\pgfpathlineto{\pgfqpoint{5.730112in}{2.433265in}}%
\pgfpathlineto{\pgfqpoint{5.731395in}{2.415089in}}%
\pgfpathlineto{\pgfqpoint{5.731692in}{2.423046in}}%
\pgfpathlineto{\pgfqpoint{5.732777in}{2.430344in}}%
\pgfpathlineto{\pgfqpoint{5.732284in}{2.417449in}}%
\pgfpathlineto{\pgfqpoint{5.732876in}{2.429081in}}%
\pgfpathlineto{\pgfqpoint{5.732975in}{2.428641in}}%
\pgfpathlineto{\pgfqpoint{5.733073in}{2.429602in}}%
\pgfpathlineto{\pgfqpoint{5.734554in}{2.448777in}}%
\pgfpathlineto{\pgfqpoint{5.734653in}{2.447271in}}%
\pgfpathlineto{\pgfqpoint{5.734850in}{2.442901in}}%
\pgfpathlineto{\pgfqpoint{5.735245in}{2.456354in}}%
\pgfpathlineto{\pgfqpoint{5.736528in}{2.478299in}}%
\pgfpathlineto{\pgfqpoint{5.736626in}{2.476305in}}%
\pgfpathlineto{\pgfqpoint{5.738403in}{2.409567in}}%
\pgfpathlineto{\pgfqpoint{5.738600in}{2.416891in}}%
\pgfpathlineto{\pgfqpoint{5.738995in}{2.444027in}}%
\pgfpathlineto{\pgfqpoint{5.739686in}{2.422622in}}%
\pgfpathlineto{\pgfqpoint{5.740772in}{2.372637in}}%
\pgfpathlineto{\pgfqpoint{5.741364in}{2.377804in}}%
\pgfpathlineto{\pgfqpoint{5.742154in}{2.422835in}}%
\pgfpathlineto{\pgfqpoint{5.742450in}{2.432955in}}%
\pgfpathlineto{\pgfqpoint{5.743042in}{2.413207in}}%
\pgfpathlineto{\pgfqpoint{5.744621in}{2.335290in}}%
\pgfpathlineto{\pgfqpoint{5.744917in}{2.351155in}}%
\pgfpathlineto{\pgfqpoint{5.745411in}{2.336064in}}%
\pgfpathlineto{\pgfqpoint{5.746299in}{2.483220in}}%
\pgfpathlineto{\pgfqpoint{5.747286in}{2.661037in}}%
\pgfpathlineto{\pgfqpoint{5.747878in}{2.644428in}}%
\pgfpathlineto{\pgfqpoint{5.748372in}{2.591941in}}%
\pgfpathlineto{\pgfqpoint{5.749063in}{2.376930in}}%
\pgfpathlineto{\pgfqpoint{5.750050in}{2.414826in}}%
\pgfpathlineto{\pgfqpoint{5.750839in}{2.426328in}}%
\pgfpathlineto{\pgfqpoint{5.752320in}{2.486767in}}%
\pgfpathlineto{\pgfqpoint{5.752714in}{2.454612in}}%
\pgfpathlineto{\pgfqpoint{5.753405in}{2.423844in}}%
\pgfpathlineto{\pgfqpoint{5.753899in}{2.448947in}}%
\pgfpathlineto{\pgfqpoint{5.755478in}{2.488679in}}%
\pgfpathlineto{\pgfqpoint{5.754392in}{2.441074in}}%
\pgfpathlineto{\pgfqpoint{5.755971in}{2.467092in}}%
\pgfpathlineto{\pgfqpoint{5.756268in}{2.456389in}}%
\pgfpathlineto{\pgfqpoint{5.757156in}{2.460655in}}%
\pgfpathlineto{\pgfqpoint{5.757945in}{2.480386in}}%
\pgfpathlineto{\pgfqpoint{5.758340in}{2.494083in}}%
\pgfpathlineto{\pgfqpoint{5.758932in}{2.477453in}}%
\pgfpathlineto{\pgfqpoint{5.759130in}{2.477034in}}%
\pgfpathlineto{\pgfqpoint{5.759623in}{2.478036in}}%
\pgfpathlineto{\pgfqpoint{5.761005in}{2.507361in}}%
\pgfpathlineto{\pgfqpoint{5.762091in}{2.497150in}}%
\pgfpathlineto{\pgfqpoint{5.762288in}{2.493994in}}%
\pgfpathlineto{\pgfqpoint{5.762782in}{2.505240in}}%
\pgfpathlineto{\pgfqpoint{5.764163in}{2.525176in}}%
\pgfpathlineto{\pgfqpoint{5.764361in}{2.516102in}}%
\pgfpathlineto{\pgfqpoint{5.765447in}{2.494758in}}%
\pgfpathlineto{\pgfqpoint{5.765644in}{2.501076in}}%
\pgfpathlineto{\pgfqpoint{5.766335in}{2.498484in}}%
\pgfpathlineto{\pgfqpoint{5.766927in}{2.509234in}}%
\pgfpathlineto{\pgfqpoint{5.767124in}{2.505252in}}%
\pgfpathlineto{\pgfqpoint{5.768605in}{2.463609in}}%
\pgfpathlineto{\pgfqpoint{5.768704in}{2.463746in}}%
\pgfpathlineto{\pgfqpoint{5.769789in}{2.478750in}}%
\pgfpathlineto{\pgfqpoint{5.770184in}{2.468097in}}%
\pgfpathlineto{\pgfqpoint{5.770875in}{2.464701in}}%
\pgfpathlineto{\pgfqpoint{5.770579in}{2.469956in}}%
\pgfpathlineto{\pgfqpoint{5.771072in}{2.467485in}}%
\pgfpathlineto{\pgfqpoint{5.771171in}{2.469247in}}%
\pgfpathlineto{\pgfqpoint{5.771566in}{2.457360in}}%
\pgfpathlineto{\pgfqpoint{5.772257in}{2.436023in}}%
\pgfpathlineto{\pgfqpoint{5.772652in}{2.452050in}}%
\pgfpathlineto{\pgfqpoint{5.772750in}{2.454615in}}%
\pgfpathlineto{\pgfqpoint{5.773145in}{2.441524in}}%
\pgfpathlineto{\pgfqpoint{5.774231in}{2.427130in}}%
\pgfpathlineto{\pgfqpoint{5.774527in}{2.429574in}}%
\pgfpathlineto{\pgfqpoint{5.774626in}{2.429666in}}%
\pgfpathlineto{\pgfqpoint{5.775218in}{2.410886in}}%
\pgfpathlineto{\pgfqpoint{5.776007in}{2.423362in}}%
\pgfpathlineto{\pgfqpoint{5.776303in}{2.428131in}}%
\pgfpathlineto{\pgfqpoint{5.776501in}{2.423654in}}%
\pgfpathlineto{\pgfqpoint{5.777883in}{2.401921in}}%
\pgfpathlineto{\pgfqpoint{5.777981in}{2.402077in}}%
\pgfpathlineto{\pgfqpoint{5.778277in}{2.407872in}}%
\pgfpathlineto{\pgfqpoint{5.778672in}{2.400125in}}%
\pgfpathlineto{\pgfqpoint{5.779067in}{2.403138in}}%
\pgfpathlineto{\pgfqpoint{5.779166in}{2.402871in}}%
\pgfpathlineto{\pgfqpoint{5.779264in}{2.403597in}}%
\pgfpathlineto{\pgfqpoint{5.780153in}{2.413567in}}%
\pgfpathlineto{\pgfqpoint{5.780449in}{2.405042in}}%
\pgfpathlineto{\pgfqpoint{5.780646in}{2.398845in}}%
\pgfpathlineto{\pgfqpoint{5.781041in}{2.410948in}}%
\pgfpathlineto{\pgfqpoint{5.781436in}{2.407496in}}%
\pgfpathlineto{\pgfqpoint{5.783212in}{2.454139in}}%
\pgfpathlineto{\pgfqpoint{5.783805in}{2.448081in}}%
\pgfpathlineto{\pgfqpoint{5.784002in}{2.450828in}}%
\pgfpathlineto{\pgfqpoint{5.784397in}{2.441522in}}%
\pgfpathlineto{\pgfqpoint{5.786963in}{2.405206in}}%
\pgfpathlineto{\pgfqpoint{5.787358in}{2.411774in}}%
\pgfpathlineto{\pgfqpoint{5.787654in}{2.403695in}}%
\pgfpathlineto{\pgfqpoint{5.788246in}{2.385252in}}%
\pgfpathlineto{\pgfqpoint{5.788937in}{2.393958in}}%
\pgfpathlineto{\pgfqpoint{5.789036in}{2.394058in}}%
\pgfpathlineto{\pgfqpoint{5.789134in}{2.393277in}}%
\pgfpathlineto{\pgfqpoint{5.789332in}{2.391759in}}%
\pgfpathlineto{\pgfqpoint{5.789628in}{2.397561in}}%
\pgfpathlineto{\pgfqpoint{5.790615in}{2.434698in}}%
\pgfpathlineto{\pgfqpoint{5.790911in}{2.411958in}}%
\pgfpathlineto{\pgfqpoint{5.791306in}{2.384231in}}%
\pgfpathlineto{\pgfqpoint{5.792095in}{2.388597in}}%
\pgfpathlineto{\pgfqpoint{5.793872in}{2.489569in}}%
\pgfpathlineto{\pgfqpoint{5.794958in}{2.664512in}}%
\pgfpathlineto{\pgfqpoint{5.795451in}{2.622575in}}%
\pgfpathlineto{\pgfqpoint{5.796142in}{2.538513in}}%
\pgfpathlineto{\pgfqpoint{5.796833in}{2.368358in}}%
\pgfpathlineto{\pgfqpoint{5.797524in}{2.425359in}}%
\pgfpathlineto{\pgfqpoint{5.798412in}{2.415871in}}%
\pgfpathlineto{\pgfqpoint{5.798609in}{2.421311in}}%
\pgfpathlineto{\pgfqpoint{5.799991in}{2.477251in}}%
\pgfpathlineto{\pgfqpoint{5.800189in}{2.463595in}}%
\pgfpathlineto{\pgfqpoint{5.801176in}{2.422380in}}%
\pgfpathlineto{\pgfqpoint{5.801472in}{2.435584in}}%
\pgfpathlineto{\pgfqpoint{5.803051in}{2.470241in}}%
\pgfpathlineto{\pgfqpoint{5.801965in}{2.430627in}}%
\pgfpathlineto{\pgfqpoint{5.803150in}{2.467735in}}%
\pgfpathlineto{\pgfqpoint{5.804235in}{2.443424in}}%
\pgfpathlineto{\pgfqpoint{5.804630in}{2.451734in}}%
\pgfpathlineto{\pgfqpoint{5.805321in}{2.442900in}}%
\pgfpathlineto{\pgfqpoint{5.805814in}{2.456844in}}%
\pgfpathlineto{\pgfqpoint{5.805913in}{2.456354in}}%
\pgfpathlineto{\pgfqpoint{5.806111in}{2.458402in}}%
\pgfpathlineto{\pgfqpoint{5.807492in}{2.474518in}}%
\pgfpathlineto{\pgfqpoint{5.807591in}{2.472410in}}%
\pgfpathlineto{\pgfqpoint{5.807887in}{2.464058in}}%
\pgfpathlineto{\pgfqpoint{5.808578in}{2.469510in}}%
\pgfpathlineto{\pgfqpoint{5.808874in}{2.481862in}}%
\pgfpathlineto{\pgfqpoint{5.809368in}{2.461607in}}%
\pgfpathlineto{\pgfqpoint{5.809565in}{2.463401in}}%
\pgfpathlineto{\pgfqpoint{5.809664in}{2.463663in}}%
\pgfpathlineto{\pgfqpoint{5.809762in}{2.462299in}}%
\pgfpathlineto{\pgfqpoint{5.811045in}{2.420334in}}%
\pgfpathlineto{\pgfqpoint{5.811342in}{2.404173in}}%
\pgfpathlineto{\pgfqpoint{5.812230in}{2.413483in}}%
\pgfpathlineto{\pgfqpoint{5.813908in}{2.493867in}}%
\pgfpathlineto{\pgfqpoint{5.815290in}{2.486528in}}%
\pgfpathlineto{\pgfqpoint{5.815684in}{2.475628in}}%
\pgfpathlineto{\pgfqpoint{5.816474in}{2.482276in}}%
\pgfpathlineto{\pgfqpoint{5.817066in}{2.484883in}}%
\pgfpathlineto{\pgfqpoint{5.817658in}{2.470328in}}%
\pgfpathlineto{\pgfqpoint{5.818152in}{2.455633in}}%
\pgfpathlineto{\pgfqpoint{5.818744in}{2.470013in}}%
\pgfpathlineto{\pgfqpoint{5.818941in}{2.470330in}}%
\pgfpathlineto{\pgfqpoint{5.819139in}{2.469592in}}%
\pgfpathlineto{\pgfqpoint{5.821014in}{2.434013in}}%
\pgfpathlineto{\pgfqpoint{5.821705in}{2.434322in}}%
\pgfpathlineto{\pgfqpoint{5.821804in}{2.434188in}}%
\pgfpathlineto{\pgfqpoint{5.821902in}{2.435857in}}%
\pgfpathlineto{\pgfqpoint{5.822100in}{2.439145in}}%
\pgfpathlineto{\pgfqpoint{5.822593in}{2.428584in}}%
\pgfpathlineto{\pgfqpoint{5.824666in}{2.408205in}}%
\pgfpathlineto{\pgfqpoint{5.824863in}{2.410835in}}%
\pgfpathlineto{\pgfqpoint{5.826245in}{2.420434in}}%
\pgfpathlineto{\pgfqpoint{5.825258in}{2.409581in}}%
\pgfpathlineto{\pgfqpoint{5.826443in}{2.416889in}}%
\pgfpathlineto{\pgfqpoint{5.827430in}{2.396588in}}%
\pgfpathlineto{\pgfqpoint{5.827726in}{2.405711in}}%
\pgfpathlineto{\pgfqpoint{5.829107in}{2.427695in}}%
\pgfpathlineto{\pgfqpoint{5.829206in}{2.426596in}}%
\pgfpathlineto{\pgfqpoint{5.829601in}{2.413713in}}%
\pgfpathlineto{\pgfqpoint{5.829996in}{2.436302in}}%
\pgfpathlineto{\pgfqpoint{5.830687in}{2.426597in}}%
\pgfpathlineto{\pgfqpoint{5.831279in}{2.443027in}}%
\pgfpathlineto{\pgfqpoint{5.832167in}{2.438016in}}%
\pgfpathlineto{\pgfqpoint{5.833648in}{2.369238in}}%
\pgfpathlineto{\pgfqpoint{5.834240in}{2.378192in}}%
\pgfpathlineto{\pgfqpoint{5.834733in}{2.386623in}}%
\pgfpathlineto{\pgfqpoint{5.835029in}{2.378994in}}%
\pgfpathlineto{\pgfqpoint{5.836016in}{2.336304in}}%
\pgfpathlineto{\pgfqpoint{5.836510in}{2.348624in}}%
\pgfpathlineto{\pgfqpoint{5.837398in}{2.374762in}}%
\pgfpathlineto{\pgfqpoint{5.837793in}{2.401067in}}%
\pgfpathlineto{\pgfqpoint{5.838385in}{2.374933in}}%
\pgfpathlineto{\pgfqpoint{5.839668in}{2.344340in}}%
\pgfpathlineto{\pgfqpoint{5.839866in}{2.351415in}}%
\pgfpathlineto{\pgfqpoint{5.841543in}{2.526631in}}%
\pgfpathlineto{\pgfqpoint{5.842136in}{2.627159in}}%
\pgfpathlineto{\pgfqpoint{5.842827in}{2.593656in}}%
\pgfpathlineto{\pgfqpoint{5.843616in}{2.505299in}}%
\pgfpathlineto{\pgfqpoint{5.844208in}{2.341400in}}%
\pgfpathlineto{\pgfqpoint{5.844998in}{2.387361in}}%
\pgfpathlineto{\pgfqpoint{5.845491in}{2.370996in}}%
\pgfpathlineto{\pgfqpoint{5.846084in}{2.386224in}}%
\pgfpathlineto{\pgfqpoint{5.846281in}{2.386000in}}%
\pgfpathlineto{\pgfqpoint{5.846380in}{2.387098in}}%
\pgfpathlineto{\pgfqpoint{5.847071in}{2.434168in}}%
\pgfpathlineto{\pgfqpoint{5.847268in}{2.444098in}}%
\pgfpathlineto{\pgfqpoint{5.847959in}{2.419714in}}%
\pgfpathlineto{\pgfqpoint{5.848748in}{2.396491in}}%
\pgfpathlineto{\pgfqpoint{5.849341in}{2.407749in}}%
\pgfpathlineto{\pgfqpoint{5.850525in}{2.456543in}}%
\pgfpathlineto{\pgfqpoint{5.851019in}{2.436219in}}%
\pgfpathlineto{\pgfqpoint{5.851611in}{2.414323in}}%
\pgfpathlineto{\pgfqpoint{5.852598in}{2.419981in}}%
\pgfpathlineto{\pgfqpoint{5.853585in}{2.463995in}}%
\pgfpathlineto{\pgfqpoint{5.854374in}{2.453174in}}%
\pgfpathlineto{\pgfqpoint{5.855361in}{2.432149in}}%
\pgfpathlineto{\pgfqpoint{5.855657in}{2.443208in}}%
\pgfpathlineto{\pgfqpoint{5.856546in}{2.470737in}}%
\pgfpathlineto{\pgfqpoint{5.857237in}{2.467972in}}%
\pgfpathlineto{\pgfqpoint{5.857533in}{2.464058in}}%
\pgfpathlineto{\pgfqpoint{5.857829in}{2.475390in}}%
\pgfpathlineto{\pgfqpoint{5.857927in}{2.478643in}}%
\pgfpathlineto{\pgfqpoint{5.858224in}{2.464032in}}%
\pgfpathlineto{\pgfqpoint{5.858520in}{2.445987in}}%
\pgfpathlineto{\pgfqpoint{5.858914in}{2.471259in}}%
\pgfpathlineto{\pgfqpoint{5.859309in}{2.456956in}}%
\pgfpathlineto{\pgfqpoint{5.859803in}{2.476225in}}%
\pgfpathlineto{\pgfqpoint{5.860790in}{2.462289in}}%
\pgfpathlineto{\pgfqpoint{5.861086in}{2.458463in}}%
\pgfpathlineto{\pgfqpoint{5.861382in}{2.467607in}}%
\pgfpathlineto{\pgfqpoint{5.862468in}{2.478932in}}%
\pgfpathlineto{\pgfqpoint{5.861974in}{2.462398in}}%
\pgfpathlineto{\pgfqpoint{5.862566in}{2.475997in}}%
\pgfpathlineto{\pgfqpoint{5.863849in}{2.460488in}}%
\pgfpathlineto{\pgfqpoint{5.863948in}{2.460564in}}%
\pgfpathlineto{\pgfqpoint{5.864244in}{2.463541in}}%
\pgfpathlineto{\pgfqpoint{5.864639in}{2.457525in}}%
\pgfpathlineto{\pgfqpoint{5.865527in}{2.447676in}}%
\pgfpathlineto{\pgfqpoint{5.865823in}{2.453621in}}%
\pgfpathlineto{\pgfqpoint{5.866120in}{2.462190in}}%
\pgfpathlineto{\pgfqpoint{5.866514in}{2.441085in}}%
\pgfpathlineto{\pgfqpoint{5.867304in}{2.419472in}}%
\pgfpathlineto{\pgfqpoint{5.868192in}{2.423327in}}%
\pgfpathlineto{\pgfqpoint{5.868390in}{2.420748in}}%
\pgfpathlineto{\pgfqpoint{5.868784in}{2.405354in}}%
\pgfpathlineto{\pgfqpoint{5.869673in}{2.407239in}}%
\pgfpathlineto{\pgfqpoint{5.869870in}{2.408648in}}%
\pgfpathlineto{\pgfqpoint{5.870067in}{2.405210in}}%
\pgfpathlineto{\pgfqpoint{5.870956in}{2.385567in}}%
\pgfpathlineto{\pgfqpoint{5.871252in}{2.394199in}}%
\pgfpathlineto{\pgfqpoint{5.871943in}{2.393403in}}%
\pgfpathlineto{\pgfqpoint{5.872535in}{2.405785in}}%
\pgfpathlineto{\pgfqpoint{5.874608in}{2.373142in}}%
\pgfpathlineto{\pgfqpoint{5.874805in}{2.384585in}}%
\pgfpathlineto{\pgfqpoint{5.875693in}{2.405853in}}%
\pgfpathlineto{\pgfqpoint{5.875891in}{2.397519in}}%
\pgfpathlineto{\pgfqpoint{5.877174in}{2.356574in}}%
\pgfpathlineto{\pgfqpoint{5.877470in}{2.360192in}}%
\pgfpathlineto{\pgfqpoint{5.878259in}{2.356884in}}%
\pgfpathlineto{\pgfqpoint{5.879345in}{2.398205in}}%
\pgfpathlineto{\pgfqpoint{5.879543in}{2.396321in}}%
\pgfpathlineto{\pgfqpoint{5.879641in}{2.394674in}}%
\pgfpathlineto{\pgfqpoint{5.880036in}{2.404805in}}%
\pgfpathlineto{\pgfqpoint{5.880135in}{2.405380in}}%
\pgfpathlineto{\pgfqpoint{5.880332in}{2.400458in}}%
\pgfpathlineto{\pgfqpoint{5.880431in}{2.399559in}}%
\pgfpathlineto{\pgfqpoint{5.880628in}{2.406734in}}%
\pgfpathlineto{\pgfqpoint{5.881813in}{2.415204in}}%
\pgfpathlineto{\pgfqpoint{5.881220in}{2.397277in}}%
\pgfpathlineto{\pgfqpoint{5.881911in}{2.414654in}}%
\pgfpathlineto{\pgfqpoint{5.882800in}{2.380203in}}%
\pgfpathlineto{\pgfqpoint{5.883194in}{2.361743in}}%
\pgfpathlineto{\pgfqpoint{5.883984in}{2.369879in}}%
\pgfpathlineto{\pgfqpoint{5.885366in}{2.417272in}}%
\pgfpathlineto{\pgfqpoint{5.885761in}{2.402160in}}%
\pgfpathlineto{\pgfqpoint{5.886353in}{2.362136in}}%
\pgfpathlineto{\pgfqpoint{5.887241in}{2.364269in}}%
\pgfpathlineto{\pgfqpoint{5.888327in}{2.410116in}}%
\pgfpathlineto{\pgfqpoint{5.890103in}{2.641601in}}%
\pgfpathlineto{\pgfqpoint{5.890893in}{2.594458in}}%
\pgfpathlineto{\pgfqpoint{5.891683in}{2.354016in}}%
\pgfpathlineto{\pgfqpoint{5.892670in}{2.398967in}}%
\pgfpathlineto{\pgfqpoint{5.894841in}{2.476312in}}%
\pgfpathlineto{\pgfqpoint{5.895334in}{2.441811in}}%
\pgfpathlineto{\pgfqpoint{5.896025in}{2.427947in}}%
\pgfpathlineto{\pgfqpoint{5.896519in}{2.434095in}}%
\pgfpathlineto{\pgfqpoint{5.897901in}{2.466224in}}%
\pgfpathlineto{\pgfqpoint{5.898295in}{2.454685in}}%
\pgfpathlineto{\pgfqpoint{5.899381in}{2.431221in}}%
\pgfpathlineto{\pgfqpoint{5.899480in}{2.433385in}}%
\pgfusepath{stroke}%
\end{pgfscope}%
\begin{pgfscope}%
\pgfsetrectcap%
\pgfsetmiterjoin%
\pgfsetlinewidth{0.803000pt}%
\definecolor{currentstroke}{rgb}{0.000000,0.000000,0.000000}%
\pgfsetstrokecolor{currentstroke}%
\pgfsetdash{}{0pt}%
\pgfpathmoveto{\pgfqpoint{0.717889in}{2.114143in}}%
\pgfpathlineto{\pgfqpoint{0.717889in}{2.901359in}}%
\pgfusepath{stroke}%
\end{pgfscope}%
\begin{pgfscope}%
\pgfsetrectcap%
\pgfsetmiterjoin%
\pgfsetlinewidth{0.803000pt}%
\definecolor{currentstroke}{rgb}{0.000000,0.000000,0.000000}%
\pgfsetstrokecolor{currentstroke}%
\pgfsetdash{}{0pt}%
\pgfpathmoveto{\pgfqpoint{6.146222in}{2.114143in}}%
\pgfpathlineto{\pgfqpoint{6.146222in}{2.901359in}}%
\pgfusepath{stroke}%
\end{pgfscope}%
\begin{pgfscope}%
\pgfsetrectcap%
\pgfsetmiterjoin%
\pgfsetlinewidth{0.803000pt}%
\definecolor{currentstroke}{rgb}{0.000000,0.000000,0.000000}%
\pgfsetstrokecolor{currentstroke}%
\pgfsetdash{}{0pt}%
\pgfpathmoveto{\pgfqpoint{0.717889in}{2.114143in}}%
\pgfpathlineto{\pgfqpoint{6.146222in}{2.114143in}}%
\pgfusepath{stroke}%
\end{pgfscope}%
\begin{pgfscope}%
\pgfsetrectcap%
\pgfsetmiterjoin%
\pgfsetlinewidth{0.803000pt}%
\definecolor{currentstroke}{rgb}{0.000000,0.000000,0.000000}%
\pgfsetstrokecolor{currentstroke}%
\pgfsetdash{}{0pt}%
\pgfpathmoveto{\pgfqpoint{0.717889in}{2.901359in}}%
\pgfpathlineto{\pgfqpoint{6.146222in}{2.901359in}}%
\pgfusepath{stroke}%
\end{pgfscope}%
\begin{pgfscope}%
\definecolor{textcolor}{rgb}{0.000000,0.000000,0.000000}%
\pgfsetstrokecolor{textcolor}%
\pgfsetfillcolor{textcolor}%
\pgftext[x=3.432055in,y=2.984692in,,base]{\color{textcolor}\rmfamily\fontsize{12.000000}{14.400000}\selectfont Filtered ECG Signal}%
\end{pgfscope}%
\begin{pgfscope}%
\pgfsetbuttcap%
\pgfsetmiterjoin%
\definecolor{currentfill}{rgb}{1.000000,1.000000,1.000000}%
\pgfsetfillcolor{currentfill}%
\pgfsetlinewidth{0.000000pt}%
\definecolor{currentstroke}{rgb}{0.000000,0.000000,0.000000}%
\pgfsetstrokecolor{currentstroke}%
\pgfsetstrokeopacity{0.000000}%
\pgfsetdash{}{0pt}%
\pgfpathmoveto{\pgfqpoint{0.717889in}{0.564143in}}%
\pgfpathlineto{\pgfqpoint{6.146222in}{0.564143in}}%
\pgfpathlineto{\pgfqpoint{6.146222in}{1.351359in}}%
\pgfpathlineto{\pgfqpoint{0.717889in}{1.351359in}}%
\pgfpathclose%
\pgfusepath{fill}%
\end{pgfscope}%
\begin{pgfscope}%
\pgfsetbuttcap%
\pgfsetroundjoin%
\definecolor{currentfill}{rgb}{0.000000,0.000000,0.000000}%
\pgfsetfillcolor{currentfill}%
\pgfsetlinewidth{0.803000pt}%
\definecolor{currentstroke}{rgb}{0.000000,0.000000,0.000000}%
\pgfsetstrokecolor{currentstroke}%
\pgfsetdash{}{0pt}%
\pgfsys@defobject{currentmarker}{\pgfqpoint{0.000000in}{-0.048611in}}{\pgfqpoint{0.000000in}{0.000000in}}{%
\pgfpathmoveto{\pgfqpoint{0.000000in}{0.000000in}}%
\pgfpathlineto{\pgfqpoint{0.000000in}{-0.048611in}}%
\pgfusepath{stroke,fill}%
}%
\begin{pgfscope}%
\pgfsys@transformshift{0.717889in}{0.564143in}%
\pgfsys@useobject{currentmarker}{}%
\end{pgfscope}%
\end{pgfscope}%
\begin{pgfscope}%
\definecolor{textcolor}{rgb}{0.000000,0.000000,0.000000}%
\pgfsetstrokecolor{textcolor}%
\pgfsetfillcolor{textcolor}%
\pgftext[x=0.717889in,y=0.466921in,,top]{\color{textcolor}\rmfamily\fontsize{10.000000}{12.000000}\selectfont \(\displaystyle {-100}\)}%
\end{pgfscope}%
\begin{pgfscope}%
\pgfsetbuttcap%
\pgfsetroundjoin%
\definecolor{currentfill}{rgb}{0.000000,0.000000,0.000000}%
\pgfsetfillcolor{currentfill}%
\pgfsetlinewidth{0.803000pt}%
\definecolor{currentstroke}{rgb}{0.000000,0.000000,0.000000}%
\pgfsetstrokecolor{currentstroke}%
\pgfsetdash{}{0pt}%
\pgfsys@defobject{currentmarker}{\pgfqpoint{0.000000in}{-0.048611in}}{\pgfqpoint{0.000000in}{0.000000in}}{%
\pgfpathmoveto{\pgfqpoint{0.000000in}{0.000000in}}%
\pgfpathlineto{\pgfqpoint{0.000000in}{-0.048611in}}%
\pgfusepath{stroke,fill}%
}%
\begin{pgfscope}%
\pgfsys@transformshift{1.396430in}{0.564143in}%
\pgfsys@useobject{currentmarker}{}%
\end{pgfscope}%
\end{pgfscope}%
\begin{pgfscope}%
\definecolor{textcolor}{rgb}{0.000000,0.000000,0.000000}%
\pgfsetstrokecolor{textcolor}%
\pgfsetfillcolor{textcolor}%
\pgftext[x=1.396430in,y=0.466921in,,top]{\color{textcolor}\rmfamily\fontsize{10.000000}{12.000000}\selectfont \(\displaystyle {-75}\)}%
\end{pgfscope}%
\begin{pgfscope}%
\pgfsetbuttcap%
\pgfsetroundjoin%
\definecolor{currentfill}{rgb}{0.000000,0.000000,0.000000}%
\pgfsetfillcolor{currentfill}%
\pgfsetlinewidth{0.803000pt}%
\definecolor{currentstroke}{rgb}{0.000000,0.000000,0.000000}%
\pgfsetstrokecolor{currentstroke}%
\pgfsetdash{}{0pt}%
\pgfsys@defobject{currentmarker}{\pgfqpoint{0.000000in}{-0.048611in}}{\pgfqpoint{0.000000in}{0.000000in}}{%
\pgfpathmoveto{\pgfqpoint{0.000000in}{0.000000in}}%
\pgfpathlineto{\pgfqpoint{0.000000in}{-0.048611in}}%
\pgfusepath{stroke,fill}%
}%
\begin{pgfscope}%
\pgfsys@transformshift{2.074972in}{0.564143in}%
\pgfsys@useobject{currentmarker}{}%
\end{pgfscope}%
\end{pgfscope}%
\begin{pgfscope}%
\definecolor{textcolor}{rgb}{0.000000,0.000000,0.000000}%
\pgfsetstrokecolor{textcolor}%
\pgfsetfillcolor{textcolor}%
\pgftext[x=2.074972in,y=0.466921in,,top]{\color{textcolor}\rmfamily\fontsize{10.000000}{12.000000}\selectfont \(\displaystyle {-50}\)}%
\end{pgfscope}%
\begin{pgfscope}%
\pgfsetbuttcap%
\pgfsetroundjoin%
\definecolor{currentfill}{rgb}{0.000000,0.000000,0.000000}%
\pgfsetfillcolor{currentfill}%
\pgfsetlinewidth{0.803000pt}%
\definecolor{currentstroke}{rgb}{0.000000,0.000000,0.000000}%
\pgfsetstrokecolor{currentstroke}%
\pgfsetdash{}{0pt}%
\pgfsys@defobject{currentmarker}{\pgfqpoint{0.000000in}{-0.048611in}}{\pgfqpoint{0.000000in}{0.000000in}}{%
\pgfpathmoveto{\pgfqpoint{0.000000in}{0.000000in}}%
\pgfpathlineto{\pgfqpoint{0.000000in}{-0.048611in}}%
\pgfusepath{stroke,fill}%
}%
\begin{pgfscope}%
\pgfsys@transformshift{2.753514in}{0.564143in}%
\pgfsys@useobject{currentmarker}{}%
\end{pgfscope}%
\end{pgfscope}%
\begin{pgfscope}%
\definecolor{textcolor}{rgb}{0.000000,0.000000,0.000000}%
\pgfsetstrokecolor{textcolor}%
\pgfsetfillcolor{textcolor}%
\pgftext[x=2.753514in,y=0.466921in,,top]{\color{textcolor}\rmfamily\fontsize{10.000000}{12.000000}\selectfont \(\displaystyle {-25}\)}%
\end{pgfscope}%
\begin{pgfscope}%
\pgfsetbuttcap%
\pgfsetroundjoin%
\definecolor{currentfill}{rgb}{0.000000,0.000000,0.000000}%
\pgfsetfillcolor{currentfill}%
\pgfsetlinewidth{0.803000pt}%
\definecolor{currentstroke}{rgb}{0.000000,0.000000,0.000000}%
\pgfsetstrokecolor{currentstroke}%
\pgfsetdash{}{0pt}%
\pgfsys@defobject{currentmarker}{\pgfqpoint{0.000000in}{-0.048611in}}{\pgfqpoint{0.000000in}{0.000000in}}{%
\pgfpathmoveto{\pgfqpoint{0.000000in}{0.000000in}}%
\pgfpathlineto{\pgfqpoint{0.000000in}{-0.048611in}}%
\pgfusepath{stroke,fill}%
}%
\begin{pgfscope}%
\pgfsys@transformshift{3.432055in}{0.564143in}%
\pgfsys@useobject{currentmarker}{}%
\end{pgfscope}%
\end{pgfscope}%
\begin{pgfscope}%
\definecolor{textcolor}{rgb}{0.000000,0.000000,0.000000}%
\pgfsetstrokecolor{textcolor}%
\pgfsetfillcolor{textcolor}%
\pgftext[x=3.432055in,y=0.466921in,,top]{\color{textcolor}\rmfamily\fontsize{10.000000}{12.000000}\selectfont \(\displaystyle {0}\)}%
\end{pgfscope}%
\begin{pgfscope}%
\pgfsetbuttcap%
\pgfsetroundjoin%
\definecolor{currentfill}{rgb}{0.000000,0.000000,0.000000}%
\pgfsetfillcolor{currentfill}%
\pgfsetlinewidth{0.803000pt}%
\definecolor{currentstroke}{rgb}{0.000000,0.000000,0.000000}%
\pgfsetstrokecolor{currentstroke}%
\pgfsetdash{}{0pt}%
\pgfsys@defobject{currentmarker}{\pgfqpoint{0.000000in}{-0.048611in}}{\pgfqpoint{0.000000in}{0.000000in}}{%
\pgfpathmoveto{\pgfqpoint{0.000000in}{0.000000in}}%
\pgfpathlineto{\pgfqpoint{0.000000in}{-0.048611in}}%
\pgfusepath{stroke,fill}%
}%
\begin{pgfscope}%
\pgfsys@transformshift{4.110597in}{0.564143in}%
\pgfsys@useobject{currentmarker}{}%
\end{pgfscope}%
\end{pgfscope}%
\begin{pgfscope}%
\definecolor{textcolor}{rgb}{0.000000,0.000000,0.000000}%
\pgfsetstrokecolor{textcolor}%
\pgfsetfillcolor{textcolor}%
\pgftext[x=4.110597in,y=0.466921in,,top]{\color{textcolor}\rmfamily\fontsize{10.000000}{12.000000}\selectfont \(\displaystyle {25}\)}%
\end{pgfscope}%
\begin{pgfscope}%
\pgfsetbuttcap%
\pgfsetroundjoin%
\definecolor{currentfill}{rgb}{0.000000,0.000000,0.000000}%
\pgfsetfillcolor{currentfill}%
\pgfsetlinewidth{0.803000pt}%
\definecolor{currentstroke}{rgb}{0.000000,0.000000,0.000000}%
\pgfsetstrokecolor{currentstroke}%
\pgfsetdash{}{0pt}%
\pgfsys@defobject{currentmarker}{\pgfqpoint{0.000000in}{-0.048611in}}{\pgfqpoint{0.000000in}{0.000000in}}{%
\pgfpathmoveto{\pgfqpoint{0.000000in}{0.000000in}}%
\pgfpathlineto{\pgfqpoint{0.000000in}{-0.048611in}}%
\pgfusepath{stroke,fill}%
}%
\begin{pgfscope}%
\pgfsys@transformshift{4.789139in}{0.564143in}%
\pgfsys@useobject{currentmarker}{}%
\end{pgfscope}%
\end{pgfscope}%
\begin{pgfscope}%
\definecolor{textcolor}{rgb}{0.000000,0.000000,0.000000}%
\pgfsetstrokecolor{textcolor}%
\pgfsetfillcolor{textcolor}%
\pgftext[x=4.789139in,y=0.466921in,,top]{\color{textcolor}\rmfamily\fontsize{10.000000}{12.000000}\selectfont \(\displaystyle {50}\)}%
\end{pgfscope}%
\begin{pgfscope}%
\pgfsetbuttcap%
\pgfsetroundjoin%
\definecolor{currentfill}{rgb}{0.000000,0.000000,0.000000}%
\pgfsetfillcolor{currentfill}%
\pgfsetlinewidth{0.803000pt}%
\definecolor{currentstroke}{rgb}{0.000000,0.000000,0.000000}%
\pgfsetstrokecolor{currentstroke}%
\pgfsetdash{}{0pt}%
\pgfsys@defobject{currentmarker}{\pgfqpoint{0.000000in}{-0.048611in}}{\pgfqpoint{0.000000in}{0.000000in}}{%
\pgfpathmoveto{\pgfqpoint{0.000000in}{0.000000in}}%
\pgfpathlineto{\pgfqpoint{0.000000in}{-0.048611in}}%
\pgfusepath{stroke,fill}%
}%
\begin{pgfscope}%
\pgfsys@transformshift{5.467680in}{0.564143in}%
\pgfsys@useobject{currentmarker}{}%
\end{pgfscope}%
\end{pgfscope}%
\begin{pgfscope}%
\definecolor{textcolor}{rgb}{0.000000,0.000000,0.000000}%
\pgfsetstrokecolor{textcolor}%
\pgfsetfillcolor{textcolor}%
\pgftext[x=5.467680in,y=0.466921in,,top]{\color{textcolor}\rmfamily\fontsize{10.000000}{12.000000}\selectfont \(\displaystyle {75}\)}%
\end{pgfscope}%
\begin{pgfscope}%
\pgfsetbuttcap%
\pgfsetroundjoin%
\definecolor{currentfill}{rgb}{0.000000,0.000000,0.000000}%
\pgfsetfillcolor{currentfill}%
\pgfsetlinewidth{0.803000pt}%
\definecolor{currentstroke}{rgb}{0.000000,0.000000,0.000000}%
\pgfsetstrokecolor{currentstroke}%
\pgfsetdash{}{0pt}%
\pgfsys@defobject{currentmarker}{\pgfqpoint{0.000000in}{-0.048611in}}{\pgfqpoint{0.000000in}{0.000000in}}{%
\pgfpathmoveto{\pgfqpoint{0.000000in}{0.000000in}}%
\pgfpathlineto{\pgfqpoint{0.000000in}{-0.048611in}}%
\pgfusepath{stroke,fill}%
}%
\begin{pgfscope}%
\pgfsys@transformshift{6.146222in}{0.564143in}%
\pgfsys@useobject{currentmarker}{}%
\end{pgfscope}%
\end{pgfscope}%
\begin{pgfscope}%
\definecolor{textcolor}{rgb}{0.000000,0.000000,0.000000}%
\pgfsetstrokecolor{textcolor}%
\pgfsetfillcolor{textcolor}%
\pgftext[x=6.146222in,y=0.466921in,,top]{\color{textcolor}\rmfamily\fontsize{10.000000}{12.000000}\selectfont \(\displaystyle {100}\)}%
\end{pgfscope}%
\begin{pgfscope}%
\definecolor{textcolor}{rgb}{0.000000,0.000000,0.000000}%
\pgfsetstrokecolor{textcolor}%
\pgfsetfillcolor{textcolor}%
\pgftext[x=3.432055in,y=0.287909in,,top]{\color{textcolor}\rmfamily\fontsize{10.000000}{12.000000}\selectfont Frequency (Hz)}%
\end{pgfscope}%
\begin{pgfscope}%
\pgfsetbuttcap%
\pgfsetroundjoin%
\definecolor{currentfill}{rgb}{0.000000,0.000000,0.000000}%
\pgfsetfillcolor{currentfill}%
\pgfsetlinewidth{0.803000pt}%
\definecolor{currentstroke}{rgb}{0.000000,0.000000,0.000000}%
\pgfsetstrokecolor{currentstroke}%
\pgfsetdash{}{0pt}%
\pgfsys@defobject{currentmarker}{\pgfqpoint{-0.048611in}{0.000000in}}{\pgfqpoint{0.000000in}{0.000000in}}{%
\pgfpathmoveto{\pgfqpoint{0.000000in}{0.000000in}}%
\pgfpathlineto{\pgfqpoint{-0.048611in}{0.000000in}}%
\pgfusepath{stroke,fill}%
}%
\begin{pgfscope}%
\pgfsys@transformshift{0.717889in}{0.599916in}%
\pgfsys@useobject{currentmarker}{}%
\end{pgfscope}%
\end{pgfscope}%
\begin{pgfscope}%
\definecolor{textcolor}{rgb}{0.000000,0.000000,0.000000}%
\pgfsetstrokecolor{textcolor}%
\pgfsetfillcolor{textcolor}%
\pgftext[x=0.551222in, y=0.551691in, left, base]{\color{textcolor}\rmfamily\fontsize{10.000000}{12.000000}\selectfont \(\displaystyle {0}\)}%
\end{pgfscope}%
\begin{pgfscope}%
\pgfsetbuttcap%
\pgfsetroundjoin%
\definecolor{currentfill}{rgb}{0.000000,0.000000,0.000000}%
\pgfsetfillcolor{currentfill}%
\pgfsetlinewidth{0.803000pt}%
\definecolor{currentstroke}{rgb}{0.000000,0.000000,0.000000}%
\pgfsetstrokecolor{currentstroke}%
\pgfsetdash{}{0pt}%
\pgfsys@defobject{currentmarker}{\pgfqpoint{-0.048611in}{0.000000in}}{\pgfqpoint{0.000000in}{0.000000in}}{%
\pgfpathmoveto{\pgfqpoint{0.000000in}{0.000000in}}%
\pgfpathlineto{\pgfqpoint{-0.048611in}{0.000000in}}%
\pgfusepath{stroke,fill}%
}%
\begin{pgfscope}%
\pgfsys@transformshift{0.717889in}{1.216441in}%
\pgfsys@useobject{currentmarker}{}%
\end{pgfscope}%
\end{pgfscope}%
\begin{pgfscope}%
\definecolor{textcolor}{rgb}{0.000000,0.000000,0.000000}%
\pgfsetstrokecolor{textcolor}%
\pgfsetfillcolor{textcolor}%
\pgftext[x=0.551222in, y=1.168215in, left, base]{\color{textcolor}\rmfamily\fontsize{10.000000}{12.000000}\selectfont \(\displaystyle {2}\)}%
\end{pgfscope}%
\begin{pgfscope}%
\definecolor{textcolor}{rgb}{0.000000,0.000000,0.000000}%
\pgfsetstrokecolor{textcolor}%
\pgfsetfillcolor{textcolor}%
\pgftext[x=0.495666in,y=0.957751in,,bottom,rotate=90.000000]{\color{textcolor}\rmfamily\fontsize{10.000000}{12.000000}\selectfont abs(Y(f)) (\(\displaystyle \mu V^2\))}%
\end{pgfscope}%
\begin{pgfscope}%
\definecolor{textcolor}{rgb}{0.000000,0.000000,0.000000}%
\pgfsetstrokecolor{textcolor}%
\pgfsetfillcolor{textcolor}%
\pgftext[x=0.717889in,y=1.393025in,left,base]{\color{textcolor}\rmfamily\fontsize{10.000000}{12.000000}\selectfont \(\displaystyle \times{10^{6}}{}\)}%
\end{pgfscope}%
\begin{pgfscope}%
\pgfpathrectangle{\pgfqpoint{0.717889in}{0.564143in}}{\pgfqpoint{5.428334in}{0.787215in}}%
\pgfusepath{clip}%
\pgfsetrectcap%
\pgfsetroundjoin%
\pgfsetlinewidth{1.505625pt}%
\definecolor{currentstroke}{rgb}{0.121569,0.466667,0.705882}%
\pgfsetstrokecolor{currentstroke}%
\pgfsetdash{}{0pt}%
\pgfpathmoveto{\pgfqpoint{3.432055in}{0.629514in}}%
\pgfpathlineto{\pgfqpoint{3.432611in}{0.623652in}}%
\pgfpathlineto{\pgfqpoint{3.433167in}{0.661967in}}%
\pgfpathlineto{\pgfqpoint{3.433723in}{0.657388in}}%
\pgfpathlineto{\pgfqpoint{3.434279in}{0.629117in}}%
\pgfpathlineto{\pgfqpoint{3.434835in}{0.738601in}}%
\pgfpathlineto{\pgfqpoint{3.435391in}{0.679911in}}%
\pgfpathlineto{\pgfqpoint{3.435946in}{0.622161in}}%
\pgfpathlineto{\pgfqpoint{3.437058in}{0.895741in}}%
\pgfpathlineto{\pgfqpoint{3.437614in}{0.823748in}}%
\pgfpathlineto{\pgfqpoint{3.438170in}{0.612408in}}%
\pgfpathlineto{\pgfqpoint{3.440393in}{1.154208in}}%
\pgfpathlineto{\pgfqpoint{3.440949in}{1.315576in}}%
\pgfpathlineto{\pgfqpoint{3.441505in}{0.732048in}}%
\pgfpathlineto{\pgfqpoint{3.442061in}{0.781868in}}%
\pgfpathlineto{\pgfqpoint{3.442617in}{1.117573in}}%
\pgfpathlineto{\pgfqpoint{3.443173in}{0.864793in}}%
\pgfpathlineto{\pgfqpoint{3.444840in}{0.758474in}}%
\pgfpathlineto{\pgfqpoint{3.445396in}{0.638568in}}%
\pgfpathlineto{\pgfqpoint{3.445952in}{0.711175in}}%
\pgfpathlineto{\pgfqpoint{3.447064in}{0.892781in}}%
\pgfpathlineto{\pgfqpoint{3.447620in}{0.873119in}}%
\pgfpathlineto{\pgfqpoint{3.448175in}{0.633703in}}%
\pgfpathlineto{\pgfqpoint{3.448731in}{0.701422in}}%
\pgfpathlineto{\pgfqpoint{3.449287in}{0.868239in}}%
\pgfpathlineto{\pgfqpoint{3.449843in}{0.695883in}}%
\pgfpathlineto{\pgfqpoint{3.450399in}{0.867669in}}%
\pgfpathlineto{\pgfqpoint{3.451511in}{0.697744in}}%
\pgfpathlineto{\pgfqpoint{3.452066in}{0.702904in}}%
\pgfpathlineto{\pgfqpoint{3.452622in}{0.871202in}}%
\pgfpathlineto{\pgfqpoint{3.453178in}{0.748957in}}%
\pgfpathlineto{\pgfqpoint{3.454290in}{0.699684in}}%
\pgfpathlineto{\pgfqpoint{3.454846in}{0.822642in}}%
\pgfpathlineto{\pgfqpoint{3.455402in}{0.617120in}}%
\pgfpathlineto{\pgfqpoint{3.455957in}{0.651910in}}%
\pgfpathlineto{\pgfqpoint{3.458737in}{0.925227in}}%
\pgfpathlineto{\pgfqpoint{3.460404in}{0.736258in}}%
\pgfpathlineto{\pgfqpoint{3.460960in}{0.786733in}}%
\pgfpathlineto{\pgfqpoint{3.461516in}{0.654914in}}%
\pgfpathlineto{\pgfqpoint{3.462072in}{0.673894in}}%
\pgfpathlineto{\pgfqpoint{3.462628in}{0.721259in}}%
\pgfpathlineto{\pgfqpoint{3.463184in}{0.712637in}}%
\pgfpathlineto{\pgfqpoint{3.463740in}{0.680063in}}%
\pgfpathlineto{\pgfqpoint{3.464295in}{0.715493in}}%
\pgfpathlineto{\pgfqpoint{3.464851in}{0.713647in}}%
\pgfpathlineto{\pgfqpoint{3.465963in}{0.638316in}}%
\pgfpathlineto{\pgfqpoint{3.466519in}{0.647823in}}%
\pgfpathlineto{\pgfqpoint{3.467075in}{0.638697in}}%
\pgfpathlineto{\pgfqpoint{3.467631in}{0.646005in}}%
\pgfpathlineto{\pgfqpoint{3.468186in}{0.651206in}}%
\pgfpathlineto{\pgfqpoint{3.468742in}{0.686150in}}%
\pgfpathlineto{\pgfqpoint{3.469298in}{0.615224in}}%
\pgfpathlineto{\pgfqpoint{3.469854in}{0.643126in}}%
\pgfpathlineto{\pgfqpoint{3.470410in}{0.766756in}}%
\pgfpathlineto{\pgfqpoint{3.470966in}{0.716852in}}%
\pgfpathlineto{\pgfqpoint{3.471522in}{0.753385in}}%
\pgfpathlineto{\pgfqpoint{3.472077in}{0.723663in}}%
\pgfpathlineto{\pgfqpoint{3.472633in}{0.710916in}}%
\pgfpathlineto{\pgfqpoint{3.473189in}{0.727291in}}%
\pgfpathlineto{\pgfqpoint{3.473745in}{0.633631in}}%
\pgfpathlineto{\pgfqpoint{3.474301in}{0.751450in}}%
\pgfpathlineto{\pgfqpoint{3.474857in}{0.712088in}}%
\pgfpathlineto{\pgfqpoint{3.475968in}{0.745982in}}%
\pgfpathlineto{\pgfqpoint{3.476524in}{0.619455in}}%
\pgfpathlineto{\pgfqpoint{3.477080in}{0.685492in}}%
\pgfpathlineto{\pgfqpoint{3.477636in}{0.620645in}}%
\pgfpathlineto{\pgfqpoint{3.478192in}{0.658847in}}%
\pgfpathlineto{\pgfqpoint{3.478748in}{0.674007in}}%
\pgfpathlineto{\pgfqpoint{3.480415in}{0.619640in}}%
\pgfpathlineto{\pgfqpoint{3.480971in}{0.690111in}}%
\pgfpathlineto{\pgfqpoint{3.481527in}{0.676128in}}%
\pgfpathlineto{\pgfqpoint{3.483195in}{0.630829in}}%
\pgfpathlineto{\pgfqpoint{3.483751in}{0.668390in}}%
\pgfpathlineto{\pgfqpoint{3.484306in}{0.645880in}}%
\pgfpathlineto{\pgfqpoint{3.485418in}{0.608832in}}%
\pgfpathlineto{\pgfqpoint{3.485974in}{0.719666in}}%
\pgfpathlineto{\pgfqpoint{3.486530in}{0.664415in}}%
\pgfpathlineto{\pgfqpoint{3.487086in}{0.713517in}}%
\pgfpathlineto{\pgfqpoint{3.487642in}{0.620669in}}%
\pgfpathlineto{\pgfqpoint{3.488197in}{0.704119in}}%
\pgfpathlineto{\pgfqpoint{3.488753in}{0.663096in}}%
\pgfpathlineto{\pgfqpoint{3.489309in}{0.960985in}}%
\pgfpathlineto{\pgfqpoint{3.489865in}{0.888329in}}%
\pgfpathlineto{\pgfqpoint{3.490977in}{0.756287in}}%
\pgfpathlineto{\pgfqpoint{3.491533in}{0.804679in}}%
\pgfpathlineto{\pgfqpoint{3.492644in}{0.635451in}}%
\pgfpathlineto{\pgfqpoint{3.493200in}{0.695227in}}%
\pgfpathlineto{\pgfqpoint{3.493756in}{0.621952in}}%
\pgfpathlineto{\pgfqpoint{3.494312in}{0.690637in}}%
\pgfpathlineto{\pgfqpoint{3.494868in}{0.699172in}}%
\pgfpathlineto{\pgfqpoint{3.495424in}{0.675830in}}%
\pgfpathlineto{\pgfqpoint{3.495979in}{0.698432in}}%
\pgfpathlineto{\pgfqpoint{3.496535in}{0.694489in}}%
\pgfpathlineto{\pgfqpoint{3.497091in}{0.818704in}}%
\pgfpathlineto{\pgfqpoint{3.497647in}{0.619290in}}%
\pgfpathlineto{\pgfqpoint{3.498203in}{0.766789in}}%
\pgfpathlineto{\pgfqpoint{3.498759in}{0.671889in}}%
\pgfpathlineto{\pgfqpoint{3.499315in}{0.718831in}}%
\pgfpathlineto{\pgfqpoint{3.499870in}{0.789158in}}%
\pgfpathlineto{\pgfqpoint{3.500426in}{0.689524in}}%
\pgfpathlineto{\pgfqpoint{3.502094in}{0.884167in}}%
\pgfpathlineto{\pgfqpoint{3.502650in}{0.699002in}}%
\pgfpathlineto{\pgfqpoint{3.503206in}{0.873234in}}%
\pgfpathlineto{\pgfqpoint{3.503762in}{0.747504in}}%
\pgfpathlineto{\pgfqpoint{3.504317in}{0.775498in}}%
\pgfpathlineto{\pgfqpoint{3.504873in}{0.785782in}}%
\pgfpathlineto{\pgfqpoint{3.505429in}{0.719423in}}%
\pgfpathlineto{\pgfqpoint{3.507097in}{0.875880in}}%
\pgfpathlineto{\pgfqpoint{3.509320in}{0.731229in}}%
\pgfpathlineto{\pgfqpoint{3.509876in}{0.709418in}}%
\pgfpathlineto{\pgfqpoint{3.510432in}{0.628539in}}%
\pgfpathlineto{\pgfqpoint{3.510988in}{0.755481in}}%
\pgfpathlineto{\pgfqpoint{3.511544in}{0.652735in}}%
\pgfpathlineto{\pgfqpoint{3.512099in}{0.659195in}}%
\pgfpathlineto{\pgfqpoint{3.513767in}{0.636054in}}%
\pgfpathlineto{\pgfqpoint{3.514323in}{0.669259in}}%
\pgfpathlineto{\pgfqpoint{3.514879in}{0.656097in}}%
\pgfpathlineto{\pgfqpoint{3.515435in}{0.611228in}}%
\pgfpathlineto{\pgfqpoint{3.515990in}{0.654944in}}%
\pgfpathlineto{\pgfqpoint{3.516546in}{0.669238in}}%
\pgfpathlineto{\pgfqpoint{3.517658in}{0.629910in}}%
\pgfpathlineto{\pgfqpoint{3.518214in}{0.634641in}}%
\pgfpathlineto{\pgfqpoint{3.518770in}{0.657444in}}%
\pgfpathlineto{\pgfqpoint{3.520437in}{0.607262in}}%
\pgfpathlineto{\pgfqpoint{3.520993in}{0.620757in}}%
\pgfpathlineto{\pgfqpoint{3.521549in}{0.668553in}}%
\pgfpathlineto{\pgfqpoint{3.522105in}{0.656674in}}%
\pgfpathlineto{\pgfqpoint{3.522661in}{0.612625in}}%
\pgfpathlineto{\pgfqpoint{3.523217in}{0.617462in}}%
\pgfpathlineto{\pgfqpoint{3.524328in}{0.641812in}}%
\pgfpathlineto{\pgfqpoint{3.524884in}{0.632037in}}%
\pgfpathlineto{\pgfqpoint{3.525440in}{0.626847in}}%
\pgfpathlineto{\pgfqpoint{3.525996in}{0.605752in}}%
\pgfpathlineto{\pgfqpoint{3.526552in}{0.626561in}}%
\pgfpathlineto{\pgfqpoint{3.527108in}{0.629392in}}%
\pgfpathlineto{\pgfqpoint{3.527664in}{0.664957in}}%
\pgfpathlineto{\pgfqpoint{3.528219in}{0.614540in}}%
\pgfpathlineto{\pgfqpoint{3.528775in}{0.642357in}}%
\pgfpathlineto{\pgfqpoint{3.529887in}{0.633163in}}%
\pgfpathlineto{\pgfqpoint{3.530999in}{0.659389in}}%
\pgfpathlineto{\pgfqpoint{3.531555in}{0.641491in}}%
\pgfpathlineto{\pgfqpoint{3.532666in}{0.677714in}}%
\pgfpathlineto{\pgfqpoint{3.534334in}{0.623523in}}%
\pgfpathlineto{\pgfqpoint{3.534890in}{0.607308in}}%
\pgfpathlineto{\pgfqpoint{3.537113in}{0.681697in}}%
\pgfpathlineto{\pgfqpoint{3.538781in}{0.644751in}}%
\pgfpathlineto{\pgfqpoint{3.539337in}{0.654078in}}%
\pgfpathlineto{\pgfqpoint{3.539893in}{0.652188in}}%
\pgfpathlineto{\pgfqpoint{3.540448in}{0.656366in}}%
\pgfpathlineto{\pgfqpoint{3.541004in}{0.651470in}}%
\pgfpathlineto{\pgfqpoint{3.541560in}{0.636386in}}%
\pgfpathlineto{\pgfqpoint{3.542116in}{0.659871in}}%
\pgfpathlineto{\pgfqpoint{3.542672in}{0.623879in}}%
\pgfpathlineto{\pgfqpoint{3.543228in}{0.665897in}}%
\pgfpathlineto{\pgfqpoint{3.543784in}{0.664445in}}%
\pgfpathlineto{\pgfqpoint{3.544895in}{0.636589in}}%
\pgfpathlineto{\pgfqpoint{3.545451in}{0.656651in}}%
\pgfpathlineto{\pgfqpoint{3.546007in}{0.625718in}}%
\pgfpathlineto{\pgfqpoint{3.546563in}{0.661839in}}%
\pgfpathlineto{\pgfqpoint{3.547119in}{0.809063in}}%
\pgfpathlineto{\pgfqpoint{3.547675in}{0.651182in}}%
\pgfpathlineto{\pgfqpoint{3.548230in}{0.687658in}}%
\pgfpathlineto{\pgfqpoint{3.549342in}{0.658046in}}%
\pgfpathlineto{\pgfqpoint{3.549898in}{0.660387in}}%
\pgfpathlineto{\pgfqpoint{3.551010in}{0.679839in}}%
\pgfpathlineto{\pgfqpoint{3.551566in}{0.649687in}}%
\pgfpathlineto{\pgfqpoint{3.552121in}{0.661274in}}%
\pgfpathlineto{\pgfqpoint{3.555457in}{0.608745in}}%
\pgfpathlineto{\pgfqpoint{3.556012in}{0.621487in}}%
\pgfpathlineto{\pgfqpoint{3.557124in}{0.629486in}}%
\pgfpathlineto{\pgfqpoint{3.557680in}{0.635194in}}%
\pgfpathlineto{\pgfqpoint{3.558792in}{0.613359in}}%
\pgfpathlineto{\pgfqpoint{3.559348in}{0.632628in}}%
\pgfpathlineto{\pgfqpoint{3.559904in}{0.616593in}}%
\pgfpathlineto{\pgfqpoint{3.561015in}{0.644620in}}%
\pgfpathlineto{\pgfqpoint{3.561571in}{0.618482in}}%
\pgfpathlineto{\pgfqpoint{3.562127in}{0.632424in}}%
\pgfpathlineto{\pgfqpoint{3.564906in}{0.613905in}}%
\pgfpathlineto{\pgfqpoint{3.565462in}{0.635416in}}%
\pgfpathlineto{\pgfqpoint{3.566018in}{0.620789in}}%
\pgfpathlineto{\pgfqpoint{3.566574in}{0.626829in}}%
\pgfpathlineto{\pgfqpoint{3.567130in}{0.622461in}}%
\pgfpathlineto{\pgfqpoint{3.567686in}{0.654399in}}%
\pgfpathlineto{\pgfqpoint{3.568241in}{0.612399in}}%
\pgfpathlineto{\pgfqpoint{3.568797in}{0.638758in}}%
\pgfpathlineto{\pgfqpoint{3.570465in}{0.604178in}}%
\pgfpathlineto{\pgfqpoint{3.572688in}{0.656454in}}%
\pgfpathlineto{\pgfqpoint{3.573244in}{0.631835in}}%
\pgfpathlineto{\pgfqpoint{3.573800in}{0.673087in}}%
\pgfpathlineto{\pgfqpoint{3.574356in}{0.628966in}}%
\pgfpathlineto{\pgfqpoint{3.574912in}{0.644000in}}%
\pgfpathlineto{\pgfqpoint{3.576579in}{0.618065in}}%
\pgfpathlineto{\pgfqpoint{3.577135in}{0.620605in}}%
\pgfpathlineto{\pgfqpoint{3.578803in}{0.648073in}}%
\pgfpathlineto{\pgfqpoint{3.579359in}{0.610924in}}%
\pgfpathlineto{\pgfqpoint{3.579915in}{0.652164in}}%
\pgfpathlineto{\pgfqpoint{3.580470in}{0.632550in}}%
\pgfpathlineto{\pgfqpoint{3.581026in}{0.631627in}}%
\pgfpathlineto{\pgfqpoint{3.582138in}{0.601082in}}%
\pgfpathlineto{\pgfqpoint{3.583806in}{0.630553in}}%
\pgfpathlineto{\pgfqpoint{3.584361in}{0.604070in}}%
\pgfpathlineto{\pgfqpoint{3.584917in}{0.607602in}}%
\pgfpathlineto{\pgfqpoint{3.586029in}{0.622866in}}%
\pgfpathlineto{\pgfqpoint{3.586585in}{0.606756in}}%
\pgfpathlineto{\pgfqpoint{3.587141in}{0.610568in}}%
\pgfpathlineto{\pgfqpoint{3.588252in}{0.616758in}}%
\pgfpathlineto{\pgfqpoint{3.588808in}{0.625309in}}%
\pgfpathlineto{\pgfqpoint{3.589364in}{0.602369in}}%
\pgfpathlineto{\pgfqpoint{3.589920in}{0.617280in}}%
\pgfpathlineto{\pgfqpoint{3.590476in}{0.616872in}}%
\pgfpathlineto{\pgfqpoint{3.591588in}{0.635872in}}%
\pgfpathlineto{\pgfqpoint{3.592143in}{0.624487in}}%
\pgfpathlineto{\pgfqpoint{3.592699in}{0.601963in}}%
\pgfpathlineto{\pgfqpoint{3.593811in}{0.636438in}}%
\pgfpathlineto{\pgfqpoint{3.595479in}{0.609985in}}%
\pgfpathlineto{\pgfqpoint{3.596035in}{0.609873in}}%
\pgfpathlineto{\pgfqpoint{3.597702in}{0.624490in}}%
\pgfpathlineto{\pgfqpoint{3.598258in}{0.618898in}}%
\pgfpathlineto{\pgfqpoint{3.598814in}{0.670562in}}%
\pgfpathlineto{\pgfqpoint{3.599370in}{0.604230in}}%
\pgfpathlineto{\pgfqpoint{3.599926in}{0.647270in}}%
\pgfpathlineto{\pgfqpoint{3.600481in}{0.614850in}}%
\pgfpathlineto{\pgfqpoint{3.601037in}{0.623442in}}%
\pgfpathlineto{\pgfqpoint{3.601593in}{0.632124in}}%
\pgfpathlineto{\pgfqpoint{3.602149in}{0.745259in}}%
\pgfpathlineto{\pgfqpoint{3.602705in}{0.650100in}}%
\pgfpathlineto{\pgfqpoint{3.603817in}{0.690842in}}%
\pgfpathlineto{\pgfqpoint{3.604372in}{0.872308in}}%
\pgfpathlineto{\pgfqpoint{3.604928in}{0.716088in}}%
\pgfpathlineto{\pgfqpoint{3.606040in}{0.627646in}}%
\pgfpathlineto{\pgfqpoint{3.606596in}{0.735463in}}%
\pgfpathlineto{\pgfqpoint{3.607152in}{0.676493in}}%
\pgfpathlineto{\pgfqpoint{3.607708in}{0.698541in}}%
\pgfpathlineto{\pgfqpoint{3.608263in}{0.689741in}}%
\pgfpathlineto{\pgfqpoint{3.609931in}{0.636102in}}%
\pgfpathlineto{\pgfqpoint{3.611599in}{0.729660in}}%
\pgfpathlineto{\pgfqpoint{3.613822in}{0.634334in}}%
\pgfpathlineto{\pgfqpoint{3.614378in}{0.639894in}}%
\pgfpathlineto{\pgfqpoint{3.616046in}{0.621286in}}%
\pgfpathlineto{\pgfqpoint{3.616601in}{0.630113in}}%
\pgfpathlineto{\pgfqpoint{3.617157in}{0.623726in}}%
\pgfpathlineto{\pgfqpoint{3.618825in}{0.605055in}}%
\pgfpathlineto{\pgfqpoint{3.620492in}{0.620729in}}%
\pgfpathlineto{\pgfqpoint{3.621604in}{0.602604in}}%
\pgfpathlineto{\pgfqpoint{3.622716in}{0.620151in}}%
\pgfpathlineto{\pgfqpoint{3.623272in}{0.604384in}}%
\pgfpathlineto{\pgfqpoint{3.623828in}{0.617520in}}%
\pgfpathlineto{\pgfqpoint{3.624383in}{0.632853in}}%
\pgfpathlineto{\pgfqpoint{3.624939in}{0.604547in}}%
\pgfpathlineto{\pgfqpoint{3.625495in}{0.615758in}}%
\pgfpathlineto{\pgfqpoint{3.626607in}{0.607012in}}%
\pgfpathlineto{\pgfqpoint{3.627163in}{0.616970in}}%
\pgfpathlineto{\pgfqpoint{3.627719in}{0.606116in}}%
\pgfpathlineto{\pgfqpoint{3.628274in}{0.613244in}}%
\pgfpathlineto{\pgfqpoint{3.628830in}{0.612211in}}%
\pgfpathlineto{\pgfqpoint{3.629386in}{0.616706in}}%
\pgfpathlineto{\pgfqpoint{3.630498in}{0.607175in}}%
\pgfpathlineto{\pgfqpoint{3.631054in}{0.629778in}}%
\pgfpathlineto{\pgfqpoint{3.631610in}{0.609064in}}%
\pgfpathlineto{\pgfqpoint{3.632165in}{0.613356in}}%
\pgfpathlineto{\pgfqpoint{3.632721in}{0.609087in}}%
\pgfpathlineto{\pgfqpoint{3.633277in}{0.636588in}}%
\pgfpathlineto{\pgfqpoint{3.633833in}{0.629675in}}%
\pgfpathlineto{\pgfqpoint{3.636057in}{0.607817in}}%
\pgfpathlineto{\pgfqpoint{3.636612in}{0.620127in}}%
\pgfpathlineto{\pgfqpoint{3.637168in}{0.616338in}}%
\pgfpathlineto{\pgfqpoint{3.637724in}{0.616447in}}%
\pgfpathlineto{\pgfqpoint{3.638280in}{0.623668in}}%
\pgfpathlineto{\pgfqpoint{3.638836in}{0.604624in}}%
\pgfpathlineto{\pgfqpoint{3.639392in}{0.607937in}}%
\pgfpathlineto{\pgfqpoint{3.641615in}{0.620093in}}%
\pgfpathlineto{\pgfqpoint{3.643283in}{0.607867in}}%
\pgfpathlineto{\pgfqpoint{3.643839in}{0.612274in}}%
\pgfpathlineto{\pgfqpoint{3.644394in}{0.604670in}}%
\pgfpathlineto{\pgfqpoint{3.644950in}{0.605101in}}%
\pgfpathlineto{\pgfqpoint{3.646062in}{0.628139in}}%
\pgfpathlineto{\pgfqpoint{3.646618in}{0.617455in}}%
\pgfpathlineto{\pgfqpoint{3.649397in}{0.642527in}}%
\pgfpathlineto{\pgfqpoint{3.651621in}{0.607186in}}%
\pgfpathlineto{\pgfqpoint{3.652177in}{0.623651in}}%
\pgfpathlineto{\pgfqpoint{3.652732in}{0.620559in}}%
\pgfpathlineto{\pgfqpoint{3.653288in}{0.608353in}}%
\pgfpathlineto{\pgfqpoint{3.654956in}{0.636125in}}%
\pgfpathlineto{\pgfqpoint{3.655512in}{0.625412in}}%
\pgfpathlineto{\pgfqpoint{3.656068in}{0.662390in}}%
\pgfpathlineto{\pgfqpoint{3.656623in}{0.608114in}}%
\pgfpathlineto{\pgfqpoint{3.657179in}{0.656464in}}%
\pgfpathlineto{\pgfqpoint{3.657735in}{0.614393in}}%
\pgfpathlineto{\pgfqpoint{3.659403in}{0.797022in}}%
\pgfpathlineto{\pgfqpoint{3.659959in}{0.652649in}}%
\pgfpathlineto{\pgfqpoint{3.660514in}{0.678065in}}%
\pgfpathlineto{\pgfqpoint{3.661070in}{0.665272in}}%
\pgfpathlineto{\pgfqpoint{3.661626in}{0.760035in}}%
\pgfpathlineto{\pgfqpoint{3.662182in}{0.754457in}}%
\pgfpathlineto{\pgfqpoint{3.662738in}{0.633489in}}%
\pgfpathlineto{\pgfqpoint{3.663294in}{0.724738in}}%
\pgfpathlineto{\pgfqpoint{3.663850in}{0.782399in}}%
\pgfpathlineto{\pgfqpoint{3.664405in}{0.734454in}}%
\pgfpathlineto{\pgfqpoint{3.665517in}{0.660894in}}%
\pgfpathlineto{\pgfqpoint{3.666073in}{0.690906in}}%
\pgfpathlineto{\pgfqpoint{3.667185in}{0.635154in}}%
\pgfpathlineto{\pgfqpoint{3.667741in}{0.665009in}}%
\pgfpathlineto{\pgfqpoint{3.668296in}{0.694092in}}%
\pgfpathlineto{\pgfqpoint{3.668852in}{0.666625in}}%
\pgfpathlineto{\pgfqpoint{3.669408in}{0.613225in}}%
\pgfpathlineto{\pgfqpoint{3.671632in}{0.726435in}}%
\pgfpathlineto{\pgfqpoint{3.672743in}{0.690400in}}%
\pgfpathlineto{\pgfqpoint{3.673299in}{0.641821in}}%
\pgfpathlineto{\pgfqpoint{3.673855in}{0.647642in}}%
\pgfpathlineto{\pgfqpoint{3.674411in}{0.644463in}}%
\pgfpathlineto{\pgfqpoint{3.676079in}{0.606720in}}%
\pgfpathlineto{\pgfqpoint{3.676634in}{0.611651in}}%
\pgfpathlineto{\pgfqpoint{3.677190in}{0.623456in}}%
\pgfpathlineto{\pgfqpoint{3.677746in}{0.606286in}}%
\pgfpathlineto{\pgfqpoint{3.678302in}{0.616774in}}%
\pgfpathlineto{\pgfqpoint{3.678858in}{0.613513in}}%
\pgfpathlineto{\pgfqpoint{3.679414in}{0.617789in}}%
\pgfpathlineto{\pgfqpoint{3.679970in}{0.601812in}}%
\pgfpathlineto{\pgfqpoint{3.680525in}{0.618975in}}%
\pgfpathlineto{\pgfqpoint{3.681081in}{0.605199in}}%
\pgfpathlineto{\pgfqpoint{3.681637in}{0.615256in}}%
\pgfpathlineto{\pgfqpoint{3.682193in}{0.603585in}}%
\pgfpathlineto{\pgfqpoint{3.682749in}{0.611890in}}%
\pgfpathlineto{\pgfqpoint{3.683305in}{0.621807in}}%
\pgfpathlineto{\pgfqpoint{3.683861in}{0.611930in}}%
\pgfpathlineto{\pgfqpoint{3.684972in}{0.618536in}}%
\pgfpathlineto{\pgfqpoint{3.685528in}{0.604837in}}%
\pgfpathlineto{\pgfqpoint{3.686084in}{0.616942in}}%
\pgfpathlineto{\pgfqpoint{3.687196in}{0.613432in}}%
\pgfpathlineto{\pgfqpoint{3.687752in}{0.624259in}}%
\pgfpathlineto{\pgfqpoint{3.688307in}{0.614799in}}%
\pgfpathlineto{\pgfqpoint{3.688863in}{0.617511in}}%
\pgfpathlineto{\pgfqpoint{3.691087in}{0.604328in}}%
\pgfpathlineto{\pgfqpoint{3.691643in}{0.628698in}}%
\pgfpathlineto{\pgfqpoint{3.692199in}{0.605173in}}%
\pgfpathlineto{\pgfqpoint{3.693310in}{0.618547in}}%
\pgfpathlineto{\pgfqpoint{3.693866in}{0.603449in}}%
\pgfpathlineto{\pgfqpoint{3.694422in}{0.624084in}}%
\pgfpathlineto{\pgfqpoint{3.694978in}{0.614049in}}%
\pgfpathlineto{\pgfqpoint{3.695534in}{0.602651in}}%
\pgfpathlineto{\pgfqpoint{3.696090in}{0.605771in}}%
\pgfpathlineto{\pgfqpoint{3.697757in}{0.626242in}}%
\pgfpathlineto{\pgfqpoint{3.698313in}{0.601763in}}%
\pgfpathlineto{\pgfqpoint{3.698869in}{0.619045in}}%
\pgfpathlineto{\pgfqpoint{3.699425in}{0.621810in}}%
\pgfpathlineto{\pgfqpoint{3.699981in}{0.631469in}}%
\pgfpathlineto{\pgfqpoint{3.700536in}{0.629145in}}%
\pgfpathlineto{\pgfqpoint{3.701648in}{0.606399in}}%
\pgfpathlineto{\pgfqpoint{3.702204in}{0.607702in}}%
\pgfpathlineto{\pgfqpoint{3.702760in}{0.615626in}}%
\pgfpathlineto{\pgfqpoint{3.704427in}{0.602272in}}%
\pgfpathlineto{\pgfqpoint{3.704983in}{0.602995in}}%
\pgfpathlineto{\pgfqpoint{3.706095in}{0.635118in}}%
\pgfpathlineto{\pgfqpoint{3.706651in}{0.615228in}}%
\pgfpathlineto{\pgfqpoint{3.707207in}{0.618457in}}%
\pgfpathlineto{\pgfqpoint{3.707763in}{0.626005in}}%
\pgfpathlineto{\pgfqpoint{3.708318in}{0.611912in}}%
\pgfpathlineto{\pgfqpoint{3.708874in}{0.615907in}}%
\pgfpathlineto{\pgfqpoint{3.709430in}{0.624722in}}%
\pgfpathlineto{\pgfqpoint{3.709986in}{0.607871in}}%
\pgfpathlineto{\pgfqpoint{3.710542in}{0.613664in}}%
\pgfpathlineto{\pgfqpoint{3.711654in}{0.621068in}}%
\pgfpathlineto{\pgfqpoint{3.712210in}{0.613129in}}%
\pgfpathlineto{\pgfqpoint{3.712765in}{0.617470in}}%
\pgfpathlineto{\pgfqpoint{3.714433in}{0.654584in}}%
\pgfpathlineto{\pgfqpoint{3.714989in}{0.609001in}}%
\pgfpathlineto{\pgfqpoint{3.716656in}{0.719067in}}%
\pgfpathlineto{\pgfqpoint{3.717212in}{0.711469in}}%
\pgfpathlineto{\pgfqpoint{3.718324in}{0.608318in}}%
\pgfpathlineto{\pgfqpoint{3.718880in}{0.653597in}}%
\pgfpathlineto{\pgfqpoint{3.719436in}{0.766297in}}%
\pgfpathlineto{\pgfqpoint{3.719992in}{0.668616in}}%
\pgfpathlineto{\pgfqpoint{3.720547in}{0.710518in}}%
\pgfpathlineto{\pgfqpoint{3.721103in}{0.661885in}}%
\pgfpathlineto{\pgfqpoint{3.721659in}{0.704371in}}%
\pgfpathlineto{\pgfqpoint{3.723327in}{0.662359in}}%
\pgfpathlineto{\pgfqpoint{3.723883in}{0.669745in}}%
\pgfpathlineto{\pgfqpoint{3.724438in}{0.665605in}}%
\pgfpathlineto{\pgfqpoint{3.726106in}{0.627495in}}%
\pgfpathlineto{\pgfqpoint{3.726662in}{0.618777in}}%
\pgfpathlineto{\pgfqpoint{3.727774in}{0.655678in}}%
\pgfpathlineto{\pgfqpoint{3.728885in}{0.649143in}}%
\pgfpathlineto{\pgfqpoint{3.729441in}{0.619864in}}%
\pgfpathlineto{\pgfqpoint{3.729997in}{0.644541in}}%
\pgfpathlineto{\pgfqpoint{3.731665in}{0.661161in}}%
\pgfpathlineto{\pgfqpoint{3.732221in}{0.656897in}}%
\pgfpathlineto{\pgfqpoint{3.732776in}{0.671143in}}%
\pgfpathlineto{\pgfqpoint{3.734444in}{0.631927in}}%
\pgfpathlineto{\pgfqpoint{3.735556in}{0.607203in}}%
\pgfpathlineto{\pgfqpoint{3.736112in}{0.617034in}}%
\pgfpathlineto{\pgfqpoint{3.736667in}{0.610025in}}%
\pgfpathlineto{\pgfqpoint{3.737223in}{0.606092in}}%
\pgfpathlineto{\pgfqpoint{3.737779in}{0.609630in}}%
\pgfpathlineto{\pgfqpoint{3.738891in}{0.616220in}}%
\pgfpathlineto{\pgfqpoint{3.739447in}{0.600595in}}%
\pgfpathlineto{\pgfqpoint{3.740003in}{0.607210in}}%
\pgfpathlineto{\pgfqpoint{3.740558in}{0.616429in}}%
\pgfpathlineto{\pgfqpoint{3.741114in}{0.607913in}}%
\pgfpathlineto{\pgfqpoint{3.741670in}{0.607774in}}%
\pgfpathlineto{\pgfqpoint{3.742226in}{0.602499in}}%
\pgfpathlineto{\pgfqpoint{3.742782in}{0.615447in}}%
\pgfpathlineto{\pgfqpoint{3.743338in}{0.604690in}}%
\pgfpathlineto{\pgfqpoint{3.745005in}{0.618854in}}%
\pgfpathlineto{\pgfqpoint{3.746117in}{0.605960in}}%
\pgfpathlineto{\pgfqpoint{3.746673in}{0.611907in}}%
\pgfpathlineto{\pgfqpoint{3.747229in}{0.616888in}}%
\pgfpathlineto{\pgfqpoint{3.747785in}{0.604758in}}%
\pgfpathlineto{\pgfqpoint{3.748341in}{0.625418in}}%
\pgfpathlineto{\pgfqpoint{3.748896in}{0.614192in}}%
\pgfpathlineto{\pgfqpoint{3.751120in}{0.606836in}}%
\pgfpathlineto{\pgfqpoint{3.751676in}{0.622371in}}%
\pgfpathlineto{\pgfqpoint{3.752232in}{0.622333in}}%
\pgfpathlineto{\pgfqpoint{3.752787in}{0.618576in}}%
\pgfpathlineto{\pgfqpoint{3.753343in}{0.623162in}}%
\pgfpathlineto{\pgfqpoint{3.753899in}{0.606451in}}%
\pgfpathlineto{\pgfqpoint{3.754455in}{0.615067in}}%
\pgfpathlineto{\pgfqpoint{3.755011in}{0.622622in}}%
\pgfpathlineto{\pgfqpoint{3.755567in}{0.608808in}}%
\pgfpathlineto{\pgfqpoint{3.756123in}{0.619062in}}%
\pgfpathlineto{\pgfqpoint{3.756678in}{0.612675in}}%
\pgfpathlineto{\pgfqpoint{3.757234in}{0.618536in}}%
\pgfpathlineto{\pgfqpoint{3.757790in}{0.625655in}}%
\pgfpathlineto{\pgfqpoint{3.758902in}{0.603841in}}%
\pgfpathlineto{\pgfqpoint{3.759458in}{0.608780in}}%
\pgfpathlineto{\pgfqpoint{3.760569in}{0.610166in}}%
\pgfpathlineto{\pgfqpoint{3.761125in}{0.625133in}}%
\pgfpathlineto{\pgfqpoint{3.762237in}{0.607833in}}%
\pgfpathlineto{\pgfqpoint{3.763905in}{0.623250in}}%
\pgfpathlineto{\pgfqpoint{3.764460in}{0.602489in}}%
\pgfpathlineto{\pgfqpoint{3.765016in}{0.616988in}}%
\pgfpathlineto{\pgfqpoint{3.766128in}{0.606291in}}%
\pgfpathlineto{\pgfqpoint{3.766684in}{0.636168in}}%
\pgfpathlineto{\pgfqpoint{3.767240in}{0.616546in}}%
\pgfpathlineto{\pgfqpoint{3.767796in}{0.610534in}}%
\pgfpathlineto{\pgfqpoint{3.768352in}{0.616224in}}%
\pgfpathlineto{\pgfqpoint{3.768907in}{0.622340in}}%
\pgfpathlineto{\pgfqpoint{3.770019in}{0.609842in}}%
\pgfpathlineto{\pgfqpoint{3.770575in}{0.611849in}}%
\pgfpathlineto{\pgfqpoint{3.771131in}{0.652299in}}%
\pgfpathlineto{\pgfqpoint{3.771687in}{0.623172in}}%
\pgfpathlineto{\pgfqpoint{3.772243in}{0.650922in}}%
\pgfpathlineto{\pgfqpoint{3.772798in}{0.640736in}}%
\pgfpathlineto{\pgfqpoint{3.773910in}{0.637948in}}%
\pgfpathlineto{\pgfqpoint{3.774466in}{0.734663in}}%
\pgfpathlineto{\pgfqpoint{3.775022in}{0.653493in}}%
\pgfpathlineto{\pgfqpoint{3.775578in}{0.620993in}}%
\pgfpathlineto{\pgfqpoint{3.776689in}{0.683762in}}%
\pgfpathlineto{\pgfqpoint{3.777245in}{0.627838in}}%
\pgfpathlineto{\pgfqpoint{3.777801in}{0.650304in}}%
\pgfpathlineto{\pgfqpoint{3.778357in}{0.644578in}}%
\pgfpathlineto{\pgfqpoint{3.778913in}{0.707877in}}%
\pgfpathlineto{\pgfqpoint{3.779469in}{0.657982in}}%
\pgfpathlineto{\pgfqpoint{3.780025in}{0.667062in}}%
\pgfpathlineto{\pgfqpoint{3.780580in}{0.619944in}}%
\pgfpathlineto{\pgfqpoint{3.781136in}{0.657917in}}%
\pgfpathlineto{\pgfqpoint{3.781692in}{0.677679in}}%
\pgfpathlineto{\pgfqpoint{3.782248in}{0.617951in}}%
\pgfpathlineto{\pgfqpoint{3.782804in}{0.631061in}}%
\pgfpathlineto{\pgfqpoint{3.783360in}{0.628891in}}%
\pgfpathlineto{\pgfqpoint{3.784472in}{0.620320in}}%
\pgfpathlineto{\pgfqpoint{3.785027in}{0.660666in}}%
\pgfpathlineto{\pgfqpoint{3.785583in}{0.625915in}}%
\pgfpathlineto{\pgfqpoint{3.786139in}{0.625363in}}%
\pgfpathlineto{\pgfqpoint{3.786695in}{0.621724in}}%
\pgfpathlineto{\pgfqpoint{3.788363in}{0.661370in}}%
\pgfpathlineto{\pgfqpoint{3.789474in}{0.617386in}}%
\pgfpathlineto{\pgfqpoint{3.791142in}{0.659111in}}%
\pgfpathlineto{\pgfqpoint{3.791698in}{0.657538in}}%
\pgfpathlineto{\pgfqpoint{3.793921in}{0.627812in}}%
\pgfpathlineto{\pgfqpoint{3.794477in}{0.630489in}}%
\pgfpathlineto{\pgfqpoint{3.795033in}{0.613565in}}%
\pgfpathlineto{\pgfqpoint{3.795589in}{0.625674in}}%
\pgfpathlineto{\pgfqpoint{3.796700in}{0.610622in}}%
\pgfpathlineto{\pgfqpoint{3.797256in}{0.617245in}}%
\pgfpathlineto{\pgfqpoint{3.797812in}{0.616116in}}%
\pgfpathlineto{\pgfqpoint{3.798368in}{0.611182in}}%
\pgfpathlineto{\pgfqpoint{3.798924in}{0.622985in}}%
\pgfpathlineto{\pgfqpoint{3.799480in}{0.600748in}}%
\pgfpathlineto{\pgfqpoint{3.800036in}{0.618673in}}%
\pgfpathlineto{\pgfqpoint{3.800591in}{0.607695in}}%
\pgfpathlineto{\pgfqpoint{3.801147in}{0.613482in}}%
\pgfpathlineto{\pgfqpoint{3.802815in}{0.621661in}}%
\pgfpathlineto{\pgfqpoint{3.803371in}{0.602884in}}%
\pgfpathlineto{\pgfqpoint{3.803927in}{0.618054in}}%
\pgfpathlineto{\pgfqpoint{3.806150in}{0.604538in}}%
\pgfpathlineto{\pgfqpoint{3.807262in}{0.614234in}}%
\pgfpathlineto{\pgfqpoint{3.807818in}{0.603717in}}%
\pgfpathlineto{\pgfqpoint{3.808374in}{0.608186in}}%
\pgfpathlineto{\pgfqpoint{3.808929in}{0.613464in}}%
\pgfpathlineto{\pgfqpoint{3.809485in}{0.608400in}}%
\pgfpathlineto{\pgfqpoint{3.811153in}{0.615123in}}%
\pgfpathlineto{\pgfqpoint{3.811709in}{0.613463in}}%
\pgfpathlineto{\pgfqpoint{3.812820in}{0.604192in}}%
\pgfpathlineto{\pgfqpoint{3.814488in}{0.609153in}}%
\pgfpathlineto{\pgfqpoint{3.815044in}{0.604791in}}%
\pgfpathlineto{\pgfqpoint{3.815600in}{0.606585in}}%
\pgfpathlineto{\pgfqpoint{3.816156in}{0.609884in}}%
\pgfpathlineto{\pgfqpoint{3.816711in}{0.605784in}}%
\pgfpathlineto{\pgfqpoint{3.817267in}{0.607320in}}%
\pgfpathlineto{\pgfqpoint{3.817823in}{0.615313in}}%
\pgfpathlineto{\pgfqpoint{3.818379in}{0.614094in}}%
\pgfpathlineto{\pgfqpoint{3.820047in}{0.612334in}}%
\pgfpathlineto{\pgfqpoint{3.820602in}{0.614156in}}%
\pgfpathlineto{\pgfqpoint{3.821158in}{0.621749in}}%
\pgfpathlineto{\pgfqpoint{3.821714in}{0.601656in}}%
\pgfpathlineto{\pgfqpoint{3.822270in}{0.619928in}}%
\pgfpathlineto{\pgfqpoint{3.822826in}{0.614566in}}%
\pgfpathlineto{\pgfqpoint{3.823382in}{0.621348in}}%
\pgfpathlineto{\pgfqpoint{3.823938in}{0.607222in}}%
\pgfpathlineto{\pgfqpoint{3.824494in}{0.617690in}}%
\pgfpathlineto{\pgfqpoint{3.826161in}{0.627852in}}%
\pgfpathlineto{\pgfqpoint{3.826717in}{0.604779in}}%
\pgfpathlineto{\pgfqpoint{3.828385in}{0.658525in}}%
\pgfpathlineto{\pgfqpoint{3.828940in}{0.649157in}}%
\pgfpathlineto{\pgfqpoint{3.829496in}{0.679927in}}%
\pgfpathlineto{\pgfqpoint{3.830052in}{0.614465in}}%
\pgfpathlineto{\pgfqpoint{3.830608in}{0.643910in}}%
\pgfpathlineto{\pgfqpoint{3.831164in}{0.653442in}}%
\pgfpathlineto{\pgfqpoint{3.831720in}{0.705039in}}%
\pgfpathlineto{\pgfqpoint{3.832276in}{0.673409in}}%
\pgfpathlineto{\pgfqpoint{3.833943in}{0.610247in}}%
\pgfpathlineto{\pgfqpoint{3.835611in}{0.694506in}}%
\pgfpathlineto{\pgfqpoint{3.837834in}{0.630270in}}%
\pgfpathlineto{\pgfqpoint{3.838946in}{0.667806in}}%
\pgfpathlineto{\pgfqpoint{3.840058in}{0.618661in}}%
\pgfpathlineto{\pgfqpoint{3.840614in}{0.633462in}}%
\pgfpathlineto{\pgfqpoint{3.841169in}{0.655435in}}%
\pgfpathlineto{\pgfqpoint{3.841725in}{0.613285in}}%
\pgfpathlineto{\pgfqpoint{3.842281in}{0.643334in}}%
\pgfpathlineto{\pgfqpoint{3.843393in}{0.609752in}}%
\pgfpathlineto{\pgfqpoint{3.843949in}{0.613055in}}%
\pgfpathlineto{\pgfqpoint{3.844505in}{0.658416in}}%
\pgfpathlineto{\pgfqpoint{3.845060in}{0.629466in}}%
\pgfpathlineto{\pgfqpoint{3.846172in}{0.613155in}}%
\pgfpathlineto{\pgfqpoint{3.847840in}{0.653051in}}%
\pgfpathlineto{\pgfqpoint{3.848951in}{0.615032in}}%
\pgfpathlineto{\pgfqpoint{3.850063in}{0.618931in}}%
\pgfpathlineto{\pgfqpoint{3.851731in}{0.645172in}}%
\pgfpathlineto{\pgfqpoint{3.852287in}{0.663267in}}%
\pgfpathlineto{\pgfqpoint{3.853954in}{0.627681in}}%
\pgfpathlineto{\pgfqpoint{3.854510in}{0.635506in}}%
\pgfpathlineto{\pgfqpoint{3.855066in}{0.628510in}}%
\pgfpathlineto{\pgfqpoint{3.856178in}{0.615243in}}%
\pgfpathlineto{\pgfqpoint{3.856733in}{0.617617in}}%
\pgfpathlineto{\pgfqpoint{3.857289in}{0.619071in}}%
\pgfpathlineto{\pgfqpoint{3.857845in}{0.605333in}}%
\pgfpathlineto{\pgfqpoint{3.858401in}{0.611697in}}%
\pgfpathlineto{\pgfqpoint{3.858957in}{0.616106in}}%
\pgfpathlineto{\pgfqpoint{3.859513in}{0.610738in}}%
\pgfpathlineto{\pgfqpoint{3.860069in}{0.618707in}}%
\pgfpathlineto{\pgfqpoint{3.860625in}{0.604381in}}%
\pgfpathlineto{\pgfqpoint{3.861180in}{0.608276in}}%
\pgfpathlineto{\pgfqpoint{3.862848in}{0.612636in}}%
\pgfpathlineto{\pgfqpoint{3.863404in}{0.606079in}}%
\pgfpathlineto{\pgfqpoint{3.864516in}{0.618616in}}%
\pgfpathlineto{\pgfqpoint{3.865071in}{0.618170in}}%
\pgfpathlineto{\pgfqpoint{3.865627in}{0.619874in}}%
\pgfpathlineto{\pgfqpoint{3.866183in}{0.606014in}}%
\pgfpathlineto{\pgfqpoint{3.866739in}{0.608559in}}%
\pgfpathlineto{\pgfqpoint{3.867295in}{0.617400in}}%
\pgfpathlineto{\pgfqpoint{3.867851in}{0.609958in}}%
\pgfpathlineto{\pgfqpoint{3.868407in}{0.611367in}}%
\pgfpathlineto{\pgfqpoint{3.868962in}{0.617364in}}%
\pgfpathlineto{\pgfqpoint{3.869518in}{0.604194in}}%
\pgfpathlineto{\pgfqpoint{3.870074in}{0.610469in}}%
\pgfpathlineto{\pgfqpoint{3.870630in}{0.613772in}}%
\pgfpathlineto{\pgfqpoint{3.871186in}{0.604102in}}%
\pgfpathlineto{\pgfqpoint{3.871742in}{0.612319in}}%
\pgfpathlineto{\pgfqpoint{3.872298in}{0.627985in}}%
\pgfpathlineto{\pgfqpoint{3.872853in}{0.613153in}}%
\pgfpathlineto{\pgfqpoint{3.873409in}{0.615418in}}%
\pgfpathlineto{\pgfqpoint{3.873965in}{0.613583in}}%
\pgfpathlineto{\pgfqpoint{3.874521in}{0.609979in}}%
\pgfpathlineto{\pgfqpoint{3.875077in}{0.619499in}}%
\pgfpathlineto{\pgfqpoint{3.875633in}{0.604370in}}%
\pgfpathlineto{\pgfqpoint{3.876189in}{0.620159in}}%
\pgfpathlineto{\pgfqpoint{3.876744in}{0.618981in}}%
\pgfpathlineto{\pgfqpoint{3.877300in}{0.617946in}}%
\pgfpathlineto{\pgfqpoint{3.877856in}{0.620916in}}%
\pgfpathlineto{\pgfqpoint{3.878968in}{0.604532in}}%
\pgfpathlineto{\pgfqpoint{3.879524in}{0.627386in}}%
\pgfpathlineto{\pgfqpoint{3.880080in}{0.611508in}}%
\pgfpathlineto{\pgfqpoint{3.880636in}{0.617118in}}%
\pgfpathlineto{\pgfqpoint{3.881191in}{0.608023in}}%
\pgfpathlineto{\pgfqpoint{3.881747in}{0.616448in}}%
\pgfpathlineto{\pgfqpoint{3.883971in}{0.627815in}}%
\pgfpathlineto{\pgfqpoint{3.884527in}{0.665089in}}%
\pgfpathlineto{\pgfqpoint{3.885082in}{0.627314in}}%
\pgfpathlineto{\pgfqpoint{3.885638in}{0.660937in}}%
\pgfpathlineto{\pgfqpoint{3.886750in}{0.679275in}}%
\pgfpathlineto{\pgfqpoint{3.888418in}{0.629755in}}%
\pgfpathlineto{\pgfqpoint{3.889529in}{0.741230in}}%
\pgfpathlineto{\pgfqpoint{3.890641in}{0.614363in}}%
\pgfpathlineto{\pgfqpoint{3.891197in}{0.667566in}}%
\pgfpathlineto{\pgfqpoint{3.891753in}{0.636854in}}%
\pgfpathlineto{\pgfqpoint{3.892864in}{0.683403in}}%
\pgfpathlineto{\pgfqpoint{3.893420in}{0.609840in}}%
\pgfpathlineto{\pgfqpoint{3.893976in}{0.706423in}}%
\pgfpathlineto{\pgfqpoint{3.894532in}{0.645637in}}%
\pgfpathlineto{\pgfqpoint{3.895088in}{0.694807in}}%
\pgfpathlineto{\pgfqpoint{3.895644in}{0.658245in}}%
\pgfpathlineto{\pgfqpoint{3.896755in}{0.706689in}}%
\pgfpathlineto{\pgfqpoint{3.897311in}{0.630941in}}%
\pgfpathlineto{\pgfqpoint{3.897867in}{0.642143in}}%
\pgfpathlineto{\pgfqpoint{3.898423in}{0.664047in}}%
\pgfpathlineto{\pgfqpoint{3.898979in}{0.644347in}}%
\pgfpathlineto{\pgfqpoint{3.899535in}{0.619025in}}%
\pgfpathlineto{\pgfqpoint{3.900091in}{0.659728in}}%
\pgfpathlineto{\pgfqpoint{3.900647in}{0.621053in}}%
\pgfpathlineto{\pgfqpoint{3.901202in}{0.608119in}}%
\pgfpathlineto{\pgfqpoint{3.901758in}{0.672204in}}%
\pgfpathlineto{\pgfqpoint{3.902314in}{0.632268in}}%
\pgfpathlineto{\pgfqpoint{3.902870in}{0.611212in}}%
\pgfpathlineto{\pgfqpoint{3.903426in}{0.626614in}}%
\pgfpathlineto{\pgfqpoint{3.905093in}{0.648235in}}%
\pgfpathlineto{\pgfqpoint{3.905649in}{0.620066in}}%
\pgfpathlineto{\pgfqpoint{3.906205in}{0.624692in}}%
\pgfpathlineto{\pgfqpoint{3.907317in}{0.673431in}}%
\pgfpathlineto{\pgfqpoint{3.909540in}{0.608285in}}%
\pgfpathlineto{\pgfqpoint{3.910096in}{0.619421in}}%
\pgfpathlineto{\pgfqpoint{3.911764in}{0.662108in}}%
\pgfpathlineto{\pgfqpoint{3.912320in}{0.661989in}}%
\pgfpathlineto{\pgfqpoint{3.913987in}{0.640597in}}%
\pgfpathlineto{\pgfqpoint{3.914543in}{0.631543in}}%
\pgfpathlineto{\pgfqpoint{3.915099in}{0.652739in}}%
\pgfpathlineto{\pgfqpoint{3.915655in}{0.642734in}}%
\pgfpathlineto{\pgfqpoint{3.917322in}{0.605640in}}%
\pgfpathlineto{\pgfqpoint{3.917878in}{0.608214in}}%
\pgfpathlineto{\pgfqpoint{3.918434in}{0.605388in}}%
\pgfpathlineto{\pgfqpoint{3.920102in}{0.613555in}}%
\pgfpathlineto{\pgfqpoint{3.920658in}{0.605103in}}%
\pgfpathlineto{\pgfqpoint{3.921213in}{0.609128in}}%
\pgfpathlineto{\pgfqpoint{3.922325in}{0.609973in}}%
\pgfpathlineto{\pgfqpoint{3.922881in}{0.611772in}}%
\pgfpathlineto{\pgfqpoint{3.923437in}{0.607432in}}%
\pgfpathlineto{\pgfqpoint{3.923993in}{0.617954in}}%
\pgfpathlineto{\pgfqpoint{3.924549in}{0.604441in}}%
\pgfpathlineto{\pgfqpoint{3.925104in}{0.607486in}}%
\pgfpathlineto{\pgfqpoint{3.925660in}{0.608451in}}%
\pgfpathlineto{\pgfqpoint{3.927328in}{0.602701in}}%
\pgfpathlineto{\pgfqpoint{3.928440in}{0.617344in}}%
\pgfpathlineto{\pgfqpoint{3.930107in}{0.606925in}}%
\pgfpathlineto{\pgfqpoint{3.931775in}{0.621602in}}%
\pgfpathlineto{\pgfqpoint{3.932331in}{0.602828in}}%
\pgfpathlineto{\pgfqpoint{3.932886in}{0.617982in}}%
\pgfpathlineto{\pgfqpoint{3.933442in}{0.621728in}}%
\pgfpathlineto{\pgfqpoint{3.933998in}{0.618939in}}%
\pgfpathlineto{\pgfqpoint{3.934554in}{0.612378in}}%
\pgfpathlineto{\pgfqpoint{3.935110in}{0.619581in}}%
\pgfpathlineto{\pgfqpoint{3.935666in}{0.602866in}}%
\pgfpathlineto{\pgfqpoint{3.936222in}{0.614294in}}%
\pgfpathlineto{\pgfqpoint{3.936778in}{0.614517in}}%
\pgfpathlineto{\pgfqpoint{3.937333in}{0.620950in}}%
\pgfpathlineto{\pgfqpoint{3.937889in}{0.613422in}}%
\pgfpathlineto{\pgfqpoint{3.939001in}{0.629011in}}%
\pgfpathlineto{\pgfqpoint{3.939557in}{0.606472in}}%
\pgfpathlineto{\pgfqpoint{3.941224in}{0.661214in}}%
\pgfpathlineto{\pgfqpoint{3.942892in}{0.628857in}}%
\pgfpathlineto{\pgfqpoint{3.944560in}{0.702307in}}%
\pgfpathlineto{\pgfqpoint{3.945671in}{0.632956in}}%
\pgfpathlineto{\pgfqpoint{3.946783in}{0.735023in}}%
\pgfpathlineto{\pgfqpoint{3.947895in}{0.618124in}}%
\pgfpathlineto{\pgfqpoint{3.948451in}{0.622118in}}%
\pgfpathlineto{\pgfqpoint{3.949006in}{0.623854in}}%
\pgfpathlineto{\pgfqpoint{3.949562in}{0.621289in}}%
\pgfpathlineto{\pgfqpoint{3.951230in}{0.690705in}}%
\pgfpathlineto{\pgfqpoint{3.952897in}{0.626968in}}%
\pgfpathlineto{\pgfqpoint{3.954009in}{0.701301in}}%
\pgfpathlineto{\pgfqpoint{3.955121in}{0.623329in}}%
\pgfpathlineto{\pgfqpoint{3.955677in}{0.625525in}}%
\pgfpathlineto{\pgfqpoint{3.956233in}{0.676947in}}%
\pgfpathlineto{\pgfqpoint{3.956789in}{0.609635in}}%
\pgfpathlineto{\pgfqpoint{3.957344in}{0.654476in}}%
\pgfpathlineto{\pgfqpoint{3.957900in}{0.619945in}}%
\pgfpathlineto{\pgfqpoint{3.958456in}{0.629102in}}%
\pgfpathlineto{\pgfqpoint{3.959012in}{0.628630in}}%
\pgfpathlineto{\pgfqpoint{3.959568in}{0.648861in}}%
\pgfpathlineto{\pgfqpoint{3.960680in}{0.607244in}}%
\pgfpathlineto{\pgfqpoint{3.961235in}{0.649217in}}%
\pgfpathlineto{\pgfqpoint{3.961791in}{0.639972in}}%
\pgfpathlineto{\pgfqpoint{3.962903in}{0.607938in}}%
\pgfpathlineto{\pgfqpoint{3.964571in}{0.650951in}}%
\pgfpathlineto{\pgfqpoint{3.965682in}{0.605049in}}%
\pgfpathlineto{\pgfqpoint{3.966238in}{0.618587in}}%
\pgfpathlineto{\pgfqpoint{3.967350in}{0.653596in}}%
\pgfpathlineto{\pgfqpoint{3.967906in}{0.647992in}}%
\pgfpathlineto{\pgfqpoint{3.969573in}{0.602224in}}%
\pgfpathlineto{\pgfqpoint{3.971797in}{0.645270in}}%
\pgfpathlineto{\pgfqpoint{3.972353in}{0.637054in}}%
\pgfpathlineto{\pgfqpoint{3.972909in}{0.638916in}}%
\pgfpathlineto{\pgfqpoint{3.973464in}{0.646307in}}%
\pgfpathlineto{\pgfqpoint{3.975688in}{0.625807in}}%
\pgfpathlineto{\pgfqpoint{3.977355in}{0.615534in}}%
\pgfpathlineto{\pgfqpoint{3.979023in}{0.607452in}}%
\pgfpathlineto{\pgfqpoint{3.979579in}{0.610542in}}%
\pgfpathlineto{\pgfqpoint{3.980135in}{0.601491in}}%
\pgfpathlineto{\pgfqpoint{3.980691in}{0.602489in}}%
\pgfpathlineto{\pgfqpoint{3.981246in}{0.601551in}}%
\pgfpathlineto{\pgfqpoint{3.981802in}{0.602231in}}%
\pgfpathlineto{\pgfqpoint{3.982914in}{0.609364in}}%
\pgfpathlineto{\pgfqpoint{3.983470in}{0.604004in}}%
\pgfpathlineto{\pgfqpoint{3.984582in}{0.614377in}}%
\pgfpathlineto{\pgfqpoint{3.985137in}{0.605204in}}%
\pgfpathlineto{\pgfqpoint{3.985693in}{0.608040in}}%
\pgfpathlineto{\pgfqpoint{3.986249in}{0.614802in}}%
\pgfpathlineto{\pgfqpoint{3.986805in}{0.608317in}}%
\pgfpathlineto{\pgfqpoint{3.987361in}{0.605129in}}%
\pgfpathlineto{\pgfqpoint{3.989028in}{0.614871in}}%
\pgfpathlineto{\pgfqpoint{3.989584in}{0.613398in}}%
\pgfpathlineto{\pgfqpoint{3.990140in}{0.614184in}}%
\pgfpathlineto{\pgfqpoint{3.991808in}{0.605723in}}%
\pgfpathlineto{\pgfqpoint{3.993475in}{0.628663in}}%
\pgfpathlineto{\pgfqpoint{3.995143in}{0.609371in}}%
\pgfpathlineto{\pgfqpoint{3.996255in}{0.635122in}}%
\pgfpathlineto{\pgfqpoint{3.996811in}{0.608475in}}%
\pgfpathlineto{\pgfqpoint{3.998478in}{0.646129in}}%
\pgfpathlineto{\pgfqpoint{3.999034in}{0.622529in}}%
\pgfpathlineto{\pgfqpoint{3.999590in}{0.659152in}}%
\pgfpathlineto{\pgfqpoint{4.000146in}{0.610680in}}%
\pgfpathlineto{\pgfqpoint{4.000702in}{0.649020in}}%
\pgfpathlineto{\pgfqpoint{4.001257in}{0.643639in}}%
\pgfpathlineto{\pgfqpoint{4.001813in}{0.696811in}}%
\pgfpathlineto{\pgfqpoint{4.002369in}{0.653499in}}%
\pgfpathlineto{\pgfqpoint{4.002925in}{0.651764in}}%
\pgfpathlineto{\pgfqpoint{4.003481in}{0.644908in}}%
\pgfpathlineto{\pgfqpoint{4.004593in}{0.673327in}}%
\pgfpathlineto{\pgfqpoint{4.005704in}{0.622731in}}%
\pgfpathlineto{\pgfqpoint{4.006260in}{0.653236in}}%
\pgfpathlineto{\pgfqpoint{4.006816in}{0.608104in}}%
\pgfpathlineto{\pgfqpoint{4.007928in}{0.677417in}}%
\pgfpathlineto{\pgfqpoint{4.008484in}{0.619238in}}%
\pgfpathlineto{\pgfqpoint{4.009039in}{0.626297in}}%
\pgfpathlineto{\pgfqpoint{4.009595in}{0.623026in}}%
\pgfpathlineto{\pgfqpoint{4.011263in}{0.672843in}}%
\pgfpathlineto{\pgfqpoint{4.012931in}{0.625073in}}%
\pgfpathlineto{\pgfqpoint{4.013486in}{0.676046in}}%
\pgfpathlineto{\pgfqpoint{4.014042in}{0.628719in}}%
\pgfpathlineto{\pgfqpoint{4.015710in}{0.640639in}}%
\pgfpathlineto{\pgfqpoint{4.016266in}{0.606988in}}%
\pgfpathlineto{\pgfqpoint{4.016822in}{0.652370in}}%
\pgfpathlineto{\pgfqpoint{4.017377in}{0.610620in}}%
\pgfpathlineto{\pgfqpoint{4.019045in}{0.630798in}}%
\pgfpathlineto{\pgfqpoint{4.019601in}{0.612735in}}%
\pgfpathlineto{\pgfqpoint{4.020157in}{0.616320in}}%
\pgfpathlineto{\pgfqpoint{4.021268in}{0.650247in}}%
\pgfpathlineto{\pgfqpoint{4.021824in}{0.632610in}}%
\pgfpathlineto{\pgfqpoint{4.022380in}{0.606557in}}%
\pgfpathlineto{\pgfqpoint{4.022936in}{0.625830in}}%
\pgfpathlineto{\pgfqpoint{4.024048in}{0.639762in}}%
\pgfpathlineto{\pgfqpoint{4.025715in}{0.607341in}}%
\pgfpathlineto{\pgfqpoint{4.027383in}{0.640602in}}%
\pgfpathlineto{\pgfqpoint{4.029051in}{0.604599in}}%
\pgfpathlineto{\pgfqpoint{4.029606in}{0.610053in}}%
\pgfpathlineto{\pgfqpoint{4.030162in}{0.616632in}}%
\pgfpathlineto{\pgfqpoint{4.030718in}{0.639117in}}%
\pgfpathlineto{\pgfqpoint{4.031274in}{0.633579in}}%
\pgfpathlineto{\pgfqpoint{4.031830in}{0.632362in}}%
\pgfpathlineto{\pgfqpoint{4.032942in}{0.649650in}}%
\pgfpathlineto{\pgfqpoint{4.034609in}{0.633244in}}%
\pgfpathlineto{\pgfqpoint{4.035165in}{0.635240in}}%
\pgfpathlineto{\pgfqpoint{4.036277in}{0.635092in}}%
\pgfpathlineto{\pgfqpoint{4.037388in}{0.612649in}}%
\pgfpathlineto{\pgfqpoint{4.037944in}{0.613598in}}%
\pgfpathlineto{\pgfqpoint{4.038500in}{0.607907in}}%
\pgfpathlineto{\pgfqpoint{4.040168in}{0.618678in}}%
\pgfpathlineto{\pgfqpoint{4.040724in}{0.614814in}}%
\pgfpathlineto{\pgfqpoint{4.041279in}{0.602279in}}%
\pgfpathlineto{\pgfqpoint{4.041835in}{0.606887in}}%
\pgfpathlineto{\pgfqpoint{4.042947in}{0.607739in}}%
\pgfpathlineto{\pgfqpoint{4.043503in}{0.619330in}}%
\pgfpathlineto{\pgfqpoint{4.044059in}{0.610120in}}%
\pgfpathlineto{\pgfqpoint{4.044615in}{0.617785in}}%
\pgfpathlineto{\pgfqpoint{4.045170in}{0.612866in}}%
\pgfpathlineto{\pgfqpoint{4.045726in}{0.611910in}}%
\pgfpathlineto{\pgfqpoint{4.046838in}{0.618519in}}%
\pgfpathlineto{\pgfqpoint{4.047950in}{0.609941in}}%
\pgfpathlineto{\pgfqpoint{4.049617in}{0.632212in}}%
\pgfpathlineto{\pgfqpoint{4.050173in}{0.606483in}}%
\pgfpathlineto{\pgfqpoint{4.050729in}{0.615430in}}%
\pgfpathlineto{\pgfqpoint{4.051841in}{0.617312in}}%
\pgfpathlineto{\pgfqpoint{4.052397in}{0.601359in}}%
\pgfpathlineto{\pgfqpoint{4.052953in}{0.613908in}}%
\pgfpathlineto{\pgfqpoint{4.053508in}{0.618627in}}%
\pgfpathlineto{\pgfqpoint{4.054064in}{0.635849in}}%
\pgfpathlineto{\pgfqpoint{4.054620in}{0.628653in}}%
\pgfpathlineto{\pgfqpoint{4.055176in}{0.628845in}}%
\pgfpathlineto{\pgfqpoint{4.055732in}{0.627066in}}%
\pgfpathlineto{\pgfqpoint{4.056844in}{0.655579in}}%
\pgfpathlineto{\pgfqpoint{4.057955in}{0.639678in}}%
\pgfpathlineto{\pgfqpoint{4.059623in}{0.670641in}}%
\pgfpathlineto{\pgfqpoint{4.060735in}{0.631502in}}%
\pgfpathlineto{\pgfqpoint{4.061846in}{0.670541in}}%
\pgfpathlineto{\pgfqpoint{4.063514in}{0.622553in}}%
\pgfpathlineto{\pgfqpoint{4.064070in}{0.624899in}}%
\pgfpathlineto{\pgfqpoint{4.065737in}{0.662297in}}%
\pgfpathlineto{\pgfqpoint{4.066293in}{0.635288in}}%
\pgfpathlineto{\pgfqpoint{4.066849in}{0.653563in}}%
\pgfpathlineto{\pgfqpoint{4.067405in}{0.668910in}}%
\pgfpathlineto{\pgfqpoint{4.067961in}{0.619696in}}%
\pgfpathlineto{\pgfqpoint{4.068517in}{0.657282in}}%
\pgfpathlineto{\pgfqpoint{4.069628in}{0.647317in}}%
\pgfpathlineto{\pgfqpoint{4.070184in}{0.611560in}}%
\pgfpathlineto{\pgfqpoint{4.071296in}{0.665346in}}%
\pgfpathlineto{\pgfqpoint{4.071852in}{0.611619in}}%
\pgfpathlineto{\pgfqpoint{4.072408in}{0.657742in}}%
\pgfpathlineto{\pgfqpoint{4.073519in}{0.607810in}}%
\pgfpathlineto{\pgfqpoint{4.074075in}{0.649271in}}%
\pgfpathlineto{\pgfqpoint{4.074631in}{0.626728in}}%
\pgfpathlineto{\pgfqpoint{4.075187in}{0.628073in}}%
\pgfpathlineto{\pgfqpoint{4.075743in}{0.605950in}}%
\pgfpathlineto{\pgfqpoint{4.076299in}{0.633153in}}%
\pgfpathlineto{\pgfqpoint{4.076855in}{0.623908in}}%
\pgfpathlineto{\pgfqpoint{4.077410in}{0.622910in}}%
\pgfpathlineto{\pgfqpoint{4.077966in}{0.627704in}}%
\pgfpathlineto{\pgfqpoint{4.078522in}{0.650776in}}%
\pgfpathlineto{\pgfqpoint{4.079634in}{0.607450in}}%
\pgfpathlineto{\pgfqpoint{4.080746in}{0.637656in}}%
\pgfpathlineto{\pgfqpoint{4.081301in}{0.627096in}}%
\pgfpathlineto{\pgfqpoint{4.081857in}{0.626487in}}%
\pgfpathlineto{\pgfqpoint{4.082969in}{0.611659in}}%
\pgfpathlineto{\pgfqpoint{4.083525in}{0.649668in}}%
\pgfpathlineto{\pgfqpoint{4.084081in}{0.632403in}}%
\pgfpathlineto{\pgfqpoint{4.085748in}{0.612672in}}%
\pgfpathlineto{\pgfqpoint{4.086304in}{0.618630in}}%
\pgfpathlineto{\pgfqpoint{4.087416in}{0.649235in}}%
\pgfpathlineto{\pgfqpoint{4.087972in}{0.645919in}}%
\pgfpathlineto{\pgfqpoint{4.089639in}{0.608994in}}%
\pgfpathlineto{\pgfqpoint{4.090195in}{0.614130in}}%
\pgfpathlineto{\pgfqpoint{4.091863in}{0.640243in}}%
\pgfpathlineto{\pgfqpoint{4.092419in}{0.641281in}}%
\pgfpathlineto{\pgfqpoint{4.092975in}{0.628091in}}%
\pgfpathlineto{\pgfqpoint{4.093530in}{0.639540in}}%
\pgfpathlineto{\pgfqpoint{4.094086in}{0.637513in}}%
\pgfpathlineto{\pgfqpoint{4.094642in}{0.647744in}}%
\pgfpathlineto{\pgfqpoint{4.095754in}{0.628617in}}%
\pgfpathlineto{\pgfqpoint{4.096310in}{0.630637in}}%
\pgfpathlineto{\pgfqpoint{4.097977in}{0.615343in}}%
\pgfpathlineto{\pgfqpoint{4.098533in}{0.606974in}}%
\pgfpathlineto{\pgfqpoint{4.099089in}{0.607329in}}%
\pgfpathlineto{\pgfqpoint{4.099645in}{0.617931in}}%
\pgfpathlineto{\pgfqpoint{4.100201in}{0.601235in}}%
\pgfpathlineto{\pgfqpoint{4.100757in}{0.608995in}}%
\pgfpathlineto{\pgfqpoint{4.101312in}{0.604585in}}%
\pgfpathlineto{\pgfqpoint{4.101868in}{0.615492in}}%
\pgfpathlineto{\pgfqpoint{4.102424in}{0.602881in}}%
\pgfpathlineto{\pgfqpoint{4.102980in}{0.609522in}}%
\pgfpathlineto{\pgfqpoint{4.103536in}{0.605674in}}%
\pgfpathlineto{\pgfqpoint{4.104092in}{0.616714in}}%
\pgfpathlineto{\pgfqpoint{4.104648in}{0.607062in}}%
\pgfpathlineto{\pgfqpoint{4.105204in}{0.601741in}}%
\pgfpathlineto{\pgfqpoint{4.105759in}{0.618222in}}%
\pgfpathlineto{\pgfqpoint{4.106315in}{0.609626in}}%
\pgfpathlineto{\pgfqpoint{4.107427in}{0.617594in}}%
\pgfpathlineto{\pgfqpoint{4.108539in}{0.609303in}}%
\pgfpathlineto{\pgfqpoint{4.109095in}{0.643542in}}%
\pgfpathlineto{\pgfqpoint{4.109650in}{0.613944in}}%
\pgfpathlineto{\pgfqpoint{4.110206in}{0.624595in}}%
\pgfpathlineto{\pgfqpoint{4.110762in}{0.609104in}}%
\pgfpathlineto{\pgfqpoint{4.111318in}{0.641517in}}%
\pgfpathlineto{\pgfqpoint{4.111874in}{0.637167in}}%
\pgfpathlineto{\pgfqpoint{4.112430in}{0.633173in}}%
\pgfpathlineto{\pgfqpoint{4.112986in}{0.620702in}}%
\pgfpathlineto{\pgfqpoint{4.114653in}{0.666701in}}%
\pgfpathlineto{\pgfqpoint{4.115765in}{0.627765in}}%
\pgfpathlineto{\pgfqpoint{4.116877in}{0.689696in}}%
\pgfpathlineto{\pgfqpoint{4.117432in}{0.621189in}}%
\pgfpathlineto{\pgfqpoint{4.117988in}{0.635056in}}%
\pgfpathlineto{\pgfqpoint{4.118544in}{0.657722in}}%
\pgfpathlineto{\pgfqpoint{4.119100in}{0.646347in}}%
\pgfpathlineto{\pgfqpoint{4.119656in}{0.611908in}}%
\pgfpathlineto{\pgfqpoint{4.120212in}{0.649593in}}%
\pgfpathlineto{\pgfqpoint{4.120768in}{0.636100in}}%
\pgfpathlineto{\pgfqpoint{4.121323in}{0.645623in}}%
\pgfpathlineto{\pgfqpoint{4.121879in}{0.608764in}}%
\pgfpathlineto{\pgfqpoint{4.122435in}{0.643651in}}%
\pgfpathlineto{\pgfqpoint{4.122991in}{0.669219in}}%
\pgfpathlineto{\pgfqpoint{4.123547in}{0.618167in}}%
\pgfpathlineto{\pgfqpoint{4.124103in}{0.640762in}}%
\pgfpathlineto{\pgfqpoint{4.124659in}{0.641991in}}%
\pgfpathlineto{\pgfqpoint{4.126326in}{0.654230in}}%
\pgfpathlineto{\pgfqpoint{4.127994in}{0.617681in}}%
\pgfpathlineto{\pgfqpoint{4.128550in}{0.689252in}}%
\pgfpathlineto{\pgfqpoint{4.129106in}{0.610857in}}%
\pgfpathlineto{\pgfqpoint{4.129661in}{0.646328in}}%
\pgfpathlineto{\pgfqpoint{4.130217in}{0.619579in}}%
\pgfpathlineto{\pgfqpoint{4.130773in}{0.626784in}}%
\pgfpathlineto{\pgfqpoint{4.131329in}{0.620049in}}%
\pgfpathlineto{\pgfqpoint{4.131885in}{0.648037in}}%
\pgfpathlineto{\pgfqpoint{4.132997in}{0.607110in}}%
\pgfpathlineto{\pgfqpoint{4.133552in}{0.631360in}}%
\pgfpathlineto{\pgfqpoint{4.134108in}{0.627494in}}%
\pgfpathlineto{\pgfqpoint{4.134664in}{0.626670in}}%
\pgfpathlineto{\pgfqpoint{4.135220in}{0.605176in}}%
\pgfpathlineto{\pgfqpoint{4.135776in}{0.629999in}}%
\pgfpathlineto{\pgfqpoint{4.136332in}{0.615679in}}%
\pgfpathlineto{\pgfqpoint{4.136888in}{0.608455in}}%
\pgfpathlineto{\pgfqpoint{4.137443in}{0.612844in}}%
\pgfpathlineto{\pgfqpoint{4.137999in}{0.634650in}}%
\pgfpathlineto{\pgfqpoint{4.138555in}{0.620111in}}%
\pgfpathlineto{\pgfqpoint{4.139111in}{0.609898in}}%
\pgfpathlineto{\pgfqpoint{4.140223in}{0.630061in}}%
\pgfpathlineto{\pgfqpoint{4.140779in}{0.624369in}}%
\pgfpathlineto{\pgfqpoint{4.141334in}{0.633950in}}%
\pgfpathlineto{\pgfqpoint{4.142446in}{0.603481in}}%
\pgfpathlineto{\pgfqpoint{4.144114in}{0.635982in}}%
\pgfpathlineto{\pgfqpoint{4.145781in}{0.612990in}}%
\pgfpathlineto{\pgfqpoint{4.146337in}{0.617469in}}%
\pgfpathlineto{\pgfqpoint{4.146893in}{0.627619in}}%
\pgfpathlineto{\pgfqpoint{4.147449in}{0.627397in}}%
\pgfpathlineto{\pgfqpoint{4.149117in}{0.617210in}}%
\pgfpathlineto{\pgfqpoint{4.149672in}{0.609487in}}%
\pgfpathlineto{\pgfqpoint{4.150228in}{0.614560in}}%
\pgfpathlineto{\pgfqpoint{4.151340in}{0.616176in}}%
\pgfpathlineto{\pgfqpoint{4.153008in}{0.637193in}}%
\pgfpathlineto{\pgfqpoint{4.153563in}{0.637260in}}%
\pgfpathlineto{\pgfqpoint{4.154675in}{0.625587in}}%
\pgfpathlineto{\pgfqpoint{4.155231in}{0.636358in}}%
\pgfpathlineto{\pgfqpoint{4.156899in}{0.612426in}}%
\pgfpathlineto{\pgfqpoint{4.157454in}{0.614249in}}%
\pgfpathlineto{\pgfqpoint{4.158010in}{0.624198in}}%
\pgfpathlineto{\pgfqpoint{4.158566in}{0.622322in}}%
\pgfpathlineto{\pgfqpoint{4.159678in}{0.604444in}}%
\pgfpathlineto{\pgfqpoint{4.160234in}{0.619652in}}%
\pgfpathlineto{\pgfqpoint{4.160790in}{0.610249in}}%
\pgfpathlineto{\pgfqpoint{4.161346in}{0.617597in}}%
\pgfpathlineto{\pgfqpoint{4.161901in}{0.607264in}}%
\pgfpathlineto{\pgfqpoint{4.162457in}{0.623045in}}%
\pgfpathlineto{\pgfqpoint{4.164125in}{0.606389in}}%
\pgfpathlineto{\pgfqpoint{4.166348in}{0.635022in}}%
\pgfpathlineto{\pgfqpoint{4.166904in}{0.603845in}}%
\pgfpathlineto{\pgfqpoint{4.167460in}{0.623429in}}%
\pgfpathlineto{\pgfqpoint{4.168016in}{0.606128in}}%
\pgfpathlineto{\pgfqpoint{4.169683in}{0.639241in}}%
\pgfpathlineto{\pgfqpoint{4.170239in}{0.616042in}}%
\pgfpathlineto{\pgfqpoint{4.170795in}{0.631178in}}%
\pgfpathlineto{\pgfqpoint{4.171351in}{0.627055in}}%
\pgfpathlineto{\pgfqpoint{4.171907in}{0.646591in}}%
\pgfpathlineto{\pgfqpoint{4.172463in}{0.625959in}}%
\pgfpathlineto{\pgfqpoint{4.173019in}{0.626954in}}%
\pgfpathlineto{\pgfqpoint{4.173574in}{0.634215in}}%
\pgfpathlineto{\pgfqpoint{4.174130in}{0.658999in}}%
\pgfpathlineto{\pgfqpoint{4.175798in}{0.627863in}}%
\pgfpathlineto{\pgfqpoint{4.176354in}{0.633725in}}%
\pgfpathlineto{\pgfqpoint{4.176910in}{0.627784in}}%
\pgfpathlineto{\pgfqpoint{4.178021in}{0.651627in}}%
\pgfpathlineto{\pgfqpoint{4.179133in}{0.609209in}}%
\pgfpathlineto{\pgfqpoint{4.179689in}{0.622427in}}%
\pgfpathlineto{\pgfqpoint{4.180245in}{0.617924in}}%
\pgfpathlineto{\pgfqpoint{4.180801in}{0.626920in}}%
\pgfpathlineto{\pgfqpoint{4.181357in}{0.608019in}}%
\pgfpathlineto{\pgfqpoint{4.182468in}{0.643409in}}%
\pgfpathlineto{\pgfqpoint{4.184136in}{0.613325in}}%
\pgfpathlineto{\pgfqpoint{4.184692in}{0.643635in}}%
\pgfpathlineto{\pgfqpoint{4.185248in}{0.604819in}}%
\pgfpathlineto{\pgfqpoint{4.185803in}{0.657737in}}%
\pgfpathlineto{\pgfqpoint{4.186359in}{0.616855in}}%
\pgfpathlineto{\pgfqpoint{4.186915in}{0.616362in}}%
\pgfpathlineto{\pgfqpoint{4.187471in}{0.618100in}}%
\pgfpathlineto{\pgfqpoint{4.188027in}{0.636014in}}%
\pgfpathlineto{\pgfqpoint{4.188583in}{0.602290in}}%
\pgfpathlineto{\pgfqpoint{4.189139in}{0.635037in}}%
\pgfpathlineto{\pgfqpoint{4.189694in}{0.609499in}}%
\pgfpathlineto{\pgfqpoint{4.190250in}{0.610524in}}%
\pgfpathlineto{\pgfqpoint{4.190806in}{0.611474in}}%
\pgfpathlineto{\pgfqpoint{4.191918in}{0.622109in}}%
\pgfpathlineto{\pgfqpoint{4.193030in}{0.618289in}}%
\pgfpathlineto{\pgfqpoint{4.193585in}{0.623667in}}%
\pgfpathlineto{\pgfqpoint{4.194697in}{0.604074in}}%
\pgfpathlineto{\pgfqpoint{4.195253in}{0.619981in}}%
\pgfpathlineto{\pgfqpoint{4.195809in}{0.614613in}}%
\pgfpathlineto{\pgfqpoint{4.196365in}{0.602630in}}%
\pgfpathlineto{\pgfqpoint{4.196921in}{0.610350in}}%
\pgfpathlineto{\pgfqpoint{4.197476in}{0.621063in}}%
\pgfpathlineto{\pgfqpoint{4.199144in}{0.604824in}}%
\pgfpathlineto{\pgfqpoint{4.201368in}{0.617938in}}%
\pgfpathlineto{\pgfqpoint{4.201923in}{0.608817in}}%
\pgfpathlineto{\pgfqpoint{4.202479in}{0.609847in}}%
\pgfpathlineto{\pgfqpoint{4.203035in}{0.610624in}}%
\pgfpathlineto{\pgfqpoint{4.203591in}{0.627545in}}%
\pgfpathlineto{\pgfqpoint{4.204147in}{0.618168in}}%
\pgfpathlineto{\pgfqpoint{4.205814in}{0.600841in}}%
\pgfpathlineto{\pgfqpoint{4.206370in}{0.605454in}}%
\pgfpathlineto{\pgfqpoint{4.206926in}{0.624270in}}%
\pgfpathlineto{\pgfqpoint{4.207482in}{0.620001in}}%
\pgfpathlineto{\pgfqpoint{4.208038in}{0.622304in}}%
\pgfpathlineto{\pgfqpoint{4.209150in}{0.605438in}}%
\pgfpathlineto{\pgfqpoint{4.209705in}{0.605719in}}%
\pgfpathlineto{\pgfqpoint{4.210817in}{0.608452in}}%
\pgfpathlineto{\pgfqpoint{4.211929in}{0.616911in}}%
\pgfpathlineto{\pgfqpoint{4.212485in}{0.614064in}}%
\pgfpathlineto{\pgfqpoint{4.213041in}{0.614783in}}%
\pgfpathlineto{\pgfqpoint{4.213596in}{0.623516in}}%
\pgfpathlineto{\pgfqpoint{4.214152in}{0.619371in}}%
\pgfpathlineto{\pgfqpoint{4.214708in}{0.621964in}}%
\pgfpathlineto{\pgfqpoint{4.215264in}{0.609137in}}%
\pgfpathlineto{\pgfqpoint{4.215820in}{0.610231in}}%
\pgfpathlineto{\pgfqpoint{4.216932in}{0.618771in}}%
\pgfpathlineto{\pgfqpoint{4.218043in}{0.608521in}}%
\pgfpathlineto{\pgfqpoint{4.218599in}{0.610190in}}%
\pgfpathlineto{\pgfqpoint{4.219155in}{0.606339in}}%
\pgfpathlineto{\pgfqpoint{4.219711in}{0.612087in}}%
\pgfpathlineto{\pgfqpoint{4.220267in}{0.605058in}}%
\pgfpathlineto{\pgfqpoint{4.220823in}{0.609009in}}%
\pgfpathlineto{\pgfqpoint{4.221379in}{0.608558in}}%
\pgfpathlineto{\pgfqpoint{4.221934in}{0.609992in}}%
\pgfpathlineto{\pgfqpoint{4.222490in}{0.604694in}}%
\pgfpathlineto{\pgfqpoint{4.223046in}{0.605714in}}%
\pgfpathlineto{\pgfqpoint{4.224158in}{0.613632in}}%
\pgfpathlineto{\pgfqpoint{4.224714in}{0.607653in}}%
\pgfpathlineto{\pgfqpoint{4.225270in}{0.609038in}}%
\pgfpathlineto{\pgfqpoint{4.225825in}{0.608616in}}%
\pgfpathlineto{\pgfqpoint{4.226381in}{0.620575in}}%
\pgfpathlineto{\pgfqpoint{4.226937in}{0.619684in}}%
\pgfpathlineto{\pgfqpoint{4.228049in}{0.610997in}}%
\pgfpathlineto{\pgfqpoint{4.229716in}{0.619410in}}%
\pgfpathlineto{\pgfqpoint{4.230828in}{0.606189in}}%
\pgfpathlineto{\pgfqpoint{4.231940in}{0.626264in}}%
\pgfpathlineto{\pgfqpoint{4.233052in}{0.608323in}}%
\pgfpathlineto{\pgfqpoint{4.233607in}{0.615907in}}%
\pgfpathlineto{\pgfqpoint{4.234163in}{0.609660in}}%
\pgfpathlineto{\pgfqpoint{4.235275in}{0.619702in}}%
\pgfpathlineto{\pgfqpoint{4.236943in}{0.603008in}}%
\pgfpathlineto{\pgfqpoint{4.238054in}{0.616715in}}%
\pgfpathlineto{\pgfqpoint{4.238610in}{0.603603in}}%
\pgfpathlineto{\pgfqpoint{4.239166in}{0.617638in}}%
\pgfpathlineto{\pgfqpoint{4.239722in}{0.614220in}}%
\pgfpathlineto{\pgfqpoint{4.240278in}{0.606645in}}%
\pgfpathlineto{\pgfqpoint{4.240834in}{0.610248in}}%
\pgfpathlineto{\pgfqpoint{4.241390in}{0.611298in}}%
\pgfpathlineto{\pgfqpoint{4.241945in}{0.618889in}}%
\pgfpathlineto{\pgfqpoint{4.242501in}{0.608523in}}%
\pgfpathlineto{\pgfqpoint{4.243057in}{0.612836in}}%
\pgfpathlineto{\pgfqpoint{4.243613in}{0.620353in}}%
\pgfpathlineto{\pgfqpoint{4.244169in}{0.608711in}}%
\pgfpathlineto{\pgfqpoint{4.244725in}{0.613721in}}%
\pgfpathlineto{\pgfqpoint{4.245281in}{0.613051in}}%
\pgfpathlineto{\pgfqpoint{4.245836in}{0.605245in}}%
\pgfpathlineto{\pgfqpoint{4.246392in}{0.608257in}}%
\pgfpathlineto{\pgfqpoint{4.246948in}{0.610167in}}%
\pgfpathlineto{\pgfqpoint{4.248060in}{0.601144in}}%
\pgfpathlineto{\pgfqpoint{4.249727in}{0.611068in}}%
\pgfpathlineto{\pgfqpoint{4.250283in}{0.602659in}}%
\pgfpathlineto{\pgfqpoint{4.250839in}{0.608460in}}%
\pgfpathlineto{\pgfqpoint{4.252507in}{0.603984in}}%
\pgfpathlineto{\pgfqpoint{4.253063in}{0.606840in}}%
\pgfpathlineto{\pgfqpoint{4.254174in}{0.601064in}}%
\pgfpathlineto{\pgfqpoint{4.254730in}{0.609254in}}%
\pgfpathlineto{\pgfqpoint{4.255286in}{0.604881in}}%
\pgfpathlineto{\pgfqpoint{4.255842in}{0.604525in}}%
\pgfpathlineto{\pgfqpoint{4.257510in}{0.601286in}}%
\pgfpathlineto{\pgfqpoint{4.258065in}{0.604712in}}%
\pgfpathlineto{\pgfqpoint{4.258621in}{0.602170in}}%
\pgfpathlineto{\pgfqpoint{4.259177in}{0.601460in}}%
\pgfpathlineto{\pgfqpoint{4.260289in}{0.607748in}}%
\pgfpathlineto{\pgfqpoint{4.260845in}{0.606997in}}%
\pgfpathlineto{\pgfqpoint{4.261956in}{0.601564in}}%
\pgfpathlineto{\pgfqpoint{4.262512in}{0.601796in}}%
\pgfpathlineto{\pgfqpoint{4.264180in}{0.606429in}}%
\pgfpathlineto{\pgfqpoint{4.265847in}{0.601744in}}%
\pgfpathlineto{\pgfqpoint{4.267515in}{0.606093in}}%
\pgfpathlineto{\pgfqpoint{4.270850in}{0.600684in}}%
\pgfpathlineto{\pgfqpoint{4.273074in}{0.605950in}}%
\pgfpathlineto{\pgfqpoint{4.274185in}{0.602221in}}%
\pgfpathlineto{\pgfqpoint{4.274741in}{0.605080in}}%
\pgfpathlineto{\pgfqpoint{4.275297in}{0.604628in}}%
\pgfpathlineto{\pgfqpoint{4.277521in}{0.601705in}}%
\pgfpathlineto{\pgfqpoint{4.278076in}{0.604228in}}%
\pgfpathlineto{\pgfqpoint{4.279744in}{0.600094in}}%
\pgfpathlineto{\pgfqpoint{4.280300in}{0.600798in}}%
\pgfpathlineto{\pgfqpoint{4.280856in}{0.600232in}}%
\pgfpathlineto{\pgfqpoint{4.281412in}{0.606151in}}%
\pgfpathlineto{\pgfqpoint{4.281967in}{0.601533in}}%
\pgfpathlineto{\pgfqpoint{4.282523in}{0.603339in}}%
\pgfpathlineto{\pgfqpoint{4.283079in}{0.600742in}}%
\pgfpathlineto{\pgfqpoint{4.284747in}{0.604224in}}%
\pgfpathlineto{\pgfqpoint{4.285303in}{0.601308in}}%
\pgfpathlineto{\pgfqpoint{4.285858in}{0.603111in}}%
\pgfpathlineto{\pgfqpoint{4.286970in}{0.605273in}}%
\pgfpathlineto{\pgfqpoint{4.288082in}{0.602369in}}%
\pgfpathlineto{\pgfqpoint{4.288638in}{0.606345in}}%
\pgfpathlineto{\pgfqpoint{4.289194in}{0.605486in}}%
\pgfpathlineto{\pgfqpoint{4.289749in}{0.600475in}}%
\pgfpathlineto{\pgfqpoint{4.290305in}{0.603301in}}%
\pgfpathlineto{\pgfqpoint{4.291417in}{0.600527in}}%
\pgfpathlineto{\pgfqpoint{4.291973in}{0.601095in}}%
\pgfpathlineto{\pgfqpoint{4.293085in}{0.605999in}}%
\pgfpathlineto{\pgfqpoint{4.294196in}{0.601052in}}%
\pgfpathlineto{\pgfqpoint{4.294752in}{0.602469in}}%
\pgfpathlineto{\pgfqpoint{4.296420in}{0.601827in}}%
\pgfpathlineto{\pgfqpoint{4.297532in}{0.603320in}}%
\pgfpathlineto{\pgfqpoint{4.298087in}{0.601657in}}%
\pgfpathlineto{\pgfqpoint{4.298643in}{0.603204in}}%
\pgfpathlineto{\pgfqpoint{4.299199in}{0.602308in}}%
\pgfpathlineto{\pgfqpoint{4.299755in}{0.603733in}}%
\pgfpathlineto{\pgfqpoint{4.300311in}{0.600868in}}%
\pgfpathlineto{\pgfqpoint{4.300867in}{0.606659in}}%
\pgfpathlineto{\pgfqpoint{4.301423in}{0.601125in}}%
\pgfpathlineto{\pgfqpoint{4.301978in}{0.603057in}}%
\pgfpathlineto{\pgfqpoint{4.302534in}{0.600547in}}%
\pgfpathlineto{\pgfqpoint{4.303090in}{0.604118in}}%
\pgfpathlineto{\pgfqpoint{4.303646in}{0.600743in}}%
\pgfpathlineto{\pgfqpoint{4.304202in}{0.603674in}}%
\pgfpathlineto{\pgfqpoint{4.304758in}{0.600879in}}%
\pgfpathlineto{\pgfqpoint{4.305869in}{0.601698in}}%
\pgfpathlineto{\pgfqpoint{4.306425in}{0.603038in}}%
\pgfpathlineto{\pgfqpoint{4.306981in}{0.602206in}}%
\pgfpathlineto{\pgfqpoint{4.307537in}{0.601712in}}%
\pgfpathlineto{\pgfqpoint{4.309205in}{0.603529in}}%
\pgfpathlineto{\pgfqpoint{4.309760in}{0.600636in}}%
\pgfpathlineto{\pgfqpoint{4.310316in}{0.602072in}}%
\pgfpathlineto{\pgfqpoint{4.311428in}{0.600567in}}%
\pgfpathlineto{\pgfqpoint{4.311984in}{0.601078in}}%
\pgfpathlineto{\pgfqpoint{4.312540in}{0.603173in}}%
\pgfpathlineto{\pgfqpoint{4.313096in}{0.601034in}}%
\pgfpathlineto{\pgfqpoint{4.314207in}{0.604405in}}%
\pgfpathlineto{\pgfqpoint{4.314763in}{0.602416in}}%
\pgfpathlineto{\pgfqpoint{4.315319in}{0.600448in}}%
\pgfpathlineto{\pgfqpoint{4.317543in}{0.660543in}}%
\pgfpathlineto{\pgfqpoint{4.318098in}{0.607387in}}%
\pgfpathlineto{\pgfqpoint{4.318654in}{0.629154in}}%
\pgfpathlineto{\pgfqpoint{4.320322in}{0.608441in}}%
\pgfpathlineto{\pgfqpoint{4.323101in}{0.600226in}}%
\pgfpathlineto{\pgfqpoint{4.324213in}{0.604712in}}%
\pgfpathlineto{\pgfqpoint{4.324769in}{0.603919in}}%
\pgfpathlineto{\pgfqpoint{4.326436in}{0.600847in}}%
\pgfpathlineto{\pgfqpoint{4.327548in}{0.603710in}}%
\pgfpathlineto{\pgfqpoint{4.329771in}{0.600586in}}%
\pgfpathlineto{\pgfqpoint{4.331439in}{0.604414in}}%
\pgfpathlineto{\pgfqpoint{4.333107in}{0.603217in}}%
\pgfpathlineto{\pgfqpoint{4.334774in}{0.607286in}}%
\pgfpathlineto{\pgfqpoint{4.335330in}{0.601795in}}%
\pgfpathlineto{\pgfqpoint{4.335886in}{0.601990in}}%
\pgfpathlineto{\pgfqpoint{4.336442in}{0.606985in}}%
\pgfpathlineto{\pgfqpoint{4.336998in}{0.605418in}}%
\pgfpathlineto{\pgfqpoint{4.337554in}{0.605740in}}%
\pgfpathlineto{\pgfqpoint{4.338109in}{0.603148in}}%
\pgfpathlineto{\pgfqpoint{4.339777in}{0.607003in}}%
\pgfpathlineto{\pgfqpoint{4.340333in}{0.601789in}}%
\pgfpathlineto{\pgfqpoint{4.340889in}{0.604097in}}%
\pgfpathlineto{\pgfqpoint{4.342000in}{0.608200in}}%
\pgfpathlineto{\pgfqpoint{4.343112in}{0.601388in}}%
\pgfpathlineto{\pgfqpoint{4.343668in}{0.608343in}}%
\pgfpathlineto{\pgfqpoint{4.344224in}{0.607587in}}%
\pgfpathlineto{\pgfqpoint{4.344780in}{0.606268in}}%
\pgfpathlineto{\pgfqpoint{4.346447in}{0.609802in}}%
\pgfpathlineto{\pgfqpoint{4.348115in}{0.603451in}}%
\pgfpathlineto{\pgfqpoint{4.349783in}{0.608236in}}%
\pgfpathlineto{\pgfqpoint{4.350338in}{0.613792in}}%
\pgfpathlineto{\pgfqpoint{4.351450in}{0.601406in}}%
\pgfpathlineto{\pgfqpoint{4.353118in}{0.608610in}}%
\pgfpathlineto{\pgfqpoint{4.353674in}{0.600635in}}%
\pgfpathlineto{\pgfqpoint{4.354229in}{0.609676in}}%
\pgfpathlineto{\pgfqpoint{4.354785in}{0.608012in}}%
\pgfpathlineto{\pgfqpoint{4.355341in}{0.601349in}}%
\pgfpathlineto{\pgfqpoint{4.355897in}{0.603758in}}%
\pgfpathlineto{\pgfqpoint{4.356453in}{0.605690in}}%
\pgfpathlineto{\pgfqpoint{4.357009in}{0.612402in}}%
\pgfpathlineto{\pgfqpoint{4.357565in}{0.603026in}}%
\pgfpathlineto{\pgfqpoint{4.358120in}{0.611767in}}%
\pgfpathlineto{\pgfqpoint{4.359788in}{0.606742in}}%
\pgfpathlineto{\pgfqpoint{4.360344in}{0.615189in}}%
\pgfpathlineto{\pgfqpoint{4.360900in}{0.602092in}}%
\pgfpathlineto{\pgfqpoint{4.361456in}{0.606492in}}%
\pgfpathlineto{\pgfqpoint{4.362567in}{0.602803in}}%
\pgfpathlineto{\pgfqpoint{4.363679in}{0.607658in}}%
\pgfpathlineto{\pgfqpoint{4.364235in}{0.607211in}}%
\pgfpathlineto{\pgfqpoint{4.364791in}{0.608357in}}%
\pgfpathlineto{\pgfqpoint{4.365902in}{0.606629in}}%
\pgfpathlineto{\pgfqpoint{4.366458in}{0.607773in}}%
\pgfpathlineto{\pgfqpoint{4.367570in}{0.605057in}}%
\pgfpathlineto{\pgfqpoint{4.368126in}{0.605965in}}%
\pgfpathlineto{\pgfqpoint{4.369238in}{0.601114in}}%
\pgfpathlineto{\pgfqpoint{4.369794in}{0.610929in}}%
\pgfpathlineto{\pgfqpoint{4.370349in}{0.603692in}}%
\pgfpathlineto{\pgfqpoint{4.372017in}{0.607690in}}%
\pgfpathlineto{\pgfqpoint{4.373685in}{0.602988in}}%
\pgfpathlineto{\pgfqpoint{4.374240in}{0.607172in}}%
\pgfpathlineto{\pgfqpoint{4.374796in}{0.606816in}}%
\pgfpathlineto{\pgfqpoint{4.375352in}{0.603346in}}%
\pgfpathlineto{\pgfqpoint{4.375908in}{0.605272in}}%
\pgfpathlineto{\pgfqpoint{4.376464in}{0.604170in}}%
\pgfpathlineto{\pgfqpoint{4.377020in}{0.606583in}}%
\pgfpathlineto{\pgfqpoint{4.377576in}{0.601693in}}%
\pgfpathlineto{\pgfqpoint{4.378131in}{0.605194in}}%
\pgfpathlineto{\pgfqpoint{4.378687in}{0.602405in}}%
\pgfpathlineto{\pgfqpoint{4.380355in}{0.610485in}}%
\pgfpathlineto{\pgfqpoint{4.382578in}{0.602217in}}%
\pgfpathlineto{\pgfqpoint{4.384246in}{0.613416in}}%
\pgfpathlineto{\pgfqpoint{4.384802in}{0.610526in}}%
\pgfpathlineto{\pgfqpoint{4.385358in}{0.600718in}}%
\pgfpathlineto{\pgfqpoint{4.385913in}{0.606764in}}%
\pgfpathlineto{\pgfqpoint{4.386469in}{0.604421in}}%
\pgfpathlineto{\pgfqpoint{4.387025in}{0.606473in}}%
\pgfpathlineto{\pgfqpoint{4.388693in}{0.613437in}}%
\pgfpathlineto{\pgfqpoint{4.390360in}{0.601005in}}%
\pgfpathlineto{\pgfqpoint{4.391472in}{0.602528in}}%
\pgfpathlineto{\pgfqpoint{4.393140in}{0.620580in}}%
\pgfpathlineto{\pgfqpoint{4.393696in}{0.604528in}}%
\pgfpathlineto{\pgfqpoint{4.394251in}{0.612116in}}%
\pgfpathlineto{\pgfqpoint{4.395363in}{0.612421in}}%
\pgfpathlineto{\pgfqpoint{4.395919in}{0.614307in}}%
\pgfpathlineto{\pgfqpoint{4.396475in}{0.624107in}}%
\pgfpathlineto{\pgfqpoint{4.397031in}{0.603516in}}%
\pgfpathlineto{\pgfqpoint{4.397587in}{0.609743in}}%
\pgfpathlineto{\pgfqpoint{4.398142in}{0.609937in}}%
\pgfpathlineto{\pgfqpoint{4.399254in}{0.619430in}}%
\pgfpathlineto{\pgfqpoint{4.399810in}{0.617836in}}%
\pgfpathlineto{\pgfqpoint{4.400366in}{0.601310in}}%
\pgfpathlineto{\pgfqpoint{4.400922in}{0.613060in}}%
\pgfpathlineto{\pgfqpoint{4.402033in}{0.618671in}}%
\pgfpathlineto{\pgfqpoint{4.402589in}{0.612452in}}%
\pgfpathlineto{\pgfqpoint{4.403145in}{0.612707in}}%
\pgfpathlineto{\pgfqpoint{4.403701in}{0.626635in}}%
\pgfpathlineto{\pgfqpoint{4.404257in}{0.617814in}}%
\pgfpathlineto{\pgfqpoint{4.405925in}{0.600270in}}%
\pgfpathlineto{\pgfqpoint{4.406480in}{0.608910in}}%
\pgfpathlineto{\pgfqpoint{4.407036in}{0.607888in}}%
\pgfpathlineto{\pgfqpoint{4.408148in}{0.625405in}}%
\pgfpathlineto{\pgfqpoint{4.408704in}{0.610117in}}%
\pgfpathlineto{\pgfqpoint{4.409260in}{0.613253in}}%
\pgfpathlineto{\pgfqpoint{4.409816in}{0.618335in}}%
\pgfpathlineto{\pgfqpoint{4.410927in}{0.602177in}}%
\pgfpathlineto{\pgfqpoint{4.412595in}{0.615423in}}%
\pgfpathlineto{\pgfqpoint{4.413151in}{0.604943in}}%
\pgfpathlineto{\pgfqpoint{4.413707in}{0.620835in}}%
\pgfpathlineto{\pgfqpoint{4.414262in}{0.619014in}}%
\pgfpathlineto{\pgfqpoint{4.415374in}{0.606910in}}%
\pgfpathlineto{\pgfqpoint{4.415930in}{0.624118in}}%
\pgfpathlineto{\pgfqpoint{4.416486in}{0.613399in}}%
\pgfpathlineto{\pgfqpoint{4.417042in}{0.620596in}}%
\pgfpathlineto{\pgfqpoint{4.417598in}{0.617581in}}%
\pgfpathlineto{\pgfqpoint{4.418153in}{0.619206in}}%
\pgfpathlineto{\pgfqpoint{4.418709in}{0.604223in}}%
\pgfpathlineto{\pgfqpoint{4.419265in}{0.616589in}}%
\pgfpathlineto{\pgfqpoint{4.420377in}{0.607218in}}%
\pgfpathlineto{\pgfqpoint{4.422044in}{0.617188in}}%
\pgfpathlineto{\pgfqpoint{4.422600in}{0.603907in}}%
\pgfpathlineto{\pgfqpoint{4.423156in}{0.612064in}}%
\pgfpathlineto{\pgfqpoint{4.423712in}{0.612191in}}%
\pgfpathlineto{\pgfqpoint{4.424268in}{0.615234in}}%
\pgfpathlineto{\pgfqpoint{4.425936in}{0.602352in}}%
\pgfpathlineto{\pgfqpoint{4.427047in}{0.616863in}}%
\pgfpathlineto{\pgfqpoint{4.428715in}{0.600343in}}%
\pgfpathlineto{\pgfqpoint{4.429827in}{0.609714in}}%
\pgfpathlineto{\pgfqpoint{4.430938in}{0.602534in}}%
\pgfpathlineto{\pgfqpoint{4.432606in}{0.610965in}}%
\pgfpathlineto{\pgfqpoint{4.433162in}{0.611146in}}%
\pgfpathlineto{\pgfqpoint{4.435941in}{0.600666in}}%
\pgfpathlineto{\pgfqpoint{4.436497in}{0.615863in}}%
\pgfpathlineto{\pgfqpoint{4.437053in}{0.606572in}}%
\pgfpathlineto{\pgfqpoint{4.437609in}{0.611214in}}%
\pgfpathlineto{\pgfqpoint{4.438164in}{0.603644in}}%
\pgfpathlineto{\pgfqpoint{4.438720in}{0.606384in}}%
\pgfpathlineto{\pgfqpoint{4.440944in}{0.615747in}}%
\pgfpathlineto{\pgfqpoint{4.441500in}{0.601512in}}%
\pgfpathlineto{\pgfqpoint{4.442055in}{0.604999in}}%
\pgfpathlineto{\pgfqpoint{4.443167in}{0.620373in}}%
\pgfpathlineto{\pgfqpoint{4.443723in}{0.613103in}}%
\pgfpathlineto{\pgfqpoint{4.444279in}{0.605584in}}%
\pgfpathlineto{\pgfqpoint{4.444835in}{0.613576in}}%
\pgfpathlineto{\pgfqpoint{4.445391in}{0.606143in}}%
\pgfpathlineto{\pgfqpoint{4.445947in}{0.606946in}}%
\pgfpathlineto{\pgfqpoint{4.447058in}{0.616739in}}%
\pgfpathlineto{\pgfqpoint{4.447614in}{0.615300in}}%
\pgfpathlineto{\pgfqpoint{4.449282in}{0.606855in}}%
\pgfpathlineto{\pgfqpoint{4.450393in}{0.610995in}}%
\pgfpathlineto{\pgfqpoint{4.450949in}{0.607539in}}%
\pgfpathlineto{\pgfqpoint{4.451505in}{0.621097in}}%
\pgfpathlineto{\pgfqpoint{4.452061in}{0.606486in}}%
\pgfpathlineto{\pgfqpoint{4.452617in}{0.619393in}}%
\pgfpathlineto{\pgfqpoint{4.453173in}{0.612637in}}%
\pgfpathlineto{\pgfqpoint{4.453729in}{0.616429in}}%
\pgfpathlineto{\pgfqpoint{4.454284in}{0.630663in}}%
\pgfpathlineto{\pgfqpoint{4.454840in}{0.630436in}}%
\pgfpathlineto{\pgfqpoint{4.455952in}{0.612116in}}%
\pgfpathlineto{\pgfqpoint{4.456508in}{0.628947in}}%
\pgfpathlineto{\pgfqpoint{4.458175in}{0.607269in}}%
\pgfpathlineto{\pgfqpoint{4.458731in}{0.640135in}}%
\pgfpathlineto{\pgfqpoint{4.459287in}{0.614173in}}%
\pgfpathlineto{\pgfqpoint{4.459843in}{0.611933in}}%
\pgfpathlineto{\pgfqpoint{4.460399in}{0.613863in}}%
\pgfpathlineto{\pgfqpoint{4.460955in}{0.629054in}}%
\pgfpathlineto{\pgfqpoint{4.461511in}{0.617726in}}%
\pgfpathlineto{\pgfqpoint{4.462066in}{0.602601in}}%
\pgfpathlineto{\pgfqpoint{4.462622in}{0.619975in}}%
\pgfpathlineto{\pgfqpoint{4.463178in}{0.611755in}}%
\pgfpathlineto{\pgfqpoint{4.464846in}{0.616509in}}%
\pgfpathlineto{\pgfqpoint{4.465402in}{0.631787in}}%
\pgfpathlineto{\pgfqpoint{4.466513in}{0.606537in}}%
\pgfpathlineto{\pgfqpoint{4.467069in}{0.620123in}}%
\pgfpathlineto{\pgfqpoint{4.467625in}{0.616525in}}%
\pgfpathlineto{\pgfqpoint{4.468737in}{0.607170in}}%
\pgfpathlineto{\pgfqpoint{4.469849in}{0.615743in}}%
\pgfpathlineto{\pgfqpoint{4.470404in}{0.609213in}}%
\pgfpathlineto{\pgfqpoint{4.470960in}{0.618966in}}%
\pgfpathlineto{\pgfqpoint{4.472628in}{0.602497in}}%
\pgfpathlineto{\pgfqpoint{4.473184in}{0.621221in}}%
\pgfpathlineto{\pgfqpoint{4.473740in}{0.610670in}}%
\pgfpathlineto{\pgfqpoint{4.474295in}{0.620920in}}%
\pgfpathlineto{\pgfqpoint{4.474851in}{0.605055in}}%
\pgfpathlineto{\pgfqpoint{4.475407in}{0.628164in}}%
\pgfpathlineto{\pgfqpoint{4.475963in}{0.605762in}}%
\pgfpathlineto{\pgfqpoint{4.476519in}{0.618312in}}%
\pgfpathlineto{\pgfqpoint{4.477075in}{0.604907in}}%
\pgfpathlineto{\pgfqpoint{4.477631in}{0.606636in}}%
\pgfpathlineto{\pgfqpoint{4.479298in}{0.613551in}}%
\pgfpathlineto{\pgfqpoint{4.479854in}{0.605566in}}%
\pgfpathlineto{\pgfqpoint{4.480410in}{0.611764in}}%
\pgfpathlineto{\pgfqpoint{4.480966in}{0.612732in}}%
\pgfpathlineto{\pgfqpoint{4.481522in}{0.618062in}}%
\pgfpathlineto{\pgfqpoint{4.483189in}{0.600435in}}%
\pgfpathlineto{\pgfqpoint{4.483745in}{0.615465in}}%
\pgfpathlineto{\pgfqpoint{4.484301in}{0.609228in}}%
\pgfpathlineto{\pgfqpoint{4.485413in}{0.606259in}}%
\pgfpathlineto{\pgfqpoint{4.486524in}{0.611908in}}%
\pgfpathlineto{\pgfqpoint{4.488192in}{0.605908in}}%
\pgfpathlineto{\pgfqpoint{4.488748in}{0.608184in}}%
\pgfpathlineto{\pgfqpoint{4.489304in}{0.603610in}}%
\pgfpathlineto{\pgfqpoint{4.489860in}{0.608942in}}%
\pgfpathlineto{\pgfqpoint{4.490415in}{0.605008in}}%
\pgfpathlineto{\pgfqpoint{4.490971in}{0.607942in}}%
\pgfpathlineto{\pgfqpoint{4.491527in}{0.605273in}}%
\pgfpathlineto{\pgfqpoint{4.492083in}{0.606334in}}%
\pgfpathlineto{\pgfqpoint{4.493195in}{0.600800in}}%
\pgfpathlineto{\pgfqpoint{4.494306in}{0.615442in}}%
\pgfpathlineto{\pgfqpoint{4.495418in}{0.602294in}}%
\pgfpathlineto{\pgfqpoint{4.497086in}{0.609646in}}%
\pgfpathlineto{\pgfqpoint{4.497642in}{0.602242in}}%
\pgfpathlineto{\pgfqpoint{4.498197in}{0.604628in}}%
\pgfpathlineto{\pgfqpoint{4.498753in}{0.605648in}}%
\pgfpathlineto{\pgfqpoint{4.500421in}{0.619831in}}%
\pgfpathlineto{\pgfqpoint{4.502089in}{0.601535in}}%
\pgfpathlineto{\pgfqpoint{4.503756in}{0.612808in}}%
\pgfpathlineto{\pgfqpoint{4.504312in}{0.610513in}}%
\pgfpathlineto{\pgfqpoint{4.504868in}{0.613195in}}%
\pgfpathlineto{\pgfqpoint{4.506535in}{0.606649in}}%
\pgfpathlineto{\pgfqpoint{4.508203in}{0.602814in}}%
\pgfpathlineto{\pgfqpoint{4.508759in}{0.629420in}}%
\pgfpathlineto{\pgfqpoint{4.509315in}{0.602240in}}%
\pgfpathlineto{\pgfqpoint{4.509871in}{0.612177in}}%
\pgfpathlineto{\pgfqpoint{4.510426in}{0.604899in}}%
\pgfpathlineto{\pgfqpoint{4.511538in}{0.618655in}}%
\pgfpathlineto{\pgfqpoint{4.513206in}{0.606964in}}%
\pgfpathlineto{\pgfqpoint{4.514873in}{0.622021in}}%
\pgfpathlineto{\pgfqpoint{4.515429in}{0.610259in}}%
\pgfpathlineto{\pgfqpoint{4.515985in}{0.621959in}}%
\pgfpathlineto{\pgfqpoint{4.516541in}{0.623044in}}%
\pgfpathlineto{\pgfqpoint{4.518208in}{0.604885in}}%
\pgfpathlineto{\pgfqpoint{4.518764in}{0.619841in}}%
\pgfpathlineto{\pgfqpoint{4.519320in}{0.603292in}}%
\pgfpathlineto{\pgfqpoint{4.519876in}{0.616440in}}%
\pgfpathlineto{\pgfqpoint{4.520432in}{0.621990in}}%
\pgfpathlineto{\pgfqpoint{4.522100in}{0.606466in}}%
\pgfpathlineto{\pgfqpoint{4.523211in}{0.615723in}}%
\pgfpathlineto{\pgfqpoint{4.523767in}{0.606554in}}%
\pgfpathlineto{\pgfqpoint{4.524323in}{0.615612in}}%
\pgfpathlineto{\pgfqpoint{4.524879in}{0.618624in}}%
\pgfpathlineto{\pgfqpoint{4.525435in}{0.605006in}}%
\pgfpathlineto{\pgfqpoint{4.525991in}{0.605742in}}%
\pgfpathlineto{\pgfqpoint{4.527102in}{0.611432in}}%
\pgfpathlineto{\pgfqpoint{4.528214in}{0.600732in}}%
\pgfpathlineto{\pgfqpoint{4.529326in}{0.612274in}}%
\pgfpathlineto{\pgfqpoint{4.529882in}{0.602733in}}%
\pgfpathlineto{\pgfqpoint{4.530437in}{0.604001in}}%
\pgfpathlineto{\pgfqpoint{4.532105in}{0.612611in}}%
\pgfpathlineto{\pgfqpoint{4.532661in}{0.618129in}}%
\pgfpathlineto{\pgfqpoint{4.533773in}{0.607839in}}%
\pgfpathlineto{\pgfqpoint{4.534328in}{0.611388in}}%
\pgfpathlineto{\pgfqpoint{4.534884in}{0.612395in}}%
\pgfpathlineto{\pgfqpoint{4.535996in}{0.604880in}}%
\pgfpathlineto{\pgfqpoint{4.536552in}{0.606766in}}%
\pgfpathlineto{\pgfqpoint{4.537108in}{0.610557in}}%
\pgfpathlineto{\pgfqpoint{4.537664in}{0.608401in}}%
\pgfpathlineto{\pgfqpoint{4.538220in}{0.603229in}}%
\pgfpathlineto{\pgfqpoint{4.538775in}{0.612718in}}%
\pgfpathlineto{\pgfqpoint{4.539331in}{0.611260in}}%
\pgfpathlineto{\pgfqpoint{4.539887in}{0.603350in}}%
\pgfpathlineto{\pgfqpoint{4.540443in}{0.605414in}}%
\pgfpathlineto{\pgfqpoint{4.541555in}{0.605617in}}%
\pgfpathlineto{\pgfqpoint{4.542111in}{0.608899in}}%
\pgfpathlineto{\pgfqpoint{4.542666in}{0.601082in}}%
\pgfpathlineto{\pgfqpoint{4.543222in}{0.610972in}}%
\pgfpathlineto{\pgfqpoint{4.543778in}{0.603517in}}%
\pgfpathlineto{\pgfqpoint{4.545446in}{0.610960in}}%
\pgfpathlineto{\pgfqpoint{4.546002in}{0.600857in}}%
\pgfpathlineto{\pgfqpoint{4.546557in}{0.609323in}}%
\pgfpathlineto{\pgfqpoint{4.547669in}{0.604354in}}%
\pgfpathlineto{\pgfqpoint{4.548225in}{0.608061in}}%
\pgfpathlineto{\pgfqpoint{4.548781in}{0.604964in}}%
\pgfpathlineto{\pgfqpoint{4.549893in}{0.600196in}}%
\pgfpathlineto{\pgfqpoint{4.550448in}{0.602894in}}%
\pgfpathlineto{\pgfqpoint{4.551004in}{0.601676in}}%
\pgfpathlineto{\pgfqpoint{4.551560in}{0.610972in}}%
\pgfpathlineto{\pgfqpoint{4.552116in}{0.601820in}}%
\pgfpathlineto{\pgfqpoint{4.553228in}{0.609078in}}%
\pgfpathlineto{\pgfqpoint{4.554895in}{0.601332in}}%
\pgfpathlineto{\pgfqpoint{4.555451in}{0.603597in}}%
\pgfpathlineto{\pgfqpoint{4.556007in}{0.602577in}}%
\pgfpathlineto{\pgfqpoint{4.557675in}{0.612740in}}%
\pgfpathlineto{\pgfqpoint{4.558231in}{0.602507in}}%
\pgfpathlineto{\pgfqpoint{4.558786in}{0.608196in}}%
\pgfpathlineto{\pgfqpoint{4.559342in}{0.609835in}}%
\pgfpathlineto{\pgfqpoint{4.559898in}{0.601359in}}%
\pgfpathlineto{\pgfqpoint{4.560454in}{0.607504in}}%
\pgfpathlineto{\pgfqpoint{4.561010in}{0.607893in}}%
\pgfpathlineto{\pgfqpoint{4.561566in}{0.601138in}}%
\pgfpathlineto{\pgfqpoint{4.562122in}{0.606091in}}%
\pgfpathlineto{\pgfqpoint{4.563233in}{0.612154in}}%
\pgfpathlineto{\pgfqpoint{4.564901in}{0.604115in}}%
\pgfpathlineto{\pgfqpoint{4.565457in}{0.603210in}}%
\pgfpathlineto{\pgfqpoint{4.566013in}{0.610098in}}%
\pgfpathlineto{\pgfqpoint{4.566568in}{0.608113in}}%
\pgfpathlineto{\pgfqpoint{4.567124in}{0.607815in}}%
\pgfpathlineto{\pgfqpoint{4.567680in}{0.602852in}}%
\pgfpathlineto{\pgfqpoint{4.568236in}{0.604397in}}%
\pgfpathlineto{\pgfqpoint{4.568792in}{0.619269in}}%
\pgfpathlineto{\pgfqpoint{4.570459in}{0.603037in}}%
\pgfpathlineto{\pgfqpoint{4.571015in}{0.608673in}}%
\pgfpathlineto{\pgfqpoint{4.571571in}{0.605233in}}%
\pgfpathlineto{\pgfqpoint{4.572683in}{0.609278in}}%
\pgfpathlineto{\pgfqpoint{4.573239in}{0.602960in}}%
\pgfpathlineto{\pgfqpoint{4.573795in}{0.625191in}}%
\pgfpathlineto{\pgfqpoint{4.574350in}{0.604142in}}%
\pgfpathlineto{\pgfqpoint{4.574906in}{0.610010in}}%
\pgfpathlineto{\pgfqpoint{4.575462in}{0.606118in}}%
\pgfpathlineto{\pgfqpoint{4.576574in}{0.612303in}}%
\pgfpathlineto{\pgfqpoint{4.577130in}{0.605953in}}%
\pgfpathlineto{\pgfqpoint{4.577686in}{0.613783in}}%
\pgfpathlineto{\pgfqpoint{4.578242in}{0.613038in}}%
\pgfpathlineto{\pgfqpoint{4.579353in}{0.603970in}}%
\pgfpathlineto{\pgfqpoint{4.579909in}{0.604084in}}%
\pgfpathlineto{\pgfqpoint{4.580465in}{0.615050in}}%
\pgfpathlineto{\pgfqpoint{4.581021in}{0.605682in}}%
\pgfpathlineto{\pgfqpoint{4.581577in}{0.608964in}}%
\pgfpathlineto{\pgfqpoint{4.582133in}{0.607414in}}%
\pgfpathlineto{\pgfqpoint{4.583244in}{0.602708in}}%
\pgfpathlineto{\pgfqpoint{4.583800in}{0.602995in}}%
\pgfpathlineto{\pgfqpoint{4.584356in}{0.609783in}}%
\pgfpathlineto{\pgfqpoint{4.584912in}{0.608324in}}%
\pgfpathlineto{\pgfqpoint{4.586579in}{0.602570in}}%
\pgfpathlineto{\pgfqpoint{4.587135in}{0.603316in}}%
\pgfpathlineto{\pgfqpoint{4.587691in}{0.601972in}}%
\pgfpathlineto{\pgfqpoint{4.589359in}{0.608113in}}%
\pgfpathlineto{\pgfqpoint{4.589915in}{0.602130in}}%
\pgfpathlineto{\pgfqpoint{4.590470in}{0.611813in}}%
\pgfpathlineto{\pgfqpoint{4.591026in}{0.606783in}}%
\pgfpathlineto{\pgfqpoint{4.591582in}{0.609700in}}%
\pgfpathlineto{\pgfqpoint{4.592138in}{0.604420in}}%
\pgfpathlineto{\pgfqpoint{4.592694in}{0.612021in}}%
\pgfpathlineto{\pgfqpoint{4.593806in}{0.601003in}}%
\pgfpathlineto{\pgfqpoint{4.594361in}{0.602257in}}%
\pgfpathlineto{\pgfqpoint{4.594917in}{0.602004in}}%
\pgfpathlineto{\pgfqpoint{4.596585in}{0.612752in}}%
\pgfpathlineto{\pgfqpoint{4.597141in}{0.602443in}}%
\pgfpathlineto{\pgfqpoint{4.597697in}{0.604679in}}%
\pgfpathlineto{\pgfqpoint{4.599364in}{0.605221in}}%
\pgfpathlineto{\pgfqpoint{4.599920in}{0.602103in}}%
\pgfpathlineto{\pgfqpoint{4.600476in}{0.604187in}}%
\pgfpathlineto{\pgfqpoint{4.601032in}{0.603676in}}%
\pgfpathlineto{\pgfqpoint{4.601588in}{0.604318in}}%
\pgfpathlineto{\pgfqpoint{4.602144in}{0.604375in}}%
\pgfpathlineto{\pgfqpoint{4.602699in}{0.606073in}}%
\pgfpathlineto{\pgfqpoint{4.603255in}{0.604988in}}%
\pgfpathlineto{\pgfqpoint{4.603811in}{0.602750in}}%
\pgfpathlineto{\pgfqpoint{4.604367in}{0.603277in}}%
\pgfpathlineto{\pgfqpoint{4.604923in}{0.606007in}}%
\pgfpathlineto{\pgfqpoint{4.605479in}{0.605177in}}%
\pgfpathlineto{\pgfqpoint{4.607146in}{0.601678in}}%
\pgfpathlineto{\pgfqpoint{4.607702in}{0.602934in}}%
\pgfpathlineto{\pgfqpoint{4.608258in}{0.601952in}}%
\pgfpathlineto{\pgfqpoint{4.608814in}{0.600995in}}%
\pgfpathlineto{\pgfqpoint{4.609370in}{0.601406in}}%
\pgfpathlineto{\pgfqpoint{4.609926in}{0.607173in}}%
\pgfpathlineto{\pgfqpoint{4.610481in}{0.602718in}}%
\pgfpathlineto{\pgfqpoint{4.611037in}{0.604784in}}%
\pgfpathlineto{\pgfqpoint{4.612705in}{0.601156in}}%
\pgfpathlineto{\pgfqpoint{4.613817in}{0.605140in}}%
\pgfpathlineto{\pgfqpoint{4.614373in}{0.604836in}}%
\pgfpathlineto{\pgfqpoint{4.615484in}{0.600268in}}%
\pgfpathlineto{\pgfqpoint{4.616040in}{0.607390in}}%
\pgfpathlineto{\pgfqpoint{4.616596in}{0.606813in}}%
\pgfpathlineto{\pgfqpoint{4.617708in}{0.601681in}}%
\pgfpathlineto{\pgfqpoint{4.618264in}{0.601994in}}%
\pgfpathlineto{\pgfqpoint{4.618819in}{0.602447in}}%
\pgfpathlineto{\pgfqpoint{4.620487in}{0.607627in}}%
\pgfpathlineto{\pgfqpoint{4.621599in}{0.601873in}}%
\pgfpathlineto{\pgfqpoint{4.622710in}{0.606213in}}%
\pgfpathlineto{\pgfqpoint{4.623266in}{0.605257in}}%
\pgfpathlineto{\pgfqpoint{4.623822in}{0.604274in}}%
\pgfpathlineto{\pgfqpoint{4.624378in}{0.605704in}}%
\pgfpathlineto{\pgfqpoint{4.625490in}{0.603775in}}%
\pgfpathlineto{\pgfqpoint{4.626601in}{0.607252in}}%
\pgfpathlineto{\pgfqpoint{4.627713in}{0.603489in}}%
\pgfpathlineto{\pgfqpoint{4.628269in}{0.604555in}}%
\pgfpathlineto{\pgfqpoint{4.629381in}{0.605120in}}%
\pgfpathlineto{\pgfqpoint{4.630492in}{0.602659in}}%
\pgfpathlineto{\pgfqpoint{4.631048in}{0.609623in}}%
\pgfpathlineto{\pgfqpoint{4.631604in}{0.604034in}}%
\pgfpathlineto{\pgfqpoint{4.632160in}{0.605620in}}%
\pgfpathlineto{\pgfqpoint{4.632716in}{0.605166in}}%
\pgfpathlineto{\pgfqpoint{4.633272in}{0.601818in}}%
\pgfpathlineto{\pgfqpoint{4.633828in}{0.606576in}}%
\pgfpathlineto{\pgfqpoint{4.634384in}{0.603112in}}%
\pgfpathlineto{\pgfqpoint{4.634939in}{0.603823in}}%
\pgfpathlineto{\pgfqpoint{4.635495in}{0.612712in}}%
\pgfpathlineto{\pgfqpoint{4.636051in}{0.602078in}}%
\pgfpathlineto{\pgfqpoint{4.636607in}{0.611386in}}%
\pgfpathlineto{\pgfqpoint{4.638275in}{0.601615in}}%
\pgfpathlineto{\pgfqpoint{4.638830in}{0.602770in}}%
\pgfpathlineto{\pgfqpoint{4.639386in}{0.601279in}}%
\pgfpathlineto{\pgfqpoint{4.639942in}{0.610094in}}%
\pgfpathlineto{\pgfqpoint{4.640498in}{0.605207in}}%
\pgfpathlineto{\pgfqpoint{4.641054in}{0.602717in}}%
\pgfpathlineto{\pgfqpoint{4.641610in}{0.604899in}}%
\pgfpathlineto{\pgfqpoint{4.643277in}{0.602740in}}%
\pgfpathlineto{\pgfqpoint{4.643833in}{0.605043in}}%
\pgfpathlineto{\pgfqpoint{4.644389in}{0.601496in}}%
\pgfpathlineto{\pgfqpoint{4.644945in}{0.602436in}}%
\pgfpathlineto{\pgfqpoint{4.646612in}{0.600981in}}%
\pgfpathlineto{\pgfqpoint{4.647724in}{0.604977in}}%
\pgfpathlineto{\pgfqpoint{4.649392in}{0.602973in}}%
\pgfpathlineto{\pgfqpoint{4.649948in}{0.604886in}}%
\pgfpathlineto{\pgfqpoint{4.650503in}{0.603818in}}%
\pgfpathlineto{\pgfqpoint{4.651059in}{0.600278in}}%
\pgfpathlineto{\pgfqpoint{4.651615in}{0.601171in}}%
\pgfpathlineto{\pgfqpoint{4.652171in}{0.601358in}}%
\pgfpathlineto{\pgfqpoint{4.652727in}{0.606174in}}%
\pgfpathlineto{\pgfqpoint{4.653283in}{0.600674in}}%
\pgfpathlineto{\pgfqpoint{4.653839in}{0.604289in}}%
\pgfpathlineto{\pgfqpoint{4.654395in}{0.603155in}}%
\pgfpathlineto{\pgfqpoint{4.654950in}{0.603968in}}%
\pgfpathlineto{\pgfqpoint{4.655506in}{0.603697in}}%
\pgfpathlineto{\pgfqpoint{4.657174in}{0.601600in}}%
\pgfpathlineto{\pgfqpoint{4.657730in}{0.602442in}}%
\pgfpathlineto{\pgfqpoint{4.658286in}{0.600845in}}%
\pgfpathlineto{\pgfqpoint{4.658841in}{0.604473in}}%
\pgfpathlineto{\pgfqpoint{4.659397in}{0.602149in}}%
\pgfpathlineto{\pgfqpoint{4.661065in}{0.603765in}}%
\pgfpathlineto{\pgfqpoint{4.662732in}{0.601682in}}%
\pgfpathlineto{\pgfqpoint{4.664400in}{0.602296in}}%
\pgfpathlineto{\pgfqpoint{4.664956in}{0.601513in}}%
\pgfpathlineto{\pgfqpoint{4.665512in}{0.606459in}}%
\pgfpathlineto{\pgfqpoint{4.666068in}{0.600921in}}%
\pgfpathlineto{\pgfqpoint{4.666623in}{0.601976in}}%
\pgfpathlineto{\pgfqpoint{4.667179in}{0.605329in}}%
\pgfpathlineto{\pgfqpoint{4.667735in}{0.604377in}}%
\pgfpathlineto{\pgfqpoint{4.668291in}{0.602392in}}%
\pgfpathlineto{\pgfqpoint{4.668847in}{0.603158in}}%
\pgfpathlineto{\pgfqpoint{4.669403in}{0.604223in}}%
\pgfpathlineto{\pgfqpoint{4.669959in}{0.603828in}}%
\pgfpathlineto{\pgfqpoint{4.671070in}{0.601565in}}%
\pgfpathlineto{\pgfqpoint{4.671626in}{0.604175in}}%
\pgfpathlineto{\pgfqpoint{4.672182in}{0.600229in}}%
\pgfpathlineto{\pgfqpoint{4.672738in}{0.600561in}}%
\pgfpathlineto{\pgfqpoint{4.674406in}{0.601815in}}%
\pgfpathlineto{\pgfqpoint{4.674961in}{0.605722in}}%
\pgfpathlineto{\pgfqpoint{4.675517in}{0.603710in}}%
\pgfpathlineto{\pgfqpoint{4.676073in}{0.601263in}}%
\pgfpathlineto{\pgfqpoint{4.676629in}{0.601587in}}%
\pgfpathlineto{\pgfqpoint{4.678852in}{0.604522in}}%
\pgfpathlineto{\pgfqpoint{4.679408in}{0.602166in}}%
\pgfpathlineto{\pgfqpoint{4.679964in}{0.604609in}}%
\pgfpathlineto{\pgfqpoint{4.680520in}{0.603895in}}%
\pgfpathlineto{\pgfqpoint{4.681076in}{0.605365in}}%
\pgfpathlineto{\pgfqpoint{4.682188in}{0.601806in}}%
\pgfpathlineto{\pgfqpoint{4.682743in}{0.602128in}}%
\pgfpathlineto{\pgfqpoint{4.683299in}{0.602965in}}%
\pgfpathlineto{\pgfqpoint{4.684411in}{0.601593in}}%
\pgfpathlineto{\pgfqpoint{4.686079in}{0.606509in}}%
\pgfpathlineto{\pgfqpoint{4.687746in}{0.600150in}}%
\pgfpathlineto{\pgfqpoint{4.689414in}{0.606713in}}%
\pgfpathlineto{\pgfqpoint{4.690526in}{0.600377in}}%
\pgfpathlineto{\pgfqpoint{4.691637in}{0.604991in}}%
\pgfpathlineto{\pgfqpoint{4.692193in}{0.600917in}}%
\pgfpathlineto{\pgfqpoint{4.692749in}{0.603990in}}%
\pgfpathlineto{\pgfqpoint{4.693305in}{0.605915in}}%
\pgfpathlineto{\pgfqpoint{4.694417in}{0.600854in}}%
\pgfpathlineto{\pgfqpoint{4.696084in}{0.607042in}}%
\pgfpathlineto{\pgfqpoint{4.697196in}{0.602374in}}%
\pgfpathlineto{\pgfqpoint{4.697752in}{0.606095in}}%
\pgfpathlineto{\pgfqpoint{4.698308in}{0.605216in}}%
\pgfpathlineto{\pgfqpoint{4.698863in}{0.600265in}}%
\pgfpathlineto{\pgfqpoint{4.699419in}{0.605180in}}%
\pgfpathlineto{\pgfqpoint{4.700531in}{0.600631in}}%
\pgfpathlineto{\pgfqpoint{4.702199in}{0.603618in}}%
\pgfpathlineto{\pgfqpoint{4.702754in}{0.602428in}}%
\pgfpathlineto{\pgfqpoint{4.704422in}{0.605431in}}%
\pgfpathlineto{\pgfqpoint{4.706090in}{0.602396in}}%
\pgfpathlineto{\pgfqpoint{4.707757in}{0.606920in}}%
\pgfpathlineto{\pgfqpoint{4.708313in}{0.601281in}}%
\pgfpathlineto{\pgfqpoint{4.708869in}{0.603402in}}%
\pgfpathlineto{\pgfqpoint{4.709425in}{0.602085in}}%
\pgfpathlineto{\pgfqpoint{4.711092in}{0.604670in}}%
\pgfpathlineto{\pgfqpoint{4.712204in}{0.605563in}}%
\pgfpathlineto{\pgfqpoint{4.712760in}{0.603926in}}%
\pgfpathlineto{\pgfqpoint{4.713316in}{0.607124in}}%
\pgfpathlineto{\pgfqpoint{4.714428in}{0.601585in}}%
\pgfpathlineto{\pgfqpoint{4.714983in}{0.602146in}}%
\pgfpathlineto{\pgfqpoint{4.715539in}{0.605026in}}%
\pgfpathlineto{\pgfqpoint{4.716095in}{0.601751in}}%
\pgfpathlineto{\pgfqpoint{4.716651in}{0.603566in}}%
\pgfpathlineto{\pgfqpoint{4.717207in}{0.602110in}}%
\pgfpathlineto{\pgfqpoint{4.717763in}{0.605331in}}%
\pgfpathlineto{\pgfqpoint{4.718319in}{0.601168in}}%
\pgfpathlineto{\pgfqpoint{4.718874in}{0.601856in}}%
\pgfpathlineto{\pgfqpoint{4.719986in}{0.602360in}}%
\pgfpathlineto{\pgfqpoint{4.720542in}{0.600359in}}%
\pgfpathlineto{\pgfqpoint{4.721098in}{0.604243in}}%
\pgfpathlineto{\pgfqpoint{4.721654in}{0.600901in}}%
\pgfpathlineto{\pgfqpoint{4.722765in}{0.605767in}}%
\pgfpathlineto{\pgfqpoint{4.723321in}{0.604315in}}%
\pgfpathlineto{\pgfqpoint{4.725545in}{0.600936in}}%
\pgfpathlineto{\pgfqpoint{4.726657in}{0.605464in}}%
\pgfpathlineto{\pgfqpoint{4.727212in}{0.602634in}}%
\pgfpathlineto{\pgfqpoint{4.727768in}{0.604925in}}%
\pgfpathlineto{\pgfqpoint{4.728880in}{0.602039in}}%
\pgfpathlineto{\pgfqpoint{4.729436in}{0.602322in}}%
\pgfpathlineto{\pgfqpoint{4.729992in}{0.603220in}}%
\pgfpathlineto{\pgfqpoint{4.730548in}{0.602840in}}%
\pgfpathlineto{\pgfqpoint{4.731659in}{0.600572in}}%
\pgfpathlineto{\pgfqpoint{4.732771in}{0.604165in}}%
\pgfpathlineto{\pgfqpoint{4.733327in}{0.600923in}}%
\pgfpathlineto{\pgfqpoint{4.733883in}{0.603099in}}%
\pgfpathlineto{\pgfqpoint{4.734439in}{0.603048in}}%
\pgfpathlineto{\pgfqpoint{4.735550in}{0.600938in}}%
\pgfpathlineto{\pgfqpoint{4.736106in}{0.609580in}}%
\pgfpathlineto{\pgfqpoint{4.736662in}{0.607291in}}%
\pgfpathlineto{\pgfqpoint{4.737774in}{0.602838in}}%
\pgfpathlineto{\pgfqpoint{4.738885in}{0.606245in}}%
\pgfpathlineto{\pgfqpoint{4.740553in}{0.603532in}}%
\pgfpathlineto{\pgfqpoint{4.741109in}{0.606569in}}%
\pgfpathlineto{\pgfqpoint{4.741665in}{0.600311in}}%
\pgfpathlineto{\pgfqpoint{4.742221in}{0.602541in}}%
\pgfpathlineto{\pgfqpoint{4.743888in}{0.608322in}}%
\pgfpathlineto{\pgfqpoint{4.745000in}{0.602483in}}%
\pgfpathlineto{\pgfqpoint{4.745556in}{0.604179in}}%
\pgfpathlineto{\pgfqpoint{4.746112in}{0.605364in}}%
\pgfpathlineto{\pgfqpoint{4.747779in}{0.601007in}}%
\pgfpathlineto{\pgfqpoint{4.748335in}{0.602614in}}%
\pgfpathlineto{\pgfqpoint{4.748891in}{0.610573in}}%
\pgfpathlineto{\pgfqpoint{4.749447in}{0.603511in}}%
\pgfpathlineto{\pgfqpoint{4.750003in}{0.607955in}}%
\pgfpathlineto{\pgfqpoint{4.750559in}{0.607026in}}%
\pgfpathlineto{\pgfqpoint{4.751114in}{0.602547in}}%
\pgfpathlineto{\pgfqpoint{4.751670in}{0.604140in}}%
\pgfpathlineto{\pgfqpoint{4.752782in}{0.602089in}}%
\pgfpathlineto{\pgfqpoint{4.753338in}{0.603182in}}%
\pgfpathlineto{\pgfqpoint{4.755005in}{0.607813in}}%
\pgfpathlineto{\pgfqpoint{4.756117in}{0.602169in}}%
\pgfpathlineto{\pgfqpoint{4.757785in}{0.607355in}}%
\pgfpathlineto{\pgfqpoint{4.758341in}{0.604327in}}%
\pgfpathlineto{\pgfqpoint{4.758896in}{0.605787in}}%
\pgfpathlineto{\pgfqpoint{4.759452in}{0.606418in}}%
\pgfpathlineto{\pgfqpoint{4.760564in}{0.602959in}}%
\pgfpathlineto{\pgfqpoint{4.761120in}{0.603451in}}%
\pgfpathlineto{\pgfqpoint{4.761676in}{0.602013in}}%
\pgfpathlineto{\pgfqpoint{4.762787in}{0.610998in}}%
\pgfpathlineto{\pgfqpoint{4.763343in}{0.605693in}}%
\pgfpathlineto{\pgfqpoint{4.765011in}{0.612036in}}%
\pgfpathlineto{\pgfqpoint{4.765567in}{0.601857in}}%
\pgfpathlineto{\pgfqpoint{4.766123in}{0.604178in}}%
\pgfpathlineto{\pgfqpoint{4.766679in}{0.602810in}}%
\pgfpathlineto{\pgfqpoint{4.767790in}{0.612113in}}%
\pgfpathlineto{\pgfqpoint{4.768346in}{0.603478in}}%
\pgfpathlineto{\pgfqpoint{4.768902in}{0.607271in}}%
\pgfpathlineto{\pgfqpoint{4.770014in}{0.605712in}}%
\pgfpathlineto{\pgfqpoint{4.770570in}{0.610790in}}%
\pgfpathlineto{\pgfqpoint{4.772237in}{0.603402in}}%
\pgfpathlineto{\pgfqpoint{4.772793in}{0.601829in}}%
\pgfpathlineto{\pgfqpoint{4.773905in}{0.603952in}}%
\pgfpathlineto{\pgfqpoint{4.774461in}{0.601442in}}%
\pgfpathlineto{\pgfqpoint{4.775016in}{0.604397in}}%
\pgfpathlineto{\pgfqpoint{4.775572in}{0.602810in}}%
\pgfpathlineto{\pgfqpoint{4.777240in}{0.605202in}}%
\pgfpathlineto{\pgfqpoint{4.777796in}{0.603071in}}%
\pgfpathlineto{\pgfqpoint{4.778352in}{0.603717in}}%
\pgfpathlineto{\pgfqpoint{4.778907in}{0.604046in}}%
\pgfpathlineto{\pgfqpoint{4.779463in}{0.601884in}}%
\pgfpathlineto{\pgfqpoint{4.780019in}{0.606630in}}%
\pgfpathlineto{\pgfqpoint{4.780575in}{0.605929in}}%
\pgfpathlineto{\pgfqpoint{4.781131in}{0.600937in}}%
\pgfpathlineto{\pgfqpoint{4.781687in}{0.601731in}}%
\pgfpathlineto{\pgfqpoint{4.782243in}{0.606118in}}%
\pgfpathlineto{\pgfqpoint{4.782798in}{0.602829in}}%
\pgfpathlineto{\pgfqpoint{4.783354in}{0.602198in}}%
\pgfpathlineto{\pgfqpoint{4.784466in}{0.608201in}}%
\pgfpathlineto{\pgfqpoint{4.785022in}{0.600049in}}%
\pgfpathlineto{\pgfqpoint{4.785578in}{0.602577in}}%
\pgfpathlineto{\pgfqpoint{4.786134in}{0.600940in}}%
\pgfpathlineto{\pgfqpoint{4.787245in}{0.604049in}}%
\pgfpathlineto{\pgfqpoint{4.787801in}{0.602883in}}%
\pgfpathlineto{\pgfqpoint{4.788357in}{0.607090in}}%
\pgfpathlineto{\pgfqpoint{4.788913in}{0.605042in}}%
\pgfpathlineto{\pgfqpoint{4.790025in}{0.603263in}}%
\pgfpathlineto{\pgfqpoint{4.790581in}{0.605001in}}%
\pgfpathlineto{\pgfqpoint{4.791692in}{0.600894in}}%
\pgfpathlineto{\pgfqpoint{4.793360in}{0.604823in}}%
\pgfpathlineto{\pgfqpoint{4.793916in}{0.608501in}}%
\pgfpathlineto{\pgfqpoint{4.794472in}{0.601486in}}%
\pgfpathlineto{\pgfqpoint{4.795027in}{0.602801in}}%
\pgfpathlineto{\pgfqpoint{4.795583in}{0.605829in}}%
\pgfpathlineto{\pgfqpoint{4.796139in}{0.615839in}}%
\pgfpathlineto{\pgfqpoint{4.796695in}{0.602549in}}%
\pgfpathlineto{\pgfqpoint{4.797251in}{0.605716in}}%
\pgfpathlineto{\pgfqpoint{4.797807in}{0.602790in}}%
\pgfpathlineto{\pgfqpoint{4.798363in}{0.605232in}}%
\pgfpathlineto{\pgfqpoint{4.798918in}{0.607508in}}%
\pgfpathlineto{\pgfqpoint{4.800030in}{0.601421in}}%
\pgfpathlineto{\pgfqpoint{4.800586in}{0.602233in}}%
\pgfpathlineto{\pgfqpoint{4.801142in}{0.612102in}}%
\pgfpathlineto{\pgfqpoint{4.802254in}{0.601430in}}%
\pgfpathlineto{\pgfqpoint{4.803921in}{0.607808in}}%
\pgfpathlineto{\pgfqpoint{4.805033in}{0.601803in}}%
\pgfpathlineto{\pgfqpoint{4.806145in}{0.608311in}}%
\pgfpathlineto{\pgfqpoint{4.806701in}{0.606121in}}%
\pgfpathlineto{\pgfqpoint{4.807256in}{0.606979in}}%
\pgfpathlineto{\pgfqpoint{4.807812in}{0.610358in}}%
\pgfpathlineto{\pgfqpoint{4.808368in}{0.600748in}}%
\pgfpathlineto{\pgfqpoint{4.808924in}{0.606674in}}%
\pgfpathlineto{\pgfqpoint{4.809480in}{0.608276in}}%
\pgfpathlineto{\pgfqpoint{4.810036in}{0.608077in}}%
\pgfpathlineto{\pgfqpoint{4.811147in}{0.603077in}}%
\pgfpathlineto{\pgfqpoint{4.811703in}{0.603223in}}%
\pgfpathlineto{\pgfqpoint{4.812815in}{0.608478in}}%
\pgfpathlineto{\pgfqpoint{4.813927in}{0.600558in}}%
\pgfpathlineto{\pgfqpoint{4.814483in}{0.605676in}}%
\pgfpathlineto{\pgfqpoint{4.815038in}{0.602481in}}%
\pgfpathlineto{\pgfqpoint{4.815594in}{0.601091in}}%
\pgfpathlineto{\pgfqpoint{4.816150in}{0.601901in}}%
\pgfpathlineto{\pgfqpoint{4.817262in}{0.601642in}}%
\pgfpathlineto{\pgfqpoint{4.817818in}{0.602474in}}%
\pgfpathlineto{\pgfqpoint{4.818374in}{0.601691in}}%
\pgfpathlineto{\pgfqpoint{4.819485in}{0.601048in}}%
\pgfpathlineto{\pgfqpoint{4.820041in}{0.601896in}}%
\pgfpathlineto{\pgfqpoint{4.820597in}{0.601472in}}%
\pgfpathlineto{\pgfqpoint{4.824488in}{0.601340in}}%
\pgfpathlineto{\pgfqpoint{4.827267in}{0.603770in}}%
\pgfpathlineto{\pgfqpoint{4.827823in}{0.603637in}}%
\pgfpathlineto{\pgfqpoint{4.828379in}{0.606633in}}%
\pgfpathlineto{\pgfqpoint{4.828935in}{0.603908in}}%
\pgfpathlineto{\pgfqpoint{4.829491in}{0.602575in}}%
\pgfpathlineto{\pgfqpoint{4.830603in}{0.607840in}}%
\pgfpathlineto{\pgfqpoint{4.831158in}{0.601620in}}%
\pgfpathlineto{\pgfqpoint{4.831714in}{0.601938in}}%
\pgfpathlineto{\pgfqpoint{4.832826in}{0.608293in}}%
\pgfpathlineto{\pgfqpoint{4.833382in}{0.606729in}}%
\pgfpathlineto{\pgfqpoint{4.833938in}{0.603472in}}%
\pgfpathlineto{\pgfqpoint{4.834494in}{0.607453in}}%
\pgfpathlineto{\pgfqpoint{4.835049in}{0.602368in}}%
\pgfpathlineto{\pgfqpoint{4.835605in}{0.605999in}}%
\pgfpathlineto{\pgfqpoint{4.836717in}{0.602551in}}%
\pgfpathlineto{\pgfqpoint{4.838385in}{0.605776in}}%
\pgfpathlineto{\pgfqpoint{4.838940in}{0.601767in}}%
\pgfpathlineto{\pgfqpoint{4.839496in}{0.606375in}}%
\pgfpathlineto{\pgfqpoint{4.840052in}{0.606005in}}%
\pgfpathlineto{\pgfqpoint{4.840608in}{0.602885in}}%
\pgfpathlineto{\pgfqpoint{4.841164in}{0.604946in}}%
\pgfpathlineto{\pgfqpoint{4.841720in}{0.605653in}}%
\pgfpathlineto{\pgfqpoint{4.842276in}{0.603303in}}%
\pgfpathlineto{\pgfqpoint{4.842832in}{0.603958in}}%
\pgfpathlineto{\pgfqpoint{4.843387in}{0.603677in}}%
\pgfpathlineto{\pgfqpoint{4.843943in}{0.601902in}}%
\pgfpathlineto{\pgfqpoint{4.844499in}{0.603537in}}%
\pgfpathlineto{\pgfqpoint{4.845055in}{0.606543in}}%
\pgfpathlineto{\pgfqpoint{4.846167in}{0.601529in}}%
\pgfpathlineto{\pgfqpoint{4.847834in}{0.605447in}}%
\pgfpathlineto{\pgfqpoint{4.849502in}{0.601188in}}%
\pgfpathlineto{\pgfqpoint{4.850614in}{0.604031in}}%
\pgfpathlineto{\pgfqpoint{4.851169in}{0.610344in}}%
\pgfpathlineto{\pgfqpoint{4.851725in}{0.606787in}}%
\pgfpathlineto{\pgfqpoint{4.852281in}{0.604344in}}%
\pgfpathlineto{\pgfqpoint{4.853949in}{0.612656in}}%
\pgfpathlineto{\pgfqpoint{4.854505in}{0.602955in}}%
\pgfpathlineto{\pgfqpoint{4.855060in}{0.607044in}}%
\pgfpathlineto{\pgfqpoint{4.856728in}{0.608648in}}%
\pgfpathlineto{\pgfqpoint{4.857840in}{0.604860in}}%
\pgfpathlineto{\pgfqpoint{4.858396in}{0.613275in}}%
\pgfpathlineto{\pgfqpoint{4.858952in}{0.608877in}}%
\pgfpathlineto{\pgfqpoint{4.860619in}{0.606509in}}%
\pgfpathlineto{\pgfqpoint{4.861175in}{0.614229in}}%
\pgfpathlineto{\pgfqpoint{4.862843in}{0.600601in}}%
\pgfpathlineto{\pgfqpoint{4.863398in}{0.615292in}}%
\pgfpathlineto{\pgfqpoint{4.863954in}{0.608568in}}%
\pgfpathlineto{\pgfqpoint{4.864510in}{0.607669in}}%
\pgfpathlineto{\pgfqpoint{4.865066in}{0.602953in}}%
\pgfpathlineto{\pgfqpoint{4.865622in}{0.605874in}}%
\pgfpathlineto{\pgfqpoint{4.867289in}{0.603149in}}%
\pgfpathlineto{\pgfqpoint{4.867845in}{0.603072in}}%
\pgfpathlineto{\pgfqpoint{4.868401in}{0.610731in}}%
\pgfpathlineto{\pgfqpoint{4.868957in}{0.609091in}}%
\pgfpathlineto{\pgfqpoint{4.870625in}{0.602472in}}%
\pgfpathlineto{\pgfqpoint{4.871180in}{0.612973in}}%
\pgfpathlineto{\pgfqpoint{4.871736in}{0.608341in}}%
\pgfpathlineto{\pgfqpoint{4.873404in}{0.600644in}}%
\pgfpathlineto{\pgfqpoint{4.873960in}{0.608838in}}%
\pgfpathlineto{\pgfqpoint{4.874516in}{0.603831in}}%
\pgfpathlineto{\pgfqpoint{4.875627in}{0.602175in}}%
\pgfpathlineto{\pgfqpoint{4.876739in}{0.610158in}}%
\pgfpathlineto{\pgfqpoint{4.877295in}{0.606068in}}%
\pgfpathlineto{\pgfqpoint{4.878407in}{0.615764in}}%
\pgfpathlineto{\pgfqpoint{4.878963in}{0.605421in}}%
\pgfpathlineto{\pgfqpoint{4.879518in}{0.606353in}}%
\pgfpathlineto{\pgfqpoint{4.880074in}{0.613150in}}%
\pgfpathlineto{\pgfqpoint{4.880630in}{0.603301in}}%
\pgfpathlineto{\pgfqpoint{4.881186in}{0.610620in}}%
\pgfpathlineto{\pgfqpoint{4.882854in}{0.611487in}}%
\pgfpathlineto{\pgfqpoint{4.883409in}{0.604486in}}%
\pgfpathlineto{\pgfqpoint{4.883965in}{0.605063in}}%
\pgfpathlineto{\pgfqpoint{4.884521in}{0.612922in}}%
\pgfpathlineto{\pgfqpoint{4.885077in}{0.603371in}}%
\pgfpathlineto{\pgfqpoint{4.885633in}{0.613212in}}%
\pgfpathlineto{\pgfqpoint{4.886189in}{0.606599in}}%
\pgfpathlineto{\pgfqpoint{4.886745in}{0.602095in}}%
\pgfpathlineto{\pgfqpoint{4.887300in}{0.612413in}}%
\pgfpathlineto{\pgfqpoint{4.887856in}{0.610347in}}%
\pgfpathlineto{\pgfqpoint{4.889524in}{0.601313in}}%
\pgfpathlineto{\pgfqpoint{4.890080in}{0.607796in}}%
\pgfpathlineto{\pgfqpoint{4.890636in}{0.605332in}}%
\pgfpathlineto{\pgfqpoint{4.891191in}{0.606253in}}%
\pgfpathlineto{\pgfqpoint{4.891747in}{0.610577in}}%
\pgfpathlineto{\pgfqpoint{4.892859in}{0.601525in}}%
\pgfpathlineto{\pgfqpoint{4.893415in}{0.603136in}}%
\pgfpathlineto{\pgfqpoint{4.895638in}{0.608943in}}%
\pgfpathlineto{\pgfqpoint{4.896194in}{0.603511in}}%
\pgfpathlineto{\pgfqpoint{4.896750in}{0.606183in}}%
\pgfpathlineto{\pgfqpoint{4.897306in}{0.606945in}}%
\pgfpathlineto{\pgfqpoint{4.897862in}{0.601670in}}%
\pgfpathlineto{\pgfqpoint{4.898418in}{0.605767in}}%
\pgfpathlineto{\pgfqpoint{4.898974in}{0.611106in}}%
\pgfpathlineto{\pgfqpoint{4.899529in}{0.606678in}}%
\pgfpathlineto{\pgfqpoint{4.901197in}{0.603016in}}%
\pgfpathlineto{\pgfqpoint{4.902309in}{0.606275in}}%
\pgfpathlineto{\pgfqpoint{4.902865in}{0.601009in}}%
\pgfpathlineto{\pgfqpoint{4.903420in}{0.602971in}}%
\pgfpathlineto{\pgfqpoint{4.903976in}{0.606705in}}%
\pgfpathlineto{\pgfqpoint{4.904532in}{0.605378in}}%
\pgfpathlineto{\pgfqpoint{4.905088in}{0.602482in}}%
\pgfpathlineto{\pgfqpoint{4.906756in}{0.606723in}}%
\pgfpathlineto{\pgfqpoint{4.907867in}{0.601207in}}%
\pgfpathlineto{\pgfqpoint{4.908979in}{0.613768in}}%
\pgfpathlineto{\pgfqpoint{4.910091in}{0.602739in}}%
\pgfpathlineto{\pgfqpoint{4.910647in}{0.604038in}}%
\pgfpathlineto{\pgfqpoint{4.911202in}{0.616057in}}%
\pgfpathlineto{\pgfqpoint{4.911758in}{0.602486in}}%
\pgfpathlineto{\pgfqpoint{4.912314in}{0.611992in}}%
\pgfpathlineto{\pgfqpoint{4.912870in}{0.616159in}}%
\pgfpathlineto{\pgfqpoint{4.913426in}{0.603834in}}%
\pgfpathlineto{\pgfqpoint{4.913982in}{0.610085in}}%
\pgfpathlineto{\pgfqpoint{4.914538in}{0.611260in}}%
\pgfpathlineto{\pgfqpoint{4.915094in}{0.604204in}}%
\pgfpathlineto{\pgfqpoint{4.915649in}{0.620487in}}%
\pgfpathlineto{\pgfqpoint{4.916205in}{0.605934in}}%
\pgfpathlineto{\pgfqpoint{4.916761in}{0.616707in}}%
\pgfpathlineto{\pgfqpoint{4.917317in}{0.601139in}}%
\pgfpathlineto{\pgfqpoint{4.917873in}{0.604675in}}%
\pgfpathlineto{\pgfqpoint{4.918429in}{0.605301in}}%
\pgfpathlineto{\pgfqpoint{4.918985in}{0.611013in}}%
\pgfpathlineto{\pgfqpoint{4.919540in}{0.601167in}}%
\pgfpathlineto{\pgfqpoint{4.920096in}{0.616238in}}%
\pgfpathlineto{\pgfqpoint{4.920652in}{0.614896in}}%
\pgfpathlineto{\pgfqpoint{4.921208in}{0.616171in}}%
\pgfpathlineto{\pgfqpoint{4.922876in}{0.607292in}}%
\pgfpathlineto{\pgfqpoint{4.923431in}{0.606353in}}%
\pgfpathlineto{\pgfqpoint{4.923987in}{0.612902in}}%
\pgfpathlineto{\pgfqpoint{4.924543in}{0.607254in}}%
\pgfpathlineto{\pgfqpoint{4.925655in}{0.605394in}}%
\pgfpathlineto{\pgfqpoint{4.926211in}{0.614393in}}%
\pgfpathlineto{\pgfqpoint{4.926767in}{0.602343in}}%
\pgfpathlineto{\pgfqpoint{4.927322in}{0.609800in}}%
\pgfpathlineto{\pgfqpoint{4.928990in}{0.603864in}}%
\pgfpathlineto{\pgfqpoint{4.929546in}{0.605626in}}%
\pgfpathlineto{\pgfqpoint{4.930102in}{0.614434in}}%
\pgfpathlineto{\pgfqpoint{4.930658in}{0.610537in}}%
\pgfpathlineto{\pgfqpoint{4.931769in}{0.602411in}}%
\pgfpathlineto{\pgfqpoint{4.932325in}{0.602583in}}%
\pgfpathlineto{\pgfqpoint{4.932881in}{0.608724in}}%
\pgfpathlineto{\pgfqpoint{4.933437in}{0.602664in}}%
\pgfpathlineto{\pgfqpoint{4.935105in}{0.608188in}}%
\pgfpathlineto{\pgfqpoint{4.935660in}{0.609059in}}%
\pgfpathlineto{\pgfqpoint{4.936216in}{0.614557in}}%
\pgfpathlineto{\pgfqpoint{4.936772in}{0.607225in}}%
\pgfpathlineto{\pgfqpoint{4.937328in}{0.621237in}}%
\pgfpathlineto{\pgfqpoint{4.937884in}{0.614370in}}%
\pgfpathlineto{\pgfqpoint{4.938996in}{0.602248in}}%
\pgfpathlineto{\pgfqpoint{4.939551in}{0.615705in}}%
\pgfpathlineto{\pgfqpoint{4.940107in}{0.615646in}}%
\pgfpathlineto{\pgfqpoint{4.940663in}{0.606729in}}%
\pgfpathlineto{\pgfqpoint{4.941219in}{0.608112in}}%
\pgfpathlineto{\pgfqpoint{4.941775in}{0.608891in}}%
\pgfpathlineto{\pgfqpoint{4.942331in}{0.617725in}}%
\pgfpathlineto{\pgfqpoint{4.942887in}{0.613162in}}%
\pgfpathlineto{\pgfqpoint{4.943442in}{0.614105in}}%
\pgfpathlineto{\pgfqpoint{4.944554in}{0.606749in}}%
\pgfpathlineto{\pgfqpoint{4.945110in}{0.615648in}}%
\pgfpathlineto{\pgfqpoint{4.945666in}{0.614241in}}%
\pgfpathlineto{\pgfqpoint{4.947333in}{0.604191in}}%
\pgfpathlineto{\pgfqpoint{4.947889in}{0.602422in}}%
\pgfpathlineto{\pgfqpoint{4.948445in}{0.611352in}}%
\pgfpathlineto{\pgfqpoint{4.949001in}{0.609382in}}%
\pgfpathlineto{\pgfqpoint{4.949557in}{0.610778in}}%
\pgfpathlineto{\pgfqpoint{4.950669in}{0.602777in}}%
\pgfpathlineto{\pgfqpoint{4.951780in}{0.606840in}}%
\pgfpathlineto{\pgfqpoint{4.952336in}{0.603875in}}%
\pgfpathlineto{\pgfqpoint{4.952892in}{0.608816in}}%
\pgfpathlineto{\pgfqpoint{4.953448in}{0.607320in}}%
\pgfpathlineto{\pgfqpoint{4.954004in}{0.603625in}}%
\pgfpathlineto{\pgfqpoint{4.954560in}{0.613064in}}%
\pgfpathlineto{\pgfqpoint{4.955116in}{0.607236in}}%
\pgfpathlineto{\pgfqpoint{4.955671in}{0.610918in}}%
\pgfpathlineto{\pgfqpoint{4.956227in}{0.606559in}}%
\pgfpathlineto{\pgfqpoint{4.956783in}{0.612133in}}%
\pgfpathlineto{\pgfqpoint{4.957339in}{0.607172in}}%
\pgfpathlineto{\pgfqpoint{4.959007in}{0.601949in}}%
\pgfpathlineto{\pgfqpoint{4.960118in}{0.608384in}}%
\pgfpathlineto{\pgfqpoint{4.960674in}{0.607705in}}%
\pgfpathlineto{\pgfqpoint{4.961230in}{0.606933in}}%
\pgfpathlineto{\pgfqpoint{4.961786in}{0.610496in}}%
\pgfpathlineto{\pgfqpoint{4.962342in}{0.601642in}}%
\pgfpathlineto{\pgfqpoint{4.962898in}{0.608910in}}%
\pgfpathlineto{\pgfqpoint{4.964565in}{0.602759in}}%
\pgfpathlineto{\pgfqpoint{4.965121in}{0.603444in}}%
\pgfpathlineto{\pgfqpoint{4.966233in}{0.610452in}}%
\pgfpathlineto{\pgfqpoint{4.966789in}{0.609978in}}%
\pgfpathlineto{\pgfqpoint{4.967344in}{0.603553in}}%
\pgfpathlineto{\pgfqpoint{4.967900in}{0.607643in}}%
\pgfpathlineto{\pgfqpoint{4.968456in}{0.609795in}}%
\pgfpathlineto{\pgfqpoint{4.969012in}{0.605367in}}%
\pgfpathlineto{\pgfqpoint{4.969568in}{0.605745in}}%
\pgfpathlineto{\pgfqpoint{4.970124in}{0.616145in}}%
\pgfpathlineto{\pgfqpoint{4.970680in}{0.611486in}}%
\pgfpathlineto{\pgfqpoint{4.971235in}{0.611987in}}%
\pgfpathlineto{\pgfqpoint{4.971791in}{0.608142in}}%
\pgfpathlineto{\pgfqpoint{4.973459in}{0.618978in}}%
\pgfpathlineto{\pgfqpoint{4.975127in}{0.602893in}}%
\pgfpathlineto{\pgfqpoint{4.976794in}{0.615989in}}%
\pgfpathlineto{\pgfqpoint{4.977350in}{0.613860in}}%
\pgfpathlineto{\pgfqpoint{4.977906in}{0.607775in}}%
\pgfpathlineto{\pgfqpoint{4.978462in}{0.610081in}}%
\pgfpathlineto{\pgfqpoint{4.979018in}{0.618142in}}%
\pgfpathlineto{\pgfqpoint{4.979573in}{0.604666in}}%
\pgfpathlineto{\pgfqpoint{4.980129in}{0.610962in}}%
\pgfpathlineto{\pgfqpoint{4.981797in}{0.605081in}}%
\pgfpathlineto{\pgfqpoint{4.983464in}{0.610376in}}%
\pgfpathlineto{\pgfqpoint{4.984020in}{0.610890in}}%
\pgfpathlineto{\pgfqpoint{4.984576in}{0.613868in}}%
\pgfpathlineto{\pgfqpoint{4.985132in}{0.603082in}}%
\pgfpathlineto{\pgfqpoint{4.985688in}{0.614031in}}%
\pgfpathlineto{\pgfqpoint{4.986244in}{0.606569in}}%
\pgfpathlineto{\pgfqpoint{4.986800in}{0.607883in}}%
\pgfpathlineto{\pgfqpoint{4.987355in}{0.606616in}}%
\pgfpathlineto{\pgfqpoint{4.988467in}{0.602478in}}%
\pgfpathlineto{\pgfqpoint{4.989023in}{0.604175in}}%
\pgfpathlineto{\pgfqpoint{4.989579in}{0.616464in}}%
\pgfpathlineto{\pgfqpoint{4.990135in}{0.611980in}}%
\pgfpathlineto{\pgfqpoint{4.991247in}{0.605257in}}%
\pgfpathlineto{\pgfqpoint{4.991802in}{0.606773in}}%
\pgfpathlineto{\pgfqpoint{4.992358in}{0.609954in}}%
\pgfpathlineto{\pgfqpoint{4.992914in}{0.605556in}}%
\pgfpathlineto{\pgfqpoint{4.993470in}{0.613690in}}%
\pgfpathlineto{\pgfqpoint{4.994026in}{0.607568in}}%
\pgfpathlineto{\pgfqpoint{4.995693in}{0.609673in}}%
\pgfpathlineto{\pgfqpoint{4.996249in}{0.616206in}}%
\pgfpathlineto{\pgfqpoint{4.996805in}{0.611377in}}%
\pgfpathlineto{\pgfqpoint{4.997361in}{0.612046in}}%
\pgfpathlineto{\pgfqpoint{4.997917in}{0.604566in}}%
\pgfpathlineto{\pgfqpoint{4.998473in}{0.611878in}}%
\pgfpathlineto{\pgfqpoint{4.999029in}{0.612508in}}%
\pgfpathlineto{\pgfqpoint{4.999584in}{0.610707in}}%
\pgfpathlineto{\pgfqpoint{5.000140in}{0.603602in}}%
\pgfpathlineto{\pgfqpoint{5.000696in}{0.614965in}}%
\pgfpathlineto{\pgfqpoint{5.001252in}{0.612458in}}%
\pgfpathlineto{\pgfqpoint{5.001808in}{0.609314in}}%
\pgfpathlineto{\pgfqpoint{5.002364in}{0.610005in}}%
\pgfpathlineto{\pgfqpoint{5.002920in}{0.612801in}}%
\pgfpathlineto{\pgfqpoint{5.003475in}{0.605530in}}%
\pgfpathlineto{\pgfqpoint{5.004031in}{0.607515in}}%
\pgfpathlineto{\pgfqpoint{5.004587in}{0.608500in}}%
\pgfpathlineto{\pgfqpoint{5.006255in}{0.604913in}}%
\pgfpathlineto{\pgfqpoint{5.007366in}{0.604795in}}%
\pgfpathlineto{\pgfqpoint{5.007922in}{0.603003in}}%
\pgfpathlineto{\pgfqpoint{5.008478in}{0.603760in}}%
\pgfpathlineto{\pgfqpoint{5.009034in}{0.606670in}}%
\pgfpathlineto{\pgfqpoint{5.009590in}{0.602573in}}%
\pgfpathlineto{\pgfqpoint{5.010146in}{0.607003in}}%
\pgfpathlineto{\pgfqpoint{5.010702in}{0.603026in}}%
\pgfpathlineto{\pgfqpoint{5.011813in}{0.609383in}}%
\pgfpathlineto{\pgfqpoint{5.012369in}{0.606479in}}%
\pgfpathlineto{\pgfqpoint{5.013481in}{0.602168in}}%
\pgfpathlineto{\pgfqpoint{5.014037in}{0.610589in}}%
\pgfpathlineto{\pgfqpoint{5.014593in}{0.608708in}}%
\pgfpathlineto{\pgfqpoint{5.015149in}{0.608621in}}%
\pgfpathlineto{\pgfqpoint{5.016260in}{0.603186in}}%
\pgfpathlineto{\pgfqpoint{5.016816in}{0.605121in}}%
\pgfpathlineto{\pgfqpoint{5.018484in}{0.600105in}}%
\pgfpathlineto{\pgfqpoint{5.020151in}{0.608760in}}%
\pgfpathlineto{\pgfqpoint{5.020707in}{0.603794in}}%
\pgfpathlineto{\pgfqpoint{5.021263in}{0.609399in}}%
\pgfpathlineto{\pgfqpoint{5.022375in}{0.602760in}}%
\pgfpathlineto{\pgfqpoint{5.022931in}{0.603698in}}%
\pgfpathlineto{\pgfqpoint{5.023486in}{0.610174in}}%
\pgfpathlineto{\pgfqpoint{5.024042in}{0.605586in}}%
\pgfpathlineto{\pgfqpoint{5.025154in}{0.603616in}}%
\pgfpathlineto{\pgfqpoint{5.025710in}{0.606840in}}%
\pgfpathlineto{\pgfqpoint{5.026822in}{0.600993in}}%
\pgfpathlineto{\pgfqpoint{5.028489in}{0.610541in}}%
\pgfpathlineto{\pgfqpoint{5.030157in}{0.602946in}}%
\pgfpathlineto{\pgfqpoint{5.030713in}{0.610213in}}%
\pgfpathlineto{\pgfqpoint{5.031269in}{0.608776in}}%
\pgfpathlineto{\pgfqpoint{5.032380in}{0.604472in}}%
\pgfpathlineto{\pgfqpoint{5.032936in}{0.606357in}}%
\pgfpathlineto{\pgfqpoint{5.033492in}{0.601363in}}%
\pgfpathlineto{\pgfqpoint{5.034048in}{0.610865in}}%
\pgfpathlineto{\pgfqpoint{5.034604in}{0.607299in}}%
\pgfpathlineto{\pgfqpoint{5.035160in}{0.605676in}}%
\pgfpathlineto{\pgfqpoint{5.036271in}{0.609275in}}%
\pgfpathlineto{\pgfqpoint{5.036827in}{0.607736in}}%
\pgfpathlineto{\pgfqpoint{5.037383in}{0.608247in}}%
\pgfpathlineto{\pgfqpoint{5.037939in}{0.601376in}}%
\pgfpathlineto{\pgfqpoint{5.038495in}{0.610545in}}%
\pgfpathlineto{\pgfqpoint{5.039051in}{0.601248in}}%
\pgfpathlineto{\pgfqpoint{5.039606in}{0.603145in}}%
\pgfpathlineto{\pgfqpoint{5.040162in}{0.603757in}}%
\pgfpathlineto{\pgfqpoint{5.040718in}{0.606140in}}%
\pgfpathlineto{\pgfqpoint{5.041274in}{0.602500in}}%
\pgfpathlineto{\pgfqpoint{5.041830in}{0.606060in}}%
\pgfpathlineto{\pgfqpoint{5.042386in}{0.602687in}}%
\pgfpathlineto{\pgfqpoint{5.042942in}{0.604073in}}%
\pgfpathlineto{\pgfqpoint{5.044053in}{0.607824in}}%
\pgfpathlineto{\pgfqpoint{5.044609in}{0.602555in}}%
\pgfpathlineto{\pgfqpoint{5.045165in}{0.603239in}}%
\pgfpathlineto{\pgfqpoint{5.045721in}{0.602741in}}%
\pgfpathlineto{\pgfqpoint{5.046277in}{0.604994in}}%
\pgfpathlineto{\pgfqpoint{5.046833in}{0.603743in}}%
\pgfpathlineto{\pgfqpoint{5.047389in}{0.601020in}}%
\pgfpathlineto{\pgfqpoint{5.047944in}{0.601912in}}%
\pgfpathlineto{\pgfqpoint{5.048500in}{0.601303in}}%
\pgfpathlineto{\pgfqpoint{5.049056in}{0.606697in}}%
\pgfpathlineto{\pgfqpoint{5.049612in}{0.605912in}}%
\pgfpathlineto{\pgfqpoint{5.050724in}{0.601859in}}%
\pgfpathlineto{\pgfqpoint{5.051835in}{0.606208in}}%
\pgfpathlineto{\pgfqpoint{5.052391in}{0.605788in}}%
\pgfpathlineto{\pgfqpoint{5.053503in}{0.606295in}}%
\pgfpathlineto{\pgfqpoint{5.055171in}{0.603447in}}%
\pgfpathlineto{\pgfqpoint{5.057394in}{0.608159in}}%
\pgfpathlineto{\pgfqpoint{5.057950in}{0.608293in}}%
\pgfpathlineto{\pgfqpoint{5.059062in}{0.601870in}}%
\pgfpathlineto{\pgfqpoint{5.059617in}{0.604129in}}%
\pgfpathlineto{\pgfqpoint{5.060173in}{0.612754in}}%
\pgfpathlineto{\pgfqpoint{5.060729in}{0.608801in}}%
\pgfpathlineto{\pgfqpoint{5.061285in}{0.602945in}}%
\pgfpathlineto{\pgfqpoint{5.061841in}{0.607781in}}%
\pgfpathlineto{\pgfqpoint{5.063508in}{0.601586in}}%
\pgfpathlineto{\pgfqpoint{5.064064in}{0.603717in}}%
\pgfpathlineto{\pgfqpoint{5.064620in}{0.602340in}}%
\pgfpathlineto{\pgfqpoint{5.065176in}{0.601826in}}%
\pgfpathlineto{\pgfqpoint{5.065732in}{0.604582in}}%
\pgfpathlineto{\pgfqpoint{5.066288in}{0.600086in}}%
\pgfpathlineto{\pgfqpoint{5.067400in}{0.604702in}}%
\pgfpathlineto{\pgfqpoint{5.068511in}{0.602917in}}%
\pgfpathlineto{\pgfqpoint{5.069067in}{0.604851in}}%
\pgfpathlineto{\pgfqpoint{5.070735in}{0.601589in}}%
\pgfpathlineto{\pgfqpoint{5.071291in}{0.600759in}}%
\pgfpathlineto{\pgfqpoint{5.071846in}{0.606153in}}%
\pgfpathlineto{\pgfqpoint{5.072402in}{0.602446in}}%
\pgfpathlineto{\pgfqpoint{5.072958in}{0.602686in}}%
\pgfpathlineto{\pgfqpoint{5.073514in}{0.604771in}}%
\pgfpathlineto{\pgfqpoint{5.074070in}{0.601818in}}%
\pgfpathlineto{\pgfqpoint{5.074626in}{0.602375in}}%
\pgfpathlineto{\pgfqpoint{5.075182in}{0.601956in}}%
\pgfpathlineto{\pgfqpoint{5.076849in}{0.603604in}}%
\pgfpathlineto{\pgfqpoint{5.077405in}{0.601237in}}%
\pgfpathlineto{\pgfqpoint{5.077961in}{0.605279in}}%
\pgfpathlineto{\pgfqpoint{5.078517in}{0.603173in}}%
\pgfpathlineto{\pgfqpoint{5.079073in}{0.604889in}}%
\pgfpathlineto{\pgfqpoint{5.079628in}{0.603706in}}%
\pgfpathlineto{\pgfqpoint{5.080740in}{0.601123in}}%
\pgfpathlineto{\pgfqpoint{5.081296in}{0.604926in}}%
\pgfpathlineto{\pgfqpoint{5.081852in}{0.603839in}}%
\pgfpathlineto{\pgfqpoint{5.082964in}{0.601152in}}%
\pgfpathlineto{\pgfqpoint{5.083519in}{0.604869in}}%
\pgfpathlineto{\pgfqpoint{5.084075in}{0.602666in}}%
\pgfpathlineto{\pgfqpoint{5.084631in}{0.603616in}}%
\pgfpathlineto{\pgfqpoint{5.085187in}{0.601487in}}%
\pgfpathlineto{\pgfqpoint{5.085743in}{0.606327in}}%
\pgfpathlineto{\pgfqpoint{5.086299in}{0.604550in}}%
\pgfpathlineto{\pgfqpoint{5.086855in}{0.602467in}}%
\pgfpathlineto{\pgfqpoint{5.087411in}{0.605624in}}%
\pgfpathlineto{\pgfqpoint{5.087966in}{0.604891in}}%
\pgfpathlineto{\pgfqpoint{5.088522in}{0.605006in}}%
\pgfpathlineto{\pgfqpoint{5.089634in}{0.600606in}}%
\pgfpathlineto{\pgfqpoint{5.090190in}{0.602624in}}%
\pgfpathlineto{\pgfqpoint{5.091302in}{0.605269in}}%
\pgfpathlineto{\pgfqpoint{5.091857in}{0.601823in}}%
\pgfpathlineto{\pgfqpoint{5.092413in}{0.603768in}}%
\pgfpathlineto{\pgfqpoint{5.092969in}{0.605325in}}%
\pgfpathlineto{\pgfqpoint{5.094637in}{0.603208in}}%
\pgfpathlineto{\pgfqpoint{5.095748in}{0.602562in}}%
\pgfpathlineto{\pgfqpoint{5.096304in}{0.608019in}}%
\pgfpathlineto{\pgfqpoint{5.096860in}{0.605229in}}%
\pgfpathlineto{\pgfqpoint{5.098528in}{0.606568in}}%
\pgfpathlineto{\pgfqpoint{5.099084in}{0.605224in}}%
\pgfpathlineto{\pgfqpoint{5.100195in}{0.608663in}}%
\pgfpathlineto{\pgfqpoint{5.100751in}{0.607946in}}%
\pgfpathlineto{\pgfqpoint{5.101307in}{0.606332in}}%
\pgfpathlineto{\pgfqpoint{5.101863in}{0.607320in}}%
\pgfpathlineto{\pgfqpoint{5.102975in}{0.607869in}}%
\pgfpathlineto{\pgfqpoint{5.104642in}{0.617756in}}%
\pgfpathlineto{\pgfqpoint{5.105198in}{0.630947in}}%
\pgfpathlineto{\pgfqpoint{5.105754in}{0.610519in}}%
\pgfpathlineto{\pgfqpoint{5.106310in}{0.699410in}}%
\pgfpathlineto{\pgfqpoint{5.106866in}{0.633427in}}%
\pgfpathlineto{\pgfqpoint{5.107422in}{0.647657in}}%
\pgfpathlineto{\pgfqpoint{5.107977in}{0.697829in}}%
\pgfpathlineto{\pgfqpoint{5.108533in}{0.614627in}}%
\pgfpathlineto{\pgfqpoint{5.109089in}{0.632822in}}%
\pgfpathlineto{\pgfqpoint{5.110757in}{0.610552in}}%
\pgfpathlineto{\pgfqpoint{5.112980in}{0.604552in}}%
\pgfpathlineto{\pgfqpoint{5.113536in}{0.605408in}}%
\pgfpathlineto{\pgfqpoint{5.114092in}{0.603724in}}%
\pgfpathlineto{\pgfqpoint{5.114648in}{0.605755in}}%
\pgfpathlineto{\pgfqpoint{5.115204in}{0.603506in}}%
\pgfpathlineto{\pgfqpoint{5.115759in}{0.605232in}}%
\pgfpathlineto{\pgfqpoint{5.116315in}{0.604695in}}%
\pgfpathlineto{\pgfqpoint{5.116871in}{0.601043in}}%
\pgfpathlineto{\pgfqpoint{5.117427in}{0.604142in}}%
\pgfpathlineto{\pgfqpoint{5.117983in}{0.602087in}}%
\pgfpathlineto{\pgfqpoint{5.118539in}{0.603681in}}%
\pgfpathlineto{\pgfqpoint{5.119095in}{0.603728in}}%
\pgfpathlineto{\pgfqpoint{5.119650in}{0.600596in}}%
\pgfpathlineto{\pgfqpoint{5.120206in}{0.602623in}}%
\pgfpathlineto{\pgfqpoint{5.122430in}{0.601017in}}%
\pgfpathlineto{\pgfqpoint{5.122986in}{0.601975in}}%
\pgfpathlineto{\pgfqpoint{5.124097in}{0.601515in}}%
\pgfpathlineto{\pgfqpoint{5.126321in}{0.600697in}}%
\pgfpathlineto{\pgfqpoint{5.127988in}{0.601225in}}%
\pgfpathlineto{\pgfqpoint{5.130768in}{0.601614in}}%
\pgfpathlineto{\pgfqpoint{5.131324in}{0.600665in}}%
\pgfpathlineto{\pgfqpoint{5.131879in}{0.601144in}}%
\pgfpathlineto{\pgfqpoint{5.135215in}{0.601810in}}%
\pgfpathlineto{\pgfqpoint{5.135770in}{0.600369in}}%
\pgfpathlineto{\pgfqpoint{5.136326in}{0.600798in}}%
\pgfpathlineto{\pgfqpoint{5.136882in}{0.601289in}}%
\pgfpathlineto{\pgfqpoint{5.137438in}{0.600132in}}%
\pgfpathlineto{\pgfqpoint{5.137994in}{0.600587in}}%
\pgfpathlineto{\pgfqpoint{5.139106in}{0.601075in}}%
\pgfpathlineto{\pgfqpoint{5.140773in}{0.600547in}}%
\pgfpathlineto{\pgfqpoint{5.142997in}{0.601215in}}%
\pgfpathlineto{\pgfqpoint{5.143553in}{0.602632in}}%
\pgfpathlineto{\pgfqpoint{5.144108in}{0.600051in}}%
\pgfpathlineto{\pgfqpoint{5.144664in}{0.601314in}}%
\pgfpathlineto{\pgfqpoint{5.148555in}{0.602001in}}%
\pgfpathlineto{\pgfqpoint{5.150223in}{0.600607in}}%
\pgfpathlineto{\pgfqpoint{5.150779in}{0.601156in}}%
\pgfpathlineto{\pgfqpoint{5.151335in}{0.600540in}}%
\pgfpathlineto{\pgfqpoint{5.153002in}{0.600605in}}%
\pgfpathlineto{\pgfqpoint{5.153558in}{0.602656in}}%
\pgfpathlineto{\pgfqpoint{5.154114in}{0.600636in}}%
\pgfpathlineto{\pgfqpoint{5.154670in}{0.601589in}}%
\pgfpathlineto{\pgfqpoint{5.155226in}{0.600495in}}%
\pgfpathlineto{\pgfqpoint{5.156337in}{0.601315in}}%
\pgfpathlineto{\pgfqpoint{5.156893in}{0.600376in}}%
\pgfpathlineto{\pgfqpoint{5.159117in}{0.601167in}}%
\pgfpathlineto{\pgfqpoint{5.160784in}{0.602115in}}%
\pgfpathlineto{\pgfqpoint{5.161340in}{0.600672in}}%
\pgfpathlineto{\pgfqpoint{5.161896in}{0.601725in}}%
\pgfpathlineto{\pgfqpoint{5.162452in}{0.601724in}}%
\pgfpathlineto{\pgfqpoint{5.164119in}{0.600502in}}%
\pgfpathlineto{\pgfqpoint{5.164675in}{0.601866in}}%
\pgfpathlineto{\pgfqpoint{5.165231in}{0.600152in}}%
\pgfpathlineto{\pgfqpoint{5.165787in}{0.600367in}}%
\pgfpathlineto{\pgfqpoint{5.166343in}{0.603284in}}%
\pgfpathlineto{\pgfqpoint{5.166899in}{0.600716in}}%
\pgfpathlineto{\pgfqpoint{5.168010in}{0.600782in}}%
\pgfpathlineto{\pgfqpoint{5.168566in}{0.602792in}}%
\pgfpathlineto{\pgfqpoint{5.169122in}{0.601953in}}%
\pgfpathlineto{\pgfqpoint{5.170234in}{0.601178in}}%
\pgfpathlineto{\pgfqpoint{5.171901in}{0.602918in}}%
\pgfpathlineto{\pgfqpoint{5.172457in}{0.600991in}}%
\pgfpathlineto{\pgfqpoint{5.173013in}{0.601636in}}%
\pgfpathlineto{\pgfqpoint{5.174681in}{0.601990in}}%
\pgfpathlineto{\pgfqpoint{5.175237in}{0.605591in}}%
\pgfpathlineto{\pgfqpoint{5.175792in}{0.600858in}}%
\pgfpathlineto{\pgfqpoint{5.176348in}{0.601714in}}%
\pgfpathlineto{\pgfqpoint{5.177460in}{0.601878in}}%
\pgfpathlineto{\pgfqpoint{5.178016in}{0.605482in}}%
\pgfpathlineto{\pgfqpoint{5.178572in}{0.602180in}}%
\pgfpathlineto{\pgfqpoint{5.180795in}{0.603515in}}%
\pgfpathlineto{\pgfqpoint{5.181907in}{0.600798in}}%
\pgfpathlineto{\pgfqpoint{5.183575in}{0.602813in}}%
\pgfpathlineto{\pgfqpoint{5.186910in}{0.602356in}}%
\pgfpathlineto{\pgfqpoint{5.187466in}{0.603948in}}%
\pgfpathlineto{\pgfqpoint{5.188021in}{0.600725in}}%
\pgfpathlineto{\pgfqpoint{5.188577in}{0.602943in}}%
\pgfpathlineto{\pgfqpoint{5.190801in}{0.600040in}}%
\pgfpathlineto{\pgfqpoint{5.191357in}{0.603676in}}%
\pgfpathlineto{\pgfqpoint{5.191912in}{0.602584in}}%
\pgfpathlineto{\pgfqpoint{5.193024in}{0.603027in}}%
\pgfpathlineto{\pgfqpoint{5.193580in}{0.603971in}}%
\pgfpathlineto{\pgfqpoint{5.195248in}{0.602479in}}%
\pgfpathlineto{\pgfqpoint{5.195803in}{0.604765in}}%
\pgfpathlineto{\pgfqpoint{5.196359in}{0.602732in}}%
\pgfpathlineto{\pgfqpoint{5.196915in}{0.602640in}}%
\pgfpathlineto{\pgfqpoint{5.197471in}{0.600886in}}%
\pgfpathlineto{\pgfqpoint{5.198027in}{0.603360in}}%
\pgfpathlineto{\pgfqpoint{5.198583in}{0.602265in}}%
\pgfpathlineto{\pgfqpoint{5.200250in}{0.602648in}}%
\pgfpathlineto{\pgfqpoint{5.200806in}{0.605567in}}%
\pgfpathlineto{\pgfqpoint{5.201362in}{0.604580in}}%
\pgfpathlineto{\pgfqpoint{5.202474in}{0.601675in}}%
\pgfpathlineto{\pgfqpoint{5.203030in}{0.603219in}}%
\pgfpathlineto{\pgfqpoint{5.204141in}{0.605200in}}%
\pgfpathlineto{\pgfqpoint{5.205253in}{0.601910in}}%
\pgfpathlineto{\pgfqpoint{5.206921in}{0.606996in}}%
\pgfpathlineto{\pgfqpoint{5.208588in}{0.600826in}}%
\pgfpathlineto{\pgfqpoint{5.209144in}{0.603688in}}%
\pgfpathlineto{\pgfqpoint{5.209700in}{0.603112in}}%
\pgfpathlineto{\pgfqpoint{5.210256in}{0.602952in}}%
\pgfpathlineto{\pgfqpoint{5.210812in}{0.604383in}}%
\pgfpathlineto{\pgfqpoint{5.211368in}{0.601736in}}%
\pgfpathlineto{\pgfqpoint{5.211923in}{0.604073in}}%
\pgfpathlineto{\pgfqpoint{5.212479in}{0.604667in}}%
\pgfpathlineto{\pgfqpoint{5.213035in}{0.607381in}}%
\pgfpathlineto{\pgfqpoint{5.213591in}{0.601786in}}%
\pgfpathlineto{\pgfqpoint{5.214147in}{0.604447in}}%
\pgfpathlineto{\pgfqpoint{5.215259in}{0.601455in}}%
\pgfpathlineto{\pgfqpoint{5.216370in}{0.606087in}}%
\pgfpathlineto{\pgfqpoint{5.217482in}{0.601215in}}%
\pgfpathlineto{\pgfqpoint{5.218038in}{0.604391in}}%
\pgfpathlineto{\pgfqpoint{5.218594in}{0.600586in}}%
\pgfpathlineto{\pgfqpoint{5.219150in}{0.604049in}}%
\pgfpathlineto{\pgfqpoint{5.220261in}{0.605171in}}%
\pgfpathlineto{\pgfqpoint{5.220817in}{0.603507in}}%
\pgfpathlineto{\pgfqpoint{5.221929in}{0.601730in}}%
\pgfpathlineto{\pgfqpoint{5.222485in}{0.604310in}}%
\pgfpathlineto{\pgfqpoint{5.223041in}{0.603658in}}%
\pgfpathlineto{\pgfqpoint{5.223597in}{0.602845in}}%
\pgfpathlineto{\pgfqpoint{5.224152in}{0.606864in}}%
\pgfpathlineto{\pgfqpoint{5.224708in}{0.602005in}}%
\pgfpathlineto{\pgfqpoint{5.225264in}{0.603115in}}%
\pgfpathlineto{\pgfqpoint{5.225820in}{0.609774in}}%
\pgfpathlineto{\pgfqpoint{5.226376in}{0.603686in}}%
\pgfpathlineto{\pgfqpoint{5.227488in}{0.600672in}}%
\pgfpathlineto{\pgfqpoint{5.228043in}{0.605967in}}%
\pgfpathlineto{\pgfqpoint{5.228599in}{0.601472in}}%
\pgfpathlineto{\pgfqpoint{5.230267in}{0.605337in}}%
\pgfpathlineto{\pgfqpoint{5.230823in}{0.604054in}}%
\pgfpathlineto{\pgfqpoint{5.231379in}{0.606501in}}%
\pgfpathlineto{\pgfqpoint{5.231934in}{0.601609in}}%
\pgfpathlineto{\pgfqpoint{5.232490in}{0.609871in}}%
\pgfpathlineto{\pgfqpoint{5.233046in}{0.608286in}}%
\pgfpathlineto{\pgfqpoint{5.233602in}{0.600631in}}%
\pgfpathlineto{\pgfqpoint{5.234158in}{0.611840in}}%
\pgfpathlineto{\pgfqpoint{5.234714in}{0.604586in}}%
\pgfpathlineto{\pgfqpoint{5.235270in}{0.605152in}}%
\pgfpathlineto{\pgfqpoint{5.235826in}{0.601907in}}%
\pgfpathlineto{\pgfqpoint{5.236381in}{0.602956in}}%
\pgfpathlineto{\pgfqpoint{5.236937in}{0.606534in}}%
\pgfpathlineto{\pgfqpoint{5.237493in}{0.606141in}}%
\pgfpathlineto{\pgfqpoint{5.238049in}{0.602017in}}%
\pgfpathlineto{\pgfqpoint{5.239161in}{0.613412in}}%
\pgfpathlineto{\pgfqpoint{5.240828in}{0.601996in}}%
\pgfpathlineto{\pgfqpoint{5.241384in}{0.605132in}}%
\pgfpathlineto{\pgfqpoint{5.241940in}{0.602853in}}%
\pgfpathlineto{\pgfqpoint{5.243052in}{0.605192in}}%
\pgfpathlineto{\pgfqpoint{5.243608in}{0.604348in}}%
\pgfpathlineto{\pgfqpoint{5.244163in}{0.605363in}}%
\pgfpathlineto{\pgfqpoint{5.244719in}{0.606751in}}%
\pgfpathlineto{\pgfqpoint{5.245275in}{0.604038in}}%
\pgfpathlineto{\pgfqpoint{5.245831in}{0.606338in}}%
\pgfpathlineto{\pgfqpoint{5.247499in}{0.600137in}}%
\pgfpathlineto{\pgfqpoint{5.248610in}{0.604059in}}%
\pgfpathlineto{\pgfqpoint{5.249722in}{0.601383in}}%
\pgfpathlineto{\pgfqpoint{5.251390in}{0.609623in}}%
\pgfpathlineto{\pgfqpoint{5.252501in}{0.603280in}}%
\pgfpathlineto{\pgfqpoint{5.253057in}{0.608952in}}%
\pgfpathlineto{\pgfqpoint{5.253613in}{0.607126in}}%
\pgfpathlineto{\pgfqpoint{5.254169in}{0.603737in}}%
\pgfpathlineto{\pgfqpoint{5.254725in}{0.604740in}}%
\pgfpathlineto{\pgfqpoint{5.255281in}{0.604439in}}%
\pgfpathlineto{\pgfqpoint{5.255837in}{0.608670in}}%
\pgfpathlineto{\pgfqpoint{5.256392in}{0.604481in}}%
\pgfpathlineto{\pgfqpoint{5.256948in}{0.604389in}}%
\pgfpathlineto{\pgfqpoint{5.257504in}{0.602177in}}%
\pgfpathlineto{\pgfqpoint{5.258060in}{0.609264in}}%
\pgfpathlineto{\pgfqpoint{5.258616in}{0.607183in}}%
\pgfpathlineto{\pgfqpoint{5.259172in}{0.603827in}}%
\pgfpathlineto{\pgfqpoint{5.259728in}{0.607014in}}%
\pgfpathlineto{\pgfqpoint{5.260839in}{0.607091in}}%
\pgfpathlineto{\pgfqpoint{5.261395in}{0.601966in}}%
\pgfpathlineto{\pgfqpoint{5.261951in}{0.607006in}}%
\pgfpathlineto{\pgfqpoint{5.262507in}{0.607191in}}%
\pgfpathlineto{\pgfqpoint{5.263063in}{0.610643in}}%
\pgfpathlineto{\pgfqpoint{5.263619in}{0.608961in}}%
\pgfpathlineto{\pgfqpoint{5.264174in}{0.603748in}}%
\pgfpathlineto{\pgfqpoint{5.264730in}{0.610469in}}%
\pgfpathlineto{\pgfqpoint{5.265286in}{0.604223in}}%
\pgfpathlineto{\pgfqpoint{5.266398in}{0.604852in}}%
\pgfpathlineto{\pgfqpoint{5.266954in}{0.603756in}}%
\pgfpathlineto{\pgfqpoint{5.267510in}{0.606456in}}%
\pgfpathlineto{\pgfqpoint{5.268065in}{0.604361in}}%
\pgfpathlineto{\pgfqpoint{5.268621in}{0.604038in}}%
\pgfpathlineto{\pgfqpoint{5.269733in}{0.611270in}}%
\pgfpathlineto{\pgfqpoint{5.270289in}{0.608273in}}%
\pgfpathlineto{\pgfqpoint{5.271401in}{0.605470in}}%
\pgfpathlineto{\pgfqpoint{5.271956in}{0.606424in}}%
\pgfpathlineto{\pgfqpoint{5.272512in}{0.603628in}}%
\pgfpathlineto{\pgfqpoint{5.273068in}{0.605008in}}%
\pgfpathlineto{\pgfqpoint{5.274736in}{0.601836in}}%
\pgfpathlineto{\pgfqpoint{5.275292in}{0.609229in}}%
\pgfpathlineto{\pgfqpoint{5.275848in}{0.603834in}}%
\pgfpathlineto{\pgfqpoint{5.276959in}{0.609324in}}%
\pgfpathlineto{\pgfqpoint{5.278071in}{0.600781in}}%
\pgfpathlineto{\pgfqpoint{5.278627in}{0.603618in}}%
\pgfpathlineto{\pgfqpoint{5.279183in}{0.606036in}}%
\pgfpathlineto{\pgfqpoint{5.279739in}{0.603145in}}%
\pgfpathlineto{\pgfqpoint{5.280294in}{0.607802in}}%
\pgfpathlineto{\pgfqpoint{5.281406in}{0.602052in}}%
\pgfpathlineto{\pgfqpoint{5.281962in}{0.607044in}}%
\pgfpathlineto{\pgfqpoint{5.282518in}{0.603744in}}%
\pgfpathlineto{\pgfqpoint{5.283630in}{0.607183in}}%
\pgfpathlineto{\pgfqpoint{5.284185in}{0.603759in}}%
\pgfpathlineto{\pgfqpoint{5.284741in}{0.604572in}}%
\pgfpathlineto{\pgfqpoint{5.285297in}{0.610508in}}%
\pgfpathlineto{\pgfqpoint{5.285853in}{0.604217in}}%
\pgfpathlineto{\pgfqpoint{5.286409in}{0.606944in}}%
\pgfpathlineto{\pgfqpoint{5.286965in}{0.611111in}}%
\pgfpathlineto{\pgfqpoint{5.288076in}{0.604868in}}%
\pgfpathlineto{\pgfqpoint{5.289744in}{0.608032in}}%
\pgfpathlineto{\pgfqpoint{5.290300in}{0.607443in}}%
\pgfpathlineto{\pgfqpoint{5.290856in}{0.601415in}}%
\pgfpathlineto{\pgfqpoint{5.291968in}{0.613101in}}%
\pgfpathlineto{\pgfqpoint{5.293635in}{0.605640in}}%
\pgfpathlineto{\pgfqpoint{5.294191in}{0.605329in}}%
\pgfpathlineto{\pgfqpoint{5.294747in}{0.608355in}}%
\pgfpathlineto{\pgfqpoint{5.295303in}{0.603746in}}%
\pgfpathlineto{\pgfqpoint{5.295859in}{0.608577in}}%
\pgfpathlineto{\pgfqpoint{5.296414in}{0.601738in}}%
\pgfpathlineto{\pgfqpoint{5.296970in}{0.606314in}}%
\pgfpathlineto{\pgfqpoint{5.297526in}{0.605512in}}%
\pgfpathlineto{\pgfqpoint{5.299194in}{0.612941in}}%
\pgfpathlineto{\pgfqpoint{5.299750in}{0.605542in}}%
\pgfpathlineto{\pgfqpoint{5.300305in}{0.606305in}}%
\pgfpathlineto{\pgfqpoint{5.300861in}{0.606402in}}%
\pgfpathlineto{\pgfqpoint{5.301417in}{0.602716in}}%
\pgfpathlineto{\pgfqpoint{5.301973in}{0.603673in}}%
\pgfpathlineto{\pgfqpoint{5.302529in}{0.609527in}}%
\pgfpathlineto{\pgfqpoint{5.303085in}{0.601691in}}%
\pgfpathlineto{\pgfqpoint{5.303641in}{0.610234in}}%
\pgfpathlineto{\pgfqpoint{5.304196in}{0.606659in}}%
\pgfpathlineto{\pgfqpoint{5.305308in}{0.601030in}}%
\pgfpathlineto{\pgfqpoint{5.305864in}{0.602758in}}%
\pgfpathlineto{\pgfqpoint{5.306420in}{0.601021in}}%
\pgfpathlineto{\pgfqpoint{5.306976in}{0.601127in}}%
\pgfpathlineto{\pgfqpoint{5.308087in}{0.611198in}}%
\pgfpathlineto{\pgfqpoint{5.308643in}{0.607252in}}%
\pgfpathlineto{\pgfqpoint{5.309755in}{0.602851in}}%
\pgfpathlineto{\pgfqpoint{5.310311in}{0.610476in}}%
\pgfpathlineto{\pgfqpoint{5.310867in}{0.607629in}}%
\pgfpathlineto{\pgfqpoint{5.311423in}{0.609633in}}%
\pgfpathlineto{\pgfqpoint{5.313646in}{0.603079in}}%
\pgfpathlineto{\pgfqpoint{5.314758in}{0.603521in}}%
\pgfpathlineto{\pgfqpoint{5.315314in}{0.605268in}}%
\pgfpathlineto{\pgfqpoint{5.315870in}{0.612451in}}%
\pgfpathlineto{\pgfqpoint{5.316425in}{0.602202in}}%
\pgfpathlineto{\pgfqpoint{5.316981in}{0.604577in}}%
\pgfpathlineto{\pgfqpoint{5.318649in}{0.606455in}}%
\pgfpathlineto{\pgfqpoint{5.319761in}{0.612970in}}%
\pgfpathlineto{\pgfqpoint{5.320316in}{0.603463in}}%
\pgfpathlineto{\pgfqpoint{5.320872in}{0.611577in}}%
\pgfpathlineto{\pgfqpoint{5.323096in}{0.602642in}}%
\pgfpathlineto{\pgfqpoint{5.323652in}{0.603309in}}%
\pgfpathlineto{\pgfqpoint{5.324207in}{0.606404in}}%
\pgfpathlineto{\pgfqpoint{5.324763in}{0.602126in}}%
\pgfpathlineto{\pgfqpoint{5.325319in}{0.603568in}}%
\pgfpathlineto{\pgfqpoint{5.325875in}{0.605151in}}%
\pgfpathlineto{\pgfqpoint{5.326431in}{0.602867in}}%
\pgfpathlineto{\pgfqpoint{5.326987in}{0.606054in}}%
\pgfpathlineto{\pgfqpoint{5.327543in}{0.600815in}}%
\pgfpathlineto{\pgfqpoint{5.329210in}{0.611914in}}%
\pgfpathlineto{\pgfqpoint{5.330322in}{0.603298in}}%
\pgfpathlineto{\pgfqpoint{5.330878in}{0.604410in}}%
\pgfpathlineto{\pgfqpoint{5.331434in}{0.608905in}}%
\pgfpathlineto{\pgfqpoint{5.331990in}{0.601502in}}%
\pgfpathlineto{\pgfqpoint{5.332545in}{0.611061in}}%
\pgfpathlineto{\pgfqpoint{5.333101in}{0.605461in}}%
\pgfpathlineto{\pgfqpoint{5.333657in}{0.603395in}}%
\pgfpathlineto{\pgfqpoint{5.334213in}{0.605332in}}%
\pgfpathlineto{\pgfqpoint{5.334769in}{0.609388in}}%
\pgfpathlineto{\pgfqpoint{5.335881in}{0.603685in}}%
\pgfpathlineto{\pgfqpoint{5.336436in}{0.617474in}}%
\pgfpathlineto{\pgfqpoint{5.336992in}{0.607054in}}%
\pgfpathlineto{\pgfqpoint{5.339216in}{0.603416in}}%
\pgfpathlineto{\pgfqpoint{5.339772in}{0.605791in}}%
\pgfpathlineto{\pgfqpoint{5.340327in}{0.607052in}}%
\pgfpathlineto{\pgfqpoint{5.340883in}{0.603765in}}%
\pgfpathlineto{\pgfqpoint{5.341439in}{0.607204in}}%
\pgfpathlineto{\pgfqpoint{5.341995in}{0.604871in}}%
\pgfpathlineto{\pgfqpoint{5.342551in}{0.603925in}}%
\pgfpathlineto{\pgfqpoint{5.343107in}{0.608805in}}%
\pgfpathlineto{\pgfqpoint{5.343663in}{0.605380in}}%
\pgfpathlineto{\pgfqpoint{5.344218in}{0.608099in}}%
\pgfpathlineto{\pgfqpoint{5.344774in}{0.605782in}}%
\pgfpathlineto{\pgfqpoint{5.345886in}{0.603756in}}%
\pgfpathlineto{\pgfqpoint{5.347554in}{0.606945in}}%
\pgfpathlineto{\pgfqpoint{5.348109in}{0.601004in}}%
\pgfpathlineto{\pgfqpoint{5.348665in}{0.606984in}}%
\pgfpathlineto{\pgfqpoint{5.349221in}{0.610152in}}%
\pgfpathlineto{\pgfqpoint{5.349777in}{0.602061in}}%
\pgfpathlineto{\pgfqpoint{5.350333in}{0.609881in}}%
\pgfpathlineto{\pgfqpoint{5.350889in}{0.606339in}}%
\pgfpathlineto{\pgfqpoint{5.351445in}{0.612785in}}%
\pgfpathlineto{\pgfqpoint{5.352001in}{0.604485in}}%
\pgfpathlineto{\pgfqpoint{5.352556in}{0.608058in}}%
\pgfpathlineto{\pgfqpoint{5.353112in}{0.608291in}}%
\pgfpathlineto{\pgfqpoint{5.354780in}{0.602453in}}%
\pgfpathlineto{\pgfqpoint{5.356447in}{0.605675in}}%
\pgfpathlineto{\pgfqpoint{5.357003in}{0.602947in}}%
\pgfpathlineto{\pgfqpoint{5.358671in}{0.609880in}}%
\pgfpathlineto{\pgfqpoint{5.359783in}{0.609725in}}%
\pgfpathlineto{\pgfqpoint{5.360894in}{0.603453in}}%
\pgfpathlineto{\pgfqpoint{5.362006in}{0.610754in}}%
\pgfpathlineto{\pgfqpoint{5.362562in}{0.605812in}}%
\pgfpathlineto{\pgfqpoint{5.363118in}{0.608094in}}%
\pgfpathlineto{\pgfqpoint{5.363674in}{0.606536in}}%
\pgfpathlineto{\pgfqpoint{5.364229in}{0.601351in}}%
\pgfpathlineto{\pgfqpoint{5.364785in}{0.602330in}}%
\pgfpathlineto{\pgfqpoint{5.365897in}{0.610155in}}%
\pgfpathlineto{\pgfqpoint{5.366453in}{0.608046in}}%
\pgfpathlineto{\pgfqpoint{5.367565in}{0.602008in}}%
\pgfpathlineto{\pgfqpoint{5.368121in}{0.609502in}}%
\pgfpathlineto{\pgfqpoint{5.368676in}{0.605587in}}%
\pgfpathlineto{\pgfqpoint{5.369232in}{0.606522in}}%
\pgfpathlineto{\pgfqpoint{5.369788in}{0.601766in}}%
\pgfpathlineto{\pgfqpoint{5.370344in}{0.603439in}}%
\pgfpathlineto{\pgfqpoint{5.371456in}{0.603193in}}%
\pgfpathlineto{\pgfqpoint{5.373123in}{0.607483in}}%
\pgfpathlineto{\pgfqpoint{5.373679in}{0.606942in}}%
\pgfpathlineto{\pgfqpoint{5.374791in}{0.604519in}}%
\pgfpathlineto{\pgfqpoint{5.377014in}{0.610537in}}%
\pgfpathlineto{\pgfqpoint{5.377570in}{0.609668in}}%
\pgfpathlineto{\pgfqpoint{5.378126in}{0.610154in}}%
\pgfpathlineto{\pgfqpoint{5.379238in}{0.601812in}}%
\pgfpathlineto{\pgfqpoint{5.379794in}{0.610825in}}%
\pgfpathlineto{\pgfqpoint{5.380349in}{0.606348in}}%
\pgfpathlineto{\pgfqpoint{5.380905in}{0.607308in}}%
\pgfpathlineto{\pgfqpoint{5.382017in}{0.603168in}}%
\pgfpathlineto{\pgfqpoint{5.382573in}{0.604470in}}%
\pgfpathlineto{\pgfqpoint{5.383129in}{0.606893in}}%
\pgfpathlineto{\pgfqpoint{5.383685in}{0.604849in}}%
\pgfpathlineto{\pgfqpoint{5.384240in}{0.600821in}}%
\pgfpathlineto{\pgfqpoint{5.384796in}{0.601379in}}%
\pgfpathlineto{\pgfqpoint{5.385908in}{0.609448in}}%
\pgfpathlineto{\pgfqpoint{5.387020in}{0.601206in}}%
\pgfpathlineto{\pgfqpoint{5.387576in}{0.602800in}}%
\pgfpathlineto{\pgfqpoint{5.388132in}{0.606883in}}%
\pgfpathlineto{\pgfqpoint{5.388687in}{0.602841in}}%
\pgfpathlineto{\pgfqpoint{5.389799in}{0.605010in}}%
\pgfpathlineto{\pgfqpoint{5.390355in}{0.606411in}}%
\pgfpathlineto{\pgfqpoint{5.390911in}{0.601535in}}%
\pgfpathlineto{\pgfqpoint{5.391467in}{0.604734in}}%
\pgfpathlineto{\pgfqpoint{5.392578in}{0.608351in}}%
\pgfpathlineto{\pgfqpoint{5.393134in}{0.604668in}}%
\pgfpathlineto{\pgfqpoint{5.393690in}{0.610291in}}%
\pgfpathlineto{\pgfqpoint{5.394246in}{0.605770in}}%
\pgfpathlineto{\pgfqpoint{5.394802in}{0.600988in}}%
\pgfpathlineto{\pgfqpoint{5.395358in}{0.602894in}}%
\pgfpathlineto{\pgfqpoint{5.396469in}{0.606383in}}%
\pgfpathlineto{\pgfqpoint{5.397581in}{0.600941in}}%
\pgfpathlineto{\pgfqpoint{5.398693in}{0.606833in}}%
\pgfpathlineto{\pgfqpoint{5.400916in}{0.603108in}}%
\pgfpathlineto{\pgfqpoint{5.402584in}{0.607004in}}%
\pgfpathlineto{\pgfqpoint{5.403140in}{0.604959in}}%
\pgfpathlineto{\pgfqpoint{5.403696in}{0.609781in}}%
\pgfpathlineto{\pgfqpoint{5.404251in}{0.604701in}}%
\pgfpathlineto{\pgfqpoint{5.404807in}{0.606943in}}%
\pgfpathlineto{\pgfqpoint{5.405363in}{0.605106in}}%
\pgfpathlineto{\pgfqpoint{5.405919in}{0.605363in}}%
\pgfpathlineto{\pgfqpoint{5.406475in}{0.611225in}}%
\pgfpathlineto{\pgfqpoint{5.407031in}{0.602066in}}%
\pgfpathlineto{\pgfqpoint{5.407587in}{0.608419in}}%
\pgfpathlineto{\pgfqpoint{5.408143in}{0.602777in}}%
\pgfpathlineto{\pgfqpoint{5.408698in}{0.608113in}}%
\pgfpathlineto{\pgfqpoint{5.409254in}{0.605831in}}%
\pgfpathlineto{\pgfqpoint{5.409810in}{0.610117in}}%
\pgfpathlineto{\pgfqpoint{5.410922in}{0.604174in}}%
\pgfpathlineto{\pgfqpoint{5.411478in}{0.607247in}}%
\pgfpathlineto{\pgfqpoint{5.413145in}{0.601994in}}%
\pgfpathlineto{\pgfqpoint{5.414257in}{0.603884in}}%
\pgfpathlineto{\pgfqpoint{5.415369in}{0.602878in}}%
\pgfpathlineto{\pgfqpoint{5.415925in}{0.607099in}}%
\pgfpathlineto{\pgfqpoint{5.416480in}{0.601611in}}%
\pgfpathlineto{\pgfqpoint{5.417036in}{0.605738in}}%
\pgfpathlineto{\pgfqpoint{5.417592in}{0.606028in}}%
\pgfpathlineto{\pgfqpoint{5.418148in}{0.603532in}}%
\pgfpathlineto{\pgfqpoint{5.418704in}{0.609097in}}%
\pgfpathlineto{\pgfqpoint{5.419260in}{0.606500in}}%
\pgfpathlineto{\pgfqpoint{5.419816in}{0.606643in}}%
\pgfpathlineto{\pgfqpoint{5.420371in}{0.602344in}}%
\pgfpathlineto{\pgfqpoint{5.420927in}{0.603187in}}%
\pgfpathlineto{\pgfqpoint{5.422595in}{0.610790in}}%
\pgfpathlineto{\pgfqpoint{5.424818in}{0.603539in}}%
\pgfpathlineto{\pgfqpoint{5.426486in}{0.606993in}}%
\pgfpathlineto{\pgfqpoint{5.427042in}{0.601750in}}%
\pgfpathlineto{\pgfqpoint{5.427598in}{0.603566in}}%
\pgfpathlineto{\pgfqpoint{5.428154in}{0.605735in}}%
\pgfpathlineto{\pgfqpoint{5.429821in}{0.600748in}}%
\pgfpathlineto{\pgfqpoint{5.430933in}{0.607610in}}%
\pgfpathlineto{\pgfqpoint{5.432045in}{0.602053in}}%
\pgfpathlineto{\pgfqpoint{5.433712in}{0.608497in}}%
\pgfpathlineto{\pgfqpoint{5.434268in}{0.606005in}}%
\pgfpathlineto{\pgfqpoint{5.434824in}{0.609591in}}%
\pgfpathlineto{\pgfqpoint{5.436491in}{0.601408in}}%
\pgfpathlineto{\pgfqpoint{5.437603in}{0.605323in}}%
\pgfpathlineto{\pgfqpoint{5.438159in}{0.603173in}}%
\pgfpathlineto{\pgfqpoint{5.439271in}{0.605884in}}%
\pgfpathlineto{\pgfqpoint{5.439827in}{0.600738in}}%
\pgfpathlineto{\pgfqpoint{5.440382in}{0.602807in}}%
\pgfpathlineto{\pgfqpoint{5.440938in}{0.605900in}}%
\pgfpathlineto{\pgfqpoint{5.441494in}{0.603427in}}%
\pgfpathlineto{\pgfqpoint{5.442050in}{0.600743in}}%
\pgfpathlineto{\pgfqpoint{5.442606in}{0.606258in}}%
\pgfpathlineto{\pgfqpoint{5.443162in}{0.605611in}}%
\pgfpathlineto{\pgfqpoint{5.444829in}{0.601635in}}%
\pgfpathlineto{\pgfqpoint{5.445385in}{0.611543in}}%
\pgfpathlineto{\pgfqpoint{5.445941in}{0.606352in}}%
\pgfpathlineto{\pgfqpoint{5.447053in}{0.601519in}}%
\pgfpathlineto{\pgfqpoint{5.447609in}{0.605604in}}%
\pgfpathlineto{\pgfqpoint{5.448165in}{0.603886in}}%
\pgfpathlineto{\pgfqpoint{5.448720in}{0.605255in}}%
\pgfpathlineto{\pgfqpoint{5.450388in}{0.600676in}}%
\pgfpathlineto{\pgfqpoint{5.450944in}{0.604480in}}%
\pgfpathlineto{\pgfqpoint{5.451500in}{0.604145in}}%
\pgfpathlineto{\pgfqpoint{5.452056in}{0.603239in}}%
\pgfpathlineto{\pgfqpoint{5.452611in}{0.607973in}}%
\pgfpathlineto{\pgfqpoint{5.453167in}{0.606354in}}%
\pgfpathlineto{\pgfqpoint{5.454835in}{0.603344in}}%
\pgfpathlineto{\pgfqpoint{5.455391in}{0.602129in}}%
\pgfpathlineto{\pgfqpoint{5.455947in}{0.607126in}}%
\pgfpathlineto{\pgfqpoint{5.456502in}{0.602894in}}%
\pgfpathlineto{\pgfqpoint{5.458726in}{0.605745in}}%
\pgfpathlineto{\pgfqpoint{5.459282in}{0.607139in}}%
\pgfpathlineto{\pgfqpoint{5.460393in}{0.603082in}}%
\pgfpathlineto{\pgfqpoint{5.460949in}{0.607683in}}%
\pgfpathlineto{\pgfqpoint{5.462061in}{0.601768in}}%
\pgfpathlineto{\pgfqpoint{5.462617in}{0.604869in}}%
\pgfpathlineto{\pgfqpoint{5.463173in}{0.600719in}}%
\pgfpathlineto{\pgfqpoint{5.463729in}{0.609555in}}%
\pgfpathlineto{\pgfqpoint{5.464285in}{0.606602in}}%
\pgfpathlineto{\pgfqpoint{5.465396in}{0.602360in}}%
\pgfpathlineto{\pgfqpoint{5.465952in}{0.607611in}}%
\pgfpathlineto{\pgfqpoint{5.466508in}{0.607368in}}%
\pgfpathlineto{\pgfqpoint{5.468176in}{0.601213in}}%
\pgfpathlineto{\pgfqpoint{5.468731in}{0.605243in}}%
\pgfpathlineto{\pgfqpoint{5.469287in}{0.600199in}}%
\pgfpathlineto{\pgfqpoint{5.470955in}{0.607011in}}%
\pgfpathlineto{\pgfqpoint{5.472622in}{0.602378in}}%
\pgfpathlineto{\pgfqpoint{5.473178in}{0.604773in}}%
\pgfpathlineto{\pgfqpoint{5.473734in}{0.601853in}}%
\pgfpathlineto{\pgfqpoint{5.474290in}{0.604718in}}%
\pgfpathlineto{\pgfqpoint{5.474846in}{0.608454in}}%
\pgfpathlineto{\pgfqpoint{5.475402in}{0.602402in}}%
\pgfpathlineto{\pgfqpoint{5.475958in}{0.604528in}}%
\pgfpathlineto{\pgfqpoint{5.476513in}{0.604419in}}%
\pgfpathlineto{\pgfqpoint{5.477069in}{0.608152in}}%
\pgfpathlineto{\pgfqpoint{5.477625in}{0.605453in}}%
\pgfpathlineto{\pgfqpoint{5.479293in}{0.601907in}}%
\pgfpathlineto{\pgfqpoint{5.480960in}{0.611323in}}%
\pgfpathlineto{\pgfqpoint{5.482072in}{0.602294in}}%
\pgfpathlineto{\pgfqpoint{5.482628in}{0.602930in}}%
\pgfpathlineto{\pgfqpoint{5.483184in}{0.604668in}}%
\pgfpathlineto{\pgfqpoint{5.483740in}{0.600793in}}%
\pgfpathlineto{\pgfqpoint{5.484296in}{0.603368in}}%
\pgfpathlineto{\pgfqpoint{5.484851in}{0.601612in}}%
\pgfpathlineto{\pgfqpoint{5.485963in}{0.604772in}}%
\pgfpathlineto{\pgfqpoint{5.486519in}{0.601114in}}%
\pgfpathlineto{\pgfqpoint{5.487075in}{0.603374in}}%
\pgfpathlineto{\pgfqpoint{5.487631in}{0.605681in}}%
\pgfpathlineto{\pgfqpoint{5.488187in}{0.603380in}}%
\pgfpathlineto{\pgfqpoint{5.489854in}{0.602002in}}%
\pgfpathlineto{\pgfqpoint{5.492078in}{0.609053in}}%
\pgfpathlineto{\pgfqpoint{5.493745in}{0.600954in}}%
\pgfpathlineto{\pgfqpoint{5.494301in}{0.602169in}}%
\pgfpathlineto{\pgfqpoint{5.494857in}{0.604300in}}%
\pgfpathlineto{\pgfqpoint{5.495413in}{0.602008in}}%
\pgfpathlineto{\pgfqpoint{5.495969in}{0.607232in}}%
\pgfpathlineto{\pgfqpoint{5.496524in}{0.603233in}}%
\pgfpathlineto{\pgfqpoint{5.497080in}{0.603680in}}%
\pgfpathlineto{\pgfqpoint{5.497636in}{0.601319in}}%
\pgfpathlineto{\pgfqpoint{5.498192in}{0.602724in}}%
\pgfpathlineto{\pgfqpoint{5.498748in}{0.603294in}}%
\pgfpathlineto{\pgfqpoint{5.499304in}{0.600870in}}%
\pgfpathlineto{\pgfqpoint{5.499860in}{0.605404in}}%
\pgfpathlineto{\pgfqpoint{5.500416in}{0.602255in}}%
\pgfpathlineto{\pgfqpoint{5.502083in}{0.606792in}}%
\pgfpathlineto{\pgfqpoint{5.502639in}{0.605962in}}%
\pgfpathlineto{\pgfqpoint{5.503751in}{0.600509in}}%
\pgfpathlineto{\pgfqpoint{5.504862in}{0.602033in}}%
\pgfpathlineto{\pgfqpoint{5.505418in}{0.605394in}}%
\pgfpathlineto{\pgfqpoint{5.505974in}{0.604247in}}%
\pgfpathlineto{\pgfqpoint{5.506530in}{0.601178in}}%
\pgfpathlineto{\pgfqpoint{5.507086in}{0.602345in}}%
\pgfpathlineto{\pgfqpoint{5.507642in}{0.602367in}}%
\pgfpathlineto{\pgfqpoint{5.508198in}{0.600274in}}%
\pgfpathlineto{\pgfqpoint{5.509309in}{0.606210in}}%
\pgfpathlineto{\pgfqpoint{5.512089in}{0.601610in}}%
\pgfpathlineto{\pgfqpoint{5.512644in}{0.601805in}}%
\pgfpathlineto{\pgfqpoint{5.513200in}{0.606554in}}%
\pgfpathlineto{\pgfqpoint{5.513756in}{0.602752in}}%
\pgfpathlineto{\pgfqpoint{5.514312in}{0.600884in}}%
\pgfpathlineto{\pgfqpoint{5.515980in}{0.605376in}}%
\pgfpathlineto{\pgfqpoint{5.517091in}{0.601865in}}%
\pgfpathlineto{\pgfqpoint{5.518759in}{0.605139in}}%
\pgfpathlineto{\pgfqpoint{5.519315in}{0.603940in}}%
\pgfpathlineto{\pgfqpoint{5.519871in}{0.607790in}}%
\pgfpathlineto{\pgfqpoint{5.520427in}{0.604400in}}%
\pgfpathlineto{\pgfqpoint{5.520982in}{0.601103in}}%
\pgfpathlineto{\pgfqpoint{5.521538in}{0.606916in}}%
\pgfpathlineto{\pgfqpoint{5.522094in}{0.603651in}}%
\pgfpathlineto{\pgfqpoint{5.524318in}{0.608500in}}%
\pgfpathlineto{\pgfqpoint{5.524873in}{0.604286in}}%
\pgfpathlineto{\pgfqpoint{5.525429in}{0.604629in}}%
\pgfpathlineto{\pgfqpoint{5.525985in}{0.604997in}}%
\pgfpathlineto{\pgfqpoint{5.527653in}{0.602479in}}%
\pgfpathlineto{\pgfqpoint{5.528209in}{0.604091in}}%
\pgfpathlineto{\pgfqpoint{5.529320in}{0.601635in}}%
\pgfpathlineto{\pgfqpoint{5.529876in}{0.604807in}}%
\pgfpathlineto{\pgfqpoint{5.530432in}{0.600423in}}%
\pgfpathlineto{\pgfqpoint{5.530988in}{0.602348in}}%
\pgfpathlineto{\pgfqpoint{5.534323in}{0.603832in}}%
\pgfpathlineto{\pgfqpoint{5.536546in}{0.602176in}}%
\pgfpathlineto{\pgfqpoint{5.538214in}{0.606004in}}%
\pgfpathlineto{\pgfqpoint{5.538770in}{0.604696in}}%
\pgfpathlineto{\pgfqpoint{5.539882in}{0.600512in}}%
\pgfpathlineto{\pgfqpoint{5.541549in}{0.607721in}}%
\pgfpathlineto{\pgfqpoint{5.542105in}{0.602118in}}%
\pgfpathlineto{\pgfqpoint{5.542661in}{0.603235in}}%
\pgfpathlineto{\pgfqpoint{5.543217in}{0.603465in}}%
\pgfpathlineto{\pgfqpoint{5.543773in}{0.600683in}}%
\pgfpathlineto{\pgfqpoint{5.544329in}{0.603204in}}%
\pgfpathlineto{\pgfqpoint{5.545440in}{0.604212in}}%
\pgfpathlineto{\pgfqpoint{5.546552in}{0.602263in}}%
\pgfpathlineto{\pgfqpoint{5.547108in}{0.602983in}}%
\pgfpathlineto{\pgfqpoint{5.547664in}{0.604026in}}%
\pgfpathlineto{\pgfqpoint{5.548220in}{0.608166in}}%
\pgfpathlineto{\pgfqpoint{5.548775in}{0.605525in}}%
\pgfpathlineto{\pgfqpoint{5.549331in}{0.603802in}}%
\pgfpathlineto{\pgfqpoint{5.549887in}{0.607571in}}%
\pgfpathlineto{\pgfqpoint{5.551555in}{0.601238in}}%
\pgfpathlineto{\pgfqpoint{5.552111in}{0.603046in}}%
\pgfpathlineto{\pgfqpoint{5.552666in}{0.601889in}}%
\pgfpathlineto{\pgfqpoint{5.553222in}{0.601148in}}%
\pgfpathlineto{\pgfqpoint{5.553778in}{0.604952in}}%
\pgfpathlineto{\pgfqpoint{5.554334in}{0.600512in}}%
\pgfpathlineto{\pgfqpoint{5.554890in}{0.600989in}}%
\pgfpathlineto{\pgfqpoint{5.555446in}{0.604045in}}%
\pgfpathlineto{\pgfqpoint{5.556002in}{0.603301in}}%
\pgfpathlineto{\pgfqpoint{5.557113in}{0.601643in}}%
\pgfpathlineto{\pgfqpoint{5.557669in}{0.603943in}}%
\pgfpathlineto{\pgfqpoint{5.559337in}{0.600796in}}%
\pgfpathlineto{\pgfqpoint{5.560449in}{0.604917in}}%
\pgfpathlineto{\pgfqpoint{5.561004in}{0.601550in}}%
\pgfpathlineto{\pgfqpoint{5.561560in}{0.602147in}}%
\pgfpathlineto{\pgfqpoint{5.563228in}{0.602243in}}%
\pgfpathlineto{\pgfqpoint{5.564895in}{0.603508in}}%
\pgfpathlineto{\pgfqpoint{5.565451in}{0.600628in}}%
\pgfpathlineto{\pgfqpoint{5.566007in}{0.607003in}}%
\pgfpathlineto{\pgfqpoint{5.566563in}{0.605419in}}%
\pgfpathlineto{\pgfqpoint{5.567675in}{0.601365in}}%
\pgfpathlineto{\pgfqpoint{5.569342in}{0.604544in}}%
\pgfpathlineto{\pgfqpoint{5.570454in}{0.600449in}}%
\pgfpathlineto{\pgfqpoint{5.572677in}{0.608594in}}%
\pgfpathlineto{\pgfqpoint{5.573789in}{0.600907in}}%
\pgfpathlineto{\pgfqpoint{5.574901in}{0.603759in}}%
\pgfpathlineto{\pgfqpoint{5.575457in}{0.602060in}}%
\pgfpathlineto{\pgfqpoint{5.576013in}{0.604167in}}%
\pgfpathlineto{\pgfqpoint{5.577680in}{0.600908in}}%
\pgfpathlineto{\pgfqpoint{5.578236in}{0.601088in}}%
\pgfpathlineto{\pgfqpoint{5.578792in}{0.603902in}}%
\pgfpathlineto{\pgfqpoint{5.579348in}{0.601001in}}%
\pgfpathlineto{\pgfqpoint{5.579904in}{0.600076in}}%
\pgfpathlineto{\pgfqpoint{5.581015in}{0.605637in}}%
\pgfpathlineto{\pgfqpoint{5.581571in}{0.605316in}}%
\pgfpathlineto{\pgfqpoint{5.582127in}{0.604566in}}%
\pgfpathlineto{\pgfqpoint{5.582683in}{0.601665in}}%
\pgfpathlineto{\pgfqpoint{5.583239in}{0.603611in}}%
\pgfpathlineto{\pgfqpoint{5.584351in}{0.600796in}}%
\pgfpathlineto{\pgfqpoint{5.585462in}{0.603057in}}%
\pgfpathlineto{\pgfqpoint{5.586018in}{0.601919in}}%
\pgfpathlineto{\pgfqpoint{5.586574in}{0.604640in}}%
\pgfpathlineto{\pgfqpoint{5.587130in}{0.604515in}}%
\pgfpathlineto{\pgfqpoint{5.587686in}{0.601196in}}%
\pgfpathlineto{\pgfqpoint{5.588242in}{0.602047in}}%
\pgfpathlineto{\pgfqpoint{5.589909in}{0.602989in}}%
\pgfpathlineto{\pgfqpoint{5.590465in}{0.601101in}}%
\pgfpathlineto{\pgfqpoint{5.591021in}{0.603845in}}%
\pgfpathlineto{\pgfqpoint{5.591577in}{0.602949in}}%
\pgfpathlineto{\pgfqpoint{5.592133in}{0.600720in}}%
\pgfpathlineto{\pgfqpoint{5.592688in}{0.601924in}}%
\pgfpathlineto{\pgfqpoint{5.593244in}{0.604792in}}%
\pgfpathlineto{\pgfqpoint{5.593800in}{0.602974in}}%
\pgfpathlineto{\pgfqpoint{5.594356in}{0.602294in}}%
\pgfpathlineto{\pgfqpoint{5.596024in}{0.606525in}}%
\pgfpathlineto{\pgfqpoint{5.597691in}{0.602345in}}%
\pgfpathlineto{\pgfqpoint{5.598803in}{0.600628in}}%
\pgfpathlineto{\pgfqpoint{5.600471in}{0.604849in}}%
\pgfpathlineto{\pgfqpoint{5.601026in}{0.605026in}}%
\pgfpathlineto{\pgfqpoint{5.601582in}{0.602106in}}%
\pgfpathlineto{\pgfqpoint{5.602138in}{0.604739in}}%
\pgfpathlineto{\pgfqpoint{5.602694in}{0.605871in}}%
\pgfpathlineto{\pgfqpoint{5.603806in}{0.601750in}}%
\pgfpathlineto{\pgfqpoint{5.604917in}{0.602562in}}%
\pgfpathlineto{\pgfqpoint{5.606029in}{0.603176in}}%
\pgfpathlineto{\pgfqpoint{5.607141in}{0.608561in}}%
\pgfpathlineto{\pgfqpoint{5.607697in}{0.600609in}}%
\pgfpathlineto{\pgfqpoint{5.608253in}{0.601503in}}%
\pgfpathlineto{\pgfqpoint{5.609920in}{0.601070in}}%
\pgfpathlineto{\pgfqpoint{5.611032in}{0.605035in}}%
\pgfpathlineto{\pgfqpoint{5.611588in}{0.601134in}}%
\pgfpathlineto{\pgfqpoint{5.612144in}{0.602090in}}%
\pgfpathlineto{\pgfqpoint{5.612700in}{0.602316in}}%
\pgfpathlineto{\pgfqpoint{5.614367in}{0.600674in}}%
\pgfpathlineto{\pgfqpoint{5.616035in}{0.603726in}}%
\pgfpathlineto{\pgfqpoint{5.617146in}{0.602224in}}%
\pgfpathlineto{\pgfqpoint{5.617702in}{0.605861in}}%
\pgfpathlineto{\pgfqpoint{5.618258in}{0.602941in}}%
\pgfpathlineto{\pgfqpoint{5.619370in}{0.600225in}}%
\pgfpathlineto{\pgfqpoint{5.619926in}{0.602973in}}%
\pgfpathlineto{\pgfqpoint{5.620482in}{0.601885in}}%
\pgfpathlineto{\pgfqpoint{5.621593in}{0.600868in}}%
\pgfpathlineto{\pgfqpoint{5.622149in}{0.604520in}}%
\pgfpathlineto{\pgfqpoint{5.622705in}{0.602867in}}%
\pgfpathlineto{\pgfqpoint{5.623261in}{0.604280in}}%
\pgfpathlineto{\pgfqpoint{5.623817in}{0.602967in}}%
\pgfpathlineto{\pgfqpoint{5.624373in}{0.603512in}}%
\pgfpathlineto{\pgfqpoint{5.624928in}{0.601949in}}%
\pgfpathlineto{\pgfqpoint{5.626596in}{0.604129in}}%
\pgfpathlineto{\pgfqpoint{5.628819in}{0.601352in}}%
\pgfpathlineto{\pgfqpoint{5.629375in}{0.604392in}}%
\pgfpathlineto{\pgfqpoint{5.629931in}{0.600320in}}%
\pgfpathlineto{\pgfqpoint{5.630487in}{0.600442in}}%
\pgfpathlineto{\pgfqpoint{5.631043in}{0.601454in}}%
\pgfpathlineto{\pgfqpoint{5.631599in}{0.605730in}}%
\pgfpathlineto{\pgfqpoint{5.632155in}{0.603366in}}%
\pgfpathlineto{\pgfqpoint{5.632711in}{0.601243in}}%
\pgfpathlineto{\pgfqpoint{5.633266in}{0.602830in}}%
\pgfpathlineto{\pgfqpoint{5.633822in}{0.603615in}}%
\pgfpathlineto{\pgfqpoint{5.634378in}{0.600219in}}%
\pgfpathlineto{\pgfqpoint{5.634934in}{0.602476in}}%
\pgfpathlineto{\pgfqpoint{5.635490in}{0.601058in}}%
\pgfpathlineto{\pgfqpoint{5.636046in}{0.604596in}}%
\pgfpathlineto{\pgfqpoint{5.636602in}{0.600295in}}%
\pgfpathlineto{\pgfqpoint{5.637157in}{0.604256in}}%
\pgfpathlineto{\pgfqpoint{5.637713in}{0.602718in}}%
\pgfpathlineto{\pgfqpoint{5.638269in}{0.604214in}}%
\pgfpathlineto{\pgfqpoint{5.638825in}{0.603491in}}%
\pgfpathlineto{\pgfqpoint{5.639381in}{0.604216in}}%
\pgfpathlineto{\pgfqpoint{5.639937in}{0.604880in}}%
\pgfpathlineto{\pgfqpoint{5.642160in}{0.601441in}}%
\pgfpathlineto{\pgfqpoint{5.643828in}{0.603782in}}%
\pgfpathlineto{\pgfqpoint{5.646051in}{0.600361in}}%
\pgfpathlineto{\pgfqpoint{5.646607in}{0.603675in}}%
\pgfpathlineto{\pgfqpoint{5.647163in}{0.601716in}}%
\pgfpathlineto{\pgfqpoint{5.648275in}{0.604422in}}%
\pgfpathlineto{\pgfqpoint{5.649386in}{0.601356in}}%
\pgfpathlineto{\pgfqpoint{5.651054in}{0.604723in}}%
\pgfpathlineto{\pgfqpoint{5.652722in}{0.601159in}}%
\pgfpathlineto{\pgfqpoint{5.653833in}{0.607274in}}%
\pgfpathlineto{\pgfqpoint{5.654945in}{0.601189in}}%
\pgfpathlineto{\pgfqpoint{5.656057in}{0.602855in}}%
\pgfpathlineto{\pgfqpoint{5.656613in}{0.601877in}}%
\pgfpathlineto{\pgfqpoint{5.657168in}{0.602280in}}%
\pgfpathlineto{\pgfqpoint{5.658836in}{0.603578in}}%
\pgfpathlineto{\pgfqpoint{5.659948in}{0.601909in}}%
\pgfpathlineto{\pgfqpoint{5.661059in}{0.606654in}}%
\pgfpathlineto{\pgfqpoint{5.662727in}{0.601030in}}%
\pgfpathlineto{\pgfqpoint{5.663283in}{0.607204in}}%
\pgfpathlineto{\pgfqpoint{5.663839in}{0.600887in}}%
\pgfpathlineto{\pgfqpoint{5.664395in}{0.604235in}}%
\pgfpathlineto{\pgfqpoint{5.665506in}{0.602075in}}%
\pgfpathlineto{\pgfqpoint{5.666618in}{0.604536in}}%
\pgfpathlineto{\pgfqpoint{5.667730in}{0.601859in}}%
\pgfpathlineto{\pgfqpoint{5.668286in}{0.603205in}}%
\pgfpathlineto{\pgfqpoint{5.668842in}{0.601927in}}%
\pgfpathlineto{\pgfqpoint{5.669397in}{0.602001in}}%
\pgfpathlineto{\pgfqpoint{5.669953in}{0.600401in}}%
\pgfpathlineto{\pgfqpoint{5.670509in}{0.604092in}}%
\pgfpathlineto{\pgfqpoint{5.671065in}{0.601024in}}%
\pgfpathlineto{\pgfqpoint{5.671621in}{0.603306in}}%
\pgfpathlineto{\pgfqpoint{5.672177in}{0.601694in}}%
\pgfpathlineto{\pgfqpoint{5.673288in}{0.601826in}}%
\pgfpathlineto{\pgfqpoint{5.673844in}{0.603937in}}%
\pgfpathlineto{\pgfqpoint{5.674400in}{0.601823in}}%
\pgfpathlineto{\pgfqpoint{5.675512in}{0.601819in}}%
\pgfpathlineto{\pgfqpoint{5.676068in}{0.603786in}}%
\pgfpathlineto{\pgfqpoint{5.676624in}{0.602149in}}%
\pgfpathlineto{\pgfqpoint{5.678847in}{0.600793in}}%
\pgfpathlineto{\pgfqpoint{5.679403in}{0.602523in}}%
\pgfpathlineto{\pgfqpoint{5.679959in}{0.600959in}}%
\pgfpathlineto{\pgfqpoint{5.682182in}{0.603151in}}%
\pgfpathlineto{\pgfqpoint{5.682738in}{0.602143in}}%
\pgfpathlineto{\pgfqpoint{5.683294in}{0.605950in}}%
\pgfpathlineto{\pgfqpoint{5.683850in}{0.603562in}}%
\pgfpathlineto{\pgfqpoint{5.685517in}{0.601819in}}%
\pgfpathlineto{\pgfqpoint{5.686073in}{0.603411in}}%
\pgfpathlineto{\pgfqpoint{5.686629in}{0.601609in}}%
\pgfpathlineto{\pgfqpoint{5.687185in}{0.602488in}}%
\pgfpathlineto{\pgfqpoint{5.688297in}{0.605235in}}%
\pgfpathlineto{\pgfqpoint{5.689964in}{0.602571in}}%
\pgfpathlineto{\pgfqpoint{5.690520in}{0.601432in}}%
\pgfpathlineto{\pgfqpoint{5.691076in}{0.604972in}}%
\pgfpathlineto{\pgfqpoint{5.691632in}{0.602128in}}%
\pgfpathlineto{\pgfqpoint{5.692188in}{0.604566in}}%
\pgfpathlineto{\pgfqpoint{5.692744in}{0.603855in}}%
\pgfpathlineto{\pgfqpoint{5.693855in}{0.601039in}}%
\pgfpathlineto{\pgfqpoint{5.694411in}{0.601678in}}%
\pgfpathlineto{\pgfqpoint{5.694967in}{0.602270in}}%
\pgfpathlineto{\pgfqpoint{5.695523in}{0.601219in}}%
\pgfpathlineto{\pgfqpoint{5.696079in}{0.601904in}}%
\pgfpathlineto{\pgfqpoint{5.696635in}{0.601721in}}%
\pgfpathlineto{\pgfqpoint{5.697190in}{0.604530in}}%
\pgfpathlineto{\pgfqpoint{5.697746in}{0.603733in}}%
\pgfpathlineto{\pgfqpoint{5.698302in}{0.603719in}}%
\pgfpathlineto{\pgfqpoint{5.698858in}{0.600348in}}%
\pgfpathlineto{\pgfqpoint{5.699970in}{0.604508in}}%
\pgfpathlineto{\pgfqpoint{5.701637in}{0.601700in}}%
\pgfpathlineto{\pgfqpoint{5.702193in}{0.602348in}}%
\pgfpathlineto{\pgfqpoint{5.702749in}{0.600478in}}%
\pgfpathlineto{\pgfqpoint{5.703305in}{0.601899in}}%
\pgfpathlineto{\pgfqpoint{5.704417in}{0.603671in}}%
\pgfpathlineto{\pgfqpoint{5.704972in}{0.601157in}}%
\pgfpathlineto{\pgfqpoint{5.705528in}{0.602787in}}%
\pgfpathlineto{\pgfqpoint{5.706084in}{0.602857in}}%
\pgfpathlineto{\pgfqpoint{5.706640in}{0.601383in}}%
\pgfpathlineto{\pgfqpoint{5.707196in}{0.602362in}}%
\pgfpathlineto{\pgfqpoint{5.707752in}{0.605544in}}%
\pgfpathlineto{\pgfqpoint{5.708308in}{0.603835in}}%
\pgfpathlineto{\pgfqpoint{5.708864in}{0.605230in}}%
\pgfpathlineto{\pgfqpoint{5.709419in}{0.602112in}}%
\pgfpathlineto{\pgfqpoint{5.709975in}{0.602639in}}%
\pgfpathlineto{\pgfqpoint{5.710531in}{0.603989in}}%
\pgfpathlineto{\pgfqpoint{5.712199in}{0.600444in}}%
\pgfpathlineto{\pgfqpoint{5.713866in}{0.604751in}}%
\pgfpathlineto{\pgfqpoint{5.714978in}{0.601138in}}%
\pgfpathlineto{\pgfqpoint{5.715534in}{0.604453in}}%
\pgfpathlineto{\pgfqpoint{5.716090in}{0.601082in}}%
\pgfpathlineto{\pgfqpoint{5.716646in}{0.602479in}}%
\pgfpathlineto{\pgfqpoint{5.717201in}{0.601502in}}%
\pgfpathlineto{\pgfqpoint{5.718313in}{0.602166in}}%
\pgfpathlineto{\pgfqpoint{5.719425in}{0.607159in}}%
\pgfpathlineto{\pgfqpoint{5.719981in}{0.605634in}}%
\pgfpathlineto{\pgfqpoint{5.720537in}{0.602180in}}%
\pgfpathlineto{\pgfqpoint{5.721092in}{0.604372in}}%
\pgfpathlineto{\pgfqpoint{5.721648in}{0.602464in}}%
\pgfpathlineto{\pgfqpoint{5.722204in}{0.609045in}}%
\pgfpathlineto{\pgfqpoint{5.722760in}{0.603302in}}%
\pgfpathlineto{\pgfqpoint{5.723872in}{0.601058in}}%
\pgfpathlineto{\pgfqpoint{5.724428in}{0.602840in}}%
\pgfpathlineto{\pgfqpoint{5.725539in}{0.604231in}}%
\pgfpathlineto{\pgfqpoint{5.727207in}{0.603262in}}%
\pgfpathlineto{\pgfqpoint{5.730542in}{0.601073in}}%
\pgfpathlineto{\pgfqpoint{5.732210in}{0.603317in}}%
\pgfpathlineto{\pgfqpoint{5.733321in}{0.600493in}}%
\pgfpathlineto{\pgfqpoint{5.733877in}{0.602238in}}%
\pgfpathlineto{\pgfqpoint{5.734989in}{0.602103in}}%
\pgfpathlineto{\pgfqpoint{5.735545in}{0.603475in}}%
\pgfpathlineto{\pgfqpoint{5.736101in}{0.602669in}}%
\pgfpathlineto{\pgfqpoint{5.737212in}{0.600503in}}%
\pgfpathlineto{\pgfqpoint{5.737768in}{0.602152in}}%
\pgfpathlineto{\pgfqpoint{5.738880in}{0.603004in}}%
\pgfpathlineto{\pgfqpoint{5.739436in}{0.604224in}}%
\pgfpathlineto{\pgfqpoint{5.739992in}{0.601517in}}%
\pgfpathlineto{\pgfqpoint{5.740548in}{0.602863in}}%
\pgfpathlineto{\pgfqpoint{5.741103in}{0.604881in}}%
\pgfpathlineto{\pgfqpoint{5.742771in}{0.601232in}}%
\pgfpathlineto{\pgfqpoint{5.743327in}{0.600473in}}%
\pgfpathlineto{\pgfqpoint{5.743883in}{0.601844in}}%
\pgfpathlineto{\pgfqpoint{5.744439in}{0.601090in}}%
\pgfpathlineto{\pgfqpoint{5.745550in}{0.601565in}}%
\pgfpathlineto{\pgfqpoint{5.746106in}{0.604241in}}%
\pgfpathlineto{\pgfqpoint{5.746662in}{0.602135in}}%
\pgfpathlineto{\pgfqpoint{5.747218in}{0.600685in}}%
\pgfpathlineto{\pgfqpoint{5.747774in}{0.604428in}}%
\pgfpathlineto{\pgfqpoint{5.748330in}{0.604278in}}%
\pgfpathlineto{\pgfqpoint{5.749441in}{0.602370in}}%
\pgfpathlineto{\pgfqpoint{5.749997in}{0.603975in}}%
\pgfpathlineto{\pgfqpoint{5.750553in}{0.602511in}}%
\pgfpathlineto{\pgfqpoint{5.751109in}{0.603320in}}%
\pgfpathlineto{\pgfqpoint{5.751665in}{0.601338in}}%
\pgfpathlineto{\pgfqpoint{5.752221in}{0.603877in}}%
\pgfpathlineto{\pgfqpoint{5.752777in}{0.601481in}}%
\pgfpathlineto{\pgfqpoint{5.754444in}{0.602889in}}%
\pgfpathlineto{\pgfqpoint{5.755000in}{0.601786in}}%
\pgfpathlineto{\pgfqpoint{5.755556in}{0.604136in}}%
\pgfpathlineto{\pgfqpoint{5.756112in}{0.603697in}}%
\pgfpathlineto{\pgfqpoint{5.756668in}{0.602731in}}%
\pgfpathlineto{\pgfqpoint{5.757223in}{0.605200in}}%
\pgfpathlineto{\pgfqpoint{5.757779in}{0.602672in}}%
\pgfpathlineto{\pgfqpoint{5.759447in}{0.600649in}}%
\pgfpathlineto{\pgfqpoint{5.760003in}{0.601413in}}%
\pgfpathlineto{\pgfqpoint{5.761670in}{0.604036in}}%
\pgfpathlineto{\pgfqpoint{5.762226in}{0.601269in}}%
\pgfpathlineto{\pgfqpoint{5.762782in}{0.604187in}}%
\pgfpathlineto{\pgfqpoint{5.763338in}{0.600550in}}%
\pgfpathlineto{\pgfqpoint{5.763894in}{0.602792in}}%
\pgfpathlineto{\pgfqpoint{5.765006in}{0.601749in}}%
\pgfpathlineto{\pgfqpoint{5.766117in}{0.605941in}}%
\pgfpathlineto{\pgfqpoint{5.766673in}{0.605330in}}%
\pgfpathlineto{\pgfqpoint{5.768341in}{0.603280in}}%
\pgfpathlineto{\pgfqpoint{5.768897in}{0.603347in}}%
\pgfpathlineto{\pgfqpoint{5.770008in}{0.601032in}}%
\pgfpathlineto{\pgfqpoint{5.770564in}{0.603870in}}%
\pgfpathlineto{\pgfqpoint{5.771120in}{0.600853in}}%
\pgfpathlineto{\pgfqpoint{5.771676in}{0.602269in}}%
\pgfpathlineto{\pgfqpoint{5.772232in}{0.603925in}}%
\pgfpathlineto{\pgfqpoint{5.772788in}{0.602837in}}%
\pgfpathlineto{\pgfqpoint{5.775011in}{0.601131in}}%
\pgfpathlineto{\pgfqpoint{5.775567in}{0.605348in}}%
\pgfpathlineto{\pgfqpoint{5.776123in}{0.601594in}}%
\pgfpathlineto{\pgfqpoint{5.777234in}{0.603768in}}%
\pgfpathlineto{\pgfqpoint{5.777790in}{0.600981in}}%
\pgfpathlineto{\pgfqpoint{5.778346in}{0.602427in}}%
\pgfpathlineto{\pgfqpoint{5.779458in}{0.602302in}}%
\pgfpathlineto{\pgfqpoint{5.780014in}{0.600819in}}%
\pgfpathlineto{\pgfqpoint{5.781125in}{0.606493in}}%
\pgfpathlineto{\pgfqpoint{5.781681in}{0.604198in}}%
\pgfpathlineto{\pgfqpoint{5.782793in}{0.602574in}}%
\pgfpathlineto{\pgfqpoint{5.783349in}{0.604395in}}%
\pgfpathlineto{\pgfqpoint{5.783905in}{0.601053in}}%
\pgfpathlineto{\pgfqpoint{5.784461in}{0.604942in}}%
\pgfpathlineto{\pgfqpoint{5.785017in}{0.604329in}}%
\pgfpathlineto{\pgfqpoint{5.786684in}{0.602810in}}%
\pgfpathlineto{\pgfqpoint{5.787796in}{0.600029in}}%
\pgfpathlineto{\pgfqpoint{5.788352in}{0.601853in}}%
\pgfpathlineto{\pgfqpoint{5.788908in}{0.603624in}}%
\pgfpathlineto{\pgfqpoint{5.789463in}{0.603082in}}%
\pgfpathlineto{\pgfqpoint{5.790019in}{0.602561in}}%
\pgfpathlineto{\pgfqpoint{5.790575in}{0.603861in}}%
\pgfpathlineto{\pgfqpoint{5.791131in}{0.600830in}}%
\pgfpathlineto{\pgfqpoint{5.791687in}{0.603070in}}%
\pgfpathlineto{\pgfqpoint{5.792243in}{0.601401in}}%
\pgfpathlineto{\pgfqpoint{5.792799in}{0.602431in}}%
\pgfpathlineto{\pgfqpoint{5.793354in}{0.604159in}}%
\pgfpathlineto{\pgfqpoint{5.794466in}{0.600966in}}%
\pgfpathlineto{\pgfqpoint{5.795022in}{0.601696in}}%
\pgfpathlineto{\pgfqpoint{5.795578in}{0.600849in}}%
\pgfpathlineto{\pgfqpoint{5.801137in}{0.603971in}}%
\pgfpathlineto{\pgfqpoint{5.802804in}{0.601773in}}%
\pgfpathlineto{\pgfqpoint{5.803360in}{0.601770in}}%
\pgfpathlineto{\pgfqpoint{5.803916in}{0.604502in}}%
\pgfpathlineto{\pgfqpoint{5.804472in}{0.602959in}}%
\pgfpathlineto{\pgfqpoint{5.807807in}{0.602026in}}%
\pgfpathlineto{\pgfqpoint{5.808363in}{0.600887in}}%
\pgfpathlineto{\pgfqpoint{5.808919in}{0.601688in}}%
\pgfpathlineto{\pgfqpoint{5.809474in}{0.602653in}}%
\pgfpathlineto{\pgfqpoint{5.811142in}{0.601279in}}%
\pgfpathlineto{\pgfqpoint{5.811698in}{0.606776in}}%
\pgfpathlineto{\pgfqpoint{5.812254in}{0.602007in}}%
\pgfpathlineto{\pgfqpoint{5.812810in}{0.604549in}}%
\pgfpathlineto{\pgfqpoint{5.814477in}{0.601001in}}%
\pgfpathlineto{\pgfqpoint{5.815589in}{0.605054in}}%
\pgfpathlineto{\pgfqpoint{5.816145in}{0.600320in}}%
\pgfpathlineto{\pgfqpoint{5.816701in}{0.601257in}}%
\pgfpathlineto{\pgfqpoint{5.817256in}{0.602786in}}%
\pgfpathlineto{\pgfqpoint{5.817812in}{0.602394in}}%
\pgfpathlineto{\pgfqpoint{5.819480in}{0.600937in}}%
\pgfpathlineto{\pgfqpoint{5.820036in}{0.601979in}}%
\pgfpathlineto{\pgfqpoint{5.820592in}{0.602633in}}%
\pgfpathlineto{\pgfqpoint{5.821148in}{0.602027in}}%
\pgfpathlineto{\pgfqpoint{5.822259in}{0.600993in}}%
\pgfpathlineto{\pgfqpoint{5.823927in}{0.605402in}}%
\pgfpathlineto{\pgfqpoint{5.825594in}{0.601717in}}%
\pgfpathlineto{\pgfqpoint{5.826150in}{0.604373in}}%
\pgfpathlineto{\pgfqpoint{5.826706in}{0.600231in}}%
\pgfpathlineto{\pgfqpoint{5.827262in}{0.602089in}}%
\pgfpathlineto{\pgfqpoint{5.828374in}{0.600428in}}%
\pgfpathlineto{\pgfqpoint{5.828930in}{0.602355in}}%
\pgfpathlineto{\pgfqpoint{5.829485in}{0.600518in}}%
\pgfpathlineto{\pgfqpoint{5.831709in}{0.604609in}}%
\pgfpathlineto{\pgfqpoint{5.833376in}{0.602148in}}%
\pgfpathlineto{\pgfqpoint{5.834488in}{0.606613in}}%
\pgfpathlineto{\pgfqpoint{5.835600in}{0.600850in}}%
\pgfpathlineto{\pgfqpoint{5.837267in}{0.604835in}}%
\pgfpathlineto{\pgfqpoint{5.838935in}{0.601331in}}%
\pgfpathlineto{\pgfqpoint{5.839491in}{0.601879in}}%
\pgfpathlineto{\pgfqpoint{5.840047in}{0.605201in}}%
\pgfpathlineto{\pgfqpoint{5.840603in}{0.602128in}}%
\pgfpathlineto{\pgfqpoint{5.841159in}{0.604436in}}%
\pgfpathlineto{\pgfqpoint{5.841714in}{0.600702in}}%
\pgfpathlineto{\pgfqpoint{5.842270in}{0.603253in}}%
\pgfpathlineto{\pgfqpoint{5.843382in}{0.601829in}}%
\pgfpathlineto{\pgfqpoint{5.843938in}{0.602288in}}%
\pgfpathlineto{\pgfqpoint{5.844494in}{0.601783in}}%
\pgfpathlineto{\pgfqpoint{5.845050in}{0.602705in}}%
\pgfpathlineto{\pgfqpoint{5.845605in}{0.602324in}}%
\pgfpathlineto{\pgfqpoint{5.846161in}{0.603655in}}%
\pgfpathlineto{\pgfqpoint{5.846717in}{0.601401in}}%
\pgfpathlineto{\pgfqpoint{5.847273in}{0.603239in}}%
\pgfpathlineto{\pgfqpoint{5.848385in}{0.603731in}}%
\pgfpathlineto{\pgfqpoint{5.849496in}{0.601744in}}%
\pgfpathlineto{\pgfqpoint{5.850608in}{0.603503in}}%
\pgfpathlineto{\pgfqpoint{5.851164in}{0.600974in}}%
\pgfpathlineto{\pgfqpoint{5.851720in}{0.604694in}}%
\pgfpathlineto{\pgfqpoint{5.852276in}{0.603711in}}%
\pgfpathlineto{\pgfqpoint{5.852832in}{0.601360in}}%
\pgfpathlineto{\pgfqpoint{5.853387in}{0.601887in}}%
\pgfpathlineto{\pgfqpoint{5.853943in}{0.602418in}}%
\pgfpathlineto{\pgfqpoint{5.854499in}{0.600801in}}%
\pgfpathlineto{\pgfqpoint{5.855611in}{0.604513in}}%
\pgfpathlineto{\pgfqpoint{5.857279in}{0.601471in}}%
\pgfpathlineto{\pgfqpoint{5.857834in}{0.602629in}}%
\pgfpathlineto{\pgfqpoint{5.858390in}{0.601907in}}%
\pgfpathlineto{\pgfqpoint{5.858946in}{0.600994in}}%
\pgfpathlineto{\pgfqpoint{5.859502in}{0.603310in}}%
\pgfpathlineto{\pgfqpoint{5.860058in}{0.602778in}}%
\pgfpathlineto{\pgfqpoint{5.860614in}{0.602041in}}%
\pgfpathlineto{\pgfqpoint{5.861170in}{0.603082in}}%
\pgfpathlineto{\pgfqpoint{5.863393in}{0.603973in}}%
\pgfpathlineto{\pgfqpoint{5.863949in}{0.601752in}}%
\pgfpathlineto{\pgfqpoint{5.864505in}{0.603687in}}%
\pgfpathlineto{\pgfqpoint{5.866172in}{0.600739in}}%
\pgfpathlineto{\pgfqpoint{5.867284in}{0.604313in}}%
\pgfpathlineto{\pgfqpoint{5.867840in}{0.603134in}}%
\pgfpathlineto{\pgfqpoint{5.869507in}{0.601172in}}%
\pgfpathlineto{\pgfqpoint{5.870063in}{0.602341in}}%
\pgfpathlineto{\pgfqpoint{5.870619in}{0.605906in}}%
\pgfpathlineto{\pgfqpoint{5.871175in}{0.603757in}}%
\pgfpathlineto{\pgfqpoint{5.871731in}{0.604346in}}%
\pgfpathlineto{\pgfqpoint{5.872843in}{0.603355in}}%
\pgfpathlineto{\pgfqpoint{5.873398in}{0.604954in}}%
\pgfpathlineto{\pgfqpoint{5.874510in}{0.601841in}}%
\pgfpathlineto{\pgfqpoint{5.875066in}{0.602007in}}%
\pgfpathlineto{\pgfqpoint{5.875622in}{0.600998in}}%
\pgfpathlineto{\pgfqpoint{5.876734in}{0.605850in}}%
\pgfpathlineto{\pgfqpoint{5.877290in}{0.603767in}}%
\pgfpathlineto{\pgfqpoint{5.877845in}{0.603836in}}%
\pgfpathlineto{\pgfqpoint{5.878957in}{0.600125in}}%
\pgfpathlineto{\pgfqpoint{5.879513in}{0.600902in}}%
\pgfpathlineto{\pgfqpoint{5.880625in}{0.605992in}}%
\pgfpathlineto{\pgfqpoint{5.881736in}{0.601099in}}%
\pgfpathlineto{\pgfqpoint{5.883404in}{0.603371in}}%
\pgfpathlineto{\pgfqpoint{5.883960in}{0.600728in}}%
\pgfpathlineto{\pgfqpoint{5.884516in}{0.601767in}}%
\pgfpathlineto{\pgfqpoint{5.885072in}{0.601158in}}%
\pgfpathlineto{\pgfqpoint{5.885627in}{0.602084in}}%
\pgfpathlineto{\pgfqpoint{5.886183in}{0.602964in}}%
\pgfpathlineto{\pgfqpoint{5.886739in}{0.601251in}}%
\pgfpathlineto{\pgfqpoint{5.887295in}{0.601918in}}%
\pgfpathlineto{\pgfqpoint{5.888407in}{0.601419in}}%
\pgfpathlineto{\pgfqpoint{5.888963in}{0.602401in}}%
\pgfpathlineto{\pgfqpoint{5.889518in}{0.600945in}}%
\pgfpathlineto{\pgfqpoint{5.890630in}{0.603824in}}%
\pgfpathlineto{\pgfqpoint{5.891186in}{0.602181in}}%
\pgfpathlineto{\pgfqpoint{5.891742in}{0.605839in}}%
\pgfpathlineto{\pgfqpoint{5.892298in}{0.602152in}}%
\pgfpathlineto{\pgfqpoint{5.893965in}{0.600846in}}%
\pgfpathlineto{\pgfqpoint{5.895077in}{0.604939in}}%
\pgfpathlineto{\pgfqpoint{5.895633in}{0.604684in}}%
\pgfpathlineto{\pgfqpoint{5.897301in}{0.601268in}}%
\pgfpathlineto{\pgfqpoint{5.897856in}{0.601883in}}%
\pgfpathlineto{\pgfqpoint{5.898412in}{0.604820in}}%
\pgfpathlineto{\pgfqpoint{5.898968in}{0.600815in}}%
\pgfpathlineto{\pgfqpoint{5.899524in}{0.605015in}}%
\pgfpathlineto{\pgfqpoint{5.900080in}{0.603974in}}%
\pgfpathlineto{\pgfqpoint{5.901192in}{0.604252in}}%
\pgfpathlineto{\pgfqpoint{5.902859in}{0.601736in}}%
\pgfpathlineto{\pgfqpoint{5.903971in}{0.604837in}}%
\pgfpathlineto{\pgfqpoint{5.904527in}{0.602372in}}%
\pgfpathlineto{\pgfqpoint{5.905083in}{0.602913in}}%
\pgfpathlineto{\pgfqpoint{5.906194in}{0.602891in}}%
\pgfpathlineto{\pgfqpoint{5.906750in}{0.600815in}}%
\pgfpathlineto{\pgfqpoint{5.907306in}{0.604069in}}%
\pgfpathlineto{\pgfqpoint{5.907862in}{0.601914in}}%
\pgfpathlineto{\pgfqpoint{5.908418in}{0.600759in}}%
\pgfpathlineto{\pgfqpoint{5.908974in}{0.601971in}}%
\pgfpathlineto{\pgfqpoint{5.909529in}{0.602187in}}%
\pgfpathlineto{\pgfqpoint{5.910085in}{0.600462in}}%
\pgfpathlineto{\pgfqpoint{5.910641in}{0.601413in}}%
\pgfpathlineto{\pgfqpoint{5.911197in}{0.601712in}}%
\pgfpathlineto{\pgfqpoint{5.911753in}{0.600642in}}%
\pgfpathlineto{\pgfqpoint{5.913420in}{0.603657in}}%
\pgfpathlineto{\pgfqpoint{5.913976in}{0.603446in}}%
\pgfpathlineto{\pgfqpoint{5.914532in}{0.600602in}}%
\pgfpathlineto{\pgfqpoint{5.915088in}{0.603437in}}%
\pgfpathlineto{\pgfqpoint{5.916200in}{0.602520in}}%
\pgfpathlineto{\pgfqpoint{5.918423in}{0.605217in}}%
\pgfpathlineto{\pgfqpoint{5.918979in}{0.602414in}}%
\pgfpathlineto{\pgfqpoint{5.919535in}{0.606581in}}%
\pgfpathlineto{\pgfqpoint{5.920091in}{0.602431in}}%
\pgfpathlineto{\pgfqpoint{5.920647in}{0.603437in}}%
\pgfpathlineto{\pgfqpoint{5.921758in}{0.601179in}}%
\pgfpathlineto{\pgfqpoint{5.922314in}{0.603764in}}%
\pgfpathlineto{\pgfqpoint{5.922870in}{0.601044in}}%
\pgfpathlineto{\pgfqpoint{5.923426in}{0.601214in}}%
\pgfpathlineto{\pgfqpoint{5.924538in}{0.607224in}}%
\pgfpathlineto{\pgfqpoint{5.925094in}{0.602449in}}%
\pgfpathlineto{\pgfqpoint{5.925649in}{0.602656in}}%
\pgfpathlineto{\pgfqpoint{5.926761in}{0.603191in}}%
\pgfpathlineto{\pgfqpoint{5.927317in}{0.603863in}}%
\pgfpathlineto{\pgfqpoint{5.927873in}{0.608144in}}%
\pgfpathlineto{\pgfqpoint{5.928429in}{0.603005in}}%
\pgfpathlineto{\pgfqpoint{5.928985in}{0.604923in}}%
\pgfpathlineto{\pgfqpoint{5.930096in}{0.601693in}}%
\pgfpathlineto{\pgfqpoint{5.930652in}{0.606258in}}%
\pgfpathlineto{\pgfqpoint{5.931208in}{0.602520in}}%
\pgfpathlineto{\pgfqpoint{5.931764in}{0.605249in}}%
\pgfpathlineto{\pgfqpoint{5.932320in}{0.603407in}}%
\pgfpathlineto{\pgfqpoint{5.932876in}{0.604182in}}%
\pgfpathlineto{\pgfqpoint{5.933432in}{0.600895in}}%
\pgfpathlineto{\pgfqpoint{5.933987in}{0.601796in}}%
\pgfpathlineto{\pgfqpoint{5.935099in}{0.601979in}}%
\pgfpathlineto{\pgfqpoint{5.935655in}{0.603503in}}%
\pgfpathlineto{\pgfqpoint{5.936211in}{0.602443in}}%
\pgfpathlineto{\pgfqpoint{5.937323in}{0.602210in}}%
\pgfpathlineto{\pgfqpoint{5.937878in}{0.607448in}}%
\pgfpathlineto{\pgfqpoint{5.938990in}{0.601754in}}%
\pgfpathlineto{\pgfqpoint{5.939546in}{0.602471in}}%
\pgfpathlineto{\pgfqpoint{5.940102in}{0.601114in}}%
\pgfpathlineto{\pgfqpoint{5.940658in}{0.604224in}}%
\pgfpathlineto{\pgfqpoint{5.941214in}{0.600146in}}%
\pgfpathlineto{\pgfqpoint{5.941769in}{0.603363in}}%
\pgfpathlineto{\pgfqpoint{5.943993in}{0.600724in}}%
\pgfpathlineto{\pgfqpoint{5.945105in}{0.602339in}}%
\pgfpathlineto{\pgfqpoint{5.946216in}{0.601107in}}%
\pgfpathlineto{\pgfqpoint{5.947328in}{0.603729in}}%
\pgfpathlineto{\pgfqpoint{5.948440in}{0.601386in}}%
\pgfpathlineto{\pgfqpoint{5.948996in}{0.607017in}}%
\pgfpathlineto{\pgfqpoint{5.949551in}{0.603559in}}%
\pgfpathlineto{\pgfqpoint{5.950107in}{0.600070in}}%
\pgfpathlineto{\pgfqpoint{5.950663in}{0.600909in}}%
\pgfpathlineto{\pgfqpoint{5.951219in}{0.600623in}}%
\pgfpathlineto{\pgfqpoint{5.952331in}{0.603012in}}%
\pgfpathlineto{\pgfqpoint{5.952887in}{0.601496in}}%
\pgfpathlineto{\pgfqpoint{5.953443in}{0.604671in}}%
\pgfpathlineto{\pgfqpoint{5.953998in}{0.602553in}}%
\pgfpathlineto{\pgfqpoint{5.955110in}{0.601823in}}%
\pgfpathlineto{\pgfqpoint{5.955666in}{0.601134in}}%
\pgfpathlineto{\pgfqpoint{5.956222in}{0.603162in}}%
\pgfpathlineto{\pgfqpoint{5.956778in}{0.601922in}}%
\pgfpathlineto{\pgfqpoint{5.957334in}{0.602571in}}%
\pgfpathlineto{\pgfqpoint{5.957889in}{0.601490in}}%
\pgfpathlineto{\pgfqpoint{5.958445in}{0.602824in}}%
\pgfpathlineto{\pgfqpoint{5.959557in}{0.603498in}}%
\pgfpathlineto{\pgfqpoint{5.960669in}{0.601191in}}%
\pgfpathlineto{\pgfqpoint{5.962336in}{0.603709in}}%
\pgfpathlineto{\pgfqpoint{5.962892in}{0.602773in}}%
\pgfpathlineto{\pgfqpoint{5.963448in}{0.602266in}}%
\pgfpathlineto{\pgfqpoint{5.964004in}{0.604727in}}%
\pgfpathlineto{\pgfqpoint{5.964560in}{0.602321in}}%
\pgfpathlineto{\pgfqpoint{5.966227in}{0.602666in}}%
\pgfpathlineto{\pgfqpoint{5.966783in}{0.603672in}}%
\pgfpathlineto{\pgfqpoint{5.968451in}{0.601083in}}%
\pgfpathlineto{\pgfqpoint{5.970674in}{0.603281in}}%
\pgfpathlineto{\pgfqpoint{5.972342in}{0.600562in}}%
\pgfpathlineto{\pgfqpoint{5.974565in}{0.602948in}}%
\pgfpathlineto{\pgfqpoint{5.975121in}{0.601537in}}%
\pgfpathlineto{\pgfqpoint{5.975677in}{0.604285in}}%
\pgfpathlineto{\pgfqpoint{5.976233in}{0.603872in}}%
\pgfpathlineto{\pgfqpoint{5.977900in}{0.600852in}}%
\pgfpathlineto{\pgfqpoint{5.979012in}{0.603810in}}%
\pgfpathlineto{\pgfqpoint{5.979568in}{0.602133in}}%
\pgfpathlineto{\pgfqpoint{5.980680in}{0.601130in}}%
\pgfpathlineto{\pgfqpoint{5.981236in}{0.606510in}}%
\pgfpathlineto{\pgfqpoint{5.981791in}{0.604239in}}%
\pgfpathlineto{\pgfqpoint{5.982347in}{0.600436in}}%
\pgfpathlineto{\pgfqpoint{5.982903in}{0.602150in}}%
\pgfpathlineto{\pgfqpoint{5.985127in}{0.601467in}}%
\pgfpathlineto{\pgfqpoint{5.986238in}{0.603755in}}%
\pgfpathlineto{\pgfqpoint{5.986794in}{0.603138in}}%
\pgfpathlineto{\pgfqpoint{5.987350in}{0.602038in}}%
\pgfpathlineto{\pgfqpoint{5.987906in}{0.603596in}}%
\pgfpathlineto{\pgfqpoint{5.988462in}{0.600298in}}%
\pgfpathlineto{\pgfqpoint{5.989018in}{0.601218in}}%
\pgfpathlineto{\pgfqpoint{5.990685in}{0.602901in}}%
\pgfpathlineto{\pgfqpoint{5.991797in}{0.600991in}}%
\pgfpathlineto{\pgfqpoint{5.993465in}{0.602324in}}%
\pgfpathlineto{\pgfqpoint{5.994020in}{0.601257in}}%
\pgfpathlineto{\pgfqpoint{5.995132in}{0.604305in}}%
\pgfpathlineto{\pgfqpoint{5.995688in}{0.603771in}}%
\pgfpathlineto{\pgfqpoint{5.996244in}{0.603635in}}%
\pgfpathlineto{\pgfqpoint{5.996800in}{0.601456in}}%
\pgfpathlineto{\pgfqpoint{5.997356in}{0.602459in}}%
\pgfpathlineto{\pgfqpoint{5.997911in}{0.601973in}}%
\pgfpathlineto{\pgfqpoint{5.998467in}{0.603329in}}%
\pgfpathlineto{\pgfqpoint{5.999579in}{0.600616in}}%
\pgfpathlineto{\pgfqpoint{6.002914in}{0.602652in}}%
\pgfpathlineto{\pgfqpoint{6.004026in}{0.602566in}}%
\pgfpathlineto{\pgfqpoint{6.005138in}{0.602373in}}%
\pgfpathlineto{\pgfqpoint{6.005693in}{0.601279in}}%
\pgfpathlineto{\pgfqpoint{6.006249in}{0.601673in}}%
\pgfpathlineto{\pgfqpoint{6.008473in}{0.603253in}}%
\pgfpathlineto{\pgfqpoint{6.009029in}{0.601089in}}%
\pgfpathlineto{\pgfqpoint{6.009585in}{0.602891in}}%
\pgfpathlineto{\pgfqpoint{6.011808in}{0.601259in}}%
\pgfpathlineto{\pgfqpoint{6.012920in}{0.603659in}}%
\pgfpathlineto{\pgfqpoint{6.013476in}{0.602594in}}%
\pgfpathlineto{\pgfqpoint{6.014587in}{0.601233in}}%
\pgfpathlineto{\pgfqpoint{6.015699in}{0.603112in}}%
\pgfpathlineto{\pgfqpoint{6.016255in}{0.600710in}}%
\pgfpathlineto{\pgfqpoint{6.016811in}{0.600869in}}%
\pgfpathlineto{\pgfqpoint{6.017922in}{0.603416in}}%
\pgfpathlineto{\pgfqpoint{6.018478in}{0.600335in}}%
\pgfpathlineto{\pgfqpoint{6.019034in}{0.603490in}}%
\pgfpathlineto{\pgfqpoint{6.020702in}{0.600615in}}%
\pgfpathlineto{\pgfqpoint{6.021813in}{0.603744in}}%
\pgfpathlineto{\pgfqpoint{6.022925in}{0.600946in}}%
\pgfpathlineto{\pgfqpoint{6.024037in}{0.602985in}}%
\pgfpathlineto{\pgfqpoint{6.024593in}{0.603474in}}%
\pgfpathlineto{\pgfqpoint{6.025149in}{0.600587in}}%
\pgfpathlineto{\pgfqpoint{6.025704in}{0.602597in}}%
\pgfpathlineto{\pgfqpoint{6.026260in}{0.601162in}}%
\pgfpathlineto{\pgfqpoint{6.026816in}{0.604446in}}%
\pgfpathlineto{\pgfqpoint{6.027372in}{0.601963in}}%
\pgfpathlineto{\pgfqpoint{6.027928in}{0.601747in}}%
\pgfpathlineto{\pgfqpoint{6.028484in}{0.600290in}}%
\pgfpathlineto{\pgfqpoint{6.029040in}{0.600640in}}%
\pgfpathlineto{\pgfqpoint{6.029596in}{0.605086in}}%
\pgfpathlineto{\pgfqpoint{6.030151in}{0.603572in}}%
\pgfpathlineto{\pgfqpoint{6.031819in}{0.600642in}}%
\pgfpathlineto{\pgfqpoint{6.034598in}{0.601152in}}%
\pgfpathlineto{\pgfqpoint{6.035154in}{0.600411in}}%
\pgfpathlineto{\pgfqpoint{6.035710in}{0.603398in}}%
\pgfpathlineto{\pgfqpoint{6.036266in}{0.602341in}}%
\pgfpathlineto{\pgfqpoint{6.037933in}{0.600519in}}%
\pgfpathlineto{\pgfqpoint{6.038489in}{0.603084in}}%
\pgfpathlineto{\pgfqpoint{6.039045in}{0.602224in}}%
\pgfpathlineto{\pgfqpoint{6.039601in}{0.602963in}}%
\pgfpathlineto{\pgfqpoint{6.041269in}{0.601221in}}%
\pgfpathlineto{\pgfqpoint{6.043492in}{0.601343in}}%
\pgfpathlineto{\pgfqpoint{6.044048in}{0.603703in}}%
\pgfpathlineto{\pgfqpoint{6.044604in}{0.601942in}}%
\pgfpathlineto{\pgfqpoint{6.046271in}{0.602976in}}%
\pgfpathlineto{\pgfqpoint{6.047383in}{0.600272in}}%
\pgfpathlineto{\pgfqpoint{6.047939in}{0.602143in}}%
\pgfpathlineto{\pgfqpoint{6.048495in}{0.601306in}}%
\pgfpathlineto{\pgfqpoint{6.049051in}{0.605305in}}%
\pgfpathlineto{\pgfqpoint{6.049607in}{0.603439in}}%
\pgfpathlineto{\pgfqpoint{6.050162in}{0.601065in}}%
\pgfpathlineto{\pgfqpoint{6.050718in}{0.602513in}}%
\pgfpathlineto{\pgfqpoint{6.051830in}{0.601135in}}%
\pgfpathlineto{\pgfqpoint{6.053498in}{0.602042in}}%
\pgfpathlineto{\pgfqpoint{6.054609in}{0.600978in}}%
\pgfpathlineto{\pgfqpoint{6.055165in}{0.602591in}}%
\pgfpathlineto{\pgfqpoint{6.056277in}{0.600147in}}%
\pgfpathlineto{\pgfqpoint{6.057944in}{0.603457in}}%
\pgfpathlineto{\pgfqpoint{6.058500in}{0.600840in}}%
\pgfpathlineto{\pgfqpoint{6.059056in}{0.602126in}}%
\pgfpathlineto{\pgfqpoint{6.060724in}{0.603824in}}%
\pgfpathlineto{\pgfqpoint{6.061280in}{0.602186in}}%
\pgfpathlineto{\pgfqpoint{6.061835in}{0.604485in}}%
\pgfpathlineto{\pgfqpoint{6.062391in}{0.603963in}}%
\pgfpathlineto{\pgfqpoint{6.062947in}{0.601076in}}%
\pgfpathlineto{\pgfqpoint{6.063503in}{0.601576in}}%
\pgfpathlineto{\pgfqpoint{6.064059in}{0.604535in}}%
\pgfpathlineto{\pgfqpoint{6.064615in}{0.603168in}}%
\pgfpathlineto{\pgfqpoint{6.065171in}{0.603509in}}%
\pgfpathlineto{\pgfqpoint{6.066282in}{0.600431in}}%
\pgfpathlineto{\pgfqpoint{6.066838in}{0.601273in}}%
\pgfpathlineto{\pgfqpoint{6.068506in}{0.603070in}}%
\pgfpathlineto{\pgfqpoint{6.069618in}{0.601066in}}%
\pgfpathlineto{\pgfqpoint{6.070173in}{0.602411in}}%
\pgfpathlineto{\pgfqpoint{6.070729in}{0.601568in}}%
\pgfpathlineto{\pgfqpoint{6.074620in}{0.602785in}}%
\pgfpathlineto{\pgfqpoint{6.075176in}{0.600575in}}%
\pgfpathlineto{\pgfqpoint{6.075732in}{0.601402in}}%
\pgfpathlineto{\pgfqpoint{6.076288in}{0.601419in}}%
\pgfpathlineto{\pgfqpoint{6.076844in}{0.600307in}}%
\pgfpathlineto{\pgfqpoint{6.077400in}{0.601216in}}%
\pgfpathlineto{\pgfqpoint{6.077955in}{0.600574in}}%
\pgfpathlineto{\pgfqpoint{6.079067in}{0.602923in}}%
\pgfpathlineto{\pgfqpoint{6.079623in}{0.601290in}}%
\pgfpathlineto{\pgfqpoint{6.080179in}{0.602913in}}%
\pgfpathlineto{\pgfqpoint{6.080735in}{0.603263in}}%
\pgfpathlineto{\pgfqpoint{6.081291in}{0.601120in}}%
\pgfpathlineto{\pgfqpoint{6.081846in}{0.602796in}}%
\pgfpathlineto{\pgfqpoint{6.082402in}{0.601370in}}%
\pgfpathlineto{\pgfqpoint{6.083514in}{0.605075in}}%
\pgfpathlineto{\pgfqpoint{6.084626in}{0.600981in}}%
\pgfpathlineto{\pgfqpoint{6.085182in}{0.603436in}}%
\pgfpathlineto{\pgfqpoint{6.085738in}{0.601852in}}%
\pgfpathlineto{\pgfqpoint{6.086293in}{0.601460in}}%
\pgfpathlineto{\pgfqpoint{6.087961in}{0.602290in}}%
\pgfpathlineto{\pgfqpoint{6.088517in}{0.602852in}}%
\pgfpathlineto{\pgfqpoint{6.089629in}{0.600288in}}%
\pgfpathlineto{\pgfqpoint{6.090184in}{0.601278in}}%
\pgfpathlineto{\pgfqpoint{6.090740in}{0.600742in}}%
\pgfpathlineto{\pgfqpoint{6.091296in}{0.603536in}}%
\pgfpathlineto{\pgfqpoint{6.091852in}{0.600825in}}%
\pgfpathlineto{\pgfqpoint{6.092408in}{0.600261in}}%
\pgfpathlineto{\pgfqpoint{6.093520in}{0.604522in}}%
\pgfpathlineto{\pgfqpoint{6.094075in}{0.600790in}}%
\pgfpathlineto{\pgfqpoint{6.094631in}{0.601618in}}%
\pgfpathlineto{\pgfqpoint{6.096299in}{0.602581in}}%
\pgfpathlineto{\pgfqpoint{6.096855in}{0.605054in}}%
\pgfpathlineto{\pgfqpoint{6.098522in}{0.601567in}}%
\pgfpathlineto{\pgfqpoint{6.099078in}{0.604573in}}%
\pgfpathlineto{\pgfqpoint{6.100746in}{0.601060in}}%
\pgfpathlineto{\pgfqpoint{6.101302in}{0.600593in}}%
\pgfpathlineto{\pgfqpoint{6.101857in}{0.604023in}}%
\pgfpathlineto{\pgfqpoint{6.102413in}{0.602315in}}%
\pgfpathlineto{\pgfqpoint{6.104081in}{0.606050in}}%
\pgfpathlineto{\pgfqpoint{6.105193in}{0.601622in}}%
\pgfpathlineto{\pgfqpoint{6.105749in}{0.605561in}}%
\pgfpathlineto{\pgfqpoint{6.106304in}{0.602870in}}%
\pgfpathlineto{\pgfqpoint{6.107416in}{0.601081in}}%
\pgfpathlineto{\pgfqpoint{6.107972in}{0.605454in}}%
\pgfpathlineto{\pgfqpoint{6.108528in}{0.603005in}}%
\pgfpathlineto{\pgfqpoint{6.109084in}{0.600577in}}%
\pgfpathlineto{\pgfqpoint{6.109640in}{0.601906in}}%
\pgfpathlineto{\pgfqpoint{6.110751in}{0.603883in}}%
\pgfpathlineto{\pgfqpoint{6.111307in}{0.601025in}}%
\pgfpathlineto{\pgfqpoint{6.111863in}{0.602015in}}%
\pgfpathlineto{\pgfqpoint{6.114642in}{0.602671in}}%
\pgfpathlineto{\pgfqpoint{6.115198in}{0.600664in}}%
\pgfpathlineto{\pgfqpoint{6.115754in}{0.602070in}}%
\pgfpathlineto{\pgfqpoint{6.116310in}{0.602441in}}%
\pgfpathlineto{\pgfqpoint{6.116866in}{0.601345in}}%
\pgfpathlineto{\pgfqpoint{6.117422in}{0.603039in}}%
\pgfpathlineto{\pgfqpoint{6.117977in}{0.602686in}}%
\pgfpathlineto{\pgfqpoint{6.118533in}{0.600286in}}%
\pgfpathlineto{\pgfqpoint{6.119089in}{0.601493in}}%
\pgfpathlineto{\pgfqpoint{6.120757in}{0.603597in}}%
\pgfpathlineto{\pgfqpoint{6.121313in}{0.603201in}}%
\pgfpathlineto{\pgfqpoint{6.123536in}{0.600631in}}%
\pgfpathlineto{\pgfqpoint{6.124648in}{0.604869in}}%
\pgfpathlineto{\pgfqpoint{6.125204in}{0.602406in}}%
\pgfpathlineto{\pgfqpoint{6.125760in}{0.600162in}}%
\pgfpathlineto{\pgfqpoint{6.126315in}{0.601099in}}%
\pgfpathlineto{\pgfqpoint{6.126871in}{0.600999in}}%
\pgfpathlineto{\pgfqpoint{6.127983in}{0.602677in}}%
\pgfpathlineto{\pgfqpoint{6.128539in}{0.601172in}}%
\pgfpathlineto{\pgfqpoint{6.129095in}{0.602497in}}%
\pgfpathlineto{\pgfqpoint{6.130762in}{0.601264in}}%
\pgfpathlineto{\pgfqpoint{6.131318in}{0.602718in}}%
\pgfpathlineto{\pgfqpoint{6.131874in}{0.601275in}}%
\pgfpathlineto{\pgfqpoint{6.132430in}{0.602338in}}%
\pgfpathlineto{\pgfqpoint{6.132986in}{0.602008in}}%
\pgfpathlineto{\pgfqpoint{6.134653in}{0.601756in}}%
\pgfpathlineto{\pgfqpoint{6.135209in}{0.601863in}}%
\pgfpathlineto{\pgfqpoint{6.135765in}{0.603142in}}%
\pgfpathlineto{\pgfqpoint{6.136321in}{0.602039in}}%
\pgfpathlineto{\pgfqpoint{6.137433in}{0.603393in}}%
\pgfpathlineto{\pgfqpoint{6.137988in}{0.600725in}}%
\pgfpathlineto{\pgfqpoint{6.138544in}{0.602463in}}%
\pgfpathlineto{\pgfqpoint{6.140212in}{0.601015in}}%
\pgfpathlineto{\pgfqpoint{6.141324in}{0.602031in}}%
\pgfpathlineto{\pgfqpoint{6.141880in}{0.600447in}}%
\pgfpathlineto{\pgfqpoint{6.142435in}{0.602925in}}%
\pgfpathlineto{\pgfqpoint{6.142991in}{0.601089in}}%
\pgfpathlineto{\pgfqpoint{6.143547in}{0.600901in}}%
\pgfpathlineto{\pgfqpoint{6.144103in}{0.603112in}}%
\pgfpathlineto{\pgfqpoint{6.144659in}{0.600929in}}%
\pgfpathlineto{\pgfqpoint{6.147438in}{0.602823in}}%
\pgfpathlineto{\pgfqpoint{6.149106in}{0.600993in}}%
\pgfpathlineto{\pgfqpoint{6.150217in}{0.605232in}}%
\pgfpathlineto{\pgfqpoint{6.151885in}{0.601237in}}%
\pgfpathlineto{\pgfqpoint{6.152441in}{0.601144in}}%
\pgfpathlineto{\pgfqpoint{6.152997in}{0.602680in}}%
\pgfpathlineto{\pgfqpoint{6.153553in}{0.601951in}}%
\pgfpathlineto{\pgfqpoint{6.154664in}{0.601387in}}%
\pgfpathlineto{\pgfqpoint{6.156222in}{0.602169in}}%
\pgfpathmoveto{\pgfqpoint{6.156222in}{0.599988in}}%
\pgfpathlineto{\pgfqpoint{0.707889in}{0.599990in}}%
\pgfpathmoveto{\pgfqpoint{0.707889in}{0.602169in}}%
\pgfpathlineto{\pgfqpoint{0.709446in}{0.601387in}}%
\pgfpathlineto{\pgfqpoint{0.713338in}{0.602479in}}%
\pgfpathlineto{\pgfqpoint{0.713893in}{0.605232in}}%
\pgfpathlineto{\pgfqpoint{0.714449in}{0.603268in}}%
\pgfpathlineto{\pgfqpoint{0.715005in}{0.600993in}}%
\pgfpathlineto{\pgfqpoint{0.715561in}{0.601327in}}%
\pgfpathlineto{\pgfqpoint{0.717784in}{0.602210in}}%
\pgfpathlineto{\pgfqpoint{0.719452in}{0.600929in}}%
\pgfpathlineto{\pgfqpoint{0.720008in}{0.603112in}}%
\pgfpathlineto{\pgfqpoint{0.720564in}{0.600901in}}%
\pgfpathlineto{\pgfqpoint{0.721120in}{0.601089in}}%
\pgfpathlineto{\pgfqpoint{0.721675in}{0.602925in}}%
\pgfpathlineto{\pgfqpoint{0.722231in}{0.600447in}}%
\pgfpathlineto{\pgfqpoint{0.722787in}{0.602031in}}%
\pgfpathlineto{\pgfqpoint{0.725011in}{0.601672in}}%
\pgfpathlineto{\pgfqpoint{0.725566in}{0.602463in}}%
\pgfpathlineto{\pgfqpoint{0.726122in}{0.600725in}}%
\pgfpathlineto{\pgfqpoint{0.726678in}{0.603393in}}%
\pgfpathlineto{\pgfqpoint{0.727234in}{0.602682in}}%
\pgfpathlineto{\pgfqpoint{0.727790in}{0.602039in}}%
\pgfpathlineto{\pgfqpoint{0.728346in}{0.603142in}}%
\pgfpathlineto{\pgfqpoint{0.728902in}{0.601863in}}%
\pgfpathlineto{\pgfqpoint{0.732237in}{0.601275in}}%
\pgfpathlineto{\pgfqpoint{0.733904in}{0.602670in}}%
\pgfpathlineto{\pgfqpoint{0.735572in}{0.601172in}}%
\pgfpathlineto{\pgfqpoint{0.736684in}{0.602606in}}%
\pgfpathlineto{\pgfqpoint{0.738351in}{0.600162in}}%
\pgfpathlineto{\pgfqpoint{0.739463in}{0.604869in}}%
\pgfpathlineto{\pgfqpoint{0.740575in}{0.600631in}}%
\pgfpathlineto{\pgfqpoint{0.742798in}{0.603201in}}%
\pgfpathlineto{\pgfqpoint{0.743910in}{0.603265in}}%
\pgfpathlineto{\pgfqpoint{0.745577in}{0.600286in}}%
\pgfpathlineto{\pgfqpoint{0.746689in}{0.603039in}}%
\pgfpathlineto{\pgfqpoint{0.747245in}{0.601345in}}%
\pgfpathlineto{\pgfqpoint{0.747801in}{0.602441in}}%
\pgfpathlineto{\pgfqpoint{0.748913in}{0.600664in}}%
\pgfpathlineto{\pgfqpoint{0.750024in}{0.602688in}}%
\pgfpathlineto{\pgfqpoint{0.752804in}{0.601025in}}%
\pgfpathlineto{\pgfqpoint{0.753360in}{0.603883in}}%
\pgfpathlineto{\pgfqpoint{0.753915in}{0.602520in}}%
\pgfpathlineto{\pgfqpoint{0.755027in}{0.600577in}}%
\pgfpathlineto{\pgfqpoint{0.756139in}{0.605454in}}%
\pgfpathlineto{\pgfqpoint{0.756695in}{0.601081in}}%
\pgfpathlineto{\pgfqpoint{0.757251in}{0.602452in}}%
\pgfpathlineto{\pgfqpoint{0.757806in}{0.602870in}}%
\pgfpathlineto{\pgfqpoint{0.758362in}{0.605561in}}%
\pgfpathlineto{\pgfqpoint{0.758918in}{0.601622in}}%
\pgfpathlineto{\pgfqpoint{0.759474in}{0.603133in}}%
\pgfpathlineto{\pgfqpoint{0.760030in}{0.606050in}}%
\pgfpathlineto{\pgfqpoint{0.760586in}{0.603638in}}%
\pgfpathlineto{\pgfqpoint{0.761697in}{0.602315in}}%
\pgfpathlineto{\pgfqpoint{0.762253in}{0.604023in}}%
\pgfpathlineto{\pgfqpoint{0.762809in}{0.600593in}}%
\pgfpathlineto{\pgfqpoint{0.763365in}{0.601060in}}%
\pgfpathlineto{\pgfqpoint{0.765033in}{0.604573in}}%
\pgfpathlineto{\pgfqpoint{0.765588in}{0.601567in}}%
\pgfpathlineto{\pgfqpoint{0.766144in}{0.602599in}}%
\pgfpathlineto{\pgfqpoint{0.766700in}{0.602330in}}%
\pgfpathlineto{\pgfqpoint{0.767256in}{0.605054in}}%
\pgfpathlineto{\pgfqpoint{0.767812in}{0.602581in}}%
\pgfpathlineto{\pgfqpoint{0.768368in}{0.601346in}}%
\pgfpathlineto{\pgfqpoint{0.768924in}{0.601734in}}%
\pgfpathlineto{\pgfqpoint{0.770035in}{0.600790in}}%
\pgfpathlineto{\pgfqpoint{0.770591in}{0.604522in}}%
\pgfpathlineto{\pgfqpoint{0.771147in}{0.601896in}}%
\pgfpathlineto{\pgfqpoint{0.771703in}{0.600261in}}%
\pgfpathlineto{\pgfqpoint{0.772259in}{0.600825in}}%
\pgfpathlineto{\pgfqpoint{0.772815in}{0.603536in}}%
\pgfpathlineto{\pgfqpoint{0.773371in}{0.600742in}}%
\pgfpathlineto{\pgfqpoint{0.776150in}{0.602290in}}%
\pgfpathlineto{\pgfqpoint{0.777817in}{0.601460in}}%
\pgfpathlineto{\pgfqpoint{0.778929in}{0.603436in}}%
\pgfpathlineto{\pgfqpoint{0.779485in}{0.600981in}}%
\pgfpathlineto{\pgfqpoint{0.780041in}{0.602018in}}%
\pgfpathlineto{\pgfqpoint{0.780597in}{0.605075in}}%
\pgfpathlineto{\pgfqpoint{0.781153in}{0.603051in}}%
\pgfpathlineto{\pgfqpoint{0.782820in}{0.601120in}}%
\pgfpathlineto{\pgfqpoint{0.783376in}{0.603263in}}%
\pgfpathlineto{\pgfqpoint{0.783932in}{0.602913in}}%
\pgfpathlineto{\pgfqpoint{0.784488in}{0.601290in}}%
\pgfpathlineto{\pgfqpoint{0.785044in}{0.602923in}}%
\pgfpathlineto{\pgfqpoint{0.785599in}{0.602622in}}%
\pgfpathlineto{\pgfqpoint{0.787267in}{0.600307in}}%
\pgfpathlineto{\pgfqpoint{0.788379in}{0.601402in}}%
\pgfpathlineto{\pgfqpoint{0.788935in}{0.600575in}}%
\pgfpathlineto{\pgfqpoint{0.790046in}{0.602666in}}%
\pgfpathlineto{\pgfqpoint{0.791158in}{0.602316in}}%
\pgfpathlineto{\pgfqpoint{0.791714in}{0.600765in}}%
\pgfpathlineto{\pgfqpoint{0.792270in}{0.601292in}}%
\pgfpathlineto{\pgfqpoint{0.793937in}{0.602411in}}%
\pgfpathlineto{\pgfqpoint{0.794493in}{0.601066in}}%
\pgfpathlineto{\pgfqpoint{0.795049in}{0.601924in}}%
\pgfpathlineto{\pgfqpoint{0.795605in}{0.603070in}}%
\pgfpathlineto{\pgfqpoint{0.796161in}{0.602703in}}%
\pgfpathlineto{\pgfqpoint{0.796717in}{0.602682in}}%
\pgfpathlineto{\pgfqpoint{0.797828in}{0.600431in}}%
\pgfpathlineto{\pgfqpoint{0.800052in}{0.604535in}}%
\pgfpathlineto{\pgfqpoint{0.801164in}{0.601076in}}%
\pgfpathlineto{\pgfqpoint{0.802275in}{0.604485in}}%
\pgfpathlineto{\pgfqpoint{0.802831in}{0.602186in}}%
\pgfpathlineto{\pgfqpoint{0.803387in}{0.603824in}}%
\pgfpathlineto{\pgfqpoint{0.805055in}{0.602126in}}%
\pgfpathlineto{\pgfqpoint{0.805610in}{0.600840in}}%
\pgfpathlineto{\pgfqpoint{0.806166in}{0.603457in}}%
\pgfpathlineto{\pgfqpoint{0.806722in}{0.602689in}}%
\pgfpathlineto{\pgfqpoint{0.808390in}{0.600210in}}%
\pgfpathlineto{\pgfqpoint{0.808946in}{0.602591in}}%
\pgfpathlineto{\pgfqpoint{0.809502in}{0.600978in}}%
\pgfpathlineto{\pgfqpoint{0.813393in}{0.602513in}}%
\pgfpathlineto{\pgfqpoint{0.813948in}{0.601065in}}%
\pgfpathlineto{\pgfqpoint{0.815060in}{0.605305in}}%
\pgfpathlineto{\pgfqpoint{0.816728in}{0.600272in}}%
\pgfpathlineto{\pgfqpoint{0.817839in}{0.602976in}}%
\pgfpathlineto{\pgfqpoint{0.818395in}{0.602498in}}%
\pgfpathlineto{\pgfqpoint{0.820063in}{0.603703in}}%
\pgfpathlineto{\pgfqpoint{0.821175in}{0.601165in}}%
\pgfpathlineto{\pgfqpoint{0.821730in}{0.602129in}}%
\pgfpathlineto{\pgfqpoint{0.822286in}{0.601112in}}%
\pgfpathlineto{\pgfqpoint{0.823954in}{0.601685in}}%
\pgfpathlineto{\pgfqpoint{0.825622in}{0.603084in}}%
\pgfpathlineto{\pgfqpoint{0.826733in}{0.600335in}}%
\pgfpathlineto{\pgfqpoint{0.828401in}{0.603398in}}%
\pgfpathlineto{\pgfqpoint{0.828957in}{0.600411in}}%
\pgfpathlineto{\pgfqpoint{0.829513in}{0.601152in}}%
\pgfpathlineto{\pgfqpoint{0.830068in}{0.601080in}}%
\pgfpathlineto{\pgfqpoint{0.830624in}{0.602164in}}%
\pgfpathlineto{\pgfqpoint{0.831180in}{0.601587in}}%
\pgfpathlineto{\pgfqpoint{0.832292in}{0.600642in}}%
\pgfpathlineto{\pgfqpoint{0.834515in}{0.605086in}}%
\pgfpathlineto{\pgfqpoint{0.835627in}{0.600290in}}%
\pgfpathlineto{\pgfqpoint{0.837295in}{0.604446in}}%
\pgfpathlineto{\pgfqpoint{0.838962in}{0.600587in}}%
\pgfpathlineto{\pgfqpoint{0.839518in}{0.603474in}}%
\pgfpathlineto{\pgfqpoint{0.840074in}{0.602985in}}%
\pgfpathlineto{\pgfqpoint{0.841741in}{0.601029in}}%
\pgfpathlineto{\pgfqpoint{0.842297in}{0.603744in}}%
\pgfpathlineto{\pgfqpoint{0.842853in}{0.601997in}}%
\pgfpathlineto{\pgfqpoint{0.843409in}{0.600615in}}%
\pgfpathlineto{\pgfqpoint{0.843965in}{0.601406in}}%
\pgfpathlineto{\pgfqpoint{0.845077in}{0.603490in}}%
\pgfpathlineto{\pgfqpoint{0.845633in}{0.600335in}}%
\pgfpathlineto{\pgfqpoint{0.846188in}{0.603416in}}%
\pgfpathlineto{\pgfqpoint{0.847856in}{0.600710in}}%
\pgfpathlineto{\pgfqpoint{0.848412in}{0.603112in}}%
\pgfpathlineto{\pgfqpoint{0.848968in}{0.602232in}}%
\pgfpathlineto{\pgfqpoint{0.850079in}{0.601146in}}%
\pgfpathlineto{\pgfqpoint{0.851191in}{0.603659in}}%
\pgfpathlineto{\pgfqpoint{0.852303in}{0.601259in}}%
\pgfpathlineto{\pgfqpoint{0.852859in}{0.601630in}}%
\pgfpathlineto{\pgfqpoint{0.854526in}{0.602891in}}%
\pgfpathlineto{\pgfqpoint{0.855082in}{0.601089in}}%
\pgfpathlineto{\pgfqpoint{0.856194in}{0.603267in}}%
\pgfpathlineto{\pgfqpoint{0.858417in}{0.601279in}}%
\pgfpathlineto{\pgfqpoint{0.860085in}{0.602566in}}%
\pgfpathlineto{\pgfqpoint{0.861752in}{0.602296in}}%
\pgfpathlineto{\pgfqpoint{0.863420in}{0.602348in}}%
\pgfpathlineto{\pgfqpoint{0.865088in}{0.600549in}}%
\pgfpathlineto{\pgfqpoint{0.865644in}{0.603329in}}%
\pgfpathlineto{\pgfqpoint{0.866199in}{0.601973in}}%
\pgfpathlineto{\pgfqpoint{0.866755in}{0.602459in}}%
\pgfpathlineto{\pgfqpoint{0.867311in}{0.601456in}}%
\pgfpathlineto{\pgfqpoint{0.868979in}{0.604305in}}%
\pgfpathlineto{\pgfqpoint{0.870090in}{0.601257in}}%
\pgfpathlineto{\pgfqpoint{0.870646in}{0.602324in}}%
\pgfpathlineto{\pgfqpoint{0.872314in}{0.600991in}}%
\pgfpathlineto{\pgfqpoint{0.874537in}{0.602807in}}%
\pgfpathlineto{\pgfqpoint{0.875649in}{0.600298in}}%
\pgfpathlineto{\pgfqpoint{0.876205in}{0.603596in}}%
\pgfpathlineto{\pgfqpoint{0.876761in}{0.602038in}}%
\pgfpathlineto{\pgfqpoint{0.877872in}{0.603755in}}%
\pgfpathlineto{\pgfqpoint{0.880096in}{0.601372in}}%
\pgfpathlineto{\pgfqpoint{0.881208in}{0.602150in}}%
\pgfpathlineto{\pgfqpoint{0.881764in}{0.600436in}}%
\pgfpathlineto{\pgfqpoint{0.882875in}{0.606510in}}%
\pgfpathlineto{\pgfqpoint{0.883431in}{0.601130in}}%
\pgfpathlineto{\pgfqpoint{0.883987in}{0.601404in}}%
\pgfpathlineto{\pgfqpoint{0.885099in}{0.603810in}}%
\pgfpathlineto{\pgfqpoint{0.885655in}{0.602161in}}%
\pgfpathlineto{\pgfqpoint{0.886210in}{0.600852in}}%
\pgfpathlineto{\pgfqpoint{0.886766in}{0.602064in}}%
\pgfpathlineto{\pgfqpoint{0.887322in}{0.601873in}}%
\pgfpathlineto{\pgfqpoint{0.888434in}{0.604285in}}%
\pgfpathlineto{\pgfqpoint{0.888990in}{0.601537in}}%
\pgfpathlineto{\pgfqpoint{0.889546in}{0.602948in}}%
\pgfpathlineto{\pgfqpoint{0.892325in}{0.600680in}}%
\pgfpathlineto{\pgfqpoint{0.892881in}{0.600897in}}%
\pgfpathlineto{\pgfqpoint{0.893437in}{0.603281in}}%
\pgfpathlineto{\pgfqpoint{0.893992in}{0.601777in}}%
\pgfpathlineto{\pgfqpoint{0.895660in}{0.601083in}}%
\pgfpathlineto{\pgfqpoint{0.897328in}{0.603672in}}%
\pgfpathlineto{\pgfqpoint{0.898439in}{0.601400in}}%
\pgfpathlineto{\pgfqpoint{0.898995in}{0.602201in}}%
\pgfpathlineto{\pgfqpoint{0.899551in}{0.602321in}}%
\pgfpathlineto{\pgfqpoint{0.900107in}{0.604727in}}%
\pgfpathlineto{\pgfqpoint{0.900663in}{0.602266in}}%
\pgfpathlineto{\pgfqpoint{0.902330in}{0.603861in}}%
\pgfpathlineto{\pgfqpoint{0.903442in}{0.601191in}}%
\pgfpathlineto{\pgfqpoint{0.903998in}{0.601638in}}%
\pgfpathlineto{\pgfqpoint{0.904554in}{0.603498in}}%
\pgfpathlineto{\pgfqpoint{0.905110in}{0.603228in}}%
\pgfpathlineto{\pgfqpoint{0.907333in}{0.601922in}}%
\pgfpathlineto{\pgfqpoint{0.907889in}{0.603162in}}%
\pgfpathlineto{\pgfqpoint{0.908445in}{0.601134in}}%
\pgfpathlineto{\pgfqpoint{0.909001in}{0.601823in}}%
\pgfpathlineto{\pgfqpoint{0.910668in}{0.604671in}}%
\pgfpathlineto{\pgfqpoint{0.911224in}{0.601496in}}%
\pgfpathlineto{\pgfqpoint{0.911780in}{0.603012in}}%
\pgfpathlineto{\pgfqpoint{0.912336in}{0.602633in}}%
\pgfpathlineto{\pgfqpoint{0.914003in}{0.600070in}}%
\pgfpathlineto{\pgfqpoint{0.915115in}{0.607017in}}%
\pgfpathlineto{\pgfqpoint{0.915671in}{0.601386in}}%
\pgfpathlineto{\pgfqpoint{0.916227in}{0.602676in}}%
\pgfpathlineto{\pgfqpoint{0.916783in}{0.603729in}}%
\pgfpathlineto{\pgfqpoint{0.917339in}{0.603166in}}%
\pgfpathlineto{\pgfqpoint{0.918450in}{0.601083in}}%
\pgfpathlineto{\pgfqpoint{0.919562in}{0.602443in}}%
\pgfpathlineto{\pgfqpoint{0.920118in}{0.600724in}}%
\pgfpathlineto{\pgfqpoint{0.920674in}{0.601689in}}%
\pgfpathlineto{\pgfqpoint{0.922341in}{0.603363in}}%
\pgfpathlineto{\pgfqpoint{0.922897in}{0.600146in}}%
\pgfpathlineto{\pgfqpoint{0.923453in}{0.604224in}}%
\pgfpathlineto{\pgfqpoint{0.924009in}{0.601114in}}%
\pgfpathlineto{\pgfqpoint{0.924565in}{0.602471in}}%
\pgfpathlineto{\pgfqpoint{0.925121in}{0.601754in}}%
\pgfpathlineto{\pgfqpoint{0.925677in}{0.601938in}}%
\pgfpathlineto{\pgfqpoint{0.926232in}{0.607448in}}%
\pgfpathlineto{\pgfqpoint{0.926788in}{0.602210in}}%
\pgfpathlineto{\pgfqpoint{0.927900in}{0.602443in}}%
\pgfpathlineto{\pgfqpoint{0.928456in}{0.603503in}}%
\pgfpathlineto{\pgfqpoint{0.930123in}{0.601796in}}%
\pgfpathlineto{\pgfqpoint{0.930679in}{0.600895in}}%
\pgfpathlineto{\pgfqpoint{0.932347in}{0.605249in}}%
\pgfpathlineto{\pgfqpoint{0.932903in}{0.602520in}}%
\pgfpathlineto{\pgfqpoint{0.933459in}{0.606258in}}%
\pgfpathlineto{\pgfqpoint{0.934014in}{0.601693in}}%
\pgfpathlineto{\pgfqpoint{0.934570in}{0.603997in}}%
\pgfpathlineto{\pgfqpoint{0.935126in}{0.604923in}}%
\pgfpathlineto{\pgfqpoint{0.935682in}{0.603005in}}%
\pgfpathlineto{\pgfqpoint{0.936238in}{0.608144in}}%
\pgfpathlineto{\pgfqpoint{0.936794in}{0.603863in}}%
\pgfpathlineto{\pgfqpoint{0.939017in}{0.602449in}}%
\pgfpathlineto{\pgfqpoint{0.939573in}{0.607224in}}%
\pgfpathlineto{\pgfqpoint{0.940129in}{0.602843in}}%
\pgfpathlineto{\pgfqpoint{0.941241in}{0.601044in}}%
\pgfpathlineto{\pgfqpoint{0.941797in}{0.603764in}}%
\pgfpathlineto{\pgfqpoint{0.942352in}{0.601179in}}%
\pgfpathlineto{\pgfqpoint{0.942908in}{0.601516in}}%
\pgfpathlineto{\pgfqpoint{0.944576in}{0.606581in}}%
\pgfpathlineto{\pgfqpoint{0.945132in}{0.602414in}}%
\pgfpathlineto{\pgfqpoint{0.945688in}{0.605217in}}%
\pgfpathlineto{\pgfqpoint{0.946799in}{0.602763in}}%
\pgfpathlineto{\pgfqpoint{0.947355in}{0.603325in}}%
\pgfpathlineto{\pgfqpoint{0.947911in}{0.602520in}}%
\pgfpathlineto{\pgfqpoint{0.949023in}{0.603437in}}%
\pgfpathlineto{\pgfqpoint{0.949579in}{0.600602in}}%
\pgfpathlineto{\pgfqpoint{0.950134in}{0.603446in}}%
\pgfpathlineto{\pgfqpoint{0.950690in}{0.603657in}}%
\pgfpathlineto{\pgfqpoint{0.952358in}{0.600642in}}%
\pgfpathlineto{\pgfqpoint{0.952914in}{0.601712in}}%
\pgfpathlineto{\pgfqpoint{0.953470in}{0.601413in}}%
\pgfpathlineto{\pgfqpoint{0.954025in}{0.600462in}}%
\pgfpathlineto{\pgfqpoint{0.954581in}{0.602187in}}%
\pgfpathlineto{\pgfqpoint{0.955137in}{0.601971in}}%
\pgfpathlineto{\pgfqpoint{0.955693in}{0.600759in}}%
\pgfpathlineto{\pgfqpoint{0.956249in}{0.601914in}}%
\pgfpathlineto{\pgfqpoint{0.956805in}{0.604069in}}%
\pgfpathlineto{\pgfqpoint{0.957361in}{0.600815in}}%
\pgfpathlineto{\pgfqpoint{0.957917in}{0.602891in}}%
\pgfpathlineto{\pgfqpoint{0.959584in}{0.602372in}}%
\pgfpathlineto{\pgfqpoint{0.960140in}{0.604837in}}%
\pgfpathlineto{\pgfqpoint{0.960696in}{0.603123in}}%
\pgfpathlineto{\pgfqpoint{0.961252in}{0.601736in}}%
\pgfpathlineto{\pgfqpoint{0.961808in}{0.602563in}}%
\pgfpathlineto{\pgfqpoint{0.962363in}{0.602621in}}%
\pgfpathlineto{\pgfqpoint{0.963475in}{0.604609in}}%
\pgfpathlineto{\pgfqpoint{0.964031in}{0.603974in}}%
\pgfpathlineto{\pgfqpoint{0.964587in}{0.605015in}}%
\pgfpathlineto{\pgfqpoint{0.965143in}{0.600815in}}%
\pgfpathlineto{\pgfqpoint{0.965699in}{0.604820in}}%
\pgfpathlineto{\pgfqpoint{0.966810in}{0.601268in}}%
\pgfpathlineto{\pgfqpoint{0.967366in}{0.601683in}}%
\pgfpathlineto{\pgfqpoint{0.969034in}{0.604939in}}%
\pgfpathlineto{\pgfqpoint{0.970145in}{0.600846in}}%
\pgfpathlineto{\pgfqpoint{0.970701in}{0.602978in}}%
\pgfpathlineto{\pgfqpoint{0.971813in}{0.602152in}}%
\pgfpathlineto{\pgfqpoint{0.972369in}{0.605839in}}%
\pgfpathlineto{\pgfqpoint{0.972925in}{0.602181in}}%
\pgfpathlineto{\pgfqpoint{0.973481in}{0.603824in}}%
\pgfpathlineto{\pgfqpoint{0.974036in}{0.603010in}}%
\pgfpathlineto{\pgfqpoint{0.974592in}{0.600945in}}%
\pgfpathlineto{\pgfqpoint{0.975148in}{0.602401in}}%
\pgfpathlineto{\pgfqpoint{0.976260in}{0.601562in}}%
\pgfpathlineto{\pgfqpoint{0.977928in}{0.602964in}}%
\pgfpathlineto{\pgfqpoint{0.978483in}{0.602084in}}%
\pgfpathlineto{\pgfqpoint{0.980151in}{0.600728in}}%
\pgfpathlineto{\pgfqpoint{0.980707in}{0.603371in}}%
\pgfpathlineto{\pgfqpoint{0.981263in}{0.601789in}}%
\pgfpathlineto{\pgfqpoint{0.982930in}{0.601358in}}%
\pgfpathlineto{\pgfqpoint{0.983486in}{0.605992in}}%
\pgfpathlineto{\pgfqpoint{0.984042in}{0.602800in}}%
\pgfpathlineto{\pgfqpoint{0.985154in}{0.600125in}}%
\pgfpathlineto{\pgfqpoint{0.987377in}{0.605850in}}%
\pgfpathlineto{\pgfqpoint{0.988489in}{0.600998in}}%
\pgfpathlineto{\pgfqpoint{0.989045in}{0.602007in}}%
\pgfpathlineto{\pgfqpoint{0.990156in}{0.602229in}}%
\pgfpathlineto{\pgfqpoint{0.990712in}{0.604954in}}%
\pgfpathlineto{\pgfqpoint{0.991268in}{0.603355in}}%
\pgfpathlineto{\pgfqpoint{0.991824in}{0.603151in}}%
\pgfpathlineto{\pgfqpoint{0.993492in}{0.605906in}}%
\pgfpathlineto{\pgfqpoint{0.994603in}{0.601172in}}%
\pgfpathlineto{\pgfqpoint{0.995159in}{0.601790in}}%
\pgfpathlineto{\pgfqpoint{0.996827in}{0.604313in}}%
\pgfpathlineto{\pgfqpoint{0.997939in}{0.600739in}}%
\pgfpathlineto{\pgfqpoint{0.999606in}{0.603687in}}%
\pgfpathlineto{\pgfqpoint{1.000162in}{0.601752in}}%
\pgfpathlineto{\pgfqpoint{1.000718in}{0.603973in}}%
\pgfpathlineto{\pgfqpoint{1.001274in}{0.603639in}}%
\pgfpathlineto{\pgfqpoint{1.001830in}{0.602328in}}%
\pgfpathlineto{\pgfqpoint{1.002385in}{0.603250in}}%
\pgfpathlineto{\pgfqpoint{1.004609in}{0.603310in}}%
\pgfpathlineto{\pgfqpoint{1.005165in}{0.600994in}}%
\pgfpathlineto{\pgfqpoint{1.005721in}{0.601907in}}%
\pgfpathlineto{\pgfqpoint{1.006276in}{0.602629in}}%
\pgfpathlineto{\pgfqpoint{1.007944in}{0.601531in}}%
\pgfpathlineto{\pgfqpoint{1.008500in}{0.604513in}}%
\pgfpathlineto{\pgfqpoint{1.009056in}{0.602896in}}%
\pgfpathlineto{\pgfqpoint{1.009612in}{0.600801in}}%
\pgfpathlineto{\pgfqpoint{1.010167in}{0.602418in}}%
\pgfpathlineto{\pgfqpoint{1.011279in}{0.601360in}}%
\pgfpathlineto{\pgfqpoint{1.012391in}{0.604694in}}%
\pgfpathlineto{\pgfqpoint{1.012947in}{0.600974in}}%
\pgfpathlineto{\pgfqpoint{1.013503in}{0.603503in}}%
\pgfpathlineto{\pgfqpoint{1.014059in}{0.603355in}}%
\pgfpathlineto{\pgfqpoint{1.015170in}{0.601728in}}%
\pgfpathlineto{\pgfqpoint{1.015726in}{0.603731in}}%
\pgfpathlineto{\pgfqpoint{1.016282in}{0.603417in}}%
\pgfpathlineto{\pgfqpoint{1.016838in}{0.603239in}}%
\pgfpathlineto{\pgfqpoint{1.017394in}{0.601401in}}%
\pgfpathlineto{\pgfqpoint{1.017950in}{0.603655in}}%
\pgfpathlineto{\pgfqpoint{1.018505in}{0.602324in}}%
\pgfpathlineto{\pgfqpoint{1.020173in}{0.602288in}}%
\pgfpathlineto{\pgfqpoint{1.021285in}{0.602095in}}%
\pgfpathlineto{\pgfqpoint{1.021841in}{0.603253in}}%
\pgfpathlineto{\pgfqpoint{1.022396in}{0.600702in}}%
\pgfpathlineto{\pgfqpoint{1.024064in}{0.605201in}}%
\pgfpathlineto{\pgfqpoint{1.025176in}{0.601331in}}%
\pgfpathlineto{\pgfqpoint{1.026843in}{0.604835in}}%
\pgfpathlineto{\pgfqpoint{1.028511in}{0.600850in}}%
\pgfpathlineto{\pgfqpoint{1.029623in}{0.606613in}}%
\pgfpathlineto{\pgfqpoint{1.030178in}{0.603480in}}%
\pgfpathlineto{\pgfqpoint{1.030734in}{0.602148in}}%
\pgfpathlineto{\pgfqpoint{1.032402in}{0.604609in}}%
\pgfpathlineto{\pgfqpoint{1.034625in}{0.600518in}}%
\pgfpathlineto{\pgfqpoint{1.035181in}{0.602355in}}%
\pgfpathlineto{\pgfqpoint{1.035737in}{0.600428in}}%
\pgfpathlineto{\pgfqpoint{1.036849in}{0.602089in}}%
\pgfpathlineto{\pgfqpoint{1.037405in}{0.600231in}}%
\pgfpathlineto{\pgfqpoint{1.037961in}{0.604373in}}%
\pgfpathlineto{\pgfqpoint{1.038516in}{0.601717in}}%
\pgfpathlineto{\pgfqpoint{1.041296in}{0.604775in}}%
\pgfpathlineto{\pgfqpoint{1.041852in}{0.600993in}}%
\pgfpathlineto{\pgfqpoint{1.042407in}{0.602054in}}%
\pgfpathlineto{\pgfqpoint{1.047410in}{0.601257in}}%
\pgfpathlineto{\pgfqpoint{1.047966in}{0.600320in}}%
\pgfpathlineto{\pgfqpoint{1.048522in}{0.605054in}}%
\pgfpathlineto{\pgfqpoint{1.049078in}{0.602740in}}%
\pgfpathlineto{\pgfqpoint{1.050745in}{0.601114in}}%
\pgfpathlineto{\pgfqpoint{1.052413in}{0.606776in}}%
\pgfpathlineto{\pgfqpoint{1.054081in}{0.601254in}}%
\pgfpathlineto{\pgfqpoint{1.054636in}{0.602653in}}%
\pgfpathlineto{\pgfqpoint{1.055192in}{0.601688in}}%
\pgfpathlineto{\pgfqpoint{1.055748in}{0.600887in}}%
\pgfpathlineto{\pgfqpoint{1.056304in}{0.602026in}}%
\pgfpathlineto{\pgfqpoint{1.056860in}{0.603479in}}%
\pgfpathlineto{\pgfqpoint{1.057416in}{0.601963in}}%
\pgfpathlineto{\pgfqpoint{1.059083in}{0.602878in}}%
\pgfpathlineto{\pgfqpoint{1.059639in}{0.602959in}}%
\pgfpathlineto{\pgfqpoint{1.060195in}{0.604502in}}%
\pgfpathlineto{\pgfqpoint{1.061307in}{0.601773in}}%
\pgfpathlineto{\pgfqpoint{1.062974in}{0.603971in}}%
\pgfpathlineto{\pgfqpoint{1.063530in}{0.602434in}}%
\pgfpathlineto{\pgfqpoint{1.064086in}{0.603701in}}%
\pgfpathlineto{\pgfqpoint{1.065754in}{0.603189in}}%
\pgfpathlineto{\pgfqpoint{1.066309in}{0.600970in}}%
\pgfpathlineto{\pgfqpoint{1.066865in}{0.601872in}}%
\pgfpathlineto{\pgfqpoint{1.067421in}{0.602134in}}%
\pgfpathlineto{\pgfqpoint{1.068533in}{0.600849in}}%
\pgfpathlineto{\pgfqpoint{1.070756in}{0.604159in}}%
\pgfpathlineto{\pgfqpoint{1.071868in}{0.601401in}}%
\pgfpathlineto{\pgfqpoint{1.072424in}{0.603070in}}%
\pgfpathlineto{\pgfqpoint{1.072980in}{0.600830in}}%
\pgfpathlineto{\pgfqpoint{1.073536in}{0.603861in}}%
\pgfpathlineto{\pgfqpoint{1.074092in}{0.602561in}}%
\pgfpathlineto{\pgfqpoint{1.075203in}{0.603624in}}%
\pgfpathlineto{\pgfqpoint{1.076315in}{0.600029in}}%
\pgfpathlineto{\pgfqpoint{1.077983in}{0.602792in}}%
\pgfpathlineto{\pgfqpoint{1.078538in}{0.602761in}}%
\pgfpathlineto{\pgfqpoint{1.079650in}{0.604942in}}%
\pgfpathlineto{\pgfqpoint{1.080206in}{0.601053in}}%
\pgfpathlineto{\pgfqpoint{1.080762in}{0.604395in}}%
\pgfpathlineto{\pgfqpoint{1.081318in}{0.602574in}}%
\pgfpathlineto{\pgfqpoint{1.081874in}{0.603754in}}%
\pgfpathlineto{\pgfqpoint{1.082429in}{0.604198in}}%
\pgfpathlineto{\pgfqpoint{1.082985in}{0.606493in}}%
\pgfpathlineto{\pgfqpoint{1.084097in}{0.600819in}}%
\pgfpathlineto{\pgfqpoint{1.084653in}{0.602302in}}%
\pgfpathlineto{\pgfqpoint{1.085765in}{0.602427in}}%
\pgfpathlineto{\pgfqpoint{1.086320in}{0.600981in}}%
\pgfpathlineto{\pgfqpoint{1.086876in}{0.603768in}}%
\pgfpathlineto{\pgfqpoint{1.087432in}{0.603049in}}%
\pgfpathlineto{\pgfqpoint{1.087988in}{0.601594in}}%
\pgfpathlineto{\pgfqpoint{1.088544in}{0.605348in}}%
\pgfpathlineto{\pgfqpoint{1.089100in}{0.601131in}}%
\pgfpathlineto{\pgfqpoint{1.089656in}{0.601314in}}%
\pgfpathlineto{\pgfqpoint{1.090212in}{0.603590in}}%
\pgfpathlineto{\pgfqpoint{1.090767in}{0.602653in}}%
\pgfpathlineto{\pgfqpoint{1.092435in}{0.602269in}}%
\pgfpathlineto{\pgfqpoint{1.092991in}{0.600853in}}%
\pgfpathlineto{\pgfqpoint{1.093547in}{0.603870in}}%
\pgfpathlineto{\pgfqpoint{1.094103in}{0.601032in}}%
\pgfpathlineto{\pgfqpoint{1.096326in}{0.603525in}}%
\pgfpathlineto{\pgfqpoint{1.097438in}{0.605330in}}%
\pgfpathlineto{\pgfqpoint{1.097994in}{0.605941in}}%
\pgfpathlineto{\pgfqpoint{1.099105in}{0.601749in}}%
\pgfpathlineto{\pgfqpoint{1.099661in}{0.602062in}}%
\pgfpathlineto{\pgfqpoint{1.100217in}{0.602792in}}%
\pgfpathlineto{\pgfqpoint{1.100773in}{0.600550in}}%
\pgfpathlineto{\pgfqpoint{1.101329in}{0.604187in}}%
\pgfpathlineto{\pgfqpoint{1.101885in}{0.601269in}}%
\pgfpathlineto{\pgfqpoint{1.102440in}{0.604036in}}%
\pgfpathlineto{\pgfqpoint{1.102996in}{0.602598in}}%
\pgfpathlineto{\pgfqpoint{1.104664in}{0.600649in}}%
\pgfpathlineto{\pgfqpoint{1.106887in}{0.605200in}}%
\pgfpathlineto{\pgfqpoint{1.107443in}{0.602731in}}%
\pgfpathlineto{\pgfqpoint{1.107999in}{0.603697in}}%
\pgfpathlineto{\pgfqpoint{1.108555in}{0.604136in}}%
\pgfpathlineto{\pgfqpoint{1.110223in}{0.601796in}}%
\pgfpathlineto{\pgfqpoint{1.111890in}{0.603877in}}%
\pgfpathlineto{\pgfqpoint{1.112446in}{0.601338in}}%
\pgfpathlineto{\pgfqpoint{1.113002in}{0.603320in}}%
\pgfpathlineto{\pgfqpoint{1.113558in}{0.602511in}}%
\pgfpathlineto{\pgfqpoint{1.114114in}{0.603975in}}%
\pgfpathlineto{\pgfqpoint{1.114669in}{0.602370in}}%
\pgfpathlineto{\pgfqpoint{1.116337in}{0.604428in}}%
\pgfpathlineto{\pgfqpoint{1.116893in}{0.600685in}}%
\pgfpathlineto{\pgfqpoint{1.117449in}{0.602135in}}%
\pgfpathlineto{\pgfqpoint{1.118005in}{0.604241in}}%
\pgfpathlineto{\pgfqpoint{1.119672in}{0.601090in}}%
\pgfpathlineto{\pgfqpoint{1.120228in}{0.601844in}}%
\pgfpathlineto{\pgfqpoint{1.120784in}{0.600473in}}%
\pgfpathlineto{\pgfqpoint{1.121340in}{0.601232in}}%
\pgfpathlineto{\pgfqpoint{1.123007in}{0.604881in}}%
\pgfpathlineto{\pgfqpoint{1.124119in}{0.601517in}}%
\pgfpathlineto{\pgfqpoint{1.124675in}{0.604224in}}%
\pgfpathlineto{\pgfqpoint{1.125231in}{0.603004in}}%
\pgfpathlineto{\pgfqpoint{1.126898in}{0.600503in}}%
\pgfpathlineto{\pgfqpoint{1.128566in}{0.603475in}}%
\pgfpathlineto{\pgfqpoint{1.130789in}{0.600493in}}%
\pgfpathlineto{\pgfqpoint{1.131901in}{0.603317in}}%
\pgfpathlineto{\pgfqpoint{1.132457in}{0.602122in}}%
\pgfpathlineto{\pgfqpoint{1.133013in}{0.602362in}}%
\pgfpathlineto{\pgfqpoint{1.134680in}{0.600467in}}%
\pgfpathlineto{\pgfqpoint{1.136348in}{0.602851in}}%
\pgfpathlineto{\pgfqpoint{1.139127in}{0.603561in}}%
\pgfpathlineto{\pgfqpoint{1.140239in}{0.601058in}}%
\pgfpathlineto{\pgfqpoint{1.140795in}{0.602727in}}%
\pgfpathlineto{\pgfqpoint{1.141351in}{0.603302in}}%
\pgfpathlineto{\pgfqpoint{1.141907in}{0.609045in}}%
\pgfpathlineto{\pgfqpoint{1.143574in}{0.602180in}}%
\pgfpathlineto{\pgfqpoint{1.144686in}{0.607159in}}%
\pgfpathlineto{\pgfqpoint{1.146354in}{0.601410in}}%
\pgfpathlineto{\pgfqpoint{1.148021in}{0.601082in}}%
\pgfpathlineto{\pgfqpoint{1.148577in}{0.604453in}}%
\pgfpathlineto{\pgfqpoint{1.149133in}{0.601138in}}%
\pgfpathlineto{\pgfqpoint{1.150245in}{0.604751in}}%
\pgfpathlineto{\pgfqpoint{1.150800in}{0.603396in}}%
\pgfpathlineto{\pgfqpoint{1.151356in}{0.603583in}}%
\pgfpathlineto{\pgfqpoint{1.151912in}{0.600444in}}%
\pgfpathlineto{\pgfqpoint{1.152468in}{0.601136in}}%
\pgfpathlineto{\pgfqpoint{1.156359in}{0.605544in}}%
\pgfpathlineto{\pgfqpoint{1.157471in}{0.601383in}}%
\pgfpathlineto{\pgfqpoint{1.158582in}{0.602787in}}%
\pgfpathlineto{\pgfqpoint{1.159138in}{0.601157in}}%
\pgfpathlineto{\pgfqpoint{1.159694in}{0.603671in}}%
\pgfpathlineto{\pgfqpoint{1.160250in}{0.603244in}}%
\pgfpathlineto{\pgfqpoint{1.161362in}{0.600478in}}%
\pgfpathlineto{\pgfqpoint{1.163029in}{0.602580in}}%
\pgfpathlineto{\pgfqpoint{1.163585in}{0.602774in}}%
\pgfpathlineto{\pgfqpoint{1.164697in}{0.604385in}}%
\pgfpathlineto{\pgfqpoint{1.165253in}{0.600348in}}%
\pgfpathlineto{\pgfqpoint{1.165809in}{0.603719in}}%
\pgfpathlineto{\pgfqpoint{1.166920in}{0.604530in}}%
\pgfpathlineto{\pgfqpoint{1.168588in}{0.601219in}}%
\pgfpathlineto{\pgfqpoint{1.169144in}{0.602270in}}%
\pgfpathlineto{\pgfqpoint{1.169700in}{0.601678in}}%
\pgfpathlineto{\pgfqpoint{1.170256in}{0.601039in}}%
\pgfpathlineto{\pgfqpoint{1.171923in}{0.604566in}}%
\pgfpathlineto{\pgfqpoint{1.172479in}{0.602128in}}%
\pgfpathlineto{\pgfqpoint{1.173035in}{0.604972in}}%
\pgfpathlineto{\pgfqpoint{1.173591in}{0.601432in}}%
\pgfpathlineto{\pgfqpoint{1.174147in}{0.602571in}}%
\pgfpathlineto{\pgfqpoint{1.175258in}{0.602628in}}%
\pgfpathlineto{\pgfqpoint{1.175814in}{0.605235in}}%
\pgfpathlineto{\pgfqpoint{1.176370in}{0.603820in}}%
\pgfpathlineto{\pgfqpoint{1.177482in}{0.601609in}}%
\pgfpathlineto{\pgfqpoint{1.178038in}{0.603411in}}%
\pgfpathlineto{\pgfqpoint{1.178593in}{0.601819in}}%
\pgfpathlineto{\pgfqpoint{1.180817in}{0.605950in}}%
\pgfpathlineto{\pgfqpoint{1.182484in}{0.601730in}}%
\pgfpathlineto{\pgfqpoint{1.185820in}{0.602402in}}%
\pgfpathlineto{\pgfqpoint{1.186376in}{0.602799in}}%
\pgfpathlineto{\pgfqpoint{1.186931in}{0.601838in}}%
\pgfpathlineto{\pgfqpoint{1.187487in}{0.602149in}}%
\pgfpathlineto{\pgfqpoint{1.188043in}{0.603786in}}%
\pgfpathlineto{\pgfqpoint{1.189155in}{0.601289in}}%
\pgfpathlineto{\pgfqpoint{1.189711in}{0.601823in}}%
\pgfpathlineto{\pgfqpoint{1.190267in}{0.603937in}}%
\pgfpathlineto{\pgfqpoint{1.190822in}{0.601826in}}%
\pgfpathlineto{\pgfqpoint{1.191934in}{0.601694in}}%
\pgfpathlineto{\pgfqpoint{1.192490in}{0.603306in}}%
\pgfpathlineto{\pgfqpoint{1.193046in}{0.601024in}}%
\pgfpathlineto{\pgfqpoint{1.193602in}{0.604092in}}%
\pgfpathlineto{\pgfqpoint{1.194158in}{0.600401in}}%
\pgfpathlineto{\pgfqpoint{1.194713in}{0.602001in}}%
\pgfpathlineto{\pgfqpoint{1.195269in}{0.601927in}}%
\pgfpathlineto{\pgfqpoint{1.195825in}{0.603205in}}%
\pgfpathlineto{\pgfqpoint{1.196381in}{0.601859in}}%
\pgfpathlineto{\pgfqpoint{1.198049in}{0.604298in}}%
\pgfpathlineto{\pgfqpoint{1.198604in}{0.602075in}}%
\pgfpathlineto{\pgfqpoint{1.199160in}{0.603220in}}%
\pgfpathlineto{\pgfqpoint{1.199716in}{0.604235in}}%
\pgfpathlineto{\pgfqpoint{1.200272in}{0.600887in}}%
\pgfpathlineto{\pgfqpoint{1.200828in}{0.607204in}}%
\pgfpathlineto{\pgfqpoint{1.201384in}{0.601030in}}%
\pgfpathlineto{\pgfqpoint{1.203051in}{0.606654in}}%
\pgfpathlineto{\pgfqpoint{1.204163in}{0.601909in}}%
\pgfpathlineto{\pgfqpoint{1.204719in}{0.602311in}}%
\pgfpathlineto{\pgfqpoint{1.205275in}{0.603578in}}%
\pgfpathlineto{\pgfqpoint{1.205831in}{0.602028in}}%
\pgfpathlineto{\pgfqpoint{1.206387in}{0.602752in}}%
\pgfpathlineto{\pgfqpoint{1.207498in}{0.601877in}}%
\pgfpathlineto{\pgfqpoint{1.208610in}{0.602836in}}%
\pgfpathlineto{\pgfqpoint{1.209166in}{0.601189in}}%
\pgfpathlineto{\pgfqpoint{1.209722in}{0.601712in}}%
\pgfpathlineto{\pgfqpoint{1.210278in}{0.607274in}}%
\pgfpathlineto{\pgfqpoint{1.210833in}{0.606737in}}%
\pgfpathlineto{\pgfqpoint{1.211389in}{0.601159in}}%
\pgfpathlineto{\pgfqpoint{1.211945in}{0.602938in}}%
\pgfpathlineto{\pgfqpoint{1.212501in}{0.602500in}}%
\pgfpathlineto{\pgfqpoint{1.213613in}{0.604613in}}%
\pgfpathlineto{\pgfqpoint{1.214724in}{0.601356in}}%
\pgfpathlineto{\pgfqpoint{1.215280in}{0.602642in}}%
\pgfpathlineto{\pgfqpoint{1.215836in}{0.604422in}}%
\pgfpathlineto{\pgfqpoint{1.216392in}{0.603018in}}%
\pgfpathlineto{\pgfqpoint{1.216948in}{0.601716in}}%
\pgfpathlineto{\pgfqpoint{1.217504in}{0.603675in}}%
\pgfpathlineto{\pgfqpoint{1.218060in}{0.600361in}}%
\pgfpathlineto{\pgfqpoint{1.218615in}{0.602318in}}%
\pgfpathlineto{\pgfqpoint{1.219171in}{0.601736in}}%
\pgfpathlineto{\pgfqpoint{1.219727in}{0.602277in}}%
\pgfpathlineto{\pgfqpoint{1.220283in}{0.603782in}}%
\pgfpathlineto{\pgfqpoint{1.220839in}{0.603101in}}%
\pgfpathlineto{\pgfqpoint{1.223062in}{0.601398in}}%
\pgfpathlineto{\pgfqpoint{1.224174in}{0.604880in}}%
\pgfpathlineto{\pgfqpoint{1.224730in}{0.604216in}}%
\pgfpathlineto{\pgfqpoint{1.226398in}{0.602718in}}%
\pgfpathlineto{\pgfqpoint{1.226953in}{0.604256in}}%
\pgfpathlineto{\pgfqpoint{1.227509in}{0.600295in}}%
\pgfpathlineto{\pgfqpoint{1.228065in}{0.604596in}}%
\pgfpathlineto{\pgfqpoint{1.228621in}{0.601058in}}%
\pgfpathlineto{\pgfqpoint{1.229177in}{0.602476in}}%
\pgfpathlineto{\pgfqpoint{1.229733in}{0.600219in}}%
\pgfpathlineto{\pgfqpoint{1.230289in}{0.603615in}}%
\pgfpathlineto{\pgfqpoint{1.230844in}{0.602830in}}%
\pgfpathlineto{\pgfqpoint{1.231400in}{0.601243in}}%
\pgfpathlineto{\pgfqpoint{1.232512in}{0.605730in}}%
\pgfpathlineto{\pgfqpoint{1.234180in}{0.600320in}}%
\pgfpathlineto{\pgfqpoint{1.234735in}{0.604392in}}%
\pgfpathlineto{\pgfqpoint{1.235291in}{0.601352in}}%
\pgfpathlineto{\pgfqpoint{1.235847in}{0.603857in}}%
\pgfpathlineto{\pgfqpoint{1.236403in}{0.602851in}}%
\pgfpathlineto{\pgfqpoint{1.237515in}{0.604129in}}%
\pgfpathlineto{\pgfqpoint{1.239182in}{0.601949in}}%
\pgfpathlineto{\pgfqpoint{1.240850in}{0.604280in}}%
\pgfpathlineto{\pgfqpoint{1.241406in}{0.602867in}}%
\pgfpathlineto{\pgfqpoint{1.241962in}{0.604520in}}%
\pgfpathlineto{\pgfqpoint{1.242518in}{0.600868in}}%
\pgfpathlineto{\pgfqpoint{1.243073in}{0.601518in}}%
\pgfpathlineto{\pgfqpoint{1.244185in}{0.602973in}}%
\pgfpathlineto{\pgfqpoint{1.244741in}{0.600225in}}%
\pgfpathlineto{\pgfqpoint{1.245297in}{0.602288in}}%
\pgfpathlineto{\pgfqpoint{1.245853in}{0.602941in}}%
\pgfpathlineto{\pgfqpoint{1.246409in}{0.605861in}}%
\pgfpathlineto{\pgfqpoint{1.246964in}{0.602224in}}%
\pgfpathlineto{\pgfqpoint{1.247520in}{0.602369in}}%
\pgfpathlineto{\pgfqpoint{1.248076in}{0.603726in}}%
\pgfpathlineto{\pgfqpoint{1.248632in}{0.600543in}}%
\pgfpathlineto{\pgfqpoint{1.249188in}{0.601836in}}%
\pgfpathlineto{\pgfqpoint{1.250855in}{0.600821in}}%
\pgfpathlineto{\pgfqpoint{1.251411in}{0.602316in}}%
\pgfpathlineto{\pgfqpoint{1.251967in}{0.602090in}}%
\pgfpathlineto{\pgfqpoint{1.252523in}{0.601134in}}%
\pgfpathlineto{\pgfqpoint{1.253079in}{0.605035in}}%
\pgfpathlineto{\pgfqpoint{1.253635in}{0.603794in}}%
\pgfpathlineto{\pgfqpoint{1.255302in}{0.600833in}}%
\pgfpathlineto{\pgfqpoint{1.255858in}{0.601503in}}%
\pgfpathlineto{\pgfqpoint{1.256414in}{0.600609in}}%
\pgfpathlineto{\pgfqpoint{1.256970in}{0.608561in}}%
\pgfpathlineto{\pgfqpoint{1.257526in}{0.608245in}}%
\pgfpathlineto{\pgfqpoint{1.259193in}{0.602562in}}%
\pgfpathlineto{\pgfqpoint{1.259749in}{0.603918in}}%
\pgfpathlineto{\pgfqpoint{1.260305in}{0.601750in}}%
\pgfpathlineto{\pgfqpoint{1.261417in}{0.605871in}}%
\pgfpathlineto{\pgfqpoint{1.261973in}{0.604739in}}%
\pgfpathlineto{\pgfqpoint{1.262529in}{0.602106in}}%
\pgfpathlineto{\pgfqpoint{1.263084in}{0.605026in}}%
\pgfpathlineto{\pgfqpoint{1.263640in}{0.604849in}}%
\pgfpathlineto{\pgfqpoint{1.265308in}{0.600628in}}%
\pgfpathlineto{\pgfqpoint{1.266975in}{0.602661in}}%
\pgfpathlineto{\pgfqpoint{1.267531in}{0.602389in}}%
\pgfpathlineto{\pgfqpoint{1.268087in}{0.606525in}}%
\pgfpathlineto{\pgfqpoint{1.268643in}{0.603861in}}%
\pgfpathlineto{\pgfqpoint{1.269199in}{0.604291in}}%
\pgfpathlineto{\pgfqpoint{1.269755in}{0.602294in}}%
\pgfpathlineto{\pgfqpoint{1.270311in}{0.602974in}}%
\pgfpathlineto{\pgfqpoint{1.270866in}{0.604792in}}%
\pgfpathlineto{\pgfqpoint{1.271978in}{0.600720in}}%
\pgfpathlineto{\pgfqpoint{1.273090in}{0.603845in}}%
\pgfpathlineto{\pgfqpoint{1.273646in}{0.601101in}}%
\pgfpathlineto{\pgfqpoint{1.274202in}{0.602989in}}%
\pgfpathlineto{\pgfqpoint{1.275869in}{0.602047in}}%
\pgfpathlineto{\pgfqpoint{1.276425in}{0.601196in}}%
\pgfpathlineto{\pgfqpoint{1.277537in}{0.604640in}}%
\pgfpathlineto{\pgfqpoint{1.278093in}{0.601919in}}%
\pgfpathlineto{\pgfqpoint{1.278649in}{0.603057in}}%
\pgfpathlineto{\pgfqpoint{1.279204in}{0.603180in}}%
\pgfpathlineto{\pgfqpoint{1.279760in}{0.600796in}}%
\pgfpathlineto{\pgfqpoint{1.280316in}{0.601817in}}%
\pgfpathlineto{\pgfqpoint{1.282540in}{0.605316in}}%
\pgfpathlineto{\pgfqpoint{1.283095in}{0.605637in}}%
\pgfpathlineto{\pgfqpoint{1.284207in}{0.600076in}}%
\pgfpathlineto{\pgfqpoint{1.284763in}{0.601001in}}%
\pgfpathlineto{\pgfqpoint{1.285319in}{0.603902in}}%
\pgfpathlineto{\pgfqpoint{1.285875in}{0.601088in}}%
\pgfpathlineto{\pgfqpoint{1.286431in}{0.600908in}}%
\pgfpathlineto{\pgfqpoint{1.288098in}{0.604167in}}%
\pgfpathlineto{\pgfqpoint{1.288654in}{0.602060in}}%
\pgfpathlineto{\pgfqpoint{1.289210in}{0.603759in}}%
\pgfpathlineto{\pgfqpoint{1.289766in}{0.603233in}}%
\pgfpathlineto{\pgfqpoint{1.290322in}{0.600907in}}%
\pgfpathlineto{\pgfqpoint{1.290877in}{0.602510in}}%
\pgfpathlineto{\pgfqpoint{1.291433in}{0.608594in}}%
\pgfpathlineto{\pgfqpoint{1.291989in}{0.604240in}}%
\pgfpathlineto{\pgfqpoint{1.293657in}{0.600449in}}%
\pgfpathlineto{\pgfqpoint{1.294213in}{0.600925in}}%
\pgfpathlineto{\pgfqpoint{1.294768in}{0.604544in}}%
\pgfpathlineto{\pgfqpoint{1.295324in}{0.602220in}}%
\pgfpathlineto{\pgfqpoint{1.296992in}{0.601760in}}%
\pgfpathlineto{\pgfqpoint{1.298104in}{0.607003in}}%
\pgfpathlineto{\pgfqpoint{1.298660in}{0.600628in}}%
\pgfpathlineto{\pgfqpoint{1.299215in}{0.603508in}}%
\pgfpathlineto{\pgfqpoint{1.300883in}{0.602243in}}%
\pgfpathlineto{\pgfqpoint{1.303106in}{0.601550in}}%
\pgfpathlineto{\pgfqpoint{1.303662in}{0.604917in}}%
\pgfpathlineto{\pgfqpoint{1.304218in}{0.602209in}}%
\pgfpathlineto{\pgfqpoint{1.304774in}{0.600796in}}%
\pgfpathlineto{\pgfqpoint{1.305330in}{0.602390in}}%
\pgfpathlineto{\pgfqpoint{1.305886in}{0.601183in}}%
\pgfpathlineto{\pgfqpoint{1.306442in}{0.603943in}}%
\pgfpathlineto{\pgfqpoint{1.306997in}{0.601643in}}%
\pgfpathlineto{\pgfqpoint{1.307553in}{0.601719in}}%
\pgfpathlineto{\pgfqpoint{1.308665in}{0.604045in}}%
\pgfpathlineto{\pgfqpoint{1.309777in}{0.600512in}}%
\pgfpathlineto{\pgfqpoint{1.310333in}{0.604952in}}%
\pgfpathlineto{\pgfqpoint{1.310888in}{0.601148in}}%
\pgfpathlineto{\pgfqpoint{1.312000in}{0.603046in}}%
\pgfpathlineto{\pgfqpoint{1.312556in}{0.601238in}}%
\pgfpathlineto{\pgfqpoint{1.313112in}{0.601778in}}%
\pgfpathlineto{\pgfqpoint{1.313668in}{0.601496in}}%
\pgfpathlineto{\pgfqpoint{1.314224in}{0.607571in}}%
\pgfpathlineto{\pgfqpoint{1.314779in}{0.603802in}}%
\pgfpathlineto{\pgfqpoint{1.315891in}{0.608166in}}%
\pgfpathlineto{\pgfqpoint{1.317559in}{0.602263in}}%
\pgfpathlineto{\pgfqpoint{1.319226in}{0.604394in}}%
\pgfpathlineto{\pgfqpoint{1.320338in}{0.600683in}}%
\pgfpathlineto{\pgfqpoint{1.320894in}{0.603465in}}%
\pgfpathlineto{\pgfqpoint{1.321450in}{0.603235in}}%
\pgfpathlineto{\pgfqpoint{1.322006in}{0.602118in}}%
\pgfpathlineto{\pgfqpoint{1.322562in}{0.607721in}}%
\pgfpathlineto{\pgfqpoint{1.323117in}{0.603991in}}%
\pgfpathlineto{\pgfqpoint{1.324229in}{0.600512in}}%
\pgfpathlineto{\pgfqpoint{1.324785in}{0.602284in}}%
\pgfpathlineto{\pgfqpoint{1.325897in}{0.606004in}}%
\pgfpathlineto{\pgfqpoint{1.326453in}{0.605104in}}%
\pgfpathlineto{\pgfqpoint{1.327564in}{0.602176in}}%
\pgfpathlineto{\pgfqpoint{1.328120in}{0.602639in}}%
\pgfpathlineto{\pgfqpoint{1.330344in}{0.604222in}}%
\pgfpathlineto{\pgfqpoint{1.332567in}{0.602631in}}%
\pgfpathlineto{\pgfqpoint{1.333123in}{0.602348in}}%
\pgfpathlineto{\pgfqpoint{1.333679in}{0.600423in}}%
\pgfpathlineto{\pgfqpoint{1.334235in}{0.604807in}}%
\pgfpathlineto{\pgfqpoint{1.334791in}{0.601635in}}%
\pgfpathlineto{\pgfqpoint{1.335346in}{0.601584in}}%
\pgfpathlineto{\pgfqpoint{1.337014in}{0.604271in}}%
\pgfpathlineto{\pgfqpoint{1.337570in}{0.603604in}}%
\pgfpathlineto{\pgfqpoint{1.338126in}{0.604997in}}%
\pgfpathlineto{\pgfqpoint{1.338682in}{0.604629in}}%
\pgfpathlineto{\pgfqpoint{1.339237in}{0.604286in}}%
\pgfpathlineto{\pgfqpoint{1.339793in}{0.608500in}}%
\pgfpathlineto{\pgfqpoint{1.340349in}{0.605886in}}%
\pgfpathlineto{\pgfqpoint{1.341461in}{0.605978in}}%
\pgfpathlineto{\pgfqpoint{1.342017in}{0.603651in}}%
\pgfpathlineto{\pgfqpoint{1.342573in}{0.606916in}}%
\pgfpathlineto{\pgfqpoint{1.343128in}{0.601103in}}%
\pgfpathlineto{\pgfqpoint{1.343684in}{0.604400in}}%
\pgfpathlineto{\pgfqpoint{1.344240in}{0.607790in}}%
\pgfpathlineto{\pgfqpoint{1.345908in}{0.603948in}}%
\pgfpathlineto{\pgfqpoint{1.346464in}{0.604181in}}%
\pgfpathlineto{\pgfqpoint{1.347019in}{0.601865in}}%
\pgfpathlineto{\pgfqpoint{1.347575in}{0.603000in}}%
\pgfpathlineto{\pgfqpoint{1.348131in}{0.605376in}}%
\pgfpathlineto{\pgfqpoint{1.348687in}{0.605018in}}%
\pgfpathlineto{\pgfqpoint{1.349799in}{0.600884in}}%
\pgfpathlineto{\pgfqpoint{1.350355in}{0.602752in}}%
\pgfpathlineto{\pgfqpoint{1.350910in}{0.606554in}}%
\pgfpathlineto{\pgfqpoint{1.352578in}{0.601579in}}%
\pgfpathlineto{\pgfqpoint{1.354802in}{0.606210in}}%
\pgfpathlineto{\pgfqpoint{1.355357in}{0.605304in}}%
\pgfpathlineto{\pgfqpoint{1.355913in}{0.600274in}}%
\pgfpathlineto{\pgfqpoint{1.356469in}{0.602367in}}%
\pgfpathlineto{\pgfqpoint{1.357025in}{0.602345in}}%
\pgfpathlineto{\pgfqpoint{1.357581in}{0.601178in}}%
\pgfpathlineto{\pgfqpoint{1.358693in}{0.605394in}}%
\pgfpathlineto{\pgfqpoint{1.360360in}{0.600509in}}%
\pgfpathlineto{\pgfqpoint{1.362028in}{0.606792in}}%
\pgfpathlineto{\pgfqpoint{1.363695in}{0.602255in}}%
\pgfpathlineto{\pgfqpoint{1.364251in}{0.605404in}}%
\pgfpathlineto{\pgfqpoint{1.364807in}{0.600870in}}%
\pgfpathlineto{\pgfqpoint{1.365363in}{0.603294in}}%
\pgfpathlineto{\pgfqpoint{1.366475in}{0.601319in}}%
\pgfpathlineto{\pgfqpoint{1.368142in}{0.607232in}}%
\pgfpathlineto{\pgfqpoint{1.368698in}{0.602008in}}%
\pgfpathlineto{\pgfqpoint{1.369254in}{0.604300in}}%
\pgfpathlineto{\pgfqpoint{1.370366in}{0.600954in}}%
\pgfpathlineto{\pgfqpoint{1.372033in}{0.609053in}}%
\pgfpathlineto{\pgfqpoint{1.374257in}{0.602002in}}%
\pgfpathlineto{\pgfqpoint{1.374813in}{0.603606in}}%
\pgfpathlineto{\pgfqpoint{1.375368in}{0.602859in}}%
\pgfpathlineto{\pgfqpoint{1.375924in}{0.603380in}}%
\pgfpathlineto{\pgfqpoint{1.376480in}{0.605681in}}%
\pgfpathlineto{\pgfqpoint{1.377036in}{0.603374in}}%
\pgfpathlineto{\pgfqpoint{1.377592in}{0.601114in}}%
\pgfpathlineto{\pgfqpoint{1.378148in}{0.604772in}}%
\pgfpathlineto{\pgfqpoint{1.378704in}{0.603864in}}%
\pgfpathlineto{\pgfqpoint{1.380371in}{0.600793in}}%
\pgfpathlineto{\pgfqpoint{1.380927in}{0.604668in}}%
\pgfpathlineto{\pgfqpoint{1.381483in}{0.602930in}}%
\pgfpathlineto{\pgfqpoint{1.382039in}{0.602294in}}%
\pgfpathlineto{\pgfqpoint{1.383150in}{0.611323in}}%
\pgfpathlineto{\pgfqpoint{1.384818in}{0.601907in}}%
\pgfpathlineto{\pgfqpoint{1.387041in}{0.608152in}}%
\pgfpathlineto{\pgfqpoint{1.388709in}{0.602402in}}%
\pgfpathlineto{\pgfqpoint{1.389265in}{0.608454in}}%
\pgfpathlineto{\pgfqpoint{1.389821in}{0.604718in}}%
\pgfpathlineto{\pgfqpoint{1.390377in}{0.601853in}}%
\pgfpathlineto{\pgfqpoint{1.390933in}{0.604773in}}%
\pgfpathlineto{\pgfqpoint{1.391488in}{0.602378in}}%
\pgfpathlineto{\pgfqpoint{1.392044in}{0.604428in}}%
\pgfpathlineto{\pgfqpoint{1.392600in}{0.602700in}}%
\pgfpathlineto{\pgfqpoint{1.393156in}{0.607011in}}%
\pgfpathlineto{\pgfqpoint{1.393712in}{0.605893in}}%
\pgfpathlineto{\pgfqpoint{1.394268in}{0.605767in}}%
\pgfpathlineto{\pgfqpoint{1.394824in}{0.600199in}}%
\pgfpathlineto{\pgfqpoint{1.395379in}{0.605243in}}%
\pgfpathlineto{\pgfqpoint{1.395935in}{0.601213in}}%
\pgfpathlineto{\pgfqpoint{1.397603in}{0.607368in}}%
\pgfpathlineto{\pgfqpoint{1.398159in}{0.607611in}}%
\pgfpathlineto{\pgfqpoint{1.398715in}{0.602360in}}%
\pgfpathlineto{\pgfqpoint{1.399270in}{0.605439in}}%
\pgfpathlineto{\pgfqpoint{1.400382in}{0.609555in}}%
\pgfpathlineto{\pgfqpoint{1.400938in}{0.600719in}}%
\pgfpathlineto{\pgfqpoint{1.401494in}{0.604869in}}%
\pgfpathlineto{\pgfqpoint{1.402050in}{0.601768in}}%
\pgfpathlineto{\pgfqpoint{1.402606in}{0.602409in}}%
\pgfpathlineto{\pgfqpoint{1.403161in}{0.607683in}}%
\pgfpathlineto{\pgfqpoint{1.403717in}{0.603082in}}%
\pgfpathlineto{\pgfqpoint{1.404829in}{0.607139in}}%
\pgfpathlineto{\pgfqpoint{1.405385in}{0.605745in}}%
\pgfpathlineto{\pgfqpoint{1.406497in}{0.602574in}}%
\pgfpathlineto{\pgfqpoint{1.407052in}{0.603382in}}%
\pgfpathlineto{\pgfqpoint{1.407608in}{0.602894in}}%
\pgfpathlineto{\pgfqpoint{1.408164in}{0.607126in}}%
\pgfpathlineto{\pgfqpoint{1.408720in}{0.602129in}}%
\pgfpathlineto{\pgfqpoint{1.409276in}{0.603344in}}%
\pgfpathlineto{\pgfqpoint{1.410388in}{0.603857in}}%
\pgfpathlineto{\pgfqpoint{1.411499in}{0.607973in}}%
\pgfpathlineto{\pgfqpoint{1.412055in}{0.603239in}}%
\pgfpathlineto{\pgfqpoint{1.412611in}{0.604145in}}%
\pgfpathlineto{\pgfqpoint{1.413167in}{0.604480in}}%
\pgfpathlineto{\pgfqpoint{1.413723in}{0.600676in}}%
\pgfpathlineto{\pgfqpoint{1.414279in}{0.605434in}}%
\pgfpathlineto{\pgfqpoint{1.414835in}{0.603609in}}%
\pgfpathlineto{\pgfqpoint{1.416502in}{0.605604in}}%
\pgfpathlineto{\pgfqpoint{1.417058in}{0.601519in}}%
\pgfpathlineto{\pgfqpoint{1.417614in}{0.603372in}}%
\pgfpathlineto{\pgfqpoint{1.418726in}{0.611543in}}%
\pgfpathlineto{\pgfqpoint{1.419281in}{0.601635in}}%
\pgfpathlineto{\pgfqpoint{1.419837in}{0.603131in}}%
\pgfpathlineto{\pgfqpoint{1.420393in}{0.601723in}}%
\pgfpathlineto{\pgfqpoint{1.421505in}{0.606258in}}%
\pgfpathlineto{\pgfqpoint{1.422061in}{0.600743in}}%
\pgfpathlineto{\pgfqpoint{1.422617in}{0.603427in}}%
\pgfpathlineto{\pgfqpoint{1.423172in}{0.605900in}}%
\pgfpathlineto{\pgfqpoint{1.424284in}{0.600738in}}%
\pgfpathlineto{\pgfqpoint{1.424840in}{0.605884in}}%
\pgfpathlineto{\pgfqpoint{1.425396in}{0.605694in}}%
\pgfpathlineto{\pgfqpoint{1.425952in}{0.603173in}}%
\pgfpathlineto{\pgfqpoint{1.426508in}{0.605323in}}%
\pgfpathlineto{\pgfqpoint{1.427619in}{0.601408in}}%
\pgfpathlineto{\pgfqpoint{1.429287in}{0.609591in}}%
\pgfpathlineto{\pgfqpoint{1.430955in}{0.604071in}}%
\pgfpathlineto{\pgfqpoint{1.431510in}{0.606833in}}%
\pgfpathlineto{\pgfqpoint{1.432066in}{0.602053in}}%
\pgfpathlineto{\pgfqpoint{1.432622in}{0.604731in}}%
\pgfpathlineto{\pgfqpoint{1.433178in}{0.607610in}}%
\pgfpathlineto{\pgfqpoint{1.433734in}{0.604823in}}%
\pgfpathlineto{\pgfqpoint{1.434290in}{0.600748in}}%
\pgfpathlineto{\pgfqpoint{1.434846in}{0.601246in}}%
\pgfpathlineto{\pgfqpoint{1.435957in}{0.605735in}}%
\pgfpathlineto{\pgfqpoint{1.436513in}{0.603566in}}%
\pgfpathlineto{\pgfqpoint{1.437069in}{0.601750in}}%
\pgfpathlineto{\pgfqpoint{1.437625in}{0.606993in}}%
\pgfpathlineto{\pgfqpoint{1.438181in}{0.605165in}}%
\pgfpathlineto{\pgfqpoint{1.439292in}{0.603539in}}%
\pgfpathlineto{\pgfqpoint{1.441516in}{0.610790in}}%
\pgfpathlineto{\pgfqpoint{1.443183in}{0.603187in}}%
\pgfpathlineto{\pgfqpoint{1.443739in}{0.602344in}}%
\pgfpathlineto{\pgfqpoint{1.445407in}{0.609097in}}%
\pgfpathlineto{\pgfqpoint{1.445963in}{0.603532in}}%
\pgfpathlineto{\pgfqpoint{1.446519in}{0.606028in}}%
\pgfpathlineto{\pgfqpoint{1.447075in}{0.605738in}}%
\pgfpathlineto{\pgfqpoint{1.447630in}{0.601611in}}%
\pgfpathlineto{\pgfqpoint{1.448186in}{0.607099in}}%
\pgfpathlineto{\pgfqpoint{1.448742in}{0.602878in}}%
\pgfpathlineto{\pgfqpoint{1.450410in}{0.603642in}}%
\pgfpathlineto{\pgfqpoint{1.450966in}{0.601994in}}%
\pgfpathlineto{\pgfqpoint{1.452633in}{0.607247in}}%
\pgfpathlineto{\pgfqpoint{1.453189in}{0.604174in}}%
\pgfpathlineto{\pgfqpoint{1.453745in}{0.605562in}}%
\pgfpathlineto{\pgfqpoint{1.454301in}{0.610117in}}%
\pgfpathlineto{\pgfqpoint{1.454857in}{0.605831in}}%
\pgfpathlineto{\pgfqpoint{1.455412in}{0.608113in}}%
\pgfpathlineto{\pgfqpoint{1.455968in}{0.602777in}}%
\pgfpathlineto{\pgfqpoint{1.456524in}{0.608419in}}%
\pgfpathlineto{\pgfqpoint{1.457080in}{0.602066in}}%
\pgfpathlineto{\pgfqpoint{1.457636in}{0.611225in}}%
\pgfpathlineto{\pgfqpoint{1.458192in}{0.605363in}}%
\pgfpathlineto{\pgfqpoint{1.458748in}{0.605106in}}%
\pgfpathlineto{\pgfqpoint{1.459303in}{0.606943in}}%
\pgfpathlineto{\pgfqpoint{1.459859in}{0.604701in}}%
\pgfpathlineto{\pgfqpoint{1.460415in}{0.609781in}}%
\pgfpathlineto{\pgfqpoint{1.460971in}{0.604959in}}%
\pgfpathlineto{\pgfqpoint{1.461527in}{0.607004in}}%
\pgfpathlineto{\pgfqpoint{1.462083in}{0.605084in}}%
\pgfpathlineto{\pgfqpoint{1.462639in}{0.605413in}}%
\pgfpathlineto{\pgfqpoint{1.463194in}{0.603108in}}%
\pgfpathlineto{\pgfqpoint{1.463750in}{0.603633in}}%
\pgfpathlineto{\pgfqpoint{1.465418in}{0.606833in}}%
\pgfpathlineto{\pgfqpoint{1.465974in}{0.606463in}}%
\pgfpathlineto{\pgfqpoint{1.466530in}{0.600941in}}%
\pgfpathlineto{\pgfqpoint{1.467086in}{0.603052in}}%
\pgfpathlineto{\pgfqpoint{1.467641in}{0.606383in}}%
\pgfpathlineto{\pgfqpoint{1.468197in}{0.605617in}}%
\pgfpathlineto{\pgfqpoint{1.469309in}{0.600988in}}%
\pgfpathlineto{\pgfqpoint{1.470421in}{0.610291in}}%
\pgfpathlineto{\pgfqpoint{1.470977in}{0.604668in}}%
\pgfpathlineto{\pgfqpoint{1.471532in}{0.608351in}}%
\pgfpathlineto{\pgfqpoint{1.473200in}{0.601535in}}%
\pgfpathlineto{\pgfqpoint{1.473756in}{0.606411in}}%
\pgfpathlineto{\pgfqpoint{1.474312in}{0.605010in}}%
\pgfpathlineto{\pgfqpoint{1.475423in}{0.602841in}}%
\pgfpathlineto{\pgfqpoint{1.475979in}{0.606883in}}%
\pgfpathlineto{\pgfqpoint{1.476535in}{0.602800in}}%
\pgfpathlineto{\pgfqpoint{1.477091in}{0.601206in}}%
\pgfpathlineto{\pgfqpoint{1.478203in}{0.609448in}}%
\pgfpathlineto{\pgfqpoint{1.478759in}{0.608077in}}%
\pgfpathlineto{\pgfqpoint{1.479870in}{0.600821in}}%
\pgfpathlineto{\pgfqpoint{1.480982in}{0.606893in}}%
\pgfpathlineto{\pgfqpoint{1.482094in}{0.603168in}}%
\pgfpathlineto{\pgfqpoint{1.484317in}{0.610825in}}%
\pgfpathlineto{\pgfqpoint{1.484873in}{0.601812in}}%
\pgfpathlineto{\pgfqpoint{1.485429in}{0.605867in}}%
\pgfpathlineto{\pgfqpoint{1.487097in}{0.610537in}}%
\pgfpathlineto{\pgfqpoint{1.489320in}{0.604519in}}%
\pgfpathlineto{\pgfqpoint{1.490988in}{0.607483in}}%
\pgfpathlineto{\pgfqpoint{1.491543in}{0.603113in}}%
\pgfpathlineto{\pgfqpoint{1.492099in}{0.604496in}}%
\pgfpathlineto{\pgfqpoint{1.493211in}{0.603029in}}%
\pgfpathlineto{\pgfqpoint{1.493767in}{0.603439in}}%
\pgfpathlineto{\pgfqpoint{1.494323in}{0.601766in}}%
\pgfpathlineto{\pgfqpoint{1.495990in}{0.609502in}}%
\pgfpathlineto{\pgfqpoint{1.496546in}{0.602008in}}%
\pgfpathlineto{\pgfqpoint{1.497102in}{0.604140in}}%
\pgfpathlineto{\pgfqpoint{1.498214in}{0.610155in}}%
\pgfpathlineto{\pgfqpoint{1.499881in}{0.601351in}}%
\pgfpathlineto{\pgfqpoint{1.500993in}{0.608094in}}%
\pgfpathlineto{\pgfqpoint{1.501549in}{0.605812in}}%
\pgfpathlineto{\pgfqpoint{1.502105in}{0.610754in}}%
\pgfpathlineto{\pgfqpoint{1.502661in}{0.607488in}}%
\pgfpathlineto{\pgfqpoint{1.503216in}{0.603453in}}%
\pgfpathlineto{\pgfqpoint{1.503772in}{0.607303in}}%
\pgfpathlineto{\pgfqpoint{1.505440in}{0.609880in}}%
\pgfpathlineto{\pgfqpoint{1.507108in}{0.602947in}}%
\pgfpathlineto{\pgfqpoint{1.508219in}{0.604113in}}%
\pgfpathlineto{\pgfqpoint{1.508775in}{0.605260in}}%
\pgfpathlineto{\pgfqpoint{1.509331in}{0.602453in}}%
\pgfpathlineto{\pgfqpoint{1.510999in}{0.608291in}}%
\pgfpathlineto{\pgfqpoint{1.511554in}{0.608058in}}%
\pgfpathlineto{\pgfqpoint{1.512110in}{0.604485in}}%
\pgfpathlineto{\pgfqpoint{1.512666in}{0.612785in}}%
\pgfpathlineto{\pgfqpoint{1.513222in}{0.606339in}}%
\pgfpathlineto{\pgfqpoint{1.513778in}{0.609881in}}%
\pgfpathlineto{\pgfqpoint{1.514334in}{0.602061in}}%
\pgfpathlineto{\pgfqpoint{1.514890in}{0.610152in}}%
\pgfpathlineto{\pgfqpoint{1.515445in}{0.606984in}}%
\pgfpathlineto{\pgfqpoint{1.516001in}{0.601004in}}%
\pgfpathlineto{\pgfqpoint{1.516557in}{0.606945in}}%
\pgfpathlineto{\pgfqpoint{1.518225in}{0.603756in}}%
\pgfpathlineto{\pgfqpoint{1.518781in}{0.603840in}}%
\pgfpathlineto{\pgfqpoint{1.519892in}{0.608099in}}%
\pgfpathlineto{\pgfqpoint{1.520448in}{0.605380in}}%
\pgfpathlineto{\pgfqpoint{1.521004in}{0.608805in}}%
\pgfpathlineto{\pgfqpoint{1.521560in}{0.603925in}}%
\pgfpathlineto{\pgfqpoint{1.522116in}{0.604871in}}%
\pgfpathlineto{\pgfqpoint{1.522672in}{0.607204in}}%
\pgfpathlineto{\pgfqpoint{1.523228in}{0.603765in}}%
\pgfpathlineto{\pgfqpoint{1.523783in}{0.607052in}}%
\pgfpathlineto{\pgfqpoint{1.524895in}{0.603416in}}%
\pgfpathlineto{\pgfqpoint{1.526563in}{0.606523in}}%
\pgfpathlineto{\pgfqpoint{1.527119in}{0.607054in}}%
\pgfpathlineto{\pgfqpoint{1.527674in}{0.617474in}}%
\pgfpathlineto{\pgfqpoint{1.528230in}{0.603685in}}%
\pgfpathlineto{\pgfqpoint{1.528786in}{0.604127in}}%
\pgfpathlineto{\pgfqpoint{1.529342in}{0.609388in}}%
\pgfpathlineto{\pgfqpoint{1.529898in}{0.605332in}}%
\pgfpathlineto{\pgfqpoint{1.530454in}{0.603395in}}%
\pgfpathlineto{\pgfqpoint{1.531010in}{0.605461in}}%
\pgfpathlineto{\pgfqpoint{1.531565in}{0.611061in}}%
\pgfpathlineto{\pgfqpoint{1.532121in}{0.601502in}}%
\pgfpathlineto{\pgfqpoint{1.532677in}{0.608905in}}%
\pgfpathlineto{\pgfqpoint{1.533789in}{0.603298in}}%
\pgfpathlineto{\pgfqpoint{1.534345in}{0.604368in}}%
\pgfpathlineto{\pgfqpoint{1.534901in}{0.611914in}}%
\pgfpathlineto{\pgfqpoint{1.535456in}{0.608865in}}%
\pgfpathlineto{\pgfqpoint{1.536012in}{0.608239in}}%
\pgfpathlineto{\pgfqpoint{1.536568in}{0.600815in}}%
\pgfpathlineto{\pgfqpoint{1.537124in}{0.606054in}}%
\pgfpathlineto{\pgfqpoint{1.537680in}{0.602867in}}%
\pgfpathlineto{\pgfqpoint{1.538236in}{0.605151in}}%
\pgfpathlineto{\pgfqpoint{1.539347in}{0.602126in}}%
\pgfpathlineto{\pgfqpoint{1.539903in}{0.606404in}}%
\pgfpathlineto{\pgfqpoint{1.540459in}{0.603309in}}%
\pgfpathlineto{\pgfqpoint{1.541015in}{0.602642in}}%
\pgfpathlineto{\pgfqpoint{1.542683in}{0.608448in}}%
\pgfpathlineto{\pgfqpoint{1.543239in}{0.611577in}}%
\pgfpathlineto{\pgfqpoint{1.543794in}{0.603463in}}%
\pgfpathlineto{\pgfqpoint{1.544350in}{0.612970in}}%
\pgfpathlineto{\pgfqpoint{1.544906in}{0.612077in}}%
\pgfpathlineto{\pgfqpoint{1.546018in}{0.605354in}}%
\pgfpathlineto{\pgfqpoint{1.546574in}{0.605786in}}%
\pgfpathlineto{\pgfqpoint{1.547685in}{0.602202in}}%
\pgfpathlineto{\pgfqpoint{1.548241in}{0.612451in}}%
\pgfpathlineto{\pgfqpoint{1.548797in}{0.605268in}}%
\pgfpathlineto{\pgfqpoint{1.550465in}{0.603079in}}%
\pgfpathlineto{\pgfqpoint{1.552132in}{0.607209in}}%
\pgfpathlineto{\pgfqpoint{1.553800in}{0.610476in}}%
\pgfpathlineto{\pgfqpoint{1.554356in}{0.602851in}}%
\pgfpathlineto{\pgfqpoint{1.554912in}{0.604474in}}%
\pgfpathlineto{\pgfqpoint{1.556023in}{0.611198in}}%
\pgfpathlineto{\pgfqpoint{1.556579in}{0.607248in}}%
\pgfpathlineto{\pgfqpoint{1.557691in}{0.601021in}}%
\pgfpathlineto{\pgfqpoint{1.558247in}{0.602758in}}%
\pgfpathlineto{\pgfqpoint{1.558803in}{0.601030in}}%
\pgfpathlineto{\pgfqpoint{1.559358in}{0.601340in}}%
\pgfpathlineto{\pgfqpoint{1.560470in}{0.610234in}}%
\pgfpathlineto{\pgfqpoint{1.561026in}{0.601691in}}%
\pgfpathlineto{\pgfqpoint{1.561582in}{0.609527in}}%
\pgfpathlineto{\pgfqpoint{1.562694in}{0.602716in}}%
\pgfpathlineto{\pgfqpoint{1.563250in}{0.606402in}}%
\pgfpathlineto{\pgfqpoint{1.563805in}{0.606305in}}%
\pgfpathlineto{\pgfqpoint{1.564361in}{0.605542in}}%
\pgfpathlineto{\pgfqpoint{1.564917in}{0.612941in}}%
\pgfpathlineto{\pgfqpoint{1.565473in}{0.608555in}}%
\pgfpathlineto{\pgfqpoint{1.566029in}{0.611056in}}%
\pgfpathlineto{\pgfqpoint{1.567696in}{0.601738in}}%
\pgfpathlineto{\pgfqpoint{1.568252in}{0.608577in}}%
\pgfpathlineto{\pgfqpoint{1.568808in}{0.603746in}}%
\pgfpathlineto{\pgfqpoint{1.569364in}{0.608355in}}%
\pgfpathlineto{\pgfqpoint{1.569920in}{0.605329in}}%
\pgfpathlineto{\pgfqpoint{1.571587in}{0.605811in}}%
\pgfpathlineto{\pgfqpoint{1.572143in}{0.613101in}}%
\pgfpathlineto{\pgfqpoint{1.572699in}{0.611769in}}%
\pgfpathlineto{\pgfqpoint{1.573255in}{0.601415in}}%
\pgfpathlineto{\pgfqpoint{1.573811in}{0.607443in}}%
\pgfpathlineto{\pgfqpoint{1.574367in}{0.608032in}}%
\pgfpathlineto{\pgfqpoint{1.576034in}{0.604868in}}%
\pgfpathlineto{\pgfqpoint{1.576590in}{0.605257in}}%
\pgfpathlineto{\pgfqpoint{1.577146in}{0.611111in}}%
\pgfpathlineto{\pgfqpoint{1.577702in}{0.606944in}}%
\pgfpathlineto{\pgfqpoint{1.578258in}{0.604217in}}%
\pgfpathlineto{\pgfqpoint{1.578814in}{0.610508in}}%
\pgfpathlineto{\pgfqpoint{1.579370in}{0.604572in}}%
\pgfpathlineto{\pgfqpoint{1.579925in}{0.603759in}}%
\pgfpathlineto{\pgfqpoint{1.580481in}{0.607183in}}%
\pgfpathlineto{\pgfqpoint{1.581037in}{0.606566in}}%
\pgfpathlineto{\pgfqpoint{1.581593in}{0.603744in}}%
\pgfpathlineto{\pgfqpoint{1.582149in}{0.607044in}}%
\pgfpathlineto{\pgfqpoint{1.582705in}{0.602052in}}%
\pgfpathlineto{\pgfqpoint{1.583261in}{0.602434in}}%
\pgfpathlineto{\pgfqpoint{1.583816in}{0.607802in}}%
\pgfpathlineto{\pgfqpoint{1.584372in}{0.603145in}}%
\pgfpathlineto{\pgfqpoint{1.584928in}{0.606036in}}%
\pgfpathlineto{\pgfqpoint{1.585484in}{0.603618in}}%
\pgfpathlineto{\pgfqpoint{1.586040in}{0.600781in}}%
\pgfpathlineto{\pgfqpoint{1.586596in}{0.603601in}}%
\pgfpathlineto{\pgfqpoint{1.587152in}{0.609324in}}%
\pgfpathlineto{\pgfqpoint{1.587707in}{0.606862in}}%
\pgfpathlineto{\pgfqpoint{1.588263in}{0.603834in}}%
\pgfpathlineto{\pgfqpoint{1.588819in}{0.609229in}}%
\pgfpathlineto{\pgfqpoint{1.589375in}{0.601836in}}%
\pgfpathlineto{\pgfqpoint{1.589931in}{0.603515in}}%
\pgfpathlineto{\pgfqpoint{1.593266in}{0.605796in}}%
\pgfpathlineto{\pgfqpoint{1.594378in}{0.611270in}}%
\pgfpathlineto{\pgfqpoint{1.595489in}{0.604038in}}%
\pgfpathlineto{\pgfqpoint{1.596045in}{0.604361in}}%
\pgfpathlineto{\pgfqpoint{1.596601in}{0.606456in}}%
\pgfpathlineto{\pgfqpoint{1.597157in}{0.603756in}}%
\pgfpathlineto{\pgfqpoint{1.597713in}{0.604852in}}%
\pgfpathlineto{\pgfqpoint{1.598825in}{0.604223in}}%
\pgfpathlineto{\pgfqpoint{1.599381in}{0.610469in}}%
\pgfpathlineto{\pgfqpoint{1.599936in}{0.603748in}}%
\pgfpathlineto{\pgfqpoint{1.600492in}{0.608961in}}%
\pgfpathlineto{\pgfqpoint{1.601048in}{0.610643in}}%
\pgfpathlineto{\pgfqpoint{1.602716in}{0.601966in}}%
\pgfpathlineto{\pgfqpoint{1.603827in}{0.607200in}}%
\pgfpathlineto{\pgfqpoint{1.604383in}{0.607014in}}%
\pgfpathlineto{\pgfqpoint{1.604939in}{0.603827in}}%
\pgfpathlineto{\pgfqpoint{1.606051in}{0.609264in}}%
\pgfpathlineto{\pgfqpoint{1.606607in}{0.602177in}}%
\pgfpathlineto{\pgfqpoint{1.607163in}{0.604389in}}%
\pgfpathlineto{\pgfqpoint{1.607718in}{0.604481in}}%
\pgfpathlineto{\pgfqpoint{1.608274in}{0.608670in}}%
\pgfpathlineto{\pgfqpoint{1.608830in}{0.604439in}}%
\pgfpathlineto{\pgfqpoint{1.609386in}{0.604740in}}%
\pgfpathlineto{\pgfqpoint{1.609942in}{0.603737in}}%
\pgfpathlineto{\pgfqpoint{1.611054in}{0.608952in}}%
\pgfpathlineto{\pgfqpoint{1.611609in}{0.603280in}}%
\pgfpathlineto{\pgfqpoint{1.612165in}{0.603393in}}%
\pgfpathlineto{\pgfqpoint{1.612721in}{0.609623in}}%
\pgfpathlineto{\pgfqpoint{1.614389in}{0.601383in}}%
\pgfpathlineto{\pgfqpoint{1.615500in}{0.604059in}}%
\pgfpathlineto{\pgfqpoint{1.616056in}{0.603295in}}%
\pgfpathlineto{\pgfqpoint{1.616612in}{0.600137in}}%
\pgfpathlineto{\pgfqpoint{1.617168in}{0.606412in}}%
\pgfpathlineto{\pgfqpoint{1.617724in}{0.604250in}}%
\pgfpathlineto{\pgfqpoint{1.618280in}{0.606338in}}%
\pgfpathlineto{\pgfqpoint{1.618836in}{0.604038in}}%
\pgfpathlineto{\pgfqpoint{1.619392in}{0.606751in}}%
\pgfpathlineto{\pgfqpoint{1.619947in}{0.605363in}}%
\pgfpathlineto{\pgfqpoint{1.620503in}{0.604348in}}%
\pgfpathlineto{\pgfqpoint{1.621059in}{0.605192in}}%
\pgfpathlineto{\pgfqpoint{1.623283in}{0.601996in}}%
\pgfpathlineto{\pgfqpoint{1.624950in}{0.613412in}}%
\pgfpathlineto{\pgfqpoint{1.626062in}{0.602017in}}%
\pgfpathlineto{\pgfqpoint{1.626618in}{0.606141in}}%
\pgfpathlineto{\pgfqpoint{1.627174in}{0.606534in}}%
\pgfpathlineto{\pgfqpoint{1.628285in}{0.601907in}}%
\pgfpathlineto{\pgfqpoint{1.629953in}{0.611840in}}%
\pgfpathlineto{\pgfqpoint{1.630509in}{0.600631in}}%
\pgfpathlineto{\pgfqpoint{1.631065in}{0.608286in}}%
\pgfpathlineto{\pgfqpoint{1.631620in}{0.609871in}}%
\pgfpathlineto{\pgfqpoint{1.632176in}{0.601609in}}%
\pgfpathlineto{\pgfqpoint{1.632732in}{0.606501in}}%
\pgfpathlineto{\pgfqpoint{1.633288in}{0.604054in}}%
\pgfpathlineto{\pgfqpoint{1.633844in}{0.605337in}}%
\pgfpathlineto{\pgfqpoint{1.634956in}{0.604341in}}%
\pgfpathlineto{\pgfqpoint{1.635512in}{0.601472in}}%
\pgfpathlineto{\pgfqpoint{1.636067in}{0.605967in}}%
\pgfpathlineto{\pgfqpoint{1.636623in}{0.600672in}}%
\pgfpathlineto{\pgfqpoint{1.637179in}{0.601643in}}%
\pgfpathlineto{\pgfqpoint{1.638291in}{0.609774in}}%
\pgfpathlineto{\pgfqpoint{1.639403in}{0.602005in}}%
\pgfpathlineto{\pgfqpoint{1.639958in}{0.606864in}}%
\pgfpathlineto{\pgfqpoint{1.640514in}{0.602845in}}%
\pgfpathlineto{\pgfqpoint{1.641626in}{0.604310in}}%
\pgfpathlineto{\pgfqpoint{1.642182in}{0.601730in}}%
\pgfpathlineto{\pgfqpoint{1.642738in}{0.602971in}}%
\pgfpathlineto{\pgfqpoint{1.644961in}{0.604049in}}%
\pgfpathlineto{\pgfqpoint{1.645517in}{0.600586in}}%
\pgfpathlineto{\pgfqpoint{1.646073in}{0.604391in}}%
\pgfpathlineto{\pgfqpoint{1.646629in}{0.601215in}}%
\pgfpathlineto{\pgfqpoint{1.647740in}{0.606087in}}%
\pgfpathlineto{\pgfqpoint{1.648296in}{0.604325in}}%
\pgfpathlineto{\pgfqpoint{1.648852in}{0.601455in}}%
\pgfpathlineto{\pgfqpoint{1.649408in}{0.602695in}}%
\pgfpathlineto{\pgfqpoint{1.649964in}{0.604447in}}%
\pgfpathlineto{\pgfqpoint{1.650520in}{0.601786in}}%
\pgfpathlineto{\pgfqpoint{1.651076in}{0.607381in}}%
\pgfpathlineto{\pgfqpoint{1.651631in}{0.604667in}}%
\pgfpathlineto{\pgfqpoint{1.652187in}{0.604073in}}%
\pgfpathlineto{\pgfqpoint{1.652743in}{0.601736in}}%
\pgfpathlineto{\pgfqpoint{1.653299in}{0.604383in}}%
\pgfpathlineto{\pgfqpoint{1.653855in}{0.602952in}}%
\pgfpathlineto{\pgfqpoint{1.654967in}{0.603688in}}%
\pgfpathlineto{\pgfqpoint{1.656078in}{0.600882in}}%
\pgfpathlineto{\pgfqpoint{1.657190in}{0.606996in}}%
\pgfpathlineto{\pgfqpoint{1.657746in}{0.605667in}}%
\pgfpathlineto{\pgfqpoint{1.658302in}{0.605612in}}%
\pgfpathlineto{\pgfqpoint{1.658858in}{0.601910in}}%
\pgfpathlineto{\pgfqpoint{1.659414in}{0.604034in}}%
\pgfpathlineto{\pgfqpoint{1.659969in}{0.605200in}}%
\pgfpathlineto{\pgfqpoint{1.660525in}{0.604538in}}%
\pgfpathlineto{\pgfqpoint{1.661637in}{0.601675in}}%
\pgfpathlineto{\pgfqpoint{1.663305in}{0.605567in}}%
\pgfpathlineto{\pgfqpoint{1.664972in}{0.601996in}}%
\pgfpathlineto{\pgfqpoint{1.666084in}{0.603360in}}%
\pgfpathlineto{\pgfqpoint{1.666640in}{0.600886in}}%
\pgfpathlineto{\pgfqpoint{1.667196in}{0.602640in}}%
\pgfpathlineto{\pgfqpoint{1.667751in}{0.602732in}}%
\pgfpathlineto{\pgfqpoint{1.668307in}{0.604765in}}%
\pgfpathlineto{\pgfqpoint{1.669975in}{0.602495in}}%
\pgfpathlineto{\pgfqpoint{1.670531in}{0.603971in}}%
\pgfpathlineto{\pgfqpoint{1.671087in}{0.603027in}}%
\pgfpathlineto{\pgfqpoint{1.672198in}{0.602584in}}%
\pgfpathlineto{\pgfqpoint{1.672754in}{0.603676in}}%
\pgfpathlineto{\pgfqpoint{1.673310in}{0.600040in}}%
\pgfpathlineto{\pgfqpoint{1.673866in}{0.600592in}}%
\pgfpathlineto{\pgfqpoint{1.675534in}{0.602943in}}%
\pgfpathlineto{\pgfqpoint{1.676089in}{0.600725in}}%
\pgfpathlineto{\pgfqpoint{1.676645in}{0.603948in}}%
\pgfpathlineto{\pgfqpoint{1.677201in}{0.602356in}}%
\pgfpathlineto{\pgfqpoint{1.677757in}{0.602832in}}%
\pgfpathlineto{\pgfqpoint{1.678313in}{0.601287in}}%
\pgfpathlineto{\pgfqpoint{1.678869in}{0.601718in}}%
\pgfpathlineto{\pgfqpoint{1.679980in}{0.602614in}}%
\pgfpathlineto{\pgfqpoint{1.681648in}{0.602830in}}%
\pgfpathlineto{\pgfqpoint{1.682760in}{0.600853in}}%
\pgfpathlineto{\pgfqpoint{1.683316in}{0.603515in}}%
\pgfpathlineto{\pgfqpoint{1.683871in}{0.603200in}}%
\pgfpathlineto{\pgfqpoint{1.685539in}{0.602180in}}%
\pgfpathlineto{\pgfqpoint{1.686095in}{0.605482in}}%
\pgfpathlineto{\pgfqpoint{1.687762in}{0.601714in}}%
\pgfpathlineto{\pgfqpoint{1.688318in}{0.600858in}}%
\pgfpathlineto{\pgfqpoint{1.688874in}{0.605591in}}%
\pgfpathlineto{\pgfqpoint{1.689430in}{0.601990in}}%
\pgfpathlineto{\pgfqpoint{1.689986in}{0.602787in}}%
\pgfpathlineto{\pgfqpoint{1.690542in}{0.601964in}}%
\pgfpathlineto{\pgfqpoint{1.691653in}{0.600991in}}%
\pgfpathlineto{\pgfqpoint{1.692209in}{0.602918in}}%
\pgfpathlineto{\pgfqpoint{1.692765in}{0.602038in}}%
\pgfpathlineto{\pgfqpoint{1.694433in}{0.600907in}}%
\pgfpathlineto{\pgfqpoint{1.695545in}{0.602792in}}%
\pgfpathlineto{\pgfqpoint{1.696656in}{0.600551in}}%
\pgfpathlineto{\pgfqpoint{1.697212in}{0.600716in}}%
\pgfpathlineto{\pgfqpoint{1.697768in}{0.603284in}}%
\pgfpathlineto{\pgfqpoint{1.698880in}{0.600152in}}%
\pgfpathlineto{\pgfqpoint{1.699436in}{0.601866in}}%
\pgfpathlineto{\pgfqpoint{1.699991in}{0.600502in}}%
\pgfpathlineto{\pgfqpoint{1.701659in}{0.601724in}}%
\pgfpathlineto{\pgfqpoint{1.706106in}{0.601265in}}%
\pgfpathlineto{\pgfqpoint{1.707218in}{0.600376in}}%
\pgfpathlineto{\pgfqpoint{1.707773in}{0.601315in}}%
\pgfpathlineto{\pgfqpoint{1.708329in}{0.600894in}}%
\pgfpathlineto{\pgfqpoint{1.708885in}{0.600495in}}%
\pgfpathlineto{\pgfqpoint{1.710553in}{0.602656in}}%
\pgfpathlineto{\pgfqpoint{1.712220in}{0.600553in}}%
\pgfpathlineto{\pgfqpoint{1.713888in}{0.600607in}}%
\pgfpathlineto{\pgfqpoint{1.715556in}{0.602001in}}%
\pgfpathlineto{\pgfqpoint{1.716667in}{0.600682in}}%
\pgfpathlineto{\pgfqpoint{1.717223in}{0.601339in}}%
\pgfpathlineto{\pgfqpoint{1.717779in}{0.600525in}}%
\pgfpathlineto{\pgfqpoint{1.719447in}{0.601314in}}%
\pgfpathlineto{\pgfqpoint{1.720002in}{0.600051in}}%
\pgfpathlineto{\pgfqpoint{1.720558in}{0.602632in}}%
\pgfpathlineto{\pgfqpoint{1.721114in}{0.601215in}}%
\pgfpathlineto{\pgfqpoint{1.726673in}{0.600132in}}%
\pgfpathlineto{\pgfqpoint{1.727229in}{0.601289in}}%
\pgfpathlineto{\pgfqpoint{1.727784in}{0.600798in}}%
\pgfpathlineto{\pgfqpoint{1.728340in}{0.600369in}}%
\pgfpathlineto{\pgfqpoint{1.728896in}{0.601810in}}%
\pgfpathlineto{\pgfqpoint{1.729452in}{0.600603in}}%
\pgfpathlineto{\pgfqpoint{1.731676in}{0.601194in}}%
\pgfpathlineto{\pgfqpoint{1.740013in}{0.601515in}}%
\pgfpathlineto{\pgfqpoint{1.741125in}{0.601975in}}%
\pgfpathlineto{\pgfqpoint{1.741681in}{0.601017in}}%
\pgfpathlineto{\pgfqpoint{1.742237in}{0.601345in}}%
\pgfpathlineto{\pgfqpoint{1.743904in}{0.602623in}}%
\pgfpathlineto{\pgfqpoint{1.744460in}{0.600596in}}%
\pgfpathlineto{\pgfqpoint{1.745572in}{0.603681in}}%
\pgfpathlineto{\pgfqpoint{1.746128in}{0.602087in}}%
\pgfpathlineto{\pgfqpoint{1.746684in}{0.604142in}}%
\pgfpathlineto{\pgfqpoint{1.747240in}{0.601043in}}%
\pgfpathlineto{\pgfqpoint{1.748351in}{0.605232in}}%
\pgfpathlineto{\pgfqpoint{1.748907in}{0.603506in}}%
\pgfpathlineto{\pgfqpoint{1.749463in}{0.605755in}}%
\pgfpathlineto{\pgfqpoint{1.750019in}{0.603724in}}%
\pgfpathlineto{\pgfqpoint{1.754466in}{0.615872in}}%
\pgfpathlineto{\pgfqpoint{1.755022in}{0.632822in}}%
\pgfpathlineto{\pgfqpoint{1.755578in}{0.614627in}}%
\pgfpathlineto{\pgfqpoint{1.756133in}{0.697829in}}%
\pgfpathlineto{\pgfqpoint{1.756689in}{0.647657in}}%
\pgfpathlineto{\pgfqpoint{1.757245in}{0.633427in}}%
\pgfpathlineto{\pgfqpoint{1.757801in}{0.699410in}}%
\pgfpathlineto{\pgfqpoint{1.758357in}{0.610519in}}%
\pgfpathlineto{\pgfqpoint{1.758913in}{0.630947in}}%
\pgfpathlineto{\pgfqpoint{1.760580in}{0.610848in}}%
\pgfpathlineto{\pgfqpoint{1.762248in}{0.607320in}}%
\pgfpathlineto{\pgfqpoint{1.762804in}{0.606332in}}%
\pgfpathlineto{\pgfqpoint{1.763915in}{0.608663in}}%
\pgfpathlineto{\pgfqpoint{1.765027in}{0.605224in}}%
\pgfpathlineto{\pgfqpoint{1.765583in}{0.606568in}}%
\pgfpathlineto{\pgfqpoint{1.767251in}{0.605229in}}%
\pgfpathlineto{\pgfqpoint{1.767807in}{0.608019in}}%
\pgfpathlineto{\pgfqpoint{1.768362in}{0.602562in}}%
\pgfpathlineto{\pgfqpoint{1.768918in}{0.602951in}}%
\pgfpathlineto{\pgfqpoint{1.770586in}{0.603400in}}%
\pgfpathlineto{\pgfqpoint{1.771142in}{0.605325in}}%
\pgfpathlineto{\pgfqpoint{1.771698in}{0.603768in}}%
\pgfpathlineto{\pgfqpoint{1.772253in}{0.601823in}}%
\pgfpathlineto{\pgfqpoint{1.772809in}{0.605269in}}%
\pgfpathlineto{\pgfqpoint{1.773365in}{0.605098in}}%
\pgfpathlineto{\pgfqpoint{1.774477in}{0.600606in}}%
\pgfpathlineto{\pgfqpoint{1.775589in}{0.605006in}}%
\pgfpathlineto{\pgfqpoint{1.776144in}{0.604891in}}%
\pgfpathlineto{\pgfqpoint{1.776700in}{0.605624in}}%
\pgfpathlineto{\pgfqpoint{1.777256in}{0.602467in}}%
\pgfpathlineto{\pgfqpoint{1.777812in}{0.604550in}}%
\pgfpathlineto{\pgfqpoint{1.778368in}{0.606327in}}%
\pgfpathlineto{\pgfqpoint{1.778924in}{0.601487in}}%
\pgfpathlineto{\pgfqpoint{1.779480in}{0.603616in}}%
\pgfpathlineto{\pgfqpoint{1.780035in}{0.602666in}}%
\pgfpathlineto{\pgfqpoint{1.780591in}{0.604869in}}%
\pgfpathlineto{\pgfqpoint{1.781147in}{0.601152in}}%
\pgfpathlineto{\pgfqpoint{1.781703in}{0.602757in}}%
\pgfpathlineto{\pgfqpoint{1.782815in}{0.604926in}}%
\pgfpathlineto{\pgfqpoint{1.783371in}{0.601123in}}%
\pgfpathlineto{\pgfqpoint{1.783926in}{0.603164in}}%
\pgfpathlineto{\pgfqpoint{1.785038in}{0.604889in}}%
\pgfpathlineto{\pgfqpoint{1.785594in}{0.603173in}}%
\pgfpathlineto{\pgfqpoint{1.786150in}{0.605279in}}%
\pgfpathlineto{\pgfqpoint{1.786706in}{0.601237in}}%
\pgfpathlineto{\pgfqpoint{1.787262in}{0.603604in}}%
\pgfpathlineto{\pgfqpoint{1.788929in}{0.601956in}}%
\pgfpathlineto{\pgfqpoint{1.790041in}{0.601818in}}%
\pgfpathlineto{\pgfqpoint{1.790597in}{0.604771in}}%
\pgfpathlineto{\pgfqpoint{1.791153in}{0.602686in}}%
\pgfpathlineto{\pgfqpoint{1.791709in}{0.602446in}}%
\pgfpathlineto{\pgfqpoint{1.792264in}{0.606153in}}%
\pgfpathlineto{\pgfqpoint{1.792820in}{0.600759in}}%
\pgfpathlineto{\pgfqpoint{1.793376in}{0.601589in}}%
\pgfpathlineto{\pgfqpoint{1.795044in}{0.604851in}}%
\pgfpathlineto{\pgfqpoint{1.795600in}{0.602917in}}%
\pgfpathlineto{\pgfqpoint{1.796155in}{0.603516in}}%
\pgfpathlineto{\pgfqpoint{1.797267in}{0.604768in}}%
\pgfpathlineto{\pgfqpoint{1.797823in}{0.600086in}}%
\pgfpathlineto{\pgfqpoint{1.798379in}{0.604582in}}%
\pgfpathlineto{\pgfqpoint{1.798935in}{0.601826in}}%
\pgfpathlineto{\pgfqpoint{1.799491in}{0.602340in}}%
\pgfpathlineto{\pgfqpoint{1.800046in}{0.603717in}}%
\pgfpathlineto{\pgfqpoint{1.800602in}{0.601586in}}%
\pgfpathlineto{\pgfqpoint{1.802270in}{0.607781in}}%
\pgfpathlineto{\pgfqpoint{1.802826in}{0.602945in}}%
\pgfpathlineto{\pgfqpoint{1.803937in}{0.612754in}}%
\pgfpathlineto{\pgfqpoint{1.805049in}{0.601870in}}%
\pgfpathlineto{\pgfqpoint{1.805605in}{0.602245in}}%
\pgfpathlineto{\pgfqpoint{1.806161in}{0.608293in}}%
\pgfpathlineto{\pgfqpoint{1.806717in}{0.608159in}}%
\pgfpathlineto{\pgfqpoint{1.808940in}{0.603447in}}%
\pgfpathlineto{\pgfqpoint{1.810608in}{0.606295in}}%
\pgfpathlineto{\pgfqpoint{1.812831in}{0.604732in}}%
\pgfpathlineto{\pgfqpoint{1.813387in}{0.601859in}}%
\pgfpathlineto{\pgfqpoint{1.813943in}{0.604350in}}%
\pgfpathlineto{\pgfqpoint{1.815055in}{0.606697in}}%
\pgfpathlineto{\pgfqpoint{1.816722in}{0.601020in}}%
\pgfpathlineto{\pgfqpoint{1.817834in}{0.604994in}}%
\pgfpathlineto{\pgfqpoint{1.819502in}{0.602555in}}%
\pgfpathlineto{\pgfqpoint{1.820057in}{0.607824in}}%
\pgfpathlineto{\pgfqpoint{1.820613in}{0.606841in}}%
\pgfpathlineto{\pgfqpoint{1.821725in}{0.602687in}}%
\pgfpathlineto{\pgfqpoint{1.822281in}{0.606060in}}%
\pgfpathlineto{\pgfqpoint{1.822837in}{0.602500in}}%
\pgfpathlineto{\pgfqpoint{1.823393in}{0.606140in}}%
\pgfpathlineto{\pgfqpoint{1.825060in}{0.601248in}}%
\pgfpathlineto{\pgfqpoint{1.825616in}{0.610545in}}%
\pgfpathlineto{\pgfqpoint{1.826172in}{0.601376in}}%
\pgfpathlineto{\pgfqpoint{1.827840in}{0.609275in}}%
\pgfpathlineto{\pgfqpoint{1.828951in}{0.605676in}}%
\pgfpathlineto{\pgfqpoint{1.829507in}{0.607299in}}%
\pgfpathlineto{\pgfqpoint{1.830063in}{0.610865in}}%
\pgfpathlineto{\pgfqpoint{1.830619in}{0.601363in}}%
\pgfpathlineto{\pgfqpoint{1.831175in}{0.606357in}}%
\pgfpathlineto{\pgfqpoint{1.831731in}{0.604472in}}%
\pgfpathlineto{\pgfqpoint{1.832286in}{0.605897in}}%
\pgfpathlineto{\pgfqpoint{1.833398in}{0.610213in}}%
\pgfpathlineto{\pgfqpoint{1.833954in}{0.602946in}}%
\pgfpathlineto{\pgfqpoint{1.834510in}{0.608971in}}%
\pgfpathlineto{\pgfqpoint{1.835066in}{0.603378in}}%
\pgfpathlineto{\pgfqpoint{1.835622in}{0.610541in}}%
\pgfpathlineto{\pgfqpoint{1.836177in}{0.610424in}}%
\pgfpathlineto{\pgfqpoint{1.836733in}{0.608923in}}%
\pgfpathlineto{\pgfqpoint{1.837289in}{0.600993in}}%
\pgfpathlineto{\pgfqpoint{1.837845in}{0.602661in}}%
\pgfpathlineto{\pgfqpoint{1.838401in}{0.606840in}}%
\pgfpathlineto{\pgfqpoint{1.838957in}{0.603616in}}%
\pgfpathlineto{\pgfqpoint{1.840624in}{0.610174in}}%
\pgfpathlineto{\pgfqpoint{1.841736in}{0.602760in}}%
\pgfpathlineto{\pgfqpoint{1.842292in}{0.603111in}}%
\pgfpathlineto{\pgfqpoint{1.842848in}{0.609399in}}%
\pgfpathlineto{\pgfqpoint{1.843404in}{0.603794in}}%
\pgfpathlineto{\pgfqpoint{1.843960in}{0.608760in}}%
\pgfpathlineto{\pgfqpoint{1.844515in}{0.606253in}}%
\pgfpathlineto{\pgfqpoint{1.845071in}{0.605429in}}%
\pgfpathlineto{\pgfqpoint{1.845627in}{0.600105in}}%
\pgfpathlineto{\pgfqpoint{1.846183in}{0.602829in}}%
\pgfpathlineto{\pgfqpoint{1.846739in}{0.602440in}}%
\pgfpathlineto{\pgfqpoint{1.847295in}{0.605121in}}%
\pgfpathlineto{\pgfqpoint{1.847851in}{0.603186in}}%
\pgfpathlineto{\pgfqpoint{1.848406in}{0.603461in}}%
\pgfpathlineto{\pgfqpoint{1.850074in}{0.610589in}}%
\pgfpathlineto{\pgfqpoint{1.850630in}{0.602168in}}%
\pgfpathlineto{\pgfqpoint{1.851186in}{0.604544in}}%
\pgfpathlineto{\pgfqpoint{1.852297in}{0.609383in}}%
\pgfpathlineto{\pgfqpoint{1.853409in}{0.603026in}}%
\pgfpathlineto{\pgfqpoint{1.853965in}{0.607003in}}%
\pgfpathlineto{\pgfqpoint{1.854521in}{0.602573in}}%
\pgfpathlineto{\pgfqpoint{1.855077in}{0.606670in}}%
\pgfpathlineto{\pgfqpoint{1.856188in}{0.603003in}}%
\pgfpathlineto{\pgfqpoint{1.858412in}{0.607304in}}%
\pgfpathlineto{\pgfqpoint{1.858968in}{0.605200in}}%
\pgfpathlineto{\pgfqpoint{1.859524in}{0.608500in}}%
\pgfpathlineto{\pgfqpoint{1.860079in}{0.607515in}}%
\pgfpathlineto{\pgfqpoint{1.860635in}{0.605530in}}%
\pgfpathlineto{\pgfqpoint{1.861191in}{0.612801in}}%
\pgfpathlineto{\pgfqpoint{1.861747in}{0.610005in}}%
\pgfpathlineto{\pgfqpoint{1.862303in}{0.609314in}}%
\pgfpathlineto{\pgfqpoint{1.863415in}{0.614965in}}%
\pgfpathlineto{\pgfqpoint{1.863971in}{0.603602in}}%
\pgfpathlineto{\pgfqpoint{1.864526in}{0.610707in}}%
\pgfpathlineto{\pgfqpoint{1.865082in}{0.612508in}}%
\pgfpathlineto{\pgfqpoint{1.865638in}{0.611878in}}%
\pgfpathlineto{\pgfqpoint{1.866194in}{0.604566in}}%
\pgfpathlineto{\pgfqpoint{1.867862in}{0.616206in}}%
\pgfpathlineto{\pgfqpoint{1.868973in}{0.608187in}}%
\pgfpathlineto{\pgfqpoint{1.869529in}{0.608637in}}%
\pgfpathlineto{\pgfqpoint{1.870085in}{0.607568in}}%
\pgfpathlineto{\pgfqpoint{1.870641in}{0.613690in}}%
\pgfpathlineto{\pgfqpoint{1.871197in}{0.605556in}}%
\pgfpathlineto{\pgfqpoint{1.871753in}{0.609954in}}%
\pgfpathlineto{\pgfqpoint{1.872864in}{0.605257in}}%
\pgfpathlineto{\pgfqpoint{1.874532in}{0.616464in}}%
\pgfpathlineto{\pgfqpoint{1.875644in}{0.602478in}}%
\pgfpathlineto{\pgfqpoint{1.878423in}{0.614031in}}%
\pgfpathlineto{\pgfqpoint{1.878979in}{0.603082in}}%
\pgfpathlineto{\pgfqpoint{1.879535in}{0.613868in}}%
\pgfpathlineto{\pgfqpoint{1.881202in}{0.606433in}}%
\pgfpathlineto{\pgfqpoint{1.881758in}{0.608081in}}%
\pgfpathlineto{\pgfqpoint{1.882870in}{0.605166in}}%
\pgfpathlineto{\pgfqpoint{1.883982in}{0.610962in}}%
\pgfpathlineto{\pgfqpoint{1.884537in}{0.604666in}}%
\pgfpathlineto{\pgfqpoint{1.885093in}{0.618142in}}%
\pgfpathlineto{\pgfqpoint{1.885649in}{0.610081in}}%
\pgfpathlineto{\pgfqpoint{1.886205in}{0.607775in}}%
\pgfpathlineto{\pgfqpoint{1.887317in}{0.615989in}}%
\pgfpathlineto{\pgfqpoint{1.888984in}{0.602893in}}%
\pgfpathlineto{\pgfqpoint{1.890096in}{0.608172in}}%
\pgfpathlineto{\pgfqpoint{1.890652in}{0.618978in}}%
\pgfpathlineto{\pgfqpoint{1.891208in}{0.609600in}}%
\pgfpathlineto{\pgfqpoint{1.891764in}{0.614082in}}%
\pgfpathlineto{\pgfqpoint{1.892319in}{0.608142in}}%
\pgfpathlineto{\pgfqpoint{1.892875in}{0.611987in}}%
\pgfpathlineto{\pgfqpoint{1.893431in}{0.611486in}}%
\pgfpathlineto{\pgfqpoint{1.893987in}{0.616145in}}%
\pgfpathlineto{\pgfqpoint{1.895099in}{0.605367in}}%
\pgfpathlineto{\pgfqpoint{1.895655in}{0.609795in}}%
\pgfpathlineto{\pgfqpoint{1.896210in}{0.607643in}}%
\pgfpathlineto{\pgfqpoint{1.896766in}{0.603553in}}%
\pgfpathlineto{\pgfqpoint{1.897878in}{0.610452in}}%
\pgfpathlineto{\pgfqpoint{1.899546in}{0.602759in}}%
\pgfpathlineto{\pgfqpoint{1.901213in}{0.608910in}}%
\pgfpathlineto{\pgfqpoint{1.901769in}{0.601642in}}%
\pgfpathlineto{\pgfqpoint{1.902325in}{0.610496in}}%
\pgfpathlineto{\pgfqpoint{1.902881in}{0.606933in}}%
\pgfpathlineto{\pgfqpoint{1.903993in}{0.608384in}}%
\pgfpathlineto{\pgfqpoint{1.905104in}{0.601949in}}%
\pgfpathlineto{\pgfqpoint{1.907328in}{0.612133in}}%
\pgfpathlineto{\pgfqpoint{1.907884in}{0.606559in}}%
\pgfpathlineto{\pgfqpoint{1.908439in}{0.610918in}}%
\pgfpathlineto{\pgfqpoint{1.908995in}{0.607236in}}%
\pgfpathlineto{\pgfqpoint{1.909551in}{0.613064in}}%
\pgfpathlineto{\pgfqpoint{1.910107in}{0.603625in}}%
\pgfpathlineto{\pgfqpoint{1.910663in}{0.607320in}}%
\pgfpathlineto{\pgfqpoint{1.911219in}{0.608816in}}%
\pgfpathlineto{\pgfqpoint{1.912886in}{0.603724in}}%
\pgfpathlineto{\pgfqpoint{1.913998in}{0.602757in}}%
\pgfpathlineto{\pgfqpoint{1.915666in}{0.611352in}}%
\pgfpathlineto{\pgfqpoint{1.916221in}{0.602422in}}%
\pgfpathlineto{\pgfqpoint{1.916777in}{0.604191in}}%
\pgfpathlineto{\pgfqpoint{1.919001in}{0.615648in}}%
\pgfpathlineto{\pgfqpoint{1.919557in}{0.606749in}}%
\pgfpathlineto{\pgfqpoint{1.920113in}{0.608523in}}%
\pgfpathlineto{\pgfqpoint{1.921780in}{0.617725in}}%
\pgfpathlineto{\pgfqpoint{1.923448in}{0.606729in}}%
\pgfpathlineto{\pgfqpoint{1.924559in}{0.615705in}}%
\pgfpathlineto{\pgfqpoint{1.925115in}{0.602248in}}%
\pgfpathlineto{\pgfqpoint{1.925671in}{0.603755in}}%
\pgfpathlineto{\pgfqpoint{1.926783in}{0.621237in}}%
\pgfpathlineto{\pgfqpoint{1.927339in}{0.607225in}}%
\pgfpathlineto{\pgfqpoint{1.927895in}{0.614557in}}%
\pgfpathlineto{\pgfqpoint{1.929562in}{0.605980in}}%
\pgfpathlineto{\pgfqpoint{1.930118in}{0.605972in}}%
\pgfpathlineto{\pgfqpoint{1.930674in}{0.602664in}}%
\pgfpathlineto{\pgfqpoint{1.931230in}{0.608724in}}%
\pgfpathlineto{\pgfqpoint{1.932341in}{0.602411in}}%
\pgfpathlineto{\pgfqpoint{1.934009in}{0.614434in}}%
\pgfpathlineto{\pgfqpoint{1.935121in}{0.603864in}}%
\pgfpathlineto{\pgfqpoint{1.935677in}{0.604268in}}%
\pgfpathlineto{\pgfqpoint{1.936788in}{0.609800in}}%
\pgfpathlineto{\pgfqpoint{1.937344in}{0.602343in}}%
\pgfpathlineto{\pgfqpoint{1.937900in}{0.614393in}}%
\pgfpathlineto{\pgfqpoint{1.938456in}{0.605394in}}%
\pgfpathlineto{\pgfqpoint{1.939012in}{0.605632in}}%
\pgfpathlineto{\pgfqpoint{1.939568in}{0.607254in}}%
\pgfpathlineto{\pgfqpoint{1.940124in}{0.612902in}}%
\pgfpathlineto{\pgfqpoint{1.940679in}{0.606353in}}%
\pgfpathlineto{\pgfqpoint{1.941235in}{0.607292in}}%
\pgfpathlineto{\pgfqpoint{1.944015in}{0.616238in}}%
\pgfpathlineto{\pgfqpoint{1.944570in}{0.601167in}}%
\pgfpathlineto{\pgfqpoint{1.945126in}{0.611013in}}%
\pgfpathlineto{\pgfqpoint{1.946794in}{0.601139in}}%
\pgfpathlineto{\pgfqpoint{1.948461in}{0.620487in}}%
\pgfpathlineto{\pgfqpoint{1.949017in}{0.604204in}}%
\pgfpathlineto{\pgfqpoint{1.949573in}{0.611260in}}%
\pgfpathlineto{\pgfqpoint{1.950129in}{0.610085in}}%
\pgfpathlineto{\pgfqpoint{1.950685in}{0.603834in}}%
\pgfpathlineto{\pgfqpoint{1.951241in}{0.616159in}}%
\pgfpathlineto{\pgfqpoint{1.951797in}{0.611992in}}%
\pgfpathlineto{\pgfqpoint{1.952352in}{0.602486in}}%
\pgfpathlineto{\pgfqpoint{1.952908in}{0.616057in}}%
\pgfpathlineto{\pgfqpoint{1.953464in}{0.604038in}}%
\pgfpathlineto{\pgfqpoint{1.954020in}{0.602739in}}%
\pgfpathlineto{\pgfqpoint{1.955132in}{0.613768in}}%
\pgfpathlineto{\pgfqpoint{1.955688in}{0.610194in}}%
\pgfpathlineto{\pgfqpoint{1.956244in}{0.601207in}}%
\pgfpathlineto{\pgfqpoint{1.956799in}{0.604360in}}%
\pgfpathlineto{\pgfqpoint{1.957355in}{0.606723in}}%
\pgfpathlineto{\pgfqpoint{1.957911in}{0.605019in}}%
\pgfpathlineto{\pgfqpoint{1.958467in}{0.605959in}}%
\pgfpathlineto{\pgfqpoint{1.959023in}{0.602482in}}%
\pgfpathlineto{\pgfqpoint{1.959579in}{0.605378in}}%
\pgfpathlineto{\pgfqpoint{1.960135in}{0.606705in}}%
\pgfpathlineto{\pgfqpoint{1.961246in}{0.601009in}}%
\pgfpathlineto{\pgfqpoint{1.962358in}{0.606288in}}%
\pgfpathlineto{\pgfqpoint{1.962914in}{0.603016in}}%
\pgfpathlineto{\pgfqpoint{1.965137in}{0.611106in}}%
\pgfpathlineto{\pgfqpoint{1.966249in}{0.601670in}}%
\pgfpathlineto{\pgfqpoint{1.966805in}{0.606945in}}%
\pgfpathlineto{\pgfqpoint{1.967361in}{0.606183in}}%
\pgfpathlineto{\pgfqpoint{1.967917in}{0.603511in}}%
\pgfpathlineto{\pgfqpoint{1.968472in}{0.608943in}}%
\pgfpathlineto{\pgfqpoint{1.969028in}{0.606560in}}%
\pgfpathlineto{\pgfqpoint{1.970696in}{0.603136in}}%
\pgfpathlineto{\pgfqpoint{1.971252in}{0.601525in}}%
\pgfpathlineto{\pgfqpoint{1.972363in}{0.610577in}}%
\pgfpathlineto{\pgfqpoint{1.972919in}{0.606253in}}%
\pgfpathlineto{\pgfqpoint{1.973475in}{0.605332in}}%
\pgfpathlineto{\pgfqpoint{1.974031in}{0.607796in}}%
\pgfpathlineto{\pgfqpoint{1.974587in}{0.601313in}}%
\pgfpathlineto{\pgfqpoint{1.975143in}{0.601600in}}%
\pgfpathlineto{\pgfqpoint{1.976810in}{0.612413in}}%
\pgfpathlineto{\pgfqpoint{1.977366in}{0.602095in}}%
\pgfpathlineto{\pgfqpoint{1.977922in}{0.606599in}}%
\pgfpathlineto{\pgfqpoint{1.978478in}{0.613212in}}%
\pgfpathlineto{\pgfqpoint{1.979034in}{0.603371in}}%
\pgfpathlineto{\pgfqpoint{1.979590in}{0.612922in}}%
\pgfpathlineto{\pgfqpoint{1.980701in}{0.604486in}}%
\pgfpathlineto{\pgfqpoint{1.981257in}{0.611487in}}%
\pgfpathlineto{\pgfqpoint{1.981813in}{0.610427in}}%
\pgfpathlineto{\pgfqpoint{1.982369in}{0.611208in}}%
\pgfpathlineto{\pgfqpoint{1.982925in}{0.610620in}}%
\pgfpathlineto{\pgfqpoint{1.983481in}{0.603301in}}%
\pgfpathlineto{\pgfqpoint{1.984037in}{0.613150in}}%
\pgfpathlineto{\pgfqpoint{1.984592in}{0.606353in}}%
\pgfpathlineto{\pgfqpoint{1.985148in}{0.605421in}}%
\pgfpathlineto{\pgfqpoint{1.985704in}{0.615764in}}%
\pgfpathlineto{\pgfqpoint{1.986260in}{0.614778in}}%
\pgfpathlineto{\pgfqpoint{1.986816in}{0.606068in}}%
\pgfpathlineto{\pgfqpoint{1.987372in}{0.610158in}}%
\pgfpathlineto{\pgfqpoint{1.987928in}{0.609274in}}%
\pgfpathlineto{\pgfqpoint{1.988483in}{0.602175in}}%
\pgfpathlineto{\pgfqpoint{1.989039in}{0.602975in}}%
\pgfpathlineto{\pgfqpoint{1.989595in}{0.603831in}}%
\pgfpathlineto{\pgfqpoint{1.990151in}{0.608838in}}%
\pgfpathlineto{\pgfqpoint{1.990707in}{0.600644in}}%
\pgfpathlineto{\pgfqpoint{1.991263in}{0.603762in}}%
\pgfpathlineto{\pgfqpoint{1.991819in}{0.602488in}}%
\pgfpathlineto{\pgfqpoint{1.992930in}{0.612973in}}%
\pgfpathlineto{\pgfqpoint{1.993486in}{0.602472in}}%
\pgfpathlineto{\pgfqpoint{1.994042in}{0.607495in}}%
\pgfpathlineto{\pgfqpoint{1.994598in}{0.605378in}}%
\pgfpathlineto{\pgfqpoint{1.995710in}{0.610731in}}%
\pgfpathlineto{\pgfqpoint{1.996266in}{0.603072in}}%
\pgfpathlineto{\pgfqpoint{1.996821in}{0.603149in}}%
\pgfpathlineto{\pgfqpoint{1.998489in}{0.605874in}}%
\pgfpathlineto{\pgfqpoint{1.999045in}{0.602953in}}%
\pgfpathlineto{\pgfqpoint{2.000712in}{0.615292in}}%
\pgfpathlineto{\pgfqpoint{2.001268in}{0.600601in}}%
\pgfpathlineto{\pgfqpoint{2.001824in}{0.608760in}}%
\pgfpathlineto{\pgfqpoint{2.002380in}{0.603205in}}%
\pgfpathlineto{\pgfqpoint{2.002936in}{0.614229in}}%
\pgfpathlineto{\pgfqpoint{2.003492in}{0.606509in}}%
\pgfpathlineto{\pgfqpoint{2.005715in}{0.613275in}}%
\pgfpathlineto{\pgfqpoint{2.006271in}{0.604860in}}%
\pgfpathlineto{\pgfqpoint{2.006827in}{0.606526in}}%
\pgfpathlineto{\pgfqpoint{2.007383in}{0.608648in}}%
\pgfpathlineto{\pgfqpoint{2.007939in}{0.608088in}}%
\pgfpathlineto{\pgfqpoint{2.009050in}{0.607044in}}%
\pgfpathlineto{\pgfqpoint{2.009606in}{0.602955in}}%
\pgfpathlineto{\pgfqpoint{2.010162in}{0.612656in}}%
\pgfpathlineto{\pgfqpoint{2.010718in}{0.609520in}}%
\pgfpathlineto{\pgfqpoint{2.011274in}{0.608701in}}%
\pgfpathlineto{\pgfqpoint{2.011830in}{0.604344in}}%
\pgfpathlineto{\pgfqpoint{2.012386in}{0.606787in}}%
\pgfpathlineto{\pgfqpoint{2.012941in}{0.610344in}}%
\pgfpathlineto{\pgfqpoint{2.014609in}{0.601188in}}%
\pgfpathlineto{\pgfqpoint{2.016277in}{0.605447in}}%
\pgfpathlineto{\pgfqpoint{2.017944in}{0.601529in}}%
\pgfpathlineto{\pgfqpoint{2.018500in}{0.602216in}}%
\pgfpathlineto{\pgfqpoint{2.019056in}{0.606543in}}%
\pgfpathlineto{\pgfqpoint{2.019612in}{0.603537in}}%
\pgfpathlineto{\pgfqpoint{2.020168in}{0.601902in}}%
\pgfpathlineto{\pgfqpoint{2.020723in}{0.603677in}}%
\pgfpathlineto{\pgfqpoint{2.022947in}{0.604946in}}%
\pgfpathlineto{\pgfqpoint{2.023503in}{0.602885in}}%
\pgfpathlineto{\pgfqpoint{2.024614in}{0.606375in}}%
\pgfpathlineto{\pgfqpoint{2.025170in}{0.601767in}}%
\pgfpathlineto{\pgfqpoint{2.025726in}{0.605776in}}%
\pgfpathlineto{\pgfqpoint{2.026838in}{0.604805in}}%
\pgfpathlineto{\pgfqpoint{2.027394in}{0.602551in}}%
\pgfpathlineto{\pgfqpoint{2.027950in}{0.603995in}}%
\pgfpathlineto{\pgfqpoint{2.028505in}{0.605999in}}%
\pgfpathlineto{\pgfqpoint{2.029061in}{0.602368in}}%
\pgfpathlineto{\pgfqpoint{2.029617in}{0.607453in}}%
\pgfpathlineto{\pgfqpoint{2.030173in}{0.603472in}}%
\pgfpathlineto{\pgfqpoint{2.031285in}{0.608293in}}%
\pgfpathlineto{\pgfqpoint{2.032952in}{0.601620in}}%
\pgfpathlineto{\pgfqpoint{2.033508in}{0.607840in}}%
\pgfpathlineto{\pgfqpoint{2.034064in}{0.607040in}}%
\pgfpathlineto{\pgfqpoint{2.034620in}{0.602575in}}%
\pgfpathlineto{\pgfqpoint{2.035176in}{0.603908in}}%
\pgfpathlineto{\pgfqpoint{2.035732in}{0.606633in}}%
\pgfpathlineto{\pgfqpoint{2.037399in}{0.603523in}}%
\pgfpathlineto{\pgfqpoint{2.037955in}{0.601718in}}%
\pgfpathlineto{\pgfqpoint{2.038511in}{0.602097in}}%
\pgfpathlineto{\pgfqpoint{2.039623in}{0.601340in}}%
\pgfpathlineto{\pgfqpoint{2.041290in}{0.600612in}}%
\pgfpathlineto{\pgfqpoint{2.043514in}{0.601472in}}%
\pgfpathlineto{\pgfqpoint{2.044070in}{0.601896in}}%
\pgfpathlineto{\pgfqpoint{2.044625in}{0.601048in}}%
\pgfpathlineto{\pgfqpoint{2.047961in}{0.601901in}}%
\pgfpathlineto{\pgfqpoint{2.048516in}{0.601091in}}%
\pgfpathlineto{\pgfqpoint{2.049628in}{0.605676in}}%
\pgfpathlineto{\pgfqpoint{2.050184in}{0.600558in}}%
\pgfpathlineto{\pgfqpoint{2.050740in}{0.603140in}}%
\pgfpathlineto{\pgfqpoint{2.051296in}{0.608478in}}%
\pgfpathlineto{\pgfqpoint{2.051852in}{0.605805in}}%
\pgfpathlineto{\pgfqpoint{2.052963in}{0.603077in}}%
\pgfpathlineto{\pgfqpoint{2.053519in}{0.603217in}}%
\pgfpathlineto{\pgfqpoint{2.054631in}{0.608276in}}%
\pgfpathlineto{\pgfqpoint{2.055187in}{0.606674in}}%
\pgfpathlineto{\pgfqpoint{2.055743in}{0.600748in}}%
\pgfpathlineto{\pgfqpoint{2.056299in}{0.610358in}}%
\pgfpathlineto{\pgfqpoint{2.056854in}{0.606979in}}%
\pgfpathlineto{\pgfqpoint{2.057410in}{0.606121in}}%
\pgfpathlineto{\pgfqpoint{2.057966in}{0.608311in}}%
\pgfpathlineto{\pgfqpoint{2.059078in}{0.601803in}}%
\pgfpathlineto{\pgfqpoint{2.059634in}{0.604750in}}%
\pgfpathlineto{\pgfqpoint{2.060190in}{0.607808in}}%
\pgfpathlineto{\pgfqpoint{2.060745in}{0.604986in}}%
\pgfpathlineto{\pgfqpoint{2.061301in}{0.605677in}}%
\pgfpathlineto{\pgfqpoint{2.061857in}{0.601430in}}%
\pgfpathlineto{\pgfqpoint{2.062413in}{0.601778in}}%
\pgfpathlineto{\pgfqpoint{2.062969in}{0.612102in}}%
\pgfpathlineto{\pgfqpoint{2.063525in}{0.602233in}}%
\pgfpathlineto{\pgfqpoint{2.064081in}{0.601421in}}%
\pgfpathlineto{\pgfqpoint{2.065192in}{0.607508in}}%
\pgfpathlineto{\pgfqpoint{2.065748in}{0.605232in}}%
\pgfpathlineto{\pgfqpoint{2.066304in}{0.602790in}}%
\pgfpathlineto{\pgfqpoint{2.066860in}{0.605716in}}%
\pgfpathlineto{\pgfqpoint{2.067416in}{0.602549in}}%
\pgfpathlineto{\pgfqpoint{2.067972in}{0.615839in}}%
\pgfpathlineto{\pgfqpoint{2.068527in}{0.605829in}}%
\pgfpathlineto{\pgfqpoint{2.069639in}{0.601486in}}%
\pgfpathlineto{\pgfqpoint{2.070195in}{0.608501in}}%
\pgfpathlineto{\pgfqpoint{2.070751in}{0.604823in}}%
\pgfpathlineto{\pgfqpoint{2.072419in}{0.600894in}}%
\pgfpathlineto{\pgfqpoint{2.072974in}{0.601768in}}%
\pgfpathlineto{\pgfqpoint{2.073530in}{0.605001in}}%
\pgfpathlineto{\pgfqpoint{2.074086in}{0.603263in}}%
\pgfpathlineto{\pgfqpoint{2.075198in}{0.605042in}}%
\pgfpathlineto{\pgfqpoint{2.075754in}{0.607090in}}%
\pgfpathlineto{\pgfqpoint{2.076310in}{0.602883in}}%
\pgfpathlineto{\pgfqpoint{2.076865in}{0.604049in}}%
\pgfpathlineto{\pgfqpoint{2.079089in}{0.600049in}}%
\pgfpathlineto{\pgfqpoint{2.079645in}{0.608201in}}%
\pgfpathlineto{\pgfqpoint{2.080201in}{0.605268in}}%
\pgfpathlineto{\pgfqpoint{2.080756in}{0.602198in}}%
\pgfpathlineto{\pgfqpoint{2.081312in}{0.602829in}}%
\pgfpathlineto{\pgfqpoint{2.081868in}{0.606118in}}%
\pgfpathlineto{\pgfqpoint{2.082980in}{0.600937in}}%
\pgfpathlineto{\pgfqpoint{2.084092in}{0.606630in}}%
\pgfpathlineto{\pgfqpoint{2.084647in}{0.601884in}}%
\pgfpathlineto{\pgfqpoint{2.085203in}{0.604046in}}%
\pgfpathlineto{\pgfqpoint{2.086315in}{0.603071in}}%
\pgfpathlineto{\pgfqpoint{2.086871in}{0.605202in}}%
\pgfpathlineto{\pgfqpoint{2.087427in}{0.604453in}}%
\pgfpathlineto{\pgfqpoint{2.088539in}{0.602810in}}%
\pgfpathlineto{\pgfqpoint{2.089094in}{0.604397in}}%
\pgfpathlineto{\pgfqpoint{2.089650in}{0.601442in}}%
\pgfpathlineto{\pgfqpoint{2.090206in}{0.603952in}}%
\pgfpathlineto{\pgfqpoint{2.090762in}{0.603979in}}%
\pgfpathlineto{\pgfqpoint{2.091318in}{0.601829in}}%
\pgfpathlineto{\pgfqpoint{2.091874in}{0.603402in}}%
\pgfpathlineto{\pgfqpoint{2.092430in}{0.603371in}}%
\pgfpathlineto{\pgfqpoint{2.093541in}{0.610790in}}%
\pgfpathlineto{\pgfqpoint{2.094097in}{0.605712in}}%
\pgfpathlineto{\pgfqpoint{2.095209in}{0.607271in}}%
\pgfpathlineto{\pgfqpoint{2.095765in}{0.603478in}}%
\pgfpathlineto{\pgfqpoint{2.096321in}{0.612113in}}%
\pgfpathlineto{\pgfqpoint{2.096876in}{0.606297in}}%
\pgfpathlineto{\pgfqpoint{2.098544in}{0.601857in}}%
\pgfpathlineto{\pgfqpoint{2.099100in}{0.612036in}}%
\pgfpathlineto{\pgfqpoint{2.099656in}{0.606685in}}%
\pgfpathlineto{\pgfqpoint{2.100212in}{0.607830in}}%
\pgfpathlineto{\pgfqpoint{2.100767in}{0.605693in}}%
\pgfpathlineto{\pgfqpoint{2.101323in}{0.610998in}}%
\pgfpathlineto{\pgfqpoint{2.102435in}{0.602013in}}%
\pgfpathlineto{\pgfqpoint{2.102991in}{0.603451in}}%
\pgfpathlineto{\pgfqpoint{2.103547in}{0.602959in}}%
\pgfpathlineto{\pgfqpoint{2.104103in}{0.603848in}}%
\pgfpathlineto{\pgfqpoint{2.104658in}{0.606418in}}%
\pgfpathlineto{\pgfqpoint{2.105214in}{0.605787in}}%
\pgfpathlineto{\pgfqpoint{2.105770in}{0.604327in}}%
\pgfpathlineto{\pgfqpoint{2.106326in}{0.607355in}}%
\pgfpathlineto{\pgfqpoint{2.107994in}{0.602169in}}%
\pgfpathlineto{\pgfqpoint{2.109105in}{0.607813in}}%
\pgfpathlineto{\pgfqpoint{2.109661in}{0.605173in}}%
\pgfpathlineto{\pgfqpoint{2.110217in}{0.605748in}}%
\pgfpathlineto{\pgfqpoint{2.111329in}{0.602089in}}%
\pgfpathlineto{\pgfqpoint{2.111885in}{0.602972in}}%
\pgfpathlineto{\pgfqpoint{2.112441in}{0.604140in}}%
\pgfpathlineto{\pgfqpoint{2.112996in}{0.602547in}}%
\pgfpathlineto{\pgfqpoint{2.114108in}{0.607955in}}%
\pgfpathlineto{\pgfqpoint{2.114664in}{0.603511in}}%
\pgfpathlineto{\pgfqpoint{2.115220in}{0.610573in}}%
\pgfpathlineto{\pgfqpoint{2.116332in}{0.601007in}}%
\pgfpathlineto{\pgfqpoint{2.116887in}{0.602185in}}%
\pgfpathlineto{\pgfqpoint{2.117443in}{0.602285in}}%
\pgfpathlineto{\pgfqpoint{2.117999in}{0.605364in}}%
\pgfpathlineto{\pgfqpoint{2.118555in}{0.604179in}}%
\pgfpathlineto{\pgfqpoint{2.119111in}{0.602483in}}%
\pgfpathlineto{\pgfqpoint{2.120223in}{0.608322in}}%
\pgfpathlineto{\pgfqpoint{2.120778in}{0.604798in}}%
\pgfpathlineto{\pgfqpoint{2.122446in}{0.600311in}}%
\pgfpathlineto{\pgfqpoint{2.123002in}{0.606569in}}%
\pgfpathlineto{\pgfqpoint{2.123558in}{0.603532in}}%
\pgfpathlineto{\pgfqpoint{2.125225in}{0.606245in}}%
\pgfpathlineto{\pgfqpoint{2.126337in}{0.602838in}}%
\pgfpathlineto{\pgfqpoint{2.126893in}{0.605051in}}%
\pgfpathlineto{\pgfqpoint{2.128005in}{0.609580in}}%
\pgfpathlineto{\pgfqpoint{2.128561in}{0.600938in}}%
\pgfpathlineto{\pgfqpoint{2.129116in}{0.601764in}}%
\pgfpathlineto{\pgfqpoint{2.130228in}{0.603099in}}%
\pgfpathlineto{\pgfqpoint{2.130784in}{0.600923in}}%
\pgfpathlineto{\pgfqpoint{2.131340in}{0.604165in}}%
\pgfpathlineto{\pgfqpoint{2.131896in}{0.602808in}}%
\pgfpathlineto{\pgfqpoint{2.132452in}{0.600572in}}%
\pgfpathlineto{\pgfqpoint{2.133007in}{0.601691in}}%
\pgfpathlineto{\pgfqpoint{2.134119in}{0.603220in}}%
\pgfpathlineto{\pgfqpoint{2.135231in}{0.602039in}}%
\pgfpathlineto{\pgfqpoint{2.138010in}{0.605054in}}%
\pgfpathlineto{\pgfqpoint{2.138566in}{0.600936in}}%
\pgfpathlineto{\pgfqpoint{2.139122in}{0.602205in}}%
\pgfpathlineto{\pgfqpoint{2.141345in}{0.605767in}}%
\pgfpathlineto{\pgfqpoint{2.142457in}{0.600901in}}%
\pgfpathlineto{\pgfqpoint{2.143013in}{0.604243in}}%
\pgfpathlineto{\pgfqpoint{2.143569in}{0.600359in}}%
\pgfpathlineto{\pgfqpoint{2.144125in}{0.602360in}}%
\pgfpathlineto{\pgfqpoint{2.145792in}{0.601168in}}%
\pgfpathlineto{\pgfqpoint{2.146348in}{0.605331in}}%
\pgfpathlineto{\pgfqpoint{2.146904in}{0.602110in}}%
\pgfpathlineto{\pgfqpoint{2.147460in}{0.603566in}}%
\pgfpathlineto{\pgfqpoint{2.148016in}{0.601751in}}%
\pgfpathlineto{\pgfqpoint{2.148572in}{0.605026in}}%
\pgfpathlineto{\pgfqpoint{2.149127in}{0.602146in}}%
\pgfpathlineto{\pgfqpoint{2.149683in}{0.601585in}}%
\pgfpathlineto{\pgfqpoint{2.150239in}{0.602648in}}%
\pgfpathlineto{\pgfqpoint{2.150795in}{0.607124in}}%
\pgfpathlineto{\pgfqpoint{2.151351in}{0.603926in}}%
\pgfpathlineto{\pgfqpoint{2.151907in}{0.605563in}}%
\pgfpathlineto{\pgfqpoint{2.152463in}{0.605252in}}%
\pgfpathlineto{\pgfqpoint{2.155798in}{0.601281in}}%
\pgfpathlineto{\pgfqpoint{2.156354in}{0.606920in}}%
\pgfpathlineto{\pgfqpoint{2.156909in}{0.606630in}}%
\pgfpathlineto{\pgfqpoint{2.158021in}{0.602396in}}%
\pgfpathlineto{\pgfqpoint{2.159689in}{0.605431in}}%
\pgfpathlineto{\pgfqpoint{2.161356in}{0.602428in}}%
\pgfpathlineto{\pgfqpoint{2.161912in}{0.603618in}}%
\pgfpathlineto{\pgfqpoint{2.162468in}{0.602500in}}%
\pgfpathlineto{\pgfqpoint{2.163580in}{0.600631in}}%
\pgfpathlineto{\pgfqpoint{2.164136in}{0.601226in}}%
\pgfpathlineto{\pgfqpoint{2.164692in}{0.605180in}}%
\pgfpathlineto{\pgfqpoint{2.165247in}{0.600265in}}%
\pgfpathlineto{\pgfqpoint{2.165803in}{0.605216in}}%
\pgfpathlineto{\pgfqpoint{2.166359in}{0.606095in}}%
\pgfpathlineto{\pgfqpoint{2.166915in}{0.602374in}}%
\pgfpathlineto{\pgfqpoint{2.167471in}{0.604021in}}%
\pgfpathlineto{\pgfqpoint{2.168027in}{0.607042in}}%
\pgfpathlineto{\pgfqpoint{2.168583in}{0.606187in}}%
\pgfpathlineto{\pgfqpoint{2.169138in}{0.606432in}}%
\pgfpathlineto{\pgfqpoint{2.169694in}{0.600854in}}%
\pgfpathlineto{\pgfqpoint{2.170250in}{0.603410in}}%
\pgfpathlineto{\pgfqpoint{2.170806in}{0.605915in}}%
\pgfpathlineto{\pgfqpoint{2.171362in}{0.603990in}}%
\pgfpathlineto{\pgfqpoint{2.171918in}{0.600917in}}%
\pgfpathlineto{\pgfqpoint{2.172474in}{0.604991in}}%
\pgfpathlineto{\pgfqpoint{2.173029in}{0.601938in}}%
\pgfpathlineto{\pgfqpoint{2.173585in}{0.600377in}}%
\pgfpathlineto{\pgfqpoint{2.174141in}{0.601312in}}%
\pgfpathlineto{\pgfqpoint{2.174697in}{0.606713in}}%
\pgfpathlineto{\pgfqpoint{2.175253in}{0.602379in}}%
\pgfpathlineto{\pgfqpoint{2.175809in}{0.605479in}}%
\pgfpathlineto{\pgfqpoint{2.176365in}{0.600150in}}%
\pgfpathlineto{\pgfqpoint{2.176920in}{0.605414in}}%
\pgfpathlineto{\pgfqpoint{2.177476in}{0.600922in}}%
\pgfpathlineto{\pgfqpoint{2.178032in}{0.606509in}}%
\pgfpathlineto{\pgfqpoint{2.178588in}{0.603562in}}%
\pgfpathlineto{\pgfqpoint{2.179144in}{0.604110in}}%
\pgfpathlineto{\pgfqpoint{2.180256in}{0.601673in}}%
\pgfpathlineto{\pgfqpoint{2.180811in}{0.602965in}}%
\pgfpathlineto{\pgfqpoint{2.181367in}{0.602128in}}%
\pgfpathlineto{\pgfqpoint{2.181923in}{0.601806in}}%
\pgfpathlineto{\pgfqpoint{2.183035in}{0.605365in}}%
\pgfpathlineto{\pgfqpoint{2.183591in}{0.603895in}}%
\pgfpathlineto{\pgfqpoint{2.184147in}{0.604609in}}%
\pgfpathlineto{\pgfqpoint{2.184703in}{0.602166in}}%
\pgfpathlineto{\pgfqpoint{2.185258in}{0.604522in}}%
\pgfpathlineto{\pgfqpoint{2.185814in}{0.604039in}}%
\pgfpathlineto{\pgfqpoint{2.186370in}{0.601459in}}%
\pgfpathlineto{\pgfqpoint{2.186926in}{0.602449in}}%
\pgfpathlineto{\pgfqpoint{2.188038in}{0.601263in}}%
\pgfpathlineto{\pgfqpoint{2.189149in}{0.605722in}}%
\pgfpathlineto{\pgfqpoint{2.190817in}{0.601262in}}%
\pgfpathlineto{\pgfqpoint{2.191929in}{0.600229in}}%
\pgfpathlineto{\pgfqpoint{2.192485in}{0.604175in}}%
\pgfpathlineto{\pgfqpoint{2.193040in}{0.601565in}}%
\pgfpathlineto{\pgfqpoint{2.194708in}{0.604223in}}%
\pgfpathlineto{\pgfqpoint{2.195820in}{0.602392in}}%
\pgfpathlineto{\pgfqpoint{2.196931in}{0.605329in}}%
\pgfpathlineto{\pgfqpoint{2.198043in}{0.600921in}}%
\pgfpathlineto{\pgfqpoint{2.198599in}{0.606459in}}%
\pgfpathlineto{\pgfqpoint{2.199155in}{0.601513in}}%
\pgfpathlineto{\pgfqpoint{2.200823in}{0.602150in}}%
\pgfpathlineto{\pgfqpoint{2.201934in}{0.602216in}}%
\pgfpathlineto{\pgfqpoint{2.203046in}{0.603765in}}%
\pgfpathlineto{\pgfqpoint{2.203602in}{0.602788in}}%
\pgfpathlineto{\pgfqpoint{2.204714in}{0.602149in}}%
\pgfpathlineto{\pgfqpoint{2.205269in}{0.604473in}}%
\pgfpathlineto{\pgfqpoint{2.205825in}{0.600845in}}%
\pgfpathlineto{\pgfqpoint{2.206381in}{0.602442in}}%
\pgfpathlineto{\pgfqpoint{2.206937in}{0.601600in}}%
\pgfpathlineto{\pgfqpoint{2.207493in}{0.602264in}}%
\pgfpathlineto{\pgfqpoint{2.208049in}{0.602189in}}%
\pgfpathlineto{\pgfqpoint{2.209160in}{0.603968in}}%
\pgfpathlineto{\pgfqpoint{2.210828in}{0.600674in}}%
\pgfpathlineto{\pgfqpoint{2.211384in}{0.606174in}}%
\pgfpathlineto{\pgfqpoint{2.211940in}{0.601358in}}%
\pgfpathlineto{\pgfqpoint{2.213051in}{0.600278in}}%
\pgfpathlineto{\pgfqpoint{2.214163in}{0.604886in}}%
\pgfpathlineto{\pgfqpoint{2.214719in}{0.602973in}}%
\pgfpathlineto{\pgfqpoint{2.215275in}{0.604788in}}%
\pgfpathlineto{\pgfqpoint{2.215831in}{0.604091in}}%
\pgfpathlineto{\pgfqpoint{2.216387in}{0.604977in}}%
\pgfpathlineto{\pgfqpoint{2.216942in}{0.605019in}}%
\pgfpathlineto{\pgfqpoint{2.217498in}{0.600981in}}%
\pgfpathlineto{\pgfqpoint{2.218054in}{0.601398in}}%
\pgfpathlineto{\pgfqpoint{2.219166in}{0.602436in}}%
\pgfpathlineto{\pgfqpoint{2.219722in}{0.601496in}}%
\pgfpathlineto{\pgfqpoint{2.220278in}{0.605043in}}%
\pgfpathlineto{\pgfqpoint{2.220834in}{0.602740in}}%
\pgfpathlineto{\pgfqpoint{2.222501in}{0.604899in}}%
\pgfpathlineto{\pgfqpoint{2.223057in}{0.602717in}}%
\pgfpathlineto{\pgfqpoint{2.224169in}{0.610094in}}%
\pgfpathlineto{\pgfqpoint{2.224725in}{0.601279in}}%
\pgfpathlineto{\pgfqpoint{2.225280in}{0.602770in}}%
\pgfpathlineto{\pgfqpoint{2.225836in}{0.601615in}}%
\pgfpathlineto{\pgfqpoint{2.227504in}{0.611386in}}%
\pgfpathlineto{\pgfqpoint{2.228060in}{0.602078in}}%
\pgfpathlineto{\pgfqpoint{2.228616in}{0.612712in}}%
\pgfpathlineto{\pgfqpoint{2.229171in}{0.603823in}}%
\pgfpathlineto{\pgfqpoint{2.229727in}{0.603112in}}%
\pgfpathlineto{\pgfqpoint{2.230283in}{0.606576in}}%
\pgfpathlineto{\pgfqpoint{2.230839in}{0.601818in}}%
\pgfpathlineto{\pgfqpoint{2.231395in}{0.605166in}}%
\pgfpathlineto{\pgfqpoint{2.231951in}{0.605620in}}%
\pgfpathlineto{\pgfqpoint{2.232507in}{0.604034in}}%
\pgfpathlineto{\pgfqpoint{2.233062in}{0.609623in}}%
\pgfpathlineto{\pgfqpoint{2.233618in}{0.602659in}}%
\pgfpathlineto{\pgfqpoint{2.234174in}{0.602930in}}%
\pgfpathlineto{\pgfqpoint{2.235286in}{0.605170in}}%
\pgfpathlineto{\pgfqpoint{2.236398in}{0.603489in}}%
\pgfpathlineto{\pgfqpoint{2.236953in}{0.604959in}}%
\pgfpathlineto{\pgfqpoint{2.238065in}{0.607181in}}%
\pgfpathlineto{\pgfqpoint{2.239177in}{0.603719in}}%
\pgfpathlineto{\pgfqpoint{2.239733in}{0.605704in}}%
\pgfpathlineto{\pgfqpoint{2.240289in}{0.604274in}}%
\pgfpathlineto{\pgfqpoint{2.241400in}{0.606213in}}%
\pgfpathlineto{\pgfqpoint{2.241956in}{0.605144in}}%
\pgfpathlineto{\pgfqpoint{2.242512in}{0.601873in}}%
\pgfpathlineto{\pgfqpoint{2.243068in}{0.602487in}}%
\pgfpathlineto{\pgfqpoint{2.243624in}{0.607627in}}%
\pgfpathlineto{\pgfqpoint{2.244180in}{0.606882in}}%
\pgfpathlineto{\pgfqpoint{2.244736in}{0.606271in}}%
\pgfpathlineto{\pgfqpoint{2.246403in}{0.601681in}}%
\pgfpathlineto{\pgfqpoint{2.246959in}{0.602480in}}%
\pgfpathlineto{\pgfqpoint{2.248071in}{0.607390in}}%
\pgfpathlineto{\pgfqpoint{2.248627in}{0.600268in}}%
\pgfpathlineto{\pgfqpoint{2.249182in}{0.602482in}}%
\pgfpathlineto{\pgfqpoint{2.250294in}{0.605140in}}%
\pgfpathlineto{\pgfqpoint{2.251406in}{0.601156in}}%
\pgfpathlineto{\pgfqpoint{2.251962in}{0.602636in}}%
\pgfpathlineto{\pgfqpoint{2.252518in}{0.601571in}}%
\pgfpathlineto{\pgfqpoint{2.254185in}{0.607173in}}%
\pgfpathlineto{\pgfqpoint{2.255297in}{0.600995in}}%
\pgfpathlineto{\pgfqpoint{2.256409in}{0.602934in}}%
\pgfpathlineto{\pgfqpoint{2.256964in}{0.601678in}}%
\pgfpathlineto{\pgfqpoint{2.259188in}{0.606007in}}%
\pgfpathlineto{\pgfqpoint{2.260300in}{0.602750in}}%
\pgfpathlineto{\pgfqpoint{2.261411in}{0.606073in}}%
\pgfpathlineto{\pgfqpoint{2.263079in}{0.603676in}}%
\pgfpathlineto{\pgfqpoint{2.263635in}{0.604187in}}%
\pgfpathlineto{\pgfqpoint{2.264191in}{0.602103in}}%
\pgfpathlineto{\pgfqpoint{2.265302in}{0.606026in}}%
\pgfpathlineto{\pgfqpoint{2.266970in}{0.602443in}}%
\pgfpathlineto{\pgfqpoint{2.267526in}{0.612752in}}%
\pgfpathlineto{\pgfqpoint{2.268082in}{0.604220in}}%
\pgfpathlineto{\pgfqpoint{2.268638in}{0.604947in}}%
\pgfpathlineto{\pgfqpoint{2.270305in}{0.601003in}}%
\pgfpathlineto{\pgfqpoint{2.271417in}{0.612021in}}%
\pgfpathlineto{\pgfqpoint{2.272529in}{0.609700in}}%
\pgfpathlineto{\pgfqpoint{2.273084in}{0.606783in}}%
\pgfpathlineto{\pgfqpoint{2.273640in}{0.611813in}}%
\pgfpathlineto{\pgfqpoint{2.274196in}{0.602130in}}%
\pgfpathlineto{\pgfqpoint{2.274752in}{0.608113in}}%
\pgfpathlineto{\pgfqpoint{2.275864in}{0.606633in}}%
\pgfpathlineto{\pgfqpoint{2.276420in}{0.601972in}}%
\pgfpathlineto{\pgfqpoint{2.276976in}{0.603316in}}%
\pgfpathlineto{\pgfqpoint{2.277531in}{0.602570in}}%
\pgfpathlineto{\pgfqpoint{2.279199in}{0.608324in}}%
\pgfpathlineto{\pgfqpoint{2.279755in}{0.609783in}}%
\pgfpathlineto{\pgfqpoint{2.280867in}{0.602708in}}%
\pgfpathlineto{\pgfqpoint{2.282534in}{0.608964in}}%
\pgfpathlineto{\pgfqpoint{2.283090in}{0.605682in}}%
\pgfpathlineto{\pgfqpoint{2.283646in}{0.615050in}}%
\pgfpathlineto{\pgfqpoint{2.284758in}{0.603970in}}%
\pgfpathlineto{\pgfqpoint{2.286425in}{0.613783in}}%
\pgfpathlineto{\pgfqpoint{2.286981in}{0.605953in}}%
\pgfpathlineto{\pgfqpoint{2.287537in}{0.612303in}}%
\pgfpathlineto{\pgfqpoint{2.288649in}{0.606118in}}%
\pgfpathlineto{\pgfqpoint{2.289204in}{0.610010in}}%
\pgfpathlineto{\pgfqpoint{2.289760in}{0.604142in}}%
\pgfpathlineto{\pgfqpoint{2.290316in}{0.625191in}}%
\pgfpathlineto{\pgfqpoint{2.290872in}{0.602960in}}%
\pgfpathlineto{\pgfqpoint{2.291428in}{0.609278in}}%
\pgfpathlineto{\pgfqpoint{2.292540in}{0.605233in}}%
\pgfpathlineto{\pgfqpoint{2.293095in}{0.608673in}}%
\pgfpathlineto{\pgfqpoint{2.293651in}{0.603037in}}%
\pgfpathlineto{\pgfqpoint{2.294207in}{0.604045in}}%
\pgfpathlineto{\pgfqpoint{2.294763in}{0.603667in}}%
\pgfpathlineto{\pgfqpoint{2.295319in}{0.619269in}}%
\pgfpathlineto{\pgfqpoint{2.295875in}{0.604397in}}%
\pgfpathlineto{\pgfqpoint{2.296431in}{0.602852in}}%
\pgfpathlineto{\pgfqpoint{2.298098in}{0.610098in}}%
\pgfpathlineto{\pgfqpoint{2.298654in}{0.603210in}}%
\pgfpathlineto{\pgfqpoint{2.299210in}{0.604115in}}%
\pgfpathlineto{\pgfqpoint{2.300878in}{0.612154in}}%
\pgfpathlineto{\pgfqpoint{2.302545in}{0.601138in}}%
\pgfpathlineto{\pgfqpoint{2.303101in}{0.607893in}}%
\pgfpathlineto{\pgfqpoint{2.303657in}{0.607504in}}%
\pgfpathlineto{\pgfqpoint{2.304213in}{0.601359in}}%
\pgfpathlineto{\pgfqpoint{2.304769in}{0.609835in}}%
\pgfpathlineto{\pgfqpoint{2.305324in}{0.608196in}}%
\pgfpathlineto{\pgfqpoint{2.305880in}{0.602507in}}%
\pgfpathlineto{\pgfqpoint{2.306436in}{0.612740in}}%
\pgfpathlineto{\pgfqpoint{2.306992in}{0.606873in}}%
\pgfpathlineto{\pgfqpoint{2.307548in}{0.612391in}}%
\pgfpathlineto{\pgfqpoint{2.309215in}{0.601332in}}%
\pgfpathlineto{\pgfqpoint{2.310883in}{0.609078in}}%
\pgfpathlineto{\pgfqpoint{2.311995in}{0.601820in}}%
\pgfpathlineto{\pgfqpoint{2.312551in}{0.610972in}}%
\pgfpathlineto{\pgfqpoint{2.314218in}{0.600196in}}%
\pgfpathlineto{\pgfqpoint{2.315886in}{0.608061in}}%
\pgfpathlineto{\pgfqpoint{2.316442in}{0.604354in}}%
\pgfpathlineto{\pgfqpoint{2.316998in}{0.606953in}}%
\pgfpathlineto{\pgfqpoint{2.317553in}{0.609323in}}%
\pgfpathlineto{\pgfqpoint{2.318109in}{0.600857in}}%
\pgfpathlineto{\pgfqpoint{2.318665in}{0.610960in}}%
\pgfpathlineto{\pgfqpoint{2.319221in}{0.610240in}}%
\pgfpathlineto{\pgfqpoint{2.320333in}{0.603517in}}%
\pgfpathlineto{\pgfqpoint{2.320889in}{0.610972in}}%
\pgfpathlineto{\pgfqpoint{2.321444in}{0.601082in}}%
\pgfpathlineto{\pgfqpoint{2.322000in}{0.608899in}}%
\pgfpathlineto{\pgfqpoint{2.323668in}{0.605414in}}%
\pgfpathlineto{\pgfqpoint{2.324224in}{0.603350in}}%
\pgfpathlineto{\pgfqpoint{2.325335in}{0.612718in}}%
\pgfpathlineto{\pgfqpoint{2.325891in}{0.603229in}}%
\pgfpathlineto{\pgfqpoint{2.326447in}{0.608401in}}%
\pgfpathlineto{\pgfqpoint{2.327003in}{0.610557in}}%
\pgfpathlineto{\pgfqpoint{2.328115in}{0.604880in}}%
\pgfpathlineto{\pgfqpoint{2.329226in}{0.612395in}}%
\pgfpathlineto{\pgfqpoint{2.329782in}{0.611388in}}%
\pgfpathlineto{\pgfqpoint{2.330338in}{0.607839in}}%
\pgfpathlineto{\pgfqpoint{2.331450in}{0.618129in}}%
\pgfpathlineto{\pgfqpoint{2.332006in}{0.612611in}}%
\pgfpathlineto{\pgfqpoint{2.333118in}{0.611787in}}%
\pgfpathlineto{\pgfqpoint{2.334229in}{0.602733in}}%
\pgfpathlineto{\pgfqpoint{2.334785in}{0.612274in}}%
\pgfpathlineto{\pgfqpoint{2.335341in}{0.610224in}}%
\pgfpathlineto{\pgfqpoint{2.335897in}{0.600732in}}%
\pgfpathlineto{\pgfqpoint{2.336453in}{0.607290in}}%
\pgfpathlineto{\pgfqpoint{2.337009in}{0.611432in}}%
\pgfpathlineto{\pgfqpoint{2.337564in}{0.608157in}}%
\pgfpathlineto{\pgfqpoint{2.338676in}{0.605006in}}%
\pgfpathlineto{\pgfqpoint{2.339232in}{0.618624in}}%
\pgfpathlineto{\pgfqpoint{2.339788in}{0.615612in}}%
\pgfpathlineto{\pgfqpoint{2.340344in}{0.606554in}}%
\pgfpathlineto{\pgfqpoint{2.340900in}{0.615723in}}%
\pgfpathlineto{\pgfqpoint{2.341455in}{0.612092in}}%
\pgfpathlineto{\pgfqpoint{2.342011in}{0.606466in}}%
\pgfpathlineto{\pgfqpoint{2.343679in}{0.621990in}}%
\pgfpathlineto{\pgfqpoint{2.344791in}{0.603292in}}%
\pgfpathlineto{\pgfqpoint{2.346458in}{0.620749in}}%
\pgfpathlineto{\pgfqpoint{2.347014in}{0.615771in}}%
\pgfpathlineto{\pgfqpoint{2.347570in}{0.623044in}}%
\pgfpathlineto{\pgfqpoint{2.348126in}{0.621959in}}%
\pgfpathlineto{\pgfqpoint{2.348682in}{0.610259in}}%
\pgfpathlineto{\pgfqpoint{2.349237in}{0.622021in}}%
\pgfpathlineto{\pgfqpoint{2.349793in}{0.616953in}}%
\pgfpathlineto{\pgfqpoint{2.350349in}{0.620534in}}%
\pgfpathlineto{\pgfqpoint{2.350905in}{0.606964in}}%
\pgfpathlineto{\pgfqpoint{2.351461in}{0.608770in}}%
\pgfpathlineto{\pgfqpoint{2.352017in}{0.607488in}}%
\pgfpathlineto{\pgfqpoint{2.352573in}{0.618655in}}%
\pgfpathlineto{\pgfqpoint{2.353129in}{0.615164in}}%
\pgfpathlineto{\pgfqpoint{2.354796in}{0.602240in}}%
\pgfpathlineto{\pgfqpoint{2.355352in}{0.629420in}}%
\pgfpathlineto{\pgfqpoint{2.355908in}{0.602814in}}%
\pgfpathlineto{\pgfqpoint{2.357575in}{0.606649in}}%
\pgfpathlineto{\pgfqpoint{2.359243in}{0.613195in}}%
\pgfpathlineto{\pgfqpoint{2.360911in}{0.610058in}}%
\pgfpathlineto{\pgfqpoint{2.361466in}{0.610931in}}%
\pgfpathlineto{\pgfqpoint{2.362022in}{0.601535in}}%
\pgfpathlineto{\pgfqpoint{2.362578in}{0.603378in}}%
\pgfpathlineto{\pgfqpoint{2.363690in}{0.619831in}}%
\pgfpathlineto{\pgfqpoint{2.364246in}{0.611864in}}%
\pgfpathlineto{\pgfqpoint{2.364802in}{0.611560in}}%
\pgfpathlineto{\pgfqpoint{2.366469in}{0.602242in}}%
\pgfpathlineto{\pgfqpoint{2.367025in}{0.609646in}}%
\pgfpathlineto{\pgfqpoint{2.367581in}{0.604924in}}%
\pgfpathlineto{\pgfqpoint{2.368137in}{0.608533in}}%
\pgfpathlineto{\pgfqpoint{2.368693in}{0.602294in}}%
\pgfpathlineto{\pgfqpoint{2.369248in}{0.606289in}}%
\pgfpathlineto{\pgfqpoint{2.369804in}{0.615442in}}%
\pgfpathlineto{\pgfqpoint{2.370360in}{0.606542in}}%
\pgfpathlineto{\pgfqpoint{2.370916in}{0.600800in}}%
\pgfpathlineto{\pgfqpoint{2.371472in}{0.603319in}}%
\pgfpathlineto{\pgfqpoint{2.373140in}{0.607942in}}%
\pgfpathlineto{\pgfqpoint{2.373695in}{0.605008in}}%
\pgfpathlineto{\pgfqpoint{2.374251in}{0.608942in}}%
\pgfpathlineto{\pgfqpoint{2.374807in}{0.603610in}}%
\pgfpathlineto{\pgfqpoint{2.375363in}{0.608184in}}%
\pgfpathlineto{\pgfqpoint{2.375919in}{0.605908in}}%
\pgfpathlineto{\pgfqpoint{2.376475in}{0.608065in}}%
\pgfpathlineto{\pgfqpoint{2.377586in}{0.611908in}}%
\pgfpathlineto{\pgfqpoint{2.378142in}{0.609661in}}%
\pgfpathlineto{\pgfqpoint{2.378698in}{0.606259in}}%
\pgfpathlineto{\pgfqpoint{2.379254in}{0.607643in}}%
\pgfpathlineto{\pgfqpoint{2.379810in}{0.609228in}}%
\pgfpathlineto{\pgfqpoint{2.380366in}{0.615465in}}%
\pgfpathlineto{\pgfqpoint{2.380922in}{0.600435in}}%
\pgfpathlineto{\pgfqpoint{2.381477in}{0.610681in}}%
\pgfpathlineto{\pgfqpoint{2.382033in}{0.607074in}}%
\pgfpathlineto{\pgfqpoint{2.382589in}{0.618062in}}%
\pgfpathlineto{\pgfqpoint{2.383145in}{0.612732in}}%
\pgfpathlineto{\pgfqpoint{2.383701in}{0.611764in}}%
\pgfpathlineto{\pgfqpoint{2.384257in}{0.605566in}}%
\pgfpathlineto{\pgfqpoint{2.384813in}{0.613551in}}%
\pgfpathlineto{\pgfqpoint{2.385368in}{0.608179in}}%
\pgfpathlineto{\pgfqpoint{2.385924in}{0.613360in}}%
\pgfpathlineto{\pgfqpoint{2.387036in}{0.604907in}}%
\pgfpathlineto{\pgfqpoint{2.388704in}{0.628164in}}%
\pgfpathlineto{\pgfqpoint{2.389260in}{0.605055in}}%
\pgfpathlineto{\pgfqpoint{2.389815in}{0.620920in}}%
\pgfpathlineto{\pgfqpoint{2.390371in}{0.610670in}}%
\pgfpathlineto{\pgfqpoint{2.390927in}{0.621221in}}%
\pgfpathlineto{\pgfqpoint{2.391483in}{0.602497in}}%
\pgfpathlineto{\pgfqpoint{2.392039in}{0.615427in}}%
\pgfpathlineto{\pgfqpoint{2.392595in}{0.607065in}}%
\pgfpathlineto{\pgfqpoint{2.393151in}{0.618966in}}%
\pgfpathlineto{\pgfqpoint{2.393706in}{0.609213in}}%
\pgfpathlineto{\pgfqpoint{2.394262in}{0.615743in}}%
\pgfpathlineto{\pgfqpoint{2.394818in}{0.612983in}}%
\pgfpathlineto{\pgfqpoint{2.395374in}{0.607170in}}%
\pgfpathlineto{\pgfqpoint{2.395930in}{0.611122in}}%
\pgfpathlineto{\pgfqpoint{2.397042in}{0.620123in}}%
\pgfpathlineto{\pgfqpoint{2.397597in}{0.606537in}}%
\pgfpathlineto{\pgfqpoint{2.398153in}{0.613920in}}%
\pgfpathlineto{\pgfqpoint{2.398709in}{0.631787in}}%
\pgfpathlineto{\pgfqpoint{2.399265in}{0.616509in}}%
\pgfpathlineto{\pgfqpoint{2.400377in}{0.613186in}}%
\pgfpathlineto{\pgfqpoint{2.400933in}{0.611755in}}%
\pgfpathlineto{\pgfqpoint{2.401488in}{0.619975in}}%
\pgfpathlineto{\pgfqpoint{2.402044in}{0.602601in}}%
\pgfpathlineto{\pgfqpoint{2.402600in}{0.617726in}}%
\pgfpathlineto{\pgfqpoint{2.403156in}{0.629054in}}%
\pgfpathlineto{\pgfqpoint{2.404268in}{0.611933in}}%
\pgfpathlineto{\pgfqpoint{2.404824in}{0.614173in}}%
\pgfpathlineto{\pgfqpoint{2.405379in}{0.640135in}}%
\pgfpathlineto{\pgfqpoint{2.405935in}{0.607269in}}%
\pgfpathlineto{\pgfqpoint{2.406491in}{0.611200in}}%
\pgfpathlineto{\pgfqpoint{2.407047in}{0.610595in}}%
\pgfpathlineto{\pgfqpoint{2.407603in}{0.628947in}}%
\pgfpathlineto{\pgfqpoint{2.408159in}{0.612116in}}%
\pgfpathlineto{\pgfqpoint{2.408715in}{0.613123in}}%
\pgfpathlineto{\pgfqpoint{2.409826in}{0.630663in}}%
\pgfpathlineto{\pgfqpoint{2.410938in}{0.612637in}}%
\pgfpathlineto{\pgfqpoint{2.411494in}{0.619393in}}%
\pgfpathlineto{\pgfqpoint{2.412050in}{0.606486in}}%
\pgfpathlineto{\pgfqpoint{2.412606in}{0.621097in}}%
\pgfpathlineto{\pgfqpoint{2.413162in}{0.607539in}}%
\pgfpathlineto{\pgfqpoint{2.413717in}{0.610995in}}%
\pgfpathlineto{\pgfqpoint{2.414273in}{0.608420in}}%
\pgfpathlineto{\pgfqpoint{2.414829in}{0.606855in}}%
\pgfpathlineto{\pgfqpoint{2.415385in}{0.608316in}}%
\pgfpathlineto{\pgfqpoint{2.417053in}{0.616739in}}%
\pgfpathlineto{\pgfqpoint{2.418720in}{0.606143in}}%
\pgfpathlineto{\pgfqpoint{2.419276in}{0.613576in}}%
\pgfpathlineto{\pgfqpoint{2.419832in}{0.605584in}}%
\pgfpathlineto{\pgfqpoint{2.420388in}{0.613103in}}%
\pgfpathlineto{\pgfqpoint{2.420944in}{0.620373in}}%
\pgfpathlineto{\pgfqpoint{2.422611in}{0.601512in}}%
\pgfpathlineto{\pgfqpoint{2.423167in}{0.615747in}}%
\pgfpathlineto{\pgfqpoint{2.423723in}{0.610224in}}%
\pgfpathlineto{\pgfqpoint{2.425946in}{0.603644in}}%
\pgfpathlineto{\pgfqpoint{2.427614in}{0.615863in}}%
\pgfpathlineto{\pgfqpoint{2.428170in}{0.600666in}}%
\pgfpathlineto{\pgfqpoint{2.428726in}{0.602113in}}%
\pgfpathlineto{\pgfqpoint{2.430393in}{0.608168in}}%
\pgfpathlineto{\pgfqpoint{2.430949in}{0.611146in}}%
\pgfpathlineto{\pgfqpoint{2.431505in}{0.610965in}}%
\pgfpathlineto{\pgfqpoint{2.433173in}{0.602534in}}%
\pgfpathlineto{\pgfqpoint{2.434284in}{0.609714in}}%
\pgfpathlineto{\pgfqpoint{2.434840in}{0.609243in}}%
\pgfpathlineto{\pgfqpoint{2.435396in}{0.600343in}}%
\pgfpathlineto{\pgfqpoint{2.437064in}{0.616863in}}%
\pgfpathlineto{\pgfqpoint{2.438175in}{0.602352in}}%
\pgfpathlineto{\pgfqpoint{2.439843in}{0.615234in}}%
\pgfpathlineto{\pgfqpoint{2.441510in}{0.603907in}}%
\pgfpathlineto{\pgfqpoint{2.442066in}{0.617188in}}%
\pgfpathlineto{\pgfqpoint{2.442622in}{0.610910in}}%
\pgfpathlineto{\pgfqpoint{2.443178in}{0.613304in}}%
\pgfpathlineto{\pgfqpoint{2.443734in}{0.607218in}}%
\pgfpathlineto{\pgfqpoint{2.444290in}{0.610593in}}%
\pgfpathlineto{\pgfqpoint{2.444846in}{0.616589in}}%
\pgfpathlineto{\pgfqpoint{2.445401in}{0.604223in}}%
\pgfpathlineto{\pgfqpoint{2.447069in}{0.620596in}}%
\pgfpathlineto{\pgfqpoint{2.447625in}{0.613399in}}%
\pgfpathlineto{\pgfqpoint{2.448181in}{0.624118in}}%
\pgfpathlineto{\pgfqpoint{2.448737in}{0.606910in}}%
\pgfpathlineto{\pgfqpoint{2.449293in}{0.608832in}}%
\pgfpathlineto{\pgfqpoint{2.450404in}{0.620835in}}%
\pgfpathlineto{\pgfqpoint{2.452072in}{0.602309in}}%
\pgfpathlineto{\pgfqpoint{2.452628in}{0.606631in}}%
\pgfpathlineto{\pgfqpoint{2.453184in}{0.602177in}}%
\pgfpathlineto{\pgfqpoint{2.453739in}{0.605184in}}%
\pgfpathlineto{\pgfqpoint{2.454295in}{0.618335in}}%
\pgfpathlineto{\pgfqpoint{2.454851in}{0.613253in}}%
\pgfpathlineto{\pgfqpoint{2.455407in}{0.610117in}}%
\pgfpathlineto{\pgfqpoint{2.455963in}{0.625405in}}%
\pgfpathlineto{\pgfqpoint{2.456519in}{0.616325in}}%
\pgfpathlineto{\pgfqpoint{2.458186in}{0.600270in}}%
\pgfpathlineto{\pgfqpoint{2.460410in}{0.626635in}}%
\pgfpathlineto{\pgfqpoint{2.461521in}{0.612452in}}%
\pgfpathlineto{\pgfqpoint{2.462077in}{0.618671in}}%
\pgfpathlineto{\pgfqpoint{2.462633in}{0.617051in}}%
\pgfpathlineto{\pgfqpoint{2.463745in}{0.601310in}}%
\pgfpathlineto{\pgfqpoint{2.464857in}{0.619430in}}%
\pgfpathlineto{\pgfqpoint{2.467080in}{0.603516in}}%
\pgfpathlineto{\pgfqpoint{2.467636in}{0.624107in}}%
\pgfpathlineto{\pgfqpoint{2.468192in}{0.614307in}}%
\pgfpathlineto{\pgfqpoint{2.470415in}{0.604528in}}%
\pgfpathlineto{\pgfqpoint{2.470971in}{0.620580in}}%
\pgfpathlineto{\pgfqpoint{2.471527in}{0.613296in}}%
\pgfpathlineto{\pgfqpoint{2.473750in}{0.601005in}}%
\pgfpathlineto{\pgfqpoint{2.475418in}{0.613437in}}%
\pgfpathlineto{\pgfqpoint{2.477641in}{0.604421in}}%
\pgfpathlineto{\pgfqpoint{2.478197in}{0.606764in}}%
\pgfpathlineto{\pgfqpoint{2.478753in}{0.600718in}}%
\pgfpathlineto{\pgfqpoint{2.479865in}{0.613416in}}%
\pgfpathlineto{\pgfqpoint{2.480421in}{0.613301in}}%
\pgfpathlineto{\pgfqpoint{2.481532in}{0.602217in}}%
\pgfpathlineto{\pgfqpoint{2.482088in}{0.603864in}}%
\pgfpathlineto{\pgfqpoint{2.483756in}{0.610485in}}%
\pgfpathlineto{\pgfqpoint{2.484312in}{0.606942in}}%
\pgfpathlineto{\pgfqpoint{2.484868in}{0.606183in}}%
\pgfpathlineto{\pgfqpoint{2.486535in}{0.601693in}}%
\pgfpathlineto{\pgfqpoint{2.487091in}{0.606583in}}%
\pgfpathlineto{\pgfqpoint{2.487647in}{0.604170in}}%
\pgfpathlineto{\pgfqpoint{2.488203in}{0.605272in}}%
\pgfpathlineto{\pgfqpoint{2.488759in}{0.603346in}}%
\pgfpathlineto{\pgfqpoint{2.489870in}{0.607172in}}%
\pgfpathlineto{\pgfqpoint{2.491538in}{0.602979in}}%
\pgfpathlineto{\pgfqpoint{2.492094in}{0.607690in}}%
\pgfpathlineto{\pgfqpoint{2.492650in}{0.604237in}}%
\pgfpathlineto{\pgfqpoint{2.493206in}{0.607466in}}%
\pgfpathlineto{\pgfqpoint{2.493761in}{0.603692in}}%
\pgfpathlineto{\pgfqpoint{2.494317in}{0.610929in}}%
\pgfpathlineto{\pgfqpoint{2.494873in}{0.601114in}}%
\pgfpathlineto{\pgfqpoint{2.495429in}{0.604724in}}%
\pgfpathlineto{\pgfqpoint{2.495985in}{0.605965in}}%
\pgfpathlineto{\pgfqpoint{2.496541in}{0.605057in}}%
\pgfpathlineto{\pgfqpoint{2.497097in}{0.605517in}}%
\pgfpathlineto{\pgfqpoint{2.497652in}{0.607773in}}%
\pgfpathlineto{\pgfqpoint{2.498208in}{0.606629in}}%
\pgfpathlineto{\pgfqpoint{2.498764in}{0.606511in}}%
\pgfpathlineto{\pgfqpoint{2.499320in}{0.608357in}}%
\pgfpathlineto{\pgfqpoint{2.499876in}{0.607211in}}%
\pgfpathlineto{\pgfqpoint{2.500432in}{0.607658in}}%
\pgfpathlineto{\pgfqpoint{2.501543in}{0.602803in}}%
\pgfpathlineto{\pgfqpoint{2.502099in}{0.603832in}}%
\pgfpathlineto{\pgfqpoint{2.502655in}{0.606492in}}%
\pgfpathlineto{\pgfqpoint{2.503211in}{0.602092in}}%
\pgfpathlineto{\pgfqpoint{2.503767in}{0.615189in}}%
\pgfpathlineto{\pgfqpoint{2.504323in}{0.606742in}}%
\pgfpathlineto{\pgfqpoint{2.505990in}{0.611767in}}%
\pgfpathlineto{\pgfqpoint{2.506546in}{0.603026in}}%
\pgfpathlineto{\pgfqpoint{2.507102in}{0.612402in}}%
\pgfpathlineto{\pgfqpoint{2.507658in}{0.605690in}}%
\pgfpathlineto{\pgfqpoint{2.508770in}{0.601349in}}%
\pgfpathlineto{\pgfqpoint{2.509881in}{0.609676in}}%
\pgfpathlineto{\pgfqpoint{2.510437in}{0.600635in}}%
\pgfpathlineto{\pgfqpoint{2.510993in}{0.608610in}}%
\pgfpathlineto{\pgfqpoint{2.512661in}{0.601406in}}%
\pgfpathlineto{\pgfqpoint{2.513772in}{0.613792in}}%
\pgfpathlineto{\pgfqpoint{2.514328in}{0.608236in}}%
\pgfpathlineto{\pgfqpoint{2.515996in}{0.603451in}}%
\pgfpathlineto{\pgfqpoint{2.517663in}{0.609802in}}%
\pgfpathlineto{\pgfqpoint{2.519331in}{0.606268in}}%
\pgfpathlineto{\pgfqpoint{2.520443in}{0.608343in}}%
\pgfpathlineto{\pgfqpoint{2.520999in}{0.601388in}}%
\pgfpathlineto{\pgfqpoint{2.521555in}{0.603219in}}%
\pgfpathlineto{\pgfqpoint{2.522110in}{0.608200in}}%
\pgfpathlineto{\pgfqpoint{2.522666in}{0.605860in}}%
\pgfpathlineto{\pgfqpoint{2.523778in}{0.601789in}}%
\pgfpathlineto{\pgfqpoint{2.524334in}{0.607003in}}%
\pgfpathlineto{\pgfqpoint{2.524890in}{0.604309in}}%
\pgfpathlineto{\pgfqpoint{2.525446in}{0.606413in}}%
\pgfpathlineto{\pgfqpoint{2.526001in}{0.603148in}}%
\pgfpathlineto{\pgfqpoint{2.526557in}{0.605740in}}%
\pgfpathlineto{\pgfqpoint{2.527113in}{0.605418in}}%
\pgfpathlineto{\pgfqpoint{2.527669in}{0.606985in}}%
\pgfpathlineto{\pgfqpoint{2.528781in}{0.601795in}}%
\pgfpathlineto{\pgfqpoint{2.529337in}{0.607286in}}%
\pgfpathlineto{\pgfqpoint{2.529892in}{0.605660in}}%
\pgfpathlineto{\pgfqpoint{2.530448in}{0.605656in}}%
\pgfpathlineto{\pgfqpoint{2.531004in}{0.603217in}}%
\pgfpathlineto{\pgfqpoint{2.531560in}{0.603429in}}%
\pgfpathlineto{\pgfqpoint{2.532672in}{0.604414in}}%
\pgfpathlineto{\pgfqpoint{2.534339in}{0.600586in}}%
\pgfpathlineto{\pgfqpoint{2.536563in}{0.603710in}}%
\pgfpathlineto{\pgfqpoint{2.537119in}{0.603342in}}%
\pgfpathlineto{\pgfqpoint{2.537674in}{0.600847in}}%
\pgfpathlineto{\pgfqpoint{2.538230in}{0.601745in}}%
\pgfpathlineto{\pgfqpoint{2.538786in}{0.601136in}}%
\pgfpathlineto{\pgfqpoint{2.539898in}{0.604712in}}%
\pgfpathlineto{\pgfqpoint{2.541010in}{0.600226in}}%
\pgfpathlineto{\pgfqpoint{2.542121in}{0.601909in}}%
\pgfpathlineto{\pgfqpoint{2.542677in}{0.602776in}}%
\pgfpathlineto{\pgfqpoint{2.544901in}{0.613679in}}%
\pgfpathlineto{\pgfqpoint{2.545457in}{0.629154in}}%
\pgfpathlineto{\pgfqpoint{2.546012in}{0.607387in}}%
\pgfpathlineto{\pgfqpoint{2.546568in}{0.660543in}}%
\pgfpathlineto{\pgfqpoint{2.547124in}{0.641909in}}%
\pgfpathlineto{\pgfqpoint{2.548792in}{0.600448in}}%
\pgfpathlineto{\pgfqpoint{2.549903in}{0.604405in}}%
\pgfpathlineto{\pgfqpoint{2.550459in}{0.602566in}}%
\pgfpathlineto{\pgfqpoint{2.551015in}{0.601034in}}%
\pgfpathlineto{\pgfqpoint{2.551571in}{0.603173in}}%
\pgfpathlineto{\pgfqpoint{2.552127in}{0.601078in}}%
\pgfpathlineto{\pgfqpoint{2.552683in}{0.600567in}}%
\pgfpathlineto{\pgfqpoint{2.553239in}{0.601272in}}%
\pgfpathlineto{\pgfqpoint{2.553794in}{0.602072in}}%
\pgfpathlineto{\pgfqpoint{2.554350in}{0.600636in}}%
\pgfpathlineto{\pgfqpoint{2.554906in}{0.603529in}}%
\pgfpathlineto{\pgfqpoint{2.555462in}{0.601611in}}%
\pgfpathlineto{\pgfqpoint{2.556018in}{0.603109in}}%
\pgfpathlineto{\pgfqpoint{2.556574in}{0.601712in}}%
\pgfpathlineto{\pgfqpoint{2.557685in}{0.603038in}}%
\pgfpathlineto{\pgfqpoint{2.559353in}{0.600879in}}%
\pgfpathlineto{\pgfqpoint{2.559909in}{0.603674in}}%
\pgfpathlineto{\pgfqpoint{2.560465in}{0.600743in}}%
\pgfpathlineto{\pgfqpoint{2.561021in}{0.604118in}}%
\pgfpathlineto{\pgfqpoint{2.561577in}{0.600547in}}%
\pgfpathlineto{\pgfqpoint{2.562132in}{0.603057in}}%
\pgfpathlineto{\pgfqpoint{2.562688in}{0.601125in}}%
\pgfpathlineto{\pgfqpoint{2.563244in}{0.606659in}}%
\pgfpathlineto{\pgfqpoint{2.563800in}{0.600868in}}%
\pgfpathlineto{\pgfqpoint{2.564356in}{0.603733in}}%
\pgfpathlineto{\pgfqpoint{2.566023in}{0.601657in}}%
\pgfpathlineto{\pgfqpoint{2.566579in}{0.603320in}}%
\pgfpathlineto{\pgfqpoint{2.567135in}{0.602346in}}%
\pgfpathlineto{\pgfqpoint{2.568803in}{0.601884in}}%
\pgfpathlineto{\pgfqpoint{2.569359in}{0.602469in}}%
\pgfpathlineto{\pgfqpoint{2.569914in}{0.601052in}}%
\pgfpathlineto{\pgfqpoint{2.570470in}{0.602650in}}%
\pgfpathlineto{\pgfqpoint{2.571026in}{0.605999in}}%
\pgfpathlineto{\pgfqpoint{2.571582in}{0.604726in}}%
\pgfpathlineto{\pgfqpoint{2.572694in}{0.600527in}}%
\pgfpathlineto{\pgfqpoint{2.573805in}{0.603301in}}%
\pgfpathlineto{\pgfqpoint{2.574361in}{0.600475in}}%
\pgfpathlineto{\pgfqpoint{2.575473in}{0.606345in}}%
\pgfpathlineto{\pgfqpoint{2.576029in}{0.602369in}}%
\pgfpathlineto{\pgfqpoint{2.576585in}{0.602799in}}%
\pgfpathlineto{\pgfqpoint{2.577141in}{0.605273in}}%
\pgfpathlineto{\pgfqpoint{2.577697in}{0.604080in}}%
\pgfpathlineto{\pgfqpoint{2.578808in}{0.601308in}}%
\pgfpathlineto{\pgfqpoint{2.579364in}{0.604224in}}%
\pgfpathlineto{\pgfqpoint{2.579920in}{0.603581in}}%
\pgfpathlineto{\pgfqpoint{2.580476in}{0.604086in}}%
\pgfpathlineto{\pgfqpoint{2.581032in}{0.600742in}}%
\pgfpathlineto{\pgfqpoint{2.581588in}{0.603339in}}%
\pgfpathlineto{\pgfqpoint{2.582143in}{0.601533in}}%
\pgfpathlineto{\pgfqpoint{2.582699in}{0.606151in}}%
\pgfpathlineto{\pgfqpoint{2.584367in}{0.600094in}}%
\pgfpathlineto{\pgfqpoint{2.586034in}{0.604228in}}%
\pgfpathlineto{\pgfqpoint{2.586590in}{0.601705in}}%
\pgfpathlineto{\pgfqpoint{2.587146in}{0.602488in}}%
\pgfpathlineto{\pgfqpoint{2.588814in}{0.604628in}}%
\pgfpathlineto{\pgfqpoint{2.589370in}{0.605080in}}%
\pgfpathlineto{\pgfqpoint{2.589925in}{0.602221in}}%
\pgfpathlineto{\pgfqpoint{2.590481in}{0.604235in}}%
\pgfpathlineto{\pgfqpoint{2.591037in}{0.605950in}}%
\pgfpathlineto{\pgfqpoint{2.591593in}{0.604868in}}%
\pgfpathlineto{\pgfqpoint{2.592149in}{0.604588in}}%
\pgfpathlineto{\pgfqpoint{2.593261in}{0.600684in}}%
\pgfpathlineto{\pgfqpoint{2.593816in}{0.600881in}}%
\pgfpathlineto{\pgfqpoint{2.594928in}{0.601834in}}%
\pgfpathlineto{\pgfqpoint{2.596596in}{0.606093in}}%
\pgfpathlineto{\pgfqpoint{2.598263in}{0.601744in}}%
\pgfpathlineto{\pgfqpoint{2.599375in}{0.604479in}}%
\pgfpathlineto{\pgfqpoint{2.599931in}{0.606429in}}%
\pgfpathlineto{\pgfqpoint{2.600487in}{0.605306in}}%
\pgfpathlineto{\pgfqpoint{2.602154in}{0.601564in}}%
\pgfpathlineto{\pgfqpoint{2.602710in}{0.602650in}}%
\pgfpathlineto{\pgfqpoint{2.603822in}{0.607748in}}%
\pgfpathlineto{\pgfqpoint{2.604934in}{0.601460in}}%
\pgfpathlineto{\pgfqpoint{2.605490in}{0.602170in}}%
\pgfpathlineto{\pgfqpoint{2.606045in}{0.604712in}}%
\pgfpathlineto{\pgfqpoint{2.606601in}{0.601286in}}%
\pgfpathlineto{\pgfqpoint{2.607157in}{0.604346in}}%
\pgfpathlineto{\pgfqpoint{2.607713in}{0.602210in}}%
\pgfpathlineto{\pgfqpoint{2.609381in}{0.609254in}}%
\pgfpathlineto{\pgfqpoint{2.609936in}{0.601064in}}%
\pgfpathlineto{\pgfqpoint{2.610492in}{0.601595in}}%
\pgfpathlineto{\pgfqpoint{2.611048in}{0.606840in}}%
\pgfpathlineto{\pgfqpoint{2.611604in}{0.603984in}}%
\pgfpathlineto{\pgfqpoint{2.612716in}{0.605134in}}%
\pgfpathlineto{\pgfqpoint{2.613272in}{0.608460in}}%
\pgfpathlineto{\pgfqpoint{2.613827in}{0.602659in}}%
\pgfpathlineto{\pgfqpoint{2.614383in}{0.611068in}}%
\pgfpathlineto{\pgfqpoint{2.614939in}{0.607074in}}%
\pgfpathlineto{\pgfqpoint{2.615495in}{0.610889in}}%
\pgfpathlineto{\pgfqpoint{2.616051in}{0.601144in}}%
\pgfpathlineto{\pgfqpoint{2.616607in}{0.603549in}}%
\pgfpathlineto{\pgfqpoint{2.617163in}{0.610167in}}%
\pgfpathlineto{\pgfqpoint{2.617719in}{0.608257in}}%
\pgfpathlineto{\pgfqpoint{2.618274in}{0.605245in}}%
\pgfpathlineto{\pgfqpoint{2.619386in}{0.613721in}}%
\pgfpathlineto{\pgfqpoint{2.619942in}{0.608711in}}%
\pgfpathlineto{\pgfqpoint{2.620498in}{0.620353in}}%
\pgfpathlineto{\pgfqpoint{2.621054in}{0.612836in}}%
\pgfpathlineto{\pgfqpoint{2.621610in}{0.608523in}}%
\pgfpathlineto{\pgfqpoint{2.622165in}{0.618889in}}%
\pgfpathlineto{\pgfqpoint{2.622721in}{0.611298in}}%
\pgfpathlineto{\pgfqpoint{2.623277in}{0.610248in}}%
\pgfpathlineto{\pgfqpoint{2.623833in}{0.606645in}}%
\pgfpathlineto{\pgfqpoint{2.624945in}{0.617638in}}%
\pgfpathlineto{\pgfqpoint{2.625501in}{0.603603in}}%
\pgfpathlineto{\pgfqpoint{2.626056in}{0.616715in}}%
\pgfpathlineto{\pgfqpoint{2.627168in}{0.603008in}}%
\pgfpathlineto{\pgfqpoint{2.628836in}{0.619702in}}%
\pgfpathlineto{\pgfqpoint{2.629947in}{0.609660in}}%
\pgfpathlineto{\pgfqpoint{2.630503in}{0.615907in}}%
\pgfpathlineto{\pgfqpoint{2.631059in}{0.608323in}}%
\pgfpathlineto{\pgfqpoint{2.631615in}{0.613101in}}%
\pgfpathlineto{\pgfqpoint{2.632171in}{0.626264in}}%
\pgfpathlineto{\pgfqpoint{2.632727in}{0.626173in}}%
\pgfpathlineto{\pgfqpoint{2.633283in}{0.606189in}}%
\pgfpathlineto{\pgfqpoint{2.633838in}{0.613576in}}%
\pgfpathlineto{\pgfqpoint{2.634394in}{0.619410in}}%
\pgfpathlineto{\pgfqpoint{2.634950in}{0.614544in}}%
\pgfpathlineto{\pgfqpoint{2.635506in}{0.617042in}}%
\pgfpathlineto{\pgfqpoint{2.636062in}{0.610997in}}%
\pgfpathlineto{\pgfqpoint{2.636618in}{0.612001in}}%
\pgfpathlineto{\pgfqpoint{2.637730in}{0.620575in}}%
\pgfpathlineto{\pgfqpoint{2.639397in}{0.607653in}}%
\pgfpathlineto{\pgfqpoint{2.639953in}{0.613632in}}%
\pgfpathlineto{\pgfqpoint{2.640509in}{0.612466in}}%
\pgfpathlineto{\pgfqpoint{2.641621in}{0.604694in}}%
\pgfpathlineto{\pgfqpoint{2.642176in}{0.609992in}}%
\pgfpathlineto{\pgfqpoint{2.642732in}{0.608558in}}%
\pgfpathlineto{\pgfqpoint{2.643288in}{0.609009in}}%
\pgfpathlineto{\pgfqpoint{2.643844in}{0.605058in}}%
\pgfpathlineto{\pgfqpoint{2.644400in}{0.612087in}}%
\pgfpathlineto{\pgfqpoint{2.644956in}{0.606339in}}%
\pgfpathlineto{\pgfqpoint{2.647179in}{0.618771in}}%
\pgfpathlineto{\pgfqpoint{2.647735in}{0.617746in}}%
\pgfpathlineto{\pgfqpoint{2.648847in}{0.609137in}}%
\pgfpathlineto{\pgfqpoint{2.650514in}{0.623516in}}%
\pgfpathlineto{\pgfqpoint{2.651626in}{0.614064in}}%
\pgfpathlineto{\pgfqpoint{2.652182in}{0.616911in}}%
\pgfpathlineto{\pgfqpoint{2.652738in}{0.615446in}}%
\pgfpathlineto{\pgfqpoint{2.654405in}{0.605719in}}%
\pgfpathlineto{\pgfqpoint{2.654961in}{0.605438in}}%
\pgfpathlineto{\pgfqpoint{2.657185in}{0.624270in}}%
\pgfpathlineto{\pgfqpoint{2.658296in}{0.600841in}}%
\pgfpathlineto{\pgfqpoint{2.660520in}{0.627545in}}%
\pgfpathlineto{\pgfqpoint{2.662187in}{0.608817in}}%
\pgfpathlineto{\pgfqpoint{2.662743in}{0.617938in}}%
\pgfpathlineto{\pgfqpoint{2.663299in}{0.616783in}}%
\pgfpathlineto{\pgfqpoint{2.664967in}{0.604824in}}%
\pgfpathlineto{\pgfqpoint{2.666634in}{0.621063in}}%
\pgfpathlineto{\pgfqpoint{2.667746in}{0.602630in}}%
\pgfpathlineto{\pgfqpoint{2.668858in}{0.619981in}}%
\pgfpathlineto{\pgfqpoint{2.669414in}{0.604074in}}%
\pgfpathlineto{\pgfqpoint{2.669969in}{0.609774in}}%
\pgfpathlineto{\pgfqpoint{2.670525in}{0.623667in}}%
\pgfpathlineto{\pgfqpoint{2.671081in}{0.618289in}}%
\pgfpathlineto{\pgfqpoint{2.672193in}{0.622109in}}%
\pgfpathlineto{\pgfqpoint{2.672749in}{0.620863in}}%
\pgfpathlineto{\pgfqpoint{2.674416in}{0.609499in}}%
\pgfpathlineto{\pgfqpoint{2.674972in}{0.635037in}}%
\pgfpathlineto{\pgfqpoint{2.675528in}{0.602290in}}%
\pgfpathlineto{\pgfqpoint{2.676084in}{0.636014in}}%
\pgfpathlineto{\pgfqpoint{2.676640in}{0.618100in}}%
\pgfpathlineto{\pgfqpoint{2.677196in}{0.616362in}}%
\pgfpathlineto{\pgfqpoint{2.677752in}{0.616855in}}%
\pgfpathlineto{\pgfqpoint{2.678307in}{0.657737in}}%
\pgfpathlineto{\pgfqpoint{2.678863in}{0.604819in}}%
\pgfpathlineto{\pgfqpoint{2.679419in}{0.643635in}}%
\pgfpathlineto{\pgfqpoint{2.679975in}{0.613325in}}%
\pgfpathlineto{\pgfqpoint{2.680531in}{0.630310in}}%
\pgfpathlineto{\pgfqpoint{2.681087in}{0.613482in}}%
\pgfpathlineto{\pgfqpoint{2.681643in}{0.643409in}}%
\pgfpathlineto{\pgfqpoint{2.682198in}{0.643112in}}%
\pgfpathlineto{\pgfqpoint{2.682754in}{0.608019in}}%
\pgfpathlineto{\pgfqpoint{2.683310in}{0.626920in}}%
\pgfpathlineto{\pgfqpoint{2.684978in}{0.609209in}}%
\pgfpathlineto{\pgfqpoint{2.686089in}{0.651627in}}%
\pgfpathlineto{\pgfqpoint{2.686645in}{0.648482in}}%
\pgfpathlineto{\pgfqpoint{2.687201in}{0.627784in}}%
\pgfpathlineto{\pgfqpoint{2.687757in}{0.633725in}}%
\pgfpathlineto{\pgfqpoint{2.688313in}{0.627863in}}%
\pgfpathlineto{\pgfqpoint{2.688869in}{0.632259in}}%
\pgfpathlineto{\pgfqpoint{2.689425in}{0.633586in}}%
\pgfpathlineto{\pgfqpoint{2.689980in}{0.658999in}}%
\pgfpathlineto{\pgfqpoint{2.690536in}{0.634215in}}%
\pgfpathlineto{\pgfqpoint{2.691648in}{0.625959in}}%
\pgfpathlineto{\pgfqpoint{2.692204in}{0.646591in}}%
\pgfpathlineto{\pgfqpoint{2.692760in}{0.627055in}}%
\pgfpathlineto{\pgfqpoint{2.693316in}{0.631178in}}%
\pgfpathlineto{\pgfqpoint{2.693872in}{0.616042in}}%
\pgfpathlineto{\pgfqpoint{2.694427in}{0.639241in}}%
\pgfpathlineto{\pgfqpoint{2.694983in}{0.626337in}}%
\pgfpathlineto{\pgfqpoint{2.695539in}{0.627335in}}%
\pgfpathlineto{\pgfqpoint{2.697207in}{0.603845in}}%
\pgfpathlineto{\pgfqpoint{2.697763in}{0.635022in}}%
\pgfpathlineto{\pgfqpoint{2.698318in}{0.620044in}}%
\pgfpathlineto{\pgfqpoint{2.699986in}{0.606389in}}%
\pgfpathlineto{\pgfqpoint{2.701654in}{0.623045in}}%
\pgfpathlineto{\pgfqpoint{2.702209in}{0.607264in}}%
\pgfpathlineto{\pgfqpoint{2.702765in}{0.617597in}}%
\pgfpathlineto{\pgfqpoint{2.703321in}{0.610249in}}%
\pgfpathlineto{\pgfqpoint{2.703877in}{0.619652in}}%
\pgfpathlineto{\pgfqpoint{2.704433in}{0.604444in}}%
\pgfpathlineto{\pgfqpoint{2.704989in}{0.607274in}}%
\pgfpathlineto{\pgfqpoint{2.706100in}{0.624198in}}%
\pgfpathlineto{\pgfqpoint{2.707212in}{0.612426in}}%
\pgfpathlineto{\pgfqpoint{2.708880in}{0.636358in}}%
\pgfpathlineto{\pgfqpoint{2.709436in}{0.625587in}}%
\pgfpathlineto{\pgfqpoint{2.709992in}{0.633952in}}%
\pgfpathlineto{\pgfqpoint{2.711103in}{0.637193in}}%
\pgfpathlineto{\pgfqpoint{2.712771in}{0.616176in}}%
\pgfpathlineto{\pgfqpoint{2.713327in}{0.616237in}}%
\pgfpathlineto{\pgfqpoint{2.714438in}{0.609487in}}%
\pgfpathlineto{\pgfqpoint{2.716662in}{0.627397in}}%
\pgfpathlineto{\pgfqpoint{2.717218in}{0.627619in}}%
\pgfpathlineto{\pgfqpoint{2.718329in}{0.612990in}}%
\pgfpathlineto{\pgfqpoint{2.718885in}{0.616887in}}%
\pgfpathlineto{\pgfqpoint{2.719997in}{0.635982in}}%
\pgfpathlineto{\pgfqpoint{2.721665in}{0.603481in}}%
\pgfpathlineto{\pgfqpoint{2.722220in}{0.604851in}}%
\pgfpathlineto{\pgfqpoint{2.722776in}{0.633950in}}%
\pgfpathlineto{\pgfqpoint{2.723332in}{0.624369in}}%
\pgfpathlineto{\pgfqpoint{2.723888in}{0.630061in}}%
\pgfpathlineto{\pgfqpoint{2.724444in}{0.624800in}}%
\pgfpathlineto{\pgfqpoint{2.725000in}{0.609898in}}%
\pgfpathlineto{\pgfqpoint{2.725556in}{0.620111in}}%
\pgfpathlineto{\pgfqpoint{2.726111in}{0.634650in}}%
\pgfpathlineto{\pgfqpoint{2.727223in}{0.608455in}}%
\pgfpathlineto{\pgfqpoint{2.728335in}{0.629999in}}%
\pgfpathlineto{\pgfqpoint{2.728891in}{0.605176in}}%
\pgfpathlineto{\pgfqpoint{2.729447in}{0.626670in}}%
\pgfpathlineto{\pgfqpoint{2.730003in}{0.627494in}}%
\pgfpathlineto{\pgfqpoint{2.730558in}{0.631360in}}%
\pgfpathlineto{\pgfqpoint{2.731114in}{0.607110in}}%
\pgfpathlineto{\pgfqpoint{2.731670in}{0.618024in}}%
\pgfpathlineto{\pgfqpoint{2.732226in}{0.648037in}}%
\pgfpathlineto{\pgfqpoint{2.732782in}{0.620049in}}%
\pgfpathlineto{\pgfqpoint{2.733338in}{0.626784in}}%
\pgfpathlineto{\pgfqpoint{2.733894in}{0.619579in}}%
\pgfpathlineto{\pgfqpoint{2.734449in}{0.646328in}}%
\pgfpathlineto{\pgfqpoint{2.735005in}{0.610857in}}%
\pgfpathlineto{\pgfqpoint{2.735561in}{0.689252in}}%
\pgfpathlineto{\pgfqpoint{2.736117in}{0.617681in}}%
\pgfpathlineto{\pgfqpoint{2.737785in}{0.654230in}}%
\pgfpathlineto{\pgfqpoint{2.738340in}{0.651426in}}%
\pgfpathlineto{\pgfqpoint{2.738896in}{0.653335in}}%
\pgfpathlineto{\pgfqpoint{2.740564in}{0.618167in}}%
\pgfpathlineto{\pgfqpoint{2.741120in}{0.669219in}}%
\pgfpathlineto{\pgfqpoint{2.741676in}{0.643651in}}%
\pgfpathlineto{\pgfqpoint{2.742231in}{0.608764in}}%
\pgfpathlineto{\pgfqpoint{2.743899in}{0.649593in}}%
\pgfpathlineto{\pgfqpoint{2.744455in}{0.611908in}}%
\pgfpathlineto{\pgfqpoint{2.745011in}{0.646347in}}%
\pgfpathlineto{\pgfqpoint{2.745567in}{0.657722in}}%
\pgfpathlineto{\pgfqpoint{2.746678in}{0.621189in}}%
\pgfpathlineto{\pgfqpoint{2.747234in}{0.689696in}}%
\pgfpathlineto{\pgfqpoint{2.747790in}{0.661083in}}%
\pgfpathlineto{\pgfqpoint{2.748346in}{0.627765in}}%
\pgfpathlineto{\pgfqpoint{2.748902in}{0.629761in}}%
\pgfpathlineto{\pgfqpoint{2.749458in}{0.666701in}}%
\pgfpathlineto{\pgfqpoint{2.750014in}{0.632407in}}%
\pgfpathlineto{\pgfqpoint{2.750569in}{0.645795in}}%
\pgfpathlineto{\pgfqpoint{2.751125in}{0.620702in}}%
\pgfpathlineto{\pgfqpoint{2.751681in}{0.633173in}}%
\pgfpathlineto{\pgfqpoint{2.752793in}{0.641517in}}%
\pgfpathlineto{\pgfqpoint{2.753349in}{0.609104in}}%
\pgfpathlineto{\pgfqpoint{2.753905in}{0.624595in}}%
\pgfpathlineto{\pgfqpoint{2.754460in}{0.613944in}}%
\pgfpathlineto{\pgfqpoint{2.755016in}{0.643542in}}%
\pgfpathlineto{\pgfqpoint{2.755572in}{0.609303in}}%
\pgfpathlineto{\pgfqpoint{2.756128in}{0.611764in}}%
\pgfpathlineto{\pgfqpoint{2.756684in}{0.617594in}}%
\pgfpathlineto{\pgfqpoint{2.757240in}{0.614716in}}%
\pgfpathlineto{\pgfqpoint{2.757796in}{0.609626in}}%
\pgfpathlineto{\pgfqpoint{2.758351in}{0.618222in}}%
\pgfpathlineto{\pgfqpoint{2.758907in}{0.601741in}}%
\pgfpathlineto{\pgfqpoint{2.759463in}{0.607062in}}%
\pgfpathlineto{\pgfqpoint{2.760019in}{0.616714in}}%
\pgfpathlineto{\pgfqpoint{2.761687in}{0.602881in}}%
\pgfpathlineto{\pgfqpoint{2.762242in}{0.615492in}}%
\pgfpathlineto{\pgfqpoint{2.762798in}{0.604585in}}%
\pgfpathlineto{\pgfqpoint{2.763354in}{0.608995in}}%
\pgfpathlineto{\pgfqpoint{2.763910in}{0.601235in}}%
\pgfpathlineto{\pgfqpoint{2.764466in}{0.617931in}}%
\pgfpathlineto{\pgfqpoint{2.765022in}{0.607329in}}%
\pgfpathlineto{\pgfqpoint{2.765578in}{0.606974in}}%
\pgfpathlineto{\pgfqpoint{2.767801in}{0.630637in}}%
\pgfpathlineto{\pgfqpoint{2.768357in}{0.628617in}}%
\pgfpathlineto{\pgfqpoint{2.768913in}{0.632624in}}%
\pgfpathlineto{\pgfqpoint{2.769469in}{0.647744in}}%
\pgfpathlineto{\pgfqpoint{2.770025in}{0.637513in}}%
\pgfpathlineto{\pgfqpoint{2.770580in}{0.639540in}}%
\pgfpathlineto{\pgfqpoint{2.771136in}{0.628091in}}%
\pgfpathlineto{\pgfqpoint{2.771692in}{0.641281in}}%
\pgfpathlineto{\pgfqpoint{2.772248in}{0.640243in}}%
\pgfpathlineto{\pgfqpoint{2.773360in}{0.630466in}}%
\pgfpathlineto{\pgfqpoint{2.774471in}{0.608994in}}%
\pgfpathlineto{\pgfqpoint{2.775027in}{0.613859in}}%
\pgfpathlineto{\pgfqpoint{2.775583in}{0.612352in}}%
\pgfpathlineto{\pgfqpoint{2.776695in}{0.649235in}}%
\pgfpathlineto{\pgfqpoint{2.778362in}{0.612672in}}%
\pgfpathlineto{\pgfqpoint{2.780586in}{0.649668in}}%
\pgfpathlineto{\pgfqpoint{2.781142in}{0.611659in}}%
\pgfpathlineto{\pgfqpoint{2.781698in}{0.612529in}}%
\pgfpathlineto{\pgfqpoint{2.783365in}{0.637656in}}%
\pgfpathlineto{\pgfqpoint{2.784477in}{0.607450in}}%
\pgfpathlineto{\pgfqpoint{2.785033in}{0.613721in}}%
\pgfpathlineto{\pgfqpoint{2.785589in}{0.650776in}}%
\pgfpathlineto{\pgfqpoint{2.786145in}{0.627704in}}%
\pgfpathlineto{\pgfqpoint{2.786700in}{0.622910in}}%
\pgfpathlineto{\pgfqpoint{2.787256in}{0.623908in}}%
\pgfpathlineto{\pgfqpoint{2.787812in}{0.633153in}}%
\pgfpathlineto{\pgfqpoint{2.788368in}{0.605950in}}%
\pgfpathlineto{\pgfqpoint{2.788924in}{0.628073in}}%
\pgfpathlineto{\pgfqpoint{2.789480in}{0.626728in}}%
\pgfpathlineto{\pgfqpoint{2.790036in}{0.649271in}}%
\pgfpathlineto{\pgfqpoint{2.790591in}{0.607810in}}%
\pgfpathlineto{\pgfqpoint{2.791147in}{0.631915in}}%
\pgfpathlineto{\pgfqpoint{2.791703in}{0.657742in}}%
\pgfpathlineto{\pgfqpoint{2.792259in}{0.611619in}}%
\pgfpathlineto{\pgfqpoint{2.792815in}{0.665346in}}%
\pgfpathlineto{\pgfqpoint{2.793371in}{0.653942in}}%
\pgfpathlineto{\pgfqpoint{2.793927in}{0.611560in}}%
\pgfpathlineto{\pgfqpoint{2.794482in}{0.647317in}}%
\pgfpathlineto{\pgfqpoint{2.795594in}{0.657282in}}%
\pgfpathlineto{\pgfqpoint{2.796150in}{0.619696in}}%
\pgfpathlineto{\pgfqpoint{2.796706in}{0.668910in}}%
\pgfpathlineto{\pgfqpoint{2.797262in}{0.653563in}}%
\pgfpathlineto{\pgfqpoint{2.797818in}{0.635288in}}%
\pgfpathlineto{\pgfqpoint{2.798373in}{0.662297in}}%
\pgfpathlineto{\pgfqpoint{2.798929in}{0.635873in}}%
\pgfpathlineto{\pgfqpoint{2.799485in}{0.638163in}}%
\pgfpathlineto{\pgfqpoint{2.800597in}{0.622553in}}%
\pgfpathlineto{\pgfqpoint{2.802264in}{0.670541in}}%
\pgfpathlineto{\pgfqpoint{2.802820in}{0.659321in}}%
\pgfpathlineto{\pgfqpoint{2.803376in}{0.631502in}}%
\pgfpathlineto{\pgfqpoint{2.803932in}{0.646087in}}%
\pgfpathlineto{\pgfqpoint{2.804488in}{0.670641in}}%
\pgfpathlineto{\pgfqpoint{2.805044in}{0.662026in}}%
\pgfpathlineto{\pgfqpoint{2.806156in}{0.639678in}}%
\pgfpathlineto{\pgfqpoint{2.806711in}{0.641261in}}%
\pgfpathlineto{\pgfqpoint{2.807267in}{0.655579in}}%
\pgfpathlineto{\pgfqpoint{2.807823in}{0.647832in}}%
\pgfpathlineto{\pgfqpoint{2.808379in}{0.627066in}}%
\pgfpathlineto{\pgfqpoint{2.808935in}{0.628845in}}%
\pgfpathlineto{\pgfqpoint{2.809491in}{0.628653in}}%
\pgfpathlineto{\pgfqpoint{2.810047in}{0.635849in}}%
\pgfpathlineto{\pgfqpoint{2.811714in}{0.601359in}}%
\pgfpathlineto{\pgfqpoint{2.812270in}{0.617312in}}%
\pgfpathlineto{\pgfqpoint{2.812826in}{0.615746in}}%
\pgfpathlineto{\pgfqpoint{2.813382in}{0.615430in}}%
\pgfpathlineto{\pgfqpoint{2.813938in}{0.606483in}}%
\pgfpathlineto{\pgfqpoint{2.814493in}{0.632212in}}%
\pgfpathlineto{\pgfqpoint{2.815049in}{0.613543in}}%
\pgfpathlineto{\pgfqpoint{2.815605in}{0.617540in}}%
\pgfpathlineto{\pgfqpoint{2.816161in}{0.609941in}}%
\pgfpathlineto{\pgfqpoint{2.816717in}{0.615073in}}%
\pgfpathlineto{\pgfqpoint{2.817273in}{0.618519in}}%
\pgfpathlineto{\pgfqpoint{2.817829in}{0.616874in}}%
\pgfpathlineto{\pgfqpoint{2.818384in}{0.611910in}}%
\pgfpathlineto{\pgfqpoint{2.818940in}{0.612866in}}%
\pgfpathlineto{\pgfqpoint{2.819496in}{0.617785in}}%
\pgfpathlineto{\pgfqpoint{2.820052in}{0.610120in}}%
\pgfpathlineto{\pgfqpoint{2.820608in}{0.619330in}}%
\pgfpathlineto{\pgfqpoint{2.822275in}{0.606887in}}%
\pgfpathlineto{\pgfqpoint{2.822831in}{0.602279in}}%
\pgfpathlineto{\pgfqpoint{2.823943in}{0.618678in}}%
\pgfpathlineto{\pgfqpoint{2.824499in}{0.616112in}}%
\pgfpathlineto{\pgfqpoint{2.825055in}{0.615260in}}%
\pgfpathlineto{\pgfqpoint{2.825611in}{0.607907in}}%
\pgfpathlineto{\pgfqpoint{2.826167in}{0.613598in}}%
\pgfpathlineto{\pgfqpoint{2.826722in}{0.612649in}}%
\pgfpathlineto{\pgfqpoint{2.827278in}{0.616034in}}%
\pgfpathlineto{\pgfqpoint{2.828390in}{0.635257in}}%
\pgfpathlineto{\pgfqpoint{2.828946in}{0.635240in}}%
\pgfpathlineto{\pgfqpoint{2.829502in}{0.633244in}}%
\pgfpathlineto{\pgfqpoint{2.830058in}{0.634995in}}%
\pgfpathlineto{\pgfqpoint{2.831169in}{0.649650in}}%
\pgfpathlineto{\pgfqpoint{2.831725in}{0.646228in}}%
\pgfpathlineto{\pgfqpoint{2.832281in}{0.632362in}}%
\pgfpathlineto{\pgfqpoint{2.832837in}{0.633579in}}%
\pgfpathlineto{\pgfqpoint{2.833393in}{0.639117in}}%
\pgfpathlineto{\pgfqpoint{2.835060in}{0.604599in}}%
\pgfpathlineto{\pgfqpoint{2.836728in}{0.640602in}}%
\pgfpathlineto{\pgfqpoint{2.838395in}{0.607341in}}%
\pgfpathlineto{\pgfqpoint{2.840063in}{0.639762in}}%
\pgfpathlineto{\pgfqpoint{2.840619in}{0.638371in}}%
\pgfpathlineto{\pgfqpoint{2.841731in}{0.606557in}}%
\pgfpathlineto{\pgfqpoint{2.842842in}{0.650247in}}%
\pgfpathlineto{\pgfqpoint{2.844510in}{0.612735in}}%
\pgfpathlineto{\pgfqpoint{2.845066in}{0.630798in}}%
\pgfpathlineto{\pgfqpoint{2.845622in}{0.622040in}}%
\pgfpathlineto{\pgfqpoint{2.846178in}{0.627226in}}%
\pgfpathlineto{\pgfqpoint{2.846733in}{0.610620in}}%
\pgfpathlineto{\pgfqpoint{2.847289in}{0.652370in}}%
\pgfpathlineto{\pgfqpoint{2.847845in}{0.606988in}}%
\pgfpathlineto{\pgfqpoint{2.848401in}{0.640639in}}%
\pgfpathlineto{\pgfqpoint{2.850069in}{0.628719in}}%
\pgfpathlineto{\pgfqpoint{2.850624in}{0.676046in}}%
\pgfpathlineto{\pgfqpoint{2.851180in}{0.625073in}}%
\pgfpathlineto{\pgfqpoint{2.851736in}{0.646031in}}%
\pgfpathlineto{\pgfqpoint{2.852292in}{0.651871in}}%
\pgfpathlineto{\pgfqpoint{2.852848in}{0.672843in}}%
\pgfpathlineto{\pgfqpoint{2.853404in}{0.657778in}}%
\pgfpathlineto{\pgfqpoint{2.853960in}{0.670518in}}%
\pgfpathlineto{\pgfqpoint{2.855627in}{0.619238in}}%
\pgfpathlineto{\pgfqpoint{2.856183in}{0.677417in}}%
\pgfpathlineto{\pgfqpoint{2.856739in}{0.664288in}}%
\pgfpathlineto{\pgfqpoint{2.857295in}{0.608104in}}%
\pgfpathlineto{\pgfqpoint{2.857851in}{0.653236in}}%
\pgfpathlineto{\pgfqpoint{2.858406in}{0.622731in}}%
\pgfpathlineto{\pgfqpoint{2.858962in}{0.644065in}}%
\pgfpathlineto{\pgfqpoint{2.859518in}{0.673327in}}%
\pgfpathlineto{\pgfqpoint{2.860074in}{0.666865in}}%
\pgfpathlineto{\pgfqpoint{2.860630in}{0.644908in}}%
\pgfpathlineto{\pgfqpoint{2.861186in}{0.651764in}}%
\pgfpathlineto{\pgfqpoint{2.861742in}{0.653499in}}%
\pgfpathlineto{\pgfqpoint{2.862298in}{0.696811in}}%
\pgfpathlineto{\pgfqpoint{2.863965in}{0.610680in}}%
\pgfpathlineto{\pgfqpoint{2.864521in}{0.659152in}}%
\pgfpathlineto{\pgfqpoint{2.865077in}{0.622529in}}%
\pgfpathlineto{\pgfqpoint{2.865633in}{0.646129in}}%
\pgfpathlineto{\pgfqpoint{2.867300in}{0.608475in}}%
\pgfpathlineto{\pgfqpoint{2.867856in}{0.635122in}}%
\pgfpathlineto{\pgfqpoint{2.868412in}{0.621414in}}%
\pgfpathlineto{\pgfqpoint{2.868968in}{0.609371in}}%
\pgfpathlineto{\pgfqpoint{2.869524in}{0.614137in}}%
\pgfpathlineto{\pgfqpoint{2.870635in}{0.628663in}}%
\pgfpathlineto{\pgfqpoint{2.872303in}{0.605723in}}%
\pgfpathlineto{\pgfqpoint{2.875082in}{0.614871in}}%
\pgfpathlineto{\pgfqpoint{2.876750in}{0.605129in}}%
\pgfpathlineto{\pgfqpoint{2.877306in}{0.608317in}}%
\pgfpathlineto{\pgfqpoint{2.877862in}{0.614802in}}%
\pgfpathlineto{\pgfqpoint{2.878973in}{0.605204in}}%
\pgfpathlineto{\pgfqpoint{2.879529in}{0.614377in}}%
\pgfpathlineto{\pgfqpoint{2.880085in}{0.613111in}}%
\pgfpathlineto{\pgfqpoint{2.880641in}{0.604004in}}%
\pgfpathlineto{\pgfqpoint{2.881197in}{0.609364in}}%
\pgfpathlineto{\pgfqpoint{2.883976in}{0.601491in}}%
\pgfpathlineto{\pgfqpoint{2.885644in}{0.614232in}}%
\pgfpathlineto{\pgfqpoint{2.886200in}{0.607939in}}%
\pgfpathlineto{\pgfqpoint{2.888979in}{0.635021in}}%
\pgfpathlineto{\pgfqpoint{2.889535in}{0.626353in}}%
\pgfpathlineto{\pgfqpoint{2.890646in}{0.646307in}}%
\pgfpathlineto{\pgfqpoint{2.891202in}{0.638916in}}%
\pgfpathlineto{\pgfqpoint{2.891758in}{0.637054in}}%
\pgfpathlineto{\pgfqpoint{2.892314in}{0.645270in}}%
\pgfpathlineto{\pgfqpoint{2.892870in}{0.644138in}}%
\pgfpathlineto{\pgfqpoint{2.894537in}{0.602224in}}%
\pgfpathlineto{\pgfqpoint{2.896761in}{0.653596in}}%
\pgfpathlineto{\pgfqpoint{2.898429in}{0.605049in}}%
\pgfpathlineto{\pgfqpoint{2.899540in}{0.650951in}}%
\pgfpathlineto{\pgfqpoint{2.900096in}{0.641494in}}%
\pgfpathlineto{\pgfqpoint{2.900652in}{0.638614in}}%
\pgfpathlineto{\pgfqpoint{2.901208in}{0.607938in}}%
\pgfpathlineto{\pgfqpoint{2.901764in}{0.614016in}}%
\pgfpathlineto{\pgfqpoint{2.902875in}{0.649217in}}%
\pgfpathlineto{\pgfqpoint{2.903431in}{0.607244in}}%
\pgfpathlineto{\pgfqpoint{2.903987in}{0.609531in}}%
\pgfpathlineto{\pgfqpoint{2.904543in}{0.648861in}}%
\pgfpathlineto{\pgfqpoint{2.905099in}{0.628630in}}%
\pgfpathlineto{\pgfqpoint{2.905655in}{0.629102in}}%
\pgfpathlineto{\pgfqpoint{2.906211in}{0.619945in}}%
\pgfpathlineto{\pgfqpoint{2.906766in}{0.654476in}}%
\pgfpathlineto{\pgfqpoint{2.907322in}{0.609635in}}%
\pgfpathlineto{\pgfqpoint{2.907878in}{0.676947in}}%
\pgfpathlineto{\pgfqpoint{2.908434in}{0.625525in}}%
\pgfpathlineto{\pgfqpoint{2.908990in}{0.623329in}}%
\pgfpathlineto{\pgfqpoint{2.910102in}{0.701301in}}%
\pgfpathlineto{\pgfqpoint{2.910657in}{0.682147in}}%
\pgfpathlineto{\pgfqpoint{2.911213in}{0.626968in}}%
\pgfpathlineto{\pgfqpoint{2.911769in}{0.674623in}}%
\pgfpathlineto{\pgfqpoint{2.912325in}{0.657281in}}%
\pgfpathlineto{\pgfqpoint{2.912881in}{0.690705in}}%
\pgfpathlineto{\pgfqpoint{2.913437in}{0.681504in}}%
\pgfpathlineto{\pgfqpoint{2.913993in}{0.669527in}}%
\pgfpathlineto{\pgfqpoint{2.914548in}{0.621289in}}%
\pgfpathlineto{\pgfqpoint{2.915104in}{0.623854in}}%
\pgfpathlineto{\pgfqpoint{2.916216in}{0.618124in}}%
\pgfpathlineto{\pgfqpoint{2.916772in}{0.631672in}}%
\pgfpathlineto{\pgfqpoint{2.917328in}{0.735023in}}%
\pgfpathlineto{\pgfqpoint{2.917884in}{0.682594in}}%
\pgfpathlineto{\pgfqpoint{2.918440in}{0.632956in}}%
\pgfpathlineto{\pgfqpoint{2.918995in}{0.653376in}}%
\pgfpathlineto{\pgfqpoint{2.919551in}{0.702307in}}%
\pgfpathlineto{\pgfqpoint{2.921219in}{0.628857in}}%
\pgfpathlineto{\pgfqpoint{2.922886in}{0.661214in}}%
\pgfpathlineto{\pgfqpoint{2.924554in}{0.606472in}}%
\pgfpathlineto{\pgfqpoint{2.925110in}{0.629011in}}%
\pgfpathlineto{\pgfqpoint{2.925666in}{0.622202in}}%
\pgfpathlineto{\pgfqpoint{2.926222in}{0.613422in}}%
\pgfpathlineto{\pgfqpoint{2.926777in}{0.620950in}}%
\pgfpathlineto{\pgfqpoint{2.928445in}{0.602866in}}%
\pgfpathlineto{\pgfqpoint{2.929001in}{0.619581in}}%
\pgfpathlineto{\pgfqpoint{2.929557in}{0.612378in}}%
\pgfpathlineto{\pgfqpoint{2.930668in}{0.621728in}}%
\pgfpathlineto{\pgfqpoint{2.931224in}{0.617982in}}%
\pgfpathlineto{\pgfqpoint{2.931780in}{0.602828in}}%
\pgfpathlineto{\pgfqpoint{2.932336in}{0.621602in}}%
\pgfpathlineto{\pgfqpoint{2.932892in}{0.609695in}}%
\pgfpathlineto{\pgfqpoint{2.933448in}{0.617924in}}%
\pgfpathlineto{\pgfqpoint{2.934004in}{0.606925in}}%
\pgfpathlineto{\pgfqpoint{2.934559in}{0.607202in}}%
\pgfpathlineto{\pgfqpoint{2.935115in}{0.608744in}}%
\pgfpathlineto{\pgfqpoint{2.935671in}{0.617344in}}%
\pgfpathlineto{\pgfqpoint{2.936227in}{0.609884in}}%
\pgfpathlineto{\pgfqpoint{2.937339in}{0.602658in}}%
\pgfpathlineto{\pgfqpoint{2.937895in}{0.603865in}}%
\pgfpathlineto{\pgfqpoint{2.938451in}{0.608451in}}%
\pgfpathlineto{\pgfqpoint{2.939006in}{0.607486in}}%
\pgfpathlineto{\pgfqpoint{2.939562in}{0.604441in}}%
\pgfpathlineto{\pgfqpoint{2.940118in}{0.617954in}}%
\pgfpathlineto{\pgfqpoint{2.940674in}{0.607432in}}%
\pgfpathlineto{\pgfqpoint{2.941230in}{0.611772in}}%
\pgfpathlineto{\pgfqpoint{2.941786in}{0.609973in}}%
\pgfpathlineto{\pgfqpoint{2.942897in}{0.609128in}}%
\pgfpathlineto{\pgfqpoint{2.943453in}{0.605103in}}%
\pgfpathlineto{\pgfqpoint{2.944009in}{0.613555in}}%
\pgfpathlineto{\pgfqpoint{2.944565in}{0.606695in}}%
\pgfpathlineto{\pgfqpoint{2.945121in}{0.612474in}}%
\pgfpathlineto{\pgfqpoint{2.945677in}{0.605388in}}%
\pgfpathlineto{\pgfqpoint{2.946233in}{0.608214in}}%
\pgfpathlineto{\pgfqpoint{2.946788in}{0.605640in}}%
\pgfpathlineto{\pgfqpoint{2.949012in}{0.652739in}}%
\pgfpathlineto{\pgfqpoint{2.949568in}{0.631543in}}%
\pgfpathlineto{\pgfqpoint{2.950124in}{0.640597in}}%
\pgfpathlineto{\pgfqpoint{2.951791in}{0.661989in}}%
\pgfpathlineto{\pgfqpoint{2.952347in}{0.662108in}}%
\pgfpathlineto{\pgfqpoint{2.954570in}{0.608285in}}%
\pgfpathlineto{\pgfqpoint{2.955126in}{0.615727in}}%
\pgfpathlineto{\pgfqpoint{2.956794in}{0.673431in}}%
\pgfpathlineto{\pgfqpoint{2.958462in}{0.620066in}}%
\pgfpathlineto{\pgfqpoint{2.959017in}{0.648235in}}%
\pgfpathlineto{\pgfqpoint{2.959573in}{0.641253in}}%
\pgfpathlineto{\pgfqpoint{2.960129in}{0.639426in}}%
\pgfpathlineto{\pgfqpoint{2.961241in}{0.611212in}}%
\pgfpathlineto{\pgfqpoint{2.962353in}{0.672204in}}%
\pgfpathlineto{\pgfqpoint{2.962908in}{0.608119in}}%
\pgfpathlineto{\pgfqpoint{2.963464in}{0.621053in}}%
\pgfpathlineto{\pgfqpoint{2.964020in}{0.659728in}}%
\pgfpathlineto{\pgfqpoint{2.964576in}{0.619025in}}%
\pgfpathlineto{\pgfqpoint{2.965132in}{0.644347in}}%
\pgfpathlineto{\pgfqpoint{2.965688in}{0.664047in}}%
\pgfpathlineto{\pgfqpoint{2.966799in}{0.630941in}}%
\pgfpathlineto{\pgfqpoint{2.967355in}{0.706689in}}%
\pgfpathlineto{\pgfqpoint{2.967911in}{0.682993in}}%
\pgfpathlineto{\pgfqpoint{2.968467in}{0.658245in}}%
\pgfpathlineto{\pgfqpoint{2.969023in}{0.694807in}}%
\pgfpathlineto{\pgfqpoint{2.969579in}{0.645637in}}%
\pgfpathlineto{\pgfqpoint{2.970135in}{0.706423in}}%
\pgfpathlineto{\pgfqpoint{2.970690in}{0.609840in}}%
\pgfpathlineto{\pgfqpoint{2.971246in}{0.683403in}}%
\pgfpathlineto{\pgfqpoint{2.972358in}{0.636854in}}%
\pgfpathlineto{\pgfqpoint{2.972914in}{0.667566in}}%
\pgfpathlineto{\pgfqpoint{2.973470in}{0.614363in}}%
\pgfpathlineto{\pgfqpoint{2.974026in}{0.655745in}}%
\pgfpathlineto{\pgfqpoint{2.974582in}{0.741230in}}%
\pgfpathlineto{\pgfqpoint{2.975137in}{0.682097in}}%
\pgfpathlineto{\pgfqpoint{2.975693in}{0.629755in}}%
\pgfpathlineto{\pgfqpoint{2.976249in}{0.655954in}}%
\pgfpathlineto{\pgfqpoint{2.976805in}{0.660190in}}%
\pgfpathlineto{\pgfqpoint{2.977361in}{0.679275in}}%
\pgfpathlineto{\pgfqpoint{2.977917in}{0.670584in}}%
\pgfpathlineto{\pgfqpoint{2.978473in}{0.660937in}}%
\pgfpathlineto{\pgfqpoint{2.979028in}{0.627314in}}%
\pgfpathlineto{\pgfqpoint{2.979584in}{0.665089in}}%
\pgfpathlineto{\pgfqpoint{2.980140in}{0.627815in}}%
\pgfpathlineto{\pgfqpoint{2.982919in}{0.608023in}}%
\pgfpathlineto{\pgfqpoint{2.984587in}{0.627386in}}%
\pgfpathlineto{\pgfqpoint{2.985143in}{0.604532in}}%
\pgfpathlineto{\pgfqpoint{2.985699in}{0.611314in}}%
\pgfpathlineto{\pgfqpoint{2.986255in}{0.620916in}}%
\pgfpathlineto{\pgfqpoint{2.986810in}{0.617946in}}%
\pgfpathlineto{\pgfqpoint{2.987922in}{0.620159in}}%
\pgfpathlineto{\pgfqpoint{2.988478in}{0.604370in}}%
\pgfpathlineto{\pgfqpoint{2.989034in}{0.619499in}}%
\pgfpathlineto{\pgfqpoint{2.989590in}{0.609979in}}%
\pgfpathlineto{\pgfqpoint{2.990146in}{0.613583in}}%
\pgfpathlineto{\pgfqpoint{2.990701in}{0.615418in}}%
\pgfpathlineto{\pgfqpoint{2.991257in}{0.613153in}}%
\pgfpathlineto{\pgfqpoint{2.991813in}{0.627985in}}%
\pgfpathlineto{\pgfqpoint{2.992925in}{0.604102in}}%
\pgfpathlineto{\pgfqpoint{2.993481in}{0.613772in}}%
\pgfpathlineto{\pgfqpoint{2.994037in}{0.610469in}}%
\pgfpathlineto{\pgfqpoint{2.994593in}{0.604194in}}%
\pgfpathlineto{\pgfqpoint{2.995148in}{0.617364in}}%
\pgfpathlineto{\pgfqpoint{2.995704in}{0.611367in}}%
\pgfpathlineto{\pgfqpoint{2.996260in}{0.609958in}}%
\pgfpathlineto{\pgfqpoint{2.996816in}{0.617400in}}%
\pgfpathlineto{\pgfqpoint{2.997928in}{0.606014in}}%
\pgfpathlineto{\pgfqpoint{2.998484in}{0.619874in}}%
\pgfpathlineto{\pgfqpoint{2.999039in}{0.618170in}}%
\pgfpathlineto{\pgfqpoint{2.999595in}{0.618616in}}%
\pgfpathlineto{\pgfqpoint{3.000151in}{0.616984in}}%
\pgfpathlineto{\pgfqpoint{3.000707in}{0.606079in}}%
\pgfpathlineto{\pgfqpoint{3.001263in}{0.612636in}}%
\pgfpathlineto{\pgfqpoint{3.003486in}{0.604381in}}%
\pgfpathlineto{\pgfqpoint{3.004042in}{0.618707in}}%
\pgfpathlineto{\pgfqpoint{3.004598in}{0.610738in}}%
\pgfpathlineto{\pgfqpoint{3.005154in}{0.616106in}}%
\pgfpathlineto{\pgfqpoint{3.005710in}{0.611697in}}%
\pgfpathlineto{\pgfqpoint{3.006266in}{0.605333in}}%
\pgfpathlineto{\pgfqpoint{3.006821in}{0.619071in}}%
\pgfpathlineto{\pgfqpoint{3.007377in}{0.617617in}}%
\pgfpathlineto{\pgfqpoint{3.007933in}{0.615243in}}%
\pgfpathlineto{\pgfqpoint{3.010712in}{0.643147in}}%
\pgfpathlineto{\pgfqpoint{3.011268in}{0.639406in}}%
\pgfpathlineto{\pgfqpoint{3.011824in}{0.663267in}}%
\pgfpathlineto{\pgfqpoint{3.012380in}{0.645172in}}%
\pgfpathlineto{\pgfqpoint{3.015159in}{0.615032in}}%
\pgfpathlineto{\pgfqpoint{3.015715in}{0.617404in}}%
\pgfpathlineto{\pgfqpoint{3.016271in}{0.653051in}}%
\pgfpathlineto{\pgfqpoint{3.016827in}{0.650152in}}%
\pgfpathlineto{\pgfqpoint{3.017939in}{0.613155in}}%
\pgfpathlineto{\pgfqpoint{3.018495in}{0.624070in}}%
\pgfpathlineto{\pgfqpoint{3.019050in}{0.629466in}}%
\pgfpathlineto{\pgfqpoint{3.019606in}{0.658416in}}%
\pgfpathlineto{\pgfqpoint{3.020718in}{0.609752in}}%
\pgfpathlineto{\pgfqpoint{3.021830in}{0.643334in}}%
\pgfpathlineto{\pgfqpoint{3.022386in}{0.613285in}}%
\pgfpathlineto{\pgfqpoint{3.022941in}{0.655435in}}%
\pgfpathlineto{\pgfqpoint{3.023497in}{0.633462in}}%
\pgfpathlineto{\pgfqpoint{3.024053in}{0.618661in}}%
\pgfpathlineto{\pgfqpoint{3.025165in}{0.667806in}}%
\pgfpathlineto{\pgfqpoint{3.025721in}{0.655966in}}%
\pgfpathlineto{\pgfqpoint{3.026277in}{0.630270in}}%
\pgfpathlineto{\pgfqpoint{3.026832in}{0.633950in}}%
\pgfpathlineto{\pgfqpoint{3.028500in}{0.694506in}}%
\pgfpathlineto{\pgfqpoint{3.029612in}{0.648375in}}%
\pgfpathlineto{\pgfqpoint{3.030168in}{0.610247in}}%
\pgfpathlineto{\pgfqpoint{3.030724in}{0.617408in}}%
\pgfpathlineto{\pgfqpoint{3.031279in}{0.624072in}}%
\pgfpathlineto{\pgfqpoint{3.032391in}{0.705039in}}%
\pgfpathlineto{\pgfqpoint{3.034059in}{0.614465in}}%
\pgfpathlineto{\pgfqpoint{3.034615in}{0.679927in}}%
\pgfpathlineto{\pgfqpoint{3.035170in}{0.649157in}}%
\pgfpathlineto{\pgfqpoint{3.035726in}{0.658525in}}%
\pgfpathlineto{\pgfqpoint{3.037394in}{0.604779in}}%
\pgfpathlineto{\pgfqpoint{3.037950in}{0.627852in}}%
\pgfpathlineto{\pgfqpoint{3.038506in}{0.625999in}}%
\pgfpathlineto{\pgfqpoint{3.039061in}{0.627060in}}%
\pgfpathlineto{\pgfqpoint{3.040173in}{0.607222in}}%
\pgfpathlineto{\pgfqpoint{3.040729in}{0.621348in}}%
\pgfpathlineto{\pgfqpoint{3.041285in}{0.614566in}}%
\pgfpathlineto{\pgfqpoint{3.041841in}{0.619928in}}%
\pgfpathlineto{\pgfqpoint{3.042397in}{0.601656in}}%
\pgfpathlineto{\pgfqpoint{3.042952in}{0.621749in}}%
\pgfpathlineto{\pgfqpoint{3.043508in}{0.614156in}}%
\pgfpathlineto{\pgfqpoint{3.045176in}{0.612311in}}%
\pgfpathlineto{\pgfqpoint{3.046288in}{0.615313in}}%
\pgfpathlineto{\pgfqpoint{3.047399in}{0.605784in}}%
\pgfpathlineto{\pgfqpoint{3.047955in}{0.609884in}}%
\pgfpathlineto{\pgfqpoint{3.048511in}{0.606585in}}%
\pgfpathlineto{\pgfqpoint{3.049067in}{0.604791in}}%
\pgfpathlineto{\pgfqpoint{3.050735in}{0.609135in}}%
\pgfpathlineto{\pgfqpoint{3.051290in}{0.604192in}}%
\pgfpathlineto{\pgfqpoint{3.051846in}{0.605392in}}%
\pgfpathlineto{\pgfqpoint{3.052958in}{0.615123in}}%
\pgfpathlineto{\pgfqpoint{3.054626in}{0.608400in}}%
\pgfpathlineto{\pgfqpoint{3.055181in}{0.613464in}}%
\pgfpathlineto{\pgfqpoint{3.056293in}{0.603717in}}%
\pgfpathlineto{\pgfqpoint{3.056849in}{0.614234in}}%
\pgfpathlineto{\pgfqpoint{3.057405in}{0.611160in}}%
\pgfpathlineto{\pgfqpoint{3.057961in}{0.604538in}}%
\pgfpathlineto{\pgfqpoint{3.059628in}{0.612357in}}%
\pgfpathlineto{\pgfqpoint{3.060184in}{0.618054in}}%
\pgfpathlineto{\pgfqpoint{3.060740in}{0.602884in}}%
\pgfpathlineto{\pgfqpoint{3.061296in}{0.621661in}}%
\pgfpathlineto{\pgfqpoint{3.061852in}{0.614020in}}%
\pgfpathlineto{\pgfqpoint{3.062408in}{0.615153in}}%
\pgfpathlineto{\pgfqpoint{3.062963in}{0.613482in}}%
\pgfpathlineto{\pgfqpoint{3.063519in}{0.607695in}}%
\pgfpathlineto{\pgfqpoint{3.064075in}{0.618673in}}%
\pgfpathlineto{\pgfqpoint{3.064631in}{0.600748in}}%
\pgfpathlineto{\pgfqpoint{3.065187in}{0.622985in}}%
\pgfpathlineto{\pgfqpoint{3.065743in}{0.611182in}}%
\pgfpathlineto{\pgfqpoint{3.066854in}{0.617245in}}%
\pgfpathlineto{\pgfqpoint{3.067410in}{0.610622in}}%
\pgfpathlineto{\pgfqpoint{3.067966in}{0.614264in}}%
\pgfpathlineto{\pgfqpoint{3.068522in}{0.625674in}}%
\pgfpathlineto{\pgfqpoint{3.069078in}{0.613565in}}%
\pgfpathlineto{\pgfqpoint{3.070746in}{0.633162in}}%
\pgfpathlineto{\pgfqpoint{3.072413in}{0.657538in}}%
\pgfpathlineto{\pgfqpoint{3.072969in}{0.659111in}}%
\pgfpathlineto{\pgfqpoint{3.074637in}{0.617386in}}%
\pgfpathlineto{\pgfqpoint{3.075192in}{0.627688in}}%
\pgfpathlineto{\pgfqpoint{3.075748in}{0.661370in}}%
\pgfpathlineto{\pgfqpoint{3.076304in}{0.645432in}}%
\pgfpathlineto{\pgfqpoint{3.077416in}{0.621724in}}%
\pgfpathlineto{\pgfqpoint{3.077972in}{0.625363in}}%
\pgfpathlineto{\pgfqpoint{3.078528in}{0.625915in}}%
\pgfpathlineto{\pgfqpoint{3.079083in}{0.660666in}}%
\pgfpathlineto{\pgfqpoint{3.079639in}{0.620320in}}%
\pgfpathlineto{\pgfqpoint{3.080195in}{0.621921in}}%
\pgfpathlineto{\pgfqpoint{3.081307in}{0.631061in}}%
\pgfpathlineto{\pgfqpoint{3.081863in}{0.617951in}}%
\pgfpathlineto{\pgfqpoint{3.082419in}{0.677679in}}%
\pgfpathlineto{\pgfqpoint{3.082974in}{0.657917in}}%
\pgfpathlineto{\pgfqpoint{3.083530in}{0.619944in}}%
\pgfpathlineto{\pgfqpoint{3.085198in}{0.707877in}}%
\pgfpathlineto{\pgfqpoint{3.086866in}{0.627838in}}%
\pgfpathlineto{\pgfqpoint{3.087421in}{0.683762in}}%
\pgfpathlineto{\pgfqpoint{3.087977in}{0.653664in}}%
\pgfpathlineto{\pgfqpoint{3.088533in}{0.620993in}}%
\pgfpathlineto{\pgfqpoint{3.089089in}{0.653493in}}%
\pgfpathlineto{\pgfqpoint{3.089645in}{0.734663in}}%
\pgfpathlineto{\pgfqpoint{3.090201in}{0.637948in}}%
\pgfpathlineto{\pgfqpoint{3.090757in}{0.638236in}}%
\pgfpathlineto{\pgfqpoint{3.091312in}{0.640736in}}%
\pgfpathlineto{\pgfqpoint{3.091868in}{0.650922in}}%
\pgfpathlineto{\pgfqpoint{3.092424in}{0.623172in}}%
\pgfpathlineto{\pgfqpoint{3.092980in}{0.652299in}}%
\pgfpathlineto{\pgfqpoint{3.094092in}{0.609842in}}%
\pgfpathlineto{\pgfqpoint{3.094648in}{0.610353in}}%
\pgfpathlineto{\pgfqpoint{3.095203in}{0.622340in}}%
\pgfpathlineto{\pgfqpoint{3.095759in}{0.616224in}}%
\pgfpathlineto{\pgfqpoint{3.096315in}{0.610534in}}%
\pgfpathlineto{\pgfqpoint{3.096871in}{0.616546in}}%
\pgfpathlineto{\pgfqpoint{3.097427in}{0.636168in}}%
\pgfpathlineto{\pgfqpoint{3.097983in}{0.606291in}}%
\pgfpathlineto{\pgfqpoint{3.098539in}{0.610853in}}%
\pgfpathlineto{\pgfqpoint{3.099094in}{0.616988in}}%
\pgfpathlineto{\pgfqpoint{3.099650in}{0.602489in}}%
\pgfpathlineto{\pgfqpoint{3.100206in}{0.623250in}}%
\pgfpathlineto{\pgfqpoint{3.100762in}{0.611335in}}%
\pgfpathlineto{\pgfqpoint{3.101318in}{0.622352in}}%
\pgfpathlineto{\pgfqpoint{3.101874in}{0.607833in}}%
\pgfpathlineto{\pgfqpoint{3.102430in}{0.608846in}}%
\pgfpathlineto{\pgfqpoint{3.102985in}{0.625133in}}%
\pgfpathlineto{\pgfqpoint{3.103541in}{0.610166in}}%
\pgfpathlineto{\pgfqpoint{3.104653in}{0.608780in}}%
\pgfpathlineto{\pgfqpoint{3.105209in}{0.603841in}}%
\pgfpathlineto{\pgfqpoint{3.106321in}{0.625655in}}%
\pgfpathlineto{\pgfqpoint{3.106877in}{0.618536in}}%
\pgfpathlineto{\pgfqpoint{3.107432in}{0.612675in}}%
\pgfpathlineto{\pgfqpoint{3.107988in}{0.619062in}}%
\pgfpathlineto{\pgfqpoint{3.108544in}{0.608808in}}%
\pgfpathlineto{\pgfqpoint{3.109100in}{0.622622in}}%
\pgfpathlineto{\pgfqpoint{3.109656in}{0.615067in}}%
\pgfpathlineto{\pgfqpoint{3.110212in}{0.606451in}}%
\pgfpathlineto{\pgfqpoint{3.110768in}{0.623162in}}%
\pgfpathlineto{\pgfqpoint{3.111323in}{0.618576in}}%
\pgfpathlineto{\pgfqpoint{3.112435in}{0.622371in}}%
\pgfpathlineto{\pgfqpoint{3.112991in}{0.606836in}}%
\pgfpathlineto{\pgfqpoint{3.113547in}{0.610373in}}%
\pgfpathlineto{\pgfqpoint{3.114103in}{0.610176in}}%
\pgfpathlineto{\pgfqpoint{3.115214in}{0.614192in}}%
\pgfpathlineto{\pgfqpoint{3.115770in}{0.625418in}}%
\pgfpathlineto{\pgfqpoint{3.116326in}{0.604758in}}%
\pgfpathlineto{\pgfqpoint{3.116882in}{0.616888in}}%
\pgfpathlineto{\pgfqpoint{3.117994in}{0.605960in}}%
\pgfpathlineto{\pgfqpoint{3.119105in}{0.618854in}}%
\pgfpathlineto{\pgfqpoint{3.120773in}{0.604690in}}%
\pgfpathlineto{\pgfqpoint{3.121329in}{0.615447in}}%
\pgfpathlineto{\pgfqpoint{3.121885in}{0.602499in}}%
\pgfpathlineto{\pgfqpoint{3.122441in}{0.607774in}}%
\pgfpathlineto{\pgfqpoint{3.122996in}{0.607913in}}%
\pgfpathlineto{\pgfqpoint{3.123552in}{0.616429in}}%
\pgfpathlineto{\pgfqpoint{3.124664in}{0.600595in}}%
\pgfpathlineto{\pgfqpoint{3.125220in}{0.616220in}}%
\pgfpathlineto{\pgfqpoint{3.125776in}{0.614882in}}%
\pgfpathlineto{\pgfqpoint{3.126888in}{0.606092in}}%
\pgfpathlineto{\pgfqpoint{3.127999in}{0.617034in}}%
\pgfpathlineto{\pgfqpoint{3.128555in}{0.607203in}}%
\pgfpathlineto{\pgfqpoint{3.129111in}{0.616827in}}%
\pgfpathlineto{\pgfqpoint{3.131334in}{0.671143in}}%
\pgfpathlineto{\pgfqpoint{3.133002in}{0.651376in}}%
\pgfpathlineto{\pgfqpoint{3.133558in}{0.652469in}}%
\pgfpathlineto{\pgfqpoint{3.134114in}{0.644541in}}%
\pgfpathlineto{\pgfqpoint{3.134670in}{0.619864in}}%
\pgfpathlineto{\pgfqpoint{3.136337in}{0.655678in}}%
\pgfpathlineto{\pgfqpoint{3.137449in}{0.618777in}}%
\pgfpathlineto{\pgfqpoint{3.138005in}{0.627495in}}%
\pgfpathlineto{\pgfqpoint{3.139672in}{0.665605in}}%
\pgfpathlineto{\pgfqpoint{3.140228in}{0.669745in}}%
\pgfpathlineto{\pgfqpoint{3.140784in}{0.662359in}}%
\pgfpathlineto{\pgfqpoint{3.141340in}{0.666155in}}%
\pgfpathlineto{\pgfqpoint{3.141896in}{0.673695in}}%
\pgfpathlineto{\pgfqpoint{3.142452in}{0.704371in}}%
\pgfpathlineto{\pgfqpoint{3.143007in}{0.661885in}}%
\pgfpathlineto{\pgfqpoint{3.144675in}{0.766297in}}%
\pgfpathlineto{\pgfqpoint{3.145787in}{0.608318in}}%
\pgfpathlineto{\pgfqpoint{3.147454in}{0.719067in}}%
\pgfpathlineto{\pgfqpoint{3.149122in}{0.609001in}}%
\pgfpathlineto{\pgfqpoint{3.149678in}{0.654584in}}%
\pgfpathlineto{\pgfqpoint{3.150234in}{0.639180in}}%
\pgfpathlineto{\pgfqpoint{3.150790in}{0.639015in}}%
\pgfpathlineto{\pgfqpoint{3.151901in}{0.613129in}}%
\pgfpathlineto{\pgfqpoint{3.152457in}{0.621068in}}%
\pgfpathlineto{\pgfqpoint{3.153013in}{0.620491in}}%
\pgfpathlineto{\pgfqpoint{3.154125in}{0.607871in}}%
\pgfpathlineto{\pgfqpoint{3.154681in}{0.624722in}}%
\pgfpathlineto{\pgfqpoint{3.155236in}{0.615907in}}%
\pgfpathlineto{\pgfqpoint{3.155792in}{0.611912in}}%
\pgfpathlineto{\pgfqpoint{3.156348in}{0.626005in}}%
\pgfpathlineto{\pgfqpoint{3.156904in}{0.618457in}}%
\pgfpathlineto{\pgfqpoint{3.157460in}{0.615228in}}%
\pgfpathlineto{\pgfqpoint{3.158016in}{0.635118in}}%
\pgfpathlineto{\pgfqpoint{3.158572in}{0.618177in}}%
\pgfpathlineto{\pgfqpoint{3.159683in}{0.602272in}}%
\pgfpathlineto{\pgfqpoint{3.161351in}{0.615626in}}%
\pgfpathlineto{\pgfqpoint{3.162463in}{0.606399in}}%
\pgfpathlineto{\pgfqpoint{3.164130in}{0.631469in}}%
\pgfpathlineto{\pgfqpoint{3.165798in}{0.601763in}}%
\pgfpathlineto{\pgfqpoint{3.166354in}{0.626242in}}%
\pgfpathlineto{\pgfqpoint{3.166910in}{0.611643in}}%
\pgfpathlineto{\pgfqpoint{3.167465in}{0.612841in}}%
\pgfpathlineto{\pgfqpoint{3.168577in}{0.602651in}}%
\pgfpathlineto{\pgfqpoint{3.169689in}{0.624084in}}%
\pgfpathlineto{\pgfqpoint{3.170245in}{0.603449in}}%
\pgfpathlineto{\pgfqpoint{3.170801in}{0.618547in}}%
\pgfpathlineto{\pgfqpoint{3.171912in}{0.605173in}}%
\pgfpathlineto{\pgfqpoint{3.172468in}{0.628698in}}%
\pgfpathlineto{\pgfqpoint{3.173024in}{0.604328in}}%
\pgfpathlineto{\pgfqpoint{3.173580in}{0.610006in}}%
\pgfpathlineto{\pgfqpoint{3.174136in}{0.609329in}}%
\pgfpathlineto{\pgfqpoint{3.176359in}{0.624259in}}%
\pgfpathlineto{\pgfqpoint{3.176915in}{0.613432in}}%
\pgfpathlineto{\pgfqpoint{3.177471in}{0.614115in}}%
\pgfpathlineto{\pgfqpoint{3.178027in}{0.616942in}}%
\pgfpathlineto{\pgfqpoint{3.178583in}{0.604837in}}%
\pgfpathlineto{\pgfqpoint{3.179138in}{0.618536in}}%
\pgfpathlineto{\pgfqpoint{3.179694in}{0.616536in}}%
\pgfpathlineto{\pgfqpoint{3.180250in}{0.611930in}}%
\pgfpathlineto{\pgfqpoint{3.180806in}{0.621807in}}%
\pgfpathlineto{\pgfqpoint{3.181918in}{0.603585in}}%
\pgfpathlineto{\pgfqpoint{3.183585in}{0.618975in}}%
\pgfpathlineto{\pgfqpoint{3.184141in}{0.601812in}}%
\pgfpathlineto{\pgfqpoint{3.184697in}{0.617789in}}%
\pgfpathlineto{\pgfqpoint{3.186365in}{0.606286in}}%
\pgfpathlineto{\pgfqpoint{3.186921in}{0.623456in}}%
\pgfpathlineto{\pgfqpoint{3.187476in}{0.611651in}}%
\pgfpathlineto{\pgfqpoint{3.188032in}{0.606720in}}%
\pgfpathlineto{\pgfqpoint{3.190256in}{0.647642in}}%
\pgfpathlineto{\pgfqpoint{3.190812in}{0.641821in}}%
\pgfpathlineto{\pgfqpoint{3.192479in}{0.726435in}}%
\pgfpathlineto{\pgfqpoint{3.194703in}{0.613225in}}%
\pgfpathlineto{\pgfqpoint{3.195814in}{0.694092in}}%
\pgfpathlineto{\pgfqpoint{3.196926in}{0.635154in}}%
\pgfpathlineto{\pgfqpoint{3.198038in}{0.690906in}}%
\pgfpathlineto{\pgfqpoint{3.198594in}{0.660894in}}%
\pgfpathlineto{\pgfqpoint{3.200261in}{0.782399in}}%
\pgfpathlineto{\pgfqpoint{3.201373in}{0.633489in}}%
\pgfpathlineto{\pgfqpoint{3.202485in}{0.760035in}}%
\pgfpathlineto{\pgfqpoint{3.204152in}{0.652649in}}%
\pgfpathlineto{\pgfqpoint{3.204708in}{0.797022in}}%
\pgfpathlineto{\pgfqpoint{3.205264in}{0.672693in}}%
\pgfpathlineto{\pgfqpoint{3.205820in}{0.674340in}}%
\pgfpathlineto{\pgfqpoint{3.207487in}{0.608114in}}%
\pgfpathlineto{\pgfqpoint{3.208043in}{0.662390in}}%
\pgfpathlineto{\pgfqpoint{3.208599in}{0.625412in}}%
\pgfpathlineto{\pgfqpoint{3.209155in}{0.636125in}}%
\pgfpathlineto{\pgfqpoint{3.210823in}{0.608353in}}%
\pgfpathlineto{\pgfqpoint{3.211934in}{0.623651in}}%
\pgfpathlineto{\pgfqpoint{3.212490in}{0.607186in}}%
\pgfpathlineto{\pgfqpoint{3.213046in}{0.619479in}}%
\pgfpathlineto{\pgfqpoint{3.213602in}{0.608694in}}%
\pgfpathlineto{\pgfqpoint{3.214714in}{0.642527in}}%
\pgfpathlineto{\pgfqpoint{3.215269in}{0.634923in}}%
\pgfpathlineto{\pgfqpoint{3.215825in}{0.637192in}}%
\pgfpathlineto{\pgfqpoint{3.217493in}{0.617455in}}%
\pgfpathlineto{\pgfqpoint{3.218049in}{0.628139in}}%
\pgfpathlineto{\pgfqpoint{3.219716in}{0.604670in}}%
\pgfpathlineto{\pgfqpoint{3.221384in}{0.612557in}}%
\pgfpathlineto{\pgfqpoint{3.221940in}{0.608619in}}%
\pgfpathlineto{\pgfqpoint{3.222496in}{0.620093in}}%
\pgfpathlineto{\pgfqpoint{3.223052in}{0.616259in}}%
\pgfpathlineto{\pgfqpoint{3.225275in}{0.604624in}}%
\pgfpathlineto{\pgfqpoint{3.225831in}{0.623668in}}%
\pgfpathlineto{\pgfqpoint{3.226387in}{0.616447in}}%
\pgfpathlineto{\pgfqpoint{3.226943in}{0.616338in}}%
\pgfpathlineto{\pgfqpoint{3.227498in}{0.620127in}}%
\pgfpathlineto{\pgfqpoint{3.228054in}{0.607817in}}%
\pgfpathlineto{\pgfqpoint{3.228610in}{0.616530in}}%
\pgfpathlineto{\pgfqpoint{3.229722in}{0.619614in}}%
\pgfpathlineto{\pgfqpoint{3.230834in}{0.636588in}}%
\pgfpathlineto{\pgfqpoint{3.232501in}{0.609064in}}%
\pgfpathlineto{\pgfqpoint{3.233057in}{0.629778in}}%
\pgfpathlineto{\pgfqpoint{3.233613in}{0.607175in}}%
\pgfpathlineto{\pgfqpoint{3.234169in}{0.609705in}}%
\pgfpathlineto{\pgfqpoint{3.234725in}{0.616706in}}%
\pgfpathlineto{\pgfqpoint{3.235280in}{0.612211in}}%
\pgfpathlineto{\pgfqpoint{3.235836in}{0.613244in}}%
\pgfpathlineto{\pgfqpoint{3.236392in}{0.606116in}}%
\pgfpathlineto{\pgfqpoint{3.236948in}{0.616970in}}%
\pgfpathlineto{\pgfqpoint{3.237504in}{0.607012in}}%
\pgfpathlineto{\pgfqpoint{3.238616in}{0.615758in}}%
\pgfpathlineto{\pgfqpoint{3.239172in}{0.604547in}}%
\pgfpathlineto{\pgfqpoint{3.239727in}{0.632853in}}%
\pgfpathlineto{\pgfqpoint{3.240283in}{0.617520in}}%
\pgfpathlineto{\pgfqpoint{3.240839in}{0.604384in}}%
\pgfpathlineto{\pgfqpoint{3.241395in}{0.620151in}}%
\pgfpathlineto{\pgfqpoint{3.241951in}{0.608546in}}%
\pgfpathlineto{\pgfqpoint{3.242507in}{0.602604in}}%
\pgfpathlineto{\pgfqpoint{3.243063in}{0.604137in}}%
\pgfpathlineto{\pgfqpoint{3.243618in}{0.620729in}}%
\pgfpathlineto{\pgfqpoint{3.244174in}{0.614401in}}%
\pgfpathlineto{\pgfqpoint{3.244730in}{0.619209in}}%
\pgfpathlineto{\pgfqpoint{3.245286in}{0.605055in}}%
\pgfpathlineto{\pgfqpoint{3.246954in}{0.623726in}}%
\pgfpathlineto{\pgfqpoint{3.247509in}{0.630113in}}%
\pgfpathlineto{\pgfqpoint{3.248065in}{0.621286in}}%
\pgfpathlineto{\pgfqpoint{3.249733in}{0.639894in}}%
\pgfpathlineto{\pgfqpoint{3.250289in}{0.634334in}}%
\pgfpathlineto{\pgfqpoint{3.252512in}{0.729660in}}%
\pgfpathlineto{\pgfqpoint{3.254180in}{0.636102in}}%
\pgfpathlineto{\pgfqpoint{3.254736in}{0.647112in}}%
\pgfpathlineto{\pgfqpoint{3.256403in}{0.698541in}}%
\pgfpathlineto{\pgfqpoint{3.256959in}{0.676493in}}%
\pgfpathlineto{\pgfqpoint{3.257515in}{0.735463in}}%
\pgfpathlineto{\pgfqpoint{3.258071in}{0.627646in}}%
\pgfpathlineto{\pgfqpoint{3.258627in}{0.689885in}}%
\pgfpathlineto{\pgfqpoint{3.259183in}{0.716088in}}%
\pgfpathlineto{\pgfqpoint{3.259738in}{0.872308in}}%
\pgfpathlineto{\pgfqpoint{3.261406in}{0.650100in}}%
\pgfpathlineto{\pgfqpoint{3.261962in}{0.745259in}}%
\pgfpathlineto{\pgfqpoint{3.263629in}{0.614850in}}%
\pgfpathlineto{\pgfqpoint{3.264185in}{0.647270in}}%
\pgfpathlineto{\pgfqpoint{3.264741in}{0.604230in}}%
\pgfpathlineto{\pgfqpoint{3.265297in}{0.670562in}}%
\pgfpathlineto{\pgfqpoint{3.265853in}{0.618898in}}%
\pgfpathlineto{\pgfqpoint{3.266409in}{0.624490in}}%
\pgfpathlineto{\pgfqpoint{3.268076in}{0.609873in}}%
\pgfpathlineto{\pgfqpoint{3.268632in}{0.609985in}}%
\pgfpathlineto{\pgfqpoint{3.269188in}{0.612934in}}%
\pgfpathlineto{\pgfqpoint{3.269744in}{0.610154in}}%
\pgfpathlineto{\pgfqpoint{3.270300in}{0.636438in}}%
\pgfpathlineto{\pgfqpoint{3.270856in}{0.627060in}}%
\pgfpathlineto{\pgfqpoint{3.271411in}{0.601963in}}%
\pgfpathlineto{\pgfqpoint{3.271967in}{0.624487in}}%
\pgfpathlineto{\pgfqpoint{3.272523in}{0.635872in}}%
\pgfpathlineto{\pgfqpoint{3.274747in}{0.602369in}}%
\pgfpathlineto{\pgfqpoint{3.275303in}{0.625309in}}%
\pgfpathlineto{\pgfqpoint{3.275858in}{0.616758in}}%
\pgfpathlineto{\pgfqpoint{3.277526in}{0.606756in}}%
\pgfpathlineto{\pgfqpoint{3.278082in}{0.622866in}}%
\pgfpathlineto{\pgfqpoint{3.278638in}{0.613638in}}%
\pgfpathlineto{\pgfqpoint{3.279749in}{0.604070in}}%
\pgfpathlineto{\pgfqpoint{3.280305in}{0.630553in}}%
\pgfpathlineto{\pgfqpoint{3.280861in}{0.612094in}}%
\pgfpathlineto{\pgfqpoint{3.281417in}{0.628512in}}%
\pgfpathlineto{\pgfqpoint{3.281973in}{0.601082in}}%
\pgfpathlineto{\pgfqpoint{3.282529in}{0.616225in}}%
\pgfpathlineto{\pgfqpoint{3.284196in}{0.652164in}}%
\pgfpathlineto{\pgfqpoint{3.284752in}{0.610924in}}%
\pgfpathlineto{\pgfqpoint{3.285308in}{0.648073in}}%
\pgfpathlineto{\pgfqpoint{3.286420in}{0.634580in}}%
\pgfpathlineto{\pgfqpoint{3.287531in}{0.618065in}}%
\pgfpathlineto{\pgfqpoint{3.289199in}{0.644000in}}%
\pgfpathlineto{\pgfqpoint{3.289755in}{0.628966in}}%
\pgfpathlineto{\pgfqpoint{3.290311in}{0.673087in}}%
\pgfpathlineto{\pgfqpoint{3.290867in}{0.631835in}}%
\pgfpathlineto{\pgfqpoint{3.291422in}{0.656454in}}%
\pgfpathlineto{\pgfqpoint{3.291978in}{0.634975in}}%
\pgfpathlineto{\pgfqpoint{3.292534in}{0.632551in}}%
\pgfpathlineto{\pgfqpoint{3.293646in}{0.604178in}}%
\pgfpathlineto{\pgfqpoint{3.295314in}{0.638758in}}%
\pgfpathlineto{\pgfqpoint{3.295869in}{0.612399in}}%
\pgfpathlineto{\pgfqpoint{3.296425in}{0.654399in}}%
\pgfpathlineto{\pgfqpoint{3.296981in}{0.622461in}}%
\pgfpathlineto{\pgfqpoint{3.297537in}{0.626829in}}%
\pgfpathlineto{\pgfqpoint{3.298093in}{0.620789in}}%
\pgfpathlineto{\pgfqpoint{3.298649in}{0.635416in}}%
\pgfpathlineto{\pgfqpoint{3.299205in}{0.613905in}}%
\pgfpathlineto{\pgfqpoint{3.299760in}{0.621178in}}%
\pgfpathlineto{\pgfqpoint{3.300316in}{0.614923in}}%
\pgfpathlineto{\pgfqpoint{3.301984in}{0.632424in}}%
\pgfpathlineto{\pgfqpoint{3.302540in}{0.618482in}}%
\pgfpathlineto{\pgfqpoint{3.303096in}{0.644620in}}%
\pgfpathlineto{\pgfqpoint{3.303651in}{0.625695in}}%
\pgfpathlineto{\pgfqpoint{3.304207in}{0.616593in}}%
\pgfpathlineto{\pgfqpoint{3.304763in}{0.632628in}}%
\pgfpathlineto{\pgfqpoint{3.305319in}{0.613359in}}%
\pgfpathlineto{\pgfqpoint{3.305875in}{0.629167in}}%
\pgfpathlineto{\pgfqpoint{3.306431in}{0.635194in}}%
\pgfpathlineto{\pgfqpoint{3.306987in}{0.629486in}}%
\pgfpathlineto{\pgfqpoint{3.308654in}{0.608745in}}%
\pgfpathlineto{\pgfqpoint{3.309210in}{0.631494in}}%
\pgfpathlineto{\pgfqpoint{3.309766in}{0.627995in}}%
\pgfpathlineto{\pgfqpoint{3.310878in}{0.637541in}}%
\pgfpathlineto{\pgfqpoint{3.311989in}{0.661274in}}%
\pgfpathlineto{\pgfqpoint{3.312545in}{0.649687in}}%
\pgfpathlineto{\pgfqpoint{3.313101in}{0.679839in}}%
\pgfpathlineto{\pgfqpoint{3.313657in}{0.666718in}}%
\pgfpathlineto{\pgfqpoint{3.314769in}{0.658046in}}%
\pgfpathlineto{\pgfqpoint{3.315880in}{0.687658in}}%
\pgfpathlineto{\pgfqpoint{3.316436in}{0.651182in}}%
\pgfpathlineto{\pgfqpoint{3.316992in}{0.809063in}}%
\pgfpathlineto{\pgfqpoint{3.317548in}{0.661839in}}%
\pgfpathlineto{\pgfqpoint{3.318104in}{0.625718in}}%
\pgfpathlineto{\pgfqpoint{3.318660in}{0.656651in}}%
\pgfpathlineto{\pgfqpoint{3.319216in}{0.636589in}}%
\pgfpathlineto{\pgfqpoint{3.319771in}{0.640934in}}%
\pgfpathlineto{\pgfqpoint{3.320883in}{0.665897in}}%
\pgfpathlineto{\pgfqpoint{3.321439in}{0.623879in}}%
\pgfpathlineto{\pgfqpoint{3.321995in}{0.659871in}}%
\pgfpathlineto{\pgfqpoint{3.322551in}{0.636386in}}%
\pgfpathlineto{\pgfqpoint{3.323107in}{0.651470in}}%
\pgfpathlineto{\pgfqpoint{3.323662in}{0.656366in}}%
\pgfpathlineto{\pgfqpoint{3.324218in}{0.652188in}}%
\pgfpathlineto{\pgfqpoint{3.324774in}{0.654078in}}%
\pgfpathlineto{\pgfqpoint{3.325330in}{0.644751in}}%
\pgfpathlineto{\pgfqpoint{3.326998in}{0.681697in}}%
\pgfpathlineto{\pgfqpoint{3.328109in}{0.620894in}}%
\pgfpathlineto{\pgfqpoint{3.328665in}{0.629312in}}%
\pgfpathlineto{\pgfqpoint{3.329221in}{0.607308in}}%
\pgfpathlineto{\pgfqpoint{3.329777in}{0.623523in}}%
\pgfpathlineto{\pgfqpoint{3.331444in}{0.677714in}}%
\pgfpathlineto{\pgfqpoint{3.332556in}{0.641491in}}%
\pgfpathlineto{\pgfqpoint{3.333668in}{0.642744in}}%
\pgfpathlineto{\pgfqpoint{3.334224in}{0.633163in}}%
\pgfpathlineto{\pgfqpoint{3.334780in}{0.636077in}}%
\pgfpathlineto{\pgfqpoint{3.335336in}{0.642357in}}%
\pgfpathlineto{\pgfqpoint{3.335891in}{0.614540in}}%
\pgfpathlineto{\pgfqpoint{3.336447in}{0.664957in}}%
\pgfpathlineto{\pgfqpoint{3.337003in}{0.629392in}}%
\pgfpathlineto{\pgfqpoint{3.337559in}{0.626561in}}%
\pgfpathlineto{\pgfqpoint{3.338115in}{0.605752in}}%
\pgfpathlineto{\pgfqpoint{3.339782in}{0.641812in}}%
\pgfpathlineto{\pgfqpoint{3.341450in}{0.612625in}}%
\pgfpathlineto{\pgfqpoint{3.342562in}{0.668553in}}%
\pgfpathlineto{\pgfqpoint{3.343673in}{0.607262in}}%
\pgfpathlineto{\pgfqpoint{3.345341in}{0.657444in}}%
\pgfpathlineto{\pgfqpoint{3.346453in}{0.629910in}}%
\pgfpathlineto{\pgfqpoint{3.347564in}{0.669238in}}%
\pgfpathlineto{\pgfqpoint{3.348120in}{0.654944in}}%
\pgfpathlineto{\pgfqpoint{3.348676in}{0.611228in}}%
\pgfpathlineto{\pgfqpoint{3.349788in}{0.669259in}}%
\pgfpathlineto{\pgfqpoint{3.350344in}{0.636054in}}%
\pgfpathlineto{\pgfqpoint{3.350900in}{0.638060in}}%
\pgfpathlineto{\pgfqpoint{3.351456in}{0.638072in}}%
\pgfpathlineto{\pgfqpoint{3.352011in}{0.659195in}}%
\pgfpathlineto{\pgfqpoint{3.352567in}{0.652735in}}%
\pgfpathlineto{\pgfqpoint{3.353123in}{0.755481in}}%
\pgfpathlineto{\pgfqpoint{3.353679in}{0.628539in}}%
\pgfpathlineto{\pgfqpoint{3.354235in}{0.709418in}}%
\pgfpathlineto{\pgfqpoint{3.355347in}{0.771067in}}%
\pgfpathlineto{\pgfqpoint{3.357014in}{0.875880in}}%
\pgfpathlineto{\pgfqpoint{3.358682in}{0.719423in}}%
\pgfpathlineto{\pgfqpoint{3.359238in}{0.785782in}}%
\pgfpathlineto{\pgfqpoint{3.359793in}{0.775498in}}%
\pgfpathlineto{\pgfqpoint{3.360349in}{0.747504in}}%
\pgfpathlineto{\pgfqpoint{3.360905in}{0.873234in}}%
\pgfpathlineto{\pgfqpoint{3.361461in}{0.699002in}}%
\pgfpathlineto{\pgfqpoint{3.362017in}{0.884167in}}%
\pgfpathlineto{\pgfqpoint{3.362573in}{0.740813in}}%
\pgfpathlineto{\pgfqpoint{3.363129in}{0.864282in}}%
\pgfpathlineto{\pgfqpoint{3.363684in}{0.689524in}}%
\pgfpathlineto{\pgfqpoint{3.364240in}{0.789158in}}%
\pgfpathlineto{\pgfqpoint{3.365352in}{0.671889in}}%
\pgfpathlineto{\pgfqpoint{3.365908in}{0.766789in}}%
\pgfpathlineto{\pgfqpoint{3.366464in}{0.619290in}}%
\pgfpathlineto{\pgfqpoint{3.367020in}{0.818704in}}%
\pgfpathlineto{\pgfqpoint{3.367575in}{0.694489in}}%
\pgfpathlineto{\pgfqpoint{3.368131in}{0.698432in}}%
\pgfpathlineto{\pgfqpoint{3.368687in}{0.675830in}}%
\pgfpathlineto{\pgfqpoint{3.369243in}{0.699172in}}%
\pgfpathlineto{\pgfqpoint{3.369799in}{0.690637in}}%
\pgfpathlineto{\pgfqpoint{3.370355in}{0.621952in}}%
\pgfpathlineto{\pgfqpoint{3.370911in}{0.695227in}}%
\pgfpathlineto{\pgfqpoint{3.371467in}{0.635451in}}%
\pgfpathlineto{\pgfqpoint{3.372022in}{0.677456in}}%
\pgfpathlineto{\pgfqpoint{3.372578in}{0.804679in}}%
\pgfpathlineto{\pgfqpoint{3.373134in}{0.756287in}}%
\pgfpathlineto{\pgfqpoint{3.373690in}{0.761746in}}%
\pgfpathlineto{\pgfqpoint{3.374802in}{0.960985in}}%
\pgfpathlineto{\pgfqpoint{3.376469in}{0.620669in}}%
\pgfpathlineto{\pgfqpoint{3.378137in}{0.719666in}}%
\pgfpathlineto{\pgfqpoint{3.378693in}{0.608832in}}%
\pgfpathlineto{\pgfqpoint{3.379249in}{0.621986in}}%
\pgfpathlineto{\pgfqpoint{3.380360in}{0.668390in}}%
\pgfpathlineto{\pgfqpoint{3.380916in}{0.630829in}}%
\pgfpathlineto{\pgfqpoint{3.381472in}{0.658869in}}%
\pgfpathlineto{\pgfqpoint{3.382028in}{0.656933in}}%
\pgfpathlineto{\pgfqpoint{3.383140in}{0.690111in}}%
\pgfpathlineto{\pgfqpoint{3.383695in}{0.619640in}}%
\pgfpathlineto{\pgfqpoint{3.384251in}{0.668566in}}%
\pgfpathlineto{\pgfqpoint{3.384807in}{0.650816in}}%
\pgfpathlineto{\pgfqpoint{3.385363in}{0.674007in}}%
\pgfpathlineto{\pgfqpoint{3.385919in}{0.658847in}}%
\pgfpathlineto{\pgfqpoint{3.386475in}{0.620645in}}%
\pgfpathlineto{\pgfqpoint{3.387031in}{0.685492in}}%
\pgfpathlineto{\pgfqpoint{3.387586in}{0.619455in}}%
\pgfpathlineto{\pgfqpoint{3.388142in}{0.745982in}}%
\pgfpathlineto{\pgfqpoint{3.388698in}{0.736691in}}%
\pgfpathlineto{\pgfqpoint{3.389254in}{0.712088in}}%
\pgfpathlineto{\pgfqpoint{3.389810in}{0.751450in}}%
\pgfpathlineto{\pgfqpoint{3.390366in}{0.633631in}}%
\pgfpathlineto{\pgfqpoint{3.390922in}{0.727291in}}%
\pgfpathlineto{\pgfqpoint{3.391478in}{0.710916in}}%
\pgfpathlineto{\pgfqpoint{3.392033in}{0.723663in}}%
\pgfpathlineto{\pgfqpoint{3.392589in}{0.753385in}}%
\pgfpathlineto{\pgfqpoint{3.393145in}{0.716852in}}%
\pgfpathlineto{\pgfqpoint{3.393701in}{0.766756in}}%
\pgfpathlineto{\pgfqpoint{3.394813in}{0.615224in}}%
\pgfpathlineto{\pgfqpoint{3.395369in}{0.686150in}}%
\pgfpathlineto{\pgfqpoint{3.395924in}{0.651206in}}%
\pgfpathlineto{\pgfqpoint{3.397036in}{0.638697in}}%
\pgfpathlineto{\pgfqpoint{3.397592in}{0.647823in}}%
\pgfpathlineto{\pgfqpoint{3.398148in}{0.638316in}}%
\pgfpathlineto{\pgfqpoint{3.399815in}{0.715493in}}%
\pgfpathlineto{\pgfqpoint{3.400371in}{0.680063in}}%
\pgfpathlineto{\pgfqpoint{3.400927in}{0.712637in}}%
\pgfpathlineto{\pgfqpoint{3.401483in}{0.721259in}}%
\pgfpathlineto{\pgfqpoint{3.402595in}{0.654914in}}%
\pgfpathlineto{\pgfqpoint{3.404262in}{0.791215in}}%
\pgfpathlineto{\pgfqpoint{3.404818in}{0.753855in}}%
\pgfpathlineto{\pgfqpoint{3.405374in}{0.925227in}}%
\pgfpathlineto{\pgfqpoint{3.405930in}{0.902594in}}%
\pgfpathlineto{\pgfqpoint{3.407598in}{0.705827in}}%
\pgfpathlineto{\pgfqpoint{3.408709in}{0.617120in}}%
\pgfpathlineto{\pgfqpoint{3.409265in}{0.822642in}}%
\pgfpathlineto{\pgfqpoint{3.409821in}{0.699684in}}%
\pgfpathlineto{\pgfqpoint{3.410377in}{0.709990in}}%
\pgfpathlineto{\pgfqpoint{3.410933in}{0.748957in}}%
\pgfpathlineto{\pgfqpoint{3.411489in}{0.871202in}}%
\pgfpathlineto{\pgfqpoint{3.412600in}{0.697744in}}%
\pgfpathlineto{\pgfqpoint{3.413156in}{0.734637in}}%
\pgfpathlineto{\pgfqpoint{3.413712in}{0.867669in}}%
\pgfpathlineto{\pgfqpoint{3.414268in}{0.695883in}}%
\pgfpathlineto{\pgfqpoint{3.414824in}{0.868239in}}%
\pgfpathlineto{\pgfqpoint{3.415380in}{0.701422in}}%
\pgfpathlineto{\pgfqpoint{3.415935in}{0.633703in}}%
\pgfpathlineto{\pgfqpoint{3.417047in}{0.892781in}}%
\pgfpathlineto{\pgfqpoint{3.418715in}{0.638568in}}%
\pgfpathlineto{\pgfqpoint{3.419271in}{0.758474in}}%
\pgfpathlineto{\pgfqpoint{3.420938in}{0.864793in}}%
\pgfpathlineto{\pgfqpoint{3.421494in}{1.117573in}}%
\pgfpathlineto{\pgfqpoint{3.422606in}{0.732048in}}%
\pgfpathlineto{\pgfqpoint{3.423162in}{1.315576in}}%
\pgfpathlineto{\pgfqpoint{3.423717in}{1.154208in}}%
\pgfpathlineto{\pgfqpoint{3.424273in}{1.114044in}}%
\pgfpathlineto{\pgfqpoint{3.425941in}{0.612408in}}%
\pgfpathlineto{\pgfqpoint{3.427053in}{0.895741in}}%
\pgfpathlineto{\pgfqpoint{3.428164in}{0.622161in}}%
\pgfpathlineto{\pgfqpoint{3.428720in}{0.679911in}}%
\pgfpathlineto{\pgfqpoint{3.429276in}{0.738601in}}%
\pgfpathlineto{\pgfqpoint{3.429832in}{0.629117in}}%
\pgfpathlineto{\pgfqpoint{3.430388in}{0.657388in}}%
\pgfpathlineto{\pgfqpoint{3.430944in}{0.661967in}}%
\pgfpathlineto{\pgfqpoint{3.431500in}{0.623652in}}%
\pgfpathlineto{\pgfqpoint{3.431500in}{0.623652in}}%
\pgfusepath{stroke}%
\end{pgfscope}%
\begin{pgfscope}%
\pgfsetrectcap%
\pgfsetmiterjoin%
\pgfsetlinewidth{0.803000pt}%
\definecolor{currentstroke}{rgb}{0.000000,0.000000,0.000000}%
\pgfsetstrokecolor{currentstroke}%
\pgfsetdash{}{0pt}%
\pgfpathmoveto{\pgfqpoint{0.717889in}{0.564143in}}%
\pgfpathlineto{\pgfqpoint{0.717889in}{1.351359in}}%
\pgfusepath{stroke}%
\end{pgfscope}%
\begin{pgfscope}%
\pgfsetrectcap%
\pgfsetmiterjoin%
\pgfsetlinewidth{0.803000pt}%
\definecolor{currentstroke}{rgb}{0.000000,0.000000,0.000000}%
\pgfsetstrokecolor{currentstroke}%
\pgfsetdash{}{0pt}%
\pgfpathmoveto{\pgfqpoint{6.146222in}{0.564143in}}%
\pgfpathlineto{\pgfqpoint{6.146222in}{1.351359in}}%
\pgfusepath{stroke}%
\end{pgfscope}%
\begin{pgfscope}%
\pgfsetrectcap%
\pgfsetmiterjoin%
\pgfsetlinewidth{0.803000pt}%
\definecolor{currentstroke}{rgb}{0.000000,0.000000,0.000000}%
\pgfsetstrokecolor{currentstroke}%
\pgfsetdash{}{0pt}%
\pgfpathmoveto{\pgfqpoint{0.717889in}{0.564143in}}%
\pgfpathlineto{\pgfqpoint{6.146222in}{0.564143in}}%
\pgfusepath{stroke}%
\end{pgfscope}%
\begin{pgfscope}%
\pgfsetrectcap%
\pgfsetmiterjoin%
\pgfsetlinewidth{0.803000pt}%
\definecolor{currentstroke}{rgb}{0.000000,0.000000,0.000000}%
\pgfsetstrokecolor{currentstroke}%
\pgfsetdash{}{0pt}%
\pgfpathmoveto{\pgfqpoint{0.717889in}{1.351359in}}%
\pgfpathlineto{\pgfqpoint{6.146222in}{1.351359in}}%
\pgfusepath{stroke}%
\end{pgfscope}%
\begin{pgfscope}%
\definecolor{textcolor}{rgb}{0.000000,0.000000,0.000000}%
\pgfsetstrokecolor{textcolor}%
\pgfsetfillcolor{textcolor}%
\pgftext[x=3.432055in,y=1.434692in,,base]{\color{textcolor}\rmfamily\fontsize{12.000000}{14.400000}\selectfont Spectrum of Filtered ECG Signal}%
\end{pgfscope}%
\end{pgfpicture}%
\makeatother%
\endgroup%

    }
    \caption{Frequency response of the frequency-sampled FIR filter (top), frequency-sampled filtered ECG signal (middle), and spectrum of filtered ECG signal (bottom).}
    \label{fig:fir-freq}
\end{figure}

\subsection{Cascaded Pole-Zero Placed IIR Filters}
\begin{figure}[H]
    \centering
    \adjustbox{max width=0.75\textwidth}{
    %% Creator: Matplotlib, PGF backend
%%
%% To include the figure in your LaTeX document, write
%%   \input{<filename>.pgf}
%%
%% Make sure the required packages are loaded in your preamble
%%   \usepackage{pgf}
%%
%% and, on pdftex
%%   \usepackage[utf8]{inputenc}\DeclareUnicodeCharacter{2212}{-}
%%
%% or, on luatex and xetex
%%   \usepackage{unicode-math}
%%
%% Figures using additional raster images can only be included by \input if
%% they are in the same directory as the main LaTeX file. For loading figures
%% from other directories you can use the `import` package
%%   \usepackage{import}
%%
%% and then include the figures with
%%   \import{<path to file>}{<filename>.pgf}
%%
%% Matplotlib used the following preamble
%%
\begingroup%
\makeatletter%
\begin{pgfpicture}%
\pgfpathrectangle{\pgfpointorigin}{\pgfqpoint{6.400000in}{4.800000in}}%
\pgfusepath{use as bounding box, clip}%
\begin{pgfscope}%
\pgfsetbuttcap%
\pgfsetmiterjoin%
\definecolor{currentfill}{rgb}{1.000000,1.000000,1.000000}%
\pgfsetfillcolor{currentfill}%
\pgfsetlinewidth{0.000000pt}%
\definecolor{currentstroke}{rgb}{1.000000,1.000000,1.000000}%
\pgfsetstrokecolor{currentstroke}%
\pgfsetdash{}{0pt}%
\pgfpathmoveto{\pgfqpoint{0.000000in}{0.000000in}}%
\pgfpathlineto{\pgfqpoint{6.400000in}{0.000000in}}%
\pgfpathlineto{\pgfqpoint{6.400000in}{4.800000in}}%
\pgfpathlineto{\pgfqpoint{0.000000in}{4.800000in}}%
\pgfpathclose%
\pgfusepath{fill}%
\end{pgfscope}%
\begin{pgfscope}%
\pgfsetbuttcap%
\pgfsetmiterjoin%
\definecolor{currentfill}{rgb}{1.000000,1.000000,1.000000}%
\pgfsetfillcolor{currentfill}%
\pgfsetlinewidth{0.000000pt}%
\definecolor{currentstroke}{rgb}{0.000000,0.000000,0.000000}%
\pgfsetstrokecolor{currentstroke}%
\pgfsetstrokeopacity{0.000000}%
\pgfsetdash{}{0pt}%
\pgfpathmoveto{\pgfqpoint{0.717889in}{3.664143in}}%
\pgfpathlineto{\pgfqpoint{6.146222in}{3.664143in}}%
\pgfpathlineto{\pgfqpoint{6.146222in}{4.451359in}}%
\pgfpathlineto{\pgfqpoint{0.717889in}{4.451359in}}%
\pgfpathclose%
\pgfusepath{fill}%
\end{pgfscope}%
\begin{pgfscope}%
\pgfsetbuttcap%
\pgfsetroundjoin%
\definecolor{currentfill}{rgb}{0.000000,0.000000,0.000000}%
\pgfsetfillcolor{currentfill}%
\pgfsetlinewidth{0.803000pt}%
\definecolor{currentstroke}{rgb}{0.000000,0.000000,0.000000}%
\pgfsetstrokecolor{currentstroke}%
\pgfsetdash{}{0pt}%
\pgfsys@defobject{currentmarker}{\pgfqpoint{0.000000in}{-0.048611in}}{\pgfqpoint{0.000000in}{0.000000in}}{%
\pgfpathmoveto{\pgfqpoint{0.000000in}{0.000000in}}%
\pgfpathlineto{\pgfqpoint{0.000000in}{-0.048611in}}%
\pgfusepath{stroke,fill}%
}%
\begin{pgfscope}%
\pgfsys@transformshift{0.717889in}{3.664143in}%
\pgfsys@useobject{currentmarker}{}%
\end{pgfscope}%
\end{pgfscope}%
\begin{pgfscope}%
\definecolor{textcolor}{rgb}{0.000000,0.000000,0.000000}%
\pgfsetstrokecolor{textcolor}%
\pgfsetfillcolor{textcolor}%
\pgftext[x=0.717889in,y=3.566921in,,top]{\color{textcolor}\rmfamily\fontsize{10.000000}{12.000000}\selectfont \(\displaystyle {0}\)}%
\end{pgfscope}%
\begin{pgfscope}%
\pgfsetbuttcap%
\pgfsetroundjoin%
\definecolor{currentfill}{rgb}{0.000000,0.000000,0.000000}%
\pgfsetfillcolor{currentfill}%
\pgfsetlinewidth{0.803000pt}%
\definecolor{currentstroke}{rgb}{0.000000,0.000000,0.000000}%
\pgfsetstrokecolor{currentstroke}%
\pgfsetdash{}{0pt}%
\pgfsys@defobject{currentmarker}{\pgfqpoint{0.000000in}{-0.048611in}}{\pgfqpoint{0.000000in}{0.000000in}}{%
\pgfpathmoveto{\pgfqpoint{0.000000in}{0.000000in}}%
\pgfpathlineto{\pgfqpoint{0.000000in}{-0.048611in}}%
\pgfusepath{stroke,fill}%
}%
\begin{pgfscope}%
\pgfsys@transformshift{1.803555in}{3.664143in}%
\pgfsys@useobject{currentmarker}{}%
\end{pgfscope}%
\end{pgfscope}%
\begin{pgfscope}%
\definecolor{textcolor}{rgb}{0.000000,0.000000,0.000000}%
\pgfsetstrokecolor{textcolor}%
\pgfsetfillcolor{textcolor}%
\pgftext[x=1.803555in,y=3.566921in,,top]{\color{textcolor}\rmfamily\fontsize{10.000000}{12.000000}\selectfont \(\displaystyle {20}\)}%
\end{pgfscope}%
\begin{pgfscope}%
\pgfsetbuttcap%
\pgfsetroundjoin%
\definecolor{currentfill}{rgb}{0.000000,0.000000,0.000000}%
\pgfsetfillcolor{currentfill}%
\pgfsetlinewidth{0.803000pt}%
\definecolor{currentstroke}{rgb}{0.000000,0.000000,0.000000}%
\pgfsetstrokecolor{currentstroke}%
\pgfsetdash{}{0pt}%
\pgfsys@defobject{currentmarker}{\pgfqpoint{0.000000in}{-0.048611in}}{\pgfqpoint{0.000000in}{0.000000in}}{%
\pgfpathmoveto{\pgfqpoint{0.000000in}{0.000000in}}%
\pgfpathlineto{\pgfqpoint{0.000000in}{-0.048611in}}%
\pgfusepath{stroke,fill}%
}%
\begin{pgfscope}%
\pgfsys@transformshift{2.889222in}{3.664143in}%
\pgfsys@useobject{currentmarker}{}%
\end{pgfscope}%
\end{pgfscope}%
\begin{pgfscope}%
\definecolor{textcolor}{rgb}{0.000000,0.000000,0.000000}%
\pgfsetstrokecolor{textcolor}%
\pgfsetfillcolor{textcolor}%
\pgftext[x=2.889222in,y=3.566921in,,top]{\color{textcolor}\rmfamily\fontsize{10.000000}{12.000000}\selectfont \(\displaystyle {40}\)}%
\end{pgfscope}%
\begin{pgfscope}%
\pgfsetbuttcap%
\pgfsetroundjoin%
\definecolor{currentfill}{rgb}{0.000000,0.000000,0.000000}%
\pgfsetfillcolor{currentfill}%
\pgfsetlinewidth{0.803000pt}%
\definecolor{currentstroke}{rgb}{0.000000,0.000000,0.000000}%
\pgfsetstrokecolor{currentstroke}%
\pgfsetdash{}{0pt}%
\pgfsys@defobject{currentmarker}{\pgfqpoint{0.000000in}{-0.048611in}}{\pgfqpoint{0.000000in}{0.000000in}}{%
\pgfpathmoveto{\pgfqpoint{0.000000in}{0.000000in}}%
\pgfpathlineto{\pgfqpoint{0.000000in}{-0.048611in}}%
\pgfusepath{stroke,fill}%
}%
\begin{pgfscope}%
\pgfsys@transformshift{3.974889in}{3.664143in}%
\pgfsys@useobject{currentmarker}{}%
\end{pgfscope}%
\end{pgfscope}%
\begin{pgfscope}%
\definecolor{textcolor}{rgb}{0.000000,0.000000,0.000000}%
\pgfsetstrokecolor{textcolor}%
\pgfsetfillcolor{textcolor}%
\pgftext[x=3.974889in,y=3.566921in,,top]{\color{textcolor}\rmfamily\fontsize{10.000000}{12.000000}\selectfont \(\displaystyle {60}\)}%
\end{pgfscope}%
\begin{pgfscope}%
\pgfsetbuttcap%
\pgfsetroundjoin%
\definecolor{currentfill}{rgb}{0.000000,0.000000,0.000000}%
\pgfsetfillcolor{currentfill}%
\pgfsetlinewidth{0.803000pt}%
\definecolor{currentstroke}{rgb}{0.000000,0.000000,0.000000}%
\pgfsetstrokecolor{currentstroke}%
\pgfsetdash{}{0pt}%
\pgfsys@defobject{currentmarker}{\pgfqpoint{0.000000in}{-0.048611in}}{\pgfqpoint{0.000000in}{0.000000in}}{%
\pgfpathmoveto{\pgfqpoint{0.000000in}{0.000000in}}%
\pgfpathlineto{\pgfqpoint{0.000000in}{-0.048611in}}%
\pgfusepath{stroke,fill}%
}%
\begin{pgfscope}%
\pgfsys@transformshift{5.060555in}{3.664143in}%
\pgfsys@useobject{currentmarker}{}%
\end{pgfscope}%
\end{pgfscope}%
\begin{pgfscope}%
\definecolor{textcolor}{rgb}{0.000000,0.000000,0.000000}%
\pgfsetstrokecolor{textcolor}%
\pgfsetfillcolor{textcolor}%
\pgftext[x=5.060555in,y=3.566921in,,top]{\color{textcolor}\rmfamily\fontsize{10.000000}{12.000000}\selectfont \(\displaystyle {80}\)}%
\end{pgfscope}%
\begin{pgfscope}%
\pgfsetbuttcap%
\pgfsetroundjoin%
\definecolor{currentfill}{rgb}{0.000000,0.000000,0.000000}%
\pgfsetfillcolor{currentfill}%
\pgfsetlinewidth{0.803000pt}%
\definecolor{currentstroke}{rgb}{0.000000,0.000000,0.000000}%
\pgfsetstrokecolor{currentstroke}%
\pgfsetdash{}{0pt}%
\pgfsys@defobject{currentmarker}{\pgfqpoint{0.000000in}{-0.048611in}}{\pgfqpoint{0.000000in}{0.000000in}}{%
\pgfpathmoveto{\pgfqpoint{0.000000in}{0.000000in}}%
\pgfpathlineto{\pgfqpoint{0.000000in}{-0.048611in}}%
\pgfusepath{stroke,fill}%
}%
\begin{pgfscope}%
\pgfsys@transformshift{6.146222in}{3.664143in}%
\pgfsys@useobject{currentmarker}{}%
\end{pgfscope}%
\end{pgfscope}%
\begin{pgfscope}%
\definecolor{textcolor}{rgb}{0.000000,0.000000,0.000000}%
\pgfsetstrokecolor{textcolor}%
\pgfsetfillcolor{textcolor}%
\pgftext[x=6.146222in,y=3.566921in,,top]{\color{textcolor}\rmfamily\fontsize{10.000000}{12.000000}\selectfont \(\displaystyle {100}\)}%
\end{pgfscope}%
\begin{pgfscope}%
\definecolor{textcolor}{rgb}{0.000000,0.000000,0.000000}%
\pgfsetstrokecolor{textcolor}%
\pgfsetfillcolor{textcolor}%
\pgftext[x=3.432055in,y=3.387909in,,top]{\color{textcolor}\rmfamily\fontsize{10.000000}{12.000000}\selectfont Frequency (Hz)}%
\end{pgfscope}%
\begin{pgfscope}%
\pgfsetbuttcap%
\pgfsetroundjoin%
\definecolor{currentfill}{rgb}{0.000000,0.000000,0.000000}%
\pgfsetfillcolor{currentfill}%
\pgfsetlinewidth{0.803000pt}%
\definecolor{currentstroke}{rgb}{0.000000,0.000000,0.000000}%
\pgfsetstrokecolor{currentstroke}%
\pgfsetdash{}{0pt}%
\pgfsys@defobject{currentmarker}{\pgfqpoint{-0.048611in}{0.000000in}}{\pgfqpoint{0.000000in}{0.000000in}}{%
\pgfpathmoveto{\pgfqpoint{0.000000in}{0.000000in}}%
\pgfpathlineto{\pgfqpoint{-0.048611in}{0.000000in}}%
\pgfusepath{stroke,fill}%
}%
\begin{pgfscope}%
\pgfsys@transformshift{0.717889in}{4.028149in}%
\pgfsys@useobject{currentmarker}{}%
\end{pgfscope}%
\end{pgfscope}%
\begin{pgfscope}%
\definecolor{textcolor}{rgb}{0.000000,0.000000,0.000000}%
\pgfsetstrokecolor{textcolor}%
\pgfsetfillcolor{textcolor}%
\pgftext[x=0.373752in, y=3.979924in, left, base]{\color{textcolor}\rmfamily\fontsize{10.000000}{12.000000}\selectfont \(\displaystyle {-10}\)}%
\end{pgfscope}%
\begin{pgfscope}%
\pgfsetbuttcap%
\pgfsetroundjoin%
\definecolor{currentfill}{rgb}{0.000000,0.000000,0.000000}%
\pgfsetfillcolor{currentfill}%
\pgfsetlinewidth{0.803000pt}%
\definecolor{currentstroke}{rgb}{0.000000,0.000000,0.000000}%
\pgfsetstrokecolor{currentstroke}%
\pgfsetdash{}{0pt}%
\pgfsys@defobject{currentmarker}{\pgfqpoint{-0.048611in}{0.000000in}}{\pgfqpoint{0.000000in}{0.000000in}}{%
\pgfpathmoveto{\pgfqpoint{0.000000in}{0.000000in}}%
\pgfpathlineto{\pgfqpoint{-0.048611in}{0.000000in}}%
\pgfusepath{stroke,fill}%
}%
\begin{pgfscope}%
\pgfsys@transformshift{0.717889in}{4.413050in}%
\pgfsys@useobject{currentmarker}{}%
\end{pgfscope}%
\end{pgfscope}%
\begin{pgfscope}%
\definecolor{textcolor}{rgb}{0.000000,0.000000,0.000000}%
\pgfsetstrokecolor{textcolor}%
\pgfsetfillcolor{textcolor}%
\pgftext[x=0.551222in, y=4.364824in, left, base]{\color{textcolor}\rmfamily\fontsize{10.000000}{12.000000}\selectfont \(\displaystyle {0}\)}%
\end{pgfscope}%
\begin{pgfscope}%
\definecolor{textcolor}{rgb}{0.000000,0.000000,0.000000}%
\pgfsetstrokecolor{textcolor}%
\pgfsetfillcolor{textcolor}%
\pgftext[x=0.318197in,y=4.057751in,,bottom,rotate=90.000000]{\color{textcolor}\rmfamily\fontsize{10.000000}{12.000000}\selectfont Magnitude (dB)}%
\end{pgfscope}%
\begin{pgfscope}%
\pgfpathrectangle{\pgfqpoint{0.717889in}{3.664143in}}{\pgfqpoint{5.428334in}{0.787215in}}%
\pgfusepath{clip}%
\pgfsetrectcap%
\pgfsetroundjoin%
\pgfsetlinewidth{1.505625pt}%
\definecolor{currentstroke}{rgb}{0.121569,0.466667,0.705882}%
\pgfsetstrokecolor{currentstroke}%
\pgfsetdash{}{0pt}%
\pgfpathmoveto{\pgfqpoint{0.717889in}{4.413009in}}%
\pgfpathlineto{\pgfqpoint{1.369289in}{4.411910in}}%
\pgfpathlineto{\pgfqpoint{1.694989in}{4.409494in}}%
\pgfpathlineto{\pgfqpoint{1.857839in}{4.406652in}}%
\pgfpathlineto{\pgfqpoint{1.966405in}{4.403176in}}%
\pgfpathlineto{\pgfqpoint{2.074972in}{4.396811in}}%
\pgfpathlineto{\pgfqpoint{2.129255in}{4.391457in}}%
\pgfpathlineto{\pgfqpoint{2.183539in}{4.383330in}}%
\pgfpathlineto{\pgfqpoint{2.237822in}{4.370224in}}%
\pgfpathlineto{\pgfqpoint{2.292105in}{4.347365in}}%
\pgfpathlineto{\pgfqpoint{2.346389in}{4.303211in}}%
\pgfpathlineto{\pgfqpoint{2.400672in}{4.205316in}}%
\pgfpathlineto{\pgfqpoint{2.454955in}{3.924347in}}%
\pgfpathlineto{\pgfqpoint{2.509239in}{3.795656in}}%
\pgfpathlineto{\pgfqpoint{2.563522in}{4.172847in}}%
\pgfpathlineto{\pgfqpoint{2.617805in}{4.289868in}}%
\pgfpathlineto{\pgfqpoint{2.672089in}{4.340722in}}%
\pgfpathlineto{\pgfqpoint{2.726372in}{4.366320in}}%
\pgfpathlineto{\pgfqpoint{2.780655in}{4.380633in}}%
\pgfpathlineto{\pgfqpoint{2.834939in}{4.389291in}}%
\pgfpathlineto{\pgfqpoint{2.889222in}{4.394842in}}%
\pgfpathlineto{\pgfqpoint{2.943505in}{4.398552in}}%
\pgfpathlineto{\pgfqpoint{3.052072in}{4.402866in}}%
\pgfpathlineto{\pgfqpoint{3.160639in}{4.404884in}}%
\pgfpathlineto{\pgfqpoint{3.323489in}{4.405441in}}%
\pgfpathlineto{\pgfqpoint{3.486339in}{4.403302in}}%
\pgfpathlineto{\pgfqpoint{3.594905in}{4.399445in}}%
\pgfpathlineto{\pgfqpoint{3.649189in}{4.396141in}}%
\pgfpathlineto{\pgfqpoint{3.703472in}{4.391241in}}%
\pgfpathlineto{\pgfqpoint{3.757755in}{4.383701in}}%
\pgfpathlineto{\pgfqpoint{3.812039in}{4.371474in}}%
\pgfpathlineto{\pgfqpoint{3.866322in}{4.350168in}}%
\pgfpathlineto{\pgfqpoint{3.920605in}{4.309291in}}%
\pgfpathlineto{\pgfqpoint{3.974889in}{4.219890in}}%
\pgfpathlineto{\pgfqpoint{4.029172in}{3.973762in}}%
\pgfpathlineto{\pgfqpoint{4.083455in}{3.699926in}}%
\pgfpathlineto{\pgfqpoint{4.137739in}{4.153510in}}%
\pgfpathlineto{\pgfqpoint{4.192022in}{4.282506in}}%
\pgfpathlineto{\pgfqpoint{4.246305in}{4.337617in}}%
\pgfpathlineto{\pgfqpoint{4.300589in}{4.365124in}}%
\pgfpathlineto{\pgfqpoint{4.354872in}{4.380464in}}%
\pgfpathlineto{\pgfqpoint{4.409155in}{4.389778in}}%
\pgfpathlineto{\pgfqpoint{4.463439in}{4.395818in}}%
\pgfpathlineto{\pgfqpoint{4.517722in}{4.399944in}}%
\pgfpathlineto{\pgfqpoint{4.626289in}{4.405047in}}%
\pgfpathlineto{\pgfqpoint{4.789139in}{4.408971in}}%
\pgfpathlineto{\pgfqpoint{5.006272in}{4.411465in}}%
\pgfpathlineto{\pgfqpoint{5.386255in}{4.413322in}}%
\pgfpathlineto{\pgfqpoint{6.156222in}{4.414558in}}%
\pgfpathlineto{\pgfqpoint{6.156222in}{4.414558in}}%
\pgfusepath{stroke}%
\end{pgfscope}%
\begin{pgfscope}%
\pgfsetrectcap%
\pgfsetmiterjoin%
\pgfsetlinewidth{0.803000pt}%
\definecolor{currentstroke}{rgb}{0.000000,0.000000,0.000000}%
\pgfsetstrokecolor{currentstroke}%
\pgfsetdash{}{0pt}%
\pgfpathmoveto{\pgfqpoint{0.717889in}{3.664143in}}%
\pgfpathlineto{\pgfqpoint{0.717889in}{4.451359in}}%
\pgfusepath{stroke}%
\end{pgfscope}%
\begin{pgfscope}%
\pgfsetrectcap%
\pgfsetmiterjoin%
\pgfsetlinewidth{0.803000pt}%
\definecolor{currentstroke}{rgb}{0.000000,0.000000,0.000000}%
\pgfsetstrokecolor{currentstroke}%
\pgfsetdash{}{0pt}%
\pgfpathmoveto{\pgfqpoint{6.146222in}{3.664143in}}%
\pgfpathlineto{\pgfqpoint{6.146222in}{4.451359in}}%
\pgfusepath{stroke}%
\end{pgfscope}%
\begin{pgfscope}%
\pgfsetrectcap%
\pgfsetmiterjoin%
\pgfsetlinewidth{0.803000pt}%
\definecolor{currentstroke}{rgb}{0.000000,0.000000,0.000000}%
\pgfsetstrokecolor{currentstroke}%
\pgfsetdash{}{0pt}%
\pgfpathmoveto{\pgfqpoint{0.717889in}{3.664143in}}%
\pgfpathlineto{\pgfqpoint{6.146222in}{3.664143in}}%
\pgfusepath{stroke}%
\end{pgfscope}%
\begin{pgfscope}%
\pgfsetrectcap%
\pgfsetmiterjoin%
\pgfsetlinewidth{0.803000pt}%
\definecolor{currentstroke}{rgb}{0.000000,0.000000,0.000000}%
\pgfsetstrokecolor{currentstroke}%
\pgfsetdash{}{0pt}%
\pgfpathmoveto{\pgfqpoint{0.717889in}{4.451359in}}%
\pgfpathlineto{\pgfqpoint{6.146222in}{4.451359in}}%
\pgfusepath{stroke}%
\end{pgfscope}%
\begin{pgfscope}%
\definecolor{textcolor}{rgb}{0.000000,0.000000,0.000000}%
\pgfsetstrokecolor{textcolor}%
\pgfsetfillcolor{textcolor}%
\pgftext[x=3.432055in,y=4.534692in,,base]{\color{textcolor}\rmfamily\fontsize{12.000000}{14.400000}\selectfont Magnitude Response of Cascaded IIR Notch Filters}%
\end{pgfscope}%
\begin{pgfscope}%
\pgfsetbuttcap%
\pgfsetmiterjoin%
\definecolor{currentfill}{rgb}{1.000000,1.000000,1.000000}%
\pgfsetfillcolor{currentfill}%
\pgfsetlinewidth{0.000000pt}%
\definecolor{currentstroke}{rgb}{0.000000,0.000000,0.000000}%
\pgfsetstrokecolor{currentstroke}%
\pgfsetstrokeopacity{0.000000}%
\pgfsetdash{}{0pt}%
\pgfpathmoveto{\pgfqpoint{0.717889in}{2.114143in}}%
\pgfpathlineto{\pgfqpoint{6.146222in}{2.114143in}}%
\pgfpathlineto{\pgfqpoint{6.146222in}{2.901359in}}%
\pgfpathlineto{\pgfqpoint{0.717889in}{2.901359in}}%
\pgfpathclose%
\pgfusepath{fill}%
\end{pgfscope}%
\begin{pgfscope}%
\pgfsetbuttcap%
\pgfsetroundjoin%
\definecolor{currentfill}{rgb}{0.000000,0.000000,0.000000}%
\pgfsetfillcolor{currentfill}%
\pgfsetlinewidth{0.803000pt}%
\definecolor{currentstroke}{rgb}{0.000000,0.000000,0.000000}%
\pgfsetstrokecolor{currentstroke}%
\pgfsetdash{}{0pt}%
\pgfsys@defobject{currentmarker}{\pgfqpoint{0.000000in}{-0.048611in}}{\pgfqpoint{0.000000in}{0.000000in}}{%
\pgfpathmoveto{\pgfqpoint{0.000000in}{0.000000in}}%
\pgfpathlineto{\pgfqpoint{0.000000in}{-0.048611in}}%
\pgfusepath{stroke,fill}%
}%
\begin{pgfscope}%
\pgfsys@transformshift{0.964631in}{2.114143in}%
\pgfsys@useobject{currentmarker}{}%
\end{pgfscope}%
\end{pgfscope}%
\begin{pgfscope}%
\definecolor{textcolor}{rgb}{0.000000,0.000000,0.000000}%
\pgfsetstrokecolor{textcolor}%
\pgfsetfillcolor{textcolor}%
\pgftext[x=0.964631in,y=2.016921in,,top]{\color{textcolor}\rmfamily\fontsize{10.000000}{12.000000}\selectfont \(\displaystyle {0}\)}%
\end{pgfscope}%
\begin{pgfscope}%
\pgfsetbuttcap%
\pgfsetroundjoin%
\definecolor{currentfill}{rgb}{0.000000,0.000000,0.000000}%
\pgfsetfillcolor{currentfill}%
\pgfsetlinewidth{0.803000pt}%
\definecolor{currentstroke}{rgb}{0.000000,0.000000,0.000000}%
\pgfsetstrokecolor{currentstroke}%
\pgfsetdash{}{0pt}%
\pgfsys@defobject{currentmarker}{\pgfqpoint{0.000000in}{-0.048611in}}{\pgfqpoint{0.000000in}{0.000000in}}{%
\pgfpathmoveto{\pgfqpoint{0.000000in}{0.000000in}}%
\pgfpathlineto{\pgfqpoint{0.000000in}{-0.048611in}}%
\pgfusepath{stroke,fill}%
}%
\begin{pgfscope}%
\pgfsys@transformshift{1.975308in}{2.114143in}%
\pgfsys@useobject{currentmarker}{}%
\end{pgfscope}%
\end{pgfscope}%
\begin{pgfscope}%
\definecolor{textcolor}{rgb}{0.000000,0.000000,0.000000}%
\pgfsetstrokecolor{textcolor}%
\pgfsetfillcolor{textcolor}%
\pgftext[x=1.975308in,y=2.016921in,,top]{\color{textcolor}\rmfamily\fontsize{10.000000}{12.000000}\selectfont \(\displaystyle {10}\)}%
\end{pgfscope}%
\begin{pgfscope}%
\pgfsetbuttcap%
\pgfsetroundjoin%
\definecolor{currentfill}{rgb}{0.000000,0.000000,0.000000}%
\pgfsetfillcolor{currentfill}%
\pgfsetlinewidth{0.803000pt}%
\definecolor{currentstroke}{rgb}{0.000000,0.000000,0.000000}%
\pgfsetstrokecolor{currentstroke}%
\pgfsetdash{}{0pt}%
\pgfsys@defobject{currentmarker}{\pgfqpoint{0.000000in}{-0.048611in}}{\pgfqpoint{0.000000in}{0.000000in}}{%
\pgfpathmoveto{\pgfqpoint{0.000000in}{0.000000in}}%
\pgfpathlineto{\pgfqpoint{0.000000in}{-0.048611in}}%
\pgfusepath{stroke,fill}%
}%
\begin{pgfscope}%
\pgfsys@transformshift{2.985986in}{2.114143in}%
\pgfsys@useobject{currentmarker}{}%
\end{pgfscope}%
\end{pgfscope}%
\begin{pgfscope}%
\definecolor{textcolor}{rgb}{0.000000,0.000000,0.000000}%
\pgfsetstrokecolor{textcolor}%
\pgfsetfillcolor{textcolor}%
\pgftext[x=2.985986in,y=2.016921in,,top]{\color{textcolor}\rmfamily\fontsize{10.000000}{12.000000}\selectfont \(\displaystyle {20}\)}%
\end{pgfscope}%
\begin{pgfscope}%
\pgfsetbuttcap%
\pgfsetroundjoin%
\definecolor{currentfill}{rgb}{0.000000,0.000000,0.000000}%
\pgfsetfillcolor{currentfill}%
\pgfsetlinewidth{0.803000pt}%
\definecolor{currentstroke}{rgb}{0.000000,0.000000,0.000000}%
\pgfsetstrokecolor{currentstroke}%
\pgfsetdash{}{0pt}%
\pgfsys@defobject{currentmarker}{\pgfqpoint{0.000000in}{-0.048611in}}{\pgfqpoint{0.000000in}{0.000000in}}{%
\pgfpathmoveto{\pgfqpoint{0.000000in}{0.000000in}}%
\pgfpathlineto{\pgfqpoint{0.000000in}{-0.048611in}}%
\pgfusepath{stroke,fill}%
}%
\begin{pgfscope}%
\pgfsys@transformshift{3.996663in}{2.114143in}%
\pgfsys@useobject{currentmarker}{}%
\end{pgfscope}%
\end{pgfscope}%
\begin{pgfscope}%
\definecolor{textcolor}{rgb}{0.000000,0.000000,0.000000}%
\pgfsetstrokecolor{textcolor}%
\pgfsetfillcolor{textcolor}%
\pgftext[x=3.996663in,y=2.016921in,,top]{\color{textcolor}\rmfamily\fontsize{10.000000}{12.000000}\selectfont \(\displaystyle {30}\)}%
\end{pgfscope}%
\begin{pgfscope}%
\pgfsetbuttcap%
\pgfsetroundjoin%
\definecolor{currentfill}{rgb}{0.000000,0.000000,0.000000}%
\pgfsetfillcolor{currentfill}%
\pgfsetlinewidth{0.803000pt}%
\definecolor{currentstroke}{rgb}{0.000000,0.000000,0.000000}%
\pgfsetstrokecolor{currentstroke}%
\pgfsetdash{}{0pt}%
\pgfsys@defobject{currentmarker}{\pgfqpoint{0.000000in}{-0.048611in}}{\pgfqpoint{0.000000in}{0.000000in}}{%
\pgfpathmoveto{\pgfqpoint{0.000000in}{0.000000in}}%
\pgfpathlineto{\pgfqpoint{0.000000in}{-0.048611in}}%
\pgfusepath{stroke,fill}%
}%
\begin{pgfscope}%
\pgfsys@transformshift{5.007340in}{2.114143in}%
\pgfsys@useobject{currentmarker}{}%
\end{pgfscope}%
\end{pgfscope}%
\begin{pgfscope}%
\definecolor{textcolor}{rgb}{0.000000,0.000000,0.000000}%
\pgfsetstrokecolor{textcolor}%
\pgfsetfillcolor{textcolor}%
\pgftext[x=5.007340in,y=2.016921in,,top]{\color{textcolor}\rmfamily\fontsize{10.000000}{12.000000}\selectfont \(\displaystyle {40}\)}%
\end{pgfscope}%
\begin{pgfscope}%
\pgfsetbuttcap%
\pgfsetroundjoin%
\definecolor{currentfill}{rgb}{0.000000,0.000000,0.000000}%
\pgfsetfillcolor{currentfill}%
\pgfsetlinewidth{0.803000pt}%
\definecolor{currentstroke}{rgb}{0.000000,0.000000,0.000000}%
\pgfsetstrokecolor{currentstroke}%
\pgfsetdash{}{0pt}%
\pgfsys@defobject{currentmarker}{\pgfqpoint{0.000000in}{-0.048611in}}{\pgfqpoint{0.000000in}{0.000000in}}{%
\pgfpathmoveto{\pgfqpoint{0.000000in}{0.000000in}}%
\pgfpathlineto{\pgfqpoint{0.000000in}{-0.048611in}}%
\pgfusepath{stroke,fill}%
}%
\begin{pgfscope}%
\pgfsys@transformshift{6.018017in}{2.114143in}%
\pgfsys@useobject{currentmarker}{}%
\end{pgfscope}%
\end{pgfscope}%
\begin{pgfscope}%
\definecolor{textcolor}{rgb}{0.000000,0.000000,0.000000}%
\pgfsetstrokecolor{textcolor}%
\pgfsetfillcolor{textcolor}%
\pgftext[x=6.018017in,y=2.016921in,,top]{\color{textcolor}\rmfamily\fontsize{10.000000}{12.000000}\selectfont \(\displaystyle {50}\)}%
\end{pgfscope}%
\begin{pgfscope}%
\definecolor{textcolor}{rgb}{0.000000,0.000000,0.000000}%
\pgfsetstrokecolor{textcolor}%
\pgfsetfillcolor{textcolor}%
\pgftext[x=3.432055in,y=1.837909in,,top]{\color{textcolor}\rmfamily\fontsize{10.000000}{12.000000}\selectfont Time (s)}%
\end{pgfscope}%
\begin{pgfscope}%
\pgfsetbuttcap%
\pgfsetroundjoin%
\definecolor{currentfill}{rgb}{0.000000,0.000000,0.000000}%
\pgfsetfillcolor{currentfill}%
\pgfsetlinewidth{0.803000pt}%
\definecolor{currentstroke}{rgb}{0.000000,0.000000,0.000000}%
\pgfsetstrokecolor{currentstroke}%
\pgfsetdash{}{0pt}%
\pgfsys@defobject{currentmarker}{\pgfqpoint{-0.048611in}{0.000000in}}{\pgfqpoint{0.000000in}{0.000000in}}{%
\pgfpathmoveto{\pgfqpoint{0.000000in}{0.000000in}}%
\pgfpathlineto{\pgfqpoint{-0.048611in}{0.000000in}}%
\pgfusepath{stroke,fill}%
}%
\begin{pgfscope}%
\pgfsys@transformshift{0.717889in}{2.455976in}%
\pgfsys@useobject{currentmarker}{}%
\end{pgfscope}%
\end{pgfscope}%
\begin{pgfscope}%
\definecolor{textcolor}{rgb}{0.000000,0.000000,0.000000}%
\pgfsetstrokecolor{textcolor}%
\pgfsetfillcolor{textcolor}%
\pgftext[x=0.551222in, y=2.407750in, left, base]{\color{textcolor}\rmfamily\fontsize{10.000000}{12.000000}\selectfont \(\displaystyle {0}\)}%
\end{pgfscope}%
\begin{pgfscope}%
\pgfsetbuttcap%
\pgfsetroundjoin%
\definecolor{currentfill}{rgb}{0.000000,0.000000,0.000000}%
\pgfsetfillcolor{currentfill}%
\pgfsetlinewidth{0.803000pt}%
\definecolor{currentstroke}{rgb}{0.000000,0.000000,0.000000}%
\pgfsetstrokecolor{currentstroke}%
\pgfsetdash{}{0pt}%
\pgfsys@defobject{currentmarker}{\pgfqpoint{-0.048611in}{0.000000in}}{\pgfqpoint{0.000000in}{0.000000in}}{%
\pgfpathmoveto{\pgfqpoint{0.000000in}{0.000000in}}%
\pgfpathlineto{\pgfqpoint{-0.048611in}{0.000000in}}%
\pgfusepath{stroke,fill}%
}%
\begin{pgfscope}%
\pgfsys@transformshift{0.717889in}{2.807449in}%
\pgfsys@useobject{currentmarker}{}%
\end{pgfscope}%
\end{pgfscope}%
\begin{pgfscope}%
\definecolor{textcolor}{rgb}{0.000000,0.000000,0.000000}%
\pgfsetstrokecolor{textcolor}%
\pgfsetfillcolor{textcolor}%
\pgftext[x=0.342888in, y=2.759224in, left, base]{\color{textcolor}\rmfamily\fontsize{10.000000}{12.000000}\selectfont \(\displaystyle {1000}\)}%
\end{pgfscope}%
\begin{pgfscope}%
\definecolor{textcolor}{rgb}{0.000000,0.000000,0.000000}%
\pgfsetstrokecolor{textcolor}%
\pgfsetfillcolor{textcolor}%
\pgftext[x=0.287332in,y=2.507751in,,bottom,rotate=90.000000]{\color{textcolor}\rmfamily\fontsize{10.000000}{12.000000}\selectfont ECG Voltage (\(\displaystyle \mu V\))}%
\end{pgfscope}%
\begin{pgfscope}%
\pgfpathrectangle{\pgfqpoint{0.717889in}{2.114143in}}{\pgfqpoint{5.428334in}{0.787215in}}%
\pgfusepath{clip}%
\pgfsetrectcap%
\pgfsetroundjoin%
\pgfsetlinewidth{1.505625pt}%
\definecolor{currentstroke}{rgb}{0.121569,0.466667,0.705882}%
\pgfsetstrokecolor{currentstroke}%
\pgfsetdash{}{0pt}%
\pgfpathmoveto{\pgfqpoint{0.964631in}{2.612328in}}%
\pgfpathlineto{\pgfqpoint{0.964828in}{2.627058in}}%
\pgfpathlineto{\pgfqpoint{0.965125in}{2.556754in}}%
\pgfpathlineto{\pgfqpoint{0.965618in}{2.433557in}}%
\pgfpathlineto{\pgfqpoint{0.966112in}{2.556941in}}%
\pgfpathlineto{\pgfqpoint{0.966408in}{2.601298in}}%
\pgfpathlineto{\pgfqpoint{0.966901in}{2.513443in}}%
\pgfpathlineto{\pgfqpoint{0.967296in}{2.449279in}}%
\pgfpathlineto{\pgfqpoint{0.967789in}{2.531859in}}%
\pgfpathlineto{\pgfqpoint{0.968086in}{2.559646in}}%
\pgfpathlineto{\pgfqpoint{0.968579in}{2.486916in}}%
\pgfpathlineto{\pgfqpoint{0.968875in}{2.452180in}}%
\pgfpathlineto{\pgfqpoint{0.969467in}{2.517421in}}%
\pgfpathlineto{\pgfqpoint{0.969763in}{2.536299in}}%
\pgfpathlineto{\pgfqpoint{0.970158in}{2.487290in}}%
\pgfpathlineto{\pgfqpoint{0.970553in}{2.456982in}}%
\pgfpathlineto{\pgfqpoint{0.971046in}{2.493891in}}%
\pgfpathlineto{\pgfqpoint{0.971343in}{2.514291in}}%
\pgfpathlineto{\pgfqpoint{0.971836in}{2.475534in}}%
\pgfpathlineto{\pgfqpoint{0.972231in}{2.449139in}}%
\pgfpathlineto{\pgfqpoint{0.972724in}{2.488825in}}%
\pgfpathlineto{\pgfqpoint{0.972823in}{2.491758in}}%
\pgfpathlineto{\pgfqpoint{0.973119in}{2.478072in}}%
\pgfpathlineto{\pgfqpoint{0.973909in}{2.421972in}}%
\pgfpathlineto{\pgfqpoint{0.974600in}{2.444900in}}%
\pgfpathlineto{\pgfqpoint{0.974896in}{2.436159in}}%
\pgfpathlineto{\pgfqpoint{0.975389in}{2.401988in}}%
\pgfpathlineto{\pgfqpoint{0.975981in}{2.429220in}}%
\pgfpathlineto{\pgfqpoint{0.976080in}{2.431190in}}%
\pgfpathlineto{\pgfqpoint{0.976376in}{2.418677in}}%
\pgfpathlineto{\pgfqpoint{0.977265in}{2.406650in}}%
\pgfpathlineto{\pgfqpoint{0.977659in}{2.410674in}}%
\pgfpathlineto{\pgfqpoint{0.978054in}{2.422116in}}%
\pgfpathlineto{\pgfqpoint{0.978350in}{2.403950in}}%
\pgfpathlineto{\pgfqpoint{0.978548in}{2.395541in}}%
\pgfpathlineto{\pgfqpoint{0.979239in}{2.417170in}}%
\pgfpathlineto{\pgfqpoint{0.979436in}{2.419892in}}%
\pgfpathlineto{\pgfqpoint{0.980028in}{2.411628in}}%
\pgfpathlineto{\pgfqpoint{0.980324in}{2.407389in}}%
\pgfpathlineto{\pgfqpoint{0.980620in}{2.414834in}}%
\pgfpathlineto{\pgfqpoint{0.981114in}{2.439186in}}%
\pgfpathlineto{\pgfqpoint{0.981805in}{2.420792in}}%
\pgfpathlineto{\pgfqpoint{0.982989in}{2.433949in}}%
\pgfpathlineto{\pgfqpoint{0.983186in}{2.428949in}}%
\pgfpathlineto{\pgfqpoint{0.983483in}{2.419208in}}%
\pgfpathlineto{\pgfqpoint{0.984173in}{2.428358in}}%
\pgfpathlineto{\pgfqpoint{0.984470in}{2.443944in}}%
\pgfpathlineto{\pgfqpoint{0.985457in}{2.442347in}}%
\pgfpathlineto{\pgfqpoint{0.987727in}{2.473739in}}%
\pgfpathlineto{\pgfqpoint{0.987825in}{2.474191in}}%
\pgfpathlineto{\pgfqpoint{0.988023in}{2.470648in}}%
\pgfpathlineto{\pgfqpoint{0.990490in}{2.409276in}}%
\pgfpathlineto{\pgfqpoint{0.990589in}{2.410930in}}%
\pgfpathlineto{\pgfqpoint{0.990885in}{2.415960in}}%
\pgfpathlineto{\pgfqpoint{0.991378in}{2.407621in}}%
\pgfpathlineto{\pgfqpoint{0.991773in}{2.390510in}}%
\pgfpathlineto{\pgfqpoint{0.992563in}{2.402444in}}%
\pgfpathlineto{\pgfqpoint{0.992859in}{2.410201in}}%
\pgfpathlineto{\pgfqpoint{0.993747in}{2.405239in}}%
\pgfpathlineto{\pgfqpoint{0.994142in}{2.427182in}}%
\pgfpathlineto{\pgfqpoint{0.994636in}{2.404626in}}%
\pgfpathlineto{\pgfqpoint{0.995030in}{2.412956in}}%
\pgfpathlineto{\pgfqpoint{0.996412in}{2.422870in}}%
\pgfpathlineto{\pgfqpoint{0.996610in}{2.419066in}}%
\pgfpathlineto{\pgfqpoint{0.997004in}{2.401916in}}%
\pgfpathlineto{\pgfqpoint{0.997300in}{2.424729in}}%
\pgfpathlineto{\pgfqpoint{0.998485in}{2.669673in}}%
\pgfpathlineto{\pgfqpoint{0.999570in}{2.632686in}}%
\pgfpathlineto{\pgfqpoint{1.000557in}{2.408579in}}%
\pgfpathlineto{\pgfqpoint{1.002038in}{2.429023in}}%
\pgfpathlineto{\pgfqpoint{1.003716in}{2.511655in}}%
\pgfpathlineto{\pgfqpoint{1.004111in}{2.474086in}}%
\pgfpathlineto{\pgfqpoint{1.004900in}{2.423647in}}%
\pgfpathlineto{\pgfqpoint{1.005394in}{2.447633in}}%
\pgfpathlineto{\pgfqpoint{1.006381in}{2.477361in}}%
\pgfpathlineto{\pgfqpoint{1.006677in}{2.491947in}}%
\pgfpathlineto{\pgfqpoint{1.007269in}{2.469963in}}%
\pgfpathlineto{\pgfqpoint{1.007861in}{2.426551in}}%
\pgfpathlineto{\pgfqpoint{1.008552in}{2.444834in}}%
\pgfpathlineto{\pgfqpoint{1.009934in}{2.468412in}}%
\pgfpathlineto{\pgfqpoint{1.010131in}{2.475090in}}%
\pgfpathlineto{\pgfqpoint{1.010625in}{2.451621in}}%
\pgfpathlineto{\pgfqpoint{1.011020in}{2.434941in}}%
\pgfpathlineto{\pgfqpoint{1.011612in}{2.456865in}}%
\pgfpathlineto{\pgfqpoint{1.013290in}{2.478978in}}%
\pgfpathlineto{\pgfqpoint{1.013487in}{2.472902in}}%
\pgfpathlineto{\pgfqpoint{1.014573in}{2.458126in}}%
\pgfpathlineto{\pgfqpoint{1.014770in}{2.461524in}}%
\pgfpathlineto{\pgfqpoint{1.015560in}{2.482823in}}%
\pgfpathlineto{\pgfqpoint{1.016547in}{2.475419in}}%
\pgfpathlineto{\pgfqpoint{1.019212in}{2.388800in}}%
\pgfpathlineto{\pgfqpoint{1.019409in}{2.378146in}}%
\pgfpathlineto{\pgfqpoint{1.019902in}{2.390824in}}%
\pgfpathlineto{\pgfqpoint{1.020297in}{2.388922in}}%
\pgfpathlineto{\pgfqpoint{1.020593in}{2.395257in}}%
\pgfpathlineto{\pgfqpoint{1.022173in}{2.461643in}}%
\pgfpathlineto{\pgfqpoint{1.022765in}{2.444898in}}%
\pgfpathlineto{\pgfqpoint{1.025627in}{2.395352in}}%
\pgfpathlineto{\pgfqpoint{1.025824in}{2.400157in}}%
\pgfpathlineto{\pgfqpoint{1.026022in}{2.404404in}}%
\pgfpathlineto{\pgfqpoint{1.026515in}{2.393685in}}%
\pgfpathlineto{\pgfqpoint{1.026811in}{2.396969in}}%
\pgfpathlineto{\pgfqpoint{1.028489in}{2.378578in}}%
\pgfpathlineto{\pgfqpoint{1.028884in}{2.387453in}}%
\pgfpathlineto{\pgfqpoint{1.029871in}{2.396622in}}%
\pgfpathlineto{\pgfqpoint{1.029476in}{2.387429in}}%
\pgfpathlineto{\pgfqpoint{1.030068in}{2.391783in}}%
\pgfpathlineto{\pgfqpoint{1.030463in}{2.379660in}}%
\pgfpathlineto{\pgfqpoint{1.031154in}{2.390341in}}%
\pgfpathlineto{\pgfqpoint{1.031253in}{2.390251in}}%
\pgfpathlineto{\pgfqpoint{1.032536in}{2.379374in}}%
\pgfpathlineto{\pgfqpoint{1.032733in}{2.382121in}}%
\pgfpathlineto{\pgfqpoint{1.032931in}{2.387490in}}%
\pgfpathlineto{\pgfqpoint{1.033326in}{2.380558in}}%
\pgfpathlineto{\pgfqpoint{1.033918in}{2.385492in}}%
\pgfpathlineto{\pgfqpoint{1.035102in}{2.410235in}}%
\pgfpathlineto{\pgfqpoint{1.036583in}{2.444112in}}%
\pgfpathlineto{\pgfqpoint{1.037175in}{2.438298in}}%
\pgfpathlineto{\pgfqpoint{1.037866in}{2.457011in}}%
\pgfpathlineto{\pgfqpoint{1.038260in}{2.441234in}}%
\pgfpathlineto{\pgfqpoint{1.040037in}{2.401861in}}%
\pgfpathlineto{\pgfqpoint{1.040234in}{2.407891in}}%
\pgfpathlineto{\pgfqpoint{1.040432in}{2.413799in}}%
\pgfpathlineto{\pgfqpoint{1.041123in}{2.404192in}}%
\pgfpathlineto{\pgfqpoint{1.041320in}{2.399315in}}%
\pgfpathlineto{\pgfqpoint{1.041715in}{2.411546in}}%
\pgfpathlineto{\pgfqpoint{1.042110in}{2.408227in}}%
\pgfpathlineto{\pgfqpoint{1.043097in}{2.416938in}}%
\pgfpathlineto{\pgfqpoint{1.042702in}{2.402722in}}%
\pgfpathlineto{\pgfqpoint{1.043294in}{2.410386in}}%
\pgfpathlineto{\pgfqpoint{1.044380in}{2.395640in}}%
\pgfpathlineto{\pgfqpoint{1.043886in}{2.410763in}}%
\pgfpathlineto{\pgfqpoint{1.044577in}{2.399686in}}%
\pgfpathlineto{\pgfqpoint{1.044873in}{2.409176in}}%
\pgfpathlineto{\pgfqpoint{1.045268in}{2.398534in}}%
\pgfpathlineto{\pgfqpoint{1.045564in}{2.400763in}}%
\pgfpathlineto{\pgfqpoint{1.046058in}{2.371309in}}%
\pgfpathlineto{\pgfqpoint{1.046551in}{2.407325in}}%
\pgfpathlineto{\pgfqpoint{1.047834in}{2.640194in}}%
\pgfpathlineto{\pgfqpoint{1.048723in}{2.618473in}}%
\pgfpathlineto{\pgfqpoint{1.050697in}{2.381556in}}%
\pgfpathlineto{\pgfqpoint{1.051387in}{2.410211in}}%
\pgfpathlineto{\pgfqpoint{1.051684in}{2.408322in}}%
\pgfpathlineto{\pgfqpoint{1.052078in}{2.430134in}}%
\pgfpathlineto{\pgfqpoint{1.052967in}{2.473757in}}%
\pgfpathlineto{\pgfqpoint{1.053263in}{2.448221in}}%
\pgfpathlineto{\pgfqpoint{1.053756in}{2.381001in}}%
\pgfpathlineto{\pgfqpoint{1.054546in}{2.410017in}}%
\pgfpathlineto{\pgfqpoint{1.055829in}{2.456852in}}%
\pgfpathlineto{\pgfqpoint{1.056224in}{2.443665in}}%
\pgfpathlineto{\pgfqpoint{1.056915in}{2.393674in}}%
\pgfpathlineto{\pgfqpoint{1.057704in}{2.417728in}}%
\pgfpathlineto{\pgfqpoint{1.059283in}{2.458827in}}%
\pgfpathlineto{\pgfqpoint{1.059382in}{2.460113in}}%
\pgfpathlineto{\pgfqpoint{1.059579in}{2.449257in}}%
\pgfpathlineto{\pgfqpoint{1.060369in}{2.414648in}}%
\pgfpathlineto{\pgfqpoint{1.060764in}{2.435638in}}%
\pgfpathlineto{\pgfqpoint{1.061948in}{2.457344in}}%
\pgfpathlineto{\pgfqpoint{1.062047in}{2.455444in}}%
\pgfpathlineto{\pgfqpoint{1.063231in}{2.434285in}}%
\pgfpathlineto{\pgfqpoint{1.062738in}{2.464558in}}%
\pgfpathlineto{\pgfqpoint{1.063330in}{2.437205in}}%
\pgfpathlineto{\pgfqpoint{1.064712in}{2.466245in}}%
\pgfpathlineto{\pgfqpoint{1.064810in}{2.463401in}}%
\pgfpathlineto{\pgfqpoint{1.065008in}{2.458238in}}%
\pgfpathlineto{\pgfqpoint{1.065501in}{2.475180in}}%
\pgfpathlineto{\pgfqpoint{1.065600in}{2.473793in}}%
\pgfpathlineto{\pgfqpoint{1.065797in}{2.471926in}}%
\pgfpathlineto{\pgfqpoint{1.066390in}{2.477760in}}%
\pgfpathlineto{\pgfqpoint{1.066883in}{2.485929in}}%
\pgfpathlineto{\pgfqpoint{1.069153in}{2.538855in}}%
\pgfpathlineto{\pgfqpoint{1.069351in}{2.534289in}}%
\pgfpathlineto{\pgfqpoint{1.069548in}{2.529199in}}%
\pgfpathlineto{\pgfqpoint{1.070140in}{2.545784in}}%
\pgfpathlineto{\pgfqpoint{1.071423in}{2.560520in}}%
\pgfpathlineto{\pgfqpoint{1.071029in}{2.543729in}}%
\pgfpathlineto{\pgfqpoint{1.071522in}{2.556104in}}%
\pgfpathlineto{\pgfqpoint{1.072608in}{2.533008in}}%
\pgfpathlineto{\pgfqpoint{1.072904in}{2.533015in}}%
\pgfpathlineto{\pgfqpoint{1.075273in}{2.474478in}}%
\pgfpathlineto{\pgfqpoint{1.076950in}{2.446855in}}%
\pgfpathlineto{\pgfqpoint{1.077148in}{2.450739in}}%
\pgfpathlineto{\pgfqpoint{1.078036in}{2.458949in}}%
\pgfpathlineto{\pgfqpoint{1.078234in}{2.454020in}}%
\pgfpathlineto{\pgfqpoint{1.078431in}{2.447443in}}%
\pgfpathlineto{\pgfqpoint{1.078826in}{2.454755in}}%
\pgfpathlineto{\pgfqpoint{1.079221in}{2.453628in}}%
\pgfpathlineto{\pgfqpoint{1.080800in}{2.471957in}}%
\pgfpathlineto{\pgfqpoint{1.080898in}{2.470501in}}%
\pgfpathlineto{\pgfqpoint{1.081589in}{2.473272in}}%
\pgfpathlineto{\pgfqpoint{1.082478in}{2.450460in}}%
\pgfpathlineto{\pgfqpoint{1.084649in}{2.469538in}}%
\pgfpathlineto{\pgfqpoint{1.085537in}{2.479933in}}%
\pgfpathlineto{\pgfqpoint{1.085044in}{2.467528in}}%
\pgfpathlineto{\pgfqpoint{1.085833in}{2.472908in}}%
\pgfpathlineto{\pgfqpoint{1.088301in}{2.424399in}}%
\pgfpathlineto{\pgfqpoint{1.088597in}{2.434160in}}%
\pgfpathlineto{\pgfqpoint{1.088696in}{2.436104in}}%
\pgfpathlineto{\pgfqpoint{1.089090in}{2.424659in}}%
\pgfpathlineto{\pgfqpoint{1.089584in}{2.417799in}}%
\pgfpathlineto{\pgfqpoint{1.089979in}{2.424814in}}%
\pgfpathlineto{\pgfqpoint{1.090275in}{2.422445in}}%
\pgfpathlineto{\pgfqpoint{1.090571in}{2.427805in}}%
\pgfpathlineto{\pgfqpoint{1.091262in}{2.422963in}}%
\pgfpathlineto{\pgfqpoint{1.092841in}{2.398097in}}%
\pgfpathlineto{\pgfqpoint{1.092940in}{2.398946in}}%
\pgfpathlineto{\pgfqpoint{1.093038in}{2.400015in}}%
\pgfpathlineto{\pgfqpoint{1.093334in}{2.391922in}}%
\pgfpathlineto{\pgfqpoint{1.094914in}{2.314363in}}%
\pgfpathlineto{\pgfqpoint{1.095111in}{2.329839in}}%
\pgfpathlineto{\pgfqpoint{1.097282in}{2.610483in}}%
\pgfpathlineto{\pgfqpoint{1.097677in}{2.567693in}}%
\pgfpathlineto{\pgfqpoint{1.099454in}{2.375930in}}%
\pgfpathlineto{\pgfqpoint{1.101428in}{2.501295in}}%
\pgfpathlineto{\pgfqpoint{1.102020in}{2.469726in}}%
\pgfpathlineto{\pgfqpoint{1.102612in}{2.414528in}}%
\pgfpathlineto{\pgfqpoint{1.103402in}{2.446071in}}%
\pgfpathlineto{\pgfqpoint{1.104981in}{2.518502in}}%
\pgfpathlineto{\pgfqpoint{1.105376in}{2.495500in}}%
\pgfpathlineto{\pgfqpoint{1.105771in}{2.461407in}}%
\pgfpathlineto{\pgfqpoint{1.106461in}{2.493171in}}%
\pgfpathlineto{\pgfqpoint{1.106659in}{2.494648in}}%
\pgfpathlineto{\pgfqpoint{1.107843in}{2.531519in}}%
\pgfpathlineto{\pgfqpoint{1.108633in}{2.519303in}}%
\pgfpathlineto{\pgfqpoint{1.108732in}{2.519319in}}%
\pgfpathlineto{\pgfqpoint{1.109126in}{2.507724in}}%
\pgfpathlineto{\pgfqpoint{1.109422in}{2.525226in}}%
\pgfpathlineto{\pgfqpoint{1.111002in}{2.567787in}}%
\pgfpathlineto{\pgfqpoint{1.111100in}{2.570503in}}%
\pgfpathlineto{\pgfqpoint{1.111692in}{2.559241in}}%
\pgfpathlineto{\pgfqpoint{1.112679in}{2.543845in}}%
\pgfpathlineto{\pgfqpoint{1.112877in}{2.550823in}}%
\pgfpathlineto{\pgfqpoint{1.113173in}{2.564396in}}%
\pgfpathlineto{\pgfqpoint{1.113864in}{2.548941in}}%
\pgfpathlineto{\pgfqpoint{1.114752in}{2.533784in}}%
\pgfpathlineto{\pgfqpoint{1.114357in}{2.549392in}}%
\pgfpathlineto{\pgfqpoint{1.115048in}{2.545828in}}%
\pgfpathlineto{\pgfqpoint{1.115147in}{2.547722in}}%
\pgfpathlineto{\pgfqpoint{1.115344in}{2.534466in}}%
\pgfpathlineto{\pgfqpoint{1.115542in}{2.520080in}}%
\pgfpathlineto{\pgfqpoint{1.115937in}{2.540452in}}%
\pgfpathlineto{\pgfqpoint{1.116529in}{2.525842in}}%
\pgfpathlineto{\pgfqpoint{1.116726in}{2.530348in}}%
\pgfpathlineto{\pgfqpoint{1.117318in}{2.517769in}}%
\pgfpathlineto{\pgfqpoint{1.118305in}{2.505462in}}%
\pgfpathlineto{\pgfqpoint{1.118503in}{2.513497in}}%
\pgfpathlineto{\pgfqpoint{1.118601in}{2.516736in}}%
\pgfpathlineto{\pgfqpoint{1.118996in}{2.496206in}}%
\pgfpathlineto{\pgfqpoint{1.119391in}{2.504873in}}%
\pgfpathlineto{\pgfqpoint{1.121069in}{2.444363in}}%
\pgfpathlineto{\pgfqpoint{1.122747in}{2.403707in}}%
\pgfpathlineto{\pgfqpoint{1.126399in}{2.358268in}}%
\pgfpathlineto{\pgfqpoint{1.126695in}{2.358705in}}%
\pgfpathlineto{\pgfqpoint{1.126892in}{2.356862in}}%
\pgfpathlineto{\pgfqpoint{1.127583in}{2.346099in}}%
\pgfpathlineto{\pgfqpoint{1.128471in}{2.348771in}}%
\pgfpathlineto{\pgfqpoint{1.128669in}{2.351522in}}%
\pgfpathlineto{\pgfqpoint{1.128965in}{2.342166in}}%
\pgfpathlineto{\pgfqpoint{1.129162in}{2.334606in}}%
\pgfpathlineto{\pgfqpoint{1.129656in}{2.351715in}}%
\pgfpathlineto{\pgfqpoint{1.130050in}{2.342804in}}%
\pgfpathlineto{\pgfqpoint{1.130347in}{2.344056in}}%
\pgfpathlineto{\pgfqpoint{1.130445in}{2.342206in}}%
\pgfpathlineto{\pgfqpoint{1.130741in}{2.337126in}}%
\pgfpathlineto{\pgfqpoint{1.131037in}{2.352342in}}%
\pgfpathlineto{\pgfqpoint{1.132123in}{2.380133in}}%
\pgfpathlineto{\pgfqpoint{1.131728in}{2.351900in}}%
\pgfpathlineto{\pgfqpoint{1.132321in}{2.369727in}}%
\pgfpathlineto{\pgfqpoint{1.132518in}{2.363348in}}%
\pgfpathlineto{\pgfqpoint{1.132913in}{2.390998in}}%
\pgfpathlineto{\pgfqpoint{1.133209in}{2.378122in}}%
\pgfpathlineto{\pgfqpoint{1.133406in}{2.382148in}}%
\pgfpathlineto{\pgfqpoint{1.133604in}{2.388879in}}%
\pgfpathlineto{\pgfqpoint{1.134492in}{2.381649in}}%
\pgfpathlineto{\pgfqpoint{1.134591in}{2.380512in}}%
\pgfpathlineto{\pgfqpoint{1.134788in}{2.387676in}}%
\pgfpathlineto{\pgfqpoint{1.135775in}{2.396148in}}%
\pgfpathlineto{\pgfqpoint{1.135380in}{2.384635in}}%
\pgfpathlineto{\pgfqpoint{1.135874in}{2.393554in}}%
\pgfpathlineto{\pgfqpoint{1.136861in}{2.377708in}}%
\pgfpathlineto{\pgfqpoint{1.137058in}{2.383306in}}%
\pgfpathlineto{\pgfqpoint{1.137848in}{2.395933in}}%
\pgfpathlineto{\pgfqpoint{1.138341in}{2.392126in}}%
\pgfpathlineto{\pgfqpoint{1.138736in}{2.401482in}}%
\pgfpathlineto{\pgfqpoint{1.140315in}{2.425190in}}%
\pgfpathlineto{\pgfqpoint{1.140414in}{2.424075in}}%
\pgfpathlineto{\pgfqpoint{1.140513in}{2.423429in}}%
\pgfpathlineto{\pgfqpoint{1.140710in}{2.426420in}}%
\pgfpathlineto{\pgfqpoint{1.141006in}{2.433775in}}%
\pgfpathlineto{\pgfqpoint{1.141500in}{2.422099in}}%
\pgfpathlineto{\pgfqpoint{1.141796in}{2.426302in}}%
\pgfpathlineto{\pgfqpoint{1.142190in}{2.432357in}}%
\pgfpathlineto{\pgfqpoint{1.142684in}{2.424441in}}%
\pgfpathlineto{\pgfqpoint{1.143177in}{2.399097in}}%
\pgfpathlineto{\pgfqpoint{1.143474in}{2.419689in}}%
\pgfpathlineto{\pgfqpoint{1.145151in}{2.684498in}}%
\pgfpathlineto{\pgfqpoint{1.145744in}{2.662964in}}%
\pgfpathlineto{\pgfqpoint{1.146040in}{2.636439in}}%
\pgfpathlineto{\pgfqpoint{1.147816in}{2.415408in}}%
\pgfpathlineto{\pgfqpoint{1.147915in}{2.415837in}}%
\pgfpathlineto{\pgfqpoint{1.149889in}{2.509782in}}%
\pgfpathlineto{\pgfqpoint{1.150284in}{2.488555in}}%
\pgfpathlineto{\pgfqpoint{1.150975in}{2.416119in}}%
\pgfpathlineto{\pgfqpoint{1.151764in}{2.427309in}}%
\pgfpathlineto{\pgfqpoint{1.153047in}{2.471766in}}%
\pgfpathlineto{\pgfqpoint{1.153343in}{2.456560in}}%
\pgfpathlineto{\pgfqpoint{1.154528in}{2.390825in}}%
\pgfpathlineto{\pgfqpoint{1.155021in}{2.410757in}}%
\pgfpathlineto{\pgfqpoint{1.155614in}{2.405194in}}%
\pgfpathlineto{\pgfqpoint{1.156502in}{2.427239in}}%
\pgfpathlineto{\pgfqpoint{1.156601in}{2.427474in}}%
\pgfpathlineto{\pgfqpoint{1.156699in}{2.424910in}}%
\pgfpathlineto{\pgfqpoint{1.157588in}{2.384453in}}%
\pgfpathlineto{\pgfqpoint{1.158081in}{2.408238in}}%
\pgfpathlineto{\pgfqpoint{1.158180in}{2.408926in}}%
\pgfpathlineto{\pgfqpoint{1.158377in}{2.404141in}}%
\pgfpathlineto{\pgfqpoint{1.159167in}{2.399732in}}%
\pgfpathlineto{\pgfqpoint{1.158772in}{2.410148in}}%
\pgfpathlineto{\pgfqpoint{1.159265in}{2.401823in}}%
\pgfpathlineto{\pgfqpoint{1.159561in}{2.413063in}}%
\pgfpathlineto{\pgfqpoint{1.159956in}{2.395881in}}%
\pgfpathlineto{\pgfqpoint{1.160450in}{2.408084in}}%
\pgfpathlineto{\pgfqpoint{1.160746in}{2.402942in}}%
\pgfpathlineto{\pgfqpoint{1.161141in}{2.412091in}}%
\pgfpathlineto{\pgfqpoint{1.162917in}{2.433979in}}%
\pgfpathlineto{\pgfqpoint{1.161733in}{2.405917in}}%
\pgfpathlineto{\pgfqpoint{1.163016in}{2.430732in}}%
\pgfpathlineto{\pgfqpoint{1.163411in}{2.409717in}}%
\pgfpathlineto{\pgfqpoint{1.164102in}{2.423588in}}%
\pgfpathlineto{\pgfqpoint{1.165681in}{2.444658in}}%
\pgfpathlineto{\pgfqpoint{1.165780in}{2.444585in}}%
\pgfpathlineto{\pgfqpoint{1.166076in}{2.447938in}}%
\pgfpathlineto{\pgfqpoint{1.166372in}{2.452692in}}%
\pgfpathlineto{\pgfqpoint{1.166964in}{2.446656in}}%
\pgfpathlineto{\pgfqpoint{1.167260in}{2.437467in}}%
\pgfpathlineto{\pgfqpoint{1.167753in}{2.455549in}}%
\pgfpathlineto{\pgfqpoint{1.167852in}{2.454825in}}%
\pgfpathlineto{\pgfqpoint{1.169925in}{2.388096in}}%
\pgfpathlineto{\pgfqpoint{1.170714in}{2.396686in}}%
\pgfpathlineto{\pgfqpoint{1.174662in}{2.519895in}}%
\pgfpathlineto{\pgfqpoint{1.174761in}{2.519699in}}%
\pgfpathlineto{\pgfqpoint{1.174959in}{2.517326in}}%
\pgfpathlineto{\pgfqpoint{1.175255in}{2.530416in}}%
\pgfpathlineto{\pgfqpoint{1.176932in}{2.553425in}}%
\pgfpathlineto{\pgfqpoint{1.175649in}{2.528320in}}%
\pgfpathlineto{\pgfqpoint{1.177031in}{2.552711in}}%
\pgfpathlineto{\pgfqpoint{1.177130in}{2.552458in}}%
\pgfpathlineto{\pgfqpoint{1.177229in}{2.553267in}}%
\pgfpathlineto{\pgfqpoint{1.177426in}{2.556007in}}%
\pgfpathlineto{\pgfqpoint{1.177722in}{2.544616in}}%
\pgfpathlineto{\pgfqpoint{1.177919in}{2.537695in}}%
\pgfpathlineto{\pgfqpoint{1.178709in}{2.544301in}}%
\pgfpathlineto{\pgfqpoint{1.180387in}{2.584903in}}%
\pgfpathlineto{\pgfqpoint{1.180683in}{2.573010in}}%
\pgfpathlineto{\pgfqpoint{1.180782in}{2.572266in}}%
\pgfpathlineto{\pgfqpoint{1.180880in}{2.575264in}}%
\pgfpathlineto{\pgfqpoint{1.181275in}{2.589972in}}%
\pgfpathlineto{\pgfqpoint{1.182164in}{2.586712in}}%
\pgfpathlineto{\pgfqpoint{1.182262in}{2.587764in}}%
\pgfpathlineto{\pgfqpoint{1.182460in}{2.581488in}}%
\pgfpathlineto{\pgfqpoint{1.183348in}{2.576693in}}%
\pgfpathlineto{\pgfqpoint{1.182953in}{2.593198in}}%
\pgfpathlineto{\pgfqpoint{1.183447in}{2.579550in}}%
\pgfpathlineto{\pgfqpoint{1.183644in}{2.587486in}}%
\pgfpathlineto{\pgfqpoint{1.184039in}{2.573788in}}%
\pgfpathlineto{\pgfqpoint{1.184335in}{2.574638in}}%
\pgfpathlineto{\pgfqpoint{1.185914in}{2.542027in}}%
\pgfpathlineto{\pgfqpoint{1.186704in}{2.534463in}}%
\pgfpathlineto{\pgfqpoint{1.186309in}{2.543420in}}%
\pgfpathlineto{\pgfqpoint{1.186901in}{2.538973in}}%
\pgfpathlineto{\pgfqpoint{1.187691in}{2.546091in}}%
\pgfpathlineto{\pgfqpoint{1.187987in}{2.539819in}}%
\pgfpathlineto{\pgfqpoint{1.190652in}{2.490027in}}%
\pgfpathlineto{\pgfqpoint{1.190849in}{2.494380in}}%
\pgfpathlineto{\pgfqpoint{1.190948in}{2.496072in}}%
\pgfpathlineto{\pgfqpoint{1.191145in}{2.486907in}}%
\pgfpathlineto{\pgfqpoint{1.191935in}{2.454086in}}%
\pgfpathlineto{\pgfqpoint{1.192231in}{2.476397in}}%
\pgfpathlineto{\pgfqpoint{1.193711in}{2.678433in}}%
\pgfpathlineto{\pgfqpoint{1.194402in}{2.651190in}}%
\pgfpathlineto{\pgfqpoint{1.194896in}{2.545654in}}%
\pgfpathlineto{\pgfqpoint{1.196376in}{2.391352in}}%
\pgfpathlineto{\pgfqpoint{1.196870in}{2.378242in}}%
\pgfpathlineto{\pgfqpoint{1.197166in}{2.391554in}}%
\pgfpathlineto{\pgfqpoint{1.198548in}{2.462311in}}%
\pgfpathlineto{\pgfqpoint{1.199140in}{2.420276in}}%
\pgfpathlineto{\pgfqpoint{1.199535in}{2.371592in}}%
\pgfpathlineto{\pgfqpoint{1.200225in}{2.404197in}}%
\pgfpathlineto{\pgfqpoint{1.201805in}{2.460673in}}%
\pgfpathlineto{\pgfqpoint{1.202002in}{2.444818in}}%
\pgfpathlineto{\pgfqpoint{1.202890in}{2.397025in}}%
\pgfpathlineto{\pgfqpoint{1.203285in}{2.418591in}}%
\pgfpathlineto{\pgfqpoint{1.204371in}{2.448853in}}%
\pgfpathlineto{\pgfqpoint{1.204667in}{2.439349in}}%
\pgfpathlineto{\pgfqpoint{1.204864in}{2.443573in}}%
\pgfpathlineto{\pgfqpoint{1.205259in}{2.457211in}}%
\pgfpathlineto{\pgfqpoint{1.205654in}{2.430566in}}%
\pgfpathlineto{\pgfqpoint{1.205950in}{2.415158in}}%
\pgfpathlineto{\pgfqpoint{1.206542in}{2.439379in}}%
\pgfpathlineto{\pgfqpoint{1.208023in}{2.472707in}}%
\pgfpathlineto{\pgfqpoint{1.208319in}{2.475629in}}%
\pgfpathlineto{\pgfqpoint{1.209010in}{2.473263in}}%
\pgfpathlineto{\pgfqpoint{1.209306in}{2.468586in}}%
\pgfpathlineto{\pgfqpoint{1.209503in}{2.477472in}}%
\pgfpathlineto{\pgfqpoint{1.211181in}{2.523942in}}%
\pgfpathlineto{\pgfqpoint{1.211477in}{2.523345in}}%
\pgfpathlineto{\pgfqpoint{1.211773in}{2.527549in}}%
\pgfpathlineto{\pgfqpoint{1.213945in}{2.550007in}}%
\pgfpathlineto{\pgfqpoint{1.214043in}{2.549527in}}%
\pgfpathlineto{\pgfqpoint{1.215326in}{2.536722in}}%
\pgfpathlineto{\pgfqpoint{1.215524in}{2.542039in}}%
\pgfpathlineto{\pgfqpoint{1.215721in}{2.547180in}}%
\pgfpathlineto{\pgfqpoint{1.216116in}{2.533378in}}%
\pgfpathlineto{\pgfqpoint{1.216511in}{2.543532in}}%
\pgfpathlineto{\pgfqpoint{1.218287in}{2.506610in}}%
\pgfpathlineto{\pgfqpoint{1.218485in}{2.508955in}}%
\pgfpathlineto{\pgfqpoint{1.218583in}{2.508862in}}%
\pgfpathlineto{\pgfqpoint{1.221939in}{2.416508in}}%
\pgfpathlineto{\pgfqpoint{1.222433in}{2.432145in}}%
\pgfpathlineto{\pgfqpoint{1.222630in}{2.435273in}}%
\pgfpathlineto{\pgfqpoint{1.223124in}{2.430499in}}%
\pgfpathlineto{\pgfqpoint{1.223518in}{2.413304in}}%
\pgfpathlineto{\pgfqpoint{1.224308in}{2.425899in}}%
\pgfpathlineto{\pgfqpoint{1.224505in}{2.435339in}}%
\pgfpathlineto{\pgfqpoint{1.224999in}{2.418862in}}%
\pgfpathlineto{\pgfqpoint{1.225295in}{2.422112in}}%
\pgfpathlineto{\pgfqpoint{1.226775in}{2.399045in}}%
\pgfpathlineto{\pgfqpoint{1.226874in}{2.400671in}}%
\pgfpathlineto{\pgfqpoint{1.227170in}{2.406558in}}%
\pgfpathlineto{\pgfqpoint{1.227664in}{2.393237in}}%
\pgfpathlineto{\pgfqpoint{1.228651in}{2.389610in}}%
\pgfpathlineto{\pgfqpoint{1.228256in}{2.399793in}}%
\pgfpathlineto{\pgfqpoint{1.228749in}{2.390965in}}%
\pgfpathlineto{\pgfqpoint{1.228848in}{2.391862in}}%
\pgfpathlineto{\pgfqpoint{1.229046in}{2.383804in}}%
\pgfpathlineto{\pgfqpoint{1.230625in}{2.343709in}}%
\pgfpathlineto{\pgfqpoint{1.235560in}{2.189770in}}%
\pgfpathlineto{\pgfqpoint{1.237435in}{2.216013in}}%
\pgfpathlineto{\pgfqpoint{1.237632in}{2.213213in}}%
\pgfpathlineto{\pgfqpoint{1.238027in}{2.223143in}}%
\pgfpathlineto{\pgfqpoint{1.238718in}{2.236910in}}%
\pgfpathlineto{\pgfqpoint{1.239409in}{2.231767in}}%
\pgfpathlineto{\pgfqpoint{1.239902in}{2.227385in}}%
\pgfpathlineto{\pgfqpoint{1.240100in}{2.233641in}}%
\pgfpathlineto{\pgfqpoint{1.240988in}{2.445282in}}%
\pgfpathlineto{\pgfqpoint{1.241876in}{2.501437in}}%
\pgfpathlineto{\pgfqpoint{1.242271in}{2.489291in}}%
\pgfpathlineto{\pgfqpoint{1.242666in}{2.442153in}}%
\pgfpathlineto{\pgfqpoint{1.244344in}{2.219119in}}%
\pgfpathlineto{\pgfqpoint{1.244443in}{2.220285in}}%
\pgfpathlineto{\pgfqpoint{1.246713in}{2.442238in}}%
\pgfpathlineto{\pgfqpoint{1.248391in}{2.421336in}}%
\pgfpathlineto{\pgfqpoint{1.248884in}{2.442412in}}%
\pgfpathlineto{\pgfqpoint{1.250167in}{2.506819in}}%
\pgfpathlineto{\pgfqpoint{1.250463in}{2.496575in}}%
\pgfpathlineto{\pgfqpoint{1.250759in}{2.479847in}}%
\pgfpathlineto{\pgfqpoint{1.251253in}{2.508806in}}%
\pgfpathlineto{\pgfqpoint{1.253029in}{2.598417in}}%
\pgfpathlineto{\pgfqpoint{1.253622in}{2.593234in}}%
\pgfpathlineto{\pgfqpoint{1.254016in}{2.585945in}}%
\pgfpathlineto{\pgfqpoint{1.254411in}{2.601178in}}%
\pgfpathlineto{\pgfqpoint{1.255201in}{2.635744in}}%
\pgfpathlineto{\pgfqpoint{1.256286in}{2.622053in}}%
\pgfpathlineto{\pgfqpoint{1.258557in}{2.588745in}}%
\pgfpathlineto{\pgfqpoint{1.256681in}{2.623483in}}%
\pgfpathlineto{\pgfqpoint{1.258754in}{2.593998in}}%
\pgfpathlineto{\pgfqpoint{1.259050in}{2.599398in}}%
\pgfpathlineto{\pgfqpoint{1.259544in}{2.586529in}}%
\pgfpathlineto{\pgfqpoint{1.260432in}{2.565117in}}%
\pgfpathlineto{\pgfqpoint{1.260925in}{2.571459in}}%
\pgfpathlineto{\pgfqpoint{1.261123in}{2.568789in}}%
\pgfpathlineto{\pgfqpoint{1.261419in}{2.578050in}}%
\pgfpathlineto{\pgfqpoint{1.261616in}{2.584084in}}%
\pgfpathlineto{\pgfqpoint{1.262110in}{2.562124in}}%
\pgfpathlineto{\pgfqpoint{1.262208in}{2.561544in}}%
\pgfpathlineto{\pgfqpoint{1.262406in}{2.566185in}}%
\pgfpathlineto{\pgfqpoint{1.262504in}{2.567344in}}%
\pgfpathlineto{\pgfqpoint{1.262801in}{2.558741in}}%
\pgfpathlineto{\pgfqpoint{1.264281in}{2.521628in}}%
\pgfpathlineto{\pgfqpoint{1.264577in}{2.528485in}}%
\pgfpathlineto{\pgfqpoint{1.265465in}{2.496331in}}%
\pgfpathlineto{\pgfqpoint{1.270104in}{2.244180in}}%
\pgfpathlineto{\pgfqpoint{1.272276in}{2.187448in}}%
\pgfpathlineto{\pgfqpoint{1.272572in}{2.189761in}}%
\pgfpathlineto{\pgfqpoint{1.272769in}{2.185670in}}%
\pgfpathlineto{\pgfqpoint{1.272967in}{2.179432in}}%
\pgfpathlineto{\pgfqpoint{1.273460in}{2.197573in}}%
\pgfpathlineto{\pgfqpoint{1.274743in}{2.206989in}}%
\pgfpathlineto{\pgfqpoint{1.274842in}{2.206273in}}%
\pgfpathlineto{\pgfqpoint{1.274941in}{2.205845in}}%
\pgfpathlineto{\pgfqpoint{1.275138in}{2.209496in}}%
\pgfpathlineto{\pgfqpoint{1.277309in}{2.250434in}}%
\pgfpathlineto{\pgfqpoint{1.277803in}{2.266332in}}%
\pgfpathlineto{\pgfqpoint{1.277902in}{2.267523in}}%
\pgfpathlineto{\pgfqpoint{1.278099in}{2.258674in}}%
\pgfpathlineto{\pgfqpoint{1.278296in}{2.249489in}}%
\pgfpathlineto{\pgfqpoint{1.278790in}{2.264631in}}%
\pgfpathlineto{\pgfqpoint{1.279086in}{2.260471in}}%
\pgfpathlineto{\pgfqpoint{1.279382in}{2.269033in}}%
\pgfpathlineto{\pgfqpoint{1.280172in}{2.266424in}}%
\pgfpathlineto{\pgfqpoint{1.281159in}{2.239186in}}%
\pgfpathlineto{\pgfqpoint{1.281553in}{2.251352in}}%
\pgfpathlineto{\pgfqpoint{1.284218in}{2.303351in}}%
\pgfpathlineto{\pgfqpoint{1.284416in}{2.303119in}}%
\pgfpathlineto{\pgfqpoint{1.284712in}{2.310662in}}%
\pgfpathlineto{\pgfqpoint{1.288364in}{2.537030in}}%
\pgfpathlineto{\pgfqpoint{1.289745in}{2.770219in}}%
\pgfpathlineto{\pgfqpoint{1.290140in}{2.743647in}}%
\pgfpathlineto{\pgfqpoint{1.290634in}{2.687129in}}%
\pgfpathlineto{\pgfqpoint{1.291325in}{2.487837in}}%
\pgfpathlineto{\pgfqpoint{1.292312in}{2.490011in}}%
\pgfpathlineto{\pgfqpoint{1.292509in}{2.496451in}}%
\pgfpathlineto{\pgfqpoint{1.294384in}{2.661918in}}%
\pgfpathlineto{\pgfqpoint{1.294976in}{2.618949in}}%
\pgfpathlineto{\pgfqpoint{1.295569in}{2.567697in}}%
\pgfpathlineto{\pgfqpoint{1.296358in}{2.602089in}}%
\pgfpathlineto{\pgfqpoint{1.296654in}{2.607864in}}%
\pgfpathlineto{\pgfqpoint{1.297641in}{2.626810in}}%
\pgfpathlineto{\pgfqpoint{1.297839in}{2.617890in}}%
\pgfpathlineto{\pgfqpoint{1.298826in}{2.531571in}}%
\pgfpathlineto{\pgfqpoint{1.299220in}{2.574256in}}%
\pgfpathlineto{\pgfqpoint{1.299319in}{2.577890in}}%
\pgfpathlineto{\pgfqpoint{1.299615in}{2.560117in}}%
\pgfpathlineto{\pgfqpoint{1.300207in}{2.575437in}}%
\pgfpathlineto{\pgfqpoint{1.301885in}{2.528973in}}%
\pgfpathlineto{\pgfqpoint{1.300701in}{2.576307in}}%
\pgfpathlineto{\pgfqpoint{1.302280in}{2.547597in}}%
\pgfpathlineto{\pgfqpoint{1.303267in}{2.563595in}}%
\pgfpathlineto{\pgfqpoint{1.303563in}{2.558120in}}%
\pgfpathlineto{\pgfqpoint{1.305636in}{2.518998in}}%
\pgfpathlineto{\pgfqpoint{1.305833in}{2.524006in}}%
\pgfpathlineto{\pgfqpoint{1.306031in}{2.530488in}}%
\pgfpathlineto{\pgfqpoint{1.306426in}{2.512729in}}%
\pgfpathlineto{\pgfqpoint{1.306722in}{2.517338in}}%
\pgfpathlineto{\pgfqpoint{1.312742in}{2.312447in}}%
\pgfpathlineto{\pgfqpoint{1.313236in}{2.304003in}}%
\pgfpathlineto{\pgfqpoint{1.315802in}{2.253116in}}%
\pgfpathlineto{\pgfqpoint{1.317282in}{2.243115in}}%
\pgfpathlineto{\pgfqpoint{1.316888in}{2.253295in}}%
\pgfpathlineto{\pgfqpoint{1.317480in}{2.245424in}}%
\pgfpathlineto{\pgfqpoint{1.318368in}{2.249781in}}%
\pgfpathlineto{\pgfqpoint{1.318565in}{2.245822in}}%
\pgfpathlineto{\pgfqpoint{1.318664in}{2.244941in}}%
\pgfpathlineto{\pgfqpoint{1.318862in}{2.250737in}}%
\pgfpathlineto{\pgfqpoint{1.319256in}{2.267218in}}%
\pgfpathlineto{\pgfqpoint{1.319849in}{2.250405in}}%
\pgfpathlineto{\pgfqpoint{1.320342in}{2.230571in}}%
\pgfpathlineto{\pgfqpoint{1.321033in}{2.245999in}}%
\pgfpathlineto{\pgfqpoint{1.321921in}{2.266775in}}%
\pgfpathlineto{\pgfqpoint{1.323204in}{2.398822in}}%
\pgfpathlineto{\pgfqpoint{1.325771in}{2.629887in}}%
\pgfpathlineto{\pgfqpoint{1.325869in}{2.629433in}}%
\pgfpathlineto{\pgfqpoint{1.326067in}{2.632919in}}%
\pgfpathlineto{\pgfqpoint{1.326461in}{2.650532in}}%
\pgfpathlineto{\pgfqpoint{1.327054in}{2.626511in}}%
\pgfpathlineto{\pgfqpoint{1.329916in}{2.568937in}}%
\pgfpathlineto{\pgfqpoint{1.330015in}{2.570763in}}%
\pgfpathlineto{\pgfqpoint{1.332186in}{2.623543in}}%
\pgfpathlineto{\pgfqpoint{1.333469in}{2.632261in}}%
\pgfpathlineto{\pgfqpoint{1.333074in}{2.622260in}}%
\pgfpathlineto{\pgfqpoint{1.333568in}{2.628023in}}%
\pgfpathlineto{\pgfqpoint{1.333864in}{2.612507in}}%
\pgfpathlineto{\pgfqpoint{1.334752in}{2.617641in}}%
\pgfpathlineto{\pgfqpoint{1.335048in}{2.605620in}}%
\pgfpathlineto{\pgfqpoint{1.335542in}{2.576491in}}%
\pgfpathlineto{\pgfqpoint{1.335937in}{2.622297in}}%
\pgfpathlineto{\pgfqpoint{1.337318in}{2.811780in}}%
\pgfpathlineto{\pgfqpoint{1.337713in}{2.774075in}}%
\pgfpathlineto{\pgfqpoint{1.338305in}{2.695576in}}%
\pgfpathlineto{\pgfqpoint{1.340082in}{2.443937in}}%
\pgfpathlineto{\pgfqpoint{1.340181in}{2.443385in}}%
\pgfpathlineto{\pgfqpoint{1.340279in}{2.447433in}}%
\pgfpathlineto{\pgfqpoint{1.341957in}{2.507024in}}%
\pgfpathlineto{\pgfqpoint{1.342155in}{2.500732in}}%
\pgfpathlineto{\pgfqpoint{1.342944in}{2.408311in}}%
\pgfpathlineto{\pgfqpoint{1.343240in}{2.390421in}}%
\pgfpathlineto{\pgfqpoint{1.344129in}{2.394958in}}%
\pgfpathlineto{\pgfqpoint{1.344425in}{2.385907in}}%
\pgfpathlineto{\pgfqpoint{1.344819in}{2.404679in}}%
\pgfpathlineto{\pgfqpoint{1.345116in}{2.399059in}}%
\pgfpathlineto{\pgfqpoint{1.345313in}{2.401382in}}%
\pgfpathlineto{\pgfqpoint{1.345510in}{2.396237in}}%
\pgfpathlineto{\pgfqpoint{1.346596in}{2.302820in}}%
\pgfpathlineto{\pgfqpoint{1.347386in}{2.316774in}}%
\pgfpathlineto{\pgfqpoint{1.348866in}{2.299228in}}%
\pgfpathlineto{\pgfqpoint{1.349458in}{2.242993in}}%
\pgfpathlineto{\pgfqpoint{1.350445in}{2.260762in}}%
\pgfpathlineto{\pgfqpoint{1.351334in}{2.267169in}}%
\pgfpathlineto{\pgfqpoint{1.350939in}{2.259682in}}%
\pgfpathlineto{\pgfqpoint{1.351531in}{2.262825in}}%
\pgfpathlineto{\pgfqpoint{1.352617in}{2.244659in}}%
\pgfpathlineto{\pgfqpoint{1.352024in}{2.268979in}}%
\pgfpathlineto{\pgfqpoint{1.352913in}{2.256174in}}%
\pgfpathlineto{\pgfqpoint{1.355281in}{2.280147in}}%
\pgfpathlineto{\pgfqpoint{1.353406in}{2.249443in}}%
\pgfpathlineto{\pgfqpoint{1.355380in}{2.278787in}}%
\pgfpathlineto{\pgfqpoint{1.355775in}{2.264869in}}%
\pgfpathlineto{\pgfqpoint{1.356466in}{2.277930in}}%
\pgfpathlineto{\pgfqpoint{1.358341in}{2.319694in}}%
\pgfpathlineto{\pgfqpoint{1.358539in}{2.314781in}}%
\pgfpathlineto{\pgfqpoint{1.358736in}{2.309038in}}%
\pgfpathlineto{\pgfqpoint{1.359131in}{2.318747in}}%
\pgfpathlineto{\pgfqpoint{1.359624in}{2.315148in}}%
\pgfpathlineto{\pgfqpoint{1.361006in}{2.351786in}}%
\pgfpathlineto{\pgfqpoint{1.361203in}{2.351072in}}%
\pgfpathlineto{\pgfqpoint{1.361302in}{2.350715in}}%
\pgfpathlineto{\pgfqpoint{1.361500in}{2.352545in}}%
\pgfpathlineto{\pgfqpoint{1.362881in}{2.388679in}}%
\pgfpathlineto{\pgfqpoint{1.366632in}{2.577370in}}%
\pgfpathlineto{\pgfqpoint{1.369001in}{2.545730in}}%
\pgfpathlineto{\pgfqpoint{1.369099in}{2.546467in}}%
\pgfpathlineto{\pgfqpoint{1.369198in}{2.546999in}}%
\pgfpathlineto{\pgfqpoint{1.369395in}{2.543223in}}%
\pgfpathlineto{\pgfqpoint{1.369692in}{2.534912in}}%
\pgfpathlineto{\pgfqpoint{1.370382in}{2.540569in}}%
\pgfpathlineto{\pgfqpoint{1.372850in}{2.597505in}}%
\pgfpathlineto{\pgfqpoint{1.372949in}{2.599251in}}%
\pgfpathlineto{\pgfqpoint{1.373245in}{2.587469in}}%
\pgfpathlineto{\pgfqpoint{1.376502in}{2.459620in}}%
\pgfpathlineto{\pgfqpoint{1.378871in}{2.415315in}}%
\pgfpathlineto{\pgfqpoint{1.381042in}{2.374927in}}%
\pgfpathlineto{\pgfqpoint{1.381930in}{2.376150in}}%
\pgfpathlineto{\pgfqpoint{1.382522in}{2.356542in}}%
\pgfpathlineto{\pgfqpoint{1.382819in}{2.325122in}}%
\pgfpathlineto{\pgfqpoint{1.383213in}{2.393946in}}%
\pgfpathlineto{\pgfqpoint{1.384792in}{2.580402in}}%
\pgfpathlineto{\pgfqpoint{1.384990in}{2.571955in}}%
\pgfpathlineto{\pgfqpoint{1.385681in}{2.465940in}}%
\pgfpathlineto{\pgfqpoint{1.387260in}{2.252885in}}%
\pgfpathlineto{\pgfqpoint{1.387359in}{2.250489in}}%
\pgfpathlineto{\pgfqpoint{1.388148in}{2.257568in}}%
\pgfpathlineto{\pgfqpoint{1.388543in}{2.255640in}}%
\pgfpathlineto{\pgfqpoint{1.388839in}{2.281385in}}%
\pgfpathlineto{\pgfqpoint{1.389530in}{2.320184in}}%
\pgfpathlineto{\pgfqpoint{1.389925in}{2.279760in}}%
\pgfpathlineto{\pgfqpoint{1.390616in}{2.240960in}}%
\pgfpathlineto{\pgfqpoint{1.391109in}{2.262577in}}%
\pgfpathlineto{\pgfqpoint{1.392590in}{2.322882in}}%
\pgfpathlineto{\pgfqpoint{1.393182in}{2.299049in}}%
\pgfpathlineto{\pgfqpoint{1.393675in}{2.272509in}}%
\pgfpathlineto{\pgfqpoint{1.394169in}{2.299635in}}%
\pgfpathlineto{\pgfqpoint{1.395649in}{2.348348in}}%
\pgfpathlineto{\pgfqpoint{1.395748in}{2.347215in}}%
\pgfpathlineto{\pgfqpoint{1.397229in}{2.293980in}}%
\pgfpathlineto{\pgfqpoint{1.397623in}{2.330939in}}%
\pgfpathlineto{\pgfqpoint{1.399203in}{2.416104in}}%
\pgfpathlineto{\pgfqpoint{1.402460in}{2.590042in}}%
\pgfpathlineto{\pgfqpoint{1.403348in}{2.575304in}}%
\pgfpathlineto{\pgfqpoint{1.403545in}{2.567848in}}%
\pgfpathlineto{\pgfqpoint{1.404039in}{2.584342in}}%
\pgfpathlineto{\pgfqpoint{1.404335in}{2.579995in}}%
\pgfpathlineto{\pgfqpoint{1.404730in}{2.581525in}}%
\pgfpathlineto{\pgfqpoint{1.405618in}{2.559769in}}%
\pgfpathlineto{\pgfqpoint{1.405914in}{2.551002in}}%
\pgfpathlineto{\pgfqpoint{1.406605in}{2.563802in}}%
\pgfpathlineto{\pgfqpoint{1.407197in}{2.573452in}}%
\pgfpathlineto{\pgfqpoint{1.407592in}{2.589518in}}%
\pgfpathlineto{\pgfqpoint{1.408184in}{2.571203in}}%
\pgfpathlineto{\pgfqpoint{1.408480in}{2.562262in}}%
\pgfpathlineto{\pgfqpoint{1.409369in}{2.552265in}}%
\pgfpathlineto{\pgfqpoint{1.409566in}{2.558010in}}%
\pgfpathlineto{\pgfqpoint{1.409665in}{2.560302in}}%
\pgfpathlineto{\pgfqpoint{1.409961in}{2.543753in}}%
\pgfpathlineto{\pgfqpoint{1.411244in}{2.515207in}}%
\pgfpathlineto{\pgfqpoint{1.410553in}{2.547887in}}%
\pgfpathlineto{\pgfqpoint{1.411342in}{2.516577in}}%
\pgfpathlineto{\pgfqpoint{1.411540in}{2.521750in}}%
\pgfpathlineto{\pgfqpoint{1.411935in}{2.510247in}}%
\pgfpathlineto{\pgfqpoint{1.412428in}{2.518069in}}%
\pgfpathlineto{\pgfqpoint{1.414106in}{2.489451in}}%
\pgfpathlineto{\pgfqpoint{1.416672in}{2.468017in}}%
\pgfpathlineto{\pgfqpoint{1.414501in}{2.491224in}}%
\pgfpathlineto{\pgfqpoint{1.416870in}{2.475176in}}%
\pgfpathlineto{\pgfqpoint{1.417955in}{2.483521in}}%
\pgfpathlineto{\pgfqpoint{1.417462in}{2.473136in}}%
\pgfpathlineto{\pgfqpoint{1.418054in}{2.482189in}}%
\pgfpathlineto{\pgfqpoint{1.418449in}{2.485045in}}%
\pgfpathlineto{\pgfqpoint{1.418745in}{2.473535in}}%
\pgfpathlineto{\pgfqpoint{1.418942in}{2.467463in}}%
\pgfpathlineto{\pgfqpoint{1.419732in}{2.479002in}}%
\pgfpathlineto{\pgfqpoint{1.420028in}{2.484476in}}%
\pgfpathlineto{\pgfqpoint{1.420522in}{2.470184in}}%
\pgfpathlineto{\pgfqpoint{1.420818in}{2.469385in}}%
\pgfpathlineto{\pgfqpoint{1.424667in}{2.376794in}}%
\pgfpathlineto{\pgfqpoint{1.424963in}{2.381958in}}%
\pgfpathlineto{\pgfqpoint{1.425160in}{2.386866in}}%
\pgfpathlineto{\pgfqpoint{1.426049in}{2.383127in}}%
\pgfpathlineto{\pgfqpoint{1.426740in}{2.373855in}}%
\pgfpathlineto{\pgfqpoint{1.427233in}{2.381251in}}%
\pgfpathlineto{\pgfqpoint{1.428023in}{2.376256in}}%
\pgfpathlineto{\pgfqpoint{1.427628in}{2.381366in}}%
\pgfpathlineto{\pgfqpoint{1.428220in}{2.380579in}}%
\pgfpathlineto{\pgfqpoint{1.429207in}{2.388505in}}%
\pgfpathlineto{\pgfqpoint{1.429404in}{2.386669in}}%
\pgfpathlineto{\pgfqpoint{1.429898in}{2.355709in}}%
\pgfpathlineto{\pgfqpoint{1.430293in}{2.393496in}}%
\pgfpathlineto{\pgfqpoint{1.431773in}{2.642443in}}%
\pgfpathlineto{\pgfqpoint{1.432464in}{2.627111in}}%
\pgfpathlineto{\pgfqpoint{1.432958in}{2.528524in}}%
\pgfpathlineto{\pgfqpoint{1.433451in}{2.391421in}}%
\pgfpathlineto{\pgfqpoint{1.434339in}{2.401952in}}%
\pgfpathlineto{\pgfqpoint{1.434537in}{2.398482in}}%
\pgfpathlineto{\pgfqpoint{1.434833in}{2.411906in}}%
\pgfpathlineto{\pgfqpoint{1.436708in}{2.531644in}}%
\pgfpathlineto{\pgfqpoint{1.436906in}{2.523481in}}%
\pgfpathlineto{\pgfqpoint{1.437794in}{2.452829in}}%
\pgfpathlineto{\pgfqpoint{1.438485in}{2.479069in}}%
\pgfpathlineto{\pgfqpoint{1.439866in}{2.541324in}}%
\pgfpathlineto{\pgfqpoint{1.440459in}{2.512959in}}%
\pgfpathlineto{\pgfqpoint{1.440952in}{2.491652in}}%
\pgfpathlineto{\pgfqpoint{1.441347in}{2.512563in}}%
\pgfpathlineto{\pgfqpoint{1.442729in}{2.538717in}}%
\pgfpathlineto{\pgfqpoint{1.443124in}{2.538825in}}%
\pgfpathlineto{\pgfqpoint{1.443420in}{2.532817in}}%
\pgfpathlineto{\pgfqpoint{1.443913in}{2.511190in}}%
\pgfpathlineto{\pgfqpoint{1.444604in}{2.526128in}}%
\pgfpathlineto{\pgfqpoint{1.445492in}{2.557937in}}%
\pgfpathlineto{\pgfqpoint{1.446085in}{2.549246in}}%
\pgfpathlineto{\pgfqpoint{1.447170in}{2.532705in}}%
\pgfpathlineto{\pgfqpoint{1.446677in}{2.549487in}}%
\pgfpathlineto{\pgfqpoint{1.447664in}{2.544121in}}%
\pgfpathlineto{\pgfqpoint{1.449144in}{2.569675in}}%
\pgfpathlineto{\pgfqpoint{1.449440in}{2.563608in}}%
\pgfpathlineto{\pgfqpoint{1.450822in}{2.538997in}}%
\pgfpathlineto{\pgfqpoint{1.451217in}{2.543986in}}%
\pgfpathlineto{\pgfqpoint{1.451316in}{2.544569in}}%
\pgfpathlineto{\pgfqpoint{1.451612in}{2.540060in}}%
\pgfpathlineto{\pgfqpoint{1.453191in}{2.519294in}}%
\pgfpathlineto{\pgfqpoint{1.453290in}{2.520500in}}%
\pgfpathlineto{\pgfqpoint{1.453487in}{2.526042in}}%
\pgfpathlineto{\pgfqpoint{1.453980in}{2.508971in}}%
\pgfpathlineto{\pgfqpoint{1.454573in}{2.518724in}}%
\pgfpathlineto{\pgfqpoint{1.455264in}{2.512096in}}%
\pgfpathlineto{\pgfqpoint{1.458619in}{2.476116in}}%
\pgfpathlineto{\pgfqpoint{1.458915in}{2.484397in}}%
\pgfpathlineto{\pgfqpoint{1.459014in}{2.485738in}}%
\pgfpathlineto{\pgfqpoint{1.459310in}{2.476520in}}%
\pgfpathlineto{\pgfqpoint{1.459409in}{2.475387in}}%
\pgfpathlineto{\pgfqpoint{1.459705in}{2.484298in}}%
\pgfpathlineto{\pgfqpoint{1.459804in}{2.485448in}}%
\pgfpathlineto{\pgfqpoint{1.460100in}{2.476835in}}%
\pgfpathlineto{\pgfqpoint{1.461087in}{2.466817in}}%
\pgfpathlineto{\pgfqpoint{1.460593in}{2.481233in}}%
\pgfpathlineto{\pgfqpoint{1.461284in}{2.470351in}}%
\pgfpathlineto{\pgfqpoint{1.461580in}{2.481702in}}%
\pgfpathlineto{\pgfqpoint{1.462370in}{2.471121in}}%
\pgfpathlineto{\pgfqpoint{1.462567in}{2.464623in}}%
\pgfpathlineto{\pgfqpoint{1.462962in}{2.477163in}}%
\pgfpathlineto{\pgfqpoint{1.463357in}{2.474190in}}%
\pgfpathlineto{\pgfqpoint{1.464048in}{2.481246in}}%
\pgfpathlineto{\pgfqpoint{1.464344in}{2.472507in}}%
\pgfpathlineto{\pgfqpoint{1.464541in}{2.467629in}}%
\pgfpathlineto{\pgfqpoint{1.464936in}{2.491842in}}%
\pgfpathlineto{\pgfqpoint{1.465133in}{2.487309in}}%
\pgfpathlineto{\pgfqpoint{1.465232in}{2.484088in}}%
\pgfpathlineto{\pgfqpoint{1.465726in}{2.499586in}}%
\pgfpathlineto{\pgfqpoint{1.465923in}{2.497786in}}%
\pgfpathlineto{\pgfqpoint{1.466811in}{2.510790in}}%
\pgfpathlineto{\pgfqpoint{1.467107in}{2.527872in}}%
\pgfpathlineto{\pgfqpoint{1.467897in}{2.513871in}}%
\pgfpathlineto{\pgfqpoint{1.468489in}{2.516619in}}%
\pgfpathlineto{\pgfqpoint{1.468785in}{2.490447in}}%
\pgfpathlineto{\pgfqpoint{1.469674in}{2.467063in}}%
\pgfpathlineto{\pgfqpoint{1.469279in}{2.499100in}}%
\pgfpathlineto{\pgfqpoint{1.469970in}{2.482196in}}%
\pgfpathlineto{\pgfqpoint{1.470167in}{2.474444in}}%
\pgfpathlineto{\pgfqpoint{1.470463in}{2.451150in}}%
\pgfpathlineto{\pgfqpoint{1.471450in}{2.452865in}}%
\pgfpathlineto{\pgfqpoint{1.471648in}{2.444387in}}%
\pgfpathlineto{\pgfqpoint{1.472141in}{2.456144in}}%
\pgfpathlineto{\pgfqpoint{1.472437in}{2.454447in}}%
\pgfpathlineto{\pgfqpoint{1.472832in}{2.461963in}}%
\pgfpathlineto{\pgfqpoint{1.473424in}{2.451374in}}%
\pgfpathlineto{\pgfqpoint{1.474016in}{2.453729in}}%
\pgfpathlineto{\pgfqpoint{1.475102in}{2.418732in}}%
\pgfpathlineto{\pgfqpoint{1.475892in}{2.415691in}}%
\pgfpathlineto{\pgfqpoint{1.477175in}{2.483972in}}%
\pgfpathlineto{\pgfqpoint{1.478655in}{2.753998in}}%
\pgfpathlineto{\pgfqpoint{1.479445in}{2.723774in}}%
\pgfpathlineto{\pgfqpoint{1.481419in}{2.468803in}}%
\pgfpathlineto{\pgfqpoint{1.482110in}{2.491650in}}%
\pgfpathlineto{\pgfqpoint{1.483393in}{2.578779in}}%
\pgfpathlineto{\pgfqpoint{1.483985in}{2.555434in}}%
\pgfpathlineto{\pgfqpoint{1.484676in}{2.503228in}}%
\pgfpathlineto{\pgfqpoint{1.485268in}{2.536744in}}%
\pgfpathlineto{\pgfqpoint{1.485564in}{2.526561in}}%
\pgfpathlineto{\pgfqpoint{1.485860in}{2.542037in}}%
\pgfpathlineto{\pgfqpoint{1.486748in}{2.566878in}}%
\pgfpathlineto{\pgfqpoint{1.487143in}{2.558350in}}%
\pgfpathlineto{\pgfqpoint{1.487637in}{2.513155in}}%
\pgfpathlineto{\pgfqpoint{1.488426in}{2.544657in}}%
\pgfpathlineto{\pgfqpoint{1.490006in}{2.579605in}}%
\pgfpathlineto{\pgfqpoint{1.490203in}{2.578621in}}%
\pgfpathlineto{\pgfqpoint{1.490696in}{2.561007in}}%
\pgfpathlineto{\pgfqpoint{1.490993in}{2.557324in}}%
\pgfpathlineto{\pgfqpoint{1.491486in}{2.566895in}}%
\pgfpathlineto{\pgfqpoint{1.492868in}{2.588594in}}%
\pgfpathlineto{\pgfqpoint{1.493263in}{2.596222in}}%
\pgfpathlineto{\pgfqpoint{1.493657in}{2.585243in}}%
\pgfpathlineto{\pgfqpoint{1.494052in}{2.571123in}}%
\pgfpathlineto{\pgfqpoint{1.494842in}{2.581172in}}%
\pgfpathlineto{\pgfqpoint{1.495927in}{2.590176in}}%
\pgfpathlineto{\pgfqpoint{1.495533in}{2.580059in}}%
\pgfpathlineto{\pgfqpoint{1.496125in}{2.583658in}}%
\pgfpathlineto{\pgfqpoint{1.497211in}{2.560311in}}%
\pgfpathlineto{\pgfqpoint{1.496717in}{2.592634in}}%
\pgfpathlineto{\pgfqpoint{1.497507in}{2.566544in}}%
\pgfpathlineto{\pgfqpoint{1.498198in}{2.582622in}}%
\pgfpathlineto{\pgfqpoint{1.498987in}{2.574520in}}%
\pgfpathlineto{\pgfqpoint{1.500468in}{2.557724in}}%
\pgfpathlineto{\pgfqpoint{1.500665in}{2.558405in}}%
\pgfpathlineto{\pgfqpoint{1.501257in}{2.578992in}}%
\pgfpathlineto{\pgfqpoint{1.501652in}{2.557710in}}%
\pgfpathlineto{\pgfqpoint{1.504021in}{2.524584in}}%
\pgfpathlineto{\pgfqpoint{1.502146in}{2.561654in}}%
\pgfpathlineto{\pgfqpoint{1.504218in}{2.525551in}}%
\pgfpathlineto{\pgfqpoint{1.504317in}{2.526764in}}%
\pgfpathlineto{\pgfqpoint{1.504613in}{2.517998in}}%
\pgfpathlineto{\pgfqpoint{1.505699in}{2.499304in}}%
\pgfpathlineto{\pgfqpoint{1.505896in}{2.503098in}}%
\pgfpathlineto{\pgfqpoint{1.506093in}{2.508417in}}%
\pgfpathlineto{\pgfqpoint{1.506686in}{2.495487in}}%
\pgfpathlineto{\pgfqpoint{1.507080in}{2.506004in}}%
\pgfpathlineto{\pgfqpoint{1.507179in}{2.507252in}}%
\pgfpathlineto{\pgfqpoint{1.507574in}{2.499305in}}%
\pgfpathlineto{\pgfqpoint{1.507673in}{2.498905in}}%
\pgfpathlineto{\pgfqpoint{1.507870in}{2.502292in}}%
\pgfpathlineto{\pgfqpoint{1.509745in}{2.528022in}}%
\pgfpathlineto{\pgfqpoint{1.509844in}{2.527876in}}%
\pgfpathlineto{\pgfqpoint{1.510140in}{2.522553in}}%
\pgfpathlineto{\pgfqpoint{1.510634in}{2.531555in}}%
\pgfpathlineto{\pgfqpoint{1.511325in}{2.540178in}}%
\pgfpathlineto{\pgfqpoint{1.513496in}{2.597596in}}%
\pgfpathlineto{\pgfqpoint{1.513693in}{2.595374in}}%
\pgfpathlineto{\pgfqpoint{1.515075in}{2.574711in}}%
\pgfpathlineto{\pgfqpoint{1.515272in}{2.575500in}}%
\pgfpathlineto{\pgfqpoint{1.515371in}{2.575426in}}%
\pgfpathlineto{\pgfqpoint{1.515963in}{2.552491in}}%
\pgfpathlineto{\pgfqpoint{1.517740in}{2.506362in}}%
\pgfpathlineto{\pgfqpoint{1.517937in}{2.510084in}}%
\pgfpathlineto{\pgfqpoint{1.518036in}{2.510086in}}%
\pgfpathlineto{\pgfqpoint{1.519813in}{2.481197in}}%
\pgfpathlineto{\pgfqpoint{1.520010in}{2.483358in}}%
\pgfpathlineto{\pgfqpoint{1.520701in}{2.490866in}}%
\pgfpathlineto{\pgfqpoint{1.520997in}{2.485507in}}%
\pgfpathlineto{\pgfqpoint{1.522872in}{2.458797in}}%
\pgfpathlineto{\pgfqpoint{1.522971in}{2.459016in}}%
\pgfpathlineto{\pgfqpoint{1.523070in}{2.457722in}}%
\pgfpathlineto{\pgfqpoint{1.523563in}{2.423229in}}%
\pgfpathlineto{\pgfqpoint{1.523958in}{2.462604in}}%
\pgfpathlineto{\pgfqpoint{1.525438in}{2.692189in}}%
\pgfpathlineto{\pgfqpoint{1.526129in}{2.662677in}}%
\pgfpathlineto{\pgfqpoint{1.528202in}{2.419603in}}%
\pgfpathlineto{\pgfqpoint{1.528498in}{2.428932in}}%
\pgfpathlineto{\pgfqpoint{1.529683in}{2.477897in}}%
\pgfpathlineto{\pgfqpoint{1.530077in}{2.530082in}}%
\pgfpathlineto{\pgfqpoint{1.530670in}{2.478256in}}%
\pgfpathlineto{\pgfqpoint{1.531360in}{2.421552in}}%
\pgfpathlineto{\pgfqpoint{1.532249in}{2.444816in}}%
\pgfpathlineto{\pgfqpoint{1.533038in}{2.469983in}}%
\pgfpathlineto{\pgfqpoint{1.533433in}{2.482928in}}%
\pgfpathlineto{\pgfqpoint{1.533828in}{2.464214in}}%
\pgfpathlineto{\pgfqpoint{1.534617in}{2.427842in}}%
\pgfpathlineto{\pgfqpoint{1.535111in}{2.450719in}}%
\pgfpathlineto{\pgfqpoint{1.536295in}{2.491650in}}%
\pgfpathlineto{\pgfqpoint{1.536591in}{2.488152in}}%
\pgfpathlineto{\pgfqpoint{1.536789in}{2.491318in}}%
\pgfpathlineto{\pgfqpoint{1.536986in}{2.485147in}}%
\pgfpathlineto{\pgfqpoint{1.537677in}{2.456049in}}%
\pgfpathlineto{\pgfqpoint{1.538072in}{2.474950in}}%
\pgfpathlineto{\pgfqpoint{1.539552in}{2.501678in}}%
\pgfpathlineto{\pgfqpoint{1.539947in}{2.499789in}}%
\pgfpathlineto{\pgfqpoint{1.540243in}{2.503256in}}%
\pgfpathlineto{\pgfqpoint{1.540539in}{2.509573in}}%
\pgfpathlineto{\pgfqpoint{1.540836in}{2.496223in}}%
\pgfpathlineto{\pgfqpoint{1.541132in}{2.483629in}}%
\pgfpathlineto{\pgfqpoint{1.541724in}{2.506059in}}%
\pgfpathlineto{\pgfqpoint{1.542020in}{2.521429in}}%
\pgfpathlineto{\pgfqpoint{1.542415in}{2.501085in}}%
\pgfpathlineto{\pgfqpoint{1.542809in}{2.507706in}}%
\pgfpathlineto{\pgfqpoint{1.543994in}{2.492336in}}%
\pgfpathlineto{\pgfqpoint{1.544290in}{2.499874in}}%
\pgfpathlineto{\pgfqpoint{1.545080in}{2.507563in}}%
\pgfpathlineto{\pgfqpoint{1.544685in}{2.499549in}}%
\pgfpathlineto{\pgfqpoint{1.545376in}{2.501242in}}%
\pgfpathlineto{\pgfqpoint{1.546757in}{2.489594in}}%
\pgfpathlineto{\pgfqpoint{1.546856in}{2.490031in}}%
\pgfpathlineto{\pgfqpoint{1.546955in}{2.490616in}}%
\pgfpathlineto{\pgfqpoint{1.547152in}{2.487889in}}%
\pgfpathlineto{\pgfqpoint{1.548041in}{2.475438in}}%
\pgfpathlineto{\pgfqpoint{1.548435in}{2.480440in}}%
\pgfpathlineto{\pgfqpoint{1.548534in}{2.481283in}}%
\pgfpathlineto{\pgfqpoint{1.548731in}{2.477232in}}%
\pgfpathlineto{\pgfqpoint{1.553173in}{2.312231in}}%
\pgfpathlineto{\pgfqpoint{1.553469in}{2.326060in}}%
\pgfpathlineto{\pgfqpoint{1.555147in}{2.392206in}}%
\pgfpathlineto{\pgfqpoint{1.556134in}{2.401272in}}%
\pgfpathlineto{\pgfqpoint{1.556331in}{2.396959in}}%
\pgfpathlineto{\pgfqpoint{1.557417in}{2.385159in}}%
\pgfpathlineto{\pgfqpoint{1.557614in}{2.387796in}}%
\pgfpathlineto{\pgfqpoint{1.560279in}{2.430171in}}%
\pgfpathlineto{\pgfqpoint{1.560575in}{2.426708in}}%
\pgfpathlineto{\pgfqpoint{1.564128in}{2.374845in}}%
\pgfpathlineto{\pgfqpoint{1.564227in}{2.376525in}}%
\pgfpathlineto{\pgfqpoint{1.565510in}{2.397728in}}%
\pgfpathlineto{\pgfqpoint{1.564918in}{2.370291in}}%
\pgfpathlineto{\pgfqpoint{1.565806in}{2.396567in}}%
\pgfpathlineto{\pgfqpoint{1.569360in}{2.470521in}}%
\pgfpathlineto{\pgfqpoint{1.569557in}{2.459104in}}%
\pgfpathlineto{\pgfqpoint{1.569952in}{2.429103in}}%
\pgfpathlineto{\pgfqpoint{1.570347in}{2.467760in}}%
\pgfpathlineto{\pgfqpoint{1.571827in}{2.726599in}}%
\pgfpathlineto{\pgfqpoint{1.572518in}{2.700355in}}%
\pgfpathlineto{\pgfqpoint{1.573110in}{2.562588in}}%
\pgfpathlineto{\pgfqpoint{1.574492in}{2.442696in}}%
\pgfpathlineto{\pgfqpoint{1.574689in}{2.438381in}}%
\pgfpathlineto{\pgfqpoint{1.575084in}{2.457653in}}%
\pgfpathlineto{\pgfqpoint{1.576565in}{2.517865in}}%
\pgfpathlineto{\pgfqpoint{1.577058in}{2.486809in}}%
\pgfpathlineto{\pgfqpoint{1.577749in}{2.415582in}}%
\pgfpathlineto{\pgfqpoint{1.578637in}{2.423121in}}%
\pgfpathlineto{\pgfqpoint{1.579427in}{2.456765in}}%
\pgfpathlineto{\pgfqpoint{1.579624in}{2.467217in}}%
\pgfpathlineto{\pgfqpoint{1.580216in}{2.447220in}}%
\pgfpathlineto{\pgfqpoint{1.581006in}{2.381561in}}%
\pgfpathlineto{\pgfqpoint{1.581697in}{2.413791in}}%
\pgfpathlineto{\pgfqpoint{1.583079in}{2.440703in}}%
\pgfpathlineto{\pgfqpoint{1.583276in}{2.436649in}}%
\pgfpathlineto{\pgfqpoint{1.584362in}{2.389035in}}%
\pgfpathlineto{\pgfqpoint{1.584954in}{2.401075in}}%
\pgfpathlineto{\pgfqpoint{1.585053in}{2.400165in}}%
\pgfpathlineto{\pgfqpoint{1.585151in}{2.403391in}}%
\pgfpathlineto{\pgfqpoint{1.585447in}{2.420404in}}%
\pgfpathlineto{\pgfqpoint{1.586336in}{2.411832in}}%
\pgfpathlineto{\pgfqpoint{1.586731in}{2.421657in}}%
\pgfpathlineto{\pgfqpoint{1.586928in}{2.411917in}}%
\pgfpathlineto{\pgfqpoint{1.587224in}{2.397287in}}%
\pgfpathlineto{\pgfqpoint{1.588014in}{2.405631in}}%
\pgfpathlineto{\pgfqpoint{1.589395in}{2.420736in}}%
\pgfpathlineto{\pgfqpoint{1.589691in}{2.417156in}}%
\pgfpathlineto{\pgfqpoint{1.590975in}{2.397601in}}%
\pgfpathlineto{\pgfqpoint{1.591172in}{2.404865in}}%
\pgfpathlineto{\pgfqpoint{1.592258in}{2.432217in}}%
\pgfpathlineto{\pgfqpoint{1.592455in}{2.427015in}}%
\pgfpathlineto{\pgfqpoint{1.592751in}{2.413871in}}%
\pgfpathlineto{\pgfqpoint{1.593639in}{2.420508in}}%
\pgfpathlineto{\pgfqpoint{1.594034in}{2.414219in}}%
\pgfpathlineto{\pgfqpoint{1.594330in}{2.422947in}}%
\pgfpathlineto{\pgfqpoint{1.594528in}{2.430041in}}%
\pgfpathlineto{\pgfqpoint{1.595021in}{2.409916in}}%
\pgfpathlineto{\pgfqpoint{1.595515in}{2.413542in}}%
\pgfpathlineto{\pgfqpoint{1.597291in}{2.386163in}}%
\pgfpathlineto{\pgfqpoint{1.597686in}{2.383187in}}%
\pgfpathlineto{\pgfqpoint{1.598870in}{2.363918in}}%
\pgfpathlineto{\pgfqpoint{1.598278in}{2.386693in}}%
\pgfpathlineto{\pgfqpoint{1.599167in}{2.369349in}}%
\pgfpathlineto{\pgfqpoint{1.599364in}{2.370273in}}%
\pgfpathlineto{\pgfqpoint{1.599759in}{2.366085in}}%
\pgfpathlineto{\pgfqpoint{1.600647in}{2.364166in}}%
\pgfpathlineto{\pgfqpoint{1.600351in}{2.367670in}}%
\pgfpathlineto{\pgfqpoint{1.600746in}{2.365394in}}%
\pgfpathlineto{\pgfqpoint{1.601042in}{2.372609in}}%
\pgfpathlineto{\pgfqpoint{1.601437in}{2.360925in}}%
\pgfpathlineto{\pgfqpoint{1.601831in}{2.365558in}}%
\pgfpathlineto{\pgfqpoint{1.602128in}{2.352254in}}%
\pgfpathlineto{\pgfqpoint{1.603016in}{2.360532in}}%
\pgfpathlineto{\pgfqpoint{1.606174in}{2.411284in}}%
\pgfpathlineto{\pgfqpoint{1.606766in}{2.408245in}}%
\pgfpathlineto{\pgfqpoint{1.607260in}{2.394698in}}%
\pgfpathlineto{\pgfqpoint{1.607457in}{2.392846in}}%
\pgfpathlineto{\pgfqpoint{1.608247in}{2.396887in}}%
\pgfpathlineto{\pgfqpoint{1.608444in}{2.391273in}}%
\pgfpathlineto{\pgfqpoint{1.609826in}{2.348544in}}%
\pgfpathlineto{\pgfqpoint{1.609925in}{2.348972in}}%
\pgfpathlineto{\pgfqpoint{1.610912in}{2.357527in}}%
\pgfpathlineto{\pgfqpoint{1.610320in}{2.347555in}}%
\pgfpathlineto{\pgfqpoint{1.611208in}{2.350759in}}%
\pgfpathlineto{\pgfqpoint{1.611504in}{2.342584in}}%
\pgfpathlineto{\pgfqpoint{1.611899in}{2.358693in}}%
\pgfpathlineto{\pgfqpoint{1.612195in}{2.354020in}}%
\pgfpathlineto{\pgfqpoint{1.612294in}{2.352794in}}%
\pgfpathlineto{\pgfqpoint{1.612787in}{2.359871in}}%
\pgfpathlineto{\pgfqpoint{1.613083in}{2.362857in}}%
\pgfpathlineto{\pgfqpoint{1.613379in}{2.352193in}}%
\pgfpathlineto{\pgfqpoint{1.613577in}{2.348805in}}%
\pgfpathlineto{\pgfqpoint{1.614070in}{2.360454in}}%
\pgfpathlineto{\pgfqpoint{1.615156in}{2.365140in}}%
\pgfpathlineto{\pgfqpoint{1.614662in}{2.357556in}}%
\pgfpathlineto{\pgfqpoint{1.615255in}{2.363971in}}%
\pgfpathlineto{\pgfqpoint{1.616242in}{2.330744in}}%
\pgfpathlineto{\pgfqpoint{1.616538in}{2.357054in}}%
\pgfpathlineto{\pgfqpoint{1.618215in}{2.622811in}}%
\pgfpathlineto{\pgfqpoint{1.618906in}{2.611019in}}%
\pgfpathlineto{\pgfqpoint{1.621078in}{2.392064in}}%
\pgfpathlineto{\pgfqpoint{1.621176in}{2.393491in}}%
\pgfpathlineto{\pgfqpoint{1.622854in}{2.488049in}}%
\pgfpathlineto{\pgfqpoint{1.623348in}{2.452931in}}%
\pgfpathlineto{\pgfqpoint{1.624137in}{2.392620in}}%
\pgfpathlineto{\pgfqpoint{1.624631in}{2.425474in}}%
\pgfpathlineto{\pgfqpoint{1.625223in}{2.416383in}}%
\pgfpathlineto{\pgfqpoint{1.625914in}{2.432679in}}%
\pgfpathlineto{\pgfqpoint{1.626210in}{2.438167in}}%
\pgfpathlineto{\pgfqpoint{1.626408in}{2.427594in}}%
\pgfpathlineto{\pgfqpoint{1.627394in}{2.324073in}}%
\pgfpathlineto{\pgfqpoint{1.628283in}{2.331300in}}%
\pgfpathlineto{\pgfqpoint{1.628579in}{2.328731in}}%
\pgfpathlineto{\pgfqpoint{1.628776in}{2.336097in}}%
\pgfpathlineto{\pgfqpoint{1.629665in}{2.380595in}}%
\pgfpathlineto{\pgfqpoint{1.630158in}{2.358630in}}%
\pgfpathlineto{\pgfqpoint{1.631737in}{2.379062in}}%
\pgfpathlineto{\pgfqpoint{1.631836in}{2.377498in}}%
\pgfpathlineto{\pgfqpoint{1.634600in}{2.318800in}}%
\pgfpathlineto{\pgfqpoint{1.634797in}{2.321660in}}%
\pgfpathlineto{\pgfqpoint{1.634896in}{2.322501in}}%
\pgfpathlineto{\pgfqpoint{1.635093in}{2.318399in}}%
\pgfpathlineto{\pgfqpoint{1.636968in}{2.269849in}}%
\pgfpathlineto{\pgfqpoint{1.637067in}{2.270982in}}%
\pgfpathlineto{\pgfqpoint{1.637560in}{2.278711in}}%
\pgfpathlineto{\pgfqpoint{1.637955in}{2.269981in}}%
\pgfpathlineto{\pgfqpoint{1.639238in}{2.264278in}}%
\pgfpathlineto{\pgfqpoint{1.640818in}{2.247035in}}%
\pgfpathlineto{\pgfqpoint{1.642002in}{2.232908in}}%
\pgfpathlineto{\pgfqpoint{1.641508in}{2.252536in}}%
\pgfpathlineto{\pgfqpoint{1.642199in}{2.235192in}}%
\pgfpathlineto{\pgfqpoint{1.642298in}{2.236247in}}%
\pgfpathlineto{\pgfqpoint{1.642594in}{2.229377in}}%
\pgfpathlineto{\pgfqpoint{1.643088in}{2.218573in}}%
\pgfpathlineto{\pgfqpoint{1.643779in}{2.224216in}}%
\pgfpathlineto{\pgfqpoint{1.643976in}{2.225489in}}%
\pgfpathlineto{\pgfqpoint{1.644272in}{2.217712in}}%
\pgfpathlineto{\pgfqpoint{1.645654in}{2.196745in}}%
\pgfpathlineto{\pgfqpoint{1.645950in}{2.204012in}}%
\pgfpathlineto{\pgfqpoint{1.646641in}{2.192571in}}%
\pgfpathlineto{\pgfqpoint{1.647134in}{2.208794in}}%
\pgfpathlineto{\pgfqpoint{1.647529in}{2.196326in}}%
\pgfpathlineto{\pgfqpoint{1.647924in}{2.214979in}}%
\pgfpathlineto{\pgfqpoint{1.648023in}{2.218069in}}%
\pgfpathlineto{\pgfqpoint{1.648615in}{2.210948in}}%
\pgfpathlineto{\pgfqpoint{1.648911in}{2.211634in}}%
\pgfpathlineto{\pgfqpoint{1.650490in}{2.259559in}}%
\pgfpathlineto{\pgfqpoint{1.652365in}{2.328841in}}%
\pgfpathlineto{\pgfqpoint{1.652563in}{2.328794in}}%
\pgfpathlineto{\pgfqpoint{1.653945in}{2.358361in}}%
\pgfpathlineto{\pgfqpoint{1.654241in}{2.351819in}}%
\pgfpathlineto{\pgfqpoint{1.656116in}{2.303818in}}%
\pgfpathlineto{\pgfqpoint{1.656708in}{2.308127in}}%
\pgfpathlineto{\pgfqpoint{1.657103in}{2.319182in}}%
\pgfpathlineto{\pgfqpoint{1.657399in}{2.302448in}}%
\pgfpathlineto{\pgfqpoint{1.657596in}{2.295764in}}%
\pgfpathlineto{\pgfqpoint{1.657991in}{2.310709in}}%
\pgfpathlineto{\pgfqpoint{1.658485in}{2.303517in}}%
\pgfpathlineto{\pgfqpoint{1.658583in}{2.303319in}}%
\pgfpathlineto{\pgfqpoint{1.660656in}{2.265520in}}%
\pgfpathlineto{\pgfqpoint{1.660755in}{2.267359in}}%
\pgfpathlineto{\pgfqpoint{1.661742in}{2.280850in}}%
\pgfpathlineto{\pgfqpoint{1.661939in}{2.273936in}}%
\pgfpathlineto{\pgfqpoint{1.662235in}{2.250063in}}%
\pgfpathlineto{\pgfqpoint{1.662729in}{2.293619in}}%
\pgfpathlineto{\pgfqpoint{1.664407in}{2.540674in}}%
\pgfpathlineto{\pgfqpoint{1.665196in}{2.493869in}}%
\pgfpathlineto{\pgfqpoint{1.665986in}{2.292739in}}%
\pgfpathlineto{\pgfqpoint{1.666973in}{2.317951in}}%
\pgfpathlineto{\pgfqpoint{1.669045in}{2.419451in}}%
\pgfpathlineto{\pgfqpoint{1.669638in}{2.376168in}}%
\pgfpathlineto{\pgfqpoint{1.670230in}{2.327821in}}%
\pgfpathlineto{\pgfqpoint{1.670822in}{2.357067in}}%
\pgfpathlineto{\pgfqpoint{1.672303in}{2.413027in}}%
\pgfpathlineto{\pgfqpoint{1.672796in}{2.393204in}}%
\pgfpathlineto{\pgfqpoint{1.673290in}{2.360599in}}%
\pgfpathlineto{\pgfqpoint{1.673882in}{2.393172in}}%
\pgfpathlineto{\pgfqpoint{1.675461in}{2.422094in}}%
\pgfpathlineto{\pgfqpoint{1.675560in}{2.422091in}}%
\pgfpathlineto{\pgfqpoint{1.676250in}{2.393711in}}%
\pgfpathlineto{\pgfqpoint{1.677139in}{2.403576in}}%
\pgfpathlineto{\pgfqpoint{1.678422in}{2.448050in}}%
\pgfpathlineto{\pgfqpoint{1.678817in}{2.444031in}}%
\pgfpathlineto{\pgfqpoint{1.679409in}{2.446572in}}%
\pgfpathlineto{\pgfqpoint{1.679804in}{2.432154in}}%
\pgfpathlineto{\pgfqpoint{1.679902in}{2.429644in}}%
\pgfpathlineto{\pgfqpoint{1.680198in}{2.447248in}}%
\pgfpathlineto{\pgfqpoint{1.680988in}{2.464751in}}%
\pgfpathlineto{\pgfqpoint{1.681482in}{2.460076in}}%
\pgfpathlineto{\pgfqpoint{1.681876in}{2.455619in}}%
\pgfpathlineto{\pgfqpoint{1.682074in}{2.462323in}}%
\pgfpathlineto{\pgfqpoint{1.682172in}{2.465987in}}%
\pgfpathlineto{\pgfqpoint{1.682666in}{2.445597in}}%
\pgfpathlineto{\pgfqpoint{1.683061in}{2.459129in}}%
\pgfpathlineto{\pgfqpoint{1.683159in}{2.459547in}}%
\pgfpathlineto{\pgfqpoint{1.683357in}{2.455556in}}%
\pgfpathlineto{\pgfqpoint{1.683455in}{2.454459in}}%
\pgfpathlineto{\pgfqpoint{1.683752in}{2.462813in}}%
\pgfpathlineto{\pgfqpoint{1.684048in}{2.474712in}}%
\pgfpathlineto{\pgfqpoint{1.684442in}{2.461351in}}%
\pgfpathlineto{\pgfqpoint{1.684936in}{2.471048in}}%
\pgfpathlineto{\pgfqpoint{1.685232in}{2.463213in}}%
\pgfpathlineto{\pgfqpoint{1.685726in}{2.478245in}}%
\pgfpathlineto{\pgfqpoint{1.685824in}{2.477886in}}%
\pgfpathlineto{\pgfqpoint{1.686614in}{2.475453in}}%
\pgfpathlineto{\pgfqpoint{1.686318in}{2.479459in}}%
\pgfpathlineto{\pgfqpoint{1.686811in}{2.478412in}}%
\pgfpathlineto{\pgfqpoint{1.687897in}{2.505149in}}%
\pgfpathlineto{\pgfqpoint{1.688094in}{2.496887in}}%
\pgfpathlineto{\pgfqpoint{1.688292in}{2.485847in}}%
\pgfpathlineto{\pgfqpoint{1.689180in}{2.489838in}}%
\pgfpathlineto{\pgfqpoint{1.690562in}{2.506216in}}%
\pgfpathlineto{\pgfqpoint{1.690759in}{2.503260in}}%
\pgfpathlineto{\pgfqpoint{1.691055in}{2.490375in}}%
\pgfpathlineto{\pgfqpoint{1.691549in}{2.507055in}}%
\pgfpathlineto{\pgfqpoint{1.691746in}{2.505455in}}%
\pgfpathlineto{\pgfqpoint{1.691845in}{2.504966in}}%
\pgfpathlineto{\pgfqpoint{1.692042in}{2.507920in}}%
\pgfpathlineto{\pgfqpoint{1.694214in}{2.555240in}}%
\pgfpathlineto{\pgfqpoint{1.694510in}{2.544680in}}%
\pgfpathlineto{\pgfqpoint{1.694608in}{2.542407in}}%
\pgfpathlineto{\pgfqpoint{1.695003in}{2.558560in}}%
\pgfpathlineto{\pgfqpoint{1.695102in}{2.560248in}}%
\pgfpathlineto{\pgfqpoint{1.695892in}{2.555935in}}%
\pgfpathlineto{\pgfqpoint{1.696089in}{2.557312in}}%
\pgfpathlineto{\pgfqpoint{1.698260in}{2.593167in}}%
\pgfpathlineto{\pgfqpoint{1.698556in}{2.587512in}}%
\pgfpathlineto{\pgfqpoint{1.698655in}{2.586597in}}%
\pgfpathlineto{\pgfqpoint{1.698951in}{2.593720in}}%
\pgfpathlineto{\pgfqpoint{1.699050in}{2.594062in}}%
\pgfpathlineto{\pgfqpoint{1.699149in}{2.592448in}}%
\pgfpathlineto{\pgfqpoint{1.705071in}{2.403273in}}%
\pgfpathlineto{\pgfqpoint{1.705367in}{2.404169in}}%
\pgfpathlineto{\pgfqpoint{1.705465in}{2.403975in}}%
\pgfpathlineto{\pgfqpoint{1.705564in}{2.405119in}}%
\pgfpathlineto{\pgfqpoint{1.707637in}{2.451417in}}%
\pgfpathlineto{\pgfqpoint{1.707834in}{2.444944in}}%
\pgfpathlineto{\pgfqpoint{1.708525in}{2.415554in}}%
\pgfpathlineto{\pgfqpoint{1.708821in}{2.434863in}}%
\pgfpathlineto{\pgfqpoint{1.710302in}{2.666614in}}%
\pgfpathlineto{\pgfqpoint{1.711190in}{2.637772in}}%
\pgfpathlineto{\pgfqpoint{1.712078in}{2.409702in}}%
\pgfpathlineto{\pgfqpoint{1.713361in}{2.416218in}}%
\pgfpathlineto{\pgfqpoint{1.715138in}{2.554148in}}%
\pgfpathlineto{\pgfqpoint{1.715631in}{2.517287in}}%
\pgfpathlineto{\pgfqpoint{1.716224in}{2.458515in}}%
\pgfpathlineto{\pgfqpoint{1.716816in}{2.485830in}}%
\pgfpathlineto{\pgfqpoint{1.718494in}{2.544717in}}%
\pgfpathlineto{\pgfqpoint{1.718592in}{2.543777in}}%
\pgfpathlineto{\pgfqpoint{1.719678in}{2.502874in}}%
\pgfpathlineto{\pgfqpoint{1.720073in}{2.522358in}}%
\pgfpathlineto{\pgfqpoint{1.721553in}{2.561419in}}%
\pgfpathlineto{\pgfqpoint{1.721652in}{2.560241in}}%
\pgfpathlineto{\pgfqpoint{1.722935in}{2.542142in}}%
\pgfpathlineto{\pgfqpoint{1.723034in}{2.544505in}}%
\pgfpathlineto{\pgfqpoint{1.724514in}{2.582238in}}%
\pgfpathlineto{\pgfqpoint{1.724810in}{2.581216in}}%
\pgfpathlineto{\pgfqpoint{1.725106in}{2.586920in}}%
\pgfpathlineto{\pgfqpoint{1.725403in}{2.572680in}}%
\pgfpathlineto{\pgfqpoint{1.726291in}{2.558479in}}%
\pgfpathlineto{\pgfqpoint{1.726488in}{2.565548in}}%
\pgfpathlineto{\pgfqpoint{1.726784in}{2.576470in}}%
\pgfpathlineto{\pgfqpoint{1.727574in}{2.569828in}}%
\pgfpathlineto{\pgfqpoint{1.729647in}{2.541789in}}%
\pgfpathlineto{\pgfqpoint{1.729745in}{2.543052in}}%
\pgfpathlineto{\pgfqpoint{1.730140in}{2.558079in}}%
\pgfpathlineto{\pgfqpoint{1.730535in}{2.534129in}}%
\pgfpathlineto{\pgfqpoint{1.731423in}{2.531741in}}%
\pgfpathlineto{\pgfqpoint{1.730930in}{2.541807in}}%
\pgfpathlineto{\pgfqpoint{1.731522in}{2.535347in}}%
\pgfpathlineto{\pgfqpoint{1.731818in}{2.546473in}}%
\pgfpathlineto{\pgfqpoint{1.732509in}{2.531632in}}%
\pgfpathlineto{\pgfqpoint{1.733200in}{2.527287in}}%
\pgfpathlineto{\pgfqpoint{1.735174in}{2.476662in}}%
\pgfpathlineto{\pgfqpoint{1.735470in}{2.478615in}}%
\pgfpathlineto{\pgfqpoint{1.735569in}{2.478970in}}%
\pgfpathlineto{\pgfqpoint{1.735667in}{2.477240in}}%
\pgfpathlineto{\pgfqpoint{1.737839in}{2.443900in}}%
\pgfpathlineto{\pgfqpoint{1.737937in}{2.443927in}}%
\pgfpathlineto{\pgfqpoint{1.738332in}{2.451570in}}%
\pgfpathlineto{\pgfqpoint{1.738826in}{2.442353in}}%
\pgfpathlineto{\pgfqpoint{1.739023in}{2.443839in}}%
\pgfpathlineto{\pgfqpoint{1.739122in}{2.443943in}}%
\pgfpathlineto{\pgfqpoint{1.739319in}{2.442822in}}%
\pgfpathlineto{\pgfqpoint{1.739714in}{2.439328in}}%
\pgfpathlineto{\pgfqpoint{1.740207in}{2.444099in}}%
\pgfpathlineto{\pgfqpoint{1.740602in}{2.452221in}}%
\pgfpathlineto{\pgfqpoint{1.740997in}{2.437512in}}%
\pgfpathlineto{\pgfqpoint{1.741490in}{2.429720in}}%
\pgfpathlineto{\pgfqpoint{1.741885in}{2.438141in}}%
\pgfpathlineto{\pgfqpoint{1.742774in}{2.464207in}}%
\pgfpathlineto{\pgfqpoint{1.743464in}{2.463782in}}%
\pgfpathlineto{\pgfqpoint{1.744945in}{2.501331in}}%
\pgfpathlineto{\pgfqpoint{1.745438in}{2.486102in}}%
\pgfpathlineto{\pgfqpoint{1.745932in}{2.482177in}}%
\pgfpathlineto{\pgfqpoint{1.746129in}{2.487822in}}%
\pgfpathlineto{\pgfqpoint{1.746327in}{2.490430in}}%
\pgfpathlineto{\pgfqpoint{1.746623in}{2.476695in}}%
\pgfpathlineto{\pgfqpoint{1.748202in}{2.448578in}}%
\pgfpathlineto{\pgfqpoint{1.748399in}{2.451820in}}%
\pgfpathlineto{\pgfqpoint{1.748597in}{2.454113in}}%
\pgfpathlineto{\pgfqpoint{1.749090in}{2.447197in}}%
\pgfpathlineto{\pgfqpoint{1.749288in}{2.443525in}}%
\pgfpathlineto{\pgfqpoint{1.749781in}{2.456160in}}%
\pgfpathlineto{\pgfqpoint{1.750176in}{2.444971in}}%
\pgfpathlineto{\pgfqpoint{1.750472in}{2.454605in}}%
\pgfpathlineto{\pgfqpoint{1.750867in}{2.437237in}}%
\pgfpathlineto{\pgfqpoint{1.752150in}{2.424436in}}%
\pgfpathlineto{\pgfqpoint{1.751360in}{2.442810in}}%
\pgfpathlineto{\pgfqpoint{1.752347in}{2.426499in}}%
\pgfpathlineto{\pgfqpoint{1.753532in}{2.438709in}}%
\pgfpathlineto{\pgfqpoint{1.753729in}{2.434183in}}%
\pgfpathlineto{\pgfqpoint{1.754519in}{2.410293in}}%
\pgfpathlineto{\pgfqpoint{1.754716in}{2.425730in}}%
\pgfpathlineto{\pgfqpoint{1.756690in}{2.687012in}}%
\pgfpathlineto{\pgfqpoint{1.757085in}{2.656227in}}%
\pgfpathlineto{\pgfqpoint{1.758960in}{2.422819in}}%
\pgfpathlineto{\pgfqpoint{1.759256in}{2.431234in}}%
\pgfpathlineto{\pgfqpoint{1.759750in}{2.421524in}}%
\pgfpathlineto{\pgfqpoint{1.760243in}{2.431692in}}%
\pgfpathlineto{\pgfqpoint{1.761033in}{2.495091in}}%
\pgfpathlineto{\pgfqpoint{1.761526in}{2.463280in}}%
\pgfpathlineto{\pgfqpoint{1.762217in}{2.391592in}}%
\pgfpathlineto{\pgfqpoint{1.762908in}{2.425729in}}%
\pgfpathlineto{\pgfqpoint{1.764290in}{2.480349in}}%
\pgfpathlineto{\pgfqpoint{1.764685in}{2.461972in}}%
\pgfpathlineto{\pgfqpoint{1.765474in}{2.420640in}}%
\pgfpathlineto{\pgfqpoint{1.765968in}{2.443766in}}%
\pgfpathlineto{\pgfqpoint{1.766264in}{2.449027in}}%
\pgfpathlineto{\pgfqpoint{1.767547in}{2.506068in}}%
\pgfpathlineto{\pgfqpoint{1.767942in}{2.485507in}}%
\pgfpathlineto{\pgfqpoint{1.768040in}{2.482526in}}%
\pgfpathlineto{\pgfqpoint{1.768435in}{2.494403in}}%
\pgfpathlineto{\pgfqpoint{1.768731in}{2.492634in}}%
\pgfpathlineto{\pgfqpoint{1.770903in}{2.553253in}}%
\pgfpathlineto{\pgfqpoint{1.771396in}{2.534707in}}%
\pgfpathlineto{\pgfqpoint{1.771594in}{2.531765in}}%
\pgfpathlineto{\pgfqpoint{1.772186in}{2.539338in}}%
\pgfpathlineto{\pgfqpoint{1.772482in}{2.544962in}}%
\pgfpathlineto{\pgfqpoint{1.773272in}{2.540288in}}%
\pgfpathlineto{\pgfqpoint{1.777516in}{2.340424in}}%
\pgfpathlineto{\pgfqpoint{1.777713in}{2.340754in}}%
\pgfpathlineto{\pgfqpoint{1.777812in}{2.339274in}}%
\pgfpathlineto{\pgfqpoint{1.782451in}{2.165859in}}%
\pgfpathlineto{\pgfqpoint{1.782747in}{2.181580in}}%
\pgfpathlineto{\pgfqpoint{1.785214in}{2.292979in}}%
\pgfpathlineto{\pgfqpoint{1.785313in}{2.289764in}}%
\pgfpathlineto{\pgfqpoint{1.785510in}{2.281965in}}%
\pgfpathlineto{\pgfqpoint{1.786004in}{2.305138in}}%
\pgfpathlineto{\pgfqpoint{1.787089in}{2.339262in}}%
\pgfpathlineto{\pgfqpoint{1.790445in}{2.478393in}}%
\pgfpathlineto{\pgfqpoint{1.790544in}{2.477891in}}%
\pgfpathlineto{\pgfqpoint{1.790643in}{2.476616in}}%
\pgfpathlineto{\pgfqpoint{1.790840in}{2.481472in}}%
\pgfpathlineto{\pgfqpoint{1.791926in}{2.530088in}}%
\pgfpathlineto{\pgfqpoint{1.792518in}{2.529326in}}%
\pgfpathlineto{\pgfqpoint{1.792617in}{2.529308in}}%
\pgfpathlineto{\pgfqpoint{1.793110in}{2.517773in}}%
\pgfpathlineto{\pgfqpoint{1.793801in}{2.526413in}}%
\pgfpathlineto{\pgfqpoint{1.795874in}{2.580928in}}%
\pgfpathlineto{\pgfqpoint{1.796663in}{2.576849in}}%
\pgfpathlineto{\pgfqpoint{1.797650in}{2.573658in}}%
\pgfpathlineto{\pgfqpoint{1.797255in}{2.584467in}}%
\pgfpathlineto{\pgfqpoint{1.797749in}{2.574108in}}%
\pgfpathlineto{\pgfqpoint{1.798045in}{2.578906in}}%
\pgfpathlineto{\pgfqpoint{1.798440in}{2.570765in}}%
\pgfpathlineto{\pgfqpoint{1.798835in}{2.577292in}}%
\pgfpathlineto{\pgfqpoint{1.800315in}{2.550220in}}%
\pgfpathlineto{\pgfqpoint{1.800414in}{2.551993in}}%
\pgfpathlineto{\pgfqpoint{1.801006in}{2.671646in}}%
\pgfpathlineto{\pgfqpoint{1.802190in}{2.821728in}}%
\pgfpathlineto{\pgfqpoint{1.802585in}{2.801617in}}%
\pgfpathlineto{\pgfqpoint{1.803177in}{2.717079in}}%
\pgfpathlineto{\pgfqpoint{1.803770in}{2.553419in}}%
\pgfpathlineto{\pgfqpoint{1.804658in}{2.571056in}}%
\pgfpathlineto{\pgfqpoint{1.805151in}{2.580017in}}%
\pgfpathlineto{\pgfqpoint{1.806928in}{2.687233in}}%
\pgfpathlineto{\pgfqpoint{1.807323in}{2.659187in}}%
\pgfpathlineto{\pgfqpoint{1.807915in}{2.602076in}}%
\pgfpathlineto{\pgfqpoint{1.808606in}{2.616884in}}%
\pgfpathlineto{\pgfqpoint{1.810185in}{2.696478in}}%
\pgfpathlineto{\pgfqpoint{1.810382in}{2.687824in}}%
\pgfpathlineto{\pgfqpoint{1.811172in}{2.645912in}}%
\pgfpathlineto{\pgfqpoint{1.811665in}{2.658951in}}%
\pgfpathlineto{\pgfqpoint{1.813146in}{2.700088in}}%
\pgfpathlineto{\pgfqpoint{1.813541in}{2.687323in}}%
\pgfpathlineto{\pgfqpoint{1.814429in}{2.668640in}}%
\pgfpathlineto{\pgfqpoint{1.814725in}{2.676779in}}%
\pgfpathlineto{\pgfqpoint{1.816008in}{2.719240in}}%
\pgfpathlineto{\pgfqpoint{1.816206in}{2.716465in}}%
\pgfpathlineto{\pgfqpoint{1.817587in}{2.690392in}}%
\pgfpathlineto{\pgfqpoint{1.817785in}{2.695645in}}%
\pgfpathlineto{\pgfqpoint{1.818673in}{2.735559in}}%
\pgfpathlineto{\pgfqpoint{1.819364in}{2.722114in}}%
\pgfpathlineto{\pgfqpoint{1.819956in}{2.727503in}}%
\pgfpathlineto{\pgfqpoint{1.820351in}{2.722857in}}%
\pgfpathlineto{\pgfqpoint{1.821239in}{2.710666in}}%
\pgfpathlineto{\pgfqpoint{1.821535in}{2.720859in}}%
\pgfpathlineto{\pgfqpoint{1.821733in}{2.725566in}}%
\pgfpathlineto{\pgfqpoint{1.822128in}{2.720213in}}%
\pgfpathlineto{\pgfqpoint{1.822522in}{2.720852in}}%
\pgfpathlineto{\pgfqpoint{1.823608in}{2.703450in}}%
\pgfpathlineto{\pgfqpoint{1.824101in}{2.708643in}}%
\pgfpathlineto{\pgfqpoint{1.824398in}{2.714298in}}%
\pgfpathlineto{\pgfqpoint{1.824891in}{2.701772in}}%
\pgfpathlineto{\pgfqpoint{1.825483in}{2.709773in}}%
\pgfpathlineto{\pgfqpoint{1.826865in}{2.678801in}}%
\pgfpathlineto{\pgfqpoint{1.827260in}{2.679106in}}%
\pgfpathlineto{\pgfqpoint{1.829135in}{2.658685in}}%
\pgfpathlineto{\pgfqpoint{1.829431in}{2.660057in}}%
\pgfpathlineto{\pgfqpoint{1.829530in}{2.660448in}}%
\pgfpathlineto{\pgfqpoint{1.829727in}{2.658290in}}%
\pgfpathlineto{\pgfqpoint{1.830122in}{2.650366in}}%
\pgfpathlineto{\pgfqpoint{1.830616in}{2.660873in}}%
\pgfpathlineto{\pgfqpoint{1.831010in}{2.652696in}}%
\pgfpathlineto{\pgfqpoint{1.831405in}{2.651592in}}%
\pgfpathlineto{\pgfqpoint{1.831701in}{2.654034in}}%
\pgfpathlineto{\pgfqpoint{1.831997in}{2.662093in}}%
\pgfpathlineto{\pgfqpoint{1.832392in}{2.640532in}}%
\pgfpathlineto{\pgfqpoint{1.832491in}{2.638597in}}%
\pgfpathlineto{\pgfqpoint{1.832688in}{2.649087in}}%
\pgfpathlineto{\pgfqpoint{1.833873in}{2.664555in}}%
\pgfpathlineto{\pgfqpoint{1.833379in}{2.646130in}}%
\pgfpathlineto{\pgfqpoint{1.833971in}{2.663716in}}%
\pgfpathlineto{\pgfqpoint{1.834070in}{2.662264in}}%
\pgfpathlineto{\pgfqpoint{1.834366in}{2.671454in}}%
\pgfpathlineto{\pgfqpoint{1.834958in}{2.667871in}}%
\pgfpathlineto{\pgfqpoint{1.835945in}{2.693364in}}%
\pgfpathlineto{\pgfqpoint{1.836241in}{2.685821in}}%
\pgfpathlineto{\pgfqpoint{1.836439in}{2.679937in}}%
\pgfpathlineto{\pgfqpoint{1.836932in}{2.691110in}}%
\pgfpathlineto{\pgfqpoint{1.837327in}{2.685060in}}%
\pgfpathlineto{\pgfqpoint{1.837722in}{2.688431in}}%
\pgfpathlineto{\pgfqpoint{1.837919in}{2.685759in}}%
\pgfpathlineto{\pgfqpoint{1.840683in}{2.614622in}}%
\pgfpathlineto{\pgfqpoint{1.841176in}{2.605307in}}%
\pgfpathlineto{\pgfqpoint{1.841473in}{2.616524in}}%
\pgfpathlineto{\pgfqpoint{1.841670in}{2.622393in}}%
\pgfpathlineto{\pgfqpoint{1.842262in}{2.609616in}}%
\pgfpathlineto{\pgfqpoint{1.842361in}{2.610012in}}%
\pgfpathlineto{\pgfqpoint{1.842756in}{2.614092in}}%
\pgfpathlineto{\pgfqpoint{1.843150in}{2.609748in}}%
\pgfpathlineto{\pgfqpoint{1.844532in}{2.600533in}}%
\pgfpathlineto{\pgfqpoint{1.844631in}{2.601394in}}%
\pgfpathlineto{\pgfqpoint{1.845026in}{2.612198in}}%
\pgfpathlineto{\pgfqpoint{1.845618in}{2.598509in}}%
\pgfpathlineto{\pgfqpoint{1.846013in}{2.586341in}}%
\pgfpathlineto{\pgfqpoint{1.846210in}{2.598253in}}%
\pgfpathlineto{\pgfqpoint{1.848085in}{2.865576in}}%
\pgfpathlineto{\pgfqpoint{1.848579in}{2.836399in}}%
\pgfpathlineto{\pgfqpoint{1.850553in}{2.582102in}}%
\pgfpathlineto{\pgfqpoint{1.850849in}{2.602895in}}%
\pgfpathlineto{\pgfqpoint{1.852527in}{2.694257in}}%
\pgfpathlineto{\pgfqpoint{1.852823in}{2.669803in}}%
\pgfpathlineto{\pgfqpoint{1.854599in}{2.557121in}}%
\pgfpathlineto{\pgfqpoint{1.855093in}{2.543851in}}%
\pgfpathlineto{\pgfqpoint{1.855488in}{2.558582in}}%
\pgfpathlineto{\pgfqpoint{1.855586in}{2.560429in}}%
\pgfpathlineto{\pgfqpoint{1.855883in}{2.546655in}}%
\pgfpathlineto{\pgfqpoint{1.858942in}{2.417730in}}%
\pgfpathlineto{\pgfqpoint{1.859732in}{2.361862in}}%
\pgfpathlineto{\pgfqpoint{1.861311in}{2.312726in}}%
\pgfpathlineto{\pgfqpoint{1.862594in}{2.281221in}}%
\pgfpathlineto{\pgfqpoint{1.862989in}{2.288477in}}%
\pgfpathlineto{\pgfqpoint{1.863482in}{2.312617in}}%
\pgfpathlineto{\pgfqpoint{1.865062in}{2.395259in}}%
\pgfpathlineto{\pgfqpoint{1.865358in}{2.390864in}}%
\pgfpathlineto{\pgfqpoint{1.865752in}{2.407579in}}%
\pgfpathlineto{\pgfqpoint{1.866147in}{2.386878in}}%
\pgfpathlineto{\pgfqpoint{1.866542in}{2.395261in}}%
\pgfpathlineto{\pgfqpoint{1.866838in}{2.388284in}}%
\pgfpathlineto{\pgfqpoint{1.867134in}{2.400295in}}%
\pgfpathlineto{\pgfqpoint{1.868713in}{2.436643in}}%
\pgfpathlineto{\pgfqpoint{1.868812in}{2.436960in}}%
\pgfpathlineto{\pgfqpoint{1.869207in}{2.421776in}}%
\pgfpathlineto{\pgfqpoint{1.869700in}{2.439277in}}%
\pgfpathlineto{\pgfqpoint{1.869997in}{2.433870in}}%
\pgfpathlineto{\pgfqpoint{1.870194in}{2.428460in}}%
\pgfpathlineto{\pgfqpoint{1.870885in}{2.437113in}}%
\pgfpathlineto{\pgfqpoint{1.871872in}{2.447821in}}%
\pgfpathlineto{\pgfqpoint{1.872069in}{2.443173in}}%
\pgfpathlineto{\pgfqpoint{1.872957in}{2.425649in}}%
\pgfpathlineto{\pgfqpoint{1.873254in}{2.436601in}}%
\pgfpathlineto{\pgfqpoint{1.873352in}{2.438462in}}%
\pgfpathlineto{\pgfqpoint{1.874043in}{2.431317in}}%
\pgfpathlineto{\pgfqpoint{1.874241in}{2.428230in}}%
\pgfpathlineto{\pgfqpoint{1.874635in}{2.435315in}}%
\pgfpathlineto{\pgfqpoint{1.875228in}{2.428924in}}%
\pgfpathlineto{\pgfqpoint{1.875622in}{2.432006in}}%
\pgfpathlineto{\pgfqpoint{1.875721in}{2.433138in}}%
\pgfpathlineto{\pgfqpoint{1.876017in}{2.425889in}}%
\pgfpathlineto{\pgfqpoint{1.876116in}{2.422663in}}%
\pgfpathlineto{\pgfqpoint{1.876609in}{2.441202in}}%
\pgfpathlineto{\pgfqpoint{1.876708in}{2.441366in}}%
\pgfpathlineto{\pgfqpoint{1.877300in}{2.430339in}}%
\pgfpathlineto{\pgfqpoint{1.877892in}{2.437430in}}%
\pgfpathlineto{\pgfqpoint{1.879472in}{2.467426in}}%
\pgfpathlineto{\pgfqpoint{1.878682in}{2.433834in}}%
\pgfpathlineto{\pgfqpoint{1.879669in}{2.461652in}}%
\pgfpathlineto{\pgfqpoint{1.879866in}{2.457666in}}%
\pgfpathlineto{\pgfqpoint{1.880162in}{2.477500in}}%
\pgfpathlineto{\pgfqpoint{1.881643in}{2.508436in}}%
\pgfpathlineto{\pgfqpoint{1.881742in}{2.509005in}}%
\pgfpathlineto{\pgfqpoint{1.881840in}{2.506508in}}%
\pgfpathlineto{\pgfqpoint{1.885097in}{2.409253in}}%
\pgfpathlineto{\pgfqpoint{1.885295in}{2.410657in}}%
\pgfpathlineto{\pgfqpoint{1.885788in}{2.400133in}}%
\pgfpathlineto{\pgfqpoint{1.886381in}{2.385334in}}%
\pgfpathlineto{\pgfqpoint{1.887071in}{2.394989in}}%
\pgfpathlineto{\pgfqpoint{1.887170in}{2.396377in}}%
\pgfpathlineto{\pgfqpoint{1.887664in}{2.388570in}}%
\pgfpathlineto{\pgfqpoint{1.887861in}{2.386950in}}%
\pgfpathlineto{\pgfqpoint{1.888947in}{2.369303in}}%
\pgfpathlineto{\pgfqpoint{1.889144in}{2.373962in}}%
\pgfpathlineto{\pgfqpoint{1.889440in}{2.382644in}}%
\pgfpathlineto{\pgfqpoint{1.890131in}{2.368718in}}%
\pgfpathlineto{\pgfqpoint{1.890230in}{2.367127in}}%
\pgfpathlineto{\pgfqpoint{1.890526in}{2.375506in}}%
\pgfpathlineto{\pgfqpoint{1.890723in}{2.384276in}}%
\pgfpathlineto{\pgfqpoint{1.891118in}{2.360630in}}%
\pgfpathlineto{\pgfqpoint{1.891414in}{2.340726in}}%
\pgfpathlineto{\pgfqpoint{1.891908in}{2.384036in}}%
\pgfpathlineto{\pgfqpoint{1.893191in}{2.604448in}}%
\pgfpathlineto{\pgfqpoint{1.894079in}{2.593485in}}%
\pgfpathlineto{\pgfqpoint{1.896053in}{2.324578in}}%
\pgfpathlineto{\pgfqpoint{1.896448in}{2.343282in}}%
\pgfpathlineto{\pgfqpoint{1.898126in}{2.439475in}}%
\pgfpathlineto{\pgfqpoint{1.898619in}{2.391309in}}%
\pgfpathlineto{\pgfqpoint{1.899310in}{2.338599in}}%
\pgfpathlineto{\pgfqpoint{1.899804in}{2.365068in}}%
\pgfpathlineto{\pgfqpoint{1.901383in}{2.418829in}}%
\pgfpathlineto{\pgfqpoint{1.901580in}{2.409926in}}%
\pgfpathlineto{\pgfqpoint{1.902468in}{2.358183in}}%
\pgfpathlineto{\pgfqpoint{1.903159in}{2.371507in}}%
\pgfpathlineto{\pgfqpoint{1.904640in}{2.412928in}}%
\pgfpathlineto{\pgfqpoint{1.905035in}{2.401811in}}%
\pgfpathlineto{\pgfqpoint{1.905824in}{2.379566in}}%
\pgfpathlineto{\pgfqpoint{1.906120in}{2.394039in}}%
\pgfpathlineto{\pgfqpoint{1.906416in}{2.407191in}}%
\pgfpathlineto{\pgfqpoint{1.907206in}{2.397784in}}%
\pgfpathlineto{\pgfqpoint{1.907403in}{2.400843in}}%
\pgfpathlineto{\pgfqpoint{1.907601in}{2.406769in}}%
\pgfpathlineto{\pgfqpoint{1.908193in}{2.392041in}}%
\pgfpathlineto{\pgfqpoint{1.908390in}{2.396691in}}%
\pgfpathlineto{\pgfqpoint{1.908588in}{2.400673in}}%
\pgfpathlineto{\pgfqpoint{1.909081in}{2.386685in}}%
\pgfpathlineto{\pgfqpoint{1.909180in}{2.386704in}}%
\pgfpathlineto{\pgfqpoint{1.909673in}{2.407758in}}%
\pgfpathlineto{\pgfqpoint{1.910957in}{2.405871in}}%
\pgfpathlineto{\pgfqpoint{1.911154in}{2.407121in}}%
\pgfpathlineto{\pgfqpoint{1.911351in}{2.401188in}}%
\pgfpathlineto{\pgfqpoint{1.911845in}{2.388038in}}%
\pgfpathlineto{\pgfqpoint{1.912536in}{2.396660in}}%
\pgfpathlineto{\pgfqpoint{1.912832in}{2.407085in}}%
\pgfpathlineto{\pgfqpoint{1.913523in}{2.396647in}}%
\pgfpathlineto{\pgfqpoint{1.917372in}{2.348737in}}%
\pgfpathlineto{\pgfqpoint{1.918458in}{2.326036in}}%
\pgfpathlineto{\pgfqpoint{1.918655in}{2.326605in}}%
\pgfpathlineto{\pgfqpoint{1.922011in}{2.288245in}}%
\pgfpathlineto{\pgfqpoint{1.922504in}{2.295880in}}%
\pgfpathlineto{\pgfqpoint{1.922702in}{2.296562in}}%
\pgfpathlineto{\pgfqpoint{1.922899in}{2.292436in}}%
\pgfpathlineto{\pgfqpoint{1.923195in}{2.280195in}}%
\pgfpathlineto{\pgfqpoint{1.923590in}{2.295079in}}%
\pgfpathlineto{\pgfqpoint{1.924084in}{2.287790in}}%
\pgfpathlineto{\pgfqpoint{1.926946in}{2.347045in}}%
\pgfpathlineto{\pgfqpoint{1.927242in}{2.336011in}}%
\pgfpathlineto{\pgfqpoint{1.927439in}{2.328982in}}%
\pgfpathlineto{\pgfqpoint{1.927933in}{2.338498in}}%
\pgfpathlineto{\pgfqpoint{1.928328in}{2.333848in}}%
\pgfpathlineto{\pgfqpoint{1.929315in}{2.312384in}}%
\pgfpathlineto{\pgfqpoint{1.931091in}{2.286329in}}%
\pgfpathlineto{\pgfqpoint{1.932966in}{2.259842in}}%
\pgfpathlineto{\pgfqpoint{1.933065in}{2.261736in}}%
\pgfpathlineto{\pgfqpoint{1.933361in}{2.275543in}}%
\pgfpathlineto{\pgfqpoint{1.934151in}{2.263033in}}%
\pgfpathlineto{\pgfqpoint{1.934546in}{2.281203in}}%
\pgfpathlineto{\pgfqpoint{1.934940in}{2.259198in}}%
\pgfpathlineto{\pgfqpoint{1.935039in}{2.257134in}}%
\pgfpathlineto{\pgfqpoint{1.935335in}{2.271793in}}%
\pgfpathlineto{\pgfqpoint{1.935434in}{2.275231in}}%
\pgfpathlineto{\pgfqpoint{1.935927in}{2.266572in}}%
\pgfpathlineto{\pgfqpoint{1.936322in}{2.269059in}}%
\pgfpathlineto{\pgfqpoint{1.937112in}{2.245316in}}%
\pgfpathlineto{\pgfqpoint{1.937408in}{2.267184in}}%
\pgfpathlineto{\pgfqpoint{1.939086in}{2.526087in}}%
\pgfpathlineto{\pgfqpoint{1.939678in}{2.518247in}}%
\pgfpathlineto{\pgfqpoint{1.940665in}{2.279022in}}%
\pgfpathlineto{\pgfqpoint{1.942244in}{2.313272in}}%
\pgfpathlineto{\pgfqpoint{1.942639in}{2.306886in}}%
\pgfpathlineto{\pgfqpoint{1.943231in}{2.332867in}}%
\pgfpathlineto{\pgfqpoint{1.943725in}{2.368629in}}%
\pgfpathlineto{\pgfqpoint{1.944119in}{2.329373in}}%
\pgfpathlineto{\pgfqpoint{1.944810in}{2.251938in}}%
\pgfpathlineto{\pgfqpoint{1.945403in}{2.297519in}}%
\pgfpathlineto{\pgfqpoint{1.946982in}{2.417232in}}%
\pgfpathlineto{\pgfqpoint{1.947475in}{2.393465in}}%
\pgfpathlineto{\pgfqpoint{1.948265in}{2.356411in}}%
\pgfpathlineto{\pgfqpoint{1.948660in}{2.385059in}}%
\pgfpathlineto{\pgfqpoint{1.949153in}{2.377841in}}%
\pgfpathlineto{\pgfqpoint{1.950337in}{2.408164in}}%
\pgfpathlineto{\pgfqpoint{1.950436in}{2.409427in}}%
\pgfpathlineto{\pgfqpoint{1.950732in}{2.401911in}}%
\pgfpathlineto{\pgfqpoint{1.951226in}{2.377572in}}%
\pgfpathlineto{\pgfqpoint{1.951917in}{2.393730in}}%
\pgfpathlineto{\pgfqpoint{1.952311in}{2.410432in}}%
\pgfpathlineto{\pgfqpoint{1.953101in}{2.399160in}}%
\pgfpathlineto{\pgfqpoint{1.953200in}{2.399903in}}%
\pgfpathlineto{\pgfqpoint{1.953693in}{2.397692in}}%
\pgfpathlineto{\pgfqpoint{1.954285in}{2.384358in}}%
\pgfpathlineto{\pgfqpoint{1.954976in}{2.391451in}}%
\pgfpathlineto{\pgfqpoint{1.955174in}{2.393099in}}%
\pgfpathlineto{\pgfqpoint{1.955470in}{2.388479in}}%
\pgfpathlineto{\pgfqpoint{1.957345in}{2.361588in}}%
\pgfpathlineto{\pgfqpoint{1.957740in}{2.367253in}}%
\pgfpathlineto{\pgfqpoint{1.958036in}{2.360511in}}%
\pgfpathlineto{\pgfqpoint{1.959122in}{2.345888in}}%
\pgfpathlineto{\pgfqpoint{1.958628in}{2.361611in}}%
\pgfpathlineto{\pgfqpoint{1.959319in}{2.352508in}}%
\pgfpathlineto{\pgfqpoint{1.959516in}{2.360650in}}%
\pgfpathlineto{\pgfqpoint{1.959911in}{2.347815in}}%
\pgfpathlineto{\pgfqpoint{1.960405in}{2.355546in}}%
\pgfpathlineto{\pgfqpoint{1.961885in}{2.330475in}}%
\pgfpathlineto{\pgfqpoint{1.962181in}{2.338049in}}%
\pgfpathlineto{\pgfqpoint{1.962477in}{2.324573in}}%
\pgfpathlineto{\pgfqpoint{1.963662in}{2.275603in}}%
\pgfpathlineto{\pgfqpoint{1.963958in}{2.283676in}}%
\pgfpathlineto{\pgfqpoint{1.964945in}{2.270300in}}%
\pgfpathlineto{\pgfqpoint{1.967807in}{2.203698in}}%
\pgfpathlineto{\pgfqpoint{1.968103in}{2.208810in}}%
\pgfpathlineto{\pgfqpoint{1.968597in}{2.199896in}}%
\pgfpathlineto{\pgfqpoint{1.968794in}{2.202642in}}%
\pgfpathlineto{\pgfqpoint{1.968893in}{2.203240in}}%
\pgfpathlineto{\pgfqpoint{1.969090in}{2.200118in}}%
\pgfpathlineto{\pgfqpoint{1.970472in}{2.191153in}}%
\pgfpathlineto{\pgfqpoint{1.970571in}{2.191977in}}%
\pgfpathlineto{\pgfqpoint{1.971854in}{2.213260in}}%
\pgfpathlineto{\pgfqpoint{1.972150in}{2.208607in}}%
\pgfpathlineto{\pgfqpoint{1.972249in}{2.208700in}}%
\pgfpathlineto{\pgfqpoint{1.972643in}{2.220009in}}%
\pgfpathlineto{\pgfqpoint{1.973038in}{2.201977in}}%
\pgfpathlineto{\pgfqpoint{1.973137in}{2.200646in}}%
\pgfpathlineto{\pgfqpoint{1.973433in}{2.209730in}}%
\pgfpathlineto{\pgfqpoint{1.973532in}{2.211771in}}%
\pgfpathlineto{\pgfqpoint{1.973927in}{2.204974in}}%
\pgfpathlineto{\pgfqpoint{1.974321in}{2.209515in}}%
\pgfpathlineto{\pgfqpoint{1.976197in}{2.158818in}}%
\pgfpathlineto{\pgfqpoint{1.976394in}{2.163243in}}%
\pgfpathlineto{\pgfqpoint{1.976690in}{2.170989in}}%
\pgfpathlineto{\pgfqpoint{1.977184in}{2.150849in}}%
\pgfpathlineto{\pgfqpoint{1.977282in}{2.149926in}}%
\pgfpathlineto{\pgfqpoint{1.977480in}{2.157390in}}%
\pgfpathlineto{\pgfqpoint{1.979158in}{2.211723in}}%
\pgfpathlineto{\pgfqpoint{1.979454in}{2.211226in}}%
\pgfpathlineto{\pgfqpoint{1.980539in}{2.238560in}}%
\pgfpathlineto{\pgfqpoint{1.980835in}{2.223747in}}%
\pgfpathlineto{\pgfqpoint{1.980934in}{2.223434in}}%
\pgfpathlineto{\pgfqpoint{1.981428in}{2.252745in}}%
\pgfpathlineto{\pgfqpoint{1.982217in}{2.234675in}}%
\pgfpathlineto{\pgfqpoint{1.982612in}{2.222405in}}%
\pgfpathlineto{\pgfqpoint{1.982908in}{2.244080in}}%
\pgfpathlineto{\pgfqpoint{1.984191in}{2.487602in}}%
\pgfpathlineto{\pgfqpoint{1.985079in}{2.470728in}}%
\pgfpathlineto{\pgfqpoint{1.985869in}{2.254749in}}%
\pgfpathlineto{\pgfqpoint{1.987152in}{2.206297in}}%
\pgfpathlineto{\pgfqpoint{1.987744in}{2.213182in}}%
\pgfpathlineto{\pgfqpoint{1.988830in}{2.264533in}}%
\pgfpathlineto{\pgfqpoint{1.989126in}{2.291637in}}%
\pgfpathlineto{\pgfqpoint{1.989817in}{2.251499in}}%
\pgfpathlineto{\pgfqpoint{1.990311in}{2.189656in}}%
\pgfpathlineto{\pgfqpoint{1.991100in}{2.208438in}}%
\pgfpathlineto{\pgfqpoint{1.992482in}{2.265517in}}%
\pgfpathlineto{\pgfqpoint{1.993074in}{2.234495in}}%
\pgfpathlineto{\pgfqpoint{1.993568in}{2.210935in}}%
\pgfpathlineto{\pgfqpoint{1.993864in}{2.234465in}}%
\pgfpathlineto{\pgfqpoint{1.995640in}{2.342731in}}%
\pgfpathlineto{\pgfqpoint{1.995838in}{2.339009in}}%
\pgfpathlineto{\pgfqpoint{1.996529in}{2.332436in}}%
\pgfpathlineto{\pgfqpoint{1.996726in}{2.338169in}}%
\pgfpathlineto{\pgfqpoint{1.999490in}{2.474352in}}%
\pgfpathlineto{\pgfqpoint{1.999884in}{2.466194in}}%
\pgfpathlineto{\pgfqpoint{2.000082in}{2.473129in}}%
\pgfpathlineto{\pgfqpoint{2.001760in}{2.531621in}}%
\pgfpathlineto{\pgfqpoint{2.001858in}{2.531110in}}%
\pgfpathlineto{\pgfqpoint{2.002253in}{2.518190in}}%
\pgfpathlineto{\pgfqpoint{2.002648in}{2.536611in}}%
\pgfpathlineto{\pgfqpoint{2.002747in}{2.537946in}}%
\pgfpathlineto{\pgfqpoint{2.002944in}{2.529220in}}%
\pgfpathlineto{\pgfqpoint{2.003240in}{2.518264in}}%
\pgfpathlineto{\pgfqpoint{2.003635in}{2.546248in}}%
\pgfpathlineto{\pgfqpoint{2.004424in}{2.564108in}}%
\pgfpathlineto{\pgfqpoint{2.004721in}{2.551831in}}%
\pgfpathlineto{\pgfqpoint{2.004819in}{2.548037in}}%
\pgfpathlineto{\pgfqpoint{2.005313in}{2.568575in}}%
\pgfpathlineto{\pgfqpoint{2.006004in}{2.574597in}}%
\pgfpathlineto{\pgfqpoint{2.005609in}{2.567545in}}%
\pgfpathlineto{\pgfqpoint{2.006300in}{2.569550in}}%
\pgfpathlineto{\pgfqpoint{2.006398in}{2.568174in}}%
\pgfpathlineto{\pgfqpoint{2.006892in}{2.575639in}}%
\pgfpathlineto{\pgfqpoint{2.008076in}{2.592295in}}%
\pgfpathlineto{\pgfqpoint{2.008372in}{2.583261in}}%
\pgfpathlineto{\pgfqpoint{2.009162in}{2.573089in}}%
\pgfpathlineto{\pgfqpoint{2.009458in}{2.579309in}}%
\pgfpathlineto{\pgfqpoint{2.009853in}{2.592871in}}%
\pgfpathlineto{\pgfqpoint{2.010346in}{2.576632in}}%
\pgfpathlineto{\pgfqpoint{2.010741in}{2.590116in}}%
\pgfpathlineto{\pgfqpoint{2.011235in}{2.577132in}}%
\pgfpathlineto{\pgfqpoint{2.011629in}{2.591726in}}%
\pgfpathlineto{\pgfqpoint{2.012024in}{2.601223in}}%
\pgfpathlineto{\pgfqpoint{2.012419in}{2.587763in}}%
\pgfpathlineto{\pgfqpoint{2.012715in}{2.576676in}}%
\pgfpathlineto{\pgfqpoint{2.013209in}{2.596814in}}%
\pgfpathlineto{\pgfqpoint{2.013505in}{2.588824in}}%
\pgfpathlineto{\pgfqpoint{2.013603in}{2.588069in}}%
\pgfpathlineto{\pgfqpoint{2.013702in}{2.590995in}}%
\pgfpathlineto{\pgfqpoint{2.013998in}{2.604786in}}%
\pgfpathlineto{\pgfqpoint{2.014492in}{2.590861in}}%
\pgfpathlineto{\pgfqpoint{2.014887in}{2.595788in}}%
\pgfpathlineto{\pgfqpoint{2.015874in}{2.613465in}}%
\pgfpathlineto{\pgfqpoint{2.018045in}{2.662581in}}%
\pgfpathlineto{\pgfqpoint{2.019920in}{2.640724in}}%
\pgfpathlineto{\pgfqpoint{2.021598in}{2.602993in}}%
\pgfpathlineto{\pgfqpoint{2.021894in}{2.603422in}}%
\pgfpathlineto{\pgfqpoint{2.022289in}{2.604106in}}%
\pgfpathlineto{\pgfqpoint{2.022585in}{2.597341in}}%
\pgfpathlineto{\pgfqpoint{2.023276in}{2.578860in}}%
\pgfpathlineto{\pgfqpoint{2.023769in}{2.592764in}}%
\pgfpathlineto{\pgfqpoint{2.025447in}{2.584537in}}%
\pgfpathlineto{\pgfqpoint{2.025645in}{2.587704in}}%
\pgfpathlineto{\pgfqpoint{2.026138in}{2.600787in}}%
\pgfpathlineto{\pgfqpoint{2.026533in}{2.588812in}}%
\pgfpathlineto{\pgfqpoint{2.028112in}{2.502559in}}%
\pgfpathlineto{\pgfqpoint{2.028408in}{2.539494in}}%
\pgfpathlineto{\pgfqpoint{2.030382in}{2.792890in}}%
\pgfpathlineto{\pgfqpoint{2.030580in}{2.782861in}}%
\pgfpathlineto{\pgfqpoint{2.032652in}{2.501985in}}%
\pgfpathlineto{\pgfqpoint{2.033047in}{2.513071in}}%
\pgfpathlineto{\pgfqpoint{2.033442in}{2.508289in}}%
\pgfpathlineto{\pgfqpoint{2.034133in}{2.539778in}}%
\pgfpathlineto{\pgfqpoint{2.034528in}{2.571465in}}%
\pgfpathlineto{\pgfqpoint{2.035120in}{2.534931in}}%
\pgfpathlineto{\pgfqpoint{2.035811in}{2.463023in}}%
\pgfpathlineto{\pgfqpoint{2.036699in}{2.490287in}}%
\pgfpathlineto{\pgfqpoint{2.037785in}{2.529750in}}%
\pgfpathlineto{\pgfqpoint{2.038081in}{2.517952in}}%
\pgfpathlineto{\pgfqpoint{2.038969in}{2.450580in}}%
\pgfpathlineto{\pgfqpoint{2.039660in}{2.475704in}}%
\pgfpathlineto{\pgfqpoint{2.039759in}{2.477053in}}%
\pgfpathlineto{\pgfqpoint{2.040055in}{2.466369in}}%
\pgfpathlineto{\pgfqpoint{2.040153in}{2.464112in}}%
\pgfpathlineto{\pgfqpoint{2.040647in}{2.473994in}}%
\pgfpathlineto{\pgfqpoint{2.041239in}{2.501972in}}%
\pgfpathlineto{\pgfqpoint{2.041930in}{2.480863in}}%
\pgfpathlineto{\pgfqpoint{2.042325in}{2.467553in}}%
\pgfpathlineto{\pgfqpoint{2.042917in}{2.485578in}}%
\pgfpathlineto{\pgfqpoint{2.043213in}{2.486679in}}%
\pgfpathlineto{\pgfqpoint{2.044003in}{2.513185in}}%
\pgfpathlineto{\pgfqpoint{2.044792in}{2.502737in}}%
\pgfpathlineto{\pgfqpoint{2.045187in}{2.490171in}}%
\pgfpathlineto{\pgfqpoint{2.045483in}{2.506011in}}%
\pgfpathlineto{\pgfqpoint{2.046372in}{2.519604in}}%
\pgfpathlineto{\pgfqpoint{2.046766in}{2.517596in}}%
\pgfpathlineto{\pgfqpoint{2.048247in}{2.506706in}}%
\pgfpathlineto{\pgfqpoint{2.048346in}{2.507691in}}%
\pgfpathlineto{\pgfqpoint{2.049431in}{2.517813in}}%
\pgfpathlineto{\pgfqpoint{2.048938in}{2.498877in}}%
\pgfpathlineto{\pgfqpoint{2.049629in}{2.513827in}}%
\pgfpathlineto{\pgfqpoint{2.051010in}{2.487625in}}%
\pgfpathlineto{\pgfqpoint{2.051109in}{2.488210in}}%
\pgfpathlineto{\pgfqpoint{2.051405in}{2.495547in}}%
\pgfpathlineto{\pgfqpoint{2.051899in}{2.482519in}}%
\pgfpathlineto{\pgfqpoint{2.052590in}{2.466896in}}%
\pgfpathlineto{\pgfqpoint{2.052984in}{2.481193in}}%
\pgfpathlineto{\pgfqpoint{2.053083in}{2.481555in}}%
\pgfpathlineto{\pgfqpoint{2.053182in}{2.478090in}}%
\pgfpathlineto{\pgfqpoint{2.054860in}{2.426673in}}%
\pgfpathlineto{\pgfqpoint{2.054958in}{2.428290in}}%
\pgfpathlineto{\pgfqpoint{2.055156in}{2.432632in}}%
\pgfpathlineto{\pgfqpoint{2.055551in}{2.417308in}}%
\pgfpathlineto{\pgfqpoint{2.055945in}{2.428220in}}%
\pgfpathlineto{\pgfqpoint{2.056439in}{2.414165in}}%
\pgfpathlineto{\pgfqpoint{2.057130in}{2.425222in}}%
\pgfpathlineto{\pgfqpoint{2.057722in}{2.422409in}}%
\pgfpathlineto{\pgfqpoint{2.058117in}{2.397827in}}%
\pgfpathlineto{\pgfqpoint{2.058906in}{2.415850in}}%
\pgfpathlineto{\pgfqpoint{2.059992in}{2.434779in}}%
\pgfpathlineto{\pgfqpoint{2.060288in}{2.419361in}}%
\pgfpathlineto{\pgfqpoint{2.060387in}{2.416688in}}%
\pgfpathlineto{\pgfqpoint{2.060782in}{2.430590in}}%
\pgfpathlineto{\pgfqpoint{2.061176in}{2.422822in}}%
\pgfpathlineto{\pgfqpoint{2.062657in}{2.467738in}}%
\pgfpathlineto{\pgfqpoint{2.062854in}{2.465804in}}%
\pgfpathlineto{\pgfqpoint{2.062953in}{2.465647in}}%
\pgfpathlineto{\pgfqpoint{2.064828in}{2.505332in}}%
\pgfpathlineto{\pgfqpoint{2.064927in}{2.502429in}}%
\pgfpathlineto{\pgfqpoint{2.067098in}{2.447119in}}%
\pgfpathlineto{\pgfqpoint{2.067296in}{2.449989in}}%
\pgfpathlineto{\pgfqpoint{2.067493in}{2.455449in}}%
\pgfpathlineto{\pgfqpoint{2.067987in}{2.440794in}}%
\pgfpathlineto{\pgfqpoint{2.068381in}{2.437207in}}%
\pgfpathlineto{\pgfqpoint{2.068776in}{2.446358in}}%
\pgfpathlineto{\pgfqpoint{2.068974in}{2.443649in}}%
\pgfpathlineto{\pgfqpoint{2.069862in}{2.430818in}}%
\pgfpathlineto{\pgfqpoint{2.070059in}{2.437675in}}%
\pgfpathlineto{\pgfqpoint{2.071046in}{2.447190in}}%
\pgfpathlineto{\pgfqpoint{2.070651in}{2.421122in}}%
\pgfpathlineto{\pgfqpoint{2.071145in}{2.444130in}}%
\pgfpathlineto{\pgfqpoint{2.072329in}{2.426798in}}%
\pgfpathlineto{\pgfqpoint{2.072428in}{2.428220in}}%
\pgfpathlineto{\pgfqpoint{2.072922in}{2.437487in}}%
\pgfpathlineto{\pgfqpoint{2.073218in}{2.426106in}}%
\pgfpathlineto{\pgfqpoint{2.073612in}{2.405763in}}%
\pgfpathlineto{\pgfqpoint{2.074007in}{2.444375in}}%
\pgfpathlineto{\pgfqpoint{2.075883in}{2.676565in}}%
\pgfpathlineto{\pgfqpoint{2.076179in}{2.660766in}}%
\pgfpathlineto{\pgfqpoint{2.078153in}{2.417029in}}%
\pgfpathlineto{\pgfqpoint{2.078547in}{2.429622in}}%
\pgfpathlineto{\pgfqpoint{2.080423in}{2.528663in}}%
\pgfpathlineto{\pgfqpoint{2.080719in}{2.503834in}}%
\pgfpathlineto{\pgfqpoint{2.081212in}{2.457621in}}%
\pgfpathlineto{\pgfqpoint{2.081903in}{2.477916in}}%
\pgfpathlineto{\pgfqpoint{2.083581in}{2.547429in}}%
\pgfpathlineto{\pgfqpoint{2.083877in}{2.525919in}}%
\pgfpathlineto{\pgfqpoint{2.084469in}{2.496994in}}%
\pgfpathlineto{\pgfqpoint{2.085062in}{2.515308in}}%
\pgfpathlineto{\pgfqpoint{2.086739in}{2.568351in}}%
\pgfpathlineto{\pgfqpoint{2.086937in}{2.564391in}}%
\pgfpathlineto{\pgfqpoint{2.087332in}{2.547484in}}%
\pgfpathlineto{\pgfqpoint{2.088121in}{2.552545in}}%
\pgfpathlineto{\pgfqpoint{2.090194in}{2.601224in}}%
\pgfpathlineto{\pgfqpoint{2.090391in}{2.593016in}}%
\pgfpathlineto{\pgfqpoint{2.090589in}{2.585905in}}%
\pgfpathlineto{\pgfqpoint{2.091280in}{2.602427in}}%
\pgfpathlineto{\pgfqpoint{2.092661in}{2.627702in}}%
\pgfpathlineto{\pgfqpoint{2.093352in}{2.624833in}}%
\pgfpathlineto{\pgfqpoint{2.094339in}{2.613977in}}%
\pgfpathlineto{\pgfqpoint{2.094635in}{2.621938in}}%
\pgfpathlineto{\pgfqpoint{2.095129in}{2.637804in}}%
\pgfpathlineto{\pgfqpoint{2.095622in}{2.622779in}}%
\pgfpathlineto{\pgfqpoint{2.095820in}{2.616913in}}%
\pgfpathlineto{\pgfqpoint{2.096708in}{2.622226in}}%
\pgfpathlineto{\pgfqpoint{2.097596in}{2.629467in}}%
\pgfpathlineto{\pgfqpoint{2.097794in}{2.627286in}}%
\pgfpathlineto{\pgfqpoint{2.099669in}{2.599790in}}%
\pgfpathlineto{\pgfqpoint{2.099768in}{2.600652in}}%
\pgfpathlineto{\pgfqpoint{2.100064in}{2.603191in}}%
\pgfpathlineto{\pgfqpoint{2.100360in}{2.599114in}}%
\pgfpathlineto{\pgfqpoint{2.100656in}{2.592351in}}%
\pgfpathlineto{\pgfqpoint{2.101248in}{2.602254in}}%
\pgfpathlineto{\pgfqpoint{2.101347in}{2.601722in}}%
\pgfpathlineto{\pgfqpoint{2.101742in}{2.595515in}}%
\pgfpathlineto{\pgfqpoint{2.102827in}{2.577658in}}%
\pgfpathlineto{\pgfqpoint{2.103123in}{2.581803in}}%
\pgfpathlineto{\pgfqpoint{2.104110in}{2.567692in}}%
\pgfpathlineto{\pgfqpoint{2.104703in}{2.575232in}}%
\pgfpathlineto{\pgfqpoint{2.104900in}{2.575429in}}%
\pgfpathlineto{\pgfqpoint{2.104999in}{2.574823in}}%
\pgfpathlineto{\pgfqpoint{2.105295in}{2.572082in}}%
\pgfpathlineto{\pgfqpoint{2.105492in}{2.578439in}}%
\pgfpathlineto{\pgfqpoint{2.105690in}{2.582887in}}%
\pgfpathlineto{\pgfqpoint{2.106084in}{2.561202in}}%
\pgfpathlineto{\pgfqpoint{2.106380in}{2.570759in}}%
\pgfpathlineto{\pgfqpoint{2.106479in}{2.571326in}}%
\pgfpathlineto{\pgfqpoint{2.106578in}{2.568415in}}%
\pgfpathlineto{\pgfqpoint{2.106775in}{2.562708in}}%
\pgfpathlineto{\pgfqpoint{2.107466in}{2.574150in}}%
\pgfpathlineto{\pgfqpoint{2.108552in}{2.596436in}}%
\pgfpathlineto{\pgfqpoint{2.108848in}{2.590538in}}%
\pgfpathlineto{\pgfqpoint{2.111414in}{2.493301in}}%
\pgfpathlineto{\pgfqpoint{2.112302in}{2.507117in}}%
\pgfpathlineto{\pgfqpoint{2.113586in}{2.552506in}}%
\pgfpathlineto{\pgfqpoint{2.113882in}{2.543034in}}%
\pgfpathlineto{\pgfqpoint{2.115066in}{2.537400in}}%
\pgfpathlineto{\pgfqpoint{2.114474in}{2.552449in}}%
\pgfpathlineto{\pgfqpoint{2.115165in}{2.537836in}}%
\pgfpathlineto{\pgfqpoint{2.115559in}{2.544060in}}%
\pgfpathlineto{\pgfqpoint{2.115856in}{2.533148in}}%
\pgfpathlineto{\pgfqpoint{2.116941in}{2.521314in}}%
\pgfpathlineto{\pgfqpoint{2.117040in}{2.523352in}}%
\pgfpathlineto{\pgfqpoint{2.118422in}{2.539142in}}%
\pgfpathlineto{\pgfqpoint{2.118520in}{2.536735in}}%
\pgfpathlineto{\pgfqpoint{2.119014in}{2.506928in}}%
\pgfpathlineto{\pgfqpoint{2.119409in}{2.537090in}}%
\pgfpathlineto{\pgfqpoint{2.121087in}{2.800338in}}%
\pgfpathlineto{\pgfqpoint{2.121679in}{2.750556in}}%
\pgfpathlineto{\pgfqpoint{2.122271in}{2.581189in}}%
\pgfpathlineto{\pgfqpoint{2.123554in}{2.516411in}}%
\pgfpathlineto{\pgfqpoint{2.123653in}{2.513960in}}%
\pgfpathlineto{\pgfqpoint{2.124048in}{2.528942in}}%
\pgfpathlineto{\pgfqpoint{2.124245in}{2.526655in}}%
\pgfpathlineto{\pgfqpoint{2.124344in}{2.525653in}}%
\pgfpathlineto{\pgfqpoint{2.124640in}{2.533264in}}%
\pgfpathlineto{\pgfqpoint{2.125824in}{2.596249in}}%
\pgfpathlineto{\pgfqpoint{2.126318in}{2.568158in}}%
\pgfpathlineto{\pgfqpoint{2.126910in}{2.505206in}}%
\pgfpathlineto{\pgfqpoint{2.127699in}{2.530544in}}%
\pgfpathlineto{\pgfqpoint{2.127996in}{2.533053in}}%
\pgfpathlineto{\pgfqpoint{2.128884in}{2.578450in}}%
\pgfpathlineto{\pgfqpoint{2.129476in}{2.548351in}}%
\pgfpathlineto{\pgfqpoint{2.129970in}{2.517811in}}%
\pgfpathlineto{\pgfqpoint{2.130660in}{2.532385in}}%
\pgfpathlineto{\pgfqpoint{2.132141in}{2.571737in}}%
\pgfpathlineto{\pgfqpoint{2.132240in}{2.566259in}}%
\pgfpathlineto{\pgfqpoint{2.133621in}{2.521522in}}%
\pgfpathlineto{\pgfqpoint{2.133720in}{2.523241in}}%
\pgfpathlineto{\pgfqpoint{2.135398in}{2.547337in}}%
\pgfpathlineto{\pgfqpoint{2.136681in}{2.514449in}}%
\pgfpathlineto{\pgfqpoint{2.137273in}{2.530981in}}%
\pgfpathlineto{\pgfqpoint{2.139839in}{2.498110in}}%
\pgfpathlineto{\pgfqpoint{2.140432in}{2.503154in}}%
\pgfpathlineto{\pgfqpoint{2.140728in}{2.509355in}}%
\pgfpathlineto{\pgfqpoint{2.141320in}{2.497238in}}%
\pgfpathlineto{\pgfqpoint{2.141616in}{2.496473in}}%
\pgfpathlineto{\pgfqpoint{2.141715in}{2.498121in}}%
\pgfpathlineto{\pgfqpoint{2.142011in}{2.509382in}}%
\pgfpathlineto{\pgfqpoint{2.142406in}{2.484463in}}%
\pgfpathlineto{\pgfqpoint{2.143195in}{2.478580in}}%
\pgfpathlineto{\pgfqpoint{2.142899in}{2.486840in}}%
\pgfpathlineto{\pgfqpoint{2.143491in}{2.483973in}}%
\pgfpathlineto{\pgfqpoint{2.144281in}{2.488110in}}%
\pgfpathlineto{\pgfqpoint{2.143886in}{2.481128in}}%
\pgfpathlineto{\pgfqpoint{2.144478in}{2.482854in}}%
\pgfpathlineto{\pgfqpoint{2.146156in}{2.446199in}}%
\pgfpathlineto{\pgfqpoint{2.146255in}{2.446527in}}%
\pgfpathlineto{\pgfqpoint{2.146354in}{2.447479in}}%
\pgfpathlineto{\pgfqpoint{2.146650in}{2.441707in}}%
\pgfpathlineto{\pgfqpoint{2.147242in}{2.450568in}}%
\pgfpathlineto{\pgfqpoint{2.147834in}{2.436742in}}%
\pgfpathlineto{\pgfqpoint{2.148525in}{2.443539in}}%
\pgfpathlineto{\pgfqpoint{2.149315in}{2.443100in}}%
\pgfpathlineto{\pgfqpoint{2.149413in}{2.443254in}}%
\pgfpathlineto{\pgfqpoint{2.149512in}{2.441973in}}%
\pgfpathlineto{\pgfqpoint{2.150499in}{2.424744in}}%
\pgfpathlineto{\pgfqpoint{2.151091in}{2.428865in}}%
\pgfpathlineto{\pgfqpoint{2.151683in}{2.422496in}}%
\pgfpathlineto{\pgfqpoint{2.152078in}{2.426871in}}%
\pgfpathlineto{\pgfqpoint{2.153361in}{2.451067in}}%
\pgfpathlineto{\pgfqpoint{2.153657in}{2.445952in}}%
\pgfpathlineto{\pgfqpoint{2.155039in}{2.465092in}}%
\pgfpathlineto{\pgfqpoint{2.155236in}{2.461249in}}%
\pgfpathlineto{\pgfqpoint{2.156026in}{2.461784in}}%
\pgfpathlineto{\pgfqpoint{2.157408in}{2.410852in}}%
\pgfpathlineto{\pgfqpoint{2.159382in}{2.387677in}}%
\pgfpathlineto{\pgfqpoint{2.159875in}{2.392150in}}%
\pgfpathlineto{\pgfqpoint{2.159974in}{2.391845in}}%
\pgfpathlineto{\pgfqpoint{2.160171in}{2.393509in}}%
\pgfpathlineto{\pgfqpoint{2.161060in}{2.397594in}}%
\pgfpathlineto{\pgfqpoint{2.160764in}{2.392295in}}%
\pgfpathlineto{\pgfqpoint{2.161257in}{2.395818in}}%
\pgfpathlineto{\pgfqpoint{2.162047in}{2.384704in}}%
\pgfpathlineto{\pgfqpoint{2.162343in}{2.394304in}}%
\pgfpathlineto{\pgfqpoint{2.162441in}{2.394692in}}%
\pgfpathlineto{\pgfqpoint{2.162540in}{2.392145in}}%
\pgfpathlineto{\pgfqpoint{2.162738in}{2.385218in}}%
\pgfpathlineto{\pgfqpoint{2.163428in}{2.399161in}}%
\pgfpathlineto{\pgfqpoint{2.163725in}{2.398230in}}%
\pgfpathlineto{\pgfqpoint{2.164317in}{2.381020in}}%
\pgfpathlineto{\pgfqpoint{2.164613in}{2.363607in}}%
\pgfpathlineto{\pgfqpoint{2.164909in}{2.404275in}}%
\pgfpathlineto{\pgfqpoint{2.166389in}{2.631283in}}%
\pgfpathlineto{\pgfqpoint{2.166883in}{2.620728in}}%
\pgfpathlineto{\pgfqpoint{2.166982in}{2.621345in}}%
\pgfpathlineto{\pgfqpoint{2.167080in}{2.617264in}}%
\pgfpathlineto{\pgfqpoint{2.167870in}{2.393287in}}%
\pgfpathlineto{\pgfqpoint{2.168067in}{2.371459in}}%
\pgfpathlineto{\pgfqpoint{2.168561in}{2.406877in}}%
\pgfpathlineto{\pgfqpoint{2.169054in}{2.372177in}}%
\pgfpathlineto{\pgfqpoint{2.169153in}{2.370453in}}%
\pgfpathlineto{\pgfqpoint{2.169449in}{2.379664in}}%
\pgfpathlineto{\pgfqpoint{2.171226in}{2.483659in}}%
\pgfpathlineto{\pgfqpoint{2.171522in}{2.458161in}}%
\pgfpathlineto{\pgfqpoint{2.172311in}{2.387087in}}%
\pgfpathlineto{\pgfqpoint{2.172904in}{2.406800in}}%
\pgfpathlineto{\pgfqpoint{2.173989in}{2.448441in}}%
\pgfpathlineto{\pgfqpoint{2.174285in}{2.461648in}}%
\pgfpathlineto{\pgfqpoint{2.174878in}{2.436334in}}%
\pgfpathlineto{\pgfqpoint{2.175470in}{2.407615in}}%
\pgfpathlineto{\pgfqpoint{2.176062in}{2.428669in}}%
\pgfpathlineto{\pgfqpoint{2.177740in}{2.460581in}}%
\pgfpathlineto{\pgfqpoint{2.177839in}{2.457786in}}%
\pgfpathlineto{\pgfqpoint{2.178727in}{2.422247in}}%
\pgfpathlineto{\pgfqpoint{2.179220in}{2.435046in}}%
\pgfpathlineto{\pgfqpoint{2.181096in}{2.460191in}}%
\pgfpathlineto{\pgfqpoint{2.181293in}{2.454368in}}%
\pgfpathlineto{\pgfqpoint{2.181688in}{2.429367in}}%
\pgfpathlineto{\pgfqpoint{2.182477in}{2.448524in}}%
\pgfpathlineto{\pgfqpoint{2.182971in}{2.447605in}}%
\pgfpathlineto{\pgfqpoint{2.183563in}{2.434394in}}%
\pgfpathlineto{\pgfqpoint{2.183859in}{2.448316in}}%
\pgfpathlineto{\pgfqpoint{2.184155in}{2.454567in}}%
\pgfpathlineto{\pgfqpoint{2.184747in}{2.441554in}}%
\pgfpathlineto{\pgfqpoint{2.184945in}{2.432857in}}%
\pgfpathlineto{\pgfqpoint{2.185833in}{2.442816in}}%
\pgfpathlineto{\pgfqpoint{2.186129in}{2.447119in}}%
\pgfpathlineto{\pgfqpoint{2.186820in}{2.441428in}}%
\pgfpathlineto{\pgfqpoint{2.187906in}{2.428314in}}%
\pgfpathlineto{\pgfqpoint{2.188103in}{2.433201in}}%
\pgfpathlineto{\pgfqpoint{2.188399in}{2.444586in}}%
\pgfpathlineto{\pgfqpoint{2.188794in}{2.423717in}}%
\pgfpathlineto{\pgfqpoint{2.189189in}{2.432252in}}%
\pgfpathlineto{\pgfqpoint{2.192545in}{2.313379in}}%
\pgfpathlineto{\pgfqpoint{2.193137in}{2.325591in}}%
\pgfpathlineto{\pgfqpoint{2.195407in}{2.389814in}}%
\pgfpathlineto{\pgfqpoint{2.195604in}{2.393520in}}%
\pgfpathlineto{\pgfqpoint{2.196295in}{2.386862in}}%
\pgfpathlineto{\pgfqpoint{2.197282in}{2.368361in}}%
\pgfpathlineto{\pgfqpoint{2.197480in}{2.371606in}}%
\pgfpathlineto{\pgfqpoint{2.198664in}{2.407855in}}%
\pgfpathlineto{\pgfqpoint{2.199059in}{2.396291in}}%
\pgfpathlineto{\pgfqpoint{2.201329in}{2.422217in}}%
\pgfpathlineto{\pgfqpoint{2.201526in}{2.414553in}}%
\pgfpathlineto{\pgfqpoint{2.203500in}{2.358991in}}%
\pgfpathlineto{\pgfqpoint{2.203796in}{2.359988in}}%
\pgfpathlineto{\pgfqpoint{2.203895in}{2.360363in}}%
\pgfpathlineto{\pgfqpoint{2.203994in}{2.358676in}}%
\pgfpathlineto{\pgfqpoint{2.204389in}{2.343698in}}%
\pgfpathlineto{\pgfqpoint{2.205277in}{2.346741in}}%
\pgfpathlineto{\pgfqpoint{2.205672in}{2.361117in}}%
\pgfpathlineto{\pgfqpoint{2.206461in}{2.353961in}}%
\pgfpathlineto{\pgfqpoint{2.206560in}{2.353890in}}%
\pgfpathlineto{\pgfqpoint{2.206659in}{2.354950in}}%
\pgfpathlineto{\pgfqpoint{2.207646in}{2.372071in}}%
\pgfpathlineto{\pgfqpoint{2.207942in}{2.361893in}}%
\pgfpathlineto{\pgfqpoint{2.208139in}{2.357838in}}%
\pgfpathlineto{\pgfqpoint{2.208830in}{2.366222in}}%
\pgfpathlineto{\pgfqpoint{2.209225in}{2.376367in}}%
\pgfpathlineto{\pgfqpoint{2.209817in}{2.362124in}}%
\pgfpathlineto{\pgfqpoint{2.210014in}{2.352846in}}%
\pgfpathlineto{\pgfqpoint{2.210409in}{2.384237in}}%
\pgfpathlineto{\pgfqpoint{2.211988in}{2.623003in}}%
\pgfpathlineto{\pgfqpoint{2.212679in}{2.600383in}}%
\pgfpathlineto{\pgfqpoint{2.214752in}{2.358165in}}%
\pgfpathlineto{\pgfqpoint{2.215147in}{2.378709in}}%
\pgfpathlineto{\pgfqpoint{2.216627in}{2.462434in}}%
\pgfpathlineto{\pgfqpoint{2.215739in}{2.377103in}}%
\pgfpathlineto{\pgfqpoint{2.217022in}{2.437087in}}%
\pgfpathlineto{\pgfqpoint{2.217713in}{2.379453in}}%
\pgfpathlineto{\pgfqpoint{2.218601in}{2.406375in}}%
\pgfpathlineto{\pgfqpoint{2.219983in}{2.456696in}}%
\pgfpathlineto{\pgfqpoint{2.220279in}{2.444682in}}%
\pgfpathlineto{\pgfqpoint{2.220970in}{2.399542in}}%
\pgfpathlineto{\pgfqpoint{2.221562in}{2.422559in}}%
\pgfpathlineto{\pgfqpoint{2.223141in}{2.458557in}}%
\pgfpathlineto{\pgfqpoint{2.223240in}{2.459945in}}%
\pgfpathlineto{\pgfqpoint{2.223536in}{2.452067in}}%
\pgfpathlineto{\pgfqpoint{2.224227in}{2.435134in}}%
\pgfpathlineto{\pgfqpoint{2.224721in}{2.445783in}}%
\pgfpathlineto{\pgfqpoint{2.225214in}{2.463099in}}%
\pgfpathlineto{\pgfqpoint{2.226300in}{2.479120in}}%
\pgfpathlineto{\pgfqpoint{2.226497in}{2.474330in}}%
\pgfpathlineto{\pgfqpoint{2.227583in}{2.462702in}}%
\pgfpathlineto{\pgfqpoint{2.227681in}{2.465812in}}%
\pgfpathlineto{\pgfqpoint{2.229162in}{2.489782in}}%
\pgfpathlineto{\pgfqpoint{2.229359in}{2.488578in}}%
\pgfpathlineto{\pgfqpoint{2.229458in}{2.487113in}}%
\pgfpathlineto{\pgfqpoint{2.229754in}{2.495600in}}%
\pgfpathlineto{\pgfqpoint{2.229853in}{2.498732in}}%
\pgfpathlineto{\pgfqpoint{2.230149in}{2.481428in}}%
\pgfpathlineto{\pgfqpoint{2.230445in}{2.465305in}}%
\pgfpathlineto{\pgfqpoint{2.231235in}{2.483326in}}%
\pgfpathlineto{\pgfqpoint{2.231827in}{2.486388in}}%
\pgfpathlineto{\pgfqpoint{2.232024in}{2.480863in}}%
\pgfpathlineto{\pgfqpoint{2.233011in}{2.465269in}}%
\pgfpathlineto{\pgfqpoint{2.233209in}{2.473986in}}%
\pgfpathlineto{\pgfqpoint{2.233406in}{2.485372in}}%
\pgfpathlineto{\pgfqpoint{2.233801in}{2.473017in}}%
\pgfpathlineto{\pgfqpoint{2.234294in}{2.479707in}}%
\pgfpathlineto{\pgfqpoint{2.235874in}{2.445135in}}%
\pgfpathlineto{\pgfqpoint{2.235972in}{2.445298in}}%
\pgfpathlineto{\pgfqpoint{2.236268in}{2.447711in}}%
\pgfpathlineto{\pgfqpoint{2.236663in}{2.441732in}}%
\pgfpathlineto{\pgfqpoint{2.236860in}{2.442737in}}%
\pgfpathlineto{\pgfqpoint{2.236959in}{2.442916in}}%
\pgfpathlineto{\pgfqpoint{2.237058in}{2.442177in}}%
\pgfpathlineto{\pgfqpoint{2.239131in}{2.416246in}}%
\pgfpathlineto{\pgfqpoint{2.239525in}{2.426644in}}%
\pgfpathlineto{\pgfqpoint{2.240019in}{2.413262in}}%
\pgfpathlineto{\pgfqpoint{2.240216in}{2.414049in}}%
\pgfpathlineto{\pgfqpoint{2.240414in}{2.408622in}}%
\pgfpathlineto{\pgfqpoint{2.240611in}{2.401833in}}%
\pgfpathlineto{\pgfqpoint{2.241105in}{2.417387in}}%
\pgfpathlineto{\pgfqpoint{2.241302in}{2.415820in}}%
\pgfpathlineto{\pgfqpoint{2.241993in}{2.403887in}}%
\pgfpathlineto{\pgfqpoint{2.242684in}{2.412459in}}%
\pgfpathlineto{\pgfqpoint{2.243868in}{2.437129in}}%
\pgfpathlineto{\pgfqpoint{2.245349in}{2.462611in}}%
\pgfpathlineto{\pgfqpoint{2.244263in}{2.432706in}}%
\pgfpathlineto{\pgfqpoint{2.245645in}{2.450600in}}%
\pgfpathlineto{\pgfqpoint{2.245743in}{2.448311in}}%
\pgfpathlineto{\pgfqpoint{2.246039in}{2.462834in}}%
\pgfpathlineto{\pgfqpoint{2.246237in}{2.467289in}}%
\pgfpathlineto{\pgfqpoint{2.246632in}{2.449301in}}%
\pgfpathlineto{\pgfqpoint{2.246928in}{2.457675in}}%
\pgfpathlineto{\pgfqpoint{2.247520in}{2.459767in}}%
\pgfpathlineto{\pgfqpoint{2.247915in}{2.424427in}}%
\pgfpathlineto{\pgfqpoint{2.249395in}{2.393504in}}%
\pgfpathlineto{\pgfqpoint{2.248310in}{2.428034in}}%
\pgfpathlineto{\pgfqpoint{2.249494in}{2.395991in}}%
\pgfpathlineto{\pgfqpoint{2.249790in}{2.403879in}}%
\pgfpathlineto{\pgfqpoint{2.250284in}{2.383853in}}%
\pgfpathlineto{\pgfqpoint{2.250580in}{2.387875in}}%
\pgfpathlineto{\pgfqpoint{2.250876in}{2.377670in}}%
\pgfpathlineto{\pgfqpoint{2.250974in}{2.375319in}}%
\pgfpathlineto{\pgfqpoint{2.251369in}{2.382861in}}%
\pgfpathlineto{\pgfqpoint{2.251863in}{2.377368in}}%
\pgfpathlineto{\pgfqpoint{2.252060in}{2.379363in}}%
\pgfpathlineto{\pgfqpoint{2.252455in}{2.372246in}}%
\pgfpathlineto{\pgfqpoint{2.252948in}{2.378111in}}%
\pgfpathlineto{\pgfqpoint{2.253935in}{2.372289in}}%
\pgfpathlineto{\pgfqpoint{2.253541in}{2.378308in}}%
\pgfpathlineto{\pgfqpoint{2.254133in}{2.375987in}}%
\pgfpathlineto{\pgfqpoint{2.254330in}{2.379931in}}%
\pgfpathlineto{\pgfqpoint{2.254922in}{2.369152in}}%
\pgfpathlineto{\pgfqpoint{2.255317in}{2.370327in}}%
\pgfpathlineto{\pgfqpoint{2.255613in}{2.360115in}}%
\pgfpathlineto{\pgfqpoint{2.255909in}{2.350498in}}%
\pgfpathlineto{\pgfqpoint{2.256205in}{2.380997in}}%
\pgfpathlineto{\pgfqpoint{2.258179in}{2.614442in}}%
\pgfpathlineto{\pgfqpoint{2.258377in}{2.602582in}}%
\pgfpathlineto{\pgfqpoint{2.259364in}{2.360581in}}%
\pgfpathlineto{\pgfqpoint{2.261239in}{2.386837in}}%
\pgfpathlineto{\pgfqpoint{2.261733in}{2.401960in}}%
\pgfpathlineto{\pgfqpoint{2.262621in}{2.470536in}}%
\pgfpathlineto{\pgfqpoint{2.263016in}{2.430561in}}%
\pgfpathlineto{\pgfqpoint{2.263411in}{2.390505in}}%
\pgfpathlineto{\pgfqpoint{2.264101in}{2.419474in}}%
\pgfpathlineto{\pgfqpoint{2.265779in}{2.461833in}}%
\pgfpathlineto{\pgfqpoint{2.265977in}{2.452212in}}%
\pgfpathlineto{\pgfqpoint{2.267062in}{2.404347in}}%
\pgfpathlineto{\pgfqpoint{2.267358in}{2.414970in}}%
\pgfpathlineto{\pgfqpoint{2.269036in}{2.468673in}}%
\pgfpathlineto{\pgfqpoint{2.269332in}{2.457641in}}%
\pgfpathlineto{\pgfqpoint{2.270122in}{2.431272in}}%
\pgfpathlineto{\pgfqpoint{2.270616in}{2.442574in}}%
\pgfpathlineto{\pgfqpoint{2.271010in}{2.448006in}}%
\pgfpathlineto{\pgfqpoint{2.271405in}{2.436057in}}%
\pgfpathlineto{\pgfqpoint{2.273083in}{2.383207in}}%
\pgfpathlineto{\pgfqpoint{2.273280in}{2.394005in}}%
\pgfpathlineto{\pgfqpoint{2.275156in}{2.495603in}}%
\pgfpathlineto{\pgfqpoint{2.275254in}{2.493868in}}%
\pgfpathlineto{\pgfqpoint{2.276636in}{2.467308in}}%
\pgfpathlineto{\pgfqpoint{2.276735in}{2.469011in}}%
\pgfpathlineto{\pgfqpoint{2.277031in}{2.482690in}}%
\pgfpathlineto{\pgfqpoint{2.277524in}{2.465965in}}%
\pgfpathlineto{\pgfqpoint{2.278018in}{2.478464in}}%
\pgfpathlineto{\pgfqpoint{2.279104in}{2.469400in}}%
\pgfpathlineto{\pgfqpoint{2.278709in}{2.481335in}}%
\pgfpathlineto{\pgfqpoint{2.279301in}{2.471074in}}%
\pgfpathlineto{\pgfqpoint{2.279400in}{2.471589in}}%
\pgfpathlineto{\pgfqpoint{2.279498in}{2.469629in}}%
\pgfpathlineto{\pgfqpoint{2.280782in}{2.448459in}}%
\pgfpathlineto{\pgfqpoint{2.280979in}{2.449929in}}%
\pgfpathlineto{\pgfqpoint{2.281275in}{2.447441in}}%
\pgfpathlineto{\pgfqpoint{2.283545in}{2.424844in}}%
\pgfpathlineto{\pgfqpoint{2.283940in}{2.428371in}}%
\pgfpathlineto{\pgfqpoint{2.284236in}{2.422446in}}%
\pgfpathlineto{\pgfqpoint{2.284729in}{2.427037in}}%
\pgfpathlineto{\pgfqpoint{2.286210in}{2.416349in}}%
\pgfpathlineto{\pgfqpoint{2.286703in}{2.434502in}}%
\pgfpathlineto{\pgfqpoint{2.287888in}{2.425461in}}%
\pgfpathlineto{\pgfqpoint{2.288085in}{2.423201in}}%
\pgfpathlineto{\pgfqpoint{2.288480in}{2.434942in}}%
\pgfpathlineto{\pgfqpoint{2.288677in}{2.432241in}}%
\pgfpathlineto{\pgfqpoint{2.288776in}{2.431241in}}%
\pgfpathlineto{\pgfqpoint{2.288974in}{2.436972in}}%
\pgfpathlineto{\pgfqpoint{2.290553in}{2.463176in}}%
\pgfpathlineto{\pgfqpoint{2.292428in}{2.485703in}}%
\pgfpathlineto{\pgfqpoint{2.292527in}{2.484727in}}%
\pgfpathlineto{\pgfqpoint{2.294895in}{2.413522in}}%
\pgfpathlineto{\pgfqpoint{2.295192in}{2.414252in}}%
\pgfpathlineto{\pgfqpoint{2.295389in}{2.412946in}}%
\pgfpathlineto{\pgfqpoint{2.296376in}{2.388042in}}%
\pgfpathlineto{\pgfqpoint{2.297659in}{2.394452in}}%
\pgfpathlineto{\pgfqpoint{2.297955in}{2.404052in}}%
\pgfpathlineto{\pgfqpoint{2.298449in}{2.389650in}}%
\pgfpathlineto{\pgfqpoint{2.298745in}{2.395706in}}%
\pgfpathlineto{\pgfqpoint{2.299830in}{2.383068in}}%
\pgfpathlineto{\pgfqpoint{2.300028in}{2.390970in}}%
\pgfpathlineto{\pgfqpoint{2.301015in}{2.400703in}}%
\pgfpathlineto{\pgfqpoint{2.301114in}{2.397091in}}%
\pgfpathlineto{\pgfqpoint{2.301508in}{2.384085in}}%
\pgfpathlineto{\pgfqpoint{2.302002in}{2.398337in}}%
\pgfpathlineto{\pgfqpoint{2.304173in}{2.633740in}}%
\pgfpathlineto{\pgfqpoint{2.304469in}{2.605005in}}%
\pgfpathlineto{\pgfqpoint{2.306246in}{2.387919in}}%
\pgfpathlineto{\pgfqpoint{2.306345in}{2.392209in}}%
\pgfpathlineto{\pgfqpoint{2.308417in}{2.511096in}}%
\pgfpathlineto{\pgfqpoint{2.308713in}{2.492292in}}%
\pgfpathlineto{\pgfqpoint{2.309602in}{2.421750in}}%
\pgfpathlineto{\pgfqpoint{2.310095in}{2.453252in}}%
\pgfpathlineto{\pgfqpoint{2.311773in}{2.493228in}}%
\pgfpathlineto{\pgfqpoint{2.311970in}{2.486645in}}%
\pgfpathlineto{\pgfqpoint{2.312563in}{2.446465in}}%
\pgfpathlineto{\pgfqpoint{2.313352in}{2.463019in}}%
\pgfpathlineto{\pgfqpoint{2.315129in}{2.502671in}}%
\pgfpathlineto{\pgfqpoint{2.315425in}{2.489839in}}%
\pgfpathlineto{\pgfqpoint{2.315820in}{2.465767in}}%
\pgfpathlineto{\pgfqpoint{2.316511in}{2.480941in}}%
\pgfpathlineto{\pgfqpoint{2.317794in}{2.500825in}}%
\pgfpathlineto{\pgfqpoint{2.318287in}{2.499639in}}%
\pgfpathlineto{\pgfqpoint{2.318386in}{2.500398in}}%
\pgfpathlineto{\pgfqpoint{2.318583in}{2.495998in}}%
\pgfpathlineto{\pgfqpoint{2.318978in}{2.471278in}}%
\pgfpathlineto{\pgfqpoint{2.319669in}{2.492573in}}%
\pgfpathlineto{\pgfqpoint{2.319965in}{2.489097in}}%
\pgfpathlineto{\pgfqpoint{2.320261in}{2.495790in}}%
\pgfpathlineto{\pgfqpoint{2.320459in}{2.498286in}}%
\pgfpathlineto{\pgfqpoint{2.320755in}{2.492934in}}%
\pgfpathlineto{\pgfqpoint{2.321248in}{2.496065in}}%
\pgfpathlineto{\pgfqpoint{2.322432in}{2.474122in}}%
\pgfpathlineto{\pgfqpoint{2.322729in}{2.483184in}}%
\pgfpathlineto{\pgfqpoint{2.323716in}{2.491827in}}%
\pgfpathlineto{\pgfqpoint{2.323321in}{2.477384in}}%
\pgfpathlineto{\pgfqpoint{2.323913in}{2.488083in}}%
\pgfpathlineto{\pgfqpoint{2.325196in}{2.481163in}}%
\pgfpathlineto{\pgfqpoint{2.328848in}{2.425752in}}%
\pgfpathlineto{\pgfqpoint{2.328947in}{2.426103in}}%
\pgfpathlineto{\pgfqpoint{2.329144in}{2.430447in}}%
\pgfpathlineto{\pgfqpoint{2.329539in}{2.414707in}}%
\pgfpathlineto{\pgfqpoint{2.329638in}{2.414204in}}%
\pgfpathlineto{\pgfqpoint{2.329736in}{2.417715in}}%
\pgfpathlineto{\pgfqpoint{2.330032in}{2.427031in}}%
\pgfpathlineto{\pgfqpoint{2.330526in}{2.405037in}}%
\pgfpathlineto{\pgfqpoint{2.330723in}{2.406118in}}%
\pgfpathlineto{\pgfqpoint{2.330921in}{2.404024in}}%
\pgfpathlineto{\pgfqpoint{2.332006in}{2.391852in}}%
\pgfpathlineto{\pgfqpoint{2.331513in}{2.404700in}}%
\pgfpathlineto{\pgfqpoint{2.332204in}{2.395321in}}%
\pgfpathlineto{\pgfqpoint{2.332697in}{2.408310in}}%
\pgfpathlineto{\pgfqpoint{2.333289in}{2.396257in}}%
\pgfpathlineto{\pgfqpoint{2.334375in}{2.390355in}}%
\pgfpathlineto{\pgfqpoint{2.334572in}{2.391547in}}%
\pgfpathlineto{\pgfqpoint{2.339014in}{2.456142in}}%
\pgfpathlineto{\pgfqpoint{2.339409in}{2.450827in}}%
\pgfpathlineto{\pgfqpoint{2.342172in}{2.385345in}}%
\pgfpathlineto{\pgfqpoint{2.342370in}{2.386667in}}%
\pgfpathlineto{\pgfqpoint{2.342567in}{2.383452in}}%
\pgfpathlineto{\pgfqpoint{2.342764in}{2.376429in}}%
\pgfpathlineto{\pgfqpoint{2.343159in}{2.391355in}}%
\pgfpathlineto{\pgfqpoint{2.343554in}{2.383219in}}%
\pgfpathlineto{\pgfqpoint{2.343949in}{2.400687in}}%
\pgfpathlineto{\pgfqpoint{2.344837in}{2.392853in}}%
\pgfpathlineto{\pgfqpoint{2.345725in}{2.396060in}}%
\pgfpathlineto{\pgfqpoint{2.345331in}{2.390826in}}%
\pgfpathlineto{\pgfqpoint{2.345923in}{2.393886in}}%
\pgfpathlineto{\pgfqpoint{2.346022in}{2.393292in}}%
\pgfpathlineto{\pgfqpoint{2.346219in}{2.398809in}}%
\pgfpathlineto{\pgfqpoint{2.346416in}{2.403595in}}%
\pgfpathlineto{\pgfqpoint{2.346811in}{2.392758in}}%
\pgfpathlineto{\pgfqpoint{2.347305in}{2.400026in}}%
\pgfpathlineto{\pgfqpoint{2.347798in}{2.377268in}}%
\pgfpathlineto{\pgfqpoint{2.348094in}{2.397973in}}%
\pgfpathlineto{\pgfqpoint{2.350068in}{2.628953in}}%
\pgfpathlineto{\pgfqpoint{2.350562in}{2.603109in}}%
\pgfpathlineto{\pgfqpoint{2.352536in}{2.376806in}}%
\pgfpathlineto{\pgfqpoint{2.352832in}{2.387165in}}%
\pgfpathlineto{\pgfqpoint{2.353029in}{2.390795in}}%
\pgfpathlineto{\pgfqpoint{2.353523in}{2.376809in}}%
\pgfpathlineto{\pgfqpoint{2.353621in}{2.376726in}}%
\pgfpathlineto{\pgfqpoint{2.354411in}{2.436737in}}%
\pgfpathlineto{\pgfqpoint{2.354904in}{2.395209in}}%
\pgfpathlineto{\pgfqpoint{2.355595in}{2.324824in}}%
\pgfpathlineto{\pgfqpoint{2.356188in}{2.364362in}}%
\pgfpathlineto{\pgfqpoint{2.358063in}{2.478095in}}%
\pgfpathlineto{\pgfqpoint{2.358260in}{2.464222in}}%
\pgfpathlineto{\pgfqpoint{2.358655in}{2.425113in}}%
\pgfpathlineto{\pgfqpoint{2.359543in}{2.439190in}}%
\pgfpathlineto{\pgfqpoint{2.359741in}{2.436969in}}%
\pgfpathlineto{\pgfqpoint{2.359938in}{2.442903in}}%
\pgfpathlineto{\pgfqpoint{2.360925in}{2.478246in}}%
\pgfpathlineto{\pgfqpoint{2.361616in}{2.475641in}}%
\pgfpathlineto{\pgfqpoint{2.362011in}{2.451804in}}%
\pgfpathlineto{\pgfqpoint{2.362702in}{2.470297in}}%
\pgfpathlineto{\pgfqpoint{2.364478in}{2.499834in}}%
\pgfpathlineto{\pgfqpoint{2.363491in}{2.469598in}}%
\pgfpathlineto{\pgfqpoint{2.364577in}{2.496995in}}%
\pgfpathlineto{\pgfqpoint{2.364972in}{2.475915in}}%
\pgfpathlineto{\pgfqpoint{2.365761in}{2.484301in}}%
\pgfpathlineto{\pgfqpoint{2.366156in}{2.494562in}}%
\pgfpathlineto{\pgfqpoint{2.366946in}{2.490958in}}%
\pgfpathlineto{\pgfqpoint{2.367242in}{2.492731in}}%
\pgfpathlineto{\pgfqpoint{2.367439in}{2.489567in}}%
\pgfpathlineto{\pgfqpoint{2.368722in}{2.469273in}}%
\pgfpathlineto{\pgfqpoint{2.368920in}{2.472564in}}%
\pgfpathlineto{\pgfqpoint{2.369314in}{2.491338in}}%
\pgfpathlineto{\pgfqpoint{2.370203in}{2.483465in}}%
\pgfpathlineto{\pgfqpoint{2.371091in}{2.476414in}}%
\pgfpathlineto{\pgfqpoint{2.370696in}{2.495012in}}%
\pgfpathlineto{\pgfqpoint{2.371288in}{2.481405in}}%
\pgfpathlineto{\pgfqpoint{2.371486in}{2.487855in}}%
\pgfpathlineto{\pgfqpoint{2.371881in}{2.477266in}}%
\pgfpathlineto{\pgfqpoint{2.372374in}{2.483736in}}%
\pgfpathlineto{\pgfqpoint{2.373657in}{2.455602in}}%
\pgfpathlineto{\pgfqpoint{2.374052in}{2.464486in}}%
\pgfpathlineto{\pgfqpoint{2.374151in}{2.464810in}}%
\pgfpathlineto{\pgfqpoint{2.374249in}{2.461792in}}%
\pgfpathlineto{\pgfqpoint{2.375533in}{2.433507in}}%
\pgfpathlineto{\pgfqpoint{2.375730in}{2.439111in}}%
\pgfpathlineto{\pgfqpoint{2.375829in}{2.440400in}}%
\pgfpathlineto{\pgfqpoint{2.376125in}{2.429817in}}%
\pgfpathlineto{\pgfqpoint{2.377309in}{2.400330in}}%
\pgfpathlineto{\pgfqpoint{2.377506in}{2.409636in}}%
\pgfpathlineto{\pgfqpoint{2.377704in}{2.417462in}}%
\pgfpathlineto{\pgfqpoint{2.378197in}{2.392701in}}%
\pgfpathlineto{\pgfqpoint{2.378395in}{2.397500in}}%
\pgfpathlineto{\pgfqpoint{2.378691in}{2.406408in}}%
\pgfpathlineto{\pgfqpoint{2.379382in}{2.395677in}}%
\pgfpathlineto{\pgfqpoint{2.379579in}{2.393512in}}%
\pgfpathlineto{\pgfqpoint{2.380073in}{2.399034in}}%
\pgfpathlineto{\pgfqpoint{2.380369in}{2.406087in}}%
\pgfpathlineto{\pgfqpoint{2.380764in}{2.389264in}}%
\pgfpathlineto{\pgfqpoint{2.380961in}{2.384231in}}%
\pgfpathlineto{\pgfqpoint{2.381356in}{2.396860in}}%
\pgfpathlineto{\pgfqpoint{2.381652in}{2.394410in}}%
\pgfpathlineto{\pgfqpoint{2.384613in}{2.442918in}}%
\pgfpathlineto{\pgfqpoint{2.384810in}{2.438573in}}%
\pgfpathlineto{\pgfqpoint{2.386488in}{2.398557in}}%
\pgfpathlineto{\pgfqpoint{2.386784in}{2.404672in}}%
\pgfpathlineto{\pgfqpoint{2.386982in}{2.399782in}}%
\pgfpathlineto{\pgfqpoint{2.388067in}{2.376632in}}%
\pgfpathlineto{\pgfqpoint{2.388265in}{2.379476in}}%
\pgfpathlineto{\pgfqpoint{2.389745in}{2.392870in}}%
\pgfpathlineto{\pgfqpoint{2.388758in}{2.371780in}}%
\pgfpathlineto{\pgfqpoint{2.389844in}{2.392377in}}%
\pgfpathlineto{\pgfqpoint{2.390732in}{2.388238in}}%
\pgfpathlineto{\pgfqpoint{2.390930in}{2.392688in}}%
\pgfpathlineto{\pgfqpoint{2.392015in}{2.413949in}}%
\pgfpathlineto{\pgfqpoint{2.392311in}{2.403913in}}%
\pgfpathlineto{\pgfqpoint{2.392410in}{2.402347in}}%
\pgfpathlineto{\pgfqpoint{2.392706in}{2.413788in}}%
\pgfpathlineto{\pgfqpoint{2.393298in}{2.418049in}}%
\pgfpathlineto{\pgfqpoint{2.393594in}{2.409323in}}%
\pgfpathlineto{\pgfqpoint{2.393891in}{2.393663in}}%
\pgfpathlineto{\pgfqpoint{2.394285in}{2.429460in}}%
\pgfpathlineto{\pgfqpoint{2.396259in}{2.634770in}}%
\pgfpathlineto{\pgfqpoint{2.396358in}{2.632206in}}%
\pgfpathlineto{\pgfqpoint{2.396851in}{2.539634in}}%
\pgfpathlineto{\pgfqpoint{2.398332in}{2.388456in}}%
\pgfpathlineto{\pgfqpoint{2.398431in}{2.389011in}}%
\pgfpathlineto{\pgfqpoint{2.400503in}{2.496498in}}%
\pgfpathlineto{\pgfqpoint{2.400799in}{2.483541in}}%
\pgfpathlineto{\pgfqpoint{2.401589in}{2.409026in}}%
\pgfpathlineto{\pgfqpoint{2.402379in}{2.435070in}}%
\pgfpathlineto{\pgfqpoint{2.402477in}{2.433607in}}%
\pgfpathlineto{\pgfqpoint{2.402872in}{2.442664in}}%
\pgfpathlineto{\pgfqpoint{2.403859in}{2.487852in}}%
\pgfpathlineto{\pgfqpoint{2.404254in}{2.461025in}}%
\pgfpathlineto{\pgfqpoint{2.404846in}{2.430814in}}%
\pgfpathlineto{\pgfqpoint{2.405340in}{2.446564in}}%
\pgfpathlineto{\pgfqpoint{2.406820in}{2.497304in}}%
\pgfpathlineto{\pgfqpoint{2.407017in}{2.493684in}}%
\pgfpathlineto{\pgfqpoint{2.408004in}{2.467916in}}%
\pgfpathlineto{\pgfqpoint{2.408498in}{2.482270in}}%
\pgfpathlineto{\pgfqpoint{2.410275in}{2.522377in}}%
\pgfpathlineto{\pgfqpoint{2.410472in}{2.517835in}}%
\pgfpathlineto{\pgfqpoint{2.411459in}{2.503937in}}%
\pgfpathlineto{\pgfqpoint{2.411656in}{2.506943in}}%
\pgfpathlineto{\pgfqpoint{2.412051in}{2.524760in}}%
\pgfpathlineto{\pgfqpoint{2.412742in}{2.515882in}}%
\pgfpathlineto{\pgfqpoint{2.413630in}{2.504104in}}%
\pgfpathlineto{\pgfqpoint{2.413828in}{2.509852in}}%
\pgfpathlineto{\pgfqpoint{2.414025in}{2.514830in}}%
\pgfpathlineto{\pgfqpoint{2.414420in}{2.495559in}}%
\pgfpathlineto{\pgfqpoint{2.414913in}{2.508825in}}%
\pgfpathlineto{\pgfqpoint{2.415900in}{2.517704in}}%
\pgfpathlineto{\pgfqpoint{2.416098in}{2.522032in}}%
\pgfpathlineto{\pgfqpoint{2.416493in}{2.502293in}}%
\pgfpathlineto{\pgfqpoint{2.416690in}{2.508875in}}%
\pgfpathlineto{\pgfqpoint{2.416887in}{2.519913in}}%
\pgfpathlineto{\pgfqpoint{2.417282in}{2.499250in}}%
\pgfpathlineto{\pgfqpoint{2.417776in}{2.510310in}}%
\pgfpathlineto{\pgfqpoint{2.418763in}{2.492038in}}%
\pgfpathlineto{\pgfqpoint{2.419256in}{2.494692in}}%
\pgfpathlineto{\pgfqpoint{2.419651in}{2.495979in}}%
\pgfpathlineto{\pgfqpoint{2.419848in}{2.494197in}}%
\pgfpathlineto{\pgfqpoint{2.421131in}{2.461268in}}%
\pgfpathlineto{\pgfqpoint{2.421822in}{2.473157in}}%
\pgfpathlineto{\pgfqpoint{2.421921in}{2.474631in}}%
\pgfpathlineto{\pgfqpoint{2.422118in}{2.466807in}}%
\pgfpathlineto{\pgfqpoint{2.422415in}{2.447281in}}%
\pgfpathlineto{\pgfqpoint{2.423204in}{2.462916in}}%
\pgfpathlineto{\pgfqpoint{2.423402in}{2.468250in}}%
\pgfpathlineto{\pgfqpoint{2.423796in}{2.451528in}}%
\pgfpathlineto{\pgfqpoint{2.424290in}{2.466194in}}%
\pgfpathlineto{\pgfqpoint{2.425277in}{2.456050in}}%
\pgfpathlineto{\pgfqpoint{2.425573in}{2.459793in}}%
\pgfpathlineto{\pgfqpoint{2.425672in}{2.460116in}}%
\pgfpathlineto{\pgfqpoint{2.425770in}{2.458529in}}%
\pgfpathlineto{\pgfqpoint{2.426955in}{2.442534in}}%
\pgfpathlineto{\pgfqpoint{2.426560in}{2.469428in}}%
\pgfpathlineto{\pgfqpoint{2.427053in}{2.442641in}}%
\pgfpathlineto{\pgfqpoint{2.428336in}{2.485106in}}%
\pgfpathlineto{\pgfqpoint{2.428633in}{2.476889in}}%
\pgfpathlineto{\pgfqpoint{2.430804in}{2.518712in}}%
\pgfpathlineto{\pgfqpoint{2.431890in}{2.511359in}}%
\pgfpathlineto{\pgfqpoint{2.436134in}{2.382147in}}%
\pgfpathlineto{\pgfqpoint{2.436627in}{2.409641in}}%
\pgfpathlineto{\pgfqpoint{2.438700in}{2.497202in}}%
\pgfpathlineto{\pgfqpoint{2.438996in}{2.491450in}}%
\pgfpathlineto{\pgfqpoint{2.439193in}{2.492298in}}%
\pgfpathlineto{\pgfqpoint{2.439292in}{2.490567in}}%
\pgfpathlineto{\pgfqpoint{2.440279in}{2.463764in}}%
\pgfpathlineto{\pgfqpoint{2.440476in}{2.475877in}}%
\pgfpathlineto{\pgfqpoint{2.442549in}{2.721962in}}%
\pgfpathlineto{\pgfqpoint{2.442845in}{2.698867in}}%
\pgfpathlineto{\pgfqpoint{2.444720in}{2.442161in}}%
\pgfpathlineto{\pgfqpoint{2.444819in}{2.445236in}}%
\pgfpathlineto{\pgfqpoint{2.446793in}{2.543772in}}%
\pgfpathlineto{\pgfqpoint{2.447089in}{2.527275in}}%
\pgfpathlineto{\pgfqpoint{2.447780in}{2.465097in}}%
\pgfpathlineto{\pgfqpoint{2.448471in}{2.486292in}}%
\pgfpathlineto{\pgfqpoint{2.450050in}{2.539366in}}%
\pgfpathlineto{\pgfqpoint{2.450149in}{2.539771in}}%
\pgfpathlineto{\pgfqpoint{2.450248in}{2.535625in}}%
\pgfpathlineto{\pgfqpoint{2.451037in}{2.490690in}}%
\pgfpathlineto{\pgfqpoint{2.451728in}{2.505711in}}%
\pgfpathlineto{\pgfqpoint{2.453406in}{2.552381in}}%
\pgfpathlineto{\pgfqpoint{2.453505in}{2.547883in}}%
\pgfpathlineto{\pgfqpoint{2.454393in}{2.518897in}}%
\pgfpathlineto{\pgfqpoint{2.454788in}{2.534166in}}%
\pgfpathlineto{\pgfqpoint{2.456268in}{2.558438in}}%
\pgfpathlineto{\pgfqpoint{2.456762in}{2.553859in}}%
\pgfpathlineto{\pgfqpoint{2.457453in}{2.531769in}}%
\pgfpathlineto{\pgfqpoint{2.458144in}{2.546677in}}%
\pgfpathlineto{\pgfqpoint{2.459131in}{2.536156in}}%
\pgfpathlineto{\pgfqpoint{2.458538in}{2.548813in}}%
\pgfpathlineto{\pgfqpoint{2.459525in}{2.542344in}}%
\pgfpathlineto{\pgfqpoint{2.459723in}{2.545323in}}%
\pgfpathlineto{\pgfqpoint{2.460118in}{2.536138in}}%
\pgfpathlineto{\pgfqpoint{2.460512in}{2.543247in}}%
\pgfpathlineto{\pgfqpoint{2.461598in}{2.526597in}}%
\pgfpathlineto{\pgfqpoint{2.461795in}{2.535922in}}%
\pgfpathlineto{\pgfqpoint{2.461993in}{2.541996in}}%
\pgfpathlineto{\pgfqpoint{2.462782in}{2.532375in}}%
\pgfpathlineto{\pgfqpoint{2.464756in}{2.504512in}}%
\pgfpathlineto{\pgfqpoint{2.464855in}{2.506153in}}%
\pgfpathlineto{\pgfqpoint{2.465151in}{2.516861in}}%
\pgfpathlineto{\pgfqpoint{2.465645in}{2.495246in}}%
\pgfpathlineto{\pgfqpoint{2.466039in}{2.497403in}}%
\pgfpathlineto{\pgfqpoint{2.466434in}{2.481326in}}%
\pgfpathlineto{\pgfqpoint{2.468013in}{2.455709in}}%
\pgfpathlineto{\pgfqpoint{2.468408in}{2.467785in}}%
\pgfpathlineto{\pgfqpoint{2.468902in}{2.454617in}}%
\pgfpathlineto{\pgfqpoint{2.470086in}{2.438413in}}%
\pgfpathlineto{\pgfqpoint{2.470284in}{2.439791in}}%
\pgfpathlineto{\pgfqpoint{2.470481in}{2.442573in}}%
\pgfpathlineto{\pgfqpoint{2.470974in}{2.436505in}}%
\pgfpathlineto{\pgfqpoint{2.471369in}{2.441427in}}%
\pgfpathlineto{\pgfqpoint{2.471764in}{2.430205in}}%
\pgfpathlineto{\pgfqpoint{2.472060in}{2.441991in}}%
\pgfpathlineto{\pgfqpoint{2.472948in}{2.447145in}}%
\pgfpathlineto{\pgfqpoint{2.472554in}{2.439382in}}%
\pgfpathlineto{\pgfqpoint{2.473047in}{2.443407in}}%
\pgfpathlineto{\pgfqpoint{2.473343in}{2.432711in}}%
\pgfpathlineto{\pgfqpoint{2.474133in}{2.441058in}}%
\pgfpathlineto{\pgfqpoint{2.476107in}{2.487517in}}%
\pgfpathlineto{\pgfqpoint{2.476304in}{2.485677in}}%
\pgfpathlineto{\pgfqpoint{2.476600in}{2.479767in}}%
\pgfpathlineto{\pgfqpoint{2.477094in}{2.492112in}}%
\pgfpathlineto{\pgfqpoint{2.477291in}{2.494751in}}%
\pgfpathlineto{\pgfqpoint{2.477686in}{2.486546in}}%
\pgfpathlineto{\pgfqpoint{2.478081in}{2.492186in}}%
\pgfpathlineto{\pgfqpoint{2.478377in}{2.482674in}}%
\pgfpathlineto{\pgfqpoint{2.479660in}{2.427807in}}%
\pgfpathlineto{\pgfqpoint{2.480153in}{2.430724in}}%
\pgfpathlineto{\pgfqpoint{2.480449in}{2.434845in}}%
\pgfpathlineto{\pgfqpoint{2.480647in}{2.424685in}}%
\pgfpathlineto{\pgfqpoint{2.480943in}{2.408107in}}%
\pgfpathlineto{\pgfqpoint{2.481831in}{2.417594in}}%
\pgfpathlineto{\pgfqpoint{2.481930in}{2.418081in}}%
\pgfpathlineto{\pgfqpoint{2.482127in}{2.413720in}}%
\pgfpathlineto{\pgfqpoint{2.483312in}{2.408542in}}%
\pgfpathlineto{\pgfqpoint{2.483410in}{2.408665in}}%
\pgfpathlineto{\pgfqpoint{2.484003in}{2.407461in}}%
\pgfpathlineto{\pgfqpoint{2.484101in}{2.408105in}}%
\pgfpathlineto{\pgfqpoint{2.484299in}{2.411239in}}%
\pgfpathlineto{\pgfqpoint{2.484694in}{2.405364in}}%
\pgfpathlineto{\pgfqpoint{2.485187in}{2.409662in}}%
\pgfpathlineto{\pgfqpoint{2.485483in}{2.400607in}}%
\pgfpathlineto{\pgfqpoint{2.486273in}{2.407493in}}%
\pgfpathlineto{\pgfqpoint{2.486371in}{2.408187in}}%
\pgfpathlineto{\pgfqpoint{2.486470in}{2.404907in}}%
\pgfpathlineto{\pgfqpoint{2.486668in}{2.392619in}}%
\pgfpathlineto{\pgfqpoint{2.486964in}{2.428196in}}%
\pgfpathlineto{\pgfqpoint{2.488642in}{2.631110in}}%
\pgfpathlineto{\pgfqpoint{2.488938in}{2.643483in}}%
\pgfpathlineto{\pgfqpoint{2.489234in}{2.620684in}}%
\pgfpathlineto{\pgfqpoint{2.490122in}{2.386237in}}%
\pgfpathlineto{\pgfqpoint{2.491208in}{2.397835in}}%
\pgfpathlineto{\pgfqpoint{2.493182in}{2.501381in}}%
\pgfpathlineto{\pgfqpoint{2.493379in}{2.493989in}}%
\pgfpathlineto{\pgfqpoint{2.494366in}{2.407468in}}%
\pgfpathlineto{\pgfqpoint{2.494958in}{2.436876in}}%
\pgfpathlineto{\pgfqpoint{2.496044in}{2.476632in}}%
\pgfpathlineto{\pgfqpoint{2.496537in}{2.493692in}}%
\pgfpathlineto{\pgfqpoint{2.496932in}{2.475995in}}%
\pgfpathlineto{\pgfqpoint{2.497919in}{2.447553in}}%
\pgfpathlineto{\pgfqpoint{2.498117in}{2.456429in}}%
\pgfpathlineto{\pgfqpoint{2.499498in}{2.482670in}}%
\pgfpathlineto{\pgfqpoint{2.499597in}{2.483302in}}%
\pgfpathlineto{\pgfqpoint{2.499696in}{2.480491in}}%
\pgfpathlineto{\pgfqpoint{2.500683in}{2.453922in}}%
\pgfpathlineto{\pgfqpoint{2.500979in}{2.464642in}}%
\pgfpathlineto{\pgfqpoint{2.501966in}{2.493451in}}%
\pgfpathlineto{\pgfqpoint{2.502361in}{2.479655in}}%
\pgfpathlineto{\pgfqpoint{2.502459in}{2.479465in}}%
\pgfpathlineto{\pgfqpoint{2.503052in}{2.493752in}}%
\pgfpathlineto{\pgfqpoint{2.503545in}{2.483651in}}%
\pgfpathlineto{\pgfqpoint{2.504137in}{2.467486in}}%
\pgfpathlineto{\pgfqpoint{2.504532in}{2.485496in}}%
\pgfpathlineto{\pgfqpoint{2.505519in}{2.493531in}}%
\pgfpathlineto{\pgfqpoint{2.505026in}{2.475006in}}%
\pgfpathlineto{\pgfqpoint{2.505716in}{2.487692in}}%
\pgfpathlineto{\pgfqpoint{2.507197in}{2.451879in}}%
\pgfpathlineto{\pgfqpoint{2.507394in}{2.460468in}}%
\pgfpathlineto{\pgfqpoint{2.508579in}{2.483364in}}%
\pgfpathlineto{\pgfqpoint{2.508776in}{2.481115in}}%
\pgfpathlineto{\pgfqpoint{2.510454in}{2.468412in}}%
\pgfpathlineto{\pgfqpoint{2.509270in}{2.483096in}}%
\pgfpathlineto{\pgfqpoint{2.510553in}{2.469875in}}%
\pgfpathlineto{\pgfqpoint{2.510849in}{2.474156in}}%
\pgfpathlineto{\pgfqpoint{2.511441in}{2.466903in}}%
\pgfpathlineto{\pgfqpoint{2.515192in}{2.346429in}}%
\pgfpathlineto{\pgfqpoint{2.515290in}{2.346072in}}%
\pgfpathlineto{\pgfqpoint{2.515389in}{2.348912in}}%
\pgfpathlineto{\pgfqpoint{2.517758in}{2.439163in}}%
\pgfpathlineto{\pgfqpoint{2.517856in}{2.438378in}}%
\pgfpathlineto{\pgfqpoint{2.518350in}{2.407395in}}%
\pgfpathlineto{\pgfqpoint{2.519337in}{2.411775in}}%
\pgfpathlineto{\pgfqpoint{2.520225in}{2.404755in}}%
\pgfpathlineto{\pgfqpoint{2.520423in}{2.408399in}}%
\pgfpathlineto{\pgfqpoint{2.522594in}{2.450780in}}%
\pgfpathlineto{\pgfqpoint{2.522693in}{2.450356in}}%
\pgfpathlineto{\pgfqpoint{2.523976in}{2.440855in}}%
\pgfpathlineto{\pgfqpoint{2.523482in}{2.452062in}}%
\pgfpathlineto{\pgfqpoint{2.524074in}{2.442594in}}%
\pgfpathlineto{\pgfqpoint{2.524469in}{2.460128in}}%
\pgfpathlineto{\pgfqpoint{2.525061in}{2.441192in}}%
\pgfpathlineto{\pgfqpoint{2.525456in}{2.424369in}}%
\pgfpathlineto{\pgfqpoint{2.526937in}{2.392249in}}%
\pgfpathlineto{\pgfqpoint{2.527134in}{2.396023in}}%
\pgfpathlineto{\pgfqpoint{2.527332in}{2.401156in}}%
\pgfpathlineto{\pgfqpoint{2.527924in}{2.384885in}}%
\pgfpathlineto{\pgfqpoint{2.529305in}{2.392078in}}%
\pgfpathlineto{\pgfqpoint{2.530391in}{2.407743in}}%
\pgfpathlineto{\pgfqpoint{2.530589in}{2.404252in}}%
\pgfpathlineto{\pgfqpoint{2.530687in}{2.402740in}}%
\pgfpathlineto{\pgfqpoint{2.531181in}{2.410102in}}%
\pgfpathlineto{\pgfqpoint{2.531279in}{2.409627in}}%
\pgfpathlineto{\pgfqpoint{2.531477in}{2.411355in}}%
\pgfpathlineto{\pgfqpoint{2.531674in}{2.415193in}}%
\pgfpathlineto{\pgfqpoint{2.532168in}{2.405903in}}%
\pgfpathlineto{\pgfqpoint{2.532266in}{2.406325in}}%
\pgfpathlineto{\pgfqpoint{2.532464in}{2.402825in}}%
\pgfpathlineto{\pgfqpoint{2.532661in}{2.393408in}}%
\pgfpathlineto{\pgfqpoint{2.532957in}{2.416057in}}%
\pgfpathlineto{\pgfqpoint{2.535129in}{2.638719in}}%
\pgfpathlineto{\pgfqpoint{2.535227in}{2.641948in}}%
\pgfpathlineto{\pgfqpoint{2.535425in}{2.616711in}}%
\pgfpathlineto{\pgfqpoint{2.536313in}{2.366987in}}%
\pgfpathlineto{\pgfqpoint{2.537399in}{2.379053in}}%
\pgfpathlineto{\pgfqpoint{2.539373in}{2.481194in}}%
\pgfpathlineto{\pgfqpoint{2.539669in}{2.468627in}}%
\pgfpathlineto{\pgfqpoint{2.540557in}{2.400215in}}%
\pgfpathlineto{\pgfqpoint{2.541051in}{2.434599in}}%
\pgfpathlineto{\pgfqpoint{2.542729in}{2.487250in}}%
\pgfpathlineto{\pgfqpoint{2.541544in}{2.429803in}}%
\pgfpathlineto{\pgfqpoint{2.543025in}{2.463905in}}%
\pgfpathlineto{\pgfqpoint{2.543814in}{2.429860in}}%
\pgfpathlineto{\pgfqpoint{2.544209in}{2.453853in}}%
\pgfpathlineto{\pgfqpoint{2.545690in}{2.475024in}}%
\pgfpathlineto{\pgfqpoint{2.546874in}{2.464215in}}%
\pgfpathlineto{\pgfqpoint{2.547269in}{2.450546in}}%
\pgfpathlineto{\pgfqpoint{2.547663in}{2.475897in}}%
\pgfpathlineto{\pgfqpoint{2.547861in}{2.481057in}}%
\pgfpathlineto{\pgfqpoint{2.548354in}{2.457500in}}%
\pgfpathlineto{\pgfqpoint{2.548749in}{2.477947in}}%
\pgfpathlineto{\pgfqpoint{2.549243in}{2.479958in}}%
\pgfpathlineto{\pgfqpoint{2.549440in}{2.474172in}}%
\pgfpathlineto{\pgfqpoint{2.549835in}{2.459410in}}%
\pgfpathlineto{\pgfqpoint{2.550624in}{2.466772in}}%
\pgfpathlineto{\pgfqpoint{2.551908in}{2.476897in}}%
\pgfpathlineto{\pgfqpoint{2.552006in}{2.476334in}}%
\pgfpathlineto{\pgfqpoint{2.553289in}{2.448424in}}%
\pgfpathlineto{\pgfqpoint{2.553980in}{2.457482in}}%
\pgfpathlineto{\pgfqpoint{2.554178in}{2.459001in}}%
\pgfpathlineto{\pgfqpoint{2.554375in}{2.456471in}}%
\pgfpathlineto{\pgfqpoint{2.554671in}{2.443433in}}%
\pgfpathlineto{\pgfqpoint{2.555658in}{2.447286in}}%
\pgfpathlineto{\pgfqpoint{2.556250in}{2.452879in}}%
\pgfpathlineto{\pgfqpoint{2.556546in}{2.445463in}}%
\pgfpathlineto{\pgfqpoint{2.557533in}{2.430180in}}%
\pgfpathlineto{\pgfqpoint{2.557731in}{2.439996in}}%
\pgfpathlineto{\pgfqpoint{2.557928in}{2.445436in}}%
\pgfpathlineto{\pgfqpoint{2.558422in}{2.427005in}}%
\pgfpathlineto{\pgfqpoint{2.558520in}{2.428145in}}%
\pgfpathlineto{\pgfqpoint{2.558816in}{2.417783in}}%
\pgfpathlineto{\pgfqpoint{2.560100in}{2.402444in}}%
\pgfpathlineto{\pgfqpoint{2.560198in}{2.401739in}}%
\pgfpathlineto{\pgfqpoint{2.560494in}{2.406104in}}%
\pgfpathlineto{\pgfqpoint{2.560790in}{2.402852in}}%
\pgfpathlineto{\pgfqpoint{2.561185in}{2.416721in}}%
\pgfpathlineto{\pgfqpoint{2.561481in}{2.394775in}}%
\pgfpathlineto{\pgfqpoint{2.561679in}{2.386840in}}%
\pgfpathlineto{\pgfqpoint{2.562370in}{2.403398in}}%
\pgfpathlineto{\pgfqpoint{2.562666in}{2.411118in}}%
\pgfpathlineto{\pgfqpoint{2.563061in}{2.390079in}}%
\pgfpathlineto{\pgfqpoint{2.563258in}{2.393799in}}%
\pgfpathlineto{\pgfqpoint{2.563850in}{2.403936in}}%
\pgfpathlineto{\pgfqpoint{2.564245in}{2.394854in}}%
\pgfpathlineto{\pgfqpoint{2.564442in}{2.391896in}}%
\pgfpathlineto{\pgfqpoint{2.565034in}{2.401646in}}%
\pgfpathlineto{\pgfqpoint{2.566219in}{2.391448in}}%
\pgfpathlineto{\pgfqpoint{2.565725in}{2.402090in}}%
\pgfpathlineto{\pgfqpoint{2.566416in}{2.398404in}}%
\pgfpathlineto{\pgfqpoint{2.568094in}{2.430175in}}%
\pgfpathlineto{\pgfqpoint{2.568193in}{2.429913in}}%
\pgfpathlineto{\pgfqpoint{2.568390in}{2.431245in}}%
\pgfpathlineto{\pgfqpoint{2.568884in}{2.453707in}}%
\pgfpathlineto{\pgfqpoint{2.569772in}{2.448611in}}%
\pgfpathlineto{\pgfqpoint{2.570660in}{2.440936in}}%
\pgfpathlineto{\pgfqpoint{2.570364in}{2.450424in}}%
\pgfpathlineto{\pgfqpoint{2.570858in}{2.446202in}}%
\pgfpathlineto{\pgfqpoint{2.571055in}{2.452627in}}%
\pgfpathlineto{\pgfqpoint{2.571351in}{2.426995in}}%
\pgfpathlineto{\pgfqpoint{2.572930in}{2.378599in}}%
\pgfpathlineto{\pgfqpoint{2.573621in}{2.374665in}}%
\pgfpathlineto{\pgfqpoint{2.573819in}{2.379632in}}%
\pgfpathlineto{\pgfqpoint{2.574016in}{2.384382in}}%
\pgfpathlineto{\pgfqpoint{2.574510in}{2.374799in}}%
\pgfpathlineto{\pgfqpoint{2.574806in}{2.377116in}}%
\pgfpathlineto{\pgfqpoint{2.575102in}{2.374326in}}%
\pgfpathlineto{\pgfqpoint{2.575398in}{2.379092in}}%
\pgfpathlineto{\pgfqpoint{2.576484in}{2.393955in}}%
\pgfpathlineto{\pgfqpoint{2.575891in}{2.377456in}}%
\pgfpathlineto{\pgfqpoint{2.576681in}{2.387808in}}%
\pgfpathlineto{\pgfqpoint{2.576977in}{2.372608in}}%
\pgfpathlineto{\pgfqpoint{2.577865in}{2.377447in}}%
\pgfpathlineto{\pgfqpoint{2.578260in}{2.385193in}}%
\pgfpathlineto{\pgfqpoint{2.578754in}{2.374976in}}%
\pgfpathlineto{\pgfqpoint{2.579050in}{2.367962in}}%
\pgfpathlineto{\pgfqpoint{2.579346in}{2.380756in}}%
\pgfpathlineto{\pgfqpoint{2.581517in}{2.605395in}}%
\pgfpathlineto{\pgfqpoint{2.581813in}{2.574570in}}%
\pgfpathlineto{\pgfqpoint{2.582702in}{2.343235in}}%
\pgfpathlineto{\pgfqpoint{2.583886in}{2.371050in}}%
\pgfpathlineto{\pgfqpoint{2.585761in}{2.456498in}}%
\pgfpathlineto{\pgfqpoint{2.586255in}{2.413190in}}%
\pgfpathlineto{\pgfqpoint{2.586748in}{2.366700in}}%
\pgfpathlineto{\pgfqpoint{2.587538in}{2.382512in}}%
\pgfpathlineto{\pgfqpoint{2.588130in}{2.386453in}}%
\pgfpathlineto{\pgfqpoint{2.588821in}{2.398703in}}%
\pgfpathlineto{\pgfqpoint{2.589117in}{2.388357in}}%
\pgfpathlineto{\pgfqpoint{2.589907in}{2.316751in}}%
\pgfpathlineto{\pgfqpoint{2.590696in}{2.353276in}}%
\pgfpathlineto{\pgfqpoint{2.592473in}{2.450750in}}%
\pgfpathlineto{\pgfqpoint{2.592769in}{2.440875in}}%
\pgfpathlineto{\pgfqpoint{2.593361in}{2.435601in}}%
\pgfpathlineto{\pgfqpoint{2.593756in}{2.442642in}}%
\pgfpathlineto{\pgfqpoint{2.595532in}{2.465735in}}%
\pgfpathlineto{\pgfqpoint{2.595730in}{2.464352in}}%
\pgfpathlineto{\pgfqpoint{2.596026in}{2.460874in}}%
\pgfpathlineto{\pgfqpoint{2.596421in}{2.468974in}}%
\pgfpathlineto{\pgfqpoint{2.597506in}{2.487660in}}%
\pgfpathlineto{\pgfqpoint{2.597605in}{2.490202in}}%
\pgfpathlineto{\pgfqpoint{2.598000in}{2.472149in}}%
\pgfpathlineto{\pgfqpoint{2.599086in}{2.464101in}}%
\pgfpathlineto{\pgfqpoint{2.598493in}{2.475894in}}%
\pgfpathlineto{\pgfqpoint{2.599283in}{2.467743in}}%
\pgfpathlineto{\pgfqpoint{2.599480in}{2.470978in}}%
\pgfpathlineto{\pgfqpoint{2.599777in}{2.460665in}}%
\pgfpathlineto{\pgfqpoint{2.599974in}{2.453320in}}%
\pgfpathlineto{\pgfqpoint{2.600566in}{2.475846in}}%
\pgfpathlineto{\pgfqpoint{2.600665in}{2.476056in}}%
\pgfpathlineto{\pgfqpoint{2.600764in}{2.474729in}}%
\pgfpathlineto{\pgfqpoint{2.601158in}{2.467688in}}%
\pgfpathlineto{\pgfqpoint{2.601454in}{2.477918in}}%
\pgfpathlineto{\pgfqpoint{2.603034in}{2.508570in}}%
\pgfpathlineto{\pgfqpoint{2.603132in}{2.508748in}}%
\pgfpathlineto{\pgfqpoint{2.604119in}{2.497235in}}%
\pgfpathlineto{\pgfqpoint{2.604317in}{2.502060in}}%
\pgfpathlineto{\pgfqpoint{2.604514in}{2.508190in}}%
\pgfpathlineto{\pgfqpoint{2.605008in}{2.485922in}}%
\pgfpathlineto{\pgfqpoint{2.605106in}{2.487095in}}%
\pgfpathlineto{\pgfqpoint{2.605304in}{2.490211in}}%
\pgfpathlineto{\pgfqpoint{2.605698in}{2.476699in}}%
\pgfpathlineto{\pgfqpoint{2.607376in}{2.445667in}}%
\pgfpathlineto{\pgfqpoint{2.608067in}{2.436472in}}%
\pgfpathlineto{\pgfqpoint{2.607672in}{2.446730in}}%
\pgfpathlineto{\pgfqpoint{2.608265in}{2.445577in}}%
\pgfpathlineto{\pgfqpoint{2.608462in}{2.450778in}}%
\pgfpathlineto{\pgfqpoint{2.608857in}{2.426675in}}%
\pgfpathlineto{\pgfqpoint{2.608956in}{2.426737in}}%
\pgfpathlineto{\pgfqpoint{2.609252in}{2.433202in}}%
\pgfpathlineto{\pgfqpoint{2.609646in}{2.419114in}}%
\pgfpathlineto{\pgfqpoint{2.610041in}{2.429487in}}%
\pgfpathlineto{\pgfqpoint{2.610535in}{2.415943in}}%
\pgfpathlineto{\pgfqpoint{2.611423in}{2.422374in}}%
\pgfpathlineto{\pgfqpoint{2.611620in}{2.430513in}}%
\pgfpathlineto{\pgfqpoint{2.612015in}{2.415961in}}%
\pgfpathlineto{\pgfqpoint{2.612509in}{2.421595in}}%
\pgfpathlineto{\pgfqpoint{2.612903in}{2.422767in}}%
\pgfpathlineto{\pgfqpoint{2.613101in}{2.421037in}}%
\pgfpathlineto{\pgfqpoint{2.613200in}{2.420533in}}%
\pgfpathlineto{\pgfqpoint{2.613397in}{2.423515in}}%
\pgfpathlineto{\pgfqpoint{2.614680in}{2.435704in}}%
\pgfpathlineto{\pgfqpoint{2.614779in}{2.435586in}}%
\pgfpathlineto{\pgfqpoint{2.615174in}{2.428152in}}%
\pgfpathlineto{\pgfqpoint{2.615371in}{2.437928in}}%
\pgfpathlineto{\pgfqpoint{2.616358in}{2.453686in}}%
\pgfpathlineto{\pgfqpoint{2.616555in}{2.446124in}}%
\pgfpathlineto{\pgfqpoint{2.619911in}{2.325876in}}%
\pgfpathlineto{\pgfqpoint{2.620010in}{2.326926in}}%
\pgfpathlineto{\pgfqpoint{2.620207in}{2.330361in}}%
\pgfpathlineto{\pgfqpoint{2.620503in}{2.318514in}}%
\pgfpathlineto{\pgfqpoint{2.620799in}{2.308566in}}%
\pgfpathlineto{\pgfqpoint{2.621688in}{2.312678in}}%
\pgfpathlineto{\pgfqpoint{2.621786in}{2.313150in}}%
\pgfpathlineto{\pgfqpoint{2.621984in}{2.309286in}}%
\pgfpathlineto{\pgfqpoint{2.622181in}{2.305248in}}%
\pgfpathlineto{\pgfqpoint{2.622576in}{2.323910in}}%
\pgfpathlineto{\pgfqpoint{2.623464in}{2.330358in}}%
\pgfpathlineto{\pgfqpoint{2.622971in}{2.316830in}}%
\pgfpathlineto{\pgfqpoint{2.623563in}{2.327212in}}%
\pgfpathlineto{\pgfqpoint{2.623859in}{2.317768in}}%
\pgfpathlineto{\pgfqpoint{2.624254in}{2.333339in}}%
\pgfpathlineto{\pgfqpoint{2.624649in}{2.324267in}}%
\pgfpathlineto{\pgfqpoint{2.625142in}{2.341034in}}%
\pgfpathlineto{\pgfqpoint{2.625734in}{2.326147in}}%
\pgfpathlineto{\pgfqpoint{2.625833in}{2.325140in}}%
\pgfpathlineto{\pgfqpoint{2.626129in}{2.331808in}}%
\pgfpathlineto{\pgfqpoint{2.626919in}{2.510884in}}%
\pgfpathlineto{\pgfqpoint{2.628301in}{2.566240in}}%
\pgfpathlineto{\pgfqpoint{2.628399in}{2.567229in}}%
\pgfpathlineto{\pgfqpoint{2.628597in}{2.562155in}}%
\pgfpathlineto{\pgfqpoint{2.629386in}{2.346700in}}%
\pgfpathlineto{\pgfqpoint{2.629584in}{2.330511in}}%
\pgfpathlineto{\pgfqpoint{2.629978in}{2.363179in}}%
\pgfpathlineto{\pgfqpoint{2.630472in}{2.338218in}}%
\pgfpathlineto{\pgfqpoint{2.632742in}{2.451297in}}%
\pgfpathlineto{\pgfqpoint{2.633038in}{2.429105in}}%
\pgfpathlineto{\pgfqpoint{2.633532in}{2.365970in}}%
\pgfpathlineto{\pgfqpoint{2.634321in}{2.390873in}}%
\pgfpathlineto{\pgfqpoint{2.635802in}{2.432477in}}%
\pgfpathlineto{\pgfqpoint{2.636098in}{2.419939in}}%
\pgfpathlineto{\pgfqpoint{2.637085in}{2.381330in}}%
\pgfpathlineto{\pgfqpoint{2.637480in}{2.394937in}}%
\pgfpathlineto{\pgfqpoint{2.639157in}{2.439001in}}%
\pgfpathlineto{\pgfqpoint{2.639256in}{2.438361in}}%
\pgfpathlineto{\pgfqpoint{2.640144in}{2.408894in}}%
\pgfpathlineto{\pgfqpoint{2.640539in}{2.423699in}}%
\pgfpathlineto{\pgfqpoint{2.641329in}{2.442464in}}%
\pgfpathlineto{\pgfqpoint{2.642118in}{2.441580in}}%
\pgfpathlineto{\pgfqpoint{2.643105in}{2.423927in}}%
\pgfpathlineto{\pgfqpoint{2.642711in}{2.443151in}}%
\pgfpathlineto{\pgfqpoint{2.643303in}{2.433521in}}%
\pgfpathlineto{\pgfqpoint{2.644191in}{2.457055in}}%
\pgfpathlineto{\pgfqpoint{2.643796in}{2.429544in}}%
\pgfpathlineto{\pgfqpoint{2.644487in}{2.444234in}}%
\pgfpathlineto{\pgfqpoint{2.644586in}{2.442851in}}%
\pgfpathlineto{\pgfqpoint{2.644783in}{2.452320in}}%
\pgfpathlineto{\pgfqpoint{2.644882in}{2.456232in}}%
\pgfpathlineto{\pgfqpoint{2.645277in}{2.443176in}}%
\pgfpathlineto{\pgfqpoint{2.645770in}{2.453077in}}%
\pgfpathlineto{\pgfqpoint{2.646955in}{2.419745in}}%
\pgfpathlineto{\pgfqpoint{2.647251in}{2.426070in}}%
\pgfpathlineto{\pgfqpoint{2.647448in}{2.431044in}}%
\pgfpathlineto{\pgfqpoint{2.647942in}{2.422534in}}%
\pgfpathlineto{\pgfqpoint{2.648336in}{2.427829in}}%
\pgfpathlineto{\pgfqpoint{2.648633in}{2.428659in}}%
\pgfpathlineto{\pgfqpoint{2.649126in}{2.427142in}}%
\pgfpathlineto{\pgfqpoint{2.649323in}{2.427434in}}%
\pgfpathlineto{\pgfqpoint{2.649817in}{2.406739in}}%
\pgfpathlineto{\pgfqpoint{2.651100in}{2.413617in}}%
\pgfpathlineto{\pgfqpoint{2.651199in}{2.414756in}}%
\pgfpathlineto{\pgfqpoint{2.651396in}{2.409498in}}%
\pgfpathlineto{\pgfqpoint{2.653173in}{2.368466in}}%
\pgfpathlineto{\pgfqpoint{2.653271in}{2.369662in}}%
\pgfpathlineto{\pgfqpoint{2.653469in}{2.372349in}}%
\pgfpathlineto{\pgfqpoint{2.654160in}{2.366815in}}%
\pgfpathlineto{\pgfqpoint{2.655344in}{2.338328in}}%
\pgfpathlineto{\pgfqpoint{2.655739in}{2.348149in}}%
\pgfpathlineto{\pgfqpoint{2.658009in}{2.327585in}}%
\pgfpathlineto{\pgfqpoint{2.658206in}{2.330036in}}%
\pgfpathlineto{\pgfqpoint{2.659292in}{2.347329in}}%
\pgfpathlineto{\pgfqpoint{2.658798in}{2.326476in}}%
\pgfpathlineto{\pgfqpoint{2.659489in}{2.338085in}}%
\pgfpathlineto{\pgfqpoint{2.660378in}{2.325916in}}%
\pgfpathlineto{\pgfqpoint{2.660575in}{2.335938in}}%
\pgfpathlineto{\pgfqpoint{2.660970in}{2.362602in}}%
\pgfpathlineto{\pgfqpoint{2.661858in}{2.358853in}}%
\pgfpathlineto{\pgfqpoint{2.661957in}{2.359446in}}%
\pgfpathlineto{\pgfqpoint{2.662154in}{2.356344in}}%
\pgfpathlineto{\pgfqpoint{2.664128in}{2.312383in}}%
\pgfpathlineto{\pgfqpoint{2.664622in}{2.315429in}}%
\pgfpathlineto{\pgfqpoint{2.666991in}{2.349926in}}%
\pgfpathlineto{\pgfqpoint{2.667287in}{2.340909in}}%
\pgfpathlineto{\pgfqpoint{2.668175in}{2.313224in}}%
\pgfpathlineto{\pgfqpoint{2.668767in}{2.324552in}}%
\pgfpathlineto{\pgfqpoint{2.669261in}{2.326127in}}%
\pgfpathlineto{\pgfqpoint{2.669359in}{2.325161in}}%
\pgfpathlineto{\pgfqpoint{2.669754in}{2.311028in}}%
\pgfpathlineto{\pgfqpoint{2.670248in}{2.328692in}}%
\pgfpathlineto{\pgfqpoint{2.670840in}{2.332850in}}%
\pgfpathlineto{\pgfqpoint{2.671136in}{2.326492in}}%
\pgfpathlineto{\pgfqpoint{2.671333in}{2.323453in}}%
\pgfpathlineto{\pgfqpoint{2.671728in}{2.335990in}}%
\pgfpathlineto{\pgfqpoint{2.672222in}{2.323989in}}%
\pgfpathlineto{\pgfqpoint{2.673110in}{2.321539in}}%
\pgfpathlineto{\pgfqpoint{2.673801in}{2.406690in}}%
\pgfpathlineto{\pgfqpoint{2.675380in}{2.548115in}}%
\pgfpathlineto{\pgfqpoint{2.675577in}{2.552782in}}%
\pgfpathlineto{\pgfqpoint{2.675775in}{2.543207in}}%
\pgfpathlineto{\pgfqpoint{2.676860in}{2.298128in}}%
\pgfpathlineto{\pgfqpoint{2.678242in}{2.333889in}}%
\pgfpathlineto{\pgfqpoint{2.679328in}{2.368235in}}%
\pgfpathlineto{\pgfqpoint{2.679920in}{2.419497in}}%
\pgfpathlineto{\pgfqpoint{2.680414in}{2.367928in}}%
\pgfpathlineto{\pgfqpoint{2.680710in}{2.334783in}}%
\pgfpathlineto{\pgfqpoint{2.681598in}{2.351227in}}%
\pgfpathlineto{\pgfqpoint{2.683177in}{2.415272in}}%
\pgfpathlineto{\pgfqpoint{2.683375in}{2.406685in}}%
\pgfpathlineto{\pgfqpoint{2.684065in}{2.355932in}}%
\pgfpathlineto{\pgfqpoint{2.684658in}{2.379088in}}%
\pgfpathlineto{\pgfqpoint{2.686336in}{2.419461in}}%
\pgfpathlineto{\pgfqpoint{2.686434in}{2.417273in}}%
\pgfpathlineto{\pgfqpoint{2.687322in}{2.383292in}}%
\pgfpathlineto{\pgfqpoint{2.687717in}{2.398898in}}%
\pgfpathlineto{\pgfqpoint{2.689099in}{2.423432in}}%
\pgfpathlineto{\pgfqpoint{2.690876in}{2.409654in}}%
\pgfpathlineto{\pgfqpoint{2.691073in}{2.414177in}}%
\pgfpathlineto{\pgfqpoint{2.692060in}{2.423787in}}%
\pgfpathlineto{\pgfqpoint{2.692257in}{2.420233in}}%
\pgfpathlineto{\pgfqpoint{2.693442in}{2.401002in}}%
\pgfpathlineto{\pgfqpoint{2.693738in}{2.401894in}}%
\pgfpathlineto{\pgfqpoint{2.694034in}{2.398297in}}%
\pgfpathlineto{\pgfqpoint{2.694330in}{2.406959in}}%
\pgfpathlineto{\pgfqpoint{2.695416in}{2.423155in}}%
\pgfpathlineto{\pgfqpoint{2.695515in}{2.419599in}}%
\pgfpathlineto{\pgfqpoint{2.696501in}{2.401103in}}%
\pgfpathlineto{\pgfqpoint{2.696798in}{2.406887in}}%
\pgfpathlineto{\pgfqpoint{2.697488in}{2.414560in}}%
\pgfpathlineto{\pgfqpoint{2.697094in}{2.406076in}}%
\pgfpathlineto{\pgfqpoint{2.698377in}{2.412072in}}%
\pgfpathlineto{\pgfqpoint{2.700153in}{2.386290in}}%
\pgfpathlineto{\pgfqpoint{2.700252in}{2.387415in}}%
\pgfpathlineto{\pgfqpoint{2.700449in}{2.388808in}}%
\pgfpathlineto{\pgfqpoint{2.700844in}{2.385128in}}%
\pgfpathlineto{\pgfqpoint{2.701239in}{2.387929in}}%
\pgfpathlineto{\pgfqpoint{2.702325in}{2.366417in}}%
\pgfpathlineto{\pgfqpoint{2.703016in}{2.376050in}}%
\pgfpathlineto{\pgfqpoint{2.703312in}{2.375462in}}%
\pgfpathlineto{\pgfqpoint{2.703707in}{2.364840in}}%
\pgfpathlineto{\pgfqpoint{2.704397in}{2.376057in}}%
\pgfpathlineto{\pgfqpoint{2.704792in}{2.383612in}}%
\pgfpathlineto{\pgfqpoint{2.705483in}{2.377067in}}%
\pgfpathlineto{\pgfqpoint{2.705779in}{2.372900in}}%
\pgfpathlineto{\pgfqpoint{2.706075in}{2.380881in}}%
\pgfpathlineto{\pgfqpoint{2.707260in}{2.390062in}}%
\pgfpathlineto{\pgfqpoint{2.706766in}{2.372438in}}%
\pgfpathlineto{\pgfqpoint{2.707358in}{2.388662in}}%
\pgfpathlineto{\pgfqpoint{2.707654in}{2.381181in}}%
\pgfpathlineto{\pgfqpoint{2.707951in}{2.397976in}}%
\pgfpathlineto{\pgfqpoint{2.709431in}{2.428145in}}%
\pgfpathlineto{\pgfqpoint{2.710517in}{2.456945in}}%
\pgfpathlineto{\pgfqpoint{2.711109in}{2.449332in}}%
\pgfpathlineto{\pgfqpoint{2.711602in}{2.452262in}}%
\pgfpathlineto{\pgfqpoint{2.711800in}{2.446307in}}%
\pgfpathlineto{\pgfqpoint{2.713478in}{2.409409in}}%
\pgfpathlineto{\pgfqpoint{2.714465in}{2.399877in}}%
\pgfpathlineto{\pgfqpoint{2.714070in}{2.414081in}}%
\pgfpathlineto{\pgfqpoint{2.714563in}{2.402330in}}%
\pgfpathlineto{\pgfqpoint{2.715945in}{2.418466in}}%
\pgfpathlineto{\pgfqpoint{2.716340in}{2.413346in}}%
\pgfpathlineto{\pgfqpoint{2.716636in}{2.420550in}}%
\pgfpathlineto{\pgfqpoint{2.716735in}{2.422130in}}%
\pgfpathlineto{\pgfqpoint{2.717228in}{2.412545in}}%
\pgfpathlineto{\pgfqpoint{2.717426in}{2.413107in}}%
\pgfpathlineto{\pgfqpoint{2.717524in}{2.411611in}}%
\pgfpathlineto{\pgfqpoint{2.717919in}{2.400770in}}%
\pgfpathlineto{\pgfqpoint{2.718314in}{2.421888in}}%
\pgfpathlineto{\pgfqpoint{2.718413in}{2.422883in}}%
\pgfpathlineto{\pgfqpoint{2.718610in}{2.416050in}}%
\pgfpathlineto{\pgfqpoint{2.718807in}{2.412318in}}%
\pgfpathlineto{\pgfqpoint{2.719104in}{2.434087in}}%
\pgfpathlineto{\pgfqpoint{2.719301in}{2.445948in}}%
\pgfpathlineto{\pgfqpoint{2.719893in}{2.411309in}}%
\pgfpathlineto{\pgfqpoint{2.720091in}{2.407224in}}%
\pgfpathlineto{\pgfqpoint{2.720387in}{2.432976in}}%
\pgfpathlineto{\pgfqpoint{2.722163in}{2.680345in}}%
\pgfpathlineto{\pgfqpoint{2.722755in}{2.662686in}}%
\pgfpathlineto{\pgfqpoint{2.723742in}{2.422805in}}%
\pgfpathlineto{\pgfqpoint{2.725223in}{2.448272in}}%
\pgfpathlineto{\pgfqpoint{2.725420in}{2.450853in}}%
\pgfpathlineto{\pgfqpoint{2.726407in}{2.524052in}}%
\pgfpathlineto{\pgfqpoint{2.726802in}{2.556556in}}%
\pgfpathlineto{\pgfqpoint{2.727296in}{2.516472in}}%
\pgfpathlineto{\pgfqpoint{2.727986in}{2.476238in}}%
\pgfpathlineto{\pgfqpoint{2.728381in}{2.506265in}}%
\pgfpathlineto{\pgfqpoint{2.730059in}{2.565818in}}%
\pgfpathlineto{\pgfqpoint{2.728875in}{2.501723in}}%
\pgfpathlineto{\pgfqpoint{2.730257in}{2.557505in}}%
\pgfpathlineto{\pgfqpoint{2.731046in}{2.500499in}}%
\pgfpathlineto{\pgfqpoint{2.731737in}{2.522556in}}%
\pgfpathlineto{\pgfqpoint{2.733020in}{2.566899in}}%
\pgfpathlineto{\pgfqpoint{2.733316in}{2.562533in}}%
\pgfpathlineto{\pgfqpoint{2.734994in}{2.508639in}}%
\pgfpathlineto{\pgfqpoint{2.735981in}{2.514425in}}%
\pgfpathlineto{\pgfqpoint{2.738251in}{2.634723in}}%
\pgfpathlineto{\pgfqpoint{2.739238in}{2.621563in}}%
\pgfpathlineto{\pgfqpoint{2.740126in}{2.602875in}}%
\pgfpathlineto{\pgfqpoint{2.740916in}{2.606757in}}%
\pgfpathlineto{\pgfqpoint{2.742495in}{2.635788in}}%
\pgfpathlineto{\pgfqpoint{2.742890in}{2.622700in}}%
\pgfpathlineto{\pgfqpoint{2.743186in}{2.617137in}}%
\pgfpathlineto{\pgfqpoint{2.743778in}{2.628748in}}%
\pgfpathlineto{\pgfqpoint{2.743877in}{2.629305in}}%
\pgfpathlineto{\pgfqpoint{2.744074in}{2.625829in}}%
\pgfpathlineto{\pgfqpoint{2.744272in}{2.621872in}}%
\pgfpathlineto{\pgfqpoint{2.744864in}{2.630402in}}%
\pgfpathlineto{\pgfqpoint{2.745259in}{2.636765in}}%
\pgfpathlineto{\pgfqpoint{2.745851in}{2.629009in}}%
\pgfpathlineto{\pgfqpoint{2.750687in}{2.574855in}}%
\pgfpathlineto{\pgfqpoint{2.750786in}{2.575609in}}%
\pgfpathlineto{\pgfqpoint{2.752266in}{2.585862in}}%
\pgfpathlineto{\pgfqpoint{2.752365in}{2.586084in}}%
\pgfpathlineto{\pgfqpoint{2.752562in}{2.584354in}}%
\pgfpathlineto{\pgfqpoint{2.753352in}{2.574715in}}%
\pgfpathlineto{\pgfqpoint{2.753648in}{2.581588in}}%
\pgfpathlineto{\pgfqpoint{2.754734in}{2.589310in}}%
\pgfpathlineto{\pgfqpoint{2.754240in}{2.569527in}}%
\pgfpathlineto{\pgfqpoint{2.754833in}{2.585807in}}%
\pgfpathlineto{\pgfqpoint{2.755721in}{2.574981in}}%
\pgfpathlineto{\pgfqpoint{2.755425in}{2.587005in}}%
\pgfpathlineto{\pgfqpoint{2.755918in}{2.582620in}}%
\pgfpathlineto{\pgfqpoint{2.757103in}{2.609434in}}%
\pgfpathlineto{\pgfqpoint{2.757300in}{2.609112in}}%
\pgfpathlineto{\pgfqpoint{2.757794in}{2.618321in}}%
\pgfpathlineto{\pgfqpoint{2.758090in}{2.608605in}}%
\pgfpathlineto{\pgfqpoint{2.759373in}{2.593100in}}%
\pgfpathlineto{\pgfqpoint{2.758781in}{2.612789in}}%
\pgfpathlineto{\pgfqpoint{2.759570in}{2.597005in}}%
\pgfpathlineto{\pgfqpoint{2.759768in}{2.599765in}}%
\pgfpathlineto{\pgfqpoint{2.760064in}{2.589223in}}%
\pgfpathlineto{\pgfqpoint{2.761544in}{2.550877in}}%
\pgfpathlineto{\pgfqpoint{2.761742in}{2.561775in}}%
\pgfpathlineto{\pgfqpoint{2.761939in}{2.575980in}}%
\pgfpathlineto{\pgfqpoint{2.762827in}{2.559570in}}%
\pgfpathlineto{\pgfqpoint{2.762926in}{2.559494in}}%
\pgfpathlineto{\pgfqpoint{2.764110in}{2.586140in}}%
\pgfpathlineto{\pgfqpoint{2.764406in}{2.572600in}}%
\pgfpathlineto{\pgfqpoint{2.765393in}{2.565349in}}%
\pgfpathlineto{\pgfqpoint{2.764900in}{2.578409in}}%
\pgfpathlineto{\pgfqpoint{2.765591in}{2.569378in}}%
\pgfpathlineto{\pgfqpoint{2.766578in}{2.593700in}}%
\pgfpathlineto{\pgfqpoint{2.767269in}{2.590901in}}%
\pgfpathlineto{\pgfqpoint{2.767663in}{2.579311in}}%
\pgfpathlineto{\pgfqpoint{2.768157in}{2.563791in}}%
\pgfpathlineto{\pgfqpoint{2.768453in}{2.583497in}}%
\pgfpathlineto{\pgfqpoint{2.770131in}{2.848265in}}%
\pgfpathlineto{\pgfqpoint{2.770822in}{2.835784in}}%
\pgfpathlineto{\pgfqpoint{2.772993in}{2.589790in}}%
\pgfpathlineto{\pgfqpoint{2.773388in}{2.608558in}}%
\pgfpathlineto{\pgfqpoint{2.774868in}{2.704658in}}%
\pgfpathlineto{\pgfqpoint{2.775362in}{2.652863in}}%
\pgfpathlineto{\pgfqpoint{2.775954in}{2.602728in}}%
\pgfpathlineto{\pgfqpoint{2.776546in}{2.633718in}}%
\pgfpathlineto{\pgfqpoint{2.778224in}{2.699964in}}%
\pgfpathlineto{\pgfqpoint{2.778619in}{2.668467in}}%
\pgfpathlineto{\pgfqpoint{2.779409in}{2.651997in}}%
\pgfpathlineto{\pgfqpoint{2.779606in}{2.663010in}}%
\pgfpathlineto{\pgfqpoint{2.781185in}{2.710075in}}%
\pgfpathlineto{\pgfqpoint{2.781876in}{2.702600in}}%
\pgfpathlineto{\pgfqpoint{2.782271in}{2.686537in}}%
\pgfpathlineto{\pgfqpoint{2.782962in}{2.694882in}}%
\pgfpathlineto{\pgfqpoint{2.784245in}{2.714528in}}%
\pgfpathlineto{\pgfqpoint{2.784344in}{2.713838in}}%
\pgfpathlineto{\pgfqpoint{2.785528in}{2.701462in}}%
\pgfpathlineto{\pgfqpoint{2.784936in}{2.714388in}}%
\pgfpathlineto{\pgfqpoint{2.786021in}{2.704298in}}%
\pgfpathlineto{\pgfqpoint{2.786515in}{2.735004in}}%
\pgfpathlineto{\pgfqpoint{2.787601in}{2.734269in}}%
\pgfpathlineto{\pgfqpoint{2.788785in}{2.724608in}}%
\pgfpathlineto{\pgfqpoint{2.788884in}{2.726266in}}%
\pgfpathlineto{\pgfqpoint{2.789969in}{2.755561in}}%
\pgfpathlineto{\pgfqpoint{2.790265in}{2.743570in}}%
\pgfpathlineto{\pgfqpoint{2.790463in}{2.733684in}}%
\pgfpathlineto{\pgfqpoint{2.790956in}{2.748316in}}%
\pgfpathlineto{\pgfqpoint{2.791351in}{2.741899in}}%
\pgfpathlineto{\pgfqpoint{2.791549in}{2.743821in}}%
\pgfpathlineto{\pgfqpoint{2.791845in}{2.734641in}}%
\pgfpathlineto{\pgfqpoint{2.792141in}{2.728850in}}%
\pgfpathlineto{\pgfqpoint{2.792536in}{2.737795in}}%
\pgfpathlineto{\pgfqpoint{2.793029in}{2.729202in}}%
\pgfpathlineto{\pgfqpoint{2.793226in}{2.731722in}}%
\pgfpathlineto{\pgfqpoint{2.793523in}{2.722399in}}%
\pgfpathlineto{\pgfqpoint{2.794016in}{2.730854in}}%
\pgfpathlineto{\pgfqpoint{2.797865in}{2.656390in}}%
\pgfpathlineto{\pgfqpoint{2.799346in}{2.644544in}}%
\pgfpathlineto{\pgfqpoint{2.798260in}{2.657019in}}%
\pgfpathlineto{\pgfqpoint{2.799444in}{2.646555in}}%
\pgfpathlineto{\pgfqpoint{2.799741in}{2.660846in}}%
\pgfpathlineto{\pgfqpoint{2.800135in}{2.641251in}}%
\pgfpathlineto{\pgfqpoint{2.800530in}{2.653028in}}%
\pgfpathlineto{\pgfqpoint{2.803787in}{2.539039in}}%
\pgfpathlineto{\pgfqpoint{2.804478in}{2.552834in}}%
\pgfpathlineto{\pgfqpoint{2.805465in}{2.607142in}}%
\pgfpathlineto{\pgfqpoint{2.806650in}{2.662789in}}%
\pgfpathlineto{\pgfqpoint{2.806946in}{2.656823in}}%
\pgfpathlineto{\pgfqpoint{2.808525in}{2.594850in}}%
\pgfpathlineto{\pgfqpoint{2.809808in}{2.545495in}}%
\pgfpathlineto{\pgfqpoint{2.810005in}{2.550405in}}%
\pgfpathlineto{\pgfqpoint{2.810203in}{2.552210in}}%
\pgfpathlineto{\pgfqpoint{2.810597in}{2.542429in}}%
\pgfpathlineto{\pgfqpoint{2.812571in}{2.519659in}}%
\pgfpathlineto{\pgfqpoint{2.812868in}{2.523370in}}%
\pgfpathlineto{\pgfqpoint{2.813065in}{2.524276in}}%
\pgfpathlineto{\pgfqpoint{2.813262in}{2.521718in}}%
\pgfpathlineto{\pgfqpoint{2.814052in}{2.526048in}}%
\pgfpathlineto{\pgfqpoint{2.814644in}{2.514069in}}%
\pgfpathlineto{\pgfqpoint{2.815039in}{2.517723in}}%
\pgfpathlineto{\pgfqpoint{2.815335in}{2.510306in}}%
\pgfpathlineto{\pgfqpoint{2.815631in}{2.510776in}}%
\pgfpathlineto{\pgfqpoint{2.815829in}{2.501699in}}%
\pgfpathlineto{\pgfqpoint{2.816223in}{2.467335in}}%
\pgfpathlineto{\pgfqpoint{2.816618in}{2.517907in}}%
\pgfpathlineto{\pgfqpoint{2.818000in}{2.749083in}}%
\pgfpathlineto{\pgfqpoint{2.818789in}{2.735878in}}%
\pgfpathlineto{\pgfqpoint{2.819480in}{2.535065in}}%
\pgfpathlineto{\pgfqpoint{2.820862in}{2.474109in}}%
\pgfpathlineto{\pgfqpoint{2.822145in}{2.515050in}}%
\pgfpathlineto{\pgfqpoint{2.822935in}{2.585374in}}%
\pgfpathlineto{\pgfqpoint{2.823330in}{2.549027in}}%
\pgfpathlineto{\pgfqpoint{2.824218in}{2.491703in}}%
\pgfpathlineto{\pgfqpoint{2.824613in}{2.511972in}}%
\pgfpathlineto{\pgfqpoint{2.824909in}{2.510673in}}%
\pgfpathlineto{\pgfqpoint{2.825304in}{2.530806in}}%
\pgfpathlineto{\pgfqpoint{2.826093in}{2.550460in}}%
\pgfpathlineto{\pgfqpoint{2.826389in}{2.539760in}}%
\pgfpathlineto{\pgfqpoint{2.826982in}{2.480689in}}%
\pgfpathlineto{\pgfqpoint{2.827870in}{2.502429in}}%
\pgfpathlineto{\pgfqpoint{2.828363in}{2.497181in}}%
\pgfpathlineto{\pgfqpoint{2.829449in}{2.521417in}}%
\pgfpathlineto{\pgfqpoint{2.829646in}{2.516715in}}%
\pgfpathlineto{\pgfqpoint{2.830140in}{2.478086in}}%
\pgfpathlineto{\pgfqpoint{2.830929in}{2.493292in}}%
\pgfpathlineto{\pgfqpoint{2.832311in}{2.530490in}}%
\pgfpathlineto{\pgfqpoint{2.832410in}{2.527792in}}%
\pgfpathlineto{\pgfqpoint{2.832805in}{2.514681in}}%
\pgfpathlineto{\pgfqpoint{2.833693in}{2.519624in}}%
\pgfpathlineto{\pgfqpoint{2.834581in}{2.517213in}}%
\pgfpathlineto{\pgfqpoint{2.834779in}{2.518600in}}%
\pgfpathlineto{\pgfqpoint{2.834976in}{2.520259in}}%
\pgfpathlineto{\pgfqpoint{2.835272in}{2.515917in}}%
\pgfpathlineto{\pgfqpoint{2.835568in}{2.517040in}}%
\pgfpathlineto{\pgfqpoint{2.836851in}{2.487941in}}%
\pgfpathlineto{\pgfqpoint{2.837345in}{2.495469in}}%
\pgfpathlineto{\pgfqpoint{2.837542in}{2.500847in}}%
\pgfpathlineto{\pgfqpoint{2.837937in}{2.487359in}}%
\pgfpathlineto{\pgfqpoint{2.838332in}{2.493783in}}%
\pgfpathlineto{\pgfqpoint{2.839023in}{2.503145in}}%
\pgfpathlineto{\pgfqpoint{2.840108in}{2.473703in}}%
\pgfpathlineto{\pgfqpoint{2.840306in}{2.476954in}}%
\pgfpathlineto{\pgfqpoint{2.840701in}{2.492032in}}%
\pgfpathlineto{\pgfqpoint{2.841095in}{2.476004in}}%
\pgfpathlineto{\pgfqpoint{2.841392in}{2.478788in}}%
\pgfpathlineto{\pgfqpoint{2.844550in}{2.430413in}}%
\pgfpathlineto{\pgfqpoint{2.846623in}{2.378235in}}%
\pgfpathlineto{\pgfqpoint{2.846721in}{2.380086in}}%
\pgfpathlineto{\pgfqpoint{2.847017in}{2.391745in}}%
\pgfpathlineto{\pgfqpoint{2.847610in}{2.373154in}}%
\pgfpathlineto{\pgfqpoint{2.847906in}{2.368027in}}%
\pgfpathlineto{\pgfqpoint{2.848103in}{2.362431in}}%
\pgfpathlineto{\pgfqpoint{2.848597in}{2.379694in}}%
\pgfpathlineto{\pgfqpoint{2.848991in}{2.364411in}}%
\pgfpathlineto{\pgfqpoint{2.849287in}{2.368815in}}%
\pgfpathlineto{\pgfqpoint{2.849978in}{2.363065in}}%
\pgfpathlineto{\pgfqpoint{2.850274in}{2.354484in}}%
\pgfpathlineto{\pgfqpoint{2.850669in}{2.368158in}}%
\pgfpathlineto{\pgfqpoint{2.850965in}{2.361114in}}%
\pgfpathlineto{\pgfqpoint{2.853235in}{2.405022in}}%
\pgfpathlineto{\pgfqpoint{2.854321in}{2.418335in}}%
\pgfpathlineto{\pgfqpoint{2.854519in}{2.409703in}}%
\pgfpathlineto{\pgfqpoint{2.854716in}{2.400967in}}%
\pgfpathlineto{\pgfqpoint{2.855111in}{2.422001in}}%
\pgfpathlineto{\pgfqpoint{2.855604in}{2.406740in}}%
\pgfpathlineto{\pgfqpoint{2.855703in}{2.408231in}}%
\pgfpathlineto{\pgfqpoint{2.855999in}{2.398514in}}%
\pgfpathlineto{\pgfqpoint{2.857578in}{2.377938in}}%
\pgfpathlineto{\pgfqpoint{2.859552in}{2.400015in}}%
\pgfpathlineto{\pgfqpoint{2.859651in}{2.398015in}}%
\pgfpathlineto{\pgfqpoint{2.859947in}{2.389644in}}%
\pgfpathlineto{\pgfqpoint{2.860638in}{2.397737in}}%
\pgfpathlineto{\pgfqpoint{2.860835in}{2.399266in}}%
\pgfpathlineto{\pgfqpoint{2.861131in}{2.390178in}}%
\pgfpathlineto{\pgfqpoint{2.861329in}{2.386284in}}%
\pgfpathlineto{\pgfqpoint{2.861921in}{2.399774in}}%
\pgfpathlineto{\pgfqpoint{2.863500in}{2.363387in}}%
\pgfpathlineto{\pgfqpoint{2.863895in}{2.376725in}}%
\pgfpathlineto{\pgfqpoint{2.865770in}{2.603668in}}%
\pgfpathlineto{\pgfqpoint{2.866264in}{2.573162in}}%
\pgfpathlineto{\pgfqpoint{2.868238in}{2.340446in}}%
\pgfpathlineto{\pgfqpoint{2.868632in}{2.350985in}}%
\pgfpathlineto{\pgfqpoint{2.869027in}{2.338438in}}%
\pgfpathlineto{\pgfqpoint{2.869422in}{2.353748in}}%
\pgfpathlineto{\pgfqpoint{2.870212in}{2.392059in}}%
\pgfpathlineto{\pgfqpoint{2.870606in}{2.362115in}}%
\pgfpathlineto{\pgfqpoint{2.871692in}{2.269700in}}%
\pgfpathlineto{\pgfqpoint{2.872087in}{2.279768in}}%
\pgfpathlineto{\pgfqpoint{2.873567in}{2.402808in}}%
\pgfpathlineto{\pgfqpoint{2.875443in}{2.394384in}}%
\pgfpathlineto{\pgfqpoint{2.875936in}{2.383988in}}%
\pgfpathlineto{\pgfqpoint{2.876627in}{2.392211in}}%
\pgfpathlineto{\pgfqpoint{2.876923in}{2.400815in}}%
\pgfpathlineto{\pgfqpoint{2.877417in}{2.383644in}}%
\pgfpathlineto{\pgfqpoint{2.877811in}{2.376137in}}%
\pgfpathlineto{\pgfqpoint{2.877910in}{2.374125in}}%
\pgfpathlineto{\pgfqpoint{2.878206in}{2.389660in}}%
\pgfpathlineto{\pgfqpoint{2.878601in}{2.413940in}}%
\pgfpathlineto{\pgfqpoint{2.879391in}{2.404518in}}%
\pgfpathlineto{\pgfqpoint{2.879687in}{2.403386in}}%
\pgfpathlineto{\pgfqpoint{2.879785in}{2.404569in}}%
\pgfpathlineto{\pgfqpoint{2.880082in}{2.414829in}}%
\pgfpathlineto{\pgfqpoint{2.880476in}{2.394069in}}%
\pgfpathlineto{\pgfqpoint{2.880674in}{2.388552in}}%
\pgfpathlineto{\pgfqpoint{2.881069in}{2.402474in}}%
\pgfpathlineto{\pgfqpoint{2.881463in}{2.399300in}}%
\pgfpathlineto{\pgfqpoint{2.882549in}{2.409379in}}%
\pgfpathlineto{\pgfqpoint{2.882845in}{2.405431in}}%
\pgfpathlineto{\pgfqpoint{2.883141in}{2.406063in}}%
\pgfpathlineto{\pgfqpoint{2.883536in}{2.390138in}}%
\pgfpathlineto{\pgfqpoint{2.883733in}{2.381891in}}%
\pgfpathlineto{\pgfqpoint{2.884227in}{2.391621in}}%
\pgfpathlineto{\pgfqpoint{2.884523in}{2.390674in}}%
\pgfpathlineto{\pgfqpoint{2.885115in}{2.407561in}}%
\pgfpathlineto{\pgfqpoint{2.886300in}{2.404906in}}%
\pgfpathlineto{\pgfqpoint{2.886694in}{2.394859in}}%
\pgfpathlineto{\pgfqpoint{2.887287in}{2.406887in}}%
\pgfpathlineto{\pgfqpoint{2.887385in}{2.405608in}}%
\pgfpathlineto{\pgfqpoint{2.887977in}{2.394959in}}%
\pgfpathlineto{\pgfqpoint{2.888570in}{2.400335in}}%
\pgfpathlineto{\pgfqpoint{2.888866in}{2.405647in}}%
\pgfpathlineto{\pgfqpoint{2.889162in}{2.388906in}}%
\pgfpathlineto{\pgfqpoint{2.890149in}{2.383703in}}%
\pgfpathlineto{\pgfqpoint{2.889655in}{2.394667in}}%
\pgfpathlineto{\pgfqpoint{2.890248in}{2.385002in}}%
\pgfpathlineto{\pgfqpoint{2.890346in}{2.386408in}}%
\pgfpathlineto{\pgfqpoint{2.890741in}{2.377616in}}%
\pgfpathlineto{\pgfqpoint{2.891432in}{2.363769in}}%
\pgfpathlineto{\pgfqpoint{2.891728in}{2.376182in}}%
\pgfpathlineto{\pgfqpoint{2.891925in}{2.380354in}}%
\pgfpathlineto{\pgfqpoint{2.892320in}{2.360035in}}%
\pgfpathlineto{\pgfqpoint{2.894195in}{2.341512in}}%
\pgfpathlineto{\pgfqpoint{2.892814in}{2.365194in}}%
\pgfpathlineto{\pgfqpoint{2.894294in}{2.342341in}}%
\pgfpathlineto{\pgfqpoint{2.894689in}{2.355186in}}%
\pgfpathlineto{\pgfqpoint{2.895084in}{2.337427in}}%
\pgfpathlineto{\pgfqpoint{2.895281in}{2.331047in}}%
\pgfpathlineto{\pgfqpoint{2.895775in}{2.340932in}}%
\pgfpathlineto{\pgfqpoint{2.896071in}{2.340104in}}%
\pgfpathlineto{\pgfqpoint{2.897255in}{2.349123in}}%
\pgfpathlineto{\pgfqpoint{2.896762in}{2.336550in}}%
\pgfpathlineto{\pgfqpoint{2.897453in}{2.346422in}}%
\pgfpathlineto{\pgfqpoint{2.897650in}{2.343046in}}%
\pgfpathlineto{\pgfqpoint{2.897946in}{2.354756in}}%
\pgfpathlineto{\pgfqpoint{2.899427in}{2.386625in}}%
\pgfpathlineto{\pgfqpoint{2.899624in}{2.384857in}}%
\pgfpathlineto{\pgfqpoint{2.899723in}{2.383750in}}%
\pgfpathlineto{\pgfqpoint{2.900315in}{2.389514in}}%
\pgfpathlineto{\pgfqpoint{2.902091in}{2.396978in}}%
\pgfpathlineto{\pgfqpoint{2.902190in}{2.396619in}}%
\pgfpathlineto{\pgfqpoint{2.905151in}{2.315040in}}%
\pgfpathlineto{\pgfqpoint{2.905546in}{2.320572in}}%
\pgfpathlineto{\pgfqpoint{2.905645in}{2.320840in}}%
\pgfpathlineto{\pgfqpoint{2.905842in}{2.318392in}}%
\pgfpathlineto{\pgfqpoint{2.906138in}{2.314463in}}%
\pgfpathlineto{\pgfqpoint{2.906533in}{2.324383in}}%
\pgfpathlineto{\pgfqpoint{2.907322in}{2.326608in}}%
\pgfpathlineto{\pgfqpoint{2.906928in}{2.323374in}}%
\pgfpathlineto{\pgfqpoint{2.907421in}{2.324820in}}%
\pgfpathlineto{\pgfqpoint{2.907717in}{2.319282in}}%
\pgfpathlineto{\pgfqpoint{2.908408in}{2.324949in}}%
\pgfpathlineto{\pgfqpoint{2.908507in}{2.324479in}}%
\pgfpathlineto{\pgfqpoint{2.908803in}{2.324927in}}%
\pgfpathlineto{\pgfqpoint{2.909000in}{2.323078in}}%
\pgfpathlineto{\pgfqpoint{2.909296in}{2.314709in}}%
\pgfpathlineto{\pgfqpoint{2.909790in}{2.329908in}}%
\pgfpathlineto{\pgfqpoint{2.909889in}{2.330711in}}%
\pgfpathlineto{\pgfqpoint{2.910086in}{2.325295in}}%
\pgfpathlineto{\pgfqpoint{2.910580in}{2.293425in}}%
\pgfpathlineto{\pgfqpoint{2.910974in}{2.331240in}}%
\pgfpathlineto{\pgfqpoint{2.913146in}{2.539926in}}%
\pgfpathlineto{\pgfqpoint{2.913244in}{2.532196in}}%
\pgfpathlineto{\pgfqpoint{2.914231in}{2.287886in}}%
\pgfpathlineto{\pgfqpoint{2.915712in}{2.316766in}}%
\pgfpathlineto{\pgfqpoint{2.917488in}{2.408211in}}%
\pgfpathlineto{\pgfqpoint{2.917785in}{2.378940in}}%
\pgfpathlineto{\pgfqpoint{2.918574in}{2.328515in}}%
\pgfpathlineto{\pgfqpoint{2.918969in}{2.350305in}}%
\pgfpathlineto{\pgfqpoint{2.920548in}{2.399249in}}%
\pgfpathlineto{\pgfqpoint{2.920943in}{2.371046in}}%
\pgfpathlineto{\pgfqpoint{2.921436in}{2.324718in}}%
\pgfpathlineto{\pgfqpoint{2.922029in}{2.361557in}}%
\pgfpathlineto{\pgfqpoint{2.922127in}{2.361747in}}%
\pgfpathlineto{\pgfqpoint{2.923706in}{2.400239in}}%
\pgfpathlineto{\pgfqpoint{2.924101in}{2.397444in}}%
\pgfpathlineto{\pgfqpoint{2.924792in}{2.360062in}}%
\pgfpathlineto{\pgfqpoint{2.925384in}{2.389840in}}%
\pgfpathlineto{\pgfqpoint{2.926371in}{2.412401in}}%
\pgfpathlineto{\pgfqpoint{2.925878in}{2.388534in}}%
\pgfpathlineto{\pgfqpoint{2.926667in}{2.400349in}}%
\pgfpathlineto{\pgfqpoint{2.926766in}{2.397393in}}%
\pgfpathlineto{\pgfqpoint{2.927161in}{2.417640in}}%
\pgfpathlineto{\pgfqpoint{2.927260in}{2.419827in}}%
\pgfpathlineto{\pgfqpoint{2.927654in}{2.405723in}}%
\pgfpathlineto{\pgfqpoint{2.928641in}{2.393536in}}%
\pgfpathlineto{\pgfqpoint{2.928839in}{2.402100in}}%
\pgfpathlineto{\pgfqpoint{2.929135in}{2.413722in}}%
\pgfpathlineto{\pgfqpoint{2.929826in}{2.400756in}}%
\pgfpathlineto{\pgfqpoint{2.930813in}{2.385427in}}%
\pgfpathlineto{\pgfqpoint{2.930418in}{2.407232in}}%
\pgfpathlineto{\pgfqpoint{2.931109in}{2.395492in}}%
\pgfpathlineto{\pgfqpoint{2.932392in}{2.403794in}}%
\pgfpathlineto{\pgfqpoint{2.932787in}{2.396473in}}%
\pgfpathlineto{\pgfqpoint{2.932885in}{2.394876in}}%
\pgfpathlineto{\pgfqpoint{2.933182in}{2.402318in}}%
\pgfpathlineto{\pgfqpoint{2.933379in}{2.409426in}}%
\pgfpathlineto{\pgfqpoint{2.933774in}{2.394709in}}%
\pgfpathlineto{\pgfqpoint{2.934267in}{2.406662in}}%
\pgfpathlineto{\pgfqpoint{2.934662in}{2.384542in}}%
\pgfpathlineto{\pgfqpoint{2.935452in}{2.399839in}}%
\pgfpathlineto{\pgfqpoint{2.939301in}{2.352441in}}%
\pgfpathlineto{\pgfqpoint{2.939597in}{2.361403in}}%
\pgfpathlineto{\pgfqpoint{2.939794in}{2.369309in}}%
\pgfpathlineto{\pgfqpoint{2.940288in}{2.354040in}}%
\pgfpathlineto{\pgfqpoint{2.940584in}{2.357875in}}%
\pgfpathlineto{\pgfqpoint{2.940781in}{2.356999in}}%
\pgfpathlineto{\pgfqpoint{2.940880in}{2.357649in}}%
\pgfpathlineto{\pgfqpoint{2.941176in}{2.363760in}}%
\pgfpathlineto{\pgfqpoint{2.941966in}{2.360629in}}%
\pgfpathlineto{\pgfqpoint{2.942163in}{2.357113in}}%
\pgfpathlineto{\pgfqpoint{2.942558in}{2.367006in}}%
\pgfpathlineto{\pgfqpoint{2.942953in}{2.361994in}}%
\pgfpathlineto{\pgfqpoint{2.943940in}{2.368868in}}%
\pgfpathlineto{\pgfqpoint{2.943446in}{2.360529in}}%
\pgfpathlineto{\pgfqpoint{2.944038in}{2.365762in}}%
\pgfpathlineto{\pgfqpoint{2.945914in}{2.326940in}}%
\pgfpathlineto{\pgfqpoint{2.946407in}{2.322799in}}%
\pgfpathlineto{\pgfqpoint{2.946703in}{2.327669in}}%
\pgfpathlineto{\pgfqpoint{2.948085in}{2.417859in}}%
\pgfpathlineto{\pgfqpoint{2.949072in}{2.430605in}}%
\pgfpathlineto{\pgfqpoint{2.949368in}{2.427096in}}%
\pgfpathlineto{\pgfqpoint{2.949566in}{2.429415in}}%
\pgfpathlineto{\pgfqpoint{2.949862in}{2.418302in}}%
\pgfpathlineto{\pgfqpoint{2.951342in}{2.356788in}}%
\pgfpathlineto{\pgfqpoint{2.951737in}{2.363297in}}%
\pgfpathlineto{\pgfqpoint{2.953119in}{2.349406in}}%
\pgfpathlineto{\pgfqpoint{2.953316in}{2.353103in}}%
\pgfpathlineto{\pgfqpoint{2.953612in}{2.365214in}}%
\pgfpathlineto{\pgfqpoint{2.954204in}{2.346418in}}%
\pgfpathlineto{\pgfqpoint{2.954797in}{2.339296in}}%
\pgfpathlineto{\pgfqpoint{2.955093in}{2.347426in}}%
\pgfpathlineto{\pgfqpoint{2.955389in}{2.358574in}}%
\pgfpathlineto{\pgfqpoint{2.956080in}{2.342989in}}%
\pgfpathlineto{\pgfqpoint{2.956376in}{2.340004in}}%
\pgfpathlineto{\pgfqpoint{2.956869in}{2.344737in}}%
\pgfpathlineto{\pgfqpoint{2.957264in}{2.354095in}}%
\pgfpathlineto{\pgfqpoint{2.957560in}{2.340381in}}%
\pgfpathlineto{\pgfqpoint{2.957856in}{2.326442in}}%
\pgfpathlineto{\pgfqpoint{2.958152in}{2.364369in}}%
\pgfpathlineto{\pgfqpoint{2.960028in}{2.562261in}}%
\pgfpathlineto{\pgfqpoint{2.960225in}{2.571370in}}%
\pgfpathlineto{\pgfqpoint{2.960521in}{2.538488in}}%
\pgfpathlineto{\pgfqpoint{2.961311in}{2.321040in}}%
\pgfpathlineto{\pgfqpoint{2.962495in}{2.329953in}}%
\pgfpathlineto{\pgfqpoint{2.964568in}{2.442977in}}%
\pgfpathlineto{\pgfqpoint{2.964765in}{2.426366in}}%
\pgfpathlineto{\pgfqpoint{2.965752in}{2.368065in}}%
\pgfpathlineto{\pgfqpoint{2.966048in}{2.384880in}}%
\pgfpathlineto{\pgfqpoint{2.966739in}{2.375802in}}%
\pgfpathlineto{\pgfqpoint{2.967529in}{2.420027in}}%
\pgfpathlineto{\pgfqpoint{2.967628in}{2.419943in}}%
\pgfpathlineto{\pgfqpoint{2.968220in}{2.390591in}}%
\pgfpathlineto{\pgfqpoint{2.968812in}{2.358493in}}%
\pgfpathlineto{\pgfqpoint{2.969305in}{2.382072in}}%
\pgfpathlineto{\pgfqpoint{2.970885in}{2.416474in}}%
\pgfpathlineto{\pgfqpoint{2.971082in}{2.422345in}}%
\pgfpathlineto{\pgfqpoint{2.971674in}{2.405496in}}%
\pgfpathlineto{\pgfqpoint{2.971970in}{2.404320in}}%
\pgfpathlineto{\pgfqpoint{2.972168in}{2.402244in}}%
\pgfpathlineto{\pgfqpoint{2.972464in}{2.411668in}}%
\pgfpathlineto{\pgfqpoint{2.973352in}{2.432365in}}%
\pgfpathlineto{\pgfqpoint{2.973747in}{2.423980in}}%
\pgfpathlineto{\pgfqpoint{2.973846in}{2.423702in}}%
\pgfpathlineto{\pgfqpoint{2.973944in}{2.425238in}}%
\pgfpathlineto{\pgfqpoint{2.974142in}{2.427735in}}%
\pgfpathlineto{\pgfqpoint{2.974833in}{2.423183in}}%
\pgfpathlineto{\pgfqpoint{2.975227in}{2.417209in}}%
\pgfpathlineto{\pgfqpoint{2.975721in}{2.427734in}}%
\pgfpathlineto{\pgfqpoint{2.976807in}{2.450845in}}%
\pgfpathlineto{\pgfqpoint{2.977103in}{2.441353in}}%
\pgfpathlineto{\pgfqpoint{2.977201in}{2.441015in}}%
\pgfpathlineto{\pgfqpoint{2.977300in}{2.444163in}}%
\pgfpathlineto{\pgfqpoint{2.977497in}{2.449612in}}%
\pgfpathlineto{\pgfqpoint{2.977991in}{2.427783in}}%
\pgfpathlineto{\pgfqpoint{2.978188in}{2.429331in}}%
\pgfpathlineto{\pgfqpoint{2.978583in}{2.422514in}}%
\pgfpathlineto{\pgfqpoint{2.978978in}{2.427534in}}%
\pgfpathlineto{\pgfqpoint{2.979175in}{2.423481in}}%
\pgfpathlineto{\pgfqpoint{2.979570in}{2.437941in}}%
\pgfpathlineto{\pgfqpoint{2.979669in}{2.439372in}}%
\pgfpathlineto{\pgfqpoint{2.979866in}{2.432313in}}%
\pgfpathlineto{\pgfqpoint{2.980360in}{2.416633in}}%
\pgfpathlineto{\pgfqpoint{2.980853in}{2.430029in}}%
\pgfpathlineto{\pgfqpoint{2.981051in}{2.437240in}}%
\pgfpathlineto{\pgfqpoint{2.981544in}{2.425698in}}%
\pgfpathlineto{\pgfqpoint{2.981939in}{2.431061in}}%
\pgfpathlineto{\pgfqpoint{2.982334in}{2.426228in}}%
\pgfpathlineto{\pgfqpoint{2.982531in}{2.431246in}}%
\pgfpathlineto{\pgfqpoint{2.982827in}{2.439790in}}%
\pgfpathlineto{\pgfqpoint{2.983321in}{2.420702in}}%
\pgfpathlineto{\pgfqpoint{2.983419in}{2.422392in}}%
\pgfpathlineto{\pgfqpoint{2.983617in}{2.425512in}}%
\pgfpathlineto{\pgfqpoint{2.984209in}{2.418873in}}%
\pgfpathlineto{\pgfqpoint{2.988453in}{2.354036in}}%
\pgfpathlineto{\pgfqpoint{2.988749in}{2.363114in}}%
\pgfpathlineto{\pgfqpoint{2.989637in}{2.371851in}}%
\pgfpathlineto{\pgfqpoint{2.989835in}{2.369199in}}%
\pgfpathlineto{\pgfqpoint{2.990723in}{2.349608in}}%
\pgfpathlineto{\pgfqpoint{2.990920in}{2.359537in}}%
\pgfpathlineto{\pgfqpoint{2.991118in}{2.372164in}}%
\pgfpathlineto{\pgfqpoint{2.991611in}{2.347661in}}%
\pgfpathlineto{\pgfqpoint{2.991907in}{2.355933in}}%
\pgfpathlineto{\pgfqpoint{2.992697in}{2.362595in}}%
\pgfpathlineto{\pgfqpoint{2.992204in}{2.354663in}}%
\pgfpathlineto{\pgfqpoint{2.993092in}{2.361783in}}%
\pgfpathlineto{\pgfqpoint{2.995165in}{2.394108in}}%
\pgfpathlineto{\pgfqpoint{2.993980in}{2.361325in}}%
\pgfpathlineto{\pgfqpoint{2.995263in}{2.393174in}}%
\pgfpathlineto{\pgfqpoint{2.996250in}{2.377759in}}%
\pgfpathlineto{\pgfqpoint{2.995658in}{2.394000in}}%
\pgfpathlineto{\pgfqpoint{2.996744in}{2.388107in}}%
\pgfpathlineto{\pgfqpoint{2.997040in}{2.394861in}}%
\pgfpathlineto{\pgfqpoint{2.997435in}{2.373990in}}%
\pgfpathlineto{\pgfqpoint{2.999014in}{2.330091in}}%
\pgfpathlineto{\pgfqpoint{2.999310in}{2.335679in}}%
\pgfpathlineto{\pgfqpoint{3.000198in}{2.315072in}}%
\pgfpathlineto{\pgfqpoint{3.000593in}{2.331068in}}%
\pgfpathlineto{\pgfqpoint{3.000790in}{2.335086in}}%
\pgfpathlineto{\pgfqpoint{3.001383in}{2.326995in}}%
\pgfpathlineto{\pgfqpoint{3.001580in}{2.327988in}}%
\pgfpathlineto{\pgfqpoint{3.001679in}{2.328245in}}%
\pgfpathlineto{\pgfqpoint{3.001777in}{2.327515in}}%
\pgfpathlineto{\pgfqpoint{3.002073in}{2.321473in}}%
\pgfpathlineto{\pgfqpoint{3.002567in}{2.329520in}}%
\pgfpathlineto{\pgfqpoint{3.002764in}{2.328943in}}%
\pgfpathlineto{\pgfqpoint{3.002962in}{2.331358in}}%
\pgfpathlineto{\pgfqpoint{3.003554in}{2.326794in}}%
\pgfpathlineto{\pgfqpoint{3.004442in}{2.339832in}}%
\pgfpathlineto{\pgfqpoint{3.004541in}{2.339855in}}%
\pgfpathlineto{\pgfqpoint{3.005034in}{2.326575in}}%
\pgfpathlineto{\pgfqpoint{3.005331in}{2.340344in}}%
\pgfpathlineto{\pgfqpoint{3.007601in}{2.559322in}}%
\pgfpathlineto{\pgfqpoint{3.007995in}{2.502403in}}%
\pgfpathlineto{\pgfqpoint{3.008686in}{2.318612in}}%
\pgfpathlineto{\pgfqpoint{3.009575in}{2.325923in}}%
\pgfpathlineto{\pgfqpoint{3.011055in}{2.379649in}}%
\pgfpathlineto{\pgfqpoint{3.011845in}{2.438381in}}%
\pgfpathlineto{\pgfqpoint{3.012338in}{2.397411in}}%
\pgfpathlineto{\pgfqpoint{3.012733in}{2.373936in}}%
\pgfpathlineto{\pgfqpoint{3.013424in}{2.393117in}}%
\pgfpathlineto{\pgfqpoint{3.015102in}{2.459692in}}%
\pgfpathlineto{\pgfqpoint{3.015200in}{2.457585in}}%
\pgfpathlineto{\pgfqpoint{3.015990in}{2.408822in}}%
\pgfpathlineto{\pgfqpoint{3.016681in}{2.432460in}}%
\pgfpathlineto{\pgfqpoint{3.017372in}{2.431274in}}%
\pgfpathlineto{\pgfqpoint{3.018063in}{2.453998in}}%
\pgfpathlineto{\pgfqpoint{3.018260in}{2.454586in}}%
\pgfpathlineto{\pgfqpoint{3.018457in}{2.450802in}}%
\pgfpathlineto{\pgfqpoint{3.019543in}{2.401074in}}%
\pgfpathlineto{\pgfqpoint{3.020431in}{2.405135in}}%
\pgfpathlineto{\pgfqpoint{3.020826in}{2.404493in}}%
\pgfpathlineto{\pgfqpoint{3.021813in}{2.432444in}}%
\pgfpathlineto{\pgfqpoint{3.023886in}{2.509465in}}%
\pgfpathlineto{\pgfqpoint{3.023985in}{2.507915in}}%
\pgfpathlineto{\pgfqpoint{3.024182in}{2.505310in}}%
\pgfpathlineto{\pgfqpoint{3.024478in}{2.514161in}}%
\pgfpathlineto{\pgfqpoint{3.024675in}{2.517489in}}%
\pgfpathlineto{\pgfqpoint{3.025070in}{2.501218in}}%
\pgfpathlineto{\pgfqpoint{3.025761in}{2.494106in}}%
\pgfpathlineto{\pgfqpoint{3.026156in}{2.501754in}}%
\pgfpathlineto{\pgfqpoint{3.027044in}{2.524951in}}%
\pgfpathlineto{\pgfqpoint{3.027636in}{2.514485in}}%
\pgfpathlineto{\pgfqpoint{3.027834in}{2.515034in}}%
\pgfpathlineto{\pgfqpoint{3.027933in}{2.514577in}}%
\pgfpathlineto{\pgfqpoint{3.028327in}{2.508793in}}%
\pgfpathlineto{\pgfqpoint{3.028623in}{2.518296in}}%
\pgfpathlineto{\pgfqpoint{3.028722in}{2.521390in}}%
\pgfpathlineto{\pgfqpoint{3.029117in}{2.501185in}}%
\pgfpathlineto{\pgfqpoint{3.029314in}{2.495466in}}%
\pgfpathlineto{\pgfqpoint{3.029808in}{2.512342in}}%
\pgfpathlineto{\pgfqpoint{3.029907in}{2.512257in}}%
\pgfpathlineto{\pgfqpoint{3.030203in}{2.513679in}}%
\pgfpathlineto{\pgfqpoint{3.030400in}{2.510312in}}%
\pgfpathlineto{\pgfqpoint{3.033558in}{2.440471in}}%
\pgfpathlineto{\pgfqpoint{3.033756in}{2.448193in}}%
\pgfpathlineto{\pgfqpoint{3.034052in}{2.455559in}}%
\pgfpathlineto{\pgfqpoint{3.034743in}{2.446583in}}%
\pgfpathlineto{\pgfqpoint{3.034940in}{2.440687in}}%
\pgfpathlineto{\pgfqpoint{3.035335in}{2.451615in}}%
\pgfpathlineto{\pgfqpoint{3.035828in}{2.442075in}}%
\pgfpathlineto{\pgfqpoint{3.036815in}{2.451712in}}%
\pgfpathlineto{\pgfqpoint{3.037013in}{2.449634in}}%
\pgfpathlineto{\pgfqpoint{3.037408in}{2.439764in}}%
\pgfpathlineto{\pgfqpoint{3.038296in}{2.441059in}}%
\pgfpathlineto{\pgfqpoint{3.038987in}{2.432649in}}%
\pgfpathlineto{\pgfqpoint{3.039875in}{2.436580in}}%
\pgfpathlineto{\pgfqpoint{3.041356in}{2.461252in}}%
\pgfpathlineto{\pgfqpoint{3.041553in}{2.455880in}}%
\pgfpathlineto{\pgfqpoint{3.041750in}{2.450579in}}%
\pgfpathlineto{\pgfqpoint{3.042244in}{2.467170in}}%
\pgfpathlineto{\pgfqpoint{3.042639in}{2.472768in}}%
\pgfpathlineto{\pgfqpoint{3.043034in}{2.464879in}}%
\pgfpathlineto{\pgfqpoint{3.044810in}{2.434926in}}%
\pgfpathlineto{\pgfqpoint{3.045896in}{2.399433in}}%
\pgfpathlineto{\pgfqpoint{3.046291in}{2.403013in}}%
\pgfpathlineto{\pgfqpoint{3.046389in}{2.402924in}}%
\pgfpathlineto{\pgfqpoint{3.046488in}{2.403288in}}%
\pgfpathlineto{\pgfqpoint{3.046784in}{2.408618in}}%
\pgfpathlineto{\pgfqpoint{3.047179in}{2.398357in}}%
\pgfpathlineto{\pgfqpoint{3.047278in}{2.396744in}}%
\pgfpathlineto{\pgfqpoint{3.047771in}{2.406272in}}%
\pgfpathlineto{\pgfqpoint{3.048363in}{2.420825in}}%
\pgfpathlineto{\pgfqpoint{3.049844in}{2.418537in}}%
\pgfpathlineto{\pgfqpoint{3.050633in}{2.414664in}}%
\pgfpathlineto{\pgfqpoint{3.050239in}{2.419211in}}%
\pgfpathlineto{\pgfqpoint{3.050732in}{2.416409in}}%
\pgfpathlineto{\pgfqpoint{3.051818in}{2.437116in}}%
\pgfpathlineto{\pgfqpoint{3.052015in}{2.426410in}}%
\pgfpathlineto{\pgfqpoint{3.052410in}{2.407531in}}%
\pgfpathlineto{\pgfqpoint{3.052805in}{2.446517in}}%
\pgfpathlineto{\pgfqpoint{3.054680in}{2.650705in}}%
\pgfpathlineto{\pgfqpoint{3.054779in}{2.652139in}}%
\pgfpathlineto{\pgfqpoint{3.054976in}{2.642432in}}%
\pgfpathlineto{\pgfqpoint{3.055963in}{2.400775in}}%
\pgfpathlineto{\pgfqpoint{3.057444in}{2.452870in}}%
\pgfpathlineto{\pgfqpoint{3.059121in}{2.526067in}}%
\pgfpathlineto{\pgfqpoint{3.058036in}{2.448224in}}%
\pgfpathlineto{\pgfqpoint{3.059418in}{2.513626in}}%
\pgfpathlineto{\pgfqpoint{3.060108in}{2.451991in}}%
\pgfpathlineto{\pgfqpoint{3.060701in}{2.479792in}}%
\pgfpathlineto{\pgfqpoint{3.062280in}{2.540044in}}%
\pgfpathlineto{\pgfqpoint{3.062576in}{2.520919in}}%
\pgfpathlineto{\pgfqpoint{3.063464in}{2.482321in}}%
\pgfpathlineto{\pgfqpoint{3.063760in}{2.502523in}}%
\pgfpathlineto{\pgfqpoint{3.065339in}{2.538177in}}%
\pgfpathlineto{\pgfqpoint{3.066425in}{2.517498in}}%
\pgfpathlineto{\pgfqpoint{3.065833in}{2.543497in}}%
\pgfpathlineto{\pgfqpoint{3.066820in}{2.534110in}}%
\pgfpathlineto{\pgfqpoint{3.067708in}{2.559912in}}%
\pgfpathlineto{\pgfqpoint{3.067215in}{2.531048in}}%
\pgfpathlineto{\pgfqpoint{3.068991in}{2.552954in}}%
\pgfpathlineto{\pgfqpoint{3.069386in}{2.534547in}}%
\pgfpathlineto{\pgfqpoint{3.070274in}{2.537538in}}%
\pgfpathlineto{\pgfqpoint{3.071656in}{2.559964in}}%
\pgfpathlineto{\pgfqpoint{3.071755in}{2.560247in}}%
\pgfpathlineto{\pgfqpoint{3.071952in}{2.558166in}}%
\pgfpathlineto{\pgfqpoint{3.072841in}{2.533454in}}%
\pgfpathlineto{\pgfqpoint{3.073630in}{2.545392in}}%
\pgfpathlineto{\pgfqpoint{3.074025in}{2.565089in}}%
\pgfpathlineto{\pgfqpoint{3.074716in}{2.545697in}}%
\pgfpathlineto{\pgfqpoint{3.075111in}{2.550742in}}%
\pgfpathlineto{\pgfqpoint{3.075900in}{2.557157in}}%
\pgfpathlineto{\pgfqpoint{3.075505in}{2.548115in}}%
\pgfpathlineto{\pgfqpoint{3.076098in}{2.552781in}}%
\pgfpathlineto{\pgfqpoint{3.076394in}{2.541327in}}%
\pgfpathlineto{\pgfqpoint{3.077183in}{2.551122in}}%
\pgfpathlineto{\pgfqpoint{3.077282in}{2.552331in}}%
\pgfpathlineto{\pgfqpoint{3.077578in}{2.543102in}}%
\pgfpathlineto{\pgfqpoint{3.080342in}{2.490468in}}%
\pgfpathlineto{\pgfqpoint{3.077973in}{2.547619in}}%
\pgfpathlineto{\pgfqpoint{3.080638in}{2.496645in}}%
\pgfpathlineto{\pgfqpoint{3.080736in}{2.497348in}}%
\pgfpathlineto{\pgfqpoint{3.080934in}{2.492054in}}%
\pgfpathlineto{\pgfqpoint{3.081033in}{2.490169in}}%
\pgfpathlineto{\pgfqpoint{3.081329in}{2.504937in}}%
\pgfpathlineto{\pgfqpoint{3.081526in}{2.511386in}}%
\pgfpathlineto{\pgfqpoint{3.081921in}{2.492530in}}%
\pgfpathlineto{\pgfqpoint{3.082316in}{2.502863in}}%
\pgfpathlineto{\pgfqpoint{3.082612in}{2.492692in}}%
\pgfpathlineto{\pgfqpoint{3.083007in}{2.504728in}}%
\pgfpathlineto{\pgfqpoint{3.083500in}{2.497415in}}%
\pgfpathlineto{\pgfqpoint{3.083796in}{2.507088in}}%
\pgfpathlineto{\pgfqpoint{3.084191in}{2.489379in}}%
\pgfpathlineto{\pgfqpoint{3.084586in}{2.497855in}}%
\pgfpathlineto{\pgfqpoint{3.085671in}{2.480214in}}%
\pgfpathlineto{\pgfqpoint{3.085869in}{2.484288in}}%
\pgfpathlineto{\pgfqpoint{3.086066in}{2.490198in}}%
\pgfpathlineto{\pgfqpoint{3.086461in}{2.480761in}}%
\pgfpathlineto{\pgfqpoint{3.086955in}{2.485649in}}%
\pgfpathlineto{\pgfqpoint{3.087251in}{2.474793in}}%
\pgfpathlineto{\pgfqpoint{3.087843in}{2.490372in}}%
\pgfpathlineto{\pgfqpoint{3.089718in}{2.524323in}}%
\pgfpathlineto{\pgfqpoint{3.088632in}{2.483713in}}%
\pgfpathlineto{\pgfqpoint{3.090113in}{2.515646in}}%
\pgfpathlineto{\pgfqpoint{3.090310in}{2.514508in}}%
\pgfpathlineto{\pgfqpoint{3.091593in}{2.466858in}}%
\pgfpathlineto{\pgfqpoint{3.093074in}{2.379874in}}%
\pgfpathlineto{\pgfqpoint{3.093271in}{2.387094in}}%
\pgfpathlineto{\pgfqpoint{3.095541in}{2.479355in}}%
\pgfpathlineto{\pgfqpoint{3.095640in}{2.479391in}}%
\pgfpathlineto{\pgfqpoint{3.097219in}{2.456760in}}%
\pgfpathlineto{\pgfqpoint{3.097417in}{2.459203in}}%
\pgfpathlineto{\pgfqpoint{3.097515in}{2.461224in}}%
\pgfpathlineto{\pgfqpoint{3.097910in}{2.446614in}}%
\pgfpathlineto{\pgfqpoint{3.098206in}{2.462934in}}%
\pgfpathlineto{\pgfqpoint{3.099095in}{2.479707in}}%
\pgfpathlineto{\pgfqpoint{3.099292in}{2.470897in}}%
\pgfpathlineto{\pgfqpoint{3.099687in}{2.454232in}}%
\pgfpathlineto{\pgfqpoint{3.099983in}{2.477796in}}%
\pgfpathlineto{\pgfqpoint{3.102154in}{2.693816in}}%
\pgfpathlineto{\pgfqpoint{3.102549in}{2.617247in}}%
\pgfpathlineto{\pgfqpoint{3.103240in}{2.433722in}}%
\pgfpathlineto{\pgfqpoint{3.104128in}{2.442535in}}%
\pgfpathlineto{\pgfqpoint{3.105905in}{2.515056in}}%
\pgfpathlineto{\pgfqpoint{3.106398in}{2.553402in}}%
\pgfpathlineto{\pgfqpoint{3.106793in}{2.508422in}}%
\pgfpathlineto{\pgfqpoint{3.107484in}{2.465586in}}%
\pgfpathlineto{\pgfqpoint{3.107977in}{2.484317in}}%
\pgfpathlineto{\pgfqpoint{3.109260in}{2.523256in}}%
\pgfpathlineto{\pgfqpoint{3.108471in}{2.483895in}}%
\pgfpathlineto{\pgfqpoint{3.109853in}{2.517134in}}%
\pgfpathlineto{\pgfqpoint{3.110741in}{2.468604in}}%
\pgfpathlineto{\pgfqpoint{3.111234in}{2.496106in}}%
\pgfpathlineto{\pgfqpoint{3.112123in}{2.510095in}}%
\pgfpathlineto{\pgfqpoint{3.112419in}{2.505157in}}%
\pgfpathlineto{\pgfqpoint{3.112814in}{2.519370in}}%
\pgfpathlineto{\pgfqpoint{3.113208in}{2.503291in}}%
\pgfpathlineto{\pgfqpoint{3.113801in}{2.481025in}}%
\pgfpathlineto{\pgfqpoint{3.114492in}{2.494744in}}%
\pgfpathlineto{\pgfqpoint{3.115479in}{2.507461in}}%
\pgfpathlineto{\pgfqpoint{3.115972in}{2.502285in}}%
\pgfpathlineto{\pgfqpoint{3.116169in}{2.503695in}}%
\pgfpathlineto{\pgfqpoint{3.116367in}{2.505400in}}%
\pgfpathlineto{\pgfqpoint{3.116762in}{2.498497in}}%
\pgfpathlineto{\pgfqpoint{3.117156in}{2.488042in}}%
\pgfpathlineto{\pgfqpoint{3.117453in}{2.498146in}}%
\pgfpathlineto{\pgfqpoint{3.118538in}{2.517640in}}%
\pgfpathlineto{\pgfqpoint{3.118736in}{2.515625in}}%
\pgfpathlineto{\pgfqpoint{3.120413in}{2.488166in}}%
\pgfpathlineto{\pgfqpoint{3.119229in}{2.517208in}}%
\pgfpathlineto{\pgfqpoint{3.121302in}{2.494692in}}%
\pgfpathlineto{\pgfqpoint{3.121894in}{2.501451in}}%
\pgfpathlineto{\pgfqpoint{3.122486in}{2.497463in}}%
\pgfpathlineto{\pgfqpoint{3.122881in}{2.485947in}}%
\pgfpathlineto{\pgfqpoint{3.123177in}{2.502719in}}%
\pgfpathlineto{\pgfqpoint{3.123276in}{2.506896in}}%
\pgfpathlineto{\pgfqpoint{3.123671in}{2.480855in}}%
\pgfpathlineto{\pgfqpoint{3.123769in}{2.480023in}}%
\pgfpathlineto{\pgfqpoint{3.123868in}{2.484379in}}%
\pgfpathlineto{\pgfqpoint{3.124164in}{2.499097in}}%
\pgfpathlineto{\pgfqpoint{3.124855in}{2.480320in}}%
\pgfpathlineto{\pgfqpoint{3.126434in}{2.453283in}}%
\pgfpathlineto{\pgfqpoint{3.126632in}{2.459289in}}%
\pgfpathlineto{\pgfqpoint{3.126829in}{2.462270in}}%
\pgfpathlineto{\pgfqpoint{3.127322in}{2.451816in}}%
\pgfpathlineto{\pgfqpoint{3.128408in}{2.435349in}}%
\pgfpathlineto{\pgfqpoint{3.127915in}{2.452254in}}%
\pgfpathlineto{\pgfqpoint{3.128507in}{2.438934in}}%
\pgfpathlineto{\pgfqpoint{3.128704in}{2.448079in}}%
\pgfpathlineto{\pgfqpoint{3.129099in}{2.430651in}}%
\pgfpathlineto{\pgfqpoint{3.129592in}{2.442478in}}%
\pgfpathlineto{\pgfqpoint{3.129889in}{2.426780in}}%
\pgfpathlineto{\pgfqpoint{3.130283in}{2.445647in}}%
\pgfpathlineto{\pgfqpoint{3.130777in}{2.436344in}}%
\pgfpathlineto{\pgfqpoint{3.131073in}{2.448482in}}%
\pgfpathlineto{\pgfqpoint{3.131863in}{2.437196in}}%
\pgfpathlineto{\pgfqpoint{3.131961in}{2.436449in}}%
\pgfpathlineto{\pgfqpoint{3.132257in}{2.440707in}}%
\pgfpathlineto{\pgfqpoint{3.133343in}{2.450304in}}%
\pgfpathlineto{\pgfqpoint{3.132948in}{2.428639in}}%
\pgfpathlineto{\pgfqpoint{3.133442in}{2.449514in}}%
\pgfpathlineto{\pgfqpoint{3.133738in}{2.438750in}}%
\pgfpathlineto{\pgfqpoint{3.134527in}{2.445603in}}%
\pgfpathlineto{\pgfqpoint{3.136501in}{2.486125in}}%
\pgfpathlineto{\pgfqpoint{3.136600in}{2.485055in}}%
\pgfpathlineto{\pgfqpoint{3.136798in}{2.481394in}}%
\pgfpathlineto{\pgfqpoint{3.137291in}{2.494566in}}%
\pgfpathlineto{\pgfqpoint{3.137587in}{2.488090in}}%
\pgfpathlineto{\pgfqpoint{3.140351in}{2.405332in}}%
\pgfpathlineto{\pgfqpoint{3.140647in}{2.409366in}}%
\pgfpathlineto{\pgfqpoint{3.140844in}{2.399986in}}%
\pgfpathlineto{\pgfqpoint{3.141732in}{2.385782in}}%
\pgfpathlineto{\pgfqpoint{3.141338in}{2.399999in}}%
\pgfpathlineto{\pgfqpoint{3.142029in}{2.394640in}}%
\pgfpathlineto{\pgfqpoint{3.142719in}{2.402624in}}%
\pgfpathlineto{\pgfqpoint{3.142325in}{2.391698in}}%
\pgfpathlineto{\pgfqpoint{3.142917in}{2.397705in}}%
\pgfpathlineto{\pgfqpoint{3.143213in}{2.384617in}}%
\pgfpathlineto{\pgfqpoint{3.143608in}{2.398153in}}%
\pgfpathlineto{\pgfqpoint{3.144003in}{2.389410in}}%
\pgfpathlineto{\pgfqpoint{3.144397in}{2.409323in}}%
\pgfpathlineto{\pgfqpoint{3.144792in}{2.386491in}}%
\pgfpathlineto{\pgfqpoint{3.145384in}{2.400790in}}%
\pgfpathlineto{\pgfqpoint{3.145680in}{2.391876in}}%
\pgfpathlineto{\pgfqpoint{3.146470in}{2.400008in}}%
\pgfpathlineto{\pgfqpoint{3.146865in}{2.379544in}}%
\pgfpathlineto{\pgfqpoint{3.147062in}{2.389504in}}%
\pgfpathlineto{\pgfqpoint{3.149332in}{2.629493in}}%
\pgfpathlineto{\pgfqpoint{3.149431in}{2.627893in}}%
\pgfpathlineto{\pgfqpoint{3.150023in}{2.458242in}}%
\pgfpathlineto{\pgfqpoint{3.150418in}{2.381784in}}%
\pgfpathlineto{\pgfqpoint{3.151306in}{2.391654in}}%
\pgfpathlineto{\pgfqpoint{3.151405in}{2.390166in}}%
\pgfpathlineto{\pgfqpoint{3.151602in}{2.396207in}}%
\pgfpathlineto{\pgfqpoint{3.153576in}{2.488232in}}%
\pgfpathlineto{\pgfqpoint{3.153774in}{2.477905in}}%
\pgfpathlineto{\pgfqpoint{3.154761in}{2.401439in}}%
\pgfpathlineto{\pgfqpoint{3.155156in}{2.435749in}}%
\pgfpathlineto{\pgfqpoint{3.156833in}{2.479940in}}%
\pgfpathlineto{\pgfqpoint{3.155649in}{2.433897in}}%
\pgfpathlineto{\pgfqpoint{3.156932in}{2.477341in}}%
\pgfpathlineto{\pgfqpoint{3.157820in}{2.425204in}}%
\pgfpathlineto{\pgfqpoint{3.158314in}{2.452932in}}%
\pgfpathlineto{\pgfqpoint{3.158413in}{2.455270in}}%
\pgfpathlineto{\pgfqpoint{3.159202in}{2.449839in}}%
\pgfpathlineto{\pgfqpoint{3.161077in}{2.383854in}}%
\pgfpathlineto{\pgfqpoint{3.161472in}{2.396119in}}%
\pgfpathlineto{\pgfqpoint{3.163249in}{2.507834in}}%
\pgfpathlineto{\pgfqpoint{3.163644in}{2.507571in}}%
\pgfpathlineto{\pgfqpoint{3.165124in}{2.519984in}}%
\pgfpathlineto{\pgfqpoint{3.165519in}{2.510239in}}%
\pgfpathlineto{\pgfqpoint{3.166012in}{2.522328in}}%
\pgfpathlineto{\pgfqpoint{3.166210in}{2.519589in}}%
\pgfpathlineto{\pgfqpoint{3.166506in}{2.519802in}}%
\pgfpathlineto{\pgfqpoint{3.166605in}{2.518420in}}%
\pgfpathlineto{\pgfqpoint{3.166999in}{2.501467in}}%
\pgfpathlineto{\pgfqpoint{3.167789in}{2.512638in}}%
\pgfpathlineto{\pgfqpoint{3.168085in}{2.518059in}}%
\pgfpathlineto{\pgfqpoint{3.168973in}{2.516207in}}%
\pgfpathlineto{\pgfqpoint{3.169269in}{2.512192in}}%
\pgfpathlineto{\pgfqpoint{3.169763in}{2.518284in}}%
\pgfpathlineto{\pgfqpoint{3.170059in}{2.514619in}}%
\pgfpathlineto{\pgfqpoint{3.171145in}{2.523756in}}%
\pgfpathlineto{\pgfqpoint{3.170750in}{2.512385in}}%
\pgfpathlineto{\pgfqpoint{3.171342in}{2.517519in}}%
\pgfpathlineto{\pgfqpoint{3.173316in}{2.478887in}}%
\pgfpathlineto{\pgfqpoint{3.174007in}{2.469921in}}%
\pgfpathlineto{\pgfqpoint{3.175487in}{2.436916in}}%
\pgfpathlineto{\pgfqpoint{3.174698in}{2.471350in}}%
\pgfpathlineto{\pgfqpoint{3.175685in}{2.442832in}}%
\pgfpathlineto{\pgfqpoint{3.176474in}{2.456411in}}%
\pgfpathlineto{\pgfqpoint{3.176771in}{2.446028in}}%
\pgfpathlineto{\pgfqpoint{3.177659in}{2.431301in}}%
\pgfpathlineto{\pgfqpoint{3.178251in}{2.433615in}}%
\pgfpathlineto{\pgfqpoint{3.179139in}{2.424368in}}%
\pgfpathlineto{\pgfqpoint{3.179435in}{2.428355in}}%
\pgfpathlineto{\pgfqpoint{3.179534in}{2.429197in}}%
\pgfpathlineto{\pgfqpoint{3.179929in}{2.423423in}}%
\pgfpathlineto{\pgfqpoint{3.181212in}{2.419875in}}%
\pgfpathlineto{\pgfqpoint{3.180422in}{2.424300in}}%
\pgfpathlineto{\pgfqpoint{3.181311in}{2.420759in}}%
\pgfpathlineto{\pgfqpoint{3.182199in}{2.441157in}}%
\pgfpathlineto{\pgfqpoint{3.182594in}{2.427737in}}%
\pgfpathlineto{\pgfqpoint{3.182693in}{2.427439in}}%
\pgfpathlineto{\pgfqpoint{3.182791in}{2.430231in}}%
\pgfpathlineto{\pgfqpoint{3.183680in}{2.461593in}}%
\pgfpathlineto{\pgfqpoint{3.184074in}{2.439832in}}%
\pgfpathlineto{\pgfqpoint{3.185357in}{2.426042in}}%
\pgfpathlineto{\pgfqpoint{3.184666in}{2.443883in}}%
\pgfpathlineto{\pgfqpoint{3.185456in}{2.426485in}}%
\pgfpathlineto{\pgfqpoint{3.185752in}{2.435320in}}%
\pgfpathlineto{\pgfqpoint{3.186147in}{2.415743in}}%
\pgfpathlineto{\pgfqpoint{3.187430in}{2.375281in}}%
\pgfpathlineto{\pgfqpoint{3.187726in}{2.385772in}}%
\pgfpathlineto{\pgfqpoint{3.187825in}{2.389903in}}%
\pgfpathlineto{\pgfqpoint{3.188220in}{2.365740in}}%
\pgfpathlineto{\pgfqpoint{3.188318in}{2.363431in}}%
\pgfpathlineto{\pgfqpoint{3.188713in}{2.379549in}}%
\pgfpathlineto{\pgfqpoint{3.189108in}{2.367483in}}%
\pgfpathlineto{\pgfqpoint{3.189996in}{2.379272in}}%
\pgfpathlineto{\pgfqpoint{3.190194in}{2.370672in}}%
\pgfpathlineto{\pgfqpoint{3.191181in}{2.360866in}}%
\pgfpathlineto{\pgfqpoint{3.190786in}{2.382105in}}%
\pgfpathlineto{\pgfqpoint{3.191378in}{2.364865in}}%
\pgfpathlineto{\pgfqpoint{3.192168in}{2.368636in}}%
\pgfpathlineto{\pgfqpoint{3.191773in}{2.359230in}}%
\pgfpathlineto{\pgfqpoint{3.192266in}{2.366345in}}%
\pgfpathlineto{\pgfqpoint{3.193253in}{2.359422in}}%
\pgfpathlineto{\pgfqpoint{3.192859in}{2.372398in}}%
\pgfpathlineto{\pgfqpoint{3.193451in}{2.361988in}}%
\pgfpathlineto{\pgfqpoint{3.193747in}{2.350116in}}%
\pgfpathlineto{\pgfqpoint{3.193944in}{2.340888in}}%
\pgfpathlineto{\pgfqpoint{3.194339in}{2.382024in}}%
\pgfpathlineto{\pgfqpoint{3.196313in}{2.583814in}}%
\pgfpathlineto{\pgfqpoint{3.196609in}{2.557676in}}%
\pgfpathlineto{\pgfqpoint{3.197497in}{2.322254in}}%
\pgfpathlineto{\pgfqpoint{3.198583in}{2.341150in}}%
\pgfpathlineto{\pgfqpoint{3.199077in}{2.359019in}}%
\pgfpathlineto{\pgfqpoint{3.200557in}{2.442093in}}%
\pgfpathlineto{\pgfqpoint{3.199570in}{2.352529in}}%
\pgfpathlineto{\pgfqpoint{3.200952in}{2.414108in}}%
\pgfpathlineto{\pgfqpoint{3.201741in}{2.352796in}}%
\pgfpathlineto{\pgfqpoint{3.202235in}{2.375107in}}%
\pgfpathlineto{\pgfqpoint{3.203814in}{2.430457in}}%
\pgfpathlineto{\pgfqpoint{3.202630in}{2.374653in}}%
\pgfpathlineto{\pgfqpoint{3.204209in}{2.402506in}}%
\pgfpathlineto{\pgfqpoint{3.204900in}{2.364442in}}%
\pgfpathlineto{\pgfqpoint{3.205393in}{2.391934in}}%
\pgfpathlineto{\pgfqpoint{3.207071in}{2.427921in}}%
\pgfpathlineto{\pgfqpoint{3.207269in}{2.427586in}}%
\pgfpathlineto{\pgfqpoint{3.207565in}{2.423705in}}%
\pgfpathlineto{\pgfqpoint{3.208058in}{2.401896in}}%
\pgfpathlineto{\pgfqpoint{3.208650in}{2.414685in}}%
\pgfpathlineto{\pgfqpoint{3.209835in}{2.438194in}}%
\pgfpathlineto{\pgfqpoint{3.209933in}{2.436399in}}%
\pgfpathlineto{\pgfqpoint{3.211315in}{2.417968in}}%
\pgfpathlineto{\pgfqpoint{3.210723in}{2.440227in}}%
\pgfpathlineto{\pgfqpoint{3.211414in}{2.418672in}}%
\pgfpathlineto{\pgfqpoint{3.213585in}{2.446357in}}%
\pgfpathlineto{\pgfqpoint{3.214079in}{2.427805in}}%
\pgfpathlineto{\pgfqpoint{3.214868in}{2.439452in}}%
\pgfpathlineto{\pgfqpoint{3.215658in}{2.450374in}}%
\pgfpathlineto{\pgfqpoint{3.215855in}{2.442498in}}%
\pgfpathlineto{\pgfqpoint{3.216053in}{2.437183in}}%
\pgfpathlineto{\pgfqpoint{3.216448in}{2.446199in}}%
\pgfpathlineto{\pgfqpoint{3.216941in}{2.442534in}}%
\pgfpathlineto{\pgfqpoint{3.217237in}{2.432439in}}%
\pgfpathlineto{\pgfqpoint{3.217731in}{2.447053in}}%
\pgfpathlineto{\pgfqpoint{3.218323in}{2.433846in}}%
\pgfpathlineto{\pgfqpoint{3.218915in}{2.434763in}}%
\pgfpathlineto{\pgfqpoint{3.219112in}{2.430940in}}%
\pgfpathlineto{\pgfqpoint{3.220790in}{2.411082in}}%
\pgfpathlineto{\pgfqpoint{3.222567in}{2.383286in}}%
\pgfpathlineto{\pgfqpoint{3.222764in}{2.378417in}}%
\pgfpathlineto{\pgfqpoint{3.223356in}{2.390729in}}%
\pgfpathlineto{\pgfqpoint{3.223455in}{2.390712in}}%
\pgfpathlineto{\pgfqpoint{3.224541in}{2.371395in}}%
\pgfpathlineto{\pgfqpoint{3.225627in}{2.380651in}}%
\pgfpathlineto{\pgfqpoint{3.225824in}{2.384072in}}%
\pgfpathlineto{\pgfqpoint{3.226416in}{2.374554in}}%
\pgfpathlineto{\pgfqpoint{3.226811in}{2.371365in}}%
\pgfpathlineto{\pgfqpoint{3.227107in}{2.377566in}}%
\pgfpathlineto{\pgfqpoint{3.227897in}{2.381815in}}%
\pgfpathlineto{\pgfqpoint{3.227601in}{2.376126in}}%
\pgfpathlineto{\pgfqpoint{3.227995in}{2.379336in}}%
\pgfpathlineto{\pgfqpoint{3.228390in}{2.359976in}}%
\pgfpathlineto{\pgfqpoint{3.228982in}{2.386268in}}%
\pgfpathlineto{\pgfqpoint{3.229278in}{2.384935in}}%
\pgfpathlineto{\pgfqpoint{3.229575in}{2.375114in}}%
\pgfpathlineto{\pgfqpoint{3.229969in}{2.394703in}}%
\pgfpathlineto{\pgfqpoint{3.230068in}{2.397676in}}%
\pgfpathlineto{\pgfqpoint{3.230364in}{2.379341in}}%
\pgfpathlineto{\pgfqpoint{3.231548in}{2.365314in}}%
\pgfpathlineto{\pgfqpoint{3.230956in}{2.384683in}}%
\pgfpathlineto{\pgfqpoint{3.231647in}{2.365918in}}%
\pgfpathlineto{\pgfqpoint{3.231746in}{2.366466in}}%
\pgfpathlineto{\pgfqpoint{3.231943in}{2.362962in}}%
\pgfpathlineto{\pgfqpoint{3.232338in}{2.349140in}}%
\pgfpathlineto{\pgfqpoint{3.232733in}{2.376197in}}%
\pgfpathlineto{\pgfqpoint{3.233819in}{2.405185in}}%
\pgfpathlineto{\pgfqpoint{3.234016in}{2.400631in}}%
\pgfpathlineto{\pgfqpoint{3.235003in}{2.395630in}}%
\pgfpathlineto{\pgfqpoint{3.234509in}{2.409703in}}%
\pgfpathlineto{\pgfqpoint{3.235200in}{2.397717in}}%
\pgfpathlineto{\pgfqpoint{3.236780in}{2.376659in}}%
\pgfpathlineto{\pgfqpoint{3.237076in}{2.382786in}}%
\pgfpathlineto{\pgfqpoint{3.238161in}{2.402628in}}%
\pgfpathlineto{\pgfqpoint{3.238359in}{2.393748in}}%
\pgfpathlineto{\pgfqpoint{3.238556in}{2.384411in}}%
\pgfpathlineto{\pgfqpoint{3.239050in}{2.400702in}}%
\pgfpathlineto{\pgfqpoint{3.239346in}{2.396462in}}%
\pgfpathlineto{\pgfqpoint{3.239642in}{2.397617in}}%
\pgfpathlineto{\pgfqpoint{3.239839in}{2.394876in}}%
\pgfpathlineto{\pgfqpoint{3.240135in}{2.390673in}}%
\pgfpathlineto{\pgfqpoint{3.240530in}{2.400639in}}%
\pgfpathlineto{\pgfqpoint{3.241320in}{2.390300in}}%
\pgfpathlineto{\pgfqpoint{3.242208in}{2.548782in}}%
\pgfpathlineto{\pgfqpoint{3.243590in}{2.651457in}}%
\pgfpathlineto{\pgfqpoint{3.243688in}{2.648753in}}%
\pgfpathlineto{\pgfqpoint{3.244478in}{2.478228in}}%
\pgfpathlineto{\pgfqpoint{3.245860in}{2.390211in}}%
\pgfpathlineto{\pgfqpoint{3.247439in}{2.459420in}}%
\pgfpathlineto{\pgfqpoint{3.248031in}{2.502248in}}%
\pgfpathlineto{\pgfqpoint{3.248525in}{2.463001in}}%
\pgfpathlineto{\pgfqpoint{3.249018in}{2.417205in}}%
\pgfpathlineto{\pgfqpoint{3.249709in}{2.442692in}}%
\pgfpathlineto{\pgfqpoint{3.249906in}{2.439488in}}%
\pgfpathlineto{\pgfqpoint{3.250301in}{2.450198in}}%
\pgfpathlineto{\pgfqpoint{3.250992in}{2.490887in}}%
\pgfpathlineto{\pgfqpoint{3.251190in}{2.498950in}}%
\pgfpathlineto{\pgfqpoint{3.251584in}{2.469039in}}%
\pgfpathlineto{\pgfqpoint{3.252177in}{2.427310in}}%
\pgfpathlineto{\pgfqpoint{3.252867in}{2.438370in}}%
\pgfpathlineto{\pgfqpoint{3.254348in}{2.472307in}}%
\pgfpathlineto{\pgfqpoint{3.254545in}{2.468754in}}%
\pgfpathlineto{\pgfqpoint{3.255532in}{2.443938in}}%
\pgfpathlineto{\pgfqpoint{3.255927in}{2.453845in}}%
\pgfpathlineto{\pgfqpoint{3.257901in}{2.480181in}}%
\pgfpathlineto{\pgfqpoint{3.258493in}{2.457166in}}%
\pgfpathlineto{\pgfqpoint{3.259283in}{2.469768in}}%
\pgfpathlineto{\pgfqpoint{3.259678in}{2.480491in}}%
\pgfpathlineto{\pgfqpoint{3.260467in}{2.476462in}}%
\pgfpathlineto{\pgfqpoint{3.261849in}{2.457075in}}%
\pgfpathlineto{\pgfqpoint{3.262046in}{2.457166in}}%
\pgfpathlineto{\pgfqpoint{3.262343in}{2.460123in}}%
\pgfpathlineto{\pgfqpoint{3.262737in}{2.471577in}}%
\pgfpathlineto{\pgfqpoint{3.263527in}{2.463711in}}%
\pgfpathlineto{\pgfqpoint{3.264514in}{2.450929in}}%
\pgfpathlineto{\pgfqpoint{3.264119in}{2.464019in}}%
\pgfpathlineto{\pgfqpoint{3.264711in}{2.460388in}}%
\pgfpathlineto{\pgfqpoint{3.264909in}{2.469439in}}%
\pgfpathlineto{\pgfqpoint{3.265304in}{2.448327in}}%
\pgfpathlineto{\pgfqpoint{3.265797in}{2.465196in}}%
\pgfpathlineto{\pgfqpoint{3.267475in}{2.443469in}}%
\pgfpathlineto{\pgfqpoint{3.268955in}{2.408248in}}%
\pgfpathlineto{\pgfqpoint{3.269350in}{2.420882in}}%
\pgfpathlineto{\pgfqpoint{3.270633in}{2.388194in}}%
\pgfpathlineto{\pgfqpoint{3.271620in}{2.392710in}}%
\pgfpathlineto{\pgfqpoint{3.272311in}{2.385363in}}%
\pgfpathlineto{\pgfqpoint{3.272805in}{2.396307in}}%
\pgfpathlineto{\pgfqpoint{3.273792in}{2.378141in}}%
\pgfpathlineto{\pgfqpoint{3.274088in}{2.365908in}}%
\pgfpathlineto{\pgfqpoint{3.274581in}{2.381112in}}%
\pgfpathlineto{\pgfqpoint{3.274976in}{2.371437in}}%
\pgfpathlineto{\pgfqpoint{3.275371in}{2.380291in}}%
\pgfpathlineto{\pgfqpoint{3.275963in}{2.371713in}}%
\pgfpathlineto{\pgfqpoint{3.276259in}{2.363889in}}%
\pgfpathlineto{\pgfqpoint{3.276753in}{2.380194in}}%
\pgfpathlineto{\pgfqpoint{3.276950in}{2.374459in}}%
\pgfpathlineto{\pgfqpoint{3.277049in}{2.372949in}}%
\pgfpathlineto{\pgfqpoint{3.277246in}{2.381282in}}%
\pgfpathlineto{\pgfqpoint{3.278529in}{2.402397in}}%
\pgfpathlineto{\pgfqpoint{3.278628in}{2.401984in}}%
\pgfpathlineto{\pgfqpoint{3.280503in}{2.418084in}}%
\pgfpathlineto{\pgfqpoint{3.280799in}{2.411424in}}%
\pgfpathlineto{\pgfqpoint{3.283069in}{2.364110in}}%
\pgfpathlineto{\pgfqpoint{3.283267in}{2.368199in}}%
\pgfpathlineto{\pgfqpoint{3.283464in}{2.372937in}}%
\pgfpathlineto{\pgfqpoint{3.283958in}{2.359060in}}%
\pgfpathlineto{\pgfqpoint{3.284451in}{2.370645in}}%
\pgfpathlineto{\pgfqpoint{3.285537in}{2.365948in}}%
\pgfpathlineto{\pgfqpoint{3.285142in}{2.372783in}}%
\pgfpathlineto{\pgfqpoint{3.285636in}{2.367352in}}%
\pgfpathlineto{\pgfqpoint{3.286721in}{2.382267in}}%
\pgfpathlineto{\pgfqpoint{3.286919in}{2.377641in}}%
\pgfpathlineto{\pgfqpoint{3.287017in}{2.376180in}}%
\pgfpathlineto{\pgfqpoint{3.287412in}{2.382980in}}%
\pgfpathlineto{\pgfqpoint{3.287807in}{2.377960in}}%
\pgfpathlineto{\pgfqpoint{3.288202in}{2.393812in}}%
\pgfpathlineto{\pgfqpoint{3.288695in}{2.377968in}}%
\pgfpathlineto{\pgfqpoint{3.289090in}{2.359086in}}%
\pgfpathlineto{\pgfqpoint{3.289485in}{2.382681in}}%
\pgfpathlineto{\pgfqpoint{3.290867in}{2.639610in}}%
\pgfpathlineto{\pgfqpoint{3.291854in}{2.604783in}}%
\pgfpathlineto{\pgfqpoint{3.293729in}{2.365078in}}%
\pgfpathlineto{\pgfqpoint{3.294025in}{2.375019in}}%
\pgfpathlineto{\pgfqpoint{3.295999in}{2.489982in}}%
\pgfpathlineto{\pgfqpoint{3.296295in}{2.451110in}}%
\pgfpathlineto{\pgfqpoint{3.296887in}{2.393581in}}%
\pgfpathlineto{\pgfqpoint{3.297479in}{2.419929in}}%
\pgfpathlineto{\pgfqpoint{3.298960in}{2.483659in}}%
\pgfpathlineto{\pgfqpoint{3.299355in}{2.463432in}}%
\pgfpathlineto{\pgfqpoint{3.300243in}{2.417505in}}%
\pgfpathlineto{\pgfqpoint{3.300835in}{2.440611in}}%
\pgfpathlineto{\pgfqpoint{3.301033in}{2.437530in}}%
\pgfpathlineto{\pgfqpoint{3.301329in}{2.456139in}}%
\pgfpathlineto{\pgfqpoint{3.302118in}{2.474156in}}%
\pgfpathlineto{\pgfqpoint{3.302414in}{2.461823in}}%
\pgfpathlineto{\pgfqpoint{3.303303in}{2.430651in}}%
\pgfpathlineto{\pgfqpoint{3.304092in}{2.434808in}}%
\pgfpathlineto{\pgfqpoint{3.304191in}{2.436814in}}%
\pgfpathlineto{\pgfqpoint{3.304684in}{2.424615in}}%
\pgfpathlineto{\pgfqpoint{3.304981in}{2.428641in}}%
\pgfpathlineto{\pgfqpoint{3.305178in}{2.419829in}}%
\pgfpathlineto{\pgfqpoint{3.305375in}{2.410810in}}%
\pgfpathlineto{\pgfqpoint{3.305869in}{2.435263in}}%
\pgfpathlineto{\pgfqpoint{3.307941in}{2.525750in}}%
\pgfpathlineto{\pgfqpoint{3.308928in}{2.523120in}}%
\pgfpathlineto{\pgfqpoint{3.309718in}{2.488603in}}%
\pgfpathlineto{\pgfqpoint{3.310409in}{2.504947in}}%
\pgfpathlineto{\pgfqpoint{3.311889in}{2.521984in}}%
\pgfpathlineto{\pgfqpoint{3.313666in}{2.500565in}}%
\pgfpathlineto{\pgfqpoint{3.313765in}{2.500898in}}%
\pgfpathlineto{\pgfqpoint{3.314160in}{2.513142in}}%
\pgfpathlineto{\pgfqpoint{3.314653in}{2.499170in}}%
\pgfpathlineto{\pgfqpoint{3.314752in}{2.499218in}}%
\pgfpathlineto{\pgfqpoint{3.314949in}{2.499964in}}%
\pgfpathlineto{\pgfqpoint{3.315245in}{2.496942in}}%
\pgfpathlineto{\pgfqpoint{3.316035in}{2.481299in}}%
\pgfpathlineto{\pgfqpoint{3.316923in}{2.470786in}}%
\pgfpathlineto{\pgfqpoint{3.317219in}{2.478612in}}%
\pgfpathlineto{\pgfqpoint{3.317318in}{2.478773in}}%
\pgfpathlineto{\pgfqpoint{3.318897in}{2.461150in}}%
\pgfpathlineto{\pgfqpoint{3.319983in}{2.472510in}}%
\pgfpathlineto{\pgfqpoint{3.319588in}{2.458767in}}%
\pgfpathlineto{\pgfqpoint{3.320180in}{2.467164in}}%
\pgfpathlineto{\pgfqpoint{3.320378in}{2.461394in}}%
\pgfpathlineto{\pgfqpoint{3.321365in}{2.464111in}}%
\pgfpathlineto{\pgfqpoint{3.323141in}{2.449067in}}%
\pgfpathlineto{\pgfqpoint{3.323832in}{2.451706in}}%
\pgfpathlineto{\pgfqpoint{3.323931in}{2.451385in}}%
\pgfpathlineto{\pgfqpoint{3.324622in}{2.452690in}}%
\pgfpathlineto{\pgfqpoint{3.326201in}{2.489164in}}%
\pgfpathlineto{\pgfqpoint{3.326990in}{2.478783in}}%
\pgfpathlineto{\pgfqpoint{3.327089in}{2.475544in}}%
\pgfpathlineto{\pgfqpoint{3.327583in}{2.494282in}}%
\pgfpathlineto{\pgfqpoint{3.327681in}{2.494231in}}%
\pgfpathlineto{\pgfqpoint{3.332715in}{2.388919in}}%
\pgfpathlineto{\pgfqpoint{3.333011in}{2.393556in}}%
\pgfpathlineto{\pgfqpoint{3.333110in}{2.394630in}}%
\pgfpathlineto{\pgfqpoint{3.333406in}{2.387854in}}%
\pgfpathlineto{\pgfqpoint{3.333505in}{2.386707in}}%
\pgfpathlineto{\pgfqpoint{3.333702in}{2.392572in}}%
\pgfpathlineto{\pgfqpoint{3.333899in}{2.399262in}}%
\pgfpathlineto{\pgfqpoint{3.334294in}{2.380291in}}%
\pgfpathlineto{\pgfqpoint{3.334689in}{2.387591in}}%
\pgfpathlineto{\pgfqpoint{3.334985in}{2.379479in}}%
\pgfpathlineto{\pgfqpoint{3.335478in}{2.397662in}}%
\pgfpathlineto{\pgfqpoint{3.336959in}{2.367913in}}%
\pgfpathlineto{\pgfqpoint{3.337255in}{2.381076in}}%
\pgfpathlineto{\pgfqpoint{3.338834in}{2.661629in}}%
\pgfpathlineto{\pgfqpoint{3.339920in}{2.585141in}}%
\pgfpathlineto{\pgfqpoint{3.341598in}{2.398611in}}%
\pgfpathlineto{\pgfqpoint{3.341697in}{2.398757in}}%
\pgfpathlineto{\pgfqpoint{3.343374in}{2.514094in}}%
\pgfpathlineto{\pgfqpoint{3.343670in}{2.534664in}}%
\pgfpathlineto{\pgfqpoint{3.344164in}{2.495671in}}%
\pgfpathlineto{\pgfqpoint{3.344756in}{2.439743in}}%
\pgfpathlineto{\pgfqpoint{3.345546in}{2.464044in}}%
\pgfpathlineto{\pgfqpoint{3.345644in}{2.463169in}}%
\pgfpathlineto{\pgfqpoint{3.345842in}{2.469080in}}%
\pgfpathlineto{\pgfqpoint{3.347125in}{2.532229in}}%
\pgfpathlineto{\pgfqpoint{3.347520in}{2.512376in}}%
\pgfpathlineto{\pgfqpoint{3.347915in}{2.484337in}}%
\pgfpathlineto{\pgfqpoint{3.348605in}{2.506499in}}%
\pgfpathlineto{\pgfqpoint{3.350283in}{2.556803in}}%
\pgfpathlineto{\pgfqpoint{3.350481in}{2.553525in}}%
\pgfpathlineto{\pgfqpoint{3.351073in}{2.533483in}}%
\pgfpathlineto{\pgfqpoint{3.351764in}{2.542494in}}%
\pgfpathlineto{\pgfqpoint{3.353442in}{2.574536in}}%
\pgfpathlineto{\pgfqpoint{3.353836in}{2.590221in}}%
\pgfpathlineto{\pgfqpoint{3.354231in}{2.567935in}}%
\pgfpathlineto{\pgfqpoint{3.354330in}{2.567508in}}%
\pgfpathlineto{\pgfqpoint{3.354429in}{2.571659in}}%
\pgfpathlineto{\pgfqpoint{3.355416in}{2.608049in}}%
\pgfpathlineto{\pgfqpoint{3.356008in}{2.602400in}}%
\pgfpathlineto{\pgfqpoint{3.356304in}{2.605709in}}%
\pgfpathlineto{\pgfqpoint{3.356600in}{2.594892in}}%
\pgfpathlineto{\pgfqpoint{3.357488in}{2.585808in}}%
\pgfpathlineto{\pgfqpoint{3.357094in}{2.600362in}}%
\pgfpathlineto{\pgfqpoint{3.357587in}{2.588850in}}%
\pgfpathlineto{\pgfqpoint{3.358771in}{2.616891in}}%
\pgfpathlineto{\pgfqpoint{3.358969in}{2.611717in}}%
\pgfpathlineto{\pgfqpoint{3.359561in}{2.603872in}}%
\pgfpathlineto{\pgfqpoint{3.360252in}{2.607537in}}%
\pgfpathlineto{\pgfqpoint{3.361436in}{2.588230in}}%
\pgfpathlineto{\pgfqpoint{3.361732in}{2.601333in}}%
\pgfpathlineto{\pgfqpoint{3.361831in}{2.603347in}}%
\pgfpathlineto{\pgfqpoint{3.362226in}{2.589682in}}%
\pgfpathlineto{\pgfqpoint{3.362621in}{2.598042in}}%
\pgfpathlineto{\pgfqpoint{3.364002in}{2.564641in}}%
\pgfpathlineto{\pgfqpoint{3.364496in}{2.577370in}}%
\pgfpathlineto{\pgfqpoint{3.364891in}{2.563339in}}%
\pgfpathlineto{\pgfqpoint{3.366371in}{2.540917in}}%
\pgfpathlineto{\pgfqpoint{3.366865in}{2.552396in}}%
\pgfpathlineto{\pgfqpoint{3.367260in}{2.538382in}}%
\pgfpathlineto{\pgfqpoint{3.367358in}{2.535394in}}%
\pgfpathlineto{\pgfqpoint{3.367753in}{2.550997in}}%
\pgfpathlineto{\pgfqpoint{3.367950in}{2.557368in}}%
\pgfpathlineto{\pgfqpoint{3.368444in}{2.536869in}}%
\pgfpathlineto{\pgfqpoint{3.368641in}{2.540687in}}%
\pgfpathlineto{\pgfqpoint{3.369431in}{2.543472in}}%
\pgfpathlineto{\pgfqpoint{3.369036in}{2.537117in}}%
\pgfpathlineto{\pgfqpoint{3.369628in}{2.539423in}}%
\pgfpathlineto{\pgfqpoint{3.370418in}{2.533105in}}%
\pgfpathlineto{\pgfqpoint{3.370714in}{2.538593in}}%
\pgfpathlineto{\pgfqpoint{3.371504in}{2.540740in}}%
\pgfpathlineto{\pgfqpoint{3.371208in}{2.535300in}}%
\pgfpathlineto{\pgfqpoint{3.371602in}{2.539377in}}%
\pgfpathlineto{\pgfqpoint{3.371898in}{2.532178in}}%
\pgfpathlineto{\pgfqpoint{3.372293in}{2.544356in}}%
\pgfpathlineto{\pgfqpoint{3.372688in}{2.540117in}}%
\pgfpathlineto{\pgfqpoint{3.374563in}{2.570824in}}%
\pgfpathlineto{\pgfqpoint{3.375155in}{2.566340in}}%
\pgfpathlineto{\pgfqpoint{3.378314in}{2.432199in}}%
\pgfpathlineto{\pgfqpoint{3.379202in}{2.457808in}}%
\pgfpathlineto{\pgfqpoint{3.380880in}{2.500712in}}%
\pgfpathlineto{\pgfqpoint{3.381077in}{2.499085in}}%
\pgfpathlineto{\pgfqpoint{3.382953in}{2.471950in}}%
\pgfpathlineto{\pgfqpoint{3.383940in}{2.464994in}}%
\pgfpathlineto{\pgfqpoint{3.383545in}{2.474437in}}%
\pgfpathlineto{\pgfqpoint{3.384137in}{2.469269in}}%
\pgfpathlineto{\pgfqpoint{3.384334in}{2.473706in}}%
\pgfpathlineto{\pgfqpoint{3.384828in}{2.459188in}}%
\pgfpathlineto{\pgfqpoint{3.385321in}{2.430863in}}%
\pgfpathlineto{\pgfqpoint{3.385618in}{2.455151in}}%
\pgfpathlineto{\pgfqpoint{3.387098in}{2.720639in}}%
\pgfpathlineto{\pgfqpoint{3.387986in}{2.672285in}}%
\pgfpathlineto{\pgfqpoint{3.389960in}{2.431086in}}%
\pgfpathlineto{\pgfqpoint{3.390059in}{2.434163in}}%
\pgfpathlineto{\pgfqpoint{3.391836in}{2.550395in}}%
\pgfpathlineto{\pgfqpoint{3.392329in}{2.523972in}}%
\pgfpathlineto{\pgfqpoint{3.393119in}{2.462681in}}%
\pgfpathlineto{\pgfqpoint{3.393711in}{2.492548in}}%
\pgfpathlineto{\pgfqpoint{3.395290in}{2.534104in}}%
\pgfpathlineto{\pgfqpoint{3.395586in}{2.519754in}}%
\pgfpathlineto{\pgfqpoint{3.396376in}{2.471339in}}%
\pgfpathlineto{\pgfqpoint{3.396869in}{2.493455in}}%
\pgfpathlineto{\pgfqpoint{3.398448in}{2.535763in}}%
\pgfpathlineto{\pgfqpoint{3.398646in}{2.524964in}}%
\pgfpathlineto{\pgfqpoint{3.399139in}{2.495284in}}%
\pgfpathlineto{\pgfqpoint{3.399929in}{2.500858in}}%
\pgfpathlineto{\pgfqpoint{3.400324in}{2.519648in}}%
\pgfpathlineto{\pgfqpoint{3.401212in}{2.512478in}}%
\pgfpathlineto{\pgfqpoint{3.402692in}{2.484633in}}%
\pgfpathlineto{\pgfqpoint{3.401607in}{2.513784in}}%
\pgfpathlineto{\pgfqpoint{3.402989in}{2.498109in}}%
\pgfpathlineto{\pgfqpoint{3.403778in}{2.516092in}}%
\pgfpathlineto{\pgfqpoint{3.404173in}{2.504516in}}%
\pgfpathlineto{\pgfqpoint{3.404568in}{2.517850in}}%
\pgfpathlineto{\pgfqpoint{3.405160in}{2.506103in}}%
\pgfpathlineto{\pgfqpoint{3.405555in}{2.492967in}}%
\pgfpathlineto{\pgfqpoint{3.405950in}{2.507069in}}%
\pgfpathlineto{\pgfqpoint{3.406344in}{2.495697in}}%
\pgfpathlineto{\pgfqpoint{3.406739in}{2.514877in}}%
\pgfpathlineto{\pgfqpoint{3.407529in}{2.500396in}}%
\pgfpathlineto{\pgfqpoint{3.408417in}{2.494795in}}%
\pgfpathlineto{\pgfqpoint{3.408022in}{2.505951in}}%
\pgfpathlineto{\pgfqpoint{3.408614in}{2.500782in}}%
\pgfpathlineto{\pgfqpoint{3.408812in}{2.505121in}}%
\pgfpathlineto{\pgfqpoint{3.409601in}{2.498500in}}%
\pgfpathlineto{\pgfqpoint{3.411279in}{2.474117in}}%
\pgfpathlineto{\pgfqpoint{3.410194in}{2.508834in}}%
\pgfpathlineto{\pgfqpoint{3.411378in}{2.475198in}}%
\pgfpathlineto{\pgfqpoint{3.411477in}{2.476071in}}%
\pgfpathlineto{\pgfqpoint{3.411674in}{2.471676in}}%
\pgfpathlineto{\pgfqpoint{3.413648in}{2.426553in}}%
\pgfpathlineto{\pgfqpoint{3.413747in}{2.425756in}}%
\pgfpathlineto{\pgfqpoint{3.414142in}{2.431009in}}%
\pgfpathlineto{\pgfqpoint{3.414339in}{2.434257in}}%
\pgfpathlineto{\pgfqpoint{3.414635in}{2.417683in}}%
\pgfpathlineto{\pgfqpoint{3.415622in}{2.400913in}}%
\pgfpathlineto{\pgfqpoint{3.415819in}{2.404837in}}%
\pgfpathlineto{\pgfqpoint{3.415918in}{2.406853in}}%
\pgfpathlineto{\pgfqpoint{3.416313in}{2.393103in}}%
\pgfpathlineto{\pgfqpoint{3.416412in}{2.390635in}}%
\pgfpathlineto{\pgfqpoint{3.416806in}{2.404238in}}%
\pgfpathlineto{\pgfqpoint{3.417004in}{2.406630in}}%
\pgfpathlineto{\pgfqpoint{3.417300in}{2.395377in}}%
\pgfpathlineto{\pgfqpoint{3.418386in}{2.387711in}}%
\pgfpathlineto{\pgfqpoint{3.417991in}{2.401984in}}%
\pgfpathlineto{\pgfqpoint{3.418484in}{2.390310in}}%
\pgfpathlineto{\pgfqpoint{3.419767in}{2.408514in}}%
\pgfpathlineto{\pgfqpoint{3.419274in}{2.388281in}}%
\pgfpathlineto{\pgfqpoint{3.419866in}{2.407578in}}%
\pgfpathlineto{\pgfqpoint{3.421149in}{2.391841in}}%
\pgfpathlineto{\pgfqpoint{3.421248in}{2.393590in}}%
\pgfpathlineto{\pgfqpoint{3.424505in}{2.435850in}}%
\pgfpathlineto{\pgfqpoint{3.424604in}{2.435803in}}%
\pgfpathlineto{\pgfqpoint{3.425492in}{2.416381in}}%
\pgfpathlineto{\pgfqpoint{3.427170in}{2.373542in}}%
\pgfpathlineto{\pgfqpoint{3.427466in}{2.379803in}}%
\pgfpathlineto{\pgfqpoint{3.427565in}{2.380145in}}%
\pgfpathlineto{\pgfqpoint{3.427663in}{2.377306in}}%
\pgfpathlineto{\pgfqpoint{3.428848in}{2.355184in}}%
\pgfpathlineto{\pgfqpoint{3.429045in}{2.359738in}}%
\pgfpathlineto{\pgfqpoint{3.430328in}{2.373478in}}%
\pgfpathlineto{\pgfqpoint{3.430427in}{2.373017in}}%
\pgfpathlineto{\pgfqpoint{3.430723in}{2.366967in}}%
\pgfpathlineto{\pgfqpoint{3.431216in}{2.379423in}}%
\pgfpathlineto{\pgfqpoint{3.432500in}{2.389070in}}%
\pgfpathlineto{\pgfqpoint{3.432697in}{2.383985in}}%
\pgfpathlineto{\pgfqpoint{3.433585in}{2.365493in}}%
\pgfpathlineto{\pgfqpoint{3.433092in}{2.384509in}}%
\pgfpathlineto{\pgfqpoint{3.433881in}{2.381643in}}%
\pgfpathlineto{\pgfqpoint{3.435658in}{2.641768in}}%
\pgfpathlineto{\pgfqpoint{3.436448in}{2.589891in}}%
\pgfpathlineto{\pgfqpoint{3.438125in}{2.381140in}}%
\pgfpathlineto{\pgfqpoint{3.438224in}{2.381697in}}%
\pgfpathlineto{\pgfqpoint{3.438619in}{2.381621in}}%
\pgfpathlineto{\pgfqpoint{3.438915in}{2.397949in}}%
\pgfpathlineto{\pgfqpoint{3.440395in}{2.509872in}}%
\pgfpathlineto{\pgfqpoint{3.440692in}{2.490692in}}%
\pgfpathlineto{\pgfqpoint{3.441382in}{2.423453in}}%
\pgfpathlineto{\pgfqpoint{3.442073in}{2.442258in}}%
\pgfpathlineto{\pgfqpoint{3.442962in}{2.472750in}}%
\pgfpathlineto{\pgfqpoint{3.443554in}{2.496868in}}%
\pgfpathlineto{\pgfqpoint{3.443949in}{2.478545in}}%
\pgfpathlineto{\pgfqpoint{3.444837in}{2.434540in}}%
\pgfpathlineto{\pgfqpoint{3.445232in}{2.458335in}}%
\pgfpathlineto{\pgfqpoint{3.446712in}{2.484725in}}%
\pgfpathlineto{\pgfqpoint{3.446910in}{2.481968in}}%
\pgfpathlineto{\pgfqpoint{3.448094in}{2.427881in}}%
\pgfpathlineto{\pgfqpoint{3.448884in}{2.429019in}}%
\pgfpathlineto{\pgfqpoint{3.448982in}{2.429571in}}%
\pgfpathlineto{\pgfqpoint{3.449180in}{2.427489in}}%
\pgfpathlineto{\pgfqpoint{3.449574in}{2.419136in}}%
\pgfpathlineto{\pgfqpoint{3.449969in}{2.432732in}}%
\pgfpathlineto{\pgfqpoint{3.450167in}{2.429624in}}%
\pgfpathlineto{\pgfqpoint{3.450265in}{2.428380in}}%
\pgfpathlineto{\pgfqpoint{3.450463in}{2.435301in}}%
\pgfpathlineto{\pgfqpoint{3.452338in}{2.523790in}}%
\pgfpathlineto{\pgfqpoint{3.452535in}{2.520541in}}%
\pgfpathlineto{\pgfqpoint{3.454312in}{2.490188in}}%
\pgfpathlineto{\pgfqpoint{3.454509in}{2.493360in}}%
\pgfpathlineto{\pgfqpoint{3.455398in}{2.517602in}}%
\pgfpathlineto{\pgfqpoint{3.455793in}{2.502046in}}%
\pgfpathlineto{\pgfqpoint{3.455891in}{2.502068in}}%
\pgfpathlineto{\pgfqpoint{3.456187in}{2.510446in}}%
\pgfpathlineto{\pgfqpoint{3.456878in}{2.504031in}}%
\pgfpathlineto{\pgfqpoint{3.458260in}{2.479857in}}%
\pgfpathlineto{\pgfqpoint{3.458457in}{2.487277in}}%
\pgfpathlineto{\pgfqpoint{3.458655in}{2.494256in}}%
\pgfpathlineto{\pgfqpoint{3.459050in}{2.479086in}}%
\pgfpathlineto{\pgfqpoint{3.459444in}{2.487072in}}%
\pgfpathlineto{\pgfqpoint{3.460925in}{2.449432in}}%
\pgfpathlineto{\pgfqpoint{3.462504in}{2.423820in}}%
\pgfpathlineto{\pgfqpoint{3.464182in}{2.405938in}}%
\pgfpathlineto{\pgfqpoint{3.462899in}{2.428750in}}%
\pgfpathlineto{\pgfqpoint{3.464675in}{2.409391in}}%
\pgfpathlineto{\pgfqpoint{3.465465in}{2.416832in}}%
\pgfpathlineto{\pgfqpoint{3.465761in}{2.412418in}}%
\pgfpathlineto{\pgfqpoint{3.465958in}{2.408740in}}%
\pgfpathlineto{\pgfqpoint{3.466353in}{2.418456in}}%
\pgfpathlineto{\pgfqpoint{3.466748in}{2.413584in}}%
\pgfpathlineto{\pgfqpoint{3.467143in}{2.426408in}}%
\pgfpathlineto{\pgfqpoint{3.467932in}{2.416192in}}%
\pgfpathlineto{\pgfqpoint{3.468722in}{2.403891in}}%
\pgfpathlineto{\pgfqpoint{3.468327in}{2.418504in}}%
\pgfpathlineto{\pgfqpoint{3.468919in}{2.413144in}}%
\pgfpathlineto{\pgfqpoint{3.470301in}{2.436079in}}%
\pgfpathlineto{\pgfqpoint{3.470499in}{2.443331in}}%
\pgfpathlineto{\pgfqpoint{3.470992in}{2.431068in}}%
\pgfpathlineto{\pgfqpoint{3.471486in}{2.441590in}}%
\pgfpathlineto{\pgfqpoint{3.471683in}{2.446847in}}%
\pgfpathlineto{\pgfqpoint{3.472177in}{2.432958in}}%
\pgfpathlineto{\pgfqpoint{3.472571in}{2.442922in}}%
\pgfpathlineto{\pgfqpoint{3.472670in}{2.443856in}}%
\pgfpathlineto{\pgfqpoint{3.472966in}{2.436945in}}%
\pgfpathlineto{\pgfqpoint{3.473657in}{2.409805in}}%
\pgfpathlineto{\pgfqpoint{3.475236in}{2.381070in}}%
\pgfpathlineto{\pgfqpoint{3.475631in}{2.372027in}}%
\pgfpathlineto{\pgfqpoint{3.476124in}{2.360054in}}%
\pgfpathlineto{\pgfqpoint{3.476815in}{2.367057in}}%
\pgfpathlineto{\pgfqpoint{3.477013in}{2.365227in}}%
\pgfpathlineto{\pgfqpoint{3.477408in}{2.375305in}}%
\pgfpathlineto{\pgfqpoint{3.477605in}{2.377519in}}%
\pgfpathlineto{\pgfqpoint{3.478296in}{2.373758in}}%
\pgfpathlineto{\pgfqpoint{3.478691in}{2.369316in}}%
\pgfpathlineto{\pgfqpoint{3.478987in}{2.375643in}}%
\pgfpathlineto{\pgfqpoint{3.480270in}{2.388747in}}%
\pgfpathlineto{\pgfqpoint{3.479776in}{2.373658in}}%
\pgfpathlineto{\pgfqpoint{3.480467in}{2.384402in}}%
\pgfpathlineto{\pgfqpoint{3.481948in}{2.360642in}}%
\pgfpathlineto{\pgfqpoint{3.481158in}{2.398486in}}%
\pgfpathlineto{\pgfqpoint{3.482046in}{2.364971in}}%
\pgfpathlineto{\pgfqpoint{3.482935in}{2.568204in}}%
\pgfpathlineto{\pgfqpoint{3.483724in}{2.646413in}}%
\pgfpathlineto{\pgfqpoint{3.484218in}{2.630034in}}%
\pgfpathlineto{\pgfqpoint{3.484711in}{2.572701in}}%
\pgfpathlineto{\pgfqpoint{3.486488in}{2.367759in}}%
\pgfpathlineto{\pgfqpoint{3.487179in}{2.384393in}}%
\pgfpathlineto{\pgfqpoint{3.488067in}{2.439504in}}%
\pgfpathlineto{\pgfqpoint{3.488561in}{2.484865in}}%
\pgfpathlineto{\pgfqpoint{3.489054in}{2.447442in}}%
\pgfpathlineto{\pgfqpoint{3.489646in}{2.393284in}}%
\pgfpathlineto{\pgfqpoint{3.490238in}{2.417213in}}%
\pgfpathlineto{\pgfqpoint{3.491719in}{2.476017in}}%
\pgfpathlineto{\pgfqpoint{3.491916in}{2.472479in}}%
\pgfpathlineto{\pgfqpoint{3.492903in}{2.420320in}}%
\pgfpathlineto{\pgfqpoint{3.493397in}{2.443700in}}%
\pgfpathlineto{\pgfqpoint{3.495173in}{2.472222in}}%
\pgfpathlineto{\pgfqpoint{3.495371in}{2.465911in}}%
\pgfpathlineto{\pgfqpoint{3.495864in}{2.443624in}}%
\pgfpathlineto{\pgfqpoint{3.496456in}{2.459319in}}%
\pgfpathlineto{\pgfqpoint{3.498036in}{2.486034in}}%
\pgfpathlineto{\pgfqpoint{3.498134in}{2.485152in}}%
\pgfpathlineto{\pgfqpoint{3.499417in}{2.463189in}}%
\pgfpathlineto{\pgfqpoint{3.499615in}{2.464763in}}%
\pgfpathlineto{\pgfqpoint{3.500404in}{2.484457in}}%
\pgfpathlineto{\pgfqpoint{3.500898in}{2.469605in}}%
\pgfpathlineto{\pgfqpoint{3.500997in}{2.468450in}}%
\pgfpathlineto{\pgfqpoint{3.501293in}{2.477017in}}%
\pgfpathlineto{\pgfqpoint{3.501490in}{2.481844in}}%
\pgfpathlineto{\pgfqpoint{3.501786in}{2.464946in}}%
\pgfpathlineto{\pgfqpoint{3.501984in}{2.455811in}}%
\pgfpathlineto{\pgfqpoint{3.502378in}{2.465787in}}%
\pgfpathlineto{\pgfqpoint{3.502872in}{2.460926in}}%
\pgfpathlineto{\pgfqpoint{3.503661in}{2.475552in}}%
\pgfpathlineto{\pgfqpoint{3.504155in}{2.471809in}}%
\pgfpathlineto{\pgfqpoint{3.504648in}{2.479431in}}%
\pgfpathlineto{\pgfqpoint{3.505043in}{2.469702in}}%
\pgfpathlineto{\pgfqpoint{3.506030in}{2.460673in}}%
\pgfpathlineto{\pgfqpoint{3.505537in}{2.472078in}}%
\pgfpathlineto{\pgfqpoint{3.506228in}{2.465024in}}%
\pgfpathlineto{\pgfqpoint{3.506425in}{2.469012in}}%
\pgfpathlineto{\pgfqpoint{3.506820in}{2.453582in}}%
\pgfpathlineto{\pgfqpoint{3.508300in}{2.434287in}}%
\pgfpathlineto{\pgfqpoint{3.508498in}{2.441421in}}%
\pgfpathlineto{\pgfqpoint{3.508596in}{2.442466in}}%
\pgfpathlineto{\pgfqpoint{3.508695in}{2.438693in}}%
\pgfpathlineto{\pgfqpoint{3.509781in}{2.414737in}}%
\pgfpathlineto{\pgfqpoint{3.509978in}{2.421391in}}%
\pgfpathlineto{\pgfqpoint{3.510176in}{2.425532in}}%
\pgfpathlineto{\pgfqpoint{3.510570in}{2.413929in}}%
\pgfpathlineto{\pgfqpoint{3.510965in}{2.422048in}}%
\pgfpathlineto{\pgfqpoint{3.512051in}{2.403105in}}%
\pgfpathlineto{\pgfqpoint{3.512446in}{2.414605in}}%
\pgfpathlineto{\pgfqpoint{3.513334in}{2.428611in}}%
\pgfpathlineto{\pgfqpoint{3.512840in}{2.412032in}}%
\pgfpathlineto{\pgfqpoint{3.513531in}{2.419991in}}%
\pgfpathlineto{\pgfqpoint{3.514420in}{2.406564in}}%
\pgfpathlineto{\pgfqpoint{3.514617in}{2.413636in}}%
\pgfpathlineto{\pgfqpoint{3.514814in}{2.418219in}}%
\pgfpathlineto{\pgfqpoint{3.515209in}{2.406473in}}%
\pgfpathlineto{\pgfqpoint{3.515801in}{2.416827in}}%
\pgfpathlineto{\pgfqpoint{3.515900in}{2.416526in}}%
\pgfpathlineto{\pgfqpoint{3.516394in}{2.418348in}}%
\pgfpathlineto{\pgfqpoint{3.516492in}{2.418227in}}%
\pgfpathlineto{\pgfqpoint{3.516986in}{2.426609in}}%
\pgfpathlineto{\pgfqpoint{3.517183in}{2.420003in}}%
\pgfpathlineto{\pgfqpoint{3.517479in}{2.411840in}}%
\pgfpathlineto{\pgfqpoint{3.517874in}{2.435902in}}%
\pgfpathlineto{\pgfqpoint{3.518762in}{2.452476in}}%
\pgfpathlineto{\pgfqpoint{3.519059in}{2.445487in}}%
\pgfpathlineto{\pgfqpoint{3.519157in}{2.442411in}}%
\pgfpathlineto{\pgfqpoint{3.519552in}{2.460216in}}%
\pgfpathlineto{\pgfqpoint{3.519749in}{2.464941in}}%
\pgfpathlineto{\pgfqpoint{3.520243in}{2.451511in}}%
\pgfpathlineto{\pgfqpoint{3.520440in}{2.453606in}}%
\pgfpathlineto{\pgfqpoint{3.520934in}{2.459262in}}%
\pgfpathlineto{\pgfqpoint{3.521230in}{2.451621in}}%
\pgfpathlineto{\pgfqpoint{3.524191in}{2.348351in}}%
\pgfpathlineto{\pgfqpoint{3.526165in}{2.301326in}}%
\pgfpathlineto{\pgfqpoint{3.526264in}{2.304046in}}%
\pgfpathlineto{\pgfqpoint{3.528830in}{2.381325in}}%
\pgfpathlineto{\pgfqpoint{3.530014in}{2.335469in}}%
\pgfpathlineto{\pgfqpoint{3.529422in}{2.390234in}}%
\pgfpathlineto{\pgfqpoint{3.530409in}{2.356481in}}%
\pgfpathlineto{\pgfqpoint{3.531988in}{2.627198in}}%
\pgfpathlineto{\pgfqpoint{3.532876in}{2.583021in}}%
\pgfpathlineto{\pgfqpoint{3.534752in}{2.341893in}}%
\pgfpathlineto{\pgfqpoint{3.534850in}{2.343037in}}%
\pgfpathlineto{\pgfqpoint{3.536824in}{2.462825in}}%
\pgfpathlineto{\pgfqpoint{3.537219in}{2.423470in}}%
\pgfpathlineto{\pgfqpoint{3.538107in}{2.363663in}}%
\pgfpathlineto{\pgfqpoint{3.538502in}{2.393710in}}%
\pgfpathlineto{\pgfqpoint{3.540081in}{2.443578in}}%
\pgfpathlineto{\pgfqpoint{3.540279in}{2.435158in}}%
\pgfpathlineto{\pgfqpoint{3.540871in}{2.395137in}}%
\pgfpathlineto{\pgfqpoint{3.541562in}{2.407415in}}%
\pgfpathlineto{\pgfqpoint{3.543536in}{2.454802in}}%
\pgfpathlineto{\pgfqpoint{3.543635in}{2.453750in}}%
\pgfpathlineto{\pgfqpoint{3.544325in}{2.425150in}}%
\pgfpathlineto{\pgfqpoint{3.544918in}{2.446149in}}%
\pgfpathlineto{\pgfqpoint{3.546102in}{2.465571in}}%
\pgfpathlineto{\pgfqpoint{3.546694in}{2.458590in}}%
\pgfpathlineto{\pgfqpoint{3.546793in}{2.458843in}}%
\pgfpathlineto{\pgfqpoint{3.546990in}{2.457123in}}%
\pgfpathlineto{\pgfqpoint{3.547681in}{2.442703in}}%
\pgfpathlineto{\pgfqpoint{3.548372in}{2.452073in}}%
\pgfpathlineto{\pgfqpoint{3.548767in}{2.459267in}}%
\pgfpathlineto{\pgfqpoint{3.548964in}{2.463269in}}%
\pgfpathlineto{\pgfqpoint{3.549458in}{2.449644in}}%
\pgfpathlineto{\pgfqpoint{3.549557in}{2.449504in}}%
\pgfpathlineto{\pgfqpoint{3.549655in}{2.450615in}}%
\pgfpathlineto{\pgfqpoint{3.549853in}{2.451500in}}%
\pgfpathlineto{\pgfqpoint{3.550050in}{2.447407in}}%
\pgfpathlineto{\pgfqpoint{3.550543in}{2.434650in}}%
\pgfpathlineto{\pgfqpoint{3.551037in}{2.447993in}}%
\pgfpathlineto{\pgfqpoint{3.551136in}{2.447277in}}%
\pgfpathlineto{\pgfqpoint{3.551333in}{2.444643in}}%
\pgfpathlineto{\pgfqpoint{3.551728in}{2.453278in}}%
\pgfpathlineto{\pgfqpoint{3.552123in}{2.459579in}}%
\pgfpathlineto{\pgfqpoint{3.552419in}{2.450755in}}%
\pgfpathlineto{\pgfqpoint{3.552616in}{2.442353in}}%
\pgfpathlineto{\pgfqpoint{3.553110in}{2.458274in}}%
\pgfpathlineto{\pgfqpoint{3.553406in}{2.454544in}}%
\pgfpathlineto{\pgfqpoint{3.553603in}{2.455175in}}%
\pgfpathlineto{\pgfqpoint{3.553702in}{2.454168in}}%
\pgfpathlineto{\pgfqpoint{3.554097in}{2.446913in}}%
\pgfpathlineto{\pgfqpoint{3.554788in}{2.453846in}}%
\pgfpathlineto{\pgfqpoint{3.554985in}{2.451790in}}%
\pgfpathlineto{\pgfqpoint{3.557354in}{2.412088in}}%
\pgfpathlineto{\pgfqpoint{3.557452in}{2.413557in}}%
\pgfpathlineto{\pgfqpoint{3.557650in}{2.416379in}}%
\pgfpathlineto{\pgfqpoint{3.557946in}{2.400863in}}%
\pgfpathlineto{\pgfqpoint{3.558439in}{2.401728in}}%
\pgfpathlineto{\pgfqpoint{3.559426in}{2.385871in}}%
\pgfpathlineto{\pgfqpoint{3.559624in}{2.385405in}}%
\pgfpathlineto{\pgfqpoint{3.559920in}{2.387565in}}%
\pgfpathlineto{\pgfqpoint{3.560216in}{2.390974in}}%
\pgfpathlineto{\pgfqpoint{3.560611in}{2.384449in}}%
\pgfpathlineto{\pgfqpoint{3.560907in}{2.385605in}}%
\pgfpathlineto{\pgfqpoint{3.561203in}{2.384131in}}%
\pgfpathlineto{\pgfqpoint{3.562091in}{2.375989in}}%
\pgfpathlineto{\pgfqpoint{3.562289in}{2.380061in}}%
\pgfpathlineto{\pgfqpoint{3.562585in}{2.388857in}}%
\pgfpathlineto{\pgfqpoint{3.563078in}{2.366769in}}%
\pgfpathlineto{\pgfqpoint{3.563473in}{2.377511in}}%
\pgfpathlineto{\pgfqpoint{3.564756in}{2.395169in}}%
\pgfpathlineto{\pgfqpoint{3.564855in}{2.393250in}}%
\pgfpathlineto{\pgfqpoint{3.565151in}{2.386635in}}%
\pgfpathlineto{\pgfqpoint{3.565842in}{2.391686in}}%
\pgfpathlineto{\pgfqpoint{3.568408in}{2.441630in}}%
\pgfpathlineto{\pgfqpoint{3.568704in}{2.440673in}}%
\pgfpathlineto{\pgfqpoint{3.568803in}{2.441350in}}%
\pgfpathlineto{\pgfqpoint{3.569000in}{2.435766in}}%
\pgfpathlineto{\pgfqpoint{3.570777in}{2.387748in}}%
\pgfpathlineto{\pgfqpoint{3.570875in}{2.388765in}}%
\pgfpathlineto{\pgfqpoint{3.571172in}{2.382072in}}%
\pgfpathlineto{\pgfqpoint{3.572159in}{2.369514in}}%
\pgfpathlineto{\pgfqpoint{3.571665in}{2.384646in}}%
\pgfpathlineto{\pgfqpoint{3.572455in}{2.374744in}}%
\pgfpathlineto{\pgfqpoint{3.572849in}{2.362529in}}%
\pgfpathlineto{\pgfqpoint{3.573343in}{2.375439in}}%
\pgfpathlineto{\pgfqpoint{3.573738in}{2.370718in}}%
\pgfpathlineto{\pgfqpoint{3.574922in}{2.377339in}}%
\pgfpathlineto{\pgfqpoint{3.574330in}{2.361608in}}%
\pgfpathlineto{\pgfqpoint{3.575021in}{2.376380in}}%
\pgfpathlineto{\pgfqpoint{3.575120in}{2.375796in}}%
\pgfpathlineto{\pgfqpoint{3.575613in}{2.378002in}}%
\pgfpathlineto{\pgfqpoint{3.577488in}{2.397289in}}%
\pgfpathlineto{\pgfqpoint{3.577686in}{2.394794in}}%
\pgfpathlineto{\pgfqpoint{3.578080in}{2.383452in}}%
\pgfpathlineto{\pgfqpoint{3.578377in}{2.398617in}}%
\pgfpathlineto{\pgfqpoint{3.580054in}{2.658332in}}%
\pgfpathlineto{\pgfqpoint{3.580844in}{2.614658in}}%
\pgfpathlineto{\pgfqpoint{3.581634in}{2.398850in}}%
\pgfpathlineto{\pgfqpoint{3.582719in}{2.400036in}}%
\pgfpathlineto{\pgfqpoint{3.583015in}{2.396116in}}%
\pgfpathlineto{\pgfqpoint{3.583213in}{2.405329in}}%
\pgfpathlineto{\pgfqpoint{3.584792in}{2.525449in}}%
\pgfpathlineto{\pgfqpoint{3.585187in}{2.479896in}}%
\pgfpathlineto{\pgfqpoint{3.585878in}{2.418625in}}%
\pgfpathlineto{\pgfqpoint{3.586470in}{2.456953in}}%
\pgfpathlineto{\pgfqpoint{3.588049in}{2.514013in}}%
\pgfpathlineto{\pgfqpoint{3.588246in}{2.505547in}}%
\pgfpathlineto{\pgfqpoint{3.589233in}{2.452604in}}%
\pgfpathlineto{\pgfqpoint{3.589628in}{2.475687in}}%
\pgfpathlineto{\pgfqpoint{3.591405in}{2.519474in}}%
\pgfpathlineto{\pgfqpoint{3.591602in}{2.515908in}}%
\pgfpathlineto{\pgfqpoint{3.591997in}{2.501640in}}%
\pgfpathlineto{\pgfqpoint{3.592787in}{2.510275in}}%
\pgfpathlineto{\pgfqpoint{3.594366in}{2.539872in}}%
\pgfpathlineto{\pgfqpoint{3.594761in}{2.546851in}}%
\pgfpathlineto{\pgfqpoint{3.595155in}{2.536752in}}%
\pgfpathlineto{\pgfqpoint{3.595452in}{2.540360in}}%
\pgfpathlineto{\pgfqpoint{3.596044in}{2.551194in}}%
\pgfpathlineto{\pgfqpoint{3.596537in}{2.571007in}}%
\pgfpathlineto{\pgfqpoint{3.597327in}{2.565437in}}%
\pgfpathlineto{\pgfqpoint{3.598018in}{2.571690in}}%
\pgfpathlineto{\pgfqpoint{3.598314in}{2.562406in}}%
\pgfpathlineto{\pgfqpoint{3.599202in}{2.551851in}}%
\pgfpathlineto{\pgfqpoint{3.598807in}{2.564601in}}%
\pgfpathlineto{\pgfqpoint{3.599399in}{2.555481in}}%
\pgfpathlineto{\pgfqpoint{3.599597in}{2.561827in}}%
\pgfpathlineto{\pgfqpoint{3.600288in}{2.548337in}}%
\pgfpathlineto{\pgfqpoint{3.603051in}{2.472187in}}%
\pgfpathlineto{\pgfqpoint{3.603347in}{2.488669in}}%
\pgfpathlineto{\pgfqpoint{3.604433in}{2.527175in}}%
\pgfpathlineto{\pgfqpoint{3.604927in}{2.521343in}}%
\pgfpathlineto{\pgfqpoint{3.605223in}{2.530088in}}%
\pgfpathlineto{\pgfqpoint{3.605914in}{2.519505in}}%
\pgfpathlineto{\pgfqpoint{3.606407in}{2.520471in}}%
\pgfpathlineto{\pgfqpoint{3.607493in}{2.498063in}}%
\pgfpathlineto{\pgfqpoint{3.608578in}{2.520636in}}%
\pgfpathlineto{\pgfqpoint{3.609565in}{2.514457in}}%
\pgfpathlineto{\pgfqpoint{3.610158in}{2.504682in}}%
\pgfpathlineto{\pgfqpoint{3.611539in}{2.491084in}}%
\pgfpathlineto{\pgfqpoint{3.611737in}{2.495487in}}%
\pgfpathlineto{\pgfqpoint{3.613020in}{2.501684in}}%
\pgfpathlineto{\pgfqpoint{3.612230in}{2.489914in}}%
\pgfpathlineto{\pgfqpoint{3.613119in}{2.501641in}}%
\pgfpathlineto{\pgfqpoint{3.613612in}{2.469277in}}%
\pgfpathlineto{\pgfqpoint{3.614402in}{2.491044in}}%
\pgfpathlineto{\pgfqpoint{3.616080in}{2.524360in}}%
\pgfpathlineto{\pgfqpoint{3.616277in}{2.518760in}}%
\pgfpathlineto{\pgfqpoint{3.617363in}{2.502056in}}%
\pgfpathlineto{\pgfqpoint{3.616869in}{2.524460in}}%
\pgfpathlineto{\pgfqpoint{3.617560in}{2.507214in}}%
\pgfpathlineto{\pgfqpoint{3.617659in}{2.510639in}}%
\pgfpathlineto{\pgfqpoint{3.618152in}{2.494683in}}%
\pgfpathlineto{\pgfqpoint{3.618350in}{2.497616in}}%
\pgfpathlineto{\pgfqpoint{3.618547in}{2.493391in}}%
\pgfpathlineto{\pgfqpoint{3.620324in}{2.447406in}}%
\pgfpathlineto{\pgfqpoint{3.620521in}{2.446359in}}%
\pgfpathlineto{\pgfqpoint{3.621607in}{2.431784in}}%
\pgfpathlineto{\pgfqpoint{3.622002in}{2.440493in}}%
\pgfpathlineto{\pgfqpoint{3.622396in}{2.449266in}}%
\pgfpathlineto{\pgfqpoint{3.622989in}{2.438241in}}%
\pgfpathlineto{\pgfqpoint{3.623383in}{2.446480in}}%
\pgfpathlineto{\pgfqpoint{3.623581in}{2.439476in}}%
\pgfpathlineto{\pgfqpoint{3.623877in}{2.422469in}}%
\pgfpathlineto{\pgfqpoint{3.624765in}{2.428371in}}%
\pgfpathlineto{\pgfqpoint{3.625061in}{2.436227in}}%
\pgfpathlineto{\pgfqpoint{3.625752in}{2.426652in}}%
\pgfpathlineto{\pgfqpoint{3.626542in}{2.403355in}}%
\pgfpathlineto{\pgfqpoint{3.626739in}{2.415280in}}%
\pgfpathlineto{\pgfqpoint{3.628516in}{2.687116in}}%
\pgfpathlineto{\pgfqpoint{3.629305in}{2.641871in}}%
\pgfpathlineto{\pgfqpoint{3.631279in}{2.398223in}}%
\pgfpathlineto{\pgfqpoint{3.632562in}{2.459386in}}%
\pgfpathlineto{\pgfqpoint{3.633253in}{2.509457in}}%
\pgfpathlineto{\pgfqpoint{3.633648in}{2.477406in}}%
\pgfpathlineto{\pgfqpoint{3.634931in}{2.417086in}}%
\pgfpathlineto{\pgfqpoint{3.635128in}{2.421415in}}%
\pgfpathlineto{\pgfqpoint{3.636412in}{2.493961in}}%
\pgfpathlineto{\pgfqpoint{3.636905in}{2.469756in}}%
\pgfpathlineto{\pgfqpoint{3.637497in}{2.430382in}}%
\pgfpathlineto{\pgfqpoint{3.638287in}{2.451682in}}%
\pgfpathlineto{\pgfqpoint{3.639570in}{2.478390in}}%
\pgfpathlineto{\pgfqpoint{3.639866in}{2.470567in}}%
\pgfpathlineto{\pgfqpoint{3.640853in}{2.451830in}}%
\pgfpathlineto{\pgfqpoint{3.641050in}{2.458106in}}%
\pgfpathlineto{\pgfqpoint{3.642531in}{2.523719in}}%
\pgfpathlineto{\pgfqpoint{3.642827in}{2.515837in}}%
\pgfpathlineto{\pgfqpoint{3.643321in}{2.521302in}}%
\pgfpathlineto{\pgfqpoint{3.644307in}{2.505134in}}%
\pgfpathlineto{\pgfqpoint{3.644406in}{2.503924in}}%
\pgfpathlineto{\pgfqpoint{3.644604in}{2.510867in}}%
\pgfpathlineto{\pgfqpoint{3.644900in}{2.525058in}}%
\pgfpathlineto{\pgfqpoint{3.645689in}{2.515802in}}%
\pgfpathlineto{\pgfqpoint{3.646676in}{2.500459in}}%
\pgfpathlineto{\pgfqpoint{3.646972in}{2.507029in}}%
\pgfpathlineto{\pgfqpoint{3.648058in}{2.499132in}}%
\pgfpathlineto{\pgfqpoint{3.647762in}{2.509449in}}%
\pgfpathlineto{\pgfqpoint{3.648157in}{2.500541in}}%
\pgfpathlineto{\pgfqpoint{3.648453in}{2.518005in}}%
\pgfpathlineto{\pgfqpoint{3.649144in}{2.500948in}}%
\pgfpathlineto{\pgfqpoint{3.650229in}{2.492859in}}%
\pgfpathlineto{\pgfqpoint{3.649835in}{2.502616in}}%
\pgfpathlineto{\pgfqpoint{3.650427in}{2.496741in}}%
\pgfpathlineto{\pgfqpoint{3.650624in}{2.501078in}}%
\pgfpathlineto{\pgfqpoint{3.651315in}{2.490199in}}%
\pgfpathlineto{\pgfqpoint{3.654375in}{2.430273in}}%
\pgfpathlineto{\pgfqpoint{3.654770in}{2.433041in}}%
\pgfpathlineto{\pgfqpoint{3.655263in}{2.429536in}}%
\pgfpathlineto{\pgfqpoint{3.657138in}{2.387173in}}%
\pgfpathlineto{\pgfqpoint{3.657928in}{2.388835in}}%
\pgfpathlineto{\pgfqpoint{3.658421in}{2.389147in}}%
\pgfpathlineto{\pgfqpoint{3.658718in}{2.384048in}}%
\pgfpathlineto{\pgfqpoint{3.659803in}{2.376816in}}%
\pgfpathlineto{\pgfqpoint{3.659408in}{2.384134in}}%
\pgfpathlineto{\pgfqpoint{3.659902in}{2.376875in}}%
\pgfpathlineto{\pgfqpoint{3.661086in}{2.389688in}}%
\pgfpathlineto{\pgfqpoint{3.660692in}{2.373368in}}%
\pgfpathlineto{\pgfqpoint{3.661284in}{2.384357in}}%
\pgfpathlineto{\pgfqpoint{3.661580in}{2.371142in}}%
\pgfpathlineto{\pgfqpoint{3.662468in}{2.378532in}}%
\pgfpathlineto{\pgfqpoint{3.662567in}{2.378339in}}%
\pgfpathlineto{\pgfqpoint{3.662665in}{2.379807in}}%
\pgfpathlineto{\pgfqpoint{3.665034in}{2.428310in}}%
\pgfpathlineto{\pgfqpoint{3.665133in}{2.428301in}}%
\pgfpathlineto{\pgfqpoint{3.669081in}{2.369690in}}%
\pgfpathlineto{\pgfqpoint{3.669673in}{2.376835in}}%
\pgfpathlineto{\pgfqpoint{3.669969in}{2.382779in}}%
\pgfpathlineto{\pgfqpoint{3.670463in}{2.373771in}}%
\pgfpathlineto{\pgfqpoint{3.670759in}{2.376869in}}%
\pgfpathlineto{\pgfqpoint{3.670956in}{2.378867in}}%
\pgfpathlineto{\pgfqpoint{3.671252in}{2.372960in}}%
\pgfpathlineto{\pgfqpoint{3.672239in}{2.364426in}}%
\pgfpathlineto{\pgfqpoint{3.671844in}{2.378177in}}%
\pgfpathlineto{\pgfqpoint{3.672338in}{2.367317in}}%
\pgfpathlineto{\pgfqpoint{3.672535in}{2.374313in}}%
\pgfpathlineto{\pgfqpoint{3.673029in}{2.350893in}}%
\pgfpathlineto{\pgfqpoint{3.673325in}{2.351732in}}%
\pgfpathlineto{\pgfqpoint{3.674213in}{2.323764in}}%
\pgfpathlineto{\pgfqpoint{3.674904in}{2.274825in}}%
\pgfpathlineto{\pgfqpoint{3.675299in}{2.324890in}}%
\pgfpathlineto{\pgfqpoint{3.677174in}{2.603050in}}%
\pgfpathlineto{\pgfqpoint{3.677470in}{2.583672in}}%
\pgfpathlineto{\pgfqpoint{3.679444in}{2.348911in}}%
\pgfpathlineto{\pgfqpoint{3.679839in}{2.365299in}}%
\pgfpathlineto{\pgfqpoint{3.680629in}{2.390227in}}%
\pgfpathlineto{\pgfqpoint{3.681122in}{2.434308in}}%
\pgfpathlineto{\pgfqpoint{3.681517in}{2.472049in}}%
\pgfpathlineto{\pgfqpoint{3.682010in}{2.418053in}}%
\pgfpathlineto{\pgfqpoint{3.682800in}{2.370533in}}%
\pgfpathlineto{\pgfqpoint{3.683195in}{2.393243in}}%
\pgfpathlineto{\pgfqpoint{3.684873in}{2.452370in}}%
\pgfpathlineto{\pgfqpoint{3.684971in}{2.447827in}}%
\pgfpathlineto{\pgfqpoint{3.685761in}{2.388749in}}%
\pgfpathlineto{\pgfqpoint{3.686353in}{2.408432in}}%
\pgfpathlineto{\pgfqpoint{3.687636in}{2.447586in}}%
\pgfpathlineto{\pgfqpoint{3.687932in}{2.444229in}}%
\pgfpathlineto{\pgfqpoint{3.688031in}{2.444580in}}%
\pgfpathlineto{\pgfqpoint{3.688229in}{2.442900in}}%
\pgfpathlineto{\pgfqpoint{3.689117in}{2.414999in}}%
\pgfpathlineto{\pgfqpoint{3.689413in}{2.437016in}}%
\pgfpathlineto{\pgfqpoint{3.690400in}{2.458441in}}%
\pgfpathlineto{\pgfqpoint{3.690696in}{2.454803in}}%
\pgfpathlineto{\pgfqpoint{3.690795in}{2.454432in}}%
\pgfpathlineto{\pgfqpoint{3.690893in}{2.456194in}}%
\pgfpathlineto{\pgfqpoint{3.691288in}{2.470493in}}%
\pgfpathlineto{\pgfqpoint{3.691979in}{2.456826in}}%
\pgfpathlineto{\pgfqpoint{3.692275in}{2.444163in}}%
\pgfpathlineto{\pgfqpoint{3.692769in}{2.466408in}}%
\pgfpathlineto{\pgfqpoint{3.693657in}{2.478985in}}%
\pgfpathlineto{\pgfqpoint{3.693163in}{2.465732in}}%
\pgfpathlineto{\pgfqpoint{3.694052in}{2.471387in}}%
\pgfpathlineto{\pgfqpoint{3.695335in}{2.444388in}}%
\pgfpathlineto{\pgfqpoint{3.695828in}{2.456092in}}%
\pgfpathlineto{\pgfqpoint{3.696618in}{2.471554in}}%
\pgfpathlineto{\pgfqpoint{3.697013in}{2.463011in}}%
\pgfpathlineto{\pgfqpoint{3.698592in}{2.447944in}}%
\pgfpathlineto{\pgfqpoint{3.697506in}{2.465455in}}%
\pgfpathlineto{\pgfqpoint{3.698691in}{2.448001in}}%
\pgfpathlineto{\pgfqpoint{3.698987in}{2.455925in}}%
\pgfpathlineto{\pgfqpoint{3.699382in}{2.441174in}}%
\pgfpathlineto{\pgfqpoint{3.699875in}{2.449238in}}%
\pgfpathlineto{\pgfqpoint{3.700862in}{2.430779in}}%
\pgfpathlineto{\pgfqpoint{3.701059in}{2.432363in}}%
\pgfpathlineto{\pgfqpoint{3.701257in}{2.429533in}}%
\pgfpathlineto{\pgfqpoint{3.702342in}{2.409221in}}%
\pgfpathlineto{\pgfqpoint{3.702737in}{2.415121in}}%
\pgfpathlineto{\pgfqpoint{3.702935in}{2.415836in}}%
\pgfpathlineto{\pgfqpoint{3.703132in}{2.414915in}}%
\pgfpathlineto{\pgfqpoint{3.703922in}{2.416102in}}%
\pgfpathlineto{\pgfqpoint{3.704810in}{2.403041in}}%
\pgfpathlineto{\pgfqpoint{3.704909in}{2.402906in}}%
\pgfpathlineto{\pgfqpoint{3.705007in}{2.404041in}}%
\pgfpathlineto{\pgfqpoint{3.706093in}{2.415169in}}%
\pgfpathlineto{\pgfqpoint{3.706389in}{2.410548in}}%
\pgfpathlineto{\pgfqpoint{3.706587in}{2.405621in}}%
\pgfpathlineto{\pgfqpoint{3.706981in}{2.414964in}}%
\pgfpathlineto{\pgfqpoint{3.707475in}{2.406515in}}%
\pgfpathlineto{\pgfqpoint{3.707870in}{2.418727in}}%
\pgfpathlineto{\pgfqpoint{3.708659in}{2.408515in}}%
\pgfpathlineto{\pgfqpoint{3.709054in}{2.401695in}}%
\pgfpathlineto{\pgfqpoint{3.709350in}{2.411795in}}%
\pgfpathlineto{\pgfqpoint{3.710041in}{2.421449in}}%
\pgfpathlineto{\pgfqpoint{3.710732in}{2.420963in}}%
\pgfpathlineto{\pgfqpoint{3.711028in}{2.415725in}}%
\pgfpathlineto{\pgfqpoint{3.711225in}{2.419348in}}%
\pgfpathlineto{\pgfqpoint{3.712508in}{2.445270in}}%
\pgfpathlineto{\pgfqpoint{3.711916in}{2.418438in}}%
\pgfpathlineto{\pgfqpoint{3.712805in}{2.444803in}}%
\pgfpathlineto{\pgfqpoint{3.713298in}{2.440310in}}%
\pgfpathlineto{\pgfqpoint{3.713693in}{2.464955in}}%
\pgfpathlineto{\pgfqpoint{3.713792in}{2.469908in}}%
\pgfpathlineto{\pgfqpoint{3.714186in}{2.456241in}}%
\pgfpathlineto{\pgfqpoint{3.714779in}{2.468925in}}%
\pgfpathlineto{\pgfqpoint{3.715173in}{2.461695in}}%
\pgfpathlineto{\pgfqpoint{3.718430in}{2.375134in}}%
\pgfpathlineto{\pgfqpoint{3.718825in}{2.389423in}}%
\pgfpathlineto{\pgfqpoint{3.719516in}{2.378617in}}%
\pgfpathlineto{\pgfqpoint{3.719812in}{2.374017in}}%
\pgfpathlineto{\pgfqpoint{3.720404in}{2.382817in}}%
\pgfpathlineto{\pgfqpoint{3.721095in}{2.376235in}}%
\pgfpathlineto{\pgfqpoint{3.721885in}{2.392879in}}%
\pgfpathlineto{\pgfqpoint{3.722181in}{2.387308in}}%
\pgfpathlineto{\pgfqpoint{3.722576in}{2.395779in}}%
\pgfpathlineto{\pgfqpoint{3.723069in}{2.405230in}}%
\pgfpathlineto{\pgfqpoint{3.723365in}{2.392803in}}%
\pgfpathlineto{\pgfqpoint{3.723661in}{2.376413in}}%
\pgfpathlineto{\pgfqpoint{3.724254in}{2.410003in}}%
\pgfpathlineto{\pgfqpoint{3.725635in}{2.656393in}}%
\pgfpathlineto{\pgfqpoint{3.726326in}{2.638778in}}%
\pgfpathlineto{\pgfqpoint{3.726820in}{2.530286in}}%
\pgfpathlineto{\pgfqpoint{3.728300in}{2.382475in}}%
\pgfpathlineto{\pgfqpoint{3.728399in}{2.380670in}}%
\pgfpathlineto{\pgfqpoint{3.728596in}{2.389508in}}%
\pgfpathlineto{\pgfqpoint{3.730669in}{2.494406in}}%
\pgfpathlineto{\pgfqpoint{3.730866in}{2.482518in}}%
\pgfpathlineto{\pgfqpoint{3.731853in}{2.403773in}}%
\pgfpathlineto{\pgfqpoint{3.732248in}{2.428569in}}%
\pgfpathlineto{\pgfqpoint{3.733926in}{2.469439in}}%
\pgfpathlineto{\pgfqpoint{3.734025in}{2.467952in}}%
\pgfpathlineto{\pgfqpoint{3.734617in}{2.407041in}}%
\pgfpathlineto{\pgfqpoint{3.735505in}{2.439385in}}%
\pgfpathlineto{\pgfqpoint{3.736986in}{2.464495in}}%
\pgfpathlineto{\pgfqpoint{3.737085in}{2.466156in}}%
\pgfpathlineto{\pgfqpoint{3.737381in}{2.454279in}}%
\pgfpathlineto{\pgfqpoint{3.738170in}{2.441512in}}%
\pgfpathlineto{\pgfqpoint{3.738466in}{2.451410in}}%
\pgfpathlineto{\pgfqpoint{3.740144in}{2.485986in}}%
\pgfpathlineto{\pgfqpoint{3.740243in}{2.487238in}}%
\pgfpathlineto{\pgfqpoint{3.740638in}{2.478398in}}%
\pgfpathlineto{\pgfqpoint{3.741230in}{2.462278in}}%
\pgfpathlineto{\pgfqpoint{3.741625in}{2.479686in}}%
\pgfpathlineto{\pgfqpoint{3.742414in}{2.496026in}}%
\pgfpathlineto{\pgfqpoint{3.743006in}{2.491872in}}%
\pgfpathlineto{\pgfqpoint{3.743599in}{2.488346in}}%
\pgfpathlineto{\pgfqpoint{3.743895in}{2.491977in}}%
\pgfpathlineto{\pgfqpoint{3.744487in}{2.498209in}}%
\pgfpathlineto{\pgfqpoint{3.744684in}{2.490258in}}%
\pgfpathlineto{\pgfqpoint{3.744882in}{2.481047in}}%
\pgfpathlineto{\pgfqpoint{3.745375in}{2.505360in}}%
\pgfpathlineto{\pgfqpoint{3.745671in}{2.496915in}}%
\pgfpathlineto{\pgfqpoint{3.746856in}{2.493191in}}%
\pgfpathlineto{\pgfqpoint{3.745967in}{2.498226in}}%
\pgfpathlineto{\pgfqpoint{3.747053in}{2.495243in}}%
\pgfpathlineto{\pgfqpoint{3.747250in}{2.499134in}}%
\pgfpathlineto{\pgfqpoint{3.747547in}{2.483003in}}%
\pgfpathlineto{\pgfqpoint{3.748632in}{2.462850in}}%
\pgfpathlineto{\pgfqpoint{3.748830in}{2.467025in}}%
\pgfpathlineto{\pgfqpoint{3.749027in}{2.462990in}}%
\pgfpathlineto{\pgfqpoint{3.750211in}{2.414229in}}%
\pgfpathlineto{\pgfqpoint{3.750705in}{2.423865in}}%
\pgfpathlineto{\pgfqpoint{3.753074in}{2.477151in}}%
\pgfpathlineto{\pgfqpoint{3.753567in}{2.473528in}}%
\pgfpathlineto{\pgfqpoint{3.753666in}{2.473376in}}%
\pgfpathlineto{\pgfqpoint{3.754159in}{2.448142in}}%
\pgfpathlineto{\pgfqpoint{3.755048in}{2.458929in}}%
\pgfpathlineto{\pgfqpoint{3.755344in}{2.475633in}}%
\pgfpathlineto{\pgfqpoint{3.755837in}{2.448522in}}%
\pgfpathlineto{\pgfqpoint{3.756133in}{2.461519in}}%
\pgfpathlineto{\pgfqpoint{3.756232in}{2.463494in}}%
\pgfpathlineto{\pgfqpoint{3.756726in}{2.455946in}}%
\pgfpathlineto{\pgfqpoint{3.757910in}{2.437968in}}%
\pgfpathlineto{\pgfqpoint{3.757416in}{2.456600in}}%
\pgfpathlineto{\pgfqpoint{3.758305in}{2.441249in}}%
\pgfpathlineto{\pgfqpoint{3.758897in}{2.444904in}}%
\pgfpathlineto{\pgfqpoint{3.759588in}{2.442601in}}%
\pgfpathlineto{\pgfqpoint{3.759884in}{2.440170in}}%
\pgfpathlineto{\pgfqpoint{3.760081in}{2.443280in}}%
\pgfpathlineto{\pgfqpoint{3.761562in}{2.476201in}}%
\pgfpathlineto{\pgfqpoint{3.761759in}{2.471848in}}%
\pgfpathlineto{\pgfqpoint{3.762055in}{2.465389in}}%
\pgfpathlineto{\pgfqpoint{3.762450in}{2.477425in}}%
\pgfpathlineto{\pgfqpoint{3.762746in}{2.489483in}}%
\pgfpathlineto{\pgfqpoint{3.763338in}{2.471213in}}%
\pgfpathlineto{\pgfqpoint{3.763437in}{2.473167in}}%
\pgfpathlineto{\pgfqpoint{3.763536in}{2.474156in}}%
\pgfpathlineto{\pgfqpoint{3.763733in}{2.469834in}}%
\pgfpathlineto{\pgfqpoint{3.765510in}{2.410883in}}%
\pgfpathlineto{\pgfqpoint{3.765707in}{2.416318in}}%
\pgfpathlineto{\pgfqpoint{3.765905in}{2.419257in}}%
\pgfpathlineto{\pgfqpoint{3.766299in}{2.406558in}}%
\pgfpathlineto{\pgfqpoint{3.766595in}{2.412961in}}%
\pgfpathlineto{\pgfqpoint{3.766892in}{2.400539in}}%
\pgfpathlineto{\pgfqpoint{3.767089in}{2.391943in}}%
\pgfpathlineto{\pgfqpoint{3.767484in}{2.409042in}}%
\pgfpathlineto{\pgfqpoint{3.767879in}{2.399743in}}%
\pgfpathlineto{\pgfqpoint{3.768273in}{2.416362in}}%
\pgfpathlineto{\pgfqpoint{3.768767in}{2.393843in}}%
\pgfpathlineto{\pgfqpoint{3.768964in}{2.396735in}}%
\pgfpathlineto{\pgfqpoint{3.770050in}{2.406638in}}%
\pgfpathlineto{\pgfqpoint{3.769556in}{2.389964in}}%
\pgfpathlineto{\pgfqpoint{3.770247in}{2.404122in}}%
\pgfpathlineto{\pgfqpoint{3.771136in}{2.401830in}}%
\pgfpathlineto{\pgfqpoint{3.770741in}{2.408950in}}%
\pgfpathlineto{\pgfqpoint{3.771234in}{2.403568in}}%
\pgfpathlineto{\pgfqpoint{3.771530in}{2.410383in}}%
\pgfpathlineto{\pgfqpoint{3.771827in}{2.393191in}}%
\pgfpathlineto{\pgfqpoint{3.772221in}{2.370008in}}%
\pgfpathlineto{\pgfqpoint{3.772616in}{2.406953in}}%
\pgfpathlineto{\pgfqpoint{3.774393in}{2.632550in}}%
\pgfpathlineto{\pgfqpoint{3.774788in}{2.621883in}}%
\pgfpathlineto{\pgfqpoint{3.775774in}{2.378835in}}%
\pgfpathlineto{\pgfqpoint{3.777551in}{2.388814in}}%
\pgfpathlineto{\pgfqpoint{3.778538in}{2.462883in}}%
\pgfpathlineto{\pgfqpoint{3.778933in}{2.491079in}}%
\pgfpathlineto{\pgfqpoint{3.779426in}{2.449202in}}%
\pgfpathlineto{\pgfqpoint{3.780019in}{2.403175in}}%
\pgfpathlineto{\pgfqpoint{3.780611in}{2.431136in}}%
\pgfpathlineto{\pgfqpoint{3.781400in}{2.456258in}}%
\pgfpathlineto{\pgfqpoint{3.781993in}{2.475895in}}%
\pgfpathlineto{\pgfqpoint{3.782585in}{2.464626in}}%
\pgfpathlineto{\pgfqpoint{3.783177in}{2.415421in}}%
\pgfpathlineto{\pgfqpoint{3.783868in}{2.450453in}}%
\pgfpathlineto{\pgfqpoint{3.785250in}{2.494491in}}%
\pgfpathlineto{\pgfqpoint{3.785743in}{2.476590in}}%
\pgfpathlineto{\pgfqpoint{3.786039in}{2.477086in}}%
\pgfpathlineto{\pgfqpoint{3.786237in}{2.471665in}}%
\pgfpathlineto{\pgfqpoint{3.786434in}{2.467595in}}%
\pgfpathlineto{\pgfqpoint{3.786829in}{2.484369in}}%
\pgfpathlineto{\pgfqpoint{3.788408in}{2.508065in}}%
\pgfpathlineto{\pgfqpoint{3.788704in}{2.502571in}}%
\pgfpathlineto{\pgfqpoint{3.789691in}{2.492830in}}%
\pgfpathlineto{\pgfqpoint{3.789888in}{2.495843in}}%
\pgfpathlineto{\pgfqpoint{3.790382in}{2.511204in}}%
\pgfpathlineto{\pgfqpoint{3.791270in}{2.508192in}}%
\pgfpathlineto{\pgfqpoint{3.792652in}{2.489710in}}%
\pgfpathlineto{\pgfqpoint{3.791862in}{2.509836in}}%
\pgfpathlineto{\pgfqpoint{3.793540in}{2.494205in}}%
\pgfpathlineto{\pgfqpoint{3.793935in}{2.505923in}}%
\pgfpathlineto{\pgfqpoint{3.794823in}{2.501750in}}%
\pgfpathlineto{\pgfqpoint{3.795810in}{2.490319in}}%
\pgfpathlineto{\pgfqpoint{3.796008in}{2.496103in}}%
\pgfpathlineto{\pgfqpoint{3.796797in}{2.503258in}}%
\pgfpathlineto{\pgfqpoint{3.797093in}{2.496619in}}%
\pgfpathlineto{\pgfqpoint{3.798278in}{2.483562in}}%
\pgfpathlineto{\pgfqpoint{3.798475in}{2.484374in}}%
\pgfpathlineto{\pgfqpoint{3.798870in}{2.485186in}}%
\pgfpathlineto{\pgfqpoint{3.799067in}{2.484431in}}%
\pgfpathlineto{\pgfqpoint{3.801041in}{2.456551in}}%
\pgfpathlineto{\pgfqpoint{3.801338in}{2.459775in}}%
\pgfpathlineto{\pgfqpoint{3.801436in}{2.460618in}}%
\pgfpathlineto{\pgfqpoint{3.801732in}{2.454500in}}%
\pgfpathlineto{\pgfqpoint{3.801930in}{2.448962in}}%
\pgfpathlineto{\pgfqpoint{3.802423in}{2.463793in}}%
\pgfpathlineto{\pgfqpoint{3.802522in}{2.464932in}}%
\pgfpathlineto{\pgfqpoint{3.802818in}{2.457148in}}%
\pgfpathlineto{\pgfqpoint{3.803015in}{2.454577in}}%
\pgfpathlineto{\pgfqpoint{3.803410in}{2.465894in}}%
\pgfpathlineto{\pgfqpoint{3.804397in}{2.470867in}}%
\pgfpathlineto{\pgfqpoint{3.803904in}{2.458604in}}%
\pgfpathlineto{\pgfqpoint{3.804496in}{2.468583in}}%
\pgfpathlineto{\pgfqpoint{3.805483in}{2.455999in}}%
\pgfpathlineto{\pgfqpoint{3.805680in}{2.460065in}}%
\pgfpathlineto{\pgfqpoint{3.806865in}{2.471698in}}%
\pgfpathlineto{\pgfqpoint{3.806272in}{2.451988in}}%
\pgfpathlineto{\pgfqpoint{3.806963in}{2.468846in}}%
\pgfpathlineto{\pgfqpoint{3.807259in}{2.460994in}}%
\pgfpathlineto{\pgfqpoint{3.807950in}{2.473413in}}%
\pgfpathlineto{\pgfqpoint{3.809924in}{2.517222in}}%
\pgfpathlineto{\pgfqpoint{3.810122in}{2.512215in}}%
\pgfpathlineto{\pgfqpoint{3.811306in}{2.498833in}}%
\pgfpathlineto{\pgfqpoint{3.810813in}{2.513725in}}%
\pgfpathlineto{\pgfqpoint{3.811405in}{2.498835in}}%
\pgfpathlineto{\pgfqpoint{3.811602in}{2.500336in}}%
\pgfpathlineto{\pgfqpoint{3.811898in}{2.493975in}}%
\pgfpathlineto{\pgfqpoint{3.812194in}{2.494122in}}%
\pgfpathlineto{\pgfqpoint{3.812490in}{2.479538in}}%
\pgfpathlineto{\pgfqpoint{3.813477in}{2.448588in}}%
\pgfpathlineto{\pgfqpoint{3.813872in}{2.454658in}}%
\pgfpathlineto{\pgfqpoint{3.814168in}{2.451610in}}%
\pgfpathlineto{\pgfqpoint{3.814464in}{2.458932in}}%
\pgfpathlineto{\pgfqpoint{3.814563in}{2.461236in}}%
\pgfpathlineto{\pgfqpoint{3.814958in}{2.445103in}}%
\pgfpathlineto{\pgfqpoint{3.815155in}{2.440682in}}%
\pgfpathlineto{\pgfqpoint{3.815748in}{2.455611in}}%
\pgfpathlineto{\pgfqpoint{3.816044in}{2.462033in}}%
\pgfpathlineto{\pgfqpoint{3.816833in}{2.454870in}}%
\pgfpathlineto{\pgfqpoint{3.817425in}{2.445607in}}%
\pgfpathlineto{\pgfqpoint{3.817820in}{2.455702in}}%
\pgfpathlineto{\pgfqpoint{3.818215in}{2.448916in}}%
\pgfpathlineto{\pgfqpoint{3.819301in}{2.462544in}}%
\pgfpathlineto{\pgfqpoint{3.819498in}{2.455006in}}%
\pgfpathlineto{\pgfqpoint{3.820288in}{2.431477in}}%
\pgfpathlineto{\pgfqpoint{3.820584in}{2.454049in}}%
\pgfpathlineto{\pgfqpoint{3.822064in}{2.669832in}}%
\pgfpathlineto{\pgfqpoint{3.822656in}{2.647777in}}%
\pgfpathlineto{\pgfqpoint{3.822854in}{2.638213in}}%
\pgfpathlineto{\pgfqpoint{3.825025in}{2.406004in}}%
\pgfpathlineto{\pgfqpoint{3.825223in}{2.409934in}}%
\pgfpathlineto{\pgfqpoint{3.826802in}{2.475284in}}%
\pgfpathlineto{\pgfqpoint{3.827098in}{2.458018in}}%
\pgfpathlineto{\pgfqpoint{3.827986in}{2.359274in}}%
\pgfpathlineto{\pgfqpoint{3.828578in}{2.394787in}}%
\pgfpathlineto{\pgfqpoint{3.830256in}{2.526077in}}%
\pgfpathlineto{\pgfqpoint{3.830947in}{2.476183in}}%
\pgfpathlineto{\pgfqpoint{3.831046in}{2.475761in}}%
\pgfpathlineto{\pgfqpoint{3.831145in}{2.478819in}}%
\pgfpathlineto{\pgfqpoint{3.833020in}{2.509852in}}%
\pgfpathlineto{\pgfqpoint{3.833415in}{2.519974in}}%
\pgfpathlineto{\pgfqpoint{3.833809in}{2.503737in}}%
\pgfpathlineto{\pgfqpoint{3.834303in}{2.486301in}}%
\pgfpathlineto{\pgfqpoint{3.834796in}{2.500359in}}%
\pgfpathlineto{\pgfqpoint{3.835586in}{2.519971in}}%
\pgfpathlineto{\pgfqpoint{3.835981in}{2.506577in}}%
\pgfpathlineto{\pgfqpoint{3.836178in}{2.499248in}}%
\pgfpathlineto{\pgfqpoint{3.836672in}{2.518604in}}%
\pgfpathlineto{\pgfqpoint{3.837165in}{2.499485in}}%
\pgfpathlineto{\pgfqpoint{3.837757in}{2.497056in}}%
\pgfpathlineto{\pgfqpoint{3.837955in}{2.500879in}}%
\pgfpathlineto{\pgfqpoint{3.839435in}{2.519878in}}%
\pgfpathlineto{\pgfqpoint{3.839633in}{2.523111in}}%
\pgfpathlineto{\pgfqpoint{3.840028in}{2.512865in}}%
\pgfpathlineto{\pgfqpoint{3.841212in}{2.504309in}}%
\pgfpathlineto{\pgfqpoint{3.840718in}{2.514453in}}%
\pgfpathlineto{\pgfqpoint{3.841311in}{2.507065in}}%
\pgfpathlineto{\pgfqpoint{3.841705in}{2.524119in}}%
\pgfpathlineto{\pgfqpoint{3.842396in}{2.512522in}}%
\pgfpathlineto{\pgfqpoint{3.842791in}{2.488081in}}%
\pgfpathlineto{\pgfqpoint{3.843679in}{2.494444in}}%
\pgfpathlineto{\pgfqpoint{3.843975in}{2.502956in}}%
\pgfpathlineto{\pgfqpoint{3.844469in}{2.484443in}}%
\pgfpathlineto{\pgfqpoint{3.844864in}{2.489289in}}%
\pgfpathlineto{\pgfqpoint{3.845160in}{2.482463in}}%
\pgfpathlineto{\pgfqpoint{3.847134in}{2.449828in}}%
\pgfpathlineto{\pgfqpoint{3.845653in}{2.489164in}}%
\pgfpathlineto{\pgfqpoint{3.847331in}{2.451313in}}%
\pgfpathlineto{\pgfqpoint{3.847529in}{2.454683in}}%
\pgfpathlineto{\pgfqpoint{3.848121in}{2.446156in}}%
\pgfpathlineto{\pgfqpoint{3.848516in}{2.438407in}}%
\pgfpathlineto{\pgfqpoint{3.849404in}{2.442012in}}%
\pgfpathlineto{\pgfqpoint{3.850095in}{2.444162in}}%
\pgfpathlineto{\pgfqpoint{3.850193in}{2.441552in}}%
\pgfpathlineto{\pgfqpoint{3.850490in}{2.431739in}}%
\pgfpathlineto{\pgfqpoint{3.850983in}{2.448626in}}%
\pgfpathlineto{\pgfqpoint{3.851279in}{2.442518in}}%
\pgfpathlineto{\pgfqpoint{3.852464in}{2.430183in}}%
\pgfpathlineto{\pgfqpoint{3.852562in}{2.432832in}}%
\pgfpathlineto{\pgfqpoint{3.852858in}{2.442772in}}%
\pgfpathlineto{\pgfqpoint{3.853154in}{2.430805in}}%
\pgfpathlineto{\pgfqpoint{3.853747in}{2.436694in}}%
\pgfpathlineto{\pgfqpoint{3.854536in}{2.421206in}}%
\pgfpathlineto{\pgfqpoint{3.855030in}{2.429534in}}%
\pgfpathlineto{\pgfqpoint{3.855819in}{2.434662in}}%
\pgfpathlineto{\pgfqpoint{3.857497in}{2.466843in}}%
\pgfpathlineto{\pgfqpoint{3.859471in}{2.446716in}}%
\pgfpathlineto{\pgfqpoint{3.859669in}{2.448733in}}%
\pgfpathlineto{\pgfqpoint{3.859767in}{2.449107in}}%
\pgfpathlineto{\pgfqpoint{3.859866in}{2.447126in}}%
\pgfpathlineto{\pgfqpoint{3.861346in}{2.402115in}}%
\pgfpathlineto{\pgfqpoint{3.861544in}{2.409201in}}%
\pgfpathlineto{\pgfqpoint{3.861741in}{2.416266in}}%
\pgfpathlineto{\pgfqpoint{3.862333in}{2.399558in}}%
\pgfpathlineto{\pgfqpoint{3.862531in}{2.394231in}}%
\pgfpathlineto{\pgfqpoint{3.862926in}{2.412560in}}%
\pgfpathlineto{\pgfqpoint{3.864110in}{2.426162in}}%
\pgfpathlineto{\pgfqpoint{3.864209in}{2.425524in}}%
\pgfpathlineto{\pgfqpoint{3.864604in}{2.417627in}}%
\pgfpathlineto{\pgfqpoint{3.865097in}{2.427226in}}%
\pgfpathlineto{\pgfqpoint{3.865196in}{2.428217in}}%
\pgfpathlineto{\pgfqpoint{3.865492in}{2.423452in}}%
\pgfpathlineto{\pgfqpoint{3.865689in}{2.419374in}}%
\pgfpathlineto{\pgfqpoint{3.865985in}{2.434520in}}%
\pgfpathlineto{\pgfqpoint{3.866183in}{2.443534in}}%
\pgfpathlineto{\pgfqpoint{3.866676in}{2.422233in}}%
\pgfpathlineto{\pgfqpoint{3.867071in}{2.434529in}}%
\pgfpathlineto{\pgfqpoint{3.867861in}{2.418258in}}%
\pgfpathlineto{\pgfqpoint{3.868157in}{2.435018in}}%
\pgfpathlineto{\pgfqpoint{3.870131in}{2.660182in}}%
\pgfpathlineto{\pgfqpoint{3.870229in}{2.657640in}}%
\pgfpathlineto{\pgfqpoint{3.870822in}{2.523323in}}%
\pgfpathlineto{\pgfqpoint{3.872302in}{2.398484in}}%
\pgfpathlineto{\pgfqpoint{3.872796in}{2.414847in}}%
\pgfpathlineto{\pgfqpoint{3.874473in}{2.526325in}}%
\pgfpathlineto{\pgfqpoint{3.874967in}{2.495732in}}%
\pgfpathlineto{\pgfqpoint{3.875460in}{2.449201in}}%
\pgfpathlineto{\pgfqpoint{3.876250in}{2.477200in}}%
\pgfpathlineto{\pgfqpoint{3.877336in}{2.524725in}}%
\pgfpathlineto{\pgfqpoint{3.877632in}{2.542286in}}%
\pgfpathlineto{\pgfqpoint{3.878224in}{2.505776in}}%
\pgfpathlineto{\pgfqpoint{3.878915in}{2.474818in}}%
\pgfpathlineto{\pgfqpoint{3.879310in}{2.495572in}}%
\pgfpathlineto{\pgfqpoint{3.880395in}{2.518269in}}%
\pgfpathlineto{\pgfqpoint{3.880593in}{2.514038in}}%
\pgfpathlineto{\pgfqpoint{3.880691in}{2.512328in}}%
\pgfpathlineto{\pgfqpoint{3.880988in}{2.522903in}}%
\pgfpathlineto{\pgfqpoint{3.881086in}{2.525996in}}%
\pgfpathlineto{\pgfqpoint{3.881481in}{2.506669in}}%
\pgfpathlineto{\pgfqpoint{3.881678in}{2.502647in}}%
\pgfpathlineto{\pgfqpoint{3.882271in}{2.508926in}}%
\pgfpathlineto{\pgfqpoint{3.884245in}{2.566028in}}%
\pgfpathlineto{\pgfqpoint{3.884343in}{2.566071in}}%
\pgfpathlineto{\pgfqpoint{3.884837in}{2.550573in}}%
\pgfpathlineto{\pgfqpoint{3.885429in}{2.565719in}}%
\pgfpathlineto{\pgfqpoint{3.886515in}{2.576716in}}%
\pgfpathlineto{\pgfqpoint{3.886712in}{2.572233in}}%
\pgfpathlineto{\pgfqpoint{3.886910in}{2.569205in}}%
\pgfpathlineto{\pgfqpoint{3.887699in}{2.573362in}}%
\pgfpathlineto{\pgfqpoint{3.887896in}{2.574157in}}%
\pgfpathlineto{\pgfqpoint{3.888193in}{2.570330in}}%
\pgfpathlineto{\pgfqpoint{3.888686in}{2.555101in}}%
\pgfpathlineto{\pgfqpoint{3.889081in}{2.572631in}}%
\pgfpathlineto{\pgfqpoint{3.889574in}{2.559580in}}%
\pgfpathlineto{\pgfqpoint{3.889870in}{2.565500in}}%
\pgfpathlineto{\pgfqpoint{3.890265in}{2.550956in}}%
\pgfpathlineto{\pgfqpoint{3.890364in}{2.551036in}}%
\pgfpathlineto{\pgfqpoint{3.891154in}{2.562379in}}%
\pgfpathlineto{\pgfqpoint{3.891450in}{2.555335in}}%
\pgfpathlineto{\pgfqpoint{3.891647in}{2.549423in}}%
\pgfpathlineto{\pgfqpoint{3.892141in}{2.563296in}}%
\pgfpathlineto{\pgfqpoint{3.892437in}{2.557136in}}%
\pgfpathlineto{\pgfqpoint{3.892733in}{2.568285in}}%
\pgfpathlineto{\pgfqpoint{3.892930in}{2.578399in}}%
\pgfpathlineto{\pgfqpoint{3.893621in}{2.554792in}}%
\pgfpathlineto{\pgfqpoint{3.893917in}{2.549562in}}%
\pgfpathlineto{\pgfqpoint{3.894509in}{2.557269in}}%
\pgfpathlineto{\pgfqpoint{3.894608in}{2.557235in}}%
\pgfpathlineto{\pgfqpoint{3.894904in}{2.558031in}}%
\pgfpathlineto{\pgfqpoint{3.895102in}{2.556891in}}%
\pgfpathlineto{\pgfqpoint{3.896187in}{2.551840in}}%
\pgfpathlineto{\pgfqpoint{3.895694in}{2.559595in}}%
\pgfpathlineto{\pgfqpoint{3.896483in}{2.553195in}}%
\pgfpathlineto{\pgfqpoint{3.897174in}{2.561764in}}%
\pgfpathlineto{\pgfqpoint{3.897470in}{2.552839in}}%
\pgfpathlineto{\pgfqpoint{3.897668in}{2.547138in}}%
\pgfpathlineto{\pgfqpoint{3.898161in}{2.554962in}}%
\pgfpathlineto{\pgfqpoint{3.898556in}{2.551710in}}%
\pgfpathlineto{\pgfqpoint{3.899740in}{2.539720in}}%
\pgfpathlineto{\pgfqpoint{3.899247in}{2.553668in}}%
\pgfpathlineto{\pgfqpoint{3.900036in}{2.546891in}}%
\pgfpathlineto{\pgfqpoint{3.900727in}{2.552481in}}%
\pgfpathlineto{\pgfqpoint{3.900925in}{2.548195in}}%
\pgfpathlineto{\pgfqpoint{3.901714in}{2.532363in}}%
\pgfpathlineto{\pgfqpoint{3.902208in}{2.539163in}}%
\pgfpathlineto{\pgfqpoint{3.903195in}{2.546649in}}%
\pgfpathlineto{\pgfqpoint{3.903688in}{2.546437in}}%
\pgfpathlineto{\pgfqpoint{3.903886in}{2.548076in}}%
\pgfpathlineto{\pgfqpoint{3.904182in}{2.539934in}}%
\pgfpathlineto{\pgfqpoint{3.906649in}{2.498324in}}%
\pgfpathlineto{\pgfqpoint{3.906748in}{2.499351in}}%
\pgfpathlineto{\pgfqpoint{3.909018in}{2.550750in}}%
\pgfpathlineto{\pgfqpoint{3.909215in}{2.549412in}}%
\pgfpathlineto{\pgfqpoint{3.909709in}{2.537754in}}%
\pgfpathlineto{\pgfqpoint{3.910893in}{2.499345in}}%
\pgfpathlineto{\pgfqpoint{3.911189in}{2.507057in}}%
\pgfpathlineto{\pgfqpoint{3.911288in}{2.508684in}}%
\pgfpathlineto{\pgfqpoint{3.911584in}{2.497208in}}%
\pgfpathlineto{\pgfqpoint{3.911782in}{2.490960in}}%
\pgfpathlineto{\pgfqpoint{3.912769in}{2.492601in}}%
\pgfpathlineto{\pgfqpoint{3.913953in}{2.471992in}}%
\pgfpathlineto{\pgfqpoint{3.914150in}{2.480990in}}%
\pgfpathlineto{\pgfqpoint{3.914447in}{2.492662in}}%
\pgfpathlineto{\pgfqpoint{3.915039in}{2.469119in}}%
\pgfpathlineto{\pgfqpoint{3.915236in}{2.467150in}}%
\pgfpathlineto{\pgfqpoint{3.915434in}{2.476690in}}%
\pgfpathlineto{\pgfqpoint{3.917013in}{2.735188in}}%
\pgfpathlineto{\pgfqpoint{3.918000in}{2.681076in}}%
\pgfpathlineto{\pgfqpoint{3.919875in}{2.471957in}}%
\pgfpathlineto{\pgfqpoint{3.919974in}{2.472714in}}%
\pgfpathlineto{\pgfqpoint{3.921454in}{2.514392in}}%
\pgfpathlineto{\pgfqpoint{3.922046in}{2.547440in}}%
\pgfpathlineto{\pgfqpoint{3.922342in}{2.516112in}}%
\pgfpathlineto{\pgfqpoint{3.923132in}{2.453736in}}%
\pgfpathlineto{\pgfqpoint{3.923626in}{2.473970in}}%
\pgfpathlineto{\pgfqpoint{3.925205in}{2.534141in}}%
\pgfpathlineto{\pgfqpoint{3.925599in}{2.511691in}}%
\pgfpathlineto{\pgfqpoint{3.926093in}{2.481980in}}%
\pgfpathlineto{\pgfqpoint{3.926685in}{2.507464in}}%
\pgfpathlineto{\pgfqpoint{3.928363in}{2.549184in}}%
\pgfpathlineto{\pgfqpoint{3.929745in}{2.502827in}}%
\pgfpathlineto{\pgfqpoint{3.930140in}{2.523068in}}%
\pgfpathlineto{\pgfqpoint{3.930337in}{2.537043in}}%
\pgfpathlineto{\pgfqpoint{3.930831in}{2.517637in}}%
\pgfpathlineto{\pgfqpoint{3.931225in}{2.521030in}}%
\pgfpathlineto{\pgfqpoint{3.931916in}{2.516199in}}%
\pgfpathlineto{\pgfqpoint{3.932311in}{2.526600in}}%
\pgfpathlineto{\pgfqpoint{3.932805in}{2.489510in}}%
\pgfpathlineto{\pgfqpoint{3.933693in}{2.512297in}}%
\pgfpathlineto{\pgfqpoint{3.935173in}{2.498236in}}%
\pgfpathlineto{\pgfqpoint{3.935371in}{2.499772in}}%
\pgfpathlineto{\pgfqpoint{3.936456in}{2.516083in}}%
\pgfpathlineto{\pgfqpoint{3.935963in}{2.497695in}}%
\pgfpathlineto{\pgfqpoint{3.936851in}{2.507574in}}%
\pgfpathlineto{\pgfqpoint{3.937443in}{2.502412in}}%
\pgfpathlineto{\pgfqpoint{3.937838in}{2.506875in}}%
\pgfpathlineto{\pgfqpoint{3.938036in}{2.510016in}}%
\pgfpathlineto{\pgfqpoint{3.938529in}{2.503136in}}%
\pgfpathlineto{\pgfqpoint{3.938726in}{2.504577in}}%
\pgfpathlineto{\pgfqpoint{3.939615in}{2.498854in}}%
\pgfpathlineto{\pgfqpoint{3.939812in}{2.496456in}}%
\pgfpathlineto{\pgfqpoint{3.940108in}{2.508397in}}%
\pgfpathlineto{\pgfqpoint{3.940306in}{2.512977in}}%
\pgfpathlineto{\pgfqpoint{3.940700in}{2.493697in}}%
\pgfpathlineto{\pgfqpoint{3.941687in}{2.479321in}}%
\pgfpathlineto{\pgfqpoint{3.941194in}{2.499119in}}%
\pgfpathlineto{\pgfqpoint{3.941885in}{2.487118in}}%
\pgfpathlineto{\pgfqpoint{3.941984in}{2.488968in}}%
\pgfpathlineto{\pgfqpoint{3.942181in}{2.480112in}}%
\pgfpathlineto{\pgfqpoint{3.943069in}{2.463877in}}%
\pgfpathlineto{\pgfqpoint{3.943365in}{2.470824in}}%
\pgfpathlineto{\pgfqpoint{3.945438in}{2.450628in}}%
\pgfpathlineto{\pgfqpoint{3.945537in}{2.454083in}}%
\pgfpathlineto{\pgfqpoint{3.946326in}{2.466532in}}%
\pgfpathlineto{\pgfqpoint{3.946622in}{2.457276in}}%
\pgfpathlineto{\pgfqpoint{3.947511in}{2.443462in}}%
\pgfpathlineto{\pgfqpoint{3.947116in}{2.462283in}}%
\pgfpathlineto{\pgfqpoint{3.947708in}{2.450483in}}%
\pgfpathlineto{\pgfqpoint{3.948892in}{2.463276in}}%
\pgfpathlineto{\pgfqpoint{3.948399in}{2.446867in}}%
\pgfpathlineto{\pgfqpoint{3.948991in}{2.462041in}}%
\pgfpathlineto{\pgfqpoint{3.949189in}{2.458999in}}%
\pgfpathlineto{\pgfqpoint{3.949682in}{2.469268in}}%
\pgfpathlineto{\pgfqpoint{3.949978in}{2.477282in}}%
\pgfpathlineto{\pgfqpoint{3.951163in}{2.497075in}}%
\pgfpathlineto{\pgfqpoint{3.951360in}{2.495255in}}%
\pgfpathlineto{\pgfqpoint{3.951557in}{2.492389in}}%
\pgfpathlineto{\pgfqpoint{3.951952in}{2.504470in}}%
\pgfpathlineto{\pgfqpoint{3.952347in}{2.519415in}}%
\pgfpathlineto{\pgfqpoint{3.953136in}{2.509687in}}%
\pgfpathlineto{\pgfqpoint{3.955505in}{2.458655in}}%
\pgfpathlineto{\pgfqpoint{3.955703in}{2.449615in}}%
\pgfpathlineto{\pgfqpoint{3.956591in}{2.455135in}}%
\pgfpathlineto{\pgfqpoint{3.958368in}{2.476155in}}%
\pgfpathlineto{\pgfqpoint{3.958565in}{2.473669in}}%
\pgfpathlineto{\pgfqpoint{3.959058in}{2.457391in}}%
\pgfpathlineto{\pgfqpoint{3.959552in}{2.475836in}}%
\pgfpathlineto{\pgfqpoint{3.959651in}{2.475887in}}%
\pgfpathlineto{\pgfqpoint{3.960342in}{2.457646in}}%
\pgfpathlineto{\pgfqpoint{3.961427in}{2.464319in}}%
\pgfpathlineto{\pgfqpoint{3.961625in}{2.470019in}}%
\pgfpathlineto{\pgfqpoint{3.962118in}{2.451754in}}%
\pgfpathlineto{\pgfqpoint{3.962414in}{2.438248in}}%
\pgfpathlineto{\pgfqpoint{3.962809in}{2.471692in}}%
\pgfpathlineto{\pgfqpoint{3.964289in}{2.684176in}}%
\pgfpathlineto{\pgfqpoint{3.965079in}{2.672776in}}%
\pgfpathlineto{\pgfqpoint{3.967250in}{2.440392in}}%
\pgfpathlineto{\pgfqpoint{3.967941in}{2.452027in}}%
\pgfpathlineto{\pgfqpoint{3.968040in}{2.450546in}}%
\pgfpathlineto{\pgfqpoint{3.968336in}{2.461835in}}%
\pgfpathlineto{\pgfqpoint{3.969224in}{2.520185in}}%
\pgfpathlineto{\pgfqpoint{3.969619in}{2.480778in}}%
\pgfpathlineto{\pgfqpoint{3.970508in}{2.417800in}}%
\pgfpathlineto{\pgfqpoint{3.970902in}{2.436840in}}%
\pgfpathlineto{\pgfqpoint{3.972383in}{2.473526in}}%
\pgfpathlineto{\pgfqpoint{3.972580in}{2.466401in}}%
\pgfpathlineto{\pgfqpoint{3.973468in}{2.409952in}}%
\pgfpathlineto{\pgfqpoint{3.973962in}{2.437990in}}%
\pgfpathlineto{\pgfqpoint{3.975640in}{2.460999in}}%
\pgfpathlineto{\pgfqpoint{3.975936in}{2.448957in}}%
\pgfpathlineto{\pgfqpoint{3.976133in}{2.440760in}}%
\pgfpathlineto{\pgfqpoint{3.976528in}{2.449694in}}%
\pgfpathlineto{\pgfqpoint{3.977022in}{2.447033in}}%
\pgfpathlineto{\pgfqpoint{3.977713in}{2.471494in}}%
\pgfpathlineto{\pgfqpoint{3.978305in}{2.457362in}}%
\pgfpathlineto{\pgfqpoint{3.978798in}{2.473330in}}%
\pgfpathlineto{\pgfqpoint{3.979489in}{2.460954in}}%
\pgfpathlineto{\pgfqpoint{3.979884in}{2.441238in}}%
\pgfpathlineto{\pgfqpoint{3.980377in}{2.461324in}}%
\pgfpathlineto{\pgfqpoint{3.980575in}{2.459794in}}%
\pgfpathlineto{\pgfqpoint{3.981167in}{2.485043in}}%
\pgfpathlineto{\pgfqpoint{3.982351in}{2.475209in}}%
\pgfpathlineto{\pgfqpoint{3.982845in}{2.456509in}}%
\pgfpathlineto{\pgfqpoint{3.983733in}{2.463933in}}%
\pgfpathlineto{\pgfqpoint{3.984029in}{2.466471in}}%
\pgfpathlineto{\pgfqpoint{3.984227in}{2.460875in}}%
\pgfpathlineto{\pgfqpoint{3.986497in}{2.385844in}}%
\pgfpathlineto{\pgfqpoint{3.986595in}{2.388541in}}%
\pgfpathlineto{\pgfqpoint{3.988471in}{2.456541in}}%
\pgfpathlineto{\pgfqpoint{3.989162in}{2.448113in}}%
\pgfpathlineto{\pgfqpoint{3.989260in}{2.450490in}}%
\pgfpathlineto{\pgfqpoint{3.990050in}{2.443307in}}%
\pgfpathlineto{\pgfqpoint{3.991629in}{2.402960in}}%
\pgfpathlineto{\pgfqpoint{3.992024in}{2.415956in}}%
\pgfpathlineto{\pgfqpoint{3.992813in}{2.419753in}}%
\pgfpathlineto{\pgfqpoint{3.992517in}{2.412257in}}%
\pgfpathlineto{\pgfqpoint{3.993011in}{2.416468in}}%
\pgfpathlineto{\pgfqpoint{3.993208in}{2.411095in}}%
\pgfpathlineto{\pgfqpoint{3.993603in}{2.425259in}}%
\pgfpathlineto{\pgfqpoint{3.993998in}{2.413310in}}%
\pgfpathlineto{\pgfqpoint{3.994294in}{2.425441in}}%
\pgfpathlineto{\pgfqpoint{3.994689in}{2.406486in}}%
\pgfpathlineto{\pgfqpoint{3.995084in}{2.414984in}}%
\pgfpathlineto{\pgfqpoint{3.995281in}{2.414013in}}%
\pgfpathlineto{\pgfqpoint{3.995478in}{2.418344in}}%
\pgfpathlineto{\pgfqpoint{3.995774in}{2.426817in}}%
\pgfpathlineto{\pgfqpoint{3.996564in}{2.419628in}}%
\pgfpathlineto{\pgfqpoint{3.996959in}{2.429891in}}%
\pgfpathlineto{\pgfqpoint{3.998735in}{2.470129in}}%
\pgfpathlineto{\pgfqpoint{4.001104in}{2.508367in}}%
\pgfpathlineto{\pgfqpoint{4.001302in}{2.504882in}}%
\pgfpathlineto{\pgfqpoint{4.002881in}{2.476318in}}%
\pgfpathlineto{\pgfqpoint{4.002979in}{2.476948in}}%
\pgfpathlineto{\pgfqpoint{4.003670in}{2.493349in}}%
\pgfpathlineto{\pgfqpoint{4.004065in}{2.480743in}}%
\pgfpathlineto{\pgfqpoint{4.004164in}{2.479446in}}%
\pgfpathlineto{\pgfqpoint{4.004460in}{2.487547in}}%
\pgfpathlineto{\pgfqpoint{4.005052in}{2.502839in}}%
\pgfpathlineto{\pgfqpoint{4.005447in}{2.486567in}}%
\pgfpathlineto{\pgfqpoint{4.005743in}{2.493746in}}%
\pgfpathlineto{\pgfqpoint{4.006039in}{2.485709in}}%
\pgfpathlineto{\pgfqpoint{4.009395in}{2.378743in}}%
\pgfpathlineto{\pgfqpoint{4.009790in}{2.400940in}}%
\pgfpathlineto{\pgfqpoint{4.011566in}{2.575519in}}%
\pgfpathlineto{\pgfqpoint{4.012060in}{2.547721in}}%
\pgfpathlineto{\pgfqpoint{4.013836in}{2.296504in}}%
\pgfpathlineto{\pgfqpoint{4.014922in}{2.298265in}}%
\pgfpathlineto{\pgfqpoint{4.015317in}{2.314796in}}%
\pgfpathlineto{\pgfqpoint{4.016205in}{2.381425in}}%
\pgfpathlineto{\pgfqpoint{4.016600in}{2.342809in}}%
\pgfpathlineto{\pgfqpoint{4.017291in}{2.288647in}}%
\pgfpathlineto{\pgfqpoint{4.017883in}{2.317853in}}%
\pgfpathlineto{\pgfqpoint{4.018080in}{2.314937in}}%
\pgfpathlineto{\pgfqpoint{4.018377in}{2.329202in}}%
\pgfpathlineto{\pgfqpoint{4.019265in}{2.363876in}}%
\pgfpathlineto{\pgfqpoint{4.019857in}{2.349088in}}%
\pgfpathlineto{\pgfqpoint{4.020350in}{2.327178in}}%
\pgfpathlineto{\pgfqpoint{4.020844in}{2.347705in}}%
\pgfpathlineto{\pgfqpoint{4.022719in}{2.412238in}}%
\pgfpathlineto{\pgfqpoint{4.023015in}{2.401479in}}%
\pgfpathlineto{\pgfqpoint{4.023706in}{2.376478in}}%
\pgfpathlineto{\pgfqpoint{4.024101in}{2.399569in}}%
\pgfpathlineto{\pgfqpoint{4.025483in}{2.431410in}}%
\pgfpathlineto{\pgfqpoint{4.025779in}{2.424058in}}%
\pgfpathlineto{\pgfqpoint{4.026667in}{2.411013in}}%
\pgfpathlineto{\pgfqpoint{4.026963in}{2.416707in}}%
\pgfpathlineto{\pgfqpoint{4.028937in}{2.445079in}}%
\pgfpathlineto{\pgfqpoint{4.029036in}{2.444447in}}%
\pgfpathlineto{\pgfqpoint{4.029233in}{2.443994in}}%
\pgfpathlineto{\pgfqpoint{4.029431in}{2.447717in}}%
\pgfpathlineto{\pgfqpoint{4.033477in}{2.519246in}}%
\pgfpathlineto{\pgfqpoint{4.033576in}{2.519036in}}%
\pgfpathlineto{\pgfqpoint{4.036340in}{2.544527in}}%
\pgfpathlineto{\pgfqpoint{4.036636in}{2.539502in}}%
\pgfpathlineto{\pgfqpoint{4.037919in}{2.523592in}}%
\pgfpathlineto{\pgfqpoint{4.037425in}{2.547449in}}%
\pgfpathlineto{\pgfqpoint{4.038116in}{2.525541in}}%
\pgfpathlineto{\pgfqpoint{4.038314in}{2.527424in}}%
\pgfpathlineto{\pgfqpoint{4.038807in}{2.521141in}}%
\pgfpathlineto{\pgfqpoint{4.041176in}{2.488177in}}%
\pgfpathlineto{\pgfqpoint{4.041472in}{2.491674in}}%
\pgfpathlineto{\pgfqpoint{4.041571in}{2.492355in}}%
\pgfpathlineto{\pgfqpoint{4.041768in}{2.487372in}}%
\pgfpathlineto{\pgfqpoint{4.043841in}{2.425349in}}%
\pgfpathlineto{\pgfqpoint{4.044038in}{2.428900in}}%
\pgfpathlineto{\pgfqpoint{4.044927in}{2.436180in}}%
\pgfpathlineto{\pgfqpoint{4.044532in}{2.421104in}}%
\pgfpathlineto{\pgfqpoint{4.045025in}{2.434273in}}%
\pgfpathlineto{\pgfqpoint{4.046111in}{2.398372in}}%
\pgfpathlineto{\pgfqpoint{4.046407in}{2.412335in}}%
\pgfpathlineto{\pgfqpoint{4.047493in}{2.376341in}}%
\pgfpathlineto{\pgfqpoint{4.049368in}{2.307750in}}%
\pgfpathlineto{\pgfqpoint{4.051441in}{2.260096in}}%
\pgfpathlineto{\pgfqpoint{4.051835in}{2.275848in}}%
\pgfpathlineto{\pgfqpoint{4.051934in}{2.279088in}}%
\pgfpathlineto{\pgfqpoint{4.052428in}{2.263264in}}%
\pgfpathlineto{\pgfqpoint{4.052921in}{2.276600in}}%
\pgfpathlineto{\pgfqpoint{4.053020in}{2.278123in}}%
\pgfpathlineto{\pgfqpoint{4.053513in}{2.271040in}}%
\pgfpathlineto{\pgfqpoint{4.053809in}{2.273988in}}%
\pgfpathlineto{\pgfqpoint{4.054303in}{2.257805in}}%
\pgfpathlineto{\pgfqpoint{4.055290in}{2.268379in}}%
\pgfpathlineto{\pgfqpoint{4.055487in}{2.269895in}}%
\pgfpathlineto{\pgfqpoint{4.055783in}{2.266769in}}%
\pgfpathlineto{\pgfqpoint{4.056080in}{2.267853in}}%
\pgfpathlineto{\pgfqpoint{4.056573in}{2.237786in}}%
\pgfpathlineto{\pgfqpoint{4.056869in}{2.270029in}}%
\pgfpathlineto{\pgfqpoint{4.058942in}{2.535830in}}%
\pgfpathlineto{\pgfqpoint{4.059040in}{2.529953in}}%
\pgfpathlineto{\pgfqpoint{4.061113in}{2.294137in}}%
\pgfpathlineto{\pgfqpoint{4.061607in}{2.321623in}}%
\pgfpathlineto{\pgfqpoint{4.061903in}{2.319378in}}%
\pgfpathlineto{\pgfqpoint{4.062100in}{2.328268in}}%
\pgfpathlineto{\pgfqpoint{4.063087in}{2.421900in}}%
\pgfpathlineto{\pgfqpoint{4.063679in}{2.389328in}}%
\pgfpathlineto{\pgfqpoint{4.064272in}{2.354426in}}%
\pgfpathlineto{\pgfqpoint{4.064864in}{2.376620in}}%
\pgfpathlineto{\pgfqpoint{4.066344in}{2.433815in}}%
\pgfpathlineto{\pgfqpoint{4.066640in}{2.421439in}}%
\pgfpathlineto{\pgfqpoint{4.067529in}{2.360612in}}%
\pgfpathlineto{\pgfqpoint{4.068318in}{2.369665in}}%
\pgfpathlineto{\pgfqpoint{4.068812in}{2.384106in}}%
\pgfpathlineto{\pgfqpoint{4.071674in}{2.525701in}}%
\pgfpathlineto{\pgfqpoint{4.071773in}{2.525166in}}%
\pgfpathlineto{\pgfqpoint{4.071970in}{2.529955in}}%
\pgfpathlineto{\pgfqpoint{4.072464in}{2.551071in}}%
\pgfpathlineto{\pgfqpoint{4.073154in}{2.540345in}}%
\pgfpathlineto{\pgfqpoint{4.073944in}{2.535760in}}%
\pgfpathlineto{\pgfqpoint{4.074043in}{2.537744in}}%
\pgfpathlineto{\pgfqpoint{4.074931in}{2.580224in}}%
\pgfpathlineto{\pgfqpoint{4.075424in}{2.553839in}}%
\pgfpathlineto{\pgfqpoint{4.075721in}{2.561942in}}%
\pgfpathlineto{\pgfqpoint{4.076115in}{2.547456in}}%
\pgfpathlineto{\pgfqpoint{4.076411in}{2.552059in}}%
\pgfpathlineto{\pgfqpoint{4.076905in}{2.531062in}}%
\pgfpathlineto{\pgfqpoint{4.077793in}{2.544484in}}%
\pgfpathlineto{\pgfqpoint{4.078978in}{2.550867in}}%
\pgfpathlineto{\pgfqpoint{4.078188in}{2.543835in}}%
\pgfpathlineto{\pgfqpoint{4.079175in}{2.546539in}}%
\pgfpathlineto{\pgfqpoint{4.080458in}{2.514304in}}%
\pgfpathlineto{\pgfqpoint{4.080656in}{2.521483in}}%
\pgfpathlineto{\pgfqpoint{4.080853in}{2.529174in}}%
\pgfpathlineto{\pgfqpoint{4.081544in}{2.510563in}}%
\pgfpathlineto{\pgfqpoint{4.089045in}{2.291356in}}%
\pgfpathlineto{\pgfqpoint{4.089144in}{2.291638in}}%
\pgfpathlineto{\pgfqpoint{4.089736in}{2.293053in}}%
\pgfpathlineto{\pgfqpoint{4.089440in}{2.291183in}}%
\pgfpathlineto{\pgfqpoint{4.089933in}{2.291384in}}%
\pgfpathlineto{\pgfqpoint{4.091019in}{2.280076in}}%
\pgfpathlineto{\pgfqpoint{4.091414in}{2.283494in}}%
\pgfpathlineto{\pgfqpoint{4.092302in}{2.297799in}}%
\pgfpathlineto{\pgfqpoint{4.092598in}{2.288879in}}%
\pgfpathlineto{\pgfqpoint{4.092796in}{2.282493in}}%
\pgfpathlineto{\pgfqpoint{4.093190in}{2.307548in}}%
\pgfpathlineto{\pgfqpoint{4.094276in}{2.318430in}}%
\pgfpathlineto{\pgfqpoint{4.093782in}{2.301558in}}%
\pgfpathlineto{\pgfqpoint{4.094473in}{2.314916in}}%
\pgfpathlineto{\pgfqpoint{4.096151in}{2.286173in}}%
\pgfpathlineto{\pgfqpoint{4.096447in}{2.280534in}}%
\pgfpathlineto{\pgfqpoint{4.097434in}{2.253482in}}%
\pgfpathlineto{\pgfqpoint{4.097928in}{2.263423in}}%
\pgfpathlineto{\pgfqpoint{4.098717in}{2.258785in}}%
\pgfpathlineto{\pgfqpoint{4.098816in}{2.261054in}}%
\pgfpathlineto{\pgfqpoint{4.100691in}{2.300102in}}%
\pgfpathlineto{\pgfqpoint{4.101678in}{2.309602in}}%
\pgfpathlineto{\pgfqpoint{4.101284in}{2.299312in}}%
\pgfpathlineto{\pgfqpoint{4.101876in}{2.306896in}}%
\pgfpathlineto{\pgfqpoint{4.101975in}{2.305357in}}%
\pgfpathlineto{\pgfqpoint{4.102172in}{2.312092in}}%
\pgfpathlineto{\pgfqpoint{4.102369in}{2.322489in}}%
\pgfpathlineto{\pgfqpoint{4.103159in}{2.306975in}}%
\pgfpathlineto{\pgfqpoint{4.103455in}{2.288351in}}%
\pgfpathlineto{\pgfqpoint{4.103948in}{2.329605in}}%
\pgfpathlineto{\pgfqpoint{4.105626in}{2.593078in}}%
\pgfpathlineto{\pgfqpoint{4.106416in}{2.539679in}}%
\pgfpathlineto{\pgfqpoint{4.107107in}{2.356633in}}%
\pgfpathlineto{\pgfqpoint{4.108094in}{2.366037in}}%
\pgfpathlineto{\pgfqpoint{4.108193in}{2.365257in}}%
\pgfpathlineto{\pgfqpoint{4.108291in}{2.369156in}}%
\pgfpathlineto{\pgfqpoint{4.110364in}{2.535975in}}%
\pgfpathlineto{\pgfqpoint{4.110759in}{2.488591in}}%
\pgfpathlineto{\pgfqpoint{4.111450in}{2.450709in}}%
\pgfpathlineto{\pgfqpoint{4.111844in}{2.484343in}}%
\pgfpathlineto{\pgfqpoint{4.113818in}{2.564142in}}%
\pgfpathlineto{\pgfqpoint{4.114411in}{2.510662in}}%
\pgfpathlineto{\pgfqpoint{4.115200in}{2.540656in}}%
\pgfpathlineto{\pgfqpoint{4.116681in}{2.580895in}}%
\pgfpathlineto{\pgfqpoint{4.116779in}{2.581121in}}%
\pgfpathlineto{\pgfqpoint{4.117766in}{2.545266in}}%
\pgfpathlineto{\pgfqpoint{4.118161in}{2.564194in}}%
\pgfpathlineto{\pgfqpoint{4.119049in}{2.571866in}}%
\pgfpathlineto{\pgfqpoint{4.118655in}{2.559198in}}%
\pgfpathlineto{\pgfqpoint{4.119247in}{2.568246in}}%
\pgfpathlineto{\pgfqpoint{4.121221in}{2.507152in}}%
\pgfpathlineto{\pgfqpoint{4.121517in}{2.514921in}}%
\pgfpathlineto{\pgfqpoint{4.121616in}{2.516503in}}%
\pgfpathlineto{\pgfqpoint{4.121912in}{2.505806in}}%
\pgfpathlineto{\pgfqpoint{4.124971in}{2.433154in}}%
\pgfpathlineto{\pgfqpoint{4.125267in}{2.434523in}}%
\pgfpathlineto{\pgfqpoint{4.125465in}{2.432160in}}%
\pgfpathlineto{\pgfqpoint{4.126847in}{2.396315in}}%
\pgfpathlineto{\pgfqpoint{4.127143in}{2.398635in}}%
\pgfpathlineto{\pgfqpoint{4.127241in}{2.398641in}}%
\pgfpathlineto{\pgfqpoint{4.130893in}{2.335769in}}%
\pgfpathlineto{\pgfqpoint{4.131091in}{2.326950in}}%
\pgfpathlineto{\pgfqpoint{4.131485in}{2.336775in}}%
\pgfpathlineto{\pgfqpoint{4.132078in}{2.327894in}}%
\pgfpathlineto{\pgfqpoint{4.132275in}{2.324422in}}%
\pgfpathlineto{\pgfqpoint{4.133657in}{2.303328in}}%
\pgfpathlineto{\pgfqpoint{4.134052in}{2.306922in}}%
\pgfpathlineto{\pgfqpoint{4.134348in}{2.301065in}}%
\pgfpathlineto{\pgfqpoint{4.134545in}{2.297136in}}%
\pgfpathlineto{\pgfqpoint{4.134940in}{2.311327in}}%
\pgfpathlineto{\pgfqpoint{4.135039in}{2.313974in}}%
\pgfpathlineto{\pgfqpoint{4.135532in}{2.301781in}}%
\pgfpathlineto{\pgfqpoint{4.135927in}{2.310157in}}%
\pgfpathlineto{\pgfqpoint{4.137111in}{2.297626in}}%
\pgfpathlineto{\pgfqpoint{4.137309in}{2.306339in}}%
\pgfpathlineto{\pgfqpoint{4.138987in}{2.343512in}}%
\pgfpathlineto{\pgfqpoint{4.139283in}{2.338129in}}%
\pgfpathlineto{\pgfqpoint{4.139678in}{2.351422in}}%
\pgfpathlineto{\pgfqpoint{4.141651in}{2.376976in}}%
\pgfpathlineto{\pgfqpoint{4.141849in}{2.370146in}}%
\pgfpathlineto{\pgfqpoint{4.143724in}{2.325651in}}%
\pgfpathlineto{\pgfqpoint{4.144218in}{2.314236in}}%
\pgfpathlineto{\pgfqpoint{4.144612in}{2.329120in}}%
\pgfpathlineto{\pgfqpoint{4.144711in}{2.330154in}}%
\pgfpathlineto{\pgfqpoint{4.145007in}{2.323509in}}%
\pgfpathlineto{\pgfqpoint{4.145205in}{2.320826in}}%
\pgfpathlineto{\pgfqpoint{4.145599in}{2.330958in}}%
\pgfpathlineto{\pgfqpoint{4.147277in}{2.351107in}}%
\pgfpathlineto{\pgfqpoint{4.148560in}{2.367968in}}%
\pgfpathlineto{\pgfqpoint{4.148067in}{2.350787in}}%
\pgfpathlineto{\pgfqpoint{4.148758in}{2.359698in}}%
\pgfpathlineto{\pgfqpoint{4.148857in}{2.357031in}}%
\pgfpathlineto{\pgfqpoint{4.149350in}{2.367015in}}%
\pgfpathlineto{\pgfqpoint{4.149547in}{2.366258in}}%
\pgfpathlineto{\pgfqpoint{4.149745in}{2.366095in}}%
\pgfpathlineto{\pgfqpoint{4.150238in}{2.333410in}}%
\pgfpathlineto{\pgfqpoint{4.150633in}{2.370357in}}%
\pgfpathlineto{\pgfqpoint{4.152804in}{2.597173in}}%
\pgfpathlineto{\pgfqpoint{4.153002in}{2.575619in}}%
\pgfpathlineto{\pgfqpoint{4.153693in}{2.409732in}}%
\pgfpathlineto{\pgfqpoint{4.154680in}{2.436885in}}%
\pgfpathlineto{\pgfqpoint{4.155075in}{2.427898in}}%
\pgfpathlineto{\pgfqpoint{4.155568in}{2.441321in}}%
\pgfpathlineto{\pgfqpoint{4.155864in}{2.436301in}}%
\pgfpathlineto{\pgfqpoint{4.156062in}{2.443969in}}%
\pgfpathlineto{\pgfqpoint{4.156950in}{2.506987in}}%
\pgfpathlineto{\pgfqpoint{4.157246in}{2.471939in}}%
\pgfpathlineto{\pgfqpoint{4.158332in}{2.371534in}}%
\pgfpathlineto{\pgfqpoint{4.158726in}{2.385348in}}%
\pgfpathlineto{\pgfqpoint{4.158825in}{2.385293in}}%
\pgfpathlineto{\pgfqpoint{4.159220in}{2.379259in}}%
\pgfpathlineto{\pgfqpoint{4.159713in}{2.387924in}}%
\pgfpathlineto{\pgfqpoint{4.159911in}{2.390422in}}%
\pgfpathlineto{\pgfqpoint{4.160207in}{2.380033in}}%
\pgfpathlineto{\pgfqpoint{4.161391in}{2.298735in}}%
\pgfpathlineto{\pgfqpoint{4.161983in}{2.312324in}}%
\pgfpathlineto{\pgfqpoint{4.163267in}{2.333550in}}%
\pgfpathlineto{\pgfqpoint{4.162576in}{2.310619in}}%
\pgfpathlineto{\pgfqpoint{4.163464in}{2.331617in}}%
\pgfpathlineto{\pgfqpoint{4.164155in}{2.300709in}}%
\pgfpathlineto{\pgfqpoint{4.164944in}{2.316094in}}%
\pgfpathlineto{\pgfqpoint{4.165241in}{2.335520in}}%
\pgfpathlineto{\pgfqpoint{4.166129in}{2.326803in}}%
\pgfpathlineto{\pgfqpoint{4.167708in}{2.310407in}}%
\pgfpathlineto{\pgfqpoint{4.166622in}{2.334948in}}%
\pgfpathlineto{\pgfqpoint{4.167807in}{2.312248in}}%
\pgfpathlineto{\pgfqpoint{4.168892in}{2.353256in}}%
\pgfpathlineto{\pgfqpoint{4.169386in}{2.340215in}}%
\pgfpathlineto{\pgfqpoint{4.170472in}{2.352313in}}%
\pgfpathlineto{\pgfqpoint{4.170768in}{2.348646in}}%
\pgfpathlineto{\pgfqpoint{4.171952in}{2.371871in}}%
\pgfpathlineto{\pgfqpoint{4.172939in}{2.362527in}}%
\pgfpathlineto{\pgfqpoint{4.174321in}{2.345759in}}%
\pgfpathlineto{\pgfqpoint{4.174420in}{2.348425in}}%
\pgfpathlineto{\pgfqpoint{4.174617in}{2.355056in}}%
\pgfpathlineto{\pgfqpoint{4.175110in}{2.334630in}}%
\pgfpathlineto{\pgfqpoint{4.175604in}{2.337523in}}%
\pgfpathlineto{\pgfqpoint{4.177874in}{2.302074in}}%
\pgfpathlineto{\pgfqpoint{4.178367in}{2.297881in}}%
\pgfpathlineto{\pgfqpoint{4.178664in}{2.303285in}}%
\pgfpathlineto{\pgfqpoint{4.179848in}{2.311989in}}%
\pgfpathlineto{\pgfqpoint{4.179354in}{2.303045in}}%
\pgfpathlineto{\pgfqpoint{4.180045in}{2.311082in}}%
\pgfpathlineto{\pgfqpoint{4.180144in}{2.310518in}}%
\pgfpathlineto{\pgfqpoint{4.180539in}{2.314479in}}%
\pgfpathlineto{\pgfqpoint{4.180736in}{2.312947in}}%
\pgfpathlineto{\pgfqpoint{4.181131in}{2.321103in}}%
\pgfpathlineto{\pgfqpoint{4.181427in}{2.312216in}}%
\pgfpathlineto{\pgfqpoint{4.181625in}{2.309178in}}%
\pgfpathlineto{\pgfqpoint{4.182217in}{2.320708in}}%
\pgfpathlineto{\pgfqpoint{4.184388in}{2.353567in}}%
\pgfpathlineto{\pgfqpoint{4.182809in}{2.315513in}}%
\pgfpathlineto{\pgfqpoint{4.184586in}{2.353082in}}%
\pgfpathlineto{\pgfqpoint{4.186560in}{2.399190in}}%
\pgfpathlineto{\pgfqpoint{4.187250in}{2.398698in}}%
\pgfpathlineto{\pgfqpoint{4.187744in}{2.409710in}}%
\pgfpathlineto{\pgfqpoint{4.190310in}{2.357994in}}%
\pgfpathlineto{\pgfqpoint{4.188632in}{2.412262in}}%
\pgfpathlineto{\pgfqpoint{4.190804in}{2.376914in}}%
\pgfpathlineto{\pgfqpoint{4.191100in}{2.368309in}}%
\pgfpathlineto{\pgfqpoint{4.192185in}{2.360874in}}%
\pgfpathlineto{\pgfqpoint{4.192284in}{2.361676in}}%
\pgfpathlineto{\pgfqpoint{4.192778in}{2.376956in}}%
\pgfpathlineto{\pgfqpoint{4.193666in}{2.371077in}}%
\pgfpathlineto{\pgfqpoint{4.193863in}{2.362293in}}%
\pgfpathlineto{\pgfqpoint{4.194357in}{2.378576in}}%
\pgfpathlineto{\pgfqpoint{4.194752in}{2.369390in}}%
\pgfpathlineto{\pgfqpoint{4.196232in}{2.384855in}}%
\pgfpathlineto{\pgfqpoint{4.196429in}{2.379336in}}%
\pgfpathlineto{\pgfqpoint{4.196824in}{2.360790in}}%
\pgfpathlineto{\pgfqpoint{4.197219in}{2.378566in}}%
\pgfpathlineto{\pgfqpoint{4.199292in}{2.623129in}}%
\pgfpathlineto{\pgfqpoint{4.199588in}{2.587909in}}%
\pgfpathlineto{\pgfqpoint{4.201364in}{2.369645in}}%
\pgfpathlineto{\pgfqpoint{4.201562in}{2.372353in}}%
\pgfpathlineto{\pgfqpoint{4.201858in}{2.367334in}}%
\pgfpathlineto{\pgfqpoint{4.202055in}{2.375824in}}%
\pgfpathlineto{\pgfqpoint{4.203536in}{2.479473in}}%
\pgfpathlineto{\pgfqpoint{4.204029in}{2.449704in}}%
\pgfpathlineto{\pgfqpoint{4.204621in}{2.393646in}}%
\pgfpathlineto{\pgfqpoint{4.205312in}{2.422835in}}%
\pgfpathlineto{\pgfqpoint{4.206891in}{2.463282in}}%
\pgfpathlineto{\pgfqpoint{4.207089in}{2.459217in}}%
\pgfpathlineto{\pgfqpoint{4.208076in}{2.415141in}}%
\pgfpathlineto{\pgfqpoint{4.208471in}{2.436897in}}%
\pgfpathlineto{\pgfqpoint{4.209754in}{2.466179in}}%
\pgfpathlineto{\pgfqpoint{4.209852in}{2.463690in}}%
\pgfpathlineto{\pgfqpoint{4.211037in}{2.418921in}}%
\pgfpathlineto{\pgfqpoint{4.211432in}{2.437940in}}%
\pgfpathlineto{\pgfqpoint{4.212813in}{2.466898in}}%
\pgfpathlineto{\pgfqpoint{4.213011in}{2.464966in}}%
\pgfpathlineto{\pgfqpoint{4.214393in}{2.441227in}}%
\pgfpathlineto{\pgfqpoint{4.214689in}{2.450962in}}%
\pgfpathlineto{\pgfqpoint{4.216268in}{2.476933in}}%
\pgfpathlineto{\pgfqpoint{4.216465in}{2.480851in}}%
\pgfpathlineto{\pgfqpoint{4.216860in}{2.465276in}}%
\pgfpathlineto{\pgfqpoint{4.216959in}{2.461852in}}%
\pgfpathlineto{\pgfqpoint{4.217354in}{2.478737in}}%
\pgfpathlineto{\pgfqpoint{4.217748in}{2.471611in}}%
\pgfpathlineto{\pgfqpoint{4.218735in}{2.491756in}}%
\pgfpathlineto{\pgfqpoint{4.219031in}{2.484488in}}%
\pgfpathlineto{\pgfqpoint{4.219130in}{2.482077in}}%
\pgfpathlineto{\pgfqpoint{4.219722in}{2.494574in}}%
\pgfpathlineto{\pgfqpoint{4.220117in}{2.492690in}}%
\pgfpathlineto{\pgfqpoint{4.220315in}{2.495517in}}%
\pgfpathlineto{\pgfqpoint{4.220512in}{2.498962in}}%
\pgfpathlineto{\pgfqpoint{4.220808in}{2.485514in}}%
\pgfpathlineto{\pgfqpoint{4.221400in}{2.493890in}}%
\pgfpathlineto{\pgfqpoint{4.221992in}{2.476749in}}%
\pgfpathlineto{\pgfqpoint{4.222091in}{2.476842in}}%
\pgfpathlineto{\pgfqpoint{4.222289in}{2.475696in}}%
\pgfpathlineto{\pgfqpoint{4.224263in}{2.439660in}}%
\pgfpathlineto{\pgfqpoint{4.224657in}{2.451294in}}%
\pgfpathlineto{\pgfqpoint{4.224756in}{2.454014in}}%
\pgfpathlineto{\pgfqpoint{4.225249in}{2.440995in}}%
\pgfpathlineto{\pgfqpoint{4.225546in}{2.445753in}}%
\pgfpathlineto{\pgfqpoint{4.225644in}{2.446235in}}%
\pgfpathlineto{\pgfqpoint{4.225743in}{2.444582in}}%
\pgfpathlineto{\pgfqpoint{4.226927in}{2.419153in}}%
\pgfpathlineto{\pgfqpoint{4.227125in}{2.427006in}}%
\pgfpathlineto{\pgfqpoint{4.228112in}{2.445930in}}%
\pgfpathlineto{\pgfqpoint{4.228309in}{2.438497in}}%
\pgfpathlineto{\pgfqpoint{4.229099in}{2.425810in}}%
\pgfpathlineto{\pgfqpoint{4.229395in}{2.434462in}}%
\pgfpathlineto{\pgfqpoint{4.230184in}{2.444439in}}%
\pgfpathlineto{\pgfqpoint{4.230579in}{2.438139in}}%
\pgfpathlineto{\pgfqpoint{4.230875in}{2.436800in}}%
\pgfpathlineto{\pgfqpoint{4.231171in}{2.432491in}}%
\pgfpathlineto{\pgfqpoint{4.231566in}{2.444432in}}%
\pgfpathlineto{\pgfqpoint{4.231764in}{2.446516in}}%
\pgfpathlineto{\pgfqpoint{4.232455in}{2.441108in}}%
\pgfpathlineto{\pgfqpoint{4.233836in}{2.412442in}}%
\pgfpathlineto{\pgfqpoint{4.234231in}{2.415790in}}%
\pgfpathlineto{\pgfqpoint{4.236501in}{2.471435in}}%
\pgfpathlineto{\pgfqpoint{4.236797in}{2.457788in}}%
\pgfpathlineto{\pgfqpoint{4.238179in}{2.425544in}}%
\pgfpathlineto{\pgfqpoint{4.238475in}{2.431259in}}%
\pgfpathlineto{\pgfqpoint{4.238870in}{2.422414in}}%
\pgfpathlineto{\pgfqpoint{4.239462in}{2.430777in}}%
\pgfpathlineto{\pgfqpoint{4.241140in}{2.445166in}}%
\pgfpathlineto{\pgfqpoint{4.240054in}{2.426920in}}%
\pgfpathlineto{\pgfqpoint{4.241239in}{2.442672in}}%
\pgfpathlineto{\pgfqpoint{4.242127in}{2.429891in}}%
\pgfpathlineto{\pgfqpoint{4.242423in}{2.436938in}}%
\pgfpathlineto{\pgfqpoint{4.242522in}{2.438564in}}%
\pgfpathlineto{\pgfqpoint{4.242917in}{2.432457in}}%
\pgfpathlineto{\pgfqpoint{4.243114in}{2.432661in}}%
\pgfpathlineto{\pgfqpoint{4.243608in}{2.417694in}}%
\pgfpathlineto{\pgfqpoint{4.243805in}{2.429207in}}%
\pgfpathlineto{\pgfqpoint{4.246075in}{2.692405in}}%
\pgfpathlineto{\pgfqpoint{4.246470in}{2.619385in}}%
\pgfpathlineto{\pgfqpoint{4.248148in}{2.434982in}}%
\pgfpathlineto{\pgfqpoint{4.250122in}{2.543086in}}%
\pgfpathlineto{\pgfqpoint{4.250714in}{2.524569in}}%
\pgfpathlineto{\pgfqpoint{4.251306in}{2.460057in}}%
\pgfpathlineto{\pgfqpoint{4.252096in}{2.474939in}}%
\pgfpathlineto{\pgfqpoint{4.253181in}{2.520019in}}%
\pgfpathlineto{\pgfqpoint{4.253379in}{2.527580in}}%
\pgfpathlineto{\pgfqpoint{4.253872in}{2.511688in}}%
\pgfpathlineto{\pgfqpoint{4.254760in}{2.470425in}}%
\pgfpathlineto{\pgfqpoint{4.255155in}{2.488533in}}%
\pgfpathlineto{\pgfqpoint{4.256537in}{2.515083in}}%
\pgfpathlineto{\pgfqpoint{4.256833in}{2.505441in}}%
\pgfpathlineto{\pgfqpoint{4.258116in}{2.478945in}}%
\pgfpathlineto{\pgfqpoint{4.258412in}{2.495108in}}%
\pgfpathlineto{\pgfqpoint{4.259202in}{2.517258in}}%
\pgfpathlineto{\pgfqpoint{4.259794in}{2.512365in}}%
\pgfpathlineto{\pgfqpoint{4.261077in}{2.492512in}}%
\pgfpathlineto{\pgfqpoint{4.260386in}{2.512561in}}%
\pgfpathlineto{\pgfqpoint{4.261472in}{2.505356in}}%
\pgfpathlineto{\pgfqpoint{4.262360in}{2.524542in}}%
\pgfpathlineto{\pgfqpoint{4.262656in}{2.516182in}}%
\pgfpathlineto{\pgfqpoint{4.262755in}{2.514222in}}%
\pgfpathlineto{\pgfqpoint{4.263150in}{2.525486in}}%
\pgfpathlineto{\pgfqpoint{4.263347in}{2.524237in}}%
\pgfpathlineto{\pgfqpoint{4.264038in}{2.511422in}}%
\pgfpathlineto{\pgfqpoint{4.264729in}{2.518861in}}%
\pgfpathlineto{\pgfqpoint{4.265223in}{2.528584in}}%
\pgfpathlineto{\pgfqpoint{4.266604in}{2.526654in}}%
\pgfpathlineto{\pgfqpoint{4.268480in}{2.506521in}}%
\pgfpathlineto{\pgfqpoint{4.268677in}{2.510931in}}%
\pgfpathlineto{\pgfqpoint{4.268973in}{2.517324in}}%
\pgfpathlineto{\pgfqpoint{4.269467in}{2.502908in}}%
\pgfpathlineto{\pgfqpoint{4.273217in}{2.445838in}}%
\pgfpathlineto{\pgfqpoint{4.273513in}{2.453696in}}%
\pgfpathlineto{\pgfqpoint{4.273809in}{2.462816in}}%
\pgfpathlineto{\pgfqpoint{4.274204in}{2.448396in}}%
\pgfpathlineto{\pgfqpoint{4.274698in}{2.458171in}}%
\pgfpathlineto{\pgfqpoint{4.275685in}{2.443554in}}%
\pgfpathlineto{\pgfqpoint{4.275981in}{2.446627in}}%
\pgfpathlineto{\pgfqpoint{4.276178in}{2.444545in}}%
\pgfpathlineto{\pgfqpoint{4.276474in}{2.438565in}}%
\pgfpathlineto{\pgfqpoint{4.276869in}{2.451644in}}%
\pgfpathlineto{\pgfqpoint{4.277066in}{2.455514in}}%
\pgfpathlineto{\pgfqpoint{4.277461in}{2.449730in}}%
\pgfpathlineto{\pgfqpoint{4.277856in}{2.450048in}}%
\pgfpathlineto{\pgfqpoint{4.278942in}{2.443140in}}%
\pgfpathlineto{\pgfqpoint{4.278547in}{2.453108in}}%
\pgfpathlineto{\pgfqpoint{4.279139in}{2.445497in}}%
\pgfpathlineto{\pgfqpoint{4.281903in}{2.489466in}}%
\pgfpathlineto{\pgfqpoint{4.282001in}{2.488176in}}%
\pgfpathlineto{\pgfqpoint{4.284568in}{2.427301in}}%
\pgfpathlineto{\pgfqpoint{4.282692in}{2.491570in}}%
\pgfpathlineto{\pgfqpoint{4.284864in}{2.430053in}}%
\pgfpathlineto{\pgfqpoint{4.286936in}{2.414864in}}%
\pgfpathlineto{\pgfqpoint{4.287035in}{2.415297in}}%
\pgfpathlineto{\pgfqpoint{4.288713in}{2.436046in}}%
\pgfpathlineto{\pgfqpoint{4.288910in}{2.432953in}}%
\pgfpathlineto{\pgfqpoint{4.289305in}{2.422046in}}%
\pgfpathlineto{\pgfqpoint{4.289799in}{2.438694in}}%
\pgfpathlineto{\pgfqpoint{4.290884in}{2.445044in}}%
\pgfpathlineto{\pgfqpoint{4.290391in}{2.438315in}}%
\pgfpathlineto{\pgfqpoint{4.290983in}{2.442570in}}%
\pgfpathlineto{\pgfqpoint{4.291575in}{2.412477in}}%
\pgfpathlineto{\pgfqpoint{4.291970in}{2.446300in}}%
\pgfpathlineto{\pgfqpoint{4.293352in}{2.684501in}}%
\pgfpathlineto{\pgfqpoint{4.294141in}{2.669268in}}%
\pgfpathlineto{\pgfqpoint{4.294734in}{2.497349in}}%
\pgfpathlineto{\pgfqpoint{4.296017in}{2.425726in}}%
\pgfpathlineto{\pgfqpoint{4.296411in}{2.417945in}}%
\pgfpathlineto{\pgfqpoint{4.296609in}{2.430471in}}%
\pgfpathlineto{\pgfqpoint{4.298287in}{2.531183in}}%
\pgfpathlineto{\pgfqpoint{4.298484in}{2.522420in}}%
\pgfpathlineto{\pgfqpoint{4.299274in}{2.432244in}}%
\pgfpathlineto{\pgfqpoint{4.300063in}{2.475130in}}%
\pgfpathlineto{\pgfqpoint{4.300557in}{2.470057in}}%
\pgfpathlineto{\pgfqpoint{4.300952in}{2.494649in}}%
\pgfpathlineto{\pgfqpoint{4.301445in}{2.525584in}}%
\pgfpathlineto{\pgfqpoint{4.301939in}{2.496008in}}%
\pgfpathlineto{\pgfqpoint{4.302827in}{2.480018in}}%
\pgfpathlineto{\pgfqpoint{4.303123in}{2.490199in}}%
\pgfpathlineto{\pgfqpoint{4.304406in}{2.516207in}}%
\pgfpathlineto{\pgfqpoint{4.304505in}{2.515880in}}%
\pgfpathlineto{\pgfqpoint{4.305393in}{2.502107in}}%
\pgfpathlineto{\pgfqpoint{4.305788in}{2.489530in}}%
\pgfpathlineto{\pgfqpoint{4.306281in}{2.508312in}}%
\pgfpathlineto{\pgfqpoint{4.307071in}{2.534007in}}%
\pgfpathlineto{\pgfqpoint{4.307959in}{2.530641in}}%
\pgfpathlineto{\pgfqpoint{4.309242in}{2.510072in}}%
\pgfpathlineto{\pgfqpoint{4.309440in}{2.517777in}}%
\pgfpathlineto{\pgfqpoint{4.309736in}{2.533260in}}%
\pgfpathlineto{\pgfqpoint{4.310624in}{2.525702in}}%
\pgfpathlineto{\pgfqpoint{4.311611in}{2.517381in}}%
\pgfpathlineto{\pgfqpoint{4.311216in}{2.530894in}}%
\pgfpathlineto{\pgfqpoint{4.311907in}{2.522013in}}%
\pgfpathlineto{\pgfqpoint{4.312105in}{2.519931in}}%
\pgfpathlineto{\pgfqpoint{4.312598in}{2.524440in}}%
\pgfpathlineto{\pgfqpoint{4.315658in}{2.444225in}}%
\pgfpathlineto{\pgfqpoint{4.315855in}{2.435596in}}%
\pgfpathlineto{\pgfqpoint{4.316151in}{2.469200in}}%
\pgfpathlineto{\pgfqpoint{4.317237in}{2.507382in}}%
\pgfpathlineto{\pgfqpoint{4.317434in}{2.503951in}}%
\pgfpathlineto{\pgfqpoint{4.317533in}{2.502507in}}%
\pgfpathlineto{\pgfqpoint{4.317928in}{2.507237in}}%
\pgfpathlineto{\pgfqpoint{4.318421in}{2.503927in}}%
\pgfpathlineto{\pgfqpoint{4.318619in}{2.505372in}}%
\pgfpathlineto{\pgfqpoint{4.318915in}{2.500891in}}%
\pgfpathlineto{\pgfqpoint{4.321284in}{2.460324in}}%
\pgfpathlineto{\pgfqpoint{4.321382in}{2.462171in}}%
\pgfpathlineto{\pgfqpoint{4.322271in}{2.474581in}}%
\pgfpathlineto{\pgfqpoint{4.322468in}{2.470113in}}%
\pgfpathlineto{\pgfqpoint{4.323554in}{2.447137in}}%
\pgfpathlineto{\pgfqpoint{4.323751in}{2.451392in}}%
\pgfpathlineto{\pgfqpoint{4.323850in}{2.452435in}}%
\pgfpathlineto{\pgfqpoint{4.324245in}{2.445825in}}%
\pgfpathlineto{\pgfqpoint{4.324837in}{2.451124in}}%
\pgfpathlineto{\pgfqpoint{4.324935in}{2.450939in}}%
\pgfpathlineto{\pgfqpoint{4.325034in}{2.452327in}}%
\pgfpathlineto{\pgfqpoint{4.325330in}{2.460769in}}%
\pgfpathlineto{\pgfqpoint{4.325824in}{2.445131in}}%
\pgfpathlineto{\pgfqpoint{4.325922in}{2.444490in}}%
\pgfpathlineto{\pgfqpoint{4.326120in}{2.448914in}}%
\pgfpathlineto{\pgfqpoint{4.327995in}{2.474780in}}%
\pgfpathlineto{\pgfqpoint{4.328390in}{2.466648in}}%
\pgfpathlineto{\pgfqpoint{4.328686in}{2.480592in}}%
\pgfpathlineto{\pgfqpoint{4.329870in}{2.495380in}}%
\pgfpathlineto{\pgfqpoint{4.329969in}{2.493611in}}%
\pgfpathlineto{\pgfqpoint{4.333029in}{2.436132in}}%
\pgfpathlineto{\pgfqpoint{4.333127in}{2.436660in}}%
\pgfpathlineto{\pgfqpoint{4.333325in}{2.434619in}}%
\pgfpathlineto{\pgfqpoint{4.333720in}{2.423760in}}%
\pgfpathlineto{\pgfqpoint{4.334114in}{2.438280in}}%
\pgfpathlineto{\pgfqpoint{4.334509in}{2.429210in}}%
\pgfpathlineto{\pgfqpoint{4.334608in}{2.428657in}}%
\pgfpathlineto{\pgfqpoint{4.334707in}{2.430588in}}%
\pgfpathlineto{\pgfqpoint{4.335694in}{2.442375in}}%
\pgfpathlineto{\pgfqpoint{4.335990in}{2.438594in}}%
\pgfpathlineto{\pgfqpoint{4.337174in}{2.425034in}}%
\pgfpathlineto{\pgfqpoint{4.337470in}{2.431502in}}%
\pgfpathlineto{\pgfqpoint{4.337668in}{2.433735in}}%
\pgfpathlineto{\pgfqpoint{4.338062in}{2.424031in}}%
\pgfpathlineto{\pgfqpoint{4.339247in}{2.387830in}}%
\pgfpathlineto{\pgfqpoint{4.338655in}{2.424835in}}%
\pgfpathlineto{\pgfqpoint{4.339543in}{2.413441in}}%
\pgfpathlineto{\pgfqpoint{4.341122in}{2.673539in}}%
\pgfpathlineto{\pgfqpoint{4.342010in}{2.619360in}}%
\pgfpathlineto{\pgfqpoint{4.342899in}{2.404908in}}%
\pgfpathlineto{\pgfqpoint{4.343886in}{2.415659in}}%
\pgfpathlineto{\pgfqpoint{4.344280in}{2.423094in}}%
\pgfpathlineto{\pgfqpoint{4.345958in}{2.525232in}}%
\pgfpathlineto{\pgfqpoint{4.346551in}{2.478952in}}%
\pgfpathlineto{\pgfqpoint{4.346945in}{2.433073in}}%
\pgfpathlineto{\pgfqpoint{4.347735in}{2.463319in}}%
\pgfpathlineto{\pgfqpoint{4.348130in}{2.472474in}}%
\pgfpathlineto{\pgfqpoint{4.349215in}{2.518040in}}%
\pgfpathlineto{\pgfqpoint{4.349511in}{2.503308in}}%
\pgfpathlineto{\pgfqpoint{4.350498in}{2.458551in}}%
\pgfpathlineto{\pgfqpoint{4.350893in}{2.471632in}}%
\pgfpathlineto{\pgfqpoint{4.352472in}{2.522283in}}%
\pgfpathlineto{\pgfqpoint{4.353065in}{2.507737in}}%
\pgfpathlineto{\pgfqpoint{4.353459in}{2.492100in}}%
\pgfpathlineto{\pgfqpoint{4.353953in}{2.513440in}}%
\pgfpathlineto{\pgfqpoint{4.354446in}{2.540860in}}%
\pgfpathlineto{\pgfqpoint{4.355335in}{2.535154in}}%
\pgfpathlineto{\pgfqpoint{4.355730in}{2.538637in}}%
\pgfpathlineto{\pgfqpoint{4.356026in}{2.533950in}}%
\pgfpathlineto{\pgfqpoint{4.356815in}{2.527101in}}%
\pgfpathlineto{\pgfqpoint{4.357013in}{2.531584in}}%
\pgfpathlineto{\pgfqpoint{4.358098in}{2.545488in}}%
\pgfpathlineto{\pgfqpoint{4.358296in}{2.541873in}}%
\pgfpathlineto{\pgfqpoint{4.359579in}{2.526321in}}%
\pgfpathlineto{\pgfqpoint{4.359875in}{2.531709in}}%
\pgfpathlineto{\pgfqpoint{4.360763in}{2.545613in}}%
\pgfpathlineto{\pgfqpoint{4.361355in}{2.540338in}}%
\pgfpathlineto{\pgfqpoint{4.361454in}{2.540621in}}%
\pgfpathlineto{\pgfqpoint{4.361553in}{2.538870in}}%
\pgfpathlineto{\pgfqpoint{4.362244in}{2.544326in}}%
\pgfpathlineto{\pgfqpoint{4.362737in}{2.532495in}}%
\pgfpathlineto{\pgfqpoint{4.363033in}{2.536949in}}%
\pgfpathlineto{\pgfqpoint{4.363428in}{2.528189in}}%
\pgfpathlineto{\pgfqpoint{4.363527in}{2.526092in}}%
\pgfpathlineto{\pgfqpoint{4.364020in}{2.538958in}}%
\pgfpathlineto{\pgfqpoint{4.364119in}{2.538800in}}%
\pgfpathlineto{\pgfqpoint{4.366784in}{2.483976in}}%
\pgfpathlineto{\pgfqpoint{4.366981in}{2.488007in}}%
\pgfpathlineto{\pgfqpoint{4.367277in}{2.494235in}}%
\pgfpathlineto{\pgfqpoint{4.367771in}{2.478213in}}%
\pgfpathlineto{\pgfqpoint{4.367869in}{2.477932in}}%
\pgfpathlineto{\pgfqpoint{4.367968in}{2.479175in}}%
\pgfpathlineto{\pgfqpoint{4.368166in}{2.481189in}}%
\pgfpathlineto{\pgfqpoint{4.368462in}{2.472105in}}%
\pgfpathlineto{\pgfqpoint{4.369449in}{2.455010in}}%
\pgfpathlineto{\pgfqpoint{4.368955in}{2.479598in}}%
\pgfpathlineto{\pgfqpoint{4.369646in}{2.464031in}}%
\pgfpathlineto{\pgfqpoint{4.369843in}{2.473302in}}%
\pgfpathlineto{\pgfqpoint{4.370633in}{2.464865in}}%
\pgfpathlineto{\pgfqpoint{4.371225in}{2.473611in}}%
\pgfpathlineto{\pgfqpoint{4.372212in}{2.447331in}}%
\pgfpathlineto{\pgfqpoint{4.372311in}{2.446839in}}%
\pgfpathlineto{\pgfqpoint{4.372410in}{2.449631in}}%
\pgfpathlineto{\pgfqpoint{4.373594in}{2.463913in}}%
\pgfpathlineto{\pgfqpoint{4.373101in}{2.448423in}}%
\pgfpathlineto{\pgfqpoint{4.373693in}{2.463010in}}%
\pgfpathlineto{\pgfqpoint{4.374976in}{2.445714in}}%
\pgfpathlineto{\pgfqpoint{4.375173in}{2.448005in}}%
\pgfpathlineto{\pgfqpoint{4.376358in}{2.475478in}}%
\pgfpathlineto{\pgfqpoint{4.376654in}{2.467563in}}%
\pgfpathlineto{\pgfqpoint{4.376950in}{2.478103in}}%
\pgfpathlineto{\pgfqpoint{4.377739in}{2.498978in}}%
\pgfpathlineto{\pgfqpoint{4.378035in}{2.481826in}}%
\pgfpathlineto{\pgfqpoint{4.378134in}{2.478827in}}%
\pgfpathlineto{\pgfqpoint{4.378430in}{2.497106in}}%
\pgfpathlineto{\pgfqpoint{4.378529in}{2.501642in}}%
\pgfpathlineto{\pgfqpoint{4.379022in}{2.478173in}}%
\pgfpathlineto{\pgfqpoint{4.381194in}{2.434534in}}%
\pgfpathlineto{\pgfqpoint{4.381490in}{2.441208in}}%
\pgfpathlineto{\pgfqpoint{4.381687in}{2.437943in}}%
\pgfpathlineto{\pgfqpoint{4.382082in}{2.427545in}}%
\pgfpathlineto{\pgfqpoint{4.382576in}{2.443049in}}%
\pgfpathlineto{\pgfqpoint{4.382773in}{2.446587in}}%
\pgfpathlineto{\pgfqpoint{4.383168in}{2.435195in}}%
\pgfpathlineto{\pgfqpoint{4.383661in}{2.443442in}}%
\pgfpathlineto{\pgfqpoint{4.383957in}{2.431719in}}%
\pgfpathlineto{\pgfqpoint{4.384747in}{2.443284in}}%
\pgfpathlineto{\pgfqpoint{4.385931in}{2.449434in}}%
\pgfpathlineto{\pgfqpoint{4.385537in}{2.441198in}}%
\pgfpathlineto{\pgfqpoint{4.386129in}{2.447828in}}%
\pgfpathlineto{\pgfqpoint{4.387214in}{2.426397in}}%
\pgfpathlineto{\pgfqpoint{4.386622in}{2.455602in}}%
\pgfpathlineto{\pgfqpoint{4.387511in}{2.433985in}}%
\pgfpathlineto{\pgfqpoint{4.389188in}{2.698333in}}%
\pgfpathlineto{\pgfqpoint{4.390077in}{2.625439in}}%
\pgfpathlineto{\pgfqpoint{4.391755in}{2.424814in}}%
\pgfpathlineto{\pgfqpoint{4.392643in}{2.434946in}}%
\pgfpathlineto{\pgfqpoint{4.393334in}{2.487196in}}%
\pgfpathlineto{\pgfqpoint{4.393926in}{2.537747in}}%
\pgfpathlineto{\pgfqpoint{4.394419in}{2.488264in}}%
\pgfpathlineto{\pgfqpoint{4.395110in}{2.431736in}}%
\pgfpathlineto{\pgfqpoint{4.395801in}{2.452495in}}%
\pgfpathlineto{\pgfqpoint{4.396887in}{2.465789in}}%
\pgfpathlineto{\pgfqpoint{4.396393in}{2.451110in}}%
\pgfpathlineto{\pgfqpoint{4.397183in}{2.458889in}}%
\pgfpathlineto{\pgfqpoint{4.398269in}{2.387768in}}%
\pgfpathlineto{\pgfqpoint{4.398762in}{2.433375in}}%
\pgfpathlineto{\pgfqpoint{4.400144in}{2.519687in}}%
\pgfpathlineto{\pgfqpoint{4.400835in}{2.514765in}}%
\pgfpathlineto{\pgfqpoint{4.400934in}{2.515692in}}%
\pgfpathlineto{\pgfqpoint{4.401032in}{2.512537in}}%
\pgfpathlineto{\pgfqpoint{4.401526in}{2.489114in}}%
\pgfpathlineto{\pgfqpoint{4.402217in}{2.500308in}}%
\pgfpathlineto{\pgfqpoint{4.403697in}{2.517703in}}%
\pgfpathlineto{\pgfqpoint{4.403796in}{2.516850in}}%
\pgfpathlineto{\pgfqpoint{4.404191in}{2.500687in}}%
\pgfpathlineto{\pgfqpoint{4.404980in}{2.511412in}}%
\pgfpathlineto{\pgfqpoint{4.405375in}{2.516199in}}%
\pgfpathlineto{\pgfqpoint{4.405671in}{2.526731in}}%
\pgfpathlineto{\pgfqpoint{4.406165in}{2.512301in}}%
\pgfpathlineto{\pgfqpoint{4.406461in}{2.515663in}}%
\pgfpathlineto{\pgfqpoint{4.406658in}{2.517329in}}%
\pgfpathlineto{\pgfqpoint{4.407349in}{2.513447in}}%
\pgfpathlineto{\pgfqpoint{4.407448in}{2.512837in}}%
\pgfpathlineto{\pgfqpoint{4.407843in}{2.516429in}}%
\pgfpathlineto{\pgfqpoint{4.409915in}{2.526458in}}%
\pgfpathlineto{\pgfqpoint{4.408336in}{2.513427in}}%
\pgfpathlineto{\pgfqpoint{4.410014in}{2.525805in}}%
\pgfpathlineto{\pgfqpoint{4.410113in}{2.525031in}}%
\pgfpathlineto{\pgfqpoint{4.410310in}{2.529239in}}%
\pgfpathlineto{\pgfqpoint{4.410507in}{2.537521in}}%
\pgfpathlineto{\pgfqpoint{4.411001in}{2.514038in}}%
\pgfpathlineto{\pgfqpoint{4.411297in}{2.527310in}}%
\pgfpathlineto{\pgfqpoint{4.411988in}{2.528351in}}%
\pgfpathlineto{\pgfqpoint{4.413666in}{2.489775in}}%
\pgfpathlineto{\pgfqpoint{4.414061in}{2.501759in}}%
\pgfpathlineto{\pgfqpoint{4.414653in}{2.488007in}}%
\pgfpathlineto{\pgfqpoint{4.416133in}{2.472089in}}%
\pgfpathlineto{\pgfqpoint{4.417416in}{2.446109in}}%
\pgfpathlineto{\pgfqpoint{4.417515in}{2.447332in}}%
\pgfpathlineto{\pgfqpoint{4.418502in}{2.455086in}}%
\pgfpathlineto{\pgfqpoint{4.418699in}{2.452255in}}%
\pgfpathlineto{\pgfqpoint{4.419785in}{2.435153in}}%
\pgfpathlineto{\pgfqpoint{4.419983in}{2.439600in}}%
\pgfpathlineto{\pgfqpoint{4.420180in}{2.443541in}}%
\pgfpathlineto{\pgfqpoint{4.420575in}{2.426722in}}%
\pgfpathlineto{\pgfqpoint{4.420970in}{2.422664in}}%
\pgfpathlineto{\pgfqpoint{4.421463in}{2.430451in}}%
\pgfpathlineto{\pgfqpoint{4.422943in}{2.421918in}}%
\pgfpathlineto{\pgfqpoint{4.423141in}{2.425629in}}%
\pgfpathlineto{\pgfqpoint{4.424720in}{2.451656in}}%
\pgfpathlineto{\pgfqpoint{4.425115in}{2.445549in}}%
\pgfpathlineto{\pgfqpoint{4.425806in}{2.448921in}}%
\pgfpathlineto{\pgfqpoint{4.426102in}{2.454700in}}%
\pgfpathlineto{\pgfqpoint{4.426497in}{2.442535in}}%
\pgfpathlineto{\pgfqpoint{4.429754in}{2.378288in}}%
\pgfpathlineto{\pgfqpoint{4.429852in}{2.379080in}}%
\pgfpathlineto{\pgfqpoint{4.430741in}{2.388557in}}%
\pgfpathlineto{\pgfqpoint{4.431136in}{2.382471in}}%
\pgfpathlineto{\pgfqpoint{4.431530in}{2.374865in}}%
\pgfpathlineto{\pgfqpoint{4.432221in}{2.381039in}}%
\pgfpathlineto{\pgfqpoint{4.432517in}{2.382158in}}%
\pgfpathlineto{\pgfqpoint{4.433307in}{2.380352in}}%
\pgfpathlineto{\pgfqpoint{4.434393in}{2.391463in}}%
\pgfpathlineto{\pgfqpoint{4.434590in}{2.389878in}}%
\pgfpathlineto{\pgfqpoint{4.435182in}{2.368477in}}%
\pgfpathlineto{\pgfqpoint{4.435478in}{2.389509in}}%
\pgfpathlineto{\pgfqpoint{4.436959in}{2.658497in}}%
\pgfpathlineto{\pgfqpoint{4.437946in}{2.612510in}}%
\pgfpathlineto{\pgfqpoint{4.439821in}{2.378822in}}%
\pgfpathlineto{\pgfqpoint{4.441696in}{2.466061in}}%
\pgfpathlineto{\pgfqpoint{4.441992in}{2.482582in}}%
\pgfpathlineto{\pgfqpoint{4.442486in}{2.442643in}}%
\pgfpathlineto{\pgfqpoint{4.442979in}{2.392750in}}%
\pgfpathlineto{\pgfqpoint{4.443572in}{2.432905in}}%
\pgfpathlineto{\pgfqpoint{4.445052in}{2.474567in}}%
\pgfpathlineto{\pgfqpoint{4.445447in}{2.455440in}}%
\pgfpathlineto{\pgfqpoint{4.446236in}{2.412441in}}%
\pgfpathlineto{\pgfqpoint{4.446730in}{2.426776in}}%
\pgfpathlineto{\pgfqpoint{4.448309in}{2.475499in}}%
\pgfpathlineto{\pgfqpoint{4.448507in}{2.467781in}}%
\pgfpathlineto{\pgfqpoint{4.449395in}{2.438052in}}%
\pgfpathlineto{\pgfqpoint{4.449888in}{2.452897in}}%
\pgfpathlineto{\pgfqpoint{4.450086in}{2.455912in}}%
\pgfpathlineto{\pgfqpoint{4.450480in}{2.478784in}}%
\pgfpathlineto{\pgfqpoint{4.451369in}{2.472170in}}%
\pgfpathlineto{\pgfqpoint{4.452553in}{2.451824in}}%
\pgfpathlineto{\pgfqpoint{4.452849in}{2.461182in}}%
\pgfpathlineto{\pgfqpoint{4.453639in}{2.473662in}}%
\pgfpathlineto{\pgfqpoint{4.454034in}{2.465843in}}%
\pgfpathlineto{\pgfqpoint{4.454725in}{2.485160in}}%
\pgfpathlineto{\pgfqpoint{4.455218in}{2.470830in}}%
\pgfpathlineto{\pgfqpoint{4.455712in}{2.464477in}}%
\pgfpathlineto{\pgfqpoint{4.456205in}{2.473817in}}%
\pgfpathlineto{\pgfqpoint{4.456304in}{2.473721in}}%
\pgfpathlineto{\pgfqpoint{4.456600in}{2.466677in}}%
\pgfpathlineto{\pgfqpoint{4.456995in}{2.476511in}}%
\pgfpathlineto{\pgfqpoint{4.457389in}{2.473216in}}%
\pgfpathlineto{\pgfqpoint{4.457587in}{2.474195in}}%
\pgfpathlineto{\pgfqpoint{4.457784in}{2.466439in}}%
\pgfpathlineto{\pgfqpoint{4.458080in}{2.454574in}}%
\pgfpathlineto{\pgfqpoint{4.458870in}{2.462211in}}%
\pgfpathlineto{\pgfqpoint{4.458969in}{2.462417in}}%
\pgfpathlineto{\pgfqpoint{4.459363in}{2.437734in}}%
\pgfpathlineto{\pgfqpoint{4.460350in}{2.450301in}}%
\pgfpathlineto{\pgfqpoint{4.463114in}{2.410019in}}%
\pgfpathlineto{\pgfqpoint{4.463213in}{2.411192in}}%
\pgfpathlineto{\pgfqpoint{4.463607in}{2.420605in}}%
\pgfpathlineto{\pgfqpoint{4.464002in}{2.405762in}}%
\pgfpathlineto{\pgfqpoint{4.464101in}{2.403999in}}%
\pgfpathlineto{\pgfqpoint{4.464496in}{2.416717in}}%
\pgfpathlineto{\pgfqpoint{4.464594in}{2.416397in}}%
\pgfpathlineto{\pgfqpoint{4.465779in}{2.394355in}}%
\pgfpathlineto{\pgfqpoint{4.466075in}{2.398714in}}%
\pgfpathlineto{\pgfqpoint{4.466470in}{2.395383in}}%
\pgfpathlineto{\pgfqpoint{4.466667in}{2.399786in}}%
\pgfpathlineto{\pgfqpoint{4.466766in}{2.400810in}}%
\pgfpathlineto{\pgfqpoint{4.467062in}{2.392792in}}%
\pgfpathlineto{\pgfqpoint{4.467161in}{2.390316in}}%
\pgfpathlineto{\pgfqpoint{4.467852in}{2.400848in}}%
\pgfpathlineto{\pgfqpoint{4.468049in}{2.401141in}}%
\pgfpathlineto{\pgfqpoint{4.468246in}{2.399751in}}%
\pgfpathlineto{\pgfqpoint{4.468542in}{2.393698in}}%
\pgfpathlineto{\pgfqpoint{4.468937in}{2.409220in}}%
\pgfpathlineto{\pgfqpoint{4.469036in}{2.412524in}}%
\pgfpathlineto{\pgfqpoint{4.469529in}{2.395424in}}%
\pgfpathlineto{\pgfqpoint{4.470023in}{2.392196in}}%
\pgfpathlineto{\pgfqpoint{4.470319in}{2.397410in}}%
\pgfpathlineto{\pgfqpoint{4.471701in}{2.421226in}}%
\pgfpathlineto{\pgfqpoint{4.472984in}{2.465036in}}%
\pgfpathlineto{\pgfqpoint{4.473181in}{2.460400in}}%
\pgfpathlineto{\pgfqpoint{4.473477in}{2.448903in}}%
\pgfpathlineto{\pgfqpoint{4.474366in}{2.455157in}}%
\pgfpathlineto{\pgfqpoint{4.474563in}{2.459165in}}%
\pgfpathlineto{\pgfqpoint{4.474958in}{2.441264in}}%
\pgfpathlineto{\pgfqpoint{4.478215in}{2.345852in}}%
\pgfpathlineto{\pgfqpoint{4.480288in}{2.457112in}}%
\pgfpathlineto{\pgfqpoint{4.481768in}{2.443922in}}%
\pgfpathlineto{\pgfqpoint{4.483150in}{2.416340in}}%
\pgfpathlineto{\pgfqpoint{4.482163in}{2.447681in}}%
\pgfpathlineto{\pgfqpoint{4.483347in}{2.432729in}}%
\pgfpathlineto{\pgfqpoint{4.484926in}{2.711217in}}%
\pgfpathlineto{\pgfqpoint{4.485815in}{2.668708in}}%
\pgfpathlineto{\pgfqpoint{4.487789in}{2.429270in}}%
\pgfpathlineto{\pgfqpoint{4.489170in}{2.490978in}}%
\pgfpathlineto{\pgfqpoint{4.489763in}{2.548120in}}%
\pgfpathlineto{\pgfqpoint{4.490355in}{2.502357in}}%
\pgfpathlineto{\pgfqpoint{4.490947in}{2.448849in}}%
\pgfpathlineto{\pgfqpoint{4.491539in}{2.485995in}}%
\pgfpathlineto{\pgfqpoint{4.491835in}{2.479467in}}%
\pgfpathlineto{\pgfqpoint{4.492329in}{2.492694in}}%
\pgfpathlineto{\pgfqpoint{4.492921in}{2.523759in}}%
\pgfpathlineto{\pgfqpoint{4.493513in}{2.500541in}}%
\pgfpathlineto{\pgfqpoint{4.494105in}{2.457261in}}%
\pgfpathlineto{\pgfqpoint{4.494796in}{2.487607in}}%
\pgfpathlineto{\pgfqpoint{4.495389in}{2.499553in}}%
\pgfpathlineto{\pgfqpoint{4.496277in}{2.520498in}}%
\pgfpathlineto{\pgfqpoint{4.496573in}{2.509545in}}%
\pgfpathlineto{\pgfqpoint{4.497362in}{2.477981in}}%
\pgfpathlineto{\pgfqpoint{4.498053in}{2.500039in}}%
\pgfpathlineto{\pgfqpoint{4.498448in}{2.508263in}}%
\pgfpathlineto{\pgfqpoint{4.499238in}{2.506066in}}%
\pgfpathlineto{\pgfqpoint{4.501409in}{2.473534in}}%
\pgfpathlineto{\pgfqpoint{4.501607in}{2.479855in}}%
\pgfpathlineto{\pgfqpoint{4.501903in}{2.490967in}}%
\pgfpathlineto{\pgfqpoint{4.502297in}{2.478124in}}%
\pgfpathlineto{\pgfqpoint{4.502791in}{2.485823in}}%
\pgfpathlineto{\pgfqpoint{4.503383in}{2.472666in}}%
\pgfpathlineto{\pgfqpoint{4.504271in}{2.477689in}}%
\pgfpathlineto{\pgfqpoint{4.504568in}{2.484257in}}%
\pgfpathlineto{\pgfqpoint{4.504962in}{2.470281in}}%
\pgfpathlineto{\pgfqpoint{4.505949in}{2.458951in}}%
\pgfpathlineto{\pgfqpoint{4.505456in}{2.477686in}}%
\pgfpathlineto{\pgfqpoint{4.506147in}{2.463825in}}%
\pgfpathlineto{\pgfqpoint{4.506245in}{2.465223in}}%
\pgfpathlineto{\pgfqpoint{4.506541in}{2.456325in}}%
\pgfpathlineto{\pgfqpoint{4.506640in}{2.454894in}}%
\pgfpathlineto{\pgfqpoint{4.507035in}{2.464123in}}%
\pgfpathlineto{\pgfqpoint{4.507134in}{2.464159in}}%
\pgfpathlineto{\pgfqpoint{4.507232in}{2.463186in}}%
\pgfpathlineto{\pgfqpoint{4.507331in}{2.462642in}}%
\pgfpathlineto{\pgfqpoint{4.507627in}{2.465677in}}%
\pgfpathlineto{\pgfqpoint{4.508022in}{2.473893in}}%
\pgfpathlineto{\pgfqpoint{4.508417in}{2.463168in}}%
\pgfpathlineto{\pgfqpoint{4.509700in}{2.437688in}}%
\pgfpathlineto{\pgfqpoint{4.509799in}{2.438516in}}%
\pgfpathlineto{\pgfqpoint{4.510095in}{2.442810in}}%
\pgfpathlineto{\pgfqpoint{4.510391in}{2.431671in}}%
\pgfpathlineto{\pgfqpoint{4.511674in}{2.412719in}}%
\pgfpathlineto{\pgfqpoint{4.511773in}{2.414165in}}%
\pgfpathlineto{\pgfqpoint{4.512069in}{2.426267in}}%
\pgfpathlineto{\pgfqpoint{4.512562in}{2.412563in}}%
\pgfpathlineto{\pgfqpoint{4.512957in}{2.419347in}}%
\pgfpathlineto{\pgfqpoint{4.513056in}{2.420706in}}%
\pgfpathlineto{\pgfqpoint{4.513352in}{2.410045in}}%
\pgfpathlineto{\pgfqpoint{4.513549in}{2.400375in}}%
\pgfpathlineto{\pgfqpoint{4.514437in}{2.410824in}}%
\pgfpathlineto{\pgfqpoint{4.514536in}{2.411956in}}%
\pgfpathlineto{\pgfqpoint{4.514832in}{2.404019in}}%
\pgfpathlineto{\pgfqpoint{4.514931in}{2.402532in}}%
\pgfpathlineto{\pgfqpoint{4.515227in}{2.413450in}}%
\pgfpathlineto{\pgfqpoint{4.515424in}{2.418349in}}%
\pgfpathlineto{\pgfqpoint{4.516115in}{2.411824in}}%
\pgfpathlineto{\pgfqpoint{4.516411in}{2.417163in}}%
\pgfpathlineto{\pgfqpoint{4.516905in}{2.417055in}}%
\pgfpathlineto{\pgfqpoint{4.518089in}{2.431836in}}%
\pgfpathlineto{\pgfqpoint{4.518484in}{2.417493in}}%
\pgfpathlineto{\pgfqpoint{4.519372in}{2.425824in}}%
\pgfpathlineto{\pgfqpoint{4.520853in}{2.452390in}}%
\pgfpathlineto{\pgfqpoint{4.522235in}{2.466946in}}%
\pgfpathlineto{\pgfqpoint{4.521741in}{2.451903in}}%
\pgfpathlineto{\pgfqpoint{4.522333in}{2.465279in}}%
\pgfpathlineto{\pgfqpoint{4.525196in}{2.402644in}}%
\pgfpathlineto{\pgfqpoint{4.525492in}{2.406491in}}%
\pgfpathlineto{\pgfqpoint{4.526183in}{2.402343in}}%
\pgfpathlineto{\pgfqpoint{4.526479in}{2.389792in}}%
\pgfpathlineto{\pgfqpoint{4.526972in}{2.404464in}}%
\pgfpathlineto{\pgfqpoint{4.527268in}{2.403213in}}%
\pgfpathlineto{\pgfqpoint{4.527564in}{2.401185in}}%
\pgfpathlineto{\pgfqpoint{4.527860in}{2.406201in}}%
\pgfpathlineto{\pgfqpoint{4.527959in}{2.406836in}}%
\pgfpathlineto{\pgfqpoint{4.528058in}{2.404250in}}%
\pgfpathlineto{\pgfqpoint{4.528354in}{2.394543in}}%
\pgfpathlineto{\pgfqpoint{4.529144in}{2.400856in}}%
\pgfpathlineto{\pgfqpoint{4.529440in}{2.413567in}}%
\pgfpathlineto{\pgfqpoint{4.530328in}{2.409368in}}%
\pgfpathlineto{\pgfqpoint{4.531414in}{2.392292in}}%
\pgfpathlineto{\pgfqpoint{4.530920in}{2.412703in}}%
\pgfpathlineto{\pgfqpoint{4.531611in}{2.400993in}}%
\pgfpathlineto{\pgfqpoint{4.532401in}{2.572300in}}%
\pgfpathlineto{\pgfqpoint{4.533388in}{2.673844in}}%
\pgfpathlineto{\pgfqpoint{4.533782in}{2.660077in}}%
\pgfpathlineto{\pgfqpoint{4.534079in}{2.649067in}}%
\pgfpathlineto{\pgfqpoint{4.535954in}{2.414582in}}%
\pgfpathlineto{\pgfqpoint{4.536349in}{2.429267in}}%
\pgfpathlineto{\pgfqpoint{4.537434in}{2.450207in}}%
\pgfpathlineto{\pgfqpoint{4.538026in}{2.527227in}}%
\pgfpathlineto{\pgfqpoint{4.538717in}{2.482216in}}%
\pgfpathlineto{\pgfqpoint{4.539310in}{2.425234in}}%
\pgfpathlineto{\pgfqpoint{4.540000in}{2.455959in}}%
\pgfpathlineto{\pgfqpoint{4.540198in}{2.453225in}}%
\pgfpathlineto{\pgfqpoint{4.540494in}{2.465146in}}%
\pgfpathlineto{\pgfqpoint{4.541382in}{2.488675in}}%
\pgfpathlineto{\pgfqpoint{4.541678in}{2.474252in}}%
\pgfpathlineto{\pgfqpoint{4.542369in}{2.433575in}}%
\pgfpathlineto{\pgfqpoint{4.543060in}{2.439325in}}%
\pgfpathlineto{\pgfqpoint{4.544442in}{2.478412in}}%
\pgfpathlineto{\pgfqpoint{4.544639in}{2.473090in}}%
\pgfpathlineto{\pgfqpoint{4.545725in}{2.439453in}}%
\pgfpathlineto{\pgfqpoint{4.545922in}{2.450890in}}%
\pgfpathlineto{\pgfqpoint{4.547008in}{2.482553in}}%
\pgfpathlineto{\pgfqpoint{4.547304in}{2.479231in}}%
\pgfpathlineto{\pgfqpoint{4.547403in}{2.478911in}}%
\pgfpathlineto{\pgfqpoint{4.547600in}{2.481084in}}%
\pgfpathlineto{\pgfqpoint{4.547798in}{2.484375in}}%
\pgfpathlineto{\pgfqpoint{4.548192in}{2.470197in}}%
\pgfpathlineto{\pgfqpoint{4.549081in}{2.457847in}}%
\pgfpathlineto{\pgfqpoint{4.549377in}{2.466446in}}%
\pgfpathlineto{\pgfqpoint{4.549969in}{2.480641in}}%
\pgfpathlineto{\pgfqpoint{4.550463in}{2.469266in}}%
\pgfpathlineto{\pgfqpoint{4.551351in}{2.471300in}}%
\pgfpathlineto{\pgfqpoint{4.552042in}{2.445426in}}%
\pgfpathlineto{\pgfqpoint{4.554509in}{2.379360in}}%
\pgfpathlineto{\pgfqpoint{4.555101in}{2.406148in}}%
\pgfpathlineto{\pgfqpoint{4.557470in}{2.480844in}}%
\pgfpathlineto{\pgfqpoint{4.557569in}{2.478927in}}%
\pgfpathlineto{\pgfqpoint{4.559444in}{2.428726in}}%
\pgfpathlineto{\pgfqpoint{4.559938in}{2.444532in}}%
\pgfpathlineto{\pgfqpoint{4.561221in}{2.438565in}}%
\pgfpathlineto{\pgfqpoint{4.562109in}{2.427971in}}%
\pgfpathlineto{\pgfqpoint{4.561616in}{2.439249in}}%
\pgfpathlineto{\pgfqpoint{4.562306in}{2.435489in}}%
\pgfpathlineto{\pgfqpoint{4.562504in}{2.440777in}}%
\pgfpathlineto{\pgfqpoint{4.563392in}{2.434707in}}%
\pgfpathlineto{\pgfqpoint{4.563589in}{2.432305in}}%
\pgfpathlineto{\pgfqpoint{4.563886in}{2.443974in}}%
\pgfpathlineto{\pgfqpoint{4.564083in}{2.449298in}}%
\pgfpathlineto{\pgfqpoint{4.564675in}{2.432729in}}%
\pgfpathlineto{\pgfqpoint{4.565267in}{2.452630in}}%
\pgfpathlineto{\pgfqpoint{4.565662in}{2.431811in}}%
\pgfpathlineto{\pgfqpoint{4.565860in}{2.426873in}}%
\pgfpathlineto{\pgfqpoint{4.566353in}{2.447975in}}%
\pgfpathlineto{\pgfqpoint{4.566748in}{2.441058in}}%
\pgfpathlineto{\pgfqpoint{4.567044in}{2.433859in}}%
\pgfpathlineto{\pgfqpoint{4.567834in}{2.438244in}}%
\pgfpathlineto{\pgfqpoint{4.570202in}{2.470356in}}%
\pgfpathlineto{\pgfqpoint{4.570400in}{2.464698in}}%
\pgfpathlineto{\pgfqpoint{4.570893in}{2.467111in}}%
\pgfpathlineto{\pgfqpoint{4.573065in}{2.403766in}}%
\pgfpathlineto{\pgfqpoint{4.573163in}{2.402260in}}%
\pgfpathlineto{\pgfqpoint{4.573459in}{2.411318in}}%
\pgfpathlineto{\pgfqpoint{4.573558in}{2.413366in}}%
\pgfpathlineto{\pgfqpoint{4.573854in}{2.401766in}}%
\pgfpathlineto{\pgfqpoint{4.574052in}{2.396610in}}%
\pgfpathlineto{\pgfqpoint{4.574545in}{2.404829in}}%
\pgfpathlineto{\pgfqpoint{4.574841in}{2.403039in}}%
\pgfpathlineto{\pgfqpoint{4.575335in}{2.414659in}}%
\pgfpathlineto{\pgfqpoint{4.575828in}{2.402656in}}%
\pgfpathlineto{\pgfqpoint{4.575927in}{2.403388in}}%
\pgfpathlineto{\pgfqpoint{4.576124in}{2.404448in}}%
\pgfpathlineto{\pgfqpoint{4.576420in}{2.400749in}}%
\pgfpathlineto{\pgfqpoint{4.577605in}{2.385560in}}%
\pgfpathlineto{\pgfqpoint{4.577901in}{2.392238in}}%
\pgfpathlineto{\pgfqpoint{4.578987in}{2.397427in}}%
\pgfpathlineto{\pgfqpoint{4.578394in}{2.386081in}}%
\pgfpathlineto{\pgfqpoint{4.579085in}{2.395980in}}%
\pgfpathlineto{\pgfqpoint{4.579776in}{2.359410in}}%
\pgfpathlineto{\pgfqpoint{4.580072in}{2.393561in}}%
\pgfpathlineto{\pgfqpoint{4.581651in}{2.640900in}}%
\pgfpathlineto{\pgfqpoint{4.582244in}{2.612203in}}%
\pgfpathlineto{\pgfqpoint{4.582638in}{2.566237in}}%
\pgfpathlineto{\pgfqpoint{4.583329in}{2.390663in}}%
\pgfpathlineto{\pgfqpoint{4.584316in}{2.392759in}}%
\pgfpathlineto{\pgfqpoint{4.584514in}{2.388439in}}%
\pgfpathlineto{\pgfqpoint{4.584908in}{2.410334in}}%
\pgfpathlineto{\pgfqpoint{4.586389in}{2.493911in}}%
\pgfpathlineto{\pgfqpoint{4.586784in}{2.457101in}}%
\pgfpathlineto{\pgfqpoint{4.587771in}{2.392737in}}%
\pgfpathlineto{\pgfqpoint{4.588166in}{2.416283in}}%
\pgfpathlineto{\pgfqpoint{4.589745in}{2.461900in}}%
\pgfpathlineto{\pgfqpoint{4.590238in}{2.435470in}}%
\pgfpathlineto{\pgfqpoint{4.590929in}{2.412604in}}%
\pgfpathlineto{\pgfqpoint{4.591225in}{2.428342in}}%
\pgfpathlineto{\pgfqpoint{4.592607in}{2.459696in}}%
\pgfpathlineto{\pgfqpoint{4.593100in}{2.474423in}}%
\pgfpathlineto{\pgfqpoint{4.593495in}{2.456941in}}%
\pgfpathlineto{\pgfqpoint{4.593989in}{2.449003in}}%
\pgfpathlineto{\pgfqpoint{4.594482in}{2.460666in}}%
\pgfpathlineto{\pgfqpoint{4.595173in}{2.487556in}}%
\pgfpathlineto{\pgfqpoint{4.596160in}{2.485714in}}%
\pgfpathlineto{\pgfqpoint{4.596456in}{2.490109in}}%
\pgfpathlineto{\pgfqpoint{4.596654in}{2.486393in}}%
\pgfpathlineto{\pgfqpoint{4.597048in}{2.467875in}}%
\pgfpathlineto{\pgfqpoint{4.597542in}{2.489198in}}%
\pgfpathlineto{\pgfqpoint{4.597739in}{2.498537in}}%
\pgfpathlineto{\pgfqpoint{4.598233in}{2.479806in}}%
\pgfpathlineto{\pgfqpoint{4.598726in}{2.496990in}}%
\pgfpathlineto{\pgfqpoint{4.599220in}{2.499745in}}%
\pgfpathlineto{\pgfqpoint{4.599516in}{2.495391in}}%
\pgfpathlineto{\pgfqpoint{4.600799in}{2.477995in}}%
\pgfpathlineto{\pgfqpoint{4.600996in}{2.483830in}}%
\pgfpathlineto{\pgfqpoint{4.601391in}{2.495479in}}%
\pgfpathlineto{\pgfqpoint{4.602082in}{2.485265in}}%
\pgfpathlineto{\pgfqpoint{4.602279in}{2.483733in}}%
\pgfpathlineto{\pgfqpoint{4.603563in}{2.466025in}}%
\pgfpathlineto{\pgfqpoint{4.603069in}{2.484509in}}%
\pgfpathlineto{\pgfqpoint{4.603760in}{2.471226in}}%
\pgfpathlineto{\pgfqpoint{4.603957in}{2.474733in}}%
\pgfpathlineto{\pgfqpoint{4.604648in}{2.468317in}}%
\pgfpathlineto{\pgfqpoint{4.606425in}{2.446873in}}%
\pgfpathlineto{\pgfqpoint{4.606524in}{2.447252in}}%
\pgfpathlineto{\pgfqpoint{4.606721in}{2.448271in}}%
\pgfpathlineto{\pgfqpoint{4.607017in}{2.444439in}}%
\pgfpathlineto{\pgfqpoint{4.607412in}{2.446000in}}%
\pgfpathlineto{\pgfqpoint{4.608892in}{2.419395in}}%
\pgfpathlineto{\pgfqpoint{4.608399in}{2.447221in}}%
\pgfpathlineto{\pgfqpoint{4.609090in}{2.430588in}}%
\pgfpathlineto{\pgfqpoint{4.609287in}{2.438390in}}%
\pgfpathlineto{\pgfqpoint{4.610077in}{2.426254in}}%
\pgfpathlineto{\pgfqpoint{4.610274in}{2.422061in}}%
\pgfpathlineto{\pgfqpoint{4.610669in}{2.433757in}}%
\pgfpathlineto{\pgfqpoint{4.611064in}{2.427506in}}%
\pgfpathlineto{\pgfqpoint{4.611360in}{2.433218in}}%
\pgfpathlineto{\pgfqpoint{4.611755in}{2.417739in}}%
\pgfpathlineto{\pgfqpoint{4.611952in}{2.421289in}}%
\pgfpathlineto{\pgfqpoint{4.612742in}{2.428962in}}%
\pgfpathlineto{\pgfqpoint{4.612939in}{2.424378in}}%
\pgfpathlineto{\pgfqpoint{4.613235in}{2.416993in}}%
\pgfpathlineto{\pgfqpoint{4.613729in}{2.429563in}}%
\pgfpathlineto{\pgfqpoint{4.614025in}{2.423758in}}%
\pgfpathlineto{\pgfqpoint{4.614123in}{2.421910in}}%
\pgfpathlineto{\pgfqpoint{4.614617in}{2.427601in}}%
\pgfpathlineto{\pgfqpoint{4.615110in}{2.424029in}}%
\pgfpathlineto{\pgfqpoint{4.615209in}{2.423028in}}%
\pgfpathlineto{\pgfqpoint{4.615505in}{2.428619in}}%
\pgfpathlineto{\pgfqpoint{4.616097in}{2.428410in}}%
\pgfpathlineto{\pgfqpoint{4.617282in}{2.457631in}}%
\pgfpathlineto{\pgfqpoint{4.619552in}{2.469520in}}%
\pgfpathlineto{\pgfqpoint{4.617578in}{2.457256in}}%
\pgfpathlineto{\pgfqpoint{4.619749in}{2.464256in}}%
\pgfpathlineto{\pgfqpoint{4.621328in}{2.417897in}}%
\pgfpathlineto{\pgfqpoint{4.621427in}{2.418822in}}%
\pgfpathlineto{\pgfqpoint{4.621723in}{2.425640in}}%
\pgfpathlineto{\pgfqpoint{4.622118in}{2.412334in}}%
\pgfpathlineto{\pgfqpoint{4.622315in}{2.413679in}}%
\pgfpathlineto{\pgfqpoint{4.622414in}{2.414072in}}%
\pgfpathlineto{\pgfqpoint{4.622513in}{2.412167in}}%
\pgfpathlineto{\pgfqpoint{4.623796in}{2.400034in}}%
\pgfpathlineto{\pgfqpoint{4.623204in}{2.422304in}}%
\pgfpathlineto{\pgfqpoint{4.623895in}{2.400296in}}%
\pgfpathlineto{\pgfqpoint{4.624289in}{2.407076in}}%
\pgfpathlineto{\pgfqpoint{4.624684in}{2.397672in}}%
\pgfpathlineto{\pgfqpoint{4.625375in}{2.393264in}}%
\pgfpathlineto{\pgfqpoint{4.624980in}{2.399844in}}%
\pgfpathlineto{\pgfqpoint{4.625572in}{2.396875in}}%
\pgfpathlineto{\pgfqpoint{4.626461in}{2.400387in}}%
\pgfpathlineto{\pgfqpoint{4.626066in}{2.394622in}}%
\pgfpathlineto{\pgfqpoint{4.626658in}{2.397041in}}%
\pgfpathlineto{\pgfqpoint{4.626757in}{2.396596in}}%
\pgfpathlineto{\pgfqpoint{4.626856in}{2.398455in}}%
\pgfpathlineto{\pgfqpoint{4.627250in}{2.416619in}}%
\pgfpathlineto{\pgfqpoint{4.627645in}{2.388364in}}%
\pgfpathlineto{\pgfqpoint{4.628139in}{2.361812in}}%
\pgfpathlineto{\pgfqpoint{4.628632in}{2.397319in}}%
\pgfpathlineto{\pgfqpoint{4.629915in}{2.576839in}}%
\pgfpathlineto{\pgfqpoint{4.630409in}{2.555399in}}%
\pgfpathlineto{\pgfqpoint{4.631001in}{2.512256in}}%
\pgfpathlineto{\pgfqpoint{4.631593in}{2.360216in}}%
\pgfpathlineto{\pgfqpoint{4.632580in}{2.380689in}}%
\pgfpathlineto{\pgfqpoint{4.634751in}{2.468501in}}%
\pgfpathlineto{\pgfqpoint{4.635146in}{2.431371in}}%
\pgfpathlineto{\pgfqpoint{4.636035in}{2.369656in}}%
\pgfpathlineto{\pgfqpoint{4.636528in}{2.392676in}}%
\pgfpathlineto{\pgfqpoint{4.637712in}{2.444681in}}%
\pgfpathlineto{\pgfqpoint{4.638403in}{2.437596in}}%
\pgfpathlineto{\pgfqpoint{4.638995in}{2.382503in}}%
\pgfpathlineto{\pgfqpoint{4.639785in}{2.416507in}}%
\pgfpathlineto{\pgfqpoint{4.641266in}{2.443343in}}%
\pgfpathlineto{\pgfqpoint{4.641562in}{2.439325in}}%
\pgfpathlineto{\pgfqpoint{4.641956in}{2.423729in}}%
\pgfpathlineto{\pgfqpoint{4.642154in}{2.418189in}}%
\pgfpathlineto{\pgfqpoint{4.642549in}{2.446779in}}%
\pgfpathlineto{\pgfqpoint{4.642845in}{2.444382in}}%
\pgfpathlineto{\pgfqpoint{4.643141in}{2.461085in}}%
\pgfpathlineto{\pgfqpoint{4.644128in}{2.487832in}}%
\pgfpathlineto{\pgfqpoint{4.644424in}{2.478723in}}%
\pgfpathlineto{\pgfqpoint{4.645115in}{2.466233in}}%
\pgfpathlineto{\pgfqpoint{4.645411in}{2.479203in}}%
\pgfpathlineto{\pgfqpoint{4.646497in}{2.504041in}}%
\pgfpathlineto{\pgfqpoint{4.645904in}{2.477504in}}%
\pgfpathlineto{\pgfqpoint{4.646793in}{2.497734in}}%
\pgfpathlineto{\pgfqpoint{4.647089in}{2.503550in}}%
\pgfpathlineto{\pgfqpoint{4.647286in}{2.507415in}}%
\pgfpathlineto{\pgfqpoint{4.647681in}{2.492570in}}%
\pgfpathlineto{\pgfqpoint{4.647780in}{2.491227in}}%
\pgfpathlineto{\pgfqpoint{4.648174in}{2.499801in}}%
\pgfpathlineto{\pgfqpoint{4.648471in}{2.496161in}}%
\pgfpathlineto{\pgfqpoint{4.649260in}{2.501284in}}%
\pgfpathlineto{\pgfqpoint{4.650050in}{2.511073in}}%
\pgfpathlineto{\pgfqpoint{4.650247in}{2.503854in}}%
\pgfpathlineto{\pgfqpoint{4.650543in}{2.496173in}}%
\pgfpathlineto{\pgfqpoint{4.651234in}{2.504909in}}%
\pgfpathlineto{\pgfqpoint{4.651333in}{2.506260in}}%
\pgfpathlineto{\pgfqpoint{4.651728in}{2.497052in}}%
\pgfpathlineto{\pgfqpoint{4.654689in}{2.458523in}}%
\pgfpathlineto{\pgfqpoint{4.655380in}{2.463092in}}%
\pgfpathlineto{\pgfqpoint{4.656268in}{2.469815in}}%
\pgfpathlineto{\pgfqpoint{4.656465in}{2.466526in}}%
\pgfpathlineto{\pgfqpoint{4.656761in}{2.457729in}}%
\pgfpathlineto{\pgfqpoint{4.657353in}{2.476569in}}%
\pgfpathlineto{\pgfqpoint{4.657551in}{2.479302in}}%
\pgfpathlineto{\pgfqpoint{4.657946in}{2.465174in}}%
\pgfpathlineto{\pgfqpoint{4.659031in}{2.481702in}}%
\pgfpathlineto{\pgfqpoint{4.659426in}{2.491971in}}%
\pgfpathlineto{\pgfqpoint{4.660216in}{2.484624in}}%
\pgfpathlineto{\pgfqpoint{4.661203in}{2.470072in}}%
\pgfpathlineto{\pgfqpoint{4.660709in}{2.487432in}}%
\pgfpathlineto{\pgfqpoint{4.661400in}{2.478785in}}%
\pgfpathlineto{\pgfqpoint{4.661696in}{2.495139in}}%
\pgfpathlineto{\pgfqpoint{4.662585in}{2.484474in}}%
\pgfpathlineto{\pgfqpoint{4.664164in}{2.506403in}}%
\pgfpathlineto{\pgfqpoint{4.666039in}{2.553072in}}%
\pgfpathlineto{\pgfqpoint{4.666335in}{2.562890in}}%
\pgfpathlineto{\pgfqpoint{4.667223in}{2.560110in}}%
\pgfpathlineto{\pgfqpoint{4.667421in}{2.561870in}}%
\pgfpathlineto{\pgfqpoint{4.667717in}{2.554284in}}%
\pgfpathlineto{\pgfqpoint{4.670086in}{2.498706in}}%
\pgfpathlineto{\pgfqpoint{4.670184in}{2.500233in}}%
\pgfpathlineto{\pgfqpoint{4.670382in}{2.503326in}}%
\pgfpathlineto{\pgfqpoint{4.670777in}{2.492427in}}%
\pgfpathlineto{\pgfqpoint{4.671171in}{2.498311in}}%
\pgfpathlineto{\pgfqpoint{4.671369in}{2.497405in}}%
\pgfpathlineto{\pgfqpoint{4.671566in}{2.499872in}}%
\pgfpathlineto{\pgfqpoint{4.671764in}{2.504892in}}%
\pgfpathlineto{\pgfqpoint{4.672060in}{2.486851in}}%
\pgfpathlineto{\pgfqpoint{4.672257in}{2.476243in}}%
\pgfpathlineto{\pgfqpoint{4.672652in}{2.497004in}}%
\pgfpathlineto{\pgfqpoint{4.673145in}{2.485076in}}%
\pgfpathlineto{\pgfqpoint{4.675021in}{2.517174in}}%
\pgfpathlineto{\pgfqpoint{4.675119in}{2.516675in}}%
\pgfpathlineto{\pgfqpoint{4.675218in}{2.517590in}}%
\pgfpathlineto{\pgfqpoint{4.675613in}{2.537739in}}%
\pgfpathlineto{\pgfqpoint{4.675909in}{2.516305in}}%
\pgfpathlineto{\pgfqpoint{4.676205in}{2.483750in}}%
\pgfpathlineto{\pgfqpoint{4.676698in}{2.549802in}}%
\pgfpathlineto{\pgfqpoint{4.678080in}{2.767385in}}%
\pgfpathlineto{\pgfqpoint{4.678771in}{2.752497in}}%
\pgfpathlineto{\pgfqpoint{4.679265in}{2.642701in}}%
\pgfpathlineto{\pgfqpoint{4.679758in}{2.514112in}}%
\pgfpathlineto{\pgfqpoint{4.680646in}{2.517919in}}%
\pgfpathlineto{\pgfqpoint{4.680943in}{2.508617in}}%
\pgfpathlineto{\pgfqpoint{4.681436in}{2.528522in}}%
\pgfpathlineto{\pgfqpoint{4.683015in}{2.627260in}}%
\pgfpathlineto{\pgfqpoint{4.683509in}{2.568740in}}%
\pgfpathlineto{\pgfqpoint{4.684002in}{2.532148in}}%
\pgfpathlineto{\pgfqpoint{4.684594in}{2.551789in}}%
\pgfpathlineto{\pgfqpoint{4.686272in}{2.617687in}}%
\pgfpathlineto{\pgfqpoint{4.686470in}{2.609071in}}%
\pgfpathlineto{\pgfqpoint{4.687358in}{2.553563in}}%
\pgfpathlineto{\pgfqpoint{4.687753in}{2.573369in}}%
\pgfpathlineto{\pgfqpoint{4.689135in}{2.595468in}}%
\pgfpathlineto{\pgfqpoint{4.689332in}{2.604089in}}%
\pgfpathlineto{\pgfqpoint{4.689924in}{2.578305in}}%
\pgfpathlineto{\pgfqpoint{4.690319in}{2.564857in}}%
\pgfpathlineto{\pgfqpoint{4.691010in}{2.573746in}}%
\pgfpathlineto{\pgfqpoint{4.691503in}{2.600366in}}%
\pgfpathlineto{\pgfqpoint{4.692392in}{2.598303in}}%
\pgfpathlineto{\pgfqpoint{4.692589in}{2.604914in}}%
\pgfpathlineto{\pgfqpoint{4.693083in}{2.584658in}}%
\pgfpathlineto{\pgfqpoint{4.693181in}{2.583934in}}%
\pgfpathlineto{\pgfqpoint{4.693280in}{2.587059in}}%
\pgfpathlineto{\pgfqpoint{4.694662in}{2.601757in}}%
\pgfpathlineto{\pgfqpoint{4.694859in}{2.604654in}}%
\pgfpathlineto{\pgfqpoint{4.695353in}{2.598781in}}%
\pgfpathlineto{\pgfqpoint{4.695550in}{2.599348in}}%
\pgfpathlineto{\pgfqpoint{4.696636in}{2.573027in}}%
\pgfpathlineto{\pgfqpoint{4.697623in}{2.580014in}}%
\pgfpathlineto{\pgfqpoint{4.697919in}{2.596084in}}%
\pgfpathlineto{\pgfqpoint{4.698412in}{2.573722in}}%
\pgfpathlineto{\pgfqpoint{4.698708in}{2.582564in}}%
\pgfpathlineto{\pgfqpoint{4.698807in}{2.583454in}}%
\pgfpathlineto{\pgfqpoint{4.699103in}{2.578648in}}%
\pgfpathlineto{\pgfqpoint{4.699301in}{2.579051in}}%
\pgfpathlineto{\pgfqpoint{4.699695in}{2.562931in}}%
\pgfpathlineto{\pgfqpoint{4.701077in}{2.546249in}}%
\pgfpathlineto{\pgfqpoint{4.701176in}{2.546997in}}%
\pgfpathlineto{\pgfqpoint{4.701373in}{2.549148in}}%
\pgfpathlineto{\pgfqpoint{4.701571in}{2.541627in}}%
\pgfpathlineto{\pgfqpoint{4.704532in}{2.398790in}}%
\pgfpathlineto{\pgfqpoint{4.704926in}{2.395084in}}%
\pgfpathlineto{\pgfqpoint{4.705222in}{2.406988in}}%
\pgfpathlineto{\pgfqpoint{4.706802in}{2.453332in}}%
\pgfpathlineto{\pgfqpoint{4.707098in}{2.450262in}}%
\pgfpathlineto{\pgfqpoint{4.707295in}{2.453492in}}%
\pgfpathlineto{\pgfqpoint{4.707690in}{2.462413in}}%
\pgfpathlineto{\pgfqpoint{4.708282in}{2.450270in}}%
\pgfpathlineto{\pgfqpoint{4.710454in}{2.425908in}}%
\pgfpathlineto{\pgfqpoint{4.711539in}{2.431417in}}%
\pgfpathlineto{\pgfqpoint{4.711638in}{2.428710in}}%
\pgfpathlineto{\pgfqpoint{4.711934in}{2.416634in}}%
\pgfpathlineto{\pgfqpoint{4.712329in}{2.441338in}}%
\pgfpathlineto{\pgfqpoint{4.712921in}{2.432108in}}%
\pgfpathlineto{\pgfqpoint{4.713612in}{2.451955in}}%
\pgfpathlineto{\pgfqpoint{4.714796in}{2.460194in}}%
\pgfpathlineto{\pgfqpoint{4.714303in}{2.447258in}}%
\pgfpathlineto{\pgfqpoint{4.714895in}{2.459672in}}%
\pgfpathlineto{\pgfqpoint{4.719238in}{2.374351in}}%
\pgfpathlineto{\pgfqpoint{4.719534in}{2.381412in}}%
\pgfpathlineto{\pgfqpoint{4.719731in}{2.383648in}}%
\pgfpathlineto{\pgfqpoint{4.720225in}{2.375756in}}%
\pgfpathlineto{\pgfqpoint{4.720620in}{2.370349in}}%
\pgfpathlineto{\pgfqpoint{4.720817in}{2.377765in}}%
\pgfpathlineto{\pgfqpoint{4.721014in}{2.386603in}}%
\pgfpathlineto{\pgfqpoint{4.721705in}{2.368027in}}%
\pgfpathlineto{\pgfqpoint{4.722791in}{2.376092in}}%
\pgfpathlineto{\pgfqpoint{4.723087in}{2.383890in}}%
\pgfpathlineto{\pgfqpoint{4.723580in}{2.368971in}}%
\pgfpathlineto{\pgfqpoint{4.724271in}{2.334060in}}%
\pgfpathlineto{\pgfqpoint{4.724567in}{2.367234in}}%
\pgfpathlineto{\pgfqpoint{4.726147in}{2.599878in}}%
\pgfpathlineto{\pgfqpoint{4.726541in}{2.586694in}}%
\pgfpathlineto{\pgfqpoint{4.726936in}{2.546695in}}%
\pgfpathlineto{\pgfqpoint{4.728812in}{2.314031in}}%
\pgfpathlineto{\pgfqpoint{4.728910in}{2.315339in}}%
\pgfpathlineto{\pgfqpoint{4.730391in}{2.394905in}}%
\pgfpathlineto{\pgfqpoint{4.730786in}{2.431765in}}%
\pgfpathlineto{\pgfqpoint{4.731378in}{2.396894in}}%
\pgfpathlineto{\pgfqpoint{4.731970in}{2.344060in}}%
\pgfpathlineto{\pgfqpoint{4.732562in}{2.379050in}}%
\pgfpathlineto{\pgfqpoint{4.734141in}{2.432423in}}%
\pgfpathlineto{\pgfqpoint{4.734240in}{2.430821in}}%
\pgfpathlineto{\pgfqpoint{4.734931in}{2.374927in}}%
\pgfpathlineto{\pgfqpoint{4.735918in}{2.407844in}}%
\pgfpathlineto{\pgfqpoint{4.737300in}{2.440978in}}%
\pgfpathlineto{\pgfqpoint{4.737596in}{2.431431in}}%
\pgfpathlineto{\pgfqpoint{4.738484in}{2.402764in}}%
\pgfpathlineto{\pgfqpoint{4.738879in}{2.413512in}}%
\pgfpathlineto{\pgfqpoint{4.739767in}{2.441542in}}%
\pgfpathlineto{\pgfqpoint{4.740458in}{2.441214in}}%
\pgfpathlineto{\pgfqpoint{4.740557in}{2.441382in}}%
\pgfpathlineto{\pgfqpoint{4.740655in}{2.440645in}}%
\pgfpathlineto{\pgfqpoint{4.741938in}{2.425682in}}%
\pgfpathlineto{\pgfqpoint{4.742136in}{2.428543in}}%
\pgfpathlineto{\pgfqpoint{4.743123in}{2.449108in}}%
\pgfpathlineto{\pgfqpoint{4.743320in}{2.441588in}}%
\pgfpathlineto{\pgfqpoint{4.744406in}{2.421074in}}%
\pgfpathlineto{\pgfqpoint{4.743912in}{2.444691in}}%
\pgfpathlineto{\pgfqpoint{4.744702in}{2.426124in}}%
\pgfpathlineto{\pgfqpoint{4.746775in}{2.441285in}}%
\pgfpathlineto{\pgfqpoint{4.747170in}{2.440277in}}%
\pgfpathlineto{\pgfqpoint{4.747860in}{2.428643in}}%
\pgfpathlineto{\pgfqpoint{4.748453in}{2.436395in}}%
\pgfpathlineto{\pgfqpoint{4.748650in}{2.439903in}}%
\pgfpathlineto{\pgfqpoint{4.749242in}{2.431556in}}%
\pgfpathlineto{\pgfqpoint{4.749341in}{2.432313in}}%
\pgfpathlineto{\pgfqpoint{4.749538in}{2.434305in}}%
\pgfpathlineto{\pgfqpoint{4.749834in}{2.425812in}}%
\pgfpathlineto{\pgfqpoint{4.751710in}{2.392262in}}%
\pgfpathlineto{\pgfqpoint{4.751808in}{2.393414in}}%
\pgfpathlineto{\pgfqpoint{4.752104in}{2.404823in}}%
\pgfpathlineto{\pgfqpoint{4.752499in}{2.376101in}}%
\pgfpathlineto{\pgfqpoint{4.752598in}{2.376294in}}%
\pgfpathlineto{\pgfqpoint{4.752894in}{2.384631in}}%
\pgfpathlineto{\pgfqpoint{4.753585in}{2.375114in}}%
\pgfpathlineto{\pgfqpoint{4.754177in}{2.380188in}}%
\pgfpathlineto{\pgfqpoint{4.754868in}{2.368029in}}%
\pgfpathlineto{\pgfqpoint{4.755263in}{2.377976in}}%
\pgfpathlineto{\pgfqpoint{4.755658in}{2.367743in}}%
\pgfpathlineto{\pgfqpoint{4.756052in}{2.370487in}}%
\pgfpathlineto{\pgfqpoint{4.757039in}{2.356422in}}%
\pgfpathlineto{\pgfqpoint{4.756546in}{2.372842in}}%
\pgfpathlineto{\pgfqpoint{4.757336in}{2.365051in}}%
\pgfpathlineto{\pgfqpoint{4.759013in}{2.381100in}}%
\pgfpathlineto{\pgfqpoint{4.757829in}{2.362779in}}%
\pgfpathlineto{\pgfqpoint{4.759112in}{2.378101in}}%
\pgfpathlineto{\pgfqpoint{4.759408in}{2.364915in}}%
\pgfpathlineto{\pgfqpoint{4.760099in}{2.375482in}}%
\pgfpathlineto{\pgfqpoint{4.761678in}{2.411594in}}%
\pgfpathlineto{\pgfqpoint{4.761777in}{2.410841in}}%
\pgfpathlineto{\pgfqpoint{4.761974in}{2.409178in}}%
\pgfpathlineto{\pgfqpoint{4.762270in}{2.413585in}}%
\pgfpathlineto{\pgfqpoint{4.763159in}{2.421894in}}%
\pgfpathlineto{\pgfqpoint{4.763356in}{2.416399in}}%
\pgfpathlineto{\pgfqpoint{4.764541in}{2.392301in}}%
\pgfpathlineto{\pgfqpoint{4.764837in}{2.394559in}}%
\pgfpathlineto{\pgfqpoint{4.767600in}{2.346251in}}%
\pgfpathlineto{\pgfqpoint{4.767699in}{2.346313in}}%
\pgfpathlineto{\pgfqpoint{4.769081in}{2.367875in}}%
\pgfpathlineto{\pgfqpoint{4.768489in}{2.344325in}}%
\pgfpathlineto{\pgfqpoint{4.769278in}{2.360266in}}%
\pgfpathlineto{\pgfqpoint{4.769475in}{2.355425in}}%
\pgfpathlineto{\pgfqpoint{4.769969in}{2.364895in}}%
\pgfpathlineto{\pgfqpoint{4.770364in}{2.360177in}}%
\pgfpathlineto{\pgfqpoint{4.771548in}{2.367882in}}%
\pgfpathlineto{\pgfqpoint{4.771746in}{2.365839in}}%
\pgfpathlineto{\pgfqpoint{4.772338in}{2.319228in}}%
\pgfpathlineto{\pgfqpoint{4.772733in}{2.354844in}}%
\pgfpathlineto{\pgfqpoint{4.774016in}{2.551071in}}%
\pgfpathlineto{\pgfqpoint{4.774904in}{2.542872in}}%
\pgfpathlineto{\pgfqpoint{4.775299in}{2.476169in}}%
\pgfpathlineto{\pgfqpoint{4.775792in}{2.357346in}}%
\pgfpathlineto{\pgfqpoint{4.776582in}{2.391686in}}%
\pgfpathlineto{\pgfqpoint{4.777075in}{2.381968in}}%
\pgfpathlineto{\pgfqpoint{4.777273in}{2.391987in}}%
\pgfpathlineto{\pgfqpoint{4.778951in}{2.491900in}}%
\pgfpathlineto{\pgfqpoint{4.779148in}{2.480680in}}%
\pgfpathlineto{\pgfqpoint{4.780135in}{2.401993in}}%
\pgfpathlineto{\pgfqpoint{4.780628in}{2.429889in}}%
\pgfpathlineto{\pgfqpoint{4.782109in}{2.481182in}}%
\pgfpathlineto{\pgfqpoint{4.782405in}{2.474429in}}%
\pgfpathlineto{\pgfqpoint{4.783293in}{2.436742in}}%
\pgfpathlineto{\pgfqpoint{4.783886in}{2.454973in}}%
\pgfpathlineto{\pgfqpoint{4.785070in}{2.474946in}}%
\pgfpathlineto{\pgfqpoint{4.785563in}{2.474011in}}%
\pgfpathlineto{\pgfqpoint{4.785662in}{2.474995in}}%
\pgfpathlineto{\pgfqpoint{4.785860in}{2.468583in}}%
\pgfpathlineto{\pgfqpoint{4.786847in}{2.455067in}}%
\pgfpathlineto{\pgfqpoint{4.787044in}{2.462630in}}%
\pgfpathlineto{\pgfqpoint{4.788426in}{2.488347in}}%
\pgfpathlineto{\pgfqpoint{4.788919in}{2.489619in}}%
\pgfpathlineto{\pgfqpoint{4.789215in}{2.480708in}}%
\pgfpathlineto{\pgfqpoint{4.789807in}{2.464503in}}%
\pgfpathlineto{\pgfqpoint{4.790202in}{2.483897in}}%
\pgfpathlineto{\pgfqpoint{4.790498in}{2.494685in}}%
\pgfpathlineto{\pgfqpoint{4.791288in}{2.486592in}}%
\pgfpathlineto{\pgfqpoint{4.791485in}{2.482623in}}%
\pgfpathlineto{\pgfqpoint{4.791979in}{2.496980in}}%
\pgfpathlineto{\pgfqpoint{4.792078in}{2.496406in}}%
\pgfpathlineto{\pgfqpoint{4.792768in}{2.484352in}}%
\pgfpathlineto{\pgfqpoint{4.793163in}{2.496545in}}%
\pgfpathlineto{\pgfqpoint{4.793262in}{2.497126in}}%
\pgfpathlineto{\pgfqpoint{4.793459in}{2.491746in}}%
\pgfpathlineto{\pgfqpoint{4.793558in}{2.489331in}}%
\pgfpathlineto{\pgfqpoint{4.793854in}{2.502420in}}%
\pgfpathlineto{\pgfqpoint{4.794052in}{2.512027in}}%
\pgfpathlineto{\pgfqpoint{4.794545in}{2.497897in}}%
\pgfpathlineto{\pgfqpoint{4.794841in}{2.500284in}}%
\pgfpathlineto{\pgfqpoint{4.795039in}{2.499551in}}%
\pgfpathlineto{\pgfqpoint{4.795532in}{2.494203in}}%
\pgfpathlineto{\pgfqpoint{4.796026in}{2.499621in}}%
\pgfpathlineto{\pgfqpoint{4.796420in}{2.499300in}}%
\pgfpathlineto{\pgfqpoint{4.796716in}{2.505123in}}%
\pgfpathlineto{\pgfqpoint{4.796914in}{2.509401in}}%
\pgfpathlineto{\pgfqpoint{4.797210in}{2.493131in}}%
\pgfpathlineto{\pgfqpoint{4.798493in}{2.479975in}}%
\pgfpathlineto{\pgfqpoint{4.799677in}{2.466509in}}%
\pgfpathlineto{\pgfqpoint{4.799875in}{2.466717in}}%
\pgfpathlineto{\pgfqpoint{4.800270in}{2.469365in}}%
\pgfpathlineto{\pgfqpoint{4.800467in}{2.466641in}}%
\pgfpathlineto{\pgfqpoint{4.802244in}{2.438192in}}%
\pgfpathlineto{\pgfqpoint{4.802540in}{2.450221in}}%
\pgfpathlineto{\pgfqpoint{4.802934in}{2.434197in}}%
\pgfpathlineto{\pgfqpoint{4.803625in}{2.447671in}}%
\pgfpathlineto{\pgfqpoint{4.803921in}{2.450363in}}%
\pgfpathlineto{\pgfqpoint{4.804316in}{2.446531in}}%
\pgfpathlineto{\pgfqpoint{4.804711in}{2.447393in}}%
\pgfpathlineto{\pgfqpoint{4.805698in}{2.444991in}}%
\pgfpathlineto{\pgfqpoint{4.805205in}{2.448069in}}%
\pgfpathlineto{\pgfqpoint{4.805895in}{2.446012in}}%
\pgfpathlineto{\pgfqpoint{4.806488in}{2.449311in}}%
\pgfpathlineto{\pgfqpoint{4.806784in}{2.445621in}}%
\pgfpathlineto{\pgfqpoint{4.807080in}{2.441677in}}%
\pgfpathlineto{\pgfqpoint{4.807869in}{2.443385in}}%
\pgfpathlineto{\pgfqpoint{4.810534in}{2.496095in}}%
\pgfpathlineto{\pgfqpoint{4.810633in}{2.495366in}}%
\pgfpathlineto{\pgfqpoint{4.815371in}{2.429093in}}%
\pgfpathlineto{\pgfqpoint{4.815667in}{2.431615in}}%
\pgfpathlineto{\pgfqpoint{4.815864in}{2.432670in}}%
\pgfpathlineto{\pgfqpoint{4.816160in}{2.428253in}}%
\pgfpathlineto{\pgfqpoint{4.816555in}{2.430949in}}%
\pgfpathlineto{\pgfqpoint{4.816851in}{2.424281in}}%
\pgfpathlineto{\pgfqpoint{4.817246in}{2.437647in}}%
\pgfpathlineto{\pgfqpoint{4.817641in}{2.429105in}}%
\pgfpathlineto{\pgfqpoint{4.817937in}{2.431353in}}%
\pgfpathlineto{\pgfqpoint{4.818233in}{2.426867in}}%
\pgfpathlineto{\pgfqpoint{4.818331in}{2.426035in}}%
\pgfpathlineto{\pgfqpoint{4.818628in}{2.430309in}}%
\pgfpathlineto{\pgfqpoint{4.819615in}{2.440958in}}%
\pgfpathlineto{\pgfqpoint{4.819812in}{2.433541in}}%
\pgfpathlineto{\pgfqpoint{4.820207in}{2.403804in}}%
\pgfpathlineto{\pgfqpoint{4.820602in}{2.456377in}}%
\pgfpathlineto{\pgfqpoint{4.821983in}{2.678267in}}%
\pgfpathlineto{\pgfqpoint{4.822674in}{2.668958in}}%
\pgfpathlineto{\pgfqpoint{4.823365in}{2.486517in}}%
\pgfpathlineto{\pgfqpoint{4.824648in}{2.414332in}}%
\pgfpathlineto{\pgfqpoint{4.826030in}{2.462478in}}%
\pgfpathlineto{\pgfqpoint{4.826918in}{2.535480in}}%
\pgfpathlineto{\pgfqpoint{4.827313in}{2.493260in}}%
\pgfpathlineto{\pgfqpoint{4.828004in}{2.449601in}}%
\pgfpathlineto{\pgfqpoint{4.828596in}{2.461933in}}%
\pgfpathlineto{\pgfqpoint{4.828991in}{2.487417in}}%
\pgfpathlineto{\pgfqpoint{4.829978in}{2.522534in}}%
\pgfpathlineto{\pgfqpoint{4.830373in}{2.515323in}}%
\pgfpathlineto{\pgfqpoint{4.830965in}{2.456814in}}%
\pgfpathlineto{\pgfqpoint{4.831952in}{2.487226in}}%
\pgfpathlineto{\pgfqpoint{4.833038in}{2.495319in}}%
\pgfpathlineto{\pgfqpoint{4.832445in}{2.482943in}}%
\pgfpathlineto{\pgfqpoint{4.833334in}{2.493959in}}%
\pgfpathlineto{\pgfqpoint{4.834518in}{2.455782in}}%
\pgfpathlineto{\pgfqpoint{4.835110in}{2.480367in}}%
\pgfpathlineto{\pgfqpoint{4.835308in}{2.485940in}}%
\pgfpathlineto{\pgfqpoint{4.835801in}{2.480116in}}%
\pgfpathlineto{\pgfqpoint{4.836196in}{2.480112in}}%
\pgfpathlineto{\pgfqpoint{4.836393in}{2.479762in}}%
\pgfpathlineto{\pgfqpoint{4.836492in}{2.481048in}}%
\pgfpathlineto{\pgfqpoint{4.836788in}{2.485423in}}%
\pgfpathlineto{\pgfqpoint{4.836986in}{2.476329in}}%
\pgfpathlineto{\pgfqpoint{4.837380in}{2.452636in}}%
\pgfpathlineto{\pgfqpoint{4.838170in}{2.468511in}}%
\pgfpathlineto{\pgfqpoint{4.839058in}{2.461601in}}%
\pgfpathlineto{\pgfqpoint{4.840736in}{2.425223in}}%
\pgfpathlineto{\pgfqpoint{4.841131in}{2.428407in}}%
\pgfpathlineto{\pgfqpoint{4.841328in}{2.429201in}}%
\pgfpathlineto{\pgfqpoint{4.841723in}{2.433547in}}%
\pgfpathlineto{\pgfqpoint{4.842118in}{2.426826in}}%
\pgfpathlineto{\pgfqpoint{4.843401in}{2.403527in}}%
\pgfpathlineto{\pgfqpoint{4.843697in}{2.412542in}}%
\pgfpathlineto{\pgfqpoint{4.843895in}{2.415688in}}%
\pgfpathlineto{\pgfqpoint{4.844289in}{2.398113in}}%
\pgfpathlineto{\pgfqpoint{4.846855in}{2.308152in}}%
\pgfpathlineto{\pgfqpoint{4.847053in}{2.314308in}}%
\pgfpathlineto{\pgfqpoint{4.849224in}{2.388342in}}%
\pgfpathlineto{\pgfqpoint{4.849323in}{2.385934in}}%
\pgfpathlineto{\pgfqpoint{4.851100in}{2.345343in}}%
\pgfpathlineto{\pgfqpoint{4.851198in}{2.345193in}}%
\pgfpathlineto{\pgfqpoint{4.851593in}{2.366585in}}%
\pgfpathlineto{\pgfqpoint{4.852383in}{2.359420in}}%
\pgfpathlineto{\pgfqpoint{4.852679in}{2.349703in}}%
\pgfpathlineto{\pgfqpoint{4.853468in}{2.360754in}}%
\pgfpathlineto{\pgfqpoint{4.854554in}{2.374505in}}%
\pgfpathlineto{\pgfqpoint{4.854060in}{2.350580in}}%
\pgfpathlineto{\pgfqpoint{4.854751in}{2.370391in}}%
\pgfpathlineto{\pgfqpoint{4.855047in}{2.360723in}}%
\pgfpathlineto{\pgfqpoint{4.855640in}{2.378423in}}%
\pgfpathlineto{\pgfqpoint{4.857021in}{2.408046in}}%
\pgfpathlineto{\pgfqpoint{4.858403in}{2.444884in}}%
\pgfpathlineto{\pgfqpoint{4.858798in}{2.437820in}}%
\pgfpathlineto{\pgfqpoint{4.859785in}{2.448457in}}%
\pgfpathlineto{\pgfqpoint{4.860180in}{2.441802in}}%
\pgfpathlineto{\pgfqpoint{4.861167in}{2.422494in}}%
\pgfpathlineto{\pgfqpoint{4.861463in}{2.429377in}}%
\pgfpathlineto{\pgfqpoint{4.861759in}{2.438158in}}%
\pgfpathlineto{\pgfqpoint{4.862647in}{2.435746in}}%
\pgfpathlineto{\pgfqpoint{4.862943in}{2.430961in}}%
\pgfpathlineto{\pgfqpoint{4.863338in}{2.440079in}}%
\pgfpathlineto{\pgfqpoint{4.863536in}{2.439810in}}%
\pgfpathlineto{\pgfqpoint{4.863733in}{2.442637in}}%
\pgfpathlineto{\pgfqpoint{4.864226in}{2.462498in}}%
\pgfpathlineto{\pgfqpoint{4.865016in}{2.454652in}}%
\pgfpathlineto{\pgfqpoint{4.865411in}{2.460600in}}%
\pgfpathlineto{\pgfqpoint{4.866398in}{2.486310in}}%
\pgfpathlineto{\pgfqpoint{4.866694in}{2.470205in}}%
\pgfpathlineto{\pgfqpoint{4.867977in}{2.450563in}}%
\pgfpathlineto{\pgfqpoint{4.867187in}{2.481936in}}%
\pgfpathlineto{\pgfqpoint{4.868076in}{2.452652in}}%
\pgfpathlineto{\pgfqpoint{4.868767in}{2.553407in}}%
\pgfpathlineto{\pgfqpoint{4.869852in}{2.736324in}}%
\pgfpathlineto{\pgfqpoint{4.870346in}{2.709608in}}%
\pgfpathlineto{\pgfqpoint{4.870445in}{2.711039in}}%
\pgfpathlineto{\pgfqpoint{4.870642in}{2.705234in}}%
\pgfpathlineto{\pgfqpoint{4.872715in}{2.455880in}}%
\pgfpathlineto{\pgfqpoint{4.873307in}{2.480687in}}%
\pgfpathlineto{\pgfqpoint{4.874590in}{2.560586in}}%
\pgfpathlineto{\pgfqpoint{4.875083in}{2.529390in}}%
\pgfpathlineto{\pgfqpoint{4.876070in}{2.456029in}}%
\pgfpathlineto{\pgfqpoint{4.876465in}{2.486597in}}%
\pgfpathlineto{\pgfqpoint{4.877847in}{2.539339in}}%
\pgfpathlineto{\pgfqpoint{4.878143in}{2.524725in}}%
\pgfpathlineto{\pgfqpoint{4.879130in}{2.469262in}}%
\pgfpathlineto{\pgfqpoint{4.879624in}{2.484920in}}%
\pgfpathlineto{\pgfqpoint{4.881005in}{2.510751in}}%
\pgfpathlineto{\pgfqpoint{4.881203in}{2.507593in}}%
\pgfpathlineto{\pgfqpoint{4.882091in}{2.482228in}}%
\pgfpathlineto{\pgfqpoint{4.882683in}{2.498694in}}%
\pgfpathlineto{\pgfqpoint{4.882979in}{2.506279in}}%
\pgfpathlineto{\pgfqpoint{4.884460in}{2.528961in}}%
\pgfpathlineto{\pgfqpoint{4.885545in}{2.508187in}}%
\pgfpathlineto{\pgfqpoint{4.885842in}{2.518307in}}%
\pgfpathlineto{\pgfqpoint{4.885940in}{2.520093in}}%
\pgfpathlineto{\pgfqpoint{4.886730in}{2.516286in}}%
\pgfpathlineto{\pgfqpoint{4.886829in}{2.516298in}}%
\pgfpathlineto{\pgfqpoint{4.887816in}{2.533816in}}%
\pgfpathlineto{\pgfqpoint{4.888112in}{2.517793in}}%
\pgfpathlineto{\pgfqpoint{4.888408in}{2.504760in}}%
\pgfpathlineto{\pgfqpoint{4.889296in}{2.512202in}}%
\pgfpathlineto{\pgfqpoint{4.889691in}{2.528363in}}%
\pgfpathlineto{\pgfqpoint{4.890382in}{2.514136in}}%
\pgfpathlineto{\pgfqpoint{4.892257in}{2.486509in}}%
\pgfpathlineto{\pgfqpoint{4.892948in}{2.495943in}}%
\pgfpathlineto{\pgfqpoint{4.893047in}{2.496322in}}%
\pgfpathlineto{\pgfqpoint{4.893145in}{2.494989in}}%
\pgfpathlineto{\pgfqpoint{4.895613in}{2.432232in}}%
\pgfpathlineto{\pgfqpoint{4.895810in}{2.438547in}}%
\pgfpathlineto{\pgfqpoint{4.895909in}{2.440757in}}%
\pgfpathlineto{\pgfqpoint{4.896304in}{2.430158in}}%
\pgfpathlineto{\pgfqpoint{4.896698in}{2.434076in}}%
\pgfpathlineto{\pgfqpoint{4.897784in}{2.416648in}}%
\pgfpathlineto{\pgfqpoint{4.898080in}{2.420852in}}%
\pgfpathlineto{\pgfqpoint{4.898574in}{2.433023in}}%
\pgfpathlineto{\pgfqpoint{4.899067in}{2.419646in}}%
\pgfpathlineto{\pgfqpoint{4.900350in}{2.429185in}}%
\pgfpathlineto{\pgfqpoint{4.900646in}{2.422870in}}%
\pgfpathlineto{\pgfqpoint{4.901831in}{2.412514in}}%
\pgfpathlineto{\pgfqpoint{4.901337in}{2.427200in}}%
\pgfpathlineto{\pgfqpoint{4.901929in}{2.412861in}}%
\pgfpathlineto{\pgfqpoint{4.902324in}{2.421218in}}%
\pgfpathlineto{\pgfqpoint{4.903114in}{2.415631in}}%
\pgfpathlineto{\pgfqpoint{4.903213in}{2.414245in}}%
\pgfpathlineto{\pgfqpoint{4.903509in}{2.422665in}}%
\pgfpathlineto{\pgfqpoint{4.904594in}{2.438613in}}%
\pgfpathlineto{\pgfqpoint{4.904890in}{2.433537in}}%
\pgfpathlineto{\pgfqpoint{4.905088in}{2.430843in}}%
\pgfpathlineto{\pgfqpoint{4.905285in}{2.439912in}}%
\pgfpathlineto{\pgfqpoint{4.905680in}{2.461504in}}%
\pgfpathlineto{\pgfqpoint{4.906470in}{2.453208in}}%
\pgfpathlineto{\pgfqpoint{4.908444in}{2.417547in}}%
\pgfpathlineto{\pgfqpoint{4.908641in}{2.416262in}}%
\pgfpathlineto{\pgfqpoint{4.909135in}{2.396204in}}%
\pgfpathlineto{\pgfqpoint{4.910121in}{2.404038in}}%
\pgfpathlineto{\pgfqpoint{4.912984in}{2.431481in}}%
\pgfpathlineto{\pgfqpoint{4.913181in}{2.437118in}}%
\pgfpathlineto{\pgfqpoint{4.913576in}{2.425559in}}%
\pgfpathlineto{\pgfqpoint{4.914168in}{2.435026in}}%
\pgfpathlineto{\pgfqpoint{4.914464in}{2.434399in}}%
\pgfpathlineto{\pgfqpoint{4.914859in}{2.437920in}}%
\pgfpathlineto{\pgfqpoint{4.914958in}{2.439121in}}%
\pgfpathlineto{\pgfqpoint{4.915353in}{2.431526in}}%
\pgfpathlineto{\pgfqpoint{4.915945in}{2.419301in}}%
\pgfpathlineto{\pgfqpoint{4.916142in}{2.430443in}}%
\pgfpathlineto{\pgfqpoint{4.917820in}{2.691686in}}%
\pgfpathlineto{\pgfqpoint{4.918610in}{2.644248in}}%
\pgfpathlineto{\pgfqpoint{4.920485in}{2.421667in}}%
\pgfpathlineto{\pgfqpoint{4.920682in}{2.425669in}}%
\pgfpathlineto{\pgfqpoint{4.922558in}{2.488544in}}%
\pgfpathlineto{\pgfqpoint{4.922755in}{2.476990in}}%
\pgfpathlineto{\pgfqpoint{4.923545in}{2.408416in}}%
\pgfpathlineto{\pgfqpoint{4.924038in}{2.446288in}}%
\pgfpathlineto{\pgfqpoint{4.925815in}{2.556508in}}%
\pgfpathlineto{\pgfqpoint{4.926012in}{2.552677in}}%
\pgfpathlineto{\pgfqpoint{4.926999in}{2.479650in}}%
\pgfpathlineto{\pgfqpoint{4.927789in}{2.508028in}}%
\pgfpathlineto{\pgfqpoint{4.928381in}{2.517952in}}%
\pgfpathlineto{\pgfqpoint{4.928874in}{2.536653in}}%
\pgfpathlineto{\pgfqpoint{4.929368in}{2.516677in}}%
\pgfpathlineto{\pgfqpoint{4.930059in}{2.501913in}}%
\pgfpathlineto{\pgfqpoint{4.930552in}{2.512808in}}%
\pgfpathlineto{\pgfqpoint{4.931835in}{2.535872in}}%
\pgfpathlineto{\pgfqpoint{4.932131in}{2.531555in}}%
\pgfpathlineto{\pgfqpoint{4.932230in}{2.530646in}}%
\pgfpathlineto{\pgfqpoint{4.932526in}{2.538088in}}%
\pgfpathlineto{\pgfqpoint{4.932625in}{2.540024in}}%
\pgfpathlineto{\pgfqpoint{4.932921in}{2.529485in}}%
\pgfpathlineto{\pgfqpoint{4.933118in}{2.523356in}}%
\pgfpathlineto{\pgfqpoint{4.933809in}{2.531513in}}%
\pgfpathlineto{\pgfqpoint{4.934105in}{2.542296in}}%
\pgfpathlineto{\pgfqpoint{4.935092in}{2.541459in}}%
\pgfpathlineto{\pgfqpoint{4.935586in}{2.544314in}}%
\pgfpathlineto{\pgfqpoint{4.935882in}{2.540992in}}%
\pgfpathlineto{\pgfqpoint{4.936869in}{2.520457in}}%
\pgfpathlineto{\pgfqpoint{4.937165in}{2.530506in}}%
\pgfpathlineto{\pgfqpoint{4.937461in}{2.540259in}}%
\pgfpathlineto{\pgfqpoint{4.937757in}{2.529496in}}%
\pgfpathlineto{\pgfqpoint{4.938251in}{2.536506in}}%
\pgfpathlineto{\pgfqpoint{4.938645in}{2.517251in}}%
\pgfpathlineto{\pgfqpoint{4.939534in}{2.522573in}}%
\pgfpathlineto{\pgfqpoint{4.940126in}{2.522844in}}%
\pgfpathlineto{\pgfqpoint{4.942495in}{2.486687in}}%
\pgfpathlineto{\pgfqpoint{4.943778in}{2.461024in}}%
\pgfpathlineto{\pgfqpoint{4.943975in}{2.465909in}}%
\pgfpathlineto{\pgfqpoint{4.944271in}{2.474111in}}%
\pgfpathlineto{\pgfqpoint{4.944666in}{2.449224in}}%
\pgfpathlineto{\pgfqpoint{4.946147in}{2.436472in}}%
\pgfpathlineto{\pgfqpoint{4.946640in}{2.447748in}}%
\pgfpathlineto{\pgfqpoint{4.947035in}{2.436467in}}%
\pgfpathlineto{\pgfqpoint{4.947923in}{2.430161in}}%
\pgfpathlineto{\pgfqpoint{4.948121in}{2.434340in}}%
\pgfpathlineto{\pgfqpoint{4.948910in}{2.441005in}}%
\pgfpathlineto{\pgfqpoint{4.949108in}{2.436934in}}%
\pgfpathlineto{\pgfqpoint{4.950193in}{2.417392in}}%
\pgfpathlineto{\pgfqpoint{4.950391in}{2.421384in}}%
\pgfpathlineto{\pgfqpoint{4.953648in}{2.476316in}}%
\pgfpathlineto{\pgfqpoint{4.954733in}{2.467008in}}%
\pgfpathlineto{\pgfqpoint{4.954832in}{2.467326in}}%
\pgfpathlineto{\pgfqpoint{4.954931in}{2.465954in}}%
\pgfpathlineto{\pgfqpoint{4.957694in}{2.402957in}}%
\pgfpathlineto{\pgfqpoint{4.957990in}{2.409308in}}%
\pgfpathlineto{\pgfqpoint{4.958188in}{2.414089in}}%
\pgfpathlineto{\pgfqpoint{4.958780in}{2.404391in}}%
\pgfpathlineto{\pgfqpoint{4.958977in}{2.405349in}}%
\pgfpathlineto{\pgfqpoint{4.959175in}{2.405126in}}%
\pgfpathlineto{\pgfqpoint{4.959372in}{2.405660in}}%
\pgfpathlineto{\pgfqpoint{4.960655in}{2.421014in}}%
\pgfpathlineto{\pgfqpoint{4.960853in}{2.413747in}}%
\pgfpathlineto{\pgfqpoint{4.961938in}{2.398936in}}%
\pgfpathlineto{\pgfqpoint{4.961445in}{2.415204in}}%
\pgfpathlineto{\pgfqpoint{4.962136in}{2.402932in}}%
\pgfpathlineto{\pgfqpoint{4.962728in}{2.418480in}}%
\pgfpathlineto{\pgfqpoint{4.962925in}{2.429811in}}%
\pgfpathlineto{\pgfqpoint{4.963419in}{2.411057in}}%
\pgfpathlineto{\pgfqpoint{4.963616in}{2.412646in}}%
\pgfpathlineto{\pgfqpoint{4.964110in}{2.379437in}}%
\pgfpathlineto{\pgfqpoint{4.964505in}{2.418030in}}%
\pgfpathlineto{\pgfqpoint{4.965886in}{2.653364in}}%
\pgfpathlineto{\pgfqpoint{4.966676in}{2.640353in}}%
\pgfpathlineto{\pgfqpoint{4.967663in}{2.395153in}}%
\pgfpathlineto{\pgfqpoint{4.969242in}{2.418758in}}%
\pgfpathlineto{\pgfqpoint{4.970723in}{2.518753in}}%
\pgfpathlineto{\pgfqpoint{4.971315in}{2.469429in}}%
\pgfpathlineto{\pgfqpoint{4.971808in}{2.426912in}}%
\pgfpathlineto{\pgfqpoint{4.972499in}{2.450895in}}%
\pgfpathlineto{\pgfqpoint{4.974078in}{2.503043in}}%
\pgfpathlineto{\pgfqpoint{4.974276in}{2.483216in}}%
\pgfpathlineto{\pgfqpoint{4.975263in}{2.425128in}}%
\pgfpathlineto{\pgfqpoint{4.975559in}{2.437983in}}%
\pgfpathlineto{\pgfqpoint{4.977237in}{2.467422in}}%
\pgfpathlineto{\pgfqpoint{4.977434in}{2.468486in}}%
\pgfpathlineto{\pgfqpoint{4.977730in}{2.463717in}}%
\pgfpathlineto{\pgfqpoint{4.978224in}{2.449068in}}%
\pgfpathlineto{\pgfqpoint{4.978717in}{2.463633in}}%
\pgfpathlineto{\pgfqpoint{4.980593in}{2.503657in}}%
\pgfpathlineto{\pgfqpoint{4.980691in}{2.505822in}}%
\pgfpathlineto{\pgfqpoint{4.980889in}{2.496581in}}%
\pgfpathlineto{\pgfqpoint{4.981185in}{2.476172in}}%
\pgfpathlineto{\pgfqpoint{4.981974in}{2.487681in}}%
\pgfpathlineto{\pgfqpoint{4.982369in}{2.499004in}}%
\pgfpathlineto{\pgfqpoint{4.983060in}{2.492952in}}%
\pgfpathlineto{\pgfqpoint{4.984442in}{2.459289in}}%
\pgfpathlineto{\pgfqpoint{4.985034in}{2.463790in}}%
\pgfpathlineto{\pgfqpoint{4.986021in}{2.473741in}}%
\pgfpathlineto{\pgfqpoint{4.985626in}{2.459255in}}%
\pgfpathlineto{\pgfqpoint{4.986120in}{2.470479in}}%
\pgfpathlineto{\pgfqpoint{4.986416in}{2.460104in}}%
\pgfpathlineto{\pgfqpoint{4.986811in}{2.470703in}}%
\pgfpathlineto{\pgfqpoint{4.987304in}{2.464232in}}%
\pgfpathlineto{\pgfqpoint{4.988291in}{2.449572in}}%
\pgfpathlineto{\pgfqpoint{4.988587in}{2.458643in}}%
\pgfpathlineto{\pgfqpoint{4.988686in}{2.459417in}}%
\pgfpathlineto{\pgfqpoint{4.988883in}{2.454304in}}%
\pgfpathlineto{\pgfqpoint{4.990166in}{2.443468in}}%
\pgfpathlineto{\pgfqpoint{4.989475in}{2.456888in}}%
\pgfpathlineto{\pgfqpoint{4.990265in}{2.444799in}}%
\pgfpathlineto{\pgfqpoint{4.990561in}{2.448920in}}%
\pgfpathlineto{\pgfqpoint{4.990857in}{2.437710in}}%
\pgfpathlineto{\pgfqpoint{4.991844in}{2.433648in}}%
\pgfpathlineto{\pgfqpoint{4.991351in}{2.444077in}}%
\pgfpathlineto{\pgfqpoint{4.992042in}{2.435586in}}%
\pgfpathlineto{\pgfqpoint{4.992436in}{2.440596in}}%
\pgfpathlineto{\pgfqpoint{4.992634in}{2.435884in}}%
\pgfpathlineto{\pgfqpoint{4.993720in}{2.410189in}}%
\pgfpathlineto{\pgfqpoint{4.994114in}{2.411377in}}%
\pgfpathlineto{\pgfqpoint{4.994410in}{2.405526in}}%
\pgfpathlineto{\pgfqpoint{4.996680in}{2.322908in}}%
\pgfpathlineto{\pgfqpoint{4.996977in}{2.332438in}}%
\pgfpathlineto{\pgfqpoint{5.001714in}{2.531964in}}%
\pgfpathlineto{\pgfqpoint{5.002010in}{2.523489in}}%
\pgfpathlineto{\pgfqpoint{5.002306in}{2.530450in}}%
\pgfpathlineto{\pgfqpoint{5.002899in}{2.545716in}}%
\pgfpathlineto{\pgfqpoint{5.003293in}{2.529244in}}%
\pgfpathlineto{\pgfqpoint{5.003392in}{2.527595in}}%
\pgfpathlineto{\pgfqpoint{5.003688in}{2.539712in}}%
\pgfpathlineto{\pgfqpoint{5.003787in}{2.540560in}}%
\pgfpathlineto{\pgfqpoint{5.003885in}{2.535988in}}%
\pgfpathlineto{\pgfqpoint{5.005267in}{2.469557in}}%
\pgfpathlineto{\pgfqpoint{5.005662in}{2.474400in}}%
\pgfpathlineto{\pgfqpoint{5.007932in}{2.457253in}}%
\pgfpathlineto{\pgfqpoint{5.008031in}{2.458453in}}%
\pgfpathlineto{\pgfqpoint{5.009314in}{2.469722in}}%
\pgfpathlineto{\pgfqpoint{5.009413in}{2.469299in}}%
\pgfpathlineto{\pgfqpoint{5.010696in}{2.445101in}}%
\pgfpathlineto{\pgfqpoint{5.010992in}{2.456882in}}%
\pgfpathlineto{\pgfqpoint{5.011091in}{2.459827in}}%
\pgfpathlineto{\pgfqpoint{5.011485in}{2.444057in}}%
\pgfpathlineto{\pgfqpoint{5.012078in}{2.411061in}}%
\pgfpathlineto{\pgfqpoint{5.012374in}{2.435697in}}%
\pgfpathlineto{\pgfqpoint{5.013755in}{2.671371in}}%
\pgfpathlineto{\pgfqpoint{5.014446in}{2.645340in}}%
\pgfpathlineto{\pgfqpoint{5.014742in}{2.623278in}}%
\pgfpathlineto{\pgfqpoint{5.016618in}{2.393420in}}%
\pgfpathlineto{\pgfqpoint{5.016815in}{2.398885in}}%
\pgfpathlineto{\pgfqpoint{5.018493in}{2.481306in}}%
\pgfpathlineto{\pgfqpoint{5.019085in}{2.460172in}}%
\pgfpathlineto{\pgfqpoint{5.019677in}{2.384078in}}%
\pgfpathlineto{\pgfqpoint{5.020566in}{2.406712in}}%
\pgfpathlineto{\pgfqpoint{5.021849in}{2.452680in}}%
\pgfpathlineto{\pgfqpoint{5.022342in}{2.433131in}}%
\pgfpathlineto{\pgfqpoint{5.022934in}{2.383813in}}%
\pgfpathlineto{\pgfqpoint{5.023724in}{2.408212in}}%
\pgfpathlineto{\pgfqpoint{5.024711in}{2.435607in}}%
\pgfpathlineto{\pgfqpoint{5.024217in}{2.407585in}}%
\pgfpathlineto{\pgfqpoint{5.025402in}{2.430774in}}%
\pgfpathlineto{\pgfqpoint{5.026389in}{2.407422in}}%
\pgfpathlineto{\pgfqpoint{5.026586in}{2.419093in}}%
\pgfpathlineto{\pgfqpoint{5.027672in}{2.446314in}}%
\pgfpathlineto{\pgfqpoint{5.027869in}{2.443791in}}%
\pgfpathlineto{\pgfqpoint{5.028955in}{2.461201in}}%
\pgfpathlineto{\pgfqpoint{5.029350in}{2.448205in}}%
\pgfpathlineto{\pgfqpoint{5.029547in}{2.442694in}}%
\pgfpathlineto{\pgfqpoint{5.030041in}{2.462825in}}%
\pgfpathlineto{\pgfqpoint{5.030337in}{2.465640in}}%
\pgfpathlineto{\pgfqpoint{5.030732in}{2.479936in}}%
\pgfpathlineto{\pgfqpoint{5.031521in}{2.471333in}}%
\pgfpathlineto{\pgfqpoint{5.031719in}{2.469581in}}%
\pgfpathlineto{\pgfqpoint{5.031916in}{2.473007in}}%
\pgfpathlineto{\pgfqpoint{5.033100in}{2.482368in}}%
\pgfpathlineto{\pgfqpoint{5.032607in}{2.463858in}}%
\pgfpathlineto{\pgfqpoint{5.033199in}{2.482070in}}%
\pgfpathlineto{\pgfqpoint{5.033495in}{2.480286in}}%
\pgfpathlineto{\pgfqpoint{5.033791in}{2.482281in}}%
\pgfpathlineto{\pgfqpoint{5.034680in}{2.493612in}}%
\pgfpathlineto{\pgfqpoint{5.034877in}{2.489463in}}%
\pgfpathlineto{\pgfqpoint{5.035272in}{2.471469in}}%
\pgfpathlineto{\pgfqpoint{5.036259in}{2.473233in}}%
\pgfpathlineto{\pgfqpoint{5.036456in}{2.476177in}}%
\pgfpathlineto{\pgfqpoint{5.036950in}{2.488770in}}%
\pgfpathlineto{\pgfqpoint{5.037542in}{2.480792in}}%
\pgfpathlineto{\pgfqpoint{5.037838in}{2.472219in}}%
\pgfpathlineto{\pgfqpoint{5.038726in}{2.475998in}}%
\pgfpathlineto{\pgfqpoint{5.039022in}{2.469613in}}%
\pgfpathlineto{\pgfqpoint{5.039713in}{2.475482in}}%
\pgfpathlineto{\pgfqpoint{5.040108in}{2.487848in}}%
\pgfpathlineto{\pgfqpoint{5.040898in}{2.479069in}}%
\pgfpathlineto{\pgfqpoint{5.041292in}{2.482104in}}%
\pgfpathlineto{\pgfqpoint{5.041490in}{2.485223in}}%
\pgfpathlineto{\pgfqpoint{5.041885in}{2.472681in}}%
\pgfpathlineto{\pgfqpoint{5.042082in}{2.468011in}}%
\pgfpathlineto{\pgfqpoint{5.042575in}{2.482641in}}%
\pgfpathlineto{\pgfqpoint{5.042773in}{2.480173in}}%
\pgfpathlineto{\pgfqpoint{5.045339in}{2.453507in}}%
\pgfpathlineto{\pgfqpoint{5.046523in}{2.424708in}}%
\pgfpathlineto{\pgfqpoint{5.046721in}{2.432457in}}%
\pgfpathlineto{\pgfqpoint{5.047017in}{2.444998in}}%
\pgfpathlineto{\pgfqpoint{5.047412in}{2.431765in}}%
\pgfpathlineto{\pgfqpoint{5.047905in}{2.444198in}}%
\pgfpathlineto{\pgfqpoint{5.048103in}{2.443170in}}%
\pgfpathlineto{\pgfqpoint{5.048201in}{2.445163in}}%
\pgfpathlineto{\pgfqpoint{5.050077in}{2.482909in}}%
\pgfpathlineto{\pgfqpoint{5.050175in}{2.481585in}}%
\pgfpathlineto{\pgfqpoint{5.054518in}{2.380869in}}%
\pgfpathlineto{\pgfqpoint{5.055406in}{2.383042in}}%
\pgfpathlineto{\pgfqpoint{5.055702in}{2.385329in}}%
\pgfpathlineto{\pgfqpoint{5.056097in}{2.381364in}}%
\pgfpathlineto{\pgfqpoint{5.056393in}{2.376753in}}%
\pgfpathlineto{\pgfqpoint{5.056887in}{2.386274in}}%
\pgfpathlineto{\pgfqpoint{5.057183in}{2.381997in}}%
\pgfpathlineto{\pgfqpoint{5.057874in}{2.372335in}}%
\pgfpathlineto{\pgfqpoint{5.058367in}{2.380330in}}%
\pgfpathlineto{\pgfqpoint{5.058663in}{2.382565in}}%
\pgfpathlineto{\pgfqpoint{5.059058in}{2.376271in}}%
\pgfpathlineto{\pgfqpoint{5.059256in}{2.378204in}}%
\pgfpathlineto{\pgfqpoint{5.059453in}{2.374416in}}%
\pgfpathlineto{\pgfqpoint{5.060045in}{2.345467in}}%
\pgfpathlineto{\pgfqpoint{5.060440in}{2.380557in}}%
\pgfpathlineto{\pgfqpoint{5.061920in}{2.616855in}}%
\pgfpathlineto{\pgfqpoint{5.062513in}{2.593850in}}%
\pgfpathlineto{\pgfqpoint{5.062907in}{2.533096in}}%
\pgfpathlineto{\pgfqpoint{5.063598in}{2.354277in}}%
\pgfpathlineto{\pgfqpoint{5.064487in}{2.358180in}}%
\pgfpathlineto{\pgfqpoint{5.064783in}{2.355745in}}%
\pgfpathlineto{\pgfqpoint{5.064980in}{2.361237in}}%
\pgfpathlineto{\pgfqpoint{5.066658in}{2.476204in}}%
\pgfpathlineto{\pgfqpoint{5.067250in}{2.429394in}}%
\pgfpathlineto{\pgfqpoint{5.068040in}{2.380061in}}%
\pgfpathlineto{\pgfqpoint{5.068533in}{2.399615in}}%
\pgfpathlineto{\pgfqpoint{5.068632in}{2.400013in}}%
\pgfpathlineto{\pgfqpoint{5.068731in}{2.396918in}}%
\pgfpathlineto{\pgfqpoint{5.069126in}{2.380674in}}%
\pgfpathlineto{\pgfqpoint{5.069619in}{2.406861in}}%
\pgfpathlineto{\pgfqpoint{5.069718in}{2.408201in}}%
\pgfpathlineto{\pgfqpoint{5.070014in}{2.400876in}}%
\pgfpathlineto{\pgfqpoint{5.070705in}{2.354651in}}%
\pgfpathlineto{\pgfqpoint{5.071396in}{2.383296in}}%
\pgfpathlineto{\pgfqpoint{5.073370in}{2.476882in}}%
\pgfpathlineto{\pgfqpoint{5.073468in}{2.473001in}}%
\pgfpathlineto{\pgfqpoint{5.074357in}{2.435926in}}%
\pgfpathlineto{\pgfqpoint{5.074751in}{2.455072in}}%
\pgfpathlineto{\pgfqpoint{5.076034in}{2.475900in}}%
\pgfpathlineto{\pgfqpoint{5.075146in}{2.454959in}}%
\pgfpathlineto{\pgfqpoint{5.076725in}{2.471377in}}%
\pgfpathlineto{\pgfqpoint{5.077614in}{2.450254in}}%
\pgfpathlineto{\pgfqpoint{5.077910in}{2.462573in}}%
\pgfpathlineto{\pgfqpoint{5.078798in}{2.481277in}}%
\pgfpathlineto{\pgfqpoint{5.079094in}{2.472126in}}%
\pgfpathlineto{\pgfqpoint{5.079390in}{2.482097in}}%
\pgfpathlineto{\pgfqpoint{5.079489in}{2.485226in}}%
\pgfpathlineto{\pgfqpoint{5.079982in}{2.474191in}}%
\pgfpathlineto{\pgfqpoint{5.080278in}{2.476661in}}%
\pgfpathlineto{\pgfqpoint{5.081167in}{2.468538in}}%
\pgfpathlineto{\pgfqpoint{5.081463in}{2.473785in}}%
\pgfpathlineto{\pgfqpoint{5.081759in}{2.487960in}}%
\pgfpathlineto{\pgfqpoint{5.082252in}{2.469259in}}%
\pgfpathlineto{\pgfqpoint{5.082647in}{2.480259in}}%
\pgfpathlineto{\pgfqpoint{5.082845in}{2.480981in}}%
\pgfpathlineto{\pgfqpoint{5.083141in}{2.485946in}}%
\pgfpathlineto{\pgfqpoint{5.083634in}{2.478264in}}%
\pgfpathlineto{\pgfqpoint{5.083733in}{2.478260in}}%
\pgfpathlineto{\pgfqpoint{5.084029in}{2.477126in}}%
\pgfpathlineto{\pgfqpoint{5.084226in}{2.478690in}}%
\pgfpathlineto{\pgfqpoint{5.085016in}{2.482562in}}%
\pgfpathlineto{\pgfqpoint{5.085213in}{2.480503in}}%
\pgfpathlineto{\pgfqpoint{5.086990in}{2.450962in}}%
\pgfpathlineto{\pgfqpoint{5.087089in}{2.453515in}}%
\pgfpathlineto{\pgfqpoint{5.087385in}{2.466532in}}%
\pgfpathlineto{\pgfqpoint{5.087977in}{2.443797in}}%
\pgfpathlineto{\pgfqpoint{5.089161in}{2.430125in}}%
\pgfpathlineto{\pgfqpoint{5.089457in}{2.432809in}}%
\pgfpathlineto{\pgfqpoint{5.092616in}{2.403685in}}%
\pgfpathlineto{\pgfqpoint{5.093011in}{2.414565in}}%
\pgfpathlineto{\pgfqpoint{5.093109in}{2.414658in}}%
\pgfpathlineto{\pgfqpoint{5.094392in}{2.392606in}}%
\pgfpathlineto{\pgfqpoint{5.094590in}{2.396765in}}%
\pgfpathlineto{\pgfqpoint{5.095873in}{2.416181in}}%
\pgfpathlineto{\pgfqpoint{5.095281in}{2.396113in}}%
\pgfpathlineto{\pgfqpoint{5.095972in}{2.414980in}}%
\pgfpathlineto{\pgfqpoint{5.096268in}{2.408251in}}%
\pgfpathlineto{\pgfqpoint{5.096663in}{2.421996in}}%
\pgfpathlineto{\pgfqpoint{5.098242in}{2.446481in}}%
\pgfpathlineto{\pgfqpoint{5.101203in}{2.383716in}}%
\pgfpathlineto{\pgfqpoint{5.101400in}{2.389981in}}%
\pgfpathlineto{\pgfqpoint{5.101597in}{2.394549in}}%
\pgfpathlineto{\pgfqpoint{5.102190in}{2.380630in}}%
\pgfpathlineto{\pgfqpoint{5.102387in}{2.378755in}}%
\pgfpathlineto{\pgfqpoint{5.102683in}{2.387080in}}%
\pgfpathlineto{\pgfqpoint{5.103177in}{2.382216in}}%
\pgfpathlineto{\pgfqpoint{5.103966in}{2.392849in}}%
\pgfpathlineto{\pgfqpoint{5.104065in}{2.393348in}}%
\pgfpathlineto{\pgfqpoint{5.104164in}{2.390712in}}%
\pgfpathlineto{\pgfqpoint{5.104460in}{2.375735in}}%
\pgfpathlineto{\pgfqpoint{5.105348in}{2.379483in}}%
\pgfpathlineto{\pgfqpoint{5.106532in}{2.392904in}}%
\pgfpathlineto{\pgfqpoint{5.106730in}{2.389301in}}%
\pgfpathlineto{\pgfqpoint{5.107618in}{2.368976in}}%
\pgfpathlineto{\pgfqpoint{5.107914in}{2.378436in}}%
\pgfpathlineto{\pgfqpoint{5.108605in}{2.524146in}}%
\pgfpathlineto{\pgfqpoint{5.109395in}{2.633650in}}%
\pgfpathlineto{\pgfqpoint{5.110086in}{2.621775in}}%
\pgfpathlineto{\pgfqpoint{5.110382in}{2.602495in}}%
\pgfpathlineto{\pgfqpoint{5.112356in}{2.355995in}}%
\pgfpathlineto{\pgfqpoint{5.112553in}{2.369427in}}%
\pgfpathlineto{\pgfqpoint{5.114527in}{2.473922in}}%
\pgfpathlineto{\pgfqpoint{5.114626in}{2.471755in}}%
\pgfpathlineto{\pgfqpoint{5.115514in}{2.385526in}}%
\pgfpathlineto{\pgfqpoint{5.116304in}{2.425020in}}%
\pgfpathlineto{\pgfqpoint{5.116698in}{2.420988in}}%
\pgfpathlineto{\pgfqpoint{5.116994in}{2.433350in}}%
\pgfpathlineto{\pgfqpoint{5.117488in}{2.469530in}}%
\pgfpathlineto{\pgfqpoint{5.118080in}{2.434801in}}%
\pgfpathlineto{\pgfqpoint{5.118672in}{2.400136in}}%
\pgfpathlineto{\pgfqpoint{5.119363in}{2.423632in}}%
\pgfpathlineto{\pgfqpoint{5.120646in}{2.463272in}}%
\pgfpathlineto{\pgfqpoint{5.121041in}{2.455201in}}%
\pgfpathlineto{\pgfqpoint{5.121929in}{2.419624in}}%
\pgfpathlineto{\pgfqpoint{5.122324in}{2.445857in}}%
\pgfpathlineto{\pgfqpoint{5.123607in}{2.476087in}}%
\pgfpathlineto{\pgfqpoint{5.123903in}{2.469126in}}%
\pgfpathlineto{\pgfqpoint{5.124989in}{2.453827in}}%
\pgfpathlineto{\pgfqpoint{5.124397in}{2.473850in}}%
\pgfpathlineto{\pgfqpoint{5.125285in}{2.458435in}}%
\pgfpathlineto{\pgfqpoint{5.126864in}{2.485649in}}%
\pgfpathlineto{\pgfqpoint{5.127062in}{2.482369in}}%
\pgfpathlineto{\pgfqpoint{5.128444in}{2.471975in}}%
\pgfpathlineto{\pgfqpoint{5.128740in}{2.475214in}}%
\pgfpathlineto{\pgfqpoint{5.128838in}{2.475971in}}%
\pgfpathlineto{\pgfqpoint{5.129134in}{2.469840in}}%
\pgfpathlineto{\pgfqpoint{5.129233in}{2.468857in}}%
\pgfpathlineto{\pgfqpoint{5.129431in}{2.475106in}}%
\pgfpathlineto{\pgfqpoint{5.130516in}{2.487764in}}%
\pgfpathlineto{\pgfqpoint{5.130023in}{2.468326in}}%
\pgfpathlineto{\pgfqpoint{5.130615in}{2.485038in}}%
\pgfpathlineto{\pgfqpoint{5.131799in}{2.470782in}}%
\pgfpathlineto{\pgfqpoint{5.131898in}{2.472524in}}%
\pgfpathlineto{\pgfqpoint{5.132194in}{2.478577in}}%
\pgfpathlineto{\pgfqpoint{5.132589in}{2.461876in}}%
\pgfpathlineto{\pgfqpoint{5.132688in}{2.461366in}}%
\pgfpathlineto{\pgfqpoint{5.132885in}{2.466040in}}%
\pgfpathlineto{\pgfqpoint{5.132984in}{2.467961in}}%
\pgfpathlineto{\pgfqpoint{5.133477in}{2.458954in}}%
\pgfpathlineto{\pgfqpoint{5.133773in}{2.463124in}}%
\pgfpathlineto{\pgfqpoint{5.136734in}{2.422654in}}%
\pgfpathlineto{\pgfqpoint{5.136932in}{2.427781in}}%
\pgfpathlineto{\pgfqpoint{5.137228in}{2.444783in}}%
\pgfpathlineto{\pgfqpoint{5.137721in}{2.422980in}}%
\pgfpathlineto{\pgfqpoint{5.138017in}{2.432571in}}%
\pgfpathlineto{\pgfqpoint{5.139202in}{2.416441in}}%
\pgfpathlineto{\pgfqpoint{5.139399in}{2.419384in}}%
\pgfpathlineto{\pgfqpoint{5.139893in}{2.428596in}}%
\pgfpathlineto{\pgfqpoint{5.140189in}{2.419176in}}%
\pgfpathlineto{\pgfqpoint{5.140386in}{2.412333in}}%
\pgfpathlineto{\pgfqpoint{5.140781in}{2.422714in}}%
\pgfpathlineto{\pgfqpoint{5.141274in}{2.417830in}}%
\pgfpathlineto{\pgfqpoint{5.142163in}{2.422996in}}%
\pgfpathlineto{\pgfqpoint{5.141768in}{2.416524in}}%
\pgfpathlineto{\pgfqpoint{5.142261in}{2.420919in}}%
\pgfpathlineto{\pgfqpoint{5.144235in}{2.377823in}}%
\pgfpathlineto{\pgfqpoint{5.144729in}{2.375201in}}%
\pgfpathlineto{\pgfqpoint{5.145025in}{2.378271in}}%
\pgfpathlineto{\pgfqpoint{5.145617in}{2.388322in}}%
\pgfpathlineto{\pgfqpoint{5.147591in}{2.443069in}}%
\pgfpathlineto{\pgfqpoint{5.147986in}{2.430728in}}%
\pgfpathlineto{\pgfqpoint{5.150256in}{2.372143in}}%
\pgfpathlineto{\pgfqpoint{5.150453in}{2.376783in}}%
\pgfpathlineto{\pgfqpoint{5.151638in}{2.399527in}}%
\pgfpathlineto{\pgfqpoint{5.152033in}{2.397638in}}%
\pgfpathlineto{\pgfqpoint{5.153513in}{2.412309in}}%
\pgfpathlineto{\pgfqpoint{5.153908in}{2.390102in}}%
\pgfpathlineto{\pgfqpoint{5.154796in}{2.404620in}}%
\pgfpathlineto{\pgfqpoint{5.154994in}{2.408493in}}%
\pgfpathlineto{\pgfqpoint{5.155388in}{2.388329in}}%
\pgfpathlineto{\pgfqpoint{5.155487in}{2.386908in}}%
\pgfpathlineto{\pgfqpoint{5.155684in}{2.398372in}}%
\pgfpathlineto{\pgfqpoint{5.157658in}{2.670266in}}%
\pgfpathlineto{\pgfqpoint{5.158251in}{2.609613in}}%
\pgfpathlineto{\pgfqpoint{5.158942in}{2.415303in}}%
\pgfpathlineto{\pgfqpoint{5.160126in}{2.416291in}}%
\pgfpathlineto{\pgfqpoint{5.161310in}{2.476964in}}%
\pgfpathlineto{\pgfqpoint{5.162199in}{2.538492in}}%
\pgfpathlineto{\pgfqpoint{5.162593in}{2.500342in}}%
\pgfpathlineto{\pgfqpoint{5.163087in}{2.447969in}}%
\pgfpathlineto{\pgfqpoint{5.163778in}{2.480551in}}%
\pgfpathlineto{\pgfqpoint{5.165160in}{2.538662in}}%
\pgfpathlineto{\pgfqpoint{5.165850in}{2.517129in}}%
\pgfpathlineto{\pgfqpoint{5.166443in}{2.485237in}}%
\pgfpathlineto{\pgfqpoint{5.166936in}{2.508591in}}%
\pgfpathlineto{\pgfqpoint{5.168417in}{2.556450in}}%
\pgfpathlineto{\pgfqpoint{5.169108in}{2.538775in}}%
\pgfpathlineto{\pgfqpoint{5.169305in}{2.535226in}}%
\pgfpathlineto{\pgfqpoint{5.169700in}{2.511202in}}%
\pgfpathlineto{\pgfqpoint{5.170193in}{2.544624in}}%
\pgfpathlineto{\pgfqpoint{5.171772in}{2.573746in}}%
\pgfpathlineto{\pgfqpoint{5.171871in}{2.572382in}}%
\pgfpathlineto{\pgfqpoint{5.172858in}{2.557175in}}%
\pgfpathlineto{\pgfqpoint{5.173253in}{2.560751in}}%
\pgfpathlineto{\pgfqpoint{5.173549in}{2.555341in}}%
\pgfpathlineto{\pgfqpoint{5.173944in}{2.566897in}}%
\pgfpathlineto{\pgfqpoint{5.175029in}{2.579138in}}%
\pgfpathlineto{\pgfqpoint{5.175227in}{2.576731in}}%
\pgfpathlineto{\pgfqpoint{5.175622in}{2.556479in}}%
\pgfpathlineto{\pgfqpoint{5.176510in}{2.566667in}}%
\pgfpathlineto{\pgfqpoint{5.177102in}{2.586248in}}%
\pgfpathlineto{\pgfqpoint{5.177793in}{2.573624in}}%
\pgfpathlineto{\pgfqpoint{5.178977in}{2.553598in}}%
\pgfpathlineto{\pgfqpoint{5.179866in}{2.560631in}}%
\pgfpathlineto{\pgfqpoint{5.180063in}{2.564536in}}%
\pgfpathlineto{\pgfqpoint{5.180655in}{2.553759in}}%
\pgfpathlineto{\pgfqpoint{5.181544in}{2.534495in}}%
\pgfpathlineto{\pgfqpoint{5.183123in}{2.507577in}}%
\pgfpathlineto{\pgfqpoint{5.183419in}{2.505383in}}%
\pgfpathlineto{\pgfqpoint{5.184406in}{2.484576in}}%
\pgfpathlineto{\pgfqpoint{5.184702in}{2.494449in}}%
\pgfpathlineto{\pgfqpoint{5.184801in}{2.496377in}}%
\pgfpathlineto{\pgfqpoint{5.185195in}{2.488224in}}%
\pgfpathlineto{\pgfqpoint{5.185492in}{2.491042in}}%
\pgfpathlineto{\pgfqpoint{5.186873in}{2.471754in}}%
\pgfpathlineto{\pgfqpoint{5.187071in}{2.475588in}}%
\pgfpathlineto{\pgfqpoint{5.187367in}{2.491924in}}%
\pgfpathlineto{\pgfqpoint{5.188354in}{2.486815in}}%
\pgfpathlineto{\pgfqpoint{5.188551in}{2.494313in}}%
\pgfpathlineto{\pgfqpoint{5.188946in}{2.476010in}}%
\pgfpathlineto{\pgfqpoint{5.189440in}{2.490549in}}%
\pgfpathlineto{\pgfqpoint{5.190328in}{2.477352in}}%
\pgfpathlineto{\pgfqpoint{5.190525in}{2.485331in}}%
\pgfpathlineto{\pgfqpoint{5.192006in}{2.511569in}}%
\pgfpathlineto{\pgfqpoint{5.192302in}{2.512841in}}%
\pgfpathlineto{\pgfqpoint{5.193486in}{2.530378in}}%
\pgfpathlineto{\pgfqpoint{5.193585in}{2.526824in}}%
\pgfpathlineto{\pgfqpoint{5.195559in}{2.486047in}}%
\pgfpathlineto{\pgfqpoint{5.197039in}{2.458171in}}%
\pgfpathlineto{\pgfqpoint{5.197335in}{2.463542in}}%
\pgfpathlineto{\pgfqpoint{5.197434in}{2.464635in}}%
\pgfpathlineto{\pgfqpoint{5.197928in}{2.458327in}}%
\pgfpathlineto{\pgfqpoint{5.198125in}{2.460329in}}%
\pgfpathlineto{\pgfqpoint{5.198224in}{2.461020in}}%
\pgfpathlineto{\pgfqpoint{5.198619in}{2.455928in}}%
\pgfpathlineto{\pgfqpoint{5.198717in}{2.455634in}}%
\pgfpathlineto{\pgfqpoint{5.198816in}{2.456615in}}%
\pgfpathlineto{\pgfqpoint{5.199211in}{2.468350in}}%
\pgfpathlineto{\pgfqpoint{5.199606in}{2.450209in}}%
\pgfpathlineto{\pgfqpoint{5.200099in}{2.465743in}}%
\pgfpathlineto{\pgfqpoint{5.200494in}{2.445763in}}%
\pgfpathlineto{\pgfqpoint{5.201185in}{2.462662in}}%
\pgfpathlineto{\pgfqpoint{5.201777in}{2.468376in}}%
\pgfpathlineto{\pgfqpoint{5.202566in}{2.465919in}}%
\pgfpathlineto{\pgfqpoint{5.203060in}{2.432739in}}%
\pgfpathlineto{\pgfqpoint{5.203455in}{2.461932in}}%
\pgfpathlineto{\pgfqpoint{5.204837in}{2.696934in}}%
\pgfpathlineto{\pgfqpoint{5.205626in}{2.669956in}}%
\pgfpathlineto{\pgfqpoint{5.207699in}{2.420182in}}%
\pgfpathlineto{\pgfqpoint{5.208291in}{2.443355in}}%
\pgfpathlineto{\pgfqpoint{5.208488in}{2.440649in}}%
\pgfpathlineto{\pgfqpoint{5.208686in}{2.449040in}}%
\pgfpathlineto{\pgfqpoint{5.209772in}{2.533546in}}%
\pgfpathlineto{\pgfqpoint{5.210265in}{2.493151in}}%
\pgfpathlineto{\pgfqpoint{5.210758in}{2.431379in}}%
\pgfpathlineto{\pgfqpoint{5.211548in}{2.455636in}}%
\pgfpathlineto{\pgfqpoint{5.211647in}{2.455140in}}%
\pgfpathlineto{\pgfqpoint{5.211745in}{2.459193in}}%
\pgfpathlineto{\pgfqpoint{5.212930in}{2.509405in}}%
\pgfpathlineto{\pgfqpoint{5.213325in}{2.498106in}}%
\pgfpathlineto{\pgfqpoint{5.214016in}{2.454174in}}%
\pgfpathlineto{\pgfqpoint{5.214509in}{2.493160in}}%
\pgfpathlineto{\pgfqpoint{5.216088in}{2.532794in}}%
\pgfpathlineto{\pgfqpoint{5.216582in}{2.514207in}}%
\pgfpathlineto{\pgfqpoint{5.218260in}{2.471495in}}%
\pgfpathlineto{\pgfqpoint{5.218556in}{2.477246in}}%
\pgfpathlineto{\pgfqpoint{5.218852in}{2.464627in}}%
\pgfpathlineto{\pgfqpoint{5.218951in}{2.460996in}}%
\pgfpathlineto{\pgfqpoint{5.219740in}{2.472529in}}%
\pgfpathlineto{\pgfqpoint{5.221517in}{2.557436in}}%
\pgfpathlineto{\pgfqpoint{5.221911in}{2.550333in}}%
\pgfpathlineto{\pgfqpoint{5.222306in}{2.561335in}}%
\pgfpathlineto{\pgfqpoint{5.222602in}{2.548315in}}%
\pgfpathlineto{\pgfqpoint{5.223688in}{2.526568in}}%
\pgfpathlineto{\pgfqpoint{5.223885in}{2.529039in}}%
\pgfpathlineto{\pgfqpoint{5.225465in}{2.543471in}}%
\pgfpathlineto{\pgfqpoint{5.226945in}{2.513155in}}%
\pgfpathlineto{\pgfqpoint{5.227241in}{2.519701in}}%
\pgfpathlineto{\pgfqpoint{5.228130in}{2.526058in}}%
\pgfpathlineto{\pgfqpoint{5.227833in}{2.512965in}}%
\pgfpathlineto{\pgfqpoint{5.228327in}{2.521296in}}%
\pgfpathlineto{\pgfqpoint{5.230005in}{2.476675in}}%
\pgfpathlineto{\pgfqpoint{5.230400in}{2.477798in}}%
\pgfpathlineto{\pgfqpoint{5.231979in}{2.452580in}}%
\pgfpathlineto{\pgfqpoint{5.232374in}{2.452622in}}%
\pgfpathlineto{\pgfqpoint{5.233755in}{2.427717in}}%
\pgfpathlineto{\pgfqpoint{5.234150in}{2.436230in}}%
\pgfpathlineto{\pgfqpoint{5.234841in}{2.429952in}}%
\pgfpathlineto{\pgfqpoint{5.237309in}{2.402517in}}%
\pgfpathlineto{\pgfqpoint{5.235532in}{2.431220in}}%
\pgfpathlineto{\pgfqpoint{5.237506in}{2.406056in}}%
\pgfpathlineto{\pgfqpoint{5.239085in}{2.425729in}}%
\pgfpathlineto{\pgfqpoint{5.239282in}{2.422499in}}%
\pgfpathlineto{\pgfqpoint{5.239480in}{2.418502in}}%
\pgfpathlineto{\pgfqpoint{5.239776in}{2.433952in}}%
\pgfpathlineto{\pgfqpoint{5.240072in}{2.446188in}}%
\pgfpathlineto{\pgfqpoint{5.240566in}{2.419374in}}%
\pgfpathlineto{\pgfqpoint{5.245204in}{2.318256in}}%
\pgfpathlineto{\pgfqpoint{5.241059in}{2.425492in}}%
\pgfpathlineto{\pgfqpoint{5.245994in}{2.326872in}}%
\pgfpathlineto{\pgfqpoint{5.247376in}{2.341609in}}%
\pgfpathlineto{\pgfqpoint{5.248363in}{2.350423in}}%
\pgfpathlineto{\pgfqpoint{5.247968in}{2.337961in}}%
\pgfpathlineto{\pgfqpoint{5.248560in}{2.343933in}}%
\pgfpathlineto{\pgfqpoint{5.248758in}{2.338311in}}%
\pgfpathlineto{\pgfqpoint{5.249152in}{2.352660in}}%
\pgfpathlineto{\pgfqpoint{5.249547in}{2.346157in}}%
\pgfpathlineto{\pgfqpoint{5.249843in}{2.348934in}}%
\pgfpathlineto{\pgfqpoint{5.250041in}{2.341404in}}%
\pgfpathlineto{\pgfqpoint{5.250337in}{2.325104in}}%
\pgfpathlineto{\pgfqpoint{5.250732in}{2.368906in}}%
\pgfpathlineto{\pgfqpoint{5.252311in}{2.614161in}}%
\pgfpathlineto{\pgfqpoint{5.252804in}{2.604313in}}%
\pgfpathlineto{\pgfqpoint{5.253298in}{2.506055in}}%
\pgfpathlineto{\pgfqpoint{5.254877in}{2.341776in}}%
\pgfpathlineto{\pgfqpoint{5.255074in}{2.348701in}}%
\pgfpathlineto{\pgfqpoint{5.257147in}{2.446729in}}%
\pgfpathlineto{\pgfqpoint{5.257246in}{2.444674in}}%
\pgfpathlineto{\pgfqpoint{5.258331in}{2.376306in}}%
\pgfpathlineto{\pgfqpoint{5.259022in}{2.389506in}}%
\pgfpathlineto{\pgfqpoint{5.259516in}{2.419322in}}%
\pgfpathlineto{\pgfqpoint{5.260108in}{2.449825in}}%
\pgfpathlineto{\pgfqpoint{5.260700in}{2.435369in}}%
\pgfpathlineto{\pgfqpoint{5.261292in}{2.375740in}}%
\pgfpathlineto{\pgfqpoint{5.261983in}{2.411264in}}%
\pgfpathlineto{\pgfqpoint{5.263661in}{2.436123in}}%
\pgfpathlineto{\pgfqpoint{5.264549in}{2.400836in}}%
\pgfpathlineto{\pgfqpoint{5.265240in}{2.421942in}}%
\pgfpathlineto{\pgfqpoint{5.266425in}{2.443535in}}%
\pgfpathlineto{\pgfqpoint{5.266721in}{2.435913in}}%
\pgfpathlineto{\pgfqpoint{5.267116in}{2.420174in}}%
\pgfpathlineto{\pgfqpoint{5.267905in}{2.428282in}}%
\pgfpathlineto{\pgfqpoint{5.270175in}{2.454292in}}%
\pgfpathlineto{\pgfqpoint{5.270274in}{2.452976in}}%
\pgfpathlineto{\pgfqpoint{5.270669in}{2.431143in}}%
\pgfpathlineto{\pgfqpoint{5.271458in}{2.448100in}}%
\pgfpathlineto{\pgfqpoint{5.272248in}{2.460759in}}%
\pgfpathlineto{\pgfqpoint{5.272643in}{2.449030in}}%
\pgfpathlineto{\pgfqpoint{5.272840in}{2.445399in}}%
\pgfpathlineto{\pgfqpoint{5.273334in}{2.456757in}}%
\pgfpathlineto{\pgfqpoint{5.273728in}{2.448306in}}%
\pgfpathlineto{\pgfqpoint{5.274913in}{2.431802in}}%
\pgfpathlineto{\pgfqpoint{5.275209in}{2.439073in}}%
\pgfpathlineto{\pgfqpoint{5.275406in}{2.442555in}}%
\pgfpathlineto{\pgfqpoint{5.275801in}{2.429333in}}%
\pgfpathlineto{\pgfqpoint{5.277380in}{2.411767in}}%
\pgfpathlineto{\pgfqpoint{5.278466in}{2.402260in}}%
\pgfpathlineto{\pgfqpoint{5.277972in}{2.412900in}}%
\pgfpathlineto{\pgfqpoint{5.278663in}{2.403994in}}%
\pgfpathlineto{\pgfqpoint{5.278762in}{2.404804in}}%
\pgfpathlineto{\pgfqpoint{5.279058in}{2.400572in}}%
\pgfpathlineto{\pgfqpoint{5.280243in}{2.386889in}}%
\pgfpathlineto{\pgfqpoint{5.279650in}{2.407350in}}%
\pgfpathlineto{\pgfqpoint{5.280539in}{2.391482in}}%
\pgfpathlineto{\pgfqpoint{5.281526in}{2.384549in}}%
\pgfpathlineto{\pgfqpoint{5.281723in}{2.390565in}}%
\pgfpathlineto{\pgfqpoint{5.282019in}{2.401839in}}%
\pgfpathlineto{\pgfqpoint{5.282414in}{2.383234in}}%
\pgfpathlineto{\pgfqpoint{5.282809in}{2.394641in}}%
\pgfpathlineto{\pgfqpoint{5.283796in}{2.385499in}}%
\pgfpathlineto{\pgfqpoint{5.283401in}{2.395434in}}%
\pgfpathlineto{\pgfqpoint{5.283993in}{2.390765in}}%
\pgfpathlineto{\pgfqpoint{5.284881in}{2.400365in}}%
\pgfpathlineto{\pgfqpoint{5.284388in}{2.387734in}}%
\pgfpathlineto{\pgfqpoint{5.285079in}{2.394804in}}%
\pgfpathlineto{\pgfqpoint{5.285177in}{2.392083in}}%
\pgfpathlineto{\pgfqpoint{5.285572in}{2.409920in}}%
\pgfpathlineto{\pgfqpoint{5.287349in}{2.454341in}}%
\pgfpathlineto{\pgfqpoint{5.289323in}{2.435990in}}%
\pgfpathlineto{\pgfqpoint{5.289422in}{2.437891in}}%
\pgfpathlineto{\pgfqpoint{5.289619in}{2.440921in}}%
\pgfpathlineto{\pgfqpoint{5.289915in}{2.424716in}}%
\pgfpathlineto{\pgfqpoint{5.290803in}{2.407309in}}%
\pgfpathlineto{\pgfqpoint{5.291198in}{2.413097in}}%
\pgfpathlineto{\pgfqpoint{5.294554in}{2.327375in}}%
\pgfpathlineto{\pgfqpoint{5.294653in}{2.330476in}}%
\pgfpathlineto{\pgfqpoint{5.297021in}{2.421264in}}%
\pgfpathlineto{\pgfqpoint{5.297219in}{2.415548in}}%
\pgfpathlineto{\pgfqpoint{5.297614in}{2.387815in}}%
\pgfpathlineto{\pgfqpoint{5.298008in}{2.425556in}}%
\pgfpathlineto{\pgfqpoint{5.299390in}{2.652723in}}%
\pgfpathlineto{\pgfqpoint{5.300081in}{2.629490in}}%
\pgfpathlineto{\pgfqpoint{5.300476in}{2.561863in}}%
\pgfpathlineto{\pgfqpoint{5.302055in}{2.388848in}}%
\pgfpathlineto{\pgfqpoint{5.302252in}{2.381821in}}%
\pgfpathlineto{\pgfqpoint{5.302746in}{2.402841in}}%
\pgfpathlineto{\pgfqpoint{5.304424in}{2.507588in}}%
\pgfpathlineto{\pgfqpoint{5.304720in}{2.474742in}}%
\pgfpathlineto{\pgfqpoint{5.305411in}{2.406355in}}%
\pgfpathlineto{\pgfqpoint{5.306003in}{2.431564in}}%
\pgfpathlineto{\pgfqpoint{5.306694in}{2.464876in}}%
\pgfpathlineto{\pgfqpoint{5.307483in}{2.491713in}}%
\pgfpathlineto{\pgfqpoint{5.307780in}{2.471183in}}%
\pgfpathlineto{\pgfqpoint{5.308569in}{2.416721in}}%
\pgfpathlineto{\pgfqpoint{5.309063in}{2.452732in}}%
\pgfpathlineto{\pgfqpoint{5.310642in}{2.479200in}}%
\pgfpathlineto{\pgfqpoint{5.310839in}{2.473630in}}%
\pgfpathlineto{\pgfqpoint{5.311629in}{2.450407in}}%
\pgfpathlineto{\pgfqpoint{5.312122in}{2.461665in}}%
\pgfpathlineto{\pgfqpoint{5.313603in}{2.483314in}}%
\pgfpathlineto{\pgfqpoint{5.313800in}{2.479354in}}%
\pgfpathlineto{\pgfqpoint{5.315281in}{2.445095in}}%
\pgfpathlineto{\pgfqpoint{5.315478in}{2.453891in}}%
\pgfpathlineto{\pgfqpoint{5.316761in}{2.475608in}}%
\pgfpathlineto{\pgfqpoint{5.317156in}{2.458578in}}%
\pgfpathlineto{\pgfqpoint{5.318143in}{2.440356in}}%
\pgfpathlineto{\pgfqpoint{5.317748in}{2.467702in}}%
\pgfpathlineto{\pgfqpoint{5.318340in}{2.451354in}}%
\pgfpathlineto{\pgfqpoint{5.318636in}{2.469865in}}%
\pgfpathlineto{\pgfqpoint{5.319426in}{2.461533in}}%
\pgfpathlineto{\pgfqpoint{5.319722in}{2.444106in}}%
\pgfpathlineto{\pgfqpoint{5.320610in}{2.447693in}}%
\pgfpathlineto{\pgfqpoint{5.320709in}{2.448220in}}%
\pgfpathlineto{\pgfqpoint{5.320907in}{2.445742in}}%
\pgfpathlineto{\pgfqpoint{5.321597in}{2.440468in}}%
\pgfpathlineto{\pgfqpoint{5.322091in}{2.442578in}}%
\pgfpathlineto{\pgfqpoint{5.322190in}{2.442897in}}%
\pgfpathlineto{\pgfqpoint{5.322387in}{2.441274in}}%
\pgfpathlineto{\pgfqpoint{5.323571in}{2.419365in}}%
\pgfpathlineto{\pgfqpoint{5.323769in}{2.427038in}}%
\pgfpathlineto{\pgfqpoint{5.324065in}{2.438125in}}%
\pgfpathlineto{\pgfqpoint{5.324558in}{2.408735in}}%
\pgfpathlineto{\pgfqpoint{5.324756in}{2.410465in}}%
\pgfpathlineto{\pgfqpoint{5.325447in}{2.416565in}}%
\pgfpathlineto{\pgfqpoint{5.325743in}{2.411087in}}%
\pgfpathlineto{\pgfqpoint{5.327717in}{2.391037in}}%
\pgfpathlineto{\pgfqpoint{5.326236in}{2.414836in}}%
\pgfpathlineto{\pgfqpoint{5.327815in}{2.393309in}}%
\pgfpathlineto{\pgfqpoint{5.328605in}{2.407345in}}%
\pgfpathlineto{\pgfqpoint{5.328901in}{2.394577in}}%
\pgfpathlineto{\pgfqpoint{5.329493in}{2.405041in}}%
\pgfpathlineto{\pgfqpoint{5.330086in}{2.388567in}}%
\pgfpathlineto{\pgfqpoint{5.330382in}{2.396325in}}%
\pgfpathlineto{\pgfqpoint{5.331467in}{2.408992in}}%
\pgfpathlineto{\pgfqpoint{5.331665in}{2.405210in}}%
\pgfpathlineto{\pgfqpoint{5.331961in}{2.397823in}}%
\pgfpathlineto{\pgfqpoint{5.332553in}{2.411231in}}%
\pgfpathlineto{\pgfqpoint{5.335119in}{2.458738in}}%
\pgfpathlineto{\pgfqpoint{5.335613in}{2.441784in}}%
\pgfpathlineto{\pgfqpoint{5.336600in}{2.443468in}}%
\pgfpathlineto{\pgfqpoint{5.339561in}{2.369875in}}%
\pgfpathlineto{\pgfqpoint{5.339758in}{2.368974in}}%
\pgfpathlineto{\pgfqpoint{5.340054in}{2.363688in}}%
\pgfpathlineto{\pgfqpoint{5.340350in}{2.373225in}}%
\pgfpathlineto{\pgfqpoint{5.341140in}{2.384030in}}%
\pgfpathlineto{\pgfqpoint{5.341535in}{2.379659in}}%
\pgfpathlineto{\pgfqpoint{5.342028in}{2.378367in}}%
\pgfpathlineto{\pgfqpoint{5.342423in}{2.393059in}}%
\pgfpathlineto{\pgfqpoint{5.342818in}{2.372797in}}%
\pgfpathlineto{\pgfqpoint{5.343311in}{2.387727in}}%
\pgfpathlineto{\pgfqpoint{5.344989in}{2.357664in}}%
\pgfpathlineto{\pgfqpoint{5.345483in}{2.399269in}}%
\pgfpathlineto{\pgfqpoint{5.346864in}{2.637667in}}%
\pgfpathlineto{\pgfqpoint{5.347555in}{2.600004in}}%
\pgfpathlineto{\pgfqpoint{5.348147in}{2.450778in}}%
\pgfpathlineto{\pgfqpoint{5.349529in}{2.354904in}}%
\pgfpathlineto{\pgfqpoint{5.349628in}{2.349445in}}%
\pgfpathlineto{\pgfqpoint{5.350023in}{2.382521in}}%
\pgfpathlineto{\pgfqpoint{5.351602in}{2.470222in}}%
\pgfpathlineto{\pgfqpoint{5.352194in}{2.434800in}}%
\pgfpathlineto{\pgfqpoint{5.352688in}{2.388010in}}%
\pgfpathlineto{\pgfqpoint{5.353378in}{2.414601in}}%
\pgfpathlineto{\pgfqpoint{5.354464in}{2.461066in}}%
\pgfpathlineto{\pgfqpoint{5.355254in}{2.441501in}}%
\pgfpathlineto{\pgfqpoint{5.356043in}{2.395498in}}%
\pgfpathlineto{\pgfqpoint{5.356537in}{2.430268in}}%
\pgfpathlineto{\pgfqpoint{5.356833in}{2.418291in}}%
\pgfpathlineto{\pgfqpoint{5.357326in}{2.438454in}}%
\pgfpathlineto{\pgfqpoint{5.357425in}{2.438392in}}%
\pgfpathlineto{\pgfqpoint{5.358610in}{2.465529in}}%
\pgfpathlineto{\pgfqpoint{5.358906in}{2.447888in}}%
\pgfpathlineto{\pgfqpoint{5.359103in}{2.441454in}}%
\pgfpathlineto{\pgfqpoint{5.359794in}{2.459484in}}%
\pgfpathlineto{\pgfqpoint{5.361176in}{2.477409in}}%
\pgfpathlineto{\pgfqpoint{5.361274in}{2.478445in}}%
\pgfpathlineto{\pgfqpoint{5.361472in}{2.471477in}}%
\pgfpathlineto{\pgfqpoint{5.362360in}{2.458849in}}%
\pgfpathlineto{\pgfqpoint{5.362656in}{2.467413in}}%
\pgfpathlineto{\pgfqpoint{5.363051in}{2.480743in}}%
\pgfpathlineto{\pgfqpoint{5.363939in}{2.476922in}}%
\pgfpathlineto{\pgfqpoint{5.365124in}{2.461764in}}%
\pgfpathlineto{\pgfqpoint{5.364630in}{2.479437in}}%
\pgfpathlineto{\pgfqpoint{5.365222in}{2.463439in}}%
\pgfpathlineto{\pgfqpoint{5.366505in}{2.495131in}}%
\pgfpathlineto{\pgfqpoint{5.367098in}{2.484873in}}%
\pgfpathlineto{\pgfqpoint{5.368282in}{2.473420in}}%
\pgfpathlineto{\pgfqpoint{5.367690in}{2.491687in}}%
\pgfpathlineto{\pgfqpoint{5.368479in}{2.477130in}}%
\pgfpathlineto{\pgfqpoint{5.368578in}{2.477916in}}%
\pgfpathlineto{\pgfqpoint{5.368776in}{2.471824in}}%
\pgfpathlineto{\pgfqpoint{5.371046in}{2.403546in}}%
\pgfpathlineto{\pgfqpoint{5.371243in}{2.404820in}}%
\pgfpathlineto{\pgfqpoint{5.371342in}{2.405361in}}%
\pgfpathlineto{\pgfqpoint{5.371539in}{2.402869in}}%
\pgfpathlineto{\pgfqpoint{5.371934in}{2.393095in}}%
\pgfpathlineto{\pgfqpoint{5.372329in}{2.408270in}}%
\pgfpathlineto{\pgfqpoint{5.374105in}{2.462517in}}%
\pgfpathlineto{\pgfqpoint{5.375487in}{2.434584in}}%
\pgfpathlineto{\pgfqpoint{5.375783in}{2.441474in}}%
\pgfpathlineto{\pgfqpoint{5.375981in}{2.447824in}}%
\pgfpathlineto{\pgfqpoint{5.376474in}{2.431749in}}%
\pgfpathlineto{\pgfqpoint{5.376968in}{2.444342in}}%
\pgfpathlineto{\pgfqpoint{5.377264in}{2.432474in}}%
\pgfpathlineto{\pgfqpoint{5.377757in}{2.445390in}}%
\pgfpathlineto{\pgfqpoint{5.378152in}{2.440151in}}%
\pgfpathlineto{\pgfqpoint{5.378941in}{2.436188in}}%
\pgfpathlineto{\pgfqpoint{5.379139in}{2.433263in}}%
\pgfpathlineto{\pgfqpoint{5.379435in}{2.437562in}}%
\pgfpathlineto{\pgfqpoint{5.379830in}{2.435991in}}%
\pgfpathlineto{\pgfqpoint{5.382001in}{2.483547in}}%
\pgfpathlineto{\pgfqpoint{5.382199in}{2.479650in}}%
\pgfpathlineto{\pgfqpoint{5.383284in}{2.459154in}}%
\pgfpathlineto{\pgfqpoint{5.383482in}{2.464952in}}%
\pgfpathlineto{\pgfqpoint{5.383778in}{2.473775in}}%
\pgfpathlineto{\pgfqpoint{5.384271in}{2.452874in}}%
\pgfpathlineto{\pgfqpoint{5.385357in}{2.432572in}}%
\pgfpathlineto{\pgfqpoint{5.386541in}{2.414491in}}%
\pgfpathlineto{\pgfqpoint{5.386640in}{2.415433in}}%
\pgfpathlineto{\pgfqpoint{5.386837in}{2.417406in}}%
\pgfpathlineto{\pgfqpoint{5.387134in}{2.406633in}}%
\pgfpathlineto{\pgfqpoint{5.387331in}{2.402938in}}%
\pgfpathlineto{\pgfqpoint{5.387726in}{2.416796in}}%
\pgfpathlineto{\pgfqpoint{5.388022in}{2.412351in}}%
\pgfpathlineto{\pgfqpoint{5.388318in}{2.417863in}}%
\pgfpathlineto{\pgfqpoint{5.388910in}{2.414374in}}%
\pgfpathlineto{\pgfqpoint{5.389502in}{2.420565in}}%
\pgfpathlineto{\pgfqpoint{5.389798in}{2.416791in}}%
\pgfpathlineto{\pgfqpoint{5.390292in}{2.422572in}}%
\pgfpathlineto{\pgfqpoint{5.390588in}{2.426315in}}%
\pgfpathlineto{\pgfqpoint{5.391378in}{2.422950in}}%
\pgfpathlineto{\pgfqpoint{5.392266in}{2.387373in}}%
\pgfpathlineto{\pgfqpoint{5.392463in}{2.404711in}}%
\pgfpathlineto{\pgfqpoint{5.394042in}{2.672836in}}%
\pgfpathlineto{\pgfqpoint{5.394635in}{2.647444in}}%
\pgfpathlineto{\pgfqpoint{5.394931in}{2.612592in}}%
\pgfpathlineto{\pgfqpoint{5.396806in}{2.390759in}}%
\pgfpathlineto{\pgfqpoint{5.397102in}{2.399703in}}%
\pgfpathlineto{\pgfqpoint{5.398879in}{2.503425in}}%
\pgfpathlineto{\pgfqpoint{5.399175in}{2.477973in}}%
\pgfpathlineto{\pgfqpoint{5.399866in}{2.418789in}}%
\pgfpathlineto{\pgfqpoint{5.400557in}{2.434605in}}%
\pgfpathlineto{\pgfqpoint{5.402037in}{2.504287in}}%
\pgfpathlineto{\pgfqpoint{5.402531in}{2.481130in}}%
\pgfpathlineto{\pgfqpoint{5.403123in}{2.426618in}}%
\pgfpathlineto{\pgfqpoint{5.403912in}{2.461453in}}%
\pgfpathlineto{\pgfqpoint{5.405195in}{2.493127in}}%
\pgfpathlineto{\pgfqpoint{5.405886in}{2.487221in}}%
\pgfpathlineto{\pgfqpoint{5.406281in}{2.468154in}}%
\pgfpathlineto{\pgfqpoint{5.406972in}{2.482133in}}%
\pgfpathlineto{\pgfqpoint{5.408255in}{2.527776in}}%
\pgfpathlineto{\pgfqpoint{5.408551in}{2.510493in}}%
\pgfpathlineto{\pgfqpoint{5.409538in}{2.493624in}}%
\pgfpathlineto{\pgfqpoint{5.409045in}{2.516309in}}%
\pgfpathlineto{\pgfqpoint{5.409736in}{2.499621in}}%
\pgfpathlineto{\pgfqpoint{5.411117in}{2.532378in}}%
\pgfpathlineto{\pgfqpoint{5.411216in}{2.531957in}}%
\pgfpathlineto{\pgfqpoint{5.411413in}{2.530269in}}%
\pgfpathlineto{\pgfqpoint{5.411808in}{2.536546in}}%
\pgfpathlineto{\pgfqpoint{5.412006in}{2.533644in}}%
\pgfpathlineto{\pgfqpoint{5.412400in}{2.513782in}}%
\pgfpathlineto{\pgfqpoint{5.413190in}{2.521473in}}%
\pgfpathlineto{\pgfqpoint{5.413585in}{2.536678in}}%
\pgfpathlineto{\pgfqpoint{5.414276in}{2.524187in}}%
\pgfpathlineto{\pgfqpoint{5.414671in}{2.525226in}}%
\pgfpathlineto{\pgfqpoint{5.414967in}{2.518710in}}%
\pgfpathlineto{\pgfqpoint{5.416348in}{2.498357in}}%
\pgfpathlineto{\pgfqpoint{5.415756in}{2.524566in}}%
\pgfpathlineto{\pgfqpoint{5.416546in}{2.503861in}}%
\pgfpathlineto{\pgfqpoint{5.416743in}{2.507535in}}%
\pgfpathlineto{\pgfqpoint{5.417138in}{2.496880in}}%
\pgfpathlineto{\pgfqpoint{5.417533in}{2.500313in}}%
\pgfpathlineto{\pgfqpoint{5.418322in}{2.466209in}}%
\pgfpathlineto{\pgfqpoint{5.419309in}{2.454227in}}%
\pgfpathlineto{\pgfqpoint{5.418816in}{2.475699in}}%
\pgfpathlineto{\pgfqpoint{5.419507in}{2.459023in}}%
\pgfpathlineto{\pgfqpoint{5.419704in}{2.462076in}}%
\pgfpathlineto{\pgfqpoint{5.420198in}{2.447800in}}%
\pgfpathlineto{\pgfqpoint{5.420987in}{2.430564in}}%
\pgfpathlineto{\pgfqpoint{5.421382in}{2.445980in}}%
\pgfpathlineto{\pgfqpoint{5.421481in}{2.447930in}}%
\pgfpathlineto{\pgfqpoint{5.421777in}{2.436881in}}%
\pgfpathlineto{\pgfqpoint{5.422566in}{2.429112in}}%
\pgfpathlineto{\pgfqpoint{5.422270in}{2.438170in}}%
\pgfpathlineto{\pgfqpoint{5.422863in}{2.435131in}}%
\pgfpathlineto{\pgfqpoint{5.424343in}{2.454131in}}%
\pgfpathlineto{\pgfqpoint{5.423455in}{2.429455in}}%
\pgfpathlineto{\pgfqpoint{5.424540in}{2.450581in}}%
\pgfpathlineto{\pgfqpoint{5.424837in}{2.445338in}}%
\pgfpathlineto{\pgfqpoint{5.425231in}{2.462764in}}%
\pgfpathlineto{\pgfqpoint{5.425429in}{2.457657in}}%
\pgfpathlineto{\pgfqpoint{5.425725in}{2.447209in}}%
\pgfpathlineto{\pgfqpoint{5.426120in}{2.462799in}}%
\pgfpathlineto{\pgfqpoint{5.426514in}{2.458968in}}%
\pgfpathlineto{\pgfqpoint{5.428488in}{2.488283in}}%
\pgfpathlineto{\pgfqpoint{5.429179in}{2.516127in}}%
\pgfpathlineto{\pgfqpoint{5.430166in}{2.507884in}}%
\pgfpathlineto{\pgfqpoint{5.432239in}{2.453577in}}%
\pgfpathlineto{\pgfqpoint{5.433917in}{2.423926in}}%
\pgfpathlineto{\pgfqpoint{5.434016in}{2.424183in}}%
\pgfpathlineto{\pgfqpoint{5.434312in}{2.433359in}}%
\pgfpathlineto{\pgfqpoint{5.435200in}{2.429501in}}%
\pgfpathlineto{\pgfqpoint{5.436088in}{2.412399in}}%
\pgfpathlineto{\pgfqpoint{5.436384in}{2.427255in}}%
\pgfpathlineto{\pgfqpoint{5.436483in}{2.429123in}}%
\pgfpathlineto{\pgfqpoint{5.436680in}{2.417236in}}%
\pgfpathlineto{\pgfqpoint{5.437667in}{2.409311in}}%
\pgfpathlineto{\pgfqpoint{5.437273in}{2.434968in}}%
\pgfpathlineto{\pgfqpoint{5.437766in}{2.414920in}}%
\pgfpathlineto{\pgfqpoint{5.437963in}{2.427439in}}%
\pgfpathlineto{\pgfqpoint{5.438457in}{2.405556in}}%
\pgfpathlineto{\pgfqpoint{5.438753in}{2.408531in}}%
\pgfpathlineto{\pgfqpoint{5.439247in}{2.382634in}}%
\pgfpathlineto{\pgfqpoint{5.439641in}{2.409610in}}%
\pgfpathlineto{\pgfqpoint{5.441221in}{2.654533in}}%
\pgfpathlineto{\pgfqpoint{5.441813in}{2.626540in}}%
\pgfpathlineto{\pgfqpoint{5.442109in}{2.590877in}}%
\pgfpathlineto{\pgfqpoint{5.443787in}{2.389665in}}%
\pgfpathlineto{\pgfqpoint{5.444280in}{2.383423in}}%
\pgfpathlineto{\pgfqpoint{5.444872in}{2.387294in}}%
\pgfpathlineto{\pgfqpoint{5.445267in}{2.400238in}}%
\pgfpathlineto{\pgfqpoint{5.445958in}{2.429835in}}%
\pgfpathlineto{\pgfqpoint{5.446254in}{2.410189in}}%
\pgfpathlineto{\pgfqpoint{5.446945in}{2.326756in}}%
\pgfpathlineto{\pgfqpoint{5.447537in}{2.362614in}}%
\pgfpathlineto{\pgfqpoint{5.449215in}{2.492595in}}%
\pgfpathlineto{\pgfqpoint{5.449807in}{2.454243in}}%
\pgfpathlineto{\pgfqpoint{5.450301in}{2.424124in}}%
\pgfpathlineto{\pgfqpoint{5.451090in}{2.437629in}}%
\pgfpathlineto{\pgfqpoint{5.452472in}{2.465898in}}%
\pgfpathlineto{\pgfqpoint{5.452670in}{2.465043in}}%
\pgfpathlineto{\pgfqpoint{5.453854in}{2.436024in}}%
\pgfpathlineto{\pgfqpoint{5.454150in}{2.455165in}}%
\pgfpathlineto{\pgfqpoint{5.455433in}{2.477300in}}%
\pgfpathlineto{\pgfqpoint{5.455828in}{2.465752in}}%
\pgfpathlineto{\pgfqpoint{5.456914in}{2.443864in}}%
\pgfpathlineto{\pgfqpoint{5.457111in}{2.454970in}}%
\pgfpathlineto{\pgfqpoint{5.457407in}{2.475676in}}%
\pgfpathlineto{\pgfqpoint{5.458295in}{2.464334in}}%
\pgfpathlineto{\pgfqpoint{5.459381in}{2.471460in}}%
\pgfpathlineto{\pgfqpoint{5.459480in}{2.469815in}}%
\pgfpathlineto{\pgfqpoint{5.459776in}{2.457405in}}%
\pgfpathlineto{\pgfqpoint{5.460566in}{2.467634in}}%
\pgfpathlineto{\pgfqpoint{5.461454in}{2.478212in}}%
\pgfpathlineto{\pgfqpoint{5.461059in}{2.465507in}}%
\pgfpathlineto{\pgfqpoint{5.461651in}{2.466760in}}%
\pgfpathlineto{\pgfqpoint{5.461849in}{2.458890in}}%
\pgfpathlineto{\pgfqpoint{5.462638in}{2.467861in}}%
\pgfpathlineto{\pgfqpoint{5.462836in}{2.470680in}}%
\pgfpathlineto{\pgfqpoint{5.463428in}{2.464679in}}%
\pgfpathlineto{\pgfqpoint{5.463823in}{2.452769in}}%
\pgfpathlineto{\pgfqpoint{5.464415in}{2.467093in}}%
\pgfpathlineto{\pgfqpoint{5.464513in}{2.468160in}}%
\pgfpathlineto{\pgfqpoint{5.464810in}{2.462989in}}%
\pgfpathlineto{\pgfqpoint{5.466290in}{2.424090in}}%
\pgfpathlineto{\pgfqpoint{5.466586in}{2.435946in}}%
\pgfpathlineto{\pgfqpoint{5.466784in}{2.440855in}}%
\pgfpathlineto{\pgfqpoint{5.467277in}{2.423339in}}%
\pgfpathlineto{\pgfqpoint{5.467573in}{2.423732in}}%
\pgfpathlineto{\pgfqpoint{5.468165in}{2.405952in}}%
\pgfpathlineto{\pgfqpoint{5.468264in}{2.404571in}}%
\pgfpathlineto{\pgfqpoint{5.468560in}{2.410936in}}%
\pgfpathlineto{\pgfqpoint{5.468856in}{2.420101in}}%
\pgfpathlineto{\pgfqpoint{5.469448in}{2.403081in}}%
\pgfpathlineto{\pgfqpoint{5.470435in}{2.401577in}}%
\pgfpathlineto{\pgfqpoint{5.470534in}{2.402194in}}%
\pgfpathlineto{\pgfqpoint{5.471028in}{2.408250in}}%
\pgfpathlineto{\pgfqpoint{5.471422in}{2.400677in}}%
\pgfpathlineto{\pgfqpoint{5.472409in}{2.394189in}}%
\pgfpathlineto{\pgfqpoint{5.472015in}{2.403571in}}%
\pgfpathlineto{\pgfqpoint{5.472508in}{2.398314in}}%
\pgfpathlineto{\pgfqpoint{5.472804in}{2.416218in}}%
\pgfpathlineto{\pgfqpoint{5.473298in}{2.395571in}}%
\pgfpathlineto{\pgfqpoint{5.473692in}{2.407196in}}%
\pgfpathlineto{\pgfqpoint{5.474383in}{2.392706in}}%
\pgfpathlineto{\pgfqpoint{5.474976in}{2.403572in}}%
\pgfpathlineto{\pgfqpoint{5.477937in}{2.452703in}}%
\pgfpathlineto{\pgfqpoint{5.475568in}{2.402850in}}%
\pgfpathlineto{\pgfqpoint{5.478134in}{2.440269in}}%
\pgfpathlineto{\pgfqpoint{5.479614in}{2.401635in}}%
\pgfpathlineto{\pgfqpoint{5.481095in}{2.374950in}}%
\pgfpathlineto{\pgfqpoint{5.481292in}{2.377185in}}%
\pgfpathlineto{\pgfqpoint{5.481490in}{2.382198in}}%
\pgfpathlineto{\pgfqpoint{5.482082in}{2.371125in}}%
\pgfpathlineto{\pgfqpoint{5.482181in}{2.370295in}}%
\pgfpathlineto{\pgfqpoint{5.482378in}{2.374404in}}%
\pgfpathlineto{\pgfqpoint{5.482773in}{2.390673in}}%
\pgfpathlineto{\pgfqpoint{5.483760in}{2.388961in}}%
\pgfpathlineto{\pgfqpoint{5.484056in}{2.393809in}}%
\pgfpathlineto{\pgfqpoint{5.485240in}{2.403213in}}%
\pgfpathlineto{\pgfqpoint{5.484648in}{2.389483in}}%
\pgfpathlineto{\pgfqpoint{5.485438in}{2.401956in}}%
\pgfpathlineto{\pgfqpoint{5.486030in}{2.395718in}}%
\pgfpathlineto{\pgfqpoint{5.486425in}{2.402570in}}%
\pgfpathlineto{\pgfqpoint{5.486622in}{2.398948in}}%
\pgfpathlineto{\pgfqpoint{5.487116in}{2.367368in}}%
\pgfpathlineto{\pgfqpoint{5.487510in}{2.403934in}}%
\pgfpathlineto{\pgfqpoint{5.488991in}{2.658329in}}%
\pgfpathlineto{\pgfqpoint{5.489583in}{2.620743in}}%
\pgfpathlineto{\pgfqpoint{5.489780in}{2.613327in}}%
\pgfpathlineto{\pgfqpoint{5.491754in}{2.374313in}}%
\pgfpathlineto{\pgfqpoint{5.492050in}{2.395987in}}%
\pgfpathlineto{\pgfqpoint{5.493728in}{2.488533in}}%
\pgfpathlineto{\pgfqpoint{5.494024in}{2.470499in}}%
\pgfpathlineto{\pgfqpoint{5.495011in}{2.398932in}}%
\pgfpathlineto{\pgfqpoint{5.495406in}{2.420750in}}%
\pgfpathlineto{\pgfqpoint{5.497084in}{2.481769in}}%
\pgfpathlineto{\pgfqpoint{5.498071in}{2.425361in}}%
\pgfpathlineto{\pgfqpoint{5.498762in}{2.458169in}}%
\pgfpathlineto{\pgfqpoint{5.498861in}{2.457967in}}%
\pgfpathlineto{\pgfqpoint{5.498959in}{2.458545in}}%
\pgfpathlineto{\pgfqpoint{5.500243in}{2.499090in}}%
\pgfpathlineto{\pgfqpoint{5.500539in}{2.481326in}}%
\pgfpathlineto{\pgfqpoint{5.501427in}{2.458207in}}%
\pgfpathlineto{\pgfqpoint{5.501723in}{2.473540in}}%
\pgfpathlineto{\pgfqpoint{5.502710in}{2.497062in}}%
\pgfpathlineto{\pgfqpoint{5.503302in}{2.492803in}}%
\pgfpathlineto{\pgfqpoint{5.504585in}{2.471896in}}%
\pgfpathlineto{\pgfqpoint{5.504783in}{2.479818in}}%
\pgfpathlineto{\pgfqpoint{5.505177in}{2.496397in}}%
\pgfpathlineto{\pgfqpoint{5.505967in}{2.491371in}}%
\pgfpathlineto{\pgfqpoint{5.508237in}{2.476044in}}%
\pgfpathlineto{\pgfqpoint{5.508336in}{2.477883in}}%
\pgfpathlineto{\pgfqpoint{5.508731in}{2.496151in}}%
\pgfpathlineto{\pgfqpoint{5.509520in}{2.483091in}}%
\pgfpathlineto{\pgfqpoint{5.509718in}{2.478235in}}%
\pgfpathlineto{\pgfqpoint{5.510112in}{2.483693in}}%
\pgfpathlineto{\pgfqpoint{5.510606in}{2.481586in}}%
\pgfpathlineto{\pgfqpoint{5.510705in}{2.481549in}}%
\pgfpathlineto{\pgfqpoint{5.512185in}{2.461453in}}%
\pgfpathlineto{\pgfqpoint{5.511692in}{2.481870in}}%
\pgfpathlineto{\pgfqpoint{5.512382in}{2.467067in}}%
\pgfpathlineto{\pgfqpoint{5.512679in}{2.478662in}}%
\pgfpathlineto{\pgfqpoint{5.513073in}{2.464228in}}%
\pgfpathlineto{\pgfqpoint{5.513468in}{2.466260in}}%
\pgfpathlineto{\pgfqpoint{5.514356in}{2.458139in}}%
\pgfpathlineto{\pgfqpoint{5.514653in}{2.464475in}}%
\pgfpathlineto{\pgfqpoint{5.514850in}{2.469166in}}%
\pgfpathlineto{\pgfqpoint{5.515343in}{2.452996in}}%
\pgfpathlineto{\pgfqpoint{5.515640in}{2.461401in}}%
\pgfpathlineto{\pgfqpoint{5.515738in}{2.462100in}}%
\pgfpathlineto{\pgfqpoint{5.515837in}{2.458260in}}%
\pgfpathlineto{\pgfqpoint{5.517910in}{2.378254in}}%
\pgfpathlineto{\pgfqpoint{5.518798in}{2.343281in}}%
\pgfpathlineto{\pgfqpoint{5.519193in}{2.363303in}}%
\pgfpathlineto{\pgfqpoint{5.520574in}{2.445955in}}%
\pgfpathlineto{\pgfqpoint{5.521858in}{2.442294in}}%
\pgfpathlineto{\pgfqpoint{5.522154in}{2.432291in}}%
\pgfpathlineto{\pgfqpoint{5.522548in}{2.451374in}}%
\pgfpathlineto{\pgfqpoint{5.523634in}{2.456074in}}%
\pgfpathlineto{\pgfqpoint{5.523141in}{2.440035in}}%
\pgfpathlineto{\pgfqpoint{5.523733in}{2.455765in}}%
\pgfpathlineto{\pgfqpoint{5.524029in}{2.449057in}}%
\pgfpathlineto{\pgfqpoint{5.524325in}{2.461447in}}%
\pgfpathlineto{\pgfqpoint{5.524522in}{2.468872in}}%
\pgfpathlineto{\pgfqpoint{5.524917in}{2.453187in}}%
\pgfpathlineto{\pgfqpoint{5.525312in}{2.464675in}}%
\pgfpathlineto{\pgfqpoint{5.527089in}{2.418281in}}%
\pgfpathlineto{\pgfqpoint{5.527780in}{2.407500in}}%
\pgfpathlineto{\pgfqpoint{5.528767in}{2.409624in}}%
\pgfpathlineto{\pgfqpoint{5.529655in}{2.411076in}}%
\pgfpathlineto{\pgfqpoint{5.529260in}{2.404960in}}%
\pgfpathlineto{\pgfqpoint{5.529753in}{2.410030in}}%
\pgfpathlineto{\pgfqpoint{5.529951in}{2.407524in}}%
\pgfpathlineto{\pgfqpoint{5.530247in}{2.419185in}}%
\pgfpathlineto{\pgfqpoint{5.530444in}{2.422264in}}%
\pgfpathlineto{\pgfqpoint{5.531135in}{2.412814in}}%
\pgfpathlineto{\pgfqpoint{5.532122in}{2.408592in}}%
\pgfpathlineto{\pgfqpoint{5.531727in}{2.425500in}}%
\pgfpathlineto{\pgfqpoint{5.532221in}{2.411490in}}%
\pgfpathlineto{\pgfqpoint{5.533800in}{2.435106in}}%
\pgfpathlineto{\pgfqpoint{5.534294in}{2.395105in}}%
\pgfpathlineto{\pgfqpoint{5.534787in}{2.434446in}}%
\pgfpathlineto{\pgfqpoint{5.536564in}{2.668343in}}%
\pgfpathlineto{\pgfqpoint{5.537057in}{2.637942in}}%
\pgfpathlineto{\pgfqpoint{5.539031in}{2.395740in}}%
\pgfpathlineto{\pgfqpoint{5.539426in}{2.406738in}}%
\pgfpathlineto{\pgfqpoint{5.541104in}{2.498812in}}%
\pgfpathlineto{\pgfqpoint{5.540018in}{2.406670in}}%
\pgfpathlineto{\pgfqpoint{5.541499in}{2.464748in}}%
\pgfpathlineto{\pgfqpoint{5.542288in}{2.391861in}}%
\pgfpathlineto{\pgfqpoint{5.542880in}{2.422735in}}%
\pgfpathlineto{\pgfqpoint{5.543670in}{2.437743in}}%
\pgfpathlineto{\pgfqpoint{5.544164in}{2.471535in}}%
\pgfpathlineto{\pgfqpoint{5.544756in}{2.444168in}}%
\pgfpathlineto{\pgfqpoint{5.545447in}{2.399180in}}%
\pgfpathlineto{\pgfqpoint{5.546039in}{2.435101in}}%
\pgfpathlineto{\pgfqpoint{5.547717in}{2.472973in}}%
\pgfpathlineto{\pgfqpoint{5.548111in}{2.456364in}}%
\pgfpathlineto{\pgfqpoint{5.548704in}{2.439017in}}%
\pgfpathlineto{\pgfqpoint{5.549000in}{2.455685in}}%
\pgfpathlineto{\pgfqpoint{5.549987in}{2.476608in}}%
\pgfpathlineto{\pgfqpoint{5.550283in}{2.474159in}}%
\pgfpathlineto{\pgfqpoint{5.550678in}{2.478817in}}%
\pgfpathlineto{\pgfqpoint{5.550974in}{2.472878in}}%
\pgfpathlineto{\pgfqpoint{5.551961in}{2.460462in}}%
\pgfpathlineto{\pgfqpoint{5.551467in}{2.476503in}}%
\pgfpathlineto{\pgfqpoint{5.552158in}{2.466885in}}%
\pgfpathlineto{\pgfqpoint{5.552750in}{2.489359in}}%
\pgfpathlineto{\pgfqpoint{5.553540in}{2.488072in}}%
\pgfpathlineto{\pgfqpoint{5.553836in}{2.492757in}}%
\pgfpathlineto{\pgfqpoint{5.554330in}{2.482897in}}%
\pgfpathlineto{\pgfqpoint{5.554428in}{2.481372in}}%
\pgfpathlineto{\pgfqpoint{5.554724in}{2.490947in}}%
\pgfpathlineto{\pgfqpoint{5.555613in}{2.501048in}}%
\pgfpathlineto{\pgfqpoint{5.555218in}{2.488481in}}%
\pgfpathlineto{\pgfqpoint{5.555909in}{2.495645in}}%
\pgfpathlineto{\pgfqpoint{5.556106in}{2.498940in}}%
\pgfpathlineto{\pgfqpoint{5.557192in}{2.510475in}}%
\pgfpathlineto{\pgfqpoint{5.556698in}{2.495758in}}%
\pgfpathlineto{\pgfqpoint{5.557389in}{2.507587in}}%
\pgfpathlineto{\pgfqpoint{5.557981in}{2.498189in}}%
\pgfpathlineto{\pgfqpoint{5.558179in}{2.505527in}}%
\pgfpathlineto{\pgfqpoint{5.558771in}{2.501832in}}%
\pgfpathlineto{\pgfqpoint{5.559462in}{2.516964in}}%
\pgfpathlineto{\pgfqpoint{5.559561in}{2.517421in}}%
\pgfpathlineto{\pgfqpoint{5.559758in}{2.514828in}}%
\pgfpathlineto{\pgfqpoint{5.560350in}{2.516900in}}%
\pgfpathlineto{\pgfqpoint{5.561633in}{2.493218in}}%
\pgfpathlineto{\pgfqpoint{5.562225in}{2.494389in}}%
\pgfpathlineto{\pgfqpoint{5.562522in}{2.483324in}}%
\pgfpathlineto{\pgfqpoint{5.563607in}{2.472784in}}%
\pgfpathlineto{\pgfqpoint{5.563114in}{2.489451in}}%
\pgfpathlineto{\pgfqpoint{5.563805in}{2.474246in}}%
\pgfpathlineto{\pgfqpoint{5.564101in}{2.479863in}}%
\pgfpathlineto{\pgfqpoint{5.564298in}{2.487000in}}%
\pgfpathlineto{\pgfqpoint{5.564693in}{2.461896in}}%
\pgfpathlineto{\pgfqpoint{5.564792in}{2.457324in}}%
\pgfpathlineto{\pgfqpoint{5.565285in}{2.470495in}}%
\pgfpathlineto{\pgfqpoint{5.565581in}{2.468524in}}%
\pgfpathlineto{\pgfqpoint{5.565779in}{2.468805in}}%
\pgfpathlineto{\pgfqpoint{5.565877in}{2.467533in}}%
\pgfpathlineto{\pgfqpoint{5.566075in}{2.464783in}}%
\pgfpathlineto{\pgfqpoint{5.566371in}{2.474596in}}%
\pgfpathlineto{\pgfqpoint{5.566568in}{2.481620in}}%
\pgfpathlineto{\pgfqpoint{5.567062in}{2.456996in}}%
\pgfpathlineto{\pgfqpoint{5.567555in}{2.470231in}}%
\pgfpathlineto{\pgfqpoint{5.568345in}{2.460258in}}%
\pgfpathlineto{\pgfqpoint{5.568937in}{2.458722in}}%
\pgfpathlineto{\pgfqpoint{5.568641in}{2.461586in}}%
\pgfpathlineto{\pgfqpoint{5.569134in}{2.460355in}}%
\pgfpathlineto{\pgfqpoint{5.570319in}{2.472974in}}%
\pgfpathlineto{\pgfqpoint{5.569825in}{2.458764in}}%
\pgfpathlineto{\pgfqpoint{5.570516in}{2.469777in}}%
\pgfpathlineto{\pgfqpoint{5.570714in}{2.467962in}}%
\pgfpathlineto{\pgfqpoint{5.571010in}{2.479857in}}%
\pgfpathlineto{\pgfqpoint{5.572391in}{2.509703in}}%
\pgfpathlineto{\pgfqpoint{5.572589in}{2.500318in}}%
\pgfpathlineto{\pgfqpoint{5.573280in}{2.505520in}}%
\pgfpathlineto{\pgfqpoint{5.574069in}{2.469850in}}%
\pgfpathlineto{\pgfqpoint{5.575550in}{2.429717in}}%
\pgfpathlineto{\pgfqpoint{5.575747in}{2.432361in}}%
\pgfpathlineto{\pgfqpoint{5.575846in}{2.433750in}}%
\pgfpathlineto{\pgfqpoint{5.576241in}{2.423641in}}%
\pgfpathlineto{\pgfqpoint{5.576833in}{2.398689in}}%
\pgfpathlineto{\pgfqpoint{5.577228in}{2.421383in}}%
\pgfpathlineto{\pgfqpoint{5.577326in}{2.425764in}}%
\pgfpathlineto{\pgfqpoint{5.577721in}{2.414982in}}%
\pgfpathlineto{\pgfqpoint{5.578215in}{2.420877in}}%
\pgfpathlineto{\pgfqpoint{5.579300in}{2.402331in}}%
\pgfpathlineto{\pgfqpoint{5.579695in}{2.412868in}}%
\pgfpathlineto{\pgfqpoint{5.580880in}{2.424712in}}%
\pgfpathlineto{\pgfqpoint{5.581077in}{2.422995in}}%
\pgfpathlineto{\pgfqpoint{5.581965in}{2.387595in}}%
\pgfpathlineto{\pgfqpoint{5.582163in}{2.409511in}}%
\pgfpathlineto{\pgfqpoint{5.583742in}{2.675707in}}%
\pgfpathlineto{\pgfqpoint{5.584433in}{2.650797in}}%
\pgfpathlineto{\pgfqpoint{5.586604in}{2.362823in}}%
\pgfpathlineto{\pgfqpoint{5.587788in}{2.375710in}}%
\pgfpathlineto{\pgfqpoint{5.588677in}{2.463666in}}%
\pgfpathlineto{\pgfqpoint{5.589269in}{2.420962in}}%
\pgfpathlineto{\pgfqpoint{5.589565in}{2.406199in}}%
\pgfpathlineto{\pgfqpoint{5.590059in}{2.441924in}}%
\pgfpathlineto{\pgfqpoint{5.591539in}{2.498863in}}%
\pgfpathlineto{\pgfqpoint{5.591835in}{2.486144in}}%
\pgfpathlineto{\pgfqpoint{5.592822in}{2.424506in}}%
\pgfpathlineto{\pgfqpoint{5.593316in}{2.455385in}}%
\pgfpathlineto{\pgfqpoint{5.594500in}{2.499854in}}%
\pgfpathlineto{\pgfqpoint{5.595388in}{2.493778in}}%
\pgfpathlineto{\pgfqpoint{5.595783in}{2.469121in}}%
\pgfpathlineto{\pgfqpoint{5.596474in}{2.485690in}}%
\pgfpathlineto{\pgfqpoint{5.597066in}{2.506671in}}%
\pgfpathlineto{\pgfqpoint{5.598448in}{2.526224in}}%
\pgfpathlineto{\pgfqpoint{5.598547in}{2.526440in}}%
\pgfpathlineto{\pgfqpoint{5.598645in}{2.524194in}}%
\pgfpathlineto{\pgfqpoint{5.599040in}{2.503185in}}%
\pgfpathlineto{\pgfqpoint{5.599632in}{2.526012in}}%
\pgfpathlineto{\pgfqpoint{5.599928in}{2.523203in}}%
\pgfpathlineto{\pgfqpoint{5.600126in}{2.528177in}}%
\pgfpathlineto{\pgfqpoint{5.600521in}{2.544501in}}%
\pgfpathlineto{\pgfqpoint{5.601212in}{2.530228in}}%
\pgfpathlineto{\pgfqpoint{5.601606in}{2.533089in}}%
\pgfpathlineto{\pgfqpoint{5.601902in}{2.529717in}}%
\pgfpathlineto{\pgfqpoint{5.602495in}{2.514034in}}%
\pgfpathlineto{\pgfqpoint{5.603087in}{2.526914in}}%
\pgfpathlineto{\pgfqpoint{5.603482in}{2.531207in}}%
\pgfpathlineto{\pgfqpoint{5.603876in}{2.524909in}}%
\pgfpathlineto{\pgfqpoint{5.604074in}{2.525624in}}%
\pgfpathlineto{\pgfqpoint{5.604863in}{2.530829in}}%
\pgfpathlineto{\pgfqpoint{5.604469in}{2.524621in}}%
\pgfpathlineto{\pgfqpoint{5.605159in}{2.525824in}}%
\pgfpathlineto{\pgfqpoint{5.605357in}{2.525976in}}%
\pgfpathlineto{\pgfqpoint{5.605554in}{2.523884in}}%
\pgfpathlineto{\pgfqpoint{5.605850in}{2.515229in}}%
\pgfpathlineto{\pgfqpoint{5.606245in}{2.537374in}}%
\pgfpathlineto{\pgfqpoint{5.606443in}{2.531985in}}%
\pgfpathlineto{\pgfqpoint{5.608219in}{2.492165in}}%
\pgfpathlineto{\pgfqpoint{5.608417in}{2.491899in}}%
\pgfpathlineto{\pgfqpoint{5.609601in}{2.474609in}}%
\pgfpathlineto{\pgfqpoint{5.609996in}{2.486160in}}%
\pgfpathlineto{\pgfqpoint{5.610094in}{2.487149in}}%
\pgfpathlineto{\pgfqpoint{5.610292in}{2.478446in}}%
\pgfpathlineto{\pgfqpoint{5.611279in}{2.458023in}}%
\pgfpathlineto{\pgfqpoint{5.611575in}{2.461205in}}%
\pgfpathlineto{\pgfqpoint{5.611772in}{2.458501in}}%
\pgfpathlineto{\pgfqpoint{5.612562in}{2.451818in}}%
\pgfpathlineto{\pgfqpoint{5.612858in}{2.455354in}}%
\pgfpathlineto{\pgfqpoint{5.613746in}{2.458591in}}%
\pgfpathlineto{\pgfqpoint{5.613944in}{2.456324in}}%
\pgfpathlineto{\pgfqpoint{5.614931in}{2.449387in}}%
\pgfpathlineto{\pgfqpoint{5.614536in}{2.460154in}}%
\pgfpathlineto{\pgfqpoint{5.615029in}{2.452000in}}%
\pgfpathlineto{\pgfqpoint{5.615325in}{2.467069in}}%
\pgfpathlineto{\pgfqpoint{5.615720in}{2.451701in}}%
\pgfpathlineto{\pgfqpoint{5.616312in}{2.463213in}}%
\pgfpathlineto{\pgfqpoint{5.617102in}{2.460039in}}%
\pgfpathlineto{\pgfqpoint{5.616707in}{2.465911in}}%
\pgfpathlineto{\pgfqpoint{5.617201in}{2.461182in}}%
\pgfpathlineto{\pgfqpoint{5.619767in}{2.515874in}}%
\pgfpathlineto{\pgfqpoint{5.619866in}{2.514239in}}%
\pgfpathlineto{\pgfqpoint{5.620458in}{2.516335in}}%
\pgfpathlineto{\pgfqpoint{5.623320in}{2.440363in}}%
\pgfpathlineto{\pgfqpoint{5.624504in}{2.433669in}}%
\pgfpathlineto{\pgfqpoint{5.623912in}{2.442349in}}%
\pgfpathlineto{\pgfqpoint{5.624603in}{2.435987in}}%
\pgfpathlineto{\pgfqpoint{5.624899in}{2.443711in}}%
\pgfpathlineto{\pgfqpoint{5.625393in}{2.431930in}}%
\pgfpathlineto{\pgfqpoint{5.625788in}{2.441113in}}%
\pgfpathlineto{\pgfqpoint{5.626182in}{2.443773in}}%
\pgfpathlineto{\pgfqpoint{5.627169in}{2.447991in}}%
\pgfpathlineto{\pgfqpoint{5.626775in}{2.436841in}}%
\pgfpathlineto{\pgfqpoint{5.627268in}{2.446811in}}%
\pgfpathlineto{\pgfqpoint{5.627564in}{2.440516in}}%
\pgfpathlineto{\pgfqpoint{5.628156in}{2.447970in}}%
\pgfpathlineto{\pgfqpoint{5.629143in}{2.461638in}}%
\pgfpathlineto{\pgfqpoint{5.629341in}{2.451286in}}%
\pgfpathlineto{\pgfqpoint{5.629736in}{2.412672in}}%
\pgfpathlineto{\pgfqpoint{5.630130in}{2.453403in}}%
\pgfpathlineto{\pgfqpoint{5.631611in}{2.692723in}}%
\pgfpathlineto{\pgfqpoint{5.632104in}{2.670425in}}%
\pgfpathlineto{\pgfqpoint{5.632203in}{2.671737in}}%
\pgfpathlineto{\pgfqpoint{5.632400in}{2.659988in}}%
\pgfpathlineto{\pgfqpoint{5.634276in}{2.424025in}}%
\pgfpathlineto{\pgfqpoint{5.634572in}{2.433195in}}%
\pgfpathlineto{\pgfqpoint{5.636447in}{2.531483in}}%
\pgfpathlineto{\pgfqpoint{5.636743in}{2.512557in}}%
\pgfpathlineto{\pgfqpoint{5.637829in}{2.421114in}}%
\pgfpathlineto{\pgfqpoint{5.638322in}{2.431695in}}%
\pgfpathlineto{\pgfqpoint{5.639408in}{2.494706in}}%
\pgfpathlineto{\pgfqpoint{5.639902in}{2.471495in}}%
\pgfpathlineto{\pgfqpoint{5.640691in}{2.436663in}}%
\pgfpathlineto{\pgfqpoint{5.641185in}{2.458187in}}%
\pgfpathlineto{\pgfqpoint{5.642862in}{2.487455in}}%
\pgfpathlineto{\pgfqpoint{5.642961in}{2.485426in}}%
\pgfpathlineto{\pgfqpoint{5.644244in}{2.448470in}}%
\pgfpathlineto{\pgfqpoint{5.644639in}{2.464613in}}%
\pgfpathlineto{\pgfqpoint{5.646021in}{2.487465in}}%
\pgfpathlineto{\pgfqpoint{5.646120in}{2.486730in}}%
\pgfpathlineto{\pgfqpoint{5.647008in}{2.459208in}}%
\pgfpathlineto{\pgfqpoint{5.647896in}{2.464490in}}%
\pgfpathlineto{\pgfqpoint{5.648094in}{2.466734in}}%
\pgfpathlineto{\pgfqpoint{5.648587in}{2.462846in}}%
\pgfpathlineto{\pgfqpoint{5.649771in}{2.437956in}}%
\pgfpathlineto{\pgfqpoint{5.649377in}{2.470144in}}%
\pgfpathlineto{\pgfqpoint{5.649969in}{2.443820in}}%
\pgfpathlineto{\pgfqpoint{5.651252in}{2.466743in}}%
\pgfpathlineto{\pgfqpoint{5.651351in}{2.466730in}}%
\pgfpathlineto{\pgfqpoint{5.652239in}{2.457273in}}%
\pgfpathlineto{\pgfqpoint{5.652535in}{2.464033in}}%
\pgfpathlineto{\pgfqpoint{5.652634in}{2.466490in}}%
\pgfpathlineto{\pgfqpoint{5.653127in}{2.451838in}}%
\pgfpathlineto{\pgfqpoint{5.653522in}{2.458926in}}%
\pgfpathlineto{\pgfqpoint{5.653818in}{2.450713in}}%
\pgfpathlineto{\pgfqpoint{5.656976in}{2.363603in}}%
\pgfpathlineto{\pgfqpoint{5.657667in}{2.367619in}}%
\pgfpathlineto{\pgfqpoint{5.659247in}{2.422665in}}%
\pgfpathlineto{\pgfqpoint{5.660332in}{2.415077in}}%
\pgfpathlineto{\pgfqpoint{5.662405in}{2.386693in}}%
\pgfpathlineto{\pgfqpoint{5.662701in}{2.400195in}}%
\pgfpathlineto{\pgfqpoint{5.662898in}{2.408629in}}%
\pgfpathlineto{\pgfqpoint{5.663392in}{2.383043in}}%
\pgfpathlineto{\pgfqpoint{5.663688in}{2.394107in}}%
\pgfpathlineto{\pgfqpoint{5.663787in}{2.396166in}}%
\pgfpathlineto{\pgfqpoint{5.664280in}{2.385977in}}%
\pgfpathlineto{\pgfqpoint{5.665267in}{2.376157in}}%
\pgfpathlineto{\pgfqpoint{5.665465in}{2.381644in}}%
\pgfpathlineto{\pgfqpoint{5.667340in}{2.428625in}}%
\pgfpathlineto{\pgfqpoint{5.667439in}{2.428688in}}%
\pgfpathlineto{\pgfqpoint{5.667537in}{2.427497in}}%
\pgfpathlineto{\pgfqpoint{5.667636in}{2.427087in}}%
\pgfpathlineto{\pgfqpoint{5.667735in}{2.428694in}}%
\pgfpathlineto{\pgfqpoint{5.668722in}{2.449620in}}%
\pgfpathlineto{\pgfqpoint{5.669215in}{2.446337in}}%
\pgfpathlineto{\pgfqpoint{5.671781in}{2.402828in}}%
\pgfpathlineto{\pgfqpoint{5.671979in}{2.409725in}}%
\pgfpathlineto{\pgfqpoint{5.672176in}{2.415255in}}%
\pgfpathlineto{\pgfqpoint{5.672670in}{2.401644in}}%
\pgfpathlineto{\pgfqpoint{5.673163in}{2.413864in}}%
\pgfpathlineto{\pgfqpoint{5.674347in}{2.402071in}}%
\pgfpathlineto{\pgfqpoint{5.674940in}{2.405292in}}%
\pgfpathlineto{\pgfqpoint{5.676124in}{2.426009in}}%
\pgfpathlineto{\pgfqpoint{5.676420in}{2.413486in}}%
\pgfpathlineto{\pgfqpoint{5.677901in}{2.368625in}}%
\pgfpathlineto{\pgfqpoint{5.677012in}{2.414068in}}%
\pgfpathlineto{\pgfqpoint{5.678098in}{2.387696in}}%
\pgfpathlineto{\pgfqpoint{5.679776in}{2.654468in}}%
\pgfpathlineto{\pgfqpoint{5.680269in}{2.632641in}}%
\pgfpathlineto{\pgfqpoint{5.680664in}{2.588445in}}%
\pgfpathlineto{\pgfqpoint{5.682441in}{2.388132in}}%
\pgfpathlineto{\pgfqpoint{5.682737in}{2.377735in}}%
\pgfpathlineto{\pgfqpoint{5.683428in}{2.394769in}}%
\pgfpathlineto{\pgfqpoint{5.684119in}{2.461503in}}%
\pgfpathlineto{\pgfqpoint{5.684513in}{2.485564in}}%
\pgfpathlineto{\pgfqpoint{5.685007in}{2.452197in}}%
\pgfpathlineto{\pgfqpoint{5.685797in}{2.399908in}}%
\pgfpathlineto{\pgfqpoint{5.686191in}{2.427789in}}%
\pgfpathlineto{\pgfqpoint{5.687672in}{2.483894in}}%
\pgfpathlineto{\pgfqpoint{5.687771in}{2.483211in}}%
\pgfpathlineto{\pgfqpoint{5.688560in}{2.435998in}}%
\pgfpathlineto{\pgfqpoint{5.688856in}{2.422686in}}%
\pgfpathlineto{\pgfqpoint{5.689448in}{2.452447in}}%
\pgfpathlineto{\pgfqpoint{5.690929in}{2.478228in}}%
\pgfpathlineto{\pgfqpoint{5.691126in}{2.475630in}}%
\pgfpathlineto{\pgfqpoint{5.692212in}{2.448204in}}%
\pgfpathlineto{\pgfqpoint{5.692409in}{2.457349in}}%
\pgfpathlineto{\pgfqpoint{5.693890in}{2.494307in}}%
\pgfpathlineto{\pgfqpoint{5.694186in}{2.501137in}}%
\pgfpathlineto{\pgfqpoint{5.694679in}{2.486106in}}%
\pgfpathlineto{\pgfqpoint{5.694976in}{2.476636in}}%
\pgfpathlineto{\pgfqpoint{5.695370in}{2.491241in}}%
\pgfpathlineto{\pgfqpoint{5.695765in}{2.480012in}}%
\pgfpathlineto{\pgfqpoint{5.697048in}{2.513986in}}%
\pgfpathlineto{\pgfqpoint{5.697246in}{2.498044in}}%
\pgfpathlineto{\pgfqpoint{5.697443in}{2.481541in}}%
\pgfpathlineto{\pgfqpoint{5.697838in}{2.505312in}}%
\pgfpathlineto{\pgfqpoint{5.698331in}{2.490902in}}%
\pgfpathlineto{\pgfqpoint{5.699614in}{2.507916in}}%
\pgfpathlineto{\pgfqpoint{5.699022in}{2.490759in}}%
\pgfpathlineto{\pgfqpoint{5.699812in}{2.499904in}}%
\pgfpathlineto{\pgfqpoint{5.701095in}{2.483877in}}%
\pgfpathlineto{\pgfqpoint{5.700503in}{2.505333in}}%
\pgfpathlineto{\pgfqpoint{5.701194in}{2.484009in}}%
\pgfpathlineto{\pgfqpoint{5.701292in}{2.484201in}}%
\pgfpathlineto{\pgfqpoint{5.701391in}{2.483756in}}%
\pgfpathlineto{\pgfqpoint{5.701687in}{2.478643in}}%
\pgfpathlineto{\pgfqpoint{5.702082in}{2.490301in}}%
\pgfpathlineto{\pgfqpoint{5.702279in}{2.486600in}}%
\pgfpathlineto{\pgfqpoint{5.702871in}{2.493053in}}%
\pgfpathlineto{\pgfqpoint{5.703957in}{2.457594in}}%
\pgfpathlineto{\pgfqpoint{5.705142in}{2.433797in}}%
\pgfpathlineto{\pgfqpoint{5.704451in}{2.459637in}}%
\pgfpathlineto{\pgfqpoint{5.705438in}{2.444219in}}%
\pgfpathlineto{\pgfqpoint{5.705635in}{2.448426in}}%
\pgfpathlineto{\pgfqpoint{5.706030in}{2.427688in}}%
\pgfpathlineto{\pgfqpoint{5.706129in}{2.426401in}}%
\pgfpathlineto{\pgfqpoint{5.706523in}{2.434568in}}%
\pgfpathlineto{\pgfqpoint{5.706721in}{2.433194in}}%
\pgfpathlineto{\pgfqpoint{5.707017in}{2.433983in}}%
\pgfpathlineto{\pgfqpoint{5.707313in}{2.428704in}}%
\pgfpathlineto{\pgfqpoint{5.707905in}{2.421545in}}%
\pgfpathlineto{\pgfqpoint{5.708201in}{2.430398in}}%
\pgfpathlineto{\pgfqpoint{5.708300in}{2.434403in}}%
\pgfpathlineto{\pgfqpoint{5.708793in}{2.413914in}}%
\pgfpathlineto{\pgfqpoint{5.709188in}{2.427643in}}%
\pgfpathlineto{\pgfqpoint{5.709484in}{2.418928in}}%
\pgfpathlineto{\pgfqpoint{5.709978in}{2.434538in}}%
\pgfpathlineto{\pgfqpoint{5.710076in}{2.433702in}}%
\pgfpathlineto{\pgfqpoint{5.710767in}{2.423647in}}%
\pgfpathlineto{\pgfqpoint{5.711162in}{2.433621in}}%
\pgfpathlineto{\pgfqpoint{5.711261in}{2.434514in}}%
\pgfpathlineto{\pgfqpoint{5.711557in}{2.428328in}}%
\pgfpathlineto{\pgfqpoint{5.712544in}{2.421391in}}%
\pgfpathlineto{\pgfqpoint{5.712149in}{2.433502in}}%
\pgfpathlineto{\pgfqpoint{5.712643in}{2.423277in}}%
\pgfpathlineto{\pgfqpoint{5.714617in}{2.458121in}}%
\pgfpathlineto{\pgfqpoint{5.715900in}{2.478217in}}%
\pgfpathlineto{\pgfqpoint{5.715209in}{2.450992in}}%
\pgfpathlineto{\pgfqpoint{5.716097in}{2.471761in}}%
\pgfpathlineto{\pgfqpoint{5.716788in}{2.477891in}}%
\pgfpathlineto{\pgfqpoint{5.717676in}{2.438862in}}%
\pgfpathlineto{\pgfqpoint{5.717775in}{2.438712in}}%
\pgfpathlineto{\pgfqpoint{5.718071in}{2.444562in}}%
\pgfpathlineto{\pgfqpoint{5.718367in}{2.432661in}}%
\pgfpathlineto{\pgfqpoint{5.719157in}{2.435631in}}%
\pgfpathlineto{\pgfqpoint{5.719848in}{2.411437in}}%
\pgfpathlineto{\pgfqpoint{5.721032in}{2.399890in}}%
\pgfpathlineto{\pgfqpoint{5.720539in}{2.415123in}}%
\pgfpathlineto{\pgfqpoint{5.721723in}{2.401163in}}%
\pgfpathlineto{\pgfqpoint{5.722414in}{2.402679in}}%
\pgfpathlineto{\pgfqpoint{5.722611in}{2.400875in}}%
\pgfpathlineto{\pgfqpoint{5.723401in}{2.403510in}}%
\pgfpathlineto{\pgfqpoint{5.725375in}{2.354116in}}%
\pgfpathlineto{\pgfqpoint{5.725770in}{2.305556in}}%
\pgfpathlineto{\pgfqpoint{5.726263in}{2.384852in}}%
\pgfpathlineto{\pgfqpoint{5.727941in}{2.676796in}}%
\pgfpathlineto{\pgfqpoint{5.728533in}{2.638625in}}%
\pgfpathlineto{\pgfqpoint{5.730310in}{2.398171in}}%
\pgfpathlineto{\pgfqpoint{5.730606in}{2.405442in}}%
\pgfpathlineto{\pgfqpoint{5.732580in}{2.518893in}}%
\pgfpathlineto{\pgfqpoint{5.732975in}{2.469636in}}%
\pgfpathlineto{\pgfqpoint{5.733666in}{2.412943in}}%
\pgfpathlineto{\pgfqpoint{5.734159in}{2.452604in}}%
\pgfpathlineto{\pgfqpoint{5.735837in}{2.505808in}}%
\pgfpathlineto{\pgfqpoint{5.734653in}{2.447022in}}%
\pgfpathlineto{\pgfqpoint{5.736133in}{2.487444in}}%
\pgfpathlineto{\pgfqpoint{5.736725in}{2.453260in}}%
\pgfpathlineto{\pgfqpoint{5.737416in}{2.467033in}}%
\pgfpathlineto{\pgfqpoint{5.738600in}{2.500490in}}%
\pgfpathlineto{\pgfqpoint{5.739094in}{2.492835in}}%
\pgfpathlineto{\pgfqpoint{5.739291in}{2.491828in}}%
\pgfpathlineto{\pgfqpoint{5.739982in}{2.473085in}}%
\pgfpathlineto{\pgfqpoint{5.740476in}{2.487584in}}%
\pgfpathlineto{\pgfqpoint{5.741167in}{2.517173in}}%
\pgfpathlineto{\pgfqpoint{5.742647in}{2.507429in}}%
\pgfpathlineto{\pgfqpoint{5.742845in}{2.507785in}}%
\pgfpathlineto{\pgfqpoint{5.742943in}{2.507290in}}%
\pgfpathlineto{\pgfqpoint{5.743239in}{2.503617in}}%
\pgfpathlineto{\pgfqpoint{5.743634in}{2.512198in}}%
\pgfpathlineto{\pgfqpoint{5.744325in}{2.533876in}}%
\pgfpathlineto{\pgfqpoint{5.744720in}{2.513452in}}%
\pgfpathlineto{\pgfqpoint{5.744917in}{2.508543in}}%
\pgfpathlineto{\pgfqpoint{5.745312in}{2.523421in}}%
\pgfpathlineto{\pgfqpoint{5.745805in}{2.511625in}}%
\pgfpathlineto{\pgfqpoint{5.746003in}{2.515341in}}%
\pgfpathlineto{\pgfqpoint{5.746398in}{2.500986in}}%
\pgfpathlineto{\pgfqpoint{5.746496in}{2.498919in}}%
\pgfpathlineto{\pgfqpoint{5.746990in}{2.509700in}}%
\pgfpathlineto{\pgfqpoint{5.747187in}{2.515800in}}%
\pgfpathlineto{\pgfqpoint{5.747582in}{2.496229in}}%
\pgfpathlineto{\pgfqpoint{5.748174in}{2.499862in}}%
\pgfpathlineto{\pgfqpoint{5.748964in}{2.477966in}}%
\pgfpathlineto{\pgfqpoint{5.749260in}{2.479035in}}%
\pgfpathlineto{\pgfqpoint{5.749457in}{2.476841in}}%
\pgfpathlineto{\pgfqpoint{5.750444in}{2.470668in}}%
\pgfpathlineto{\pgfqpoint{5.750050in}{2.477627in}}%
\pgfpathlineto{\pgfqpoint{5.750642in}{2.474148in}}%
\pgfpathlineto{\pgfqpoint{5.751530in}{2.479029in}}%
\pgfpathlineto{\pgfqpoint{5.751135in}{2.473049in}}%
\pgfpathlineto{\pgfqpoint{5.751629in}{2.477139in}}%
\pgfpathlineto{\pgfqpoint{5.753603in}{2.436339in}}%
\pgfpathlineto{\pgfqpoint{5.753701in}{2.436639in}}%
\pgfpathlineto{\pgfqpoint{5.754096in}{2.439814in}}%
\pgfpathlineto{\pgfqpoint{5.754491in}{2.433669in}}%
\pgfpathlineto{\pgfqpoint{5.754590in}{2.433550in}}%
\pgfpathlineto{\pgfqpoint{5.754688in}{2.434437in}}%
\pgfpathlineto{\pgfqpoint{5.754886in}{2.436655in}}%
\pgfpathlineto{\pgfqpoint{5.755281in}{2.426707in}}%
\pgfpathlineto{\pgfqpoint{5.756268in}{2.419717in}}%
\pgfpathlineto{\pgfqpoint{5.755873in}{2.431594in}}%
\pgfpathlineto{\pgfqpoint{5.756465in}{2.422251in}}%
\pgfpathlineto{\pgfqpoint{5.756662in}{2.425436in}}%
\pgfpathlineto{\pgfqpoint{5.757057in}{2.417370in}}%
\pgfpathlineto{\pgfqpoint{5.757551in}{2.422913in}}%
\pgfpathlineto{\pgfqpoint{5.759426in}{2.401650in}}%
\pgfpathlineto{\pgfqpoint{5.759525in}{2.401267in}}%
\pgfpathlineto{\pgfqpoint{5.759623in}{2.402766in}}%
\pgfpathlineto{\pgfqpoint{5.760413in}{2.418536in}}%
\pgfpathlineto{\pgfqpoint{5.760808in}{2.404937in}}%
\pgfpathlineto{\pgfqpoint{5.760906in}{2.403248in}}%
\pgfpathlineto{\pgfqpoint{5.761203in}{2.414004in}}%
\pgfpathlineto{\pgfqpoint{5.763571in}{2.455139in}}%
\pgfpathlineto{\pgfqpoint{5.763670in}{2.453798in}}%
\pgfpathlineto{\pgfqpoint{5.763966in}{2.449398in}}%
\pgfpathlineto{\pgfqpoint{5.764361in}{2.455423in}}%
\pgfpathlineto{\pgfqpoint{5.764854in}{2.451800in}}%
\pgfpathlineto{\pgfqpoint{5.765052in}{2.455294in}}%
\pgfpathlineto{\pgfqpoint{5.765348in}{2.444955in}}%
\pgfpathlineto{\pgfqpoint{5.767124in}{2.401320in}}%
\pgfpathlineto{\pgfqpoint{5.767223in}{2.400293in}}%
\pgfpathlineto{\pgfqpoint{5.767519in}{2.406845in}}%
\pgfpathlineto{\pgfqpoint{5.767717in}{2.409576in}}%
\pgfpathlineto{\pgfqpoint{5.768309in}{2.400340in}}%
\pgfpathlineto{\pgfqpoint{5.768408in}{2.399922in}}%
\pgfpathlineto{\pgfqpoint{5.768605in}{2.402905in}}%
\pgfpathlineto{\pgfqpoint{5.768901in}{2.407961in}}%
\pgfpathlineto{\pgfqpoint{5.769592in}{2.401368in}}%
\pgfpathlineto{\pgfqpoint{5.770382in}{2.398123in}}%
\pgfpathlineto{\pgfqpoint{5.770085in}{2.402467in}}%
\pgfpathlineto{\pgfqpoint{5.770480in}{2.400099in}}%
\pgfpathlineto{\pgfqpoint{5.770875in}{2.421882in}}%
\pgfpathlineto{\pgfqpoint{5.771763in}{2.411989in}}%
\pgfpathlineto{\pgfqpoint{5.772750in}{2.417940in}}%
\pgfpathlineto{\pgfqpoint{5.772257in}{2.404771in}}%
\pgfpathlineto{\pgfqpoint{5.772849in}{2.415377in}}%
\pgfpathlineto{\pgfqpoint{5.773540in}{2.381814in}}%
\pgfpathlineto{\pgfqpoint{5.773836in}{2.401824in}}%
\pgfpathlineto{\pgfqpoint{5.775316in}{2.665808in}}%
\pgfpathlineto{\pgfqpoint{5.776205in}{2.623773in}}%
\pgfpathlineto{\pgfqpoint{5.778080in}{2.396747in}}%
\pgfpathlineto{\pgfqpoint{5.778277in}{2.402116in}}%
\pgfpathlineto{\pgfqpoint{5.779264in}{2.437873in}}%
\pgfpathlineto{\pgfqpoint{5.780153in}{2.515721in}}%
\pgfpathlineto{\pgfqpoint{5.780646in}{2.454411in}}%
\pgfpathlineto{\pgfqpoint{5.781436in}{2.404541in}}%
\pgfpathlineto{\pgfqpoint{5.781831in}{2.439984in}}%
\pgfpathlineto{\pgfqpoint{5.783311in}{2.493453in}}%
\pgfpathlineto{\pgfqpoint{5.782225in}{2.437505in}}%
\pgfpathlineto{\pgfqpoint{5.783805in}{2.471528in}}%
\pgfpathlineto{\pgfqpoint{5.784495in}{2.427957in}}%
\pgfpathlineto{\pgfqpoint{5.785088in}{2.460197in}}%
\pgfpathlineto{\pgfqpoint{5.786766in}{2.482080in}}%
\pgfpathlineto{\pgfqpoint{5.785581in}{2.455720in}}%
\pgfpathlineto{\pgfqpoint{5.786963in}{2.479417in}}%
\pgfpathlineto{\pgfqpoint{5.787950in}{2.459174in}}%
\pgfpathlineto{\pgfqpoint{5.788246in}{2.468385in}}%
\pgfpathlineto{\pgfqpoint{5.789134in}{2.493964in}}%
\pgfpathlineto{\pgfqpoint{5.789430in}{2.478922in}}%
\pgfpathlineto{\pgfqpoint{5.792293in}{2.416046in}}%
\pgfpathlineto{\pgfqpoint{5.789924in}{2.480394in}}%
\pgfpathlineto{\pgfqpoint{5.792391in}{2.416960in}}%
\pgfpathlineto{\pgfqpoint{5.795352in}{2.504376in}}%
\pgfpathlineto{\pgfqpoint{5.795846in}{2.488483in}}%
\pgfpathlineto{\pgfqpoint{5.797129in}{2.476817in}}%
\pgfpathlineto{\pgfqpoint{5.796339in}{2.493398in}}%
\pgfpathlineto{\pgfqpoint{5.797524in}{2.480212in}}%
\pgfpathlineto{\pgfqpoint{5.797820in}{2.480563in}}%
\pgfpathlineto{\pgfqpoint{5.798116in}{2.479496in}}%
\pgfpathlineto{\pgfqpoint{5.800583in}{2.445230in}}%
\pgfpathlineto{\pgfqpoint{5.798807in}{2.481323in}}%
\pgfpathlineto{\pgfqpoint{5.800781in}{2.448501in}}%
\pgfpathlineto{\pgfqpoint{5.801570in}{2.458031in}}%
\pgfpathlineto{\pgfqpoint{5.801866in}{2.450012in}}%
\pgfpathlineto{\pgfqpoint{5.803742in}{2.414651in}}%
\pgfpathlineto{\pgfqpoint{5.804038in}{2.413844in}}%
\pgfpathlineto{\pgfqpoint{5.804137in}{2.415652in}}%
\pgfpathlineto{\pgfqpoint{5.804531in}{2.427551in}}%
\pgfpathlineto{\pgfqpoint{5.805222in}{2.418282in}}%
\pgfpathlineto{\pgfqpoint{5.806111in}{2.410706in}}%
\pgfpathlineto{\pgfqpoint{5.806308in}{2.415238in}}%
\pgfpathlineto{\pgfqpoint{5.807295in}{2.423478in}}%
\pgfpathlineto{\pgfqpoint{5.806900in}{2.412758in}}%
\pgfpathlineto{\pgfqpoint{5.807394in}{2.420775in}}%
\pgfpathlineto{\pgfqpoint{5.807690in}{2.406118in}}%
\pgfpathlineto{\pgfqpoint{5.808479in}{2.417407in}}%
\pgfpathlineto{\pgfqpoint{5.809368in}{2.428497in}}%
\pgfpathlineto{\pgfqpoint{5.809664in}{2.419894in}}%
\pgfpathlineto{\pgfqpoint{5.809762in}{2.417626in}}%
\pgfpathlineto{\pgfqpoint{5.810059in}{2.431499in}}%
\pgfpathlineto{\pgfqpoint{5.810453in}{2.448094in}}%
\pgfpathlineto{\pgfqpoint{5.811243in}{2.441869in}}%
\pgfpathlineto{\pgfqpoint{5.811440in}{2.445283in}}%
\pgfpathlineto{\pgfqpoint{5.811934in}{2.436863in}}%
\pgfpathlineto{\pgfqpoint{5.812329in}{2.443341in}}%
\pgfpathlineto{\pgfqpoint{5.812723in}{2.426858in}}%
\pgfpathlineto{\pgfqpoint{5.815487in}{2.361838in}}%
\pgfpathlineto{\pgfqpoint{5.816277in}{2.353678in}}%
\pgfpathlineto{\pgfqpoint{5.816474in}{2.359936in}}%
\pgfpathlineto{\pgfqpoint{5.817856in}{2.372079in}}%
\pgfpathlineto{\pgfqpoint{5.818448in}{2.365839in}}%
\pgfpathlineto{\pgfqpoint{5.819040in}{2.375007in}}%
\pgfpathlineto{\pgfqpoint{5.819139in}{2.374861in}}%
\pgfpathlineto{\pgfqpoint{5.819238in}{2.375340in}}%
\pgfpathlineto{\pgfqpoint{5.820323in}{2.383205in}}%
\pgfpathlineto{\pgfqpoint{5.819928in}{2.370925in}}%
\pgfpathlineto{\pgfqpoint{5.820422in}{2.381001in}}%
\pgfpathlineto{\pgfqpoint{5.820915in}{2.354893in}}%
\pgfpathlineto{\pgfqpoint{5.821310in}{2.386337in}}%
\pgfpathlineto{\pgfqpoint{5.822791in}{2.630544in}}%
\pgfpathlineto{\pgfqpoint{5.823580in}{2.616259in}}%
\pgfpathlineto{\pgfqpoint{5.825554in}{2.355111in}}%
\pgfpathlineto{\pgfqpoint{5.825949in}{2.371062in}}%
\pgfpathlineto{\pgfqpoint{5.827528in}{2.471672in}}%
\pgfpathlineto{\pgfqpoint{5.827923in}{2.447136in}}%
\pgfpathlineto{\pgfqpoint{5.828713in}{2.383902in}}%
\pgfpathlineto{\pgfqpoint{5.829305in}{2.408033in}}%
\pgfpathlineto{\pgfqpoint{5.830785in}{2.473247in}}%
\pgfpathlineto{\pgfqpoint{5.831180in}{2.446154in}}%
\pgfpathlineto{\pgfqpoint{5.831871in}{2.401793in}}%
\pgfpathlineto{\pgfqpoint{5.832364in}{2.436586in}}%
\pgfpathlineto{\pgfqpoint{5.832661in}{2.431440in}}%
\pgfpathlineto{\pgfqpoint{5.832957in}{2.440323in}}%
\pgfpathlineto{\pgfqpoint{5.834141in}{2.471927in}}%
\pgfpathlineto{\pgfqpoint{5.834437in}{2.465839in}}%
\pgfpathlineto{\pgfqpoint{5.835424in}{2.437570in}}%
\pgfpathlineto{\pgfqpoint{5.835720in}{2.446699in}}%
\pgfpathlineto{\pgfqpoint{5.836806in}{2.473082in}}%
\pgfpathlineto{\pgfqpoint{5.837102in}{2.469228in}}%
\pgfpathlineto{\pgfqpoint{5.837497in}{2.473757in}}%
\pgfpathlineto{\pgfqpoint{5.837694in}{2.468945in}}%
\pgfpathlineto{\pgfqpoint{5.838681in}{2.454613in}}%
\pgfpathlineto{\pgfqpoint{5.838188in}{2.475517in}}%
\pgfpathlineto{\pgfqpoint{5.838879in}{2.465971in}}%
\pgfpathlineto{\pgfqpoint{5.839175in}{2.490978in}}%
\pgfpathlineto{\pgfqpoint{5.840063in}{2.477858in}}%
\pgfpathlineto{\pgfqpoint{5.841346in}{2.460796in}}%
\pgfpathlineto{\pgfqpoint{5.841543in}{2.466388in}}%
\pgfpathlineto{\pgfqpoint{5.842629in}{2.491830in}}%
\pgfpathlineto{\pgfqpoint{5.842827in}{2.483651in}}%
\pgfpathlineto{\pgfqpoint{5.844110in}{2.468121in}}%
\pgfpathlineto{\pgfqpoint{5.844504in}{2.471285in}}%
\pgfpathlineto{\pgfqpoint{5.844801in}{2.466860in}}%
\pgfpathlineto{\pgfqpoint{5.845886in}{2.457277in}}%
\pgfpathlineto{\pgfqpoint{5.846084in}{2.459672in}}%
\pgfpathlineto{\pgfqpoint{5.846281in}{2.463544in}}%
\pgfpathlineto{\pgfqpoint{5.846577in}{2.452635in}}%
\pgfpathlineto{\pgfqpoint{5.847564in}{2.429764in}}%
\pgfpathlineto{\pgfqpoint{5.847860in}{2.437678in}}%
\pgfpathlineto{\pgfqpoint{5.847959in}{2.438757in}}%
\pgfpathlineto{\pgfqpoint{5.848255in}{2.431098in}}%
\pgfpathlineto{\pgfqpoint{5.848551in}{2.431901in}}%
\pgfpathlineto{\pgfqpoint{5.848748in}{2.424620in}}%
\pgfpathlineto{\pgfqpoint{5.849834in}{2.405495in}}%
\pgfpathlineto{\pgfqpoint{5.850032in}{2.408613in}}%
\pgfpathlineto{\pgfqpoint{5.850229in}{2.412945in}}%
\pgfpathlineto{\pgfqpoint{5.850821in}{2.401316in}}%
\pgfpathlineto{\pgfqpoint{5.851315in}{2.393515in}}%
\pgfpathlineto{\pgfqpoint{5.851611in}{2.403340in}}%
\pgfpathlineto{\pgfqpoint{5.851709in}{2.405425in}}%
\pgfpathlineto{\pgfqpoint{5.852203in}{2.396746in}}%
\pgfpathlineto{\pgfqpoint{5.852696in}{2.403377in}}%
\pgfpathlineto{\pgfqpoint{5.852993in}{2.396880in}}%
\pgfpathlineto{\pgfqpoint{5.853289in}{2.390674in}}%
\pgfpathlineto{\pgfqpoint{5.853683in}{2.397339in}}%
\pgfpathlineto{\pgfqpoint{5.854078in}{2.394158in}}%
\pgfpathlineto{\pgfqpoint{5.854374in}{2.398648in}}%
\pgfpathlineto{\pgfqpoint{5.854572in}{2.392610in}}%
\pgfpathlineto{\pgfqpoint{5.854868in}{2.377258in}}%
\pgfpathlineto{\pgfqpoint{5.855263in}{2.401144in}}%
\pgfpathlineto{\pgfqpoint{5.855657in}{2.390838in}}%
\pgfpathlineto{\pgfqpoint{5.855954in}{2.399285in}}%
\pgfpathlineto{\pgfqpoint{5.856250in}{2.385227in}}%
\pgfpathlineto{\pgfqpoint{5.856743in}{2.396247in}}%
\pgfpathlineto{\pgfqpoint{5.857631in}{2.375009in}}%
\pgfpathlineto{\pgfqpoint{5.858026in}{2.382979in}}%
\pgfpathlineto{\pgfqpoint{5.858224in}{2.372605in}}%
\pgfpathlineto{\pgfqpoint{5.858520in}{2.352854in}}%
\pgfpathlineto{\pgfqpoint{5.859309in}{2.373272in}}%
\pgfpathlineto{\pgfqpoint{5.860987in}{2.426800in}}%
\pgfpathlineto{\pgfqpoint{5.861185in}{2.422197in}}%
\pgfpathlineto{\pgfqpoint{5.862665in}{2.388290in}}%
\pgfpathlineto{\pgfqpoint{5.862862in}{2.389726in}}%
\pgfpathlineto{\pgfqpoint{5.863159in}{2.384655in}}%
\pgfpathlineto{\pgfqpoint{5.863356in}{2.380913in}}%
\pgfpathlineto{\pgfqpoint{5.863849in}{2.390813in}}%
\pgfpathlineto{\pgfqpoint{5.864146in}{2.387399in}}%
\pgfpathlineto{\pgfqpoint{5.864738in}{2.398995in}}%
\pgfpathlineto{\pgfqpoint{5.866021in}{2.395640in}}%
\pgfpathlineto{\pgfqpoint{5.866416in}{2.385695in}}%
\pgfpathlineto{\pgfqpoint{5.866909in}{2.397880in}}%
\pgfpathlineto{\pgfqpoint{5.867106in}{2.400509in}}%
\pgfpathlineto{\pgfqpoint{5.867403in}{2.390233in}}%
\pgfpathlineto{\pgfqpoint{5.868488in}{2.372839in}}%
\pgfpathlineto{\pgfqpoint{5.867896in}{2.400384in}}%
\pgfpathlineto{\pgfqpoint{5.868587in}{2.377996in}}%
\pgfpathlineto{\pgfqpoint{5.870364in}{2.659929in}}%
\pgfpathlineto{\pgfqpoint{5.871252in}{2.593191in}}%
\pgfpathlineto{\pgfqpoint{5.872831in}{2.385876in}}%
\pgfpathlineto{\pgfqpoint{5.872930in}{2.383817in}}%
\pgfpathlineto{\pgfqpoint{5.873226in}{2.397243in}}%
\pgfpathlineto{\pgfqpoint{5.875101in}{2.506552in}}%
\pgfpathlineto{\pgfqpoint{5.875496in}{2.468470in}}%
\pgfpathlineto{\pgfqpoint{5.876285in}{2.417047in}}%
\pgfpathlineto{\pgfqpoint{5.876779in}{2.434814in}}%
\pgfpathlineto{\pgfqpoint{5.878259in}{2.484358in}}%
\pgfpathlineto{\pgfqpoint{5.878556in}{2.472990in}}%
\pgfpathlineto{\pgfqpoint{5.879543in}{2.421823in}}%
\pgfpathlineto{\pgfqpoint{5.879937in}{2.441318in}}%
\pgfpathlineto{\pgfqpoint{5.881023in}{2.464133in}}%
\pgfpathlineto{\pgfqpoint{5.881319in}{2.457433in}}%
\pgfpathlineto{\pgfqpoint{5.881714in}{2.460081in}}%
\pgfpathlineto{\pgfqpoint{5.882109in}{2.448671in}}%
\pgfpathlineto{\pgfqpoint{5.882602in}{2.431486in}}%
\pgfpathlineto{\pgfqpoint{5.883194in}{2.445708in}}%
\pgfpathlineto{\pgfqpoint{5.884379in}{2.479729in}}%
\pgfpathlineto{\pgfqpoint{5.884675in}{2.465788in}}%
\pgfpathlineto{\pgfqpoint{5.885859in}{2.447423in}}%
\pgfpathlineto{\pgfqpoint{5.885168in}{2.469627in}}%
\pgfpathlineto{\pgfqpoint{5.886057in}{2.451575in}}%
\pgfpathlineto{\pgfqpoint{5.887241in}{2.478611in}}%
\pgfpathlineto{\pgfqpoint{5.887537in}{2.475678in}}%
\pgfpathlineto{\pgfqpoint{5.888524in}{2.480588in}}%
\pgfpathlineto{\pgfqpoint{5.888129in}{2.468793in}}%
\pgfpathlineto{\pgfqpoint{5.888623in}{2.478953in}}%
\pgfpathlineto{\pgfqpoint{5.889018in}{2.462156in}}%
\pgfpathlineto{\pgfqpoint{5.889709in}{2.478681in}}%
\pgfpathlineto{\pgfqpoint{5.890794in}{2.463545in}}%
\pgfpathlineto{\pgfqpoint{5.891189in}{2.471650in}}%
\pgfpathlineto{\pgfqpoint{5.891683in}{2.472334in}}%
\pgfpathlineto{\pgfqpoint{5.891880in}{2.470575in}}%
\pgfpathlineto{\pgfqpoint{5.892472in}{2.460591in}}%
\pgfpathlineto{\pgfqpoint{5.892867in}{2.470205in}}%
\pgfpathlineto{\pgfqpoint{5.893064in}{2.471845in}}%
\pgfpathlineto{\pgfqpoint{5.893459in}{2.465330in}}%
\pgfpathlineto{\pgfqpoint{5.894643in}{2.451156in}}%
\pgfpathlineto{\pgfqpoint{5.894150in}{2.467913in}}%
\pgfpathlineto{\pgfqpoint{5.894742in}{2.453553in}}%
\pgfpathlineto{\pgfqpoint{5.894841in}{2.456131in}}%
\pgfpathlineto{\pgfqpoint{5.895236in}{2.439437in}}%
\pgfpathlineto{\pgfqpoint{5.895334in}{2.437938in}}%
\pgfpathlineto{\pgfqpoint{5.895532in}{2.448888in}}%
\pgfpathlineto{\pgfqpoint{5.895729in}{2.459547in}}%
\pgfpathlineto{\pgfqpoint{5.896519in}{2.440372in}}%
\pgfpathlineto{\pgfqpoint{5.896815in}{2.436971in}}%
\pgfpathlineto{\pgfqpoint{5.897111in}{2.441907in}}%
\pgfpathlineto{\pgfqpoint{5.897604in}{2.438005in}}%
\pgfpathlineto{\pgfqpoint{5.897901in}{2.441521in}}%
\pgfpathlineto{\pgfqpoint{5.898197in}{2.430449in}}%
\pgfpathlineto{\pgfqpoint{5.899085in}{2.424413in}}%
\pgfpathlineto{\pgfqpoint{5.898690in}{2.434170in}}%
\pgfpathlineto{\pgfqpoint{5.899282in}{2.429914in}}%
\pgfpathlineto{\pgfqpoint{5.899480in}{2.433758in}}%
\pgfpathlineto{\pgfqpoint{5.899480in}{2.433758in}}%
\pgfusepath{stroke}%
\end{pgfscope}%
\begin{pgfscope}%
\pgfsetrectcap%
\pgfsetmiterjoin%
\pgfsetlinewidth{0.803000pt}%
\definecolor{currentstroke}{rgb}{0.000000,0.000000,0.000000}%
\pgfsetstrokecolor{currentstroke}%
\pgfsetdash{}{0pt}%
\pgfpathmoveto{\pgfqpoint{0.717889in}{2.114143in}}%
\pgfpathlineto{\pgfqpoint{0.717889in}{2.901359in}}%
\pgfusepath{stroke}%
\end{pgfscope}%
\begin{pgfscope}%
\pgfsetrectcap%
\pgfsetmiterjoin%
\pgfsetlinewidth{0.803000pt}%
\definecolor{currentstroke}{rgb}{0.000000,0.000000,0.000000}%
\pgfsetstrokecolor{currentstroke}%
\pgfsetdash{}{0pt}%
\pgfpathmoveto{\pgfqpoint{6.146222in}{2.114143in}}%
\pgfpathlineto{\pgfqpoint{6.146222in}{2.901359in}}%
\pgfusepath{stroke}%
\end{pgfscope}%
\begin{pgfscope}%
\pgfsetrectcap%
\pgfsetmiterjoin%
\pgfsetlinewidth{0.803000pt}%
\definecolor{currentstroke}{rgb}{0.000000,0.000000,0.000000}%
\pgfsetstrokecolor{currentstroke}%
\pgfsetdash{}{0pt}%
\pgfpathmoveto{\pgfqpoint{0.717889in}{2.114143in}}%
\pgfpathlineto{\pgfqpoint{6.146222in}{2.114143in}}%
\pgfusepath{stroke}%
\end{pgfscope}%
\begin{pgfscope}%
\pgfsetrectcap%
\pgfsetmiterjoin%
\pgfsetlinewidth{0.803000pt}%
\definecolor{currentstroke}{rgb}{0.000000,0.000000,0.000000}%
\pgfsetstrokecolor{currentstroke}%
\pgfsetdash{}{0pt}%
\pgfpathmoveto{\pgfqpoint{0.717889in}{2.901359in}}%
\pgfpathlineto{\pgfqpoint{6.146222in}{2.901359in}}%
\pgfusepath{stroke}%
\end{pgfscope}%
\begin{pgfscope}%
\definecolor{textcolor}{rgb}{0.000000,0.000000,0.000000}%
\pgfsetstrokecolor{textcolor}%
\pgfsetfillcolor{textcolor}%
\pgftext[x=3.432055in,y=2.984692in,,base]{\color{textcolor}\rmfamily\fontsize{12.000000}{14.400000}\selectfont Final Filtered ECG Signal}%
\end{pgfscope}%
\begin{pgfscope}%
\pgfsetbuttcap%
\pgfsetmiterjoin%
\definecolor{currentfill}{rgb}{1.000000,1.000000,1.000000}%
\pgfsetfillcolor{currentfill}%
\pgfsetlinewidth{0.000000pt}%
\definecolor{currentstroke}{rgb}{0.000000,0.000000,0.000000}%
\pgfsetstrokecolor{currentstroke}%
\pgfsetstrokeopacity{0.000000}%
\pgfsetdash{}{0pt}%
\pgfpathmoveto{\pgfqpoint{0.717889in}{0.564143in}}%
\pgfpathlineto{\pgfqpoint{6.146222in}{0.564143in}}%
\pgfpathlineto{\pgfqpoint{6.146222in}{1.351359in}}%
\pgfpathlineto{\pgfqpoint{0.717889in}{1.351359in}}%
\pgfpathclose%
\pgfusepath{fill}%
\end{pgfscope}%
\begin{pgfscope}%
\pgfsetbuttcap%
\pgfsetroundjoin%
\definecolor{currentfill}{rgb}{0.000000,0.000000,0.000000}%
\pgfsetfillcolor{currentfill}%
\pgfsetlinewidth{0.803000pt}%
\definecolor{currentstroke}{rgb}{0.000000,0.000000,0.000000}%
\pgfsetstrokecolor{currentstroke}%
\pgfsetdash{}{0pt}%
\pgfsys@defobject{currentmarker}{\pgfqpoint{0.000000in}{-0.048611in}}{\pgfqpoint{0.000000in}{0.000000in}}{%
\pgfpathmoveto{\pgfqpoint{0.000000in}{0.000000in}}%
\pgfpathlineto{\pgfqpoint{0.000000in}{-0.048611in}}%
\pgfusepath{stroke,fill}%
}%
\begin{pgfscope}%
\pgfsys@transformshift{0.717889in}{0.564143in}%
\pgfsys@useobject{currentmarker}{}%
\end{pgfscope}%
\end{pgfscope}%
\begin{pgfscope}%
\definecolor{textcolor}{rgb}{0.000000,0.000000,0.000000}%
\pgfsetstrokecolor{textcolor}%
\pgfsetfillcolor{textcolor}%
\pgftext[x=0.717889in,y=0.466921in,,top]{\color{textcolor}\rmfamily\fontsize{10.000000}{12.000000}\selectfont \(\displaystyle {-100}\)}%
\end{pgfscope}%
\begin{pgfscope}%
\pgfsetbuttcap%
\pgfsetroundjoin%
\definecolor{currentfill}{rgb}{0.000000,0.000000,0.000000}%
\pgfsetfillcolor{currentfill}%
\pgfsetlinewidth{0.803000pt}%
\definecolor{currentstroke}{rgb}{0.000000,0.000000,0.000000}%
\pgfsetstrokecolor{currentstroke}%
\pgfsetdash{}{0pt}%
\pgfsys@defobject{currentmarker}{\pgfqpoint{0.000000in}{-0.048611in}}{\pgfqpoint{0.000000in}{0.000000in}}{%
\pgfpathmoveto{\pgfqpoint{0.000000in}{0.000000in}}%
\pgfpathlineto{\pgfqpoint{0.000000in}{-0.048611in}}%
\pgfusepath{stroke,fill}%
}%
\begin{pgfscope}%
\pgfsys@transformshift{1.396430in}{0.564143in}%
\pgfsys@useobject{currentmarker}{}%
\end{pgfscope}%
\end{pgfscope}%
\begin{pgfscope}%
\definecolor{textcolor}{rgb}{0.000000,0.000000,0.000000}%
\pgfsetstrokecolor{textcolor}%
\pgfsetfillcolor{textcolor}%
\pgftext[x=1.396430in,y=0.466921in,,top]{\color{textcolor}\rmfamily\fontsize{10.000000}{12.000000}\selectfont \(\displaystyle {-75}\)}%
\end{pgfscope}%
\begin{pgfscope}%
\pgfsetbuttcap%
\pgfsetroundjoin%
\definecolor{currentfill}{rgb}{0.000000,0.000000,0.000000}%
\pgfsetfillcolor{currentfill}%
\pgfsetlinewidth{0.803000pt}%
\definecolor{currentstroke}{rgb}{0.000000,0.000000,0.000000}%
\pgfsetstrokecolor{currentstroke}%
\pgfsetdash{}{0pt}%
\pgfsys@defobject{currentmarker}{\pgfqpoint{0.000000in}{-0.048611in}}{\pgfqpoint{0.000000in}{0.000000in}}{%
\pgfpathmoveto{\pgfqpoint{0.000000in}{0.000000in}}%
\pgfpathlineto{\pgfqpoint{0.000000in}{-0.048611in}}%
\pgfusepath{stroke,fill}%
}%
\begin{pgfscope}%
\pgfsys@transformshift{2.074972in}{0.564143in}%
\pgfsys@useobject{currentmarker}{}%
\end{pgfscope}%
\end{pgfscope}%
\begin{pgfscope}%
\definecolor{textcolor}{rgb}{0.000000,0.000000,0.000000}%
\pgfsetstrokecolor{textcolor}%
\pgfsetfillcolor{textcolor}%
\pgftext[x=2.074972in,y=0.466921in,,top]{\color{textcolor}\rmfamily\fontsize{10.000000}{12.000000}\selectfont \(\displaystyle {-50}\)}%
\end{pgfscope}%
\begin{pgfscope}%
\pgfsetbuttcap%
\pgfsetroundjoin%
\definecolor{currentfill}{rgb}{0.000000,0.000000,0.000000}%
\pgfsetfillcolor{currentfill}%
\pgfsetlinewidth{0.803000pt}%
\definecolor{currentstroke}{rgb}{0.000000,0.000000,0.000000}%
\pgfsetstrokecolor{currentstroke}%
\pgfsetdash{}{0pt}%
\pgfsys@defobject{currentmarker}{\pgfqpoint{0.000000in}{-0.048611in}}{\pgfqpoint{0.000000in}{0.000000in}}{%
\pgfpathmoveto{\pgfqpoint{0.000000in}{0.000000in}}%
\pgfpathlineto{\pgfqpoint{0.000000in}{-0.048611in}}%
\pgfusepath{stroke,fill}%
}%
\begin{pgfscope}%
\pgfsys@transformshift{2.753514in}{0.564143in}%
\pgfsys@useobject{currentmarker}{}%
\end{pgfscope}%
\end{pgfscope}%
\begin{pgfscope}%
\definecolor{textcolor}{rgb}{0.000000,0.000000,0.000000}%
\pgfsetstrokecolor{textcolor}%
\pgfsetfillcolor{textcolor}%
\pgftext[x=2.753514in,y=0.466921in,,top]{\color{textcolor}\rmfamily\fontsize{10.000000}{12.000000}\selectfont \(\displaystyle {-25}\)}%
\end{pgfscope}%
\begin{pgfscope}%
\pgfsetbuttcap%
\pgfsetroundjoin%
\definecolor{currentfill}{rgb}{0.000000,0.000000,0.000000}%
\pgfsetfillcolor{currentfill}%
\pgfsetlinewidth{0.803000pt}%
\definecolor{currentstroke}{rgb}{0.000000,0.000000,0.000000}%
\pgfsetstrokecolor{currentstroke}%
\pgfsetdash{}{0pt}%
\pgfsys@defobject{currentmarker}{\pgfqpoint{0.000000in}{-0.048611in}}{\pgfqpoint{0.000000in}{0.000000in}}{%
\pgfpathmoveto{\pgfqpoint{0.000000in}{0.000000in}}%
\pgfpathlineto{\pgfqpoint{0.000000in}{-0.048611in}}%
\pgfusepath{stroke,fill}%
}%
\begin{pgfscope}%
\pgfsys@transformshift{3.432055in}{0.564143in}%
\pgfsys@useobject{currentmarker}{}%
\end{pgfscope}%
\end{pgfscope}%
\begin{pgfscope}%
\definecolor{textcolor}{rgb}{0.000000,0.000000,0.000000}%
\pgfsetstrokecolor{textcolor}%
\pgfsetfillcolor{textcolor}%
\pgftext[x=3.432055in,y=0.466921in,,top]{\color{textcolor}\rmfamily\fontsize{10.000000}{12.000000}\selectfont \(\displaystyle {0}\)}%
\end{pgfscope}%
\begin{pgfscope}%
\pgfsetbuttcap%
\pgfsetroundjoin%
\definecolor{currentfill}{rgb}{0.000000,0.000000,0.000000}%
\pgfsetfillcolor{currentfill}%
\pgfsetlinewidth{0.803000pt}%
\definecolor{currentstroke}{rgb}{0.000000,0.000000,0.000000}%
\pgfsetstrokecolor{currentstroke}%
\pgfsetdash{}{0pt}%
\pgfsys@defobject{currentmarker}{\pgfqpoint{0.000000in}{-0.048611in}}{\pgfqpoint{0.000000in}{0.000000in}}{%
\pgfpathmoveto{\pgfqpoint{0.000000in}{0.000000in}}%
\pgfpathlineto{\pgfqpoint{0.000000in}{-0.048611in}}%
\pgfusepath{stroke,fill}%
}%
\begin{pgfscope}%
\pgfsys@transformshift{4.110597in}{0.564143in}%
\pgfsys@useobject{currentmarker}{}%
\end{pgfscope}%
\end{pgfscope}%
\begin{pgfscope}%
\definecolor{textcolor}{rgb}{0.000000,0.000000,0.000000}%
\pgfsetstrokecolor{textcolor}%
\pgfsetfillcolor{textcolor}%
\pgftext[x=4.110597in,y=0.466921in,,top]{\color{textcolor}\rmfamily\fontsize{10.000000}{12.000000}\selectfont \(\displaystyle {25}\)}%
\end{pgfscope}%
\begin{pgfscope}%
\pgfsetbuttcap%
\pgfsetroundjoin%
\definecolor{currentfill}{rgb}{0.000000,0.000000,0.000000}%
\pgfsetfillcolor{currentfill}%
\pgfsetlinewidth{0.803000pt}%
\definecolor{currentstroke}{rgb}{0.000000,0.000000,0.000000}%
\pgfsetstrokecolor{currentstroke}%
\pgfsetdash{}{0pt}%
\pgfsys@defobject{currentmarker}{\pgfqpoint{0.000000in}{-0.048611in}}{\pgfqpoint{0.000000in}{0.000000in}}{%
\pgfpathmoveto{\pgfqpoint{0.000000in}{0.000000in}}%
\pgfpathlineto{\pgfqpoint{0.000000in}{-0.048611in}}%
\pgfusepath{stroke,fill}%
}%
\begin{pgfscope}%
\pgfsys@transformshift{4.789139in}{0.564143in}%
\pgfsys@useobject{currentmarker}{}%
\end{pgfscope}%
\end{pgfscope}%
\begin{pgfscope}%
\definecolor{textcolor}{rgb}{0.000000,0.000000,0.000000}%
\pgfsetstrokecolor{textcolor}%
\pgfsetfillcolor{textcolor}%
\pgftext[x=4.789139in,y=0.466921in,,top]{\color{textcolor}\rmfamily\fontsize{10.000000}{12.000000}\selectfont \(\displaystyle {50}\)}%
\end{pgfscope}%
\begin{pgfscope}%
\pgfsetbuttcap%
\pgfsetroundjoin%
\definecolor{currentfill}{rgb}{0.000000,0.000000,0.000000}%
\pgfsetfillcolor{currentfill}%
\pgfsetlinewidth{0.803000pt}%
\definecolor{currentstroke}{rgb}{0.000000,0.000000,0.000000}%
\pgfsetstrokecolor{currentstroke}%
\pgfsetdash{}{0pt}%
\pgfsys@defobject{currentmarker}{\pgfqpoint{0.000000in}{-0.048611in}}{\pgfqpoint{0.000000in}{0.000000in}}{%
\pgfpathmoveto{\pgfqpoint{0.000000in}{0.000000in}}%
\pgfpathlineto{\pgfqpoint{0.000000in}{-0.048611in}}%
\pgfusepath{stroke,fill}%
}%
\begin{pgfscope}%
\pgfsys@transformshift{5.467680in}{0.564143in}%
\pgfsys@useobject{currentmarker}{}%
\end{pgfscope}%
\end{pgfscope}%
\begin{pgfscope}%
\definecolor{textcolor}{rgb}{0.000000,0.000000,0.000000}%
\pgfsetstrokecolor{textcolor}%
\pgfsetfillcolor{textcolor}%
\pgftext[x=5.467680in,y=0.466921in,,top]{\color{textcolor}\rmfamily\fontsize{10.000000}{12.000000}\selectfont \(\displaystyle {75}\)}%
\end{pgfscope}%
\begin{pgfscope}%
\pgfsetbuttcap%
\pgfsetroundjoin%
\definecolor{currentfill}{rgb}{0.000000,0.000000,0.000000}%
\pgfsetfillcolor{currentfill}%
\pgfsetlinewidth{0.803000pt}%
\definecolor{currentstroke}{rgb}{0.000000,0.000000,0.000000}%
\pgfsetstrokecolor{currentstroke}%
\pgfsetdash{}{0pt}%
\pgfsys@defobject{currentmarker}{\pgfqpoint{0.000000in}{-0.048611in}}{\pgfqpoint{0.000000in}{0.000000in}}{%
\pgfpathmoveto{\pgfqpoint{0.000000in}{0.000000in}}%
\pgfpathlineto{\pgfqpoint{0.000000in}{-0.048611in}}%
\pgfusepath{stroke,fill}%
}%
\begin{pgfscope}%
\pgfsys@transformshift{6.146222in}{0.564143in}%
\pgfsys@useobject{currentmarker}{}%
\end{pgfscope}%
\end{pgfscope}%
\begin{pgfscope}%
\definecolor{textcolor}{rgb}{0.000000,0.000000,0.000000}%
\pgfsetstrokecolor{textcolor}%
\pgfsetfillcolor{textcolor}%
\pgftext[x=6.146222in,y=0.466921in,,top]{\color{textcolor}\rmfamily\fontsize{10.000000}{12.000000}\selectfont \(\displaystyle {100}\)}%
\end{pgfscope}%
\begin{pgfscope}%
\definecolor{textcolor}{rgb}{0.000000,0.000000,0.000000}%
\pgfsetstrokecolor{textcolor}%
\pgfsetfillcolor{textcolor}%
\pgftext[x=3.432055in,y=0.287909in,,top]{\color{textcolor}\rmfamily\fontsize{10.000000}{12.000000}\selectfont Frequency (Hz)}%
\end{pgfscope}%
\begin{pgfscope}%
\pgfsetbuttcap%
\pgfsetroundjoin%
\definecolor{currentfill}{rgb}{0.000000,0.000000,0.000000}%
\pgfsetfillcolor{currentfill}%
\pgfsetlinewidth{0.803000pt}%
\definecolor{currentstroke}{rgb}{0.000000,0.000000,0.000000}%
\pgfsetstrokecolor{currentstroke}%
\pgfsetdash{}{0pt}%
\pgfsys@defobject{currentmarker}{\pgfqpoint{-0.048611in}{0.000000in}}{\pgfqpoint{0.000000in}{0.000000in}}{%
\pgfpathmoveto{\pgfqpoint{0.000000in}{0.000000in}}%
\pgfpathlineto{\pgfqpoint{-0.048611in}{0.000000in}}%
\pgfusepath{stroke,fill}%
}%
\begin{pgfscope}%
\pgfsys@transformshift{0.717889in}{0.599907in}%
\pgfsys@useobject{currentmarker}{}%
\end{pgfscope}%
\end{pgfscope}%
\begin{pgfscope}%
\definecolor{textcolor}{rgb}{0.000000,0.000000,0.000000}%
\pgfsetstrokecolor{textcolor}%
\pgfsetfillcolor{textcolor}%
\pgftext[x=0.551222in, y=0.551682in, left, base]{\color{textcolor}\rmfamily\fontsize{10.000000}{12.000000}\selectfont \(\displaystyle {0}\)}%
\end{pgfscope}%
\begin{pgfscope}%
\pgfsetbuttcap%
\pgfsetroundjoin%
\definecolor{currentfill}{rgb}{0.000000,0.000000,0.000000}%
\pgfsetfillcolor{currentfill}%
\pgfsetlinewidth{0.803000pt}%
\definecolor{currentstroke}{rgb}{0.000000,0.000000,0.000000}%
\pgfsetstrokecolor{currentstroke}%
\pgfsetdash{}{0pt}%
\pgfsys@defobject{currentmarker}{\pgfqpoint{-0.048611in}{0.000000in}}{\pgfqpoint{0.000000in}{0.000000in}}{%
\pgfpathmoveto{\pgfqpoint{0.000000in}{0.000000in}}%
\pgfpathlineto{\pgfqpoint{-0.048611in}{0.000000in}}%
\pgfusepath{stroke,fill}%
}%
\begin{pgfscope}%
\pgfsys@transformshift{0.717889in}{1.216364in}%
\pgfsys@useobject{currentmarker}{}%
\end{pgfscope}%
\end{pgfscope}%
\begin{pgfscope}%
\definecolor{textcolor}{rgb}{0.000000,0.000000,0.000000}%
\pgfsetstrokecolor{textcolor}%
\pgfsetfillcolor{textcolor}%
\pgftext[x=0.551222in, y=1.168139in, left, base]{\color{textcolor}\rmfamily\fontsize{10.000000}{12.000000}\selectfont \(\displaystyle {2}\)}%
\end{pgfscope}%
\begin{pgfscope}%
\definecolor{textcolor}{rgb}{0.000000,0.000000,0.000000}%
\pgfsetstrokecolor{textcolor}%
\pgfsetfillcolor{textcolor}%
\pgftext[x=0.495666in,y=0.957751in,,bottom,rotate=90.000000]{\color{textcolor}\rmfamily\fontsize{10.000000}{12.000000}\selectfont abs(Y(f)) (\(\displaystyle \mu V^2\))}%
\end{pgfscope}%
\begin{pgfscope}%
\definecolor{textcolor}{rgb}{0.000000,0.000000,0.000000}%
\pgfsetstrokecolor{textcolor}%
\pgfsetfillcolor{textcolor}%
\pgftext[x=0.717889in,y=1.393025in,left,base]{\color{textcolor}\rmfamily\fontsize{10.000000}{12.000000}\selectfont \(\displaystyle \times{10^{6}}{}\)}%
\end{pgfscope}%
\begin{pgfscope}%
\pgfpathrectangle{\pgfqpoint{0.717889in}{0.564143in}}{\pgfqpoint{5.428334in}{0.787215in}}%
\pgfusepath{clip}%
\pgfsetrectcap%
\pgfsetroundjoin%
\pgfsetlinewidth{1.505625pt}%
\definecolor{currentstroke}{rgb}{0.121569,0.466667,0.705882}%
\pgfsetstrokecolor{currentstroke}%
\pgfsetdash{}{0pt}%
\pgfpathmoveto{\pgfqpoint{3.432055in}{0.629233in}}%
\pgfpathlineto{\pgfqpoint{3.432611in}{0.623407in}}%
\pgfpathlineto{\pgfqpoint{3.433167in}{0.661682in}}%
\pgfpathlineto{\pgfqpoint{3.433723in}{0.657101in}}%
\pgfpathlineto{\pgfqpoint{3.434279in}{0.629247in}}%
\pgfpathlineto{\pgfqpoint{3.434835in}{0.738605in}}%
\pgfpathlineto{\pgfqpoint{3.435391in}{0.679721in}}%
\pgfpathlineto{\pgfqpoint{3.435946in}{0.622217in}}%
\pgfpathlineto{\pgfqpoint{3.437058in}{0.895512in}}%
\pgfpathlineto{\pgfqpoint{3.437614in}{0.823820in}}%
\pgfpathlineto{\pgfqpoint{3.438170in}{0.612393in}}%
\pgfpathlineto{\pgfqpoint{3.440393in}{1.153808in}}%
\pgfpathlineto{\pgfqpoint{3.440949in}{1.315576in}}%
\pgfpathlineto{\pgfqpoint{3.441505in}{0.732196in}}%
\pgfpathlineto{\pgfqpoint{3.442061in}{0.782073in}}%
\pgfpathlineto{\pgfqpoint{3.442617in}{1.117438in}}%
\pgfpathlineto{\pgfqpoint{3.443173in}{0.864424in}}%
\pgfpathlineto{\pgfqpoint{3.444840in}{0.758113in}}%
\pgfpathlineto{\pgfqpoint{3.445396in}{0.638460in}}%
\pgfpathlineto{\pgfqpoint{3.445952in}{0.710830in}}%
\pgfpathlineto{\pgfqpoint{3.447064in}{0.892871in}}%
\pgfpathlineto{\pgfqpoint{3.447620in}{0.873032in}}%
\pgfpathlineto{\pgfqpoint{3.448175in}{0.633400in}}%
\pgfpathlineto{\pgfqpoint{3.448731in}{0.701473in}}%
\pgfpathlineto{\pgfqpoint{3.449287in}{0.867872in}}%
\pgfpathlineto{\pgfqpoint{3.449843in}{0.696136in}}%
\pgfpathlineto{\pgfqpoint{3.450399in}{0.867898in}}%
\pgfpathlineto{\pgfqpoint{3.450955in}{0.734231in}}%
\pgfpathlineto{\pgfqpoint{3.451511in}{0.697573in}}%
\pgfpathlineto{\pgfqpoint{3.452066in}{0.702565in}}%
\pgfpathlineto{\pgfqpoint{3.452622in}{0.871400in}}%
\pgfpathlineto{\pgfqpoint{3.453178in}{0.749088in}}%
\pgfpathlineto{\pgfqpoint{3.454290in}{0.699809in}}%
\pgfpathlineto{\pgfqpoint{3.454846in}{0.822671in}}%
\pgfpathlineto{\pgfqpoint{3.455402in}{0.617433in}}%
\pgfpathlineto{\pgfqpoint{3.455957in}{0.651559in}}%
\pgfpathlineto{\pgfqpoint{3.458737in}{0.924690in}}%
\pgfpathlineto{\pgfqpoint{3.460404in}{0.736684in}}%
\pgfpathlineto{\pgfqpoint{3.460960in}{0.786414in}}%
\pgfpathlineto{\pgfqpoint{3.461516in}{0.654412in}}%
\pgfpathlineto{\pgfqpoint{3.462072in}{0.673757in}}%
\pgfpathlineto{\pgfqpoint{3.462628in}{0.721444in}}%
\pgfpathlineto{\pgfqpoint{3.463184in}{0.712678in}}%
\pgfpathlineto{\pgfqpoint{3.463740in}{0.679741in}}%
\pgfpathlineto{\pgfqpoint{3.464295in}{0.715129in}}%
\pgfpathlineto{\pgfqpoint{3.464851in}{0.713875in}}%
\pgfpathlineto{\pgfqpoint{3.465963in}{0.638731in}}%
\pgfpathlineto{\pgfqpoint{3.466519in}{0.648200in}}%
\pgfpathlineto{\pgfqpoint{3.467075in}{0.638636in}}%
\pgfpathlineto{\pgfqpoint{3.467631in}{0.646447in}}%
\pgfpathlineto{\pgfqpoint{3.468186in}{0.651672in}}%
\pgfpathlineto{\pgfqpoint{3.468742in}{0.685551in}}%
\pgfpathlineto{\pgfqpoint{3.469298in}{0.615326in}}%
\pgfpathlineto{\pgfqpoint{3.469854in}{0.643187in}}%
\pgfpathlineto{\pgfqpoint{3.470410in}{0.767071in}}%
\pgfpathlineto{\pgfqpoint{3.470966in}{0.716371in}}%
\pgfpathlineto{\pgfqpoint{3.471522in}{0.752765in}}%
\pgfpathlineto{\pgfqpoint{3.472077in}{0.723119in}}%
\pgfpathlineto{\pgfqpoint{3.472633in}{0.711492in}}%
\pgfpathlineto{\pgfqpoint{3.473189in}{0.727736in}}%
\pgfpathlineto{\pgfqpoint{3.473745in}{0.633351in}}%
\pgfpathlineto{\pgfqpoint{3.474301in}{0.750767in}}%
\pgfpathlineto{\pgfqpoint{3.474857in}{0.712061in}}%
\pgfpathlineto{\pgfqpoint{3.475968in}{0.746430in}}%
\pgfpathlineto{\pgfqpoint{3.476524in}{0.620094in}}%
\pgfpathlineto{\pgfqpoint{3.477080in}{0.685213in}}%
\pgfpathlineto{\pgfqpoint{3.477636in}{0.621289in}}%
\pgfpathlineto{\pgfqpoint{3.478192in}{0.659378in}}%
\pgfpathlineto{\pgfqpoint{3.478748in}{0.674413in}}%
\pgfpathlineto{\pgfqpoint{3.480415in}{0.618972in}}%
\pgfpathlineto{\pgfqpoint{3.480971in}{0.690790in}}%
\pgfpathlineto{\pgfqpoint{3.481527in}{0.676156in}}%
\pgfpathlineto{\pgfqpoint{3.483195in}{0.631179in}}%
\pgfpathlineto{\pgfqpoint{3.483751in}{0.669101in}}%
\pgfpathlineto{\pgfqpoint{3.484306in}{0.646387in}}%
\pgfpathlineto{\pgfqpoint{3.485418in}{0.608870in}}%
\pgfpathlineto{\pgfqpoint{3.485974in}{0.720313in}}%
\pgfpathlineto{\pgfqpoint{3.486530in}{0.665139in}}%
\pgfpathlineto{\pgfqpoint{3.487086in}{0.713919in}}%
\pgfpathlineto{\pgfqpoint{3.487642in}{0.619903in}}%
\pgfpathlineto{\pgfqpoint{3.488197in}{0.703588in}}%
\pgfpathlineto{\pgfqpoint{3.488753in}{0.663442in}}%
\pgfpathlineto{\pgfqpoint{3.489309in}{0.961308in}}%
\pgfpathlineto{\pgfqpoint{3.489865in}{0.888559in}}%
\pgfpathlineto{\pgfqpoint{3.490977in}{0.755403in}}%
\pgfpathlineto{\pgfqpoint{3.491533in}{0.804293in}}%
\pgfpathlineto{\pgfqpoint{3.492644in}{0.636244in}}%
\pgfpathlineto{\pgfqpoint{3.493200in}{0.694597in}}%
\pgfpathlineto{\pgfqpoint{3.493756in}{0.622249in}}%
\pgfpathlineto{\pgfqpoint{3.494312in}{0.691438in}}%
\pgfpathlineto{\pgfqpoint{3.494868in}{0.698800in}}%
\pgfpathlineto{\pgfqpoint{3.495424in}{0.675702in}}%
\pgfpathlineto{\pgfqpoint{3.495979in}{0.698703in}}%
\pgfpathlineto{\pgfqpoint{3.496535in}{0.695116in}}%
\pgfpathlineto{\pgfqpoint{3.497091in}{0.819327in}}%
\pgfpathlineto{\pgfqpoint{3.497647in}{0.620203in}}%
\pgfpathlineto{\pgfqpoint{3.498203in}{0.766250in}}%
\pgfpathlineto{\pgfqpoint{3.498759in}{0.671623in}}%
\pgfpathlineto{\pgfqpoint{3.499315in}{0.717872in}}%
\pgfpathlineto{\pgfqpoint{3.499870in}{0.788299in}}%
\pgfpathlineto{\pgfqpoint{3.500426in}{0.688675in}}%
\pgfpathlineto{\pgfqpoint{3.502094in}{0.884417in}}%
\pgfpathlineto{\pgfqpoint{3.502650in}{0.699739in}}%
\pgfpathlineto{\pgfqpoint{3.503206in}{0.874126in}}%
\pgfpathlineto{\pgfqpoint{3.503762in}{0.748441in}}%
\pgfpathlineto{\pgfqpoint{3.504317in}{0.775657in}}%
\pgfpathlineto{\pgfqpoint{3.504873in}{0.786679in}}%
\pgfpathlineto{\pgfqpoint{3.505429in}{0.718709in}}%
\pgfpathlineto{\pgfqpoint{3.507097in}{0.874959in}}%
\pgfpathlineto{\pgfqpoint{3.509320in}{0.731504in}}%
\pgfpathlineto{\pgfqpoint{3.509876in}{0.709648in}}%
\pgfpathlineto{\pgfqpoint{3.510432in}{0.628820in}}%
\pgfpathlineto{\pgfqpoint{3.510988in}{0.755908in}}%
\pgfpathlineto{\pgfqpoint{3.511544in}{0.652373in}}%
\pgfpathlineto{\pgfqpoint{3.512099in}{0.660032in}}%
\pgfpathlineto{\pgfqpoint{3.513767in}{0.635006in}}%
\pgfpathlineto{\pgfqpoint{3.514323in}{0.668255in}}%
\pgfpathlineto{\pgfqpoint{3.514879in}{0.654959in}}%
\pgfpathlineto{\pgfqpoint{3.515435in}{0.610603in}}%
\pgfpathlineto{\pgfqpoint{3.515990in}{0.654415in}}%
\pgfpathlineto{\pgfqpoint{3.516546in}{0.670204in}}%
\pgfpathlineto{\pgfqpoint{3.517658in}{0.630515in}}%
\pgfpathlineto{\pgfqpoint{3.518214in}{0.635043in}}%
\pgfpathlineto{\pgfqpoint{3.518770in}{0.656796in}}%
\pgfpathlineto{\pgfqpoint{3.520437in}{0.608115in}}%
\pgfpathlineto{\pgfqpoint{3.520993in}{0.620910in}}%
\pgfpathlineto{\pgfqpoint{3.521549in}{0.667640in}}%
\pgfpathlineto{\pgfqpoint{3.522105in}{0.655483in}}%
\pgfpathlineto{\pgfqpoint{3.522661in}{0.611730in}}%
\pgfpathlineto{\pgfqpoint{3.523217in}{0.618525in}}%
\pgfpathlineto{\pgfqpoint{3.524328in}{0.640573in}}%
\pgfpathlineto{\pgfqpoint{3.525996in}{0.604507in}}%
\pgfpathlineto{\pgfqpoint{3.527664in}{0.663668in}}%
\pgfpathlineto{\pgfqpoint{3.528219in}{0.615018in}}%
\pgfpathlineto{\pgfqpoint{3.528775in}{0.642336in}}%
\pgfpathlineto{\pgfqpoint{3.529887in}{0.634385in}}%
\pgfpathlineto{\pgfqpoint{3.530999in}{0.659027in}}%
\pgfpathlineto{\pgfqpoint{3.531555in}{0.640254in}}%
\pgfpathlineto{\pgfqpoint{3.532666in}{0.678615in}}%
\pgfpathlineto{\pgfqpoint{3.534334in}{0.622181in}}%
\pgfpathlineto{\pgfqpoint{3.534890in}{0.606031in}}%
\pgfpathlineto{\pgfqpoint{3.535446in}{0.629530in}}%
\pgfpathlineto{\pgfqpoint{3.536001in}{0.622157in}}%
\pgfpathlineto{\pgfqpoint{3.536557in}{0.629236in}}%
\pgfpathlineto{\pgfqpoint{3.537113in}{0.680371in}}%
\pgfpathlineto{\pgfqpoint{3.537669in}{0.670019in}}%
\pgfpathlineto{\pgfqpoint{3.538225in}{0.668038in}}%
\pgfpathlineto{\pgfqpoint{3.538781in}{0.645546in}}%
\pgfpathlineto{\pgfqpoint{3.539337in}{0.654611in}}%
\pgfpathlineto{\pgfqpoint{3.539893in}{0.652368in}}%
\pgfpathlineto{\pgfqpoint{3.540448in}{0.655218in}}%
\pgfpathlineto{\pgfqpoint{3.541560in}{0.636451in}}%
\pgfpathlineto{\pgfqpoint{3.542116in}{0.660302in}}%
\pgfpathlineto{\pgfqpoint{3.542672in}{0.622964in}}%
\pgfpathlineto{\pgfqpoint{3.543228in}{0.664679in}}%
\pgfpathlineto{\pgfqpoint{3.543784in}{0.663157in}}%
\pgfpathlineto{\pgfqpoint{3.544895in}{0.635987in}}%
\pgfpathlineto{\pgfqpoint{3.545451in}{0.656190in}}%
\pgfpathlineto{\pgfqpoint{3.546007in}{0.625576in}}%
\pgfpathlineto{\pgfqpoint{3.546563in}{0.660382in}}%
\pgfpathlineto{\pgfqpoint{3.547119in}{0.810080in}}%
\pgfpathlineto{\pgfqpoint{3.547675in}{0.651340in}}%
\pgfpathlineto{\pgfqpoint{3.548230in}{0.688945in}}%
\pgfpathlineto{\pgfqpoint{3.549342in}{0.658394in}}%
\pgfpathlineto{\pgfqpoint{3.549898in}{0.660944in}}%
\pgfpathlineto{\pgfqpoint{3.551010in}{0.678443in}}%
\pgfpathlineto{\pgfqpoint{3.551566in}{0.650044in}}%
\pgfpathlineto{\pgfqpoint{3.552121in}{0.662672in}}%
\pgfpathlineto{\pgfqpoint{3.555457in}{0.608650in}}%
\pgfpathlineto{\pgfqpoint{3.557124in}{0.630923in}}%
\pgfpathlineto{\pgfqpoint{3.557680in}{0.635378in}}%
\pgfpathlineto{\pgfqpoint{3.558792in}{0.612408in}}%
\pgfpathlineto{\pgfqpoint{3.559348in}{0.632552in}}%
\pgfpathlineto{\pgfqpoint{3.559904in}{0.618045in}}%
\pgfpathlineto{\pgfqpoint{3.561015in}{0.643104in}}%
\pgfpathlineto{\pgfqpoint{3.561571in}{0.617105in}}%
\pgfpathlineto{\pgfqpoint{3.562127in}{0.631900in}}%
\pgfpathlineto{\pgfqpoint{3.564906in}{0.613215in}}%
\pgfpathlineto{\pgfqpoint{3.565462in}{0.636720in}}%
\pgfpathlineto{\pgfqpoint{3.566018in}{0.622246in}}%
\pgfpathlineto{\pgfqpoint{3.566574in}{0.626232in}}%
\pgfpathlineto{\pgfqpoint{3.567130in}{0.620957in}}%
\pgfpathlineto{\pgfqpoint{3.567686in}{0.652865in}}%
\pgfpathlineto{\pgfqpoint{3.568241in}{0.611210in}}%
\pgfpathlineto{\pgfqpoint{3.568797in}{0.638417in}}%
\pgfpathlineto{\pgfqpoint{3.570465in}{0.605335in}}%
\pgfpathlineto{\pgfqpoint{3.572688in}{0.657838in}}%
\pgfpathlineto{\pgfqpoint{3.573244in}{0.633067in}}%
\pgfpathlineto{\pgfqpoint{3.573800in}{0.672023in}}%
\pgfpathlineto{\pgfqpoint{3.574356in}{0.628456in}}%
\pgfpathlineto{\pgfqpoint{3.574912in}{0.642998in}}%
\pgfpathlineto{\pgfqpoint{3.576579in}{0.616557in}}%
\pgfpathlineto{\pgfqpoint{3.577135in}{0.619251in}}%
\pgfpathlineto{\pgfqpoint{3.578803in}{0.649489in}}%
\pgfpathlineto{\pgfqpoint{3.579359in}{0.611940in}}%
\pgfpathlineto{\pgfqpoint{3.579915in}{0.651304in}}%
\pgfpathlineto{\pgfqpoint{3.580470in}{0.632969in}}%
\pgfpathlineto{\pgfqpoint{3.581026in}{0.630094in}}%
\pgfpathlineto{\pgfqpoint{3.582138in}{0.602543in}}%
\pgfpathlineto{\pgfqpoint{3.583806in}{0.630938in}}%
\pgfpathlineto{\pgfqpoint{3.584361in}{0.604459in}}%
\pgfpathlineto{\pgfqpoint{3.584917in}{0.607722in}}%
\pgfpathlineto{\pgfqpoint{3.586029in}{0.621721in}}%
\pgfpathlineto{\pgfqpoint{3.586585in}{0.607933in}}%
\pgfpathlineto{\pgfqpoint{3.587141in}{0.610688in}}%
\pgfpathlineto{\pgfqpoint{3.588252in}{0.615526in}}%
\pgfpathlineto{\pgfqpoint{3.588808in}{0.624913in}}%
\pgfpathlineto{\pgfqpoint{3.589364in}{0.603122in}}%
\pgfpathlineto{\pgfqpoint{3.589920in}{0.616594in}}%
\pgfpathlineto{\pgfqpoint{3.591032in}{0.621834in}}%
\pgfpathlineto{\pgfqpoint{3.591588in}{0.635674in}}%
\pgfpathlineto{\pgfqpoint{3.592143in}{0.625909in}}%
\pgfpathlineto{\pgfqpoint{3.592699in}{0.603388in}}%
\pgfpathlineto{\pgfqpoint{3.593811in}{0.636412in}}%
\pgfpathlineto{\pgfqpoint{3.594367in}{0.608683in}}%
\pgfpathlineto{\pgfqpoint{3.594923in}{0.611642in}}%
\pgfpathlineto{\pgfqpoint{3.595479in}{0.609139in}}%
\pgfpathlineto{\pgfqpoint{3.596035in}{0.609263in}}%
\pgfpathlineto{\pgfqpoint{3.597702in}{0.624024in}}%
\pgfpathlineto{\pgfqpoint{3.598258in}{0.618011in}}%
\pgfpathlineto{\pgfqpoint{3.598814in}{0.669552in}}%
\pgfpathlineto{\pgfqpoint{3.599370in}{0.603295in}}%
\pgfpathlineto{\pgfqpoint{3.599926in}{0.648544in}}%
\pgfpathlineto{\pgfqpoint{3.600481in}{0.613705in}}%
\pgfpathlineto{\pgfqpoint{3.601037in}{0.623326in}}%
\pgfpathlineto{\pgfqpoint{3.601593in}{0.631333in}}%
\pgfpathlineto{\pgfqpoint{3.602149in}{0.746334in}}%
\pgfpathlineto{\pgfqpoint{3.602705in}{0.649867in}}%
\pgfpathlineto{\pgfqpoint{3.603817in}{0.691037in}}%
\pgfpathlineto{\pgfqpoint{3.604372in}{0.873391in}}%
\pgfpathlineto{\pgfqpoint{3.604928in}{0.715644in}}%
\pgfpathlineto{\pgfqpoint{3.606040in}{0.627019in}}%
\pgfpathlineto{\pgfqpoint{3.606596in}{0.734056in}}%
\pgfpathlineto{\pgfqpoint{3.607152in}{0.675998in}}%
\pgfpathlineto{\pgfqpoint{3.607708in}{0.697106in}}%
\pgfpathlineto{\pgfqpoint{3.608263in}{0.688950in}}%
\pgfpathlineto{\pgfqpoint{3.609931in}{0.636957in}}%
\pgfpathlineto{\pgfqpoint{3.611599in}{0.728176in}}%
\pgfpathlineto{\pgfqpoint{3.613822in}{0.633659in}}%
\pgfpathlineto{\pgfqpoint{3.614378in}{0.638816in}}%
\pgfpathlineto{\pgfqpoint{3.616046in}{0.622261in}}%
\pgfpathlineto{\pgfqpoint{3.616601in}{0.628809in}}%
\pgfpathlineto{\pgfqpoint{3.617157in}{0.624921in}}%
\pgfpathlineto{\pgfqpoint{3.618825in}{0.606216in}}%
\pgfpathlineto{\pgfqpoint{3.620492in}{0.621903in}}%
\pgfpathlineto{\pgfqpoint{3.621604in}{0.603637in}}%
\pgfpathlineto{\pgfqpoint{3.622160in}{0.607877in}}%
\pgfpathlineto{\pgfqpoint{3.622716in}{0.620722in}}%
\pgfpathlineto{\pgfqpoint{3.623272in}{0.605324in}}%
\pgfpathlineto{\pgfqpoint{3.623828in}{0.618139in}}%
\pgfpathlineto{\pgfqpoint{3.624383in}{0.633409in}}%
\pgfpathlineto{\pgfqpoint{3.624939in}{0.605511in}}%
\pgfpathlineto{\pgfqpoint{3.625495in}{0.615532in}}%
\pgfpathlineto{\pgfqpoint{3.626607in}{0.608144in}}%
\pgfpathlineto{\pgfqpoint{3.627163in}{0.618109in}}%
\pgfpathlineto{\pgfqpoint{3.627719in}{0.606064in}}%
\pgfpathlineto{\pgfqpoint{3.628274in}{0.612403in}}%
\pgfpathlineto{\pgfqpoint{3.628830in}{0.611039in}}%
\pgfpathlineto{\pgfqpoint{3.629386in}{0.615736in}}%
\pgfpathlineto{\pgfqpoint{3.630498in}{0.606885in}}%
\pgfpathlineto{\pgfqpoint{3.631054in}{0.628859in}}%
\pgfpathlineto{\pgfqpoint{3.631610in}{0.610124in}}%
\pgfpathlineto{\pgfqpoint{3.632165in}{0.612646in}}%
\pgfpathlineto{\pgfqpoint{3.632721in}{0.609949in}}%
\pgfpathlineto{\pgfqpoint{3.633277in}{0.637563in}}%
\pgfpathlineto{\pgfqpoint{3.633833in}{0.628788in}}%
\pgfpathlineto{\pgfqpoint{3.636057in}{0.608693in}}%
\pgfpathlineto{\pgfqpoint{3.636612in}{0.621122in}}%
\pgfpathlineto{\pgfqpoint{3.637168in}{0.616949in}}%
\pgfpathlineto{\pgfqpoint{3.637724in}{0.616483in}}%
\pgfpathlineto{\pgfqpoint{3.638280in}{0.622621in}}%
\pgfpathlineto{\pgfqpoint{3.638836in}{0.604953in}}%
\pgfpathlineto{\pgfqpoint{3.639392in}{0.608883in}}%
\pgfpathlineto{\pgfqpoint{3.641615in}{0.619532in}}%
\pgfpathlineto{\pgfqpoint{3.643283in}{0.607529in}}%
\pgfpathlineto{\pgfqpoint{3.643839in}{0.613185in}}%
\pgfpathlineto{\pgfqpoint{3.644394in}{0.605582in}}%
\pgfpathlineto{\pgfqpoint{3.644950in}{0.605899in}}%
\pgfpathlineto{\pgfqpoint{3.646062in}{0.629005in}}%
\pgfpathlineto{\pgfqpoint{3.646618in}{0.618021in}}%
\pgfpathlineto{\pgfqpoint{3.649397in}{0.641687in}}%
\pgfpathlineto{\pgfqpoint{3.651621in}{0.607806in}}%
\pgfpathlineto{\pgfqpoint{3.652177in}{0.623532in}}%
\pgfpathlineto{\pgfqpoint{3.652732in}{0.619962in}}%
\pgfpathlineto{\pgfqpoint{3.653288in}{0.607597in}}%
\pgfpathlineto{\pgfqpoint{3.654956in}{0.636003in}}%
\pgfpathlineto{\pgfqpoint{3.655512in}{0.624984in}}%
\pgfpathlineto{\pgfqpoint{3.656068in}{0.661655in}}%
\pgfpathlineto{\pgfqpoint{3.656623in}{0.607740in}}%
\pgfpathlineto{\pgfqpoint{3.657179in}{0.655626in}}%
\pgfpathlineto{\pgfqpoint{3.657735in}{0.614063in}}%
\pgfpathlineto{\pgfqpoint{3.659403in}{0.796002in}}%
\pgfpathlineto{\pgfqpoint{3.659959in}{0.652944in}}%
\pgfpathlineto{\pgfqpoint{3.660514in}{0.677952in}}%
\pgfpathlineto{\pgfqpoint{3.661070in}{0.665110in}}%
\pgfpathlineto{\pgfqpoint{3.661626in}{0.759179in}}%
\pgfpathlineto{\pgfqpoint{3.662182in}{0.754172in}}%
\pgfpathlineto{\pgfqpoint{3.662738in}{0.633950in}}%
\pgfpathlineto{\pgfqpoint{3.663294in}{0.725159in}}%
\pgfpathlineto{\pgfqpoint{3.663850in}{0.782380in}}%
\pgfpathlineto{\pgfqpoint{3.664405in}{0.734736in}}%
\pgfpathlineto{\pgfqpoint{3.665517in}{0.660142in}}%
\pgfpathlineto{\pgfqpoint{3.666073in}{0.690675in}}%
\pgfpathlineto{\pgfqpoint{3.667185in}{0.634768in}}%
\pgfpathlineto{\pgfqpoint{3.667741in}{0.665038in}}%
\pgfpathlineto{\pgfqpoint{3.668296in}{0.694525in}}%
\pgfpathlineto{\pgfqpoint{3.668852in}{0.666811in}}%
\pgfpathlineto{\pgfqpoint{3.669408in}{0.612605in}}%
\pgfpathlineto{\pgfqpoint{3.671632in}{0.726604in}}%
\pgfpathlineto{\pgfqpoint{3.672743in}{0.690276in}}%
\pgfpathlineto{\pgfqpoint{3.673299in}{0.641318in}}%
\pgfpathlineto{\pgfqpoint{3.673855in}{0.647428in}}%
\pgfpathlineto{\pgfqpoint{3.674411in}{0.644685in}}%
\pgfpathlineto{\pgfqpoint{3.676079in}{0.606503in}}%
\pgfpathlineto{\pgfqpoint{3.676634in}{0.611689in}}%
\pgfpathlineto{\pgfqpoint{3.677190in}{0.623760in}}%
\pgfpathlineto{\pgfqpoint{3.677746in}{0.606456in}}%
\pgfpathlineto{\pgfqpoint{3.678302in}{0.616693in}}%
\pgfpathlineto{\pgfqpoint{3.678858in}{0.613578in}}%
\pgfpathlineto{\pgfqpoint{3.679414in}{0.617972in}}%
\pgfpathlineto{\pgfqpoint{3.679970in}{0.601381in}}%
\pgfpathlineto{\pgfqpoint{3.680525in}{0.618489in}}%
\pgfpathlineto{\pgfqpoint{3.681081in}{0.605594in}}%
\pgfpathlineto{\pgfqpoint{3.681637in}{0.615122in}}%
\pgfpathlineto{\pgfqpoint{3.682193in}{0.603861in}}%
\pgfpathlineto{\pgfqpoint{3.682749in}{0.612158in}}%
\pgfpathlineto{\pgfqpoint{3.683305in}{0.622137in}}%
\pgfpathlineto{\pgfqpoint{3.683861in}{0.612060in}}%
\pgfpathlineto{\pgfqpoint{3.684416in}{0.616201in}}%
\pgfpathlineto{\pgfqpoint{3.684972in}{0.618615in}}%
\pgfpathlineto{\pgfqpoint{3.685528in}{0.604558in}}%
\pgfpathlineto{\pgfqpoint{3.686084in}{0.616543in}}%
\pgfpathlineto{\pgfqpoint{3.687196in}{0.613111in}}%
\pgfpathlineto{\pgfqpoint{3.687752in}{0.624558in}}%
\pgfpathlineto{\pgfqpoint{3.688307in}{0.614694in}}%
\pgfpathlineto{\pgfqpoint{3.688863in}{0.617116in}}%
\pgfpathlineto{\pgfqpoint{3.691087in}{0.604602in}}%
\pgfpathlineto{\pgfqpoint{3.691643in}{0.628883in}}%
\pgfpathlineto{\pgfqpoint{3.692199in}{0.605427in}}%
\pgfpathlineto{\pgfqpoint{3.692754in}{0.608329in}}%
\pgfpathlineto{\pgfqpoint{3.693310in}{0.618808in}}%
\pgfpathlineto{\pgfqpoint{3.693866in}{0.603742in}}%
\pgfpathlineto{\pgfqpoint{3.694422in}{0.624196in}}%
\pgfpathlineto{\pgfqpoint{3.694978in}{0.614180in}}%
\pgfpathlineto{\pgfqpoint{3.695534in}{0.602350in}}%
\pgfpathlineto{\pgfqpoint{3.696090in}{0.605997in}}%
\pgfpathlineto{\pgfqpoint{3.697757in}{0.626403in}}%
\pgfpathlineto{\pgfqpoint{3.698313in}{0.601630in}}%
\pgfpathlineto{\pgfqpoint{3.698869in}{0.619275in}}%
\pgfpathlineto{\pgfqpoint{3.699425in}{0.621848in}}%
\pgfpathlineto{\pgfqpoint{3.699981in}{0.631671in}}%
\pgfpathlineto{\pgfqpoint{3.700536in}{0.629246in}}%
\pgfpathlineto{\pgfqpoint{3.701648in}{0.606448in}}%
\pgfpathlineto{\pgfqpoint{3.702204in}{0.607587in}}%
\pgfpathlineto{\pgfqpoint{3.702760in}{0.615686in}}%
\pgfpathlineto{\pgfqpoint{3.704427in}{0.602347in}}%
\pgfpathlineto{\pgfqpoint{3.704983in}{0.603125in}}%
\pgfpathlineto{\pgfqpoint{3.706095in}{0.635239in}}%
\pgfpathlineto{\pgfqpoint{3.706651in}{0.615427in}}%
\pgfpathlineto{\pgfqpoint{3.707207in}{0.618449in}}%
\pgfpathlineto{\pgfqpoint{3.707763in}{0.626212in}}%
\pgfpathlineto{\pgfqpoint{3.708318in}{0.612121in}}%
\pgfpathlineto{\pgfqpoint{3.708874in}{0.616117in}}%
\pgfpathlineto{\pgfqpoint{3.709430in}{0.624916in}}%
\pgfpathlineto{\pgfqpoint{3.709986in}{0.607838in}}%
\pgfpathlineto{\pgfqpoint{3.710542in}{0.613906in}}%
\pgfpathlineto{\pgfqpoint{3.711654in}{0.621152in}}%
\pgfpathlineto{\pgfqpoint{3.712210in}{0.613323in}}%
\pgfpathlineto{\pgfqpoint{3.712765in}{0.617582in}}%
\pgfpathlineto{\pgfqpoint{3.714433in}{0.654705in}}%
\pgfpathlineto{\pgfqpoint{3.714989in}{0.608683in}}%
\pgfpathlineto{\pgfqpoint{3.716656in}{0.719045in}}%
\pgfpathlineto{\pgfqpoint{3.717212in}{0.710844in}}%
\pgfpathlineto{\pgfqpoint{3.718324in}{0.607957in}}%
\pgfpathlineto{\pgfqpoint{3.718880in}{0.653512in}}%
\pgfpathlineto{\pgfqpoint{3.719436in}{0.765597in}}%
\pgfpathlineto{\pgfqpoint{3.719992in}{0.668116in}}%
\pgfpathlineto{\pgfqpoint{3.720547in}{0.709865in}}%
\pgfpathlineto{\pgfqpoint{3.721103in}{0.661485in}}%
\pgfpathlineto{\pgfqpoint{3.721659in}{0.704175in}}%
\pgfpathlineto{\pgfqpoint{3.723327in}{0.662393in}}%
\pgfpathlineto{\pgfqpoint{3.723883in}{0.669692in}}%
\pgfpathlineto{\pgfqpoint{3.724438in}{0.665380in}}%
\pgfpathlineto{\pgfqpoint{3.726106in}{0.627189in}}%
\pgfpathlineto{\pgfqpoint{3.726662in}{0.618399in}}%
\pgfpathlineto{\pgfqpoint{3.727774in}{0.655865in}}%
\pgfpathlineto{\pgfqpoint{3.728885in}{0.648698in}}%
\pgfpathlineto{\pgfqpoint{3.729441in}{0.620065in}}%
\pgfpathlineto{\pgfqpoint{3.729997in}{0.644412in}}%
\pgfpathlineto{\pgfqpoint{3.731665in}{0.661381in}}%
\pgfpathlineto{\pgfqpoint{3.732221in}{0.656870in}}%
\pgfpathlineto{\pgfqpoint{3.732776in}{0.670559in}}%
\pgfpathlineto{\pgfqpoint{3.734444in}{0.632188in}}%
\pgfpathlineto{\pgfqpoint{3.735556in}{0.606859in}}%
\pgfpathlineto{\pgfqpoint{3.736112in}{0.616542in}}%
\pgfpathlineto{\pgfqpoint{3.737223in}{0.605802in}}%
\pgfpathlineto{\pgfqpoint{3.737779in}{0.609708in}}%
\pgfpathlineto{\pgfqpoint{3.738891in}{0.616319in}}%
\pgfpathlineto{\pgfqpoint{3.739447in}{0.600639in}}%
\pgfpathlineto{\pgfqpoint{3.740003in}{0.607544in}}%
\pgfpathlineto{\pgfqpoint{3.740558in}{0.616528in}}%
\pgfpathlineto{\pgfqpoint{3.742226in}{0.602797in}}%
\pgfpathlineto{\pgfqpoint{3.742782in}{0.614933in}}%
\pgfpathlineto{\pgfqpoint{3.743338in}{0.605033in}}%
\pgfpathlineto{\pgfqpoint{3.745005in}{0.618655in}}%
\pgfpathlineto{\pgfqpoint{3.746117in}{0.605495in}}%
\pgfpathlineto{\pgfqpoint{3.746673in}{0.612350in}}%
\pgfpathlineto{\pgfqpoint{3.747229in}{0.616724in}}%
\pgfpathlineto{\pgfqpoint{3.747785in}{0.605222in}}%
\pgfpathlineto{\pgfqpoint{3.748341in}{0.625272in}}%
\pgfpathlineto{\pgfqpoint{3.748896in}{0.614522in}}%
\pgfpathlineto{\pgfqpoint{3.751120in}{0.606287in}}%
\pgfpathlineto{\pgfqpoint{3.752232in}{0.621903in}}%
\pgfpathlineto{\pgfqpoint{3.752787in}{0.618523in}}%
\pgfpathlineto{\pgfqpoint{3.753343in}{0.622595in}}%
\pgfpathlineto{\pgfqpoint{3.753899in}{0.606836in}}%
\pgfpathlineto{\pgfqpoint{3.754455in}{0.615427in}}%
\pgfpathlineto{\pgfqpoint{3.755011in}{0.622010in}}%
\pgfpathlineto{\pgfqpoint{3.755567in}{0.608802in}}%
\pgfpathlineto{\pgfqpoint{3.756123in}{0.618524in}}%
\pgfpathlineto{\pgfqpoint{3.756678in}{0.612094in}}%
\pgfpathlineto{\pgfqpoint{3.757234in}{0.617927in}}%
\pgfpathlineto{\pgfqpoint{3.757790in}{0.625154in}}%
\pgfpathlineto{\pgfqpoint{3.758902in}{0.603686in}}%
\pgfpathlineto{\pgfqpoint{3.759458in}{0.608216in}}%
\pgfpathlineto{\pgfqpoint{3.760569in}{0.609991in}}%
\pgfpathlineto{\pgfqpoint{3.761125in}{0.625145in}}%
\pgfpathlineto{\pgfqpoint{3.762237in}{0.607251in}}%
\pgfpathlineto{\pgfqpoint{3.763905in}{0.623310in}}%
\pgfpathlineto{\pgfqpoint{3.764460in}{0.602235in}}%
\pgfpathlineto{\pgfqpoint{3.765016in}{0.617256in}}%
\pgfpathlineto{\pgfqpoint{3.766128in}{0.606204in}}%
\pgfpathlineto{\pgfqpoint{3.766684in}{0.636413in}}%
\pgfpathlineto{\pgfqpoint{3.767240in}{0.616866in}}%
\pgfpathlineto{\pgfqpoint{3.767796in}{0.610239in}}%
\pgfpathlineto{\pgfqpoint{3.768352in}{0.616375in}}%
\pgfpathlineto{\pgfqpoint{3.768907in}{0.622075in}}%
\pgfpathlineto{\pgfqpoint{3.770019in}{0.610334in}}%
\pgfpathlineto{\pgfqpoint{3.770575in}{0.611644in}}%
\pgfpathlineto{\pgfqpoint{3.771131in}{0.652624in}}%
\pgfpathlineto{\pgfqpoint{3.771687in}{0.622524in}}%
\pgfpathlineto{\pgfqpoint{3.772243in}{0.651224in}}%
\pgfpathlineto{\pgfqpoint{3.772798in}{0.640791in}}%
\pgfpathlineto{\pgfqpoint{3.773910in}{0.637275in}}%
\pgfpathlineto{\pgfqpoint{3.774466in}{0.734405in}}%
\pgfpathlineto{\pgfqpoint{3.775022in}{0.653695in}}%
\pgfpathlineto{\pgfqpoint{3.775578in}{0.620600in}}%
\pgfpathlineto{\pgfqpoint{3.776134in}{0.653327in}}%
\pgfpathlineto{\pgfqpoint{3.776689in}{0.683522in}}%
\pgfpathlineto{\pgfqpoint{3.777245in}{0.628249in}}%
\pgfpathlineto{\pgfqpoint{3.777801in}{0.650604in}}%
\pgfpathlineto{\pgfqpoint{3.778357in}{0.644332in}}%
\pgfpathlineto{\pgfqpoint{3.778913in}{0.707782in}}%
\pgfpathlineto{\pgfqpoint{3.779469in}{0.657210in}}%
\pgfpathlineto{\pgfqpoint{3.780025in}{0.666980in}}%
\pgfpathlineto{\pgfqpoint{3.780580in}{0.620395in}}%
\pgfpathlineto{\pgfqpoint{3.781136in}{0.657192in}}%
\pgfpathlineto{\pgfqpoint{3.781692in}{0.677888in}}%
\pgfpathlineto{\pgfqpoint{3.782248in}{0.617637in}}%
\pgfpathlineto{\pgfqpoint{3.782804in}{0.631010in}}%
\pgfpathlineto{\pgfqpoint{3.783360in}{0.629256in}}%
\pgfpathlineto{\pgfqpoint{3.784472in}{0.619749in}}%
\pgfpathlineto{\pgfqpoint{3.785027in}{0.660467in}}%
\pgfpathlineto{\pgfqpoint{3.785583in}{0.626281in}}%
\pgfpathlineto{\pgfqpoint{3.786139in}{0.625769in}}%
\pgfpathlineto{\pgfqpoint{3.786695in}{0.622021in}}%
\pgfpathlineto{\pgfqpoint{3.788363in}{0.660714in}}%
\pgfpathlineto{\pgfqpoint{3.789474in}{0.617366in}}%
\pgfpathlineto{\pgfqpoint{3.791142in}{0.659115in}}%
\pgfpathlineto{\pgfqpoint{3.791698in}{0.656928in}}%
\pgfpathlineto{\pgfqpoint{3.793921in}{0.627512in}}%
\pgfpathlineto{\pgfqpoint{3.794477in}{0.629859in}}%
\pgfpathlineto{\pgfqpoint{3.795033in}{0.613367in}}%
\pgfpathlineto{\pgfqpoint{3.795589in}{0.625893in}}%
\pgfpathlineto{\pgfqpoint{3.796700in}{0.610150in}}%
\pgfpathlineto{\pgfqpoint{3.797256in}{0.616968in}}%
\pgfpathlineto{\pgfqpoint{3.797812in}{0.616421in}}%
\pgfpathlineto{\pgfqpoint{3.798368in}{0.610968in}}%
\pgfpathlineto{\pgfqpoint{3.798924in}{0.622508in}}%
\pgfpathlineto{\pgfqpoint{3.799480in}{0.600948in}}%
\pgfpathlineto{\pgfqpoint{3.800036in}{0.618958in}}%
\pgfpathlineto{\pgfqpoint{3.800591in}{0.607182in}}%
\pgfpathlineto{\pgfqpoint{3.801147in}{0.613720in}}%
\pgfpathlineto{\pgfqpoint{3.801703in}{0.614706in}}%
\pgfpathlineto{\pgfqpoint{3.802259in}{0.613631in}}%
\pgfpathlineto{\pgfqpoint{3.802815in}{0.622016in}}%
\pgfpathlineto{\pgfqpoint{3.803371in}{0.603182in}}%
\pgfpathlineto{\pgfqpoint{3.803927in}{0.617948in}}%
\pgfpathlineto{\pgfqpoint{3.806150in}{0.604076in}}%
\pgfpathlineto{\pgfqpoint{3.807262in}{0.613991in}}%
\pgfpathlineto{\pgfqpoint{3.807818in}{0.603652in}}%
\pgfpathlineto{\pgfqpoint{3.808374in}{0.607899in}}%
\pgfpathlineto{\pgfqpoint{3.808929in}{0.613302in}}%
\pgfpathlineto{\pgfqpoint{3.809485in}{0.608181in}}%
\pgfpathlineto{\pgfqpoint{3.811153in}{0.614788in}}%
\pgfpathlineto{\pgfqpoint{3.811709in}{0.612989in}}%
\pgfpathlineto{\pgfqpoint{3.812820in}{0.604529in}}%
\pgfpathlineto{\pgfqpoint{3.813376in}{0.609170in}}%
\pgfpathlineto{\pgfqpoint{3.813932in}{0.608640in}}%
\pgfpathlineto{\pgfqpoint{3.814488in}{0.608794in}}%
\pgfpathlineto{\pgfqpoint{3.815044in}{0.605105in}}%
\pgfpathlineto{\pgfqpoint{3.815600in}{0.606938in}}%
\pgfpathlineto{\pgfqpoint{3.816156in}{0.610183in}}%
\pgfpathlineto{\pgfqpoint{3.816711in}{0.605370in}}%
\pgfpathlineto{\pgfqpoint{3.817267in}{0.606967in}}%
\pgfpathlineto{\pgfqpoint{3.817823in}{0.615168in}}%
\pgfpathlineto{\pgfqpoint{3.818379in}{0.613673in}}%
\pgfpathlineto{\pgfqpoint{3.820047in}{0.611884in}}%
\pgfpathlineto{\pgfqpoint{3.820602in}{0.614182in}}%
\pgfpathlineto{\pgfqpoint{3.821158in}{0.621361in}}%
\pgfpathlineto{\pgfqpoint{3.821714in}{0.601330in}}%
\pgfpathlineto{\pgfqpoint{3.822270in}{0.619785in}}%
\pgfpathlineto{\pgfqpoint{3.822826in}{0.614810in}}%
\pgfpathlineto{\pgfqpoint{3.823382in}{0.621274in}}%
\pgfpathlineto{\pgfqpoint{3.823938in}{0.606827in}}%
\pgfpathlineto{\pgfqpoint{3.824494in}{0.617243in}}%
\pgfpathlineto{\pgfqpoint{3.826161in}{0.627465in}}%
\pgfpathlineto{\pgfqpoint{3.826717in}{0.604420in}}%
\pgfpathlineto{\pgfqpoint{3.828385in}{0.658334in}}%
\pgfpathlineto{\pgfqpoint{3.828940in}{0.649056in}}%
\pgfpathlineto{\pgfqpoint{3.829496in}{0.679663in}}%
\pgfpathlineto{\pgfqpoint{3.830052in}{0.614057in}}%
\pgfpathlineto{\pgfqpoint{3.830608in}{0.643943in}}%
\pgfpathlineto{\pgfqpoint{3.831164in}{0.653395in}}%
\pgfpathlineto{\pgfqpoint{3.831720in}{0.704674in}}%
\pgfpathlineto{\pgfqpoint{3.832276in}{0.672674in}}%
\pgfpathlineto{\pgfqpoint{3.833943in}{0.610116in}}%
\pgfpathlineto{\pgfqpoint{3.835611in}{0.693659in}}%
\pgfpathlineto{\pgfqpoint{3.837834in}{0.630342in}}%
\pgfpathlineto{\pgfqpoint{3.838946in}{0.667399in}}%
\pgfpathlineto{\pgfqpoint{3.840058in}{0.618739in}}%
\pgfpathlineto{\pgfqpoint{3.840614in}{0.633016in}}%
\pgfpathlineto{\pgfqpoint{3.841169in}{0.654970in}}%
\pgfpathlineto{\pgfqpoint{3.841725in}{0.612955in}}%
\pgfpathlineto{\pgfqpoint{3.842281in}{0.642814in}}%
\pgfpathlineto{\pgfqpoint{3.843393in}{0.609838in}}%
\pgfpathlineto{\pgfqpoint{3.843949in}{0.613179in}}%
\pgfpathlineto{\pgfqpoint{3.844505in}{0.658270in}}%
\pgfpathlineto{\pgfqpoint{3.845060in}{0.629274in}}%
\pgfpathlineto{\pgfqpoint{3.846172in}{0.613276in}}%
\pgfpathlineto{\pgfqpoint{3.847840in}{0.652840in}}%
\pgfpathlineto{\pgfqpoint{3.848951in}{0.615059in}}%
\pgfpathlineto{\pgfqpoint{3.850063in}{0.618909in}}%
\pgfpathlineto{\pgfqpoint{3.851731in}{0.644949in}}%
\pgfpathlineto{\pgfqpoint{3.852287in}{0.663034in}}%
\pgfpathlineto{\pgfqpoint{3.853954in}{0.627350in}}%
\pgfpathlineto{\pgfqpoint{3.854510in}{0.635396in}}%
\pgfpathlineto{\pgfqpoint{3.855066in}{0.628499in}}%
\pgfpathlineto{\pgfqpoint{3.856178in}{0.614942in}}%
\pgfpathlineto{\pgfqpoint{3.856733in}{0.617446in}}%
\pgfpathlineto{\pgfqpoint{3.857289in}{0.619121in}}%
\pgfpathlineto{\pgfqpoint{3.857845in}{0.605316in}}%
\pgfpathlineto{\pgfqpoint{3.858401in}{0.611455in}}%
\pgfpathlineto{\pgfqpoint{3.858957in}{0.616134in}}%
\pgfpathlineto{\pgfqpoint{3.859513in}{0.610814in}}%
\pgfpathlineto{\pgfqpoint{3.860069in}{0.618735in}}%
\pgfpathlineto{\pgfqpoint{3.860625in}{0.604357in}}%
\pgfpathlineto{\pgfqpoint{3.861180in}{0.608330in}}%
\pgfpathlineto{\pgfqpoint{3.862848in}{0.612401in}}%
\pgfpathlineto{\pgfqpoint{3.863404in}{0.605887in}}%
\pgfpathlineto{\pgfqpoint{3.864516in}{0.618622in}}%
\pgfpathlineto{\pgfqpoint{3.865071in}{0.618209in}}%
\pgfpathlineto{\pgfqpoint{3.865627in}{0.619901in}}%
\pgfpathlineto{\pgfqpoint{3.866183in}{0.605846in}}%
\pgfpathlineto{\pgfqpoint{3.866739in}{0.608657in}}%
\pgfpathlineto{\pgfqpoint{3.867295in}{0.617425in}}%
\pgfpathlineto{\pgfqpoint{3.867851in}{0.609894in}}%
\pgfpathlineto{\pgfqpoint{3.868407in}{0.611427in}}%
\pgfpathlineto{\pgfqpoint{3.868962in}{0.617408in}}%
\pgfpathlineto{\pgfqpoint{3.869518in}{0.603999in}}%
\pgfpathlineto{\pgfqpoint{3.870074in}{0.610435in}}%
\pgfpathlineto{\pgfqpoint{3.870630in}{0.613653in}}%
\pgfpathlineto{\pgfqpoint{3.871186in}{0.604150in}}%
\pgfpathlineto{\pgfqpoint{3.871742in}{0.612299in}}%
\pgfpathlineto{\pgfqpoint{3.872298in}{0.627808in}}%
\pgfpathlineto{\pgfqpoint{3.872853in}{0.613231in}}%
\pgfpathlineto{\pgfqpoint{3.873409in}{0.615478in}}%
\pgfpathlineto{\pgfqpoint{3.873965in}{0.613657in}}%
\pgfpathlineto{\pgfqpoint{3.874521in}{0.609920in}}%
\pgfpathlineto{\pgfqpoint{3.875077in}{0.619388in}}%
\pgfpathlineto{\pgfqpoint{3.875633in}{0.604512in}}%
\pgfpathlineto{\pgfqpoint{3.876189in}{0.620183in}}%
\pgfpathlineto{\pgfqpoint{3.876744in}{0.618727in}}%
\pgfpathlineto{\pgfqpoint{3.877300in}{0.617973in}}%
\pgfpathlineto{\pgfqpoint{3.877856in}{0.620759in}}%
\pgfpathlineto{\pgfqpoint{3.878968in}{0.604676in}}%
\pgfpathlineto{\pgfqpoint{3.879524in}{0.627226in}}%
\pgfpathlineto{\pgfqpoint{3.880080in}{0.611236in}}%
\pgfpathlineto{\pgfqpoint{3.880636in}{0.617107in}}%
\pgfpathlineto{\pgfqpoint{3.881191in}{0.607910in}}%
\pgfpathlineto{\pgfqpoint{3.881747in}{0.616093in}}%
\pgfpathlineto{\pgfqpoint{3.883971in}{0.627379in}}%
\pgfpathlineto{\pgfqpoint{3.884527in}{0.664395in}}%
\pgfpathlineto{\pgfqpoint{3.885082in}{0.626939in}}%
\pgfpathlineto{\pgfqpoint{3.885638in}{0.660324in}}%
\pgfpathlineto{\pgfqpoint{3.886750in}{0.678510in}}%
\pgfpathlineto{\pgfqpoint{3.888418in}{0.629266in}}%
\pgfpathlineto{\pgfqpoint{3.889529in}{0.740169in}}%
\pgfpathlineto{\pgfqpoint{3.890641in}{0.614294in}}%
\pgfpathlineto{\pgfqpoint{3.891197in}{0.667178in}}%
\pgfpathlineto{\pgfqpoint{3.891753in}{0.636736in}}%
\pgfpathlineto{\pgfqpoint{3.892864in}{0.682803in}}%
\pgfpathlineto{\pgfqpoint{3.893420in}{0.609928in}}%
\pgfpathlineto{\pgfqpoint{3.893976in}{0.705271in}}%
\pgfpathlineto{\pgfqpoint{3.894532in}{0.645062in}}%
\pgfpathlineto{\pgfqpoint{3.895088in}{0.693693in}}%
\pgfpathlineto{\pgfqpoint{3.895644in}{0.657548in}}%
\pgfpathlineto{\pgfqpoint{3.896755in}{0.705711in}}%
\pgfpathlineto{\pgfqpoint{3.897311in}{0.630887in}}%
\pgfpathlineto{\pgfqpoint{3.897867in}{0.641955in}}%
\pgfpathlineto{\pgfqpoint{3.898423in}{0.663523in}}%
\pgfpathlineto{\pgfqpoint{3.898979in}{0.643909in}}%
\pgfpathlineto{\pgfqpoint{3.899535in}{0.619083in}}%
\pgfpathlineto{\pgfqpoint{3.900091in}{0.658919in}}%
\pgfpathlineto{\pgfqpoint{3.900647in}{0.620634in}}%
\pgfpathlineto{\pgfqpoint{3.901202in}{0.607820in}}%
\pgfpathlineto{\pgfqpoint{3.901758in}{0.671281in}}%
\pgfpathlineto{\pgfqpoint{3.902314in}{0.632087in}}%
\pgfpathlineto{\pgfqpoint{3.902870in}{0.610932in}}%
\pgfpathlineto{\pgfqpoint{3.903426in}{0.626161in}}%
\pgfpathlineto{\pgfqpoint{3.905093in}{0.647525in}}%
\pgfpathlineto{\pgfqpoint{3.905649in}{0.620030in}}%
\pgfpathlineto{\pgfqpoint{3.906205in}{0.624203in}}%
\pgfpathlineto{\pgfqpoint{3.907317in}{0.672832in}}%
\pgfpathlineto{\pgfqpoint{3.909540in}{0.608264in}}%
\pgfpathlineto{\pgfqpoint{3.910096in}{0.618962in}}%
\pgfpathlineto{\pgfqpoint{3.911764in}{0.661711in}}%
\pgfpathlineto{\pgfqpoint{3.912320in}{0.661374in}}%
\pgfpathlineto{\pgfqpoint{3.913987in}{0.640346in}}%
\pgfpathlineto{\pgfqpoint{3.914543in}{0.631428in}}%
\pgfpathlineto{\pgfqpoint{3.915099in}{0.652083in}}%
\pgfpathlineto{\pgfqpoint{3.915655in}{0.642024in}}%
\pgfpathlineto{\pgfqpoint{3.917322in}{0.605395in}}%
\pgfpathlineto{\pgfqpoint{3.917878in}{0.607850in}}%
\pgfpathlineto{\pgfqpoint{3.918434in}{0.605193in}}%
\pgfpathlineto{\pgfqpoint{3.920102in}{0.613464in}}%
\pgfpathlineto{\pgfqpoint{3.920658in}{0.605108in}}%
\pgfpathlineto{\pgfqpoint{3.921213in}{0.609000in}}%
\pgfpathlineto{\pgfqpoint{3.922325in}{0.609957in}}%
\pgfpathlineto{\pgfqpoint{3.922881in}{0.611680in}}%
\pgfpathlineto{\pgfqpoint{3.923437in}{0.607375in}}%
\pgfpathlineto{\pgfqpoint{3.923993in}{0.618009in}}%
\pgfpathlineto{\pgfqpoint{3.924549in}{0.604488in}}%
\pgfpathlineto{\pgfqpoint{3.925104in}{0.607658in}}%
\pgfpathlineto{\pgfqpoint{3.925660in}{0.608294in}}%
\pgfpathlineto{\pgfqpoint{3.926772in}{0.602841in}}%
\pgfpathlineto{\pgfqpoint{3.927328in}{0.602924in}}%
\pgfpathlineto{\pgfqpoint{3.928440in}{0.617073in}}%
\pgfpathlineto{\pgfqpoint{3.930107in}{0.607088in}}%
\pgfpathlineto{\pgfqpoint{3.931775in}{0.621516in}}%
\pgfpathlineto{\pgfqpoint{3.932331in}{0.602646in}}%
\pgfpathlineto{\pgfqpoint{3.932886in}{0.617968in}}%
\pgfpathlineto{\pgfqpoint{3.933442in}{0.621322in}}%
\pgfpathlineto{\pgfqpoint{3.933998in}{0.618905in}}%
\pgfpathlineto{\pgfqpoint{3.934554in}{0.612037in}}%
\pgfpathlineto{\pgfqpoint{3.935110in}{0.619594in}}%
\pgfpathlineto{\pgfqpoint{3.935666in}{0.603025in}}%
\pgfpathlineto{\pgfqpoint{3.936222in}{0.614198in}}%
\pgfpathlineto{\pgfqpoint{3.936778in}{0.614589in}}%
\pgfpathlineto{\pgfqpoint{3.937333in}{0.620767in}}%
\pgfpathlineto{\pgfqpoint{3.937889in}{0.613473in}}%
\pgfpathlineto{\pgfqpoint{3.939001in}{0.628736in}}%
\pgfpathlineto{\pgfqpoint{3.939557in}{0.606556in}}%
\pgfpathlineto{\pgfqpoint{3.941224in}{0.660457in}}%
\pgfpathlineto{\pgfqpoint{3.942892in}{0.628733in}}%
\pgfpathlineto{\pgfqpoint{3.944560in}{0.700789in}}%
\pgfpathlineto{\pgfqpoint{3.945671in}{0.632482in}}%
\pgfpathlineto{\pgfqpoint{3.946783in}{0.733088in}}%
\pgfpathlineto{\pgfqpoint{3.947895in}{0.617851in}}%
\pgfpathlineto{\pgfqpoint{3.948451in}{0.621789in}}%
\pgfpathlineto{\pgfqpoint{3.949006in}{0.623637in}}%
\pgfpathlineto{\pgfqpoint{3.949562in}{0.620890in}}%
\pgfpathlineto{\pgfqpoint{3.951230in}{0.689538in}}%
\pgfpathlineto{\pgfqpoint{3.952897in}{0.626414in}}%
\pgfpathlineto{\pgfqpoint{3.954009in}{0.699741in}}%
\pgfpathlineto{\pgfqpoint{3.955121in}{0.622859in}}%
\pgfpathlineto{\pgfqpoint{3.955677in}{0.625028in}}%
\pgfpathlineto{\pgfqpoint{3.956233in}{0.676106in}}%
\pgfpathlineto{\pgfqpoint{3.956789in}{0.609462in}}%
\pgfpathlineto{\pgfqpoint{3.957344in}{0.653882in}}%
\pgfpathlineto{\pgfqpoint{3.957900in}{0.619863in}}%
\pgfpathlineto{\pgfqpoint{3.958456in}{0.628583in}}%
\pgfpathlineto{\pgfqpoint{3.959012in}{0.628206in}}%
\pgfpathlineto{\pgfqpoint{3.959568in}{0.648107in}}%
\pgfpathlineto{\pgfqpoint{3.960680in}{0.607322in}}%
\pgfpathlineto{\pgfqpoint{3.961235in}{0.648597in}}%
\pgfpathlineto{\pgfqpoint{3.961791in}{0.639555in}}%
\pgfpathlineto{\pgfqpoint{3.962903in}{0.607758in}}%
\pgfpathlineto{\pgfqpoint{3.964571in}{0.650121in}}%
\pgfpathlineto{\pgfqpoint{3.965682in}{0.604796in}}%
\pgfpathlineto{\pgfqpoint{3.966238in}{0.618391in}}%
\pgfpathlineto{\pgfqpoint{3.967350in}{0.652880in}}%
\pgfpathlineto{\pgfqpoint{3.967906in}{0.647126in}}%
\pgfpathlineto{\pgfqpoint{3.969573in}{0.602088in}}%
\pgfpathlineto{\pgfqpoint{3.971797in}{0.644450in}}%
\pgfpathlineto{\pgfqpoint{3.972353in}{0.636421in}}%
\pgfpathlineto{\pgfqpoint{3.972909in}{0.638440in}}%
\pgfpathlineto{\pgfqpoint{3.973464in}{0.645718in}}%
\pgfpathlineto{\pgfqpoint{3.975688in}{0.625504in}}%
\pgfpathlineto{\pgfqpoint{3.981246in}{0.601546in}}%
\pgfpathlineto{\pgfqpoint{3.981802in}{0.602103in}}%
\pgfpathlineto{\pgfqpoint{3.982914in}{0.609109in}}%
\pgfpathlineto{\pgfqpoint{3.983470in}{0.603917in}}%
\pgfpathlineto{\pgfqpoint{3.984582in}{0.614047in}}%
\pgfpathlineto{\pgfqpoint{3.985137in}{0.605183in}}%
\pgfpathlineto{\pgfqpoint{3.985693in}{0.607941in}}%
\pgfpathlineto{\pgfqpoint{3.986249in}{0.614505in}}%
\pgfpathlineto{\pgfqpoint{3.986805in}{0.608131in}}%
\pgfpathlineto{\pgfqpoint{3.987361in}{0.605006in}}%
\pgfpathlineto{\pgfqpoint{3.989028in}{0.614549in}}%
\pgfpathlineto{\pgfqpoint{3.989584in}{0.613224in}}%
\pgfpathlineto{\pgfqpoint{3.990140in}{0.613913in}}%
\pgfpathlineto{\pgfqpoint{3.991808in}{0.605660in}}%
\pgfpathlineto{\pgfqpoint{3.993475in}{0.628165in}}%
\pgfpathlineto{\pgfqpoint{3.995143in}{0.609188in}}%
\pgfpathlineto{\pgfqpoint{3.996255in}{0.634470in}}%
\pgfpathlineto{\pgfqpoint{3.996811in}{0.608295in}}%
\pgfpathlineto{\pgfqpoint{3.998478in}{0.645287in}}%
\pgfpathlineto{\pgfqpoint{3.999034in}{0.622080in}}%
\pgfpathlineto{\pgfqpoint{3.999590in}{0.658072in}}%
\pgfpathlineto{\pgfqpoint{4.000146in}{0.610449in}}%
\pgfpathlineto{\pgfqpoint{4.000702in}{0.648123in}}%
\pgfpathlineto{\pgfqpoint{4.001257in}{0.642790in}}%
\pgfpathlineto{\pgfqpoint{4.001813in}{0.694989in}}%
\pgfpathlineto{\pgfqpoint{4.002369in}{0.652418in}}%
\pgfpathlineto{\pgfqpoint{4.002925in}{0.650751in}}%
\pgfpathlineto{\pgfqpoint{4.003481in}{0.644062in}}%
\pgfpathlineto{\pgfqpoint{4.004593in}{0.671839in}}%
\pgfpathlineto{\pgfqpoint{4.005704in}{0.622219in}}%
\pgfpathlineto{\pgfqpoint{4.006260in}{0.652122in}}%
\pgfpathlineto{\pgfqpoint{4.006816in}{0.607949in}}%
\pgfpathlineto{\pgfqpoint{4.007928in}{0.675792in}}%
\pgfpathlineto{\pgfqpoint{4.008484in}{0.618785in}}%
\pgfpathlineto{\pgfqpoint{4.009039in}{0.625726in}}%
\pgfpathlineto{\pgfqpoint{4.009595in}{0.622508in}}%
\pgfpathlineto{\pgfqpoint{4.011263in}{0.671370in}}%
\pgfpathlineto{\pgfqpoint{4.012931in}{0.624461in}}%
\pgfpathlineto{\pgfqpoint{4.013486in}{0.674374in}}%
\pgfpathlineto{\pgfqpoint{4.014042in}{0.628136in}}%
\pgfpathlineto{\pgfqpoint{4.015710in}{0.639813in}}%
\pgfpathlineto{\pgfqpoint{4.016266in}{0.606781in}}%
\pgfpathlineto{\pgfqpoint{4.016822in}{0.651314in}}%
\pgfpathlineto{\pgfqpoint{4.017377in}{0.610339in}}%
\pgfpathlineto{\pgfqpoint{4.019045in}{0.630012in}}%
\pgfpathlineto{\pgfqpoint{4.019601in}{0.612534in}}%
\pgfpathlineto{\pgfqpoint{4.020157in}{0.616047in}}%
\pgfpathlineto{\pgfqpoint{4.021268in}{0.649217in}}%
\pgfpathlineto{\pgfqpoint{4.021824in}{0.631961in}}%
\pgfpathlineto{\pgfqpoint{4.022380in}{0.606282in}}%
\pgfpathlineto{\pgfqpoint{4.022936in}{0.625341in}}%
\pgfpathlineto{\pgfqpoint{4.024048in}{0.638720in}}%
\pgfpathlineto{\pgfqpoint{4.025715in}{0.607094in}}%
\pgfpathlineto{\pgfqpoint{4.027383in}{0.639610in}}%
\pgfpathlineto{\pgfqpoint{4.029051in}{0.604617in}}%
\pgfpathlineto{\pgfqpoint{4.029606in}{0.609942in}}%
\pgfpathlineto{\pgfqpoint{4.030162in}{0.616357in}}%
\pgfpathlineto{\pgfqpoint{4.030718in}{0.638268in}}%
\pgfpathlineto{\pgfqpoint{4.031274in}{0.632764in}}%
\pgfpathlineto{\pgfqpoint{4.031830in}{0.631413in}}%
\pgfpathlineto{\pgfqpoint{4.032942in}{0.648579in}}%
\pgfpathlineto{\pgfqpoint{4.034609in}{0.632257in}}%
\pgfpathlineto{\pgfqpoint{4.035165in}{0.634375in}}%
\pgfpathlineto{\pgfqpoint{4.036277in}{0.634242in}}%
\pgfpathlineto{\pgfqpoint{4.037388in}{0.612318in}}%
\pgfpathlineto{\pgfqpoint{4.037944in}{0.613416in}}%
\pgfpathlineto{\pgfqpoint{4.038500in}{0.607866in}}%
\pgfpathlineto{\pgfqpoint{4.040168in}{0.618366in}}%
\pgfpathlineto{\pgfqpoint{4.040724in}{0.614308in}}%
\pgfpathlineto{\pgfqpoint{4.041279in}{0.602042in}}%
\pgfpathlineto{\pgfqpoint{4.041835in}{0.606887in}}%
\pgfpathlineto{\pgfqpoint{4.042947in}{0.607564in}}%
\pgfpathlineto{\pgfqpoint{4.043503in}{0.618967in}}%
\pgfpathlineto{\pgfqpoint{4.044059in}{0.609939in}}%
\pgfpathlineto{\pgfqpoint{4.044615in}{0.617470in}}%
\pgfpathlineto{\pgfqpoint{4.045170in}{0.612714in}}%
\pgfpathlineto{\pgfqpoint{4.045726in}{0.611516in}}%
\pgfpathlineto{\pgfqpoint{4.046838in}{0.617953in}}%
\pgfpathlineto{\pgfqpoint{4.047950in}{0.609651in}}%
\pgfpathlineto{\pgfqpoint{4.049617in}{0.631348in}}%
\pgfpathlineto{\pgfqpoint{4.050173in}{0.606140in}}%
\pgfpathlineto{\pgfqpoint{4.050729in}{0.614882in}}%
\pgfpathlineto{\pgfqpoint{4.051841in}{0.616617in}}%
\pgfpathlineto{\pgfqpoint{4.052397in}{0.601079in}}%
\pgfpathlineto{\pgfqpoint{4.052953in}{0.613269in}}%
\pgfpathlineto{\pgfqpoint{4.053508in}{0.617970in}}%
\pgfpathlineto{\pgfqpoint{4.054064in}{0.634736in}}%
\pgfpathlineto{\pgfqpoint{4.054620in}{0.627586in}}%
\pgfpathlineto{\pgfqpoint{4.055176in}{0.627898in}}%
\pgfpathlineto{\pgfqpoint{4.055732in}{0.626048in}}%
\pgfpathlineto{\pgfqpoint{4.056844in}{0.653690in}}%
\pgfpathlineto{\pgfqpoint{4.057955in}{0.638310in}}%
\pgfpathlineto{\pgfqpoint{4.059623in}{0.668754in}}%
\pgfpathlineto{\pgfqpoint{4.060735in}{0.630311in}}%
\pgfpathlineto{\pgfqpoint{4.061846in}{0.668561in}}%
\pgfpathlineto{\pgfqpoint{4.063514in}{0.621847in}}%
\pgfpathlineto{\pgfqpoint{4.064070in}{0.623977in}}%
\pgfpathlineto{\pgfqpoint{4.065737in}{0.660441in}}%
\pgfpathlineto{\pgfqpoint{4.066293in}{0.634334in}}%
\pgfpathlineto{\pgfqpoint{4.066849in}{0.651930in}}%
\pgfpathlineto{\pgfqpoint{4.067405in}{0.666646in}}%
\pgfpathlineto{\pgfqpoint{4.067961in}{0.619287in}}%
\pgfpathlineto{\pgfqpoint{4.068517in}{0.655190in}}%
\pgfpathlineto{\pgfqpoint{4.069628in}{0.645779in}}%
\pgfpathlineto{\pgfqpoint{4.070184in}{0.611457in}}%
\pgfpathlineto{\pgfqpoint{4.071296in}{0.663020in}}%
\pgfpathlineto{\pgfqpoint{4.071852in}{0.611440in}}%
\pgfpathlineto{\pgfqpoint{4.072408in}{0.655722in}}%
\pgfpathlineto{\pgfqpoint{4.073519in}{0.607795in}}%
\pgfpathlineto{\pgfqpoint{4.074075in}{0.647597in}}%
\pgfpathlineto{\pgfqpoint{4.074631in}{0.625870in}}%
\pgfpathlineto{\pgfqpoint{4.075187in}{0.627395in}}%
\pgfpathlineto{\pgfqpoint{4.075743in}{0.606004in}}%
\pgfpathlineto{\pgfqpoint{4.076299in}{0.631986in}}%
\pgfpathlineto{\pgfqpoint{4.076855in}{0.623052in}}%
\pgfpathlineto{\pgfqpoint{4.077410in}{0.621860in}}%
\pgfpathlineto{\pgfqpoint{4.077966in}{0.626546in}}%
\pgfpathlineto{\pgfqpoint{4.078522in}{0.649064in}}%
\pgfpathlineto{\pgfqpoint{4.079634in}{0.607458in}}%
\pgfpathlineto{\pgfqpoint{4.080746in}{0.636582in}}%
\pgfpathlineto{\pgfqpoint{4.081301in}{0.625933in}}%
\pgfpathlineto{\pgfqpoint{4.081857in}{0.625426in}}%
\pgfpathlineto{\pgfqpoint{4.082969in}{0.610960in}}%
\pgfpathlineto{\pgfqpoint{4.083525in}{0.647888in}}%
\pgfpathlineto{\pgfqpoint{4.084081in}{0.631590in}}%
\pgfpathlineto{\pgfqpoint{4.085748in}{0.611939in}}%
\pgfpathlineto{\pgfqpoint{4.086304in}{0.618027in}}%
\pgfpathlineto{\pgfqpoint{4.087416in}{0.647499in}}%
\pgfpathlineto{\pgfqpoint{4.087972in}{0.644625in}}%
\pgfpathlineto{\pgfqpoint{4.089639in}{0.608498in}}%
\pgfpathlineto{\pgfqpoint{4.090195in}{0.613399in}}%
\pgfpathlineto{\pgfqpoint{4.091863in}{0.639168in}}%
\pgfpathlineto{\pgfqpoint{4.092419in}{0.639980in}}%
\pgfpathlineto{\pgfqpoint{4.092975in}{0.626895in}}%
\pgfpathlineto{\pgfqpoint{4.093530in}{0.637975in}}%
\pgfpathlineto{\pgfqpoint{4.094086in}{0.636332in}}%
\pgfpathlineto{\pgfqpoint{4.094642in}{0.646435in}}%
\pgfpathlineto{\pgfqpoint{4.095754in}{0.627342in}}%
\pgfpathlineto{\pgfqpoint{4.096310in}{0.629583in}}%
\pgfpathlineto{\pgfqpoint{4.098533in}{0.606503in}}%
\pgfpathlineto{\pgfqpoint{4.099089in}{0.607078in}}%
\pgfpathlineto{\pgfqpoint{4.099645in}{0.617624in}}%
\pgfpathlineto{\pgfqpoint{4.100201in}{0.601336in}}%
\pgfpathlineto{\pgfqpoint{4.100757in}{0.608881in}}%
\pgfpathlineto{\pgfqpoint{4.101312in}{0.604667in}}%
\pgfpathlineto{\pgfqpoint{4.101868in}{0.615267in}}%
\pgfpathlineto{\pgfqpoint{4.102424in}{0.603084in}}%
\pgfpathlineto{\pgfqpoint{4.102980in}{0.609462in}}%
\pgfpathlineto{\pgfqpoint{4.103536in}{0.605740in}}%
\pgfpathlineto{\pgfqpoint{4.104092in}{0.616440in}}%
\pgfpathlineto{\pgfqpoint{4.104648in}{0.607043in}}%
\pgfpathlineto{\pgfqpoint{4.105204in}{0.601672in}}%
\pgfpathlineto{\pgfqpoint{4.105759in}{0.617940in}}%
\pgfpathlineto{\pgfqpoint{4.106315in}{0.609405in}}%
\pgfpathlineto{\pgfqpoint{4.107427in}{0.616911in}}%
\pgfpathlineto{\pgfqpoint{4.108539in}{0.608805in}}%
\pgfpathlineto{\pgfqpoint{4.109095in}{0.642079in}}%
\pgfpathlineto{\pgfqpoint{4.109650in}{0.613576in}}%
\pgfpathlineto{\pgfqpoint{4.110206in}{0.623679in}}%
\pgfpathlineto{\pgfqpoint{4.110762in}{0.608688in}}%
\pgfpathlineto{\pgfqpoint{4.111318in}{0.639981in}}%
\pgfpathlineto{\pgfqpoint{4.111874in}{0.636206in}}%
\pgfpathlineto{\pgfqpoint{4.112986in}{0.620369in}}%
\pgfpathlineto{\pgfqpoint{4.114653in}{0.664768in}}%
\pgfpathlineto{\pgfqpoint{4.115765in}{0.626961in}}%
\pgfpathlineto{\pgfqpoint{4.116877in}{0.687273in}}%
\pgfpathlineto{\pgfqpoint{4.117432in}{0.620325in}}%
\pgfpathlineto{\pgfqpoint{4.117988in}{0.634278in}}%
\pgfpathlineto{\pgfqpoint{4.118544in}{0.656533in}}%
\pgfpathlineto{\pgfqpoint{4.119100in}{0.645211in}}%
\pgfpathlineto{\pgfqpoint{4.119656in}{0.611617in}}%
\pgfpathlineto{\pgfqpoint{4.120212in}{0.648124in}}%
\pgfpathlineto{\pgfqpoint{4.120768in}{0.635033in}}%
\pgfpathlineto{\pgfqpoint{4.121323in}{0.644324in}}%
\pgfpathlineto{\pgfqpoint{4.121879in}{0.608831in}}%
\pgfpathlineto{\pgfqpoint{4.122435in}{0.642377in}}%
\pgfpathlineto{\pgfqpoint{4.122991in}{0.667411in}}%
\pgfpathlineto{\pgfqpoint{4.123547in}{0.617693in}}%
\pgfpathlineto{\pgfqpoint{4.124103in}{0.639626in}}%
\pgfpathlineto{\pgfqpoint{4.124659in}{0.641017in}}%
\pgfpathlineto{\pgfqpoint{4.126326in}{0.653366in}}%
\pgfpathlineto{\pgfqpoint{4.127994in}{0.617288in}}%
\pgfpathlineto{\pgfqpoint{4.128550in}{0.687655in}}%
\pgfpathlineto{\pgfqpoint{4.129106in}{0.610392in}}%
\pgfpathlineto{\pgfqpoint{4.129661in}{0.645466in}}%
\pgfpathlineto{\pgfqpoint{4.130217in}{0.619363in}}%
\pgfpathlineto{\pgfqpoint{4.130773in}{0.626607in}}%
\pgfpathlineto{\pgfqpoint{4.131329in}{0.619503in}}%
\pgfpathlineto{\pgfqpoint{4.131885in}{0.647721in}}%
\pgfpathlineto{\pgfqpoint{4.132997in}{0.607278in}}%
\pgfpathlineto{\pgfqpoint{4.133552in}{0.631279in}}%
\pgfpathlineto{\pgfqpoint{4.134108in}{0.626972in}}%
\pgfpathlineto{\pgfqpoint{4.134664in}{0.626592in}}%
\pgfpathlineto{\pgfqpoint{4.135220in}{0.605403in}}%
\pgfpathlineto{\pgfqpoint{4.135776in}{0.629563in}}%
\pgfpathlineto{\pgfqpoint{4.136332in}{0.615838in}}%
\pgfpathlineto{\pgfqpoint{4.136888in}{0.608443in}}%
\pgfpathlineto{\pgfqpoint{4.137443in}{0.612665in}}%
\pgfpathlineto{\pgfqpoint{4.137999in}{0.634816in}}%
\pgfpathlineto{\pgfqpoint{4.138555in}{0.620072in}}%
\pgfpathlineto{\pgfqpoint{4.139111in}{0.609666in}}%
\pgfpathlineto{\pgfqpoint{4.140223in}{0.629839in}}%
\pgfpathlineto{\pgfqpoint{4.140779in}{0.624151in}}%
\pgfpathlineto{\pgfqpoint{4.141334in}{0.634161in}}%
\pgfpathlineto{\pgfqpoint{4.142446in}{0.603231in}}%
\pgfpathlineto{\pgfqpoint{4.144114in}{0.636082in}}%
\pgfpathlineto{\pgfqpoint{4.145781in}{0.613062in}}%
\pgfpathlineto{\pgfqpoint{4.146337in}{0.617786in}}%
\pgfpathlineto{\pgfqpoint{4.146893in}{0.628276in}}%
\pgfpathlineto{\pgfqpoint{4.147449in}{0.627946in}}%
\pgfpathlineto{\pgfqpoint{4.149117in}{0.617736in}}%
\pgfpathlineto{\pgfqpoint{4.149672in}{0.609669in}}%
\pgfpathlineto{\pgfqpoint{4.150228in}{0.614932in}}%
\pgfpathlineto{\pgfqpoint{4.150784in}{0.616875in}}%
\pgfpathlineto{\pgfqpoint{4.151340in}{0.616655in}}%
\pgfpathlineto{\pgfqpoint{4.153008in}{0.638435in}}%
\pgfpathlineto{\pgfqpoint{4.153563in}{0.638694in}}%
\pgfpathlineto{\pgfqpoint{4.154675in}{0.626271in}}%
\pgfpathlineto{\pgfqpoint{4.155231in}{0.637681in}}%
\pgfpathlineto{\pgfqpoint{4.156899in}{0.612746in}}%
\pgfpathlineto{\pgfqpoint{4.157454in}{0.614683in}}%
\pgfpathlineto{\pgfqpoint{4.158010in}{0.625373in}}%
\pgfpathlineto{\pgfqpoint{4.158566in}{0.623622in}}%
\pgfpathlineto{\pgfqpoint{4.159678in}{0.604705in}}%
\pgfpathlineto{\pgfqpoint{4.160234in}{0.620867in}}%
\pgfpathlineto{\pgfqpoint{4.160790in}{0.611026in}}%
\pgfpathlineto{\pgfqpoint{4.161346in}{0.618818in}}%
\pgfpathlineto{\pgfqpoint{4.161901in}{0.607582in}}%
\pgfpathlineto{\pgfqpoint{4.162457in}{0.624700in}}%
\pgfpathlineto{\pgfqpoint{4.164125in}{0.607034in}}%
\pgfpathlineto{\pgfqpoint{4.166348in}{0.637942in}}%
\pgfpathlineto{\pgfqpoint{4.166904in}{0.604170in}}%
\pgfpathlineto{\pgfqpoint{4.167460in}{0.625528in}}%
\pgfpathlineto{\pgfqpoint{4.168016in}{0.606494in}}%
\pgfpathlineto{\pgfqpoint{4.169683in}{0.642812in}}%
\pgfpathlineto{\pgfqpoint{4.170239in}{0.617540in}}%
\pgfpathlineto{\pgfqpoint{4.170795in}{0.634228in}}%
\pgfpathlineto{\pgfqpoint{4.171351in}{0.629811in}}%
\pgfpathlineto{\pgfqpoint{4.171907in}{0.651321in}}%
\pgfpathlineto{\pgfqpoint{4.172463in}{0.628941in}}%
\pgfpathlineto{\pgfqpoint{4.173019in}{0.629840in}}%
\pgfpathlineto{\pgfqpoint{4.173574in}{0.638016in}}%
\pgfpathlineto{\pgfqpoint{4.174130in}{0.665767in}}%
\pgfpathlineto{\pgfqpoint{4.175798in}{0.631272in}}%
\pgfpathlineto{\pgfqpoint{4.178021in}{0.658974in}}%
\pgfpathlineto{\pgfqpoint{4.179133in}{0.610437in}}%
\pgfpathlineto{\pgfqpoint{4.179689in}{0.625830in}}%
\pgfpathlineto{\pgfqpoint{4.180245in}{0.620777in}}%
\pgfpathlineto{\pgfqpoint{4.180801in}{0.631196in}}%
\pgfpathlineto{\pgfqpoint{4.181357in}{0.609451in}}%
\pgfpathlineto{\pgfqpoint{4.182468in}{0.650663in}}%
\pgfpathlineto{\pgfqpoint{4.184136in}{0.615856in}}%
\pgfpathlineto{\pgfqpoint{4.184692in}{0.651689in}}%
\pgfpathlineto{\pgfqpoint{4.185248in}{0.605843in}}%
\pgfpathlineto{\pgfqpoint{4.185803in}{0.668952in}}%
\pgfpathlineto{\pgfqpoint{4.186359in}{0.620060in}}%
\pgfpathlineto{\pgfqpoint{4.186915in}{0.619764in}}%
\pgfpathlineto{\pgfqpoint{4.187471in}{0.621718in}}%
\pgfpathlineto{\pgfqpoint{4.188027in}{0.643445in}}%
\pgfpathlineto{\pgfqpoint{4.188583in}{0.602857in}}%
\pgfpathlineto{\pgfqpoint{4.189139in}{0.642620in}}%
\pgfpathlineto{\pgfqpoint{4.189694in}{0.611647in}}%
\pgfpathlineto{\pgfqpoint{4.190250in}{0.613021in}}%
\pgfpathlineto{\pgfqpoint{4.190806in}{0.614235in}}%
\pgfpathlineto{\pgfqpoint{4.191918in}{0.627522in}}%
\pgfpathlineto{\pgfqpoint{4.193030in}{0.622841in}}%
\pgfpathlineto{\pgfqpoint{4.193585in}{0.629717in}}%
\pgfpathlineto{\pgfqpoint{4.194697in}{0.605216in}}%
\pgfpathlineto{\pgfqpoint{4.195253in}{0.625457in}}%
\pgfpathlineto{\pgfqpoint{4.195809in}{0.618783in}}%
\pgfpathlineto{\pgfqpoint{4.196365in}{0.603472in}}%
\pgfpathlineto{\pgfqpoint{4.196921in}{0.613438in}}%
\pgfpathlineto{\pgfqpoint{4.197476in}{0.627279in}}%
\pgfpathlineto{\pgfqpoint{4.199144in}{0.606290in}}%
\pgfpathlineto{\pgfqpoint{4.201368in}{0.624004in}}%
\pgfpathlineto{\pgfqpoint{4.201923in}{0.611781in}}%
\pgfpathlineto{\pgfqpoint{4.202479in}{0.613358in}}%
\pgfpathlineto{\pgfqpoint{4.203035in}{0.614381in}}%
\pgfpathlineto{\pgfqpoint{4.203591in}{0.637492in}}%
\pgfpathlineto{\pgfqpoint{4.204147in}{0.624932in}}%
\pgfpathlineto{\pgfqpoint{4.205814in}{0.601131in}}%
\pgfpathlineto{\pgfqpoint{4.206370in}{0.607676in}}%
\pgfpathlineto{\pgfqpoint{4.206926in}{0.634079in}}%
\pgfpathlineto{\pgfqpoint{4.207482in}{0.628213in}}%
\pgfpathlineto{\pgfqpoint{4.208038in}{0.631710in}}%
\pgfpathlineto{\pgfqpoint{4.209150in}{0.607840in}}%
\pgfpathlineto{\pgfqpoint{4.209705in}{0.608220in}}%
\pgfpathlineto{\pgfqpoint{4.210817in}{0.612302in}}%
\pgfpathlineto{\pgfqpoint{4.211929in}{0.624975in}}%
\pgfpathlineto{\pgfqpoint{4.212485in}{0.620954in}}%
\pgfpathlineto{\pgfqpoint{4.213041in}{0.622120in}}%
\pgfpathlineto{\pgfqpoint{4.213596in}{0.635284in}}%
\pgfpathlineto{\pgfqpoint{4.214152in}{0.629233in}}%
\pgfpathlineto{\pgfqpoint{4.214708in}{0.633396in}}%
\pgfpathlineto{\pgfqpoint{4.215264in}{0.614044in}}%
\pgfpathlineto{\pgfqpoint{4.215820in}{0.615742in}}%
\pgfpathlineto{\pgfqpoint{4.216932in}{0.629202in}}%
\pgfpathlineto{\pgfqpoint{4.218043in}{0.613511in}}%
\pgfpathlineto{\pgfqpoint{4.218599in}{0.616136in}}%
\pgfpathlineto{\pgfqpoint{4.219155in}{0.610135in}}%
\pgfpathlineto{\pgfqpoint{4.219711in}{0.619410in}}%
\pgfpathlineto{\pgfqpoint{4.220267in}{0.608230in}}%
\pgfpathlineto{\pgfqpoint{4.220823in}{0.614670in}}%
\pgfpathlineto{\pgfqpoint{4.221379in}{0.613991in}}%
\pgfpathlineto{\pgfqpoint{4.221934in}{0.616492in}}%
\pgfpathlineto{\pgfqpoint{4.222490in}{0.607841in}}%
\pgfpathlineto{\pgfqpoint{4.223046in}{0.609587in}}%
\pgfpathlineto{\pgfqpoint{4.224158in}{0.623159in}}%
\pgfpathlineto{\pgfqpoint{4.224714in}{0.613073in}}%
\pgfpathlineto{\pgfqpoint{4.225270in}{0.615606in}}%
\pgfpathlineto{\pgfqpoint{4.225825in}{0.614940in}}%
\pgfpathlineto{\pgfqpoint{4.226381in}{0.635854in}}%
\pgfpathlineto{\pgfqpoint{4.226937in}{0.634568in}}%
\pgfpathlineto{\pgfqpoint{4.228049in}{0.619619in}}%
\pgfpathlineto{\pgfqpoint{4.229716in}{0.635219in}}%
\pgfpathlineto{\pgfqpoint{4.230828in}{0.611527in}}%
\pgfpathlineto{\pgfqpoint{4.231940in}{0.649079in}}%
\pgfpathlineto{\pgfqpoint{4.233052in}{0.615939in}}%
\pgfpathlineto{\pgfqpoint{4.233607in}{0.630437in}}%
\pgfpathlineto{\pgfqpoint{4.234163in}{0.618612in}}%
\pgfpathlineto{\pgfqpoint{4.235275in}{0.638546in}}%
\pgfpathlineto{\pgfqpoint{4.236943in}{0.606073in}}%
\pgfpathlineto{\pgfqpoint{4.238054in}{0.634110in}}%
\pgfpathlineto{\pgfqpoint{4.238610in}{0.607551in}}%
\pgfpathlineto{\pgfqpoint{4.239166in}{0.636557in}}%
\pgfpathlineto{\pgfqpoint{4.239722in}{0.629822in}}%
\pgfpathlineto{\pgfqpoint{4.240278in}{0.614083in}}%
\pgfpathlineto{\pgfqpoint{4.240834in}{0.621901in}}%
\pgfpathlineto{\pgfqpoint{4.241390in}{0.624187in}}%
\pgfpathlineto{\pgfqpoint{4.241945in}{0.640746in}}%
\pgfpathlineto{\pgfqpoint{4.242501in}{0.618629in}}%
\pgfpathlineto{\pgfqpoint{4.243057in}{0.628106in}}%
\pgfpathlineto{\pgfqpoint{4.243613in}{0.645161in}}%
\pgfpathlineto{\pgfqpoint{4.244169in}{0.619533in}}%
\pgfpathlineto{\pgfqpoint{4.244725in}{0.631045in}}%
\pgfpathlineto{\pgfqpoint{4.245281in}{0.629786in}}%
\pgfpathlineto{\pgfqpoint{4.245836in}{0.612045in}}%
\pgfpathlineto{\pgfqpoint{4.246392in}{0.619204in}}%
\pgfpathlineto{\pgfqpoint{4.246948in}{0.623679in}}%
\pgfpathlineto{\pgfqpoint{4.248060in}{0.602703in}}%
\pgfpathlineto{\pgfqpoint{4.249727in}{0.627104in}}%
\pgfpathlineto{\pgfqpoint{4.250283in}{0.606723in}}%
\pgfpathlineto{\pgfqpoint{4.250839in}{0.621020in}}%
\pgfpathlineto{\pgfqpoint{4.252507in}{0.610136in}}%
\pgfpathlineto{\pgfqpoint{4.253063in}{0.617627in}}%
\pgfpathlineto{\pgfqpoint{4.254174in}{0.603041in}}%
\pgfpathlineto{\pgfqpoint{4.254730in}{0.624561in}}%
\pgfpathlineto{\pgfqpoint{4.255286in}{0.613175in}}%
\pgfpathlineto{\pgfqpoint{4.255842in}{0.612365in}}%
\pgfpathlineto{\pgfqpoint{4.257510in}{0.603505in}}%
\pgfpathlineto{\pgfqpoint{4.258065in}{0.613143in}}%
\pgfpathlineto{\pgfqpoint{4.258621in}{0.606382in}}%
\pgfpathlineto{\pgfqpoint{4.259177in}{0.604422in}}%
\pgfpathlineto{\pgfqpoint{4.260289in}{0.622442in}}%
\pgfpathlineto{\pgfqpoint{4.260845in}{0.620370in}}%
\pgfpathlineto{\pgfqpoint{4.261956in}{0.604869in}}%
\pgfpathlineto{\pgfqpoint{4.262512in}{0.605396in}}%
\pgfpathlineto{\pgfqpoint{4.264180in}{0.619880in}}%
\pgfpathlineto{\pgfqpoint{4.265847in}{0.605630in}}%
\pgfpathlineto{\pgfqpoint{4.267515in}{0.619859in}}%
\pgfpathlineto{\pgfqpoint{4.270850in}{0.602538in}}%
\pgfpathlineto{\pgfqpoint{4.273074in}{0.620250in}}%
\pgfpathlineto{\pgfqpoint{4.274185in}{0.608075in}}%
\pgfpathlineto{\pgfqpoint{4.274741in}{0.617615in}}%
\pgfpathlineto{\pgfqpoint{4.275297in}{0.616040in}}%
\pgfpathlineto{\pgfqpoint{4.277521in}{0.605828in}}%
\pgfpathlineto{\pgfqpoint{4.278076in}{0.615058in}}%
\pgfpathlineto{\pgfqpoint{4.279744in}{0.600613in}}%
\pgfpathlineto{\pgfqpoint{4.280300in}{0.603299in}}%
\pgfpathlineto{\pgfqpoint{4.280856in}{0.601458in}}%
\pgfpathlineto{\pgfqpoint{4.281412in}{0.622942in}}%
\pgfpathlineto{\pgfqpoint{4.281967in}{0.605663in}}%
\pgfpathlineto{\pgfqpoint{4.282523in}{0.612805in}}%
\pgfpathlineto{\pgfqpoint{4.283079in}{0.603038in}}%
\pgfpathlineto{\pgfqpoint{4.283635in}{0.615344in}}%
\pgfpathlineto{\pgfqpoint{4.284191in}{0.613325in}}%
\pgfpathlineto{\pgfqpoint{4.284747in}{0.615267in}}%
\pgfpathlineto{\pgfqpoint{4.285303in}{0.605187in}}%
\pgfpathlineto{\pgfqpoint{4.285858in}{0.611640in}}%
\pgfpathlineto{\pgfqpoint{4.286970in}{0.618429in}}%
\pgfpathlineto{\pgfqpoint{4.288082in}{0.608730in}}%
\pgfpathlineto{\pgfqpoint{4.288638in}{0.621826in}}%
\pgfpathlineto{\pgfqpoint{4.289194in}{0.618647in}}%
\pgfpathlineto{\pgfqpoint{4.289749in}{0.601662in}}%
\pgfpathlineto{\pgfqpoint{4.290305in}{0.612198in}}%
\pgfpathlineto{\pgfqpoint{4.291417in}{0.602717in}}%
\pgfpathlineto{\pgfqpoint{4.291973in}{0.603079in}}%
\pgfpathlineto{\pgfqpoint{4.293085in}{0.620082in}}%
\pgfpathlineto{\pgfqpoint{4.294196in}{0.603507in}}%
\pgfpathlineto{\pgfqpoint{4.294752in}{0.608586in}}%
\pgfpathlineto{\pgfqpoint{4.295864in}{0.605397in}}%
\pgfpathlineto{\pgfqpoint{4.296420in}{0.605546in}}%
\pgfpathlineto{\pgfqpoint{4.297532in}{0.609613in}}%
\pgfpathlineto{\pgfqpoint{4.298087in}{0.605117in}}%
\pgfpathlineto{\pgfqpoint{4.298643in}{0.608441in}}%
\pgfpathlineto{\pgfqpoint{4.299199in}{0.605372in}}%
\pgfpathlineto{\pgfqpoint{4.299755in}{0.609442in}}%
\pgfpathlineto{\pgfqpoint{4.300311in}{0.601286in}}%
\pgfpathlineto{\pgfqpoint{4.300867in}{0.616318in}}%
\pgfpathlineto{\pgfqpoint{4.301423in}{0.603088in}}%
\pgfpathlineto{\pgfqpoint{4.301978in}{0.607837in}}%
\pgfpathlineto{\pgfqpoint{4.302534in}{0.602172in}}%
\pgfpathlineto{\pgfqpoint{4.303090in}{0.608039in}}%
\pgfpathlineto{\pgfqpoint{4.303646in}{0.600350in}}%
\pgfpathlineto{\pgfqpoint{4.304202in}{0.606453in}}%
\pgfpathlineto{\pgfqpoint{4.304758in}{0.600410in}}%
\pgfpathlineto{\pgfqpoint{4.305314in}{0.600919in}}%
\pgfpathlineto{\pgfqpoint{4.306981in}{0.604247in}}%
\pgfpathlineto{\pgfqpoint{4.307537in}{0.601107in}}%
\pgfpathlineto{\pgfqpoint{4.308093in}{0.602846in}}%
\pgfpathlineto{\pgfqpoint{4.311428in}{0.601239in}}%
\pgfpathlineto{\pgfqpoint{4.311984in}{0.601819in}}%
\pgfpathlineto{\pgfqpoint{4.312540in}{0.601161in}}%
\pgfpathlineto{\pgfqpoint{4.314207in}{0.600411in}}%
\pgfpathlineto{\pgfqpoint{4.315319in}{0.601013in}}%
\pgfpathlineto{\pgfqpoint{4.315875in}{0.600018in}}%
\pgfpathlineto{\pgfqpoint{4.316431in}{0.601190in}}%
\pgfpathlineto{\pgfqpoint{4.316987in}{0.601518in}}%
\pgfpathlineto{\pgfqpoint{4.317543in}{0.605848in}}%
\pgfpathlineto{\pgfqpoint{4.318098in}{0.602188in}}%
\pgfpathlineto{\pgfqpoint{4.318654in}{0.605199in}}%
\pgfpathlineto{\pgfqpoint{4.319210in}{0.602627in}}%
\pgfpathlineto{\pgfqpoint{4.319766in}{0.603115in}}%
\pgfpathlineto{\pgfqpoint{4.320322in}{0.602277in}}%
\pgfpathlineto{\pgfqpoint{4.321989in}{0.600212in}}%
\pgfpathlineto{\pgfqpoint{4.322545in}{0.601065in}}%
\pgfpathlineto{\pgfqpoint{4.323101in}{0.601177in}}%
\pgfpathlineto{\pgfqpoint{4.324213in}{0.602491in}}%
\pgfpathlineto{\pgfqpoint{4.326436in}{0.600880in}}%
\pgfpathlineto{\pgfqpoint{4.326992in}{0.604140in}}%
\pgfpathlineto{\pgfqpoint{4.327548in}{0.603584in}}%
\pgfpathlineto{\pgfqpoint{4.328660in}{0.601365in}}%
\pgfpathlineto{\pgfqpoint{4.329216in}{0.603042in}}%
\pgfpathlineto{\pgfqpoint{4.329771in}{0.600399in}}%
\pgfpathlineto{\pgfqpoint{4.331439in}{0.605276in}}%
\pgfpathlineto{\pgfqpoint{4.332551in}{0.603432in}}%
\pgfpathlineto{\pgfqpoint{4.334774in}{0.608939in}}%
\pgfpathlineto{\pgfqpoint{4.335330in}{0.602226in}}%
\pgfpathlineto{\pgfqpoint{4.335886in}{0.603518in}}%
\pgfpathlineto{\pgfqpoint{4.336442in}{0.609307in}}%
\pgfpathlineto{\pgfqpoint{4.336998in}{0.606878in}}%
\pgfpathlineto{\pgfqpoint{4.337554in}{0.607490in}}%
\pgfpathlineto{\pgfqpoint{4.338109in}{0.604412in}}%
\pgfpathlineto{\pgfqpoint{4.339777in}{0.609704in}}%
\pgfpathlineto{\pgfqpoint{4.340333in}{0.603420in}}%
\pgfpathlineto{\pgfqpoint{4.340889in}{0.605777in}}%
\pgfpathlineto{\pgfqpoint{4.342000in}{0.612720in}}%
\pgfpathlineto{\pgfqpoint{4.343112in}{0.602370in}}%
\pgfpathlineto{\pgfqpoint{4.343668in}{0.613364in}}%
\pgfpathlineto{\pgfqpoint{4.344224in}{0.611761in}}%
\pgfpathlineto{\pgfqpoint{4.344780in}{0.609058in}}%
\pgfpathlineto{\pgfqpoint{4.346447in}{0.615225in}}%
\pgfpathlineto{\pgfqpoint{4.348115in}{0.605263in}}%
\pgfpathlineto{\pgfqpoint{4.349227in}{0.606959in}}%
\pgfpathlineto{\pgfqpoint{4.350338in}{0.621029in}}%
\pgfpathlineto{\pgfqpoint{4.351450in}{0.602832in}}%
\pgfpathlineto{\pgfqpoint{4.353118in}{0.613300in}}%
\pgfpathlineto{\pgfqpoint{4.353674in}{0.600537in}}%
\pgfpathlineto{\pgfqpoint{4.354229in}{0.614968in}}%
\pgfpathlineto{\pgfqpoint{4.354785in}{0.612355in}}%
\pgfpathlineto{\pgfqpoint{4.355341in}{0.602696in}}%
\pgfpathlineto{\pgfqpoint{4.355897in}{0.605983in}}%
\pgfpathlineto{\pgfqpoint{4.357009in}{0.619165in}}%
\pgfpathlineto{\pgfqpoint{4.357565in}{0.604239in}}%
\pgfpathlineto{\pgfqpoint{4.358120in}{0.617861in}}%
\pgfpathlineto{\pgfqpoint{4.358676in}{0.610475in}}%
\pgfpathlineto{\pgfqpoint{4.359232in}{0.614931in}}%
\pgfpathlineto{\pgfqpoint{4.359788in}{0.610808in}}%
\pgfpathlineto{\pgfqpoint{4.360344in}{0.623449in}}%
\pgfpathlineto{\pgfqpoint{4.360900in}{0.603529in}}%
\pgfpathlineto{\pgfqpoint{4.361456in}{0.610198in}}%
\pgfpathlineto{\pgfqpoint{4.362567in}{0.604043in}}%
\pgfpathlineto{\pgfqpoint{4.363679in}{0.611822in}}%
\pgfpathlineto{\pgfqpoint{4.364235in}{0.610460in}}%
\pgfpathlineto{\pgfqpoint{4.364791in}{0.613025in}}%
\pgfpathlineto{\pgfqpoint{4.365902in}{0.609645in}}%
\pgfpathlineto{\pgfqpoint{4.366458in}{0.612035in}}%
\pgfpathlineto{\pgfqpoint{4.367570in}{0.607116in}}%
\pgfpathlineto{\pgfqpoint{4.368126in}{0.608862in}}%
\pgfpathlineto{\pgfqpoint{4.369238in}{0.601288in}}%
\pgfpathlineto{\pgfqpoint{4.369794in}{0.616269in}}%
\pgfpathlineto{\pgfqpoint{4.370349in}{0.605351in}}%
\pgfpathlineto{\pgfqpoint{4.370905in}{0.611254in}}%
\pgfpathlineto{\pgfqpoint{4.371461in}{0.606573in}}%
\pgfpathlineto{\pgfqpoint{4.372017in}{0.610752in}}%
\pgfpathlineto{\pgfqpoint{4.373685in}{0.603914in}}%
\pgfpathlineto{\pgfqpoint{4.374240in}{0.610297in}}%
\pgfpathlineto{\pgfqpoint{4.374796in}{0.609887in}}%
\pgfpathlineto{\pgfqpoint{4.375352in}{0.604674in}}%
\pgfpathlineto{\pgfqpoint{4.375908in}{0.607745in}}%
\pgfpathlineto{\pgfqpoint{4.376464in}{0.606170in}}%
\pgfpathlineto{\pgfqpoint{4.377020in}{0.609013in}}%
\pgfpathlineto{\pgfqpoint{4.377576in}{0.602158in}}%
\pgfpathlineto{\pgfqpoint{4.378131in}{0.607283in}}%
\pgfpathlineto{\pgfqpoint{4.378687in}{0.603437in}}%
\pgfpathlineto{\pgfqpoint{4.380355in}{0.614505in}}%
\pgfpathlineto{\pgfqpoint{4.382578in}{0.602930in}}%
\pgfpathlineto{\pgfqpoint{4.384246in}{0.618053in}}%
\pgfpathlineto{\pgfqpoint{4.385358in}{0.601238in}}%
\pgfpathlineto{\pgfqpoint{4.386469in}{0.605375in}}%
\pgfpathlineto{\pgfqpoint{4.388693in}{0.617054in}}%
\pgfpathlineto{\pgfqpoint{4.390360in}{0.601571in}}%
\pgfpathlineto{\pgfqpoint{4.391472in}{0.603573in}}%
\pgfpathlineto{\pgfqpoint{4.393140in}{0.625472in}}%
\pgfpathlineto{\pgfqpoint{4.393696in}{0.605568in}}%
\pgfpathlineto{\pgfqpoint{4.394251in}{0.615406in}}%
\pgfpathlineto{\pgfqpoint{4.395363in}{0.614939in}}%
\pgfpathlineto{\pgfqpoint{4.395919in}{0.617322in}}%
\pgfpathlineto{\pgfqpoint{4.396475in}{0.629864in}}%
\pgfpathlineto{\pgfqpoint{4.397031in}{0.604447in}}%
\pgfpathlineto{\pgfqpoint{4.397587in}{0.611922in}}%
\pgfpathlineto{\pgfqpoint{4.398142in}{0.611794in}}%
\pgfpathlineto{\pgfqpoint{4.399254in}{0.623638in}}%
\pgfpathlineto{\pgfqpoint{4.399810in}{0.621287in}}%
\pgfpathlineto{\pgfqpoint{4.400366in}{0.601236in}}%
\pgfpathlineto{\pgfqpoint{4.400922in}{0.615938in}}%
\pgfpathlineto{\pgfqpoint{4.402033in}{0.621807in}}%
\pgfpathlineto{\pgfqpoint{4.402589in}{0.614697in}}%
\pgfpathlineto{\pgfqpoint{4.403145in}{0.615235in}}%
\pgfpathlineto{\pgfqpoint{4.403701in}{0.630994in}}%
\pgfpathlineto{\pgfqpoint{4.404257in}{0.620463in}}%
\pgfpathlineto{\pgfqpoint{4.405925in}{0.600675in}}%
\pgfpathlineto{\pgfqpoint{4.406480in}{0.610410in}}%
\pgfpathlineto{\pgfqpoint{4.407036in}{0.608689in}}%
\pgfpathlineto{\pgfqpoint{4.408148in}{0.629229in}}%
\pgfpathlineto{\pgfqpoint{4.408704in}{0.611885in}}%
\pgfpathlineto{\pgfqpoint{4.409260in}{0.615020in}}%
\pgfpathlineto{\pgfqpoint{4.409816in}{0.621159in}}%
\pgfpathlineto{\pgfqpoint{4.410927in}{0.602645in}}%
\pgfpathlineto{\pgfqpoint{4.412595in}{0.617540in}}%
\pgfpathlineto{\pgfqpoint{4.413151in}{0.605679in}}%
\pgfpathlineto{\pgfqpoint{4.413707in}{0.623328in}}%
\pgfpathlineto{\pgfqpoint{4.414262in}{0.620981in}}%
\pgfpathlineto{\pgfqpoint{4.415374in}{0.607409in}}%
\pgfpathlineto{\pgfqpoint{4.415930in}{0.626686in}}%
\pgfpathlineto{\pgfqpoint{4.416486in}{0.614852in}}%
\pgfpathlineto{\pgfqpoint{4.417042in}{0.622927in}}%
\pgfpathlineto{\pgfqpoint{4.417598in}{0.619469in}}%
\pgfpathlineto{\pgfqpoint{4.418153in}{0.620967in}}%
\pgfpathlineto{\pgfqpoint{4.418709in}{0.604939in}}%
\pgfpathlineto{\pgfqpoint{4.419265in}{0.617821in}}%
\pgfpathlineto{\pgfqpoint{4.420377in}{0.607408in}}%
\pgfpathlineto{\pgfqpoint{4.422044in}{0.618776in}}%
\pgfpathlineto{\pgfqpoint{4.422600in}{0.604548in}}%
\pgfpathlineto{\pgfqpoint{4.423156in}{0.612550in}}%
\pgfpathlineto{\pgfqpoint{4.423712in}{0.613257in}}%
\pgfpathlineto{\pgfqpoint{4.424268in}{0.615794in}}%
\pgfpathlineto{\pgfqpoint{4.425936in}{0.602126in}}%
\pgfpathlineto{\pgfqpoint{4.427047in}{0.618045in}}%
\pgfpathlineto{\pgfqpoint{4.428715in}{0.600196in}}%
\pgfpathlineto{\pgfqpoint{4.429827in}{0.610410in}}%
\pgfpathlineto{\pgfqpoint{4.430938in}{0.602897in}}%
\pgfpathlineto{\pgfqpoint{4.432606in}{0.611371in}}%
\pgfpathlineto{\pgfqpoint{4.433162in}{0.611841in}}%
\pgfpathlineto{\pgfqpoint{4.435941in}{0.600396in}}%
\pgfpathlineto{\pgfqpoint{4.436497in}{0.616269in}}%
\pgfpathlineto{\pgfqpoint{4.437053in}{0.607040in}}%
\pgfpathlineto{\pgfqpoint{4.437609in}{0.611581in}}%
\pgfpathlineto{\pgfqpoint{4.438164in}{0.603434in}}%
\pgfpathlineto{\pgfqpoint{4.438720in}{0.606402in}}%
\pgfpathlineto{\pgfqpoint{4.440388in}{0.610396in}}%
\pgfpathlineto{\pgfqpoint{4.440944in}{0.616169in}}%
\pgfpathlineto{\pgfqpoint{4.441500in}{0.601733in}}%
\pgfpathlineto{\pgfqpoint{4.442055in}{0.604987in}}%
\pgfpathlineto{\pgfqpoint{4.443167in}{0.620089in}}%
\pgfpathlineto{\pgfqpoint{4.443723in}{0.612760in}}%
\pgfpathlineto{\pgfqpoint{4.444279in}{0.605661in}}%
\pgfpathlineto{\pgfqpoint{4.444835in}{0.613779in}}%
\pgfpathlineto{\pgfqpoint{4.445391in}{0.605748in}}%
\pgfpathlineto{\pgfqpoint{4.446502in}{0.609988in}}%
\pgfpathlineto{\pgfqpoint{4.447058in}{0.616291in}}%
\pgfpathlineto{\pgfqpoint{4.447614in}{0.614968in}}%
\pgfpathlineto{\pgfqpoint{4.449282in}{0.606553in}}%
\pgfpathlineto{\pgfqpoint{4.450393in}{0.610659in}}%
\pgfpathlineto{\pgfqpoint{4.450949in}{0.607324in}}%
\pgfpathlineto{\pgfqpoint{4.451505in}{0.620792in}}%
\pgfpathlineto{\pgfqpoint{4.452061in}{0.606417in}}%
\pgfpathlineto{\pgfqpoint{4.452617in}{0.619351in}}%
\pgfpathlineto{\pgfqpoint{4.453173in}{0.612226in}}%
\pgfpathlineto{\pgfqpoint{4.453729in}{0.616296in}}%
\pgfpathlineto{\pgfqpoint{4.454840in}{0.630052in}}%
\pgfpathlineto{\pgfqpoint{4.455952in}{0.611925in}}%
\pgfpathlineto{\pgfqpoint{4.456508in}{0.627906in}}%
\pgfpathlineto{\pgfqpoint{4.458175in}{0.607152in}}%
\pgfpathlineto{\pgfqpoint{4.458731in}{0.638647in}}%
\pgfpathlineto{\pgfqpoint{4.459287in}{0.613935in}}%
\pgfpathlineto{\pgfqpoint{4.459843in}{0.611847in}}%
\pgfpathlineto{\pgfqpoint{4.460399in}{0.613290in}}%
\pgfpathlineto{\pgfqpoint{4.460955in}{0.627804in}}%
\pgfpathlineto{\pgfqpoint{4.461511in}{0.617232in}}%
\pgfpathlineto{\pgfqpoint{4.462066in}{0.602214in}}%
\pgfpathlineto{\pgfqpoint{4.462622in}{0.619595in}}%
\pgfpathlineto{\pgfqpoint{4.463178in}{0.611538in}}%
\pgfpathlineto{\pgfqpoint{4.464846in}{0.616172in}}%
\pgfpathlineto{\pgfqpoint{4.465402in}{0.630895in}}%
\pgfpathlineto{\pgfqpoint{4.466513in}{0.606571in}}%
\pgfpathlineto{\pgfqpoint{4.467069in}{0.619425in}}%
\pgfpathlineto{\pgfqpoint{4.467625in}{0.615563in}}%
\pgfpathlineto{\pgfqpoint{4.468737in}{0.606938in}}%
\pgfpathlineto{\pgfqpoint{4.469849in}{0.615049in}}%
\pgfpathlineto{\pgfqpoint{4.470404in}{0.608566in}}%
\pgfpathlineto{\pgfqpoint{4.470960in}{0.618077in}}%
\pgfpathlineto{\pgfqpoint{4.472628in}{0.602496in}}%
\pgfpathlineto{\pgfqpoint{4.473184in}{0.620554in}}%
\pgfpathlineto{\pgfqpoint{4.473740in}{0.610440in}}%
\pgfpathlineto{\pgfqpoint{4.474295in}{0.619897in}}%
\pgfpathlineto{\pgfqpoint{4.474851in}{0.604568in}}%
\pgfpathlineto{\pgfqpoint{4.475407in}{0.626657in}}%
\pgfpathlineto{\pgfqpoint{4.475963in}{0.605214in}}%
\pgfpathlineto{\pgfqpoint{4.476519in}{0.617185in}}%
\pgfpathlineto{\pgfqpoint{4.477075in}{0.604414in}}%
\pgfpathlineto{\pgfqpoint{4.477631in}{0.606349in}}%
\pgfpathlineto{\pgfqpoint{4.479298in}{0.612973in}}%
\pgfpathlineto{\pgfqpoint{4.479854in}{0.605143in}}%
\pgfpathlineto{\pgfqpoint{4.480410in}{0.611109in}}%
\pgfpathlineto{\pgfqpoint{4.480966in}{0.612084in}}%
\pgfpathlineto{\pgfqpoint{4.481522in}{0.617035in}}%
\pgfpathlineto{\pgfqpoint{4.483189in}{0.600464in}}%
\pgfpathlineto{\pgfqpoint{4.483745in}{0.614761in}}%
\pgfpathlineto{\pgfqpoint{4.484301in}{0.608836in}}%
\pgfpathlineto{\pgfqpoint{4.485413in}{0.606046in}}%
\pgfpathlineto{\pgfqpoint{4.486524in}{0.611304in}}%
\pgfpathlineto{\pgfqpoint{4.488192in}{0.605797in}}%
\pgfpathlineto{\pgfqpoint{4.488748in}{0.607807in}}%
\pgfpathlineto{\pgfqpoint{4.489304in}{0.603589in}}%
\pgfpathlineto{\pgfqpoint{4.489860in}{0.608686in}}%
\pgfpathlineto{\pgfqpoint{4.490415in}{0.604933in}}%
\pgfpathlineto{\pgfqpoint{4.490971in}{0.607636in}}%
\pgfpathlineto{\pgfqpoint{4.493195in}{0.600945in}}%
\pgfpathlineto{\pgfqpoint{4.494306in}{0.614804in}}%
\pgfpathlineto{\pgfqpoint{4.495418in}{0.602348in}}%
\pgfpathlineto{\pgfqpoint{4.497086in}{0.608970in}}%
\pgfpathlineto{\pgfqpoint{4.497642in}{0.602230in}}%
\pgfpathlineto{\pgfqpoint{4.498197in}{0.604550in}}%
\pgfpathlineto{\pgfqpoint{4.498753in}{0.605172in}}%
\pgfpathlineto{\pgfqpoint{4.500421in}{0.618682in}}%
\pgfpathlineto{\pgfqpoint{4.502089in}{0.601283in}}%
\pgfpathlineto{\pgfqpoint{4.503756in}{0.612086in}}%
\pgfpathlineto{\pgfqpoint{4.504312in}{0.609818in}}%
\pgfpathlineto{\pgfqpoint{4.504868in}{0.612383in}}%
\pgfpathlineto{\pgfqpoint{4.506535in}{0.606295in}}%
\pgfpathlineto{\pgfqpoint{4.508203in}{0.602553in}}%
\pgfpathlineto{\pgfqpoint{4.508759in}{0.627821in}}%
\pgfpathlineto{\pgfqpoint{4.509315in}{0.602236in}}%
\pgfpathlineto{\pgfqpoint{4.509871in}{0.611626in}}%
\pgfpathlineto{\pgfqpoint{4.510426in}{0.604717in}}%
\pgfpathlineto{\pgfqpoint{4.511538in}{0.617614in}}%
\pgfpathlineto{\pgfqpoint{4.513206in}{0.606662in}}%
\pgfpathlineto{\pgfqpoint{4.514873in}{0.620803in}}%
\pgfpathlineto{\pgfqpoint{4.515429in}{0.609681in}}%
\pgfpathlineto{\pgfqpoint{4.516541in}{0.621975in}}%
\pgfpathlineto{\pgfqpoint{4.518208in}{0.604667in}}%
\pgfpathlineto{\pgfqpoint{4.518764in}{0.618931in}}%
\pgfpathlineto{\pgfqpoint{4.519320in}{0.603171in}}%
\pgfpathlineto{\pgfqpoint{4.519876in}{0.615644in}}%
\pgfpathlineto{\pgfqpoint{4.520432in}{0.620873in}}%
\pgfpathlineto{\pgfqpoint{4.522100in}{0.606088in}}%
\pgfpathlineto{\pgfqpoint{4.523211in}{0.614880in}}%
\pgfpathlineto{\pgfqpoint{4.523767in}{0.606158in}}%
\pgfpathlineto{\pgfqpoint{4.524323in}{0.614791in}}%
\pgfpathlineto{\pgfqpoint{4.524879in}{0.617658in}}%
\pgfpathlineto{\pgfqpoint{4.525435in}{0.604703in}}%
\pgfpathlineto{\pgfqpoint{4.525991in}{0.605399in}}%
\pgfpathlineto{\pgfqpoint{4.527102in}{0.610903in}}%
\pgfpathlineto{\pgfqpoint{4.528214in}{0.600665in}}%
\pgfpathlineto{\pgfqpoint{4.529326in}{0.611725in}}%
\pgfpathlineto{\pgfqpoint{4.529882in}{0.602626in}}%
\pgfpathlineto{\pgfqpoint{4.530437in}{0.603854in}}%
\pgfpathlineto{\pgfqpoint{4.532105in}{0.611969in}}%
\pgfpathlineto{\pgfqpoint{4.532661in}{0.617236in}}%
\pgfpathlineto{\pgfqpoint{4.533773in}{0.607422in}}%
\pgfpathlineto{\pgfqpoint{4.534328in}{0.610906in}}%
\pgfpathlineto{\pgfqpoint{4.534884in}{0.611822in}}%
\pgfpathlineto{\pgfqpoint{4.535440in}{0.610098in}}%
\pgfpathlineto{\pgfqpoint{4.535996in}{0.604645in}}%
\pgfpathlineto{\pgfqpoint{4.536552in}{0.606480in}}%
\pgfpathlineto{\pgfqpoint{4.537108in}{0.610023in}}%
\pgfpathlineto{\pgfqpoint{4.537664in}{0.607956in}}%
\pgfpathlineto{\pgfqpoint{4.538220in}{0.603131in}}%
\pgfpathlineto{\pgfqpoint{4.538775in}{0.612087in}}%
\pgfpathlineto{\pgfqpoint{4.539331in}{0.610812in}}%
\pgfpathlineto{\pgfqpoint{4.539887in}{0.603231in}}%
\pgfpathlineto{\pgfqpoint{4.540443in}{0.605150in}}%
\pgfpathlineto{\pgfqpoint{4.541555in}{0.605301in}}%
\pgfpathlineto{\pgfqpoint{4.542111in}{0.608454in}}%
\pgfpathlineto{\pgfqpoint{4.542666in}{0.601091in}}%
\pgfpathlineto{\pgfqpoint{4.543222in}{0.610414in}}%
\pgfpathlineto{\pgfqpoint{4.543778in}{0.603402in}}%
\pgfpathlineto{\pgfqpoint{4.545446in}{0.610559in}}%
\pgfpathlineto{\pgfqpoint{4.546002in}{0.600728in}}%
\pgfpathlineto{\pgfqpoint{4.546557in}{0.608858in}}%
\pgfpathlineto{\pgfqpoint{4.547669in}{0.604078in}}%
\pgfpathlineto{\pgfqpoint{4.548225in}{0.607623in}}%
\pgfpathlineto{\pgfqpoint{4.548781in}{0.604689in}}%
\pgfpathlineto{\pgfqpoint{4.549893in}{0.600170in}}%
\pgfpathlineto{\pgfqpoint{4.551560in}{0.610457in}}%
\pgfpathlineto{\pgfqpoint{4.552116in}{0.601637in}}%
\pgfpathlineto{\pgfqpoint{4.552672in}{0.606337in}}%
\pgfpathlineto{\pgfqpoint{4.553228in}{0.608595in}}%
\pgfpathlineto{\pgfqpoint{4.554895in}{0.601154in}}%
\pgfpathlineto{\pgfqpoint{4.555451in}{0.603403in}}%
\pgfpathlineto{\pgfqpoint{4.556007in}{0.602375in}}%
\pgfpathlineto{\pgfqpoint{4.557675in}{0.612135in}}%
\pgfpathlineto{\pgfqpoint{4.558231in}{0.602448in}}%
\pgfpathlineto{\pgfqpoint{4.558786in}{0.607802in}}%
\pgfpathlineto{\pgfqpoint{4.559342in}{0.609498in}}%
\pgfpathlineto{\pgfqpoint{4.559898in}{0.601410in}}%
\pgfpathlineto{\pgfqpoint{4.560454in}{0.607112in}}%
\pgfpathlineto{\pgfqpoint{4.561010in}{0.607470in}}%
\pgfpathlineto{\pgfqpoint{4.561566in}{0.601152in}}%
\pgfpathlineto{\pgfqpoint{4.562122in}{0.605805in}}%
\pgfpathlineto{\pgfqpoint{4.563233in}{0.611821in}}%
\pgfpathlineto{\pgfqpoint{4.563789in}{0.607327in}}%
\pgfpathlineto{\pgfqpoint{4.564345in}{0.608300in}}%
\pgfpathlineto{\pgfqpoint{4.565457in}{0.602948in}}%
\pgfpathlineto{\pgfqpoint{4.566013in}{0.609743in}}%
\pgfpathlineto{\pgfqpoint{4.566568in}{0.607669in}}%
\pgfpathlineto{\pgfqpoint{4.567124in}{0.607662in}}%
\pgfpathlineto{\pgfqpoint{4.567680in}{0.602758in}}%
\pgfpathlineto{\pgfqpoint{4.568236in}{0.604256in}}%
\pgfpathlineto{\pgfqpoint{4.568792in}{0.618401in}}%
\pgfpathlineto{\pgfqpoint{4.570459in}{0.602862in}}%
\pgfpathlineto{\pgfqpoint{4.571015in}{0.608197in}}%
\pgfpathlineto{\pgfqpoint{4.571571in}{0.605087in}}%
\pgfpathlineto{\pgfqpoint{4.572683in}{0.608778in}}%
\pgfpathlineto{\pgfqpoint{4.573239in}{0.602840in}}%
\pgfpathlineto{\pgfqpoint{4.573795in}{0.624456in}}%
\pgfpathlineto{\pgfqpoint{4.574350in}{0.604034in}}%
\pgfpathlineto{\pgfqpoint{4.574906in}{0.609486in}}%
\pgfpathlineto{\pgfqpoint{4.575462in}{0.605721in}}%
\pgfpathlineto{\pgfqpoint{4.576574in}{0.612035in}}%
\pgfpathlineto{\pgfqpoint{4.577130in}{0.605899in}}%
\pgfpathlineto{\pgfqpoint{4.577686in}{0.613154in}}%
\pgfpathlineto{\pgfqpoint{4.578242in}{0.612433in}}%
\pgfpathlineto{\pgfqpoint{4.579909in}{0.603811in}}%
\pgfpathlineto{\pgfqpoint{4.580465in}{0.614370in}}%
\pgfpathlineto{\pgfqpoint{4.581021in}{0.605388in}}%
\pgfpathlineto{\pgfqpoint{4.581577in}{0.608481in}}%
\pgfpathlineto{\pgfqpoint{4.582133in}{0.607002in}}%
\pgfpathlineto{\pgfqpoint{4.583800in}{0.602823in}}%
\pgfpathlineto{\pgfqpoint{4.584356in}{0.609604in}}%
\pgfpathlineto{\pgfqpoint{4.584912in}{0.607855in}}%
\pgfpathlineto{\pgfqpoint{4.586579in}{0.602475in}}%
\pgfpathlineto{\pgfqpoint{4.587135in}{0.603287in}}%
\pgfpathlineto{\pgfqpoint{4.587691in}{0.601779in}}%
\pgfpathlineto{\pgfqpoint{4.589359in}{0.607703in}}%
\pgfpathlineto{\pgfqpoint{4.589915in}{0.601994in}}%
\pgfpathlineto{\pgfqpoint{4.590470in}{0.611473in}}%
\pgfpathlineto{\pgfqpoint{4.591026in}{0.606713in}}%
\pgfpathlineto{\pgfqpoint{4.591582in}{0.609573in}}%
\pgfpathlineto{\pgfqpoint{4.592138in}{0.604324in}}%
\pgfpathlineto{\pgfqpoint{4.592694in}{0.611773in}}%
\pgfpathlineto{\pgfqpoint{4.593806in}{0.600771in}}%
\pgfpathlineto{\pgfqpoint{4.594361in}{0.602016in}}%
\pgfpathlineto{\pgfqpoint{4.594917in}{0.601744in}}%
\pgfpathlineto{\pgfqpoint{4.596585in}{0.612544in}}%
\pgfpathlineto{\pgfqpoint{4.597141in}{0.602478in}}%
\pgfpathlineto{\pgfqpoint{4.597697in}{0.604603in}}%
\pgfpathlineto{\pgfqpoint{4.599364in}{0.604970in}}%
\pgfpathlineto{\pgfqpoint{4.599920in}{0.602012in}}%
\pgfpathlineto{\pgfqpoint{4.600476in}{0.603880in}}%
\pgfpathlineto{\pgfqpoint{4.602144in}{0.604236in}}%
\pgfpathlineto{\pgfqpoint{4.602699in}{0.606078in}}%
\pgfpathlineto{\pgfqpoint{4.603255in}{0.604657in}}%
\pgfpathlineto{\pgfqpoint{4.603811in}{0.602741in}}%
\pgfpathlineto{\pgfqpoint{4.604367in}{0.603344in}}%
\pgfpathlineto{\pgfqpoint{4.604923in}{0.605622in}}%
\pgfpathlineto{\pgfqpoint{4.605479in}{0.604885in}}%
\pgfpathlineto{\pgfqpoint{4.607146in}{0.601712in}}%
\pgfpathlineto{\pgfqpoint{4.607702in}{0.602991in}}%
\pgfpathlineto{\pgfqpoint{4.608258in}{0.601901in}}%
\pgfpathlineto{\pgfqpoint{4.609370in}{0.601250in}}%
\pgfpathlineto{\pgfqpoint{4.609926in}{0.606881in}}%
\pgfpathlineto{\pgfqpoint{4.610481in}{0.602432in}}%
\pgfpathlineto{\pgfqpoint{4.611037in}{0.604464in}}%
\pgfpathlineto{\pgfqpoint{4.612705in}{0.601259in}}%
\pgfpathlineto{\pgfqpoint{4.613817in}{0.604914in}}%
\pgfpathlineto{\pgfqpoint{4.614373in}{0.604556in}}%
\pgfpathlineto{\pgfqpoint{4.615484in}{0.600207in}}%
\pgfpathlineto{\pgfqpoint{4.616040in}{0.607096in}}%
\pgfpathlineto{\pgfqpoint{4.616596in}{0.606731in}}%
\pgfpathlineto{\pgfqpoint{4.617708in}{0.601811in}}%
\pgfpathlineto{\pgfqpoint{4.618264in}{0.602066in}}%
\pgfpathlineto{\pgfqpoint{4.618819in}{0.602183in}}%
\pgfpathlineto{\pgfqpoint{4.620487in}{0.607590in}}%
\pgfpathlineto{\pgfqpoint{4.621599in}{0.601984in}}%
\pgfpathlineto{\pgfqpoint{4.622710in}{0.606016in}}%
\pgfpathlineto{\pgfqpoint{4.623266in}{0.604961in}}%
\pgfpathlineto{\pgfqpoint{4.623822in}{0.604284in}}%
\pgfpathlineto{\pgfqpoint{4.624378in}{0.605686in}}%
\pgfpathlineto{\pgfqpoint{4.625490in}{0.603601in}}%
\pgfpathlineto{\pgfqpoint{4.626601in}{0.607200in}}%
\pgfpathlineto{\pgfqpoint{4.627713in}{0.603300in}}%
\pgfpathlineto{\pgfqpoint{4.628269in}{0.604593in}}%
\pgfpathlineto{\pgfqpoint{4.629381in}{0.604802in}}%
\pgfpathlineto{\pgfqpoint{4.630492in}{0.602748in}}%
\pgfpathlineto{\pgfqpoint{4.631048in}{0.609393in}}%
\pgfpathlineto{\pgfqpoint{4.631604in}{0.603765in}}%
\pgfpathlineto{\pgfqpoint{4.632160in}{0.605633in}}%
\pgfpathlineto{\pgfqpoint{4.632716in}{0.605187in}}%
\pgfpathlineto{\pgfqpoint{4.633272in}{0.601645in}}%
\pgfpathlineto{\pgfqpoint{4.633828in}{0.606305in}}%
\pgfpathlineto{\pgfqpoint{4.634384in}{0.602936in}}%
\pgfpathlineto{\pgfqpoint{4.634939in}{0.603568in}}%
\pgfpathlineto{\pgfqpoint{4.635495in}{0.612424in}}%
\pgfpathlineto{\pgfqpoint{4.636051in}{0.602170in}}%
\pgfpathlineto{\pgfqpoint{4.636607in}{0.610936in}}%
\pgfpathlineto{\pgfqpoint{4.638275in}{0.601522in}}%
\pgfpathlineto{\pgfqpoint{4.638830in}{0.602606in}}%
\pgfpathlineto{\pgfqpoint{4.639386in}{0.601366in}}%
\pgfpathlineto{\pgfqpoint{4.639942in}{0.609976in}}%
\pgfpathlineto{\pgfqpoint{4.640498in}{0.605219in}}%
\pgfpathlineto{\pgfqpoint{4.641054in}{0.602721in}}%
\pgfpathlineto{\pgfqpoint{4.641610in}{0.604815in}}%
\pgfpathlineto{\pgfqpoint{4.643277in}{0.602771in}}%
\pgfpathlineto{\pgfqpoint{4.643833in}{0.604798in}}%
\pgfpathlineto{\pgfqpoint{4.644389in}{0.601433in}}%
\pgfpathlineto{\pgfqpoint{4.644945in}{0.602223in}}%
\pgfpathlineto{\pgfqpoint{4.646612in}{0.601081in}}%
\pgfpathlineto{\pgfqpoint{4.647724in}{0.604927in}}%
\pgfpathlineto{\pgfqpoint{4.649392in}{0.602757in}}%
\pgfpathlineto{\pgfqpoint{4.649948in}{0.604656in}}%
\pgfpathlineto{\pgfqpoint{4.650503in}{0.603705in}}%
\pgfpathlineto{\pgfqpoint{4.651059in}{0.600135in}}%
\pgfpathlineto{\pgfqpoint{4.651615in}{0.601263in}}%
\pgfpathlineto{\pgfqpoint{4.652171in}{0.601272in}}%
\pgfpathlineto{\pgfqpoint{4.652727in}{0.606079in}}%
\pgfpathlineto{\pgfqpoint{4.653283in}{0.600585in}}%
\pgfpathlineto{\pgfqpoint{4.653839in}{0.604213in}}%
\pgfpathlineto{\pgfqpoint{4.654395in}{0.603160in}}%
\pgfpathlineto{\pgfqpoint{4.654950in}{0.603915in}}%
\pgfpathlineto{\pgfqpoint{4.655506in}{0.603706in}}%
\pgfpathlineto{\pgfqpoint{4.657174in}{0.601458in}}%
\pgfpathlineto{\pgfqpoint{4.657730in}{0.602482in}}%
\pgfpathlineto{\pgfqpoint{4.658286in}{0.600861in}}%
\pgfpathlineto{\pgfqpoint{4.658841in}{0.604238in}}%
\pgfpathlineto{\pgfqpoint{4.659397in}{0.602105in}}%
\pgfpathlineto{\pgfqpoint{4.661065in}{0.603743in}}%
\pgfpathlineto{\pgfqpoint{4.662732in}{0.601654in}}%
\pgfpathlineto{\pgfqpoint{4.664400in}{0.602263in}}%
\pgfpathlineto{\pgfqpoint{4.664956in}{0.601354in}}%
\pgfpathlineto{\pgfqpoint{4.665512in}{0.606389in}}%
\pgfpathlineto{\pgfqpoint{4.666068in}{0.600995in}}%
\pgfpathlineto{\pgfqpoint{4.666623in}{0.602007in}}%
\pgfpathlineto{\pgfqpoint{4.667179in}{0.605303in}}%
\pgfpathlineto{\pgfqpoint{4.667735in}{0.604152in}}%
\pgfpathlineto{\pgfqpoint{4.668291in}{0.602267in}}%
\pgfpathlineto{\pgfqpoint{4.668847in}{0.603118in}}%
\pgfpathlineto{\pgfqpoint{4.669959in}{0.603758in}}%
\pgfpathlineto{\pgfqpoint{4.671070in}{0.601506in}}%
\pgfpathlineto{\pgfqpoint{4.671626in}{0.603967in}}%
\pgfpathlineto{\pgfqpoint{4.672182in}{0.600324in}}%
\pgfpathlineto{\pgfqpoint{4.672738in}{0.600566in}}%
\pgfpathlineto{\pgfqpoint{4.674406in}{0.601709in}}%
\pgfpathlineto{\pgfqpoint{4.674961in}{0.605524in}}%
\pgfpathlineto{\pgfqpoint{4.675517in}{0.603656in}}%
\pgfpathlineto{\pgfqpoint{4.676629in}{0.601427in}}%
\pgfpathlineto{\pgfqpoint{4.678852in}{0.604339in}}%
\pgfpathlineto{\pgfqpoint{4.679408in}{0.602210in}}%
\pgfpathlineto{\pgfqpoint{4.681076in}{0.605322in}}%
\pgfpathlineto{\pgfqpoint{4.682188in}{0.601663in}}%
\pgfpathlineto{\pgfqpoint{4.682743in}{0.601956in}}%
\pgfpathlineto{\pgfqpoint{4.683299in}{0.602963in}}%
\pgfpathlineto{\pgfqpoint{4.684411in}{0.601488in}}%
\pgfpathlineto{\pgfqpoint{4.686079in}{0.606313in}}%
\pgfpathlineto{\pgfqpoint{4.687746in}{0.600031in}}%
\pgfpathlineto{\pgfqpoint{4.689414in}{0.606488in}}%
\pgfpathlineto{\pgfqpoint{4.690526in}{0.600346in}}%
\pgfpathlineto{\pgfqpoint{4.691637in}{0.605001in}}%
\pgfpathlineto{\pgfqpoint{4.692193in}{0.600754in}}%
\pgfpathlineto{\pgfqpoint{4.692749in}{0.603803in}}%
\pgfpathlineto{\pgfqpoint{4.693305in}{0.605693in}}%
\pgfpathlineto{\pgfqpoint{4.694417in}{0.600963in}}%
\pgfpathlineto{\pgfqpoint{4.696084in}{0.606864in}}%
\pgfpathlineto{\pgfqpoint{4.697196in}{0.602162in}}%
\pgfpathlineto{\pgfqpoint{4.697752in}{0.605799in}}%
\pgfpathlineto{\pgfqpoint{4.698308in}{0.604983in}}%
\pgfpathlineto{\pgfqpoint{4.698863in}{0.600124in}}%
\pgfpathlineto{\pgfqpoint{4.699419in}{0.605188in}}%
\pgfpathlineto{\pgfqpoint{4.699975in}{0.601052in}}%
\pgfpathlineto{\pgfqpoint{4.700531in}{0.600525in}}%
\pgfpathlineto{\pgfqpoint{4.702199in}{0.603642in}}%
\pgfpathlineto{\pgfqpoint{4.702754in}{0.602246in}}%
\pgfpathlineto{\pgfqpoint{4.704422in}{0.605135in}}%
\pgfpathlineto{\pgfqpoint{4.706090in}{0.602199in}}%
\pgfpathlineto{\pgfqpoint{4.707757in}{0.606859in}}%
\pgfpathlineto{\pgfqpoint{4.708313in}{0.601132in}}%
\pgfpathlineto{\pgfqpoint{4.708869in}{0.603197in}}%
\pgfpathlineto{\pgfqpoint{4.709425in}{0.602199in}}%
\pgfpathlineto{\pgfqpoint{4.711092in}{0.604385in}}%
\pgfpathlineto{\pgfqpoint{4.712204in}{0.605457in}}%
\pgfpathlineto{\pgfqpoint{4.712760in}{0.603721in}}%
\pgfpathlineto{\pgfqpoint{4.713316in}{0.607127in}}%
\pgfpathlineto{\pgfqpoint{4.714428in}{0.601355in}}%
\pgfpathlineto{\pgfqpoint{4.714983in}{0.602093in}}%
\pgfpathlineto{\pgfqpoint{4.715539in}{0.604716in}}%
\pgfpathlineto{\pgfqpoint{4.716095in}{0.601856in}}%
\pgfpathlineto{\pgfqpoint{4.716651in}{0.603660in}}%
\pgfpathlineto{\pgfqpoint{4.717207in}{0.602091in}}%
\pgfpathlineto{\pgfqpoint{4.717763in}{0.605333in}}%
\pgfpathlineto{\pgfqpoint{4.718319in}{0.601320in}}%
\pgfpathlineto{\pgfqpoint{4.718874in}{0.601631in}}%
\pgfpathlineto{\pgfqpoint{4.719986in}{0.602109in}}%
\pgfpathlineto{\pgfqpoint{4.720542in}{0.600499in}}%
\pgfpathlineto{\pgfqpoint{4.721098in}{0.604083in}}%
\pgfpathlineto{\pgfqpoint{4.721654in}{0.601051in}}%
\pgfpathlineto{\pgfqpoint{4.722765in}{0.605678in}}%
\pgfpathlineto{\pgfqpoint{4.723321in}{0.604358in}}%
\pgfpathlineto{\pgfqpoint{4.725545in}{0.601104in}}%
\pgfpathlineto{\pgfqpoint{4.726657in}{0.605370in}}%
\pgfpathlineto{\pgfqpoint{4.727212in}{0.602358in}}%
\pgfpathlineto{\pgfqpoint{4.727768in}{0.604868in}}%
\pgfpathlineto{\pgfqpoint{4.728880in}{0.601967in}}%
\pgfpathlineto{\pgfqpoint{4.729436in}{0.602367in}}%
\pgfpathlineto{\pgfqpoint{4.729992in}{0.603093in}}%
\pgfpathlineto{\pgfqpoint{4.730548in}{0.602555in}}%
\pgfpathlineto{\pgfqpoint{4.731659in}{0.600732in}}%
\pgfpathlineto{\pgfqpoint{4.732771in}{0.603939in}}%
\pgfpathlineto{\pgfqpoint{4.733327in}{0.600944in}}%
\pgfpathlineto{\pgfqpoint{4.733883in}{0.602807in}}%
\pgfpathlineto{\pgfqpoint{4.734439in}{0.602899in}}%
\pgfpathlineto{\pgfqpoint{4.735550in}{0.601071in}}%
\pgfpathlineto{\pgfqpoint{4.736106in}{0.609136in}}%
\pgfpathlineto{\pgfqpoint{4.736662in}{0.607295in}}%
\pgfpathlineto{\pgfqpoint{4.737774in}{0.602786in}}%
\pgfpathlineto{\pgfqpoint{4.738330in}{0.605015in}}%
\pgfpathlineto{\pgfqpoint{4.738885in}{0.606223in}}%
\pgfpathlineto{\pgfqpoint{4.739441in}{0.603236in}}%
\pgfpathlineto{\pgfqpoint{4.739997in}{0.604772in}}%
\pgfpathlineto{\pgfqpoint{4.740553in}{0.603616in}}%
\pgfpathlineto{\pgfqpoint{4.741109in}{0.606427in}}%
\pgfpathlineto{\pgfqpoint{4.741665in}{0.600487in}}%
\pgfpathlineto{\pgfqpoint{4.742221in}{0.602293in}}%
\pgfpathlineto{\pgfqpoint{4.743888in}{0.608090in}}%
\pgfpathlineto{\pgfqpoint{4.745000in}{0.602631in}}%
\pgfpathlineto{\pgfqpoint{4.745556in}{0.604211in}}%
\pgfpathlineto{\pgfqpoint{4.746112in}{0.605257in}}%
\pgfpathlineto{\pgfqpoint{4.747779in}{0.600814in}}%
\pgfpathlineto{\pgfqpoint{4.748335in}{0.602726in}}%
\pgfpathlineto{\pgfqpoint{4.748891in}{0.610502in}}%
\pgfpathlineto{\pgfqpoint{4.749447in}{0.603406in}}%
\pgfpathlineto{\pgfqpoint{4.750003in}{0.607565in}}%
\pgfpathlineto{\pgfqpoint{4.750559in}{0.606682in}}%
\pgfpathlineto{\pgfqpoint{4.751114in}{0.602694in}}%
\pgfpathlineto{\pgfqpoint{4.751670in}{0.603866in}}%
\pgfpathlineto{\pgfqpoint{4.752782in}{0.602206in}}%
\pgfpathlineto{\pgfqpoint{4.753338in}{0.602945in}}%
\pgfpathlineto{\pgfqpoint{4.755005in}{0.607831in}}%
\pgfpathlineto{\pgfqpoint{4.756117in}{0.602232in}}%
\pgfpathlineto{\pgfqpoint{4.757785in}{0.607388in}}%
\pgfpathlineto{\pgfqpoint{4.758341in}{0.604235in}}%
\pgfpathlineto{\pgfqpoint{4.758896in}{0.605495in}}%
\pgfpathlineto{\pgfqpoint{4.759452in}{0.606248in}}%
\pgfpathlineto{\pgfqpoint{4.760564in}{0.602659in}}%
\pgfpathlineto{\pgfqpoint{4.761120in}{0.603146in}}%
\pgfpathlineto{\pgfqpoint{4.761676in}{0.602151in}}%
\pgfpathlineto{\pgfqpoint{4.762787in}{0.610936in}}%
\pgfpathlineto{\pgfqpoint{4.763343in}{0.605765in}}%
\pgfpathlineto{\pgfqpoint{4.765011in}{0.611664in}}%
\pgfpathlineto{\pgfqpoint{4.765567in}{0.601724in}}%
\pgfpathlineto{\pgfqpoint{4.766123in}{0.604068in}}%
\pgfpathlineto{\pgfqpoint{4.766679in}{0.602787in}}%
\pgfpathlineto{\pgfqpoint{4.767790in}{0.611835in}}%
\pgfpathlineto{\pgfqpoint{4.768346in}{0.603480in}}%
\pgfpathlineto{\pgfqpoint{4.768902in}{0.607036in}}%
\pgfpathlineto{\pgfqpoint{4.770014in}{0.605358in}}%
\pgfpathlineto{\pgfqpoint{4.770570in}{0.610620in}}%
\pgfpathlineto{\pgfqpoint{4.771125in}{0.605462in}}%
\pgfpathlineto{\pgfqpoint{4.772793in}{0.601629in}}%
\pgfpathlineto{\pgfqpoint{4.773905in}{0.603844in}}%
\pgfpathlineto{\pgfqpoint{4.774461in}{0.601411in}}%
\pgfpathlineto{\pgfqpoint{4.776128in}{0.604127in}}%
\pgfpathlineto{\pgfqpoint{4.777240in}{0.605084in}}%
\pgfpathlineto{\pgfqpoint{4.777796in}{0.603077in}}%
\pgfpathlineto{\pgfqpoint{4.778352in}{0.603414in}}%
\pgfpathlineto{\pgfqpoint{4.778907in}{0.604138in}}%
\pgfpathlineto{\pgfqpoint{4.779463in}{0.601990in}}%
\pgfpathlineto{\pgfqpoint{4.780019in}{0.606354in}}%
\pgfpathlineto{\pgfqpoint{4.780575in}{0.605943in}}%
\pgfpathlineto{\pgfqpoint{4.781131in}{0.600740in}}%
\pgfpathlineto{\pgfqpoint{4.781687in}{0.601769in}}%
\pgfpathlineto{\pgfqpoint{4.782243in}{0.606113in}}%
\pgfpathlineto{\pgfqpoint{4.782798in}{0.602581in}}%
\pgfpathlineto{\pgfqpoint{4.783354in}{0.601990in}}%
\pgfpathlineto{\pgfqpoint{4.784466in}{0.607820in}}%
\pgfpathlineto{\pgfqpoint{4.785022in}{0.600194in}}%
\pgfpathlineto{\pgfqpoint{4.785578in}{0.602683in}}%
\pgfpathlineto{\pgfqpoint{4.786134in}{0.601071in}}%
\pgfpathlineto{\pgfqpoint{4.786690in}{0.602780in}}%
\pgfpathlineto{\pgfqpoint{4.788357in}{0.606711in}}%
\pgfpathlineto{\pgfqpoint{4.790025in}{0.603128in}}%
\pgfpathlineto{\pgfqpoint{4.790581in}{0.604946in}}%
\pgfpathlineto{\pgfqpoint{4.791692in}{0.601011in}}%
\pgfpathlineto{\pgfqpoint{4.793360in}{0.604518in}}%
\pgfpathlineto{\pgfqpoint{4.793916in}{0.608363in}}%
\pgfpathlineto{\pgfqpoint{4.794472in}{0.601264in}}%
\pgfpathlineto{\pgfqpoint{4.795027in}{0.602809in}}%
\pgfpathlineto{\pgfqpoint{4.795583in}{0.605748in}}%
\pgfpathlineto{\pgfqpoint{4.796139in}{0.615592in}}%
\pgfpathlineto{\pgfqpoint{4.796695in}{0.602321in}}%
\pgfpathlineto{\pgfqpoint{4.797251in}{0.605445in}}%
\pgfpathlineto{\pgfqpoint{4.797807in}{0.602763in}}%
\pgfpathlineto{\pgfqpoint{4.798363in}{0.605226in}}%
\pgfpathlineto{\pgfqpoint{4.798918in}{0.607190in}}%
\pgfpathlineto{\pgfqpoint{4.800030in}{0.601387in}}%
\pgfpathlineto{\pgfqpoint{4.800586in}{0.602282in}}%
\pgfpathlineto{\pgfqpoint{4.801142in}{0.611601in}}%
\pgfpathlineto{\pgfqpoint{4.802254in}{0.601213in}}%
\pgfpathlineto{\pgfqpoint{4.803921in}{0.607677in}}%
\pgfpathlineto{\pgfqpoint{4.805033in}{0.601832in}}%
\pgfpathlineto{\pgfqpoint{4.806145in}{0.607929in}}%
\pgfpathlineto{\pgfqpoint{4.806701in}{0.605972in}}%
\pgfpathlineto{\pgfqpoint{4.807256in}{0.606862in}}%
\pgfpathlineto{\pgfqpoint{4.807812in}{0.610221in}}%
\pgfpathlineto{\pgfqpoint{4.808368in}{0.600623in}}%
\pgfpathlineto{\pgfqpoint{4.808924in}{0.606346in}}%
\pgfpathlineto{\pgfqpoint{4.810036in}{0.607943in}}%
\pgfpathlineto{\pgfqpoint{4.811703in}{0.602983in}}%
\pgfpathlineto{\pgfqpoint{4.812815in}{0.608129in}}%
\pgfpathlineto{\pgfqpoint{4.813927in}{0.600571in}}%
\pgfpathlineto{\pgfqpoint{4.814483in}{0.605594in}}%
\pgfpathlineto{\pgfqpoint{4.815038in}{0.602384in}}%
\pgfpathlineto{\pgfqpoint{4.815594in}{0.601072in}}%
\pgfpathlineto{\pgfqpoint{4.816150in}{0.601879in}}%
\pgfpathlineto{\pgfqpoint{4.817262in}{0.601691in}}%
\pgfpathlineto{\pgfqpoint{4.817818in}{0.602245in}}%
\pgfpathlineto{\pgfqpoint{4.818374in}{0.601563in}}%
\pgfpathlineto{\pgfqpoint{4.820597in}{0.601320in}}%
\pgfpathlineto{\pgfqpoint{4.823932in}{0.600432in}}%
\pgfpathlineto{\pgfqpoint{4.825600in}{0.602125in}}%
\pgfpathlineto{\pgfqpoint{4.826156in}{0.601518in}}%
\pgfpathlineto{\pgfqpoint{4.828379in}{0.606551in}}%
\pgfpathlineto{\pgfqpoint{4.829491in}{0.602493in}}%
\pgfpathlineto{\pgfqpoint{4.830603in}{0.607452in}}%
\pgfpathlineto{\pgfqpoint{4.831158in}{0.601550in}}%
\pgfpathlineto{\pgfqpoint{4.831714in}{0.601970in}}%
\pgfpathlineto{\pgfqpoint{4.832826in}{0.608095in}}%
\pgfpathlineto{\pgfqpoint{4.833382in}{0.606620in}}%
\pgfpathlineto{\pgfqpoint{4.833938in}{0.603231in}}%
\pgfpathlineto{\pgfqpoint{4.834494in}{0.607291in}}%
\pgfpathlineto{\pgfqpoint{4.835049in}{0.602234in}}%
\pgfpathlineto{\pgfqpoint{4.835605in}{0.605880in}}%
\pgfpathlineto{\pgfqpoint{4.836717in}{0.602333in}}%
\pgfpathlineto{\pgfqpoint{4.838385in}{0.605703in}}%
\pgfpathlineto{\pgfqpoint{4.838940in}{0.601782in}}%
\pgfpathlineto{\pgfqpoint{4.839496in}{0.606232in}}%
\pgfpathlineto{\pgfqpoint{4.840052in}{0.605868in}}%
\pgfpathlineto{\pgfqpoint{4.840608in}{0.602905in}}%
\pgfpathlineto{\pgfqpoint{4.841164in}{0.604669in}}%
\pgfpathlineto{\pgfqpoint{4.841720in}{0.605510in}}%
\pgfpathlineto{\pgfqpoint{4.842276in}{0.603083in}}%
\pgfpathlineto{\pgfqpoint{4.842832in}{0.603906in}}%
\pgfpathlineto{\pgfqpoint{4.843387in}{0.603608in}}%
\pgfpathlineto{\pgfqpoint{4.843943in}{0.601778in}}%
\pgfpathlineto{\pgfqpoint{4.844499in}{0.603496in}}%
\pgfpathlineto{\pgfqpoint{4.845055in}{0.606296in}}%
\pgfpathlineto{\pgfqpoint{4.846167in}{0.601561in}}%
\pgfpathlineto{\pgfqpoint{4.847834in}{0.605119in}}%
\pgfpathlineto{\pgfqpoint{4.849502in}{0.601232in}}%
\pgfpathlineto{\pgfqpoint{4.850614in}{0.603993in}}%
\pgfpathlineto{\pgfqpoint{4.851169in}{0.609839in}}%
\pgfpathlineto{\pgfqpoint{4.851725in}{0.606654in}}%
\pgfpathlineto{\pgfqpoint{4.852281in}{0.604090in}}%
\pgfpathlineto{\pgfqpoint{4.853949in}{0.612250in}}%
\pgfpathlineto{\pgfqpoint{4.854505in}{0.602779in}}%
\pgfpathlineto{\pgfqpoint{4.855060in}{0.606893in}}%
\pgfpathlineto{\pgfqpoint{4.856172in}{0.607776in}}%
\pgfpathlineto{\pgfqpoint{4.856728in}{0.608335in}}%
\pgfpathlineto{\pgfqpoint{4.857840in}{0.604595in}}%
\pgfpathlineto{\pgfqpoint{4.858396in}{0.612890in}}%
\pgfpathlineto{\pgfqpoint{4.858952in}{0.608589in}}%
\pgfpathlineto{\pgfqpoint{4.860619in}{0.606335in}}%
\pgfpathlineto{\pgfqpoint{4.861175in}{0.613769in}}%
\pgfpathlineto{\pgfqpoint{4.862843in}{0.600466in}}%
\pgfpathlineto{\pgfqpoint{4.863398in}{0.614721in}}%
\pgfpathlineto{\pgfqpoint{4.863954in}{0.608255in}}%
\pgfpathlineto{\pgfqpoint{4.864510in}{0.607290in}}%
\pgfpathlineto{\pgfqpoint{4.865066in}{0.602815in}}%
\pgfpathlineto{\pgfqpoint{4.865622in}{0.605689in}}%
\pgfpathlineto{\pgfqpoint{4.867289in}{0.603052in}}%
\pgfpathlineto{\pgfqpoint{4.867845in}{0.602887in}}%
\pgfpathlineto{\pgfqpoint{4.868401in}{0.610169in}}%
\pgfpathlineto{\pgfqpoint{4.868957in}{0.608837in}}%
\pgfpathlineto{\pgfqpoint{4.870625in}{0.602385in}}%
\pgfpathlineto{\pgfqpoint{4.871180in}{0.612525in}}%
\pgfpathlineto{\pgfqpoint{4.871736in}{0.608065in}}%
\pgfpathlineto{\pgfqpoint{4.873404in}{0.600638in}}%
\pgfpathlineto{\pgfqpoint{4.873960in}{0.608499in}}%
\pgfpathlineto{\pgfqpoint{4.874516in}{0.603701in}}%
\pgfpathlineto{\pgfqpoint{4.875627in}{0.602191in}}%
\pgfpathlineto{\pgfqpoint{4.876739in}{0.609679in}}%
\pgfpathlineto{\pgfqpoint{4.877295in}{0.605826in}}%
\pgfpathlineto{\pgfqpoint{4.878407in}{0.615197in}}%
\pgfpathlineto{\pgfqpoint{4.878963in}{0.605139in}}%
\pgfpathlineto{\pgfqpoint{4.879518in}{0.606121in}}%
\pgfpathlineto{\pgfqpoint{4.880074in}{0.612483in}}%
\pgfpathlineto{\pgfqpoint{4.880630in}{0.603190in}}%
\pgfpathlineto{\pgfqpoint{4.881186in}{0.610170in}}%
\pgfpathlineto{\pgfqpoint{4.882854in}{0.610905in}}%
\pgfpathlineto{\pgfqpoint{4.883409in}{0.604395in}}%
\pgfpathlineto{\pgfqpoint{4.883965in}{0.604775in}}%
\pgfpathlineto{\pgfqpoint{4.884521in}{0.612317in}}%
\pgfpathlineto{\pgfqpoint{4.885077in}{0.603149in}}%
\pgfpathlineto{\pgfqpoint{4.885633in}{0.612563in}}%
\pgfpathlineto{\pgfqpoint{4.886189in}{0.606376in}}%
\pgfpathlineto{\pgfqpoint{4.886745in}{0.601869in}}%
\pgfpathlineto{\pgfqpoint{4.887300in}{0.611924in}}%
\pgfpathlineto{\pgfqpoint{4.887856in}{0.609880in}}%
\pgfpathlineto{\pgfqpoint{4.889524in}{0.601170in}}%
\pgfpathlineto{\pgfqpoint{4.890080in}{0.607323in}}%
\pgfpathlineto{\pgfqpoint{4.890636in}{0.604990in}}%
\pgfpathlineto{\pgfqpoint{4.891191in}{0.606062in}}%
\pgfpathlineto{\pgfqpoint{4.891747in}{0.609989in}}%
\pgfpathlineto{\pgfqpoint{4.892859in}{0.601491in}}%
\pgfpathlineto{\pgfqpoint{4.893415in}{0.602862in}}%
\pgfpathlineto{\pgfqpoint{4.895638in}{0.608524in}}%
\pgfpathlineto{\pgfqpoint{4.896194in}{0.603325in}}%
\pgfpathlineto{\pgfqpoint{4.896750in}{0.605818in}}%
\pgfpathlineto{\pgfqpoint{4.897306in}{0.606694in}}%
\pgfpathlineto{\pgfqpoint{4.897862in}{0.601465in}}%
\pgfpathlineto{\pgfqpoint{4.898418in}{0.605537in}}%
\pgfpathlineto{\pgfqpoint{4.898974in}{0.610638in}}%
\pgfpathlineto{\pgfqpoint{4.899529in}{0.606262in}}%
\pgfpathlineto{\pgfqpoint{4.901197in}{0.602749in}}%
\pgfpathlineto{\pgfqpoint{4.902309in}{0.606048in}}%
\pgfpathlineto{\pgfqpoint{4.902865in}{0.600841in}}%
\pgfpathlineto{\pgfqpoint{4.903420in}{0.602801in}}%
\pgfpathlineto{\pgfqpoint{4.903976in}{0.606435in}}%
\pgfpathlineto{\pgfqpoint{4.904532in}{0.605049in}}%
\pgfpathlineto{\pgfqpoint{4.905088in}{0.602392in}}%
\pgfpathlineto{\pgfqpoint{4.906756in}{0.606326in}}%
\pgfpathlineto{\pgfqpoint{4.907867in}{0.601054in}}%
\pgfpathlineto{\pgfqpoint{4.908979in}{0.613037in}}%
\pgfpathlineto{\pgfqpoint{4.910091in}{0.602486in}}%
\pgfpathlineto{\pgfqpoint{4.910647in}{0.603843in}}%
\pgfpathlineto{\pgfqpoint{4.911202in}{0.615249in}}%
\pgfpathlineto{\pgfqpoint{4.911758in}{0.602273in}}%
\pgfpathlineto{\pgfqpoint{4.912314in}{0.611363in}}%
\pgfpathlineto{\pgfqpoint{4.912870in}{0.615350in}}%
\pgfpathlineto{\pgfqpoint{4.913426in}{0.603582in}}%
\pgfpathlineto{\pgfqpoint{4.913982in}{0.609434in}}%
\pgfpathlineto{\pgfqpoint{4.914538in}{0.610717in}}%
\pgfpathlineto{\pgfqpoint{4.915094in}{0.603899in}}%
\pgfpathlineto{\pgfqpoint{4.915649in}{0.619321in}}%
\pgfpathlineto{\pgfqpoint{4.916205in}{0.605567in}}%
\pgfpathlineto{\pgfqpoint{4.916761in}{0.615826in}}%
\pgfpathlineto{\pgfqpoint{4.917317in}{0.600986in}}%
\pgfpathlineto{\pgfqpoint{4.917873in}{0.604329in}}%
\pgfpathlineto{\pgfqpoint{4.918429in}{0.604925in}}%
\pgfpathlineto{\pgfqpoint{4.918985in}{0.610313in}}%
\pgfpathlineto{\pgfqpoint{4.919540in}{0.601165in}}%
\pgfpathlineto{\pgfqpoint{4.920096in}{0.615388in}}%
\pgfpathlineto{\pgfqpoint{4.920652in}{0.614114in}}%
\pgfpathlineto{\pgfqpoint{4.921208in}{0.615166in}}%
\pgfpathlineto{\pgfqpoint{4.922876in}{0.606887in}}%
\pgfpathlineto{\pgfqpoint{4.923431in}{0.606040in}}%
\pgfpathlineto{\pgfqpoint{4.923987in}{0.612184in}}%
\pgfpathlineto{\pgfqpoint{4.924543in}{0.606892in}}%
\pgfpathlineto{\pgfqpoint{4.925655in}{0.605035in}}%
\pgfpathlineto{\pgfqpoint{4.926211in}{0.613545in}}%
\pgfpathlineto{\pgfqpoint{4.926767in}{0.602128in}}%
\pgfpathlineto{\pgfqpoint{4.927322in}{0.609184in}}%
\pgfpathlineto{\pgfqpoint{4.928990in}{0.603601in}}%
\pgfpathlineto{\pgfqpoint{4.929546in}{0.605224in}}%
\pgfpathlineto{\pgfqpoint{4.930102in}{0.613600in}}%
\pgfpathlineto{\pgfqpoint{4.930658in}{0.609963in}}%
\pgfpathlineto{\pgfqpoint{4.931769in}{0.602207in}}%
\pgfpathlineto{\pgfqpoint{4.932325in}{0.602430in}}%
\pgfpathlineto{\pgfqpoint{4.932881in}{0.608202in}}%
\pgfpathlineto{\pgfqpoint{4.933437in}{0.602439in}}%
\pgfpathlineto{\pgfqpoint{4.935105in}{0.607651in}}%
\pgfpathlineto{\pgfqpoint{4.935660in}{0.608463in}}%
\pgfpathlineto{\pgfqpoint{4.936216in}{0.613625in}}%
\pgfpathlineto{\pgfqpoint{4.936772in}{0.606787in}}%
\pgfpathlineto{\pgfqpoint{4.937328in}{0.619989in}}%
\pgfpathlineto{\pgfqpoint{4.937884in}{0.613527in}}%
\pgfpathlineto{\pgfqpoint{4.938996in}{0.602094in}}%
\pgfpathlineto{\pgfqpoint{4.939551in}{0.614769in}}%
\pgfpathlineto{\pgfqpoint{4.940107in}{0.614690in}}%
\pgfpathlineto{\pgfqpoint{4.940663in}{0.606270in}}%
\pgfpathlineto{\pgfqpoint{4.941219in}{0.607587in}}%
\pgfpathlineto{\pgfqpoint{4.941775in}{0.608302in}}%
\pgfpathlineto{\pgfqpoint{4.942331in}{0.616683in}}%
\pgfpathlineto{\pgfqpoint{4.942887in}{0.612332in}}%
\pgfpathlineto{\pgfqpoint{4.943442in}{0.613255in}}%
\pgfpathlineto{\pgfqpoint{4.944554in}{0.606291in}}%
\pgfpathlineto{\pgfqpoint{4.945110in}{0.614728in}}%
\pgfpathlineto{\pgfqpoint{4.945666in}{0.613359in}}%
\pgfpathlineto{\pgfqpoint{4.947333in}{0.603912in}}%
\pgfpathlineto{\pgfqpoint{4.947889in}{0.602272in}}%
\pgfpathlineto{\pgfqpoint{4.948445in}{0.610638in}}%
\pgfpathlineto{\pgfqpoint{4.949001in}{0.608810in}}%
\pgfpathlineto{\pgfqpoint{4.949557in}{0.610105in}}%
\pgfpathlineto{\pgfqpoint{4.950669in}{0.602625in}}%
\pgfpathlineto{\pgfqpoint{4.951224in}{0.603460in}}%
\pgfpathlineto{\pgfqpoint{4.952892in}{0.608255in}}%
\pgfpathlineto{\pgfqpoint{4.954004in}{0.603394in}}%
\pgfpathlineto{\pgfqpoint{4.954560in}{0.612250in}}%
\pgfpathlineto{\pgfqpoint{4.955116in}{0.606814in}}%
\pgfpathlineto{\pgfqpoint{4.955671in}{0.610241in}}%
\pgfpathlineto{\pgfqpoint{4.956227in}{0.606128in}}%
\pgfpathlineto{\pgfqpoint{4.956783in}{0.611440in}}%
\pgfpathlineto{\pgfqpoint{4.957339in}{0.606708in}}%
\pgfpathlineto{\pgfqpoint{4.959007in}{0.601807in}}%
\pgfpathlineto{\pgfqpoint{4.960118in}{0.607926in}}%
\pgfpathlineto{\pgfqpoint{4.960674in}{0.607248in}}%
\pgfpathlineto{\pgfqpoint{4.961230in}{0.606529in}}%
\pgfpathlineto{\pgfqpoint{4.961786in}{0.609930in}}%
\pgfpathlineto{\pgfqpoint{4.962342in}{0.601505in}}%
\pgfpathlineto{\pgfqpoint{4.962898in}{0.608445in}}%
\pgfpathlineto{\pgfqpoint{4.964565in}{0.602609in}}%
\pgfpathlineto{\pgfqpoint{4.965121in}{0.603283in}}%
\pgfpathlineto{\pgfqpoint{4.966233in}{0.609908in}}%
\pgfpathlineto{\pgfqpoint{4.966789in}{0.609449in}}%
\pgfpathlineto{\pgfqpoint{4.967344in}{0.603388in}}%
\pgfpathlineto{\pgfqpoint{4.967900in}{0.607198in}}%
\pgfpathlineto{\pgfqpoint{4.968456in}{0.609254in}}%
\pgfpathlineto{\pgfqpoint{4.969012in}{0.605088in}}%
\pgfpathlineto{\pgfqpoint{4.969568in}{0.605391in}}%
\pgfpathlineto{\pgfqpoint{4.970124in}{0.615274in}}%
\pgfpathlineto{\pgfqpoint{4.970680in}{0.610953in}}%
\pgfpathlineto{\pgfqpoint{4.971235in}{0.611407in}}%
\pgfpathlineto{\pgfqpoint{4.971791in}{0.607678in}}%
\pgfpathlineto{\pgfqpoint{4.973459in}{0.618031in}}%
\pgfpathlineto{\pgfqpoint{4.975127in}{0.602723in}}%
\pgfpathlineto{\pgfqpoint{4.976794in}{0.615206in}}%
\pgfpathlineto{\pgfqpoint{4.977350in}{0.613181in}}%
\pgfpathlineto{\pgfqpoint{4.977906in}{0.607449in}}%
\pgfpathlineto{\pgfqpoint{4.978462in}{0.609691in}}%
\pgfpathlineto{\pgfqpoint{4.979018in}{0.617307in}}%
\pgfpathlineto{\pgfqpoint{4.979573in}{0.604468in}}%
\pgfpathlineto{\pgfqpoint{4.980129in}{0.610559in}}%
\pgfpathlineto{\pgfqpoint{4.981797in}{0.604923in}}%
\pgfpathlineto{\pgfqpoint{4.983464in}{0.610070in}}%
\pgfpathlineto{\pgfqpoint{4.984020in}{0.610555in}}%
\pgfpathlineto{\pgfqpoint{4.984576in}{0.613424in}}%
\pgfpathlineto{\pgfqpoint{4.985132in}{0.602951in}}%
\pgfpathlineto{\pgfqpoint{4.985688in}{0.613558in}}%
\pgfpathlineto{\pgfqpoint{4.986244in}{0.606345in}}%
\pgfpathlineto{\pgfqpoint{4.986800in}{0.607567in}}%
\pgfpathlineto{\pgfqpoint{4.987355in}{0.606318in}}%
\pgfpathlineto{\pgfqpoint{4.988467in}{0.602406in}}%
\pgfpathlineto{\pgfqpoint{4.989023in}{0.604001in}}%
\pgfpathlineto{\pgfqpoint{4.989579in}{0.616009in}}%
\pgfpathlineto{\pgfqpoint{4.990135in}{0.611717in}}%
\pgfpathlineto{\pgfqpoint{4.991247in}{0.605099in}}%
\pgfpathlineto{\pgfqpoint{4.991802in}{0.606705in}}%
\pgfpathlineto{\pgfqpoint{4.992358in}{0.609738in}}%
\pgfpathlineto{\pgfqpoint{4.992914in}{0.605388in}}%
\pgfpathlineto{\pgfqpoint{4.993470in}{0.613435in}}%
\pgfpathlineto{\pgfqpoint{4.994026in}{0.607342in}}%
\pgfpathlineto{\pgfqpoint{4.995693in}{0.609429in}}%
\pgfpathlineto{\pgfqpoint{4.996249in}{0.616032in}}%
\pgfpathlineto{\pgfqpoint{4.996805in}{0.611334in}}%
\pgfpathlineto{\pgfqpoint{4.997361in}{0.612011in}}%
\pgfpathlineto{\pgfqpoint{4.997917in}{0.604449in}}%
\pgfpathlineto{\pgfqpoint{4.998473in}{0.611739in}}%
\pgfpathlineto{\pgfqpoint{4.999029in}{0.612499in}}%
\pgfpathlineto{\pgfqpoint{4.999584in}{0.610566in}}%
\pgfpathlineto{\pgfqpoint{5.000140in}{0.603690in}}%
\pgfpathlineto{\pgfqpoint{5.000696in}{0.614824in}}%
\pgfpathlineto{\pgfqpoint{5.001252in}{0.612427in}}%
\pgfpathlineto{\pgfqpoint{5.001808in}{0.609422in}}%
\pgfpathlineto{\pgfqpoint{5.002364in}{0.609901in}}%
\pgfpathlineto{\pgfqpoint{5.002920in}{0.612926in}}%
\pgfpathlineto{\pgfqpoint{5.003475in}{0.605432in}}%
\pgfpathlineto{\pgfqpoint{5.004031in}{0.607582in}}%
\pgfpathlineto{\pgfqpoint{5.004587in}{0.608673in}}%
\pgfpathlineto{\pgfqpoint{5.006255in}{0.605057in}}%
\pgfpathlineto{\pgfqpoint{5.007366in}{0.604974in}}%
\pgfpathlineto{\pgfqpoint{5.007922in}{0.602960in}}%
\pgfpathlineto{\pgfqpoint{5.008478in}{0.603916in}}%
\pgfpathlineto{\pgfqpoint{5.009034in}{0.606822in}}%
\pgfpathlineto{\pgfqpoint{5.009590in}{0.602618in}}%
\pgfpathlineto{\pgfqpoint{5.010146in}{0.607260in}}%
\pgfpathlineto{\pgfqpoint{5.010702in}{0.602979in}}%
\pgfpathlineto{\pgfqpoint{5.011813in}{0.609622in}}%
\pgfpathlineto{\pgfqpoint{5.012369in}{0.606563in}}%
\pgfpathlineto{\pgfqpoint{5.013481in}{0.602340in}}%
\pgfpathlineto{\pgfqpoint{5.014037in}{0.610889in}}%
\pgfpathlineto{\pgfqpoint{5.014593in}{0.609168in}}%
\pgfpathlineto{\pgfqpoint{5.015149in}{0.608961in}}%
\pgfpathlineto{\pgfqpoint{5.016260in}{0.603355in}}%
\pgfpathlineto{\pgfqpoint{5.016816in}{0.605242in}}%
\pgfpathlineto{\pgfqpoint{5.018484in}{0.600035in}}%
\pgfpathlineto{\pgfqpoint{5.020151in}{0.609373in}}%
\pgfpathlineto{\pgfqpoint{5.020707in}{0.604006in}}%
\pgfpathlineto{\pgfqpoint{5.021263in}{0.610124in}}%
\pgfpathlineto{\pgfqpoint{5.022375in}{0.603052in}}%
\pgfpathlineto{\pgfqpoint{5.022931in}{0.604028in}}%
\pgfpathlineto{\pgfqpoint{5.023486in}{0.611019in}}%
\pgfpathlineto{\pgfqpoint{5.024042in}{0.605861in}}%
\pgfpathlineto{\pgfqpoint{5.024598in}{0.605721in}}%
\pgfpathlineto{\pgfqpoint{5.025154in}{0.604011in}}%
\pgfpathlineto{\pgfqpoint{5.025710in}{0.607526in}}%
\pgfpathlineto{\pgfqpoint{5.026822in}{0.601211in}}%
\pgfpathlineto{\pgfqpoint{5.027933in}{0.611432in}}%
\pgfpathlineto{\pgfqpoint{5.028489in}{0.611379in}}%
\pgfpathlineto{\pgfqpoint{5.030157in}{0.603191in}}%
\pgfpathlineto{\pgfqpoint{5.030713in}{0.611125in}}%
\pgfpathlineto{\pgfqpoint{5.031269in}{0.609809in}}%
\pgfpathlineto{\pgfqpoint{5.032380in}{0.604876in}}%
\pgfpathlineto{\pgfqpoint{5.032936in}{0.607200in}}%
\pgfpathlineto{\pgfqpoint{5.033492in}{0.601485in}}%
\pgfpathlineto{\pgfqpoint{5.034048in}{0.612160in}}%
\pgfpathlineto{\pgfqpoint{5.034604in}{0.608320in}}%
\pgfpathlineto{\pgfqpoint{5.035160in}{0.606516in}}%
\pgfpathlineto{\pgfqpoint{5.036271in}{0.610321in}}%
\pgfpathlineto{\pgfqpoint{5.036827in}{0.608864in}}%
\pgfpathlineto{\pgfqpoint{5.037383in}{0.609485in}}%
\pgfpathlineto{\pgfqpoint{5.037939in}{0.601515in}}%
\pgfpathlineto{\pgfqpoint{5.038495in}{0.612124in}}%
\pgfpathlineto{\pgfqpoint{5.039051in}{0.601323in}}%
\pgfpathlineto{\pgfqpoint{5.039606in}{0.603475in}}%
\pgfpathlineto{\pgfqpoint{5.040162in}{0.604224in}}%
\pgfpathlineto{\pgfqpoint{5.040718in}{0.607171in}}%
\pgfpathlineto{\pgfqpoint{5.041274in}{0.602741in}}%
\pgfpathlineto{\pgfqpoint{5.041830in}{0.607078in}}%
\pgfpathlineto{\pgfqpoint{5.042386in}{0.602970in}}%
\pgfpathlineto{\pgfqpoint{5.042942in}{0.604581in}}%
\pgfpathlineto{\pgfqpoint{5.044053in}{0.608994in}}%
\pgfpathlineto{\pgfqpoint{5.044609in}{0.603122in}}%
\pgfpathlineto{\pgfqpoint{5.045165in}{0.603940in}}%
\pgfpathlineto{\pgfqpoint{5.045721in}{0.603352in}}%
\pgfpathlineto{\pgfqpoint{5.046277in}{0.605801in}}%
\pgfpathlineto{\pgfqpoint{5.046833in}{0.604367in}}%
\pgfpathlineto{\pgfqpoint{5.047389in}{0.601284in}}%
\pgfpathlineto{\pgfqpoint{5.047944in}{0.602414in}}%
\pgfpathlineto{\pgfqpoint{5.048500in}{0.601595in}}%
\pgfpathlineto{\pgfqpoint{5.049056in}{0.608097in}}%
\pgfpathlineto{\pgfqpoint{5.049612in}{0.606915in}}%
\pgfpathlineto{\pgfqpoint{5.050724in}{0.602258in}}%
\pgfpathlineto{\pgfqpoint{5.051835in}{0.607521in}}%
\pgfpathlineto{\pgfqpoint{5.052391in}{0.607072in}}%
\pgfpathlineto{\pgfqpoint{5.053503in}{0.607732in}}%
\pgfpathlineto{\pgfqpoint{5.055171in}{0.604322in}}%
\pgfpathlineto{\pgfqpoint{5.057394in}{0.610097in}}%
\pgfpathlineto{\pgfqpoint{5.057950in}{0.610180in}}%
\pgfpathlineto{\pgfqpoint{5.059062in}{0.602450in}}%
\pgfpathlineto{\pgfqpoint{5.059617in}{0.604950in}}%
\pgfpathlineto{\pgfqpoint{5.060173in}{0.615805in}}%
\pgfpathlineto{\pgfqpoint{5.060729in}{0.610817in}}%
\pgfpathlineto{\pgfqpoint{5.061285in}{0.603617in}}%
\pgfpathlineto{\pgfqpoint{5.061841in}{0.609699in}}%
\pgfpathlineto{\pgfqpoint{5.063508in}{0.602045in}}%
\pgfpathlineto{\pgfqpoint{5.064064in}{0.604516in}}%
\pgfpathlineto{\pgfqpoint{5.064620in}{0.603061in}}%
\pgfpathlineto{\pgfqpoint{5.065176in}{0.602219in}}%
\pgfpathlineto{\pgfqpoint{5.065732in}{0.605698in}}%
\pgfpathlineto{\pgfqpoint{5.066288in}{0.600187in}}%
\pgfpathlineto{\pgfqpoint{5.067400in}{0.605878in}}%
\pgfpathlineto{\pgfqpoint{5.068511in}{0.603631in}}%
\pgfpathlineto{\pgfqpoint{5.069067in}{0.605916in}}%
\pgfpathlineto{\pgfqpoint{5.070735in}{0.601986in}}%
\pgfpathlineto{\pgfqpoint{5.071291in}{0.600762in}}%
\pgfpathlineto{\pgfqpoint{5.071846in}{0.607638in}}%
\pgfpathlineto{\pgfqpoint{5.072402in}{0.602822in}}%
\pgfpathlineto{\pgfqpoint{5.072958in}{0.603167in}}%
\pgfpathlineto{\pgfqpoint{5.073514in}{0.605860in}}%
\pgfpathlineto{\pgfqpoint{5.074070in}{0.602028in}}%
\pgfpathlineto{\pgfqpoint{5.074626in}{0.602832in}}%
\pgfpathlineto{\pgfqpoint{5.075182in}{0.602333in}}%
\pgfpathlineto{\pgfqpoint{5.076849in}{0.604317in}}%
\pgfpathlineto{\pgfqpoint{5.077405in}{0.601586in}}%
\pgfpathlineto{\pgfqpoint{5.077961in}{0.606199in}}%
\pgfpathlineto{\pgfqpoint{5.078517in}{0.603822in}}%
\pgfpathlineto{\pgfqpoint{5.079073in}{0.605559in}}%
\pgfpathlineto{\pgfqpoint{5.079628in}{0.604280in}}%
\pgfpathlineto{\pgfqpoint{5.080740in}{0.601490in}}%
\pgfpathlineto{\pgfqpoint{5.081296in}{0.605467in}}%
\pgfpathlineto{\pgfqpoint{5.081852in}{0.604393in}}%
\pgfpathlineto{\pgfqpoint{5.082964in}{0.601115in}}%
\pgfpathlineto{\pgfqpoint{5.083519in}{0.605185in}}%
\pgfpathlineto{\pgfqpoint{5.084075in}{0.602728in}}%
\pgfpathlineto{\pgfqpoint{5.084631in}{0.603702in}}%
\pgfpathlineto{\pgfqpoint{5.085187in}{0.601583in}}%
\pgfpathlineto{\pgfqpoint{5.085743in}{0.606523in}}%
\pgfpathlineto{\pgfqpoint{5.086299in}{0.604884in}}%
\pgfpathlineto{\pgfqpoint{5.086855in}{0.602598in}}%
\pgfpathlineto{\pgfqpoint{5.087411in}{0.605559in}}%
\pgfpathlineto{\pgfqpoint{5.087966in}{0.604725in}}%
\pgfpathlineto{\pgfqpoint{5.088522in}{0.604954in}}%
\pgfpathlineto{\pgfqpoint{5.089634in}{0.600388in}}%
\pgfpathlineto{\pgfqpoint{5.090190in}{0.602386in}}%
\pgfpathlineto{\pgfqpoint{5.091302in}{0.604624in}}%
\pgfpathlineto{\pgfqpoint{5.091857in}{0.601709in}}%
\pgfpathlineto{\pgfqpoint{5.092413in}{0.602993in}}%
\pgfpathlineto{\pgfqpoint{5.092969in}{0.604295in}}%
\pgfpathlineto{\pgfqpoint{5.094637in}{0.602175in}}%
\pgfpathlineto{\pgfqpoint{5.095748in}{0.601635in}}%
\pgfpathlineto{\pgfqpoint{5.096304in}{0.605356in}}%
\pgfpathlineto{\pgfqpoint{5.096860in}{0.603205in}}%
\pgfpathlineto{\pgfqpoint{5.098528in}{0.603513in}}%
\pgfpathlineto{\pgfqpoint{5.099084in}{0.602548in}}%
\pgfpathlineto{\pgfqpoint{5.099639in}{0.603331in}}%
\pgfpathlineto{\pgfqpoint{5.100195in}{0.603885in}}%
\pgfpathlineto{\pgfqpoint{5.100751in}{0.603228in}}%
\pgfpathlineto{\pgfqpoint{5.102419in}{0.602079in}}%
\pgfpathlineto{\pgfqpoint{5.104086in}{0.602575in}}%
\pgfpathlineto{\pgfqpoint{5.105198in}{0.603611in}}%
\pgfpathlineto{\pgfqpoint{5.106866in}{0.600842in}}%
\pgfpathlineto{\pgfqpoint{5.107422in}{0.602597in}}%
\pgfpathlineto{\pgfqpoint{5.107977in}{0.611266in}}%
\pgfpathlineto{\pgfqpoint{5.108533in}{0.602814in}}%
\pgfpathlineto{\pgfqpoint{5.109089in}{0.607060in}}%
\pgfpathlineto{\pgfqpoint{5.109645in}{0.604059in}}%
\pgfpathlineto{\pgfqpoint{5.114092in}{0.602706in}}%
\pgfpathlineto{\pgfqpoint{5.115759in}{0.604864in}}%
\pgfpathlineto{\pgfqpoint{5.116315in}{0.604861in}}%
\pgfpathlineto{\pgfqpoint{5.116871in}{0.600773in}}%
\pgfpathlineto{\pgfqpoint{5.117427in}{0.604736in}}%
\pgfpathlineto{\pgfqpoint{5.117983in}{0.602433in}}%
\pgfpathlineto{\pgfqpoint{5.119095in}{0.605694in}}%
\pgfpathlineto{\pgfqpoint{5.119650in}{0.600340in}}%
\pgfpathlineto{\pgfqpoint{5.120206in}{0.603879in}}%
\pgfpathlineto{\pgfqpoint{5.122430in}{0.601576in}}%
\pgfpathlineto{\pgfqpoint{5.123542in}{0.604255in}}%
\pgfpathlineto{\pgfqpoint{5.124097in}{0.603935in}}%
\pgfpathlineto{\pgfqpoint{5.125209in}{0.601444in}}%
\pgfpathlineto{\pgfqpoint{5.126877in}{0.603401in}}%
\pgfpathlineto{\pgfqpoint{5.127433in}{0.604047in}}%
\pgfpathlineto{\pgfqpoint{5.127988in}{0.602988in}}%
\pgfpathlineto{\pgfqpoint{5.128544in}{0.603994in}}%
\pgfpathlineto{\pgfqpoint{5.129100in}{0.605932in}}%
\pgfpathlineto{\pgfqpoint{5.129656in}{0.602170in}}%
\pgfpathlineto{\pgfqpoint{5.130212in}{0.605551in}}%
\pgfpathlineto{\pgfqpoint{5.130768in}{0.605632in}}%
\pgfpathlineto{\pgfqpoint{5.131324in}{0.601685in}}%
\pgfpathlineto{\pgfqpoint{5.131879in}{0.603987in}}%
\pgfpathlineto{\pgfqpoint{5.134103in}{0.604326in}}%
\pgfpathlineto{\pgfqpoint{5.134659in}{0.602489in}}%
\pgfpathlineto{\pgfqpoint{5.135215in}{0.607887in}}%
\pgfpathlineto{\pgfqpoint{5.135770in}{0.601067in}}%
\pgfpathlineto{\pgfqpoint{5.136326in}{0.604251in}}%
\pgfpathlineto{\pgfqpoint{5.136882in}{0.605545in}}%
\pgfpathlineto{\pgfqpoint{5.137438in}{0.600689in}}%
\pgfpathlineto{\pgfqpoint{5.137994in}{0.602086in}}%
\pgfpathlineto{\pgfqpoint{5.138550in}{0.606505in}}%
\pgfpathlineto{\pgfqpoint{5.139106in}{0.604554in}}%
\pgfpathlineto{\pgfqpoint{5.140217in}{0.601147in}}%
\pgfpathlineto{\pgfqpoint{5.140773in}{0.603416in}}%
\pgfpathlineto{\pgfqpoint{5.141329in}{0.603091in}}%
\pgfpathlineto{\pgfqpoint{5.141885in}{0.601484in}}%
\pgfpathlineto{\pgfqpoint{5.143553in}{0.612030in}}%
\pgfpathlineto{\pgfqpoint{5.144108in}{0.600849in}}%
\pgfpathlineto{\pgfqpoint{5.144664in}{0.606939in}}%
\pgfpathlineto{\pgfqpoint{5.145220in}{0.606783in}}%
\pgfpathlineto{\pgfqpoint{5.146332in}{0.602101in}}%
\pgfpathlineto{\pgfqpoint{5.146888in}{0.605816in}}%
\pgfpathlineto{\pgfqpoint{5.147444in}{0.603790in}}%
\pgfpathlineto{\pgfqpoint{5.147999in}{0.604873in}}%
\pgfpathlineto{\pgfqpoint{5.148555in}{0.609422in}}%
\pgfpathlineto{\pgfqpoint{5.149111in}{0.603370in}}%
\pgfpathlineto{\pgfqpoint{5.149667in}{0.608460in}}%
\pgfpathlineto{\pgfqpoint{5.151335in}{0.602985in}}%
\pgfpathlineto{\pgfqpoint{5.151890in}{0.602363in}}%
\pgfpathlineto{\pgfqpoint{5.153558in}{0.612504in}}%
\pgfpathlineto{\pgfqpoint{5.155226in}{0.602872in}}%
\pgfpathlineto{\pgfqpoint{5.156337in}{0.606040in}}%
\pgfpathlineto{\pgfqpoint{5.157449in}{0.602055in}}%
\pgfpathlineto{\pgfqpoint{5.158005in}{0.605504in}}%
\pgfpathlineto{\pgfqpoint{5.158561in}{0.602768in}}%
\pgfpathlineto{\pgfqpoint{5.160784in}{0.609189in}}%
\pgfpathlineto{\pgfqpoint{5.161340in}{0.603279in}}%
\pgfpathlineto{\pgfqpoint{5.161896in}{0.607403in}}%
\pgfpathlineto{\pgfqpoint{5.162452in}{0.607147in}}%
\pgfpathlineto{\pgfqpoint{5.163008in}{0.601796in}}%
\pgfpathlineto{\pgfqpoint{5.163564in}{0.605271in}}%
\pgfpathlineto{\pgfqpoint{5.164119in}{0.602176in}}%
\pgfpathlineto{\pgfqpoint{5.164675in}{0.607565in}}%
\pgfpathlineto{\pgfqpoint{5.165231in}{0.601034in}}%
\pgfpathlineto{\pgfqpoint{5.165787in}{0.601283in}}%
\pgfpathlineto{\pgfqpoint{5.166343in}{0.612267in}}%
\pgfpathlineto{\pgfqpoint{5.166899in}{0.602793in}}%
\pgfpathlineto{\pgfqpoint{5.167455in}{0.602051in}}%
\pgfpathlineto{\pgfqpoint{5.168010in}{0.602744in}}%
\pgfpathlineto{\pgfqpoint{5.168566in}{0.610363in}}%
\pgfpathlineto{\pgfqpoint{5.169122in}{0.607142in}}%
\pgfpathlineto{\pgfqpoint{5.169678in}{0.603616in}}%
\pgfpathlineto{\pgfqpoint{5.170234in}{0.604506in}}%
\pgfpathlineto{\pgfqpoint{5.171901in}{0.609713in}}%
\pgfpathlineto{\pgfqpoint{5.172457in}{0.603528in}}%
\pgfpathlineto{\pgfqpoint{5.173013in}{0.605309in}}%
\pgfpathlineto{\pgfqpoint{5.174125in}{0.609242in}}%
\pgfpathlineto{\pgfqpoint{5.174681in}{0.606672in}}%
\pgfpathlineto{\pgfqpoint{5.175237in}{0.617875in}}%
\pgfpathlineto{\pgfqpoint{5.175792in}{0.602626in}}%
\pgfpathlineto{\pgfqpoint{5.176348in}{0.605585in}}%
\pgfpathlineto{\pgfqpoint{5.176904in}{0.606870in}}%
\pgfpathlineto{\pgfqpoint{5.177460in}{0.606008in}}%
\pgfpathlineto{\pgfqpoint{5.178016in}{0.616446in}}%
\pgfpathlineto{\pgfqpoint{5.178572in}{0.606455in}}%
\pgfpathlineto{\pgfqpoint{5.180239in}{0.609445in}}%
\pgfpathlineto{\pgfqpoint{5.180795in}{0.609944in}}%
\pgfpathlineto{\pgfqpoint{5.181907in}{0.602261in}}%
\pgfpathlineto{\pgfqpoint{5.182463in}{0.607883in}}%
\pgfpathlineto{\pgfqpoint{5.183019in}{0.607133in}}%
\pgfpathlineto{\pgfqpoint{5.184130in}{0.606960in}}%
\pgfpathlineto{\pgfqpoint{5.185798in}{0.603605in}}%
\pgfpathlineto{\pgfqpoint{5.187466in}{0.610220in}}%
\pgfpathlineto{\pgfqpoint{5.188021in}{0.601984in}}%
\pgfpathlineto{\pgfqpoint{5.188577in}{0.607218in}}%
\pgfpathlineto{\pgfqpoint{5.189133in}{0.606187in}}%
\pgfpathlineto{\pgfqpoint{5.190801in}{0.600359in}}%
\pgfpathlineto{\pgfqpoint{5.191357in}{0.608821in}}%
\pgfpathlineto{\pgfqpoint{5.191912in}{0.606063in}}%
\pgfpathlineto{\pgfqpoint{5.192468in}{0.605780in}}%
\pgfpathlineto{\pgfqpoint{5.193580in}{0.609134in}}%
\pgfpathlineto{\pgfqpoint{5.195248in}{0.605430in}}%
\pgfpathlineto{\pgfqpoint{5.195803in}{0.610642in}}%
\pgfpathlineto{\pgfqpoint{5.196359in}{0.605910in}}%
\pgfpathlineto{\pgfqpoint{5.196915in}{0.605778in}}%
\pgfpathlineto{\pgfqpoint{5.197471in}{0.601868in}}%
\pgfpathlineto{\pgfqpoint{5.198027in}{0.607183in}}%
\pgfpathlineto{\pgfqpoint{5.198583in}{0.604867in}}%
\pgfpathlineto{\pgfqpoint{5.199139in}{0.604076in}}%
\pgfpathlineto{\pgfqpoint{5.200806in}{0.611384in}}%
\pgfpathlineto{\pgfqpoint{5.201362in}{0.609156in}}%
\pgfpathlineto{\pgfqpoint{5.202474in}{0.603427in}}%
\pgfpathlineto{\pgfqpoint{5.203030in}{0.606225in}}%
\pgfpathlineto{\pgfqpoint{5.204141in}{0.609957in}}%
\pgfpathlineto{\pgfqpoint{5.205253in}{0.603535in}}%
\pgfpathlineto{\pgfqpoint{5.206921in}{0.612922in}}%
\pgfpathlineto{\pgfqpoint{5.208588in}{0.601410in}}%
\pgfpathlineto{\pgfqpoint{5.209144in}{0.606717in}}%
\pgfpathlineto{\pgfqpoint{5.209700in}{0.605706in}}%
\pgfpathlineto{\pgfqpoint{5.210256in}{0.605166in}}%
\pgfpathlineto{\pgfqpoint{5.210812in}{0.607743in}}%
\pgfpathlineto{\pgfqpoint{5.211368in}{0.603205in}}%
\pgfpathlineto{\pgfqpoint{5.211923in}{0.606972in}}%
\pgfpathlineto{\pgfqpoint{5.212479in}{0.608016in}}%
\pgfpathlineto{\pgfqpoint{5.213035in}{0.612730in}}%
\pgfpathlineto{\pgfqpoint{5.213591in}{0.603048in}}%
\pgfpathlineto{\pgfqpoint{5.214147in}{0.607413in}}%
\pgfpathlineto{\pgfqpoint{5.215259in}{0.602534in}}%
\pgfpathlineto{\pgfqpoint{5.216370in}{0.610062in}}%
\pgfpathlineto{\pgfqpoint{5.217482in}{0.602023in}}%
\pgfpathlineto{\pgfqpoint{5.218038in}{0.607134in}}%
\pgfpathlineto{\pgfqpoint{5.218594in}{0.601126in}}%
\pgfpathlineto{\pgfqpoint{5.219150in}{0.606542in}}%
\pgfpathlineto{\pgfqpoint{5.219706in}{0.606622in}}%
\pgfpathlineto{\pgfqpoint{5.220261in}{0.608026in}}%
\pgfpathlineto{\pgfqpoint{5.221929in}{0.602532in}}%
\pgfpathlineto{\pgfqpoint{5.222485in}{0.606494in}}%
\pgfpathlineto{\pgfqpoint{5.223041in}{0.605496in}}%
\pgfpathlineto{\pgfqpoint{5.223597in}{0.604399in}}%
\pgfpathlineto{\pgfqpoint{5.224152in}{0.610181in}}%
\pgfpathlineto{\pgfqpoint{5.224708in}{0.603004in}}%
\pgfpathlineto{\pgfqpoint{5.225264in}{0.604577in}}%
\pgfpathlineto{\pgfqpoint{5.225820in}{0.614418in}}%
\pgfpathlineto{\pgfqpoint{5.226376in}{0.605317in}}%
\pgfpathlineto{\pgfqpoint{5.227488in}{0.600905in}}%
\pgfpathlineto{\pgfqpoint{5.228043in}{0.608583in}}%
\pgfpathlineto{\pgfqpoint{5.228599in}{0.602199in}}%
\pgfpathlineto{\pgfqpoint{5.230267in}{0.607401in}}%
\pgfpathlineto{\pgfqpoint{5.230823in}{0.605880in}}%
\pgfpathlineto{\pgfqpoint{5.231379in}{0.609051in}}%
\pgfpathlineto{\pgfqpoint{5.231934in}{0.602432in}}%
\pgfpathlineto{\pgfqpoint{5.232490in}{0.613608in}}%
\pgfpathlineto{\pgfqpoint{5.233046in}{0.611314in}}%
\pgfpathlineto{\pgfqpoint{5.233602in}{0.600978in}}%
\pgfpathlineto{\pgfqpoint{5.234158in}{0.615980in}}%
\pgfpathlineto{\pgfqpoint{5.234714in}{0.606312in}}%
\pgfpathlineto{\pgfqpoint{5.235270in}{0.606824in}}%
\pgfpathlineto{\pgfqpoint{5.235826in}{0.602529in}}%
\pgfpathlineto{\pgfqpoint{5.236381in}{0.603954in}}%
\pgfpathlineto{\pgfqpoint{5.236937in}{0.608589in}}%
\pgfpathlineto{\pgfqpoint{5.237493in}{0.608289in}}%
\pgfpathlineto{\pgfqpoint{5.238049in}{0.602541in}}%
\pgfpathlineto{\pgfqpoint{5.239161in}{0.617383in}}%
\pgfpathlineto{\pgfqpoint{5.240828in}{0.602566in}}%
\pgfpathlineto{\pgfqpoint{5.241384in}{0.606444in}}%
\pgfpathlineto{\pgfqpoint{5.241940in}{0.603756in}}%
\pgfpathlineto{\pgfqpoint{5.243052in}{0.606774in}}%
\pgfpathlineto{\pgfqpoint{5.243608in}{0.605543in}}%
\pgfpathlineto{\pgfqpoint{5.244163in}{0.606601in}}%
\pgfpathlineto{\pgfqpoint{5.244719in}{0.608644in}}%
\pgfpathlineto{\pgfqpoint{5.245275in}{0.604917in}}%
\pgfpathlineto{\pgfqpoint{5.245831in}{0.607697in}}%
\pgfpathlineto{\pgfqpoint{5.247499in}{0.600194in}}%
\pgfpathlineto{\pgfqpoint{5.248610in}{0.604860in}}%
\pgfpathlineto{\pgfqpoint{5.249722in}{0.601850in}}%
\pgfpathlineto{\pgfqpoint{5.251390in}{0.611479in}}%
\pgfpathlineto{\pgfqpoint{5.252501in}{0.603717in}}%
\pgfpathlineto{\pgfqpoint{5.253057in}{0.610789in}}%
\pgfpathlineto{\pgfqpoint{5.253613in}{0.608346in}}%
\pgfpathlineto{\pgfqpoint{5.254169in}{0.604439in}}%
\pgfpathlineto{\pgfqpoint{5.254725in}{0.605453in}}%
\pgfpathlineto{\pgfqpoint{5.255281in}{0.605377in}}%
\pgfpathlineto{\pgfqpoint{5.255837in}{0.610115in}}%
\pgfpathlineto{\pgfqpoint{5.257504in}{0.602704in}}%
\pgfpathlineto{\pgfqpoint{5.258060in}{0.610547in}}%
\pgfpathlineto{\pgfqpoint{5.258616in}{0.608131in}}%
\pgfpathlineto{\pgfqpoint{5.259172in}{0.604314in}}%
\pgfpathlineto{\pgfqpoint{5.259728in}{0.607909in}}%
\pgfpathlineto{\pgfqpoint{5.260839in}{0.607935in}}%
\pgfpathlineto{\pgfqpoint{5.261395in}{0.602045in}}%
\pgfpathlineto{\pgfqpoint{5.263063in}{0.612033in}}%
\pgfpathlineto{\pgfqpoint{5.264174in}{0.604341in}}%
\pgfpathlineto{\pgfqpoint{5.264730in}{0.611830in}}%
\pgfpathlineto{\pgfqpoint{5.265286in}{0.604574in}}%
\pgfpathlineto{\pgfqpoint{5.266398in}{0.605339in}}%
\pgfpathlineto{\pgfqpoint{5.266954in}{0.604148in}}%
\pgfpathlineto{\pgfqpoint{5.267510in}{0.606896in}}%
\pgfpathlineto{\pgfqpoint{5.268065in}{0.604948in}}%
\pgfpathlineto{\pgfqpoint{5.268621in}{0.604402in}}%
\pgfpathlineto{\pgfqpoint{5.269733in}{0.612180in}}%
\pgfpathlineto{\pgfqpoint{5.270289in}{0.608982in}}%
\pgfpathlineto{\pgfqpoint{5.271401in}{0.605900in}}%
\pgfpathlineto{\pgfqpoint{5.271956in}{0.607104in}}%
\pgfpathlineto{\pgfqpoint{5.272512in}{0.603862in}}%
\pgfpathlineto{\pgfqpoint{5.273068in}{0.605323in}}%
\pgfpathlineto{\pgfqpoint{5.274736in}{0.601899in}}%
\pgfpathlineto{\pgfqpoint{5.275292in}{0.610001in}}%
\pgfpathlineto{\pgfqpoint{5.275848in}{0.604266in}}%
\pgfpathlineto{\pgfqpoint{5.276959in}{0.609719in}}%
\pgfpathlineto{\pgfqpoint{5.278071in}{0.601013in}}%
\pgfpathlineto{\pgfqpoint{5.278627in}{0.603613in}}%
\pgfpathlineto{\pgfqpoint{5.279183in}{0.606529in}}%
\pgfpathlineto{\pgfqpoint{5.279739in}{0.603477in}}%
\pgfpathlineto{\pgfqpoint{5.280294in}{0.608236in}}%
\pgfpathlineto{\pgfqpoint{5.281406in}{0.602194in}}%
\pgfpathlineto{\pgfqpoint{5.281962in}{0.607506in}}%
\pgfpathlineto{\pgfqpoint{5.282518in}{0.604008in}}%
\pgfpathlineto{\pgfqpoint{5.283630in}{0.607633in}}%
\pgfpathlineto{\pgfqpoint{5.284185in}{0.604046in}}%
\pgfpathlineto{\pgfqpoint{5.284741in}{0.604532in}}%
\pgfpathlineto{\pgfqpoint{5.285297in}{0.610756in}}%
\pgfpathlineto{\pgfqpoint{5.285853in}{0.604534in}}%
\pgfpathlineto{\pgfqpoint{5.286965in}{0.611468in}}%
\pgfpathlineto{\pgfqpoint{5.288076in}{0.605118in}}%
\pgfpathlineto{\pgfqpoint{5.289744in}{0.608261in}}%
\pgfpathlineto{\pgfqpoint{5.290300in}{0.607394in}}%
\pgfpathlineto{\pgfqpoint{5.290856in}{0.601255in}}%
\pgfpathlineto{\pgfqpoint{5.291968in}{0.613412in}}%
\pgfpathlineto{\pgfqpoint{5.293635in}{0.605795in}}%
\pgfpathlineto{\pgfqpoint{5.294191in}{0.605223in}}%
\pgfpathlineto{\pgfqpoint{5.294747in}{0.608483in}}%
\pgfpathlineto{\pgfqpoint{5.295303in}{0.603623in}}%
\pgfpathlineto{\pgfqpoint{5.295859in}{0.608816in}}%
\pgfpathlineto{\pgfqpoint{5.296414in}{0.601623in}}%
\pgfpathlineto{\pgfqpoint{5.296970in}{0.606459in}}%
\pgfpathlineto{\pgfqpoint{5.297526in}{0.605525in}}%
\pgfpathlineto{\pgfqpoint{5.299194in}{0.613134in}}%
\pgfpathlineto{\pgfqpoint{5.299750in}{0.605361in}}%
\pgfpathlineto{\pgfqpoint{5.300305in}{0.606261in}}%
\pgfpathlineto{\pgfqpoint{5.300861in}{0.606572in}}%
\pgfpathlineto{\pgfqpoint{5.301417in}{0.602640in}}%
\pgfpathlineto{\pgfqpoint{5.301973in}{0.603816in}}%
\pgfpathlineto{\pgfqpoint{5.302529in}{0.609306in}}%
\pgfpathlineto{\pgfqpoint{5.303085in}{0.601599in}}%
\pgfpathlineto{\pgfqpoint{5.303641in}{0.610332in}}%
\pgfpathlineto{\pgfqpoint{5.304196in}{0.606574in}}%
\pgfpathlineto{\pgfqpoint{5.305308in}{0.601124in}}%
\pgfpathlineto{\pgfqpoint{5.305864in}{0.602905in}}%
\pgfpathlineto{\pgfqpoint{5.306976in}{0.600928in}}%
\pgfpathlineto{\pgfqpoint{5.308087in}{0.611102in}}%
\pgfpathlineto{\pgfqpoint{5.308643in}{0.606982in}}%
\pgfpathlineto{\pgfqpoint{5.309755in}{0.602822in}}%
\pgfpathlineto{\pgfqpoint{5.310311in}{0.610432in}}%
\pgfpathlineto{\pgfqpoint{5.310867in}{0.607371in}}%
\pgfpathlineto{\pgfqpoint{5.311423in}{0.609661in}}%
\pgfpathlineto{\pgfqpoint{5.313646in}{0.603147in}}%
\pgfpathlineto{\pgfqpoint{5.314758in}{0.603539in}}%
\pgfpathlineto{\pgfqpoint{5.315314in}{0.605099in}}%
\pgfpathlineto{\pgfqpoint{5.315870in}{0.612359in}}%
\pgfpathlineto{\pgfqpoint{5.316425in}{0.602310in}}%
\pgfpathlineto{\pgfqpoint{5.316981in}{0.604366in}}%
\pgfpathlineto{\pgfqpoint{5.319761in}{0.612663in}}%
\pgfpathlineto{\pgfqpoint{5.320316in}{0.603216in}}%
\pgfpathlineto{\pgfqpoint{5.320872in}{0.611392in}}%
\pgfpathlineto{\pgfqpoint{5.323096in}{0.602456in}}%
\pgfpathlineto{\pgfqpoint{5.323652in}{0.603096in}}%
\pgfpathlineto{\pgfqpoint{5.324207in}{0.606397in}}%
\pgfpathlineto{\pgfqpoint{5.324763in}{0.601927in}}%
\pgfpathlineto{\pgfqpoint{5.325319in}{0.603605in}}%
\pgfpathlineto{\pgfqpoint{5.325875in}{0.604950in}}%
\pgfpathlineto{\pgfqpoint{5.326431in}{0.602626in}}%
\pgfpathlineto{\pgfqpoint{5.326987in}{0.605759in}}%
\pgfpathlineto{\pgfqpoint{5.327543in}{0.600723in}}%
\pgfpathlineto{\pgfqpoint{5.329210in}{0.611747in}}%
\pgfpathlineto{\pgfqpoint{5.330322in}{0.603305in}}%
\pgfpathlineto{\pgfqpoint{5.330878in}{0.604435in}}%
\pgfpathlineto{\pgfqpoint{5.331434in}{0.608782in}}%
\pgfpathlineto{\pgfqpoint{5.331990in}{0.601369in}}%
\pgfpathlineto{\pgfqpoint{5.332545in}{0.610863in}}%
\pgfpathlineto{\pgfqpoint{5.333101in}{0.605393in}}%
\pgfpathlineto{\pgfqpoint{5.333657in}{0.603437in}}%
\pgfpathlineto{\pgfqpoint{5.334213in}{0.605058in}}%
\pgfpathlineto{\pgfqpoint{5.334769in}{0.609268in}}%
\pgfpathlineto{\pgfqpoint{5.335881in}{0.603711in}}%
\pgfpathlineto{\pgfqpoint{5.336436in}{0.617087in}}%
\pgfpathlineto{\pgfqpoint{5.336992in}{0.606972in}}%
\pgfpathlineto{\pgfqpoint{5.338104in}{0.605558in}}%
\pgfpathlineto{\pgfqpoint{5.338660in}{0.606426in}}%
\pgfpathlineto{\pgfqpoint{5.339216in}{0.603289in}}%
\pgfpathlineto{\pgfqpoint{5.339772in}{0.605628in}}%
\pgfpathlineto{\pgfqpoint{5.340327in}{0.606907in}}%
\pgfpathlineto{\pgfqpoint{5.340883in}{0.603521in}}%
\pgfpathlineto{\pgfqpoint{5.341439in}{0.606923in}}%
\pgfpathlineto{\pgfqpoint{5.342551in}{0.603666in}}%
\pgfpathlineto{\pgfqpoint{5.343107in}{0.608679in}}%
\pgfpathlineto{\pgfqpoint{5.343663in}{0.605349in}}%
\pgfpathlineto{\pgfqpoint{5.344218in}{0.607922in}}%
\pgfpathlineto{\pgfqpoint{5.344774in}{0.605725in}}%
\pgfpathlineto{\pgfqpoint{5.345886in}{0.603571in}}%
\pgfpathlineto{\pgfqpoint{5.347554in}{0.606663in}}%
\pgfpathlineto{\pgfqpoint{5.348109in}{0.600886in}}%
\pgfpathlineto{\pgfqpoint{5.349221in}{0.609730in}}%
\pgfpathlineto{\pgfqpoint{5.349777in}{0.602003in}}%
\pgfpathlineto{\pgfqpoint{5.350333in}{0.609471in}}%
\pgfpathlineto{\pgfqpoint{5.350889in}{0.606109in}}%
\pgfpathlineto{\pgfqpoint{5.351445in}{0.612481in}}%
\pgfpathlineto{\pgfqpoint{5.352001in}{0.604219in}}%
\pgfpathlineto{\pgfqpoint{5.352556in}{0.607843in}}%
\pgfpathlineto{\pgfqpoint{5.353112in}{0.607983in}}%
\pgfpathlineto{\pgfqpoint{5.354780in}{0.602332in}}%
\pgfpathlineto{\pgfqpoint{5.356447in}{0.605424in}}%
\pgfpathlineto{\pgfqpoint{5.357003in}{0.602882in}}%
\pgfpathlineto{\pgfqpoint{5.358671in}{0.609589in}}%
\pgfpathlineto{\pgfqpoint{5.359783in}{0.609562in}}%
\pgfpathlineto{\pgfqpoint{5.360894in}{0.603245in}}%
\pgfpathlineto{\pgfqpoint{5.361450in}{0.607239in}}%
\pgfpathlineto{\pgfqpoint{5.362006in}{0.610369in}}%
\pgfpathlineto{\pgfqpoint{5.362562in}{0.605760in}}%
\pgfpathlineto{\pgfqpoint{5.363118in}{0.607902in}}%
\pgfpathlineto{\pgfqpoint{5.363674in}{0.606328in}}%
\pgfpathlineto{\pgfqpoint{5.364229in}{0.601251in}}%
\pgfpathlineto{\pgfqpoint{5.364785in}{0.602359in}}%
\pgfpathlineto{\pgfqpoint{5.365897in}{0.609982in}}%
\pgfpathlineto{\pgfqpoint{5.366453in}{0.607772in}}%
\pgfpathlineto{\pgfqpoint{5.367565in}{0.601847in}}%
\pgfpathlineto{\pgfqpoint{5.368121in}{0.609345in}}%
\pgfpathlineto{\pgfqpoint{5.368676in}{0.605379in}}%
\pgfpathlineto{\pgfqpoint{5.369232in}{0.606421in}}%
\pgfpathlineto{\pgfqpoint{5.369788in}{0.601800in}}%
\pgfpathlineto{\pgfqpoint{5.370344in}{0.603407in}}%
\pgfpathlineto{\pgfqpoint{5.371456in}{0.602995in}}%
\pgfpathlineto{\pgfqpoint{5.373123in}{0.607198in}}%
\pgfpathlineto{\pgfqpoint{5.373679in}{0.606681in}}%
\pgfpathlineto{\pgfqpoint{5.374791in}{0.604318in}}%
\pgfpathlineto{\pgfqpoint{5.377014in}{0.610220in}}%
\pgfpathlineto{\pgfqpoint{5.377570in}{0.609527in}}%
\pgfpathlineto{\pgfqpoint{5.378126in}{0.609947in}}%
\pgfpathlineto{\pgfqpoint{5.379238in}{0.601817in}}%
\pgfpathlineto{\pgfqpoint{5.379794in}{0.610659in}}%
\pgfpathlineto{\pgfqpoint{5.380349in}{0.606074in}}%
\pgfpathlineto{\pgfqpoint{5.380905in}{0.607233in}}%
\pgfpathlineto{\pgfqpoint{5.382017in}{0.603058in}}%
\pgfpathlineto{\pgfqpoint{5.382573in}{0.604263in}}%
\pgfpathlineto{\pgfqpoint{5.383129in}{0.606827in}}%
\pgfpathlineto{\pgfqpoint{5.383685in}{0.604632in}}%
\pgfpathlineto{\pgfqpoint{5.384240in}{0.600756in}}%
\pgfpathlineto{\pgfqpoint{5.384796in}{0.601215in}}%
\pgfpathlineto{\pgfqpoint{5.385908in}{0.609079in}}%
\pgfpathlineto{\pgfqpoint{5.387020in}{0.601141in}}%
\pgfpathlineto{\pgfqpoint{5.387576in}{0.602592in}}%
\pgfpathlineto{\pgfqpoint{5.388132in}{0.606801in}}%
\pgfpathlineto{\pgfqpoint{5.388687in}{0.602882in}}%
\pgfpathlineto{\pgfqpoint{5.389243in}{0.603071in}}%
\pgfpathlineto{\pgfqpoint{5.390355in}{0.606381in}}%
\pgfpathlineto{\pgfqpoint{5.390911in}{0.601355in}}%
\pgfpathlineto{\pgfqpoint{5.391467in}{0.604727in}}%
\pgfpathlineto{\pgfqpoint{5.392578in}{0.608086in}}%
\pgfpathlineto{\pgfqpoint{5.393134in}{0.604457in}}%
\pgfpathlineto{\pgfqpoint{5.393690in}{0.609918in}}%
\pgfpathlineto{\pgfqpoint{5.394246in}{0.605689in}}%
\pgfpathlineto{\pgfqpoint{5.394802in}{0.601017in}}%
\pgfpathlineto{\pgfqpoint{5.395358in}{0.602737in}}%
\pgfpathlineto{\pgfqpoint{5.396469in}{0.606271in}}%
\pgfpathlineto{\pgfqpoint{5.397581in}{0.601042in}}%
\pgfpathlineto{\pgfqpoint{5.398693in}{0.606813in}}%
\pgfpathlineto{\pgfqpoint{5.400916in}{0.602925in}}%
\pgfpathlineto{\pgfqpoint{5.402584in}{0.606993in}}%
\pgfpathlineto{\pgfqpoint{5.403140in}{0.604992in}}%
\pgfpathlineto{\pgfqpoint{5.403696in}{0.609715in}}%
\pgfpathlineto{\pgfqpoint{5.404251in}{0.604745in}}%
\pgfpathlineto{\pgfqpoint{5.404807in}{0.606933in}}%
\pgfpathlineto{\pgfqpoint{5.405363in}{0.604835in}}%
\pgfpathlineto{\pgfqpoint{5.405919in}{0.605286in}}%
\pgfpathlineto{\pgfqpoint{5.406475in}{0.611003in}}%
\pgfpathlineto{\pgfqpoint{5.407031in}{0.602164in}}%
\pgfpathlineto{\pgfqpoint{5.407587in}{0.608322in}}%
\pgfpathlineto{\pgfqpoint{5.408143in}{0.602551in}}%
\pgfpathlineto{\pgfqpoint{5.408698in}{0.607997in}}%
\pgfpathlineto{\pgfqpoint{5.409254in}{0.605670in}}%
\pgfpathlineto{\pgfqpoint{5.409810in}{0.609977in}}%
\pgfpathlineto{\pgfqpoint{5.410922in}{0.604237in}}%
\pgfpathlineto{\pgfqpoint{5.411478in}{0.606973in}}%
\pgfpathlineto{\pgfqpoint{5.413145in}{0.602108in}}%
\pgfpathlineto{\pgfqpoint{5.414257in}{0.603926in}}%
\pgfpathlineto{\pgfqpoint{5.415369in}{0.602806in}}%
\pgfpathlineto{\pgfqpoint{5.415925in}{0.607100in}}%
\pgfpathlineto{\pgfqpoint{5.416480in}{0.601442in}}%
\pgfpathlineto{\pgfqpoint{5.417036in}{0.605781in}}%
\pgfpathlineto{\pgfqpoint{5.417592in}{0.605806in}}%
\pgfpathlineto{\pgfqpoint{5.418148in}{0.603347in}}%
\pgfpathlineto{\pgfqpoint{5.418704in}{0.608970in}}%
\pgfpathlineto{\pgfqpoint{5.419260in}{0.606519in}}%
\pgfpathlineto{\pgfqpoint{5.419816in}{0.606683in}}%
\pgfpathlineto{\pgfqpoint{5.420371in}{0.602457in}}%
\pgfpathlineto{\pgfqpoint{5.420927in}{0.603024in}}%
\pgfpathlineto{\pgfqpoint{5.422595in}{0.610675in}}%
\pgfpathlineto{\pgfqpoint{5.424818in}{0.603448in}}%
\pgfpathlineto{\pgfqpoint{5.426486in}{0.606919in}}%
\pgfpathlineto{\pgfqpoint{5.427042in}{0.601630in}}%
\pgfpathlineto{\pgfqpoint{5.427598in}{0.603662in}}%
\pgfpathlineto{\pgfqpoint{5.428154in}{0.605618in}}%
\pgfpathlineto{\pgfqpoint{5.429821in}{0.600561in}}%
\pgfpathlineto{\pgfqpoint{5.430933in}{0.607656in}}%
\pgfpathlineto{\pgfqpoint{5.431489in}{0.604823in}}%
\pgfpathlineto{\pgfqpoint{5.432045in}{0.602081in}}%
\pgfpathlineto{\pgfqpoint{5.433712in}{0.608490in}}%
\pgfpathlineto{\pgfqpoint{5.434268in}{0.605924in}}%
\pgfpathlineto{\pgfqpoint{5.434824in}{0.609300in}}%
\pgfpathlineto{\pgfqpoint{5.436491in}{0.601236in}}%
\pgfpathlineto{\pgfqpoint{5.437603in}{0.605358in}}%
\pgfpathlineto{\pgfqpoint{5.438159in}{0.602942in}}%
\pgfpathlineto{\pgfqpoint{5.439271in}{0.605778in}}%
\pgfpathlineto{\pgfqpoint{5.439827in}{0.600899in}}%
\pgfpathlineto{\pgfqpoint{5.440382in}{0.602875in}}%
\pgfpathlineto{\pgfqpoint{5.440938in}{0.605832in}}%
\pgfpathlineto{\pgfqpoint{5.441494in}{0.603205in}}%
\pgfpathlineto{\pgfqpoint{5.442050in}{0.600644in}}%
\pgfpathlineto{\pgfqpoint{5.442606in}{0.606043in}}%
\pgfpathlineto{\pgfqpoint{5.443162in}{0.605537in}}%
\pgfpathlineto{\pgfqpoint{5.444829in}{0.601511in}}%
\pgfpathlineto{\pgfqpoint{5.445385in}{0.611302in}}%
\pgfpathlineto{\pgfqpoint{5.445941in}{0.606437in}}%
\pgfpathlineto{\pgfqpoint{5.447053in}{0.601575in}}%
\pgfpathlineto{\pgfqpoint{5.447609in}{0.605701in}}%
\pgfpathlineto{\pgfqpoint{5.448165in}{0.603710in}}%
\pgfpathlineto{\pgfqpoint{5.449832in}{0.605431in}}%
\pgfpathlineto{\pgfqpoint{5.450388in}{0.600487in}}%
\pgfpathlineto{\pgfqpoint{5.450944in}{0.604556in}}%
\pgfpathlineto{\pgfqpoint{5.452056in}{0.603048in}}%
\pgfpathlineto{\pgfqpoint{5.452611in}{0.607750in}}%
\pgfpathlineto{\pgfqpoint{5.453167in}{0.606351in}}%
\pgfpathlineto{\pgfqpoint{5.454835in}{0.603232in}}%
\pgfpathlineto{\pgfqpoint{5.455391in}{0.602242in}}%
\pgfpathlineto{\pgfqpoint{5.455947in}{0.606847in}}%
\pgfpathlineto{\pgfqpoint{5.456502in}{0.602663in}}%
\pgfpathlineto{\pgfqpoint{5.459282in}{0.607110in}}%
\pgfpathlineto{\pgfqpoint{5.460393in}{0.602905in}}%
\pgfpathlineto{\pgfqpoint{5.460949in}{0.607607in}}%
\pgfpathlineto{\pgfqpoint{5.462061in}{0.601742in}}%
\pgfpathlineto{\pgfqpoint{5.462617in}{0.604824in}}%
\pgfpathlineto{\pgfqpoint{5.463173in}{0.600866in}}%
\pgfpathlineto{\pgfqpoint{5.463729in}{0.609563in}}%
\pgfpathlineto{\pgfqpoint{5.464285in}{0.606332in}}%
\pgfpathlineto{\pgfqpoint{5.464840in}{0.605457in}}%
\pgfpathlineto{\pgfqpoint{5.465396in}{0.602247in}}%
\pgfpathlineto{\pgfqpoint{5.466508in}{0.607413in}}%
\pgfpathlineto{\pgfqpoint{5.468176in}{0.601072in}}%
\pgfpathlineto{\pgfqpoint{5.468731in}{0.605304in}}%
\pgfpathlineto{\pgfqpoint{5.469287in}{0.600237in}}%
\pgfpathlineto{\pgfqpoint{5.470955in}{0.606893in}}%
\pgfpathlineto{\pgfqpoint{5.472622in}{0.602240in}}%
\pgfpathlineto{\pgfqpoint{5.473178in}{0.604683in}}%
\pgfpathlineto{\pgfqpoint{5.473734in}{0.601992in}}%
\pgfpathlineto{\pgfqpoint{5.474290in}{0.604708in}}%
\pgfpathlineto{\pgfqpoint{5.474846in}{0.608518in}}%
\pgfpathlineto{\pgfqpoint{5.475402in}{0.602524in}}%
\pgfpathlineto{\pgfqpoint{5.475958in}{0.604302in}}%
\pgfpathlineto{\pgfqpoint{5.476513in}{0.604436in}}%
\pgfpathlineto{\pgfqpoint{5.477069in}{0.607896in}}%
\pgfpathlineto{\pgfqpoint{5.477625in}{0.605373in}}%
\pgfpathlineto{\pgfqpoint{5.479293in}{0.601760in}}%
\pgfpathlineto{\pgfqpoint{5.480960in}{0.611346in}}%
\pgfpathlineto{\pgfqpoint{5.482072in}{0.602102in}}%
\pgfpathlineto{\pgfqpoint{5.482628in}{0.602901in}}%
\pgfpathlineto{\pgfqpoint{5.483184in}{0.604764in}}%
\pgfpathlineto{\pgfqpoint{5.483740in}{0.600745in}}%
\pgfpathlineto{\pgfqpoint{5.484296in}{0.603476in}}%
\pgfpathlineto{\pgfqpoint{5.484851in}{0.601655in}}%
\pgfpathlineto{\pgfqpoint{5.485963in}{0.604862in}}%
\pgfpathlineto{\pgfqpoint{5.486519in}{0.600947in}}%
\pgfpathlineto{\pgfqpoint{5.487075in}{0.603413in}}%
\pgfpathlineto{\pgfqpoint{5.487631in}{0.605675in}}%
\pgfpathlineto{\pgfqpoint{5.488187in}{0.603371in}}%
\pgfpathlineto{\pgfqpoint{5.489854in}{0.602101in}}%
\pgfpathlineto{\pgfqpoint{5.491522in}{0.606522in}}%
\pgfpathlineto{\pgfqpoint{5.492078in}{0.608894in}}%
\pgfpathlineto{\pgfqpoint{5.493745in}{0.601055in}}%
\pgfpathlineto{\pgfqpoint{5.494301in}{0.602227in}}%
\pgfpathlineto{\pgfqpoint{5.494857in}{0.604318in}}%
\pgfpathlineto{\pgfqpoint{5.495413in}{0.601917in}}%
\pgfpathlineto{\pgfqpoint{5.495969in}{0.607287in}}%
\pgfpathlineto{\pgfqpoint{5.496524in}{0.603222in}}%
\pgfpathlineto{\pgfqpoint{5.497080in}{0.603652in}}%
\pgfpathlineto{\pgfqpoint{5.497636in}{0.601398in}}%
\pgfpathlineto{\pgfqpoint{5.498192in}{0.602558in}}%
\pgfpathlineto{\pgfqpoint{5.498748in}{0.603269in}}%
\pgfpathlineto{\pgfqpoint{5.499304in}{0.600973in}}%
\pgfpathlineto{\pgfqpoint{5.499860in}{0.605389in}}%
\pgfpathlineto{\pgfqpoint{5.500416in}{0.602098in}}%
\pgfpathlineto{\pgfqpoint{5.502083in}{0.606599in}}%
\pgfpathlineto{\pgfqpoint{5.502639in}{0.605775in}}%
\pgfpathlineto{\pgfqpoint{5.503751in}{0.600452in}}%
\pgfpathlineto{\pgfqpoint{5.504862in}{0.602105in}}%
\pgfpathlineto{\pgfqpoint{5.505418in}{0.605347in}}%
\pgfpathlineto{\pgfqpoint{5.505974in}{0.604102in}}%
\pgfpathlineto{\pgfqpoint{5.506530in}{0.601231in}}%
\pgfpathlineto{\pgfqpoint{5.507086in}{0.602214in}}%
\pgfpathlineto{\pgfqpoint{5.507642in}{0.602222in}}%
\pgfpathlineto{\pgfqpoint{5.508198in}{0.600357in}}%
\pgfpathlineto{\pgfqpoint{5.509309in}{0.606079in}}%
\pgfpathlineto{\pgfqpoint{5.512089in}{0.601596in}}%
\pgfpathlineto{\pgfqpoint{5.512644in}{0.601756in}}%
\pgfpathlineto{\pgfqpoint{5.513200in}{0.606425in}}%
\pgfpathlineto{\pgfqpoint{5.513756in}{0.602775in}}%
\pgfpathlineto{\pgfqpoint{5.514312in}{0.600767in}}%
\pgfpathlineto{\pgfqpoint{5.515980in}{0.605320in}}%
\pgfpathlineto{\pgfqpoint{5.517091in}{0.601902in}}%
\pgfpathlineto{\pgfqpoint{5.518759in}{0.605180in}}%
\pgfpathlineto{\pgfqpoint{5.519315in}{0.603988in}}%
\pgfpathlineto{\pgfqpoint{5.519871in}{0.607804in}}%
\pgfpathlineto{\pgfqpoint{5.520427in}{0.604420in}}%
\pgfpathlineto{\pgfqpoint{5.520982in}{0.601026in}}%
\pgfpathlineto{\pgfqpoint{5.521538in}{0.606768in}}%
\pgfpathlineto{\pgfqpoint{5.522094in}{0.603528in}}%
\pgfpathlineto{\pgfqpoint{5.522650in}{0.605994in}}%
\pgfpathlineto{\pgfqpoint{5.523206in}{0.605681in}}%
\pgfpathlineto{\pgfqpoint{5.523762in}{0.605846in}}%
\pgfpathlineto{\pgfqpoint{5.524318in}{0.608452in}}%
\pgfpathlineto{\pgfqpoint{5.524873in}{0.604264in}}%
\pgfpathlineto{\pgfqpoint{5.525429in}{0.604594in}}%
\pgfpathlineto{\pgfqpoint{5.525985in}{0.604874in}}%
\pgfpathlineto{\pgfqpoint{5.527653in}{0.602493in}}%
\pgfpathlineto{\pgfqpoint{5.528209in}{0.603988in}}%
\pgfpathlineto{\pgfqpoint{5.529320in}{0.601626in}}%
\pgfpathlineto{\pgfqpoint{5.529876in}{0.604690in}}%
\pgfpathlineto{\pgfqpoint{5.530432in}{0.600390in}}%
\pgfpathlineto{\pgfqpoint{5.530988in}{0.602374in}}%
\pgfpathlineto{\pgfqpoint{5.535435in}{0.603452in}}%
\pgfpathlineto{\pgfqpoint{5.536546in}{0.602216in}}%
\pgfpathlineto{\pgfqpoint{5.538214in}{0.605887in}}%
\pgfpathlineto{\pgfqpoint{5.538770in}{0.604680in}}%
\pgfpathlineto{\pgfqpoint{5.539882in}{0.600543in}}%
\pgfpathlineto{\pgfqpoint{5.541549in}{0.607665in}}%
\pgfpathlineto{\pgfqpoint{5.542105in}{0.602121in}}%
\pgfpathlineto{\pgfqpoint{5.542661in}{0.603258in}}%
\pgfpathlineto{\pgfqpoint{5.543217in}{0.603418in}}%
\pgfpathlineto{\pgfqpoint{5.543773in}{0.600730in}}%
\pgfpathlineto{\pgfqpoint{5.544329in}{0.603203in}}%
\pgfpathlineto{\pgfqpoint{5.545440in}{0.604195in}}%
\pgfpathlineto{\pgfqpoint{5.546552in}{0.602265in}}%
\pgfpathlineto{\pgfqpoint{5.547108in}{0.603017in}}%
\pgfpathlineto{\pgfqpoint{5.547664in}{0.604020in}}%
\pgfpathlineto{\pgfqpoint{5.548220in}{0.608112in}}%
\pgfpathlineto{\pgfqpoint{5.548775in}{0.605539in}}%
\pgfpathlineto{\pgfqpoint{5.549331in}{0.603817in}}%
\pgfpathlineto{\pgfqpoint{5.549887in}{0.607455in}}%
\pgfpathlineto{\pgfqpoint{5.551555in}{0.601274in}}%
\pgfpathlineto{\pgfqpoint{5.552111in}{0.603011in}}%
\pgfpathlineto{\pgfqpoint{5.552666in}{0.601911in}}%
\pgfpathlineto{\pgfqpoint{5.553222in}{0.601121in}}%
\pgfpathlineto{\pgfqpoint{5.553778in}{0.604969in}}%
\pgfpathlineto{\pgfqpoint{5.554334in}{0.600500in}}%
\pgfpathlineto{\pgfqpoint{5.554890in}{0.600952in}}%
\pgfpathlineto{\pgfqpoint{5.555446in}{0.604064in}}%
\pgfpathlineto{\pgfqpoint{5.556002in}{0.603334in}}%
\pgfpathlineto{\pgfqpoint{5.557113in}{0.601692in}}%
\pgfpathlineto{\pgfqpoint{5.557669in}{0.603913in}}%
\pgfpathlineto{\pgfqpoint{5.559337in}{0.600789in}}%
\pgfpathlineto{\pgfqpoint{5.560449in}{0.604819in}}%
\pgfpathlineto{\pgfqpoint{5.561004in}{0.601589in}}%
\pgfpathlineto{\pgfqpoint{5.561560in}{0.602129in}}%
\pgfpathlineto{\pgfqpoint{5.563228in}{0.602245in}}%
\pgfpathlineto{\pgfqpoint{5.564895in}{0.603522in}}%
\pgfpathlineto{\pgfqpoint{5.565451in}{0.600546in}}%
\pgfpathlineto{\pgfqpoint{5.566007in}{0.607024in}}%
\pgfpathlineto{\pgfqpoint{5.566563in}{0.605448in}}%
\pgfpathlineto{\pgfqpoint{5.567675in}{0.601360in}}%
\pgfpathlineto{\pgfqpoint{5.569342in}{0.604452in}}%
\pgfpathlineto{\pgfqpoint{5.570454in}{0.600357in}}%
\pgfpathlineto{\pgfqpoint{5.572677in}{0.608506in}}%
\pgfpathlineto{\pgfqpoint{5.573789in}{0.600974in}}%
\pgfpathlineto{\pgfqpoint{5.574901in}{0.603661in}}%
\pgfpathlineto{\pgfqpoint{5.575457in}{0.601960in}}%
\pgfpathlineto{\pgfqpoint{5.576013in}{0.604067in}}%
\pgfpathlineto{\pgfqpoint{5.577680in}{0.600834in}}%
\pgfpathlineto{\pgfqpoint{5.578236in}{0.601040in}}%
\pgfpathlineto{\pgfqpoint{5.578792in}{0.603959in}}%
\pgfpathlineto{\pgfqpoint{5.579348in}{0.601055in}}%
\pgfpathlineto{\pgfqpoint{5.579904in}{0.600070in}}%
\pgfpathlineto{\pgfqpoint{5.581015in}{0.605527in}}%
\pgfpathlineto{\pgfqpoint{5.581571in}{0.605333in}}%
\pgfpathlineto{\pgfqpoint{5.582127in}{0.604533in}}%
\pgfpathlineto{\pgfqpoint{5.582683in}{0.601676in}}%
\pgfpathlineto{\pgfqpoint{5.583239in}{0.603664in}}%
\pgfpathlineto{\pgfqpoint{5.584351in}{0.600863in}}%
\pgfpathlineto{\pgfqpoint{5.585462in}{0.603092in}}%
\pgfpathlineto{\pgfqpoint{5.586018in}{0.601831in}}%
\pgfpathlineto{\pgfqpoint{5.587130in}{0.604484in}}%
\pgfpathlineto{\pgfqpoint{5.587686in}{0.601090in}}%
\pgfpathlineto{\pgfqpoint{5.588242in}{0.601949in}}%
\pgfpathlineto{\pgfqpoint{5.589909in}{0.603048in}}%
\pgfpathlineto{\pgfqpoint{5.590465in}{0.601115in}}%
\pgfpathlineto{\pgfqpoint{5.591021in}{0.603727in}}%
\pgfpathlineto{\pgfqpoint{5.591577in}{0.603013in}}%
\pgfpathlineto{\pgfqpoint{5.592133in}{0.600647in}}%
\pgfpathlineto{\pgfqpoint{5.592688in}{0.601821in}}%
\pgfpathlineto{\pgfqpoint{5.593244in}{0.604795in}}%
\pgfpathlineto{\pgfqpoint{5.593800in}{0.602867in}}%
\pgfpathlineto{\pgfqpoint{5.594356in}{0.602368in}}%
\pgfpathlineto{\pgfqpoint{5.596024in}{0.606579in}}%
\pgfpathlineto{\pgfqpoint{5.597691in}{0.602233in}}%
\pgfpathlineto{\pgfqpoint{5.598803in}{0.600538in}}%
\pgfpathlineto{\pgfqpoint{5.600471in}{0.604773in}}%
\pgfpathlineto{\pgfqpoint{5.601026in}{0.605058in}}%
\pgfpathlineto{\pgfqpoint{5.601582in}{0.601997in}}%
\pgfpathlineto{\pgfqpoint{5.602138in}{0.604730in}}%
\pgfpathlineto{\pgfqpoint{5.602694in}{0.605757in}}%
\pgfpathlineto{\pgfqpoint{5.603806in}{0.601642in}}%
\pgfpathlineto{\pgfqpoint{5.604917in}{0.602466in}}%
\pgfpathlineto{\pgfqpoint{5.606029in}{0.603075in}}%
\pgfpathlineto{\pgfqpoint{5.607141in}{0.608556in}}%
\pgfpathlineto{\pgfqpoint{5.607697in}{0.600503in}}%
\pgfpathlineto{\pgfqpoint{5.608253in}{0.601527in}}%
\pgfpathlineto{\pgfqpoint{5.609920in}{0.601128in}}%
\pgfpathlineto{\pgfqpoint{5.611032in}{0.605044in}}%
\pgfpathlineto{\pgfqpoint{5.611588in}{0.601209in}}%
\pgfpathlineto{\pgfqpoint{5.612144in}{0.602122in}}%
\pgfpathlineto{\pgfqpoint{5.612700in}{0.602387in}}%
\pgfpathlineto{\pgfqpoint{5.614367in}{0.600574in}}%
\pgfpathlineto{\pgfqpoint{5.616035in}{0.603617in}}%
\pgfpathlineto{\pgfqpoint{5.617146in}{0.602264in}}%
\pgfpathlineto{\pgfqpoint{5.617702in}{0.605801in}}%
\pgfpathlineto{\pgfqpoint{5.618258in}{0.603009in}}%
\pgfpathlineto{\pgfqpoint{5.619370in}{0.600206in}}%
\pgfpathlineto{\pgfqpoint{5.619926in}{0.602951in}}%
\pgfpathlineto{\pgfqpoint{5.620482in}{0.601925in}}%
\pgfpathlineto{\pgfqpoint{5.621593in}{0.600886in}}%
\pgfpathlineto{\pgfqpoint{5.622149in}{0.604535in}}%
\pgfpathlineto{\pgfqpoint{5.622705in}{0.602783in}}%
\pgfpathlineto{\pgfqpoint{5.623261in}{0.604344in}}%
\pgfpathlineto{\pgfqpoint{5.623817in}{0.603033in}}%
\pgfpathlineto{\pgfqpoint{5.624373in}{0.603457in}}%
\pgfpathlineto{\pgfqpoint{5.624928in}{0.601937in}}%
\pgfpathlineto{\pgfqpoint{5.625484in}{0.604223in}}%
\pgfpathlineto{\pgfqpoint{5.626040in}{0.602574in}}%
\pgfpathlineto{\pgfqpoint{5.626596in}{0.604026in}}%
\pgfpathlineto{\pgfqpoint{5.627152in}{0.603593in}}%
\pgfpathlineto{\pgfqpoint{5.628819in}{0.601418in}}%
\pgfpathlineto{\pgfqpoint{5.629375in}{0.604378in}}%
\pgfpathlineto{\pgfqpoint{5.629931in}{0.600232in}}%
\pgfpathlineto{\pgfqpoint{5.630487in}{0.600353in}}%
\pgfpathlineto{\pgfqpoint{5.631043in}{0.601406in}}%
\pgfpathlineto{\pgfqpoint{5.631599in}{0.605635in}}%
\pgfpathlineto{\pgfqpoint{5.632155in}{0.603275in}}%
\pgfpathlineto{\pgfqpoint{5.632711in}{0.601217in}}%
\pgfpathlineto{\pgfqpoint{5.633266in}{0.602846in}}%
\pgfpathlineto{\pgfqpoint{5.633822in}{0.603585in}}%
\pgfpathlineto{\pgfqpoint{5.634378in}{0.600208in}}%
\pgfpathlineto{\pgfqpoint{5.634934in}{0.602510in}}%
\pgfpathlineto{\pgfqpoint{5.635490in}{0.601113in}}%
\pgfpathlineto{\pgfqpoint{5.636046in}{0.604612in}}%
\pgfpathlineto{\pgfqpoint{5.636602in}{0.600333in}}%
\pgfpathlineto{\pgfqpoint{5.637157in}{0.604235in}}%
\pgfpathlineto{\pgfqpoint{5.637713in}{0.602743in}}%
\pgfpathlineto{\pgfqpoint{5.638269in}{0.604199in}}%
\pgfpathlineto{\pgfqpoint{5.638825in}{0.603488in}}%
\pgfpathlineto{\pgfqpoint{5.639381in}{0.604241in}}%
\pgfpathlineto{\pgfqpoint{5.639937in}{0.604887in}}%
\pgfpathlineto{\pgfqpoint{5.642160in}{0.601442in}}%
\pgfpathlineto{\pgfqpoint{5.643828in}{0.603770in}}%
\pgfpathlineto{\pgfqpoint{5.646051in}{0.600407in}}%
\pgfpathlineto{\pgfqpoint{5.646607in}{0.603599in}}%
\pgfpathlineto{\pgfqpoint{5.647163in}{0.601646in}}%
\pgfpathlineto{\pgfqpoint{5.648275in}{0.604407in}}%
\pgfpathlineto{\pgfqpoint{5.649386in}{0.601352in}}%
\pgfpathlineto{\pgfqpoint{5.651054in}{0.604652in}}%
\pgfpathlineto{\pgfqpoint{5.652722in}{0.601196in}}%
\pgfpathlineto{\pgfqpoint{5.653833in}{0.607282in}}%
\pgfpathlineto{\pgfqpoint{5.654945in}{0.601175in}}%
\pgfpathlineto{\pgfqpoint{5.656057in}{0.602796in}}%
\pgfpathlineto{\pgfqpoint{5.656613in}{0.601860in}}%
\pgfpathlineto{\pgfqpoint{5.657168in}{0.602269in}}%
\pgfpathlineto{\pgfqpoint{5.658836in}{0.603564in}}%
\pgfpathlineto{\pgfqpoint{5.659948in}{0.601928in}}%
\pgfpathlineto{\pgfqpoint{5.661059in}{0.606627in}}%
\pgfpathlineto{\pgfqpoint{5.662727in}{0.600995in}}%
\pgfpathlineto{\pgfqpoint{5.663283in}{0.607185in}}%
\pgfpathlineto{\pgfqpoint{5.663839in}{0.600904in}}%
\pgfpathlineto{\pgfqpoint{5.664395in}{0.604245in}}%
\pgfpathlineto{\pgfqpoint{5.665506in}{0.602041in}}%
\pgfpathlineto{\pgfqpoint{5.666618in}{0.604507in}}%
\pgfpathlineto{\pgfqpoint{5.667730in}{0.601825in}}%
\pgfpathlineto{\pgfqpoint{5.668286in}{0.603169in}}%
\pgfpathlineto{\pgfqpoint{5.668842in}{0.601940in}}%
\pgfpathlineto{\pgfqpoint{5.669397in}{0.601974in}}%
\pgfpathlineto{\pgfqpoint{5.669953in}{0.600368in}}%
\pgfpathlineto{\pgfqpoint{5.670509in}{0.604096in}}%
\pgfpathlineto{\pgfqpoint{5.671065in}{0.601021in}}%
\pgfpathlineto{\pgfqpoint{5.671621in}{0.603298in}}%
\pgfpathlineto{\pgfqpoint{5.672177in}{0.601693in}}%
\pgfpathlineto{\pgfqpoint{5.673288in}{0.601795in}}%
\pgfpathlineto{\pgfqpoint{5.673844in}{0.603941in}}%
\pgfpathlineto{\pgfqpoint{5.674400in}{0.601826in}}%
\pgfpathlineto{\pgfqpoint{5.675512in}{0.601790in}}%
\pgfpathlineto{\pgfqpoint{5.676068in}{0.603782in}}%
\pgfpathlineto{\pgfqpoint{5.676624in}{0.602124in}}%
\pgfpathlineto{\pgfqpoint{5.677179in}{0.601818in}}%
\pgfpathlineto{\pgfqpoint{5.677735in}{0.602800in}}%
\pgfpathlineto{\pgfqpoint{5.678291in}{0.602403in}}%
\pgfpathlineto{\pgfqpoint{5.679959in}{0.600961in}}%
\pgfpathlineto{\pgfqpoint{5.682182in}{0.603129in}}%
\pgfpathlineto{\pgfqpoint{5.682738in}{0.602125in}}%
\pgfpathlineto{\pgfqpoint{5.683294in}{0.605925in}}%
\pgfpathlineto{\pgfqpoint{5.683850in}{0.603536in}}%
\pgfpathlineto{\pgfqpoint{5.685517in}{0.601794in}}%
\pgfpathlineto{\pgfqpoint{5.686073in}{0.603387in}}%
\pgfpathlineto{\pgfqpoint{5.686629in}{0.601614in}}%
\pgfpathlineto{\pgfqpoint{5.688297in}{0.605234in}}%
\pgfpathlineto{\pgfqpoint{5.689964in}{0.602543in}}%
\pgfpathlineto{\pgfqpoint{5.690520in}{0.601404in}}%
\pgfpathlineto{\pgfqpoint{5.691076in}{0.604950in}}%
\pgfpathlineto{\pgfqpoint{5.691632in}{0.602096in}}%
\pgfpathlineto{\pgfqpoint{5.692188in}{0.604551in}}%
\pgfpathlineto{\pgfqpoint{5.692744in}{0.603856in}}%
\pgfpathlineto{\pgfqpoint{5.693855in}{0.601007in}}%
\pgfpathlineto{\pgfqpoint{5.694411in}{0.601643in}}%
\pgfpathlineto{\pgfqpoint{5.694967in}{0.602234in}}%
\pgfpathlineto{\pgfqpoint{5.695523in}{0.601187in}}%
\pgfpathlineto{\pgfqpoint{5.696079in}{0.601907in}}%
\pgfpathlineto{\pgfqpoint{5.696635in}{0.601686in}}%
\pgfpathlineto{\pgfqpoint{5.697190in}{0.604526in}}%
\pgfpathlineto{\pgfqpoint{5.697746in}{0.603693in}}%
\pgfpathlineto{\pgfqpoint{5.698302in}{0.603721in}}%
\pgfpathlineto{\pgfqpoint{5.698858in}{0.600336in}}%
\pgfpathlineto{\pgfqpoint{5.699970in}{0.604523in}}%
\pgfpathlineto{\pgfqpoint{5.701637in}{0.601677in}}%
\pgfpathlineto{\pgfqpoint{5.702193in}{0.602364in}}%
\pgfpathlineto{\pgfqpoint{5.702749in}{0.600487in}}%
\pgfpathlineto{\pgfqpoint{5.703305in}{0.601919in}}%
\pgfpathlineto{\pgfqpoint{5.704417in}{0.603667in}}%
\pgfpathlineto{\pgfqpoint{5.704972in}{0.601181in}}%
\pgfpathlineto{\pgfqpoint{5.705528in}{0.602741in}}%
\pgfpathlineto{\pgfqpoint{5.706084in}{0.602882in}}%
\pgfpathlineto{\pgfqpoint{5.706640in}{0.601367in}}%
\pgfpathlineto{\pgfqpoint{5.707196in}{0.602377in}}%
\pgfpathlineto{\pgfqpoint{5.707752in}{0.605539in}}%
\pgfpathlineto{\pgfqpoint{5.708308in}{0.603852in}}%
\pgfpathlineto{\pgfqpoint{5.708864in}{0.605183in}}%
\pgfpathlineto{\pgfqpoint{5.709419in}{0.602121in}}%
\pgfpathlineto{\pgfqpoint{5.709975in}{0.602593in}}%
\pgfpathlineto{\pgfqpoint{5.710531in}{0.604018in}}%
\pgfpathlineto{\pgfqpoint{5.712199in}{0.600430in}}%
\pgfpathlineto{\pgfqpoint{5.713866in}{0.604764in}}%
\pgfpathlineto{\pgfqpoint{5.714978in}{0.601151in}}%
\pgfpathlineto{\pgfqpoint{5.715534in}{0.604453in}}%
\pgfpathlineto{\pgfqpoint{5.716090in}{0.601037in}}%
\pgfpathlineto{\pgfqpoint{5.716646in}{0.602486in}}%
\pgfpathlineto{\pgfqpoint{5.717757in}{0.601383in}}%
\pgfpathlineto{\pgfqpoint{5.718869in}{0.603865in}}%
\pgfpathlineto{\pgfqpoint{5.719425in}{0.607177in}}%
\pgfpathlineto{\pgfqpoint{5.719981in}{0.605668in}}%
\pgfpathlineto{\pgfqpoint{5.720537in}{0.602129in}}%
\pgfpathlineto{\pgfqpoint{5.721092in}{0.604323in}}%
\pgfpathlineto{\pgfqpoint{5.721648in}{0.602427in}}%
\pgfpathlineto{\pgfqpoint{5.722204in}{0.609080in}}%
\pgfpathlineto{\pgfqpoint{5.722760in}{0.603307in}}%
\pgfpathlineto{\pgfqpoint{5.723872in}{0.601004in}}%
\pgfpathlineto{\pgfqpoint{5.724428in}{0.602828in}}%
\pgfpathlineto{\pgfqpoint{5.725539in}{0.604244in}}%
\pgfpathlineto{\pgfqpoint{5.727207in}{0.603201in}}%
\pgfpathlineto{\pgfqpoint{5.730542in}{0.601011in}}%
\pgfpathlineto{\pgfqpoint{5.732210in}{0.603309in}}%
\pgfpathlineto{\pgfqpoint{5.733321in}{0.600487in}}%
\pgfpathlineto{\pgfqpoint{5.733877in}{0.602203in}}%
\pgfpathlineto{\pgfqpoint{5.734989in}{0.602042in}}%
\pgfpathlineto{\pgfqpoint{5.735545in}{0.603458in}}%
\pgfpathlineto{\pgfqpoint{5.736101in}{0.602608in}}%
\pgfpathlineto{\pgfqpoint{5.737212in}{0.600549in}}%
\pgfpathlineto{\pgfqpoint{5.737768in}{0.602096in}}%
\pgfpathlineto{\pgfqpoint{5.739436in}{0.604197in}}%
\pgfpathlineto{\pgfqpoint{5.739992in}{0.601558in}}%
\pgfpathlineto{\pgfqpoint{5.740548in}{0.602800in}}%
\pgfpathlineto{\pgfqpoint{5.741103in}{0.604895in}}%
\pgfpathlineto{\pgfqpoint{5.742771in}{0.601179in}}%
\pgfpathlineto{\pgfqpoint{5.743327in}{0.600505in}}%
\pgfpathlineto{\pgfqpoint{5.743883in}{0.601877in}}%
\pgfpathlineto{\pgfqpoint{5.744439in}{0.601047in}}%
\pgfpathlineto{\pgfqpoint{5.745550in}{0.601613in}}%
\pgfpathlineto{\pgfqpoint{5.746106in}{0.604276in}}%
\pgfpathlineto{\pgfqpoint{5.746662in}{0.602166in}}%
\pgfpathlineto{\pgfqpoint{5.747218in}{0.600725in}}%
\pgfpathlineto{\pgfqpoint{5.747774in}{0.604453in}}%
\pgfpathlineto{\pgfqpoint{5.748330in}{0.604289in}}%
\pgfpathlineto{\pgfqpoint{5.749441in}{0.602413in}}%
\pgfpathlineto{\pgfqpoint{5.749997in}{0.604011in}}%
\pgfpathlineto{\pgfqpoint{5.750553in}{0.602496in}}%
\pgfpathlineto{\pgfqpoint{5.751109in}{0.603282in}}%
\pgfpathlineto{\pgfqpoint{5.751665in}{0.601275in}}%
\pgfpathlineto{\pgfqpoint{5.752221in}{0.603843in}}%
\pgfpathlineto{\pgfqpoint{5.752777in}{0.601528in}}%
\pgfpathlineto{\pgfqpoint{5.754444in}{0.602937in}}%
\pgfpathlineto{\pgfqpoint{5.755000in}{0.601827in}}%
\pgfpathlineto{\pgfqpoint{5.755556in}{0.604120in}}%
\pgfpathlineto{\pgfqpoint{5.756112in}{0.603678in}}%
\pgfpathlineto{\pgfqpoint{5.756668in}{0.602684in}}%
\pgfpathlineto{\pgfqpoint{5.757223in}{0.605237in}}%
\pgfpathlineto{\pgfqpoint{5.757779in}{0.602642in}}%
\pgfpathlineto{\pgfqpoint{5.759447in}{0.600626in}}%
\pgfpathlineto{\pgfqpoint{5.760003in}{0.601364in}}%
\pgfpathlineto{\pgfqpoint{5.761670in}{0.604005in}}%
\pgfpathlineto{\pgfqpoint{5.762226in}{0.601205in}}%
\pgfpathlineto{\pgfqpoint{5.762782in}{0.604203in}}%
\pgfpathlineto{\pgfqpoint{5.763338in}{0.600591in}}%
\pgfpathlineto{\pgfqpoint{5.763894in}{0.602730in}}%
\pgfpathlineto{\pgfqpoint{5.765006in}{0.601781in}}%
\pgfpathlineto{\pgfqpoint{5.766117in}{0.605904in}}%
\pgfpathlineto{\pgfqpoint{5.766673in}{0.605314in}}%
\pgfpathlineto{\pgfqpoint{5.768341in}{0.603222in}}%
\pgfpathlineto{\pgfqpoint{5.768897in}{0.603308in}}%
\pgfpathlineto{\pgfqpoint{5.770008in}{0.600969in}}%
\pgfpathlineto{\pgfqpoint{5.770564in}{0.603913in}}%
\pgfpathlineto{\pgfqpoint{5.771120in}{0.600815in}}%
\pgfpathlineto{\pgfqpoint{5.771676in}{0.602281in}}%
\pgfpathlineto{\pgfqpoint{5.772232in}{0.603876in}}%
\pgfpathlineto{\pgfqpoint{5.772788in}{0.602774in}}%
\pgfpathlineto{\pgfqpoint{5.773343in}{0.602603in}}%
\pgfpathlineto{\pgfqpoint{5.773899in}{0.603625in}}%
\pgfpathlineto{\pgfqpoint{5.775011in}{0.601178in}}%
\pgfpathlineto{\pgfqpoint{5.775567in}{0.605307in}}%
\pgfpathlineto{\pgfqpoint{5.776123in}{0.601638in}}%
\pgfpathlineto{\pgfqpoint{5.777234in}{0.603801in}}%
\pgfpathlineto{\pgfqpoint{5.777790in}{0.600952in}}%
\pgfpathlineto{\pgfqpoint{5.778346in}{0.602431in}}%
\pgfpathlineto{\pgfqpoint{5.779458in}{0.602262in}}%
\pgfpathlineto{\pgfqpoint{5.780014in}{0.600758in}}%
\pgfpathlineto{\pgfqpoint{5.781125in}{0.606447in}}%
\pgfpathlineto{\pgfqpoint{5.781681in}{0.604140in}}%
\pgfpathlineto{\pgfqpoint{5.782793in}{0.602598in}}%
\pgfpathlineto{\pgfqpoint{5.783349in}{0.604335in}}%
\pgfpathlineto{\pgfqpoint{5.783905in}{0.600989in}}%
\pgfpathlineto{\pgfqpoint{5.784461in}{0.604886in}}%
\pgfpathlineto{\pgfqpoint{5.785017in}{0.604304in}}%
\pgfpathlineto{\pgfqpoint{5.786684in}{0.602812in}}%
\pgfpathlineto{\pgfqpoint{5.787796in}{0.600076in}}%
\pgfpathlineto{\pgfqpoint{5.788352in}{0.601780in}}%
\pgfpathlineto{\pgfqpoint{5.788908in}{0.603557in}}%
\pgfpathlineto{\pgfqpoint{5.789463in}{0.603058in}}%
\pgfpathlineto{\pgfqpoint{5.790019in}{0.602588in}}%
\pgfpathlineto{\pgfqpoint{5.790575in}{0.603907in}}%
\pgfpathlineto{\pgfqpoint{5.791131in}{0.600837in}}%
\pgfpathlineto{\pgfqpoint{5.791687in}{0.603038in}}%
\pgfpathlineto{\pgfqpoint{5.792243in}{0.601412in}}%
\pgfpathlineto{\pgfqpoint{5.792799in}{0.602401in}}%
\pgfpathlineto{\pgfqpoint{5.793354in}{0.604148in}}%
\pgfpathlineto{\pgfqpoint{5.794466in}{0.600899in}}%
\pgfpathlineto{\pgfqpoint{5.795022in}{0.601713in}}%
\pgfpathlineto{\pgfqpoint{5.795578in}{0.600901in}}%
\pgfpathlineto{\pgfqpoint{5.801137in}{0.604051in}}%
\pgfpathlineto{\pgfqpoint{5.802804in}{0.601766in}}%
\pgfpathlineto{\pgfqpoint{5.803360in}{0.601777in}}%
\pgfpathlineto{\pgfqpoint{5.803916in}{0.604449in}}%
\pgfpathlineto{\pgfqpoint{5.804472in}{0.602936in}}%
\pgfpathlineto{\pgfqpoint{5.806695in}{0.601984in}}%
\pgfpathlineto{\pgfqpoint{5.807251in}{0.603470in}}%
\pgfpathlineto{\pgfqpoint{5.807807in}{0.602113in}}%
\pgfpathlineto{\pgfqpoint{5.808363in}{0.600826in}}%
\pgfpathlineto{\pgfqpoint{5.808919in}{0.601585in}}%
\pgfpathlineto{\pgfqpoint{5.809474in}{0.602695in}}%
\pgfpathlineto{\pgfqpoint{5.811142in}{0.601320in}}%
\pgfpathlineto{\pgfqpoint{5.811698in}{0.606878in}}%
\pgfpathlineto{\pgfqpoint{5.812254in}{0.602004in}}%
\pgfpathlineto{\pgfqpoint{5.812810in}{0.604496in}}%
\pgfpathlineto{\pgfqpoint{5.814477in}{0.601091in}}%
\pgfpathlineto{\pgfqpoint{5.815589in}{0.605161in}}%
\pgfpathlineto{\pgfqpoint{5.816145in}{0.600215in}}%
\pgfpathlineto{\pgfqpoint{5.816701in}{0.601362in}}%
\pgfpathlineto{\pgfqpoint{5.817812in}{0.602486in}}%
\pgfpathlineto{\pgfqpoint{5.819480in}{0.600852in}}%
\pgfpathlineto{\pgfqpoint{5.820036in}{0.602086in}}%
\pgfpathlineto{\pgfqpoint{5.821703in}{0.602124in}}%
\pgfpathlineto{\pgfqpoint{5.822259in}{0.601053in}}%
\pgfpathlineto{\pgfqpoint{5.823927in}{0.605478in}}%
\pgfpathlineto{\pgfqpoint{5.824483in}{0.601588in}}%
\pgfpathlineto{\pgfqpoint{5.825039in}{0.602509in}}%
\pgfpathlineto{\pgfqpoint{5.825594in}{0.601834in}}%
\pgfpathlineto{\pgfqpoint{5.826150in}{0.604489in}}%
\pgfpathlineto{\pgfqpoint{5.826706in}{0.600125in}}%
\pgfpathlineto{\pgfqpoint{5.827262in}{0.602048in}}%
\pgfpathlineto{\pgfqpoint{5.828374in}{0.600493in}}%
\pgfpathlineto{\pgfqpoint{5.828930in}{0.602441in}}%
\pgfpathlineto{\pgfqpoint{5.829485in}{0.600565in}}%
\pgfpathlineto{\pgfqpoint{5.831709in}{0.604622in}}%
\pgfpathlineto{\pgfqpoint{5.832265in}{0.603805in}}%
\pgfpathlineto{\pgfqpoint{5.832821in}{0.604116in}}%
\pgfpathlineto{\pgfqpoint{5.833376in}{0.602181in}}%
\pgfpathlineto{\pgfqpoint{5.833932in}{0.603590in}}%
\pgfpathlineto{\pgfqpoint{5.834488in}{0.606609in}}%
\pgfpathlineto{\pgfqpoint{5.835600in}{0.600987in}}%
\pgfpathlineto{\pgfqpoint{5.837267in}{0.604702in}}%
\pgfpathlineto{\pgfqpoint{5.838935in}{0.601233in}}%
\pgfpathlineto{\pgfqpoint{5.839491in}{0.601899in}}%
\pgfpathlineto{\pgfqpoint{5.840047in}{0.605046in}}%
\pgfpathlineto{\pgfqpoint{5.840603in}{0.602268in}}%
\pgfpathlineto{\pgfqpoint{5.841159in}{0.604587in}}%
\pgfpathlineto{\pgfqpoint{5.841714in}{0.600741in}}%
\pgfpathlineto{\pgfqpoint{5.842270in}{0.603174in}}%
\pgfpathlineto{\pgfqpoint{5.843382in}{0.601882in}}%
\pgfpathlineto{\pgfqpoint{5.843938in}{0.602389in}}%
\pgfpathlineto{\pgfqpoint{5.844494in}{0.601815in}}%
\pgfpathlineto{\pgfqpoint{5.845050in}{0.602615in}}%
\pgfpathlineto{\pgfqpoint{5.845605in}{0.602340in}}%
\pgfpathlineto{\pgfqpoint{5.846161in}{0.603810in}}%
\pgfpathlineto{\pgfqpoint{5.846717in}{0.601513in}}%
\pgfpathlineto{\pgfqpoint{5.847273in}{0.603266in}}%
\pgfpathlineto{\pgfqpoint{5.848385in}{0.603584in}}%
\pgfpathlineto{\pgfqpoint{5.849496in}{0.601764in}}%
\pgfpathlineto{\pgfqpoint{5.850608in}{0.603652in}}%
\pgfpathlineto{\pgfqpoint{5.851164in}{0.601137in}}%
\pgfpathlineto{\pgfqpoint{5.851720in}{0.604668in}}%
\pgfpathlineto{\pgfqpoint{5.852276in}{0.603835in}}%
\pgfpathlineto{\pgfqpoint{5.852832in}{0.601504in}}%
\pgfpathlineto{\pgfqpoint{5.853387in}{0.601946in}}%
\pgfpathlineto{\pgfqpoint{5.853943in}{0.602337in}}%
\pgfpathlineto{\pgfqpoint{5.854499in}{0.600732in}}%
\pgfpathlineto{\pgfqpoint{5.855611in}{0.604595in}}%
\pgfpathlineto{\pgfqpoint{5.857279in}{0.601454in}}%
\pgfpathlineto{\pgfqpoint{5.857834in}{0.602797in}}%
\pgfpathlineto{\pgfqpoint{5.858390in}{0.601717in}}%
\pgfpathlineto{\pgfqpoint{5.858946in}{0.600923in}}%
\pgfpathlineto{\pgfqpoint{5.859502in}{0.603271in}}%
\pgfpathlineto{\pgfqpoint{5.860058in}{0.602803in}}%
\pgfpathlineto{\pgfqpoint{5.860614in}{0.601980in}}%
\pgfpathlineto{\pgfqpoint{5.861725in}{0.603262in}}%
\pgfpathlineto{\pgfqpoint{5.862281in}{0.602228in}}%
\pgfpathlineto{\pgfqpoint{5.863393in}{0.603790in}}%
\pgfpathlineto{\pgfqpoint{5.863949in}{0.601586in}}%
\pgfpathlineto{\pgfqpoint{5.864505in}{0.603658in}}%
\pgfpathlineto{\pgfqpoint{5.866172in}{0.600604in}}%
\pgfpathlineto{\pgfqpoint{5.867284in}{0.604154in}}%
\pgfpathlineto{\pgfqpoint{5.867840in}{0.603130in}}%
\pgfpathlineto{\pgfqpoint{5.869507in}{0.600969in}}%
\pgfpathlineto{\pgfqpoint{5.870063in}{0.602383in}}%
\pgfpathlineto{\pgfqpoint{5.870619in}{0.605986in}}%
\pgfpathlineto{\pgfqpoint{5.871175in}{0.603792in}}%
\pgfpathlineto{\pgfqpoint{5.871731in}{0.604538in}}%
\pgfpathlineto{\pgfqpoint{5.872843in}{0.603544in}}%
\pgfpathlineto{\pgfqpoint{5.873398in}{0.604873in}}%
\pgfpathlineto{\pgfqpoint{5.874510in}{0.601819in}}%
\pgfpathlineto{\pgfqpoint{5.875066in}{0.602157in}}%
\pgfpathlineto{\pgfqpoint{5.875622in}{0.600976in}}%
\pgfpathlineto{\pgfqpoint{5.876734in}{0.606007in}}%
\pgfpathlineto{\pgfqpoint{5.877290in}{0.603953in}}%
\pgfpathlineto{\pgfqpoint{5.877845in}{0.603634in}}%
\pgfpathlineto{\pgfqpoint{5.878957in}{0.600136in}}%
\pgfpathlineto{\pgfqpoint{5.879513in}{0.600817in}}%
\pgfpathlineto{\pgfqpoint{5.880625in}{0.606055in}}%
\pgfpathlineto{\pgfqpoint{5.881181in}{0.601146in}}%
\pgfpathlineto{\pgfqpoint{5.881736in}{0.601257in}}%
\pgfpathlineto{\pgfqpoint{5.883404in}{0.603561in}}%
\pgfpathlineto{\pgfqpoint{5.883960in}{0.600914in}}%
\pgfpathlineto{\pgfqpoint{5.884516in}{0.601879in}}%
\pgfpathlineto{\pgfqpoint{5.885627in}{0.601900in}}%
\pgfpathlineto{\pgfqpoint{5.886183in}{0.602873in}}%
\pgfpathlineto{\pgfqpoint{5.886739in}{0.601136in}}%
\pgfpathlineto{\pgfqpoint{5.887295in}{0.601739in}}%
\pgfpathlineto{\pgfqpoint{5.889518in}{0.601070in}}%
\pgfpathlineto{\pgfqpoint{5.890630in}{0.603872in}}%
\pgfpathlineto{\pgfqpoint{5.891186in}{0.602009in}}%
\pgfpathlineto{\pgfqpoint{5.891742in}{0.605645in}}%
\pgfpathlineto{\pgfqpoint{5.892298in}{0.602056in}}%
\pgfpathlineto{\pgfqpoint{5.893409in}{0.603000in}}%
\pgfpathlineto{\pgfqpoint{5.893965in}{0.600687in}}%
\pgfpathlineto{\pgfqpoint{5.894521in}{0.602729in}}%
\pgfpathlineto{\pgfqpoint{5.895633in}{0.604709in}}%
\pgfpathlineto{\pgfqpoint{5.897301in}{0.601423in}}%
\pgfpathlineto{\pgfqpoint{5.897856in}{0.602089in}}%
\pgfpathlineto{\pgfqpoint{5.898412in}{0.604654in}}%
\pgfpathlineto{\pgfqpoint{5.898968in}{0.600724in}}%
\pgfpathlineto{\pgfqpoint{5.899524in}{0.605201in}}%
\pgfpathlineto{\pgfqpoint{5.900080in}{0.604185in}}%
\pgfpathlineto{\pgfqpoint{5.901192in}{0.604123in}}%
\pgfpathlineto{\pgfqpoint{5.902859in}{0.601532in}}%
\pgfpathlineto{\pgfqpoint{5.903971in}{0.604680in}}%
\pgfpathlineto{\pgfqpoint{5.904527in}{0.602335in}}%
\pgfpathlineto{\pgfqpoint{5.905083in}{0.602886in}}%
\pgfpathlineto{\pgfqpoint{5.906194in}{0.602777in}}%
\pgfpathlineto{\pgfqpoint{5.906750in}{0.600965in}}%
\pgfpathlineto{\pgfqpoint{5.907306in}{0.604183in}}%
\pgfpathlineto{\pgfqpoint{5.907862in}{0.601774in}}%
\pgfpathlineto{\pgfqpoint{5.908418in}{0.600592in}}%
\pgfpathlineto{\pgfqpoint{5.909529in}{0.602383in}}%
\pgfpathlineto{\pgfqpoint{5.910085in}{0.600641in}}%
\pgfpathlineto{\pgfqpoint{5.910641in}{0.601229in}}%
\pgfpathlineto{\pgfqpoint{5.913976in}{0.603355in}}%
\pgfpathlineto{\pgfqpoint{5.914532in}{0.600760in}}%
\pgfpathlineto{\pgfqpoint{5.915088in}{0.603639in}}%
\pgfpathlineto{\pgfqpoint{5.915644in}{0.602945in}}%
\pgfpathlineto{\pgfqpoint{5.917312in}{0.602711in}}%
\pgfpathlineto{\pgfqpoint{5.917867in}{0.603152in}}%
\pgfpathlineto{\pgfqpoint{5.918423in}{0.605395in}}%
\pgfpathlineto{\pgfqpoint{5.918979in}{0.602237in}}%
\pgfpathlineto{\pgfqpoint{5.919535in}{0.606452in}}%
\pgfpathlineto{\pgfqpoint{5.920091in}{0.602540in}}%
\pgfpathlineto{\pgfqpoint{5.920647in}{0.603473in}}%
\pgfpathlineto{\pgfqpoint{5.921758in}{0.601090in}}%
\pgfpathlineto{\pgfqpoint{5.922314in}{0.603963in}}%
\pgfpathlineto{\pgfqpoint{5.922870in}{0.601235in}}%
\pgfpathlineto{\pgfqpoint{5.923426in}{0.601349in}}%
\pgfpathlineto{\pgfqpoint{5.924538in}{0.607049in}}%
\pgfpathlineto{\pgfqpoint{5.925094in}{0.602287in}}%
\pgfpathlineto{\pgfqpoint{5.925649in}{0.602813in}}%
\pgfpathlineto{\pgfqpoint{5.926761in}{0.603324in}}%
\pgfpathlineto{\pgfqpoint{5.927317in}{0.604058in}}%
\pgfpathlineto{\pgfqpoint{5.927873in}{0.608350in}}%
\pgfpathlineto{\pgfqpoint{5.928429in}{0.602824in}}%
\pgfpathlineto{\pgfqpoint{5.928985in}{0.605118in}}%
\pgfpathlineto{\pgfqpoint{5.930096in}{0.601848in}}%
\pgfpathlineto{\pgfqpoint{5.930652in}{0.606309in}}%
\pgfpathlineto{\pgfqpoint{5.931208in}{0.602343in}}%
\pgfpathlineto{\pgfqpoint{5.931764in}{0.605420in}}%
\pgfpathlineto{\pgfqpoint{5.932320in}{0.603299in}}%
\pgfpathlineto{\pgfqpoint{5.932876in}{0.604003in}}%
\pgfpathlineto{\pgfqpoint{5.933432in}{0.601073in}}%
\pgfpathlineto{\pgfqpoint{5.933987in}{0.601643in}}%
\pgfpathlineto{\pgfqpoint{5.936767in}{0.601934in}}%
\pgfpathlineto{\pgfqpoint{5.937323in}{0.602096in}}%
\pgfpathlineto{\pgfqpoint{5.937878in}{0.607614in}}%
\pgfpathlineto{\pgfqpoint{5.938990in}{0.601804in}}%
\pgfpathlineto{\pgfqpoint{5.939546in}{0.602591in}}%
\pgfpathlineto{\pgfqpoint{5.940102in}{0.601065in}}%
\pgfpathlineto{\pgfqpoint{5.940658in}{0.604400in}}%
\pgfpathlineto{\pgfqpoint{5.941214in}{0.600216in}}%
\pgfpathlineto{\pgfqpoint{5.941769in}{0.603427in}}%
\pgfpathlineto{\pgfqpoint{5.943993in}{0.600701in}}%
\pgfpathlineto{\pgfqpoint{5.945105in}{0.602358in}}%
\pgfpathlineto{\pgfqpoint{5.946216in}{0.600969in}}%
\pgfpathlineto{\pgfqpoint{5.947328in}{0.603776in}}%
\pgfpathlineto{\pgfqpoint{5.948440in}{0.601437in}}%
\pgfpathlineto{\pgfqpoint{5.948996in}{0.607037in}}%
\pgfpathlineto{\pgfqpoint{5.949551in}{0.603710in}}%
\pgfpathlineto{\pgfqpoint{5.950107in}{0.599956in}}%
\pgfpathlineto{\pgfqpoint{5.950663in}{0.600949in}}%
\pgfpathlineto{\pgfqpoint{5.951219in}{0.600766in}}%
\pgfpathlineto{\pgfqpoint{5.952331in}{0.603078in}}%
\pgfpathlineto{\pgfqpoint{5.952887in}{0.601580in}}%
\pgfpathlineto{\pgfqpoint{5.953443in}{0.604539in}}%
\pgfpathlineto{\pgfqpoint{5.953998in}{0.602397in}}%
\pgfpathlineto{\pgfqpoint{5.955110in}{0.601935in}}%
\pgfpathlineto{\pgfqpoint{5.955666in}{0.601268in}}%
\pgfpathlineto{\pgfqpoint{5.956222in}{0.603295in}}%
\pgfpathlineto{\pgfqpoint{5.956778in}{0.601913in}}%
\pgfpathlineto{\pgfqpoint{5.959557in}{0.603350in}}%
\pgfpathlineto{\pgfqpoint{5.960669in}{0.601039in}}%
\pgfpathlineto{\pgfqpoint{5.961225in}{0.601749in}}%
\pgfpathlineto{\pgfqpoint{5.962336in}{0.603613in}}%
\pgfpathlineto{\pgfqpoint{5.963448in}{0.602409in}}%
\pgfpathlineto{\pgfqpoint{5.964004in}{0.604666in}}%
\pgfpathlineto{\pgfqpoint{5.964560in}{0.602393in}}%
\pgfpathlineto{\pgfqpoint{5.966227in}{0.602682in}}%
\pgfpathlineto{\pgfqpoint{5.966783in}{0.603585in}}%
\pgfpathlineto{\pgfqpoint{5.968451in}{0.601223in}}%
\pgfpathlineto{\pgfqpoint{5.970674in}{0.603369in}}%
\pgfpathlineto{\pgfqpoint{5.972342in}{0.600650in}}%
\pgfpathlineto{\pgfqpoint{5.976233in}{0.603735in}}%
\pgfpathlineto{\pgfqpoint{5.977900in}{0.600796in}}%
\pgfpathlineto{\pgfqpoint{5.979012in}{0.603929in}}%
\pgfpathlineto{\pgfqpoint{5.979568in}{0.602205in}}%
\pgfpathlineto{\pgfqpoint{5.980680in}{0.600987in}}%
\pgfpathlineto{\pgfqpoint{5.981236in}{0.606410in}}%
\pgfpathlineto{\pgfqpoint{5.981791in}{0.604106in}}%
\pgfpathlineto{\pgfqpoint{5.982347in}{0.600553in}}%
\pgfpathlineto{\pgfqpoint{5.982903in}{0.602268in}}%
\pgfpathlineto{\pgfqpoint{5.985682in}{0.602544in}}%
\pgfpathlineto{\pgfqpoint{5.986238in}{0.603853in}}%
\pgfpathlineto{\pgfqpoint{5.986794in}{0.603040in}}%
\pgfpathlineto{\pgfqpoint{5.987350in}{0.601970in}}%
\pgfpathlineto{\pgfqpoint{5.987906in}{0.603729in}}%
\pgfpathlineto{\pgfqpoint{5.988462in}{0.600322in}}%
\pgfpathlineto{\pgfqpoint{5.989018in}{0.601177in}}%
\pgfpathlineto{\pgfqpoint{5.990685in}{0.602925in}}%
\pgfpathlineto{\pgfqpoint{5.991797in}{0.600973in}}%
\pgfpathlineto{\pgfqpoint{5.993465in}{0.602228in}}%
\pgfpathlineto{\pgfqpoint{5.994020in}{0.601364in}}%
\pgfpathlineto{\pgfqpoint{5.995132in}{0.604386in}}%
\pgfpathlineto{\pgfqpoint{5.995688in}{0.603723in}}%
\pgfpathlineto{\pgfqpoint{5.996244in}{0.603766in}}%
\pgfpathlineto{\pgfqpoint{5.996800in}{0.601562in}}%
\pgfpathlineto{\pgfqpoint{5.997356in}{0.602353in}}%
\pgfpathlineto{\pgfqpoint{5.997911in}{0.602106in}}%
\pgfpathlineto{\pgfqpoint{5.998467in}{0.603221in}}%
\pgfpathlineto{\pgfqpoint{5.999023in}{0.600411in}}%
\pgfpathlineto{\pgfqpoint{5.999579in}{0.600678in}}%
\pgfpathlineto{\pgfqpoint{6.002914in}{0.602705in}}%
\pgfpathlineto{\pgfqpoint{6.004026in}{0.602706in}}%
\pgfpathlineto{\pgfqpoint{6.012364in}{0.602298in}}%
\pgfpathlineto{\pgfqpoint{6.012920in}{0.603797in}}%
\pgfpathlineto{\pgfqpoint{6.013476in}{0.602662in}}%
\pgfpathlineto{\pgfqpoint{6.014587in}{0.601141in}}%
\pgfpathlineto{\pgfqpoint{6.015699in}{0.602967in}}%
\pgfpathlineto{\pgfqpoint{6.016811in}{0.600718in}}%
\pgfpathlineto{\pgfqpoint{6.017922in}{0.603453in}}%
\pgfpathlineto{\pgfqpoint{6.018478in}{0.600278in}}%
\pgfpathlineto{\pgfqpoint{6.019034in}{0.603581in}}%
\pgfpathlineto{\pgfqpoint{6.019590in}{0.602465in}}%
\pgfpathlineto{\pgfqpoint{6.020702in}{0.600551in}}%
\pgfpathlineto{\pgfqpoint{6.021813in}{0.603900in}}%
\pgfpathlineto{\pgfqpoint{6.022925in}{0.600952in}}%
\pgfpathlineto{\pgfqpoint{6.024593in}{0.603631in}}%
\pgfpathlineto{\pgfqpoint{6.025149in}{0.600540in}}%
\pgfpathlineto{\pgfqpoint{6.025704in}{0.602558in}}%
\pgfpathlineto{\pgfqpoint{6.026260in}{0.601101in}}%
\pgfpathlineto{\pgfqpoint{6.026816in}{0.604601in}}%
\pgfpathlineto{\pgfqpoint{6.027372in}{0.602064in}}%
\pgfpathlineto{\pgfqpoint{6.029040in}{0.600519in}}%
\pgfpathlineto{\pgfqpoint{6.029596in}{0.604938in}}%
\pgfpathlineto{\pgfqpoint{6.030151in}{0.603419in}}%
\pgfpathlineto{\pgfqpoint{6.031819in}{0.600788in}}%
\pgfpathlineto{\pgfqpoint{6.032931in}{0.601434in}}%
\pgfpathlineto{\pgfqpoint{6.033487in}{0.602268in}}%
\pgfpathlineto{\pgfqpoint{6.035154in}{0.600378in}}%
\pgfpathlineto{\pgfqpoint{6.035710in}{0.603422in}}%
\pgfpathlineto{\pgfqpoint{6.036266in}{0.602401in}}%
\pgfpathlineto{\pgfqpoint{6.037933in}{0.600358in}}%
\pgfpathlineto{\pgfqpoint{6.039601in}{0.603121in}}%
\pgfpathlineto{\pgfqpoint{6.041269in}{0.601353in}}%
\pgfpathlineto{\pgfqpoint{6.041824in}{0.601159in}}%
\pgfpathlineto{\pgfqpoint{6.042380in}{0.602244in}}%
\pgfpathlineto{\pgfqpoint{6.042936in}{0.601284in}}%
\pgfpathlineto{\pgfqpoint{6.043492in}{0.601323in}}%
\pgfpathlineto{\pgfqpoint{6.044048in}{0.603677in}}%
\pgfpathlineto{\pgfqpoint{6.044604in}{0.602058in}}%
\pgfpathlineto{\pgfqpoint{6.046271in}{0.603111in}}%
\pgfpathlineto{\pgfqpoint{6.047383in}{0.600155in}}%
\pgfpathlineto{\pgfqpoint{6.047939in}{0.602090in}}%
\pgfpathlineto{\pgfqpoint{6.048495in}{0.601186in}}%
\pgfpathlineto{\pgfqpoint{6.049051in}{0.605485in}}%
\pgfpathlineto{\pgfqpoint{6.049607in}{0.603361in}}%
\pgfpathlineto{\pgfqpoint{6.050162in}{0.601022in}}%
\pgfpathlineto{\pgfqpoint{6.050718in}{0.602372in}}%
\pgfpathlineto{\pgfqpoint{6.051830in}{0.601028in}}%
\pgfpathlineto{\pgfqpoint{6.053498in}{0.602197in}}%
\pgfpathlineto{\pgfqpoint{6.054609in}{0.601115in}}%
\pgfpathlineto{\pgfqpoint{6.055165in}{0.602708in}}%
\pgfpathlineto{\pgfqpoint{6.056277in}{0.600190in}}%
\pgfpathlineto{\pgfqpoint{6.057944in}{0.603336in}}%
\pgfpathlineto{\pgfqpoint{6.058500in}{0.600988in}}%
\pgfpathlineto{\pgfqpoint{6.059056in}{0.602189in}}%
\pgfpathlineto{\pgfqpoint{6.060724in}{0.603973in}}%
\pgfpathlineto{\pgfqpoint{6.061280in}{0.602334in}}%
\pgfpathlineto{\pgfqpoint{6.061835in}{0.604388in}}%
\pgfpathlineto{\pgfqpoint{6.062391in}{0.603791in}}%
\pgfpathlineto{\pgfqpoint{6.062947in}{0.601175in}}%
\pgfpathlineto{\pgfqpoint{6.063503in}{0.601585in}}%
\pgfpathlineto{\pgfqpoint{6.064059in}{0.604382in}}%
\pgfpathlineto{\pgfqpoint{6.064615in}{0.602994in}}%
\pgfpathlineto{\pgfqpoint{6.065171in}{0.603350in}}%
\pgfpathlineto{\pgfqpoint{6.066282in}{0.600500in}}%
\pgfpathlineto{\pgfqpoint{6.066838in}{0.601381in}}%
\pgfpathlineto{\pgfqpoint{6.068506in}{0.603177in}}%
\pgfpathlineto{\pgfqpoint{6.069618in}{0.601025in}}%
\pgfpathlineto{\pgfqpoint{6.070173in}{0.602584in}}%
\pgfpathlineto{\pgfqpoint{6.070729in}{0.601385in}}%
\pgfpathlineto{\pgfqpoint{6.071841in}{0.601384in}}%
\pgfpathlineto{\pgfqpoint{6.072397in}{0.600682in}}%
\pgfpathlineto{\pgfqpoint{6.073509in}{0.602778in}}%
\pgfpathlineto{\pgfqpoint{6.074064in}{0.602550in}}%
\pgfpathlineto{\pgfqpoint{6.074620in}{0.602623in}}%
\pgfpathlineto{\pgfqpoint{6.075176in}{0.600430in}}%
\pgfpathlineto{\pgfqpoint{6.075732in}{0.601518in}}%
\pgfpathlineto{\pgfqpoint{6.076288in}{0.601567in}}%
\pgfpathlineto{\pgfqpoint{6.077955in}{0.600469in}}%
\pgfpathlineto{\pgfqpoint{6.079067in}{0.602762in}}%
\pgfpathlineto{\pgfqpoint{6.079623in}{0.601195in}}%
\pgfpathlineto{\pgfqpoint{6.080735in}{0.603434in}}%
\pgfpathlineto{\pgfqpoint{6.082402in}{0.601250in}}%
\pgfpathlineto{\pgfqpoint{6.083514in}{0.604934in}}%
\pgfpathlineto{\pgfqpoint{6.084626in}{0.600839in}}%
\pgfpathlineto{\pgfqpoint{6.085182in}{0.603553in}}%
\pgfpathlineto{\pgfqpoint{6.085738in}{0.601674in}}%
\pgfpathlineto{\pgfqpoint{6.086293in}{0.601281in}}%
\pgfpathlineto{\pgfqpoint{6.087961in}{0.602388in}}%
\pgfpathlineto{\pgfqpoint{6.088517in}{0.602788in}}%
\pgfpathlineto{\pgfqpoint{6.089629in}{0.600457in}}%
\pgfpathlineto{\pgfqpoint{6.090184in}{0.601110in}}%
\pgfpathlineto{\pgfqpoint{6.090740in}{0.600896in}}%
\pgfpathlineto{\pgfqpoint{6.091296in}{0.603681in}}%
\pgfpathlineto{\pgfqpoint{6.091852in}{0.600840in}}%
\pgfpathlineto{\pgfqpoint{6.092408in}{0.600401in}}%
\pgfpathlineto{\pgfqpoint{6.093520in}{0.604700in}}%
\pgfpathlineto{\pgfqpoint{6.094075in}{0.600757in}}%
\pgfpathlineto{\pgfqpoint{6.094631in}{0.601452in}}%
\pgfpathlineto{\pgfqpoint{6.096299in}{0.602755in}}%
\pgfpathlineto{\pgfqpoint{6.096855in}{0.605232in}}%
\pgfpathlineto{\pgfqpoint{6.098522in}{0.601671in}}%
\pgfpathlineto{\pgfqpoint{6.099078in}{0.604628in}}%
\pgfpathlineto{\pgfqpoint{6.100746in}{0.601072in}}%
\pgfpathlineto{\pgfqpoint{6.101302in}{0.600555in}}%
\pgfpathlineto{\pgfqpoint{6.101857in}{0.603949in}}%
\pgfpathlineto{\pgfqpoint{6.102413in}{0.602441in}}%
\pgfpathlineto{\pgfqpoint{6.103525in}{0.603803in}}%
\pgfpathlineto{\pgfqpoint{6.104081in}{0.606076in}}%
\pgfpathlineto{\pgfqpoint{6.105193in}{0.601463in}}%
\pgfpathlineto{\pgfqpoint{6.105749in}{0.605719in}}%
\pgfpathlineto{\pgfqpoint{6.106304in}{0.603034in}}%
\pgfpathlineto{\pgfqpoint{6.107416in}{0.600914in}}%
\pgfpathlineto{\pgfqpoint{6.107972in}{0.605530in}}%
\pgfpathlineto{\pgfqpoint{6.108528in}{0.602911in}}%
\pgfpathlineto{\pgfqpoint{6.109084in}{0.600574in}}%
\pgfpathlineto{\pgfqpoint{6.109640in}{0.602010in}}%
\pgfpathlineto{\pgfqpoint{6.110751in}{0.603999in}}%
\pgfpathlineto{\pgfqpoint{6.111307in}{0.601181in}}%
\pgfpathlineto{\pgfqpoint{6.111863in}{0.602093in}}%
\pgfpathlineto{\pgfqpoint{6.114642in}{0.602529in}}%
\pgfpathlineto{\pgfqpoint{6.115198in}{0.600815in}}%
\pgfpathlineto{\pgfqpoint{6.115754in}{0.602031in}}%
\pgfpathlineto{\pgfqpoint{6.117977in}{0.602740in}}%
\pgfpathlineto{\pgfqpoint{6.118533in}{0.600341in}}%
\pgfpathlineto{\pgfqpoint{6.119089in}{0.601371in}}%
\pgfpathlineto{\pgfqpoint{6.120757in}{0.603470in}}%
\pgfpathlineto{\pgfqpoint{6.121313in}{0.603081in}}%
\pgfpathlineto{\pgfqpoint{6.123536in}{0.600482in}}%
\pgfpathlineto{\pgfqpoint{6.124648in}{0.604997in}}%
\pgfpathlineto{\pgfqpoint{6.125204in}{0.602532in}}%
\pgfpathlineto{\pgfqpoint{6.125760in}{0.600068in}}%
\pgfpathlineto{\pgfqpoint{6.126315in}{0.600959in}}%
\pgfpathlineto{\pgfqpoint{6.126871in}{0.601036in}}%
\pgfpathlineto{\pgfqpoint{6.127983in}{0.602703in}}%
\pgfpathlineto{\pgfqpoint{6.129651in}{0.601451in}}%
\pgfpathlineto{\pgfqpoint{6.131318in}{0.602837in}}%
\pgfpathlineto{\pgfqpoint{6.131874in}{0.601343in}}%
\pgfpathlineto{\pgfqpoint{6.132430in}{0.602470in}}%
\pgfpathlineto{\pgfqpoint{6.134653in}{0.601853in}}%
\pgfpathlineto{\pgfqpoint{6.137433in}{0.603525in}}%
\pgfpathlineto{\pgfqpoint{6.137988in}{0.600743in}}%
\pgfpathlineto{\pgfqpoint{6.138544in}{0.602589in}}%
\pgfpathlineto{\pgfqpoint{6.140212in}{0.600913in}}%
\pgfpathlineto{\pgfqpoint{6.141324in}{0.602093in}}%
\pgfpathlineto{\pgfqpoint{6.141880in}{0.600346in}}%
\pgfpathlineto{\pgfqpoint{6.142435in}{0.602927in}}%
\pgfpathlineto{\pgfqpoint{6.142991in}{0.601077in}}%
\pgfpathlineto{\pgfqpoint{6.143547in}{0.601008in}}%
\pgfpathlineto{\pgfqpoint{6.144103in}{0.603224in}}%
\pgfpathlineto{\pgfqpoint{6.144659in}{0.600934in}}%
\pgfpathlineto{\pgfqpoint{6.147438in}{0.602744in}}%
\pgfpathlineto{\pgfqpoint{6.149106in}{0.601092in}}%
\pgfpathlineto{\pgfqpoint{6.150217in}{0.605258in}}%
\pgfpathlineto{\pgfqpoint{6.151885in}{0.601283in}}%
\pgfpathlineto{\pgfqpoint{6.152441in}{0.601150in}}%
\pgfpathlineto{\pgfqpoint{6.152997in}{0.602629in}}%
\pgfpathlineto{\pgfqpoint{6.153553in}{0.602004in}}%
\pgfpathlineto{\pgfqpoint{6.154664in}{0.601317in}}%
\pgfpathlineto{\pgfqpoint{6.156222in}{0.602235in}}%
\pgfpathmoveto{\pgfqpoint{6.156222in}{0.599981in}}%
\pgfpathlineto{\pgfqpoint{0.707889in}{0.599984in}}%
\pgfpathmoveto{\pgfqpoint{0.707889in}{0.602235in}}%
\pgfpathlineto{\pgfqpoint{0.709446in}{0.601317in}}%
\pgfpathlineto{\pgfqpoint{0.711114in}{0.602629in}}%
\pgfpathlineto{\pgfqpoint{0.712226in}{0.601283in}}%
\pgfpathlineto{\pgfqpoint{0.713893in}{0.605258in}}%
\pgfpathlineto{\pgfqpoint{0.714449in}{0.603170in}}%
\pgfpathlineto{\pgfqpoint{0.715561in}{0.601232in}}%
\pgfpathlineto{\pgfqpoint{0.717784in}{0.602129in}}%
\pgfpathlineto{\pgfqpoint{0.719452in}{0.600934in}}%
\pgfpathlineto{\pgfqpoint{0.720008in}{0.603224in}}%
\pgfpathlineto{\pgfqpoint{0.720564in}{0.601008in}}%
\pgfpathlineto{\pgfqpoint{0.721120in}{0.601077in}}%
\pgfpathlineto{\pgfqpoint{0.721675in}{0.602927in}}%
\pgfpathlineto{\pgfqpoint{0.722231in}{0.600346in}}%
\pgfpathlineto{\pgfqpoint{0.722787in}{0.602093in}}%
\pgfpathlineto{\pgfqpoint{0.725011in}{0.601638in}}%
\pgfpathlineto{\pgfqpoint{0.725566in}{0.602589in}}%
\pgfpathlineto{\pgfqpoint{0.726122in}{0.600743in}}%
\pgfpathlineto{\pgfqpoint{0.726678in}{0.603525in}}%
\pgfpathlineto{\pgfqpoint{0.727234in}{0.602780in}}%
\pgfpathlineto{\pgfqpoint{0.729457in}{0.601853in}}%
\pgfpathlineto{\pgfqpoint{0.735016in}{0.602629in}}%
\pgfpathlineto{\pgfqpoint{0.735572in}{0.601280in}}%
\pgfpathlineto{\pgfqpoint{0.736128in}{0.602703in}}%
\pgfpathlineto{\pgfqpoint{0.736684in}{0.602483in}}%
\pgfpathlineto{\pgfqpoint{0.738351in}{0.600068in}}%
\pgfpathlineto{\pgfqpoint{0.739463in}{0.604997in}}%
\pgfpathlineto{\pgfqpoint{0.740575in}{0.600482in}}%
\pgfpathlineto{\pgfqpoint{0.742242in}{0.602335in}}%
\pgfpathlineto{\pgfqpoint{0.743910in}{0.603214in}}%
\pgfpathlineto{\pgfqpoint{0.745577in}{0.600341in}}%
\pgfpathlineto{\pgfqpoint{0.746689in}{0.603140in}}%
\pgfpathlineto{\pgfqpoint{0.747245in}{0.601497in}}%
\pgfpathlineto{\pgfqpoint{0.747801in}{0.602380in}}%
\pgfpathlineto{\pgfqpoint{0.748913in}{0.600815in}}%
\pgfpathlineto{\pgfqpoint{0.749468in}{0.602529in}}%
\pgfpathlineto{\pgfqpoint{0.750580in}{0.602364in}}%
\pgfpathlineto{\pgfqpoint{0.752804in}{0.601181in}}%
\pgfpathlineto{\pgfqpoint{0.753360in}{0.603999in}}%
\pgfpathlineto{\pgfqpoint{0.753915in}{0.602530in}}%
\pgfpathlineto{\pgfqpoint{0.755027in}{0.600574in}}%
\pgfpathlineto{\pgfqpoint{0.756139in}{0.605530in}}%
\pgfpathlineto{\pgfqpoint{0.756695in}{0.600914in}}%
\pgfpathlineto{\pgfqpoint{0.757251in}{0.602592in}}%
\pgfpathlineto{\pgfqpoint{0.757806in}{0.603034in}}%
\pgfpathlineto{\pgfqpoint{0.758362in}{0.605719in}}%
\pgfpathlineto{\pgfqpoint{0.758918in}{0.601463in}}%
\pgfpathlineto{\pgfqpoint{0.759474in}{0.603303in}}%
\pgfpathlineto{\pgfqpoint{0.760030in}{0.606076in}}%
\pgfpathlineto{\pgfqpoint{0.760586in}{0.603803in}}%
\pgfpathlineto{\pgfqpoint{0.761697in}{0.602441in}}%
\pgfpathlineto{\pgfqpoint{0.762253in}{0.603949in}}%
\pgfpathlineto{\pgfqpoint{0.762809in}{0.600555in}}%
\pgfpathlineto{\pgfqpoint{0.763365in}{0.601072in}}%
\pgfpathlineto{\pgfqpoint{0.765033in}{0.604628in}}%
\pgfpathlineto{\pgfqpoint{0.765588in}{0.601671in}}%
\pgfpathlineto{\pgfqpoint{0.766144in}{0.602589in}}%
\pgfpathlineto{\pgfqpoint{0.766700in}{0.602477in}}%
\pgfpathlineto{\pgfqpoint{0.767256in}{0.605232in}}%
\pgfpathlineto{\pgfqpoint{0.767812in}{0.602755in}}%
\pgfpathlineto{\pgfqpoint{0.768368in}{0.601230in}}%
\pgfpathlineto{\pgfqpoint{0.768924in}{0.601890in}}%
\pgfpathlineto{\pgfqpoint{0.770035in}{0.600757in}}%
\pgfpathlineto{\pgfqpoint{0.770591in}{0.604700in}}%
\pgfpathlineto{\pgfqpoint{0.771147in}{0.601797in}}%
\pgfpathlineto{\pgfqpoint{0.771703in}{0.600401in}}%
\pgfpathlineto{\pgfqpoint{0.772259in}{0.600840in}}%
\pgfpathlineto{\pgfqpoint{0.772815in}{0.603681in}}%
\pgfpathlineto{\pgfqpoint{0.773371in}{0.600896in}}%
\pgfpathlineto{\pgfqpoint{0.777262in}{0.602480in}}%
\pgfpathlineto{\pgfqpoint{0.777817in}{0.601281in}}%
\pgfpathlineto{\pgfqpoint{0.778373in}{0.601674in}}%
\pgfpathlineto{\pgfqpoint{0.778929in}{0.603553in}}%
\pgfpathlineto{\pgfqpoint{0.779485in}{0.600839in}}%
\pgfpathlineto{\pgfqpoint{0.780041in}{0.601865in}}%
\pgfpathlineto{\pgfqpoint{0.780597in}{0.604934in}}%
\pgfpathlineto{\pgfqpoint{0.781153in}{0.603223in}}%
\pgfpathlineto{\pgfqpoint{0.782820in}{0.601230in}}%
\pgfpathlineto{\pgfqpoint{0.783376in}{0.603434in}}%
\pgfpathlineto{\pgfqpoint{0.783932in}{0.603095in}}%
\pgfpathlineto{\pgfqpoint{0.784488in}{0.601195in}}%
\pgfpathlineto{\pgfqpoint{0.785044in}{0.602762in}}%
\pgfpathlineto{\pgfqpoint{0.785599in}{0.602564in}}%
\pgfpathlineto{\pgfqpoint{0.787267in}{0.600407in}}%
\pgfpathlineto{\pgfqpoint{0.788379in}{0.601518in}}%
\pgfpathlineto{\pgfqpoint{0.788935in}{0.600430in}}%
\pgfpathlineto{\pgfqpoint{0.790602in}{0.602778in}}%
\pgfpathlineto{\pgfqpoint{0.793382in}{0.601385in}}%
\pgfpathlineto{\pgfqpoint{0.793937in}{0.602584in}}%
\pgfpathlineto{\pgfqpoint{0.794493in}{0.601025in}}%
\pgfpathlineto{\pgfqpoint{0.795049in}{0.601752in}}%
\pgfpathlineto{\pgfqpoint{0.795605in}{0.603177in}}%
\pgfpathlineto{\pgfqpoint{0.796161in}{0.602761in}}%
\pgfpathlineto{\pgfqpoint{0.796717in}{0.602766in}}%
\pgfpathlineto{\pgfqpoint{0.797828in}{0.600500in}}%
\pgfpathlineto{\pgfqpoint{0.800052in}{0.604382in}}%
\pgfpathlineto{\pgfqpoint{0.801164in}{0.601175in}}%
\pgfpathlineto{\pgfqpoint{0.802275in}{0.604388in}}%
\pgfpathlineto{\pgfqpoint{0.802831in}{0.602334in}}%
\pgfpathlineto{\pgfqpoint{0.803387in}{0.603973in}}%
\pgfpathlineto{\pgfqpoint{0.805055in}{0.602189in}}%
\pgfpathlineto{\pgfqpoint{0.805610in}{0.600988in}}%
\pgfpathlineto{\pgfqpoint{0.806166in}{0.603336in}}%
\pgfpathlineto{\pgfqpoint{0.806722in}{0.602741in}}%
\pgfpathlineto{\pgfqpoint{0.808390in}{0.600377in}}%
\pgfpathlineto{\pgfqpoint{0.808946in}{0.602708in}}%
\pgfpathlineto{\pgfqpoint{0.809502in}{0.601115in}}%
\pgfpathlineto{\pgfqpoint{0.811725in}{0.601653in}}%
\pgfpathlineto{\pgfqpoint{0.812837in}{0.601346in}}%
\pgfpathlineto{\pgfqpoint{0.813393in}{0.602372in}}%
\pgfpathlineto{\pgfqpoint{0.813948in}{0.601022in}}%
\pgfpathlineto{\pgfqpoint{0.815060in}{0.605485in}}%
\pgfpathlineto{\pgfqpoint{0.816728in}{0.600155in}}%
\pgfpathlineto{\pgfqpoint{0.817839in}{0.603111in}}%
\pgfpathlineto{\pgfqpoint{0.818395in}{0.602344in}}%
\pgfpathlineto{\pgfqpoint{0.820063in}{0.603677in}}%
\pgfpathlineto{\pgfqpoint{0.821175in}{0.601284in}}%
\pgfpathlineto{\pgfqpoint{0.821730in}{0.602244in}}%
\pgfpathlineto{\pgfqpoint{0.822286in}{0.601159in}}%
\pgfpathlineto{\pgfqpoint{0.823954in}{0.601719in}}%
\pgfpathlineto{\pgfqpoint{0.825622in}{0.602925in}}%
\pgfpathlineto{\pgfqpoint{0.826177in}{0.600358in}}%
\pgfpathlineto{\pgfqpoint{0.826733in}{0.600491in}}%
\pgfpathlineto{\pgfqpoint{0.828401in}{0.603422in}}%
\pgfpathlineto{\pgfqpoint{0.828957in}{0.600378in}}%
\pgfpathlineto{\pgfqpoint{0.829513in}{0.601134in}}%
\pgfpathlineto{\pgfqpoint{0.830068in}{0.600969in}}%
\pgfpathlineto{\pgfqpoint{0.830624in}{0.602268in}}%
\pgfpathlineto{\pgfqpoint{0.831180in}{0.601434in}}%
\pgfpathlineto{\pgfqpoint{0.832292in}{0.600788in}}%
\pgfpathlineto{\pgfqpoint{0.834515in}{0.604938in}}%
\pgfpathlineto{\pgfqpoint{0.835627in}{0.600394in}}%
\pgfpathlineto{\pgfqpoint{0.837295in}{0.604601in}}%
\pgfpathlineto{\pgfqpoint{0.838962in}{0.600540in}}%
\pgfpathlineto{\pgfqpoint{0.839518in}{0.603631in}}%
\pgfpathlineto{\pgfqpoint{0.840074in}{0.603116in}}%
\pgfpathlineto{\pgfqpoint{0.841741in}{0.600920in}}%
\pgfpathlineto{\pgfqpoint{0.842297in}{0.603900in}}%
\pgfpathlineto{\pgfqpoint{0.842853in}{0.602034in}}%
\pgfpathlineto{\pgfqpoint{0.843409in}{0.600551in}}%
\pgfpathlineto{\pgfqpoint{0.843965in}{0.601253in}}%
\pgfpathlineto{\pgfqpoint{0.845077in}{0.603581in}}%
\pgfpathlineto{\pgfqpoint{0.845633in}{0.600278in}}%
\pgfpathlineto{\pgfqpoint{0.846188in}{0.603453in}}%
\pgfpathlineto{\pgfqpoint{0.847856in}{0.600612in}}%
\pgfpathlineto{\pgfqpoint{0.848412in}{0.602967in}}%
\pgfpathlineto{\pgfqpoint{0.848968in}{0.602340in}}%
\pgfpathlineto{\pgfqpoint{0.850079in}{0.601226in}}%
\pgfpathlineto{\pgfqpoint{0.851191in}{0.603797in}}%
\pgfpathlineto{\pgfqpoint{0.852859in}{0.601528in}}%
\pgfpathlineto{\pgfqpoint{0.854526in}{0.602762in}}%
\pgfpathlineto{\pgfqpoint{0.855082in}{0.601049in}}%
\pgfpathlineto{\pgfqpoint{0.856194in}{0.603271in}}%
\pgfpathlineto{\pgfqpoint{0.858417in}{0.601411in}}%
\pgfpathlineto{\pgfqpoint{0.860085in}{0.602706in}}%
\pgfpathlineto{\pgfqpoint{0.861752in}{0.602427in}}%
\pgfpathlineto{\pgfqpoint{0.863420in}{0.602408in}}%
\pgfpathlineto{\pgfqpoint{0.865088in}{0.600411in}}%
\pgfpathlineto{\pgfqpoint{0.865644in}{0.603221in}}%
\pgfpathlineto{\pgfqpoint{0.866199in}{0.602106in}}%
\pgfpathlineto{\pgfqpoint{0.868979in}{0.604386in}}%
\pgfpathlineto{\pgfqpoint{0.869535in}{0.602825in}}%
\pgfpathlineto{\pgfqpoint{0.870090in}{0.601364in}}%
\pgfpathlineto{\pgfqpoint{0.870646in}{0.602228in}}%
\pgfpathlineto{\pgfqpoint{0.872870in}{0.601446in}}%
\pgfpathlineto{\pgfqpoint{0.873426in}{0.602925in}}%
\pgfpathlineto{\pgfqpoint{0.873981in}{0.601369in}}%
\pgfpathlineto{\pgfqpoint{0.874537in}{0.602696in}}%
\pgfpathlineto{\pgfqpoint{0.875093in}{0.601177in}}%
\pgfpathlineto{\pgfqpoint{0.875649in}{0.600322in}}%
\pgfpathlineto{\pgfqpoint{0.876205in}{0.603729in}}%
\pgfpathlineto{\pgfqpoint{0.876761in}{0.601970in}}%
\pgfpathlineto{\pgfqpoint{0.877872in}{0.603853in}}%
\pgfpathlineto{\pgfqpoint{0.880096in}{0.601309in}}%
\pgfpathlineto{\pgfqpoint{0.881208in}{0.602268in}}%
\pgfpathlineto{\pgfqpoint{0.881764in}{0.600553in}}%
\pgfpathlineto{\pgfqpoint{0.882875in}{0.606410in}}%
\pgfpathlineto{\pgfqpoint{0.883431in}{0.600987in}}%
\pgfpathlineto{\pgfqpoint{0.883987in}{0.601424in}}%
\pgfpathlineto{\pgfqpoint{0.885099in}{0.603929in}}%
\pgfpathlineto{\pgfqpoint{0.885655in}{0.602250in}}%
\pgfpathlineto{\pgfqpoint{0.886210in}{0.600796in}}%
\pgfpathlineto{\pgfqpoint{0.886766in}{0.601951in}}%
\pgfpathlineto{\pgfqpoint{0.887322in}{0.601748in}}%
\pgfpathlineto{\pgfqpoint{0.888434in}{0.604232in}}%
\pgfpathlineto{\pgfqpoint{0.888990in}{0.601481in}}%
\pgfpathlineto{\pgfqpoint{0.889546in}{0.602807in}}%
\pgfpathlineto{\pgfqpoint{0.892881in}{0.600967in}}%
\pgfpathlineto{\pgfqpoint{0.893437in}{0.603369in}}%
\pgfpathlineto{\pgfqpoint{0.893992in}{0.601882in}}%
\pgfpathlineto{\pgfqpoint{0.895660in}{0.601223in}}%
\pgfpathlineto{\pgfqpoint{0.897328in}{0.603585in}}%
\pgfpathlineto{\pgfqpoint{0.898439in}{0.601543in}}%
\pgfpathlineto{\pgfqpoint{0.898995in}{0.602185in}}%
\pgfpathlineto{\pgfqpoint{0.899551in}{0.602393in}}%
\pgfpathlineto{\pgfqpoint{0.900107in}{0.604666in}}%
\pgfpathlineto{\pgfqpoint{0.900663in}{0.602409in}}%
\pgfpathlineto{\pgfqpoint{0.902330in}{0.603711in}}%
\pgfpathlineto{\pgfqpoint{0.903442in}{0.601039in}}%
\pgfpathlineto{\pgfqpoint{0.903998in}{0.601705in}}%
\pgfpathlineto{\pgfqpoint{0.905110in}{0.603216in}}%
\pgfpathlineto{\pgfqpoint{0.907333in}{0.601913in}}%
\pgfpathlineto{\pgfqpoint{0.907889in}{0.603295in}}%
\pgfpathlineto{\pgfqpoint{0.908445in}{0.601268in}}%
\pgfpathlineto{\pgfqpoint{0.909001in}{0.601935in}}%
\pgfpathlineto{\pgfqpoint{0.910668in}{0.604539in}}%
\pgfpathlineto{\pgfqpoint{0.911224in}{0.601580in}}%
\pgfpathlineto{\pgfqpoint{0.911780in}{0.603078in}}%
\pgfpathlineto{\pgfqpoint{0.914003in}{0.599956in}}%
\pgfpathlineto{\pgfqpoint{0.915115in}{0.607037in}}%
\pgfpathlineto{\pgfqpoint{0.915671in}{0.601437in}}%
\pgfpathlineto{\pgfqpoint{0.916227in}{0.602559in}}%
\pgfpathlineto{\pgfqpoint{0.916783in}{0.603776in}}%
\pgfpathlineto{\pgfqpoint{0.917339in}{0.603304in}}%
\pgfpathlineto{\pgfqpoint{0.918450in}{0.601101in}}%
\pgfpathlineto{\pgfqpoint{0.919562in}{0.602311in}}%
\pgfpathlineto{\pgfqpoint{0.920118in}{0.600701in}}%
\pgfpathlineto{\pgfqpoint{0.920674in}{0.601736in}}%
\pgfpathlineto{\pgfqpoint{0.922341in}{0.603427in}}%
\pgfpathlineto{\pgfqpoint{0.922897in}{0.600216in}}%
\pgfpathlineto{\pgfqpoint{0.923453in}{0.604400in}}%
\pgfpathlineto{\pgfqpoint{0.924009in}{0.601065in}}%
\pgfpathlineto{\pgfqpoint{0.924565in}{0.602591in}}%
\pgfpathlineto{\pgfqpoint{0.925121in}{0.601804in}}%
\pgfpathlineto{\pgfqpoint{0.925677in}{0.601754in}}%
\pgfpathlineto{\pgfqpoint{0.926232in}{0.607614in}}%
\pgfpathlineto{\pgfqpoint{0.926788in}{0.602096in}}%
\pgfpathlineto{\pgfqpoint{0.927900in}{0.602316in}}%
\pgfpathlineto{\pgfqpoint{0.928456in}{0.603494in}}%
\pgfpathlineto{\pgfqpoint{0.930123in}{0.601643in}}%
\pgfpathlineto{\pgfqpoint{0.930679in}{0.601073in}}%
\pgfpathlineto{\pgfqpoint{0.932347in}{0.605420in}}%
\pgfpathlineto{\pgfqpoint{0.932903in}{0.602343in}}%
\pgfpathlineto{\pgfqpoint{0.933459in}{0.606309in}}%
\pgfpathlineto{\pgfqpoint{0.934014in}{0.601848in}}%
\pgfpathlineto{\pgfqpoint{0.934570in}{0.603934in}}%
\pgfpathlineto{\pgfqpoint{0.935126in}{0.605118in}}%
\pgfpathlineto{\pgfqpoint{0.935682in}{0.602824in}}%
\pgfpathlineto{\pgfqpoint{0.936238in}{0.608350in}}%
\pgfpathlineto{\pgfqpoint{0.936794in}{0.604058in}}%
\pgfpathlineto{\pgfqpoint{0.939017in}{0.602287in}}%
\pgfpathlineto{\pgfqpoint{0.939573in}{0.607049in}}%
\pgfpathlineto{\pgfqpoint{0.940129in}{0.603024in}}%
\pgfpathlineto{\pgfqpoint{0.941241in}{0.601235in}}%
\pgfpathlineto{\pgfqpoint{0.941797in}{0.603963in}}%
\pgfpathlineto{\pgfqpoint{0.942352in}{0.601090in}}%
\pgfpathlineto{\pgfqpoint{0.942908in}{0.601706in}}%
\pgfpathlineto{\pgfqpoint{0.944576in}{0.606452in}}%
\pgfpathlineto{\pgfqpoint{0.945132in}{0.602237in}}%
\pgfpathlineto{\pgfqpoint{0.945688in}{0.605395in}}%
\pgfpathlineto{\pgfqpoint{0.946799in}{0.602711in}}%
\pgfpathlineto{\pgfqpoint{0.949023in}{0.603639in}}%
\pgfpathlineto{\pgfqpoint{0.949579in}{0.600760in}}%
\pgfpathlineto{\pgfqpoint{0.950134in}{0.603355in}}%
\pgfpathlineto{\pgfqpoint{0.950690in}{0.603743in}}%
\pgfpathlineto{\pgfqpoint{0.952358in}{0.600816in}}%
\pgfpathlineto{\pgfqpoint{0.952914in}{0.601816in}}%
\pgfpathlineto{\pgfqpoint{0.953470in}{0.601229in}}%
\pgfpathlineto{\pgfqpoint{0.954025in}{0.600641in}}%
\pgfpathlineto{\pgfqpoint{0.954581in}{0.602383in}}%
\pgfpathlineto{\pgfqpoint{0.955137in}{0.602157in}}%
\pgfpathlineto{\pgfqpoint{0.955693in}{0.600592in}}%
\pgfpathlineto{\pgfqpoint{0.956249in}{0.601774in}}%
\pgfpathlineto{\pgfqpoint{0.956805in}{0.604183in}}%
\pgfpathlineto{\pgfqpoint{0.957361in}{0.600965in}}%
\pgfpathlineto{\pgfqpoint{0.957917in}{0.602777in}}%
\pgfpathlineto{\pgfqpoint{0.959584in}{0.602335in}}%
\pgfpathlineto{\pgfqpoint{0.960140in}{0.604680in}}%
\pgfpathlineto{\pgfqpoint{0.960696in}{0.602942in}}%
\pgfpathlineto{\pgfqpoint{0.961252in}{0.601532in}}%
\pgfpathlineto{\pgfqpoint{0.961808in}{0.602468in}}%
\pgfpathlineto{\pgfqpoint{0.962363in}{0.602429in}}%
\pgfpathlineto{\pgfqpoint{0.963475in}{0.604484in}}%
\pgfpathlineto{\pgfqpoint{0.964031in}{0.604185in}}%
\pgfpathlineto{\pgfqpoint{0.964587in}{0.605201in}}%
\pgfpathlineto{\pgfqpoint{0.965143in}{0.600724in}}%
\pgfpathlineto{\pgfqpoint{0.965699in}{0.604654in}}%
\pgfpathlineto{\pgfqpoint{0.967366in}{0.601472in}}%
\pgfpathlineto{\pgfqpoint{0.969034in}{0.604788in}}%
\pgfpathlineto{\pgfqpoint{0.970145in}{0.600687in}}%
\pgfpathlineto{\pgfqpoint{0.970701in}{0.603000in}}%
\pgfpathlineto{\pgfqpoint{0.971257in}{0.602235in}}%
\pgfpathlineto{\pgfqpoint{0.971813in}{0.602056in}}%
\pgfpathlineto{\pgfqpoint{0.972369in}{0.605645in}}%
\pgfpathlineto{\pgfqpoint{0.972925in}{0.602009in}}%
\pgfpathlineto{\pgfqpoint{0.973481in}{0.603872in}}%
\pgfpathlineto{\pgfqpoint{0.974036in}{0.603131in}}%
\pgfpathlineto{\pgfqpoint{0.974592in}{0.601070in}}%
\pgfpathlineto{\pgfqpoint{0.975148in}{0.602296in}}%
\pgfpathlineto{\pgfqpoint{0.977372in}{0.601136in}}%
\pgfpathlineto{\pgfqpoint{0.977928in}{0.602873in}}%
\pgfpathlineto{\pgfqpoint{0.978483in}{0.601900in}}%
\pgfpathlineto{\pgfqpoint{0.980151in}{0.600914in}}%
\pgfpathlineto{\pgfqpoint{0.980707in}{0.603561in}}%
\pgfpathlineto{\pgfqpoint{0.981263in}{0.601961in}}%
\pgfpathlineto{\pgfqpoint{0.982930in}{0.601146in}}%
\pgfpathlineto{\pgfqpoint{0.983486in}{0.606055in}}%
\pgfpathlineto{\pgfqpoint{0.984042in}{0.602592in}}%
\pgfpathlineto{\pgfqpoint{0.985154in}{0.600136in}}%
\pgfpathlineto{\pgfqpoint{0.987377in}{0.606007in}}%
\pgfpathlineto{\pgfqpoint{0.988489in}{0.600976in}}%
\pgfpathlineto{\pgfqpoint{0.989601in}{0.601819in}}%
\pgfpathlineto{\pgfqpoint{0.990156in}{0.602329in}}%
\pgfpathlineto{\pgfqpoint{0.990712in}{0.604873in}}%
\pgfpathlineto{\pgfqpoint{0.991268in}{0.603544in}}%
\pgfpathlineto{\pgfqpoint{0.991824in}{0.603347in}}%
\pgfpathlineto{\pgfqpoint{0.993492in}{0.605986in}}%
\pgfpathlineto{\pgfqpoint{0.994603in}{0.600969in}}%
\pgfpathlineto{\pgfqpoint{0.995159in}{0.601902in}}%
\pgfpathlineto{\pgfqpoint{0.996827in}{0.604154in}}%
\pgfpathlineto{\pgfqpoint{0.997939in}{0.600604in}}%
\pgfpathlineto{\pgfqpoint{0.999606in}{0.603658in}}%
\pgfpathlineto{\pgfqpoint{1.000162in}{0.601586in}}%
\pgfpathlineto{\pgfqpoint{1.000718in}{0.603790in}}%
\pgfpathlineto{\pgfqpoint{1.001274in}{0.603630in}}%
\pgfpathlineto{\pgfqpoint{1.001830in}{0.602228in}}%
\pgfpathlineto{\pgfqpoint{1.002385in}{0.603262in}}%
\pgfpathlineto{\pgfqpoint{1.002941in}{0.603233in}}%
\pgfpathlineto{\pgfqpoint{1.003497in}{0.601980in}}%
\pgfpathlineto{\pgfqpoint{1.004053in}{0.602803in}}%
\pgfpathlineto{\pgfqpoint{1.004609in}{0.603271in}}%
\pgfpathlineto{\pgfqpoint{1.005165in}{0.600923in}}%
\pgfpathlineto{\pgfqpoint{1.005721in}{0.601717in}}%
\pgfpathlineto{\pgfqpoint{1.006276in}{0.602797in}}%
\pgfpathlineto{\pgfqpoint{1.006832in}{0.601454in}}%
\pgfpathlineto{\pgfqpoint{1.007944in}{0.601502in}}%
\pgfpathlineto{\pgfqpoint{1.008500in}{0.604595in}}%
\pgfpathlineto{\pgfqpoint{1.009056in}{0.602809in}}%
\pgfpathlineto{\pgfqpoint{1.009612in}{0.600732in}}%
\pgfpathlineto{\pgfqpoint{1.010167in}{0.602337in}}%
\pgfpathlineto{\pgfqpoint{1.011279in}{0.601504in}}%
\pgfpathlineto{\pgfqpoint{1.012391in}{0.604668in}}%
\pgfpathlineto{\pgfqpoint{1.012947in}{0.601137in}}%
\pgfpathlineto{\pgfqpoint{1.013503in}{0.603652in}}%
\pgfpathlineto{\pgfqpoint{1.014059in}{0.603485in}}%
\pgfpathlineto{\pgfqpoint{1.015170in}{0.601752in}}%
\pgfpathlineto{\pgfqpoint{1.015726in}{0.603584in}}%
\pgfpathlineto{\pgfqpoint{1.016282in}{0.603315in}}%
\pgfpathlineto{\pgfqpoint{1.016838in}{0.603266in}}%
\pgfpathlineto{\pgfqpoint{1.017394in}{0.601513in}}%
\pgfpathlineto{\pgfqpoint{1.017950in}{0.603810in}}%
\pgfpathlineto{\pgfqpoint{1.018505in}{0.602340in}}%
\pgfpathlineto{\pgfqpoint{1.020173in}{0.602389in}}%
\pgfpathlineto{\pgfqpoint{1.021285in}{0.602154in}}%
\pgfpathlineto{\pgfqpoint{1.021841in}{0.603174in}}%
\pgfpathlineto{\pgfqpoint{1.022396in}{0.600741in}}%
\pgfpathlineto{\pgfqpoint{1.024064in}{0.605046in}}%
\pgfpathlineto{\pgfqpoint{1.025176in}{0.601233in}}%
\pgfpathlineto{\pgfqpoint{1.026843in}{0.604702in}}%
\pgfpathlineto{\pgfqpoint{1.028511in}{0.600987in}}%
\pgfpathlineto{\pgfqpoint{1.029623in}{0.606609in}}%
\pgfpathlineto{\pgfqpoint{1.030178in}{0.603590in}}%
\pgfpathlineto{\pgfqpoint{1.030734in}{0.602181in}}%
\pgfpathlineto{\pgfqpoint{1.032402in}{0.604622in}}%
\pgfpathlineto{\pgfqpoint{1.034625in}{0.600565in}}%
\pgfpathlineto{\pgfqpoint{1.035181in}{0.602441in}}%
\pgfpathlineto{\pgfqpoint{1.035737in}{0.600493in}}%
\pgfpathlineto{\pgfqpoint{1.036849in}{0.602048in}}%
\pgfpathlineto{\pgfqpoint{1.037405in}{0.600125in}}%
\pgfpathlineto{\pgfqpoint{1.037961in}{0.604489in}}%
\pgfpathlineto{\pgfqpoint{1.038516in}{0.601834in}}%
\pgfpathlineto{\pgfqpoint{1.041296in}{0.604709in}}%
\pgfpathlineto{\pgfqpoint{1.041852in}{0.601053in}}%
\pgfpathlineto{\pgfqpoint{1.042407in}{0.602124in}}%
\pgfpathlineto{\pgfqpoint{1.044631in}{0.600852in}}%
\pgfpathlineto{\pgfqpoint{1.046298in}{0.602486in}}%
\pgfpathlineto{\pgfqpoint{1.046854in}{0.602683in}}%
\pgfpathlineto{\pgfqpoint{1.047966in}{0.600215in}}%
\pgfpathlineto{\pgfqpoint{1.048522in}{0.605161in}}%
\pgfpathlineto{\pgfqpoint{1.049078in}{0.602798in}}%
\pgfpathlineto{\pgfqpoint{1.050745in}{0.601148in}}%
\pgfpathlineto{\pgfqpoint{1.052413in}{0.606878in}}%
\pgfpathlineto{\pgfqpoint{1.054081in}{0.601166in}}%
\pgfpathlineto{\pgfqpoint{1.054636in}{0.602695in}}%
\pgfpathlineto{\pgfqpoint{1.055192in}{0.601585in}}%
\pgfpathlineto{\pgfqpoint{1.055748in}{0.600826in}}%
\pgfpathlineto{\pgfqpoint{1.056860in}{0.603470in}}%
\pgfpathlineto{\pgfqpoint{1.057416in}{0.601984in}}%
\pgfpathlineto{\pgfqpoint{1.059083in}{0.602795in}}%
\pgfpathlineto{\pgfqpoint{1.059639in}{0.602936in}}%
\pgfpathlineto{\pgfqpoint{1.060195in}{0.604449in}}%
\pgfpathlineto{\pgfqpoint{1.061307in}{0.601766in}}%
\pgfpathlineto{\pgfqpoint{1.062974in}{0.604051in}}%
\pgfpathlineto{\pgfqpoint{1.063530in}{0.602398in}}%
\pgfpathlineto{\pgfqpoint{1.064086in}{0.603681in}}%
\pgfpathlineto{\pgfqpoint{1.065754in}{0.603221in}}%
\pgfpathlineto{\pgfqpoint{1.066309in}{0.601041in}}%
\pgfpathlineto{\pgfqpoint{1.066865in}{0.601944in}}%
\pgfpathlineto{\pgfqpoint{1.067977in}{0.601194in}}%
\pgfpathlineto{\pgfqpoint{1.069645in}{0.600899in}}%
\pgfpathlineto{\pgfqpoint{1.070201in}{0.601589in}}%
\pgfpathlineto{\pgfqpoint{1.070756in}{0.604148in}}%
\pgfpathlineto{\pgfqpoint{1.071312in}{0.602401in}}%
\pgfpathlineto{\pgfqpoint{1.071868in}{0.601412in}}%
\pgfpathlineto{\pgfqpoint{1.072424in}{0.603038in}}%
\pgfpathlineto{\pgfqpoint{1.072980in}{0.600837in}}%
\pgfpathlineto{\pgfqpoint{1.073536in}{0.603907in}}%
\pgfpathlineto{\pgfqpoint{1.074092in}{0.602588in}}%
\pgfpathlineto{\pgfqpoint{1.075203in}{0.603557in}}%
\pgfpathlineto{\pgfqpoint{1.076315in}{0.600076in}}%
\pgfpathlineto{\pgfqpoint{1.077983in}{0.602775in}}%
\pgfpathlineto{\pgfqpoint{1.078538in}{0.602815in}}%
\pgfpathlineto{\pgfqpoint{1.079650in}{0.604886in}}%
\pgfpathlineto{\pgfqpoint{1.080206in}{0.600989in}}%
\pgfpathlineto{\pgfqpoint{1.080762in}{0.604335in}}%
\pgfpathlineto{\pgfqpoint{1.081318in}{0.602598in}}%
\pgfpathlineto{\pgfqpoint{1.081874in}{0.603731in}}%
\pgfpathlineto{\pgfqpoint{1.082429in}{0.604140in}}%
\pgfpathlineto{\pgfqpoint{1.082985in}{0.606447in}}%
\pgfpathlineto{\pgfqpoint{1.084097in}{0.600758in}}%
\pgfpathlineto{\pgfqpoint{1.084653in}{0.602262in}}%
\pgfpathlineto{\pgfqpoint{1.085765in}{0.602431in}}%
\pgfpathlineto{\pgfqpoint{1.086320in}{0.600952in}}%
\pgfpathlineto{\pgfqpoint{1.086876in}{0.603801in}}%
\pgfpathlineto{\pgfqpoint{1.087432in}{0.603083in}}%
\pgfpathlineto{\pgfqpoint{1.087988in}{0.601638in}}%
\pgfpathlineto{\pgfqpoint{1.088544in}{0.605307in}}%
\pgfpathlineto{\pgfqpoint{1.089656in}{0.601249in}}%
\pgfpathlineto{\pgfqpoint{1.090212in}{0.603625in}}%
\pgfpathlineto{\pgfqpoint{1.090767in}{0.602603in}}%
\pgfpathlineto{\pgfqpoint{1.092435in}{0.602281in}}%
\pgfpathlineto{\pgfqpoint{1.092991in}{0.600815in}}%
\pgfpathlineto{\pgfqpoint{1.093547in}{0.603913in}}%
\pgfpathlineto{\pgfqpoint{1.094103in}{0.600969in}}%
\pgfpathlineto{\pgfqpoint{1.096326in}{0.603551in}}%
\pgfpathlineto{\pgfqpoint{1.097438in}{0.605314in}}%
\pgfpathlineto{\pgfqpoint{1.097994in}{0.605904in}}%
\pgfpathlineto{\pgfqpoint{1.099105in}{0.601781in}}%
\pgfpathlineto{\pgfqpoint{1.099661in}{0.602102in}}%
\pgfpathlineto{\pgfqpoint{1.100217in}{0.602730in}}%
\pgfpathlineto{\pgfqpoint{1.100773in}{0.600591in}}%
\pgfpathlineto{\pgfqpoint{1.101329in}{0.604203in}}%
\pgfpathlineto{\pgfqpoint{1.101885in}{0.601205in}}%
\pgfpathlineto{\pgfqpoint{1.102440in}{0.604005in}}%
\pgfpathlineto{\pgfqpoint{1.102996in}{0.602614in}}%
\pgfpathlineto{\pgfqpoint{1.104664in}{0.600626in}}%
\pgfpathlineto{\pgfqpoint{1.106887in}{0.605237in}}%
\pgfpathlineto{\pgfqpoint{1.107443in}{0.602684in}}%
\pgfpathlineto{\pgfqpoint{1.107999in}{0.603678in}}%
\pgfpathlineto{\pgfqpoint{1.108555in}{0.604120in}}%
\pgfpathlineto{\pgfqpoint{1.110223in}{0.601755in}}%
\pgfpathlineto{\pgfqpoint{1.111890in}{0.603843in}}%
\pgfpathlineto{\pgfqpoint{1.112446in}{0.601275in}}%
\pgfpathlineto{\pgfqpoint{1.113002in}{0.603282in}}%
\pgfpathlineto{\pgfqpoint{1.113558in}{0.602496in}}%
\pgfpathlineto{\pgfqpoint{1.114114in}{0.604011in}}%
\pgfpathlineto{\pgfqpoint{1.114669in}{0.602413in}}%
\pgfpathlineto{\pgfqpoint{1.115225in}{0.602694in}}%
\pgfpathlineto{\pgfqpoint{1.116337in}{0.604453in}}%
\pgfpathlineto{\pgfqpoint{1.116893in}{0.600725in}}%
\pgfpathlineto{\pgfqpoint{1.117449in}{0.602166in}}%
\pgfpathlineto{\pgfqpoint{1.118005in}{0.604276in}}%
\pgfpathlineto{\pgfqpoint{1.119672in}{0.601047in}}%
\pgfpathlineto{\pgfqpoint{1.120228in}{0.601877in}}%
\pgfpathlineto{\pgfqpoint{1.120784in}{0.600505in}}%
\pgfpathlineto{\pgfqpoint{1.121340in}{0.601179in}}%
\pgfpathlineto{\pgfqpoint{1.123007in}{0.604895in}}%
\pgfpathlineto{\pgfqpoint{1.124119in}{0.601558in}}%
\pgfpathlineto{\pgfqpoint{1.124675in}{0.604197in}}%
\pgfpathlineto{\pgfqpoint{1.125231in}{0.602989in}}%
\pgfpathlineto{\pgfqpoint{1.126898in}{0.600549in}}%
\pgfpathlineto{\pgfqpoint{1.127454in}{0.602169in}}%
\pgfpathlineto{\pgfqpoint{1.128566in}{0.603458in}}%
\pgfpathlineto{\pgfqpoint{1.130789in}{0.600487in}}%
\pgfpathlineto{\pgfqpoint{1.131901in}{0.603309in}}%
\pgfpathlineto{\pgfqpoint{1.132457in}{0.602162in}}%
\pgfpathlineto{\pgfqpoint{1.133013in}{0.602321in}}%
\pgfpathlineto{\pgfqpoint{1.134680in}{0.600405in}}%
\pgfpathlineto{\pgfqpoint{1.136348in}{0.602790in}}%
\pgfpathlineto{\pgfqpoint{1.139127in}{0.603566in}}%
\pgfpathlineto{\pgfqpoint{1.140239in}{0.601004in}}%
\pgfpathlineto{\pgfqpoint{1.140795in}{0.602759in}}%
\pgfpathlineto{\pgfqpoint{1.141351in}{0.603307in}}%
\pgfpathlineto{\pgfqpoint{1.141907in}{0.609080in}}%
\pgfpathlineto{\pgfqpoint{1.143574in}{0.602129in}}%
\pgfpathlineto{\pgfqpoint{1.144686in}{0.607177in}}%
\pgfpathlineto{\pgfqpoint{1.146354in}{0.601383in}}%
\pgfpathlineto{\pgfqpoint{1.148021in}{0.601037in}}%
\pgfpathlineto{\pgfqpoint{1.148577in}{0.604453in}}%
\pgfpathlineto{\pgfqpoint{1.149133in}{0.601151in}}%
\pgfpathlineto{\pgfqpoint{1.150245in}{0.604764in}}%
\pgfpathlineto{\pgfqpoint{1.150800in}{0.603423in}}%
\pgfpathlineto{\pgfqpoint{1.151356in}{0.603583in}}%
\pgfpathlineto{\pgfqpoint{1.151912in}{0.600430in}}%
\pgfpathlineto{\pgfqpoint{1.152468in}{0.601090in}}%
\pgfpathlineto{\pgfqpoint{1.156359in}{0.605539in}}%
\pgfpathlineto{\pgfqpoint{1.157471in}{0.601367in}}%
\pgfpathlineto{\pgfqpoint{1.158582in}{0.602741in}}%
\pgfpathlineto{\pgfqpoint{1.159138in}{0.601181in}}%
\pgfpathlineto{\pgfqpoint{1.159694in}{0.603667in}}%
\pgfpathlineto{\pgfqpoint{1.160250in}{0.603264in}}%
\pgfpathlineto{\pgfqpoint{1.161362in}{0.600487in}}%
\pgfpathlineto{\pgfqpoint{1.163029in}{0.602592in}}%
\pgfpathlineto{\pgfqpoint{1.163585in}{0.602787in}}%
\pgfpathlineto{\pgfqpoint{1.164697in}{0.604369in}}%
\pgfpathlineto{\pgfqpoint{1.165253in}{0.600336in}}%
\pgfpathlineto{\pgfqpoint{1.165809in}{0.603721in}}%
\pgfpathlineto{\pgfqpoint{1.166920in}{0.604526in}}%
\pgfpathlineto{\pgfqpoint{1.168588in}{0.601187in}}%
\pgfpathlineto{\pgfqpoint{1.169144in}{0.602234in}}%
\pgfpathlineto{\pgfqpoint{1.169700in}{0.601643in}}%
\pgfpathlineto{\pgfqpoint{1.170256in}{0.601007in}}%
\pgfpathlineto{\pgfqpoint{1.171923in}{0.604551in}}%
\pgfpathlineto{\pgfqpoint{1.172479in}{0.602096in}}%
\pgfpathlineto{\pgfqpoint{1.173035in}{0.604950in}}%
\pgfpathlineto{\pgfqpoint{1.173591in}{0.601404in}}%
\pgfpathlineto{\pgfqpoint{1.174147in}{0.602543in}}%
\pgfpathlineto{\pgfqpoint{1.175258in}{0.602624in}}%
\pgfpathlineto{\pgfqpoint{1.175814in}{0.605234in}}%
\pgfpathlineto{\pgfqpoint{1.176370in}{0.603823in}}%
\pgfpathlineto{\pgfqpoint{1.177482in}{0.601614in}}%
\pgfpathlineto{\pgfqpoint{1.178038in}{0.603387in}}%
\pgfpathlineto{\pgfqpoint{1.178593in}{0.601794in}}%
\pgfpathlineto{\pgfqpoint{1.180817in}{0.605925in}}%
\pgfpathlineto{\pgfqpoint{1.182484in}{0.601726in}}%
\pgfpathlineto{\pgfqpoint{1.185820in}{0.602403in}}%
\pgfpathlineto{\pgfqpoint{1.186376in}{0.602800in}}%
\pgfpathlineto{\pgfqpoint{1.187487in}{0.602124in}}%
\pgfpathlineto{\pgfqpoint{1.188043in}{0.603782in}}%
\pgfpathlineto{\pgfqpoint{1.189155in}{0.601295in}}%
\pgfpathlineto{\pgfqpoint{1.189711in}{0.601826in}}%
\pgfpathlineto{\pgfqpoint{1.190267in}{0.603941in}}%
\pgfpathlineto{\pgfqpoint{1.190822in}{0.601795in}}%
\pgfpathlineto{\pgfqpoint{1.191934in}{0.601693in}}%
\pgfpathlineto{\pgfqpoint{1.192490in}{0.603298in}}%
\pgfpathlineto{\pgfqpoint{1.193046in}{0.601021in}}%
\pgfpathlineto{\pgfqpoint{1.193602in}{0.604096in}}%
\pgfpathlineto{\pgfqpoint{1.194158in}{0.600368in}}%
\pgfpathlineto{\pgfqpoint{1.194713in}{0.601974in}}%
\pgfpathlineto{\pgfqpoint{1.195269in}{0.601940in}}%
\pgfpathlineto{\pgfqpoint{1.195825in}{0.603169in}}%
\pgfpathlineto{\pgfqpoint{1.196381in}{0.601825in}}%
\pgfpathlineto{\pgfqpoint{1.198049in}{0.604260in}}%
\pgfpathlineto{\pgfqpoint{1.198604in}{0.602041in}}%
\pgfpathlineto{\pgfqpoint{1.199160in}{0.603176in}}%
\pgfpathlineto{\pgfqpoint{1.199716in}{0.604245in}}%
\pgfpathlineto{\pgfqpoint{1.200272in}{0.600904in}}%
\pgfpathlineto{\pgfqpoint{1.200828in}{0.607185in}}%
\pgfpathlineto{\pgfqpoint{1.201384in}{0.600995in}}%
\pgfpathlineto{\pgfqpoint{1.203051in}{0.606627in}}%
\pgfpathlineto{\pgfqpoint{1.204163in}{0.601928in}}%
\pgfpathlineto{\pgfqpoint{1.204719in}{0.602299in}}%
\pgfpathlineto{\pgfqpoint{1.205275in}{0.603564in}}%
\pgfpathlineto{\pgfqpoint{1.205831in}{0.601992in}}%
\pgfpathlineto{\pgfqpoint{1.206387in}{0.602706in}}%
\pgfpathlineto{\pgfqpoint{1.207498in}{0.601860in}}%
\pgfpathlineto{\pgfqpoint{1.208054in}{0.602796in}}%
\pgfpathlineto{\pgfqpoint{1.208610in}{0.602817in}}%
\pgfpathlineto{\pgfqpoint{1.209166in}{0.601175in}}%
\pgfpathlineto{\pgfqpoint{1.209722in}{0.601723in}}%
\pgfpathlineto{\pgfqpoint{1.210278in}{0.607282in}}%
\pgfpathlineto{\pgfqpoint{1.210833in}{0.606684in}}%
\pgfpathlineto{\pgfqpoint{1.211389in}{0.601196in}}%
\pgfpathlineto{\pgfqpoint{1.211945in}{0.602911in}}%
\pgfpathlineto{\pgfqpoint{1.212501in}{0.602537in}}%
\pgfpathlineto{\pgfqpoint{1.213613in}{0.604627in}}%
\pgfpathlineto{\pgfqpoint{1.214724in}{0.601352in}}%
\pgfpathlineto{\pgfqpoint{1.215280in}{0.602579in}}%
\pgfpathlineto{\pgfqpoint{1.215836in}{0.604407in}}%
\pgfpathlineto{\pgfqpoint{1.216392in}{0.602986in}}%
\pgfpathlineto{\pgfqpoint{1.216948in}{0.601646in}}%
\pgfpathlineto{\pgfqpoint{1.217504in}{0.603599in}}%
\pgfpathlineto{\pgfqpoint{1.218060in}{0.600407in}}%
\pgfpathlineto{\pgfqpoint{1.218615in}{0.602294in}}%
\pgfpathlineto{\pgfqpoint{1.219171in}{0.601666in}}%
\pgfpathlineto{\pgfqpoint{1.219727in}{0.602275in}}%
\pgfpathlineto{\pgfqpoint{1.220283in}{0.603770in}}%
\pgfpathlineto{\pgfqpoint{1.220839in}{0.603106in}}%
\pgfpathlineto{\pgfqpoint{1.223062in}{0.601369in}}%
\pgfpathlineto{\pgfqpoint{1.224174in}{0.604887in}}%
\pgfpathlineto{\pgfqpoint{1.224730in}{0.604241in}}%
\pgfpathlineto{\pgfqpoint{1.226398in}{0.602743in}}%
\pgfpathlineto{\pgfqpoint{1.226953in}{0.604235in}}%
\pgfpathlineto{\pgfqpoint{1.227509in}{0.600333in}}%
\pgfpathlineto{\pgfqpoint{1.228065in}{0.604612in}}%
\pgfpathlineto{\pgfqpoint{1.228621in}{0.601113in}}%
\pgfpathlineto{\pgfqpoint{1.229177in}{0.602510in}}%
\pgfpathlineto{\pgfqpoint{1.229733in}{0.600208in}}%
\pgfpathlineto{\pgfqpoint{1.230289in}{0.603585in}}%
\pgfpathlineto{\pgfqpoint{1.230844in}{0.602846in}}%
\pgfpathlineto{\pgfqpoint{1.231400in}{0.601217in}}%
\pgfpathlineto{\pgfqpoint{1.232512in}{0.605635in}}%
\pgfpathlineto{\pgfqpoint{1.234180in}{0.600232in}}%
\pgfpathlineto{\pgfqpoint{1.234735in}{0.604378in}}%
\pgfpathlineto{\pgfqpoint{1.235291in}{0.601418in}}%
\pgfpathlineto{\pgfqpoint{1.235847in}{0.603828in}}%
\pgfpathlineto{\pgfqpoint{1.236403in}{0.602913in}}%
\pgfpathlineto{\pgfqpoint{1.237515in}{0.604026in}}%
\pgfpathlineto{\pgfqpoint{1.239182in}{0.601937in}}%
\pgfpathlineto{\pgfqpoint{1.240850in}{0.604344in}}%
\pgfpathlineto{\pgfqpoint{1.241406in}{0.602783in}}%
\pgfpathlineto{\pgfqpoint{1.241962in}{0.604535in}}%
\pgfpathlineto{\pgfqpoint{1.242518in}{0.600886in}}%
\pgfpathlineto{\pgfqpoint{1.243073in}{0.601586in}}%
\pgfpathlineto{\pgfqpoint{1.244185in}{0.602951in}}%
\pgfpathlineto{\pgfqpoint{1.244741in}{0.600206in}}%
\pgfpathlineto{\pgfqpoint{1.245297in}{0.602209in}}%
\pgfpathlineto{\pgfqpoint{1.245853in}{0.603009in}}%
\pgfpathlineto{\pgfqpoint{1.246409in}{0.605801in}}%
\pgfpathlineto{\pgfqpoint{1.247520in}{0.602280in}}%
\pgfpathlineto{\pgfqpoint{1.248076in}{0.603617in}}%
\pgfpathlineto{\pgfqpoint{1.249744in}{0.600574in}}%
\pgfpathlineto{\pgfqpoint{1.251967in}{0.602122in}}%
\pgfpathlineto{\pgfqpoint{1.252523in}{0.601209in}}%
\pgfpathlineto{\pgfqpoint{1.253079in}{0.605044in}}%
\pgfpathlineto{\pgfqpoint{1.253635in}{0.603860in}}%
\pgfpathlineto{\pgfqpoint{1.255302in}{0.600799in}}%
\pgfpathlineto{\pgfqpoint{1.255858in}{0.601527in}}%
\pgfpathlineto{\pgfqpoint{1.256414in}{0.600503in}}%
\pgfpathlineto{\pgfqpoint{1.256970in}{0.608556in}}%
\pgfpathlineto{\pgfqpoint{1.257526in}{0.608213in}}%
\pgfpathlineto{\pgfqpoint{1.259193in}{0.602466in}}%
\pgfpathlineto{\pgfqpoint{1.259749in}{0.603882in}}%
\pgfpathlineto{\pgfqpoint{1.260305in}{0.601642in}}%
\pgfpathlineto{\pgfqpoint{1.261417in}{0.605757in}}%
\pgfpathlineto{\pgfqpoint{1.261973in}{0.604730in}}%
\pgfpathlineto{\pgfqpoint{1.262529in}{0.601997in}}%
\pgfpathlineto{\pgfqpoint{1.263084in}{0.605058in}}%
\pgfpathlineto{\pgfqpoint{1.263640in}{0.604773in}}%
\pgfpathlineto{\pgfqpoint{1.265308in}{0.600538in}}%
\pgfpathlineto{\pgfqpoint{1.266975in}{0.602576in}}%
\pgfpathlineto{\pgfqpoint{1.267531in}{0.602444in}}%
\pgfpathlineto{\pgfqpoint{1.268087in}{0.606579in}}%
\pgfpathlineto{\pgfqpoint{1.268643in}{0.603825in}}%
\pgfpathlineto{\pgfqpoint{1.269199in}{0.604281in}}%
\pgfpathlineto{\pgfqpoint{1.269755in}{0.602368in}}%
\pgfpathlineto{\pgfqpoint{1.270311in}{0.602867in}}%
\pgfpathlineto{\pgfqpoint{1.270866in}{0.604795in}}%
\pgfpathlineto{\pgfqpoint{1.271978in}{0.600647in}}%
\pgfpathlineto{\pgfqpoint{1.273090in}{0.603727in}}%
\pgfpathlineto{\pgfqpoint{1.273646in}{0.601115in}}%
\pgfpathlineto{\pgfqpoint{1.274202in}{0.603048in}}%
\pgfpathlineto{\pgfqpoint{1.276425in}{0.601090in}}%
\pgfpathlineto{\pgfqpoint{1.277537in}{0.604593in}}%
\pgfpathlineto{\pgfqpoint{1.278093in}{0.601831in}}%
\pgfpathlineto{\pgfqpoint{1.278649in}{0.603092in}}%
\pgfpathlineto{\pgfqpoint{1.279204in}{0.603093in}}%
\pgfpathlineto{\pgfqpoint{1.279760in}{0.600863in}}%
\pgfpathlineto{\pgfqpoint{1.280316in}{0.601878in}}%
\pgfpathlineto{\pgfqpoint{1.280872in}{0.603664in}}%
\pgfpathlineto{\pgfqpoint{1.281428in}{0.601676in}}%
\pgfpathlineto{\pgfqpoint{1.283095in}{0.605527in}}%
\pgfpathlineto{\pgfqpoint{1.284207in}{0.600070in}}%
\pgfpathlineto{\pgfqpoint{1.284763in}{0.601055in}}%
\pgfpathlineto{\pgfqpoint{1.285319in}{0.603959in}}%
\pgfpathlineto{\pgfqpoint{1.285875in}{0.601040in}}%
\pgfpathlineto{\pgfqpoint{1.286431in}{0.600834in}}%
\pgfpathlineto{\pgfqpoint{1.288098in}{0.604067in}}%
\pgfpathlineto{\pgfqpoint{1.288654in}{0.601960in}}%
\pgfpathlineto{\pgfqpoint{1.289210in}{0.603661in}}%
\pgfpathlineto{\pgfqpoint{1.289766in}{0.603169in}}%
\pgfpathlineto{\pgfqpoint{1.290322in}{0.600974in}}%
\pgfpathlineto{\pgfqpoint{1.290877in}{0.602496in}}%
\pgfpathlineto{\pgfqpoint{1.291433in}{0.608506in}}%
\pgfpathlineto{\pgfqpoint{1.291989in}{0.604203in}}%
\pgfpathlineto{\pgfqpoint{1.293657in}{0.600357in}}%
\pgfpathlineto{\pgfqpoint{1.294213in}{0.600924in}}%
\pgfpathlineto{\pgfqpoint{1.294768in}{0.604452in}}%
\pgfpathlineto{\pgfqpoint{1.295324in}{0.602141in}}%
\pgfpathlineto{\pgfqpoint{1.296992in}{0.601664in}}%
\pgfpathlineto{\pgfqpoint{1.298104in}{0.607024in}}%
\pgfpathlineto{\pgfqpoint{1.298660in}{0.600546in}}%
\pgfpathlineto{\pgfqpoint{1.299215in}{0.603522in}}%
\pgfpathlineto{\pgfqpoint{1.300883in}{0.602245in}}%
\pgfpathlineto{\pgfqpoint{1.303106in}{0.601589in}}%
\pgfpathlineto{\pgfqpoint{1.303662in}{0.604819in}}%
\pgfpathlineto{\pgfqpoint{1.304218in}{0.602256in}}%
\pgfpathlineto{\pgfqpoint{1.304774in}{0.600789in}}%
\pgfpathlineto{\pgfqpoint{1.305330in}{0.602428in}}%
\pgfpathlineto{\pgfqpoint{1.305886in}{0.601228in}}%
\pgfpathlineto{\pgfqpoint{1.306442in}{0.603913in}}%
\pgfpathlineto{\pgfqpoint{1.306997in}{0.601692in}}%
\pgfpathlineto{\pgfqpoint{1.307553in}{0.601634in}}%
\pgfpathlineto{\pgfqpoint{1.308665in}{0.604064in}}%
\pgfpathlineto{\pgfqpoint{1.309777in}{0.600500in}}%
\pgfpathlineto{\pgfqpoint{1.310333in}{0.604969in}}%
\pgfpathlineto{\pgfqpoint{1.310888in}{0.601121in}}%
\pgfpathlineto{\pgfqpoint{1.312000in}{0.603011in}}%
\pgfpathlineto{\pgfqpoint{1.312556in}{0.601274in}}%
\pgfpathlineto{\pgfqpoint{1.313112in}{0.601778in}}%
\pgfpathlineto{\pgfqpoint{1.313668in}{0.601524in}}%
\pgfpathlineto{\pgfqpoint{1.314224in}{0.607455in}}%
\pgfpathlineto{\pgfqpoint{1.314779in}{0.603817in}}%
\pgfpathlineto{\pgfqpoint{1.315891in}{0.608112in}}%
\pgfpathlineto{\pgfqpoint{1.317559in}{0.602265in}}%
\pgfpathlineto{\pgfqpoint{1.319226in}{0.604364in}}%
\pgfpathlineto{\pgfqpoint{1.320338in}{0.600730in}}%
\pgfpathlineto{\pgfqpoint{1.320894in}{0.603418in}}%
\pgfpathlineto{\pgfqpoint{1.322006in}{0.602121in}}%
\pgfpathlineto{\pgfqpoint{1.322562in}{0.607665in}}%
\pgfpathlineto{\pgfqpoint{1.323117in}{0.603918in}}%
\pgfpathlineto{\pgfqpoint{1.324229in}{0.600543in}}%
\pgfpathlineto{\pgfqpoint{1.324785in}{0.602311in}}%
\pgfpathlineto{\pgfqpoint{1.325897in}{0.605887in}}%
\pgfpathlineto{\pgfqpoint{1.326453in}{0.605126in}}%
\pgfpathlineto{\pgfqpoint{1.327564in}{0.602216in}}%
\pgfpathlineto{\pgfqpoint{1.328120in}{0.602554in}}%
\pgfpathlineto{\pgfqpoint{1.330344in}{0.604240in}}%
\pgfpathlineto{\pgfqpoint{1.332567in}{0.602590in}}%
\pgfpathlineto{\pgfqpoint{1.333123in}{0.602374in}}%
\pgfpathlineto{\pgfqpoint{1.333679in}{0.600390in}}%
\pgfpathlineto{\pgfqpoint{1.334235in}{0.604690in}}%
\pgfpathlineto{\pgfqpoint{1.334791in}{0.601626in}}%
\pgfpathlineto{\pgfqpoint{1.335346in}{0.601572in}}%
\pgfpathlineto{\pgfqpoint{1.337014in}{0.604258in}}%
\pgfpathlineto{\pgfqpoint{1.337570in}{0.603645in}}%
\pgfpathlineto{\pgfqpoint{1.338126in}{0.604874in}}%
\pgfpathlineto{\pgfqpoint{1.338682in}{0.604594in}}%
\pgfpathlineto{\pgfqpoint{1.339237in}{0.604264in}}%
\pgfpathlineto{\pgfqpoint{1.339793in}{0.608452in}}%
\pgfpathlineto{\pgfqpoint{1.340349in}{0.605846in}}%
\pgfpathlineto{\pgfqpoint{1.341461in}{0.605994in}}%
\pgfpathlineto{\pgfqpoint{1.342017in}{0.603528in}}%
\pgfpathlineto{\pgfqpoint{1.342573in}{0.606768in}}%
\pgfpathlineto{\pgfqpoint{1.343128in}{0.601026in}}%
\pgfpathlineto{\pgfqpoint{1.343684in}{0.604420in}}%
\pgfpathlineto{\pgfqpoint{1.344240in}{0.607804in}}%
\pgfpathlineto{\pgfqpoint{1.345908in}{0.603982in}}%
\pgfpathlineto{\pgfqpoint{1.346464in}{0.604182in}}%
\pgfpathlineto{\pgfqpoint{1.347019in}{0.601902in}}%
\pgfpathlineto{\pgfqpoint{1.347575in}{0.602896in}}%
\pgfpathlineto{\pgfqpoint{1.348131in}{0.605320in}}%
\pgfpathlineto{\pgfqpoint{1.348687in}{0.604890in}}%
\pgfpathlineto{\pgfqpoint{1.349799in}{0.600767in}}%
\pgfpathlineto{\pgfqpoint{1.350355in}{0.602775in}}%
\pgfpathlineto{\pgfqpoint{1.350910in}{0.606425in}}%
\pgfpathlineto{\pgfqpoint{1.352578in}{0.601477in}}%
\pgfpathlineto{\pgfqpoint{1.354802in}{0.606079in}}%
\pgfpathlineto{\pgfqpoint{1.355357in}{0.605174in}}%
\pgfpathlineto{\pgfqpoint{1.355913in}{0.600357in}}%
\pgfpathlineto{\pgfqpoint{1.356469in}{0.602222in}}%
\pgfpathlineto{\pgfqpoint{1.357581in}{0.601231in}}%
\pgfpathlineto{\pgfqpoint{1.358693in}{0.605347in}}%
\pgfpathlineto{\pgfqpoint{1.360360in}{0.600452in}}%
\pgfpathlineto{\pgfqpoint{1.362028in}{0.606599in}}%
\pgfpathlineto{\pgfqpoint{1.363695in}{0.602098in}}%
\pgfpathlineto{\pgfqpoint{1.364251in}{0.605389in}}%
\pgfpathlineto{\pgfqpoint{1.364807in}{0.600973in}}%
\pgfpathlineto{\pgfqpoint{1.365363in}{0.603269in}}%
\pgfpathlineto{\pgfqpoint{1.366475in}{0.601398in}}%
\pgfpathlineto{\pgfqpoint{1.368142in}{0.607287in}}%
\pgfpathlineto{\pgfqpoint{1.368698in}{0.601917in}}%
\pgfpathlineto{\pgfqpoint{1.369254in}{0.604318in}}%
\pgfpathlineto{\pgfqpoint{1.370366in}{0.601055in}}%
\pgfpathlineto{\pgfqpoint{1.372033in}{0.608894in}}%
\pgfpathlineto{\pgfqpoint{1.374257in}{0.602101in}}%
\pgfpathlineto{\pgfqpoint{1.376480in}{0.605675in}}%
\pgfpathlineto{\pgfqpoint{1.377592in}{0.600947in}}%
\pgfpathlineto{\pgfqpoint{1.378148in}{0.604862in}}%
\pgfpathlineto{\pgfqpoint{1.378704in}{0.603965in}}%
\pgfpathlineto{\pgfqpoint{1.380371in}{0.600745in}}%
\pgfpathlineto{\pgfqpoint{1.380927in}{0.604764in}}%
\pgfpathlineto{\pgfqpoint{1.381483in}{0.602901in}}%
\pgfpathlineto{\pgfqpoint{1.382039in}{0.602102in}}%
\pgfpathlineto{\pgfqpoint{1.383150in}{0.611346in}}%
\pgfpathlineto{\pgfqpoint{1.384818in}{0.601760in}}%
\pgfpathlineto{\pgfqpoint{1.387041in}{0.607896in}}%
\pgfpathlineto{\pgfqpoint{1.388709in}{0.602524in}}%
\pgfpathlineto{\pgfqpoint{1.389265in}{0.608518in}}%
\pgfpathlineto{\pgfqpoint{1.389821in}{0.604708in}}%
\pgfpathlineto{\pgfqpoint{1.390377in}{0.601992in}}%
\pgfpathlineto{\pgfqpoint{1.390933in}{0.604683in}}%
\pgfpathlineto{\pgfqpoint{1.391488in}{0.602240in}}%
\pgfpathlineto{\pgfqpoint{1.392044in}{0.604454in}}%
\pgfpathlineto{\pgfqpoint{1.392600in}{0.602740in}}%
\pgfpathlineto{\pgfqpoint{1.393156in}{0.606893in}}%
\pgfpathlineto{\pgfqpoint{1.393712in}{0.605659in}}%
\pgfpathlineto{\pgfqpoint{1.394268in}{0.605513in}}%
\pgfpathlineto{\pgfqpoint{1.394824in}{0.600237in}}%
\pgfpathlineto{\pgfqpoint{1.395379in}{0.605304in}}%
\pgfpathlineto{\pgfqpoint{1.395935in}{0.601072in}}%
\pgfpathlineto{\pgfqpoint{1.397603in}{0.607413in}}%
\pgfpathlineto{\pgfqpoint{1.398159in}{0.607380in}}%
\pgfpathlineto{\pgfqpoint{1.398715in}{0.602247in}}%
\pgfpathlineto{\pgfqpoint{1.399270in}{0.605457in}}%
\pgfpathlineto{\pgfqpoint{1.399826in}{0.606332in}}%
\pgfpathlineto{\pgfqpoint{1.400382in}{0.609563in}}%
\pgfpathlineto{\pgfqpoint{1.400938in}{0.600866in}}%
\pgfpathlineto{\pgfqpoint{1.401494in}{0.604824in}}%
\pgfpathlineto{\pgfqpoint{1.402050in}{0.601742in}}%
\pgfpathlineto{\pgfqpoint{1.402606in}{0.602528in}}%
\pgfpathlineto{\pgfqpoint{1.403161in}{0.607607in}}%
\pgfpathlineto{\pgfqpoint{1.403717in}{0.602905in}}%
\pgfpathlineto{\pgfqpoint{1.404829in}{0.607110in}}%
\pgfpathlineto{\pgfqpoint{1.405385in}{0.605820in}}%
\pgfpathlineto{\pgfqpoint{1.406497in}{0.602653in}}%
\pgfpathlineto{\pgfqpoint{1.408164in}{0.606847in}}%
\pgfpathlineto{\pgfqpoint{1.408720in}{0.602242in}}%
\pgfpathlineto{\pgfqpoint{1.409276in}{0.603232in}}%
\pgfpathlineto{\pgfqpoint{1.410388in}{0.603890in}}%
\pgfpathlineto{\pgfqpoint{1.411499in}{0.607750in}}%
\pgfpathlineto{\pgfqpoint{1.412055in}{0.603048in}}%
\pgfpathlineto{\pgfqpoint{1.412611in}{0.604243in}}%
\pgfpathlineto{\pgfqpoint{1.413167in}{0.604556in}}%
\pgfpathlineto{\pgfqpoint{1.413723in}{0.600487in}}%
\pgfpathlineto{\pgfqpoint{1.414279in}{0.605431in}}%
\pgfpathlineto{\pgfqpoint{1.414835in}{0.603655in}}%
\pgfpathlineto{\pgfqpoint{1.416502in}{0.605701in}}%
\pgfpathlineto{\pgfqpoint{1.417058in}{0.601575in}}%
\pgfpathlineto{\pgfqpoint{1.417614in}{0.603461in}}%
\pgfpathlineto{\pgfqpoint{1.418726in}{0.611302in}}%
\pgfpathlineto{\pgfqpoint{1.420393in}{0.601499in}}%
\pgfpathlineto{\pgfqpoint{1.421505in}{0.606043in}}%
\pgfpathlineto{\pgfqpoint{1.422061in}{0.600644in}}%
\pgfpathlineto{\pgfqpoint{1.422617in}{0.603205in}}%
\pgfpathlineto{\pgfqpoint{1.423172in}{0.605832in}}%
\pgfpathlineto{\pgfqpoint{1.424284in}{0.600899in}}%
\pgfpathlineto{\pgfqpoint{1.424840in}{0.605778in}}%
\pgfpathlineto{\pgfqpoint{1.425396in}{0.605589in}}%
\pgfpathlineto{\pgfqpoint{1.425952in}{0.602942in}}%
\pgfpathlineto{\pgfqpoint{1.426508in}{0.605358in}}%
\pgfpathlineto{\pgfqpoint{1.427619in}{0.601236in}}%
\pgfpathlineto{\pgfqpoint{1.429287in}{0.609300in}}%
\pgfpathlineto{\pgfqpoint{1.430955in}{0.603878in}}%
\pgfpathlineto{\pgfqpoint{1.431510in}{0.606894in}}%
\pgfpathlineto{\pgfqpoint{1.432066in}{0.602081in}}%
\pgfpathlineto{\pgfqpoint{1.432622in}{0.604823in}}%
\pgfpathlineto{\pgfqpoint{1.433178in}{0.607656in}}%
\pgfpathlineto{\pgfqpoint{1.433734in}{0.604820in}}%
\pgfpathlineto{\pgfqpoint{1.434290in}{0.600561in}}%
\pgfpathlineto{\pgfqpoint{1.434846in}{0.601389in}}%
\pgfpathlineto{\pgfqpoint{1.435957in}{0.605618in}}%
\pgfpathlineto{\pgfqpoint{1.436513in}{0.603662in}}%
\pgfpathlineto{\pgfqpoint{1.437069in}{0.601630in}}%
\pgfpathlineto{\pgfqpoint{1.437625in}{0.606919in}}%
\pgfpathlineto{\pgfqpoint{1.438181in}{0.605000in}}%
\pgfpathlineto{\pgfqpoint{1.439292in}{0.603448in}}%
\pgfpathlineto{\pgfqpoint{1.441516in}{0.610675in}}%
\pgfpathlineto{\pgfqpoint{1.443183in}{0.603024in}}%
\pgfpathlineto{\pgfqpoint{1.443739in}{0.602457in}}%
\pgfpathlineto{\pgfqpoint{1.445407in}{0.608970in}}%
\pgfpathlineto{\pgfqpoint{1.445963in}{0.603347in}}%
\pgfpathlineto{\pgfqpoint{1.446519in}{0.605806in}}%
\pgfpathlineto{\pgfqpoint{1.447075in}{0.605781in}}%
\pgfpathlineto{\pgfqpoint{1.447630in}{0.601442in}}%
\pgfpathlineto{\pgfqpoint{1.448186in}{0.607100in}}%
\pgfpathlineto{\pgfqpoint{1.448742in}{0.602806in}}%
\pgfpathlineto{\pgfqpoint{1.450410in}{0.603716in}}%
\pgfpathlineto{\pgfqpoint{1.450966in}{0.602108in}}%
\pgfpathlineto{\pgfqpoint{1.452633in}{0.606973in}}%
\pgfpathlineto{\pgfqpoint{1.453189in}{0.604237in}}%
\pgfpathlineto{\pgfqpoint{1.453745in}{0.605528in}}%
\pgfpathlineto{\pgfqpoint{1.454301in}{0.609977in}}%
\pgfpathlineto{\pgfqpoint{1.454857in}{0.605670in}}%
\pgfpathlineto{\pgfqpoint{1.455412in}{0.607997in}}%
\pgfpathlineto{\pgfqpoint{1.455968in}{0.602551in}}%
\pgfpathlineto{\pgfqpoint{1.456524in}{0.608322in}}%
\pgfpathlineto{\pgfqpoint{1.457080in}{0.602164in}}%
\pgfpathlineto{\pgfqpoint{1.457636in}{0.611003in}}%
\pgfpathlineto{\pgfqpoint{1.458192in}{0.605286in}}%
\pgfpathlineto{\pgfqpoint{1.458748in}{0.604835in}}%
\pgfpathlineto{\pgfqpoint{1.460415in}{0.609715in}}%
\pgfpathlineto{\pgfqpoint{1.460971in}{0.604992in}}%
\pgfpathlineto{\pgfqpoint{1.461527in}{0.606993in}}%
\pgfpathlineto{\pgfqpoint{1.463194in}{0.602925in}}%
\pgfpathlineto{\pgfqpoint{1.463750in}{0.603398in}}%
\pgfpathlineto{\pgfqpoint{1.465418in}{0.606813in}}%
\pgfpathlineto{\pgfqpoint{1.465974in}{0.606450in}}%
\pgfpathlineto{\pgfqpoint{1.466530in}{0.601042in}}%
\pgfpathlineto{\pgfqpoint{1.467086in}{0.603006in}}%
\pgfpathlineto{\pgfqpoint{1.467641in}{0.606271in}}%
\pgfpathlineto{\pgfqpoint{1.468197in}{0.605607in}}%
\pgfpathlineto{\pgfqpoint{1.469309in}{0.601017in}}%
\pgfpathlineto{\pgfqpoint{1.470421in}{0.609918in}}%
\pgfpathlineto{\pgfqpoint{1.470977in}{0.604457in}}%
\pgfpathlineto{\pgfqpoint{1.471532in}{0.608086in}}%
\pgfpathlineto{\pgfqpoint{1.473200in}{0.601355in}}%
\pgfpathlineto{\pgfqpoint{1.473756in}{0.606381in}}%
\pgfpathlineto{\pgfqpoint{1.474312in}{0.604832in}}%
\pgfpathlineto{\pgfqpoint{1.475423in}{0.602882in}}%
\pgfpathlineto{\pgfqpoint{1.475979in}{0.606801in}}%
\pgfpathlineto{\pgfqpoint{1.477091in}{0.601141in}}%
\pgfpathlineto{\pgfqpoint{1.478203in}{0.609079in}}%
\pgfpathlineto{\pgfqpoint{1.478759in}{0.607784in}}%
\pgfpathlineto{\pgfqpoint{1.479870in}{0.600756in}}%
\pgfpathlineto{\pgfqpoint{1.480982in}{0.606827in}}%
\pgfpathlineto{\pgfqpoint{1.482094in}{0.603058in}}%
\pgfpathlineto{\pgfqpoint{1.484317in}{0.610659in}}%
\pgfpathlineto{\pgfqpoint{1.484873in}{0.601817in}}%
\pgfpathlineto{\pgfqpoint{1.485429in}{0.605611in}}%
\pgfpathlineto{\pgfqpoint{1.487097in}{0.610220in}}%
\pgfpathlineto{\pgfqpoint{1.489320in}{0.604318in}}%
\pgfpathlineto{\pgfqpoint{1.490988in}{0.607198in}}%
\pgfpathlineto{\pgfqpoint{1.492655in}{0.602995in}}%
\pgfpathlineto{\pgfqpoint{1.494323in}{0.601800in}}%
\pgfpathlineto{\pgfqpoint{1.495990in}{0.609345in}}%
\pgfpathlineto{\pgfqpoint{1.496546in}{0.601847in}}%
\pgfpathlineto{\pgfqpoint{1.497102in}{0.603895in}}%
\pgfpathlineto{\pgfqpoint{1.498214in}{0.609982in}}%
\pgfpathlineto{\pgfqpoint{1.499881in}{0.601251in}}%
\pgfpathlineto{\pgfqpoint{1.500993in}{0.607902in}}%
\pgfpathlineto{\pgfqpoint{1.501549in}{0.605760in}}%
\pgfpathlineto{\pgfqpoint{1.502105in}{0.610369in}}%
\pgfpathlineto{\pgfqpoint{1.502661in}{0.607239in}}%
\pgfpathlineto{\pgfqpoint{1.503216in}{0.603245in}}%
\pgfpathlineto{\pgfqpoint{1.503772in}{0.607207in}}%
\pgfpathlineto{\pgfqpoint{1.505440in}{0.609589in}}%
\pgfpathlineto{\pgfqpoint{1.506552in}{0.607460in}}%
\pgfpathlineto{\pgfqpoint{1.507108in}{0.602882in}}%
\pgfpathlineto{\pgfqpoint{1.507663in}{0.605424in}}%
\pgfpathlineto{\pgfqpoint{1.509331in}{0.602332in}}%
\pgfpathlineto{\pgfqpoint{1.510999in}{0.607983in}}%
\pgfpathlineto{\pgfqpoint{1.511554in}{0.607843in}}%
\pgfpathlineto{\pgfqpoint{1.512110in}{0.604219in}}%
\pgfpathlineto{\pgfqpoint{1.512666in}{0.612481in}}%
\pgfpathlineto{\pgfqpoint{1.513222in}{0.606109in}}%
\pgfpathlineto{\pgfqpoint{1.513778in}{0.609471in}}%
\pgfpathlineto{\pgfqpoint{1.514334in}{0.602003in}}%
\pgfpathlineto{\pgfqpoint{1.514890in}{0.609730in}}%
\pgfpathlineto{\pgfqpoint{1.515445in}{0.606854in}}%
\pgfpathlineto{\pgfqpoint{1.516001in}{0.600886in}}%
\pgfpathlineto{\pgfqpoint{1.516557in}{0.606663in}}%
\pgfpathlineto{\pgfqpoint{1.518225in}{0.603571in}}%
\pgfpathlineto{\pgfqpoint{1.518781in}{0.603803in}}%
\pgfpathlineto{\pgfqpoint{1.519892in}{0.607922in}}%
\pgfpathlineto{\pgfqpoint{1.520448in}{0.605349in}}%
\pgfpathlineto{\pgfqpoint{1.521004in}{0.608679in}}%
\pgfpathlineto{\pgfqpoint{1.521560in}{0.603666in}}%
\pgfpathlineto{\pgfqpoint{1.522116in}{0.604589in}}%
\pgfpathlineto{\pgfqpoint{1.522672in}{0.606923in}}%
\pgfpathlineto{\pgfqpoint{1.523228in}{0.603521in}}%
\pgfpathlineto{\pgfqpoint{1.523783in}{0.606907in}}%
\pgfpathlineto{\pgfqpoint{1.524895in}{0.603289in}}%
\pgfpathlineto{\pgfqpoint{1.526563in}{0.606460in}}%
\pgfpathlineto{\pgfqpoint{1.527119in}{0.606972in}}%
\pgfpathlineto{\pgfqpoint{1.527674in}{0.617087in}}%
\pgfpathlineto{\pgfqpoint{1.528230in}{0.603711in}}%
\pgfpathlineto{\pgfqpoint{1.528786in}{0.604149in}}%
\pgfpathlineto{\pgfqpoint{1.529342in}{0.609268in}}%
\pgfpathlineto{\pgfqpoint{1.529898in}{0.605058in}}%
\pgfpathlineto{\pgfqpoint{1.530454in}{0.603437in}}%
\pgfpathlineto{\pgfqpoint{1.531565in}{0.610863in}}%
\pgfpathlineto{\pgfqpoint{1.532121in}{0.601369in}}%
\pgfpathlineto{\pgfqpoint{1.532677in}{0.608782in}}%
\pgfpathlineto{\pgfqpoint{1.533789in}{0.603305in}}%
\pgfpathlineto{\pgfqpoint{1.534345in}{0.604317in}}%
\pgfpathlineto{\pgfqpoint{1.534901in}{0.611747in}}%
\pgfpathlineto{\pgfqpoint{1.535456in}{0.608729in}}%
\pgfpathlineto{\pgfqpoint{1.536012in}{0.608152in}}%
\pgfpathlineto{\pgfqpoint{1.536568in}{0.600723in}}%
\pgfpathlineto{\pgfqpoint{1.537124in}{0.605759in}}%
\pgfpathlineto{\pgfqpoint{1.537680in}{0.602626in}}%
\pgfpathlineto{\pgfqpoint{1.538236in}{0.604950in}}%
\pgfpathlineto{\pgfqpoint{1.539347in}{0.601927in}}%
\pgfpathlineto{\pgfqpoint{1.539903in}{0.606397in}}%
\pgfpathlineto{\pgfqpoint{1.540459in}{0.603096in}}%
\pgfpathlineto{\pgfqpoint{1.541015in}{0.602456in}}%
\pgfpathlineto{\pgfqpoint{1.542683in}{0.608362in}}%
\pgfpathlineto{\pgfqpoint{1.543239in}{0.611392in}}%
\pgfpathlineto{\pgfqpoint{1.543794in}{0.603216in}}%
\pgfpathlineto{\pgfqpoint{1.544350in}{0.612663in}}%
\pgfpathlineto{\pgfqpoint{1.544906in}{0.611942in}}%
\pgfpathlineto{\pgfqpoint{1.546018in}{0.605143in}}%
\pgfpathlineto{\pgfqpoint{1.546574in}{0.605776in}}%
\pgfpathlineto{\pgfqpoint{1.547685in}{0.602310in}}%
\pgfpathlineto{\pgfqpoint{1.548241in}{0.612359in}}%
\pgfpathlineto{\pgfqpoint{1.548797in}{0.605099in}}%
\pgfpathlineto{\pgfqpoint{1.550465in}{0.603147in}}%
\pgfpathlineto{\pgfqpoint{1.552132in}{0.606918in}}%
\pgfpathlineto{\pgfqpoint{1.553800in}{0.610432in}}%
\pgfpathlineto{\pgfqpoint{1.554356in}{0.602822in}}%
\pgfpathlineto{\pgfqpoint{1.554912in}{0.604526in}}%
\pgfpathlineto{\pgfqpoint{1.556023in}{0.611102in}}%
\pgfpathlineto{\pgfqpoint{1.556579in}{0.607008in}}%
\pgfpathlineto{\pgfqpoint{1.557691in}{0.600896in}}%
\pgfpathlineto{\pgfqpoint{1.558247in}{0.602905in}}%
\pgfpathlineto{\pgfqpoint{1.558803in}{0.601124in}}%
\pgfpathlineto{\pgfqpoint{1.559358in}{0.601503in}}%
\pgfpathlineto{\pgfqpoint{1.560470in}{0.610332in}}%
\pgfpathlineto{\pgfqpoint{1.561026in}{0.601599in}}%
\pgfpathlineto{\pgfqpoint{1.561582in}{0.609306in}}%
\pgfpathlineto{\pgfqpoint{1.562694in}{0.602640in}}%
\pgfpathlineto{\pgfqpoint{1.563250in}{0.606572in}}%
\pgfpathlineto{\pgfqpoint{1.563805in}{0.606261in}}%
\pgfpathlineto{\pgfqpoint{1.564361in}{0.605361in}}%
\pgfpathlineto{\pgfqpoint{1.564917in}{0.613134in}}%
\pgfpathlineto{\pgfqpoint{1.565473in}{0.608739in}}%
\pgfpathlineto{\pgfqpoint{1.566029in}{0.611104in}}%
\pgfpathlineto{\pgfqpoint{1.567696in}{0.601623in}}%
\pgfpathlineto{\pgfqpoint{1.568252in}{0.608816in}}%
\pgfpathlineto{\pgfqpoint{1.568808in}{0.603623in}}%
\pgfpathlineto{\pgfqpoint{1.569364in}{0.608483in}}%
\pgfpathlineto{\pgfqpoint{1.569920in}{0.605223in}}%
\pgfpathlineto{\pgfqpoint{1.571587in}{0.605937in}}%
\pgfpathlineto{\pgfqpoint{1.572143in}{0.613412in}}%
\pgfpathlineto{\pgfqpoint{1.572699in}{0.611785in}}%
\pgfpathlineto{\pgfqpoint{1.573255in}{0.601255in}}%
\pgfpathlineto{\pgfqpoint{1.573811in}{0.607394in}}%
\pgfpathlineto{\pgfqpoint{1.574367in}{0.608261in}}%
\pgfpathlineto{\pgfqpoint{1.576034in}{0.605118in}}%
\pgfpathlineto{\pgfqpoint{1.576590in}{0.605461in}}%
\pgfpathlineto{\pgfqpoint{1.577146in}{0.611468in}}%
\pgfpathlineto{\pgfqpoint{1.577702in}{0.606993in}}%
\pgfpathlineto{\pgfqpoint{1.578258in}{0.604534in}}%
\pgfpathlineto{\pgfqpoint{1.578814in}{0.610756in}}%
\pgfpathlineto{\pgfqpoint{1.579370in}{0.604532in}}%
\pgfpathlineto{\pgfqpoint{1.579925in}{0.604046in}}%
\pgfpathlineto{\pgfqpoint{1.580481in}{0.607633in}}%
\pgfpathlineto{\pgfqpoint{1.581037in}{0.606952in}}%
\pgfpathlineto{\pgfqpoint{1.581593in}{0.604008in}}%
\pgfpathlineto{\pgfqpoint{1.582149in}{0.607506in}}%
\pgfpathlineto{\pgfqpoint{1.582705in}{0.602194in}}%
\pgfpathlineto{\pgfqpoint{1.583261in}{0.602411in}}%
\pgfpathlineto{\pgfqpoint{1.583816in}{0.608236in}}%
\pgfpathlineto{\pgfqpoint{1.584372in}{0.603477in}}%
\pgfpathlineto{\pgfqpoint{1.584928in}{0.606529in}}%
\pgfpathlineto{\pgfqpoint{1.585484in}{0.603613in}}%
\pgfpathlineto{\pgfqpoint{1.586040in}{0.601013in}}%
\pgfpathlineto{\pgfqpoint{1.586596in}{0.603608in}}%
\pgfpathlineto{\pgfqpoint{1.587152in}{0.609719in}}%
\pgfpathlineto{\pgfqpoint{1.587707in}{0.607326in}}%
\pgfpathlineto{\pgfqpoint{1.588263in}{0.604266in}}%
\pgfpathlineto{\pgfqpoint{1.588819in}{0.610001in}}%
\pgfpathlineto{\pgfqpoint{1.589375in}{0.601899in}}%
\pgfpathlineto{\pgfqpoint{1.589931in}{0.603755in}}%
\pgfpathlineto{\pgfqpoint{1.593266in}{0.606082in}}%
\pgfpathlineto{\pgfqpoint{1.594378in}{0.612180in}}%
\pgfpathlineto{\pgfqpoint{1.595489in}{0.604402in}}%
\pgfpathlineto{\pgfqpoint{1.596045in}{0.604948in}}%
\pgfpathlineto{\pgfqpoint{1.596601in}{0.606896in}}%
\pgfpathlineto{\pgfqpoint{1.597157in}{0.604148in}}%
\pgfpathlineto{\pgfqpoint{1.597713in}{0.605339in}}%
\pgfpathlineto{\pgfqpoint{1.598825in}{0.604574in}}%
\pgfpathlineto{\pgfqpoint{1.599381in}{0.611830in}}%
\pgfpathlineto{\pgfqpoint{1.599936in}{0.604341in}}%
\pgfpathlineto{\pgfqpoint{1.600492in}{0.609883in}}%
\pgfpathlineto{\pgfqpoint{1.601048in}{0.612033in}}%
\pgfpathlineto{\pgfqpoint{1.602716in}{0.602045in}}%
\pgfpathlineto{\pgfqpoint{1.603827in}{0.608004in}}%
\pgfpathlineto{\pgfqpoint{1.604383in}{0.607909in}}%
\pgfpathlineto{\pgfqpoint{1.604939in}{0.604314in}}%
\pgfpathlineto{\pgfqpoint{1.606051in}{0.610547in}}%
\pgfpathlineto{\pgfqpoint{1.606607in}{0.602704in}}%
\pgfpathlineto{\pgfqpoint{1.607163in}{0.605092in}}%
\pgfpathlineto{\pgfqpoint{1.607718in}{0.605032in}}%
\pgfpathlineto{\pgfqpoint{1.608274in}{0.610115in}}%
\pgfpathlineto{\pgfqpoint{1.608830in}{0.605377in}}%
\pgfpathlineto{\pgfqpoint{1.609942in}{0.604439in}}%
\pgfpathlineto{\pgfqpoint{1.611054in}{0.610789in}}%
\pgfpathlineto{\pgfqpoint{1.611609in}{0.603717in}}%
\pgfpathlineto{\pgfqpoint{1.612165in}{0.603863in}}%
\pgfpathlineto{\pgfqpoint{1.612721in}{0.611479in}}%
\pgfpathlineto{\pgfqpoint{1.613277in}{0.604141in}}%
\pgfpathlineto{\pgfqpoint{1.613833in}{0.606280in}}%
\pgfpathlineto{\pgfqpoint{1.614389in}{0.601850in}}%
\pgfpathlineto{\pgfqpoint{1.614945in}{0.602928in}}%
\pgfpathlineto{\pgfqpoint{1.615500in}{0.604860in}}%
\pgfpathlineto{\pgfqpoint{1.616056in}{0.603958in}}%
\pgfpathlineto{\pgfqpoint{1.616612in}{0.600194in}}%
\pgfpathlineto{\pgfqpoint{1.618280in}{0.607697in}}%
\pgfpathlineto{\pgfqpoint{1.618836in}{0.604917in}}%
\pgfpathlineto{\pgfqpoint{1.619392in}{0.608644in}}%
\pgfpathlineto{\pgfqpoint{1.619947in}{0.606601in}}%
\pgfpathlineto{\pgfqpoint{1.620503in}{0.605543in}}%
\pgfpathlineto{\pgfqpoint{1.621059in}{0.606774in}}%
\pgfpathlineto{\pgfqpoint{1.623283in}{0.602566in}}%
\pgfpathlineto{\pgfqpoint{1.624950in}{0.617383in}}%
\pgfpathlineto{\pgfqpoint{1.626062in}{0.602541in}}%
\pgfpathlineto{\pgfqpoint{1.626618in}{0.608289in}}%
\pgfpathlineto{\pgfqpoint{1.627174in}{0.608589in}}%
\pgfpathlineto{\pgfqpoint{1.628285in}{0.602529in}}%
\pgfpathlineto{\pgfqpoint{1.629953in}{0.615980in}}%
\pgfpathlineto{\pgfqpoint{1.630509in}{0.600978in}}%
\pgfpathlineto{\pgfqpoint{1.631065in}{0.611314in}}%
\pgfpathlineto{\pgfqpoint{1.631620in}{0.613608in}}%
\pgfpathlineto{\pgfqpoint{1.632176in}{0.602432in}}%
\pgfpathlineto{\pgfqpoint{1.632732in}{0.609051in}}%
\pgfpathlineto{\pgfqpoint{1.633288in}{0.605880in}}%
\pgfpathlineto{\pgfqpoint{1.633844in}{0.607401in}}%
\pgfpathlineto{\pgfqpoint{1.634956in}{0.606121in}}%
\pgfpathlineto{\pgfqpoint{1.635512in}{0.602199in}}%
\pgfpathlineto{\pgfqpoint{1.636067in}{0.608583in}}%
\pgfpathlineto{\pgfqpoint{1.636623in}{0.600905in}}%
\pgfpathlineto{\pgfqpoint{1.637179in}{0.602590in}}%
\pgfpathlineto{\pgfqpoint{1.637735in}{0.605317in}}%
\pgfpathlineto{\pgfqpoint{1.638291in}{0.614418in}}%
\pgfpathlineto{\pgfqpoint{1.639403in}{0.603004in}}%
\pgfpathlineto{\pgfqpoint{1.639958in}{0.610181in}}%
\pgfpathlineto{\pgfqpoint{1.640514in}{0.604399in}}%
\pgfpathlineto{\pgfqpoint{1.641626in}{0.606494in}}%
\pgfpathlineto{\pgfqpoint{1.642182in}{0.602532in}}%
\pgfpathlineto{\pgfqpoint{1.642738in}{0.604506in}}%
\pgfpathlineto{\pgfqpoint{1.643849in}{0.608026in}}%
\pgfpathlineto{\pgfqpoint{1.644961in}{0.606542in}}%
\pgfpathlineto{\pgfqpoint{1.645517in}{0.601126in}}%
\pgfpathlineto{\pgfqpoint{1.646073in}{0.607134in}}%
\pgfpathlineto{\pgfqpoint{1.646629in}{0.602023in}}%
\pgfpathlineto{\pgfqpoint{1.647740in}{0.610062in}}%
\pgfpathlineto{\pgfqpoint{1.648296in}{0.607044in}}%
\pgfpathlineto{\pgfqpoint{1.648852in}{0.602534in}}%
\pgfpathlineto{\pgfqpoint{1.649408in}{0.604553in}}%
\pgfpathlineto{\pgfqpoint{1.649964in}{0.607413in}}%
\pgfpathlineto{\pgfqpoint{1.650520in}{0.603048in}}%
\pgfpathlineto{\pgfqpoint{1.651076in}{0.612730in}}%
\pgfpathlineto{\pgfqpoint{1.651631in}{0.608016in}}%
\pgfpathlineto{\pgfqpoint{1.652187in}{0.606972in}}%
\pgfpathlineto{\pgfqpoint{1.652743in}{0.603205in}}%
\pgfpathlineto{\pgfqpoint{1.653299in}{0.607743in}}%
\pgfpathlineto{\pgfqpoint{1.653855in}{0.605166in}}%
\pgfpathlineto{\pgfqpoint{1.654967in}{0.606717in}}%
\pgfpathlineto{\pgfqpoint{1.655523in}{0.601410in}}%
\pgfpathlineto{\pgfqpoint{1.656078in}{0.601825in}}%
\pgfpathlineto{\pgfqpoint{1.657190in}{0.612922in}}%
\pgfpathlineto{\pgfqpoint{1.657746in}{0.610496in}}%
\pgfpathlineto{\pgfqpoint{1.658302in}{0.610735in}}%
\pgfpathlineto{\pgfqpoint{1.658858in}{0.603535in}}%
\pgfpathlineto{\pgfqpoint{1.659414in}{0.607715in}}%
\pgfpathlineto{\pgfqpoint{1.659969in}{0.609957in}}%
\pgfpathlineto{\pgfqpoint{1.660525in}{0.608744in}}%
\pgfpathlineto{\pgfqpoint{1.661637in}{0.603427in}}%
\pgfpathlineto{\pgfqpoint{1.663305in}{0.611384in}}%
\pgfpathlineto{\pgfqpoint{1.664972in}{0.604076in}}%
\pgfpathlineto{\pgfqpoint{1.666084in}{0.607183in}}%
\pgfpathlineto{\pgfqpoint{1.666640in}{0.601868in}}%
\pgfpathlineto{\pgfqpoint{1.667196in}{0.605778in}}%
\pgfpathlineto{\pgfqpoint{1.667751in}{0.605910in}}%
\pgfpathlineto{\pgfqpoint{1.668307in}{0.610642in}}%
\pgfpathlineto{\pgfqpoint{1.668863in}{0.605430in}}%
\pgfpathlineto{\pgfqpoint{1.669419in}{0.606352in}}%
\pgfpathlineto{\pgfqpoint{1.669975in}{0.605686in}}%
\pgfpathlineto{\pgfqpoint{1.670531in}{0.609134in}}%
\pgfpathlineto{\pgfqpoint{1.671087in}{0.606900in}}%
\pgfpathlineto{\pgfqpoint{1.671642in}{0.605780in}}%
\pgfpathlineto{\pgfqpoint{1.672198in}{0.606063in}}%
\pgfpathlineto{\pgfqpoint{1.672754in}{0.608821in}}%
\pgfpathlineto{\pgfqpoint{1.673310in}{0.600359in}}%
\pgfpathlineto{\pgfqpoint{1.673866in}{0.601406in}}%
\pgfpathlineto{\pgfqpoint{1.675534in}{0.607218in}}%
\pgfpathlineto{\pgfqpoint{1.676089in}{0.601984in}}%
\pgfpathlineto{\pgfqpoint{1.676645in}{0.610220in}}%
\pgfpathlineto{\pgfqpoint{1.677201in}{0.605960in}}%
\pgfpathlineto{\pgfqpoint{1.677757in}{0.607556in}}%
\pgfpathlineto{\pgfqpoint{1.678313in}{0.603605in}}%
\pgfpathlineto{\pgfqpoint{1.678869in}{0.604552in}}%
\pgfpathlineto{\pgfqpoint{1.679425in}{0.604793in}}%
\pgfpathlineto{\pgfqpoint{1.680536in}{0.607542in}}%
\pgfpathlineto{\pgfqpoint{1.681092in}{0.607133in}}%
\pgfpathlineto{\pgfqpoint{1.681648in}{0.607883in}}%
\pgfpathlineto{\pgfqpoint{1.682204in}{0.602261in}}%
\pgfpathlineto{\pgfqpoint{1.682760in}{0.602712in}}%
\pgfpathlineto{\pgfqpoint{1.683316in}{0.609944in}}%
\pgfpathlineto{\pgfqpoint{1.683871in}{0.609445in}}%
\pgfpathlineto{\pgfqpoint{1.685539in}{0.606455in}}%
\pgfpathlineto{\pgfqpoint{1.686095in}{0.616446in}}%
\pgfpathlineto{\pgfqpoint{1.687762in}{0.605585in}}%
\pgfpathlineto{\pgfqpoint{1.688318in}{0.602626in}}%
\pgfpathlineto{\pgfqpoint{1.688874in}{0.617875in}}%
\pgfpathlineto{\pgfqpoint{1.689430in}{0.606672in}}%
\pgfpathlineto{\pgfqpoint{1.689986in}{0.609242in}}%
\pgfpathlineto{\pgfqpoint{1.690542in}{0.606716in}}%
\pgfpathlineto{\pgfqpoint{1.691653in}{0.603528in}}%
\pgfpathlineto{\pgfqpoint{1.692209in}{0.609713in}}%
\pgfpathlineto{\pgfqpoint{1.692765in}{0.607315in}}%
\pgfpathlineto{\pgfqpoint{1.694433in}{0.603616in}}%
\pgfpathlineto{\pgfqpoint{1.695545in}{0.610363in}}%
\pgfpathlineto{\pgfqpoint{1.696656in}{0.602051in}}%
\pgfpathlineto{\pgfqpoint{1.697212in}{0.602793in}}%
\pgfpathlineto{\pgfqpoint{1.697768in}{0.612267in}}%
\pgfpathlineto{\pgfqpoint{1.698880in}{0.601034in}}%
\pgfpathlineto{\pgfqpoint{1.699436in}{0.607565in}}%
\pgfpathlineto{\pgfqpoint{1.699991in}{0.602176in}}%
\pgfpathlineto{\pgfqpoint{1.700547in}{0.605271in}}%
\pgfpathlineto{\pgfqpoint{1.701103in}{0.601796in}}%
\pgfpathlineto{\pgfqpoint{1.702215in}{0.607403in}}%
\pgfpathlineto{\pgfqpoint{1.702771in}{0.603279in}}%
\pgfpathlineto{\pgfqpoint{1.703327in}{0.609189in}}%
\pgfpathlineto{\pgfqpoint{1.703882in}{0.606041in}}%
\pgfpathlineto{\pgfqpoint{1.704438in}{0.608394in}}%
\pgfpathlineto{\pgfqpoint{1.705550in}{0.602768in}}%
\pgfpathlineto{\pgfqpoint{1.706106in}{0.605504in}}%
\pgfpathlineto{\pgfqpoint{1.706662in}{0.602055in}}%
\pgfpathlineto{\pgfqpoint{1.707218in}{0.602262in}}%
\pgfpathlineto{\pgfqpoint{1.707773in}{0.606040in}}%
\pgfpathlineto{\pgfqpoint{1.708329in}{0.604123in}}%
\pgfpathlineto{\pgfqpoint{1.708885in}{0.602872in}}%
\pgfpathlineto{\pgfqpoint{1.710553in}{0.612504in}}%
\pgfpathlineto{\pgfqpoint{1.712220in}{0.602363in}}%
\pgfpathlineto{\pgfqpoint{1.712776in}{0.602985in}}%
\pgfpathlineto{\pgfqpoint{1.714444in}{0.608460in}}%
\pgfpathlineto{\pgfqpoint{1.715000in}{0.603370in}}%
\pgfpathlineto{\pgfqpoint{1.715556in}{0.609422in}}%
\pgfpathlineto{\pgfqpoint{1.716111in}{0.604873in}}%
\pgfpathlineto{\pgfqpoint{1.716667in}{0.603790in}}%
\pgfpathlineto{\pgfqpoint{1.717223in}{0.605816in}}%
\pgfpathlineto{\pgfqpoint{1.717779in}{0.602101in}}%
\pgfpathlineto{\pgfqpoint{1.718335in}{0.604823in}}%
\pgfpathlineto{\pgfqpoint{1.719447in}{0.606939in}}%
\pgfpathlineto{\pgfqpoint{1.720002in}{0.600849in}}%
\pgfpathlineto{\pgfqpoint{1.720558in}{0.612030in}}%
\pgfpathlineto{\pgfqpoint{1.721114in}{0.606410in}}%
\pgfpathlineto{\pgfqpoint{1.723893in}{0.601147in}}%
\pgfpathlineto{\pgfqpoint{1.725005in}{0.604554in}}%
\pgfpathlineto{\pgfqpoint{1.725561in}{0.606505in}}%
\pgfpathlineto{\pgfqpoint{1.726673in}{0.600689in}}%
\pgfpathlineto{\pgfqpoint{1.727229in}{0.605545in}}%
\pgfpathlineto{\pgfqpoint{1.727784in}{0.604251in}}%
\pgfpathlineto{\pgfqpoint{1.728340in}{0.601067in}}%
\pgfpathlineto{\pgfqpoint{1.728896in}{0.607887in}}%
\pgfpathlineto{\pgfqpoint{1.729452in}{0.602489in}}%
\pgfpathlineto{\pgfqpoint{1.731120in}{0.604302in}}%
\pgfpathlineto{\pgfqpoint{1.732231in}{0.603987in}}%
\pgfpathlineto{\pgfqpoint{1.732787in}{0.601685in}}%
\pgfpathlineto{\pgfqpoint{1.733343in}{0.605632in}}%
\pgfpathlineto{\pgfqpoint{1.733899in}{0.605551in}}%
\pgfpathlineto{\pgfqpoint{1.734455in}{0.602170in}}%
\pgfpathlineto{\pgfqpoint{1.735011in}{0.605932in}}%
\pgfpathlineto{\pgfqpoint{1.735567in}{0.603994in}}%
\pgfpathlineto{\pgfqpoint{1.736122in}{0.602988in}}%
\pgfpathlineto{\pgfqpoint{1.736678in}{0.604047in}}%
\pgfpathlineto{\pgfqpoint{1.738902in}{0.601444in}}%
\pgfpathlineto{\pgfqpoint{1.741125in}{0.604008in}}%
\pgfpathlineto{\pgfqpoint{1.741681in}{0.601576in}}%
\pgfpathlineto{\pgfqpoint{1.742237in}{0.602223in}}%
\pgfpathlineto{\pgfqpoint{1.743904in}{0.603879in}}%
\pgfpathlineto{\pgfqpoint{1.744460in}{0.600340in}}%
\pgfpathlineto{\pgfqpoint{1.745016in}{0.605694in}}%
\pgfpathlineto{\pgfqpoint{1.745572in}{0.604918in}}%
\pgfpathlineto{\pgfqpoint{1.747240in}{0.600773in}}%
\pgfpathlineto{\pgfqpoint{1.748351in}{0.604864in}}%
\pgfpathlineto{\pgfqpoint{1.750019in}{0.602706in}}%
\pgfpathlineto{\pgfqpoint{1.750575in}{0.603544in}}%
\pgfpathlineto{\pgfqpoint{1.751131in}{0.602578in}}%
\pgfpathlineto{\pgfqpoint{1.753910in}{0.603868in}}%
\pgfpathlineto{\pgfqpoint{1.754466in}{0.604059in}}%
\pgfpathlineto{\pgfqpoint{1.755022in}{0.607060in}}%
\pgfpathlineto{\pgfqpoint{1.755578in}{0.602814in}}%
\pgfpathlineto{\pgfqpoint{1.756133in}{0.611266in}}%
\pgfpathlineto{\pgfqpoint{1.757245in}{0.600842in}}%
\pgfpathlineto{\pgfqpoint{1.758913in}{0.603611in}}%
\pgfpathlineto{\pgfqpoint{1.761136in}{0.601929in}}%
\pgfpathlineto{\pgfqpoint{1.763360in}{0.603228in}}%
\pgfpathlineto{\pgfqpoint{1.763915in}{0.603885in}}%
\pgfpathlineto{\pgfqpoint{1.764471in}{0.603331in}}%
\pgfpathlineto{\pgfqpoint{1.765027in}{0.602548in}}%
\pgfpathlineto{\pgfqpoint{1.765583in}{0.603513in}}%
\pgfpathlineto{\pgfqpoint{1.767251in}{0.603205in}}%
\pgfpathlineto{\pgfqpoint{1.767807in}{0.605356in}}%
\pgfpathlineto{\pgfqpoint{1.768362in}{0.601635in}}%
\pgfpathlineto{\pgfqpoint{1.768918in}{0.601890in}}%
\pgfpathlineto{\pgfqpoint{1.770586in}{0.602519in}}%
\pgfpathlineto{\pgfqpoint{1.771142in}{0.604295in}}%
\pgfpathlineto{\pgfqpoint{1.771698in}{0.602993in}}%
\pgfpathlineto{\pgfqpoint{1.772253in}{0.601709in}}%
\pgfpathlineto{\pgfqpoint{1.772809in}{0.604624in}}%
\pgfpathlineto{\pgfqpoint{1.773365in}{0.604470in}}%
\pgfpathlineto{\pgfqpoint{1.774477in}{0.600388in}}%
\pgfpathlineto{\pgfqpoint{1.775589in}{0.604954in}}%
\pgfpathlineto{\pgfqpoint{1.776144in}{0.604725in}}%
\pgfpathlineto{\pgfqpoint{1.776700in}{0.605559in}}%
\pgfpathlineto{\pgfqpoint{1.777256in}{0.602598in}}%
\pgfpathlineto{\pgfqpoint{1.777812in}{0.604884in}}%
\pgfpathlineto{\pgfqpoint{1.778368in}{0.606523in}}%
\pgfpathlineto{\pgfqpoint{1.778924in}{0.601583in}}%
\pgfpathlineto{\pgfqpoint{1.779480in}{0.603702in}}%
\pgfpathlineto{\pgfqpoint{1.780035in}{0.602728in}}%
\pgfpathlineto{\pgfqpoint{1.780591in}{0.605185in}}%
\pgfpathlineto{\pgfqpoint{1.781147in}{0.601115in}}%
\pgfpathlineto{\pgfqpoint{1.781703in}{0.602943in}}%
\pgfpathlineto{\pgfqpoint{1.782815in}{0.605467in}}%
\pgfpathlineto{\pgfqpoint{1.783371in}{0.601490in}}%
\pgfpathlineto{\pgfqpoint{1.783926in}{0.603557in}}%
\pgfpathlineto{\pgfqpoint{1.785038in}{0.605559in}}%
\pgfpathlineto{\pgfqpoint{1.785594in}{0.603822in}}%
\pgfpathlineto{\pgfqpoint{1.786150in}{0.606199in}}%
\pgfpathlineto{\pgfqpoint{1.786706in}{0.601586in}}%
\pgfpathlineto{\pgfqpoint{1.787262in}{0.604317in}}%
\pgfpathlineto{\pgfqpoint{1.788929in}{0.602333in}}%
\pgfpathlineto{\pgfqpoint{1.789485in}{0.602832in}}%
\pgfpathlineto{\pgfqpoint{1.790041in}{0.602028in}}%
\pgfpathlineto{\pgfqpoint{1.790597in}{0.605860in}}%
\pgfpathlineto{\pgfqpoint{1.791153in}{0.603167in}}%
\pgfpathlineto{\pgfqpoint{1.791709in}{0.602822in}}%
\pgfpathlineto{\pgfqpoint{1.792264in}{0.607638in}}%
\pgfpathlineto{\pgfqpoint{1.792820in}{0.600762in}}%
\pgfpathlineto{\pgfqpoint{1.793376in}{0.601986in}}%
\pgfpathlineto{\pgfqpoint{1.795044in}{0.605916in}}%
\pgfpathlineto{\pgfqpoint{1.795600in}{0.603631in}}%
\pgfpathlineto{\pgfqpoint{1.796155in}{0.604268in}}%
\pgfpathlineto{\pgfqpoint{1.797267in}{0.605904in}}%
\pgfpathlineto{\pgfqpoint{1.797823in}{0.600187in}}%
\pgfpathlineto{\pgfqpoint{1.798379in}{0.605698in}}%
\pgfpathlineto{\pgfqpoint{1.798935in}{0.602219in}}%
\pgfpathlineto{\pgfqpoint{1.799491in}{0.603061in}}%
\pgfpathlineto{\pgfqpoint{1.800046in}{0.604516in}}%
\pgfpathlineto{\pgfqpoint{1.800602in}{0.602045in}}%
\pgfpathlineto{\pgfqpoint{1.802270in}{0.609699in}}%
\pgfpathlineto{\pgfqpoint{1.802826in}{0.603617in}}%
\pgfpathlineto{\pgfqpoint{1.803937in}{0.615805in}}%
\pgfpathlineto{\pgfqpoint{1.805049in}{0.602450in}}%
\pgfpathlineto{\pgfqpoint{1.805605in}{0.602832in}}%
\pgfpathlineto{\pgfqpoint{1.806161in}{0.610180in}}%
\pgfpathlineto{\pgfqpoint{1.806717in}{0.610097in}}%
\pgfpathlineto{\pgfqpoint{1.808940in}{0.604322in}}%
\pgfpathlineto{\pgfqpoint{1.810608in}{0.607732in}}%
\pgfpathlineto{\pgfqpoint{1.812831in}{0.605612in}}%
\pgfpathlineto{\pgfqpoint{1.813387in}{0.602258in}}%
\pgfpathlineto{\pgfqpoint{1.813943in}{0.605050in}}%
\pgfpathlineto{\pgfqpoint{1.815055in}{0.608097in}}%
\pgfpathlineto{\pgfqpoint{1.816722in}{0.601284in}}%
\pgfpathlineto{\pgfqpoint{1.817834in}{0.605801in}}%
\pgfpathlineto{\pgfqpoint{1.819502in}{0.603122in}}%
\pgfpathlineto{\pgfqpoint{1.820057in}{0.608994in}}%
\pgfpathlineto{\pgfqpoint{1.820613in}{0.607875in}}%
\pgfpathlineto{\pgfqpoint{1.821725in}{0.602970in}}%
\pgfpathlineto{\pgfqpoint{1.822281in}{0.607078in}}%
\pgfpathlineto{\pgfqpoint{1.822837in}{0.602741in}}%
\pgfpathlineto{\pgfqpoint{1.823393in}{0.607171in}}%
\pgfpathlineto{\pgfqpoint{1.823949in}{0.604224in}}%
\pgfpathlineto{\pgfqpoint{1.825060in}{0.601323in}}%
\pgfpathlineto{\pgfqpoint{1.825616in}{0.612124in}}%
\pgfpathlineto{\pgfqpoint{1.826172in}{0.601515in}}%
\pgfpathlineto{\pgfqpoint{1.827840in}{0.610321in}}%
\pgfpathlineto{\pgfqpoint{1.828951in}{0.606516in}}%
\pgfpathlineto{\pgfqpoint{1.829507in}{0.608320in}}%
\pgfpathlineto{\pgfqpoint{1.830063in}{0.612160in}}%
\pgfpathlineto{\pgfqpoint{1.830619in}{0.601485in}}%
\pgfpathlineto{\pgfqpoint{1.831175in}{0.607200in}}%
\pgfpathlineto{\pgfqpoint{1.831731in}{0.604876in}}%
\pgfpathlineto{\pgfqpoint{1.832286in}{0.606626in}}%
\pgfpathlineto{\pgfqpoint{1.833398in}{0.611125in}}%
\pgfpathlineto{\pgfqpoint{1.833954in}{0.603191in}}%
\pgfpathlineto{\pgfqpoint{1.834510in}{0.609942in}}%
\pgfpathlineto{\pgfqpoint{1.835066in}{0.603811in}}%
\pgfpathlineto{\pgfqpoint{1.836177in}{0.611432in}}%
\pgfpathlineto{\pgfqpoint{1.836733in}{0.609570in}}%
\pgfpathlineto{\pgfqpoint{1.837289in}{0.601211in}}%
\pgfpathlineto{\pgfqpoint{1.837845in}{0.602747in}}%
\pgfpathlineto{\pgfqpoint{1.838401in}{0.607526in}}%
\pgfpathlineto{\pgfqpoint{1.838957in}{0.604011in}}%
\pgfpathlineto{\pgfqpoint{1.840624in}{0.611019in}}%
\pgfpathlineto{\pgfqpoint{1.841736in}{0.603052in}}%
\pgfpathlineto{\pgfqpoint{1.842292in}{0.603180in}}%
\pgfpathlineto{\pgfqpoint{1.842848in}{0.610124in}}%
\pgfpathlineto{\pgfqpoint{1.843404in}{0.604006in}}%
\pgfpathlineto{\pgfqpoint{1.843960in}{0.609373in}}%
\pgfpathlineto{\pgfqpoint{1.844515in}{0.606736in}}%
\pgfpathlineto{\pgfqpoint{1.845071in}{0.605811in}}%
\pgfpathlineto{\pgfqpoint{1.845627in}{0.600035in}}%
\pgfpathlineto{\pgfqpoint{1.846183in}{0.602838in}}%
\pgfpathlineto{\pgfqpoint{1.846739in}{0.602493in}}%
\pgfpathlineto{\pgfqpoint{1.847295in}{0.605242in}}%
\pgfpathlineto{\pgfqpoint{1.847851in}{0.603355in}}%
\pgfpathlineto{\pgfqpoint{1.848406in}{0.603497in}}%
\pgfpathlineto{\pgfqpoint{1.850074in}{0.610889in}}%
\pgfpathlineto{\pgfqpoint{1.850630in}{0.602340in}}%
\pgfpathlineto{\pgfqpoint{1.851186in}{0.604748in}}%
\pgfpathlineto{\pgfqpoint{1.852297in}{0.609622in}}%
\pgfpathlineto{\pgfqpoint{1.853409in}{0.602979in}}%
\pgfpathlineto{\pgfqpoint{1.853965in}{0.607260in}}%
\pgfpathlineto{\pgfqpoint{1.854521in}{0.602618in}}%
\pgfpathlineto{\pgfqpoint{1.855077in}{0.606822in}}%
\pgfpathlineto{\pgfqpoint{1.856188in}{0.602960in}}%
\pgfpathlineto{\pgfqpoint{1.858412in}{0.607476in}}%
\pgfpathlineto{\pgfqpoint{1.858968in}{0.605221in}}%
\pgfpathlineto{\pgfqpoint{1.859524in}{0.608673in}}%
\pgfpathlineto{\pgfqpoint{1.860079in}{0.607582in}}%
\pgfpathlineto{\pgfqpoint{1.860635in}{0.605432in}}%
\pgfpathlineto{\pgfqpoint{1.861191in}{0.612926in}}%
\pgfpathlineto{\pgfqpoint{1.861747in}{0.609901in}}%
\pgfpathlineto{\pgfqpoint{1.862303in}{0.609422in}}%
\pgfpathlineto{\pgfqpoint{1.863415in}{0.614824in}}%
\pgfpathlineto{\pgfqpoint{1.863971in}{0.603690in}}%
\pgfpathlineto{\pgfqpoint{1.864526in}{0.610566in}}%
\pgfpathlineto{\pgfqpoint{1.865082in}{0.612499in}}%
\pgfpathlineto{\pgfqpoint{1.865638in}{0.611739in}}%
\pgfpathlineto{\pgfqpoint{1.866194in}{0.604449in}}%
\pgfpathlineto{\pgfqpoint{1.867862in}{0.616032in}}%
\pgfpathlineto{\pgfqpoint{1.868973in}{0.607951in}}%
\pgfpathlineto{\pgfqpoint{1.869529in}{0.608407in}}%
\pgfpathlineto{\pgfqpoint{1.870085in}{0.607342in}}%
\pgfpathlineto{\pgfqpoint{1.870641in}{0.613435in}}%
\pgfpathlineto{\pgfqpoint{1.871197in}{0.605388in}}%
\pgfpathlineto{\pgfqpoint{1.871753in}{0.609738in}}%
\pgfpathlineto{\pgfqpoint{1.872864in}{0.605099in}}%
\pgfpathlineto{\pgfqpoint{1.874532in}{0.616009in}}%
\pgfpathlineto{\pgfqpoint{1.875644in}{0.602406in}}%
\pgfpathlineto{\pgfqpoint{1.878423in}{0.613558in}}%
\pgfpathlineto{\pgfqpoint{1.878979in}{0.602951in}}%
\pgfpathlineto{\pgfqpoint{1.879535in}{0.613424in}}%
\pgfpathlineto{\pgfqpoint{1.881202in}{0.606189in}}%
\pgfpathlineto{\pgfqpoint{1.881758in}{0.607841in}}%
\pgfpathlineto{\pgfqpoint{1.882870in}{0.605028in}}%
\pgfpathlineto{\pgfqpoint{1.883982in}{0.610559in}}%
\pgfpathlineto{\pgfqpoint{1.884537in}{0.604468in}}%
\pgfpathlineto{\pgfqpoint{1.885093in}{0.617307in}}%
\pgfpathlineto{\pgfqpoint{1.885649in}{0.609691in}}%
\pgfpathlineto{\pgfqpoint{1.886205in}{0.607449in}}%
\pgfpathlineto{\pgfqpoint{1.887317in}{0.615206in}}%
\pgfpathlineto{\pgfqpoint{1.888984in}{0.602723in}}%
\pgfpathlineto{\pgfqpoint{1.890652in}{0.618031in}}%
\pgfpathlineto{\pgfqpoint{1.891764in}{0.613439in}}%
\pgfpathlineto{\pgfqpoint{1.892319in}{0.607678in}}%
\pgfpathlineto{\pgfqpoint{1.892875in}{0.611407in}}%
\pgfpathlineto{\pgfqpoint{1.893431in}{0.610953in}}%
\pgfpathlineto{\pgfqpoint{1.893987in}{0.615274in}}%
\pgfpathlineto{\pgfqpoint{1.895099in}{0.605088in}}%
\pgfpathlineto{\pgfqpoint{1.895655in}{0.609254in}}%
\pgfpathlineto{\pgfqpoint{1.896210in}{0.607198in}}%
\pgfpathlineto{\pgfqpoint{1.896766in}{0.603388in}}%
\pgfpathlineto{\pgfqpoint{1.897878in}{0.609908in}}%
\pgfpathlineto{\pgfqpoint{1.899546in}{0.602609in}}%
\pgfpathlineto{\pgfqpoint{1.901213in}{0.608445in}}%
\pgfpathlineto{\pgfqpoint{1.901769in}{0.601505in}}%
\pgfpathlineto{\pgfqpoint{1.902325in}{0.609930in}}%
\pgfpathlineto{\pgfqpoint{1.902881in}{0.606529in}}%
\pgfpathlineto{\pgfqpoint{1.903993in}{0.607926in}}%
\pgfpathlineto{\pgfqpoint{1.905104in}{0.601807in}}%
\pgfpathlineto{\pgfqpoint{1.907328in}{0.611440in}}%
\pgfpathlineto{\pgfqpoint{1.907884in}{0.606128in}}%
\pgfpathlineto{\pgfqpoint{1.908439in}{0.610241in}}%
\pgfpathlineto{\pgfqpoint{1.908995in}{0.606814in}}%
\pgfpathlineto{\pgfqpoint{1.909551in}{0.612250in}}%
\pgfpathlineto{\pgfqpoint{1.910107in}{0.603394in}}%
\pgfpathlineto{\pgfqpoint{1.910663in}{0.606875in}}%
\pgfpathlineto{\pgfqpoint{1.911219in}{0.608255in}}%
\pgfpathlineto{\pgfqpoint{1.912886in}{0.603460in}}%
\pgfpathlineto{\pgfqpoint{1.913998in}{0.602586in}}%
\pgfpathlineto{\pgfqpoint{1.915666in}{0.610638in}}%
\pgfpathlineto{\pgfqpoint{1.916221in}{0.602272in}}%
\pgfpathlineto{\pgfqpoint{1.916777in}{0.603912in}}%
\pgfpathlineto{\pgfqpoint{1.919001in}{0.614728in}}%
\pgfpathlineto{\pgfqpoint{1.919557in}{0.606291in}}%
\pgfpathlineto{\pgfqpoint{1.920113in}{0.608031in}}%
\pgfpathlineto{\pgfqpoint{1.921780in}{0.616683in}}%
\pgfpathlineto{\pgfqpoint{1.923448in}{0.606270in}}%
\pgfpathlineto{\pgfqpoint{1.924559in}{0.614769in}}%
\pgfpathlineto{\pgfqpoint{1.925115in}{0.602094in}}%
\pgfpathlineto{\pgfqpoint{1.925671in}{0.603551in}}%
\pgfpathlineto{\pgfqpoint{1.926783in}{0.619989in}}%
\pgfpathlineto{\pgfqpoint{1.927339in}{0.606787in}}%
\pgfpathlineto{\pgfqpoint{1.927895in}{0.613625in}}%
\pgfpathlineto{\pgfqpoint{1.929562in}{0.605560in}}%
\pgfpathlineto{\pgfqpoint{1.930118in}{0.605656in}}%
\pgfpathlineto{\pgfqpoint{1.930674in}{0.602439in}}%
\pgfpathlineto{\pgfqpoint{1.931230in}{0.608202in}}%
\pgfpathlineto{\pgfqpoint{1.931786in}{0.602430in}}%
\pgfpathlineto{\pgfqpoint{1.932341in}{0.602207in}}%
\pgfpathlineto{\pgfqpoint{1.934009in}{0.613600in}}%
\pgfpathlineto{\pgfqpoint{1.935121in}{0.603601in}}%
\pgfpathlineto{\pgfqpoint{1.935677in}{0.603946in}}%
\pgfpathlineto{\pgfqpoint{1.936788in}{0.609184in}}%
\pgfpathlineto{\pgfqpoint{1.937344in}{0.602128in}}%
\pgfpathlineto{\pgfqpoint{1.937900in}{0.613545in}}%
\pgfpathlineto{\pgfqpoint{1.938456in}{0.605035in}}%
\pgfpathlineto{\pgfqpoint{1.939012in}{0.605267in}}%
\pgfpathlineto{\pgfqpoint{1.939568in}{0.606892in}}%
\pgfpathlineto{\pgfqpoint{1.940124in}{0.612184in}}%
\pgfpathlineto{\pgfqpoint{1.940679in}{0.606040in}}%
\pgfpathlineto{\pgfqpoint{1.941235in}{0.606887in}}%
\pgfpathlineto{\pgfqpoint{1.944015in}{0.615388in}}%
\pgfpathlineto{\pgfqpoint{1.944570in}{0.601165in}}%
\pgfpathlineto{\pgfqpoint{1.945126in}{0.610313in}}%
\pgfpathlineto{\pgfqpoint{1.946794in}{0.600986in}}%
\pgfpathlineto{\pgfqpoint{1.948461in}{0.619321in}}%
\pgfpathlineto{\pgfqpoint{1.949017in}{0.603899in}}%
\pgfpathlineto{\pgfqpoint{1.949573in}{0.610717in}}%
\pgfpathlineto{\pgfqpoint{1.950129in}{0.609434in}}%
\pgfpathlineto{\pgfqpoint{1.950685in}{0.603582in}}%
\pgfpathlineto{\pgfqpoint{1.951241in}{0.615350in}}%
\pgfpathlineto{\pgfqpoint{1.951797in}{0.611363in}}%
\pgfpathlineto{\pgfqpoint{1.952352in}{0.602273in}}%
\pgfpathlineto{\pgfqpoint{1.952908in}{0.615249in}}%
\pgfpathlineto{\pgfqpoint{1.953464in}{0.603843in}}%
\pgfpathlineto{\pgfqpoint{1.954020in}{0.602486in}}%
\pgfpathlineto{\pgfqpoint{1.955132in}{0.613037in}}%
\pgfpathlineto{\pgfqpoint{1.955688in}{0.609680in}}%
\pgfpathlineto{\pgfqpoint{1.956244in}{0.601054in}}%
\pgfpathlineto{\pgfqpoint{1.956799in}{0.604199in}}%
\pgfpathlineto{\pgfqpoint{1.957355in}{0.606326in}}%
\pgfpathlineto{\pgfqpoint{1.957911in}{0.604678in}}%
\pgfpathlineto{\pgfqpoint{1.958467in}{0.605585in}}%
\pgfpathlineto{\pgfqpoint{1.959023in}{0.602392in}}%
\pgfpathlineto{\pgfqpoint{1.959579in}{0.605049in}}%
\pgfpathlineto{\pgfqpoint{1.960135in}{0.606435in}}%
\pgfpathlineto{\pgfqpoint{1.961246in}{0.600841in}}%
\pgfpathlineto{\pgfqpoint{1.961802in}{0.606048in}}%
\pgfpathlineto{\pgfqpoint{1.962358in}{0.605957in}}%
\pgfpathlineto{\pgfqpoint{1.962914in}{0.602749in}}%
\pgfpathlineto{\pgfqpoint{1.965137in}{0.610638in}}%
\pgfpathlineto{\pgfqpoint{1.966249in}{0.601465in}}%
\pgfpathlineto{\pgfqpoint{1.966805in}{0.606694in}}%
\pgfpathlineto{\pgfqpoint{1.967361in}{0.605818in}}%
\pgfpathlineto{\pgfqpoint{1.967917in}{0.603325in}}%
\pgfpathlineto{\pgfqpoint{1.968472in}{0.608524in}}%
\pgfpathlineto{\pgfqpoint{1.969028in}{0.606125in}}%
\pgfpathlineto{\pgfqpoint{1.970696in}{0.602862in}}%
\pgfpathlineto{\pgfqpoint{1.971252in}{0.601491in}}%
\pgfpathlineto{\pgfqpoint{1.972363in}{0.609989in}}%
\pgfpathlineto{\pgfqpoint{1.972919in}{0.606062in}}%
\pgfpathlineto{\pgfqpoint{1.973475in}{0.604990in}}%
\pgfpathlineto{\pgfqpoint{1.974031in}{0.607323in}}%
\pgfpathlineto{\pgfqpoint{1.974587in}{0.601170in}}%
\pgfpathlineto{\pgfqpoint{1.975143in}{0.601461in}}%
\pgfpathlineto{\pgfqpoint{1.976810in}{0.611924in}}%
\pgfpathlineto{\pgfqpoint{1.977366in}{0.601869in}}%
\pgfpathlineto{\pgfqpoint{1.977922in}{0.606376in}}%
\pgfpathlineto{\pgfqpoint{1.978478in}{0.612563in}}%
\pgfpathlineto{\pgfqpoint{1.979034in}{0.603149in}}%
\pgfpathlineto{\pgfqpoint{1.979590in}{0.612317in}}%
\pgfpathlineto{\pgfqpoint{1.980701in}{0.604395in}}%
\pgfpathlineto{\pgfqpoint{1.981257in}{0.610905in}}%
\pgfpathlineto{\pgfqpoint{1.981813in}{0.610004in}}%
\pgfpathlineto{\pgfqpoint{1.982925in}{0.610170in}}%
\pgfpathlineto{\pgfqpoint{1.983481in}{0.603190in}}%
\pgfpathlineto{\pgfqpoint{1.984037in}{0.612483in}}%
\pgfpathlineto{\pgfqpoint{1.984592in}{0.606121in}}%
\pgfpathlineto{\pgfqpoint{1.985148in}{0.605139in}}%
\pgfpathlineto{\pgfqpoint{1.985704in}{0.615197in}}%
\pgfpathlineto{\pgfqpoint{1.986260in}{0.614211in}}%
\pgfpathlineto{\pgfqpoint{1.986816in}{0.605826in}}%
\pgfpathlineto{\pgfqpoint{1.987372in}{0.609679in}}%
\pgfpathlineto{\pgfqpoint{1.987928in}{0.608927in}}%
\pgfpathlineto{\pgfqpoint{1.988483in}{0.602191in}}%
\pgfpathlineto{\pgfqpoint{1.989039in}{0.602932in}}%
\pgfpathlineto{\pgfqpoint{1.989595in}{0.603701in}}%
\pgfpathlineto{\pgfqpoint{1.990151in}{0.608499in}}%
\pgfpathlineto{\pgfqpoint{1.990707in}{0.600638in}}%
\pgfpathlineto{\pgfqpoint{1.991263in}{0.603718in}}%
\pgfpathlineto{\pgfqpoint{1.991819in}{0.602438in}}%
\pgfpathlineto{\pgfqpoint{1.992930in}{0.612525in}}%
\pgfpathlineto{\pgfqpoint{1.993486in}{0.602385in}}%
\pgfpathlineto{\pgfqpoint{1.994042in}{0.607245in}}%
\pgfpathlineto{\pgfqpoint{1.994598in}{0.605200in}}%
\pgfpathlineto{\pgfqpoint{1.995710in}{0.610169in}}%
\pgfpathlineto{\pgfqpoint{1.996266in}{0.602887in}}%
\pgfpathlineto{\pgfqpoint{1.996821in}{0.603052in}}%
\pgfpathlineto{\pgfqpoint{1.998489in}{0.605689in}}%
\pgfpathlineto{\pgfqpoint{1.999045in}{0.602815in}}%
\pgfpathlineto{\pgfqpoint{2.000712in}{0.614721in}}%
\pgfpathlineto{\pgfqpoint{2.001268in}{0.600466in}}%
\pgfpathlineto{\pgfqpoint{2.001824in}{0.608333in}}%
\pgfpathlineto{\pgfqpoint{2.002380in}{0.603100in}}%
\pgfpathlineto{\pgfqpoint{2.002936in}{0.613769in}}%
\pgfpathlineto{\pgfqpoint{2.003492in}{0.606335in}}%
\pgfpathlineto{\pgfqpoint{2.005715in}{0.612890in}}%
\pgfpathlineto{\pgfqpoint{2.006271in}{0.604595in}}%
\pgfpathlineto{\pgfqpoint{2.006827in}{0.606397in}}%
\pgfpathlineto{\pgfqpoint{2.007383in}{0.608335in}}%
\pgfpathlineto{\pgfqpoint{2.007939in}{0.607776in}}%
\pgfpathlineto{\pgfqpoint{2.009050in}{0.606893in}}%
\pgfpathlineto{\pgfqpoint{2.009606in}{0.602779in}}%
\pgfpathlineto{\pgfqpoint{2.010162in}{0.612250in}}%
\pgfpathlineto{\pgfqpoint{2.010718in}{0.609162in}}%
\pgfpathlineto{\pgfqpoint{2.011274in}{0.608307in}}%
\pgfpathlineto{\pgfqpoint{2.011830in}{0.604090in}}%
\pgfpathlineto{\pgfqpoint{2.012386in}{0.606654in}}%
\pgfpathlineto{\pgfqpoint{2.012941in}{0.609839in}}%
\pgfpathlineto{\pgfqpoint{2.014609in}{0.601232in}}%
\pgfpathlineto{\pgfqpoint{2.016277in}{0.605119in}}%
\pgfpathlineto{\pgfqpoint{2.017944in}{0.601561in}}%
\pgfpathlineto{\pgfqpoint{2.018500in}{0.602167in}}%
\pgfpathlineto{\pgfqpoint{2.019056in}{0.606296in}}%
\pgfpathlineto{\pgfqpoint{2.019612in}{0.603496in}}%
\pgfpathlineto{\pgfqpoint{2.020168in}{0.601778in}}%
\pgfpathlineto{\pgfqpoint{2.020723in}{0.603608in}}%
\pgfpathlineto{\pgfqpoint{2.022947in}{0.604669in}}%
\pgfpathlineto{\pgfqpoint{2.023503in}{0.602905in}}%
\pgfpathlineto{\pgfqpoint{2.024614in}{0.606232in}}%
\pgfpathlineto{\pgfqpoint{2.025170in}{0.601782in}}%
\pgfpathlineto{\pgfqpoint{2.025726in}{0.605703in}}%
\pgfpathlineto{\pgfqpoint{2.026838in}{0.604715in}}%
\pgfpathlineto{\pgfqpoint{2.027394in}{0.602333in}}%
\pgfpathlineto{\pgfqpoint{2.027950in}{0.603774in}}%
\pgfpathlineto{\pgfqpoint{2.028505in}{0.605880in}}%
\pgfpathlineto{\pgfqpoint{2.029061in}{0.602234in}}%
\pgfpathlineto{\pgfqpoint{2.029617in}{0.607291in}}%
\pgfpathlineto{\pgfqpoint{2.030173in}{0.603231in}}%
\pgfpathlineto{\pgfqpoint{2.031285in}{0.608095in}}%
\pgfpathlineto{\pgfqpoint{2.032952in}{0.601550in}}%
\pgfpathlineto{\pgfqpoint{2.033508in}{0.607452in}}%
\pgfpathlineto{\pgfqpoint{2.034064in}{0.606870in}}%
\pgfpathlineto{\pgfqpoint{2.034620in}{0.602493in}}%
\pgfpathlineto{\pgfqpoint{2.035176in}{0.603678in}}%
\pgfpathlineto{\pgfqpoint{2.035732in}{0.606551in}}%
\pgfpathlineto{\pgfqpoint{2.037399in}{0.603432in}}%
\pgfpathlineto{\pgfqpoint{2.037955in}{0.601518in}}%
\pgfpathlineto{\pgfqpoint{2.038511in}{0.602125in}}%
\pgfpathlineto{\pgfqpoint{2.042402in}{0.601114in}}%
\pgfpathlineto{\pgfqpoint{2.049072in}{0.602384in}}%
\pgfpathlineto{\pgfqpoint{2.049628in}{0.605594in}}%
\pgfpathlineto{\pgfqpoint{2.050184in}{0.600571in}}%
\pgfpathlineto{\pgfqpoint{2.050740in}{0.603011in}}%
\pgfpathlineto{\pgfqpoint{2.051296in}{0.608129in}}%
\pgfpathlineto{\pgfqpoint{2.051852in}{0.605642in}}%
\pgfpathlineto{\pgfqpoint{2.052963in}{0.602967in}}%
\pgfpathlineto{\pgfqpoint{2.053519in}{0.603101in}}%
\pgfpathlineto{\pgfqpoint{2.054631in}{0.608049in}}%
\pgfpathlineto{\pgfqpoint{2.055187in}{0.606346in}}%
\pgfpathlineto{\pgfqpoint{2.055743in}{0.600623in}}%
\pgfpathlineto{\pgfqpoint{2.056299in}{0.610221in}}%
\pgfpathlineto{\pgfqpoint{2.056854in}{0.606862in}}%
\pgfpathlineto{\pgfqpoint{2.057410in}{0.605972in}}%
\pgfpathlineto{\pgfqpoint{2.057966in}{0.607929in}}%
\pgfpathlineto{\pgfqpoint{2.059078in}{0.601832in}}%
\pgfpathlineto{\pgfqpoint{2.059634in}{0.604690in}}%
\pgfpathlineto{\pgfqpoint{2.060190in}{0.607677in}}%
\pgfpathlineto{\pgfqpoint{2.060745in}{0.604736in}}%
\pgfpathlineto{\pgfqpoint{2.061301in}{0.605407in}}%
\pgfpathlineto{\pgfqpoint{2.061857in}{0.601213in}}%
\pgfpathlineto{\pgfqpoint{2.062413in}{0.601866in}}%
\pgfpathlineto{\pgfqpoint{2.062969in}{0.611601in}}%
\pgfpathlineto{\pgfqpoint{2.063525in}{0.602282in}}%
\pgfpathlineto{\pgfqpoint{2.064081in}{0.601387in}}%
\pgfpathlineto{\pgfqpoint{2.065192in}{0.607190in}}%
\pgfpathlineto{\pgfqpoint{2.065748in}{0.605226in}}%
\pgfpathlineto{\pgfqpoint{2.066304in}{0.602763in}}%
\pgfpathlineto{\pgfqpoint{2.066860in}{0.605445in}}%
\pgfpathlineto{\pgfqpoint{2.067416in}{0.602321in}}%
\pgfpathlineto{\pgfqpoint{2.067972in}{0.615592in}}%
\pgfpathlineto{\pgfqpoint{2.068527in}{0.605748in}}%
\pgfpathlineto{\pgfqpoint{2.069639in}{0.601264in}}%
\pgfpathlineto{\pgfqpoint{2.070195in}{0.608363in}}%
\pgfpathlineto{\pgfqpoint{2.070751in}{0.604518in}}%
\pgfpathlineto{\pgfqpoint{2.072419in}{0.601011in}}%
\pgfpathlineto{\pgfqpoint{2.072974in}{0.601555in}}%
\pgfpathlineto{\pgfqpoint{2.073530in}{0.604946in}}%
\pgfpathlineto{\pgfqpoint{2.074086in}{0.603128in}}%
\pgfpathlineto{\pgfqpoint{2.075198in}{0.604871in}}%
\pgfpathlineto{\pgfqpoint{2.075754in}{0.606711in}}%
\pgfpathlineto{\pgfqpoint{2.076310in}{0.602623in}}%
\pgfpathlineto{\pgfqpoint{2.076865in}{0.604076in}}%
\pgfpathlineto{\pgfqpoint{2.079089in}{0.600194in}}%
\pgfpathlineto{\pgfqpoint{2.079645in}{0.607820in}}%
\pgfpathlineto{\pgfqpoint{2.080201in}{0.605312in}}%
\pgfpathlineto{\pgfqpoint{2.080756in}{0.601990in}}%
\pgfpathlineto{\pgfqpoint{2.081312in}{0.602581in}}%
\pgfpathlineto{\pgfqpoint{2.081868in}{0.606113in}}%
\pgfpathlineto{\pgfqpoint{2.082980in}{0.600740in}}%
\pgfpathlineto{\pgfqpoint{2.084092in}{0.606354in}}%
\pgfpathlineto{\pgfqpoint{2.084647in}{0.601990in}}%
\pgfpathlineto{\pgfqpoint{2.085203in}{0.604138in}}%
\pgfpathlineto{\pgfqpoint{2.086315in}{0.603077in}}%
\pgfpathlineto{\pgfqpoint{2.086871in}{0.605084in}}%
\pgfpathlineto{\pgfqpoint{2.087427in}{0.604269in}}%
\pgfpathlineto{\pgfqpoint{2.089094in}{0.604179in}}%
\pgfpathlineto{\pgfqpoint{2.089650in}{0.601411in}}%
\pgfpathlineto{\pgfqpoint{2.090206in}{0.603844in}}%
\pgfpathlineto{\pgfqpoint{2.090762in}{0.603865in}}%
\pgfpathlineto{\pgfqpoint{2.091318in}{0.601629in}}%
\pgfpathlineto{\pgfqpoint{2.091874in}{0.603488in}}%
\pgfpathlineto{\pgfqpoint{2.092430in}{0.603282in}}%
\pgfpathlineto{\pgfqpoint{2.093541in}{0.610620in}}%
\pgfpathlineto{\pgfqpoint{2.094097in}{0.605358in}}%
\pgfpathlineto{\pgfqpoint{2.095209in}{0.607036in}}%
\pgfpathlineto{\pgfqpoint{2.095765in}{0.603480in}}%
\pgfpathlineto{\pgfqpoint{2.096321in}{0.611835in}}%
\pgfpathlineto{\pgfqpoint{2.096876in}{0.606203in}}%
\pgfpathlineto{\pgfqpoint{2.098544in}{0.601724in}}%
\pgfpathlineto{\pgfqpoint{2.099100in}{0.611664in}}%
\pgfpathlineto{\pgfqpoint{2.099656in}{0.606690in}}%
\pgfpathlineto{\pgfqpoint{2.100212in}{0.607653in}}%
\pgfpathlineto{\pgfqpoint{2.100767in}{0.605765in}}%
\pgfpathlineto{\pgfqpoint{2.101323in}{0.610936in}}%
\pgfpathlineto{\pgfqpoint{2.102435in}{0.602151in}}%
\pgfpathlineto{\pgfqpoint{2.102991in}{0.603146in}}%
\pgfpathlineto{\pgfqpoint{2.103547in}{0.602659in}}%
\pgfpathlineto{\pgfqpoint{2.104103in}{0.603606in}}%
\pgfpathlineto{\pgfqpoint{2.104658in}{0.606248in}}%
\pgfpathlineto{\pgfqpoint{2.105214in}{0.605495in}}%
\pgfpathlineto{\pgfqpoint{2.105770in}{0.604235in}}%
\pgfpathlineto{\pgfqpoint{2.106326in}{0.607388in}}%
\pgfpathlineto{\pgfqpoint{2.107994in}{0.602232in}}%
\pgfpathlineto{\pgfqpoint{2.109105in}{0.607831in}}%
\pgfpathlineto{\pgfqpoint{2.111329in}{0.602206in}}%
\pgfpathlineto{\pgfqpoint{2.112441in}{0.603866in}}%
\pgfpathlineto{\pgfqpoint{2.112996in}{0.602694in}}%
\pgfpathlineto{\pgfqpoint{2.114108in}{0.607565in}}%
\pgfpathlineto{\pgfqpoint{2.114664in}{0.603406in}}%
\pgfpathlineto{\pgfqpoint{2.115220in}{0.610502in}}%
\pgfpathlineto{\pgfqpoint{2.116332in}{0.600814in}}%
\pgfpathlineto{\pgfqpoint{2.116887in}{0.602020in}}%
\pgfpathlineto{\pgfqpoint{2.117443in}{0.602024in}}%
\pgfpathlineto{\pgfqpoint{2.117999in}{0.605257in}}%
\pgfpathlineto{\pgfqpoint{2.118555in}{0.604211in}}%
\pgfpathlineto{\pgfqpoint{2.119111in}{0.602631in}}%
\pgfpathlineto{\pgfqpoint{2.120223in}{0.608090in}}%
\pgfpathlineto{\pgfqpoint{2.122446in}{0.600487in}}%
\pgfpathlineto{\pgfqpoint{2.123002in}{0.606427in}}%
\pgfpathlineto{\pgfqpoint{2.123558in}{0.603616in}}%
\pgfpathlineto{\pgfqpoint{2.124114in}{0.604772in}}%
\pgfpathlineto{\pgfqpoint{2.124669in}{0.603236in}}%
\pgfpathlineto{\pgfqpoint{2.125225in}{0.606223in}}%
\pgfpathlineto{\pgfqpoint{2.125781in}{0.605015in}}%
\pgfpathlineto{\pgfqpoint{2.126337in}{0.602786in}}%
\pgfpathlineto{\pgfqpoint{2.126893in}{0.605135in}}%
\pgfpathlineto{\pgfqpoint{2.128005in}{0.609136in}}%
\pgfpathlineto{\pgfqpoint{2.128561in}{0.601071in}}%
\pgfpathlineto{\pgfqpoint{2.129116in}{0.601836in}}%
\pgfpathlineto{\pgfqpoint{2.130228in}{0.602807in}}%
\pgfpathlineto{\pgfqpoint{2.130784in}{0.600944in}}%
\pgfpathlineto{\pgfqpoint{2.131340in}{0.603939in}}%
\pgfpathlineto{\pgfqpoint{2.131896in}{0.602595in}}%
\pgfpathlineto{\pgfqpoint{2.132452in}{0.600732in}}%
\pgfpathlineto{\pgfqpoint{2.133007in}{0.601494in}}%
\pgfpathlineto{\pgfqpoint{2.134119in}{0.603093in}}%
\pgfpathlineto{\pgfqpoint{2.134675in}{0.602367in}}%
\pgfpathlineto{\pgfqpoint{2.135787in}{0.602606in}}%
\pgfpathlineto{\pgfqpoint{2.136343in}{0.604868in}}%
\pgfpathlineto{\pgfqpoint{2.136898in}{0.602358in}}%
\pgfpathlineto{\pgfqpoint{2.137454in}{0.605370in}}%
\pgfpathlineto{\pgfqpoint{2.138010in}{0.604884in}}%
\pgfpathlineto{\pgfqpoint{2.138566in}{0.601104in}}%
\pgfpathlineto{\pgfqpoint{2.139122in}{0.602275in}}%
\pgfpathlineto{\pgfqpoint{2.141345in}{0.605678in}}%
\pgfpathlineto{\pgfqpoint{2.142457in}{0.601051in}}%
\pgfpathlineto{\pgfqpoint{2.143013in}{0.604083in}}%
\pgfpathlineto{\pgfqpoint{2.143569in}{0.600499in}}%
\pgfpathlineto{\pgfqpoint{2.144125in}{0.602109in}}%
\pgfpathlineto{\pgfqpoint{2.145792in}{0.601320in}}%
\pgfpathlineto{\pgfqpoint{2.146348in}{0.605333in}}%
\pgfpathlineto{\pgfqpoint{2.146904in}{0.602091in}}%
\pgfpathlineto{\pgfqpoint{2.147460in}{0.603660in}}%
\pgfpathlineto{\pgfqpoint{2.148016in}{0.601856in}}%
\pgfpathlineto{\pgfqpoint{2.148572in}{0.604716in}}%
\pgfpathlineto{\pgfqpoint{2.149127in}{0.602093in}}%
\pgfpathlineto{\pgfqpoint{2.149683in}{0.601355in}}%
\pgfpathlineto{\pgfqpoint{2.150239in}{0.602408in}}%
\pgfpathlineto{\pgfqpoint{2.150795in}{0.607127in}}%
\pgfpathlineto{\pgfqpoint{2.151351in}{0.603721in}}%
\pgfpathlineto{\pgfqpoint{2.151907in}{0.605457in}}%
\pgfpathlineto{\pgfqpoint{2.152463in}{0.605258in}}%
\pgfpathlineto{\pgfqpoint{2.154686in}{0.602199in}}%
\pgfpathlineto{\pgfqpoint{2.155242in}{0.603197in}}%
\pgfpathlineto{\pgfqpoint{2.155798in}{0.601132in}}%
\pgfpathlineto{\pgfqpoint{2.156354in}{0.606859in}}%
\pgfpathlineto{\pgfqpoint{2.156909in}{0.606324in}}%
\pgfpathlineto{\pgfqpoint{2.158021in}{0.602199in}}%
\pgfpathlineto{\pgfqpoint{2.159689in}{0.605135in}}%
\pgfpathlineto{\pgfqpoint{2.161356in}{0.602246in}}%
\pgfpathlineto{\pgfqpoint{2.161912in}{0.603642in}}%
\pgfpathlineto{\pgfqpoint{2.162468in}{0.602313in}}%
\pgfpathlineto{\pgfqpoint{2.163580in}{0.600525in}}%
\pgfpathlineto{\pgfqpoint{2.164136in}{0.601052in}}%
\pgfpathlineto{\pgfqpoint{2.164692in}{0.605188in}}%
\pgfpathlineto{\pgfqpoint{2.165247in}{0.600124in}}%
\pgfpathlineto{\pgfqpoint{2.165803in}{0.604983in}}%
\pgfpathlineto{\pgfqpoint{2.166359in}{0.605799in}}%
\pgfpathlineto{\pgfqpoint{2.166915in}{0.602162in}}%
\pgfpathlineto{\pgfqpoint{2.167471in}{0.603801in}}%
\pgfpathlineto{\pgfqpoint{2.168027in}{0.606864in}}%
\pgfpathlineto{\pgfqpoint{2.168583in}{0.605986in}}%
\pgfpathlineto{\pgfqpoint{2.169138in}{0.606135in}}%
\pgfpathlineto{\pgfqpoint{2.169694in}{0.600963in}}%
\pgfpathlineto{\pgfqpoint{2.170250in}{0.603336in}}%
\pgfpathlineto{\pgfqpoint{2.170806in}{0.605693in}}%
\pgfpathlineto{\pgfqpoint{2.171362in}{0.603803in}}%
\pgfpathlineto{\pgfqpoint{2.171918in}{0.600754in}}%
\pgfpathlineto{\pgfqpoint{2.172474in}{0.605001in}}%
\pgfpathlineto{\pgfqpoint{2.173029in}{0.601757in}}%
\pgfpathlineto{\pgfqpoint{2.173585in}{0.600346in}}%
\pgfpathlineto{\pgfqpoint{2.174141in}{0.601367in}}%
\pgfpathlineto{\pgfqpoint{2.174697in}{0.606488in}}%
\pgfpathlineto{\pgfqpoint{2.175253in}{0.602424in}}%
\pgfpathlineto{\pgfqpoint{2.175809in}{0.605303in}}%
\pgfpathlineto{\pgfqpoint{2.176365in}{0.600031in}}%
\pgfpathlineto{\pgfqpoint{2.176920in}{0.605307in}}%
\pgfpathlineto{\pgfqpoint{2.177476in}{0.601016in}}%
\pgfpathlineto{\pgfqpoint{2.178032in}{0.606313in}}%
\pgfpathlineto{\pgfqpoint{2.178588in}{0.603458in}}%
\pgfpathlineto{\pgfqpoint{2.179144in}{0.603895in}}%
\pgfpathlineto{\pgfqpoint{2.180256in}{0.601522in}}%
\pgfpathlineto{\pgfqpoint{2.180811in}{0.602963in}}%
\pgfpathlineto{\pgfqpoint{2.181367in}{0.601956in}}%
\pgfpathlineto{\pgfqpoint{2.181923in}{0.601663in}}%
\pgfpathlineto{\pgfqpoint{2.183035in}{0.605322in}}%
\pgfpathlineto{\pgfqpoint{2.183591in}{0.603714in}}%
\pgfpathlineto{\pgfqpoint{2.184147in}{0.604607in}}%
\pgfpathlineto{\pgfqpoint{2.184703in}{0.602210in}}%
\pgfpathlineto{\pgfqpoint{2.185258in}{0.604339in}}%
\pgfpathlineto{\pgfqpoint{2.185814in}{0.603879in}}%
\pgfpathlineto{\pgfqpoint{2.187482in}{0.601427in}}%
\pgfpathlineto{\pgfqpoint{2.188038in}{0.601330in}}%
\pgfpathlineto{\pgfqpoint{2.189149in}{0.605524in}}%
\pgfpathlineto{\pgfqpoint{2.190817in}{0.601153in}}%
\pgfpathlineto{\pgfqpoint{2.191929in}{0.600324in}}%
\pgfpathlineto{\pgfqpoint{2.192485in}{0.603967in}}%
\pgfpathlineto{\pgfqpoint{2.193040in}{0.601506in}}%
\pgfpathlineto{\pgfqpoint{2.194708in}{0.604006in}}%
\pgfpathlineto{\pgfqpoint{2.195820in}{0.602267in}}%
\pgfpathlineto{\pgfqpoint{2.196931in}{0.605303in}}%
\pgfpathlineto{\pgfqpoint{2.198043in}{0.600995in}}%
\pgfpathlineto{\pgfqpoint{2.198599in}{0.606389in}}%
\pgfpathlineto{\pgfqpoint{2.199155in}{0.601354in}}%
\pgfpathlineto{\pgfqpoint{2.200823in}{0.602016in}}%
\pgfpathlineto{\pgfqpoint{2.201934in}{0.602268in}}%
\pgfpathlineto{\pgfqpoint{2.204158in}{0.602651in}}%
\pgfpathlineto{\pgfqpoint{2.204714in}{0.602105in}}%
\pgfpathlineto{\pgfqpoint{2.205269in}{0.604238in}}%
\pgfpathlineto{\pgfqpoint{2.205825in}{0.600861in}}%
\pgfpathlineto{\pgfqpoint{2.206381in}{0.602482in}}%
\pgfpathlineto{\pgfqpoint{2.206937in}{0.601458in}}%
\pgfpathlineto{\pgfqpoint{2.207493in}{0.602219in}}%
\pgfpathlineto{\pgfqpoint{2.208049in}{0.602150in}}%
\pgfpathlineto{\pgfqpoint{2.209160in}{0.603915in}}%
\pgfpathlineto{\pgfqpoint{2.210828in}{0.600585in}}%
\pgfpathlineto{\pgfqpoint{2.211384in}{0.606079in}}%
\pgfpathlineto{\pgfqpoint{2.211940in}{0.601272in}}%
\pgfpathlineto{\pgfqpoint{2.212496in}{0.601263in}}%
\pgfpathlineto{\pgfqpoint{2.213051in}{0.600135in}}%
\pgfpathlineto{\pgfqpoint{2.214163in}{0.604656in}}%
\pgfpathlineto{\pgfqpoint{2.214719in}{0.602757in}}%
\pgfpathlineto{\pgfqpoint{2.215275in}{0.604718in}}%
\pgfpathlineto{\pgfqpoint{2.215831in}{0.603897in}}%
\pgfpathlineto{\pgfqpoint{2.216387in}{0.604927in}}%
\pgfpathlineto{\pgfqpoint{2.216942in}{0.604981in}}%
\pgfpathlineto{\pgfqpoint{2.217498in}{0.601081in}}%
\pgfpathlineto{\pgfqpoint{2.218054in}{0.601256in}}%
\pgfpathlineto{\pgfqpoint{2.219166in}{0.602223in}}%
\pgfpathlineto{\pgfqpoint{2.219722in}{0.601433in}}%
\pgfpathlineto{\pgfqpoint{2.220278in}{0.604798in}}%
\pgfpathlineto{\pgfqpoint{2.220834in}{0.602771in}}%
\pgfpathlineto{\pgfqpoint{2.222501in}{0.604815in}}%
\pgfpathlineto{\pgfqpoint{2.223057in}{0.602721in}}%
\pgfpathlineto{\pgfqpoint{2.224169in}{0.609976in}}%
\pgfpathlineto{\pgfqpoint{2.224725in}{0.601366in}}%
\pgfpathlineto{\pgfqpoint{2.225280in}{0.602606in}}%
\pgfpathlineto{\pgfqpoint{2.225836in}{0.601522in}}%
\pgfpathlineto{\pgfqpoint{2.227504in}{0.610936in}}%
\pgfpathlineto{\pgfqpoint{2.228060in}{0.602170in}}%
\pgfpathlineto{\pgfqpoint{2.228616in}{0.612424in}}%
\pgfpathlineto{\pgfqpoint{2.229171in}{0.603568in}}%
\pgfpathlineto{\pgfqpoint{2.229727in}{0.602936in}}%
\pgfpathlineto{\pgfqpoint{2.230283in}{0.606305in}}%
\pgfpathlineto{\pgfqpoint{2.230839in}{0.601645in}}%
\pgfpathlineto{\pgfqpoint{2.231395in}{0.605187in}}%
\pgfpathlineto{\pgfqpoint{2.231951in}{0.605633in}}%
\pgfpathlineto{\pgfqpoint{2.232507in}{0.603765in}}%
\pgfpathlineto{\pgfqpoint{2.233062in}{0.609393in}}%
\pgfpathlineto{\pgfqpoint{2.234174in}{0.602685in}}%
\pgfpathlineto{\pgfqpoint{2.235286in}{0.605095in}}%
\pgfpathlineto{\pgfqpoint{2.236398in}{0.603300in}}%
\pgfpathlineto{\pgfqpoint{2.236953in}{0.604644in}}%
\pgfpathlineto{\pgfqpoint{2.237509in}{0.607200in}}%
\pgfpathlineto{\pgfqpoint{2.238065in}{0.606998in}}%
\pgfpathlineto{\pgfqpoint{2.239177in}{0.603602in}}%
\pgfpathlineto{\pgfqpoint{2.239733in}{0.605686in}}%
\pgfpathlineto{\pgfqpoint{2.240289in}{0.604284in}}%
\pgfpathlineto{\pgfqpoint{2.241400in}{0.606016in}}%
\pgfpathlineto{\pgfqpoint{2.241956in}{0.605054in}}%
\pgfpathlineto{\pgfqpoint{2.242512in}{0.601984in}}%
\pgfpathlineto{\pgfqpoint{2.243068in}{0.602299in}}%
\pgfpathlineto{\pgfqpoint{2.243624in}{0.607590in}}%
\pgfpathlineto{\pgfqpoint{2.244180in}{0.606652in}}%
\pgfpathlineto{\pgfqpoint{2.244736in}{0.605901in}}%
\pgfpathlineto{\pgfqpoint{2.246403in}{0.601811in}}%
\pgfpathlineto{\pgfqpoint{2.246959in}{0.602526in}}%
\pgfpathlineto{\pgfqpoint{2.248071in}{0.607096in}}%
\pgfpathlineto{\pgfqpoint{2.248627in}{0.600207in}}%
\pgfpathlineto{\pgfqpoint{2.249182in}{0.602244in}}%
\pgfpathlineto{\pgfqpoint{2.250294in}{0.604914in}}%
\pgfpathlineto{\pgfqpoint{2.251406in}{0.601259in}}%
\pgfpathlineto{\pgfqpoint{2.251962in}{0.602357in}}%
\pgfpathlineto{\pgfqpoint{2.252518in}{0.601348in}}%
\pgfpathlineto{\pgfqpoint{2.254185in}{0.606881in}}%
\pgfpathlineto{\pgfqpoint{2.255297in}{0.601144in}}%
\pgfpathlineto{\pgfqpoint{2.259188in}{0.605622in}}%
\pgfpathlineto{\pgfqpoint{2.260300in}{0.602741in}}%
\pgfpathlineto{\pgfqpoint{2.261411in}{0.606078in}}%
\pgfpathlineto{\pgfqpoint{2.263079in}{0.603745in}}%
\pgfpathlineto{\pgfqpoint{2.263635in}{0.603880in}}%
\pgfpathlineto{\pgfqpoint{2.264191in}{0.602012in}}%
\pgfpathlineto{\pgfqpoint{2.265302in}{0.605675in}}%
\pgfpathlineto{\pgfqpoint{2.265858in}{0.604926in}}%
\pgfpathlineto{\pgfqpoint{2.266414in}{0.604603in}}%
\pgfpathlineto{\pgfqpoint{2.266970in}{0.602478in}}%
\pgfpathlineto{\pgfqpoint{2.267526in}{0.612544in}}%
\pgfpathlineto{\pgfqpoint{2.268082in}{0.603920in}}%
\pgfpathlineto{\pgfqpoint{2.268638in}{0.604980in}}%
\pgfpathlineto{\pgfqpoint{2.270305in}{0.600771in}}%
\pgfpathlineto{\pgfqpoint{2.271417in}{0.611773in}}%
\pgfpathlineto{\pgfqpoint{2.272529in}{0.609573in}}%
\pgfpathlineto{\pgfqpoint{2.273084in}{0.606713in}}%
\pgfpathlineto{\pgfqpoint{2.273640in}{0.611473in}}%
\pgfpathlineto{\pgfqpoint{2.274196in}{0.601994in}}%
\pgfpathlineto{\pgfqpoint{2.274752in}{0.607703in}}%
\pgfpathlineto{\pgfqpoint{2.275864in}{0.606291in}}%
\pgfpathlineto{\pgfqpoint{2.276420in}{0.601779in}}%
\pgfpathlineto{\pgfqpoint{2.276976in}{0.603287in}}%
\pgfpathlineto{\pgfqpoint{2.277531in}{0.602475in}}%
\pgfpathlineto{\pgfqpoint{2.279199in}{0.607855in}}%
\pgfpathlineto{\pgfqpoint{2.279755in}{0.609604in}}%
\pgfpathlineto{\pgfqpoint{2.280867in}{0.602790in}}%
\pgfpathlineto{\pgfqpoint{2.282534in}{0.608481in}}%
\pgfpathlineto{\pgfqpoint{2.283090in}{0.605388in}}%
\pgfpathlineto{\pgfqpoint{2.283646in}{0.614370in}}%
\pgfpathlineto{\pgfqpoint{2.284758in}{0.603834in}}%
\pgfpathlineto{\pgfqpoint{2.286425in}{0.613154in}}%
\pgfpathlineto{\pgfqpoint{2.286981in}{0.605899in}}%
\pgfpathlineto{\pgfqpoint{2.287537in}{0.612035in}}%
\pgfpathlineto{\pgfqpoint{2.288649in}{0.605721in}}%
\pgfpathlineto{\pgfqpoint{2.289204in}{0.609486in}}%
\pgfpathlineto{\pgfqpoint{2.289760in}{0.604034in}}%
\pgfpathlineto{\pgfqpoint{2.290316in}{0.624456in}}%
\pgfpathlineto{\pgfqpoint{2.290872in}{0.602840in}}%
\pgfpathlineto{\pgfqpoint{2.291428in}{0.608778in}}%
\pgfpathlineto{\pgfqpoint{2.292540in}{0.605087in}}%
\pgfpathlineto{\pgfqpoint{2.293095in}{0.608197in}}%
\pgfpathlineto{\pgfqpoint{2.293651in}{0.602862in}}%
\pgfpathlineto{\pgfqpoint{2.294207in}{0.603736in}}%
\pgfpathlineto{\pgfqpoint{2.294763in}{0.603677in}}%
\pgfpathlineto{\pgfqpoint{2.295319in}{0.618401in}}%
\pgfpathlineto{\pgfqpoint{2.295875in}{0.604256in}}%
\pgfpathlineto{\pgfqpoint{2.296431in}{0.602758in}}%
\pgfpathlineto{\pgfqpoint{2.298098in}{0.609743in}}%
\pgfpathlineto{\pgfqpoint{2.298654in}{0.602948in}}%
\pgfpathlineto{\pgfqpoint{2.299210in}{0.604036in}}%
\pgfpathlineto{\pgfqpoint{2.300878in}{0.611821in}}%
\pgfpathlineto{\pgfqpoint{2.302545in}{0.601152in}}%
\pgfpathlineto{\pgfqpoint{2.303101in}{0.607470in}}%
\pgfpathlineto{\pgfqpoint{2.303657in}{0.607112in}}%
\pgfpathlineto{\pgfqpoint{2.304213in}{0.601410in}}%
\pgfpathlineto{\pgfqpoint{2.304769in}{0.609498in}}%
\pgfpathlineto{\pgfqpoint{2.305324in}{0.607802in}}%
\pgfpathlineto{\pgfqpoint{2.305880in}{0.602448in}}%
\pgfpathlineto{\pgfqpoint{2.306436in}{0.612135in}}%
\pgfpathlineto{\pgfqpoint{2.306992in}{0.606505in}}%
\pgfpathlineto{\pgfqpoint{2.307548in}{0.611815in}}%
\pgfpathlineto{\pgfqpoint{2.309215in}{0.601154in}}%
\pgfpathlineto{\pgfqpoint{2.310883in}{0.608595in}}%
\pgfpathlineto{\pgfqpoint{2.311995in}{0.601637in}}%
\pgfpathlineto{\pgfqpoint{2.312551in}{0.610457in}}%
\pgfpathlineto{\pgfqpoint{2.313106in}{0.601654in}}%
\pgfpathlineto{\pgfqpoint{2.313662in}{0.602845in}}%
\pgfpathlineto{\pgfqpoint{2.314218in}{0.600170in}}%
\pgfpathlineto{\pgfqpoint{2.314774in}{0.602571in}}%
\pgfpathlineto{\pgfqpoint{2.315886in}{0.607623in}}%
\pgfpathlineto{\pgfqpoint{2.316442in}{0.604078in}}%
\pgfpathlineto{\pgfqpoint{2.316998in}{0.606677in}}%
\pgfpathlineto{\pgfqpoint{2.317553in}{0.608858in}}%
\pgfpathlineto{\pgfqpoint{2.318109in}{0.600728in}}%
\pgfpathlineto{\pgfqpoint{2.318665in}{0.610559in}}%
\pgfpathlineto{\pgfqpoint{2.319221in}{0.609721in}}%
\pgfpathlineto{\pgfqpoint{2.320333in}{0.603402in}}%
\pgfpathlineto{\pgfqpoint{2.320889in}{0.610414in}}%
\pgfpathlineto{\pgfqpoint{2.321444in}{0.601091in}}%
\pgfpathlineto{\pgfqpoint{2.322000in}{0.608454in}}%
\pgfpathlineto{\pgfqpoint{2.323668in}{0.605150in}}%
\pgfpathlineto{\pgfqpoint{2.324224in}{0.603231in}}%
\pgfpathlineto{\pgfqpoint{2.325335in}{0.612087in}}%
\pgfpathlineto{\pgfqpoint{2.325891in}{0.603131in}}%
\pgfpathlineto{\pgfqpoint{2.326447in}{0.607956in}}%
\pgfpathlineto{\pgfqpoint{2.327003in}{0.610023in}}%
\pgfpathlineto{\pgfqpoint{2.328115in}{0.604645in}}%
\pgfpathlineto{\pgfqpoint{2.329226in}{0.611822in}}%
\pgfpathlineto{\pgfqpoint{2.329782in}{0.610906in}}%
\pgfpathlineto{\pgfqpoint{2.330338in}{0.607422in}}%
\pgfpathlineto{\pgfqpoint{2.331450in}{0.617236in}}%
\pgfpathlineto{\pgfqpoint{2.332006in}{0.611969in}}%
\pgfpathlineto{\pgfqpoint{2.333118in}{0.611194in}}%
\pgfpathlineto{\pgfqpoint{2.334229in}{0.602626in}}%
\pgfpathlineto{\pgfqpoint{2.334785in}{0.611725in}}%
\pgfpathlineto{\pgfqpoint{2.335341in}{0.609757in}}%
\pgfpathlineto{\pgfqpoint{2.335897in}{0.600665in}}%
\pgfpathlineto{\pgfqpoint{2.336453in}{0.606886in}}%
\pgfpathlineto{\pgfqpoint{2.337009in}{0.610903in}}%
\pgfpathlineto{\pgfqpoint{2.337564in}{0.607698in}}%
\pgfpathlineto{\pgfqpoint{2.338676in}{0.604703in}}%
\pgfpathlineto{\pgfqpoint{2.339232in}{0.617658in}}%
\pgfpathlineto{\pgfqpoint{2.339788in}{0.614791in}}%
\pgfpathlineto{\pgfqpoint{2.340344in}{0.606158in}}%
\pgfpathlineto{\pgfqpoint{2.340900in}{0.614880in}}%
\pgfpathlineto{\pgfqpoint{2.341455in}{0.611509in}}%
\pgfpathlineto{\pgfqpoint{2.342011in}{0.606088in}}%
\pgfpathlineto{\pgfqpoint{2.343679in}{0.620873in}}%
\pgfpathlineto{\pgfqpoint{2.344791in}{0.603171in}}%
\pgfpathlineto{\pgfqpoint{2.346458in}{0.619627in}}%
\pgfpathlineto{\pgfqpoint{2.347014in}{0.614879in}}%
\pgfpathlineto{\pgfqpoint{2.347570in}{0.621975in}}%
\pgfpathlineto{\pgfqpoint{2.348126in}{0.620874in}}%
\pgfpathlineto{\pgfqpoint{2.348682in}{0.609681in}}%
\pgfpathlineto{\pgfqpoint{2.349237in}{0.620803in}}%
\pgfpathlineto{\pgfqpoint{2.350905in}{0.606662in}}%
\pgfpathlineto{\pgfqpoint{2.351461in}{0.608200in}}%
\pgfpathlineto{\pgfqpoint{2.352017in}{0.607196in}}%
\pgfpathlineto{\pgfqpoint{2.352573in}{0.617614in}}%
\pgfpathlineto{\pgfqpoint{2.353129in}{0.614408in}}%
\pgfpathlineto{\pgfqpoint{2.354796in}{0.602236in}}%
\pgfpathlineto{\pgfqpoint{2.355352in}{0.627821in}}%
\pgfpathlineto{\pgfqpoint{2.355908in}{0.602553in}}%
\pgfpathlineto{\pgfqpoint{2.357575in}{0.606295in}}%
\pgfpathlineto{\pgfqpoint{2.359243in}{0.612383in}}%
\pgfpathlineto{\pgfqpoint{2.360911in}{0.609677in}}%
\pgfpathlineto{\pgfqpoint{2.361466in}{0.610205in}}%
\pgfpathlineto{\pgfqpoint{2.362022in}{0.601283in}}%
\pgfpathlineto{\pgfqpoint{2.362578in}{0.603077in}}%
\pgfpathlineto{\pgfqpoint{2.363690in}{0.618682in}}%
\pgfpathlineto{\pgfqpoint{2.364246in}{0.611324in}}%
\pgfpathlineto{\pgfqpoint{2.364802in}{0.611085in}}%
\pgfpathlineto{\pgfqpoint{2.366469in}{0.602230in}}%
\pgfpathlineto{\pgfqpoint{2.367025in}{0.608970in}}%
\pgfpathlineto{\pgfqpoint{2.367581in}{0.604582in}}%
\pgfpathlineto{\pgfqpoint{2.368137in}{0.608218in}}%
\pgfpathlineto{\pgfqpoint{2.368693in}{0.602348in}}%
\pgfpathlineto{\pgfqpoint{2.369248in}{0.606051in}}%
\pgfpathlineto{\pgfqpoint{2.369804in}{0.614804in}}%
\pgfpathlineto{\pgfqpoint{2.370360in}{0.606167in}}%
\pgfpathlineto{\pgfqpoint{2.370916in}{0.600945in}}%
\pgfpathlineto{\pgfqpoint{2.371472in}{0.603186in}}%
\pgfpathlineto{\pgfqpoint{2.373140in}{0.607636in}}%
\pgfpathlineto{\pgfqpoint{2.373695in}{0.604933in}}%
\pgfpathlineto{\pgfqpoint{2.374251in}{0.608686in}}%
\pgfpathlineto{\pgfqpoint{2.374807in}{0.603589in}}%
\pgfpathlineto{\pgfqpoint{2.375363in}{0.607807in}}%
\pgfpathlineto{\pgfqpoint{2.375919in}{0.605797in}}%
\pgfpathlineto{\pgfqpoint{2.376475in}{0.607514in}}%
\pgfpathlineto{\pgfqpoint{2.377586in}{0.611304in}}%
\pgfpathlineto{\pgfqpoint{2.378142in}{0.609145in}}%
\pgfpathlineto{\pgfqpoint{2.378698in}{0.606046in}}%
\pgfpathlineto{\pgfqpoint{2.379254in}{0.607293in}}%
\pgfpathlineto{\pgfqpoint{2.379810in}{0.608836in}}%
\pgfpathlineto{\pgfqpoint{2.380366in}{0.614761in}}%
\pgfpathlineto{\pgfqpoint{2.380922in}{0.600464in}}%
\pgfpathlineto{\pgfqpoint{2.381477in}{0.610061in}}%
\pgfpathlineto{\pgfqpoint{2.382033in}{0.606746in}}%
\pgfpathlineto{\pgfqpoint{2.382589in}{0.617035in}}%
\pgfpathlineto{\pgfqpoint{2.383145in}{0.612084in}}%
\pgfpathlineto{\pgfqpoint{2.383701in}{0.611109in}}%
\pgfpathlineto{\pgfqpoint{2.384257in}{0.605143in}}%
\pgfpathlineto{\pgfqpoint{2.384813in}{0.612973in}}%
\pgfpathlineto{\pgfqpoint{2.385368in}{0.608008in}}%
\pgfpathlineto{\pgfqpoint{2.385924in}{0.612884in}}%
\pgfpathlineto{\pgfqpoint{2.387036in}{0.604414in}}%
\pgfpathlineto{\pgfqpoint{2.388704in}{0.626657in}}%
\pgfpathlineto{\pgfqpoint{2.389260in}{0.604568in}}%
\pgfpathlineto{\pgfqpoint{2.389815in}{0.619897in}}%
\pgfpathlineto{\pgfqpoint{2.390371in}{0.610440in}}%
\pgfpathlineto{\pgfqpoint{2.390927in}{0.620554in}}%
\pgfpathlineto{\pgfqpoint{2.391483in}{0.602496in}}%
\pgfpathlineto{\pgfqpoint{2.392039in}{0.614981in}}%
\pgfpathlineto{\pgfqpoint{2.392595in}{0.606548in}}%
\pgfpathlineto{\pgfqpoint{2.393151in}{0.618077in}}%
\pgfpathlineto{\pgfqpoint{2.393706in}{0.608566in}}%
\pgfpathlineto{\pgfqpoint{2.394262in}{0.615049in}}%
\pgfpathlineto{\pgfqpoint{2.394818in}{0.612189in}}%
\pgfpathlineto{\pgfqpoint{2.395374in}{0.606938in}}%
\pgfpathlineto{\pgfqpoint{2.395930in}{0.610470in}}%
\pgfpathlineto{\pgfqpoint{2.397042in}{0.619425in}}%
\pgfpathlineto{\pgfqpoint{2.397597in}{0.606571in}}%
\pgfpathlineto{\pgfqpoint{2.398153in}{0.613418in}}%
\pgfpathlineto{\pgfqpoint{2.398709in}{0.630895in}}%
\pgfpathlineto{\pgfqpoint{2.399265in}{0.616172in}}%
\pgfpathlineto{\pgfqpoint{2.400933in}{0.611538in}}%
\pgfpathlineto{\pgfqpoint{2.401488in}{0.619595in}}%
\pgfpathlineto{\pgfqpoint{2.402044in}{0.602214in}}%
\pgfpathlineto{\pgfqpoint{2.402600in}{0.617232in}}%
\pgfpathlineto{\pgfqpoint{2.403156in}{0.627804in}}%
\pgfpathlineto{\pgfqpoint{2.404268in}{0.611847in}}%
\pgfpathlineto{\pgfqpoint{2.404824in}{0.613935in}}%
\pgfpathlineto{\pgfqpoint{2.405379in}{0.638647in}}%
\pgfpathlineto{\pgfqpoint{2.405935in}{0.607152in}}%
\pgfpathlineto{\pgfqpoint{2.406491in}{0.611207in}}%
\pgfpathlineto{\pgfqpoint{2.407047in}{0.610521in}}%
\pgfpathlineto{\pgfqpoint{2.407603in}{0.627906in}}%
\pgfpathlineto{\pgfqpoint{2.408159in}{0.611925in}}%
\pgfpathlineto{\pgfqpoint{2.408715in}{0.612894in}}%
\pgfpathlineto{\pgfqpoint{2.409271in}{0.630052in}}%
\pgfpathlineto{\pgfqpoint{2.409826in}{0.629821in}}%
\pgfpathlineto{\pgfqpoint{2.410938in}{0.612226in}}%
\pgfpathlineto{\pgfqpoint{2.411494in}{0.619351in}}%
\pgfpathlineto{\pgfqpoint{2.412050in}{0.606417in}}%
\pgfpathlineto{\pgfqpoint{2.412606in}{0.620792in}}%
\pgfpathlineto{\pgfqpoint{2.413162in}{0.607324in}}%
\pgfpathlineto{\pgfqpoint{2.413717in}{0.610659in}}%
\pgfpathlineto{\pgfqpoint{2.414273in}{0.607942in}}%
\pgfpathlineto{\pgfqpoint{2.414829in}{0.606553in}}%
\pgfpathlineto{\pgfqpoint{2.417053in}{0.616291in}}%
\pgfpathlineto{\pgfqpoint{2.418720in}{0.605748in}}%
\pgfpathlineto{\pgfqpoint{2.419276in}{0.613779in}}%
\pgfpathlineto{\pgfqpoint{2.419832in}{0.605661in}}%
\pgfpathlineto{\pgfqpoint{2.420388in}{0.612760in}}%
\pgfpathlineto{\pgfqpoint{2.420944in}{0.620089in}}%
\pgfpathlineto{\pgfqpoint{2.422611in}{0.601733in}}%
\pgfpathlineto{\pgfqpoint{2.423167in}{0.616169in}}%
\pgfpathlineto{\pgfqpoint{2.423723in}{0.610396in}}%
\pgfpathlineto{\pgfqpoint{2.425946in}{0.603434in}}%
\pgfpathlineto{\pgfqpoint{2.427614in}{0.616269in}}%
\pgfpathlineto{\pgfqpoint{2.428170in}{0.600396in}}%
\pgfpathlineto{\pgfqpoint{2.428726in}{0.602266in}}%
\pgfpathlineto{\pgfqpoint{2.430393in}{0.608482in}}%
\pgfpathlineto{\pgfqpoint{2.430949in}{0.611841in}}%
\pgfpathlineto{\pgfqpoint{2.431505in}{0.611371in}}%
\pgfpathlineto{\pgfqpoint{2.433173in}{0.602897in}}%
\pgfpathlineto{\pgfqpoint{2.434284in}{0.610410in}}%
\pgfpathlineto{\pgfqpoint{2.434840in}{0.609402in}}%
\pgfpathlineto{\pgfqpoint{2.435396in}{0.600196in}}%
\pgfpathlineto{\pgfqpoint{2.437064in}{0.618045in}}%
\pgfpathlineto{\pgfqpoint{2.438175in}{0.602126in}}%
\pgfpathlineto{\pgfqpoint{2.439843in}{0.615794in}}%
\pgfpathlineto{\pgfqpoint{2.441510in}{0.604548in}}%
\pgfpathlineto{\pgfqpoint{2.442066in}{0.618776in}}%
\pgfpathlineto{\pgfqpoint{2.442622in}{0.611494in}}%
\pgfpathlineto{\pgfqpoint{2.443178in}{0.614649in}}%
\pgfpathlineto{\pgfqpoint{2.443734in}{0.607408in}}%
\pgfpathlineto{\pgfqpoint{2.444290in}{0.611304in}}%
\pgfpathlineto{\pgfqpoint{2.444846in}{0.617821in}}%
\pgfpathlineto{\pgfqpoint{2.445401in}{0.604939in}}%
\pgfpathlineto{\pgfqpoint{2.447069in}{0.622927in}}%
\pgfpathlineto{\pgfqpoint{2.447625in}{0.614852in}}%
\pgfpathlineto{\pgfqpoint{2.448181in}{0.626686in}}%
\pgfpathlineto{\pgfqpoint{2.448737in}{0.607409in}}%
\pgfpathlineto{\pgfqpoint{2.449293in}{0.609779in}}%
\pgfpathlineto{\pgfqpoint{2.450404in}{0.623328in}}%
\pgfpathlineto{\pgfqpoint{2.452072in}{0.602891in}}%
\pgfpathlineto{\pgfqpoint{2.452628in}{0.607420in}}%
\pgfpathlineto{\pgfqpoint{2.453184in}{0.602645in}}%
\pgfpathlineto{\pgfqpoint{2.453739in}{0.605791in}}%
\pgfpathlineto{\pgfqpoint{2.454295in}{0.621159in}}%
\pgfpathlineto{\pgfqpoint{2.454851in}{0.615020in}}%
\pgfpathlineto{\pgfqpoint{2.455407in}{0.611885in}}%
\pgfpathlineto{\pgfqpoint{2.455963in}{0.629229in}}%
\pgfpathlineto{\pgfqpoint{2.456519in}{0.618406in}}%
\pgfpathlineto{\pgfqpoint{2.458186in}{0.600675in}}%
\pgfpathlineto{\pgfqpoint{2.460410in}{0.630994in}}%
\pgfpathlineto{\pgfqpoint{2.461521in}{0.614697in}}%
\pgfpathlineto{\pgfqpoint{2.462077in}{0.621807in}}%
\pgfpathlineto{\pgfqpoint{2.462633in}{0.619996in}}%
\pgfpathlineto{\pgfqpoint{2.463189in}{0.615938in}}%
\pgfpathlineto{\pgfqpoint{2.463745in}{0.601236in}}%
\pgfpathlineto{\pgfqpoint{2.464857in}{0.623638in}}%
\pgfpathlineto{\pgfqpoint{2.465968in}{0.611794in}}%
\pgfpathlineto{\pgfqpoint{2.466524in}{0.611922in}}%
\pgfpathlineto{\pgfqpoint{2.467080in}{0.604447in}}%
\pgfpathlineto{\pgfqpoint{2.467636in}{0.629864in}}%
\pgfpathlineto{\pgfqpoint{2.468192in}{0.617322in}}%
\pgfpathlineto{\pgfqpoint{2.469304in}{0.614842in}}%
\pgfpathlineto{\pgfqpoint{2.469859in}{0.615406in}}%
\pgfpathlineto{\pgfqpoint{2.470415in}{0.605568in}}%
\pgfpathlineto{\pgfqpoint{2.470971in}{0.625472in}}%
\pgfpathlineto{\pgfqpoint{2.471527in}{0.616281in}}%
\pgfpathlineto{\pgfqpoint{2.473750in}{0.601571in}}%
\pgfpathlineto{\pgfqpoint{2.475418in}{0.617054in}}%
\pgfpathlineto{\pgfqpoint{2.477641in}{0.605375in}}%
\pgfpathlineto{\pgfqpoint{2.478197in}{0.608612in}}%
\pgfpathlineto{\pgfqpoint{2.478753in}{0.601238in}}%
\pgfpathlineto{\pgfqpoint{2.480421in}{0.618083in}}%
\pgfpathlineto{\pgfqpoint{2.481532in}{0.602930in}}%
\pgfpathlineto{\pgfqpoint{2.482088in}{0.605291in}}%
\pgfpathlineto{\pgfqpoint{2.483200in}{0.609605in}}%
\pgfpathlineto{\pgfqpoint{2.483756in}{0.614505in}}%
\pgfpathlineto{\pgfqpoint{2.484312in}{0.609874in}}%
\pgfpathlineto{\pgfqpoint{2.484868in}{0.608760in}}%
\pgfpathlineto{\pgfqpoint{2.486535in}{0.602158in}}%
\pgfpathlineto{\pgfqpoint{2.487091in}{0.609013in}}%
\pgfpathlineto{\pgfqpoint{2.487647in}{0.606170in}}%
\pgfpathlineto{\pgfqpoint{2.488203in}{0.607745in}}%
\pgfpathlineto{\pgfqpoint{2.488759in}{0.604674in}}%
\pgfpathlineto{\pgfqpoint{2.489870in}{0.610297in}}%
\pgfpathlineto{\pgfqpoint{2.490426in}{0.603914in}}%
\pgfpathlineto{\pgfqpoint{2.490982in}{0.606048in}}%
\pgfpathlineto{\pgfqpoint{2.491538in}{0.604726in}}%
\pgfpathlineto{\pgfqpoint{2.493206in}{0.611254in}}%
\pgfpathlineto{\pgfqpoint{2.493761in}{0.605351in}}%
\pgfpathlineto{\pgfqpoint{2.494317in}{0.616269in}}%
\pgfpathlineto{\pgfqpoint{2.494873in}{0.601288in}}%
\pgfpathlineto{\pgfqpoint{2.495429in}{0.606518in}}%
\pgfpathlineto{\pgfqpoint{2.495985in}{0.608862in}}%
\pgfpathlineto{\pgfqpoint{2.496541in}{0.607116in}}%
\pgfpathlineto{\pgfqpoint{2.497097in}{0.607724in}}%
\pgfpathlineto{\pgfqpoint{2.497652in}{0.612035in}}%
\pgfpathlineto{\pgfqpoint{2.498208in}{0.609645in}}%
\pgfpathlineto{\pgfqpoint{2.498764in}{0.610228in}}%
\pgfpathlineto{\pgfqpoint{2.499320in}{0.613025in}}%
\pgfpathlineto{\pgfqpoint{2.499876in}{0.610460in}}%
\pgfpathlineto{\pgfqpoint{2.500432in}{0.611822in}}%
\pgfpathlineto{\pgfqpoint{2.501543in}{0.604043in}}%
\pgfpathlineto{\pgfqpoint{2.502099in}{0.605807in}}%
\pgfpathlineto{\pgfqpoint{2.502655in}{0.610198in}}%
\pgfpathlineto{\pgfqpoint{2.503211in}{0.603529in}}%
\pgfpathlineto{\pgfqpoint{2.503767in}{0.623449in}}%
\pgfpathlineto{\pgfqpoint{2.504323in}{0.610808in}}%
\pgfpathlineto{\pgfqpoint{2.504879in}{0.614931in}}%
\pgfpathlineto{\pgfqpoint{2.505435in}{0.610475in}}%
\pgfpathlineto{\pgfqpoint{2.505990in}{0.617861in}}%
\pgfpathlineto{\pgfqpoint{2.506546in}{0.604239in}}%
\pgfpathlineto{\pgfqpoint{2.507102in}{0.619165in}}%
\pgfpathlineto{\pgfqpoint{2.507658in}{0.609264in}}%
\pgfpathlineto{\pgfqpoint{2.508770in}{0.602696in}}%
\pgfpathlineto{\pgfqpoint{2.509881in}{0.614968in}}%
\pgfpathlineto{\pgfqpoint{2.510437in}{0.600537in}}%
\pgfpathlineto{\pgfqpoint{2.510993in}{0.613300in}}%
\pgfpathlineto{\pgfqpoint{2.512661in}{0.602832in}}%
\pgfpathlineto{\pgfqpoint{2.513772in}{0.621029in}}%
\pgfpathlineto{\pgfqpoint{2.514328in}{0.612599in}}%
\pgfpathlineto{\pgfqpoint{2.515996in}{0.605263in}}%
\pgfpathlineto{\pgfqpoint{2.517663in}{0.615225in}}%
\pgfpathlineto{\pgfqpoint{2.519331in}{0.609058in}}%
\pgfpathlineto{\pgfqpoint{2.520443in}{0.613364in}}%
\pgfpathlineto{\pgfqpoint{2.520999in}{0.602370in}}%
\pgfpathlineto{\pgfqpoint{2.521555in}{0.605153in}}%
\pgfpathlineto{\pgfqpoint{2.522110in}{0.612720in}}%
\pgfpathlineto{\pgfqpoint{2.522666in}{0.609005in}}%
\pgfpathlineto{\pgfqpoint{2.523778in}{0.603420in}}%
\pgfpathlineto{\pgfqpoint{2.524334in}{0.609704in}}%
\pgfpathlineto{\pgfqpoint{2.524890in}{0.606604in}}%
\pgfpathlineto{\pgfqpoint{2.525446in}{0.608543in}}%
\pgfpathlineto{\pgfqpoint{2.526001in}{0.604412in}}%
\pgfpathlineto{\pgfqpoint{2.526557in}{0.607490in}}%
\pgfpathlineto{\pgfqpoint{2.527113in}{0.606878in}}%
\pgfpathlineto{\pgfqpoint{2.527669in}{0.609307in}}%
\pgfpathlineto{\pgfqpoint{2.528781in}{0.602226in}}%
\pgfpathlineto{\pgfqpoint{2.529337in}{0.608939in}}%
\pgfpathlineto{\pgfqpoint{2.529892in}{0.607026in}}%
\pgfpathlineto{\pgfqpoint{2.530448in}{0.607551in}}%
\pgfpathlineto{\pgfqpoint{2.531560in}{0.603432in}}%
\pgfpathlineto{\pgfqpoint{2.532116in}{0.604177in}}%
\pgfpathlineto{\pgfqpoint{2.532672in}{0.605276in}}%
\pgfpathlineto{\pgfqpoint{2.534339in}{0.600399in}}%
\pgfpathlineto{\pgfqpoint{2.534895in}{0.603042in}}%
\pgfpathlineto{\pgfqpoint{2.535451in}{0.601365in}}%
\pgfpathlineto{\pgfqpoint{2.537119in}{0.604140in}}%
\pgfpathlineto{\pgfqpoint{2.538230in}{0.600488in}}%
\pgfpathlineto{\pgfqpoint{2.540454in}{0.602525in}}%
\pgfpathlineto{\pgfqpoint{2.542121in}{0.600212in}}%
\pgfpathlineto{\pgfqpoint{2.542677in}{0.600242in}}%
\pgfpathlineto{\pgfqpoint{2.544345in}{0.603115in}}%
\pgfpathlineto{\pgfqpoint{2.544901in}{0.602627in}}%
\pgfpathlineto{\pgfqpoint{2.545457in}{0.605199in}}%
\pgfpathlineto{\pgfqpoint{2.546012in}{0.602188in}}%
\pgfpathlineto{\pgfqpoint{2.546568in}{0.605848in}}%
\pgfpathlineto{\pgfqpoint{2.548236in}{0.600018in}}%
\pgfpathlineto{\pgfqpoint{2.549348in}{0.600856in}}%
\pgfpathlineto{\pgfqpoint{2.551015in}{0.600568in}}%
\pgfpathlineto{\pgfqpoint{2.552127in}{0.601819in}}%
\pgfpathlineto{\pgfqpoint{2.552683in}{0.601239in}}%
\pgfpathlineto{\pgfqpoint{2.553239in}{0.600812in}}%
\pgfpathlineto{\pgfqpoint{2.553794in}{0.602420in}}%
\pgfpathlineto{\pgfqpoint{2.554350in}{0.600424in}}%
\pgfpathlineto{\pgfqpoint{2.554906in}{0.602559in}}%
\pgfpathlineto{\pgfqpoint{2.556018in}{0.602846in}}%
\pgfpathlineto{\pgfqpoint{2.556574in}{0.601107in}}%
\pgfpathlineto{\pgfqpoint{2.557685in}{0.604204in}}%
\pgfpathlineto{\pgfqpoint{2.558241in}{0.603789in}}%
\pgfpathlineto{\pgfqpoint{2.559353in}{0.600410in}}%
\pgfpathlineto{\pgfqpoint{2.559909in}{0.606453in}}%
\pgfpathlineto{\pgfqpoint{2.560465in}{0.600350in}}%
\pgfpathlineto{\pgfqpoint{2.561021in}{0.608039in}}%
\pgfpathlineto{\pgfqpoint{2.561577in}{0.602172in}}%
\pgfpathlineto{\pgfqpoint{2.563244in}{0.616318in}}%
\pgfpathlineto{\pgfqpoint{2.563800in}{0.601286in}}%
\pgfpathlineto{\pgfqpoint{2.564356in}{0.609442in}}%
\pgfpathlineto{\pgfqpoint{2.566023in}{0.605117in}}%
\pgfpathlineto{\pgfqpoint{2.566579in}{0.609613in}}%
\pgfpathlineto{\pgfqpoint{2.567135in}{0.607634in}}%
\pgfpathlineto{\pgfqpoint{2.568247in}{0.605397in}}%
\pgfpathlineto{\pgfqpoint{2.568803in}{0.605604in}}%
\pgfpathlineto{\pgfqpoint{2.569359in}{0.608586in}}%
\pgfpathlineto{\pgfqpoint{2.569914in}{0.603507in}}%
\pgfpathlineto{\pgfqpoint{2.571026in}{0.620082in}}%
\pgfpathlineto{\pgfqpoint{2.571582in}{0.615326in}}%
\pgfpathlineto{\pgfqpoint{2.572694in}{0.602717in}}%
\pgfpathlineto{\pgfqpoint{2.573805in}{0.612198in}}%
\pgfpathlineto{\pgfqpoint{2.574361in}{0.601662in}}%
\pgfpathlineto{\pgfqpoint{2.575473in}{0.621826in}}%
\pgfpathlineto{\pgfqpoint{2.576029in}{0.608730in}}%
\pgfpathlineto{\pgfqpoint{2.576585in}{0.610716in}}%
\pgfpathlineto{\pgfqpoint{2.577141in}{0.618429in}}%
\pgfpathlineto{\pgfqpoint{2.577697in}{0.614466in}}%
\pgfpathlineto{\pgfqpoint{2.578808in}{0.605187in}}%
\pgfpathlineto{\pgfqpoint{2.580476in}{0.615344in}}%
\pgfpathlineto{\pgfqpoint{2.581032in}{0.603038in}}%
\pgfpathlineto{\pgfqpoint{2.581588in}{0.612805in}}%
\pgfpathlineto{\pgfqpoint{2.582143in}{0.605663in}}%
\pgfpathlineto{\pgfqpoint{2.582699in}{0.622942in}}%
\pgfpathlineto{\pgfqpoint{2.584367in}{0.600613in}}%
\pgfpathlineto{\pgfqpoint{2.586034in}{0.615058in}}%
\pgfpathlineto{\pgfqpoint{2.586590in}{0.605828in}}%
\pgfpathlineto{\pgfqpoint{2.587146in}{0.608970in}}%
\pgfpathlineto{\pgfqpoint{2.588814in}{0.616040in}}%
\pgfpathlineto{\pgfqpoint{2.589370in}{0.617615in}}%
\pgfpathlineto{\pgfqpoint{2.589925in}{0.608075in}}%
\pgfpathlineto{\pgfqpoint{2.590481in}{0.615006in}}%
\pgfpathlineto{\pgfqpoint{2.591037in}{0.620250in}}%
\pgfpathlineto{\pgfqpoint{2.591593in}{0.616445in}}%
\pgfpathlineto{\pgfqpoint{2.592149in}{0.615525in}}%
\pgfpathlineto{\pgfqpoint{2.593261in}{0.602538in}}%
\pgfpathlineto{\pgfqpoint{2.593816in}{0.602897in}}%
\pgfpathlineto{\pgfqpoint{2.594372in}{0.602598in}}%
\pgfpathlineto{\pgfqpoint{2.596596in}{0.619859in}}%
\pgfpathlineto{\pgfqpoint{2.598263in}{0.605630in}}%
\pgfpathlineto{\pgfqpoint{2.599375in}{0.613741in}}%
\pgfpathlineto{\pgfqpoint{2.599931in}{0.619880in}}%
\pgfpathlineto{\pgfqpoint{2.600487in}{0.616377in}}%
\pgfpathlineto{\pgfqpoint{2.602154in}{0.604869in}}%
\pgfpathlineto{\pgfqpoint{2.602710in}{0.607734in}}%
\pgfpathlineto{\pgfqpoint{2.603822in}{0.622442in}}%
\pgfpathlineto{\pgfqpoint{2.604934in}{0.604422in}}%
\pgfpathlineto{\pgfqpoint{2.605490in}{0.606382in}}%
\pgfpathlineto{\pgfqpoint{2.606045in}{0.613143in}}%
\pgfpathlineto{\pgfqpoint{2.606601in}{0.603505in}}%
\pgfpathlineto{\pgfqpoint{2.607157in}{0.611958in}}%
\pgfpathlineto{\pgfqpoint{2.607713in}{0.606238in}}%
\pgfpathlineto{\pgfqpoint{2.609381in}{0.624561in}}%
\pgfpathlineto{\pgfqpoint{2.609936in}{0.603041in}}%
\pgfpathlineto{\pgfqpoint{2.610492in}{0.604358in}}%
\pgfpathlineto{\pgfqpoint{2.611048in}{0.617627in}}%
\pgfpathlineto{\pgfqpoint{2.611604in}{0.610136in}}%
\pgfpathlineto{\pgfqpoint{2.612716in}{0.613011in}}%
\pgfpathlineto{\pgfqpoint{2.613272in}{0.621020in}}%
\pgfpathlineto{\pgfqpoint{2.613827in}{0.606723in}}%
\pgfpathlineto{\pgfqpoint{2.614383in}{0.627104in}}%
\pgfpathlineto{\pgfqpoint{2.614939in}{0.617151in}}%
\pgfpathlineto{\pgfqpoint{2.615495in}{0.626129in}}%
\pgfpathlineto{\pgfqpoint{2.616051in}{0.602703in}}%
\pgfpathlineto{\pgfqpoint{2.616607in}{0.608320in}}%
\pgfpathlineto{\pgfqpoint{2.617163in}{0.623679in}}%
\pgfpathlineto{\pgfqpoint{2.617719in}{0.619204in}}%
\pgfpathlineto{\pgfqpoint{2.618274in}{0.612045in}}%
\pgfpathlineto{\pgfqpoint{2.619386in}{0.631045in}}%
\pgfpathlineto{\pgfqpoint{2.619942in}{0.619533in}}%
\pgfpathlineto{\pgfqpoint{2.620498in}{0.645161in}}%
\pgfpathlineto{\pgfqpoint{2.621054in}{0.628106in}}%
\pgfpathlineto{\pgfqpoint{2.621610in}{0.618629in}}%
\pgfpathlineto{\pgfqpoint{2.622165in}{0.640746in}}%
\pgfpathlineto{\pgfqpoint{2.622721in}{0.624187in}}%
\pgfpathlineto{\pgfqpoint{2.623277in}{0.621901in}}%
\pgfpathlineto{\pgfqpoint{2.623833in}{0.614083in}}%
\pgfpathlineto{\pgfqpoint{2.624945in}{0.636557in}}%
\pgfpathlineto{\pgfqpoint{2.625501in}{0.607551in}}%
\pgfpathlineto{\pgfqpoint{2.626056in}{0.634110in}}%
\pgfpathlineto{\pgfqpoint{2.627168in}{0.606073in}}%
\pgfpathlineto{\pgfqpoint{2.628836in}{0.638546in}}%
\pgfpathlineto{\pgfqpoint{2.629947in}{0.618612in}}%
\pgfpathlineto{\pgfqpoint{2.630503in}{0.630437in}}%
\pgfpathlineto{\pgfqpoint{2.631059in}{0.615939in}}%
\pgfpathlineto{\pgfqpoint{2.631615in}{0.624670in}}%
\pgfpathlineto{\pgfqpoint{2.632171in}{0.649079in}}%
\pgfpathlineto{\pgfqpoint{2.632727in}{0.648604in}}%
\pgfpathlineto{\pgfqpoint{2.633283in}{0.611527in}}%
\pgfpathlineto{\pgfqpoint{2.633838in}{0.624913in}}%
\pgfpathlineto{\pgfqpoint{2.634394in}{0.635219in}}%
\pgfpathlineto{\pgfqpoint{2.634950in}{0.626303in}}%
\pgfpathlineto{\pgfqpoint{2.635506in}{0.630561in}}%
\pgfpathlineto{\pgfqpoint{2.636062in}{0.619619in}}%
\pgfpathlineto{\pgfqpoint{2.636618in}{0.621219in}}%
\pgfpathlineto{\pgfqpoint{2.637730in}{0.635854in}}%
\pgfpathlineto{\pgfqpoint{2.639397in}{0.613073in}}%
\pgfpathlineto{\pgfqpoint{2.639953in}{0.623159in}}%
\pgfpathlineto{\pgfqpoint{2.640509in}{0.620943in}}%
\pgfpathlineto{\pgfqpoint{2.641621in}{0.607841in}}%
\pgfpathlineto{\pgfqpoint{2.642176in}{0.616492in}}%
\pgfpathlineto{\pgfqpoint{2.642732in}{0.613991in}}%
\pgfpathlineto{\pgfqpoint{2.643288in}{0.614670in}}%
\pgfpathlineto{\pgfqpoint{2.643844in}{0.608230in}}%
\pgfpathlineto{\pgfqpoint{2.644400in}{0.619410in}}%
\pgfpathlineto{\pgfqpoint{2.644956in}{0.610135in}}%
\pgfpathlineto{\pgfqpoint{2.647179in}{0.629202in}}%
\pgfpathlineto{\pgfqpoint{2.647735in}{0.627424in}}%
\pgfpathlineto{\pgfqpoint{2.648847in}{0.614044in}}%
\pgfpathlineto{\pgfqpoint{2.650514in}{0.635284in}}%
\pgfpathlineto{\pgfqpoint{2.651626in}{0.620954in}}%
\pgfpathlineto{\pgfqpoint{2.652182in}{0.624975in}}%
\pgfpathlineto{\pgfqpoint{2.652738in}{0.622638in}}%
\pgfpathlineto{\pgfqpoint{2.654405in}{0.608220in}}%
\pgfpathlineto{\pgfqpoint{2.654961in}{0.607840in}}%
\pgfpathlineto{\pgfqpoint{2.657185in}{0.634079in}}%
\pgfpathlineto{\pgfqpoint{2.658296in}{0.601131in}}%
\pgfpathlineto{\pgfqpoint{2.660520in}{0.637492in}}%
\pgfpathlineto{\pgfqpoint{2.662187in}{0.611781in}}%
\pgfpathlineto{\pgfqpoint{2.662743in}{0.624004in}}%
\pgfpathlineto{\pgfqpoint{2.663299in}{0.622434in}}%
\pgfpathlineto{\pgfqpoint{2.664967in}{0.606290in}}%
\pgfpathlineto{\pgfqpoint{2.666634in}{0.627279in}}%
\pgfpathlineto{\pgfqpoint{2.667746in}{0.603472in}}%
\pgfpathlineto{\pgfqpoint{2.668858in}{0.625457in}}%
\pgfpathlineto{\pgfqpoint{2.669414in}{0.605216in}}%
\pgfpathlineto{\pgfqpoint{2.669969in}{0.612242in}}%
\pgfpathlineto{\pgfqpoint{2.670525in}{0.629717in}}%
\pgfpathlineto{\pgfqpoint{2.671081in}{0.622841in}}%
\pgfpathlineto{\pgfqpoint{2.672193in}{0.627522in}}%
\pgfpathlineto{\pgfqpoint{2.672749in}{0.625832in}}%
\pgfpathlineto{\pgfqpoint{2.674416in}{0.611647in}}%
\pgfpathlineto{\pgfqpoint{2.674972in}{0.642620in}}%
\pgfpathlineto{\pgfqpoint{2.675528in}{0.602857in}}%
\pgfpathlineto{\pgfqpoint{2.676084in}{0.643445in}}%
\pgfpathlineto{\pgfqpoint{2.676640in}{0.621718in}}%
\pgfpathlineto{\pgfqpoint{2.677196in}{0.619764in}}%
\pgfpathlineto{\pgfqpoint{2.677752in}{0.620060in}}%
\pgfpathlineto{\pgfqpoint{2.678307in}{0.668952in}}%
\pgfpathlineto{\pgfqpoint{2.678863in}{0.605843in}}%
\pgfpathlineto{\pgfqpoint{2.679419in}{0.651689in}}%
\pgfpathlineto{\pgfqpoint{2.679975in}{0.615856in}}%
\pgfpathlineto{\pgfqpoint{2.680531in}{0.635507in}}%
\pgfpathlineto{\pgfqpoint{2.681087in}{0.615918in}}%
\pgfpathlineto{\pgfqpoint{2.681643in}{0.650663in}}%
\pgfpathlineto{\pgfqpoint{2.682198in}{0.650276in}}%
\pgfpathlineto{\pgfqpoint{2.682754in}{0.609451in}}%
\pgfpathlineto{\pgfqpoint{2.683310in}{0.631196in}}%
\pgfpathlineto{\pgfqpoint{2.684978in}{0.610437in}}%
\pgfpathlineto{\pgfqpoint{2.686089in}{0.658974in}}%
\pgfpathlineto{\pgfqpoint{2.686645in}{0.655103in}}%
\pgfpathlineto{\pgfqpoint{2.688313in}{0.631272in}}%
\pgfpathlineto{\pgfqpoint{2.689425in}{0.637778in}}%
\pgfpathlineto{\pgfqpoint{2.689980in}{0.665767in}}%
\pgfpathlineto{\pgfqpoint{2.690536in}{0.638016in}}%
\pgfpathlineto{\pgfqpoint{2.691648in}{0.628941in}}%
\pgfpathlineto{\pgfqpoint{2.692204in}{0.651321in}}%
\pgfpathlineto{\pgfqpoint{2.692760in}{0.629811in}}%
\pgfpathlineto{\pgfqpoint{2.693316in}{0.634228in}}%
\pgfpathlineto{\pgfqpoint{2.693872in}{0.617540in}}%
\pgfpathlineto{\pgfqpoint{2.694427in}{0.642812in}}%
\pgfpathlineto{\pgfqpoint{2.694983in}{0.628788in}}%
\pgfpathlineto{\pgfqpoint{2.695539in}{0.629779in}}%
\pgfpathlineto{\pgfqpoint{2.697207in}{0.604170in}}%
\pgfpathlineto{\pgfqpoint{2.697763in}{0.637942in}}%
\pgfpathlineto{\pgfqpoint{2.698318in}{0.621614in}}%
\pgfpathlineto{\pgfqpoint{2.699986in}{0.607034in}}%
\pgfpathlineto{\pgfqpoint{2.701654in}{0.624700in}}%
\pgfpathlineto{\pgfqpoint{2.702209in}{0.607582in}}%
\pgfpathlineto{\pgfqpoint{2.702765in}{0.618818in}}%
\pgfpathlineto{\pgfqpoint{2.703321in}{0.611026in}}%
\pgfpathlineto{\pgfqpoint{2.703877in}{0.620867in}}%
\pgfpathlineto{\pgfqpoint{2.704433in}{0.604705in}}%
\pgfpathlineto{\pgfqpoint{2.704989in}{0.607834in}}%
\pgfpathlineto{\pgfqpoint{2.706100in}{0.625373in}}%
\pgfpathlineto{\pgfqpoint{2.707212in}{0.612746in}}%
\pgfpathlineto{\pgfqpoint{2.708880in}{0.637681in}}%
\pgfpathlineto{\pgfqpoint{2.709436in}{0.626271in}}%
\pgfpathlineto{\pgfqpoint{2.709992in}{0.635064in}}%
\pgfpathlineto{\pgfqpoint{2.710547in}{0.638694in}}%
\pgfpathlineto{\pgfqpoint{2.711103in}{0.638435in}}%
\pgfpathlineto{\pgfqpoint{2.712771in}{0.616655in}}%
\pgfpathlineto{\pgfqpoint{2.713327in}{0.616875in}}%
\pgfpathlineto{\pgfqpoint{2.714438in}{0.609669in}}%
\pgfpathlineto{\pgfqpoint{2.716662in}{0.627946in}}%
\pgfpathlineto{\pgfqpoint{2.717218in}{0.628276in}}%
\pgfpathlineto{\pgfqpoint{2.718329in}{0.613062in}}%
\pgfpathlineto{\pgfqpoint{2.718885in}{0.617027in}}%
\pgfpathlineto{\pgfqpoint{2.719997in}{0.636082in}}%
\pgfpathlineto{\pgfqpoint{2.721665in}{0.603231in}}%
\pgfpathlineto{\pgfqpoint{2.722220in}{0.605085in}}%
\pgfpathlineto{\pgfqpoint{2.722776in}{0.634161in}}%
\pgfpathlineto{\pgfqpoint{2.723332in}{0.624151in}}%
\pgfpathlineto{\pgfqpoint{2.723888in}{0.629839in}}%
\pgfpathlineto{\pgfqpoint{2.724444in}{0.624964in}}%
\pgfpathlineto{\pgfqpoint{2.725000in}{0.609666in}}%
\pgfpathlineto{\pgfqpoint{2.725556in}{0.620072in}}%
\pgfpathlineto{\pgfqpoint{2.726111in}{0.634816in}}%
\pgfpathlineto{\pgfqpoint{2.727223in}{0.608443in}}%
\pgfpathlineto{\pgfqpoint{2.728335in}{0.629563in}}%
\pgfpathlineto{\pgfqpoint{2.728891in}{0.605403in}}%
\pgfpathlineto{\pgfqpoint{2.729447in}{0.626592in}}%
\pgfpathlineto{\pgfqpoint{2.730003in}{0.626972in}}%
\pgfpathlineto{\pgfqpoint{2.730558in}{0.631279in}}%
\pgfpathlineto{\pgfqpoint{2.731114in}{0.607278in}}%
\pgfpathlineto{\pgfqpoint{2.731670in}{0.617677in}}%
\pgfpathlineto{\pgfqpoint{2.732226in}{0.647721in}}%
\pgfpathlineto{\pgfqpoint{2.732782in}{0.619503in}}%
\pgfpathlineto{\pgfqpoint{2.733338in}{0.626607in}}%
\pgfpathlineto{\pgfqpoint{2.733894in}{0.619363in}}%
\pgfpathlineto{\pgfqpoint{2.734449in}{0.645466in}}%
\pgfpathlineto{\pgfqpoint{2.735005in}{0.610392in}}%
\pgfpathlineto{\pgfqpoint{2.735561in}{0.687655in}}%
\pgfpathlineto{\pgfqpoint{2.736117in}{0.617288in}}%
\pgfpathlineto{\pgfqpoint{2.737785in}{0.653366in}}%
\pgfpathlineto{\pgfqpoint{2.738340in}{0.650563in}}%
\pgfpathlineto{\pgfqpoint{2.738896in}{0.652566in}}%
\pgfpathlineto{\pgfqpoint{2.740564in}{0.617693in}}%
\pgfpathlineto{\pgfqpoint{2.741120in}{0.667411in}}%
\pgfpathlineto{\pgfqpoint{2.741676in}{0.642377in}}%
\pgfpathlineto{\pgfqpoint{2.742231in}{0.608831in}}%
\pgfpathlineto{\pgfqpoint{2.743899in}{0.648124in}}%
\pgfpathlineto{\pgfqpoint{2.744455in}{0.611617in}}%
\pgfpathlineto{\pgfqpoint{2.745011in}{0.645211in}}%
\pgfpathlineto{\pgfqpoint{2.745567in}{0.656533in}}%
\pgfpathlineto{\pgfqpoint{2.746678in}{0.620325in}}%
\pgfpathlineto{\pgfqpoint{2.747234in}{0.687273in}}%
\pgfpathlineto{\pgfqpoint{2.747790in}{0.659729in}}%
\pgfpathlineto{\pgfqpoint{2.748346in}{0.626961in}}%
\pgfpathlineto{\pgfqpoint{2.748902in}{0.628766in}}%
\pgfpathlineto{\pgfqpoint{2.749458in}{0.664768in}}%
\pgfpathlineto{\pgfqpoint{2.750014in}{0.631634in}}%
\pgfpathlineto{\pgfqpoint{2.750569in}{0.644185in}}%
\pgfpathlineto{\pgfqpoint{2.751125in}{0.620369in}}%
\pgfpathlineto{\pgfqpoint{2.751681in}{0.631910in}}%
\pgfpathlineto{\pgfqpoint{2.752793in}{0.639981in}}%
\pgfpathlineto{\pgfqpoint{2.753349in}{0.608688in}}%
\pgfpathlineto{\pgfqpoint{2.753905in}{0.623679in}}%
\pgfpathlineto{\pgfqpoint{2.754460in}{0.613576in}}%
\pgfpathlineto{\pgfqpoint{2.755016in}{0.642079in}}%
\pgfpathlineto{\pgfqpoint{2.755572in}{0.608805in}}%
\pgfpathlineto{\pgfqpoint{2.756128in}{0.611695in}}%
\pgfpathlineto{\pgfqpoint{2.756684in}{0.616911in}}%
\pgfpathlineto{\pgfqpoint{2.757240in}{0.614533in}}%
\pgfpathlineto{\pgfqpoint{2.757796in}{0.609405in}}%
\pgfpathlineto{\pgfqpoint{2.758351in}{0.617940in}}%
\pgfpathlineto{\pgfqpoint{2.758907in}{0.601672in}}%
\pgfpathlineto{\pgfqpoint{2.759463in}{0.607043in}}%
\pgfpathlineto{\pgfqpoint{2.760019in}{0.616440in}}%
\pgfpathlineto{\pgfqpoint{2.761687in}{0.603084in}}%
\pgfpathlineto{\pgfqpoint{2.762242in}{0.615267in}}%
\pgfpathlineto{\pgfqpoint{2.762798in}{0.604667in}}%
\pgfpathlineto{\pgfqpoint{2.763354in}{0.608881in}}%
\pgfpathlineto{\pgfqpoint{2.763910in}{0.601336in}}%
\pgfpathlineto{\pgfqpoint{2.764466in}{0.617624in}}%
\pgfpathlineto{\pgfqpoint{2.765022in}{0.607078in}}%
\pgfpathlineto{\pgfqpoint{2.765578in}{0.606503in}}%
\pgfpathlineto{\pgfqpoint{2.767801in}{0.629583in}}%
\pgfpathlineto{\pgfqpoint{2.768357in}{0.627342in}}%
\pgfpathlineto{\pgfqpoint{2.768913in}{0.631529in}}%
\pgfpathlineto{\pgfqpoint{2.769469in}{0.646435in}}%
\pgfpathlineto{\pgfqpoint{2.770025in}{0.636332in}}%
\pgfpathlineto{\pgfqpoint{2.770580in}{0.637975in}}%
\pgfpathlineto{\pgfqpoint{2.771136in}{0.626895in}}%
\pgfpathlineto{\pgfqpoint{2.771692in}{0.639980in}}%
\pgfpathlineto{\pgfqpoint{2.772248in}{0.639168in}}%
\pgfpathlineto{\pgfqpoint{2.773360in}{0.629141in}}%
\pgfpathlineto{\pgfqpoint{2.774471in}{0.608498in}}%
\pgfpathlineto{\pgfqpoint{2.775027in}{0.613131in}}%
\pgfpathlineto{\pgfqpoint{2.775583in}{0.612012in}}%
\pgfpathlineto{\pgfqpoint{2.776695in}{0.647499in}}%
\pgfpathlineto{\pgfqpoint{2.778362in}{0.611939in}}%
\pgfpathlineto{\pgfqpoint{2.780586in}{0.647888in}}%
\pgfpathlineto{\pgfqpoint{2.781142in}{0.610960in}}%
\pgfpathlineto{\pgfqpoint{2.781698in}{0.611814in}}%
\pgfpathlineto{\pgfqpoint{2.783365in}{0.636582in}}%
\pgfpathlineto{\pgfqpoint{2.784477in}{0.607458in}}%
\pgfpathlineto{\pgfqpoint{2.785033in}{0.613547in}}%
\pgfpathlineto{\pgfqpoint{2.785589in}{0.649064in}}%
\pgfpathlineto{\pgfqpoint{2.786145in}{0.626546in}}%
\pgfpathlineto{\pgfqpoint{2.786700in}{0.621860in}}%
\pgfpathlineto{\pgfqpoint{2.787256in}{0.623052in}}%
\pgfpathlineto{\pgfqpoint{2.787812in}{0.631986in}}%
\pgfpathlineto{\pgfqpoint{2.788368in}{0.606004in}}%
\pgfpathlineto{\pgfqpoint{2.788924in}{0.627395in}}%
\pgfpathlineto{\pgfqpoint{2.789480in}{0.625870in}}%
\pgfpathlineto{\pgfqpoint{2.790036in}{0.647597in}}%
\pgfpathlineto{\pgfqpoint{2.790591in}{0.607795in}}%
\pgfpathlineto{\pgfqpoint{2.791147in}{0.630617in}}%
\pgfpathlineto{\pgfqpoint{2.791703in}{0.655722in}}%
\pgfpathlineto{\pgfqpoint{2.792259in}{0.611440in}}%
\pgfpathlineto{\pgfqpoint{2.792815in}{0.663020in}}%
\pgfpathlineto{\pgfqpoint{2.793371in}{0.652483in}}%
\pgfpathlineto{\pgfqpoint{2.793927in}{0.611457in}}%
\pgfpathlineto{\pgfqpoint{2.794482in}{0.645779in}}%
\pgfpathlineto{\pgfqpoint{2.795594in}{0.655190in}}%
\pgfpathlineto{\pgfqpoint{2.796150in}{0.619287in}}%
\pgfpathlineto{\pgfqpoint{2.796706in}{0.666646in}}%
\pgfpathlineto{\pgfqpoint{2.797262in}{0.651930in}}%
\pgfpathlineto{\pgfqpoint{2.797818in}{0.634334in}}%
\pgfpathlineto{\pgfqpoint{2.798373in}{0.660441in}}%
\pgfpathlineto{\pgfqpoint{2.798929in}{0.635016in}}%
\pgfpathlineto{\pgfqpoint{2.799485in}{0.636963in}}%
\pgfpathlineto{\pgfqpoint{2.800597in}{0.621847in}}%
\pgfpathlineto{\pgfqpoint{2.802264in}{0.668561in}}%
\pgfpathlineto{\pgfqpoint{2.802820in}{0.657457in}}%
\pgfpathlineto{\pgfqpoint{2.803376in}{0.630311in}}%
\pgfpathlineto{\pgfqpoint{2.803932in}{0.644858in}}%
\pgfpathlineto{\pgfqpoint{2.804488in}{0.668754in}}%
\pgfpathlineto{\pgfqpoint{2.805044in}{0.659940in}}%
\pgfpathlineto{\pgfqpoint{2.806156in}{0.638310in}}%
\pgfpathlineto{\pgfqpoint{2.806711in}{0.640257in}}%
\pgfpathlineto{\pgfqpoint{2.807267in}{0.653690in}}%
\pgfpathlineto{\pgfqpoint{2.807823in}{0.646497in}}%
\pgfpathlineto{\pgfqpoint{2.808379in}{0.626048in}}%
\pgfpathlineto{\pgfqpoint{2.808935in}{0.627898in}}%
\pgfpathlineto{\pgfqpoint{2.809491in}{0.627586in}}%
\pgfpathlineto{\pgfqpoint{2.810047in}{0.634736in}}%
\pgfpathlineto{\pgfqpoint{2.811714in}{0.601079in}}%
\pgfpathlineto{\pgfqpoint{2.812270in}{0.616617in}}%
\pgfpathlineto{\pgfqpoint{2.812826in}{0.615436in}}%
\pgfpathlineto{\pgfqpoint{2.813382in}{0.614882in}}%
\pgfpathlineto{\pgfqpoint{2.813938in}{0.606140in}}%
\pgfpathlineto{\pgfqpoint{2.814493in}{0.631348in}}%
\pgfpathlineto{\pgfqpoint{2.815049in}{0.613033in}}%
\pgfpathlineto{\pgfqpoint{2.815605in}{0.617037in}}%
\pgfpathlineto{\pgfqpoint{2.816161in}{0.609651in}}%
\pgfpathlineto{\pgfqpoint{2.816717in}{0.614732in}}%
\pgfpathlineto{\pgfqpoint{2.817273in}{0.617953in}}%
\pgfpathlineto{\pgfqpoint{2.817829in}{0.616611in}}%
\pgfpathlineto{\pgfqpoint{2.818384in}{0.611516in}}%
\pgfpathlineto{\pgfqpoint{2.818940in}{0.612714in}}%
\pgfpathlineto{\pgfqpoint{2.819496in}{0.617470in}}%
\pgfpathlineto{\pgfqpoint{2.820052in}{0.609939in}}%
\pgfpathlineto{\pgfqpoint{2.820608in}{0.618967in}}%
\pgfpathlineto{\pgfqpoint{2.822275in}{0.606887in}}%
\pgfpathlineto{\pgfqpoint{2.822831in}{0.602042in}}%
\pgfpathlineto{\pgfqpoint{2.823943in}{0.618366in}}%
\pgfpathlineto{\pgfqpoint{2.824499in}{0.615698in}}%
\pgfpathlineto{\pgfqpoint{2.825055in}{0.615004in}}%
\pgfpathlineto{\pgfqpoint{2.825611in}{0.607866in}}%
\pgfpathlineto{\pgfqpoint{2.826167in}{0.613416in}}%
\pgfpathlineto{\pgfqpoint{2.826722in}{0.612318in}}%
\pgfpathlineto{\pgfqpoint{2.827278in}{0.615441in}}%
\pgfpathlineto{\pgfqpoint{2.828390in}{0.634521in}}%
\pgfpathlineto{\pgfqpoint{2.828946in}{0.634375in}}%
\pgfpathlineto{\pgfqpoint{2.829502in}{0.632257in}}%
\pgfpathlineto{\pgfqpoint{2.830058in}{0.634050in}}%
\pgfpathlineto{\pgfqpoint{2.831169in}{0.648579in}}%
\pgfpathlineto{\pgfqpoint{2.831725in}{0.645086in}}%
\pgfpathlineto{\pgfqpoint{2.832281in}{0.631413in}}%
\pgfpathlineto{\pgfqpoint{2.832837in}{0.632764in}}%
\pgfpathlineto{\pgfqpoint{2.833393in}{0.638268in}}%
\pgfpathlineto{\pgfqpoint{2.835060in}{0.604617in}}%
\pgfpathlineto{\pgfqpoint{2.836728in}{0.639610in}}%
\pgfpathlineto{\pgfqpoint{2.838395in}{0.607094in}}%
\pgfpathlineto{\pgfqpoint{2.840063in}{0.638720in}}%
\pgfpathlineto{\pgfqpoint{2.840619in}{0.637442in}}%
\pgfpathlineto{\pgfqpoint{2.841731in}{0.606282in}}%
\pgfpathlineto{\pgfqpoint{2.842842in}{0.649217in}}%
\pgfpathlineto{\pgfqpoint{2.844510in}{0.612534in}}%
\pgfpathlineto{\pgfqpoint{2.845066in}{0.630012in}}%
\pgfpathlineto{\pgfqpoint{2.845622in}{0.621578in}}%
\pgfpathlineto{\pgfqpoint{2.846178in}{0.626636in}}%
\pgfpathlineto{\pgfqpoint{2.846733in}{0.610339in}}%
\pgfpathlineto{\pgfqpoint{2.847289in}{0.651314in}}%
\pgfpathlineto{\pgfqpoint{2.847845in}{0.606781in}}%
\pgfpathlineto{\pgfqpoint{2.848401in}{0.639813in}}%
\pgfpathlineto{\pgfqpoint{2.850069in}{0.628136in}}%
\pgfpathlineto{\pgfqpoint{2.850624in}{0.674374in}}%
\pgfpathlineto{\pgfqpoint{2.851180in}{0.624461in}}%
\pgfpathlineto{\pgfqpoint{2.851736in}{0.644991in}}%
\pgfpathlineto{\pgfqpoint{2.852292in}{0.650754in}}%
\pgfpathlineto{\pgfqpoint{2.852848in}{0.671370in}}%
\pgfpathlineto{\pgfqpoint{2.853404in}{0.656589in}}%
\pgfpathlineto{\pgfqpoint{2.853960in}{0.669124in}}%
\pgfpathlineto{\pgfqpoint{2.855627in}{0.618785in}}%
\pgfpathlineto{\pgfqpoint{2.856183in}{0.675792in}}%
\pgfpathlineto{\pgfqpoint{2.856739in}{0.662961in}}%
\pgfpathlineto{\pgfqpoint{2.857295in}{0.607949in}}%
\pgfpathlineto{\pgfqpoint{2.857851in}{0.652122in}}%
\pgfpathlineto{\pgfqpoint{2.858406in}{0.622219in}}%
\pgfpathlineto{\pgfqpoint{2.858962in}{0.643144in}}%
\pgfpathlineto{\pgfqpoint{2.859518in}{0.671839in}}%
\pgfpathlineto{\pgfqpoint{2.860074in}{0.665587in}}%
\pgfpathlineto{\pgfqpoint{2.860630in}{0.644062in}}%
\pgfpathlineto{\pgfqpoint{2.861186in}{0.650751in}}%
\pgfpathlineto{\pgfqpoint{2.861742in}{0.652418in}}%
\pgfpathlineto{\pgfqpoint{2.862298in}{0.694989in}}%
\pgfpathlineto{\pgfqpoint{2.863965in}{0.610449in}}%
\pgfpathlineto{\pgfqpoint{2.864521in}{0.658072in}}%
\pgfpathlineto{\pgfqpoint{2.865077in}{0.622080in}}%
\pgfpathlineto{\pgfqpoint{2.865633in}{0.645287in}}%
\pgfpathlineto{\pgfqpoint{2.867300in}{0.608295in}}%
\pgfpathlineto{\pgfqpoint{2.867856in}{0.634470in}}%
\pgfpathlineto{\pgfqpoint{2.868412in}{0.621040in}}%
\pgfpathlineto{\pgfqpoint{2.868968in}{0.609188in}}%
\pgfpathlineto{\pgfqpoint{2.869524in}{0.613879in}}%
\pgfpathlineto{\pgfqpoint{2.870635in}{0.628165in}}%
\pgfpathlineto{\pgfqpoint{2.872303in}{0.605660in}}%
\pgfpathlineto{\pgfqpoint{2.875082in}{0.614549in}}%
\pgfpathlineto{\pgfqpoint{2.876750in}{0.605006in}}%
\pgfpathlineto{\pgfqpoint{2.877306in}{0.608131in}}%
\pgfpathlineto{\pgfqpoint{2.877862in}{0.614505in}}%
\pgfpathlineto{\pgfqpoint{2.878973in}{0.605183in}}%
\pgfpathlineto{\pgfqpoint{2.879529in}{0.614047in}}%
\pgfpathlineto{\pgfqpoint{2.880085in}{0.612788in}}%
\pgfpathlineto{\pgfqpoint{2.880641in}{0.603917in}}%
\pgfpathlineto{\pgfqpoint{2.881197in}{0.609109in}}%
\pgfpathlineto{\pgfqpoint{2.883976in}{0.601356in}}%
\pgfpathlineto{\pgfqpoint{2.885644in}{0.614084in}}%
\pgfpathlineto{\pgfqpoint{2.886200in}{0.607786in}}%
\pgfpathlineto{\pgfqpoint{2.888979in}{0.634440in}}%
\pgfpathlineto{\pgfqpoint{2.889535in}{0.625801in}}%
\pgfpathlineto{\pgfqpoint{2.890646in}{0.645718in}}%
\pgfpathlineto{\pgfqpoint{2.891202in}{0.638440in}}%
\pgfpathlineto{\pgfqpoint{2.891758in}{0.636421in}}%
\pgfpathlineto{\pgfqpoint{2.892314in}{0.644450in}}%
\pgfpathlineto{\pgfqpoint{2.892870in}{0.643446in}}%
\pgfpathlineto{\pgfqpoint{2.894537in}{0.602088in}}%
\pgfpathlineto{\pgfqpoint{2.896761in}{0.652880in}}%
\pgfpathlineto{\pgfqpoint{2.898429in}{0.604796in}}%
\pgfpathlineto{\pgfqpoint{2.899540in}{0.650121in}}%
\pgfpathlineto{\pgfqpoint{2.900096in}{0.640744in}}%
\pgfpathlineto{\pgfqpoint{2.900652in}{0.638080in}}%
\pgfpathlineto{\pgfqpoint{2.901208in}{0.607758in}}%
\pgfpathlineto{\pgfqpoint{2.901764in}{0.613960in}}%
\pgfpathlineto{\pgfqpoint{2.902875in}{0.648597in}}%
\pgfpathlineto{\pgfqpoint{2.903431in}{0.607322in}}%
\pgfpathlineto{\pgfqpoint{2.903987in}{0.609354in}}%
\pgfpathlineto{\pgfqpoint{2.904543in}{0.648107in}}%
\pgfpathlineto{\pgfqpoint{2.905099in}{0.628206in}}%
\pgfpathlineto{\pgfqpoint{2.905655in}{0.628583in}}%
\pgfpathlineto{\pgfqpoint{2.906211in}{0.619863in}}%
\pgfpathlineto{\pgfqpoint{2.906766in}{0.653882in}}%
\pgfpathlineto{\pgfqpoint{2.907322in}{0.609462in}}%
\pgfpathlineto{\pgfqpoint{2.907878in}{0.676106in}}%
\pgfpathlineto{\pgfqpoint{2.908434in}{0.625028in}}%
\pgfpathlineto{\pgfqpoint{2.908990in}{0.622859in}}%
\pgfpathlineto{\pgfqpoint{2.910102in}{0.699741in}}%
\pgfpathlineto{\pgfqpoint{2.910657in}{0.681188in}}%
\pgfpathlineto{\pgfqpoint{2.911213in}{0.626414in}}%
\pgfpathlineto{\pgfqpoint{2.911769in}{0.673737in}}%
\pgfpathlineto{\pgfqpoint{2.912325in}{0.656736in}}%
\pgfpathlineto{\pgfqpoint{2.912881in}{0.689538in}}%
\pgfpathlineto{\pgfqpoint{2.913437in}{0.680535in}}%
\pgfpathlineto{\pgfqpoint{2.913993in}{0.668466in}}%
\pgfpathlineto{\pgfqpoint{2.914548in}{0.620890in}}%
\pgfpathlineto{\pgfqpoint{2.915104in}{0.623637in}}%
\pgfpathlineto{\pgfqpoint{2.916216in}{0.617851in}}%
\pgfpathlineto{\pgfqpoint{2.916772in}{0.631159in}}%
\pgfpathlineto{\pgfqpoint{2.917328in}{0.733088in}}%
\pgfpathlineto{\pgfqpoint{2.917884in}{0.681547in}}%
\pgfpathlineto{\pgfqpoint{2.918440in}{0.632482in}}%
\pgfpathlineto{\pgfqpoint{2.918995in}{0.652714in}}%
\pgfpathlineto{\pgfqpoint{2.919551in}{0.700789in}}%
\pgfpathlineto{\pgfqpoint{2.921219in}{0.628733in}}%
\pgfpathlineto{\pgfqpoint{2.922886in}{0.660457in}}%
\pgfpathlineto{\pgfqpoint{2.924554in}{0.606556in}}%
\pgfpathlineto{\pgfqpoint{2.925110in}{0.628736in}}%
\pgfpathlineto{\pgfqpoint{2.925666in}{0.622164in}}%
\pgfpathlineto{\pgfqpoint{2.926222in}{0.613473in}}%
\pgfpathlineto{\pgfqpoint{2.926777in}{0.620767in}}%
\pgfpathlineto{\pgfqpoint{2.928445in}{0.603025in}}%
\pgfpathlineto{\pgfqpoint{2.929001in}{0.619594in}}%
\pgfpathlineto{\pgfqpoint{2.929557in}{0.612037in}}%
\pgfpathlineto{\pgfqpoint{2.930668in}{0.621322in}}%
\pgfpathlineto{\pgfqpoint{2.931224in}{0.617968in}}%
\pgfpathlineto{\pgfqpoint{2.931780in}{0.602646in}}%
\pgfpathlineto{\pgfqpoint{2.932336in}{0.621516in}}%
\pgfpathlineto{\pgfqpoint{2.932892in}{0.609698in}}%
\pgfpathlineto{\pgfqpoint{2.933448in}{0.617855in}}%
\pgfpathlineto{\pgfqpoint{2.934004in}{0.607088in}}%
\pgfpathlineto{\pgfqpoint{2.934559in}{0.607218in}}%
\pgfpathlineto{\pgfqpoint{2.935115in}{0.608789in}}%
\pgfpathlineto{\pgfqpoint{2.935671in}{0.617073in}}%
\pgfpathlineto{\pgfqpoint{2.936227in}{0.609931in}}%
\pgfpathlineto{\pgfqpoint{2.937339in}{0.602841in}}%
\pgfpathlineto{\pgfqpoint{2.937895in}{0.603835in}}%
\pgfpathlineto{\pgfqpoint{2.938451in}{0.608294in}}%
\pgfpathlineto{\pgfqpoint{2.939006in}{0.607658in}}%
\pgfpathlineto{\pgfqpoint{2.939562in}{0.604488in}}%
\pgfpathlineto{\pgfqpoint{2.940118in}{0.618009in}}%
\pgfpathlineto{\pgfqpoint{2.940674in}{0.607375in}}%
\pgfpathlineto{\pgfqpoint{2.941230in}{0.611680in}}%
\pgfpathlineto{\pgfqpoint{2.941786in}{0.609957in}}%
\pgfpathlineto{\pgfqpoint{2.942897in}{0.609000in}}%
\pgfpathlineto{\pgfqpoint{2.943453in}{0.605108in}}%
\pgfpathlineto{\pgfqpoint{2.944009in}{0.613464in}}%
\pgfpathlineto{\pgfqpoint{2.944565in}{0.606878in}}%
\pgfpathlineto{\pgfqpoint{2.945121in}{0.612349in}}%
\pgfpathlineto{\pgfqpoint{2.945677in}{0.605193in}}%
\pgfpathlineto{\pgfqpoint{2.946233in}{0.607850in}}%
\pgfpathlineto{\pgfqpoint{2.946788in}{0.605395in}}%
\pgfpathlineto{\pgfqpoint{2.949012in}{0.652083in}}%
\pgfpathlineto{\pgfqpoint{2.949568in}{0.631428in}}%
\pgfpathlineto{\pgfqpoint{2.950124in}{0.640346in}}%
\pgfpathlineto{\pgfqpoint{2.951791in}{0.661374in}}%
\pgfpathlineto{\pgfqpoint{2.952347in}{0.661711in}}%
\pgfpathlineto{\pgfqpoint{2.954570in}{0.608264in}}%
\pgfpathlineto{\pgfqpoint{2.955126in}{0.615301in}}%
\pgfpathlineto{\pgfqpoint{2.956794in}{0.672832in}}%
\pgfpathlineto{\pgfqpoint{2.958462in}{0.620030in}}%
\pgfpathlineto{\pgfqpoint{2.959017in}{0.647525in}}%
\pgfpathlineto{\pgfqpoint{2.959573in}{0.640766in}}%
\pgfpathlineto{\pgfqpoint{2.960129in}{0.639290in}}%
\pgfpathlineto{\pgfqpoint{2.961241in}{0.610932in}}%
\pgfpathlineto{\pgfqpoint{2.962353in}{0.671281in}}%
\pgfpathlineto{\pgfqpoint{2.962908in}{0.607820in}}%
\pgfpathlineto{\pgfqpoint{2.963464in}{0.620634in}}%
\pgfpathlineto{\pgfqpoint{2.964020in}{0.658919in}}%
\pgfpathlineto{\pgfqpoint{2.964576in}{0.619083in}}%
\pgfpathlineto{\pgfqpoint{2.965132in}{0.643909in}}%
\pgfpathlineto{\pgfqpoint{2.965688in}{0.663523in}}%
\pgfpathlineto{\pgfqpoint{2.966799in}{0.630887in}}%
\pgfpathlineto{\pgfqpoint{2.967355in}{0.705711in}}%
\pgfpathlineto{\pgfqpoint{2.967911in}{0.682223in}}%
\pgfpathlineto{\pgfqpoint{2.968467in}{0.657548in}}%
\pgfpathlineto{\pgfqpoint{2.969023in}{0.693693in}}%
\pgfpathlineto{\pgfqpoint{2.969579in}{0.645062in}}%
\pgfpathlineto{\pgfqpoint{2.970135in}{0.705271in}}%
\pgfpathlineto{\pgfqpoint{2.970690in}{0.609928in}}%
\pgfpathlineto{\pgfqpoint{2.971246in}{0.682803in}}%
\pgfpathlineto{\pgfqpoint{2.972358in}{0.636736in}}%
\pgfpathlineto{\pgfqpoint{2.972914in}{0.667178in}}%
\pgfpathlineto{\pgfqpoint{2.973470in}{0.614294in}}%
\pgfpathlineto{\pgfqpoint{2.974026in}{0.655256in}}%
\pgfpathlineto{\pgfqpoint{2.974582in}{0.740169in}}%
\pgfpathlineto{\pgfqpoint{2.975137in}{0.681382in}}%
\pgfpathlineto{\pgfqpoint{2.975693in}{0.629266in}}%
\pgfpathlineto{\pgfqpoint{2.976249in}{0.655253in}}%
\pgfpathlineto{\pgfqpoint{2.976805in}{0.659800in}}%
\pgfpathlineto{\pgfqpoint{2.977361in}{0.678510in}}%
\pgfpathlineto{\pgfqpoint{2.977917in}{0.669837in}}%
\pgfpathlineto{\pgfqpoint{2.978473in}{0.660324in}}%
\pgfpathlineto{\pgfqpoint{2.979028in}{0.626939in}}%
\pgfpathlineto{\pgfqpoint{2.979584in}{0.664395in}}%
\pgfpathlineto{\pgfqpoint{2.980140in}{0.627379in}}%
\pgfpathlineto{\pgfqpoint{2.982919in}{0.607910in}}%
\pgfpathlineto{\pgfqpoint{2.984587in}{0.627226in}}%
\pgfpathlineto{\pgfqpoint{2.985143in}{0.604676in}}%
\pgfpathlineto{\pgfqpoint{2.985699in}{0.611202in}}%
\pgfpathlineto{\pgfqpoint{2.986255in}{0.620759in}}%
\pgfpathlineto{\pgfqpoint{2.986810in}{0.617973in}}%
\pgfpathlineto{\pgfqpoint{2.987922in}{0.620183in}}%
\pgfpathlineto{\pgfqpoint{2.988478in}{0.604512in}}%
\pgfpathlineto{\pgfqpoint{2.989034in}{0.619388in}}%
\pgfpathlineto{\pgfqpoint{2.989590in}{0.609920in}}%
\pgfpathlineto{\pgfqpoint{2.990146in}{0.613657in}}%
\pgfpathlineto{\pgfqpoint{2.990701in}{0.615478in}}%
\pgfpathlineto{\pgfqpoint{2.991257in}{0.613231in}}%
\pgfpathlineto{\pgfqpoint{2.991813in}{0.627808in}}%
\pgfpathlineto{\pgfqpoint{2.992925in}{0.604150in}}%
\pgfpathlineto{\pgfqpoint{2.993481in}{0.613653in}}%
\pgfpathlineto{\pgfqpoint{2.994037in}{0.610435in}}%
\pgfpathlineto{\pgfqpoint{2.994593in}{0.603999in}}%
\pgfpathlineto{\pgfqpoint{2.995148in}{0.617408in}}%
\pgfpathlineto{\pgfqpoint{2.995704in}{0.611427in}}%
\pgfpathlineto{\pgfqpoint{2.996260in}{0.609894in}}%
\pgfpathlineto{\pgfqpoint{2.996816in}{0.617425in}}%
\pgfpathlineto{\pgfqpoint{2.997928in}{0.605846in}}%
\pgfpathlineto{\pgfqpoint{2.998484in}{0.619901in}}%
\pgfpathlineto{\pgfqpoint{2.999039in}{0.618209in}}%
\pgfpathlineto{\pgfqpoint{2.999595in}{0.618622in}}%
\pgfpathlineto{\pgfqpoint{3.000151in}{0.617029in}}%
\pgfpathlineto{\pgfqpoint{3.000707in}{0.605887in}}%
\pgfpathlineto{\pgfqpoint{3.001263in}{0.612401in}}%
\pgfpathlineto{\pgfqpoint{3.003486in}{0.604357in}}%
\pgfpathlineto{\pgfqpoint{3.004042in}{0.618735in}}%
\pgfpathlineto{\pgfqpoint{3.004598in}{0.610814in}}%
\pgfpathlineto{\pgfqpoint{3.005154in}{0.616134in}}%
\pgfpathlineto{\pgfqpoint{3.005710in}{0.611455in}}%
\pgfpathlineto{\pgfqpoint{3.006266in}{0.605316in}}%
\pgfpathlineto{\pgfqpoint{3.006821in}{0.619121in}}%
\pgfpathlineto{\pgfqpoint{3.007377in}{0.617446in}}%
\pgfpathlineto{\pgfqpoint{3.007933in}{0.614942in}}%
\pgfpathlineto{\pgfqpoint{3.010712in}{0.642661in}}%
\pgfpathlineto{\pgfqpoint{3.011268in}{0.639171in}}%
\pgfpathlineto{\pgfqpoint{3.011824in}{0.663034in}}%
\pgfpathlineto{\pgfqpoint{3.012380in}{0.644949in}}%
\pgfpathlineto{\pgfqpoint{3.015159in}{0.615059in}}%
\pgfpathlineto{\pgfqpoint{3.015715in}{0.617493in}}%
\pgfpathlineto{\pgfqpoint{3.016271in}{0.652840in}}%
\pgfpathlineto{\pgfqpoint{3.016827in}{0.649612in}}%
\pgfpathlineto{\pgfqpoint{3.017939in}{0.613276in}}%
\pgfpathlineto{\pgfqpoint{3.018495in}{0.623816in}}%
\pgfpathlineto{\pgfqpoint{3.019050in}{0.629274in}}%
\pgfpathlineto{\pgfqpoint{3.019606in}{0.658270in}}%
\pgfpathlineto{\pgfqpoint{3.020718in}{0.609838in}}%
\pgfpathlineto{\pgfqpoint{3.021830in}{0.642814in}}%
\pgfpathlineto{\pgfqpoint{3.022386in}{0.612955in}}%
\pgfpathlineto{\pgfqpoint{3.022941in}{0.654970in}}%
\pgfpathlineto{\pgfqpoint{3.023497in}{0.633016in}}%
\pgfpathlineto{\pgfqpoint{3.024053in}{0.618739in}}%
\pgfpathlineto{\pgfqpoint{3.025165in}{0.667399in}}%
\pgfpathlineto{\pgfqpoint{3.025721in}{0.655810in}}%
\pgfpathlineto{\pgfqpoint{3.026277in}{0.630342in}}%
\pgfpathlineto{\pgfqpoint{3.026832in}{0.633582in}}%
\pgfpathlineto{\pgfqpoint{3.028500in}{0.693659in}}%
\pgfpathlineto{\pgfqpoint{3.029612in}{0.647883in}}%
\pgfpathlineto{\pgfqpoint{3.030168in}{0.610116in}}%
\pgfpathlineto{\pgfqpoint{3.030724in}{0.617529in}}%
\pgfpathlineto{\pgfqpoint{3.031279in}{0.623869in}}%
\pgfpathlineto{\pgfqpoint{3.032391in}{0.704674in}}%
\pgfpathlineto{\pgfqpoint{3.034059in}{0.614057in}}%
\pgfpathlineto{\pgfqpoint{3.034615in}{0.679663in}}%
\pgfpathlineto{\pgfqpoint{3.035170in}{0.649056in}}%
\pgfpathlineto{\pgfqpoint{3.035726in}{0.658334in}}%
\pgfpathlineto{\pgfqpoint{3.037394in}{0.604420in}}%
\pgfpathlineto{\pgfqpoint{3.037950in}{0.627465in}}%
\pgfpathlineto{\pgfqpoint{3.038506in}{0.625753in}}%
\pgfpathlineto{\pgfqpoint{3.039061in}{0.626559in}}%
\pgfpathlineto{\pgfqpoint{3.040173in}{0.606827in}}%
\pgfpathlineto{\pgfqpoint{3.040729in}{0.621274in}}%
\pgfpathlineto{\pgfqpoint{3.041285in}{0.614810in}}%
\pgfpathlineto{\pgfqpoint{3.041841in}{0.619785in}}%
\pgfpathlineto{\pgfqpoint{3.042397in}{0.601330in}}%
\pgfpathlineto{\pgfqpoint{3.042952in}{0.621361in}}%
\pgfpathlineto{\pgfqpoint{3.043508in}{0.614182in}}%
\pgfpathlineto{\pgfqpoint{3.045176in}{0.611855in}}%
\pgfpathlineto{\pgfqpoint{3.046288in}{0.615168in}}%
\pgfpathlineto{\pgfqpoint{3.047399in}{0.605370in}}%
\pgfpathlineto{\pgfqpoint{3.047955in}{0.610183in}}%
\pgfpathlineto{\pgfqpoint{3.048511in}{0.606938in}}%
\pgfpathlineto{\pgfqpoint{3.049067in}{0.605105in}}%
\pgfpathlineto{\pgfqpoint{3.050735in}{0.609170in}}%
\pgfpathlineto{\pgfqpoint{3.051290in}{0.604529in}}%
\pgfpathlineto{\pgfqpoint{3.051846in}{0.605716in}}%
\pgfpathlineto{\pgfqpoint{3.052958in}{0.614788in}}%
\pgfpathlineto{\pgfqpoint{3.054626in}{0.608181in}}%
\pgfpathlineto{\pgfqpoint{3.055181in}{0.613302in}}%
\pgfpathlineto{\pgfqpoint{3.056293in}{0.603652in}}%
\pgfpathlineto{\pgfqpoint{3.056849in}{0.613991in}}%
\pgfpathlineto{\pgfqpoint{3.057405in}{0.611541in}}%
\pgfpathlineto{\pgfqpoint{3.057961in}{0.604076in}}%
\pgfpathlineto{\pgfqpoint{3.058517in}{0.610968in}}%
\pgfpathlineto{\pgfqpoint{3.059072in}{0.604720in}}%
\pgfpathlineto{\pgfqpoint{3.060184in}{0.617948in}}%
\pgfpathlineto{\pgfqpoint{3.060740in}{0.603182in}}%
\pgfpathlineto{\pgfqpoint{3.061296in}{0.622016in}}%
\pgfpathlineto{\pgfqpoint{3.061852in}{0.613631in}}%
\pgfpathlineto{\pgfqpoint{3.062408in}{0.614706in}}%
\pgfpathlineto{\pgfqpoint{3.062963in}{0.613720in}}%
\pgfpathlineto{\pgfqpoint{3.063519in}{0.607182in}}%
\pgfpathlineto{\pgfqpoint{3.064075in}{0.618958in}}%
\pgfpathlineto{\pgfqpoint{3.064631in}{0.600948in}}%
\pgfpathlineto{\pgfqpoint{3.065187in}{0.622508in}}%
\pgfpathlineto{\pgfqpoint{3.065743in}{0.610968in}}%
\pgfpathlineto{\pgfqpoint{3.066854in}{0.616968in}}%
\pgfpathlineto{\pgfqpoint{3.067410in}{0.610150in}}%
\pgfpathlineto{\pgfqpoint{3.067966in}{0.614669in}}%
\pgfpathlineto{\pgfqpoint{3.068522in}{0.625893in}}%
\pgfpathlineto{\pgfqpoint{3.069078in}{0.613367in}}%
\pgfpathlineto{\pgfqpoint{3.071301in}{0.647039in}}%
\pgfpathlineto{\pgfqpoint{3.071857in}{0.644481in}}%
\pgfpathlineto{\pgfqpoint{3.072969in}{0.659115in}}%
\pgfpathlineto{\pgfqpoint{3.074637in}{0.617366in}}%
\pgfpathlineto{\pgfqpoint{3.075192in}{0.627689in}}%
\pgfpathlineto{\pgfqpoint{3.075748in}{0.660714in}}%
\pgfpathlineto{\pgfqpoint{3.076304in}{0.644772in}}%
\pgfpathlineto{\pgfqpoint{3.077416in}{0.622021in}}%
\pgfpathlineto{\pgfqpoint{3.077972in}{0.625769in}}%
\pgfpathlineto{\pgfqpoint{3.078528in}{0.626281in}}%
\pgfpathlineto{\pgfqpoint{3.079083in}{0.660467in}}%
\pgfpathlineto{\pgfqpoint{3.079639in}{0.619749in}}%
\pgfpathlineto{\pgfqpoint{3.080195in}{0.621333in}}%
\pgfpathlineto{\pgfqpoint{3.081307in}{0.631010in}}%
\pgfpathlineto{\pgfqpoint{3.081863in}{0.617637in}}%
\pgfpathlineto{\pgfqpoint{3.082419in}{0.677888in}}%
\pgfpathlineto{\pgfqpoint{3.082974in}{0.657192in}}%
\pgfpathlineto{\pgfqpoint{3.083530in}{0.620395in}}%
\pgfpathlineto{\pgfqpoint{3.085198in}{0.707782in}}%
\pgfpathlineto{\pgfqpoint{3.086866in}{0.628249in}}%
\pgfpathlineto{\pgfqpoint{3.087421in}{0.683522in}}%
\pgfpathlineto{\pgfqpoint{3.087977in}{0.653327in}}%
\pgfpathlineto{\pgfqpoint{3.088533in}{0.620600in}}%
\pgfpathlineto{\pgfqpoint{3.089645in}{0.734405in}}%
\pgfpathlineto{\pgfqpoint{3.090201in}{0.637275in}}%
\pgfpathlineto{\pgfqpoint{3.090757in}{0.638456in}}%
\pgfpathlineto{\pgfqpoint{3.091312in}{0.640791in}}%
\pgfpathlineto{\pgfqpoint{3.091868in}{0.651224in}}%
\pgfpathlineto{\pgfqpoint{3.092424in}{0.622524in}}%
\pgfpathlineto{\pgfqpoint{3.092980in}{0.652624in}}%
\pgfpathlineto{\pgfqpoint{3.094092in}{0.610334in}}%
\pgfpathlineto{\pgfqpoint{3.094648in}{0.610844in}}%
\pgfpathlineto{\pgfqpoint{3.095203in}{0.622075in}}%
\pgfpathlineto{\pgfqpoint{3.095759in}{0.616375in}}%
\pgfpathlineto{\pgfqpoint{3.096315in}{0.610239in}}%
\pgfpathlineto{\pgfqpoint{3.097427in}{0.636413in}}%
\pgfpathlineto{\pgfqpoint{3.097983in}{0.606204in}}%
\pgfpathlineto{\pgfqpoint{3.098539in}{0.610464in}}%
\pgfpathlineto{\pgfqpoint{3.099094in}{0.617256in}}%
\pgfpathlineto{\pgfqpoint{3.099650in}{0.602235in}}%
\pgfpathlineto{\pgfqpoint{3.100206in}{0.623310in}}%
\pgfpathlineto{\pgfqpoint{3.100762in}{0.611094in}}%
\pgfpathlineto{\pgfqpoint{3.101318in}{0.622475in}}%
\pgfpathlineto{\pgfqpoint{3.101874in}{0.607251in}}%
\pgfpathlineto{\pgfqpoint{3.102430in}{0.608262in}}%
\pgfpathlineto{\pgfqpoint{3.102985in}{0.625145in}}%
\pgfpathlineto{\pgfqpoint{3.103541in}{0.609991in}}%
\pgfpathlineto{\pgfqpoint{3.104097in}{0.609797in}}%
\pgfpathlineto{\pgfqpoint{3.105209in}{0.603686in}}%
\pgfpathlineto{\pgfqpoint{3.106321in}{0.625154in}}%
\pgfpathlineto{\pgfqpoint{3.106877in}{0.617927in}}%
\pgfpathlineto{\pgfqpoint{3.107432in}{0.612094in}}%
\pgfpathlineto{\pgfqpoint{3.107988in}{0.618524in}}%
\pgfpathlineto{\pgfqpoint{3.108544in}{0.608802in}}%
\pgfpathlineto{\pgfqpoint{3.109100in}{0.622010in}}%
\pgfpathlineto{\pgfqpoint{3.109656in}{0.615427in}}%
\pgfpathlineto{\pgfqpoint{3.110212in}{0.606836in}}%
\pgfpathlineto{\pgfqpoint{3.110768in}{0.622595in}}%
\pgfpathlineto{\pgfqpoint{3.111323in}{0.618523in}}%
\pgfpathlineto{\pgfqpoint{3.111879in}{0.621903in}}%
\pgfpathlineto{\pgfqpoint{3.112435in}{0.621795in}}%
\pgfpathlineto{\pgfqpoint{3.112991in}{0.606287in}}%
\pgfpathlineto{\pgfqpoint{3.113547in}{0.609830in}}%
\pgfpathlineto{\pgfqpoint{3.114103in}{0.610386in}}%
\pgfpathlineto{\pgfqpoint{3.115214in}{0.614522in}}%
\pgfpathlineto{\pgfqpoint{3.115770in}{0.625272in}}%
\pgfpathlineto{\pgfqpoint{3.116326in}{0.605222in}}%
\pgfpathlineto{\pgfqpoint{3.116882in}{0.616724in}}%
\pgfpathlineto{\pgfqpoint{3.117994in}{0.605495in}}%
\pgfpathlineto{\pgfqpoint{3.119105in}{0.618655in}}%
\pgfpathlineto{\pgfqpoint{3.120773in}{0.605033in}}%
\pgfpathlineto{\pgfqpoint{3.121329in}{0.614933in}}%
\pgfpathlineto{\pgfqpoint{3.121885in}{0.602797in}}%
\pgfpathlineto{\pgfqpoint{3.122441in}{0.607638in}}%
\pgfpathlineto{\pgfqpoint{3.122996in}{0.607410in}}%
\pgfpathlineto{\pgfqpoint{3.123552in}{0.616528in}}%
\pgfpathlineto{\pgfqpoint{3.124108in}{0.607544in}}%
\pgfpathlineto{\pgfqpoint{3.124664in}{0.600639in}}%
\pgfpathlineto{\pgfqpoint{3.125220in}{0.616319in}}%
\pgfpathlineto{\pgfqpoint{3.125776in}{0.614429in}}%
\pgfpathlineto{\pgfqpoint{3.126888in}{0.605802in}}%
\pgfpathlineto{\pgfqpoint{3.127999in}{0.616542in}}%
\pgfpathlineto{\pgfqpoint{3.128555in}{0.606859in}}%
\pgfpathlineto{\pgfqpoint{3.131334in}{0.670559in}}%
\pgfpathlineto{\pgfqpoint{3.132446in}{0.661381in}}%
\pgfpathlineto{\pgfqpoint{3.134670in}{0.620065in}}%
\pgfpathlineto{\pgfqpoint{3.136337in}{0.655865in}}%
\pgfpathlineto{\pgfqpoint{3.137449in}{0.618399in}}%
\pgfpathlineto{\pgfqpoint{3.138005in}{0.627189in}}%
\pgfpathlineto{\pgfqpoint{3.139672in}{0.665380in}}%
\pgfpathlineto{\pgfqpoint{3.140228in}{0.669692in}}%
\pgfpathlineto{\pgfqpoint{3.140784in}{0.662393in}}%
\pgfpathlineto{\pgfqpoint{3.141340in}{0.666266in}}%
\pgfpathlineto{\pgfqpoint{3.141896in}{0.673589in}}%
\pgfpathlineto{\pgfqpoint{3.142452in}{0.704175in}}%
\pgfpathlineto{\pgfqpoint{3.143007in}{0.661485in}}%
\pgfpathlineto{\pgfqpoint{3.144675in}{0.765597in}}%
\pgfpathlineto{\pgfqpoint{3.145787in}{0.607957in}}%
\pgfpathlineto{\pgfqpoint{3.147454in}{0.719045in}}%
\pgfpathlineto{\pgfqpoint{3.149122in}{0.608683in}}%
\pgfpathlineto{\pgfqpoint{3.149678in}{0.654705in}}%
\pgfpathlineto{\pgfqpoint{3.150234in}{0.639146in}}%
\pgfpathlineto{\pgfqpoint{3.150790in}{0.639085in}}%
\pgfpathlineto{\pgfqpoint{3.151901in}{0.613323in}}%
\pgfpathlineto{\pgfqpoint{3.152457in}{0.621152in}}%
\pgfpathlineto{\pgfqpoint{3.153013in}{0.620258in}}%
\pgfpathlineto{\pgfqpoint{3.154125in}{0.607838in}}%
\pgfpathlineto{\pgfqpoint{3.154681in}{0.624916in}}%
\pgfpathlineto{\pgfqpoint{3.155236in}{0.616117in}}%
\pgfpathlineto{\pgfqpoint{3.155792in}{0.612121in}}%
\pgfpathlineto{\pgfqpoint{3.156348in}{0.626212in}}%
\pgfpathlineto{\pgfqpoint{3.156904in}{0.618449in}}%
\pgfpathlineto{\pgfqpoint{3.157460in}{0.615427in}}%
\pgfpathlineto{\pgfqpoint{3.158016in}{0.635239in}}%
\pgfpathlineto{\pgfqpoint{3.158572in}{0.618341in}}%
\pgfpathlineto{\pgfqpoint{3.159683in}{0.602347in}}%
\pgfpathlineto{\pgfqpoint{3.161351in}{0.615686in}}%
\pgfpathlineto{\pgfqpoint{3.162463in}{0.606448in}}%
\pgfpathlineto{\pgfqpoint{3.164130in}{0.631671in}}%
\pgfpathlineto{\pgfqpoint{3.165798in}{0.601630in}}%
\pgfpathlineto{\pgfqpoint{3.166354in}{0.626403in}}%
\pgfpathlineto{\pgfqpoint{3.166910in}{0.611755in}}%
\pgfpathlineto{\pgfqpoint{3.167465in}{0.612975in}}%
\pgfpathlineto{\pgfqpoint{3.168577in}{0.602350in}}%
\pgfpathlineto{\pgfqpoint{3.169689in}{0.624196in}}%
\pgfpathlineto{\pgfqpoint{3.170245in}{0.603742in}}%
\pgfpathlineto{\pgfqpoint{3.170801in}{0.618808in}}%
\pgfpathlineto{\pgfqpoint{3.171912in}{0.605427in}}%
\pgfpathlineto{\pgfqpoint{3.172468in}{0.628883in}}%
\pgfpathlineto{\pgfqpoint{3.173024in}{0.604602in}}%
\pgfpathlineto{\pgfqpoint{3.173580in}{0.610300in}}%
\pgfpathlineto{\pgfqpoint{3.174136in}{0.608982in}}%
\pgfpathlineto{\pgfqpoint{3.176359in}{0.624558in}}%
\pgfpathlineto{\pgfqpoint{3.176915in}{0.613111in}}%
\pgfpathlineto{\pgfqpoint{3.177471in}{0.614270in}}%
\pgfpathlineto{\pgfqpoint{3.178027in}{0.616543in}}%
\pgfpathlineto{\pgfqpoint{3.178583in}{0.604558in}}%
\pgfpathlineto{\pgfqpoint{3.179138in}{0.618615in}}%
\pgfpathlineto{\pgfqpoint{3.179694in}{0.616201in}}%
\pgfpathlineto{\pgfqpoint{3.180250in}{0.612060in}}%
\pgfpathlineto{\pgfqpoint{3.180806in}{0.622137in}}%
\pgfpathlineto{\pgfqpoint{3.181362in}{0.612158in}}%
\pgfpathlineto{\pgfqpoint{3.181918in}{0.603861in}}%
\pgfpathlineto{\pgfqpoint{3.183585in}{0.618489in}}%
\pgfpathlineto{\pgfqpoint{3.184141in}{0.601381in}}%
\pgfpathlineto{\pgfqpoint{3.184697in}{0.617972in}}%
\pgfpathlineto{\pgfqpoint{3.186365in}{0.606456in}}%
\pgfpathlineto{\pgfqpoint{3.186921in}{0.623760in}}%
\pgfpathlineto{\pgfqpoint{3.187476in}{0.611689in}}%
\pgfpathlineto{\pgfqpoint{3.188032in}{0.606503in}}%
\pgfpathlineto{\pgfqpoint{3.190256in}{0.647428in}}%
\pgfpathlineto{\pgfqpoint{3.190812in}{0.641318in}}%
\pgfpathlineto{\pgfqpoint{3.192479in}{0.726604in}}%
\pgfpathlineto{\pgfqpoint{3.194703in}{0.612605in}}%
\pgfpathlineto{\pgfqpoint{3.195814in}{0.694525in}}%
\pgfpathlineto{\pgfqpoint{3.196926in}{0.634768in}}%
\pgfpathlineto{\pgfqpoint{3.198038in}{0.690675in}}%
\pgfpathlineto{\pgfqpoint{3.198594in}{0.660142in}}%
\pgfpathlineto{\pgfqpoint{3.200261in}{0.782380in}}%
\pgfpathlineto{\pgfqpoint{3.201373in}{0.633950in}}%
\pgfpathlineto{\pgfqpoint{3.202485in}{0.759179in}}%
\pgfpathlineto{\pgfqpoint{3.204152in}{0.652944in}}%
\pgfpathlineto{\pgfqpoint{3.204708in}{0.796002in}}%
\pgfpathlineto{\pgfqpoint{3.205264in}{0.672738in}}%
\pgfpathlineto{\pgfqpoint{3.205820in}{0.673537in}}%
\pgfpathlineto{\pgfqpoint{3.207487in}{0.607740in}}%
\pgfpathlineto{\pgfqpoint{3.208043in}{0.661655in}}%
\pgfpathlineto{\pgfqpoint{3.208599in}{0.624984in}}%
\pgfpathlineto{\pgfqpoint{3.209155in}{0.636003in}}%
\pgfpathlineto{\pgfqpoint{3.210823in}{0.607597in}}%
\pgfpathlineto{\pgfqpoint{3.211934in}{0.623532in}}%
\pgfpathlineto{\pgfqpoint{3.212490in}{0.607806in}}%
\pgfpathlineto{\pgfqpoint{3.213046in}{0.619255in}}%
\pgfpathlineto{\pgfqpoint{3.213602in}{0.609207in}}%
\pgfpathlineto{\pgfqpoint{3.214714in}{0.641687in}}%
\pgfpathlineto{\pgfqpoint{3.215269in}{0.633996in}}%
\pgfpathlineto{\pgfqpoint{3.215825in}{0.636829in}}%
\pgfpathlineto{\pgfqpoint{3.217493in}{0.618021in}}%
\pgfpathlineto{\pgfqpoint{3.218049in}{0.629005in}}%
\pgfpathlineto{\pgfqpoint{3.219716in}{0.605582in}}%
\pgfpathlineto{\pgfqpoint{3.221384in}{0.613480in}}%
\pgfpathlineto{\pgfqpoint{3.221940in}{0.609008in}}%
\pgfpathlineto{\pgfqpoint{3.222496in}{0.619532in}}%
\pgfpathlineto{\pgfqpoint{3.223052in}{0.615266in}}%
\pgfpathlineto{\pgfqpoint{3.225275in}{0.604953in}}%
\pgfpathlineto{\pgfqpoint{3.225831in}{0.622621in}}%
\pgfpathlineto{\pgfqpoint{3.226387in}{0.616483in}}%
\pgfpathlineto{\pgfqpoint{3.226943in}{0.616949in}}%
\pgfpathlineto{\pgfqpoint{3.227498in}{0.621122in}}%
\pgfpathlineto{\pgfqpoint{3.228054in}{0.608693in}}%
\pgfpathlineto{\pgfqpoint{3.228610in}{0.615571in}}%
\pgfpathlineto{\pgfqpoint{3.230278in}{0.628788in}}%
\pgfpathlineto{\pgfqpoint{3.230834in}{0.637563in}}%
\pgfpathlineto{\pgfqpoint{3.231389in}{0.609949in}}%
\pgfpathlineto{\pgfqpoint{3.231945in}{0.612646in}}%
\pgfpathlineto{\pgfqpoint{3.232501in}{0.610124in}}%
\pgfpathlineto{\pgfqpoint{3.233057in}{0.628859in}}%
\pgfpathlineto{\pgfqpoint{3.233613in}{0.606885in}}%
\pgfpathlineto{\pgfqpoint{3.234169in}{0.610533in}}%
\pgfpathlineto{\pgfqpoint{3.234725in}{0.615736in}}%
\pgfpathlineto{\pgfqpoint{3.235280in}{0.611039in}}%
\pgfpathlineto{\pgfqpoint{3.235836in}{0.612403in}}%
\pgfpathlineto{\pgfqpoint{3.236392in}{0.606064in}}%
\pgfpathlineto{\pgfqpoint{3.236948in}{0.618109in}}%
\pgfpathlineto{\pgfqpoint{3.237504in}{0.608144in}}%
\pgfpathlineto{\pgfqpoint{3.238060in}{0.609181in}}%
\pgfpathlineto{\pgfqpoint{3.238616in}{0.615532in}}%
\pgfpathlineto{\pgfqpoint{3.239172in}{0.605511in}}%
\pgfpathlineto{\pgfqpoint{3.239727in}{0.633409in}}%
\pgfpathlineto{\pgfqpoint{3.240283in}{0.618139in}}%
\pgfpathlineto{\pgfqpoint{3.240839in}{0.605324in}}%
\pgfpathlineto{\pgfqpoint{3.241395in}{0.620722in}}%
\pgfpathlineto{\pgfqpoint{3.241951in}{0.607877in}}%
\pgfpathlineto{\pgfqpoint{3.242507in}{0.603637in}}%
\pgfpathlineto{\pgfqpoint{3.243063in}{0.605098in}}%
\pgfpathlineto{\pgfqpoint{3.243618in}{0.621903in}}%
\pgfpathlineto{\pgfqpoint{3.244174in}{0.614617in}}%
\pgfpathlineto{\pgfqpoint{3.244730in}{0.618533in}}%
\pgfpathlineto{\pgfqpoint{3.245286in}{0.606216in}}%
\pgfpathlineto{\pgfqpoint{3.246954in}{0.624921in}}%
\pgfpathlineto{\pgfqpoint{3.247509in}{0.628809in}}%
\pgfpathlineto{\pgfqpoint{3.248065in}{0.622261in}}%
\pgfpathlineto{\pgfqpoint{3.249733in}{0.638816in}}%
\pgfpathlineto{\pgfqpoint{3.250289in}{0.633659in}}%
\pgfpathlineto{\pgfqpoint{3.252512in}{0.728176in}}%
\pgfpathlineto{\pgfqpoint{3.254180in}{0.636957in}}%
\pgfpathlineto{\pgfqpoint{3.254736in}{0.648303in}}%
\pgfpathlineto{\pgfqpoint{3.256403in}{0.697106in}}%
\pgfpathlineto{\pgfqpoint{3.256959in}{0.675998in}}%
\pgfpathlineto{\pgfqpoint{3.257515in}{0.734056in}}%
\pgfpathlineto{\pgfqpoint{3.258071in}{0.627019in}}%
\pgfpathlineto{\pgfqpoint{3.258627in}{0.691151in}}%
\pgfpathlineto{\pgfqpoint{3.259183in}{0.715644in}}%
\pgfpathlineto{\pgfqpoint{3.259738in}{0.873391in}}%
\pgfpathlineto{\pgfqpoint{3.261406in}{0.649867in}}%
\pgfpathlineto{\pgfqpoint{3.261962in}{0.746334in}}%
\pgfpathlineto{\pgfqpoint{3.263629in}{0.613705in}}%
\pgfpathlineto{\pgfqpoint{3.264185in}{0.648544in}}%
\pgfpathlineto{\pgfqpoint{3.264741in}{0.603295in}}%
\pgfpathlineto{\pgfqpoint{3.265297in}{0.669552in}}%
\pgfpathlineto{\pgfqpoint{3.265853in}{0.618011in}}%
\pgfpathlineto{\pgfqpoint{3.266409in}{0.624024in}}%
\pgfpathlineto{\pgfqpoint{3.268076in}{0.609263in}}%
\pgfpathlineto{\pgfqpoint{3.268632in}{0.609139in}}%
\pgfpathlineto{\pgfqpoint{3.269188in}{0.611642in}}%
\pgfpathlineto{\pgfqpoint{3.269744in}{0.608683in}}%
\pgfpathlineto{\pgfqpoint{3.270300in}{0.636412in}}%
\pgfpathlineto{\pgfqpoint{3.270856in}{0.628413in}}%
\pgfpathlineto{\pgfqpoint{3.271411in}{0.603388in}}%
\pgfpathlineto{\pgfqpoint{3.271967in}{0.625909in}}%
\pgfpathlineto{\pgfqpoint{3.272523in}{0.635674in}}%
\pgfpathlineto{\pgfqpoint{3.274747in}{0.603122in}}%
\pgfpathlineto{\pgfqpoint{3.275303in}{0.624913in}}%
\pgfpathlineto{\pgfqpoint{3.275858in}{0.615526in}}%
\pgfpathlineto{\pgfqpoint{3.277526in}{0.607933in}}%
\pgfpathlineto{\pgfqpoint{3.278082in}{0.621721in}}%
\pgfpathlineto{\pgfqpoint{3.278638in}{0.614213in}}%
\pgfpathlineto{\pgfqpoint{3.279749in}{0.604459in}}%
\pgfpathlineto{\pgfqpoint{3.280305in}{0.630938in}}%
\pgfpathlineto{\pgfqpoint{3.280861in}{0.611775in}}%
\pgfpathlineto{\pgfqpoint{3.281417in}{0.627450in}}%
\pgfpathlineto{\pgfqpoint{3.281973in}{0.602543in}}%
\pgfpathlineto{\pgfqpoint{3.282529in}{0.614960in}}%
\pgfpathlineto{\pgfqpoint{3.284196in}{0.651304in}}%
\pgfpathlineto{\pgfqpoint{3.284752in}{0.611940in}}%
\pgfpathlineto{\pgfqpoint{3.285308in}{0.649489in}}%
\pgfpathlineto{\pgfqpoint{3.287531in}{0.616557in}}%
\pgfpathlineto{\pgfqpoint{3.289199in}{0.642998in}}%
\pgfpathlineto{\pgfqpoint{3.289755in}{0.628456in}}%
\pgfpathlineto{\pgfqpoint{3.290311in}{0.672023in}}%
\pgfpathlineto{\pgfqpoint{3.290867in}{0.633067in}}%
\pgfpathlineto{\pgfqpoint{3.291422in}{0.657838in}}%
\pgfpathlineto{\pgfqpoint{3.291978in}{0.635534in}}%
\pgfpathlineto{\pgfqpoint{3.292534in}{0.632919in}}%
\pgfpathlineto{\pgfqpoint{3.293646in}{0.605335in}}%
\pgfpathlineto{\pgfqpoint{3.295314in}{0.638417in}}%
\pgfpathlineto{\pgfqpoint{3.295869in}{0.611210in}}%
\pgfpathlineto{\pgfqpoint{3.296425in}{0.652865in}}%
\pgfpathlineto{\pgfqpoint{3.296981in}{0.620957in}}%
\pgfpathlineto{\pgfqpoint{3.298649in}{0.636720in}}%
\pgfpathlineto{\pgfqpoint{3.299205in}{0.613215in}}%
\pgfpathlineto{\pgfqpoint{3.299760in}{0.620920in}}%
\pgfpathlineto{\pgfqpoint{3.300316in}{0.614328in}}%
\pgfpathlineto{\pgfqpoint{3.301984in}{0.631900in}}%
\pgfpathlineto{\pgfqpoint{3.302540in}{0.617105in}}%
\pgfpathlineto{\pgfqpoint{3.303096in}{0.643104in}}%
\pgfpathlineto{\pgfqpoint{3.303651in}{0.626302in}}%
\pgfpathlineto{\pgfqpoint{3.304207in}{0.618045in}}%
\pgfpathlineto{\pgfqpoint{3.304763in}{0.632552in}}%
\pgfpathlineto{\pgfqpoint{3.305319in}{0.612408in}}%
\pgfpathlineto{\pgfqpoint{3.305875in}{0.628778in}}%
\pgfpathlineto{\pgfqpoint{3.306431in}{0.635378in}}%
\pgfpathlineto{\pgfqpoint{3.306987in}{0.630923in}}%
\pgfpathlineto{\pgfqpoint{3.308654in}{0.608650in}}%
\pgfpathlineto{\pgfqpoint{3.309210in}{0.630628in}}%
\pgfpathlineto{\pgfqpoint{3.309766in}{0.629298in}}%
\pgfpathlineto{\pgfqpoint{3.310322in}{0.630636in}}%
\pgfpathlineto{\pgfqpoint{3.311433in}{0.654424in}}%
\pgfpathlineto{\pgfqpoint{3.311989in}{0.662672in}}%
\pgfpathlineto{\pgfqpoint{3.312545in}{0.650044in}}%
\pgfpathlineto{\pgfqpoint{3.313101in}{0.678443in}}%
\pgfpathlineto{\pgfqpoint{3.313657in}{0.665937in}}%
\pgfpathlineto{\pgfqpoint{3.314769in}{0.658394in}}%
\pgfpathlineto{\pgfqpoint{3.315880in}{0.688945in}}%
\pgfpathlineto{\pgfqpoint{3.316436in}{0.651340in}}%
\pgfpathlineto{\pgfqpoint{3.316992in}{0.810080in}}%
\pgfpathlineto{\pgfqpoint{3.317548in}{0.660382in}}%
\pgfpathlineto{\pgfqpoint{3.318104in}{0.625576in}}%
\pgfpathlineto{\pgfqpoint{3.318660in}{0.656190in}}%
\pgfpathlineto{\pgfqpoint{3.319216in}{0.635987in}}%
\pgfpathlineto{\pgfqpoint{3.319771in}{0.639565in}}%
\pgfpathlineto{\pgfqpoint{3.320883in}{0.664679in}}%
\pgfpathlineto{\pgfqpoint{3.321439in}{0.622964in}}%
\pgfpathlineto{\pgfqpoint{3.321995in}{0.660302in}}%
\pgfpathlineto{\pgfqpoint{3.322551in}{0.636451in}}%
\pgfpathlineto{\pgfqpoint{3.323107in}{0.650116in}}%
\pgfpathlineto{\pgfqpoint{3.323662in}{0.655218in}}%
\pgfpathlineto{\pgfqpoint{3.324218in}{0.652368in}}%
\pgfpathlineto{\pgfqpoint{3.324774in}{0.654611in}}%
\pgfpathlineto{\pgfqpoint{3.325330in}{0.645546in}}%
\pgfpathlineto{\pgfqpoint{3.326998in}{0.680371in}}%
\pgfpathlineto{\pgfqpoint{3.328109in}{0.622157in}}%
\pgfpathlineto{\pgfqpoint{3.328665in}{0.629530in}}%
\pgfpathlineto{\pgfqpoint{3.329221in}{0.606031in}}%
\pgfpathlineto{\pgfqpoint{3.329777in}{0.622181in}}%
\pgfpathlineto{\pgfqpoint{3.331444in}{0.678615in}}%
\pgfpathlineto{\pgfqpoint{3.332556in}{0.640254in}}%
\pgfpathlineto{\pgfqpoint{3.333112in}{0.659027in}}%
\pgfpathlineto{\pgfqpoint{3.334224in}{0.634385in}}%
\pgfpathlineto{\pgfqpoint{3.334780in}{0.635988in}}%
\pgfpathlineto{\pgfqpoint{3.335336in}{0.642336in}}%
\pgfpathlineto{\pgfqpoint{3.335891in}{0.615018in}}%
\pgfpathlineto{\pgfqpoint{3.336447in}{0.663668in}}%
\pgfpathlineto{\pgfqpoint{3.337003in}{0.628631in}}%
\pgfpathlineto{\pgfqpoint{3.337559in}{0.625753in}}%
\pgfpathlineto{\pgfqpoint{3.338115in}{0.604507in}}%
\pgfpathlineto{\pgfqpoint{3.339782in}{0.640573in}}%
\pgfpathlineto{\pgfqpoint{3.341450in}{0.611730in}}%
\pgfpathlineto{\pgfqpoint{3.342562in}{0.667640in}}%
\pgfpathlineto{\pgfqpoint{3.343673in}{0.608115in}}%
\pgfpathlineto{\pgfqpoint{3.345341in}{0.656796in}}%
\pgfpathlineto{\pgfqpoint{3.346453in}{0.630515in}}%
\pgfpathlineto{\pgfqpoint{3.347564in}{0.670204in}}%
\pgfpathlineto{\pgfqpoint{3.348120in}{0.654415in}}%
\pgfpathlineto{\pgfqpoint{3.348676in}{0.610603in}}%
\pgfpathlineto{\pgfqpoint{3.349788in}{0.668255in}}%
\pgfpathlineto{\pgfqpoint{3.350344in}{0.635006in}}%
\pgfpathlineto{\pgfqpoint{3.350900in}{0.636932in}}%
\pgfpathlineto{\pgfqpoint{3.351456in}{0.638834in}}%
\pgfpathlineto{\pgfqpoint{3.352011in}{0.660032in}}%
\pgfpathlineto{\pgfqpoint{3.352567in}{0.652373in}}%
\pgfpathlineto{\pgfqpoint{3.353123in}{0.755908in}}%
\pgfpathlineto{\pgfqpoint{3.353679in}{0.628820in}}%
\pgfpathlineto{\pgfqpoint{3.354235in}{0.709648in}}%
\pgfpathlineto{\pgfqpoint{3.355347in}{0.770010in}}%
\pgfpathlineto{\pgfqpoint{3.357014in}{0.874959in}}%
\pgfpathlineto{\pgfqpoint{3.358682in}{0.718709in}}%
\pgfpathlineto{\pgfqpoint{3.359238in}{0.786679in}}%
\pgfpathlineto{\pgfqpoint{3.359793in}{0.775657in}}%
\pgfpathlineto{\pgfqpoint{3.360349in}{0.748441in}}%
\pgfpathlineto{\pgfqpoint{3.360905in}{0.874126in}}%
\pgfpathlineto{\pgfqpoint{3.361461in}{0.699739in}}%
\pgfpathlineto{\pgfqpoint{3.362017in}{0.884417in}}%
\pgfpathlineto{\pgfqpoint{3.362573in}{0.740823in}}%
\pgfpathlineto{\pgfqpoint{3.363129in}{0.863464in}}%
\pgfpathlineto{\pgfqpoint{3.363684in}{0.688675in}}%
\pgfpathlineto{\pgfqpoint{3.364240in}{0.788299in}}%
\pgfpathlineto{\pgfqpoint{3.365352in}{0.671623in}}%
\pgfpathlineto{\pgfqpoint{3.365908in}{0.766250in}}%
\pgfpathlineto{\pgfqpoint{3.366464in}{0.620203in}}%
\pgfpathlineto{\pgfqpoint{3.367020in}{0.819327in}}%
\pgfpathlineto{\pgfqpoint{3.367575in}{0.695116in}}%
\pgfpathlineto{\pgfqpoint{3.368131in}{0.698703in}}%
\pgfpathlineto{\pgfqpoint{3.368687in}{0.675702in}}%
\pgfpathlineto{\pgfqpoint{3.369243in}{0.698800in}}%
\pgfpathlineto{\pgfqpoint{3.369799in}{0.691438in}}%
\pgfpathlineto{\pgfqpoint{3.370355in}{0.622249in}}%
\pgfpathlineto{\pgfqpoint{3.370911in}{0.694597in}}%
\pgfpathlineto{\pgfqpoint{3.371467in}{0.636244in}}%
\pgfpathlineto{\pgfqpoint{3.372022in}{0.677061in}}%
\pgfpathlineto{\pgfqpoint{3.372578in}{0.804293in}}%
\pgfpathlineto{\pgfqpoint{3.373134in}{0.755403in}}%
\pgfpathlineto{\pgfqpoint{3.373690in}{0.761644in}}%
\pgfpathlineto{\pgfqpoint{3.374802in}{0.961308in}}%
\pgfpathlineto{\pgfqpoint{3.376469in}{0.619903in}}%
\pgfpathlineto{\pgfqpoint{3.378137in}{0.720313in}}%
\pgfpathlineto{\pgfqpoint{3.378693in}{0.608870in}}%
\pgfpathlineto{\pgfqpoint{3.379249in}{0.622476in}}%
\pgfpathlineto{\pgfqpoint{3.380360in}{0.669101in}}%
\pgfpathlineto{\pgfqpoint{3.380916in}{0.631179in}}%
\pgfpathlineto{\pgfqpoint{3.381472in}{0.658103in}}%
\pgfpathlineto{\pgfqpoint{3.382028in}{0.656367in}}%
\pgfpathlineto{\pgfqpoint{3.383140in}{0.690790in}}%
\pgfpathlineto{\pgfqpoint{3.383695in}{0.618972in}}%
\pgfpathlineto{\pgfqpoint{3.384251in}{0.667885in}}%
\pgfpathlineto{\pgfqpoint{3.384807in}{0.650178in}}%
\pgfpathlineto{\pgfqpoint{3.385363in}{0.674413in}}%
\pgfpathlineto{\pgfqpoint{3.385919in}{0.659378in}}%
\pgfpathlineto{\pgfqpoint{3.386475in}{0.621289in}}%
\pgfpathlineto{\pgfqpoint{3.387031in}{0.685213in}}%
\pgfpathlineto{\pgfqpoint{3.387586in}{0.620094in}}%
\pgfpathlineto{\pgfqpoint{3.388142in}{0.746430in}}%
\pgfpathlineto{\pgfqpoint{3.388698in}{0.737170in}}%
\pgfpathlineto{\pgfqpoint{3.389254in}{0.712061in}}%
\pgfpathlineto{\pgfqpoint{3.389810in}{0.750767in}}%
\pgfpathlineto{\pgfqpoint{3.390366in}{0.633351in}}%
\pgfpathlineto{\pgfqpoint{3.390922in}{0.727736in}}%
\pgfpathlineto{\pgfqpoint{3.391478in}{0.711492in}}%
\pgfpathlineto{\pgfqpoint{3.392033in}{0.723119in}}%
\pgfpathlineto{\pgfqpoint{3.392589in}{0.752765in}}%
\pgfpathlineto{\pgfqpoint{3.393145in}{0.716371in}}%
\pgfpathlineto{\pgfqpoint{3.393701in}{0.767071in}}%
\pgfpathlineto{\pgfqpoint{3.394813in}{0.615326in}}%
\pgfpathlineto{\pgfqpoint{3.395369in}{0.685551in}}%
\pgfpathlineto{\pgfqpoint{3.395924in}{0.651672in}}%
\pgfpathlineto{\pgfqpoint{3.397036in}{0.638636in}}%
\pgfpathlineto{\pgfqpoint{3.398704in}{0.670136in}}%
\pgfpathlineto{\pgfqpoint{3.399815in}{0.715129in}}%
\pgfpathlineto{\pgfqpoint{3.400371in}{0.679741in}}%
\pgfpathlineto{\pgfqpoint{3.400927in}{0.712678in}}%
\pgfpathlineto{\pgfqpoint{3.401483in}{0.721444in}}%
\pgfpathlineto{\pgfqpoint{3.402595in}{0.654412in}}%
\pgfpathlineto{\pgfqpoint{3.404262in}{0.791554in}}%
\pgfpathlineto{\pgfqpoint{3.404818in}{0.753376in}}%
\pgfpathlineto{\pgfqpoint{3.405374in}{0.924690in}}%
\pgfpathlineto{\pgfqpoint{3.405930in}{0.902274in}}%
\pgfpathlineto{\pgfqpoint{3.407598in}{0.705514in}}%
\pgfpathlineto{\pgfqpoint{3.408709in}{0.617433in}}%
\pgfpathlineto{\pgfqpoint{3.409265in}{0.822671in}}%
\pgfpathlineto{\pgfqpoint{3.409821in}{0.699809in}}%
\pgfpathlineto{\pgfqpoint{3.410377in}{0.710234in}}%
\pgfpathlineto{\pgfqpoint{3.410933in}{0.749088in}}%
\pgfpathlineto{\pgfqpoint{3.411489in}{0.871400in}}%
\pgfpathlineto{\pgfqpoint{3.412600in}{0.697573in}}%
\pgfpathlineto{\pgfqpoint{3.413156in}{0.734231in}}%
\pgfpathlineto{\pgfqpoint{3.413712in}{0.867898in}}%
\pgfpathlineto{\pgfqpoint{3.414268in}{0.696136in}}%
\pgfpathlineto{\pgfqpoint{3.414824in}{0.867872in}}%
\pgfpathlineto{\pgfqpoint{3.415380in}{0.701473in}}%
\pgfpathlineto{\pgfqpoint{3.415935in}{0.633400in}}%
\pgfpathlineto{\pgfqpoint{3.417047in}{0.892871in}}%
\pgfpathlineto{\pgfqpoint{3.418715in}{0.638460in}}%
\pgfpathlineto{\pgfqpoint{3.419271in}{0.758113in}}%
\pgfpathlineto{\pgfqpoint{3.420938in}{0.864424in}}%
\pgfpathlineto{\pgfqpoint{3.421494in}{1.117438in}}%
\pgfpathlineto{\pgfqpoint{3.422606in}{0.732196in}}%
\pgfpathlineto{\pgfqpoint{3.423162in}{1.315576in}}%
\pgfpathlineto{\pgfqpoint{3.423717in}{1.153808in}}%
\pgfpathlineto{\pgfqpoint{3.424273in}{1.113753in}}%
\pgfpathlineto{\pgfqpoint{3.425941in}{0.612393in}}%
\pgfpathlineto{\pgfqpoint{3.427053in}{0.895512in}}%
\pgfpathlineto{\pgfqpoint{3.428164in}{0.622217in}}%
\pgfpathlineto{\pgfqpoint{3.428720in}{0.679721in}}%
\pgfpathlineto{\pgfqpoint{3.429276in}{0.738605in}}%
\pgfpathlineto{\pgfqpoint{3.429832in}{0.629247in}}%
\pgfpathlineto{\pgfqpoint{3.430388in}{0.657101in}}%
\pgfpathlineto{\pgfqpoint{3.430944in}{0.661682in}}%
\pgfpathlineto{\pgfqpoint{3.431500in}{0.623407in}}%
\pgfpathlineto{\pgfqpoint{3.431500in}{0.623407in}}%
\pgfusepath{stroke}%
\end{pgfscope}%
\begin{pgfscope}%
\pgfsetrectcap%
\pgfsetmiterjoin%
\pgfsetlinewidth{0.803000pt}%
\definecolor{currentstroke}{rgb}{0.000000,0.000000,0.000000}%
\pgfsetstrokecolor{currentstroke}%
\pgfsetdash{}{0pt}%
\pgfpathmoveto{\pgfqpoint{0.717889in}{0.564143in}}%
\pgfpathlineto{\pgfqpoint{0.717889in}{1.351359in}}%
\pgfusepath{stroke}%
\end{pgfscope}%
\begin{pgfscope}%
\pgfsetrectcap%
\pgfsetmiterjoin%
\pgfsetlinewidth{0.803000pt}%
\definecolor{currentstroke}{rgb}{0.000000,0.000000,0.000000}%
\pgfsetstrokecolor{currentstroke}%
\pgfsetdash{}{0pt}%
\pgfpathmoveto{\pgfqpoint{6.146222in}{0.564143in}}%
\pgfpathlineto{\pgfqpoint{6.146222in}{1.351359in}}%
\pgfusepath{stroke}%
\end{pgfscope}%
\begin{pgfscope}%
\pgfsetrectcap%
\pgfsetmiterjoin%
\pgfsetlinewidth{0.803000pt}%
\definecolor{currentstroke}{rgb}{0.000000,0.000000,0.000000}%
\pgfsetstrokecolor{currentstroke}%
\pgfsetdash{}{0pt}%
\pgfpathmoveto{\pgfqpoint{0.717889in}{0.564143in}}%
\pgfpathlineto{\pgfqpoint{6.146222in}{0.564143in}}%
\pgfusepath{stroke}%
\end{pgfscope}%
\begin{pgfscope}%
\pgfsetrectcap%
\pgfsetmiterjoin%
\pgfsetlinewidth{0.803000pt}%
\definecolor{currentstroke}{rgb}{0.000000,0.000000,0.000000}%
\pgfsetstrokecolor{currentstroke}%
\pgfsetdash{}{0pt}%
\pgfpathmoveto{\pgfqpoint{0.717889in}{1.351359in}}%
\pgfpathlineto{\pgfqpoint{6.146222in}{1.351359in}}%
\pgfusepath{stroke}%
\end{pgfscope}%
\begin{pgfscope}%
\definecolor{textcolor}{rgb}{0.000000,0.000000,0.000000}%
\pgfsetstrokecolor{textcolor}%
\pgfsetfillcolor{textcolor}%
\pgftext[x=3.432055in,y=1.434692in,,base]{\color{textcolor}\rmfamily\fontsize{12.000000}{14.400000}\selectfont Spectrum of Filtered ECG Signal}%
\end{pgfscope}%
\end{pgfpicture}%
\makeatother%
\endgroup%

    }
    \caption{Frequency response of the cascaded IIR filters (top), IIR filtered ECG signal (middle), and spectrum of filtered ECG signal (bottom).}
    \label{fig:iir-resp}
\end{figure}

\subsection{Noise Power Estimate}
Noise power was estimated by subtracting the variance of the ECG signals after filtering from the variance of the
ECG signals before filtering. To estimate the relative noise powers, the power at individual interference frequencies
was computed. This was trivial for the IIR filter as it is the cascade of two notch filters. The FIR filters are single
filters with two notches, so the power spectral density of the noisy ECG signal, computed with the SciPy function
\texttt{periodogram}, was first plotted. The relative height of the noise peaks was used to estimate the relative
noise powers; it was identified that 62\% of the noise was at 61.7 Hz and 38\% was at 32.6 Hz. Table 1 summarises 
the noise power estimates.
\begin{table}[H]
    \caption{Noise power estimates obtained after each filter.}
    \label{table:noise-power}
    \adjustbox{max width=1.1\textwidth}{
    \centering
    \begin{tabularx}{\textwidth}{ | X || X | X | X || X | X | }
        \hline
        Filter & Total Noise Power $(\mu V^2)$ & 32.6 Hz Noise Power $(\mu V^2)$ & 61.7 Hz Noise Power $(\mu V^2)$ & 32.6 Hz Relative Noise Power (\%) & 61.7 Hz Relative Noise Power (\%) \\
        \hline
        Window FIR & 16386 & 6227 & 10159 & 38 & 62\\
        Optimal FIR & 15265 & 5801 & 9496 & 38 & 62\\
        Sampled FIR & 15486 & 5885 & 9601 & 38 & 62\\
        IIR & 15376 & 5683 & 9682 & 30 & 70\\
        \hline
    \end{tabularx}
    }
\end{table}

\section{Discussion}
An FIR filter may exhibit linear phase, which means that the frequency components of the input signal are delayed 
in time by the same constant amount. An IIR filter cannot have linear phase. The window and optimal FIR filters
in this assignment were designed to have linear phase; the frequency-sampled FIR filter was not. Linear phase is
useful in applications where preserving wave shape is important. In ECGs, linear phase filters may be used to remove
baseline wander without distorting clinical information [4].\\

\noindent A disadvantage of FIR filters is that they use more coefficients for similar attenuation characteristics.
Consequently, the input-to-output delay of an FIR filter is larger. This delay is given by:
\begin{equation}
    \textrm{Delay} = \frac{N-1}{2f_s}
\end{equation}

\noindent Where N is the number of coefficients and fs is the sampling frequency. Figure 10 shows the delays of the window
FIR filter and the cascaded IIR notch filters. The delay of the FIR filter is 194.3 ms, whereas the IIR filter has
near-zero delay. The FIR delay may be untenable in real-time systems where fast processing is required, such as an
embedded system. An IIR filter may be more suitable for such an application if a non-linear phase response is acceptable.\\

\begin{figure}[H]
    \begin{subfigure}{0.5\textwidth}
        \resizebox{\linewidth}{!}{%% Creator: Matplotlib, PGF backend
%%
%% To include the figure in your LaTeX document, write
%%   \input{<filename>.pgf}
%%
%% Make sure the required packages are loaded in your preamble
%%   \usepackage{pgf}
%%
%% and, on pdftex
%%   \usepackage[utf8]{inputenc}\DeclareUnicodeCharacter{2212}{-}
%%
%% or, on luatex and xetex
%%   \usepackage{unicode-math}
%%
%% Figures using additional raster images can only be included by \input if
%% they are in the same directory as the main LaTeX file. For loading figures
%% from other directories you can use the `import` package
%%   \usepackage{import}
%%
%% and then include the figures with
%%   \import{<path to file>}{<filename>.pgf}
%%
%% Matplotlib used the following preamble
%%
\begingroup%
\makeatletter%
\begin{pgfpicture}%
\pgfpathrectangle{\pgfpointorigin}{\pgfqpoint{6.400000in}{4.800000in}}%
\pgfusepath{use as bounding box, clip}%
\begin{pgfscope}%
\pgfsetbuttcap%
\pgfsetmiterjoin%
\definecolor{currentfill}{rgb}{1.000000,1.000000,1.000000}%
\pgfsetfillcolor{currentfill}%
\pgfsetlinewidth{0.000000pt}%
\definecolor{currentstroke}{rgb}{1.000000,1.000000,1.000000}%
\pgfsetstrokecolor{currentstroke}%
\pgfsetdash{}{0pt}%
\pgfpathmoveto{\pgfqpoint{0.000000in}{0.000000in}}%
\pgfpathlineto{\pgfqpoint{6.400000in}{0.000000in}}%
\pgfpathlineto{\pgfqpoint{6.400000in}{4.800000in}}%
\pgfpathlineto{\pgfqpoint{0.000000in}{4.800000in}}%
\pgfpathclose%
\pgfusepath{fill}%
\end{pgfscope}%
\begin{pgfscope}%
\pgfsetbuttcap%
\pgfsetmiterjoin%
\definecolor{currentfill}{rgb}{1.000000,1.000000,1.000000}%
\pgfsetfillcolor{currentfill}%
\pgfsetlinewidth{0.000000pt}%
\definecolor{currentstroke}{rgb}{0.000000,0.000000,0.000000}%
\pgfsetstrokecolor{currentstroke}%
\pgfsetstrokeopacity{0.000000}%
\pgfsetdash{}{0pt}%
\pgfpathmoveto{\pgfqpoint{0.800000in}{0.528000in}}%
\pgfpathlineto{\pgfqpoint{5.760000in}{0.528000in}}%
\pgfpathlineto{\pgfqpoint{5.760000in}{4.224000in}}%
\pgfpathlineto{\pgfqpoint{0.800000in}{4.224000in}}%
\pgfpathclose%
\pgfusepath{fill}%
\end{pgfscope}%
\begin{pgfscope}%
\pgfsetbuttcap%
\pgfsetroundjoin%
\definecolor{currentfill}{rgb}{0.000000,0.000000,0.000000}%
\pgfsetfillcolor{currentfill}%
\pgfsetlinewidth{0.803000pt}%
\definecolor{currentstroke}{rgb}{0.000000,0.000000,0.000000}%
\pgfsetstrokecolor{currentstroke}%
\pgfsetdash{}{0pt}%
\pgfsys@defobject{currentmarker}{\pgfqpoint{0.000000in}{-0.048611in}}{\pgfqpoint{0.000000in}{0.000000in}}{%
\pgfpathmoveto{\pgfqpoint{0.000000in}{0.000000in}}%
\pgfpathlineto{\pgfqpoint{0.000000in}{-0.048611in}}%
\pgfusepath{stroke,fill}%
}%
\begin{pgfscope}%
\pgfsys@transformshift{0.800000in}{0.528000in}%
\pgfsys@useobject{currentmarker}{}%
\end{pgfscope}%
\end{pgfscope}%
\begin{pgfscope}%
\definecolor{textcolor}{rgb}{0.000000,0.000000,0.000000}%
\pgfsetstrokecolor{textcolor}%
\pgfsetfillcolor{textcolor}%
\pgftext[x=0.800000in,y=0.430778in,,top]{\color{textcolor}\rmfamily\fontsize{10.000000}{12.000000}\selectfont \(\displaystyle {10.0}\)}%
\end{pgfscope}%
\begin{pgfscope}%
\pgfsetbuttcap%
\pgfsetroundjoin%
\definecolor{currentfill}{rgb}{0.000000,0.000000,0.000000}%
\pgfsetfillcolor{currentfill}%
\pgfsetlinewidth{0.803000pt}%
\definecolor{currentstroke}{rgb}{0.000000,0.000000,0.000000}%
\pgfsetstrokecolor{currentstroke}%
\pgfsetdash{}{0pt}%
\pgfsys@defobject{currentmarker}{\pgfqpoint{0.000000in}{-0.048611in}}{\pgfqpoint{0.000000in}{0.000000in}}{%
\pgfpathmoveto{\pgfqpoint{0.000000in}{0.000000in}}%
\pgfpathlineto{\pgfqpoint{0.000000in}{-0.048611in}}%
\pgfusepath{stroke,fill}%
}%
\begin{pgfscope}%
\pgfsys@transformshift{1.792000in}{0.528000in}%
\pgfsys@useobject{currentmarker}{}%
\end{pgfscope}%
\end{pgfscope}%
\begin{pgfscope}%
\definecolor{textcolor}{rgb}{0.000000,0.000000,0.000000}%
\pgfsetstrokecolor{textcolor}%
\pgfsetfillcolor{textcolor}%
\pgftext[x=1.792000in,y=0.430778in,,top]{\color{textcolor}\rmfamily\fontsize{10.000000}{12.000000}\selectfont \(\displaystyle {10.2}\)}%
\end{pgfscope}%
\begin{pgfscope}%
\pgfsetbuttcap%
\pgfsetroundjoin%
\definecolor{currentfill}{rgb}{0.000000,0.000000,0.000000}%
\pgfsetfillcolor{currentfill}%
\pgfsetlinewidth{0.803000pt}%
\definecolor{currentstroke}{rgb}{0.000000,0.000000,0.000000}%
\pgfsetstrokecolor{currentstroke}%
\pgfsetdash{}{0pt}%
\pgfsys@defobject{currentmarker}{\pgfqpoint{0.000000in}{-0.048611in}}{\pgfqpoint{0.000000in}{0.000000in}}{%
\pgfpathmoveto{\pgfqpoint{0.000000in}{0.000000in}}%
\pgfpathlineto{\pgfqpoint{0.000000in}{-0.048611in}}%
\pgfusepath{stroke,fill}%
}%
\begin{pgfscope}%
\pgfsys@transformshift{2.784000in}{0.528000in}%
\pgfsys@useobject{currentmarker}{}%
\end{pgfscope}%
\end{pgfscope}%
\begin{pgfscope}%
\definecolor{textcolor}{rgb}{0.000000,0.000000,0.000000}%
\pgfsetstrokecolor{textcolor}%
\pgfsetfillcolor{textcolor}%
\pgftext[x=2.784000in,y=0.430778in,,top]{\color{textcolor}\rmfamily\fontsize{10.000000}{12.000000}\selectfont \(\displaystyle {10.4}\)}%
\end{pgfscope}%
\begin{pgfscope}%
\pgfsetbuttcap%
\pgfsetroundjoin%
\definecolor{currentfill}{rgb}{0.000000,0.000000,0.000000}%
\pgfsetfillcolor{currentfill}%
\pgfsetlinewidth{0.803000pt}%
\definecolor{currentstroke}{rgb}{0.000000,0.000000,0.000000}%
\pgfsetstrokecolor{currentstroke}%
\pgfsetdash{}{0pt}%
\pgfsys@defobject{currentmarker}{\pgfqpoint{0.000000in}{-0.048611in}}{\pgfqpoint{0.000000in}{0.000000in}}{%
\pgfpathmoveto{\pgfqpoint{0.000000in}{0.000000in}}%
\pgfpathlineto{\pgfqpoint{0.000000in}{-0.048611in}}%
\pgfusepath{stroke,fill}%
}%
\begin{pgfscope}%
\pgfsys@transformshift{3.776000in}{0.528000in}%
\pgfsys@useobject{currentmarker}{}%
\end{pgfscope}%
\end{pgfscope}%
\begin{pgfscope}%
\definecolor{textcolor}{rgb}{0.000000,0.000000,0.000000}%
\pgfsetstrokecolor{textcolor}%
\pgfsetfillcolor{textcolor}%
\pgftext[x=3.776000in,y=0.430778in,,top]{\color{textcolor}\rmfamily\fontsize{10.000000}{12.000000}\selectfont \(\displaystyle {10.6}\)}%
\end{pgfscope}%
\begin{pgfscope}%
\pgfsetbuttcap%
\pgfsetroundjoin%
\definecolor{currentfill}{rgb}{0.000000,0.000000,0.000000}%
\pgfsetfillcolor{currentfill}%
\pgfsetlinewidth{0.803000pt}%
\definecolor{currentstroke}{rgb}{0.000000,0.000000,0.000000}%
\pgfsetstrokecolor{currentstroke}%
\pgfsetdash{}{0pt}%
\pgfsys@defobject{currentmarker}{\pgfqpoint{0.000000in}{-0.048611in}}{\pgfqpoint{0.000000in}{0.000000in}}{%
\pgfpathmoveto{\pgfqpoint{0.000000in}{0.000000in}}%
\pgfpathlineto{\pgfqpoint{0.000000in}{-0.048611in}}%
\pgfusepath{stroke,fill}%
}%
\begin{pgfscope}%
\pgfsys@transformshift{4.768000in}{0.528000in}%
\pgfsys@useobject{currentmarker}{}%
\end{pgfscope}%
\end{pgfscope}%
\begin{pgfscope}%
\definecolor{textcolor}{rgb}{0.000000,0.000000,0.000000}%
\pgfsetstrokecolor{textcolor}%
\pgfsetfillcolor{textcolor}%
\pgftext[x=4.768000in,y=0.430778in,,top]{\color{textcolor}\rmfamily\fontsize{10.000000}{12.000000}\selectfont \(\displaystyle {10.8}\)}%
\end{pgfscope}%
\begin{pgfscope}%
\pgfsetbuttcap%
\pgfsetroundjoin%
\definecolor{currentfill}{rgb}{0.000000,0.000000,0.000000}%
\pgfsetfillcolor{currentfill}%
\pgfsetlinewidth{0.803000pt}%
\definecolor{currentstroke}{rgb}{0.000000,0.000000,0.000000}%
\pgfsetstrokecolor{currentstroke}%
\pgfsetdash{}{0pt}%
\pgfsys@defobject{currentmarker}{\pgfqpoint{0.000000in}{-0.048611in}}{\pgfqpoint{0.000000in}{0.000000in}}{%
\pgfpathmoveto{\pgfqpoint{0.000000in}{0.000000in}}%
\pgfpathlineto{\pgfqpoint{0.000000in}{-0.048611in}}%
\pgfusepath{stroke,fill}%
}%
\begin{pgfscope}%
\pgfsys@transformshift{5.760000in}{0.528000in}%
\pgfsys@useobject{currentmarker}{}%
\end{pgfscope}%
\end{pgfscope}%
\begin{pgfscope}%
\definecolor{textcolor}{rgb}{0.000000,0.000000,0.000000}%
\pgfsetstrokecolor{textcolor}%
\pgfsetfillcolor{textcolor}%
\pgftext[x=5.760000in,y=0.430778in,,top]{\color{textcolor}\rmfamily\fontsize{10.000000}{12.000000}\selectfont \(\displaystyle {11.0}\)}%
\end{pgfscope}%
\begin{pgfscope}%
\definecolor{textcolor}{rgb}{0.000000,0.000000,0.000000}%
\pgfsetstrokecolor{textcolor}%
\pgfsetfillcolor{textcolor}%
\pgftext[x=3.280000in,y=0.251766in,,top]{\color{textcolor}\rmfamily\fontsize{10.000000}{12.000000}\selectfont Time (s)}%
\end{pgfscope}%
\begin{pgfscope}%
\pgfsetbuttcap%
\pgfsetroundjoin%
\definecolor{currentfill}{rgb}{0.000000,0.000000,0.000000}%
\pgfsetfillcolor{currentfill}%
\pgfsetlinewidth{0.803000pt}%
\definecolor{currentstroke}{rgb}{0.000000,0.000000,0.000000}%
\pgfsetstrokecolor{currentstroke}%
\pgfsetdash{}{0pt}%
\pgfsys@defobject{currentmarker}{\pgfqpoint{-0.048611in}{0.000000in}}{\pgfqpoint{0.000000in}{0.000000in}}{%
\pgfpathmoveto{\pgfqpoint{0.000000in}{0.000000in}}%
\pgfpathlineto{\pgfqpoint{-0.048611in}{0.000000in}}%
\pgfusepath{stroke,fill}%
}%
\begin{pgfscope}%
\pgfsys@transformshift{0.800000in}{0.823350in}%
\pgfsys@useobject{currentmarker}{}%
\end{pgfscope}%
\end{pgfscope}%
\begin{pgfscope}%
\definecolor{textcolor}{rgb}{0.000000,0.000000,0.000000}%
\pgfsetstrokecolor{textcolor}%
\pgfsetfillcolor{textcolor}%
\pgftext[x=0.316974in, y=0.775125in, left, base]{\color{textcolor}\rmfamily\fontsize{10.000000}{12.000000}\selectfont \(\displaystyle {-1000}\)}%
\end{pgfscope}%
\begin{pgfscope}%
\pgfsetbuttcap%
\pgfsetroundjoin%
\definecolor{currentfill}{rgb}{0.000000,0.000000,0.000000}%
\pgfsetfillcolor{currentfill}%
\pgfsetlinewidth{0.803000pt}%
\definecolor{currentstroke}{rgb}{0.000000,0.000000,0.000000}%
\pgfsetstrokecolor{currentstroke}%
\pgfsetdash{}{0pt}%
\pgfsys@defobject{currentmarker}{\pgfqpoint{-0.048611in}{0.000000in}}{\pgfqpoint{0.000000in}{0.000000in}}{%
\pgfpathmoveto{\pgfqpoint{0.000000in}{0.000000in}}%
\pgfpathlineto{\pgfqpoint{-0.048611in}{0.000000in}}%
\pgfusepath{stroke,fill}%
}%
\begin{pgfscope}%
\pgfsys@transformshift{0.800000in}{1.467604in}%
\pgfsys@useobject{currentmarker}{}%
\end{pgfscope}%
\end{pgfscope}%
\begin{pgfscope}%
\definecolor{textcolor}{rgb}{0.000000,0.000000,0.000000}%
\pgfsetstrokecolor{textcolor}%
\pgfsetfillcolor{textcolor}%
\pgftext[x=0.386419in, y=1.419378in, left, base]{\color{textcolor}\rmfamily\fontsize{10.000000}{12.000000}\selectfont \(\displaystyle {-500}\)}%
\end{pgfscope}%
\begin{pgfscope}%
\pgfsetbuttcap%
\pgfsetroundjoin%
\definecolor{currentfill}{rgb}{0.000000,0.000000,0.000000}%
\pgfsetfillcolor{currentfill}%
\pgfsetlinewidth{0.803000pt}%
\definecolor{currentstroke}{rgb}{0.000000,0.000000,0.000000}%
\pgfsetstrokecolor{currentstroke}%
\pgfsetdash{}{0pt}%
\pgfsys@defobject{currentmarker}{\pgfqpoint{-0.048611in}{0.000000in}}{\pgfqpoint{0.000000in}{0.000000in}}{%
\pgfpathmoveto{\pgfqpoint{0.000000in}{0.000000in}}%
\pgfpathlineto{\pgfqpoint{-0.048611in}{0.000000in}}%
\pgfusepath{stroke,fill}%
}%
\begin{pgfscope}%
\pgfsys@transformshift{0.800000in}{2.111857in}%
\pgfsys@useobject{currentmarker}{}%
\end{pgfscope}%
\end{pgfscope}%
\begin{pgfscope}%
\definecolor{textcolor}{rgb}{0.000000,0.000000,0.000000}%
\pgfsetstrokecolor{textcolor}%
\pgfsetfillcolor{textcolor}%
\pgftext[x=0.633333in, y=2.063632in, left, base]{\color{textcolor}\rmfamily\fontsize{10.000000}{12.000000}\selectfont \(\displaystyle {0}\)}%
\end{pgfscope}%
\begin{pgfscope}%
\pgfsetbuttcap%
\pgfsetroundjoin%
\definecolor{currentfill}{rgb}{0.000000,0.000000,0.000000}%
\pgfsetfillcolor{currentfill}%
\pgfsetlinewidth{0.803000pt}%
\definecolor{currentstroke}{rgb}{0.000000,0.000000,0.000000}%
\pgfsetstrokecolor{currentstroke}%
\pgfsetdash{}{0pt}%
\pgfsys@defobject{currentmarker}{\pgfqpoint{-0.048611in}{0.000000in}}{\pgfqpoint{0.000000in}{0.000000in}}{%
\pgfpathmoveto{\pgfqpoint{0.000000in}{0.000000in}}%
\pgfpathlineto{\pgfqpoint{-0.048611in}{0.000000in}}%
\pgfusepath{stroke,fill}%
}%
\begin{pgfscope}%
\pgfsys@transformshift{0.800000in}{2.756111in}%
\pgfsys@useobject{currentmarker}{}%
\end{pgfscope}%
\end{pgfscope}%
\begin{pgfscope}%
\definecolor{textcolor}{rgb}{0.000000,0.000000,0.000000}%
\pgfsetstrokecolor{textcolor}%
\pgfsetfillcolor{textcolor}%
\pgftext[x=0.494444in, y=2.707885in, left, base]{\color{textcolor}\rmfamily\fontsize{10.000000}{12.000000}\selectfont \(\displaystyle {500}\)}%
\end{pgfscope}%
\begin{pgfscope}%
\pgfsetbuttcap%
\pgfsetroundjoin%
\definecolor{currentfill}{rgb}{0.000000,0.000000,0.000000}%
\pgfsetfillcolor{currentfill}%
\pgfsetlinewidth{0.803000pt}%
\definecolor{currentstroke}{rgb}{0.000000,0.000000,0.000000}%
\pgfsetstrokecolor{currentstroke}%
\pgfsetdash{}{0pt}%
\pgfsys@defobject{currentmarker}{\pgfqpoint{-0.048611in}{0.000000in}}{\pgfqpoint{0.000000in}{0.000000in}}{%
\pgfpathmoveto{\pgfqpoint{0.000000in}{0.000000in}}%
\pgfpathlineto{\pgfqpoint{-0.048611in}{0.000000in}}%
\pgfusepath{stroke,fill}%
}%
\begin{pgfscope}%
\pgfsys@transformshift{0.800000in}{3.400364in}%
\pgfsys@useobject{currentmarker}{}%
\end{pgfscope}%
\end{pgfscope}%
\begin{pgfscope}%
\definecolor{textcolor}{rgb}{0.000000,0.000000,0.000000}%
\pgfsetstrokecolor{textcolor}%
\pgfsetfillcolor{textcolor}%
\pgftext[x=0.424999in, y=3.352139in, left, base]{\color{textcolor}\rmfamily\fontsize{10.000000}{12.000000}\selectfont \(\displaystyle {1000}\)}%
\end{pgfscope}%
\begin{pgfscope}%
\pgfsetbuttcap%
\pgfsetroundjoin%
\definecolor{currentfill}{rgb}{0.000000,0.000000,0.000000}%
\pgfsetfillcolor{currentfill}%
\pgfsetlinewidth{0.803000pt}%
\definecolor{currentstroke}{rgb}{0.000000,0.000000,0.000000}%
\pgfsetstrokecolor{currentstroke}%
\pgfsetdash{}{0pt}%
\pgfsys@defobject{currentmarker}{\pgfqpoint{-0.048611in}{0.000000in}}{\pgfqpoint{0.000000in}{0.000000in}}{%
\pgfpathmoveto{\pgfqpoint{0.000000in}{0.000000in}}%
\pgfpathlineto{\pgfqpoint{-0.048611in}{0.000000in}}%
\pgfusepath{stroke,fill}%
}%
\begin{pgfscope}%
\pgfsys@transformshift{0.800000in}{4.044618in}%
\pgfsys@useobject{currentmarker}{}%
\end{pgfscope}%
\end{pgfscope}%
\begin{pgfscope}%
\definecolor{textcolor}{rgb}{0.000000,0.000000,0.000000}%
\pgfsetstrokecolor{textcolor}%
\pgfsetfillcolor{textcolor}%
\pgftext[x=0.424999in, y=3.996392in, left, base]{\color{textcolor}\rmfamily\fontsize{10.000000}{12.000000}\selectfont \(\displaystyle {1500}\)}%
\end{pgfscope}%
\begin{pgfscope}%
\definecolor{textcolor}{rgb}{0.000000,0.000000,0.000000}%
\pgfsetstrokecolor{textcolor}%
\pgfsetfillcolor{textcolor}%
\pgftext[x=0.261419in,y=2.376000in,,bottom,rotate=90.000000]{\color{textcolor}\rmfamily\fontsize{10.000000}{12.000000}\selectfont ECG Voltage (\(\displaystyle \mu V\))}%
\end{pgfscope}%
\begin{pgfscope}%
\pgfpathrectangle{\pgfqpoint{0.800000in}{0.528000in}}{\pgfqpoint{4.960000in}{3.696000in}}%
\pgfusepath{clip}%
\pgfsetrectcap%
\pgfsetroundjoin%
\pgfsetlinewidth{1.505625pt}%
\definecolor{currentstroke}{rgb}{0.121569,0.466667,0.705882}%
\pgfsetstrokecolor{currentstroke}%
\pgfsetdash{}{0pt}%
\pgfpathmoveto{\pgfqpoint{0.795156in}{1.840209in}}%
\pgfpathlineto{\pgfqpoint{0.800000in}{1.830400in}}%
\pgfpathlineto{\pgfqpoint{0.804844in}{1.826122in}}%
\pgfpathlineto{\pgfqpoint{0.809688in}{1.825980in}}%
\pgfpathlineto{\pgfqpoint{0.819375in}{1.828682in}}%
\pgfpathlineto{\pgfqpoint{0.824219in}{1.824922in}}%
\pgfpathlineto{\pgfqpoint{0.833906in}{1.810806in}}%
\pgfpathlineto{\pgfqpoint{0.838750in}{1.805578in}}%
\pgfpathlineto{\pgfqpoint{0.843594in}{1.802871in}}%
\pgfpathlineto{\pgfqpoint{0.853281in}{1.805052in}}%
\pgfpathlineto{\pgfqpoint{0.858125in}{1.803471in}}%
\pgfpathlineto{\pgfqpoint{0.862969in}{1.800463in}}%
\pgfpathlineto{\pgfqpoint{0.877500in}{1.779357in}}%
\pgfpathlineto{\pgfqpoint{0.882344in}{1.777418in}}%
\pgfpathlineto{\pgfqpoint{0.887188in}{1.782661in}}%
\pgfpathlineto{\pgfqpoint{0.896875in}{1.800723in}}%
\pgfpathlineto{\pgfqpoint{0.901719in}{1.803500in}}%
\pgfpathlineto{\pgfqpoint{0.906563in}{1.799787in}}%
\pgfpathlineto{\pgfqpoint{0.911406in}{1.785958in}}%
\pgfpathlineto{\pgfqpoint{0.921094in}{1.751983in}}%
\pgfpathlineto{\pgfqpoint{0.925937in}{1.742469in}}%
\pgfpathlineto{\pgfqpoint{0.930781in}{1.740189in}}%
\pgfpathlineto{\pgfqpoint{0.935625in}{1.743828in}}%
\pgfpathlineto{\pgfqpoint{0.940469in}{1.749409in}}%
\pgfpathlineto{\pgfqpoint{0.945312in}{1.752879in}}%
\pgfpathlineto{\pgfqpoint{0.950156in}{1.750133in}}%
\pgfpathlineto{\pgfqpoint{0.959844in}{1.723671in}}%
\pgfpathlineto{\pgfqpoint{0.964688in}{1.708512in}}%
\pgfpathlineto{\pgfqpoint{0.969531in}{1.703029in}}%
\pgfpathlineto{\pgfqpoint{0.974375in}{1.708487in}}%
\pgfpathlineto{\pgfqpoint{0.988906in}{1.754374in}}%
\pgfpathlineto{\pgfqpoint{0.993750in}{1.751040in}}%
\pgfpathlineto{\pgfqpoint{0.998594in}{1.736397in}}%
\pgfpathlineto{\pgfqpoint{1.003437in}{1.717021in}}%
\pgfpathlineto{\pgfqpoint{1.008281in}{1.707117in}}%
\pgfpathlineto{\pgfqpoint{1.013125in}{1.710498in}}%
\pgfpathlineto{\pgfqpoint{1.027656in}{1.757365in}}%
\pgfpathlineto{\pgfqpoint{1.032500in}{1.753915in}}%
\pgfpathlineto{\pgfqpoint{1.037344in}{1.735574in}}%
\pgfpathlineto{\pgfqpoint{1.047031in}{1.691033in}}%
\pgfpathlineto{\pgfqpoint{1.051875in}{1.677774in}}%
\pgfpathlineto{\pgfqpoint{1.056719in}{1.672941in}}%
\pgfpathlineto{\pgfqpoint{1.066406in}{1.675847in}}%
\pgfpathlineto{\pgfqpoint{1.071250in}{1.675413in}}%
\pgfpathlineto{\pgfqpoint{1.076094in}{1.673381in}}%
\pgfpathlineto{\pgfqpoint{1.080938in}{1.667458in}}%
\pgfpathlineto{\pgfqpoint{1.090625in}{1.649478in}}%
\pgfpathlineto{\pgfqpoint{1.095469in}{1.639776in}}%
\pgfpathlineto{\pgfqpoint{1.100313in}{1.633483in}}%
\pgfpathlineto{\pgfqpoint{1.105156in}{1.634015in}}%
\pgfpathlineto{\pgfqpoint{1.110000in}{1.643457in}}%
\pgfpathlineto{\pgfqpoint{1.119687in}{1.671870in}}%
\pgfpathlineto{\pgfqpoint{1.124531in}{1.674776in}}%
\pgfpathlineto{\pgfqpoint{1.129375in}{1.660982in}}%
\pgfpathlineto{\pgfqpoint{1.134219in}{1.630577in}}%
\pgfpathlineto{\pgfqpoint{1.143906in}{1.558355in}}%
\pgfpathlineto{\pgfqpoint{1.148750in}{1.541780in}}%
\pgfpathlineto{\pgfqpoint{1.153594in}{1.543528in}}%
\pgfpathlineto{\pgfqpoint{1.163281in}{1.571181in}}%
\pgfpathlineto{\pgfqpoint{1.168125in}{1.570401in}}%
\pgfpathlineto{\pgfqpoint{1.172969in}{1.548661in}}%
\pgfpathlineto{\pgfqpoint{1.182656in}{1.478394in}}%
\pgfpathlineto{\pgfqpoint{1.187500in}{1.456010in}}%
\pgfpathlineto{\pgfqpoint{1.192344in}{1.450939in}}%
\pgfpathlineto{\pgfqpoint{1.197188in}{1.460226in}}%
\pgfpathlineto{\pgfqpoint{1.202031in}{1.473343in}}%
\pgfpathlineto{\pgfqpoint{1.206875in}{1.479953in}}%
\pgfpathlineto{\pgfqpoint{1.211719in}{1.476352in}}%
\pgfpathlineto{\pgfqpoint{1.226250in}{1.449604in}}%
\pgfpathlineto{\pgfqpoint{1.231094in}{1.447104in}}%
\pgfpathlineto{\pgfqpoint{1.235938in}{1.445935in}}%
\pgfpathlineto{\pgfqpoint{1.240781in}{1.441322in}}%
\pgfpathlineto{\pgfqpoint{1.245625in}{1.434713in}}%
\pgfpathlineto{\pgfqpoint{1.260156in}{1.408839in}}%
\pgfpathlineto{\pgfqpoint{1.265000in}{1.395356in}}%
\pgfpathlineto{\pgfqpoint{1.269844in}{1.377755in}}%
\pgfpathlineto{\pgfqpoint{1.279531in}{1.336853in}}%
\pgfpathlineto{\pgfqpoint{1.284375in}{1.324323in}}%
\pgfpathlineto{\pgfqpoint{1.289219in}{1.321611in}}%
\pgfpathlineto{\pgfqpoint{1.298906in}{1.325912in}}%
\pgfpathlineto{\pgfqpoint{1.303750in}{1.322246in}}%
\pgfpathlineto{\pgfqpoint{1.308594in}{1.310761in}}%
\pgfpathlineto{\pgfqpoint{1.318281in}{1.281985in}}%
\pgfpathlineto{\pgfqpoint{1.323125in}{1.274478in}}%
\pgfpathlineto{\pgfqpoint{1.327969in}{1.276034in}}%
\pgfpathlineto{\pgfqpoint{1.337656in}{1.283705in}}%
\pgfpathlineto{\pgfqpoint{1.342500in}{1.282538in}}%
\pgfpathlineto{\pgfqpoint{1.352188in}{1.262248in}}%
\pgfpathlineto{\pgfqpoint{1.357031in}{1.249808in}}%
\pgfpathlineto{\pgfqpoint{1.361875in}{1.242262in}}%
\pgfpathlineto{\pgfqpoint{1.371563in}{1.231631in}}%
\pgfpathlineto{\pgfqpoint{1.376406in}{1.222762in}}%
\pgfpathlineto{\pgfqpoint{1.390937in}{1.177698in}}%
\pgfpathlineto{\pgfqpoint{1.395781in}{1.174152in}}%
\pgfpathlineto{\pgfqpoint{1.400625in}{1.177773in}}%
\pgfpathlineto{\pgfqpoint{1.410313in}{1.195196in}}%
\pgfpathlineto{\pgfqpoint{1.415156in}{1.195593in}}%
\pgfpathlineto{\pgfqpoint{1.420000in}{1.190558in}}%
\pgfpathlineto{\pgfqpoint{1.424844in}{1.181928in}}%
\pgfpathlineto{\pgfqpoint{1.429688in}{1.176652in}}%
\pgfpathlineto{\pgfqpoint{1.434531in}{1.177895in}}%
\pgfpathlineto{\pgfqpoint{1.444219in}{1.189062in}}%
\pgfpathlineto{\pgfqpoint{1.449062in}{1.190953in}}%
\pgfpathlineto{\pgfqpoint{1.453906in}{1.187186in}}%
\pgfpathlineto{\pgfqpoint{1.468438in}{1.158879in}}%
\pgfpathlineto{\pgfqpoint{1.473281in}{1.156235in}}%
\pgfpathlineto{\pgfqpoint{1.478125in}{1.157285in}}%
\pgfpathlineto{\pgfqpoint{1.482969in}{1.159762in}}%
\pgfpathlineto{\pgfqpoint{1.492656in}{1.168994in}}%
\pgfpathlineto{\pgfqpoint{1.502344in}{1.175306in}}%
\pgfpathlineto{\pgfqpoint{1.507188in}{1.175221in}}%
\pgfpathlineto{\pgfqpoint{1.512031in}{1.171791in}}%
\pgfpathlineto{\pgfqpoint{1.516875in}{1.164584in}}%
\pgfpathlineto{\pgfqpoint{1.526563in}{1.145834in}}%
\pgfpathlineto{\pgfqpoint{1.531406in}{1.144803in}}%
\pgfpathlineto{\pgfqpoint{1.536250in}{1.153273in}}%
\pgfpathlineto{\pgfqpoint{1.541094in}{1.169245in}}%
\pgfpathlineto{\pgfqpoint{1.550781in}{1.206863in}}%
\pgfpathlineto{\pgfqpoint{1.555625in}{1.211548in}}%
\pgfpathlineto{\pgfqpoint{1.560469in}{1.205089in}}%
\pgfpathlineto{\pgfqpoint{1.570156in}{1.175340in}}%
\pgfpathlineto{\pgfqpoint{1.575000in}{1.170726in}}%
\pgfpathlineto{\pgfqpoint{1.579844in}{1.178269in}}%
\pgfpathlineto{\pgfqpoint{1.594375in}{1.224285in}}%
\pgfpathlineto{\pgfqpoint{1.599219in}{1.223646in}}%
\pgfpathlineto{\pgfqpoint{1.608906in}{1.207948in}}%
\pgfpathlineto{\pgfqpoint{1.613750in}{1.207508in}}%
\pgfpathlineto{\pgfqpoint{1.618594in}{1.216867in}}%
\pgfpathlineto{\pgfqpoint{1.628281in}{1.247712in}}%
\pgfpathlineto{\pgfqpoint{1.633125in}{1.254621in}}%
\pgfpathlineto{\pgfqpoint{1.637969in}{1.248466in}}%
\pgfpathlineto{\pgfqpoint{1.642813in}{1.231822in}}%
\pgfpathlineto{\pgfqpoint{1.652500in}{1.193132in}}%
\pgfpathlineto{\pgfqpoint{1.657344in}{1.185695in}}%
\pgfpathlineto{\pgfqpoint{1.662187in}{1.189407in}}%
\pgfpathlineto{\pgfqpoint{1.671875in}{1.211896in}}%
\pgfpathlineto{\pgfqpoint{1.676719in}{1.218609in}}%
\pgfpathlineto{\pgfqpoint{1.681563in}{1.214982in}}%
\pgfpathlineto{\pgfqpoint{1.691250in}{1.195654in}}%
\pgfpathlineto{\pgfqpoint{1.696094in}{1.191003in}}%
\pgfpathlineto{\pgfqpoint{1.700937in}{1.192844in}}%
\pgfpathlineto{\pgfqpoint{1.705781in}{1.201558in}}%
\pgfpathlineto{\pgfqpoint{1.710625in}{1.207126in}}%
\pgfpathlineto{\pgfqpoint{1.715469in}{1.204519in}}%
\pgfpathlineto{\pgfqpoint{1.720313in}{1.187272in}}%
\pgfpathlineto{\pgfqpoint{1.739688in}{1.074447in}}%
\pgfpathlineto{\pgfqpoint{1.744531in}{1.067044in}}%
\pgfpathlineto{\pgfqpoint{1.749375in}{1.064682in}}%
\pgfpathlineto{\pgfqpoint{1.754219in}{1.063743in}}%
\pgfpathlineto{\pgfqpoint{1.759062in}{1.058174in}}%
\pgfpathlineto{\pgfqpoint{1.768750in}{1.036290in}}%
\pgfpathlineto{\pgfqpoint{1.773594in}{1.029827in}}%
\pgfpathlineto{\pgfqpoint{1.778438in}{1.031834in}}%
\pgfpathlineto{\pgfqpoint{1.792969in}{1.056171in}}%
\pgfpathlineto{\pgfqpoint{1.797813in}{1.059520in}}%
\pgfpathlineto{\pgfqpoint{1.802656in}{1.060985in}}%
\pgfpathlineto{\pgfqpoint{1.807500in}{1.064750in}}%
\pgfpathlineto{\pgfqpoint{1.812344in}{1.072529in}}%
\pgfpathlineto{\pgfqpoint{1.822031in}{1.100442in}}%
\pgfpathlineto{\pgfqpoint{1.826875in}{1.109969in}}%
\pgfpathlineto{\pgfqpoint{1.831719in}{1.110057in}}%
\pgfpathlineto{\pgfqpoint{1.836563in}{1.098795in}}%
\pgfpathlineto{\pgfqpoint{1.846250in}{1.050881in}}%
\pgfpathlineto{\pgfqpoint{1.855938in}{1.006607in}}%
\pgfpathlineto{\pgfqpoint{1.860781in}{0.996005in}}%
\pgfpathlineto{\pgfqpoint{1.865625in}{0.998437in}}%
\pgfpathlineto{\pgfqpoint{1.870469in}{1.011088in}}%
\pgfpathlineto{\pgfqpoint{1.885000in}{1.058714in}}%
\pgfpathlineto{\pgfqpoint{1.889844in}{1.064414in}}%
\pgfpathlineto{\pgfqpoint{1.899531in}{1.066600in}}%
\pgfpathlineto{\pgfqpoint{1.904375in}{1.071741in}}%
\pgfpathlineto{\pgfqpoint{1.909219in}{1.080414in}}%
\pgfpathlineto{\pgfqpoint{1.928594in}{1.129398in}}%
\pgfpathlineto{\pgfqpoint{1.933438in}{1.151212in}}%
\pgfpathlineto{\pgfqpoint{1.943125in}{1.216934in}}%
\pgfpathlineto{\pgfqpoint{1.952813in}{1.284532in}}%
\pgfpathlineto{\pgfqpoint{1.962500in}{1.328474in}}%
\pgfpathlineto{\pgfqpoint{1.977031in}{1.380840in}}%
\pgfpathlineto{\pgfqpoint{1.981875in}{1.384588in}}%
\pgfpathlineto{\pgfqpoint{1.986719in}{1.373100in}}%
\pgfpathlineto{\pgfqpoint{2.001250in}{1.296704in}}%
\pgfpathlineto{\pgfqpoint{2.006094in}{1.286671in}}%
\pgfpathlineto{\pgfqpoint{2.010937in}{1.284552in}}%
\pgfpathlineto{\pgfqpoint{2.015781in}{1.285198in}}%
\pgfpathlineto{\pgfqpoint{2.020625in}{1.275004in}}%
\pgfpathlineto{\pgfqpoint{2.025469in}{1.252472in}}%
\pgfpathlineto{\pgfqpoint{2.035156in}{1.201199in}}%
\pgfpathlineto{\pgfqpoint{2.040000in}{1.194094in}}%
\pgfpathlineto{\pgfqpoint{2.044844in}{1.208217in}}%
\pgfpathlineto{\pgfqpoint{2.054531in}{1.262356in}}%
\pgfpathlineto{\pgfqpoint{2.059375in}{1.276817in}}%
\pgfpathlineto{\pgfqpoint{2.064219in}{1.274352in}}%
\pgfpathlineto{\pgfqpoint{2.073906in}{1.240236in}}%
\pgfpathlineto{\pgfqpoint{2.078750in}{1.229536in}}%
\pgfpathlineto{\pgfqpoint{2.083594in}{1.231548in}}%
\pgfpathlineto{\pgfqpoint{2.088437in}{1.244779in}}%
\pgfpathlineto{\pgfqpoint{2.098125in}{1.292738in}}%
\pgfpathlineto{\pgfqpoint{2.112656in}{1.374708in}}%
\pgfpathlineto{\pgfqpoint{2.127188in}{1.436152in}}%
\pgfpathlineto{\pgfqpoint{2.132031in}{1.466693in}}%
\pgfpathlineto{\pgfqpoint{2.136875in}{1.513352in}}%
\pgfpathlineto{\pgfqpoint{2.146562in}{1.650349in}}%
\pgfpathlineto{\pgfqpoint{2.156250in}{1.790798in}}%
\pgfpathlineto{\pgfqpoint{2.175625in}{2.040102in}}%
\pgfpathlineto{\pgfqpoint{2.185313in}{2.172783in}}%
\pgfpathlineto{\pgfqpoint{2.190156in}{2.219469in}}%
\pgfpathlineto{\pgfqpoint{2.195000in}{2.244875in}}%
\pgfpathlineto{\pgfqpoint{2.199844in}{2.247931in}}%
\pgfpathlineto{\pgfqpoint{2.204687in}{2.232887in}}%
\pgfpathlineto{\pgfqpoint{2.209531in}{2.203273in}}%
\pgfpathlineto{\pgfqpoint{2.224063in}{2.098662in}}%
\pgfpathlineto{\pgfqpoint{2.228906in}{2.083140in}}%
\pgfpathlineto{\pgfqpoint{2.233750in}{2.077093in}}%
\pgfpathlineto{\pgfqpoint{2.238594in}{2.075132in}}%
\pgfpathlineto{\pgfqpoint{2.243438in}{2.061166in}}%
\pgfpathlineto{\pgfqpoint{2.248281in}{2.026020in}}%
\pgfpathlineto{\pgfqpoint{2.253125in}{1.964582in}}%
\pgfpathlineto{\pgfqpoint{2.262812in}{1.777533in}}%
\pgfpathlineto{\pgfqpoint{2.287031in}{1.227518in}}%
\pgfpathlineto{\pgfqpoint{2.291875in}{1.169719in}}%
\pgfpathlineto{\pgfqpoint{2.296719in}{1.155599in}}%
\pgfpathlineto{\pgfqpoint{2.301563in}{1.182642in}}%
\pgfpathlineto{\pgfqpoint{2.316094in}{1.326772in}}%
\pgfpathlineto{\pgfqpoint{2.320937in}{1.336235in}}%
\pgfpathlineto{\pgfqpoint{2.325781in}{1.321522in}}%
\pgfpathlineto{\pgfqpoint{2.335469in}{1.268292in}}%
\pgfpathlineto{\pgfqpoint{2.340313in}{1.252410in}}%
\pgfpathlineto{\pgfqpoint{2.345156in}{1.247278in}}%
\pgfpathlineto{\pgfqpoint{2.350000in}{1.249794in}}%
\pgfpathlineto{\pgfqpoint{2.359688in}{1.256122in}}%
\pgfpathlineto{\pgfqpoint{2.369375in}{1.256600in}}%
\pgfpathlineto{\pgfqpoint{2.374219in}{1.260580in}}%
\pgfpathlineto{\pgfqpoint{2.379062in}{1.269119in}}%
\pgfpathlineto{\pgfqpoint{2.393594in}{1.306055in}}%
\pgfpathlineto{\pgfqpoint{2.398438in}{1.310129in}}%
\pgfpathlineto{\pgfqpoint{2.408125in}{1.303439in}}%
\pgfpathlineto{\pgfqpoint{2.412969in}{1.302736in}}%
\pgfpathlineto{\pgfqpoint{2.417813in}{1.314515in}}%
\pgfpathlineto{\pgfqpoint{2.422656in}{1.338275in}}%
\pgfpathlineto{\pgfqpoint{2.437187in}{1.430470in}}%
\pgfpathlineto{\pgfqpoint{2.442031in}{1.439189in}}%
\pgfpathlineto{\pgfqpoint{2.446875in}{1.434442in}}%
\pgfpathlineto{\pgfqpoint{2.456562in}{1.411995in}}%
\pgfpathlineto{\pgfqpoint{2.466250in}{1.399223in}}%
\pgfpathlineto{\pgfqpoint{2.471094in}{1.387385in}}%
\pgfpathlineto{\pgfqpoint{2.475938in}{1.363656in}}%
\pgfpathlineto{\pgfqpoint{2.485625in}{1.280820in}}%
\pgfpathlineto{\pgfqpoint{2.490469in}{1.235172in}}%
\pgfpathlineto{\pgfqpoint{2.495313in}{1.201525in}}%
\pgfpathlineto{\pgfqpoint{2.500156in}{1.183680in}}%
\pgfpathlineto{\pgfqpoint{2.505000in}{1.181979in}}%
\pgfpathlineto{\pgfqpoint{2.509844in}{1.193511in}}%
\pgfpathlineto{\pgfqpoint{2.519531in}{1.223616in}}%
\pgfpathlineto{\pgfqpoint{2.524375in}{1.229148in}}%
\pgfpathlineto{\pgfqpoint{2.529219in}{1.226236in}}%
\pgfpathlineto{\pgfqpoint{2.534063in}{1.219179in}}%
\pgfpathlineto{\pgfqpoint{2.538906in}{1.216812in}}%
\pgfpathlineto{\pgfqpoint{2.543750in}{1.223451in}}%
\pgfpathlineto{\pgfqpoint{2.548594in}{1.241327in}}%
\pgfpathlineto{\pgfqpoint{2.558281in}{1.287415in}}%
\pgfpathlineto{\pgfqpoint{2.563125in}{1.302504in}}%
\pgfpathlineto{\pgfqpoint{2.567969in}{1.307294in}}%
\pgfpathlineto{\pgfqpoint{2.572812in}{1.305778in}}%
\pgfpathlineto{\pgfqpoint{2.577656in}{1.302679in}}%
\pgfpathlineto{\pgfqpoint{2.582500in}{1.304796in}}%
\pgfpathlineto{\pgfqpoint{2.587344in}{1.313421in}}%
\pgfpathlineto{\pgfqpoint{2.601875in}{1.351484in}}%
\pgfpathlineto{\pgfqpoint{2.606719in}{1.355171in}}%
\pgfpathlineto{\pgfqpoint{2.611563in}{1.351484in}}%
\pgfpathlineto{\pgfqpoint{2.616406in}{1.344657in}}%
\pgfpathlineto{\pgfqpoint{2.621250in}{1.335580in}}%
\pgfpathlineto{\pgfqpoint{2.626094in}{1.330673in}}%
\pgfpathlineto{\pgfqpoint{2.630937in}{1.331440in}}%
\pgfpathlineto{\pgfqpoint{2.635781in}{1.334267in}}%
\pgfpathlineto{\pgfqpoint{2.640625in}{1.335378in}}%
\pgfpathlineto{\pgfqpoint{2.645469in}{1.329385in}}%
\pgfpathlineto{\pgfqpoint{2.660000in}{1.289145in}}%
\pgfpathlineto{\pgfqpoint{2.664844in}{1.293410in}}%
\pgfpathlineto{\pgfqpoint{2.669688in}{1.313708in}}%
\pgfpathlineto{\pgfqpoint{2.679375in}{1.373799in}}%
\pgfpathlineto{\pgfqpoint{2.684219in}{1.393775in}}%
\pgfpathlineto{\pgfqpoint{2.689062in}{1.404513in}}%
\pgfpathlineto{\pgfqpoint{2.693906in}{1.409647in}}%
\pgfpathlineto{\pgfqpoint{2.698750in}{1.416972in}}%
\pgfpathlineto{\pgfqpoint{2.703594in}{1.430683in}}%
\pgfpathlineto{\pgfqpoint{2.713281in}{1.466483in}}%
\pgfpathlineto{\pgfqpoint{2.718125in}{1.479393in}}%
\pgfpathlineto{\pgfqpoint{2.732656in}{1.502415in}}%
\pgfpathlineto{\pgfqpoint{2.737500in}{1.522521in}}%
\pgfpathlineto{\pgfqpoint{2.747187in}{1.589447in}}%
\pgfpathlineto{\pgfqpoint{2.752031in}{1.622621in}}%
\pgfpathlineto{\pgfqpoint{2.756875in}{1.646146in}}%
\pgfpathlineto{\pgfqpoint{2.761719in}{1.658534in}}%
\pgfpathlineto{\pgfqpoint{2.766563in}{1.659738in}}%
\pgfpathlineto{\pgfqpoint{2.771406in}{1.658864in}}%
\pgfpathlineto{\pgfqpoint{2.776250in}{1.661291in}}%
\pgfpathlineto{\pgfqpoint{2.781094in}{1.669910in}}%
\pgfpathlineto{\pgfqpoint{2.795625in}{1.720711in}}%
\pgfpathlineto{\pgfqpoint{2.800469in}{1.731234in}}%
\pgfpathlineto{\pgfqpoint{2.805312in}{1.738597in}}%
\pgfpathlineto{\pgfqpoint{2.810156in}{1.749055in}}%
\pgfpathlineto{\pgfqpoint{2.815000in}{1.767478in}}%
\pgfpathlineto{\pgfqpoint{2.834375in}{1.865862in}}%
\pgfpathlineto{\pgfqpoint{2.844063in}{1.896303in}}%
\pgfpathlineto{\pgfqpoint{2.848906in}{1.918234in}}%
\pgfpathlineto{\pgfqpoint{2.863437in}{2.003598in}}%
\pgfpathlineto{\pgfqpoint{2.868281in}{2.010296in}}%
\pgfpathlineto{\pgfqpoint{2.873125in}{2.002116in}}%
\pgfpathlineto{\pgfqpoint{2.877969in}{1.986814in}}%
\pgfpathlineto{\pgfqpoint{2.882812in}{1.978485in}}%
\pgfpathlineto{\pgfqpoint{2.887656in}{1.986011in}}%
\pgfpathlineto{\pgfqpoint{2.892500in}{2.010525in}}%
\pgfpathlineto{\pgfqpoint{2.902188in}{2.081657in}}%
\pgfpathlineto{\pgfqpoint{2.907031in}{2.106249in}}%
\pgfpathlineto{\pgfqpoint{2.911875in}{2.115394in}}%
\pgfpathlineto{\pgfqpoint{2.916719in}{2.114336in}}%
\pgfpathlineto{\pgfqpoint{2.921562in}{2.108381in}}%
\pgfpathlineto{\pgfqpoint{2.926406in}{2.108001in}}%
\pgfpathlineto{\pgfqpoint{2.931250in}{2.117227in}}%
\pgfpathlineto{\pgfqpoint{2.936094in}{2.136179in}}%
\pgfpathlineto{\pgfqpoint{2.950625in}{2.206172in}}%
\pgfpathlineto{\pgfqpoint{2.955469in}{2.215697in}}%
\pgfpathlineto{\pgfqpoint{2.960313in}{2.217353in}}%
\pgfpathlineto{\pgfqpoint{2.965156in}{2.213321in}}%
\pgfpathlineto{\pgfqpoint{2.970000in}{2.212227in}}%
\pgfpathlineto{\pgfqpoint{2.974844in}{2.218908in}}%
\pgfpathlineto{\pgfqpoint{2.979688in}{2.233169in}}%
\pgfpathlineto{\pgfqpoint{2.989375in}{2.270214in}}%
\pgfpathlineto{\pgfqpoint{2.994219in}{2.276609in}}%
\pgfpathlineto{\pgfqpoint{2.999062in}{2.270624in}}%
\pgfpathlineto{\pgfqpoint{3.008750in}{2.243746in}}%
\pgfpathlineto{\pgfqpoint{3.013594in}{2.239107in}}%
\pgfpathlineto{\pgfqpoint{3.018438in}{2.244215in}}%
\pgfpathlineto{\pgfqpoint{3.023281in}{2.257996in}}%
\pgfpathlineto{\pgfqpoint{3.037813in}{2.307922in}}%
\pgfpathlineto{\pgfqpoint{3.047500in}{2.340410in}}%
\pgfpathlineto{\pgfqpoint{3.057187in}{2.374774in}}%
\pgfpathlineto{\pgfqpoint{3.062031in}{2.384667in}}%
\pgfpathlineto{\pgfqpoint{3.066875in}{2.384065in}}%
\pgfpathlineto{\pgfqpoint{3.071719in}{2.373036in}}%
\pgfpathlineto{\pgfqpoint{3.081406in}{2.337032in}}%
\pgfpathlineto{\pgfqpoint{3.086250in}{2.329156in}}%
\pgfpathlineto{\pgfqpoint{3.091094in}{2.335971in}}%
\pgfpathlineto{\pgfqpoint{3.095938in}{2.356281in}}%
\pgfpathlineto{\pgfqpoint{3.105625in}{2.414119in}}%
\pgfpathlineto{\pgfqpoint{3.110469in}{2.430752in}}%
\pgfpathlineto{\pgfqpoint{3.115312in}{2.431974in}}%
\pgfpathlineto{\pgfqpoint{3.120156in}{2.420018in}}%
\pgfpathlineto{\pgfqpoint{3.129844in}{2.387380in}}%
\pgfpathlineto{\pgfqpoint{3.134688in}{2.382883in}}%
\pgfpathlineto{\pgfqpoint{3.139531in}{2.391220in}}%
\pgfpathlineto{\pgfqpoint{3.154063in}{2.451569in}}%
\pgfpathlineto{\pgfqpoint{3.158906in}{2.459872in}}%
\pgfpathlineto{\pgfqpoint{3.163750in}{2.457730in}}%
\pgfpathlineto{\pgfqpoint{3.168594in}{2.449389in}}%
\pgfpathlineto{\pgfqpoint{3.173437in}{2.445475in}}%
\pgfpathlineto{\pgfqpoint{3.178281in}{2.451389in}}%
\pgfpathlineto{\pgfqpoint{3.183125in}{2.465994in}}%
\pgfpathlineto{\pgfqpoint{3.187969in}{2.484182in}}%
\pgfpathlineto{\pgfqpoint{3.192813in}{2.496877in}}%
\pgfpathlineto{\pgfqpoint{3.197656in}{2.496198in}}%
\pgfpathlineto{\pgfqpoint{3.202500in}{2.483213in}}%
\pgfpathlineto{\pgfqpoint{3.207344in}{2.464108in}}%
\pgfpathlineto{\pgfqpoint{3.212188in}{2.450235in}}%
\pgfpathlineto{\pgfqpoint{3.217031in}{2.449133in}}%
\pgfpathlineto{\pgfqpoint{3.221875in}{2.463269in}}%
\pgfpathlineto{\pgfqpoint{3.231562in}{2.506364in}}%
\pgfpathlineto{\pgfqpoint{3.236406in}{2.520040in}}%
\pgfpathlineto{\pgfqpoint{3.241250in}{2.523491in}}%
\pgfpathlineto{\pgfqpoint{3.250938in}{2.523360in}}%
\pgfpathlineto{\pgfqpoint{3.255781in}{2.530210in}}%
\pgfpathlineto{\pgfqpoint{3.270313in}{2.565026in}}%
\pgfpathlineto{\pgfqpoint{3.275156in}{2.567133in}}%
\pgfpathlineto{\pgfqpoint{3.280000in}{2.561127in}}%
\pgfpathlineto{\pgfqpoint{3.289687in}{2.544533in}}%
\pgfpathlineto{\pgfqpoint{3.294531in}{2.542671in}}%
\pgfpathlineto{\pgfqpoint{3.309063in}{2.548784in}}%
\pgfpathlineto{\pgfqpoint{3.313906in}{2.546687in}}%
\pgfpathlineto{\pgfqpoint{3.318750in}{2.547023in}}%
\pgfpathlineto{\pgfqpoint{3.323594in}{2.553941in}}%
\pgfpathlineto{\pgfqpoint{3.333281in}{2.576961in}}%
\pgfpathlineto{\pgfqpoint{3.338125in}{2.583117in}}%
\pgfpathlineto{\pgfqpoint{3.342969in}{2.579784in}}%
\pgfpathlineto{\pgfqpoint{3.347812in}{2.573343in}}%
\pgfpathlineto{\pgfqpoint{3.352656in}{2.568811in}}%
\pgfpathlineto{\pgfqpoint{3.357500in}{2.573727in}}%
\pgfpathlineto{\pgfqpoint{3.367188in}{2.600847in}}%
\pgfpathlineto{\pgfqpoint{3.372031in}{2.608577in}}%
\pgfpathlineto{\pgfqpoint{3.376875in}{2.604567in}}%
\pgfpathlineto{\pgfqpoint{3.391406in}{2.564620in}}%
\pgfpathlineto{\pgfqpoint{3.396250in}{2.560283in}}%
\pgfpathlineto{\pgfqpoint{3.405938in}{2.562527in}}%
\pgfpathlineto{\pgfqpoint{3.410781in}{2.560220in}}%
\pgfpathlineto{\pgfqpoint{3.420469in}{2.547241in}}%
\pgfpathlineto{\pgfqpoint{3.425312in}{2.545274in}}%
\pgfpathlineto{\pgfqpoint{3.430156in}{2.548190in}}%
\pgfpathlineto{\pgfqpoint{3.435000in}{2.557054in}}%
\pgfpathlineto{\pgfqpoint{3.444688in}{2.589351in}}%
\pgfpathlineto{\pgfqpoint{3.449531in}{2.607468in}}%
\pgfpathlineto{\pgfqpoint{3.454375in}{2.619652in}}%
\pgfpathlineto{\pgfqpoint{3.459219in}{2.620916in}}%
\pgfpathlineto{\pgfqpoint{3.464063in}{2.610214in}}%
\pgfpathlineto{\pgfqpoint{3.468906in}{2.590412in}}%
\pgfpathlineto{\pgfqpoint{3.473750in}{2.565678in}}%
\pgfpathlineto{\pgfqpoint{3.478594in}{2.547873in}}%
\pgfpathlineto{\pgfqpoint{3.483437in}{2.541081in}}%
\pgfpathlineto{\pgfqpoint{3.488281in}{2.549295in}}%
\pgfpathlineto{\pgfqpoint{3.502813in}{2.596847in}}%
\pgfpathlineto{\pgfqpoint{3.507656in}{2.600034in}}%
\pgfpathlineto{\pgfqpoint{3.512500in}{2.593011in}}%
\pgfpathlineto{\pgfqpoint{3.522188in}{2.572989in}}%
\pgfpathlineto{\pgfqpoint{3.527031in}{2.568629in}}%
\pgfpathlineto{\pgfqpoint{3.531875in}{2.572674in}}%
\pgfpathlineto{\pgfqpoint{3.541562in}{2.597377in}}%
\pgfpathlineto{\pgfqpoint{3.551250in}{2.621588in}}%
\pgfpathlineto{\pgfqpoint{3.560938in}{2.638082in}}%
\pgfpathlineto{\pgfqpoint{3.565781in}{2.641344in}}%
\pgfpathlineto{\pgfqpoint{3.570625in}{2.639965in}}%
\pgfpathlineto{\pgfqpoint{3.575469in}{2.631209in}}%
\pgfpathlineto{\pgfqpoint{3.580313in}{2.617397in}}%
\pgfpathlineto{\pgfqpoint{3.594844in}{2.563210in}}%
\pgfpathlineto{\pgfqpoint{3.599687in}{2.553388in}}%
\pgfpathlineto{\pgfqpoint{3.604531in}{2.555467in}}%
\pgfpathlineto{\pgfqpoint{3.609375in}{2.569565in}}%
\pgfpathlineto{\pgfqpoint{3.619063in}{2.607925in}}%
\pgfpathlineto{\pgfqpoint{3.623906in}{2.617531in}}%
\pgfpathlineto{\pgfqpoint{3.628750in}{2.614880in}}%
\pgfpathlineto{\pgfqpoint{3.638438in}{2.589569in}}%
\pgfpathlineto{\pgfqpoint{3.643281in}{2.586552in}}%
\pgfpathlineto{\pgfqpoint{3.648125in}{2.597652in}}%
\pgfpathlineto{\pgfqpoint{3.657812in}{2.643217in}}%
\pgfpathlineto{\pgfqpoint{3.662656in}{2.658902in}}%
\pgfpathlineto{\pgfqpoint{3.667500in}{2.661536in}}%
\pgfpathlineto{\pgfqpoint{3.672344in}{2.653266in}}%
\pgfpathlineto{\pgfqpoint{3.682031in}{2.628540in}}%
\pgfpathlineto{\pgfqpoint{3.686875in}{2.622141in}}%
\pgfpathlineto{\pgfqpoint{3.691719in}{2.621117in}}%
\pgfpathlineto{\pgfqpoint{3.701406in}{2.628599in}}%
\pgfpathlineto{\pgfqpoint{3.706250in}{2.632687in}}%
\pgfpathlineto{\pgfqpoint{3.711094in}{2.634485in}}%
\pgfpathlineto{\pgfqpoint{3.715937in}{2.638124in}}%
\pgfpathlineto{\pgfqpoint{3.720781in}{2.645042in}}%
\pgfpathlineto{\pgfqpoint{3.740156in}{2.688366in}}%
\pgfpathlineto{\pgfqpoint{3.749844in}{2.686318in}}%
\pgfpathlineto{\pgfqpoint{3.754688in}{2.689316in}}%
\pgfpathlineto{\pgfqpoint{3.759531in}{2.703005in}}%
\pgfpathlineto{\pgfqpoint{3.769219in}{2.751364in}}%
\pgfpathlineto{\pgfqpoint{3.774062in}{2.764748in}}%
\pgfpathlineto{\pgfqpoint{3.778906in}{2.764168in}}%
\pgfpathlineto{\pgfqpoint{3.788594in}{2.739413in}}%
\pgfpathlineto{\pgfqpoint{3.793438in}{2.735125in}}%
\pgfpathlineto{\pgfqpoint{3.798281in}{2.743705in}}%
\pgfpathlineto{\pgfqpoint{3.807969in}{2.779822in}}%
\pgfpathlineto{\pgfqpoint{3.812813in}{2.791410in}}%
\pgfpathlineto{\pgfqpoint{3.817656in}{2.795617in}}%
\pgfpathlineto{\pgfqpoint{3.827344in}{2.793183in}}%
\pgfpathlineto{\pgfqpoint{3.832187in}{2.797492in}}%
\pgfpathlineto{\pgfqpoint{3.837031in}{2.808686in}}%
\pgfpathlineto{\pgfqpoint{3.846719in}{2.845392in}}%
\pgfpathlineto{\pgfqpoint{3.856406in}{2.877491in}}%
\pgfpathlineto{\pgfqpoint{3.861250in}{2.885013in}}%
\pgfpathlineto{\pgfqpoint{3.866094in}{2.883689in}}%
\pgfpathlineto{\pgfqpoint{3.870938in}{2.878758in}}%
\pgfpathlineto{\pgfqpoint{3.885469in}{2.854523in}}%
\pgfpathlineto{\pgfqpoint{3.895156in}{2.841298in}}%
\pgfpathlineto{\pgfqpoint{3.900000in}{2.836146in}}%
\pgfpathlineto{\pgfqpoint{3.914531in}{2.825665in}}%
\pgfpathlineto{\pgfqpoint{3.924219in}{2.815420in}}%
\pgfpathlineto{\pgfqpoint{3.929063in}{2.811004in}}%
\pgfpathlineto{\pgfqpoint{3.938750in}{2.806396in}}%
\pgfpathlineto{\pgfqpoint{3.943594in}{2.798853in}}%
\pgfpathlineto{\pgfqpoint{3.948438in}{2.785721in}}%
\pgfpathlineto{\pgfqpoint{3.958125in}{2.748608in}}%
\pgfpathlineto{\pgfqpoint{3.962969in}{2.737081in}}%
\pgfpathlineto{\pgfqpoint{3.967812in}{2.734692in}}%
\pgfpathlineto{\pgfqpoint{3.972656in}{2.736188in}}%
\pgfpathlineto{\pgfqpoint{3.977500in}{2.733868in}}%
\pgfpathlineto{\pgfqpoint{3.982344in}{2.718881in}}%
\pgfpathlineto{\pgfqpoint{3.996875in}{2.643512in}}%
\pgfpathlineto{\pgfqpoint{4.001719in}{2.640329in}}%
\pgfpathlineto{\pgfqpoint{4.006563in}{2.650471in}}%
\pgfpathlineto{\pgfqpoint{4.016250in}{2.679578in}}%
\pgfpathlineto{\pgfqpoint{4.021094in}{2.686194in}}%
\pgfpathlineto{\pgfqpoint{4.025938in}{2.687892in}}%
\pgfpathlineto{\pgfqpoint{4.030781in}{2.692780in}}%
\pgfpathlineto{\pgfqpoint{4.040469in}{2.707139in}}%
\pgfpathlineto{\pgfqpoint{4.045312in}{2.711191in}}%
\pgfpathlineto{\pgfqpoint{4.050156in}{2.707815in}}%
\pgfpathlineto{\pgfqpoint{4.055000in}{2.700824in}}%
\pgfpathlineto{\pgfqpoint{4.064687in}{2.682775in}}%
\pgfpathlineto{\pgfqpoint{4.069531in}{2.675867in}}%
\pgfpathlineto{\pgfqpoint{4.074375in}{2.663149in}}%
\pgfpathlineto{\pgfqpoint{4.079219in}{2.645659in}}%
\pgfpathlineto{\pgfqpoint{4.093750in}{2.579638in}}%
\pgfpathlineto{\pgfqpoint{4.103438in}{2.550416in}}%
\pgfpathlineto{\pgfqpoint{4.113125in}{2.524748in}}%
\pgfpathlineto{\pgfqpoint{4.117969in}{2.515356in}}%
\pgfpathlineto{\pgfqpoint{4.122813in}{2.514041in}}%
\pgfpathlineto{\pgfqpoint{4.127656in}{2.518163in}}%
\pgfpathlineto{\pgfqpoint{4.137344in}{2.531708in}}%
\pgfpathlineto{\pgfqpoint{4.142188in}{2.532733in}}%
\pgfpathlineto{\pgfqpoint{4.147031in}{2.531171in}}%
\pgfpathlineto{\pgfqpoint{4.151875in}{2.532172in}}%
\pgfpathlineto{\pgfqpoint{4.156719in}{2.538090in}}%
\pgfpathlineto{\pgfqpoint{4.161563in}{2.552127in}}%
\pgfpathlineto{\pgfqpoint{4.171250in}{2.597575in}}%
\pgfpathlineto{\pgfqpoint{4.195469in}{2.737524in}}%
\pgfpathlineto{\pgfqpoint{4.200312in}{2.749486in}}%
\pgfpathlineto{\pgfqpoint{4.205156in}{2.749121in}}%
\pgfpathlineto{\pgfqpoint{4.210000in}{2.736082in}}%
\pgfpathlineto{\pgfqpoint{4.219688in}{2.689105in}}%
\pgfpathlineto{\pgfqpoint{4.229375in}{2.635740in}}%
\pgfpathlineto{\pgfqpoint{4.239063in}{2.597424in}}%
\pgfpathlineto{\pgfqpoint{4.243906in}{2.587150in}}%
\pgfpathlineto{\pgfqpoint{4.248750in}{2.583096in}}%
\pgfpathlineto{\pgfqpoint{4.258438in}{2.579575in}}%
\pgfpathlineto{\pgfqpoint{4.263281in}{2.574537in}}%
\pgfpathlineto{\pgfqpoint{4.268125in}{2.563695in}}%
\pgfpathlineto{\pgfqpoint{4.272969in}{2.544191in}}%
\pgfpathlineto{\pgfqpoint{4.282656in}{2.487319in}}%
\pgfpathlineto{\pgfqpoint{4.297187in}{2.389876in}}%
\pgfpathlineto{\pgfqpoint{4.302031in}{2.369677in}}%
\pgfpathlineto{\pgfqpoint{4.306875in}{2.360258in}}%
\pgfpathlineto{\pgfqpoint{4.311719in}{2.362353in}}%
\pgfpathlineto{\pgfqpoint{4.316562in}{2.373624in}}%
\pgfpathlineto{\pgfqpoint{4.331094in}{2.415503in}}%
\pgfpathlineto{\pgfqpoint{4.335938in}{2.422763in}}%
\pgfpathlineto{\pgfqpoint{4.340781in}{2.424741in}}%
\pgfpathlineto{\pgfqpoint{4.345625in}{2.424157in}}%
\pgfpathlineto{\pgfqpoint{4.350469in}{2.428859in}}%
\pgfpathlineto{\pgfqpoint{4.355313in}{2.444555in}}%
\pgfpathlineto{\pgfqpoint{4.360156in}{2.475224in}}%
\pgfpathlineto{\pgfqpoint{4.365000in}{2.522150in}}%
\pgfpathlineto{\pgfqpoint{4.379531in}{2.712068in}}%
\pgfpathlineto{\pgfqpoint{4.394063in}{2.931444in}}%
\pgfpathlineto{\pgfqpoint{4.403750in}{3.082892in}}%
\pgfpathlineto{\pgfqpoint{4.408594in}{3.139703in}}%
\pgfpathlineto{\pgfqpoint{4.413437in}{3.179269in}}%
\pgfpathlineto{\pgfqpoint{4.432812in}{3.286944in}}%
\pgfpathlineto{\pgfqpoint{4.437656in}{3.302132in}}%
\pgfpathlineto{\pgfqpoint{4.442500in}{3.301049in}}%
\pgfpathlineto{\pgfqpoint{4.457031in}{3.244018in}}%
\pgfpathlineto{\pgfqpoint{4.461875in}{3.241426in}}%
\pgfpathlineto{\pgfqpoint{4.466719in}{3.244291in}}%
\pgfpathlineto{\pgfqpoint{4.471563in}{3.239360in}}%
\pgfpathlineto{\pgfqpoint{4.476406in}{3.208923in}}%
\pgfpathlineto{\pgfqpoint{4.481250in}{3.147180in}}%
\pgfpathlineto{\pgfqpoint{4.490938in}{2.953461in}}%
\pgfpathlineto{\pgfqpoint{4.515156in}{2.399889in}}%
\pgfpathlineto{\pgfqpoint{4.520000in}{2.337509in}}%
\pgfpathlineto{\pgfqpoint{4.524844in}{2.314944in}}%
\pgfpathlineto{\pgfqpoint{4.529688in}{2.327600in}}%
\pgfpathlineto{\pgfqpoint{4.539375in}{2.392846in}}%
\pgfpathlineto{\pgfqpoint{4.544219in}{2.413754in}}%
\pgfpathlineto{\pgfqpoint{4.549062in}{2.417243in}}%
\pgfpathlineto{\pgfqpoint{4.553906in}{2.405944in}}%
\pgfpathlineto{\pgfqpoint{4.563594in}{2.366467in}}%
\pgfpathlineto{\pgfqpoint{4.573281in}{2.332833in}}%
\pgfpathlineto{\pgfqpoint{4.578125in}{2.323428in}}%
\pgfpathlineto{\pgfqpoint{4.582969in}{2.324196in}}%
\pgfpathlineto{\pgfqpoint{4.587813in}{2.332641in}}%
\pgfpathlineto{\pgfqpoint{4.592656in}{2.346259in}}%
\pgfpathlineto{\pgfqpoint{4.597500in}{2.356495in}}%
\pgfpathlineto{\pgfqpoint{4.602344in}{2.360628in}}%
\pgfpathlineto{\pgfqpoint{4.607188in}{2.355525in}}%
\pgfpathlineto{\pgfqpoint{4.616875in}{2.337801in}}%
\pgfpathlineto{\pgfqpoint{4.621719in}{2.335739in}}%
\pgfpathlineto{\pgfqpoint{4.626563in}{2.339839in}}%
\pgfpathlineto{\pgfqpoint{4.636250in}{2.356401in}}%
\pgfpathlineto{\pgfqpoint{4.645938in}{2.378734in}}%
\pgfpathlineto{\pgfqpoint{4.655625in}{2.422931in}}%
\pgfpathlineto{\pgfqpoint{4.660469in}{2.447696in}}%
\pgfpathlineto{\pgfqpoint{4.665312in}{2.466104in}}%
\pgfpathlineto{\pgfqpoint{4.670156in}{2.472827in}}%
\pgfpathlineto{\pgfqpoint{4.675000in}{2.464363in}}%
\pgfpathlineto{\pgfqpoint{4.679844in}{2.445774in}}%
\pgfpathlineto{\pgfqpoint{4.689531in}{2.399828in}}%
\pgfpathlineto{\pgfqpoint{4.699219in}{2.369151in}}%
\pgfpathlineto{\pgfqpoint{4.704063in}{2.353213in}}%
\pgfpathlineto{\pgfqpoint{4.708906in}{2.331574in}}%
\pgfpathlineto{\pgfqpoint{4.718594in}{2.261466in}}%
\pgfpathlineto{\pgfqpoint{4.728281in}{2.195280in}}%
\pgfpathlineto{\pgfqpoint{4.733125in}{2.181309in}}%
\pgfpathlineto{\pgfqpoint{4.737969in}{2.181384in}}%
\pgfpathlineto{\pgfqpoint{4.752500in}{2.215718in}}%
\pgfpathlineto{\pgfqpoint{4.757344in}{2.221631in}}%
\pgfpathlineto{\pgfqpoint{4.762188in}{2.221647in}}%
\pgfpathlineto{\pgfqpoint{4.767031in}{2.224254in}}%
\pgfpathlineto{\pgfqpoint{4.771875in}{2.230858in}}%
\pgfpathlineto{\pgfqpoint{4.776719in}{2.243414in}}%
\pgfpathlineto{\pgfqpoint{4.791250in}{2.296915in}}%
\pgfpathlineto{\pgfqpoint{4.796094in}{2.305554in}}%
\pgfpathlineto{\pgfqpoint{4.800937in}{2.308003in}}%
\pgfpathlineto{\pgfqpoint{4.805781in}{2.305394in}}%
\pgfpathlineto{\pgfqpoint{4.810625in}{2.304755in}}%
\pgfpathlineto{\pgfqpoint{4.815469in}{2.308816in}}%
\pgfpathlineto{\pgfqpoint{4.825156in}{2.325294in}}%
\pgfpathlineto{\pgfqpoint{4.830000in}{2.328405in}}%
\pgfpathlineto{\pgfqpoint{4.834844in}{2.323155in}}%
\pgfpathlineto{\pgfqpoint{4.839688in}{2.308958in}}%
\pgfpathlineto{\pgfqpoint{4.849375in}{2.268286in}}%
\pgfpathlineto{\pgfqpoint{4.873594in}{2.153819in}}%
\pgfpathlineto{\pgfqpoint{4.878438in}{2.141650in}}%
\pgfpathlineto{\pgfqpoint{4.883281in}{2.135997in}}%
\pgfpathlineto{\pgfqpoint{4.888125in}{2.135692in}}%
\pgfpathlineto{\pgfqpoint{4.892969in}{2.136545in}}%
\pgfpathlineto{\pgfqpoint{4.902656in}{2.142531in}}%
\pgfpathlineto{\pgfqpoint{4.907500in}{2.150821in}}%
\pgfpathlineto{\pgfqpoint{4.922031in}{2.190930in}}%
\pgfpathlineto{\pgfqpoint{4.926875in}{2.192977in}}%
\pgfpathlineto{\pgfqpoint{4.931719in}{2.183185in}}%
\pgfpathlineto{\pgfqpoint{4.941406in}{2.148776in}}%
\pgfpathlineto{\pgfqpoint{4.946250in}{2.138067in}}%
\pgfpathlineto{\pgfqpoint{4.951094in}{2.135417in}}%
\pgfpathlineto{\pgfqpoint{4.960781in}{2.145522in}}%
\pgfpathlineto{\pgfqpoint{4.965625in}{2.150834in}}%
\pgfpathlineto{\pgfqpoint{4.970469in}{2.153147in}}%
\pgfpathlineto{\pgfqpoint{4.975313in}{2.158537in}}%
\pgfpathlineto{\pgfqpoint{4.980156in}{2.170395in}}%
\pgfpathlineto{\pgfqpoint{4.985000in}{2.188216in}}%
\pgfpathlineto{\pgfqpoint{4.994688in}{2.232244in}}%
\pgfpathlineto{\pgfqpoint{4.999531in}{2.247579in}}%
\pgfpathlineto{\pgfqpoint{5.004375in}{2.252357in}}%
\pgfpathlineto{\pgfqpoint{5.009219in}{2.253161in}}%
\pgfpathlineto{\pgfqpoint{5.014063in}{2.251591in}}%
\pgfpathlineto{\pgfqpoint{5.023750in}{2.254692in}}%
\pgfpathlineto{\pgfqpoint{5.028594in}{2.253904in}}%
\pgfpathlineto{\pgfqpoint{5.033437in}{2.248705in}}%
\pgfpathlineto{\pgfqpoint{5.038281in}{2.236828in}}%
\pgfpathlineto{\pgfqpoint{5.047969in}{2.208982in}}%
\pgfpathlineto{\pgfqpoint{5.052812in}{2.204481in}}%
\pgfpathlineto{\pgfqpoint{5.057656in}{2.207274in}}%
\pgfpathlineto{\pgfqpoint{5.067344in}{2.224485in}}%
\pgfpathlineto{\pgfqpoint{5.072188in}{2.229251in}}%
\pgfpathlineto{\pgfqpoint{5.077031in}{2.228516in}}%
\pgfpathlineto{\pgfqpoint{5.091563in}{2.208686in}}%
\pgfpathlineto{\pgfqpoint{5.096406in}{2.206949in}}%
\pgfpathlineto{\pgfqpoint{5.101250in}{2.212585in}}%
\pgfpathlineto{\pgfqpoint{5.106094in}{2.224513in}}%
\pgfpathlineto{\pgfqpoint{5.115781in}{2.259069in}}%
\pgfpathlineto{\pgfqpoint{5.125469in}{2.291270in}}%
\pgfpathlineto{\pgfqpoint{5.130313in}{2.300972in}}%
\pgfpathlineto{\pgfqpoint{5.135156in}{2.303325in}}%
\pgfpathlineto{\pgfqpoint{5.140000in}{2.296912in}}%
\pgfpathlineto{\pgfqpoint{5.144844in}{2.282851in}}%
\pgfpathlineto{\pgfqpoint{5.154531in}{2.250384in}}%
\pgfpathlineto{\pgfqpoint{5.159375in}{2.247115in}}%
\pgfpathlineto{\pgfqpoint{5.164219in}{2.256225in}}%
\pgfpathlineto{\pgfqpoint{5.169062in}{2.273841in}}%
\pgfpathlineto{\pgfqpoint{5.173906in}{2.287152in}}%
\pgfpathlineto{\pgfqpoint{5.178750in}{2.289828in}}%
\pgfpathlineto{\pgfqpoint{5.183594in}{2.283296in}}%
\pgfpathlineto{\pgfqpoint{5.188438in}{2.274146in}}%
\pgfpathlineto{\pgfqpoint{5.193281in}{2.274108in}}%
\pgfpathlineto{\pgfqpoint{5.198125in}{2.288466in}}%
\pgfpathlineto{\pgfqpoint{5.207813in}{2.337588in}}%
\pgfpathlineto{\pgfqpoint{5.212656in}{2.353123in}}%
\pgfpathlineto{\pgfqpoint{5.217500in}{2.354878in}}%
\pgfpathlineto{\pgfqpoint{5.222344in}{2.347906in}}%
\pgfpathlineto{\pgfqpoint{5.227188in}{2.338514in}}%
\pgfpathlineto{\pgfqpoint{5.232031in}{2.331717in}}%
\pgfpathlineto{\pgfqpoint{5.246563in}{2.327123in}}%
\pgfpathlineto{\pgfqpoint{5.256250in}{2.320738in}}%
\pgfpathlineto{\pgfqpoint{5.261094in}{2.319221in}}%
\pgfpathlineto{\pgfqpoint{5.280469in}{2.317450in}}%
\pgfpathlineto{\pgfqpoint{5.295000in}{2.321544in}}%
\pgfpathlineto{\pgfqpoint{5.299844in}{2.319647in}}%
\pgfpathlineto{\pgfqpoint{5.309531in}{2.305106in}}%
\pgfpathlineto{\pgfqpoint{5.314375in}{2.301222in}}%
\pgfpathlineto{\pgfqpoint{5.319219in}{2.300770in}}%
\pgfpathlineto{\pgfqpoint{5.324063in}{2.303727in}}%
\pgfpathlineto{\pgfqpoint{5.328906in}{2.303702in}}%
\pgfpathlineto{\pgfqpoint{5.338594in}{2.297707in}}%
\pgfpathlineto{\pgfqpoint{5.343438in}{2.299778in}}%
\pgfpathlineto{\pgfqpoint{5.348281in}{2.309766in}}%
\pgfpathlineto{\pgfqpoint{5.353125in}{2.325275in}}%
\pgfpathlineto{\pgfqpoint{5.357969in}{2.336161in}}%
\pgfpathlineto{\pgfqpoint{5.362813in}{2.337364in}}%
\pgfpathlineto{\pgfqpoint{5.367656in}{2.327617in}}%
\pgfpathlineto{\pgfqpoint{5.377344in}{2.294308in}}%
\pgfpathlineto{\pgfqpoint{5.382187in}{2.288482in}}%
\pgfpathlineto{\pgfqpoint{5.387031in}{2.294355in}}%
\pgfpathlineto{\pgfqpoint{5.396719in}{2.322473in}}%
\pgfpathlineto{\pgfqpoint{5.401562in}{2.330287in}}%
\pgfpathlineto{\pgfqpoint{5.406406in}{2.326116in}}%
\pgfpathlineto{\pgfqpoint{5.411250in}{2.311236in}}%
\pgfpathlineto{\pgfqpoint{5.425781in}{2.249240in}}%
\pgfpathlineto{\pgfqpoint{5.430625in}{2.239825in}}%
\pgfpathlineto{\pgfqpoint{5.435469in}{2.240697in}}%
\pgfpathlineto{\pgfqpoint{5.440313in}{2.249347in}}%
\pgfpathlineto{\pgfqpoint{5.450000in}{2.272043in}}%
\pgfpathlineto{\pgfqpoint{5.454844in}{2.277014in}}%
\pgfpathlineto{\pgfqpoint{5.459688in}{2.271623in}}%
\pgfpathlineto{\pgfqpoint{5.464531in}{2.259628in}}%
\pgfpathlineto{\pgfqpoint{5.474219in}{2.228349in}}%
\pgfpathlineto{\pgfqpoint{5.479063in}{2.217990in}}%
\pgfpathlineto{\pgfqpoint{5.483906in}{2.217616in}}%
\pgfpathlineto{\pgfqpoint{5.488750in}{2.224747in}}%
\pgfpathlineto{\pgfqpoint{5.498438in}{2.247956in}}%
\pgfpathlineto{\pgfqpoint{5.503281in}{2.252811in}}%
\pgfpathlineto{\pgfqpoint{5.508125in}{2.251687in}}%
\pgfpathlineto{\pgfqpoint{5.537187in}{2.203203in}}%
\pgfpathlineto{\pgfqpoint{5.556563in}{2.145155in}}%
\pgfpathlineto{\pgfqpoint{5.561406in}{2.146656in}}%
\pgfpathlineto{\pgfqpoint{5.566250in}{2.156495in}}%
\pgfpathlineto{\pgfqpoint{5.575938in}{2.185712in}}%
\pgfpathlineto{\pgfqpoint{5.580781in}{2.189784in}}%
\pgfpathlineto{\pgfqpoint{5.585625in}{2.180558in}}%
\pgfpathlineto{\pgfqpoint{5.590469in}{2.159470in}}%
\pgfpathlineto{\pgfqpoint{5.600156in}{2.109338in}}%
\pgfpathlineto{\pgfqpoint{5.605000in}{2.093432in}}%
\pgfpathlineto{\pgfqpoint{5.619531in}{2.067980in}}%
\pgfpathlineto{\pgfqpoint{5.624375in}{2.059708in}}%
\pgfpathlineto{\pgfqpoint{5.629219in}{2.055570in}}%
\pgfpathlineto{\pgfqpoint{5.634062in}{2.059359in}}%
\pgfpathlineto{\pgfqpoint{5.643750in}{2.077582in}}%
\pgfpathlineto{\pgfqpoint{5.648594in}{2.076461in}}%
\pgfpathlineto{\pgfqpoint{5.653437in}{2.062465in}}%
\pgfpathlineto{\pgfqpoint{5.663125in}{2.020288in}}%
\pgfpathlineto{\pgfqpoint{5.667969in}{2.014043in}}%
\pgfpathlineto{\pgfqpoint{5.672813in}{2.020787in}}%
\pgfpathlineto{\pgfqpoint{5.677656in}{2.032887in}}%
\pgfpathlineto{\pgfqpoint{5.682500in}{2.037562in}}%
\pgfpathlineto{\pgfqpoint{5.687344in}{2.028437in}}%
\pgfpathlineto{\pgfqpoint{5.697031in}{1.984250in}}%
\pgfpathlineto{\pgfqpoint{5.701875in}{1.972227in}}%
\pgfpathlineto{\pgfqpoint{5.706719in}{1.974478in}}%
\pgfpathlineto{\pgfqpoint{5.716406in}{2.000516in}}%
\pgfpathlineto{\pgfqpoint{5.721250in}{2.007175in}}%
\pgfpathlineto{\pgfqpoint{5.726094in}{2.001614in}}%
\pgfpathlineto{\pgfqpoint{5.740625in}{1.961806in}}%
\pgfpathlineto{\pgfqpoint{5.745469in}{1.956662in}}%
\pgfpathlineto{\pgfqpoint{5.750312in}{1.956006in}}%
\pgfpathlineto{\pgfqpoint{5.755156in}{1.960089in}}%
\pgfpathlineto{\pgfqpoint{5.764844in}{1.974737in}}%
\pgfpathlineto{\pgfqpoint{5.764844in}{1.974737in}}%
\pgfusepath{stroke}%
\end{pgfscope}%
\begin{pgfscope}%
\pgfpathrectangle{\pgfqpoint{0.800000in}{0.528000in}}{\pgfqpoint{4.960000in}{3.696000in}}%
\pgfusepath{clip}%
\pgfsetrectcap%
\pgfsetroundjoin%
\pgfsetlinewidth{1.505625pt}%
\definecolor{currentstroke}{rgb}{1.000000,0.498039,0.054902}%
\pgfsetstrokecolor{currentstroke}%
\pgfsetdash{}{0pt}%
\pgfpathmoveto{\pgfqpoint{0.795156in}{1.308026in}}%
\pgfpathlineto{\pgfqpoint{0.800000in}{1.261696in}}%
\pgfpathlineto{\pgfqpoint{0.804844in}{1.189850in}}%
\pgfpathlineto{\pgfqpoint{0.829063in}{0.764732in}}%
\pgfpathlineto{\pgfqpoint{0.833906in}{0.716448in}}%
\pgfpathlineto{\pgfqpoint{0.838750in}{0.696000in}}%
\pgfpathlineto{\pgfqpoint{0.843594in}{0.709334in}}%
\pgfpathlineto{\pgfqpoint{0.848438in}{0.756228in}}%
\pgfpathlineto{\pgfqpoint{0.858125in}{0.926977in}}%
\pgfpathlineto{\pgfqpoint{0.867812in}{1.105559in}}%
\pgfpathlineto{\pgfqpoint{0.872656in}{1.165046in}}%
\pgfpathlineto{\pgfqpoint{0.877500in}{1.195286in}}%
\pgfpathlineto{\pgfqpoint{0.882344in}{1.198713in}}%
\pgfpathlineto{\pgfqpoint{0.887188in}{1.181093in}}%
\pgfpathlineto{\pgfqpoint{0.901719in}{1.081420in}}%
\pgfpathlineto{\pgfqpoint{0.906563in}{1.063306in}}%
\pgfpathlineto{\pgfqpoint{0.911406in}{1.059130in}}%
\pgfpathlineto{\pgfqpoint{0.916250in}{1.069976in}}%
\pgfpathlineto{\pgfqpoint{0.921094in}{1.094168in}}%
\pgfpathlineto{\pgfqpoint{0.925937in}{1.128227in}}%
\pgfpathlineto{\pgfqpoint{0.935625in}{1.220977in}}%
\pgfpathlineto{\pgfqpoint{0.945312in}{1.318976in}}%
\pgfpathlineto{\pgfqpoint{0.950156in}{1.349256in}}%
\pgfpathlineto{\pgfqpoint{0.955000in}{1.352848in}}%
\pgfpathlineto{\pgfqpoint{0.959844in}{1.328932in}}%
\pgfpathlineto{\pgfqpoint{0.964688in}{1.280001in}}%
\pgfpathlineto{\pgfqpoint{0.974375in}{1.140636in}}%
\pgfpathlineto{\pgfqpoint{0.984063in}{0.997990in}}%
\pgfpathlineto{\pgfqpoint{0.988906in}{0.941568in}}%
\pgfpathlineto{\pgfqpoint{0.993750in}{0.903272in}}%
\pgfpathlineto{\pgfqpoint{0.998594in}{0.895282in}}%
\pgfpathlineto{\pgfqpoint{1.003437in}{0.921168in}}%
\pgfpathlineto{\pgfqpoint{1.008281in}{0.983474in}}%
\pgfpathlineto{\pgfqpoint{1.032500in}{1.408773in}}%
\pgfpathlineto{\pgfqpoint{1.037344in}{1.458106in}}%
\pgfpathlineto{\pgfqpoint{1.042188in}{1.495375in}}%
\pgfpathlineto{\pgfqpoint{1.047031in}{1.512905in}}%
\pgfpathlineto{\pgfqpoint{1.051875in}{1.507576in}}%
\pgfpathlineto{\pgfqpoint{1.056719in}{1.471225in}}%
\pgfpathlineto{\pgfqpoint{1.071250in}{1.282019in}}%
\pgfpathlineto{\pgfqpoint{1.076094in}{1.252571in}}%
\pgfpathlineto{\pgfqpoint{1.080938in}{1.262118in}}%
\pgfpathlineto{\pgfqpoint{1.085781in}{1.303056in}}%
\pgfpathlineto{\pgfqpoint{1.095469in}{1.415951in}}%
\pgfpathlineto{\pgfqpoint{1.100313in}{1.455952in}}%
\pgfpathlineto{\pgfqpoint{1.105156in}{1.474439in}}%
\pgfpathlineto{\pgfqpoint{1.110000in}{1.474626in}}%
\pgfpathlineto{\pgfqpoint{1.114844in}{1.457635in}}%
\pgfpathlineto{\pgfqpoint{1.119687in}{1.424671in}}%
\pgfpathlineto{\pgfqpoint{1.124531in}{1.373660in}}%
\pgfpathlineto{\pgfqpoint{1.134219in}{1.223579in}}%
\pgfpathlineto{\pgfqpoint{1.143906in}{1.052582in}}%
\pgfpathlineto{\pgfqpoint{1.148750in}{0.984563in}}%
\pgfpathlineto{\pgfqpoint{1.153594in}{0.940123in}}%
\pgfpathlineto{\pgfqpoint{1.158438in}{0.928970in}}%
\pgfpathlineto{\pgfqpoint{1.163281in}{0.961319in}}%
\pgfpathlineto{\pgfqpoint{1.168125in}{1.042521in}}%
\pgfpathlineto{\pgfqpoint{1.172969in}{1.174613in}}%
\pgfpathlineto{\pgfqpoint{1.182656in}{1.547746in}}%
\pgfpathlineto{\pgfqpoint{1.192344in}{1.927603in}}%
\pgfpathlineto{\pgfqpoint{1.202031in}{2.197099in}}%
\pgfpathlineto{\pgfqpoint{1.206875in}{2.286901in}}%
\pgfpathlineto{\pgfqpoint{1.211719in}{2.352642in}}%
\pgfpathlineto{\pgfqpoint{1.216562in}{2.392498in}}%
\pgfpathlineto{\pgfqpoint{1.221406in}{2.405332in}}%
\pgfpathlineto{\pgfqpoint{1.226250in}{2.394690in}}%
\pgfpathlineto{\pgfqpoint{1.231094in}{2.364971in}}%
\pgfpathlineto{\pgfqpoint{1.245625in}{2.247118in}}%
\pgfpathlineto{\pgfqpoint{1.250469in}{2.222948in}}%
\pgfpathlineto{\pgfqpoint{1.255313in}{2.213236in}}%
\pgfpathlineto{\pgfqpoint{1.260156in}{2.216449in}}%
\pgfpathlineto{\pgfqpoint{1.269844in}{2.243334in}}%
\pgfpathlineto{\pgfqpoint{1.274687in}{2.238560in}}%
\pgfpathlineto{\pgfqpoint{1.279531in}{2.195781in}}%
\pgfpathlineto{\pgfqpoint{1.284375in}{2.104375in}}%
\pgfpathlineto{\pgfqpoint{1.289219in}{1.962028in}}%
\pgfpathlineto{\pgfqpoint{1.298906in}{1.569938in}}%
\pgfpathlineto{\pgfqpoint{1.313438in}{0.980925in}}%
\pgfpathlineto{\pgfqpoint{1.318281in}{0.848160in}}%
\pgfpathlineto{\pgfqpoint{1.323125in}{0.775547in}}%
\pgfpathlineto{\pgfqpoint{1.327969in}{0.777766in}}%
\pgfpathlineto{\pgfqpoint{1.332812in}{0.858447in}}%
\pgfpathlineto{\pgfqpoint{1.337656in}{1.005661in}}%
\pgfpathlineto{\pgfqpoint{1.352188in}{1.533346in}}%
\pgfpathlineto{\pgfqpoint{1.357031in}{1.635295in}}%
\pgfpathlineto{\pgfqpoint{1.361875in}{1.679592in}}%
\pgfpathlineto{\pgfqpoint{1.366719in}{1.674555in}}%
\pgfpathlineto{\pgfqpoint{1.371563in}{1.634577in}}%
\pgfpathlineto{\pgfqpoint{1.381250in}{1.503715in}}%
\pgfpathlineto{\pgfqpoint{1.395781in}{1.295999in}}%
\pgfpathlineto{\pgfqpoint{1.400625in}{1.249585in}}%
\pgfpathlineto{\pgfqpoint{1.405469in}{1.226156in}}%
\pgfpathlineto{\pgfqpoint{1.410313in}{1.228386in}}%
\pgfpathlineto{\pgfqpoint{1.415156in}{1.253462in}}%
\pgfpathlineto{\pgfqpoint{1.429688in}{1.371002in}}%
\pgfpathlineto{\pgfqpoint{1.434531in}{1.382539in}}%
\pgfpathlineto{\pgfqpoint{1.439375in}{1.365991in}}%
\pgfpathlineto{\pgfqpoint{1.444219in}{1.325280in}}%
\pgfpathlineto{\pgfqpoint{1.463594in}{1.101595in}}%
\pgfpathlineto{\pgfqpoint{1.468438in}{1.072065in}}%
\pgfpathlineto{\pgfqpoint{1.473281in}{1.059380in}}%
\pgfpathlineto{\pgfqpoint{1.478125in}{1.061898in}}%
\pgfpathlineto{\pgfqpoint{1.482969in}{1.086526in}}%
\pgfpathlineto{\pgfqpoint{1.487812in}{1.139341in}}%
\pgfpathlineto{\pgfqpoint{1.492656in}{1.221838in}}%
\pgfpathlineto{\pgfqpoint{1.512031in}{1.630874in}}%
\pgfpathlineto{\pgfqpoint{1.516875in}{1.667958in}}%
\pgfpathlineto{\pgfqpoint{1.521719in}{1.663442in}}%
\pgfpathlineto{\pgfqpoint{1.526563in}{1.621537in}}%
\pgfpathlineto{\pgfqpoint{1.536250in}{1.477842in}}%
\pgfpathlineto{\pgfqpoint{1.550781in}{1.244766in}}%
\pgfpathlineto{\pgfqpoint{1.560469in}{1.133086in}}%
\pgfpathlineto{\pgfqpoint{1.565313in}{1.099136in}}%
\pgfpathlineto{\pgfqpoint{1.570156in}{1.085802in}}%
\pgfpathlineto{\pgfqpoint{1.575000in}{1.098533in}}%
\pgfpathlineto{\pgfqpoint{1.579844in}{1.135175in}}%
\pgfpathlineto{\pgfqpoint{1.594375in}{1.292838in}}%
\pgfpathlineto{\pgfqpoint{1.599219in}{1.314541in}}%
\pgfpathlineto{\pgfqpoint{1.604062in}{1.306426in}}%
\pgfpathlineto{\pgfqpoint{1.608906in}{1.273208in}}%
\pgfpathlineto{\pgfqpoint{1.623438in}{1.128468in}}%
\pgfpathlineto{\pgfqpoint{1.628281in}{1.100618in}}%
\pgfpathlineto{\pgfqpoint{1.633125in}{1.091783in}}%
\pgfpathlineto{\pgfqpoint{1.637969in}{1.105687in}}%
\pgfpathlineto{\pgfqpoint{1.642813in}{1.140631in}}%
\pgfpathlineto{\pgfqpoint{1.647656in}{1.195505in}}%
\pgfpathlineto{\pgfqpoint{1.657344in}{1.356438in}}%
\pgfpathlineto{\pgfqpoint{1.671875in}{1.623731in}}%
\pgfpathlineto{\pgfqpoint{1.676719in}{1.674065in}}%
\pgfpathlineto{\pgfqpoint{1.681563in}{1.685404in}}%
\pgfpathlineto{\pgfqpoint{1.686406in}{1.655320in}}%
\pgfpathlineto{\pgfqpoint{1.691250in}{1.591402in}}%
\pgfpathlineto{\pgfqpoint{1.705781in}{1.348728in}}%
\pgfpathlineto{\pgfqpoint{1.710625in}{1.288944in}}%
\pgfpathlineto{\pgfqpoint{1.715469in}{1.243683in}}%
\pgfpathlineto{\pgfqpoint{1.720313in}{1.210357in}}%
\pgfpathlineto{\pgfqpoint{1.725156in}{1.193688in}}%
\pgfpathlineto{\pgfqpoint{1.730000in}{1.197620in}}%
\pgfpathlineto{\pgfqpoint{1.734844in}{1.226728in}}%
\pgfpathlineto{\pgfqpoint{1.739688in}{1.279049in}}%
\pgfpathlineto{\pgfqpoint{1.754219in}{1.461857in}}%
\pgfpathlineto{\pgfqpoint{1.759062in}{1.494164in}}%
\pgfpathlineto{\pgfqpoint{1.763906in}{1.507190in}}%
\pgfpathlineto{\pgfqpoint{1.768750in}{1.507202in}}%
\pgfpathlineto{\pgfqpoint{1.773594in}{1.502179in}}%
\pgfpathlineto{\pgfqpoint{1.778438in}{1.498774in}}%
\pgfpathlineto{\pgfqpoint{1.783281in}{1.498895in}}%
\pgfpathlineto{\pgfqpoint{1.788125in}{1.502464in}}%
\pgfpathlineto{\pgfqpoint{1.792969in}{1.510083in}}%
\pgfpathlineto{\pgfqpoint{1.797813in}{1.525775in}}%
\pgfpathlineto{\pgfqpoint{1.802656in}{1.552234in}}%
\pgfpathlineto{\pgfqpoint{1.807500in}{1.598045in}}%
\pgfpathlineto{\pgfqpoint{1.812344in}{1.664734in}}%
\pgfpathlineto{\pgfqpoint{1.831719in}{1.999266in}}%
\pgfpathlineto{\pgfqpoint{1.836563in}{2.038887in}}%
\pgfpathlineto{\pgfqpoint{1.841406in}{2.046758in}}%
\pgfpathlineto{\pgfqpoint{1.846250in}{2.027994in}}%
\pgfpathlineto{\pgfqpoint{1.851094in}{1.989968in}}%
\pgfpathlineto{\pgfqpoint{1.860781in}{1.876934in}}%
\pgfpathlineto{\pgfqpoint{1.875312in}{1.683080in}}%
\pgfpathlineto{\pgfqpoint{1.880156in}{1.651753in}}%
\pgfpathlineto{\pgfqpoint{1.885000in}{1.653033in}}%
\pgfpathlineto{\pgfqpoint{1.889844in}{1.690020in}}%
\pgfpathlineto{\pgfqpoint{1.909219in}{1.939608in}}%
\pgfpathlineto{\pgfqpoint{1.918906in}{2.007855in}}%
\pgfpathlineto{\pgfqpoint{1.928594in}{2.065110in}}%
\pgfpathlineto{\pgfqpoint{1.933438in}{2.088338in}}%
\pgfpathlineto{\pgfqpoint{1.938281in}{2.098624in}}%
\pgfpathlineto{\pgfqpoint{1.943125in}{2.093513in}}%
\pgfpathlineto{\pgfqpoint{1.952813in}{2.056318in}}%
\pgfpathlineto{\pgfqpoint{1.957656in}{2.047412in}}%
\pgfpathlineto{\pgfqpoint{1.962500in}{2.062231in}}%
\pgfpathlineto{\pgfqpoint{1.967344in}{2.105000in}}%
\pgfpathlineto{\pgfqpoint{1.972187in}{2.172950in}}%
\pgfpathlineto{\pgfqpoint{1.986719in}{2.417432in}}%
\pgfpathlineto{\pgfqpoint{1.991563in}{2.466949in}}%
\pgfpathlineto{\pgfqpoint{1.996406in}{2.487137in}}%
\pgfpathlineto{\pgfqpoint{2.001250in}{2.475844in}}%
\pgfpathlineto{\pgfqpoint{2.006094in}{2.440540in}}%
\pgfpathlineto{\pgfqpoint{2.015781in}{2.324378in}}%
\pgfpathlineto{\pgfqpoint{2.030312in}{2.103606in}}%
\pgfpathlineto{\pgfqpoint{2.035156in}{2.031087in}}%
\pgfpathlineto{\pgfqpoint{2.040000in}{1.975373in}}%
\pgfpathlineto{\pgfqpoint{2.044844in}{1.945894in}}%
\pgfpathlineto{\pgfqpoint{2.049688in}{1.953810in}}%
\pgfpathlineto{\pgfqpoint{2.054531in}{1.996507in}}%
\pgfpathlineto{\pgfqpoint{2.064219in}{2.152447in}}%
\pgfpathlineto{\pgfqpoint{2.073906in}{2.319383in}}%
\pgfpathlineto{\pgfqpoint{2.078750in}{2.384786in}}%
\pgfpathlineto{\pgfqpoint{2.083594in}{2.434938in}}%
\pgfpathlineto{\pgfqpoint{2.088437in}{2.465190in}}%
\pgfpathlineto{\pgfqpoint{2.093281in}{2.475341in}}%
\pgfpathlineto{\pgfqpoint{2.098125in}{2.464206in}}%
\pgfpathlineto{\pgfqpoint{2.102969in}{2.434081in}}%
\pgfpathlineto{\pgfqpoint{2.112656in}{2.348843in}}%
\pgfpathlineto{\pgfqpoint{2.117500in}{2.318899in}}%
\pgfpathlineto{\pgfqpoint{2.122344in}{2.314283in}}%
\pgfpathlineto{\pgfqpoint{2.127188in}{2.341400in}}%
\pgfpathlineto{\pgfqpoint{2.132031in}{2.397151in}}%
\pgfpathlineto{\pgfqpoint{2.146562in}{2.612303in}}%
\pgfpathlineto{\pgfqpoint{2.151406in}{2.646521in}}%
\pgfpathlineto{\pgfqpoint{2.156250in}{2.648596in}}%
\pgfpathlineto{\pgfqpoint{2.161094in}{2.621415in}}%
\pgfpathlineto{\pgfqpoint{2.165938in}{2.572155in}}%
\pgfpathlineto{\pgfqpoint{2.195000in}{2.193386in}}%
\pgfpathlineto{\pgfqpoint{2.199844in}{2.152081in}}%
\pgfpathlineto{\pgfqpoint{2.204687in}{2.132624in}}%
\pgfpathlineto{\pgfqpoint{2.209531in}{2.146415in}}%
\pgfpathlineto{\pgfqpoint{2.214375in}{2.196179in}}%
\pgfpathlineto{\pgfqpoint{2.224063in}{2.369881in}}%
\pgfpathlineto{\pgfqpoint{2.233750in}{2.535728in}}%
\pgfpathlineto{\pgfqpoint{2.238594in}{2.582961in}}%
\pgfpathlineto{\pgfqpoint{2.243438in}{2.604767in}}%
\pgfpathlineto{\pgfqpoint{2.248281in}{2.609884in}}%
\pgfpathlineto{\pgfqpoint{2.253125in}{2.606418in}}%
\pgfpathlineto{\pgfqpoint{2.257969in}{2.600369in}}%
\pgfpathlineto{\pgfqpoint{2.262812in}{2.589863in}}%
\pgfpathlineto{\pgfqpoint{2.267656in}{2.574501in}}%
\pgfpathlineto{\pgfqpoint{2.277344in}{2.533181in}}%
\pgfpathlineto{\pgfqpoint{2.282188in}{2.522748in}}%
\pgfpathlineto{\pgfqpoint{2.287031in}{2.528982in}}%
\pgfpathlineto{\pgfqpoint{2.291875in}{2.557527in}}%
\pgfpathlineto{\pgfqpoint{2.311250in}{2.734044in}}%
\pgfpathlineto{\pgfqpoint{2.316094in}{2.740812in}}%
\pgfpathlineto{\pgfqpoint{2.320937in}{2.723640in}}%
\pgfpathlineto{\pgfqpoint{2.325781in}{2.685452in}}%
\pgfpathlineto{\pgfqpoint{2.330625in}{2.631131in}}%
\pgfpathlineto{\pgfqpoint{2.340313in}{2.486098in}}%
\pgfpathlineto{\pgfqpoint{2.350000in}{2.328238in}}%
\pgfpathlineto{\pgfqpoint{2.354844in}{2.270532in}}%
\pgfpathlineto{\pgfqpoint{2.359688in}{2.242970in}}%
\pgfpathlineto{\pgfqpoint{2.364531in}{2.247617in}}%
\pgfpathlineto{\pgfqpoint{2.369375in}{2.281887in}}%
\pgfpathlineto{\pgfqpoint{2.379062in}{2.403838in}}%
\pgfpathlineto{\pgfqpoint{2.393594in}{2.649218in}}%
\pgfpathlineto{\pgfqpoint{2.398438in}{2.729283in}}%
\pgfpathlineto{\pgfqpoint{2.403281in}{2.790326in}}%
\pgfpathlineto{\pgfqpoint{2.408125in}{2.821106in}}%
\pgfpathlineto{\pgfqpoint{2.412969in}{2.816260in}}%
\pgfpathlineto{\pgfqpoint{2.417813in}{2.781358in}}%
\pgfpathlineto{\pgfqpoint{2.432344in}{2.623544in}}%
\pgfpathlineto{\pgfqpoint{2.437187in}{2.586873in}}%
\pgfpathlineto{\pgfqpoint{2.442031in}{2.563337in}}%
\pgfpathlineto{\pgfqpoint{2.446875in}{2.552264in}}%
\pgfpathlineto{\pgfqpoint{2.451719in}{2.553876in}}%
\pgfpathlineto{\pgfqpoint{2.456562in}{2.568006in}}%
\pgfpathlineto{\pgfqpoint{2.461406in}{2.595828in}}%
\pgfpathlineto{\pgfqpoint{2.475938in}{2.693812in}}%
\pgfpathlineto{\pgfqpoint{2.480781in}{2.707817in}}%
\pgfpathlineto{\pgfqpoint{2.485625in}{2.702083in}}%
\pgfpathlineto{\pgfqpoint{2.490469in}{2.670533in}}%
\pgfpathlineto{\pgfqpoint{2.495313in}{2.611550in}}%
\pgfpathlineto{\pgfqpoint{2.505000in}{2.437267in}}%
\pgfpathlineto{\pgfqpoint{2.509844in}{2.345937in}}%
\pgfpathlineto{\pgfqpoint{2.514687in}{2.276639in}}%
\pgfpathlineto{\pgfqpoint{2.519531in}{2.240547in}}%
\pgfpathlineto{\pgfqpoint{2.524375in}{2.246251in}}%
\pgfpathlineto{\pgfqpoint{2.529219in}{2.288766in}}%
\pgfpathlineto{\pgfqpoint{2.534063in}{2.358307in}}%
\pgfpathlineto{\pgfqpoint{2.553438in}{2.691086in}}%
\pgfpathlineto{\pgfqpoint{2.558281in}{2.752797in}}%
\pgfpathlineto{\pgfqpoint{2.563125in}{2.798059in}}%
\pgfpathlineto{\pgfqpoint{2.567969in}{2.826725in}}%
\pgfpathlineto{\pgfqpoint{2.572812in}{2.836846in}}%
\pgfpathlineto{\pgfqpoint{2.577656in}{2.825936in}}%
\pgfpathlineto{\pgfqpoint{2.582500in}{2.795653in}}%
\pgfpathlineto{\pgfqpoint{2.601875in}{2.630462in}}%
\pgfpathlineto{\pgfqpoint{2.606719in}{2.606762in}}%
\pgfpathlineto{\pgfqpoint{2.611563in}{2.593093in}}%
\pgfpathlineto{\pgfqpoint{2.616406in}{2.590206in}}%
\pgfpathlineto{\pgfqpoint{2.621250in}{2.594880in}}%
\pgfpathlineto{\pgfqpoint{2.640625in}{2.627094in}}%
\pgfpathlineto{\pgfqpoint{2.645469in}{2.627383in}}%
\pgfpathlineto{\pgfqpoint{2.650313in}{2.615238in}}%
\pgfpathlineto{\pgfqpoint{2.655156in}{2.585623in}}%
\pgfpathlineto{\pgfqpoint{2.660000in}{2.536313in}}%
\pgfpathlineto{\pgfqpoint{2.674531in}{2.342701in}}%
\pgfpathlineto{\pgfqpoint{2.679375in}{2.314679in}}%
\pgfpathlineto{\pgfqpoint{2.684219in}{2.325682in}}%
\pgfpathlineto{\pgfqpoint{2.689062in}{2.374795in}}%
\pgfpathlineto{\pgfqpoint{2.698750in}{2.538274in}}%
\pgfpathlineto{\pgfqpoint{2.708438in}{2.708996in}}%
\pgfpathlineto{\pgfqpoint{2.718125in}{2.840083in}}%
\pgfpathlineto{\pgfqpoint{2.722969in}{2.882998in}}%
\pgfpathlineto{\pgfqpoint{2.727813in}{2.905338in}}%
\pgfpathlineto{\pgfqpoint{2.732656in}{2.904817in}}%
\pgfpathlineto{\pgfqpoint{2.737500in}{2.880000in}}%
\pgfpathlineto{\pgfqpoint{2.742344in}{2.835048in}}%
\pgfpathlineto{\pgfqpoint{2.756875in}{2.657621in}}%
\pgfpathlineto{\pgfqpoint{2.761719in}{2.618753in}}%
\pgfpathlineto{\pgfqpoint{2.766563in}{2.598443in}}%
\pgfpathlineto{\pgfqpoint{2.771406in}{2.596239in}}%
\pgfpathlineto{\pgfqpoint{2.776250in}{2.605998in}}%
\pgfpathlineto{\pgfqpoint{2.781094in}{2.623879in}}%
\pgfpathlineto{\pgfqpoint{2.785938in}{2.650302in}}%
\pgfpathlineto{\pgfqpoint{2.795625in}{2.721758in}}%
\pgfpathlineto{\pgfqpoint{2.800469in}{2.759825in}}%
\pgfpathlineto{\pgfqpoint{2.805312in}{2.781756in}}%
\pgfpathlineto{\pgfqpoint{2.810156in}{2.775966in}}%
\pgfpathlineto{\pgfqpoint{2.815000in}{2.740346in}}%
\pgfpathlineto{\pgfqpoint{2.829531in}{2.563441in}}%
\pgfpathlineto{\pgfqpoint{2.834375in}{2.533524in}}%
\pgfpathlineto{\pgfqpoint{2.839219in}{2.530422in}}%
\pgfpathlineto{\pgfqpoint{2.844063in}{2.551296in}}%
\pgfpathlineto{\pgfqpoint{2.848906in}{2.590346in}}%
\pgfpathlineto{\pgfqpoint{2.853750in}{2.646159in}}%
\pgfpathlineto{\pgfqpoint{2.863437in}{2.799534in}}%
\pgfpathlineto{\pgfqpoint{2.877969in}{3.069883in}}%
\pgfpathlineto{\pgfqpoint{2.882812in}{3.134690in}}%
\pgfpathlineto{\pgfqpoint{2.887656in}{3.170357in}}%
\pgfpathlineto{\pgfqpoint{2.892500in}{3.173069in}}%
\pgfpathlineto{\pgfqpoint{2.897344in}{3.142496in}}%
\pgfpathlineto{\pgfqpoint{2.902188in}{3.081451in}}%
\pgfpathlineto{\pgfqpoint{2.921562in}{2.764591in}}%
\pgfpathlineto{\pgfqpoint{2.926406in}{2.714274in}}%
\pgfpathlineto{\pgfqpoint{2.931250in}{2.687822in}}%
\pgfpathlineto{\pgfqpoint{2.936094in}{2.683684in}}%
\pgfpathlineto{\pgfqpoint{2.940937in}{2.699870in}}%
\pgfpathlineto{\pgfqpoint{2.960313in}{2.816590in}}%
\pgfpathlineto{\pgfqpoint{2.965156in}{2.824765in}}%
\pgfpathlineto{\pgfqpoint{2.970000in}{2.817815in}}%
\pgfpathlineto{\pgfqpoint{2.974844in}{2.795205in}}%
\pgfpathlineto{\pgfqpoint{2.979688in}{2.754128in}}%
\pgfpathlineto{\pgfqpoint{2.994219in}{2.592687in}}%
\pgfpathlineto{\pgfqpoint{2.999062in}{2.566032in}}%
\pgfpathlineto{\pgfqpoint{3.003906in}{2.569670in}}%
\pgfpathlineto{\pgfqpoint{3.008750in}{2.600738in}}%
\pgfpathlineto{\pgfqpoint{3.023281in}{2.761568in}}%
\pgfpathlineto{\pgfqpoint{3.042656in}{2.969982in}}%
\pgfpathlineto{\pgfqpoint{3.047500in}{2.989285in}}%
\pgfpathlineto{\pgfqpoint{3.052344in}{2.975317in}}%
\pgfpathlineto{\pgfqpoint{3.057187in}{2.926294in}}%
\pgfpathlineto{\pgfqpoint{3.066875in}{2.763580in}}%
\pgfpathlineto{\pgfqpoint{3.076563in}{2.595724in}}%
\pgfpathlineto{\pgfqpoint{3.081406in}{2.528620in}}%
\pgfpathlineto{\pgfqpoint{3.086250in}{2.477314in}}%
\pgfpathlineto{\pgfqpoint{3.091094in}{2.449256in}}%
\pgfpathlineto{\pgfqpoint{3.095938in}{2.444778in}}%
\pgfpathlineto{\pgfqpoint{3.100781in}{2.466203in}}%
\pgfpathlineto{\pgfqpoint{3.115312in}{2.584847in}}%
\pgfpathlineto{\pgfqpoint{3.120156in}{2.608491in}}%
\pgfpathlineto{\pgfqpoint{3.125000in}{2.616783in}}%
\pgfpathlineto{\pgfqpoint{3.129844in}{2.610399in}}%
\pgfpathlineto{\pgfqpoint{3.134688in}{2.590363in}}%
\pgfpathlineto{\pgfqpoint{3.139531in}{2.556764in}}%
\pgfpathlineto{\pgfqpoint{3.154063in}{2.437598in}}%
\pgfpathlineto{\pgfqpoint{3.158906in}{2.425876in}}%
\pgfpathlineto{\pgfqpoint{3.163750in}{2.438818in}}%
\pgfpathlineto{\pgfqpoint{3.168594in}{2.476608in}}%
\pgfpathlineto{\pgfqpoint{3.178281in}{2.601782in}}%
\pgfpathlineto{\pgfqpoint{3.192813in}{2.820014in}}%
\pgfpathlineto{\pgfqpoint{3.197656in}{2.876403in}}%
\pgfpathlineto{\pgfqpoint{3.202500in}{2.911542in}}%
\pgfpathlineto{\pgfqpoint{3.207344in}{2.918045in}}%
\pgfpathlineto{\pgfqpoint{3.212188in}{2.895615in}}%
\pgfpathlineto{\pgfqpoint{3.217031in}{2.846144in}}%
\pgfpathlineto{\pgfqpoint{3.226719in}{2.693200in}}%
\pgfpathlineto{\pgfqpoint{3.241250in}{2.443035in}}%
\pgfpathlineto{\pgfqpoint{3.246094in}{2.383921in}}%
\pgfpathlineto{\pgfqpoint{3.250938in}{2.349862in}}%
\pgfpathlineto{\pgfqpoint{3.255781in}{2.344231in}}%
\pgfpathlineto{\pgfqpoint{3.260625in}{2.364532in}}%
\pgfpathlineto{\pgfqpoint{3.265469in}{2.408007in}}%
\pgfpathlineto{\pgfqpoint{3.284844in}{2.638833in}}%
\pgfpathlineto{\pgfqpoint{3.289687in}{2.669708in}}%
\pgfpathlineto{\pgfqpoint{3.294531in}{2.678090in}}%
\pgfpathlineto{\pgfqpoint{3.299375in}{2.664431in}}%
\pgfpathlineto{\pgfqpoint{3.304219in}{2.631846in}}%
\pgfpathlineto{\pgfqpoint{3.318750in}{2.492452in}}%
\pgfpathlineto{\pgfqpoint{3.323594in}{2.462503in}}%
\pgfpathlineto{\pgfqpoint{3.328438in}{2.452224in}}%
\pgfpathlineto{\pgfqpoint{3.333281in}{2.465930in}}%
\pgfpathlineto{\pgfqpoint{3.338125in}{2.502285in}}%
\pgfpathlineto{\pgfqpoint{3.347812in}{2.617874in}}%
\pgfpathlineto{\pgfqpoint{3.352656in}{2.675813in}}%
\pgfpathlineto{\pgfqpoint{3.357500in}{2.714730in}}%
\pgfpathlineto{\pgfqpoint{3.362344in}{2.724835in}}%
\pgfpathlineto{\pgfqpoint{3.367188in}{2.698690in}}%
\pgfpathlineto{\pgfqpoint{3.372031in}{2.635809in}}%
\pgfpathlineto{\pgfqpoint{3.381719in}{2.423846in}}%
\pgfpathlineto{\pgfqpoint{3.391406in}{2.203163in}}%
\pgfpathlineto{\pgfqpoint{3.396250in}{2.133710in}}%
\pgfpathlineto{\pgfqpoint{3.401094in}{2.108679in}}%
\pgfpathlineto{\pgfqpoint{3.405938in}{2.131460in}}%
\pgfpathlineto{\pgfqpoint{3.410781in}{2.197493in}}%
\pgfpathlineto{\pgfqpoint{3.415625in}{2.302760in}}%
\pgfpathlineto{\pgfqpoint{3.425312in}{2.606104in}}%
\pgfpathlineto{\pgfqpoint{3.439844in}{3.135005in}}%
\pgfpathlineto{\pgfqpoint{3.444688in}{3.263235in}}%
\pgfpathlineto{\pgfqpoint{3.449531in}{3.348977in}}%
\pgfpathlineto{\pgfqpoint{3.454375in}{3.396499in}}%
\pgfpathlineto{\pgfqpoint{3.459219in}{3.418293in}}%
\pgfpathlineto{\pgfqpoint{3.464063in}{3.424519in}}%
\pgfpathlineto{\pgfqpoint{3.468906in}{3.419775in}}%
\pgfpathlineto{\pgfqpoint{3.473750in}{3.403424in}}%
\pgfpathlineto{\pgfqpoint{3.483437in}{3.350581in}}%
\pgfpathlineto{\pgfqpoint{3.488281in}{3.335799in}}%
\pgfpathlineto{\pgfqpoint{3.493125in}{3.347312in}}%
\pgfpathlineto{\pgfqpoint{3.497969in}{3.384707in}}%
\pgfpathlineto{\pgfqpoint{3.507656in}{3.474332in}}%
\pgfpathlineto{\pgfqpoint{3.512500in}{3.477021in}}%
\pgfpathlineto{\pgfqpoint{3.517344in}{3.427933in}}%
\pgfpathlineto{\pgfqpoint{3.522188in}{3.325150in}}%
\pgfpathlineto{\pgfqpoint{3.527031in}{3.175331in}}%
\pgfpathlineto{\pgfqpoint{3.536719in}{2.772344in}}%
\pgfpathlineto{\pgfqpoint{3.551250in}{2.104187in}}%
\pgfpathlineto{\pgfqpoint{3.556094in}{1.949586in}}%
\pgfpathlineto{\pgfqpoint{3.560938in}{1.864393in}}%
\pgfpathlineto{\pgfqpoint{3.565781in}{1.850869in}}%
\pgfpathlineto{\pgfqpoint{3.570625in}{1.895666in}}%
\pgfpathlineto{\pgfqpoint{3.580313in}{2.082415in}}%
\pgfpathlineto{\pgfqpoint{3.594844in}{2.389257in}}%
\pgfpathlineto{\pgfqpoint{3.599687in}{2.468051in}}%
\pgfpathlineto{\pgfqpoint{3.604531in}{2.528466in}}%
\pgfpathlineto{\pgfqpoint{3.609375in}{2.564687in}}%
\pgfpathlineto{\pgfqpoint{3.614219in}{2.577598in}}%
\pgfpathlineto{\pgfqpoint{3.619063in}{2.573798in}}%
\pgfpathlineto{\pgfqpoint{3.623906in}{2.556427in}}%
\pgfpathlineto{\pgfqpoint{3.628750in}{2.531027in}}%
\pgfpathlineto{\pgfqpoint{3.643281in}{2.432569in}}%
\pgfpathlineto{\pgfqpoint{3.648125in}{2.412483in}}%
\pgfpathlineto{\pgfqpoint{3.652969in}{2.412432in}}%
\pgfpathlineto{\pgfqpoint{3.657812in}{2.432997in}}%
\pgfpathlineto{\pgfqpoint{3.677188in}{2.570182in}}%
\pgfpathlineto{\pgfqpoint{3.682031in}{2.572964in}}%
\pgfpathlineto{\pgfqpoint{3.686875in}{2.558429in}}%
\pgfpathlineto{\pgfqpoint{3.691719in}{2.522937in}}%
\pgfpathlineto{\pgfqpoint{3.696563in}{2.466024in}}%
\pgfpathlineto{\pgfqpoint{3.706250in}{2.292574in}}%
\pgfpathlineto{\pgfqpoint{3.715937in}{2.088967in}}%
\pgfpathlineto{\pgfqpoint{3.720781in}{2.010797in}}%
\pgfpathlineto{\pgfqpoint{3.725625in}{1.967008in}}%
\pgfpathlineto{\pgfqpoint{3.730469in}{1.963713in}}%
\pgfpathlineto{\pgfqpoint{3.735313in}{1.998606in}}%
\pgfpathlineto{\pgfqpoint{3.740156in}{2.060761in}}%
\pgfpathlineto{\pgfqpoint{3.754688in}{2.288772in}}%
\pgfpathlineto{\pgfqpoint{3.764375in}{2.399127in}}%
\pgfpathlineto{\pgfqpoint{3.769219in}{2.436740in}}%
\pgfpathlineto{\pgfqpoint{3.774062in}{2.459416in}}%
\pgfpathlineto{\pgfqpoint{3.778906in}{2.465413in}}%
\pgfpathlineto{\pgfqpoint{3.783750in}{2.450966in}}%
\pgfpathlineto{\pgfqpoint{3.788594in}{2.416222in}}%
\pgfpathlineto{\pgfqpoint{3.803125in}{2.283510in}}%
\pgfpathlineto{\pgfqpoint{3.807969in}{2.264718in}}%
\pgfpathlineto{\pgfqpoint{3.812813in}{2.269269in}}%
\pgfpathlineto{\pgfqpoint{3.817656in}{2.296180in}}%
\pgfpathlineto{\pgfqpoint{3.832187in}{2.416799in}}%
\pgfpathlineto{\pgfqpoint{3.837031in}{2.437598in}}%
\pgfpathlineto{\pgfqpoint{3.841875in}{2.437811in}}%
\pgfpathlineto{\pgfqpoint{3.846719in}{2.419878in}}%
\pgfpathlineto{\pgfqpoint{3.851562in}{2.384871in}}%
\pgfpathlineto{\pgfqpoint{3.856406in}{2.333020in}}%
\pgfpathlineto{\pgfqpoint{3.866094in}{2.188507in}}%
\pgfpathlineto{\pgfqpoint{3.880625in}{1.949100in}}%
\pgfpathlineto{\pgfqpoint{3.885469in}{1.904614in}}%
\pgfpathlineto{\pgfqpoint{3.890313in}{1.889760in}}%
\pgfpathlineto{\pgfqpoint{3.895156in}{1.904965in}}%
\pgfpathlineto{\pgfqpoint{3.900000in}{1.945922in}}%
\pgfpathlineto{\pgfqpoint{3.909687in}{2.088989in}}%
\pgfpathlineto{\pgfqpoint{3.924219in}{2.340506in}}%
\pgfpathlineto{\pgfqpoint{3.929063in}{2.392772in}}%
\pgfpathlineto{\pgfqpoint{3.933906in}{2.419264in}}%
\pgfpathlineto{\pgfqpoint{3.938750in}{2.419055in}}%
\pgfpathlineto{\pgfqpoint{3.943594in}{2.399630in}}%
\pgfpathlineto{\pgfqpoint{3.948438in}{2.367423in}}%
\pgfpathlineto{\pgfqpoint{3.958125in}{2.281289in}}%
\pgfpathlineto{\pgfqpoint{3.967812in}{2.185747in}}%
\pgfpathlineto{\pgfqpoint{3.972656in}{2.147756in}}%
\pgfpathlineto{\pgfqpoint{3.977500in}{2.127975in}}%
\pgfpathlineto{\pgfqpoint{3.982344in}{2.129899in}}%
\pgfpathlineto{\pgfqpoint{3.987188in}{2.150518in}}%
\pgfpathlineto{\pgfqpoint{3.996875in}{2.211714in}}%
\pgfpathlineto{\pgfqpoint{4.001719in}{2.230476in}}%
\pgfpathlineto{\pgfqpoint{4.006563in}{2.230424in}}%
\pgfpathlineto{\pgfqpoint{4.011406in}{2.214652in}}%
\pgfpathlineto{\pgfqpoint{4.016250in}{2.186533in}}%
\pgfpathlineto{\pgfqpoint{4.025938in}{2.106241in}}%
\pgfpathlineto{\pgfqpoint{4.040469in}{1.971845in}}%
\pgfpathlineto{\pgfqpoint{4.045312in}{1.948718in}}%
\pgfpathlineto{\pgfqpoint{4.050156in}{1.950033in}}%
\pgfpathlineto{\pgfqpoint{4.055000in}{1.982829in}}%
\pgfpathlineto{\pgfqpoint{4.059844in}{2.041746in}}%
\pgfpathlineto{\pgfqpoint{4.084062in}{2.424461in}}%
\pgfpathlineto{\pgfqpoint{4.088906in}{2.471205in}}%
\pgfpathlineto{\pgfqpoint{4.093750in}{2.498423in}}%
\pgfpathlineto{\pgfqpoint{4.098594in}{2.503552in}}%
\pgfpathlineto{\pgfqpoint{4.103438in}{2.483514in}}%
\pgfpathlineto{\pgfqpoint{4.108281in}{2.439499in}}%
\pgfpathlineto{\pgfqpoint{4.117969in}{2.305909in}}%
\pgfpathlineto{\pgfqpoint{4.127656in}{2.183130in}}%
\pgfpathlineto{\pgfqpoint{4.132500in}{2.149290in}}%
\pgfpathlineto{\pgfqpoint{4.137344in}{2.141571in}}%
\pgfpathlineto{\pgfqpoint{4.142188in}{2.157881in}}%
\pgfpathlineto{\pgfqpoint{4.147031in}{2.192047in}}%
\pgfpathlineto{\pgfqpoint{4.161563in}{2.317270in}}%
\pgfpathlineto{\pgfqpoint{4.166406in}{2.335593in}}%
\pgfpathlineto{\pgfqpoint{4.171250in}{2.330361in}}%
\pgfpathlineto{\pgfqpoint{4.176094in}{2.298965in}}%
\pgfpathlineto{\pgfqpoint{4.180937in}{2.244336in}}%
\pgfpathlineto{\pgfqpoint{4.195469in}{2.049748in}}%
\pgfpathlineto{\pgfqpoint{4.200312in}{2.019329in}}%
\pgfpathlineto{\pgfqpoint{4.205156in}{2.016646in}}%
\pgfpathlineto{\pgfqpoint{4.210000in}{2.033795in}}%
\pgfpathlineto{\pgfqpoint{4.214844in}{2.066441in}}%
\pgfpathlineto{\pgfqpoint{4.219688in}{2.114704in}}%
\pgfpathlineto{\pgfqpoint{4.224531in}{2.180632in}}%
\pgfpathlineto{\pgfqpoint{4.234219in}{2.375075in}}%
\pgfpathlineto{\pgfqpoint{4.243906in}{2.581360in}}%
\pgfpathlineto{\pgfqpoint{4.248750in}{2.645339in}}%
\pgfpathlineto{\pgfqpoint{4.253594in}{2.668217in}}%
\pgfpathlineto{\pgfqpoint{4.258438in}{2.653627in}}%
\pgfpathlineto{\pgfqpoint{4.263281in}{2.609902in}}%
\pgfpathlineto{\pgfqpoint{4.272969in}{2.475456in}}%
\pgfpathlineto{\pgfqpoint{4.287500in}{2.262209in}}%
\pgfpathlineto{\pgfqpoint{4.292344in}{2.216980in}}%
\pgfpathlineto{\pgfqpoint{4.297187in}{2.193214in}}%
\pgfpathlineto{\pgfqpoint{4.302031in}{2.190856in}}%
\pgfpathlineto{\pgfqpoint{4.306875in}{2.205616in}}%
\pgfpathlineto{\pgfqpoint{4.311719in}{2.230814in}}%
\pgfpathlineto{\pgfqpoint{4.321406in}{2.292327in}}%
\pgfpathlineto{\pgfqpoint{4.326250in}{2.313553in}}%
\pgfpathlineto{\pgfqpoint{4.331094in}{2.319780in}}%
\pgfpathlineto{\pgfqpoint{4.335938in}{2.307492in}}%
\pgfpathlineto{\pgfqpoint{4.340781in}{2.274779in}}%
\pgfpathlineto{\pgfqpoint{4.355313in}{2.142329in}}%
\pgfpathlineto{\pgfqpoint{4.360156in}{2.113881in}}%
\pgfpathlineto{\pgfqpoint{4.365000in}{2.098849in}}%
\pgfpathlineto{\pgfqpoint{4.369844in}{2.102324in}}%
\pgfpathlineto{\pgfqpoint{4.374688in}{2.129298in}}%
\pgfpathlineto{\pgfqpoint{4.379531in}{2.184197in}}%
\pgfpathlineto{\pgfqpoint{4.389219in}{2.364582in}}%
\pgfpathlineto{\pgfqpoint{4.398906in}{2.543213in}}%
\pgfpathlineto{\pgfqpoint{4.403750in}{2.598045in}}%
\pgfpathlineto{\pgfqpoint{4.408594in}{2.621550in}}%
\pgfpathlineto{\pgfqpoint{4.413437in}{2.620416in}}%
\pgfpathlineto{\pgfqpoint{4.418281in}{2.599389in}}%
\pgfpathlineto{\pgfqpoint{4.423125in}{2.562780in}}%
\pgfpathlineto{\pgfqpoint{4.427969in}{2.511648in}}%
\pgfpathlineto{\pgfqpoint{4.437656in}{2.370081in}}%
\pgfpathlineto{\pgfqpoint{4.452187in}{2.131399in}}%
\pgfpathlineto{\pgfqpoint{4.457031in}{2.080690in}}%
\pgfpathlineto{\pgfqpoint{4.461875in}{2.057785in}}%
\pgfpathlineto{\pgfqpoint{4.466719in}{2.064026in}}%
\pgfpathlineto{\pgfqpoint{4.471563in}{2.095569in}}%
\pgfpathlineto{\pgfqpoint{4.486094in}{2.237383in}}%
\pgfpathlineto{\pgfqpoint{4.490938in}{2.262785in}}%
\pgfpathlineto{\pgfqpoint{4.495781in}{2.262608in}}%
\pgfpathlineto{\pgfqpoint{4.500625in}{2.240376in}}%
\pgfpathlineto{\pgfqpoint{4.510313in}{2.155191in}}%
\pgfpathlineto{\pgfqpoint{4.515156in}{2.112931in}}%
\pgfpathlineto{\pgfqpoint{4.520000in}{2.086908in}}%
\pgfpathlineto{\pgfqpoint{4.524844in}{2.081403in}}%
\pgfpathlineto{\pgfqpoint{4.529688in}{2.099120in}}%
\pgfpathlineto{\pgfqpoint{4.534531in}{2.136445in}}%
\pgfpathlineto{\pgfqpoint{4.544219in}{2.250720in}}%
\pgfpathlineto{\pgfqpoint{4.558750in}{2.448578in}}%
\pgfpathlineto{\pgfqpoint{4.563594in}{2.497124in}}%
\pgfpathlineto{\pgfqpoint{4.568437in}{2.523888in}}%
\pgfpathlineto{\pgfqpoint{4.573281in}{2.521325in}}%
\pgfpathlineto{\pgfqpoint{4.578125in}{2.483732in}}%
\pgfpathlineto{\pgfqpoint{4.582969in}{2.417829in}}%
\pgfpathlineto{\pgfqpoint{4.602344in}{2.068815in}}%
\pgfpathlineto{\pgfqpoint{4.607188in}{2.011554in}}%
\pgfpathlineto{\pgfqpoint{4.612031in}{1.972668in}}%
\pgfpathlineto{\pgfqpoint{4.616875in}{1.948666in}}%
\pgfpathlineto{\pgfqpoint{4.621719in}{1.936737in}}%
\pgfpathlineto{\pgfqpoint{4.626563in}{1.936003in}}%
\pgfpathlineto{\pgfqpoint{4.631406in}{1.947946in}}%
\pgfpathlineto{\pgfqpoint{4.636250in}{1.973332in}}%
\pgfpathlineto{\pgfqpoint{4.650781in}{2.075223in}}%
\pgfpathlineto{\pgfqpoint{4.655625in}{2.086771in}}%
\pgfpathlineto{\pgfqpoint{4.660469in}{2.081808in}}%
\pgfpathlineto{\pgfqpoint{4.670156in}{2.046758in}}%
\pgfpathlineto{\pgfqpoint{4.689531in}{1.980053in}}%
\pgfpathlineto{\pgfqpoint{4.694375in}{1.974159in}}%
\pgfpathlineto{\pgfqpoint{4.699219in}{1.990883in}}%
\pgfpathlineto{\pgfqpoint{4.704063in}{2.036768in}}%
\pgfpathlineto{\pgfqpoint{4.718594in}{2.251275in}}%
\pgfpathlineto{\pgfqpoint{4.723438in}{2.289912in}}%
\pgfpathlineto{\pgfqpoint{4.728281in}{2.295147in}}%
\pgfpathlineto{\pgfqpoint{4.733125in}{2.273588in}}%
\pgfpathlineto{\pgfqpoint{4.737969in}{2.234836in}}%
\pgfpathlineto{\pgfqpoint{4.747656in}{2.122127in}}%
\pgfpathlineto{\pgfqpoint{4.757344in}{1.960495in}}%
\pgfpathlineto{\pgfqpoint{4.767031in}{1.779950in}}%
\pgfpathlineto{\pgfqpoint{4.771875in}{1.712722in}}%
\pgfpathlineto{\pgfqpoint{4.776719in}{1.674783in}}%
\pgfpathlineto{\pgfqpoint{4.781562in}{1.671405in}}%
\pgfpathlineto{\pgfqpoint{4.786406in}{1.698109in}}%
\pgfpathlineto{\pgfqpoint{4.791250in}{1.750032in}}%
\pgfpathlineto{\pgfqpoint{4.810625in}{2.013919in}}%
\pgfpathlineto{\pgfqpoint{4.815469in}{2.048612in}}%
\pgfpathlineto{\pgfqpoint{4.820312in}{2.064125in}}%
\pgfpathlineto{\pgfqpoint{4.825156in}{2.062463in}}%
\pgfpathlineto{\pgfqpoint{4.830000in}{2.047108in}}%
\pgfpathlineto{\pgfqpoint{4.839688in}{2.001618in}}%
\pgfpathlineto{\pgfqpoint{4.849375in}{1.945864in}}%
\pgfpathlineto{\pgfqpoint{4.854219in}{1.924699in}}%
\pgfpathlineto{\pgfqpoint{4.859063in}{1.919268in}}%
\pgfpathlineto{\pgfqpoint{4.863906in}{1.938291in}}%
\pgfpathlineto{\pgfqpoint{4.868750in}{1.985132in}}%
\pgfpathlineto{\pgfqpoint{4.883281in}{2.187912in}}%
\pgfpathlineto{\pgfqpoint{4.888125in}{2.221357in}}%
\pgfpathlineto{\pgfqpoint{4.892969in}{2.222237in}}%
\pgfpathlineto{\pgfqpoint{4.897812in}{2.190937in}}%
\pgfpathlineto{\pgfqpoint{4.902656in}{2.137522in}}%
\pgfpathlineto{\pgfqpoint{4.912344in}{1.987952in}}%
\pgfpathlineto{\pgfqpoint{4.926875in}{1.732428in}}%
\pgfpathlineto{\pgfqpoint{4.931719in}{1.673462in}}%
\pgfpathlineto{\pgfqpoint{4.936562in}{1.651244in}}%
\pgfpathlineto{\pgfqpoint{4.941406in}{1.671299in}}%
\pgfpathlineto{\pgfqpoint{4.946250in}{1.728045in}}%
\pgfpathlineto{\pgfqpoint{4.960781in}{1.956416in}}%
\pgfpathlineto{\pgfqpoint{4.965625in}{2.009496in}}%
\pgfpathlineto{\pgfqpoint{4.975313in}{2.082789in}}%
\pgfpathlineto{\pgfqpoint{4.980156in}{2.113458in}}%
\pgfpathlineto{\pgfqpoint{4.985000in}{2.134778in}}%
\pgfpathlineto{\pgfqpoint{4.989844in}{2.136734in}}%
\pgfpathlineto{\pgfqpoint{4.994688in}{2.114257in}}%
\pgfpathlineto{\pgfqpoint{5.009219in}{1.989171in}}%
\pgfpathlineto{\pgfqpoint{5.014063in}{1.985204in}}%
\pgfpathlineto{\pgfqpoint{5.018906in}{2.016279in}}%
\pgfpathlineto{\pgfqpoint{5.028594in}{2.147340in}}%
\pgfpathlineto{\pgfqpoint{5.038281in}{2.273559in}}%
\pgfpathlineto{\pgfqpoint{5.043125in}{2.309023in}}%
\pgfpathlineto{\pgfqpoint{5.047969in}{2.322176in}}%
\pgfpathlineto{\pgfqpoint{5.052812in}{2.313058in}}%
\pgfpathlineto{\pgfqpoint{5.057656in}{2.281773in}}%
\pgfpathlineto{\pgfqpoint{5.062500in}{2.230239in}}%
\pgfpathlineto{\pgfqpoint{5.072188in}{2.094596in}}%
\pgfpathlineto{\pgfqpoint{5.081875in}{1.956534in}}%
\pgfpathlineto{\pgfqpoint{5.086719in}{1.896747in}}%
\pgfpathlineto{\pgfqpoint{5.091563in}{1.851044in}}%
\pgfpathlineto{\pgfqpoint{5.096406in}{1.829570in}}%
\pgfpathlineto{\pgfqpoint{5.101250in}{1.841887in}}%
\pgfpathlineto{\pgfqpoint{5.106094in}{1.895452in}}%
\pgfpathlineto{\pgfqpoint{5.110938in}{1.986821in}}%
\pgfpathlineto{\pgfqpoint{5.125469in}{2.323540in}}%
\pgfpathlineto{\pgfqpoint{5.130313in}{2.395026in}}%
\pgfpathlineto{\pgfqpoint{5.135156in}{2.434612in}}%
\pgfpathlineto{\pgfqpoint{5.140000in}{2.448841in}}%
\pgfpathlineto{\pgfqpoint{5.144844in}{2.444767in}}%
\pgfpathlineto{\pgfqpoint{5.149687in}{2.428632in}}%
\pgfpathlineto{\pgfqpoint{5.154531in}{2.400245in}}%
\pgfpathlineto{\pgfqpoint{5.169062in}{2.281516in}}%
\pgfpathlineto{\pgfqpoint{5.173906in}{2.261123in}}%
\pgfpathlineto{\pgfqpoint{5.178750in}{2.263401in}}%
\pgfpathlineto{\pgfqpoint{5.183594in}{2.288724in}}%
\pgfpathlineto{\pgfqpoint{5.198125in}{2.390922in}}%
\pgfpathlineto{\pgfqpoint{5.202969in}{2.408044in}}%
\pgfpathlineto{\pgfqpoint{5.207813in}{2.412450in}}%
\pgfpathlineto{\pgfqpoint{5.212656in}{2.402732in}}%
\pgfpathlineto{\pgfqpoint{5.217500in}{2.377316in}}%
\pgfpathlineto{\pgfqpoint{5.222344in}{2.332428in}}%
\pgfpathlineto{\pgfqpoint{5.227188in}{2.264785in}}%
\pgfpathlineto{\pgfqpoint{5.236875in}{2.079178in}}%
\pgfpathlineto{\pgfqpoint{5.246563in}{1.901310in}}%
\pgfpathlineto{\pgfqpoint{5.251406in}{1.847189in}}%
\pgfpathlineto{\pgfqpoint{5.256250in}{1.825934in}}%
\pgfpathlineto{\pgfqpoint{5.261094in}{1.836933in}}%
\pgfpathlineto{\pgfqpoint{5.265937in}{1.874915in}}%
\pgfpathlineto{\pgfqpoint{5.275625in}{2.009430in}}%
\pgfpathlineto{\pgfqpoint{5.290156in}{2.239641in}}%
\pgfpathlineto{\pgfqpoint{5.295000in}{2.288251in}}%
\pgfpathlineto{\pgfqpoint{5.299844in}{2.312257in}}%
\pgfpathlineto{\pgfqpoint{5.304688in}{2.313066in}}%
\pgfpathlineto{\pgfqpoint{5.309531in}{2.293967in}}%
\pgfpathlineto{\pgfqpoint{5.314375in}{2.263606in}}%
\pgfpathlineto{\pgfqpoint{5.324063in}{2.184304in}}%
\pgfpathlineto{\pgfqpoint{5.333750in}{2.098042in}}%
\pgfpathlineto{\pgfqpoint{5.338594in}{2.065316in}}%
\pgfpathlineto{\pgfqpoint{5.343438in}{2.049985in}}%
\pgfpathlineto{\pgfqpoint{5.348281in}{2.054403in}}%
\pgfpathlineto{\pgfqpoint{5.353125in}{2.073184in}}%
\pgfpathlineto{\pgfqpoint{5.362813in}{2.127578in}}%
\pgfpathlineto{\pgfqpoint{5.367656in}{2.144295in}}%
\pgfpathlineto{\pgfqpoint{5.372500in}{2.150094in}}%
\pgfpathlineto{\pgfqpoint{5.377344in}{2.141619in}}%
\pgfpathlineto{\pgfqpoint{5.382187in}{2.117001in}}%
\pgfpathlineto{\pgfqpoint{5.387031in}{2.070914in}}%
\pgfpathlineto{\pgfqpoint{5.396719in}{1.922902in}}%
\pgfpathlineto{\pgfqpoint{5.406406in}{1.778851in}}%
\pgfpathlineto{\pgfqpoint{5.411250in}{1.746403in}}%
\pgfpathlineto{\pgfqpoint{5.416094in}{1.746954in}}%
\pgfpathlineto{\pgfqpoint{5.420937in}{1.774872in}}%
\pgfpathlineto{\pgfqpoint{5.425781in}{1.821222in}}%
\pgfpathlineto{\pgfqpoint{5.435469in}{1.951412in}}%
\pgfpathlineto{\pgfqpoint{5.445156in}{2.134738in}}%
\pgfpathlineto{\pgfqpoint{5.450000in}{2.230939in}}%
\pgfpathlineto{\pgfqpoint{5.454844in}{2.304395in}}%
\pgfpathlineto{\pgfqpoint{5.459688in}{2.338815in}}%
\pgfpathlineto{\pgfqpoint{5.464531in}{2.324930in}}%
\pgfpathlineto{\pgfqpoint{5.469375in}{2.270915in}}%
\pgfpathlineto{\pgfqpoint{5.479063in}{2.135554in}}%
\pgfpathlineto{\pgfqpoint{5.483906in}{2.088503in}}%
\pgfpathlineto{\pgfqpoint{5.488750in}{2.062588in}}%
\pgfpathlineto{\pgfqpoint{5.508125in}{1.993226in}}%
\pgfpathlineto{\pgfqpoint{5.512969in}{1.991129in}}%
\pgfpathlineto{\pgfqpoint{5.517812in}{2.008522in}}%
\pgfpathlineto{\pgfqpoint{5.532344in}{2.100516in}}%
\pgfpathlineto{\pgfqpoint{5.537187in}{2.097853in}}%
\pgfpathlineto{\pgfqpoint{5.542031in}{2.065397in}}%
\pgfpathlineto{\pgfqpoint{5.551719in}{1.938040in}}%
\pgfpathlineto{\pgfqpoint{5.561406in}{1.806534in}}%
\pgfpathlineto{\pgfqpoint{5.566250in}{1.763939in}}%
\pgfpathlineto{\pgfqpoint{5.571094in}{1.743517in}}%
\pgfpathlineto{\pgfqpoint{5.575938in}{1.749051in}}%
\pgfpathlineto{\pgfqpoint{5.580781in}{1.784110in}}%
\pgfpathlineto{\pgfqpoint{5.585625in}{1.843903in}}%
\pgfpathlineto{\pgfqpoint{5.605000in}{2.154671in}}%
\pgfpathlineto{\pgfqpoint{5.609844in}{2.199920in}}%
\pgfpathlineto{\pgfqpoint{5.614688in}{2.222883in}}%
\pgfpathlineto{\pgfqpoint{5.619531in}{2.227639in}}%
\pgfpathlineto{\pgfqpoint{5.624375in}{2.217554in}}%
\pgfpathlineto{\pgfqpoint{5.629219in}{2.199322in}}%
\pgfpathlineto{\pgfqpoint{5.638906in}{2.155358in}}%
\pgfpathlineto{\pgfqpoint{5.643750in}{2.143727in}}%
\pgfpathlineto{\pgfqpoint{5.648594in}{2.152581in}}%
\pgfpathlineto{\pgfqpoint{5.653437in}{2.188344in}}%
\pgfpathlineto{\pgfqpoint{5.658281in}{2.259370in}}%
\pgfpathlineto{\pgfqpoint{5.663125in}{2.364911in}}%
\pgfpathlineto{\pgfqpoint{5.682500in}{2.887809in}}%
\pgfpathlineto{\pgfqpoint{5.687344in}{2.962947in}}%
\pgfpathlineto{\pgfqpoint{5.692188in}{3.003702in}}%
\pgfpathlineto{\pgfqpoint{5.697031in}{3.013145in}}%
\pgfpathlineto{\pgfqpoint{5.701875in}{2.999357in}}%
\pgfpathlineto{\pgfqpoint{5.706719in}{2.966786in}}%
\pgfpathlineto{\pgfqpoint{5.711563in}{2.920076in}}%
\pgfpathlineto{\pgfqpoint{5.726094in}{2.746530in}}%
\pgfpathlineto{\pgfqpoint{5.730938in}{2.712023in}}%
\pgfpathlineto{\pgfqpoint{5.735781in}{2.703540in}}%
\pgfpathlineto{\pgfqpoint{5.740625in}{2.715888in}}%
\pgfpathlineto{\pgfqpoint{5.750312in}{2.759155in}}%
\pgfpathlineto{\pgfqpoint{5.755156in}{2.764555in}}%
\pgfpathlineto{\pgfqpoint{5.760000in}{2.747347in}}%
\pgfpathlineto{\pgfqpoint{5.764844in}{2.700840in}}%
\pgfpathlineto{\pgfqpoint{5.764844in}{2.700840in}}%
\pgfusepath{stroke}%
\end{pgfscope}%
\begin{pgfscope}%
\pgfsetrectcap%
\pgfsetmiterjoin%
\pgfsetlinewidth{0.803000pt}%
\definecolor{currentstroke}{rgb}{0.000000,0.000000,0.000000}%
\pgfsetstrokecolor{currentstroke}%
\pgfsetdash{}{0pt}%
\pgfpathmoveto{\pgfqpoint{0.800000in}{0.528000in}}%
\pgfpathlineto{\pgfqpoint{0.800000in}{4.224000in}}%
\pgfusepath{stroke}%
\end{pgfscope}%
\begin{pgfscope}%
\pgfsetrectcap%
\pgfsetmiterjoin%
\pgfsetlinewidth{0.803000pt}%
\definecolor{currentstroke}{rgb}{0.000000,0.000000,0.000000}%
\pgfsetstrokecolor{currentstroke}%
\pgfsetdash{}{0pt}%
\pgfpathmoveto{\pgfqpoint{5.760000in}{0.528000in}}%
\pgfpathlineto{\pgfqpoint{5.760000in}{4.224000in}}%
\pgfusepath{stroke}%
\end{pgfscope}%
\begin{pgfscope}%
\pgfsetrectcap%
\pgfsetmiterjoin%
\pgfsetlinewidth{0.803000pt}%
\definecolor{currentstroke}{rgb}{0.000000,0.000000,0.000000}%
\pgfsetstrokecolor{currentstroke}%
\pgfsetdash{}{0pt}%
\pgfpathmoveto{\pgfqpoint{0.800000in}{0.528000in}}%
\pgfpathlineto{\pgfqpoint{5.760000in}{0.528000in}}%
\pgfusepath{stroke}%
\end{pgfscope}%
\begin{pgfscope}%
\pgfsetrectcap%
\pgfsetmiterjoin%
\pgfsetlinewidth{0.803000pt}%
\definecolor{currentstroke}{rgb}{0.000000,0.000000,0.000000}%
\pgfsetstrokecolor{currentstroke}%
\pgfsetdash{}{0pt}%
\pgfpathmoveto{\pgfqpoint{0.800000in}{4.224000in}}%
\pgfpathlineto{\pgfqpoint{5.760000in}{4.224000in}}%
\pgfusepath{stroke}%
\end{pgfscope}%
\begin{pgfscope}%
\definecolor{textcolor}{rgb}{0.000000,0.000000,0.000000}%
\pgfsetstrokecolor{textcolor}%
\pgfsetfillcolor{textcolor}%
\pgftext[x=3.280000in,y=4.307333in,,base]{\color{textcolor}\rmfamily\fontsize{12.000000}{14.400000}\selectfont Delay of Window FIR Filter}%
\end{pgfscope}%
\end{pgfpicture}%
\makeatother%
\endgroup%
}
    \end{subfigure}
    \begin{subfigure}{0.5\textwidth}
        \resizebox{\linewidth}{!}{%% Creator: Matplotlib, PGF backend
%%
%% To include the figure in your LaTeX document, write
%%   \input{<filename>.pgf}
%%
%% Make sure the required packages are loaded in your preamble
%%   \usepackage{pgf}
%%
%% and, on pdftex
%%   \usepackage[utf8]{inputenc}\DeclareUnicodeCharacter{2212}{-}
%%
%% or, on luatex and xetex
%%   \usepackage{unicode-math}
%%
%% Figures using additional raster images can only be included by \input if
%% they are in the same directory as the main LaTeX file. For loading figures
%% from other directories you can use the `import` package
%%   \usepackage{import}
%%
%% and then include the figures with
%%   \import{<path to file>}{<filename>.pgf}
%%
%% Matplotlib used the following preamble
%%
\begingroup%
\makeatletter%
\begin{pgfpicture}%
\pgfpathrectangle{\pgfpointorigin}{\pgfqpoint{6.400000in}{4.800000in}}%
\pgfusepath{use as bounding box, clip}%
\begin{pgfscope}%
\pgfsetbuttcap%
\pgfsetmiterjoin%
\definecolor{currentfill}{rgb}{1.000000,1.000000,1.000000}%
\pgfsetfillcolor{currentfill}%
\pgfsetlinewidth{0.000000pt}%
\definecolor{currentstroke}{rgb}{1.000000,1.000000,1.000000}%
\pgfsetstrokecolor{currentstroke}%
\pgfsetdash{}{0pt}%
\pgfpathmoveto{\pgfqpoint{0.000000in}{0.000000in}}%
\pgfpathlineto{\pgfqpoint{6.400000in}{0.000000in}}%
\pgfpathlineto{\pgfqpoint{6.400000in}{4.800000in}}%
\pgfpathlineto{\pgfqpoint{0.000000in}{4.800000in}}%
\pgfpathclose%
\pgfusepath{fill}%
\end{pgfscope}%
\begin{pgfscope}%
\pgfsetbuttcap%
\pgfsetmiterjoin%
\definecolor{currentfill}{rgb}{1.000000,1.000000,1.000000}%
\pgfsetfillcolor{currentfill}%
\pgfsetlinewidth{0.000000pt}%
\definecolor{currentstroke}{rgb}{0.000000,0.000000,0.000000}%
\pgfsetstrokecolor{currentstroke}%
\pgfsetstrokeopacity{0.000000}%
\pgfsetdash{}{0pt}%
\pgfpathmoveto{\pgfqpoint{0.800000in}{0.528000in}}%
\pgfpathlineto{\pgfqpoint{5.760000in}{0.528000in}}%
\pgfpathlineto{\pgfqpoint{5.760000in}{4.224000in}}%
\pgfpathlineto{\pgfqpoint{0.800000in}{4.224000in}}%
\pgfpathclose%
\pgfusepath{fill}%
\end{pgfscope}%
\begin{pgfscope}%
\pgfsetbuttcap%
\pgfsetroundjoin%
\definecolor{currentfill}{rgb}{0.000000,0.000000,0.000000}%
\pgfsetfillcolor{currentfill}%
\pgfsetlinewidth{0.803000pt}%
\definecolor{currentstroke}{rgb}{0.000000,0.000000,0.000000}%
\pgfsetstrokecolor{currentstroke}%
\pgfsetdash{}{0pt}%
\pgfsys@defobject{currentmarker}{\pgfqpoint{0.000000in}{-0.048611in}}{\pgfqpoint{0.000000in}{0.000000in}}{%
\pgfpathmoveto{\pgfqpoint{0.000000in}{0.000000in}}%
\pgfpathlineto{\pgfqpoint{0.000000in}{-0.048611in}}%
\pgfusepath{stroke,fill}%
}%
\begin{pgfscope}%
\pgfsys@transformshift{0.800000in}{0.528000in}%
\pgfsys@useobject{currentmarker}{}%
\end{pgfscope}%
\end{pgfscope}%
\begin{pgfscope}%
\definecolor{textcolor}{rgb}{0.000000,0.000000,0.000000}%
\pgfsetstrokecolor{textcolor}%
\pgfsetfillcolor{textcolor}%
\pgftext[x=0.800000in,y=0.430778in,,top]{\color{textcolor}\rmfamily\fontsize{10.000000}{12.000000}\selectfont \(\displaystyle {10.0}\)}%
\end{pgfscope}%
\begin{pgfscope}%
\pgfsetbuttcap%
\pgfsetroundjoin%
\definecolor{currentfill}{rgb}{0.000000,0.000000,0.000000}%
\pgfsetfillcolor{currentfill}%
\pgfsetlinewidth{0.803000pt}%
\definecolor{currentstroke}{rgb}{0.000000,0.000000,0.000000}%
\pgfsetstrokecolor{currentstroke}%
\pgfsetdash{}{0pt}%
\pgfsys@defobject{currentmarker}{\pgfqpoint{0.000000in}{-0.048611in}}{\pgfqpoint{0.000000in}{0.000000in}}{%
\pgfpathmoveto{\pgfqpoint{0.000000in}{0.000000in}}%
\pgfpathlineto{\pgfqpoint{0.000000in}{-0.048611in}}%
\pgfusepath{stroke,fill}%
}%
\begin{pgfscope}%
\pgfsys@transformshift{1.792000in}{0.528000in}%
\pgfsys@useobject{currentmarker}{}%
\end{pgfscope}%
\end{pgfscope}%
\begin{pgfscope}%
\definecolor{textcolor}{rgb}{0.000000,0.000000,0.000000}%
\pgfsetstrokecolor{textcolor}%
\pgfsetfillcolor{textcolor}%
\pgftext[x=1.792000in,y=0.430778in,,top]{\color{textcolor}\rmfamily\fontsize{10.000000}{12.000000}\selectfont \(\displaystyle {10.2}\)}%
\end{pgfscope}%
\begin{pgfscope}%
\pgfsetbuttcap%
\pgfsetroundjoin%
\definecolor{currentfill}{rgb}{0.000000,0.000000,0.000000}%
\pgfsetfillcolor{currentfill}%
\pgfsetlinewidth{0.803000pt}%
\definecolor{currentstroke}{rgb}{0.000000,0.000000,0.000000}%
\pgfsetstrokecolor{currentstroke}%
\pgfsetdash{}{0pt}%
\pgfsys@defobject{currentmarker}{\pgfqpoint{0.000000in}{-0.048611in}}{\pgfqpoint{0.000000in}{0.000000in}}{%
\pgfpathmoveto{\pgfqpoint{0.000000in}{0.000000in}}%
\pgfpathlineto{\pgfqpoint{0.000000in}{-0.048611in}}%
\pgfusepath{stroke,fill}%
}%
\begin{pgfscope}%
\pgfsys@transformshift{2.784000in}{0.528000in}%
\pgfsys@useobject{currentmarker}{}%
\end{pgfscope}%
\end{pgfscope}%
\begin{pgfscope}%
\definecolor{textcolor}{rgb}{0.000000,0.000000,0.000000}%
\pgfsetstrokecolor{textcolor}%
\pgfsetfillcolor{textcolor}%
\pgftext[x=2.784000in,y=0.430778in,,top]{\color{textcolor}\rmfamily\fontsize{10.000000}{12.000000}\selectfont \(\displaystyle {10.4}\)}%
\end{pgfscope}%
\begin{pgfscope}%
\pgfsetbuttcap%
\pgfsetroundjoin%
\definecolor{currentfill}{rgb}{0.000000,0.000000,0.000000}%
\pgfsetfillcolor{currentfill}%
\pgfsetlinewidth{0.803000pt}%
\definecolor{currentstroke}{rgb}{0.000000,0.000000,0.000000}%
\pgfsetstrokecolor{currentstroke}%
\pgfsetdash{}{0pt}%
\pgfsys@defobject{currentmarker}{\pgfqpoint{0.000000in}{-0.048611in}}{\pgfqpoint{0.000000in}{0.000000in}}{%
\pgfpathmoveto{\pgfqpoint{0.000000in}{0.000000in}}%
\pgfpathlineto{\pgfqpoint{0.000000in}{-0.048611in}}%
\pgfusepath{stroke,fill}%
}%
\begin{pgfscope}%
\pgfsys@transformshift{3.776000in}{0.528000in}%
\pgfsys@useobject{currentmarker}{}%
\end{pgfscope}%
\end{pgfscope}%
\begin{pgfscope}%
\definecolor{textcolor}{rgb}{0.000000,0.000000,0.000000}%
\pgfsetstrokecolor{textcolor}%
\pgfsetfillcolor{textcolor}%
\pgftext[x=3.776000in,y=0.430778in,,top]{\color{textcolor}\rmfamily\fontsize{10.000000}{12.000000}\selectfont \(\displaystyle {10.6}\)}%
\end{pgfscope}%
\begin{pgfscope}%
\pgfsetbuttcap%
\pgfsetroundjoin%
\definecolor{currentfill}{rgb}{0.000000,0.000000,0.000000}%
\pgfsetfillcolor{currentfill}%
\pgfsetlinewidth{0.803000pt}%
\definecolor{currentstroke}{rgb}{0.000000,0.000000,0.000000}%
\pgfsetstrokecolor{currentstroke}%
\pgfsetdash{}{0pt}%
\pgfsys@defobject{currentmarker}{\pgfqpoint{0.000000in}{-0.048611in}}{\pgfqpoint{0.000000in}{0.000000in}}{%
\pgfpathmoveto{\pgfqpoint{0.000000in}{0.000000in}}%
\pgfpathlineto{\pgfqpoint{0.000000in}{-0.048611in}}%
\pgfusepath{stroke,fill}%
}%
\begin{pgfscope}%
\pgfsys@transformshift{4.768000in}{0.528000in}%
\pgfsys@useobject{currentmarker}{}%
\end{pgfscope}%
\end{pgfscope}%
\begin{pgfscope}%
\definecolor{textcolor}{rgb}{0.000000,0.000000,0.000000}%
\pgfsetstrokecolor{textcolor}%
\pgfsetfillcolor{textcolor}%
\pgftext[x=4.768000in,y=0.430778in,,top]{\color{textcolor}\rmfamily\fontsize{10.000000}{12.000000}\selectfont \(\displaystyle {10.8}\)}%
\end{pgfscope}%
\begin{pgfscope}%
\pgfsetbuttcap%
\pgfsetroundjoin%
\definecolor{currentfill}{rgb}{0.000000,0.000000,0.000000}%
\pgfsetfillcolor{currentfill}%
\pgfsetlinewidth{0.803000pt}%
\definecolor{currentstroke}{rgb}{0.000000,0.000000,0.000000}%
\pgfsetstrokecolor{currentstroke}%
\pgfsetdash{}{0pt}%
\pgfsys@defobject{currentmarker}{\pgfqpoint{0.000000in}{-0.048611in}}{\pgfqpoint{0.000000in}{0.000000in}}{%
\pgfpathmoveto{\pgfqpoint{0.000000in}{0.000000in}}%
\pgfpathlineto{\pgfqpoint{0.000000in}{-0.048611in}}%
\pgfusepath{stroke,fill}%
}%
\begin{pgfscope}%
\pgfsys@transformshift{5.760000in}{0.528000in}%
\pgfsys@useobject{currentmarker}{}%
\end{pgfscope}%
\end{pgfscope}%
\begin{pgfscope}%
\definecolor{textcolor}{rgb}{0.000000,0.000000,0.000000}%
\pgfsetstrokecolor{textcolor}%
\pgfsetfillcolor{textcolor}%
\pgftext[x=5.760000in,y=0.430778in,,top]{\color{textcolor}\rmfamily\fontsize{10.000000}{12.000000}\selectfont \(\displaystyle {11.0}\)}%
\end{pgfscope}%
\begin{pgfscope}%
\definecolor{textcolor}{rgb}{0.000000,0.000000,0.000000}%
\pgfsetstrokecolor{textcolor}%
\pgfsetfillcolor{textcolor}%
\pgftext[x=3.280000in,y=0.251766in,,top]{\color{textcolor}\rmfamily\fontsize{10.000000}{12.000000}\selectfont Time (s)}%
\end{pgfscope}%
\begin{pgfscope}%
\pgfsetbuttcap%
\pgfsetroundjoin%
\definecolor{currentfill}{rgb}{0.000000,0.000000,0.000000}%
\pgfsetfillcolor{currentfill}%
\pgfsetlinewidth{0.803000pt}%
\definecolor{currentstroke}{rgb}{0.000000,0.000000,0.000000}%
\pgfsetstrokecolor{currentstroke}%
\pgfsetdash{}{0pt}%
\pgfsys@defobject{currentmarker}{\pgfqpoint{-0.048611in}{0.000000in}}{\pgfqpoint{0.000000in}{0.000000in}}{%
\pgfpathmoveto{\pgfqpoint{0.000000in}{0.000000in}}%
\pgfpathlineto{\pgfqpoint{-0.048611in}{0.000000in}}%
\pgfusepath{stroke,fill}%
}%
\begin{pgfscope}%
\pgfsys@transformshift{0.800000in}{0.823350in}%
\pgfsys@useobject{currentmarker}{}%
\end{pgfscope}%
\end{pgfscope}%
\begin{pgfscope}%
\definecolor{textcolor}{rgb}{0.000000,0.000000,0.000000}%
\pgfsetstrokecolor{textcolor}%
\pgfsetfillcolor{textcolor}%
\pgftext[x=0.316974in, y=0.775125in, left, base]{\color{textcolor}\rmfamily\fontsize{10.000000}{12.000000}\selectfont \(\displaystyle {-1000}\)}%
\end{pgfscope}%
\begin{pgfscope}%
\pgfsetbuttcap%
\pgfsetroundjoin%
\definecolor{currentfill}{rgb}{0.000000,0.000000,0.000000}%
\pgfsetfillcolor{currentfill}%
\pgfsetlinewidth{0.803000pt}%
\definecolor{currentstroke}{rgb}{0.000000,0.000000,0.000000}%
\pgfsetstrokecolor{currentstroke}%
\pgfsetdash{}{0pt}%
\pgfsys@defobject{currentmarker}{\pgfqpoint{-0.048611in}{0.000000in}}{\pgfqpoint{0.000000in}{0.000000in}}{%
\pgfpathmoveto{\pgfqpoint{0.000000in}{0.000000in}}%
\pgfpathlineto{\pgfqpoint{-0.048611in}{0.000000in}}%
\pgfusepath{stroke,fill}%
}%
\begin{pgfscope}%
\pgfsys@transformshift{0.800000in}{1.467604in}%
\pgfsys@useobject{currentmarker}{}%
\end{pgfscope}%
\end{pgfscope}%
\begin{pgfscope}%
\definecolor{textcolor}{rgb}{0.000000,0.000000,0.000000}%
\pgfsetstrokecolor{textcolor}%
\pgfsetfillcolor{textcolor}%
\pgftext[x=0.386419in, y=1.419378in, left, base]{\color{textcolor}\rmfamily\fontsize{10.000000}{12.000000}\selectfont \(\displaystyle {-500}\)}%
\end{pgfscope}%
\begin{pgfscope}%
\pgfsetbuttcap%
\pgfsetroundjoin%
\definecolor{currentfill}{rgb}{0.000000,0.000000,0.000000}%
\pgfsetfillcolor{currentfill}%
\pgfsetlinewidth{0.803000pt}%
\definecolor{currentstroke}{rgb}{0.000000,0.000000,0.000000}%
\pgfsetstrokecolor{currentstroke}%
\pgfsetdash{}{0pt}%
\pgfsys@defobject{currentmarker}{\pgfqpoint{-0.048611in}{0.000000in}}{\pgfqpoint{0.000000in}{0.000000in}}{%
\pgfpathmoveto{\pgfqpoint{0.000000in}{0.000000in}}%
\pgfpathlineto{\pgfqpoint{-0.048611in}{0.000000in}}%
\pgfusepath{stroke,fill}%
}%
\begin{pgfscope}%
\pgfsys@transformshift{0.800000in}{2.111857in}%
\pgfsys@useobject{currentmarker}{}%
\end{pgfscope}%
\end{pgfscope}%
\begin{pgfscope}%
\definecolor{textcolor}{rgb}{0.000000,0.000000,0.000000}%
\pgfsetstrokecolor{textcolor}%
\pgfsetfillcolor{textcolor}%
\pgftext[x=0.633333in, y=2.063632in, left, base]{\color{textcolor}\rmfamily\fontsize{10.000000}{12.000000}\selectfont \(\displaystyle {0}\)}%
\end{pgfscope}%
\begin{pgfscope}%
\pgfsetbuttcap%
\pgfsetroundjoin%
\definecolor{currentfill}{rgb}{0.000000,0.000000,0.000000}%
\pgfsetfillcolor{currentfill}%
\pgfsetlinewidth{0.803000pt}%
\definecolor{currentstroke}{rgb}{0.000000,0.000000,0.000000}%
\pgfsetstrokecolor{currentstroke}%
\pgfsetdash{}{0pt}%
\pgfsys@defobject{currentmarker}{\pgfqpoint{-0.048611in}{0.000000in}}{\pgfqpoint{0.000000in}{0.000000in}}{%
\pgfpathmoveto{\pgfqpoint{0.000000in}{0.000000in}}%
\pgfpathlineto{\pgfqpoint{-0.048611in}{0.000000in}}%
\pgfusepath{stroke,fill}%
}%
\begin{pgfscope}%
\pgfsys@transformshift{0.800000in}{2.756111in}%
\pgfsys@useobject{currentmarker}{}%
\end{pgfscope}%
\end{pgfscope}%
\begin{pgfscope}%
\definecolor{textcolor}{rgb}{0.000000,0.000000,0.000000}%
\pgfsetstrokecolor{textcolor}%
\pgfsetfillcolor{textcolor}%
\pgftext[x=0.494444in, y=2.707885in, left, base]{\color{textcolor}\rmfamily\fontsize{10.000000}{12.000000}\selectfont \(\displaystyle {500}\)}%
\end{pgfscope}%
\begin{pgfscope}%
\pgfsetbuttcap%
\pgfsetroundjoin%
\definecolor{currentfill}{rgb}{0.000000,0.000000,0.000000}%
\pgfsetfillcolor{currentfill}%
\pgfsetlinewidth{0.803000pt}%
\definecolor{currentstroke}{rgb}{0.000000,0.000000,0.000000}%
\pgfsetstrokecolor{currentstroke}%
\pgfsetdash{}{0pt}%
\pgfsys@defobject{currentmarker}{\pgfqpoint{-0.048611in}{0.000000in}}{\pgfqpoint{0.000000in}{0.000000in}}{%
\pgfpathmoveto{\pgfqpoint{0.000000in}{0.000000in}}%
\pgfpathlineto{\pgfqpoint{-0.048611in}{0.000000in}}%
\pgfusepath{stroke,fill}%
}%
\begin{pgfscope}%
\pgfsys@transformshift{0.800000in}{3.400364in}%
\pgfsys@useobject{currentmarker}{}%
\end{pgfscope}%
\end{pgfscope}%
\begin{pgfscope}%
\definecolor{textcolor}{rgb}{0.000000,0.000000,0.000000}%
\pgfsetstrokecolor{textcolor}%
\pgfsetfillcolor{textcolor}%
\pgftext[x=0.424999in, y=3.352139in, left, base]{\color{textcolor}\rmfamily\fontsize{10.000000}{12.000000}\selectfont \(\displaystyle {1000}\)}%
\end{pgfscope}%
\begin{pgfscope}%
\pgfsetbuttcap%
\pgfsetroundjoin%
\definecolor{currentfill}{rgb}{0.000000,0.000000,0.000000}%
\pgfsetfillcolor{currentfill}%
\pgfsetlinewidth{0.803000pt}%
\definecolor{currentstroke}{rgb}{0.000000,0.000000,0.000000}%
\pgfsetstrokecolor{currentstroke}%
\pgfsetdash{}{0pt}%
\pgfsys@defobject{currentmarker}{\pgfqpoint{-0.048611in}{0.000000in}}{\pgfqpoint{0.000000in}{0.000000in}}{%
\pgfpathmoveto{\pgfqpoint{0.000000in}{0.000000in}}%
\pgfpathlineto{\pgfqpoint{-0.048611in}{0.000000in}}%
\pgfusepath{stroke,fill}%
}%
\begin{pgfscope}%
\pgfsys@transformshift{0.800000in}{4.044618in}%
\pgfsys@useobject{currentmarker}{}%
\end{pgfscope}%
\end{pgfscope}%
\begin{pgfscope}%
\definecolor{textcolor}{rgb}{0.000000,0.000000,0.000000}%
\pgfsetstrokecolor{textcolor}%
\pgfsetfillcolor{textcolor}%
\pgftext[x=0.424999in, y=3.996392in, left, base]{\color{textcolor}\rmfamily\fontsize{10.000000}{12.000000}\selectfont \(\displaystyle {1500}\)}%
\end{pgfscope}%
\begin{pgfscope}%
\definecolor{textcolor}{rgb}{0.000000,0.000000,0.000000}%
\pgfsetstrokecolor{textcolor}%
\pgfsetfillcolor{textcolor}%
\pgftext[x=0.261419in,y=2.376000in,,bottom,rotate=90.000000]{\color{textcolor}\rmfamily\fontsize{10.000000}{12.000000}\selectfont ECG Voltage (\(\displaystyle \mu V\))}%
\end{pgfscope}%
\begin{pgfscope}%
\pgfpathrectangle{\pgfqpoint{0.800000in}{0.528000in}}{\pgfqpoint{4.960000in}{3.696000in}}%
\pgfusepath{clip}%
\pgfsetrectcap%
\pgfsetroundjoin%
\pgfsetlinewidth{1.505625pt}%
\definecolor{currentstroke}{rgb}{0.121569,0.466667,0.705882}%
\pgfsetstrokecolor{currentstroke}%
\pgfsetdash{}{0pt}%
\pgfpathmoveto{\pgfqpoint{0.795156in}{1.106603in}}%
\pgfpathlineto{\pgfqpoint{0.809688in}{1.061495in}}%
\pgfpathlineto{\pgfqpoint{0.814531in}{1.053533in}}%
\pgfpathlineto{\pgfqpoint{0.829063in}{1.041020in}}%
\pgfpathlineto{\pgfqpoint{0.838750in}{1.025225in}}%
\pgfpathlineto{\pgfqpoint{0.843594in}{1.022472in}}%
\pgfpathlineto{\pgfqpoint{0.848438in}{1.026450in}}%
\pgfpathlineto{\pgfqpoint{0.862969in}{1.064268in}}%
\pgfpathlineto{\pgfqpoint{0.867812in}{1.067092in}}%
\pgfpathlineto{\pgfqpoint{0.872656in}{1.059712in}}%
\pgfpathlineto{\pgfqpoint{0.882344in}{1.023121in}}%
\pgfpathlineto{\pgfqpoint{0.887188in}{1.005600in}}%
\pgfpathlineto{\pgfqpoint{0.892031in}{0.993257in}}%
\pgfpathlineto{\pgfqpoint{0.896875in}{0.989874in}}%
\pgfpathlineto{\pgfqpoint{0.901719in}{0.998909in}}%
\pgfpathlineto{\pgfqpoint{0.911406in}{1.039367in}}%
\pgfpathlineto{\pgfqpoint{0.921094in}{1.079531in}}%
\pgfpathlineto{\pgfqpoint{0.925937in}{1.091594in}}%
\pgfpathlineto{\pgfqpoint{0.935625in}{1.111044in}}%
\pgfpathlineto{\pgfqpoint{0.945312in}{1.142707in}}%
\pgfpathlineto{\pgfqpoint{0.950156in}{1.160347in}}%
\pgfpathlineto{\pgfqpoint{0.955000in}{1.172146in}}%
\pgfpathlineto{\pgfqpoint{0.959844in}{1.179627in}}%
\pgfpathlineto{\pgfqpoint{0.969531in}{1.190108in}}%
\pgfpathlineto{\pgfqpoint{0.984063in}{1.214459in}}%
\pgfpathlineto{\pgfqpoint{0.988906in}{1.216423in}}%
\pgfpathlineto{\pgfqpoint{0.993750in}{1.212741in}}%
\pgfpathlineto{\pgfqpoint{0.998594in}{1.211358in}}%
\pgfpathlineto{\pgfqpoint{1.003437in}{1.214602in}}%
\pgfpathlineto{\pgfqpoint{1.008281in}{1.227099in}}%
\pgfpathlineto{\pgfqpoint{1.017969in}{1.255906in}}%
\pgfpathlineto{\pgfqpoint{1.022813in}{1.260618in}}%
\pgfpathlineto{\pgfqpoint{1.032500in}{1.256132in}}%
\pgfpathlineto{\pgfqpoint{1.037344in}{1.259840in}}%
\pgfpathlineto{\pgfqpoint{1.042188in}{1.275852in}}%
\pgfpathlineto{\pgfqpoint{1.051875in}{1.313333in}}%
\pgfpathlineto{\pgfqpoint{1.056719in}{1.314810in}}%
\pgfpathlineto{\pgfqpoint{1.061562in}{1.300194in}}%
\pgfpathlineto{\pgfqpoint{1.071250in}{1.260504in}}%
\pgfpathlineto{\pgfqpoint{1.076094in}{1.259356in}}%
\pgfpathlineto{\pgfqpoint{1.080938in}{1.279727in}}%
\pgfpathlineto{\pgfqpoint{1.090625in}{1.345457in}}%
\pgfpathlineto{\pgfqpoint{1.095469in}{1.364867in}}%
\pgfpathlineto{\pgfqpoint{1.100313in}{1.366812in}}%
\pgfpathlineto{\pgfqpoint{1.105156in}{1.353998in}}%
\pgfpathlineto{\pgfqpoint{1.114844in}{1.324715in}}%
\pgfpathlineto{\pgfqpoint{1.119687in}{1.318532in}}%
\pgfpathlineto{\pgfqpoint{1.129375in}{1.314396in}}%
\pgfpathlineto{\pgfqpoint{1.134219in}{1.309271in}}%
\pgfpathlineto{\pgfqpoint{1.139063in}{1.300567in}}%
\pgfpathlineto{\pgfqpoint{1.148750in}{1.276844in}}%
\pgfpathlineto{\pgfqpoint{1.153594in}{1.263957in}}%
\pgfpathlineto{\pgfqpoint{1.158438in}{1.255583in}}%
\pgfpathlineto{\pgfqpoint{1.163281in}{1.260145in}}%
\pgfpathlineto{\pgfqpoint{1.168125in}{1.284436in}}%
\pgfpathlineto{\pgfqpoint{1.172969in}{1.335044in}}%
\pgfpathlineto{\pgfqpoint{1.182656in}{1.502473in}}%
\pgfpathlineto{\pgfqpoint{1.197188in}{1.768335in}}%
\pgfpathlineto{\pgfqpoint{1.221406in}{2.149217in}}%
\pgfpathlineto{\pgfqpoint{1.226250in}{2.196210in}}%
\pgfpathlineto{\pgfqpoint{1.231094in}{2.221945in}}%
\pgfpathlineto{\pgfqpoint{1.235938in}{2.227800in}}%
\pgfpathlineto{\pgfqpoint{1.240781in}{2.219914in}}%
\pgfpathlineto{\pgfqpoint{1.255313in}{2.166940in}}%
\pgfpathlineto{\pgfqpoint{1.260156in}{2.158734in}}%
\pgfpathlineto{\pgfqpoint{1.265000in}{2.163548in}}%
\pgfpathlineto{\pgfqpoint{1.269844in}{2.172929in}}%
\pgfpathlineto{\pgfqpoint{1.274687in}{2.179348in}}%
\pgfpathlineto{\pgfqpoint{1.279531in}{2.165941in}}%
\pgfpathlineto{\pgfqpoint{1.284375in}{2.124477in}}%
\pgfpathlineto{\pgfqpoint{1.289219in}{2.052004in}}%
\pgfpathlineto{\pgfqpoint{1.298906in}{1.839856in}}%
\pgfpathlineto{\pgfqpoint{1.318281in}{1.374159in}}%
\pgfpathlineto{\pgfqpoint{1.323125in}{1.290517in}}%
\pgfpathlineto{\pgfqpoint{1.327969in}{1.242775in}}%
\pgfpathlineto{\pgfqpoint{1.332812in}{1.236451in}}%
\pgfpathlineto{\pgfqpoint{1.337656in}{1.265680in}}%
\pgfpathlineto{\pgfqpoint{1.347344in}{1.352956in}}%
\pgfpathlineto{\pgfqpoint{1.352188in}{1.369989in}}%
\pgfpathlineto{\pgfqpoint{1.357031in}{1.355952in}}%
\pgfpathlineto{\pgfqpoint{1.361875in}{1.318218in}}%
\pgfpathlineto{\pgfqpoint{1.371563in}{1.230974in}}%
\pgfpathlineto{\pgfqpoint{1.376406in}{1.206041in}}%
\pgfpathlineto{\pgfqpoint{1.381250in}{1.196533in}}%
\pgfpathlineto{\pgfqpoint{1.386094in}{1.198199in}}%
\pgfpathlineto{\pgfqpoint{1.395781in}{1.208804in}}%
\pgfpathlineto{\pgfqpoint{1.405469in}{1.215581in}}%
\pgfpathlineto{\pgfqpoint{1.410313in}{1.221774in}}%
\pgfpathlineto{\pgfqpoint{1.415156in}{1.231365in}}%
\pgfpathlineto{\pgfqpoint{1.429688in}{1.267321in}}%
\pgfpathlineto{\pgfqpoint{1.434531in}{1.272904in}}%
\pgfpathlineto{\pgfqpoint{1.439375in}{1.274603in}}%
\pgfpathlineto{\pgfqpoint{1.444219in}{1.279419in}}%
\pgfpathlineto{\pgfqpoint{1.449062in}{1.291354in}}%
\pgfpathlineto{\pgfqpoint{1.453906in}{1.318957in}}%
\pgfpathlineto{\pgfqpoint{1.463594in}{1.410024in}}%
\pgfpathlineto{\pgfqpoint{1.468438in}{1.457850in}}%
\pgfpathlineto{\pgfqpoint{1.473281in}{1.494160in}}%
\pgfpathlineto{\pgfqpoint{1.478125in}{1.509389in}}%
\pgfpathlineto{\pgfqpoint{1.482969in}{1.506795in}}%
\pgfpathlineto{\pgfqpoint{1.487812in}{1.493212in}}%
\pgfpathlineto{\pgfqpoint{1.502344in}{1.435655in}}%
\pgfpathlineto{\pgfqpoint{1.507188in}{1.405396in}}%
\pgfpathlineto{\pgfqpoint{1.512031in}{1.362243in}}%
\pgfpathlineto{\pgfqpoint{1.526563in}{1.191373in}}%
\pgfpathlineto{\pgfqpoint{1.531406in}{1.153277in}}%
\pgfpathlineto{\pgfqpoint{1.536250in}{1.135526in}}%
\pgfpathlineto{\pgfqpoint{1.541094in}{1.137190in}}%
\pgfpathlineto{\pgfqpoint{1.545937in}{1.153860in}}%
\pgfpathlineto{\pgfqpoint{1.555625in}{1.194836in}}%
\pgfpathlineto{\pgfqpoint{1.560469in}{1.205033in}}%
\pgfpathlineto{\pgfqpoint{1.565313in}{1.206298in}}%
\pgfpathlineto{\pgfqpoint{1.570156in}{1.203130in}}%
\pgfpathlineto{\pgfqpoint{1.575000in}{1.204382in}}%
\pgfpathlineto{\pgfqpoint{1.579844in}{1.214147in}}%
\pgfpathlineto{\pgfqpoint{1.584688in}{1.234441in}}%
\pgfpathlineto{\pgfqpoint{1.594375in}{1.283829in}}%
\pgfpathlineto{\pgfqpoint{1.599219in}{1.301298in}}%
\pgfpathlineto{\pgfqpoint{1.604062in}{1.310332in}}%
\pgfpathlineto{\pgfqpoint{1.613750in}{1.322009in}}%
\pgfpathlineto{\pgfqpoint{1.618594in}{1.335337in}}%
\pgfpathlineto{\pgfqpoint{1.637969in}{1.412507in}}%
\pgfpathlineto{\pgfqpoint{1.642813in}{1.413633in}}%
\pgfpathlineto{\pgfqpoint{1.647656in}{1.402449in}}%
\pgfpathlineto{\pgfqpoint{1.652500in}{1.383650in}}%
\pgfpathlineto{\pgfqpoint{1.662187in}{1.335730in}}%
\pgfpathlineto{\pgfqpoint{1.691250in}{1.220414in}}%
\pgfpathlineto{\pgfqpoint{1.696094in}{1.213537in}}%
\pgfpathlineto{\pgfqpoint{1.700937in}{1.226086in}}%
\pgfpathlineto{\pgfqpoint{1.705781in}{1.257634in}}%
\pgfpathlineto{\pgfqpoint{1.715469in}{1.340883in}}%
\pgfpathlineto{\pgfqpoint{1.720313in}{1.370220in}}%
\pgfpathlineto{\pgfqpoint{1.725156in}{1.388385in}}%
\pgfpathlineto{\pgfqpoint{1.734844in}{1.410252in}}%
\pgfpathlineto{\pgfqpoint{1.744531in}{1.445020in}}%
\pgfpathlineto{\pgfqpoint{1.749375in}{1.462961in}}%
\pgfpathlineto{\pgfqpoint{1.754219in}{1.476428in}}%
\pgfpathlineto{\pgfqpoint{1.763906in}{1.495611in}}%
\pgfpathlineto{\pgfqpoint{1.768750in}{1.512692in}}%
\pgfpathlineto{\pgfqpoint{1.773594in}{1.541128in}}%
\pgfpathlineto{\pgfqpoint{1.788125in}{1.661530in}}%
\pgfpathlineto{\pgfqpoint{1.792969in}{1.686409in}}%
\pgfpathlineto{\pgfqpoint{1.797813in}{1.696700in}}%
\pgfpathlineto{\pgfqpoint{1.802656in}{1.692467in}}%
\pgfpathlineto{\pgfqpoint{1.812344in}{1.673962in}}%
\pgfpathlineto{\pgfqpoint{1.817187in}{1.668344in}}%
\pgfpathlineto{\pgfqpoint{1.831719in}{1.667732in}}%
\pgfpathlineto{\pgfqpoint{1.841406in}{1.658959in}}%
\pgfpathlineto{\pgfqpoint{1.846250in}{1.662908in}}%
\pgfpathlineto{\pgfqpoint{1.851094in}{1.679978in}}%
\pgfpathlineto{\pgfqpoint{1.855938in}{1.709375in}}%
\pgfpathlineto{\pgfqpoint{1.870469in}{1.811333in}}%
\pgfpathlineto{\pgfqpoint{1.880156in}{1.868883in}}%
\pgfpathlineto{\pgfqpoint{1.889844in}{1.943058in}}%
\pgfpathlineto{\pgfqpoint{1.894688in}{1.980871in}}%
\pgfpathlineto{\pgfqpoint{1.899531in}{2.006619in}}%
\pgfpathlineto{\pgfqpoint{1.904375in}{2.013981in}}%
\pgfpathlineto{\pgfqpoint{1.909219in}{2.006044in}}%
\pgfpathlineto{\pgfqpoint{1.914062in}{1.991999in}}%
\pgfpathlineto{\pgfqpoint{1.918906in}{1.986588in}}%
\pgfpathlineto{\pgfqpoint{1.923750in}{1.998326in}}%
\pgfpathlineto{\pgfqpoint{1.928594in}{2.027366in}}%
\pgfpathlineto{\pgfqpoint{1.938281in}{2.104620in}}%
\pgfpathlineto{\pgfqpoint{1.943125in}{2.130139in}}%
\pgfpathlineto{\pgfqpoint{1.947969in}{2.139182in}}%
\pgfpathlineto{\pgfqpoint{1.952813in}{2.137401in}}%
\pgfpathlineto{\pgfqpoint{1.957656in}{2.130105in}}%
\pgfpathlineto{\pgfqpoint{1.962500in}{2.127338in}}%
\pgfpathlineto{\pgfqpoint{1.967344in}{2.132319in}}%
\pgfpathlineto{\pgfqpoint{1.972187in}{2.144457in}}%
\pgfpathlineto{\pgfqpoint{1.981875in}{2.173094in}}%
\pgfpathlineto{\pgfqpoint{1.986719in}{2.179224in}}%
\pgfpathlineto{\pgfqpoint{1.991563in}{2.175522in}}%
\pgfpathlineto{\pgfqpoint{2.001250in}{2.154060in}}%
\pgfpathlineto{\pgfqpoint{2.006094in}{2.149316in}}%
\pgfpathlineto{\pgfqpoint{2.010937in}{2.156487in}}%
\pgfpathlineto{\pgfqpoint{2.015781in}{2.174742in}}%
\pgfpathlineto{\pgfqpoint{2.025469in}{2.226706in}}%
\pgfpathlineto{\pgfqpoint{2.030312in}{2.242720in}}%
\pgfpathlineto{\pgfqpoint{2.035156in}{2.247406in}}%
\pgfpathlineto{\pgfqpoint{2.040000in}{2.246001in}}%
\pgfpathlineto{\pgfqpoint{2.044844in}{2.243435in}}%
\pgfpathlineto{\pgfqpoint{2.049688in}{2.249405in}}%
\pgfpathlineto{\pgfqpoint{2.054531in}{2.263039in}}%
\pgfpathlineto{\pgfqpoint{2.069063in}{2.318731in}}%
\pgfpathlineto{\pgfqpoint{2.093281in}{2.381953in}}%
\pgfpathlineto{\pgfqpoint{2.098125in}{2.389173in}}%
\pgfpathlineto{\pgfqpoint{2.102969in}{2.387302in}}%
\pgfpathlineto{\pgfqpoint{2.107813in}{2.376640in}}%
\pgfpathlineto{\pgfqpoint{2.117500in}{2.345318in}}%
\pgfpathlineto{\pgfqpoint{2.122344in}{2.339938in}}%
\pgfpathlineto{\pgfqpoint{2.127188in}{2.347470in}}%
\pgfpathlineto{\pgfqpoint{2.132031in}{2.365409in}}%
\pgfpathlineto{\pgfqpoint{2.141719in}{2.407469in}}%
\pgfpathlineto{\pgfqpoint{2.146562in}{2.412361in}}%
\pgfpathlineto{\pgfqpoint{2.151406in}{2.401848in}}%
\pgfpathlineto{\pgfqpoint{2.165938in}{2.341197in}}%
\pgfpathlineto{\pgfqpoint{2.170781in}{2.340209in}}%
\pgfpathlineto{\pgfqpoint{2.175625in}{2.355433in}}%
\pgfpathlineto{\pgfqpoint{2.190156in}{2.442797in}}%
\pgfpathlineto{\pgfqpoint{2.195000in}{2.459110in}}%
\pgfpathlineto{\pgfqpoint{2.199844in}{2.463934in}}%
\pgfpathlineto{\pgfqpoint{2.204687in}{2.461447in}}%
\pgfpathlineto{\pgfqpoint{2.209531in}{2.461976in}}%
\pgfpathlineto{\pgfqpoint{2.214375in}{2.470306in}}%
\pgfpathlineto{\pgfqpoint{2.224063in}{2.500258in}}%
\pgfpathlineto{\pgfqpoint{2.228906in}{2.508270in}}%
\pgfpathlineto{\pgfqpoint{2.233750in}{2.502242in}}%
\pgfpathlineto{\pgfqpoint{2.248281in}{2.449356in}}%
\pgfpathlineto{\pgfqpoint{2.253125in}{2.450004in}}%
\pgfpathlineto{\pgfqpoint{2.257969in}{2.466496in}}%
\pgfpathlineto{\pgfqpoint{2.267656in}{2.511846in}}%
\pgfpathlineto{\pgfqpoint{2.272500in}{2.524648in}}%
\pgfpathlineto{\pgfqpoint{2.277344in}{2.526237in}}%
\pgfpathlineto{\pgfqpoint{2.287031in}{2.520872in}}%
\pgfpathlineto{\pgfqpoint{2.291875in}{2.524471in}}%
\pgfpathlineto{\pgfqpoint{2.301563in}{2.542380in}}%
\pgfpathlineto{\pgfqpoint{2.306406in}{2.546723in}}%
\pgfpathlineto{\pgfqpoint{2.311250in}{2.545254in}}%
\pgfpathlineto{\pgfqpoint{2.325781in}{2.523176in}}%
\pgfpathlineto{\pgfqpoint{2.330625in}{2.525422in}}%
\pgfpathlineto{\pgfqpoint{2.345156in}{2.547607in}}%
\pgfpathlineto{\pgfqpoint{2.350000in}{2.550543in}}%
\pgfpathlineto{\pgfqpoint{2.354844in}{2.555215in}}%
\pgfpathlineto{\pgfqpoint{2.359688in}{2.565269in}}%
\pgfpathlineto{\pgfqpoint{2.369375in}{2.589539in}}%
\pgfpathlineto{\pgfqpoint{2.374219in}{2.594109in}}%
\pgfpathlineto{\pgfqpoint{2.379062in}{2.588661in}}%
\pgfpathlineto{\pgfqpoint{2.383906in}{2.580404in}}%
\pgfpathlineto{\pgfqpoint{2.388750in}{2.574727in}}%
\pgfpathlineto{\pgfqpoint{2.393594in}{2.579055in}}%
\pgfpathlineto{\pgfqpoint{2.403281in}{2.604872in}}%
\pgfpathlineto{\pgfqpoint{2.408125in}{2.611606in}}%
\pgfpathlineto{\pgfqpoint{2.412969in}{2.606827in}}%
\pgfpathlineto{\pgfqpoint{2.427500in}{2.568979in}}%
\pgfpathlineto{\pgfqpoint{2.432344in}{2.565749in}}%
\pgfpathlineto{\pgfqpoint{2.442031in}{2.567317in}}%
\pgfpathlineto{\pgfqpoint{2.446875in}{2.563082in}}%
\pgfpathlineto{\pgfqpoint{2.456562in}{2.545234in}}%
\pgfpathlineto{\pgfqpoint{2.461406in}{2.541197in}}%
\pgfpathlineto{\pgfqpoint{2.466250in}{2.542477in}}%
\pgfpathlineto{\pgfqpoint{2.471094in}{2.550166in}}%
\pgfpathlineto{\pgfqpoint{2.480781in}{2.581186in}}%
\pgfpathlineto{\pgfqpoint{2.485625in}{2.599099in}}%
\pgfpathlineto{\pgfqpoint{2.490469in}{2.611488in}}%
\pgfpathlineto{\pgfqpoint{2.495313in}{2.613719in}}%
\pgfpathlineto{\pgfqpoint{2.500156in}{2.605156in}}%
\pgfpathlineto{\pgfqpoint{2.514687in}{2.556358in}}%
\pgfpathlineto{\pgfqpoint{2.519531in}{2.554184in}}%
\pgfpathlineto{\pgfqpoint{2.524375in}{2.565085in}}%
\pgfpathlineto{\pgfqpoint{2.534063in}{2.596097in}}%
\pgfpathlineto{\pgfqpoint{2.538906in}{2.603616in}}%
\pgfpathlineto{\pgfqpoint{2.543750in}{2.600101in}}%
\pgfpathlineto{\pgfqpoint{2.558281in}{2.560091in}}%
\pgfpathlineto{\pgfqpoint{2.563125in}{2.556017in}}%
\pgfpathlineto{\pgfqpoint{2.567969in}{2.562323in}}%
\pgfpathlineto{\pgfqpoint{2.592188in}{2.634293in}}%
\pgfpathlineto{\pgfqpoint{2.597031in}{2.642244in}}%
\pgfpathlineto{\pgfqpoint{2.601875in}{2.644335in}}%
\pgfpathlineto{\pgfqpoint{2.606719in}{2.640883in}}%
\pgfpathlineto{\pgfqpoint{2.611563in}{2.629728in}}%
\pgfpathlineto{\pgfqpoint{2.621250in}{2.594990in}}%
\pgfpathlineto{\pgfqpoint{2.626094in}{2.575692in}}%
\pgfpathlineto{\pgfqpoint{2.630937in}{2.561011in}}%
\pgfpathlineto{\pgfqpoint{2.635781in}{2.554345in}}%
\pgfpathlineto{\pgfqpoint{2.640625in}{2.560135in}}%
\pgfpathlineto{\pgfqpoint{2.645469in}{2.577570in}}%
\pgfpathlineto{\pgfqpoint{2.655156in}{2.618920in}}%
\pgfpathlineto{\pgfqpoint{2.660000in}{2.628173in}}%
\pgfpathlineto{\pgfqpoint{2.664844in}{2.624700in}}%
\pgfpathlineto{\pgfqpoint{2.674531in}{2.598883in}}%
\pgfpathlineto{\pgfqpoint{2.679375in}{2.596115in}}%
\pgfpathlineto{\pgfqpoint{2.684219in}{2.606841in}}%
\pgfpathlineto{\pgfqpoint{2.693906in}{2.646827in}}%
\pgfpathlineto{\pgfqpoint{2.698750in}{2.657399in}}%
\pgfpathlineto{\pgfqpoint{2.703594in}{2.654520in}}%
\pgfpathlineto{\pgfqpoint{2.708438in}{2.641507in}}%
\pgfpathlineto{\pgfqpoint{2.718125in}{2.612378in}}%
\pgfpathlineto{\pgfqpoint{2.722969in}{2.606350in}}%
\pgfpathlineto{\pgfqpoint{2.727813in}{2.606803in}}%
\pgfpathlineto{\pgfqpoint{2.752031in}{2.632985in}}%
\pgfpathlineto{\pgfqpoint{2.756875in}{2.640967in}}%
\pgfpathlineto{\pgfqpoint{2.771406in}{2.678472in}}%
\pgfpathlineto{\pgfqpoint{2.776250in}{2.684215in}}%
\pgfpathlineto{\pgfqpoint{2.785938in}{2.683961in}}%
\pgfpathlineto{\pgfqpoint{2.790781in}{2.689214in}}%
\pgfpathlineto{\pgfqpoint{2.795625in}{2.705381in}}%
\pgfpathlineto{\pgfqpoint{2.805312in}{2.755967in}}%
\pgfpathlineto{\pgfqpoint{2.810156in}{2.768624in}}%
\pgfpathlineto{\pgfqpoint{2.815000in}{2.766843in}}%
\pgfpathlineto{\pgfqpoint{2.824688in}{2.741047in}}%
\pgfpathlineto{\pgfqpoint{2.829531in}{2.737105in}}%
\pgfpathlineto{\pgfqpoint{2.834375in}{2.745796in}}%
\pgfpathlineto{\pgfqpoint{2.844063in}{2.779035in}}%
\pgfpathlineto{\pgfqpoint{2.848906in}{2.787768in}}%
\pgfpathlineto{\pgfqpoint{2.853750in}{2.789065in}}%
\pgfpathlineto{\pgfqpoint{2.863437in}{2.783081in}}%
\pgfpathlineto{\pgfqpoint{2.868281in}{2.787044in}}%
\pgfpathlineto{\pgfqpoint{2.873125in}{2.798338in}}%
\pgfpathlineto{\pgfqpoint{2.892500in}{2.863664in}}%
\pgfpathlineto{\pgfqpoint{2.897344in}{2.869277in}}%
\pgfpathlineto{\pgfqpoint{2.902188in}{2.866415in}}%
\pgfpathlineto{\pgfqpoint{2.907031in}{2.860822in}}%
\pgfpathlineto{\pgfqpoint{2.916719in}{2.845920in}}%
\pgfpathlineto{\pgfqpoint{2.926406in}{2.837217in}}%
\pgfpathlineto{\pgfqpoint{2.936094in}{2.834655in}}%
\pgfpathlineto{\pgfqpoint{2.945781in}{2.835875in}}%
\pgfpathlineto{\pgfqpoint{2.950625in}{2.835176in}}%
\pgfpathlineto{\pgfqpoint{2.965156in}{2.826785in}}%
\pgfpathlineto{\pgfqpoint{2.974844in}{2.824143in}}%
\pgfpathlineto{\pgfqpoint{2.979688in}{2.817416in}}%
\pgfpathlineto{\pgfqpoint{2.984531in}{2.805638in}}%
\pgfpathlineto{\pgfqpoint{2.994219in}{2.774339in}}%
\pgfpathlineto{\pgfqpoint{2.999062in}{2.767031in}}%
\pgfpathlineto{\pgfqpoint{3.003906in}{2.768772in}}%
\pgfpathlineto{\pgfqpoint{3.008750in}{2.773441in}}%
\pgfpathlineto{\pgfqpoint{3.013594in}{2.772860in}}%
\pgfpathlineto{\pgfqpoint{3.018438in}{2.758240in}}%
\pgfpathlineto{\pgfqpoint{3.032969in}{2.678312in}}%
\pgfpathlineto{\pgfqpoint{3.037813in}{2.670515in}}%
\pgfpathlineto{\pgfqpoint{3.042656in}{2.673135in}}%
\pgfpathlineto{\pgfqpoint{3.047500in}{2.678271in}}%
\pgfpathlineto{\pgfqpoint{3.052344in}{2.678629in}}%
\pgfpathlineto{\pgfqpoint{3.057187in}{2.671197in}}%
\pgfpathlineto{\pgfqpoint{3.062031in}{2.659412in}}%
\pgfpathlineto{\pgfqpoint{3.066875in}{2.652675in}}%
\pgfpathlineto{\pgfqpoint{3.071719in}{2.650825in}}%
\pgfpathlineto{\pgfqpoint{3.081406in}{2.654163in}}%
\pgfpathlineto{\pgfqpoint{3.095938in}{2.648617in}}%
\pgfpathlineto{\pgfqpoint{3.100781in}{2.651026in}}%
\pgfpathlineto{\pgfqpoint{3.105625in}{2.654906in}}%
\pgfpathlineto{\pgfqpoint{3.110469in}{2.652574in}}%
\pgfpathlineto{\pgfqpoint{3.115312in}{2.644528in}}%
\pgfpathlineto{\pgfqpoint{3.134688in}{2.589382in}}%
\pgfpathlineto{\pgfqpoint{3.144375in}{2.571994in}}%
\pgfpathlineto{\pgfqpoint{3.149219in}{2.564878in}}%
\pgfpathlineto{\pgfqpoint{3.154063in}{2.562351in}}%
\pgfpathlineto{\pgfqpoint{3.158906in}{2.568938in}}%
\pgfpathlineto{\pgfqpoint{3.173437in}{2.608628in}}%
\pgfpathlineto{\pgfqpoint{3.178281in}{2.613324in}}%
\pgfpathlineto{\pgfqpoint{3.183125in}{2.612156in}}%
\pgfpathlineto{\pgfqpoint{3.192813in}{2.606356in}}%
\pgfpathlineto{\pgfqpoint{3.197656in}{2.605820in}}%
\pgfpathlineto{\pgfqpoint{3.202500in}{2.606426in}}%
\pgfpathlineto{\pgfqpoint{3.217031in}{2.604431in}}%
\pgfpathlineto{\pgfqpoint{3.221875in}{2.605296in}}%
\pgfpathlineto{\pgfqpoint{3.231562in}{2.609830in}}%
\pgfpathlineto{\pgfqpoint{3.236406in}{2.608444in}}%
\pgfpathlineto{\pgfqpoint{3.241250in}{2.603634in}}%
\pgfpathlineto{\pgfqpoint{3.250938in}{2.588381in}}%
\pgfpathlineto{\pgfqpoint{3.255781in}{2.584317in}}%
\pgfpathlineto{\pgfqpoint{3.260625in}{2.583167in}}%
\pgfpathlineto{\pgfqpoint{3.265469in}{2.587194in}}%
\pgfpathlineto{\pgfqpoint{3.270313in}{2.594776in}}%
\pgfpathlineto{\pgfqpoint{3.280000in}{2.616134in}}%
\pgfpathlineto{\pgfqpoint{3.289687in}{2.638184in}}%
\pgfpathlineto{\pgfqpoint{3.294531in}{2.642737in}}%
\pgfpathlineto{\pgfqpoint{3.299375in}{2.641460in}}%
\pgfpathlineto{\pgfqpoint{3.304219in}{2.633511in}}%
\pgfpathlineto{\pgfqpoint{3.313906in}{2.598837in}}%
\pgfpathlineto{\pgfqpoint{3.328438in}{2.534585in}}%
\pgfpathlineto{\pgfqpoint{3.333281in}{2.518605in}}%
\pgfpathlineto{\pgfqpoint{3.338125in}{2.507624in}}%
\pgfpathlineto{\pgfqpoint{3.342969in}{2.500978in}}%
\pgfpathlineto{\pgfqpoint{3.352656in}{2.491996in}}%
\pgfpathlineto{\pgfqpoint{3.357500in}{2.480075in}}%
\pgfpathlineto{\pgfqpoint{3.362344in}{2.459762in}}%
\pgfpathlineto{\pgfqpoint{3.367188in}{2.429706in}}%
\pgfpathlineto{\pgfqpoint{3.386563in}{2.286908in}}%
\pgfpathlineto{\pgfqpoint{3.391406in}{2.282633in}}%
\pgfpathlineto{\pgfqpoint{3.396250in}{2.303755in}}%
\pgfpathlineto{\pgfqpoint{3.401094in}{2.350736in}}%
\pgfpathlineto{\pgfqpoint{3.410781in}{2.495904in}}%
\pgfpathlineto{\pgfqpoint{3.425312in}{2.756197in}}%
\pgfpathlineto{\pgfqpoint{3.439844in}{3.019797in}}%
\pgfpathlineto{\pgfqpoint{3.444688in}{3.080779in}}%
\pgfpathlineto{\pgfqpoint{3.449531in}{3.123471in}}%
\pgfpathlineto{\pgfqpoint{3.464063in}{3.224985in}}%
\pgfpathlineto{\pgfqpoint{3.473750in}{3.295036in}}%
\pgfpathlineto{\pgfqpoint{3.478594in}{3.312593in}}%
\pgfpathlineto{\pgfqpoint{3.483437in}{3.314594in}}%
\pgfpathlineto{\pgfqpoint{3.488281in}{3.313310in}}%
\pgfpathlineto{\pgfqpoint{3.493125in}{3.319506in}}%
\pgfpathlineto{\pgfqpoint{3.502813in}{3.346990in}}%
\pgfpathlineto{\pgfqpoint{3.507656in}{3.344855in}}%
\pgfpathlineto{\pgfqpoint{3.512500in}{3.310225in}}%
\pgfpathlineto{\pgfqpoint{3.517344in}{3.239461in}}%
\pgfpathlineto{\pgfqpoint{3.527031in}{3.024094in}}%
\pgfpathlineto{\pgfqpoint{3.546406in}{2.554196in}}%
\pgfpathlineto{\pgfqpoint{3.551250in}{2.468141in}}%
\pgfpathlineto{\pgfqpoint{3.556094in}{2.413292in}}%
\pgfpathlineto{\pgfqpoint{3.560938in}{2.396213in}}%
\pgfpathlineto{\pgfqpoint{3.565781in}{2.409486in}}%
\pgfpathlineto{\pgfqpoint{3.575469in}{2.456029in}}%
\pgfpathlineto{\pgfqpoint{3.580313in}{2.459069in}}%
\pgfpathlineto{\pgfqpoint{3.585156in}{2.442083in}}%
\pgfpathlineto{\pgfqpoint{3.594844in}{2.374897in}}%
\pgfpathlineto{\pgfqpoint{3.604531in}{2.311283in}}%
\pgfpathlineto{\pgfqpoint{3.609375in}{2.291112in}}%
\pgfpathlineto{\pgfqpoint{3.614219in}{2.280528in}}%
\pgfpathlineto{\pgfqpoint{3.619063in}{2.282692in}}%
\pgfpathlineto{\pgfqpoint{3.623906in}{2.293630in}}%
\pgfpathlineto{\pgfqpoint{3.628750in}{2.309564in}}%
\pgfpathlineto{\pgfqpoint{3.633594in}{2.321169in}}%
\pgfpathlineto{\pgfqpoint{3.638438in}{2.325671in}}%
\pgfpathlineto{\pgfqpoint{3.643281in}{2.320391in}}%
\pgfpathlineto{\pgfqpoint{3.648125in}{2.310308in}}%
\pgfpathlineto{\pgfqpoint{3.652969in}{2.303640in}}%
\pgfpathlineto{\pgfqpoint{3.657812in}{2.303766in}}%
\pgfpathlineto{\pgfqpoint{3.662656in}{2.311627in}}%
\pgfpathlineto{\pgfqpoint{3.672344in}{2.341538in}}%
\pgfpathlineto{\pgfqpoint{3.682031in}{2.385918in}}%
\pgfpathlineto{\pgfqpoint{3.691719in}{2.456700in}}%
\pgfpathlineto{\pgfqpoint{3.696563in}{2.493521in}}%
\pgfpathlineto{\pgfqpoint{3.701406in}{2.521699in}}%
\pgfpathlineto{\pgfqpoint{3.706250in}{2.535244in}}%
\pgfpathlineto{\pgfqpoint{3.711094in}{2.530378in}}%
\pgfpathlineto{\pgfqpoint{3.715937in}{2.512081in}}%
\pgfpathlineto{\pgfqpoint{3.735313in}{2.401308in}}%
\pgfpathlineto{\pgfqpoint{3.745000in}{2.330226in}}%
\pgfpathlineto{\pgfqpoint{3.759531in}{2.184835in}}%
\pgfpathlineto{\pgfqpoint{3.764375in}{2.151522in}}%
\pgfpathlineto{\pgfqpoint{3.769219in}{2.137692in}}%
\pgfpathlineto{\pgfqpoint{3.774062in}{2.141047in}}%
\pgfpathlineto{\pgfqpoint{3.793438in}{2.202935in}}%
\pgfpathlineto{\pgfqpoint{3.798281in}{2.207502in}}%
\pgfpathlineto{\pgfqpoint{3.803125in}{2.213910in}}%
\pgfpathlineto{\pgfqpoint{3.807969in}{2.223327in}}%
\pgfpathlineto{\pgfqpoint{3.812813in}{2.237642in}}%
\pgfpathlineto{\pgfqpoint{3.827344in}{2.292908in}}%
\pgfpathlineto{\pgfqpoint{3.832187in}{2.303514in}}%
\pgfpathlineto{\pgfqpoint{3.837031in}{2.310135in}}%
\pgfpathlineto{\pgfqpoint{3.841875in}{2.314326in}}%
\pgfpathlineto{\pgfqpoint{3.846719in}{2.322872in}}%
\pgfpathlineto{\pgfqpoint{3.856406in}{2.355266in}}%
\pgfpathlineto{\pgfqpoint{3.861250in}{2.373237in}}%
\pgfpathlineto{\pgfqpoint{3.866094in}{2.382314in}}%
\pgfpathlineto{\pgfqpoint{3.870938in}{2.379442in}}%
\pgfpathlineto{\pgfqpoint{3.875781in}{2.363737in}}%
\pgfpathlineto{\pgfqpoint{3.885469in}{2.308791in}}%
\pgfpathlineto{\pgfqpoint{3.895156in}{2.237315in}}%
\pgfpathlineto{\pgfqpoint{3.904844in}{2.160526in}}%
\pgfpathlineto{\pgfqpoint{3.909687in}{2.127815in}}%
\pgfpathlineto{\pgfqpoint{3.914531in}{2.106442in}}%
\pgfpathlineto{\pgfqpoint{3.919375in}{2.094862in}}%
\pgfpathlineto{\pgfqpoint{3.924219in}{2.092078in}}%
\pgfpathlineto{\pgfqpoint{3.929063in}{2.093801in}}%
\pgfpathlineto{\pgfqpoint{3.933906in}{2.100268in}}%
\pgfpathlineto{\pgfqpoint{3.938750in}{2.109830in}}%
\pgfpathlineto{\pgfqpoint{3.943594in}{2.125582in}}%
\pgfpathlineto{\pgfqpoint{3.958125in}{2.184182in}}%
\pgfpathlineto{\pgfqpoint{3.962969in}{2.189126in}}%
\pgfpathlineto{\pgfqpoint{3.967812in}{2.181091in}}%
\pgfpathlineto{\pgfqpoint{3.977500in}{2.149959in}}%
\pgfpathlineto{\pgfqpoint{3.982344in}{2.141685in}}%
\pgfpathlineto{\pgfqpoint{3.987188in}{2.142104in}}%
\pgfpathlineto{\pgfqpoint{3.996875in}{2.160005in}}%
\pgfpathlineto{\pgfqpoint{4.011406in}{2.189102in}}%
\pgfpathlineto{\pgfqpoint{4.016250in}{2.206260in}}%
\pgfpathlineto{\pgfqpoint{4.030781in}{2.270715in}}%
\pgfpathlineto{\pgfqpoint{4.035625in}{2.280482in}}%
\pgfpathlineto{\pgfqpoint{4.040469in}{2.276775in}}%
\pgfpathlineto{\pgfqpoint{4.045312in}{2.266955in}}%
\pgfpathlineto{\pgfqpoint{4.055000in}{2.242172in}}%
\pgfpathlineto{\pgfqpoint{4.064687in}{2.218175in}}%
\pgfpathlineto{\pgfqpoint{4.069531in}{2.203096in}}%
\pgfpathlineto{\pgfqpoint{4.079219in}{2.165499in}}%
\pgfpathlineto{\pgfqpoint{4.084062in}{2.154419in}}%
\pgfpathlineto{\pgfqpoint{4.088906in}{2.154301in}}%
\pgfpathlineto{\pgfqpoint{4.093750in}{2.163949in}}%
\pgfpathlineto{\pgfqpoint{4.108281in}{2.211538in}}%
\pgfpathlineto{\pgfqpoint{4.113125in}{2.219040in}}%
\pgfpathlineto{\pgfqpoint{4.117969in}{2.220378in}}%
\pgfpathlineto{\pgfqpoint{4.127656in}{2.220626in}}%
\pgfpathlineto{\pgfqpoint{4.132500in}{2.224416in}}%
\pgfpathlineto{\pgfqpoint{4.137344in}{2.234358in}}%
\pgfpathlineto{\pgfqpoint{4.147031in}{2.266277in}}%
\pgfpathlineto{\pgfqpoint{4.156719in}{2.301428in}}%
\pgfpathlineto{\pgfqpoint{4.161563in}{2.313874in}}%
\pgfpathlineto{\pgfqpoint{4.166406in}{2.321348in}}%
\pgfpathlineto{\pgfqpoint{4.171250in}{2.321588in}}%
\pgfpathlineto{\pgfqpoint{4.176094in}{2.313517in}}%
\pgfpathlineto{\pgfqpoint{4.190625in}{2.265094in}}%
\pgfpathlineto{\pgfqpoint{4.195469in}{2.260555in}}%
\pgfpathlineto{\pgfqpoint{4.200312in}{2.266475in}}%
\pgfpathlineto{\pgfqpoint{4.205156in}{2.278282in}}%
\pgfpathlineto{\pgfqpoint{4.210000in}{2.283284in}}%
\pgfpathlineto{\pgfqpoint{4.214844in}{2.276270in}}%
\pgfpathlineto{\pgfqpoint{4.224531in}{2.243034in}}%
\pgfpathlineto{\pgfqpoint{4.229375in}{2.237219in}}%
\pgfpathlineto{\pgfqpoint{4.234219in}{2.247995in}}%
\pgfpathlineto{\pgfqpoint{4.248750in}{2.312699in}}%
\pgfpathlineto{\pgfqpoint{4.253594in}{2.318718in}}%
\pgfpathlineto{\pgfqpoint{4.263281in}{2.318351in}}%
\pgfpathlineto{\pgfqpoint{4.268125in}{2.321930in}}%
\pgfpathlineto{\pgfqpoint{4.277813in}{2.339030in}}%
\pgfpathlineto{\pgfqpoint{4.282656in}{2.344014in}}%
\pgfpathlineto{\pgfqpoint{4.287500in}{2.345121in}}%
\pgfpathlineto{\pgfqpoint{4.292344in}{2.344861in}}%
\pgfpathlineto{\pgfqpoint{4.302031in}{2.341241in}}%
\pgfpathlineto{\pgfqpoint{4.306875in}{2.337758in}}%
\pgfpathlineto{\pgfqpoint{4.316562in}{2.328901in}}%
\pgfpathlineto{\pgfqpoint{4.321406in}{2.327414in}}%
\pgfpathlineto{\pgfqpoint{4.331094in}{2.326781in}}%
\pgfpathlineto{\pgfqpoint{4.335938in}{2.325112in}}%
\pgfpathlineto{\pgfqpoint{4.345625in}{2.314075in}}%
\pgfpathlineto{\pgfqpoint{4.350469in}{2.312494in}}%
\pgfpathlineto{\pgfqpoint{4.355313in}{2.313644in}}%
\pgfpathlineto{\pgfqpoint{4.360156in}{2.316784in}}%
\pgfpathlineto{\pgfqpoint{4.365000in}{2.315154in}}%
\pgfpathlineto{\pgfqpoint{4.379531in}{2.297834in}}%
\pgfpathlineto{\pgfqpoint{4.384375in}{2.301447in}}%
\pgfpathlineto{\pgfqpoint{4.389219in}{2.310088in}}%
\pgfpathlineto{\pgfqpoint{4.394063in}{2.314298in}}%
\pgfpathlineto{\pgfqpoint{4.398906in}{2.310181in}}%
\pgfpathlineto{\pgfqpoint{4.403750in}{2.297664in}}%
\pgfpathlineto{\pgfqpoint{4.408594in}{2.280125in}}%
\pgfpathlineto{\pgfqpoint{4.413437in}{2.269133in}}%
\pgfpathlineto{\pgfqpoint{4.418281in}{2.270095in}}%
\pgfpathlineto{\pgfqpoint{4.423125in}{2.283986in}}%
\pgfpathlineto{\pgfqpoint{4.432812in}{2.326229in}}%
\pgfpathlineto{\pgfqpoint{4.437656in}{2.338552in}}%
\pgfpathlineto{\pgfqpoint{4.442500in}{2.337184in}}%
\pgfpathlineto{\pgfqpoint{4.447344in}{2.323941in}}%
\pgfpathlineto{\pgfqpoint{4.461875in}{2.264232in}}%
\pgfpathlineto{\pgfqpoint{4.466719in}{2.255081in}}%
\pgfpathlineto{\pgfqpoint{4.471563in}{2.255580in}}%
\pgfpathlineto{\pgfqpoint{4.476406in}{2.262976in}}%
\pgfpathlineto{\pgfqpoint{4.481250in}{2.273438in}}%
\pgfpathlineto{\pgfqpoint{4.486094in}{2.280840in}}%
\pgfpathlineto{\pgfqpoint{4.490938in}{2.283151in}}%
\pgfpathlineto{\pgfqpoint{4.495781in}{2.275839in}}%
\pgfpathlineto{\pgfqpoint{4.515156in}{2.227884in}}%
\pgfpathlineto{\pgfqpoint{4.520000in}{2.230028in}}%
\pgfpathlineto{\pgfqpoint{4.524844in}{2.238215in}}%
\pgfpathlineto{\pgfqpoint{4.529688in}{2.249324in}}%
\pgfpathlineto{\pgfqpoint{4.534531in}{2.256928in}}%
\pgfpathlineto{\pgfqpoint{4.539375in}{2.256544in}}%
\pgfpathlineto{\pgfqpoint{4.544219in}{2.249236in}}%
\pgfpathlineto{\pgfqpoint{4.558750in}{2.209165in}}%
\pgfpathlineto{\pgfqpoint{4.563594in}{2.199928in}}%
\pgfpathlineto{\pgfqpoint{4.573281in}{2.186265in}}%
\pgfpathlineto{\pgfqpoint{4.587813in}{2.154197in}}%
\pgfpathlineto{\pgfqpoint{4.592656in}{2.151892in}}%
\pgfpathlineto{\pgfqpoint{4.597500in}{2.159750in}}%
\pgfpathlineto{\pgfqpoint{4.612031in}{2.204305in}}%
\pgfpathlineto{\pgfqpoint{4.616875in}{2.205631in}}%
\pgfpathlineto{\pgfqpoint{4.621719in}{2.192930in}}%
\pgfpathlineto{\pgfqpoint{4.636250in}{2.118324in}}%
\pgfpathlineto{\pgfqpoint{4.641094in}{2.104695in}}%
\pgfpathlineto{\pgfqpoint{4.650781in}{2.094135in}}%
\pgfpathlineto{\pgfqpoint{4.660469in}{2.082165in}}%
\pgfpathlineto{\pgfqpoint{4.665312in}{2.078870in}}%
\pgfpathlineto{\pgfqpoint{4.670156in}{2.081755in}}%
\pgfpathlineto{\pgfqpoint{4.675000in}{2.089027in}}%
\pgfpathlineto{\pgfqpoint{4.679844in}{2.091685in}}%
\pgfpathlineto{\pgfqpoint{4.684687in}{2.084041in}}%
\pgfpathlineto{\pgfqpoint{4.689531in}{2.063453in}}%
\pgfpathlineto{\pgfqpoint{4.699219in}{2.012442in}}%
\pgfpathlineto{\pgfqpoint{4.704063in}{2.004432in}}%
\pgfpathlineto{\pgfqpoint{4.708906in}{2.010362in}}%
\pgfpathlineto{\pgfqpoint{4.713750in}{2.021941in}}%
\pgfpathlineto{\pgfqpoint{4.718594in}{2.026280in}}%
\pgfpathlineto{\pgfqpoint{4.723438in}{2.017504in}}%
\pgfpathlineto{\pgfqpoint{4.733125in}{1.978122in}}%
\pgfpathlineto{\pgfqpoint{4.737969in}{1.970103in}}%
\pgfpathlineto{\pgfqpoint{4.742813in}{1.976036in}}%
\pgfpathlineto{\pgfqpoint{4.752500in}{2.004696in}}%
\pgfpathlineto{\pgfqpoint{4.757344in}{2.010106in}}%
\pgfpathlineto{\pgfqpoint{4.762188in}{2.002607in}}%
\pgfpathlineto{\pgfqpoint{4.776719in}{1.961695in}}%
\pgfpathlineto{\pgfqpoint{4.781562in}{1.958580in}}%
\pgfpathlineto{\pgfqpoint{4.786406in}{1.960626in}}%
\pgfpathlineto{\pgfqpoint{4.791250in}{1.967606in}}%
\pgfpathlineto{\pgfqpoint{4.805781in}{1.994892in}}%
\pgfpathlineto{\pgfqpoint{4.810625in}{1.998944in}}%
\pgfpathlineto{\pgfqpoint{4.815469in}{1.999115in}}%
\pgfpathlineto{\pgfqpoint{4.830000in}{1.996706in}}%
\pgfpathlineto{\pgfqpoint{4.834844in}{1.998669in}}%
\pgfpathlineto{\pgfqpoint{4.839688in}{1.998066in}}%
\pgfpathlineto{\pgfqpoint{4.844531in}{1.988802in}}%
\pgfpathlineto{\pgfqpoint{4.849375in}{1.968949in}}%
\pgfpathlineto{\pgfqpoint{4.859063in}{1.914983in}}%
\pgfpathlineto{\pgfqpoint{4.863906in}{1.898684in}}%
\pgfpathlineto{\pgfqpoint{4.868750in}{1.899474in}}%
\pgfpathlineto{\pgfqpoint{4.873594in}{1.914821in}}%
\pgfpathlineto{\pgfqpoint{4.883281in}{1.960528in}}%
\pgfpathlineto{\pgfqpoint{4.888125in}{1.970664in}}%
\pgfpathlineto{\pgfqpoint{4.892969in}{1.971435in}}%
\pgfpathlineto{\pgfqpoint{4.897812in}{1.966313in}}%
\pgfpathlineto{\pgfqpoint{4.902656in}{1.964757in}}%
\pgfpathlineto{\pgfqpoint{4.907500in}{1.967942in}}%
\pgfpathlineto{\pgfqpoint{4.912344in}{1.975547in}}%
\pgfpathlineto{\pgfqpoint{4.917187in}{1.980839in}}%
\pgfpathlineto{\pgfqpoint{4.922031in}{1.979908in}}%
\pgfpathlineto{\pgfqpoint{4.931719in}{1.965718in}}%
\pgfpathlineto{\pgfqpoint{4.936562in}{1.968373in}}%
\pgfpathlineto{\pgfqpoint{4.941406in}{1.983583in}}%
\pgfpathlineto{\pgfqpoint{4.951094in}{2.027086in}}%
\pgfpathlineto{\pgfqpoint{4.955938in}{2.034148in}}%
\pgfpathlineto{\pgfqpoint{4.960781in}{2.022952in}}%
\pgfpathlineto{\pgfqpoint{4.970469in}{1.977629in}}%
\pgfpathlineto{\pgfqpoint{4.975313in}{1.967828in}}%
\pgfpathlineto{\pgfqpoint{4.980156in}{1.976572in}}%
\pgfpathlineto{\pgfqpoint{4.989844in}{2.013833in}}%
\pgfpathlineto{\pgfqpoint{4.994688in}{2.018794in}}%
\pgfpathlineto{\pgfqpoint{4.999531in}{2.009252in}}%
\pgfpathlineto{\pgfqpoint{5.004375in}{1.993308in}}%
\pgfpathlineto{\pgfqpoint{5.009219in}{1.983509in}}%
\pgfpathlineto{\pgfqpoint{5.014063in}{1.990316in}}%
\pgfpathlineto{\pgfqpoint{5.018906in}{2.014786in}}%
\pgfpathlineto{\pgfqpoint{5.028594in}{2.081880in}}%
\pgfpathlineto{\pgfqpoint{5.033437in}{2.104225in}}%
\pgfpathlineto{\pgfqpoint{5.038281in}{2.112489in}}%
\pgfpathlineto{\pgfqpoint{5.043125in}{2.109485in}}%
\pgfpathlineto{\pgfqpoint{5.052812in}{2.094566in}}%
\pgfpathlineto{\pgfqpoint{5.057656in}{2.091092in}}%
\pgfpathlineto{\pgfqpoint{5.062500in}{2.092106in}}%
\pgfpathlineto{\pgfqpoint{5.067344in}{2.100761in}}%
\pgfpathlineto{\pgfqpoint{5.077031in}{2.132094in}}%
\pgfpathlineto{\pgfqpoint{5.081875in}{2.148105in}}%
\pgfpathlineto{\pgfqpoint{5.086719in}{2.154979in}}%
\pgfpathlineto{\pgfqpoint{5.091563in}{2.153062in}}%
\pgfpathlineto{\pgfqpoint{5.096406in}{2.147887in}}%
\pgfpathlineto{\pgfqpoint{5.101250in}{2.147311in}}%
\pgfpathlineto{\pgfqpoint{5.106094in}{2.160102in}}%
\pgfpathlineto{\pgfqpoint{5.110938in}{2.187075in}}%
\pgfpathlineto{\pgfqpoint{5.120625in}{2.253716in}}%
\pgfpathlineto{\pgfqpoint{5.125469in}{2.269383in}}%
\pgfpathlineto{\pgfqpoint{5.130313in}{2.268943in}}%
\pgfpathlineto{\pgfqpoint{5.140000in}{2.246218in}}%
\pgfpathlineto{\pgfqpoint{5.144844in}{2.241995in}}%
\pgfpathlineto{\pgfqpoint{5.149687in}{2.247590in}}%
\pgfpathlineto{\pgfqpoint{5.154531in}{2.256443in}}%
\pgfpathlineto{\pgfqpoint{5.159375in}{2.262313in}}%
\pgfpathlineto{\pgfqpoint{5.164219in}{2.264441in}}%
\pgfpathlineto{\pgfqpoint{5.169062in}{2.263552in}}%
\pgfpathlineto{\pgfqpoint{5.173906in}{2.265616in}}%
\pgfpathlineto{\pgfqpoint{5.178750in}{2.272777in}}%
\pgfpathlineto{\pgfqpoint{5.183594in}{2.284876in}}%
\pgfpathlineto{\pgfqpoint{5.188438in}{2.292034in}}%
\pgfpathlineto{\pgfqpoint{5.193281in}{2.292797in}}%
\pgfpathlineto{\pgfqpoint{5.198125in}{2.282154in}}%
\pgfpathlineto{\pgfqpoint{5.207813in}{2.256076in}}%
\pgfpathlineto{\pgfqpoint{5.212656in}{2.250501in}}%
\pgfpathlineto{\pgfqpoint{5.217500in}{2.252077in}}%
\pgfpathlineto{\pgfqpoint{5.222344in}{2.256254in}}%
\pgfpathlineto{\pgfqpoint{5.227188in}{2.255994in}}%
\pgfpathlineto{\pgfqpoint{5.232031in}{2.248312in}}%
\pgfpathlineto{\pgfqpoint{5.236875in}{2.232842in}}%
\pgfpathlineto{\pgfqpoint{5.246563in}{2.194682in}}%
\pgfpathlineto{\pgfqpoint{5.251406in}{2.179706in}}%
\pgfpathlineto{\pgfqpoint{5.270781in}{2.135134in}}%
\pgfpathlineto{\pgfqpoint{5.280469in}{2.108282in}}%
\pgfpathlineto{\pgfqpoint{5.285312in}{2.098476in}}%
\pgfpathlineto{\pgfqpoint{5.295000in}{2.086036in}}%
\pgfpathlineto{\pgfqpoint{5.299844in}{2.081140in}}%
\pgfpathlineto{\pgfqpoint{5.304688in}{2.079387in}}%
\pgfpathlineto{\pgfqpoint{5.309531in}{2.081353in}}%
\pgfpathlineto{\pgfqpoint{5.314375in}{2.089910in}}%
\pgfpathlineto{\pgfqpoint{5.319219in}{2.101397in}}%
\pgfpathlineto{\pgfqpoint{5.324063in}{2.109928in}}%
\pgfpathlineto{\pgfqpoint{5.328906in}{2.108895in}}%
\pgfpathlineto{\pgfqpoint{5.333750in}{2.099874in}}%
\pgfpathlineto{\pgfqpoint{5.343438in}{2.067406in}}%
\pgfpathlineto{\pgfqpoint{5.348281in}{2.056202in}}%
\pgfpathlineto{\pgfqpoint{5.353125in}{2.048609in}}%
\pgfpathlineto{\pgfqpoint{5.362813in}{2.044175in}}%
\pgfpathlineto{\pgfqpoint{5.367656in}{2.043050in}}%
\pgfpathlineto{\pgfqpoint{5.372500in}{2.046971in}}%
\pgfpathlineto{\pgfqpoint{5.377344in}{2.055904in}}%
\pgfpathlineto{\pgfqpoint{5.382187in}{2.068391in}}%
\pgfpathlineto{\pgfqpoint{5.387031in}{2.076600in}}%
\pgfpathlineto{\pgfqpoint{5.391875in}{2.075541in}}%
\pgfpathlineto{\pgfqpoint{5.396719in}{2.066668in}}%
\pgfpathlineto{\pgfqpoint{5.401562in}{2.053715in}}%
\pgfpathlineto{\pgfqpoint{5.406406in}{2.046373in}}%
\pgfpathlineto{\pgfqpoint{5.411250in}{2.048676in}}%
\pgfpathlineto{\pgfqpoint{5.416094in}{2.056949in}}%
\pgfpathlineto{\pgfqpoint{5.420937in}{2.062914in}}%
\pgfpathlineto{\pgfqpoint{5.425781in}{2.059071in}}%
\pgfpathlineto{\pgfqpoint{5.440313in}{2.019631in}}%
\pgfpathlineto{\pgfqpoint{5.445156in}{2.026684in}}%
\pgfpathlineto{\pgfqpoint{5.454844in}{2.059876in}}%
\pgfpathlineto{\pgfqpoint{5.459688in}{2.061353in}}%
\pgfpathlineto{\pgfqpoint{5.464531in}{2.042838in}}%
\pgfpathlineto{\pgfqpoint{5.474219in}{1.988091in}}%
\pgfpathlineto{\pgfqpoint{5.479063in}{1.984083in}}%
\pgfpathlineto{\pgfqpoint{5.483906in}{2.002952in}}%
\pgfpathlineto{\pgfqpoint{5.493594in}{2.067205in}}%
\pgfpathlineto{\pgfqpoint{5.498438in}{2.079649in}}%
\pgfpathlineto{\pgfqpoint{5.503281in}{2.068430in}}%
\pgfpathlineto{\pgfqpoint{5.512969in}{2.023243in}}%
\pgfpathlineto{\pgfqpoint{5.517812in}{2.013595in}}%
\pgfpathlineto{\pgfqpoint{5.522656in}{2.018709in}}%
\pgfpathlineto{\pgfqpoint{5.532344in}{2.046290in}}%
\pgfpathlineto{\pgfqpoint{5.537187in}{2.050143in}}%
\pgfpathlineto{\pgfqpoint{5.542031in}{2.042787in}}%
\pgfpathlineto{\pgfqpoint{5.551719in}{2.014357in}}%
\pgfpathlineto{\pgfqpoint{5.556563in}{2.005476in}}%
\pgfpathlineto{\pgfqpoint{5.561406in}{2.004890in}}%
\pgfpathlineto{\pgfqpoint{5.566250in}{2.010106in}}%
\pgfpathlineto{\pgfqpoint{5.575938in}{2.025476in}}%
\pgfpathlineto{\pgfqpoint{5.580781in}{2.034761in}}%
\pgfpathlineto{\pgfqpoint{5.585625in}{2.041644in}}%
\pgfpathlineto{\pgfqpoint{5.590469in}{2.044078in}}%
\pgfpathlineto{\pgfqpoint{5.595313in}{2.040354in}}%
\pgfpathlineto{\pgfqpoint{5.600156in}{2.026452in}}%
\pgfpathlineto{\pgfqpoint{5.609844in}{1.974169in}}%
\pgfpathlineto{\pgfqpoint{5.614688in}{1.947372in}}%
\pgfpathlineto{\pgfqpoint{5.619531in}{1.930566in}}%
\pgfpathlineto{\pgfqpoint{5.624375in}{1.927777in}}%
\pgfpathlineto{\pgfqpoint{5.629219in}{1.942618in}}%
\pgfpathlineto{\pgfqpoint{5.634062in}{1.972181in}}%
\pgfpathlineto{\pgfqpoint{5.638906in}{2.015148in}}%
\pgfpathlineto{\pgfqpoint{5.643750in}{2.069328in}}%
\pgfpathlineto{\pgfqpoint{5.653437in}{2.216334in}}%
\pgfpathlineto{\pgfqpoint{5.663125in}{2.416999in}}%
\pgfpathlineto{\pgfqpoint{5.672813in}{2.635804in}}%
\pgfpathlineto{\pgfqpoint{5.677656in}{2.723192in}}%
\pgfpathlineto{\pgfqpoint{5.682500in}{2.787207in}}%
\pgfpathlineto{\pgfqpoint{5.687344in}{2.825210in}}%
\pgfpathlineto{\pgfqpoint{5.692188in}{2.846518in}}%
\pgfpathlineto{\pgfqpoint{5.706719in}{2.886339in}}%
\pgfpathlineto{\pgfqpoint{5.711563in}{2.901460in}}%
\pgfpathlineto{\pgfqpoint{5.716406in}{2.910830in}}%
\pgfpathlineto{\pgfqpoint{5.721250in}{2.915170in}}%
\pgfpathlineto{\pgfqpoint{5.735781in}{2.920542in}}%
\pgfpathlineto{\pgfqpoint{5.740625in}{2.917503in}}%
\pgfpathlineto{\pgfqpoint{5.745469in}{2.901322in}}%
\pgfpathlineto{\pgfqpoint{5.750312in}{2.862622in}}%
\pgfpathlineto{\pgfqpoint{5.755156in}{2.800014in}}%
\pgfpathlineto{\pgfqpoint{5.764844in}{2.612550in}}%
\pgfpathlineto{\pgfqpoint{5.764844in}{2.612550in}}%
\pgfusepath{stroke}%
\end{pgfscope}%
\begin{pgfscope}%
\pgfpathrectangle{\pgfqpoint{0.800000in}{0.528000in}}{\pgfqpoint{4.960000in}{3.696000in}}%
\pgfusepath{clip}%
\pgfsetrectcap%
\pgfsetroundjoin%
\pgfsetlinewidth{1.505625pt}%
\definecolor{currentstroke}{rgb}{1.000000,0.498039,0.054902}%
\pgfsetstrokecolor{currentstroke}%
\pgfsetdash{}{0pt}%
\pgfpathmoveto{\pgfqpoint{0.795156in}{1.308026in}}%
\pgfpathlineto{\pgfqpoint{0.800000in}{1.261696in}}%
\pgfpathlineto{\pgfqpoint{0.804844in}{1.189850in}}%
\pgfpathlineto{\pgfqpoint{0.829063in}{0.764732in}}%
\pgfpathlineto{\pgfqpoint{0.833906in}{0.716448in}}%
\pgfpathlineto{\pgfqpoint{0.838750in}{0.696000in}}%
\pgfpathlineto{\pgfqpoint{0.843594in}{0.709334in}}%
\pgfpathlineto{\pgfqpoint{0.848438in}{0.756228in}}%
\pgfpathlineto{\pgfqpoint{0.858125in}{0.926977in}}%
\pgfpathlineto{\pgfqpoint{0.867812in}{1.105559in}}%
\pgfpathlineto{\pgfqpoint{0.872656in}{1.165046in}}%
\pgfpathlineto{\pgfqpoint{0.877500in}{1.195286in}}%
\pgfpathlineto{\pgfqpoint{0.882344in}{1.198713in}}%
\pgfpathlineto{\pgfqpoint{0.887188in}{1.181093in}}%
\pgfpathlineto{\pgfqpoint{0.901719in}{1.081420in}}%
\pgfpathlineto{\pgfqpoint{0.906563in}{1.063306in}}%
\pgfpathlineto{\pgfqpoint{0.911406in}{1.059130in}}%
\pgfpathlineto{\pgfqpoint{0.916250in}{1.069976in}}%
\pgfpathlineto{\pgfqpoint{0.921094in}{1.094168in}}%
\pgfpathlineto{\pgfqpoint{0.925937in}{1.128227in}}%
\pgfpathlineto{\pgfqpoint{0.935625in}{1.220977in}}%
\pgfpathlineto{\pgfqpoint{0.945312in}{1.318976in}}%
\pgfpathlineto{\pgfqpoint{0.950156in}{1.349256in}}%
\pgfpathlineto{\pgfqpoint{0.955000in}{1.352848in}}%
\pgfpathlineto{\pgfqpoint{0.959844in}{1.328932in}}%
\pgfpathlineto{\pgfqpoint{0.964688in}{1.280001in}}%
\pgfpathlineto{\pgfqpoint{0.974375in}{1.140636in}}%
\pgfpathlineto{\pgfqpoint{0.984063in}{0.997990in}}%
\pgfpathlineto{\pgfqpoint{0.988906in}{0.941568in}}%
\pgfpathlineto{\pgfqpoint{0.993750in}{0.903272in}}%
\pgfpathlineto{\pgfqpoint{0.998594in}{0.895282in}}%
\pgfpathlineto{\pgfqpoint{1.003437in}{0.921168in}}%
\pgfpathlineto{\pgfqpoint{1.008281in}{0.983474in}}%
\pgfpathlineto{\pgfqpoint{1.032500in}{1.408773in}}%
\pgfpathlineto{\pgfqpoint{1.037344in}{1.458106in}}%
\pgfpathlineto{\pgfqpoint{1.042188in}{1.495375in}}%
\pgfpathlineto{\pgfqpoint{1.047031in}{1.512905in}}%
\pgfpathlineto{\pgfqpoint{1.051875in}{1.507576in}}%
\pgfpathlineto{\pgfqpoint{1.056719in}{1.471225in}}%
\pgfpathlineto{\pgfqpoint{1.071250in}{1.282019in}}%
\pgfpathlineto{\pgfqpoint{1.076094in}{1.252571in}}%
\pgfpathlineto{\pgfqpoint{1.080938in}{1.262118in}}%
\pgfpathlineto{\pgfqpoint{1.085781in}{1.303056in}}%
\pgfpathlineto{\pgfqpoint{1.095469in}{1.415951in}}%
\pgfpathlineto{\pgfqpoint{1.100313in}{1.455952in}}%
\pgfpathlineto{\pgfqpoint{1.105156in}{1.474439in}}%
\pgfpathlineto{\pgfqpoint{1.110000in}{1.474626in}}%
\pgfpathlineto{\pgfqpoint{1.114844in}{1.457635in}}%
\pgfpathlineto{\pgfqpoint{1.119687in}{1.424671in}}%
\pgfpathlineto{\pgfqpoint{1.124531in}{1.373660in}}%
\pgfpathlineto{\pgfqpoint{1.134219in}{1.223579in}}%
\pgfpathlineto{\pgfqpoint{1.143906in}{1.052582in}}%
\pgfpathlineto{\pgfqpoint{1.148750in}{0.984563in}}%
\pgfpathlineto{\pgfqpoint{1.153594in}{0.940123in}}%
\pgfpathlineto{\pgfqpoint{1.158438in}{0.928970in}}%
\pgfpathlineto{\pgfqpoint{1.163281in}{0.961319in}}%
\pgfpathlineto{\pgfqpoint{1.168125in}{1.042521in}}%
\pgfpathlineto{\pgfqpoint{1.172969in}{1.174613in}}%
\pgfpathlineto{\pgfqpoint{1.182656in}{1.547746in}}%
\pgfpathlineto{\pgfqpoint{1.192344in}{1.927603in}}%
\pgfpathlineto{\pgfqpoint{1.202031in}{2.197099in}}%
\pgfpathlineto{\pgfqpoint{1.206875in}{2.286901in}}%
\pgfpathlineto{\pgfqpoint{1.211719in}{2.352642in}}%
\pgfpathlineto{\pgfqpoint{1.216562in}{2.392498in}}%
\pgfpathlineto{\pgfqpoint{1.221406in}{2.405332in}}%
\pgfpathlineto{\pgfqpoint{1.226250in}{2.394690in}}%
\pgfpathlineto{\pgfqpoint{1.231094in}{2.364971in}}%
\pgfpathlineto{\pgfqpoint{1.245625in}{2.247118in}}%
\pgfpathlineto{\pgfqpoint{1.250469in}{2.222948in}}%
\pgfpathlineto{\pgfqpoint{1.255313in}{2.213236in}}%
\pgfpathlineto{\pgfqpoint{1.260156in}{2.216449in}}%
\pgfpathlineto{\pgfqpoint{1.269844in}{2.243334in}}%
\pgfpathlineto{\pgfqpoint{1.274687in}{2.238560in}}%
\pgfpathlineto{\pgfqpoint{1.279531in}{2.195781in}}%
\pgfpathlineto{\pgfqpoint{1.284375in}{2.104375in}}%
\pgfpathlineto{\pgfqpoint{1.289219in}{1.962028in}}%
\pgfpathlineto{\pgfqpoint{1.298906in}{1.569938in}}%
\pgfpathlineto{\pgfqpoint{1.313438in}{0.980925in}}%
\pgfpathlineto{\pgfqpoint{1.318281in}{0.848160in}}%
\pgfpathlineto{\pgfqpoint{1.323125in}{0.775547in}}%
\pgfpathlineto{\pgfqpoint{1.327969in}{0.777766in}}%
\pgfpathlineto{\pgfqpoint{1.332812in}{0.858447in}}%
\pgfpathlineto{\pgfqpoint{1.337656in}{1.005661in}}%
\pgfpathlineto{\pgfqpoint{1.352188in}{1.533346in}}%
\pgfpathlineto{\pgfqpoint{1.357031in}{1.635295in}}%
\pgfpathlineto{\pgfqpoint{1.361875in}{1.679592in}}%
\pgfpathlineto{\pgfqpoint{1.366719in}{1.674555in}}%
\pgfpathlineto{\pgfqpoint{1.371563in}{1.634577in}}%
\pgfpathlineto{\pgfqpoint{1.381250in}{1.503715in}}%
\pgfpathlineto{\pgfqpoint{1.395781in}{1.295999in}}%
\pgfpathlineto{\pgfqpoint{1.400625in}{1.249585in}}%
\pgfpathlineto{\pgfqpoint{1.405469in}{1.226156in}}%
\pgfpathlineto{\pgfqpoint{1.410313in}{1.228386in}}%
\pgfpathlineto{\pgfqpoint{1.415156in}{1.253462in}}%
\pgfpathlineto{\pgfqpoint{1.429688in}{1.371002in}}%
\pgfpathlineto{\pgfqpoint{1.434531in}{1.382539in}}%
\pgfpathlineto{\pgfqpoint{1.439375in}{1.365991in}}%
\pgfpathlineto{\pgfqpoint{1.444219in}{1.325280in}}%
\pgfpathlineto{\pgfqpoint{1.463594in}{1.101595in}}%
\pgfpathlineto{\pgfqpoint{1.468438in}{1.072065in}}%
\pgfpathlineto{\pgfqpoint{1.473281in}{1.059380in}}%
\pgfpathlineto{\pgfqpoint{1.478125in}{1.061898in}}%
\pgfpathlineto{\pgfqpoint{1.482969in}{1.086526in}}%
\pgfpathlineto{\pgfqpoint{1.487812in}{1.139341in}}%
\pgfpathlineto{\pgfqpoint{1.492656in}{1.221838in}}%
\pgfpathlineto{\pgfqpoint{1.512031in}{1.630874in}}%
\pgfpathlineto{\pgfqpoint{1.516875in}{1.667958in}}%
\pgfpathlineto{\pgfqpoint{1.521719in}{1.663442in}}%
\pgfpathlineto{\pgfqpoint{1.526563in}{1.621537in}}%
\pgfpathlineto{\pgfqpoint{1.536250in}{1.477842in}}%
\pgfpathlineto{\pgfqpoint{1.550781in}{1.244766in}}%
\pgfpathlineto{\pgfqpoint{1.560469in}{1.133086in}}%
\pgfpathlineto{\pgfqpoint{1.565313in}{1.099136in}}%
\pgfpathlineto{\pgfqpoint{1.570156in}{1.085802in}}%
\pgfpathlineto{\pgfqpoint{1.575000in}{1.098533in}}%
\pgfpathlineto{\pgfqpoint{1.579844in}{1.135175in}}%
\pgfpathlineto{\pgfqpoint{1.594375in}{1.292838in}}%
\pgfpathlineto{\pgfqpoint{1.599219in}{1.314541in}}%
\pgfpathlineto{\pgfqpoint{1.604062in}{1.306426in}}%
\pgfpathlineto{\pgfqpoint{1.608906in}{1.273208in}}%
\pgfpathlineto{\pgfqpoint{1.623438in}{1.128468in}}%
\pgfpathlineto{\pgfqpoint{1.628281in}{1.100618in}}%
\pgfpathlineto{\pgfqpoint{1.633125in}{1.091783in}}%
\pgfpathlineto{\pgfqpoint{1.637969in}{1.105687in}}%
\pgfpathlineto{\pgfqpoint{1.642813in}{1.140631in}}%
\pgfpathlineto{\pgfqpoint{1.647656in}{1.195505in}}%
\pgfpathlineto{\pgfqpoint{1.657344in}{1.356438in}}%
\pgfpathlineto{\pgfqpoint{1.671875in}{1.623731in}}%
\pgfpathlineto{\pgfqpoint{1.676719in}{1.674065in}}%
\pgfpathlineto{\pgfqpoint{1.681563in}{1.685404in}}%
\pgfpathlineto{\pgfqpoint{1.686406in}{1.655320in}}%
\pgfpathlineto{\pgfqpoint{1.691250in}{1.591402in}}%
\pgfpathlineto{\pgfqpoint{1.705781in}{1.348728in}}%
\pgfpathlineto{\pgfqpoint{1.710625in}{1.288944in}}%
\pgfpathlineto{\pgfqpoint{1.715469in}{1.243683in}}%
\pgfpathlineto{\pgfqpoint{1.720313in}{1.210357in}}%
\pgfpathlineto{\pgfqpoint{1.725156in}{1.193688in}}%
\pgfpathlineto{\pgfqpoint{1.730000in}{1.197620in}}%
\pgfpathlineto{\pgfqpoint{1.734844in}{1.226728in}}%
\pgfpathlineto{\pgfqpoint{1.739688in}{1.279049in}}%
\pgfpathlineto{\pgfqpoint{1.754219in}{1.461857in}}%
\pgfpathlineto{\pgfqpoint{1.759062in}{1.494164in}}%
\pgfpathlineto{\pgfqpoint{1.763906in}{1.507190in}}%
\pgfpathlineto{\pgfqpoint{1.768750in}{1.507202in}}%
\pgfpathlineto{\pgfqpoint{1.773594in}{1.502179in}}%
\pgfpathlineto{\pgfqpoint{1.778438in}{1.498774in}}%
\pgfpathlineto{\pgfqpoint{1.783281in}{1.498895in}}%
\pgfpathlineto{\pgfqpoint{1.788125in}{1.502464in}}%
\pgfpathlineto{\pgfqpoint{1.792969in}{1.510083in}}%
\pgfpathlineto{\pgfqpoint{1.797813in}{1.525775in}}%
\pgfpathlineto{\pgfqpoint{1.802656in}{1.552234in}}%
\pgfpathlineto{\pgfqpoint{1.807500in}{1.598045in}}%
\pgfpathlineto{\pgfqpoint{1.812344in}{1.664734in}}%
\pgfpathlineto{\pgfqpoint{1.831719in}{1.999266in}}%
\pgfpathlineto{\pgfqpoint{1.836563in}{2.038887in}}%
\pgfpathlineto{\pgfqpoint{1.841406in}{2.046758in}}%
\pgfpathlineto{\pgfqpoint{1.846250in}{2.027994in}}%
\pgfpathlineto{\pgfqpoint{1.851094in}{1.989968in}}%
\pgfpathlineto{\pgfqpoint{1.860781in}{1.876934in}}%
\pgfpathlineto{\pgfqpoint{1.875312in}{1.683080in}}%
\pgfpathlineto{\pgfqpoint{1.880156in}{1.651753in}}%
\pgfpathlineto{\pgfqpoint{1.885000in}{1.653033in}}%
\pgfpathlineto{\pgfqpoint{1.889844in}{1.690020in}}%
\pgfpathlineto{\pgfqpoint{1.909219in}{1.939608in}}%
\pgfpathlineto{\pgfqpoint{1.918906in}{2.007855in}}%
\pgfpathlineto{\pgfqpoint{1.928594in}{2.065110in}}%
\pgfpathlineto{\pgfqpoint{1.933438in}{2.088338in}}%
\pgfpathlineto{\pgfqpoint{1.938281in}{2.098624in}}%
\pgfpathlineto{\pgfqpoint{1.943125in}{2.093513in}}%
\pgfpathlineto{\pgfqpoint{1.952813in}{2.056318in}}%
\pgfpathlineto{\pgfqpoint{1.957656in}{2.047412in}}%
\pgfpathlineto{\pgfqpoint{1.962500in}{2.062231in}}%
\pgfpathlineto{\pgfqpoint{1.967344in}{2.105000in}}%
\pgfpathlineto{\pgfqpoint{1.972187in}{2.172950in}}%
\pgfpathlineto{\pgfqpoint{1.986719in}{2.417432in}}%
\pgfpathlineto{\pgfqpoint{1.991563in}{2.466949in}}%
\pgfpathlineto{\pgfqpoint{1.996406in}{2.487137in}}%
\pgfpathlineto{\pgfqpoint{2.001250in}{2.475844in}}%
\pgfpathlineto{\pgfqpoint{2.006094in}{2.440540in}}%
\pgfpathlineto{\pgfqpoint{2.015781in}{2.324378in}}%
\pgfpathlineto{\pgfqpoint{2.030312in}{2.103606in}}%
\pgfpathlineto{\pgfqpoint{2.035156in}{2.031087in}}%
\pgfpathlineto{\pgfqpoint{2.040000in}{1.975373in}}%
\pgfpathlineto{\pgfqpoint{2.044844in}{1.945894in}}%
\pgfpathlineto{\pgfqpoint{2.049688in}{1.953810in}}%
\pgfpathlineto{\pgfqpoint{2.054531in}{1.996507in}}%
\pgfpathlineto{\pgfqpoint{2.064219in}{2.152447in}}%
\pgfpathlineto{\pgfqpoint{2.073906in}{2.319383in}}%
\pgfpathlineto{\pgfqpoint{2.078750in}{2.384786in}}%
\pgfpathlineto{\pgfqpoint{2.083594in}{2.434938in}}%
\pgfpathlineto{\pgfqpoint{2.088437in}{2.465190in}}%
\pgfpathlineto{\pgfqpoint{2.093281in}{2.475341in}}%
\pgfpathlineto{\pgfqpoint{2.098125in}{2.464206in}}%
\pgfpathlineto{\pgfqpoint{2.102969in}{2.434081in}}%
\pgfpathlineto{\pgfqpoint{2.112656in}{2.348843in}}%
\pgfpathlineto{\pgfqpoint{2.117500in}{2.318899in}}%
\pgfpathlineto{\pgfqpoint{2.122344in}{2.314283in}}%
\pgfpathlineto{\pgfqpoint{2.127188in}{2.341400in}}%
\pgfpathlineto{\pgfqpoint{2.132031in}{2.397151in}}%
\pgfpathlineto{\pgfqpoint{2.146562in}{2.612303in}}%
\pgfpathlineto{\pgfqpoint{2.151406in}{2.646521in}}%
\pgfpathlineto{\pgfqpoint{2.156250in}{2.648596in}}%
\pgfpathlineto{\pgfqpoint{2.161094in}{2.621415in}}%
\pgfpathlineto{\pgfqpoint{2.165938in}{2.572155in}}%
\pgfpathlineto{\pgfqpoint{2.195000in}{2.193386in}}%
\pgfpathlineto{\pgfqpoint{2.199844in}{2.152081in}}%
\pgfpathlineto{\pgfqpoint{2.204687in}{2.132624in}}%
\pgfpathlineto{\pgfqpoint{2.209531in}{2.146415in}}%
\pgfpathlineto{\pgfqpoint{2.214375in}{2.196179in}}%
\pgfpathlineto{\pgfqpoint{2.224063in}{2.369881in}}%
\pgfpathlineto{\pgfqpoint{2.233750in}{2.535728in}}%
\pgfpathlineto{\pgfqpoint{2.238594in}{2.582961in}}%
\pgfpathlineto{\pgfqpoint{2.243438in}{2.604767in}}%
\pgfpathlineto{\pgfqpoint{2.248281in}{2.609884in}}%
\pgfpathlineto{\pgfqpoint{2.253125in}{2.606418in}}%
\pgfpathlineto{\pgfqpoint{2.257969in}{2.600369in}}%
\pgfpathlineto{\pgfqpoint{2.262812in}{2.589863in}}%
\pgfpathlineto{\pgfqpoint{2.267656in}{2.574501in}}%
\pgfpathlineto{\pgfqpoint{2.277344in}{2.533181in}}%
\pgfpathlineto{\pgfqpoint{2.282188in}{2.522748in}}%
\pgfpathlineto{\pgfqpoint{2.287031in}{2.528982in}}%
\pgfpathlineto{\pgfqpoint{2.291875in}{2.557527in}}%
\pgfpathlineto{\pgfqpoint{2.311250in}{2.734044in}}%
\pgfpathlineto{\pgfqpoint{2.316094in}{2.740812in}}%
\pgfpathlineto{\pgfqpoint{2.320937in}{2.723640in}}%
\pgfpathlineto{\pgfqpoint{2.325781in}{2.685452in}}%
\pgfpathlineto{\pgfqpoint{2.330625in}{2.631131in}}%
\pgfpathlineto{\pgfqpoint{2.340313in}{2.486098in}}%
\pgfpathlineto{\pgfqpoint{2.350000in}{2.328238in}}%
\pgfpathlineto{\pgfqpoint{2.354844in}{2.270532in}}%
\pgfpathlineto{\pgfqpoint{2.359688in}{2.242970in}}%
\pgfpathlineto{\pgfqpoint{2.364531in}{2.247617in}}%
\pgfpathlineto{\pgfqpoint{2.369375in}{2.281887in}}%
\pgfpathlineto{\pgfqpoint{2.379062in}{2.403838in}}%
\pgfpathlineto{\pgfqpoint{2.393594in}{2.649218in}}%
\pgfpathlineto{\pgfqpoint{2.398438in}{2.729283in}}%
\pgfpathlineto{\pgfqpoint{2.403281in}{2.790326in}}%
\pgfpathlineto{\pgfqpoint{2.408125in}{2.821106in}}%
\pgfpathlineto{\pgfqpoint{2.412969in}{2.816260in}}%
\pgfpathlineto{\pgfqpoint{2.417813in}{2.781358in}}%
\pgfpathlineto{\pgfqpoint{2.432344in}{2.623544in}}%
\pgfpathlineto{\pgfqpoint{2.437187in}{2.586873in}}%
\pgfpathlineto{\pgfqpoint{2.442031in}{2.563337in}}%
\pgfpathlineto{\pgfqpoint{2.446875in}{2.552264in}}%
\pgfpathlineto{\pgfqpoint{2.451719in}{2.553876in}}%
\pgfpathlineto{\pgfqpoint{2.456562in}{2.568006in}}%
\pgfpathlineto{\pgfqpoint{2.461406in}{2.595828in}}%
\pgfpathlineto{\pgfqpoint{2.475938in}{2.693812in}}%
\pgfpathlineto{\pgfqpoint{2.480781in}{2.707817in}}%
\pgfpathlineto{\pgfqpoint{2.485625in}{2.702083in}}%
\pgfpathlineto{\pgfqpoint{2.490469in}{2.670533in}}%
\pgfpathlineto{\pgfqpoint{2.495313in}{2.611550in}}%
\pgfpathlineto{\pgfqpoint{2.505000in}{2.437267in}}%
\pgfpathlineto{\pgfqpoint{2.509844in}{2.345937in}}%
\pgfpathlineto{\pgfqpoint{2.514687in}{2.276639in}}%
\pgfpathlineto{\pgfqpoint{2.519531in}{2.240547in}}%
\pgfpathlineto{\pgfqpoint{2.524375in}{2.246251in}}%
\pgfpathlineto{\pgfqpoint{2.529219in}{2.288766in}}%
\pgfpathlineto{\pgfqpoint{2.534063in}{2.358307in}}%
\pgfpathlineto{\pgfqpoint{2.553438in}{2.691086in}}%
\pgfpathlineto{\pgfqpoint{2.558281in}{2.752797in}}%
\pgfpathlineto{\pgfqpoint{2.563125in}{2.798059in}}%
\pgfpathlineto{\pgfqpoint{2.567969in}{2.826725in}}%
\pgfpathlineto{\pgfqpoint{2.572812in}{2.836846in}}%
\pgfpathlineto{\pgfqpoint{2.577656in}{2.825936in}}%
\pgfpathlineto{\pgfqpoint{2.582500in}{2.795653in}}%
\pgfpathlineto{\pgfqpoint{2.601875in}{2.630462in}}%
\pgfpathlineto{\pgfqpoint{2.606719in}{2.606762in}}%
\pgfpathlineto{\pgfqpoint{2.611563in}{2.593093in}}%
\pgfpathlineto{\pgfqpoint{2.616406in}{2.590206in}}%
\pgfpathlineto{\pgfqpoint{2.621250in}{2.594880in}}%
\pgfpathlineto{\pgfqpoint{2.640625in}{2.627094in}}%
\pgfpathlineto{\pgfqpoint{2.645469in}{2.627383in}}%
\pgfpathlineto{\pgfqpoint{2.650313in}{2.615238in}}%
\pgfpathlineto{\pgfqpoint{2.655156in}{2.585623in}}%
\pgfpathlineto{\pgfqpoint{2.660000in}{2.536313in}}%
\pgfpathlineto{\pgfqpoint{2.674531in}{2.342701in}}%
\pgfpathlineto{\pgfqpoint{2.679375in}{2.314679in}}%
\pgfpathlineto{\pgfqpoint{2.684219in}{2.325682in}}%
\pgfpathlineto{\pgfqpoint{2.689062in}{2.374795in}}%
\pgfpathlineto{\pgfqpoint{2.698750in}{2.538274in}}%
\pgfpathlineto{\pgfqpoint{2.708438in}{2.708996in}}%
\pgfpathlineto{\pgfqpoint{2.718125in}{2.840083in}}%
\pgfpathlineto{\pgfqpoint{2.722969in}{2.882998in}}%
\pgfpathlineto{\pgfqpoint{2.727813in}{2.905338in}}%
\pgfpathlineto{\pgfqpoint{2.732656in}{2.904817in}}%
\pgfpathlineto{\pgfqpoint{2.737500in}{2.880000in}}%
\pgfpathlineto{\pgfqpoint{2.742344in}{2.835048in}}%
\pgfpathlineto{\pgfqpoint{2.756875in}{2.657621in}}%
\pgfpathlineto{\pgfqpoint{2.761719in}{2.618753in}}%
\pgfpathlineto{\pgfqpoint{2.766563in}{2.598443in}}%
\pgfpathlineto{\pgfqpoint{2.771406in}{2.596239in}}%
\pgfpathlineto{\pgfqpoint{2.776250in}{2.605998in}}%
\pgfpathlineto{\pgfqpoint{2.781094in}{2.623879in}}%
\pgfpathlineto{\pgfqpoint{2.785938in}{2.650302in}}%
\pgfpathlineto{\pgfqpoint{2.795625in}{2.721758in}}%
\pgfpathlineto{\pgfqpoint{2.800469in}{2.759825in}}%
\pgfpathlineto{\pgfqpoint{2.805312in}{2.781756in}}%
\pgfpathlineto{\pgfqpoint{2.810156in}{2.775966in}}%
\pgfpathlineto{\pgfqpoint{2.815000in}{2.740346in}}%
\pgfpathlineto{\pgfqpoint{2.829531in}{2.563441in}}%
\pgfpathlineto{\pgfqpoint{2.834375in}{2.533524in}}%
\pgfpathlineto{\pgfqpoint{2.839219in}{2.530422in}}%
\pgfpathlineto{\pgfqpoint{2.844063in}{2.551296in}}%
\pgfpathlineto{\pgfqpoint{2.848906in}{2.590346in}}%
\pgfpathlineto{\pgfqpoint{2.853750in}{2.646159in}}%
\pgfpathlineto{\pgfqpoint{2.863437in}{2.799534in}}%
\pgfpathlineto{\pgfqpoint{2.877969in}{3.069883in}}%
\pgfpathlineto{\pgfqpoint{2.882812in}{3.134690in}}%
\pgfpathlineto{\pgfqpoint{2.887656in}{3.170357in}}%
\pgfpathlineto{\pgfqpoint{2.892500in}{3.173069in}}%
\pgfpathlineto{\pgfqpoint{2.897344in}{3.142496in}}%
\pgfpathlineto{\pgfqpoint{2.902188in}{3.081451in}}%
\pgfpathlineto{\pgfqpoint{2.921562in}{2.764591in}}%
\pgfpathlineto{\pgfqpoint{2.926406in}{2.714274in}}%
\pgfpathlineto{\pgfqpoint{2.931250in}{2.687822in}}%
\pgfpathlineto{\pgfqpoint{2.936094in}{2.683684in}}%
\pgfpathlineto{\pgfqpoint{2.940937in}{2.699870in}}%
\pgfpathlineto{\pgfqpoint{2.960313in}{2.816590in}}%
\pgfpathlineto{\pgfqpoint{2.965156in}{2.824765in}}%
\pgfpathlineto{\pgfqpoint{2.970000in}{2.817815in}}%
\pgfpathlineto{\pgfqpoint{2.974844in}{2.795205in}}%
\pgfpathlineto{\pgfqpoint{2.979688in}{2.754128in}}%
\pgfpathlineto{\pgfqpoint{2.994219in}{2.592687in}}%
\pgfpathlineto{\pgfqpoint{2.999062in}{2.566032in}}%
\pgfpathlineto{\pgfqpoint{3.003906in}{2.569670in}}%
\pgfpathlineto{\pgfqpoint{3.008750in}{2.600738in}}%
\pgfpathlineto{\pgfqpoint{3.023281in}{2.761568in}}%
\pgfpathlineto{\pgfqpoint{3.042656in}{2.969982in}}%
\pgfpathlineto{\pgfqpoint{3.047500in}{2.989285in}}%
\pgfpathlineto{\pgfqpoint{3.052344in}{2.975317in}}%
\pgfpathlineto{\pgfqpoint{3.057187in}{2.926294in}}%
\pgfpathlineto{\pgfqpoint{3.066875in}{2.763580in}}%
\pgfpathlineto{\pgfqpoint{3.076563in}{2.595724in}}%
\pgfpathlineto{\pgfqpoint{3.081406in}{2.528620in}}%
\pgfpathlineto{\pgfqpoint{3.086250in}{2.477314in}}%
\pgfpathlineto{\pgfqpoint{3.091094in}{2.449256in}}%
\pgfpathlineto{\pgfqpoint{3.095938in}{2.444778in}}%
\pgfpathlineto{\pgfqpoint{3.100781in}{2.466203in}}%
\pgfpathlineto{\pgfqpoint{3.115312in}{2.584847in}}%
\pgfpathlineto{\pgfqpoint{3.120156in}{2.608491in}}%
\pgfpathlineto{\pgfqpoint{3.125000in}{2.616783in}}%
\pgfpathlineto{\pgfqpoint{3.129844in}{2.610399in}}%
\pgfpathlineto{\pgfqpoint{3.134688in}{2.590363in}}%
\pgfpathlineto{\pgfqpoint{3.139531in}{2.556764in}}%
\pgfpathlineto{\pgfqpoint{3.154063in}{2.437598in}}%
\pgfpathlineto{\pgfqpoint{3.158906in}{2.425876in}}%
\pgfpathlineto{\pgfqpoint{3.163750in}{2.438818in}}%
\pgfpathlineto{\pgfqpoint{3.168594in}{2.476608in}}%
\pgfpathlineto{\pgfqpoint{3.178281in}{2.601782in}}%
\pgfpathlineto{\pgfqpoint{3.192813in}{2.820014in}}%
\pgfpathlineto{\pgfqpoint{3.197656in}{2.876403in}}%
\pgfpathlineto{\pgfqpoint{3.202500in}{2.911542in}}%
\pgfpathlineto{\pgfqpoint{3.207344in}{2.918045in}}%
\pgfpathlineto{\pgfqpoint{3.212188in}{2.895615in}}%
\pgfpathlineto{\pgfqpoint{3.217031in}{2.846144in}}%
\pgfpathlineto{\pgfqpoint{3.226719in}{2.693200in}}%
\pgfpathlineto{\pgfqpoint{3.241250in}{2.443035in}}%
\pgfpathlineto{\pgfqpoint{3.246094in}{2.383921in}}%
\pgfpathlineto{\pgfqpoint{3.250938in}{2.349862in}}%
\pgfpathlineto{\pgfqpoint{3.255781in}{2.344231in}}%
\pgfpathlineto{\pgfqpoint{3.260625in}{2.364532in}}%
\pgfpathlineto{\pgfqpoint{3.265469in}{2.408007in}}%
\pgfpathlineto{\pgfqpoint{3.284844in}{2.638833in}}%
\pgfpathlineto{\pgfqpoint{3.289687in}{2.669708in}}%
\pgfpathlineto{\pgfqpoint{3.294531in}{2.678090in}}%
\pgfpathlineto{\pgfqpoint{3.299375in}{2.664431in}}%
\pgfpathlineto{\pgfqpoint{3.304219in}{2.631846in}}%
\pgfpathlineto{\pgfqpoint{3.318750in}{2.492452in}}%
\pgfpathlineto{\pgfqpoint{3.323594in}{2.462503in}}%
\pgfpathlineto{\pgfqpoint{3.328438in}{2.452224in}}%
\pgfpathlineto{\pgfqpoint{3.333281in}{2.465930in}}%
\pgfpathlineto{\pgfqpoint{3.338125in}{2.502285in}}%
\pgfpathlineto{\pgfqpoint{3.347812in}{2.617874in}}%
\pgfpathlineto{\pgfqpoint{3.352656in}{2.675813in}}%
\pgfpathlineto{\pgfqpoint{3.357500in}{2.714730in}}%
\pgfpathlineto{\pgfqpoint{3.362344in}{2.724835in}}%
\pgfpathlineto{\pgfqpoint{3.367188in}{2.698690in}}%
\pgfpathlineto{\pgfqpoint{3.372031in}{2.635809in}}%
\pgfpathlineto{\pgfqpoint{3.381719in}{2.423846in}}%
\pgfpathlineto{\pgfqpoint{3.391406in}{2.203163in}}%
\pgfpathlineto{\pgfqpoint{3.396250in}{2.133710in}}%
\pgfpathlineto{\pgfqpoint{3.401094in}{2.108679in}}%
\pgfpathlineto{\pgfqpoint{3.405938in}{2.131460in}}%
\pgfpathlineto{\pgfqpoint{3.410781in}{2.197493in}}%
\pgfpathlineto{\pgfqpoint{3.415625in}{2.302760in}}%
\pgfpathlineto{\pgfqpoint{3.425312in}{2.606104in}}%
\pgfpathlineto{\pgfqpoint{3.439844in}{3.135005in}}%
\pgfpathlineto{\pgfqpoint{3.444688in}{3.263235in}}%
\pgfpathlineto{\pgfqpoint{3.449531in}{3.348977in}}%
\pgfpathlineto{\pgfqpoint{3.454375in}{3.396499in}}%
\pgfpathlineto{\pgfqpoint{3.459219in}{3.418293in}}%
\pgfpathlineto{\pgfqpoint{3.464063in}{3.424519in}}%
\pgfpathlineto{\pgfqpoint{3.468906in}{3.419775in}}%
\pgfpathlineto{\pgfqpoint{3.473750in}{3.403424in}}%
\pgfpathlineto{\pgfqpoint{3.483437in}{3.350581in}}%
\pgfpathlineto{\pgfqpoint{3.488281in}{3.335799in}}%
\pgfpathlineto{\pgfqpoint{3.493125in}{3.347312in}}%
\pgfpathlineto{\pgfqpoint{3.497969in}{3.384707in}}%
\pgfpathlineto{\pgfqpoint{3.507656in}{3.474332in}}%
\pgfpathlineto{\pgfqpoint{3.512500in}{3.477021in}}%
\pgfpathlineto{\pgfqpoint{3.517344in}{3.427933in}}%
\pgfpathlineto{\pgfqpoint{3.522188in}{3.325150in}}%
\pgfpathlineto{\pgfqpoint{3.527031in}{3.175331in}}%
\pgfpathlineto{\pgfqpoint{3.536719in}{2.772344in}}%
\pgfpathlineto{\pgfqpoint{3.551250in}{2.104187in}}%
\pgfpathlineto{\pgfqpoint{3.556094in}{1.949586in}}%
\pgfpathlineto{\pgfqpoint{3.560938in}{1.864393in}}%
\pgfpathlineto{\pgfqpoint{3.565781in}{1.850869in}}%
\pgfpathlineto{\pgfqpoint{3.570625in}{1.895666in}}%
\pgfpathlineto{\pgfqpoint{3.580313in}{2.082415in}}%
\pgfpathlineto{\pgfqpoint{3.594844in}{2.389257in}}%
\pgfpathlineto{\pgfqpoint{3.599687in}{2.468051in}}%
\pgfpathlineto{\pgfqpoint{3.604531in}{2.528466in}}%
\pgfpathlineto{\pgfqpoint{3.609375in}{2.564687in}}%
\pgfpathlineto{\pgfqpoint{3.614219in}{2.577598in}}%
\pgfpathlineto{\pgfqpoint{3.619063in}{2.573798in}}%
\pgfpathlineto{\pgfqpoint{3.623906in}{2.556427in}}%
\pgfpathlineto{\pgfqpoint{3.628750in}{2.531027in}}%
\pgfpathlineto{\pgfqpoint{3.643281in}{2.432569in}}%
\pgfpathlineto{\pgfqpoint{3.648125in}{2.412483in}}%
\pgfpathlineto{\pgfqpoint{3.652969in}{2.412432in}}%
\pgfpathlineto{\pgfqpoint{3.657812in}{2.432997in}}%
\pgfpathlineto{\pgfqpoint{3.677188in}{2.570182in}}%
\pgfpathlineto{\pgfqpoint{3.682031in}{2.572964in}}%
\pgfpathlineto{\pgfqpoint{3.686875in}{2.558429in}}%
\pgfpathlineto{\pgfqpoint{3.691719in}{2.522937in}}%
\pgfpathlineto{\pgfqpoint{3.696563in}{2.466024in}}%
\pgfpathlineto{\pgfqpoint{3.706250in}{2.292574in}}%
\pgfpathlineto{\pgfqpoint{3.715937in}{2.088967in}}%
\pgfpathlineto{\pgfqpoint{3.720781in}{2.010797in}}%
\pgfpathlineto{\pgfqpoint{3.725625in}{1.967008in}}%
\pgfpathlineto{\pgfqpoint{3.730469in}{1.963713in}}%
\pgfpathlineto{\pgfqpoint{3.735313in}{1.998606in}}%
\pgfpathlineto{\pgfqpoint{3.740156in}{2.060761in}}%
\pgfpathlineto{\pgfqpoint{3.754688in}{2.288772in}}%
\pgfpathlineto{\pgfqpoint{3.764375in}{2.399127in}}%
\pgfpathlineto{\pgfqpoint{3.769219in}{2.436740in}}%
\pgfpathlineto{\pgfqpoint{3.774062in}{2.459416in}}%
\pgfpathlineto{\pgfqpoint{3.778906in}{2.465413in}}%
\pgfpathlineto{\pgfqpoint{3.783750in}{2.450966in}}%
\pgfpathlineto{\pgfqpoint{3.788594in}{2.416222in}}%
\pgfpathlineto{\pgfqpoint{3.803125in}{2.283510in}}%
\pgfpathlineto{\pgfqpoint{3.807969in}{2.264718in}}%
\pgfpathlineto{\pgfqpoint{3.812813in}{2.269269in}}%
\pgfpathlineto{\pgfqpoint{3.817656in}{2.296180in}}%
\pgfpathlineto{\pgfqpoint{3.832187in}{2.416799in}}%
\pgfpathlineto{\pgfqpoint{3.837031in}{2.437598in}}%
\pgfpathlineto{\pgfqpoint{3.841875in}{2.437811in}}%
\pgfpathlineto{\pgfqpoint{3.846719in}{2.419878in}}%
\pgfpathlineto{\pgfqpoint{3.851562in}{2.384871in}}%
\pgfpathlineto{\pgfqpoint{3.856406in}{2.333020in}}%
\pgfpathlineto{\pgfqpoint{3.866094in}{2.188507in}}%
\pgfpathlineto{\pgfqpoint{3.880625in}{1.949100in}}%
\pgfpathlineto{\pgfqpoint{3.885469in}{1.904614in}}%
\pgfpathlineto{\pgfqpoint{3.890313in}{1.889760in}}%
\pgfpathlineto{\pgfqpoint{3.895156in}{1.904965in}}%
\pgfpathlineto{\pgfqpoint{3.900000in}{1.945922in}}%
\pgfpathlineto{\pgfqpoint{3.909687in}{2.088989in}}%
\pgfpathlineto{\pgfqpoint{3.924219in}{2.340506in}}%
\pgfpathlineto{\pgfqpoint{3.929063in}{2.392772in}}%
\pgfpathlineto{\pgfqpoint{3.933906in}{2.419264in}}%
\pgfpathlineto{\pgfqpoint{3.938750in}{2.419055in}}%
\pgfpathlineto{\pgfqpoint{3.943594in}{2.399630in}}%
\pgfpathlineto{\pgfqpoint{3.948438in}{2.367423in}}%
\pgfpathlineto{\pgfqpoint{3.958125in}{2.281289in}}%
\pgfpathlineto{\pgfqpoint{3.967812in}{2.185747in}}%
\pgfpathlineto{\pgfqpoint{3.972656in}{2.147756in}}%
\pgfpathlineto{\pgfqpoint{3.977500in}{2.127975in}}%
\pgfpathlineto{\pgfqpoint{3.982344in}{2.129899in}}%
\pgfpathlineto{\pgfqpoint{3.987188in}{2.150518in}}%
\pgfpathlineto{\pgfqpoint{3.996875in}{2.211714in}}%
\pgfpathlineto{\pgfqpoint{4.001719in}{2.230476in}}%
\pgfpathlineto{\pgfqpoint{4.006563in}{2.230424in}}%
\pgfpathlineto{\pgfqpoint{4.011406in}{2.214652in}}%
\pgfpathlineto{\pgfqpoint{4.016250in}{2.186533in}}%
\pgfpathlineto{\pgfqpoint{4.025938in}{2.106241in}}%
\pgfpathlineto{\pgfqpoint{4.040469in}{1.971845in}}%
\pgfpathlineto{\pgfqpoint{4.045312in}{1.948718in}}%
\pgfpathlineto{\pgfqpoint{4.050156in}{1.950033in}}%
\pgfpathlineto{\pgfqpoint{4.055000in}{1.982829in}}%
\pgfpathlineto{\pgfqpoint{4.059844in}{2.041746in}}%
\pgfpathlineto{\pgfqpoint{4.084062in}{2.424461in}}%
\pgfpathlineto{\pgfqpoint{4.088906in}{2.471205in}}%
\pgfpathlineto{\pgfqpoint{4.093750in}{2.498423in}}%
\pgfpathlineto{\pgfqpoint{4.098594in}{2.503552in}}%
\pgfpathlineto{\pgfqpoint{4.103438in}{2.483514in}}%
\pgfpathlineto{\pgfqpoint{4.108281in}{2.439499in}}%
\pgfpathlineto{\pgfqpoint{4.117969in}{2.305909in}}%
\pgfpathlineto{\pgfqpoint{4.127656in}{2.183130in}}%
\pgfpathlineto{\pgfqpoint{4.132500in}{2.149290in}}%
\pgfpathlineto{\pgfqpoint{4.137344in}{2.141571in}}%
\pgfpathlineto{\pgfqpoint{4.142188in}{2.157881in}}%
\pgfpathlineto{\pgfqpoint{4.147031in}{2.192047in}}%
\pgfpathlineto{\pgfqpoint{4.161563in}{2.317270in}}%
\pgfpathlineto{\pgfqpoint{4.166406in}{2.335593in}}%
\pgfpathlineto{\pgfqpoint{4.171250in}{2.330361in}}%
\pgfpathlineto{\pgfqpoint{4.176094in}{2.298965in}}%
\pgfpathlineto{\pgfqpoint{4.180937in}{2.244336in}}%
\pgfpathlineto{\pgfqpoint{4.195469in}{2.049748in}}%
\pgfpathlineto{\pgfqpoint{4.200312in}{2.019329in}}%
\pgfpathlineto{\pgfqpoint{4.205156in}{2.016646in}}%
\pgfpathlineto{\pgfqpoint{4.210000in}{2.033795in}}%
\pgfpathlineto{\pgfqpoint{4.214844in}{2.066441in}}%
\pgfpathlineto{\pgfqpoint{4.219688in}{2.114704in}}%
\pgfpathlineto{\pgfqpoint{4.224531in}{2.180632in}}%
\pgfpathlineto{\pgfqpoint{4.234219in}{2.375075in}}%
\pgfpathlineto{\pgfqpoint{4.243906in}{2.581360in}}%
\pgfpathlineto{\pgfqpoint{4.248750in}{2.645339in}}%
\pgfpathlineto{\pgfqpoint{4.253594in}{2.668217in}}%
\pgfpathlineto{\pgfqpoint{4.258438in}{2.653627in}}%
\pgfpathlineto{\pgfqpoint{4.263281in}{2.609902in}}%
\pgfpathlineto{\pgfqpoint{4.272969in}{2.475456in}}%
\pgfpathlineto{\pgfqpoint{4.287500in}{2.262209in}}%
\pgfpathlineto{\pgfqpoint{4.292344in}{2.216980in}}%
\pgfpathlineto{\pgfqpoint{4.297187in}{2.193214in}}%
\pgfpathlineto{\pgfqpoint{4.302031in}{2.190856in}}%
\pgfpathlineto{\pgfqpoint{4.306875in}{2.205616in}}%
\pgfpathlineto{\pgfqpoint{4.311719in}{2.230814in}}%
\pgfpathlineto{\pgfqpoint{4.321406in}{2.292327in}}%
\pgfpathlineto{\pgfqpoint{4.326250in}{2.313553in}}%
\pgfpathlineto{\pgfqpoint{4.331094in}{2.319780in}}%
\pgfpathlineto{\pgfqpoint{4.335938in}{2.307492in}}%
\pgfpathlineto{\pgfqpoint{4.340781in}{2.274779in}}%
\pgfpathlineto{\pgfqpoint{4.355313in}{2.142329in}}%
\pgfpathlineto{\pgfqpoint{4.360156in}{2.113881in}}%
\pgfpathlineto{\pgfqpoint{4.365000in}{2.098849in}}%
\pgfpathlineto{\pgfqpoint{4.369844in}{2.102324in}}%
\pgfpathlineto{\pgfqpoint{4.374688in}{2.129298in}}%
\pgfpathlineto{\pgfqpoint{4.379531in}{2.184197in}}%
\pgfpathlineto{\pgfqpoint{4.389219in}{2.364582in}}%
\pgfpathlineto{\pgfqpoint{4.398906in}{2.543213in}}%
\pgfpathlineto{\pgfqpoint{4.403750in}{2.598045in}}%
\pgfpathlineto{\pgfqpoint{4.408594in}{2.621550in}}%
\pgfpathlineto{\pgfqpoint{4.413437in}{2.620416in}}%
\pgfpathlineto{\pgfqpoint{4.418281in}{2.599389in}}%
\pgfpathlineto{\pgfqpoint{4.423125in}{2.562780in}}%
\pgfpathlineto{\pgfqpoint{4.427969in}{2.511648in}}%
\pgfpathlineto{\pgfqpoint{4.437656in}{2.370081in}}%
\pgfpathlineto{\pgfqpoint{4.452187in}{2.131399in}}%
\pgfpathlineto{\pgfqpoint{4.457031in}{2.080690in}}%
\pgfpathlineto{\pgfqpoint{4.461875in}{2.057785in}}%
\pgfpathlineto{\pgfqpoint{4.466719in}{2.064026in}}%
\pgfpathlineto{\pgfqpoint{4.471563in}{2.095569in}}%
\pgfpathlineto{\pgfqpoint{4.486094in}{2.237383in}}%
\pgfpathlineto{\pgfqpoint{4.490938in}{2.262785in}}%
\pgfpathlineto{\pgfqpoint{4.495781in}{2.262608in}}%
\pgfpathlineto{\pgfqpoint{4.500625in}{2.240376in}}%
\pgfpathlineto{\pgfqpoint{4.510313in}{2.155191in}}%
\pgfpathlineto{\pgfqpoint{4.515156in}{2.112931in}}%
\pgfpathlineto{\pgfqpoint{4.520000in}{2.086908in}}%
\pgfpathlineto{\pgfqpoint{4.524844in}{2.081403in}}%
\pgfpathlineto{\pgfqpoint{4.529688in}{2.099120in}}%
\pgfpathlineto{\pgfqpoint{4.534531in}{2.136445in}}%
\pgfpathlineto{\pgfqpoint{4.544219in}{2.250720in}}%
\pgfpathlineto{\pgfqpoint{4.558750in}{2.448578in}}%
\pgfpathlineto{\pgfqpoint{4.563594in}{2.497124in}}%
\pgfpathlineto{\pgfqpoint{4.568437in}{2.523888in}}%
\pgfpathlineto{\pgfqpoint{4.573281in}{2.521325in}}%
\pgfpathlineto{\pgfqpoint{4.578125in}{2.483732in}}%
\pgfpathlineto{\pgfqpoint{4.582969in}{2.417829in}}%
\pgfpathlineto{\pgfqpoint{4.602344in}{2.068815in}}%
\pgfpathlineto{\pgfqpoint{4.607188in}{2.011554in}}%
\pgfpathlineto{\pgfqpoint{4.612031in}{1.972668in}}%
\pgfpathlineto{\pgfqpoint{4.616875in}{1.948666in}}%
\pgfpathlineto{\pgfqpoint{4.621719in}{1.936737in}}%
\pgfpathlineto{\pgfqpoint{4.626563in}{1.936003in}}%
\pgfpathlineto{\pgfqpoint{4.631406in}{1.947946in}}%
\pgfpathlineto{\pgfqpoint{4.636250in}{1.973332in}}%
\pgfpathlineto{\pgfqpoint{4.650781in}{2.075223in}}%
\pgfpathlineto{\pgfqpoint{4.655625in}{2.086771in}}%
\pgfpathlineto{\pgfqpoint{4.660469in}{2.081808in}}%
\pgfpathlineto{\pgfqpoint{4.670156in}{2.046758in}}%
\pgfpathlineto{\pgfqpoint{4.689531in}{1.980053in}}%
\pgfpathlineto{\pgfqpoint{4.694375in}{1.974159in}}%
\pgfpathlineto{\pgfqpoint{4.699219in}{1.990883in}}%
\pgfpathlineto{\pgfqpoint{4.704063in}{2.036768in}}%
\pgfpathlineto{\pgfqpoint{4.718594in}{2.251275in}}%
\pgfpathlineto{\pgfqpoint{4.723438in}{2.289912in}}%
\pgfpathlineto{\pgfqpoint{4.728281in}{2.295147in}}%
\pgfpathlineto{\pgfqpoint{4.733125in}{2.273588in}}%
\pgfpathlineto{\pgfqpoint{4.737969in}{2.234836in}}%
\pgfpathlineto{\pgfqpoint{4.747656in}{2.122127in}}%
\pgfpathlineto{\pgfqpoint{4.757344in}{1.960495in}}%
\pgfpathlineto{\pgfqpoint{4.767031in}{1.779950in}}%
\pgfpathlineto{\pgfqpoint{4.771875in}{1.712722in}}%
\pgfpathlineto{\pgfqpoint{4.776719in}{1.674783in}}%
\pgfpathlineto{\pgfqpoint{4.781562in}{1.671405in}}%
\pgfpathlineto{\pgfqpoint{4.786406in}{1.698109in}}%
\pgfpathlineto{\pgfqpoint{4.791250in}{1.750032in}}%
\pgfpathlineto{\pgfqpoint{4.810625in}{2.013919in}}%
\pgfpathlineto{\pgfqpoint{4.815469in}{2.048612in}}%
\pgfpathlineto{\pgfqpoint{4.820312in}{2.064125in}}%
\pgfpathlineto{\pgfqpoint{4.825156in}{2.062463in}}%
\pgfpathlineto{\pgfqpoint{4.830000in}{2.047108in}}%
\pgfpathlineto{\pgfqpoint{4.839688in}{2.001618in}}%
\pgfpathlineto{\pgfqpoint{4.849375in}{1.945864in}}%
\pgfpathlineto{\pgfqpoint{4.854219in}{1.924699in}}%
\pgfpathlineto{\pgfqpoint{4.859063in}{1.919268in}}%
\pgfpathlineto{\pgfqpoint{4.863906in}{1.938291in}}%
\pgfpathlineto{\pgfqpoint{4.868750in}{1.985132in}}%
\pgfpathlineto{\pgfqpoint{4.883281in}{2.187912in}}%
\pgfpathlineto{\pgfqpoint{4.888125in}{2.221357in}}%
\pgfpathlineto{\pgfqpoint{4.892969in}{2.222237in}}%
\pgfpathlineto{\pgfqpoint{4.897812in}{2.190937in}}%
\pgfpathlineto{\pgfqpoint{4.902656in}{2.137522in}}%
\pgfpathlineto{\pgfqpoint{4.912344in}{1.987952in}}%
\pgfpathlineto{\pgfqpoint{4.926875in}{1.732428in}}%
\pgfpathlineto{\pgfqpoint{4.931719in}{1.673462in}}%
\pgfpathlineto{\pgfqpoint{4.936562in}{1.651244in}}%
\pgfpathlineto{\pgfqpoint{4.941406in}{1.671299in}}%
\pgfpathlineto{\pgfqpoint{4.946250in}{1.728045in}}%
\pgfpathlineto{\pgfqpoint{4.960781in}{1.956416in}}%
\pgfpathlineto{\pgfqpoint{4.965625in}{2.009496in}}%
\pgfpathlineto{\pgfqpoint{4.975313in}{2.082789in}}%
\pgfpathlineto{\pgfqpoint{4.980156in}{2.113458in}}%
\pgfpathlineto{\pgfqpoint{4.985000in}{2.134778in}}%
\pgfpathlineto{\pgfqpoint{4.989844in}{2.136734in}}%
\pgfpathlineto{\pgfqpoint{4.994688in}{2.114257in}}%
\pgfpathlineto{\pgfqpoint{5.009219in}{1.989171in}}%
\pgfpathlineto{\pgfqpoint{5.014063in}{1.985204in}}%
\pgfpathlineto{\pgfqpoint{5.018906in}{2.016279in}}%
\pgfpathlineto{\pgfqpoint{5.028594in}{2.147340in}}%
\pgfpathlineto{\pgfqpoint{5.038281in}{2.273559in}}%
\pgfpathlineto{\pgfqpoint{5.043125in}{2.309023in}}%
\pgfpathlineto{\pgfqpoint{5.047969in}{2.322176in}}%
\pgfpathlineto{\pgfqpoint{5.052812in}{2.313058in}}%
\pgfpathlineto{\pgfqpoint{5.057656in}{2.281773in}}%
\pgfpathlineto{\pgfqpoint{5.062500in}{2.230239in}}%
\pgfpathlineto{\pgfqpoint{5.072188in}{2.094596in}}%
\pgfpathlineto{\pgfqpoint{5.081875in}{1.956534in}}%
\pgfpathlineto{\pgfqpoint{5.086719in}{1.896747in}}%
\pgfpathlineto{\pgfqpoint{5.091563in}{1.851044in}}%
\pgfpathlineto{\pgfqpoint{5.096406in}{1.829570in}}%
\pgfpathlineto{\pgfqpoint{5.101250in}{1.841887in}}%
\pgfpathlineto{\pgfqpoint{5.106094in}{1.895452in}}%
\pgfpathlineto{\pgfqpoint{5.110938in}{1.986821in}}%
\pgfpathlineto{\pgfqpoint{5.125469in}{2.323540in}}%
\pgfpathlineto{\pgfqpoint{5.130313in}{2.395026in}}%
\pgfpathlineto{\pgfqpoint{5.135156in}{2.434612in}}%
\pgfpathlineto{\pgfqpoint{5.140000in}{2.448841in}}%
\pgfpathlineto{\pgfqpoint{5.144844in}{2.444767in}}%
\pgfpathlineto{\pgfqpoint{5.149687in}{2.428632in}}%
\pgfpathlineto{\pgfqpoint{5.154531in}{2.400245in}}%
\pgfpathlineto{\pgfqpoint{5.169062in}{2.281516in}}%
\pgfpathlineto{\pgfqpoint{5.173906in}{2.261123in}}%
\pgfpathlineto{\pgfqpoint{5.178750in}{2.263401in}}%
\pgfpathlineto{\pgfqpoint{5.183594in}{2.288724in}}%
\pgfpathlineto{\pgfqpoint{5.198125in}{2.390922in}}%
\pgfpathlineto{\pgfqpoint{5.202969in}{2.408044in}}%
\pgfpathlineto{\pgfqpoint{5.207813in}{2.412450in}}%
\pgfpathlineto{\pgfqpoint{5.212656in}{2.402732in}}%
\pgfpathlineto{\pgfqpoint{5.217500in}{2.377316in}}%
\pgfpathlineto{\pgfqpoint{5.222344in}{2.332428in}}%
\pgfpathlineto{\pgfqpoint{5.227188in}{2.264785in}}%
\pgfpathlineto{\pgfqpoint{5.236875in}{2.079178in}}%
\pgfpathlineto{\pgfqpoint{5.246563in}{1.901310in}}%
\pgfpathlineto{\pgfqpoint{5.251406in}{1.847189in}}%
\pgfpathlineto{\pgfqpoint{5.256250in}{1.825934in}}%
\pgfpathlineto{\pgfqpoint{5.261094in}{1.836933in}}%
\pgfpathlineto{\pgfqpoint{5.265937in}{1.874915in}}%
\pgfpathlineto{\pgfqpoint{5.275625in}{2.009430in}}%
\pgfpathlineto{\pgfqpoint{5.290156in}{2.239641in}}%
\pgfpathlineto{\pgfqpoint{5.295000in}{2.288251in}}%
\pgfpathlineto{\pgfqpoint{5.299844in}{2.312257in}}%
\pgfpathlineto{\pgfqpoint{5.304688in}{2.313066in}}%
\pgfpathlineto{\pgfqpoint{5.309531in}{2.293967in}}%
\pgfpathlineto{\pgfqpoint{5.314375in}{2.263606in}}%
\pgfpathlineto{\pgfqpoint{5.324063in}{2.184304in}}%
\pgfpathlineto{\pgfqpoint{5.333750in}{2.098042in}}%
\pgfpathlineto{\pgfqpoint{5.338594in}{2.065316in}}%
\pgfpathlineto{\pgfqpoint{5.343438in}{2.049985in}}%
\pgfpathlineto{\pgfqpoint{5.348281in}{2.054403in}}%
\pgfpathlineto{\pgfqpoint{5.353125in}{2.073184in}}%
\pgfpathlineto{\pgfqpoint{5.362813in}{2.127578in}}%
\pgfpathlineto{\pgfqpoint{5.367656in}{2.144295in}}%
\pgfpathlineto{\pgfqpoint{5.372500in}{2.150094in}}%
\pgfpathlineto{\pgfqpoint{5.377344in}{2.141619in}}%
\pgfpathlineto{\pgfqpoint{5.382187in}{2.117001in}}%
\pgfpathlineto{\pgfqpoint{5.387031in}{2.070914in}}%
\pgfpathlineto{\pgfqpoint{5.396719in}{1.922902in}}%
\pgfpathlineto{\pgfqpoint{5.406406in}{1.778851in}}%
\pgfpathlineto{\pgfqpoint{5.411250in}{1.746403in}}%
\pgfpathlineto{\pgfqpoint{5.416094in}{1.746954in}}%
\pgfpathlineto{\pgfqpoint{5.420937in}{1.774872in}}%
\pgfpathlineto{\pgfqpoint{5.425781in}{1.821222in}}%
\pgfpathlineto{\pgfqpoint{5.435469in}{1.951412in}}%
\pgfpathlineto{\pgfqpoint{5.445156in}{2.134738in}}%
\pgfpathlineto{\pgfqpoint{5.450000in}{2.230939in}}%
\pgfpathlineto{\pgfqpoint{5.454844in}{2.304395in}}%
\pgfpathlineto{\pgfqpoint{5.459688in}{2.338815in}}%
\pgfpathlineto{\pgfqpoint{5.464531in}{2.324930in}}%
\pgfpathlineto{\pgfqpoint{5.469375in}{2.270915in}}%
\pgfpathlineto{\pgfqpoint{5.479063in}{2.135554in}}%
\pgfpathlineto{\pgfqpoint{5.483906in}{2.088503in}}%
\pgfpathlineto{\pgfqpoint{5.488750in}{2.062588in}}%
\pgfpathlineto{\pgfqpoint{5.508125in}{1.993226in}}%
\pgfpathlineto{\pgfqpoint{5.512969in}{1.991129in}}%
\pgfpathlineto{\pgfqpoint{5.517812in}{2.008522in}}%
\pgfpathlineto{\pgfqpoint{5.532344in}{2.100516in}}%
\pgfpathlineto{\pgfqpoint{5.537187in}{2.097853in}}%
\pgfpathlineto{\pgfqpoint{5.542031in}{2.065397in}}%
\pgfpathlineto{\pgfqpoint{5.551719in}{1.938040in}}%
\pgfpathlineto{\pgfqpoint{5.561406in}{1.806534in}}%
\pgfpathlineto{\pgfqpoint{5.566250in}{1.763939in}}%
\pgfpathlineto{\pgfqpoint{5.571094in}{1.743517in}}%
\pgfpathlineto{\pgfqpoint{5.575938in}{1.749051in}}%
\pgfpathlineto{\pgfqpoint{5.580781in}{1.784110in}}%
\pgfpathlineto{\pgfqpoint{5.585625in}{1.843903in}}%
\pgfpathlineto{\pgfqpoint{5.605000in}{2.154671in}}%
\pgfpathlineto{\pgfqpoint{5.609844in}{2.199920in}}%
\pgfpathlineto{\pgfqpoint{5.614688in}{2.222883in}}%
\pgfpathlineto{\pgfqpoint{5.619531in}{2.227639in}}%
\pgfpathlineto{\pgfqpoint{5.624375in}{2.217554in}}%
\pgfpathlineto{\pgfqpoint{5.629219in}{2.199322in}}%
\pgfpathlineto{\pgfqpoint{5.638906in}{2.155358in}}%
\pgfpathlineto{\pgfqpoint{5.643750in}{2.143727in}}%
\pgfpathlineto{\pgfqpoint{5.648594in}{2.152581in}}%
\pgfpathlineto{\pgfqpoint{5.653437in}{2.188344in}}%
\pgfpathlineto{\pgfqpoint{5.658281in}{2.259370in}}%
\pgfpathlineto{\pgfqpoint{5.663125in}{2.364911in}}%
\pgfpathlineto{\pgfqpoint{5.682500in}{2.887809in}}%
\pgfpathlineto{\pgfqpoint{5.687344in}{2.962947in}}%
\pgfpathlineto{\pgfqpoint{5.692188in}{3.003702in}}%
\pgfpathlineto{\pgfqpoint{5.697031in}{3.013145in}}%
\pgfpathlineto{\pgfqpoint{5.701875in}{2.999357in}}%
\pgfpathlineto{\pgfqpoint{5.706719in}{2.966786in}}%
\pgfpathlineto{\pgfqpoint{5.711563in}{2.920076in}}%
\pgfpathlineto{\pgfqpoint{5.726094in}{2.746530in}}%
\pgfpathlineto{\pgfqpoint{5.730938in}{2.712023in}}%
\pgfpathlineto{\pgfqpoint{5.735781in}{2.703540in}}%
\pgfpathlineto{\pgfqpoint{5.740625in}{2.715888in}}%
\pgfpathlineto{\pgfqpoint{5.750312in}{2.759155in}}%
\pgfpathlineto{\pgfqpoint{5.755156in}{2.764555in}}%
\pgfpathlineto{\pgfqpoint{5.760000in}{2.747347in}}%
\pgfpathlineto{\pgfqpoint{5.764844in}{2.700840in}}%
\pgfpathlineto{\pgfqpoint{5.764844in}{2.700840in}}%
\pgfusepath{stroke}%
\end{pgfscope}%
\begin{pgfscope}%
\pgfsetrectcap%
\pgfsetmiterjoin%
\pgfsetlinewidth{0.803000pt}%
\definecolor{currentstroke}{rgb}{0.000000,0.000000,0.000000}%
\pgfsetstrokecolor{currentstroke}%
\pgfsetdash{}{0pt}%
\pgfpathmoveto{\pgfqpoint{0.800000in}{0.528000in}}%
\pgfpathlineto{\pgfqpoint{0.800000in}{4.224000in}}%
\pgfusepath{stroke}%
\end{pgfscope}%
\begin{pgfscope}%
\pgfsetrectcap%
\pgfsetmiterjoin%
\pgfsetlinewidth{0.803000pt}%
\definecolor{currentstroke}{rgb}{0.000000,0.000000,0.000000}%
\pgfsetstrokecolor{currentstroke}%
\pgfsetdash{}{0pt}%
\pgfpathmoveto{\pgfqpoint{5.760000in}{0.528000in}}%
\pgfpathlineto{\pgfqpoint{5.760000in}{4.224000in}}%
\pgfusepath{stroke}%
\end{pgfscope}%
\begin{pgfscope}%
\pgfsetrectcap%
\pgfsetmiterjoin%
\pgfsetlinewidth{0.803000pt}%
\definecolor{currentstroke}{rgb}{0.000000,0.000000,0.000000}%
\pgfsetstrokecolor{currentstroke}%
\pgfsetdash{}{0pt}%
\pgfpathmoveto{\pgfqpoint{0.800000in}{0.528000in}}%
\pgfpathlineto{\pgfqpoint{5.760000in}{0.528000in}}%
\pgfusepath{stroke}%
\end{pgfscope}%
\begin{pgfscope}%
\pgfsetrectcap%
\pgfsetmiterjoin%
\pgfsetlinewidth{0.803000pt}%
\definecolor{currentstroke}{rgb}{0.000000,0.000000,0.000000}%
\pgfsetstrokecolor{currentstroke}%
\pgfsetdash{}{0pt}%
\pgfpathmoveto{\pgfqpoint{0.800000in}{4.224000in}}%
\pgfpathlineto{\pgfqpoint{5.760000in}{4.224000in}}%
\pgfusepath{stroke}%
\end{pgfscope}%
\begin{pgfscope}%
\definecolor{textcolor}{rgb}{0.000000,0.000000,0.000000}%
\pgfsetstrokecolor{textcolor}%
\pgfsetfillcolor{textcolor}%
\pgftext[x=3.280000in,y=4.307333in,,base]{\color{textcolor}\rmfamily\fontsize{12.000000}{14.400000}\selectfont Delay of Cascaded IIR Notch Filters}%
\end{pgfscope}%
\end{pgfpicture}%
\makeatother%
\endgroup%
}
    \end{subfigure}
    \caption{Comparison of delays for the window FIR filter (left) and cascaded IIR filters (right). The noisy ECG input is shown in orange and the filtered outputs are blue.}
\end{figure}

\noindent Of the four filters, the window FIR filter shows the greatest attenuation at the noise frequencies.
The trade-off is that it has the largest transition bandwidth. As a filters transition bandwidth increases,
the attenuation of desired spectral characteristics increases. This is undesirable as it can remove important
information from a signal. The effect of an increased transition bandwidth can be seen by comparing Fig. 5 with
Fig. 8. In the spectrum of Fig. 5, frequencies around the noise frequencies are attenuated. In the spectrum of Fig. 8,
the impulses representing the noise have been removed, with less obvious attenuation on the surrounding frequencies.
If the notch attenuation is too small, then noise will still be present in the spectrum. This is shown in Fig. 7.
Table 2 summaries the transition bandwidths and notch attenuations for the filters.\\

\begin{table}[H]
    \caption{Notch frequency attenuations and 3 dB bandwidths for the filters.}
    \label{table:noise-freq-atten}
    \adjustbox{max width=1.1\textwidth}{
    \centering
    \begin{tabularx}{\textwidth}{ | X || X | X || X | X | }
        \hline
        \empty & \multicolumn{2}{c ||}{32.6 Hz Noise} & \multicolumn{2}{c | }{62.7 Hz Noise}\\
        \hline
        Filter & 3dB Bandwidth (Hz) & Notch Attenuation & 3dB Bandwidth (Hz) & Notch Attenuation \\
        \hline
        Window FIR & 10 & -32 & 10 & -36\\
        Optimal FIR & 5 & -24 & 5 & -26\\
        Sampled FIR & 8 & -20 & 8 & -19\\
        IIR & 5 & -16 & 5 & -19\\
        \hline
    \end{tabularx}
    }
\end{table}

\noindent Experimentation found that a Hamming window gave the best response for the window FIR filter.
In general, the window choice is a trade-off between transition width and stop-band attenuation.
The maximum stopband attenuation is constant for a given window. A Blackman window has a greater
maximum stopband attenuation than a Hamming window, but it was found that using a Hamming window
resulted in more attenuation at the notch frequencies. This is because the Blackman window has a
larger transition bandwidth, so the attenuation it reaches for the same notch bandwidth is less 
than the Hamming window.\\

\noindent Pass-band ripple is another filter property which can undesirably affect the ECG signal
spectrum; this refers to fluctuations in the passband of a filter’s magnitude response. A small amount
of passband ripple can be seen in the filter frequency response of Fig. 6. Because of passband ripple,
frequencies that would ideally be unaffected are instead multiplied by slightly different amounts. This
could cause issues in detecting the R peak voltage of an ECG, which is the maximum amplitude of the QRS
complex ECG component. Figure 10 shows a typical representation of the QRS complex in an ECG. The R amplitude
is important for heart rate measurement and is used when detecting arrhythmia (abnormal heart rate) and other
heart issues [5], [6].\\

\begin{figure}[H]
    \centering
    \includegraphics[width=0.25\textwidth]{qrs.png}
    \caption{QRS complex of a typical ECG.}
    \label{fig:qrs}
\end{figure}

The best filter is the cascade of IIR filters. The noise frequencies are not present in the IIR filtered
ECG spectrum. This is also true for the window and optimal FIR filters, but the window filter has a large
transition bandwidth, and the optimal filter has obvious passband ripple. Another advantage of the IIR
filter is that it is far more computationally efficient than the FIR filters. Only 4 memory elements are
needed for the IIR filter, whereas the FIR filters required 399 or 400. The IIR filter is more suitable
for applications with limited resources than the FIR filters. The IIR filter output is faster to compute
because fewer operations are required. The result is less delay in the output of the IIR filter compared
to the FIR filters. The non-linear phase response shown in Fig. 9 must be tolerated with the IIR filter. 

\section{Conclusion}
An ECG signal corrupted by additive noise at 32.6 Hz and 61.7 Hz was filtered by four different notch
filters – three FIR and one IIR. Each FIR filter has two notches at the noise frequencies. They were
designed using the window, optimal, and frequency-sampled methods. Functions from Python’s SciPy
library were used to design the window and optimal filters, and a script was written to implement
the frequency-sampled method. The IIR filter consists of two cascaded IIR notch filters, each with
two zeros and two poles. These were designed analytically with pole-zero placement.\\

\noindent The IIR filter was determined to be the best filter. It was the most computationally efficient.
The noise was not visible in the ECG spectrum after IIR filtering. It also showed less input-to-output
delay than the FIR filters. The disadvantage is that it has a non-linear phase response. The frequency-sampled
FIR filtered ECG signal still had noise present in its spectrum. The window FIR filter has a wide transition
bandwidth at 10 Hz, so frequencies neighbouring the noise were significantly attenuated. The optimal FIR
filter has good attenuation at the noise frequencies and a transition bandwidth equal to the IIR filters
but shows some passband ripple. 

\section*{References}
 

\noindent[1] J. O. Smith, "Spectral Audio Signal Processing," W3K Publishing, 2011.\\

\noindent[2] B. Porr, "Digital Signal Processing," [Online]. Available: \\
https://www.berndporr.me.uk/teaching/dsp\_handout45.pdf. [Accessed 10 08 2020].\\

\noindent[3] L. R. Rabiner, B. Gold and C. A. McGonegal, "An Approach to the Approximation Problem for Nonrecursive Digital Filters," IEEE Transactions on Audio and Electroacoustics, Vols. AU-18, no. 2, pp. 83-106, 1970.\\

\noindent[4] S. Hargittai, "Efficient and fast ECG baseline wander reduction without distortion of important clinical information," in Computers in Cardiology, Bologna, 2008.\\

\noindent[5] S. S. Joshi and P. Shrivastava, "ECG beat detection using wavelet denoising," in Proceedings of the First International Conference on Contours of Computing Technology, Mumbai, 2010.\\

\noindent[6] "ECG interpretation: Characteristics of the normal ECG (P-wave, QRS complex, ST segment, T-wave)," ECG \& Echo Learning, [Online]. Available:\\
https://ecgwaves.com/topic/ecg-normal-p-wave-qrs-complex-st-segment-t-wave-j-point/. [Accessed 10 08 2020].\\
\end{document}